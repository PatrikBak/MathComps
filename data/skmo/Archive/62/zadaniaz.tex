{%%%%%   Z4-I-1
...}
\podpis{...}

{%%%%%   Z4-I-2
...}
\podpis{...}

{%%%%%   Z4-I-3
...}
\podpis{...}

{%%%%%   Z4-I-4
...}
\podpis{...}

{%%%%%   Z4-I-5
...}
\podpis{...}

{%%%%%   Z4-I-6
...}
\podpis{...}

{%%%%%   Z5-I-1
Mamička zaplatila v~kníhkupectve 270 €.
Platila dvoma druhmi bankoviek, dvadsaťeurovými a~päťdesiateurovými, a~presne.
Koľko ktorých bankoviek mohla použiť? Uveďte všetky možnosti.}
\podpis{Marie Krejčová}

{%%%%%   Z5-I-2
Pat napísal na tabuľu čudný príklad:
$$
  550+460+359+340=2\,012.
$$
Mat to chcel napraviť, preto pátral po neznámom čísle, ktoré by pripočítal ku
každému z~piatich uvedených čísel, aby potom bol príklad vypočítaný správne.
Aké to bolo číslo?}
\podpis{Libuše Hozová}

{%%%%%   Z5-I-3
Rudo dostal na narodeniny budík. Mal z~neho radosť a~nastavil na ňom presný čas.
Odvtedy každé ráno, keď vstával (vrátane sobôt, nedieľ a~prázdnin), stlačil presne na 4 sekundy tlačidlo, ktorým sa osvetľuje
ciferník. Pritom si všimol, že počas stlačenia tlačidla je čas na budíku
zastavený. Inak sa ale budík vôbec neoneskoruje. Popoludní 11.~decembra sa
Rudo pozrel na svoj budík a~zistil, že ukazuje presne o~3~minúty menej, ako
by mal. Aký je dátum Rudových narodenín?}
\podpis{Michaela Petrová}

{%%%%%   Z5-I-4
Červík sa skladá z~bielej hlavy a~niekoľkých článkov, pozri \obr.
\insp{z62.1}%

\noindent
Keď sa červík narodí, má hlavu a~jeden biely článok.
Každý deň pribudne červíkovi nový článok jedným z~nasledujúcich spôsobov:
\begin{itemize}
  \item buď sa niektorý biely článok rozdelí na biely a~sivý
  \item alebo sa niektorý sivý článok rozdelí na sivý a~biely
\end{itemize}
\noindent
(V~oboch prípadoch opisujeme situáciu pri pohľade na červíka od hlavy.)
Na štvrtý deň červík dospieva a~ďalej nerastie -- jeho telo sa skladá z~hlavy
a~štyroch článkov. Koľko najviac rôznych farebných variantov dospelých červíkov tohto druhu môže existovať?}
\podpis{Erika Novotná}

{%%%%%   Z5-I-5
Vypočítajte $3\cdot 15+20:4+1$. Potom doplňte do zadania jeden alebo viac párov zátvoriek tak, aby výsledok bol:
\begin{enumerate}
  \item čo najväčšie celé číslo,
  \item čo najmenšie celé číslo.
\end{enumerate}
}
\podpis{Marta Volfová}

{%%%%%   Z5-I-6
Sedem trpaslíkov sa postavilo po obvode svojej záhradky, do každého rohu jeden,
a~napli medzi sebou povraz okolo celej záhrady. Snehulienka vyšla od Kýblika
a~išla pozdĺž povrazu. Najskôr išla štyri metre na východ, kde stretla Vedka.
Od neho pokračovala dva metre na sever, kým dorazila k~Dudrošovi. Od
Dudroša išla na západ a~po dvoch metroch natrafila na Plaška. Ďalej
pokračovala tri metre na sever, až došla k~Smejkovi. Vydala sa na západ
a~po štyroch metroch stretla Kýchala, odkiaľ jej zostávali tri metre na juh
k~Spachtošovi. Nakoniec pozdĺž povrázka došla najkratšou cestou ku Kýblikovi
a~tým obišla celú záhradu. Koľko metrov štvorcových má celá záhrada?
%Kejchal = Kýchal, Dřímal = Spachtoš, Stydlín= Plaško, Prófa = Vedko, Štístko = Smejko,
%Šmudla = Kýblik, Rejpal=Dudroš
}
\podpis{Martin Mach}

{%%%%%   Z6-I-1
Ľuboš si myslí trojciferné prirodzené číslo, ktoré má všetky svoje cifry
nepárne. Ak k~nemu pripočíta $421$, dostane trojciferné číslo, ktoré nemá ani
jednu svoju cifru nepárnu. Nájdite všetky čísla, ktoré si môže Ľuboš myslieť.}
\podpis{Libor Šimůnek}

{%%%%%   Z6-I-2
Na \obr{} sú vyznačené mrežové body štvorčekovej siete,
z~ktorých dva sú pomenované $A$ a~$B$.
Nech bod~$C$ je jeden zo zvyšných mrežových bodov.
Nájdite všetky možné polohy bodu~$C$ tak, aby trojuholník $ABC$ mal obsah
4{,}5 štvorčeka.
\insp{z62.200}%
}
\podpis{Erika Novotná}

{%%%%%   Z6-I-3
Obri Koloman a~Bartolomej hovoria niektoré dni iba pravdu a~niekedy iba klamú.
Koloman hovorí pravdu v~pondelok, v~piatok a~v~nedeľu, ostatné dni klame.
Bartolomej hovorí pravdu v~stredu, štvrtok a~piatok, ostatné dni klame.
\begin{enumerate}
  \item Určte, kedy môže Koloman povedať: "Včera som hovoril pravdu."
  \item Jedného dňa obaja povedali: "Včera som klamal."
    V~ktorý deň to bolo?
\end{enumerate}
}
\podpis{Marta Volfová}

{%%%%%   Z6-I-4
Eva má tri lístočky a~na každom z~nich je napísané jedno prirodzené číslo.
Keď vynásobí medzi sebou všetky možné dvojice čísel z~lístočkov, dostane výsledky $48$, $192$ a~$36$.
Ktoré čísla sú napísané na Eviných lístočkoch?}
\podpis{Erika Novotná}

{%%%%%   Z6-I-5
Na \obr{} je útvar zložený z~dvanástich zhodných kociek.
Na koľko rôznych miest môžeme premiestniť tmavú kocku (označenú šípkou), ak chceme, aby sa
povrch zostaveného telesa nezmenil?
\insp{z62.300}%

\noindent
Rovnako ako pri pôvodnom telese sa aj pri novom telese musia kocky dotýkať celými
stenami. Poloha svetlých kociek sa meniť nemôže.
}
\podpis{David Reichmann}

{%%%%%   Z6-I-6
Obsluhujúci v~bufete U~Švindliara vždy započítava platiacemu hosťovi do účtu
aj dátum: celkovú minutú sumu zväčší o~toľko centov, koľký deň
v~mesiaci práve je.
V~septembri sa v~bufete dvakrát zišla trojica priateľov. Prvýkrát platil každý
z~nich zvlášť, obsluhujúci teda vždy pripísal dátum a~žiadal od každého 1{,}68 €.
O~štyri dni tam olovrantovali znova a~dali si presne to isté čo minule. Tentoraz
však jeden platil za všetkých dokopy. Obsluhujúci teda pripísal dátum do účtu
iba raz a~vypýtal si 4{,}86 €. Priateľom sa nezdalo, že aj keď sa ceny v~jedálnom
lístku nezmenili, majú olovrant lacnejší ako minule, a~podvod odhalili. Koľkého septembra práve bolo, keď podvod odhalili?}
\podpis{Libor Šimůnek}

{%%%%%   Z7-I-1
Na \obr{} je šesť krúžkov, ktoré tvoria vrcholy štyroch trojuholníkov.
Napíšte do krúžkov navzájom rôzne jednociferné prirodzené čísla tak, aby
v~každom trojuholníku platilo, že číslo vnútri je súčtom čísel napísaných
v~jeho vrcholoch.
Nájdite všetky riešenia.
\insp{z62.4}%
}
\podpis{Erika Novotná}

{%%%%%   Z7-I-2
Pred našou školou je kvetinový záhon. Jednu pätinu všetkých kvetov tvoria tulipány, dve devätiny narcisy, štyri pätnástiny hyacinty a~zvyšok sú sirôtky. Koľko kvetov je celkom na záhone, ak zo žiadneho druhu ich nie je viac ako 60 ani menej ako 30?}
\podpis{Michaela Petrová}

{%%%%%   Z7-I-3
Obri Bartolomej a~Koloman hovoria niektoré dni iba pravdu a~niektoré dni iba klamú.
Bartolomej hovorí pravdu iba cez víkendy, ostatné dni klame. Koloman hovorí pravdu v~pondelok, v~piatok a~v~nedeľu, ostatné dni klame.

Jedného dňa Bartolomej povedal: "Včera sme obaja klamali."

Koloman však nesúhlasil: "Aspoň jeden z~nás hovoril včera pravdu."

\noindent
Ktorý deň v~týždni môžu obri viesť takýto rozhovor?}
\podpis{Marta Volfová, Vojtěch Žádník}

{%%%%%   Z7-I-4
Pani učiteľka napísala na tabuľu nasledujúce čísla:
$$
1,\ 4,\ 7,\  10,\ 13,\ 16,\ 19,\ 22,\ 25,\ 28,\ 31,\ 34,\ 37,\ 40,\ 43.
$$
Dve susedné čísla sa líšia vždy o~rovnakú hodnotu, v~tomto prípade o~$3$.
Potom z~tabule zotrela všetky čísla okrem $1$, $19$ a~$43$.
Ďalej medzi čísla $1$ a~$43$ dopísala niekoľko celých čísel tak, že sa každé dve susedné
čísla opäť líšili o~rovnakú hodnotu a~pritom žiadne číslo nebolo napísané viackrát.
Koľkými spôsobmi mohla pani učiteľka čísla doplniť?}
\podpis{Karel Pazourek}

{%%%%%   Z7-I-5
V~športovom areáli je upravená plocha tvaru obdĺžnika $ABCD$ s~dlhšou
stranou~$AB$.
Uhlopriečky $AC$ a~$BD$ zvierajú uhol $60\st$.
Bežci trénujú na veľkom okruhu $ACBDA$ alebo na malej dráhe $ADA$.
Mojmír bežal desaťkrát po veľkom okruhu a~Vojtech
pätnásťkrát po malej dráhe, teda pätnásťkrát z~$A$ do $D$ a~pätnásťkrát z~$D$ do $A$.
Dokopy ubehli celkom 4{,}5\,km.
Aká dlhá je uhlopriečka~$AC$?}
\podpis{Libuše Hozová}

{%%%%%   Z7-I-6
Máme štvorčekovú sieť so 77 mrežovými bodmi. Dva z~nich sú označené
$A$ a~$B$ ako na \obr{}. Bod~$C$ nech je jeden zo zvyšných mrežových
bodov. Nájdite všetky možné polohy bodu~$C$ tak, aby trojuholník $ABC$ mal
obsah 6~štvorčekov.
\insp{z62.5}%
}
\podpis{Erika Novotná}

{%%%%%   Z8-I-1
Súčin troch prirodzených čísel je $600$.
Keby sme jedného činiteľa zmenšili o~$10$, zmenšil by sa súčin o~$400$.
Keby sme namiesto toho jedného činiteľa zväčšili o~$5$, zväčšil by sa súčin na
dvojnásobok pôvodnej hodnoty.
Ktoré tri prirodzené čísla majú túto vlastnosť?}
\podpis{Libuše Hozová}

{%%%%%   Z8-I-2
Stano zložil 7 zhodných útvarov, každý zlepený z~8 rovnakých sivých
kociek s~hranou 1\,cm tak ako na \obr{}.
\insp{z62.6}%

\noindent
Potom všetky ponoril do bielej farby a~následne každý z~útvarov rozobral na
pôvodných 8 dielov, ktoré tak mali niektoré steny sivé a~iné biele.
Pridal k~nim ešte 8~nových kociek, ktoré boli rovnaké ako ostatné,
akurát celé biele. Zo všetkých kociek dokopy zložil jednu veľkú kocku a~snažil sa pritom, aby čo najväčšia časť povrchu vzniknutej kocky bola sivá.
Koľko cm$^2$ povrchu bude určite bielych?}
\podpis{Martin Mach}

{%%%%%   Z8-I-3
Dedo zabudol štvorciferný kód svojho mobilu. Vedel len, že na prvom mieste
nebola nula, že uprostred boli buď dve štvorky alebo dve sedmičky alebo tiež štvorka
so sedmičkou (v~neznámom poradí)
a~že sa jednalo o~číslo deliteľné číslom~$15$.
Koľko je možností pre zabudnutý kód?
Aká cifra mohla byť na prvom mieste?}
\podpis{Marta Volfová}

{%%%%%   Z8-I-4
Daná je pravidelná sedemcípa hviezda ako na \obr{}. Aká je veľkosť vyznačeného uhla?
\insp{z62.7}%
}
\podpis{Eva Patáková}

{%%%%%   Z8-I-5
Dňa 1.~septembra 2007 bola založená jazyková škola, v~ktorej vyučovalo sedem pedagógov.
Dňa 1.~septembra 2010 k~týmto siedmim učiteľom pribudol nový kolega, ktorý mal práve
25~rokov. Do 1.~septembra 2012 jeden z~učiteľov zo školy odišiel, a~tak ich zostalo
opäť sedem. Priemerný vek pedagógov na škole bol vo všetkých troch spomenutých dátumoch
rovnaký.
Koľko rokov mal 1.~septembra 2012 učiteľ, ktorý v~škole už nepracoval?
Aký bol v~ten deň priemerný vek učiteľov na škole?}
\podpis{Libor Šimůnek}

{%%%%%   Z8-I-6
Anička a~Hanka chodili v~labyrinte po špirálovitej cestičke, ktorej
začiatok je schematicky znázornený na \obr{}.
Strana štvorčeka v~štvorčekovej sieti má dĺžku 1\,m a~celá cestička od stredu
bludiska až k~východu je dlhá 210\,m.
\insp{z62.8}%

\noindent
Dievčatá vyšli zo stredu bludiska,
nikde sa nevracali a~po čase každá zastavila v~niektorom z~rohov.
Anička pritom prešla o~24\,m viac ako Hanka.
V~ktorých rohoch mohli dievčatá stáť?
Určte všetky riešenia.}
\podpis{Erika Novotná}

{%%%%%   Z9-I-1
Na tabuli bolo napísané trojciferné prirodzené číslo.
Pripísali sme k~nemu všetky ďalšie trojciferné čísla, ktoré možno získať zmenou
poradia jeho cifier.
Na tabuli tak boli okrem pôvodného čísla ešte tri nové.
Súčet najmenších dvoch zo všetkých štyroch čísel je $1\,088$.
Aké cifry obsahuje pôvodné číslo?}
\podpis{Libuše Hozová}

{%%%%%   Z9-I-2
Trojuholník má dve strany, ktorých dĺžky sa líšia o~$12\cm$, a~dve strany,
ktorých dĺžky sa líšia o~$15\cm$.
Obvod tohto trojuholníka je $75\cm$.
Určte dĺžky jeho strán. Nájdite všetky možnosti.}
\podpis{Libor Šimůnek}

{%%%%%   Z9-I-3
Pri horskej chate nám tréner povedal:
"Ak pôjdeme ďalej týmto pohodlným tempom 4\,km za hodinu, prídeme na stanicu
45~minút po odchode nášho vlaku."
Potom ukázal na skupinu, ktorá nás práve míňala:
"Tí majú lepšiu obuv, a~tak dosahujú priemernú rýchlosť 6\,km za hodinu. Na
stanici budú už pol hodiny pred odchodom nášho vlaku."
Ako ďaleko bola stanica od horskej chaty?}
\podpis{Marta Volfová}

{%%%%%   Z9-I-4
Do kružnice s~polomerom $5\cm$ je vpísaný pravidelný osemuholník $ABCDEFGH$.
Zostrojte trojuholník $ABX$ tak, aby
bod~$D$ bol ortocentrom (priesečníkom výšok) trojuholníka $ABX$.}
\podpis{Martin Mach}

{%%%%%   Z9-I-5
Do každého políčka schémy na \obr{} máme zapísať štvorciferné prirodzené
číslo tak, aby všetky naznačené výpočtové operácie boli správne.
Koľkými rôznymi spôsobmi možno schému vyplniť?
\insp{z62.9}%
}
\podpis{Libor Šimůnek}

{%%%%%   Z9-I-6
Daný je pravouhlý lichobežník $ABCD$ s~pravým uhlom pri vrchole~$B$
a~s~rovnobežnými stranami $AB$ a~$CD$.
Uhlopriečky lichobežníka sú na seba kolmé a~majú dĺžky $|AC|=12\cm$, $|BD|=9\cm$.
Vypočítajte obvod a~obsah tohto lichobežníka.}
\podpis{Marie Krejčová}

{%%%%%   Z4-II-1
...}
\podpis{...}

{%%%%%   Z4-II-2
...}
\podpis{...}

{%%%%%   Z4-II-3
...}
\podpis{...}

{%%%%%   Z5-II-1
Na tábore sa skauti vážili na starodávnej váhe. Vedúci ich upozornil, že váha síce neváži správne, ale rozdiel medzi skutočnou a~nameranou hmotnosťou je vždy rovnaký. Mišovi váha ukázala 30\,kg, Emilovi 28\,kg, ale keď si stúpli na váhu obaja naraz, ukázala 56\,kg. Aká bola skutočná hmotnosť Miša a~Emila?}
\podpis{Marta Volfová}

{%%%%%   Z5-II-2
Na \ifobrazkyvedla{}obrázku\else\obr{}\fi{} je stavba zlepená zo 14 rovnakých kocôčok. Stavbu chceme zo všetkých
strán ofarbiť, teda aj zospodu. Aká bude celková spotreba farby, keď 10 mililitrov farby stačí na ofarbenie jednej celej kocôčky?
\ifobrazkyvedla\else\insp{z62ii.51}\fi%
}
\podpis{Martin Mach}

{%%%%%   Z5-II-3
Radka dostala ráno v~deň svojich narodenín veľké balenie lentiliek. Každý deň poobede lentilky maškrtila, a~to tak, že v~pracovný deň (\tj. mimo víkendu) si vzala vždy 3~lentilky a~každú sobotu aj nedeľu si ich vzala 5. Istý deň večer zistila, že zjedla práve 111 lentiliek. Ktorý deň v~týždni mohla mať Radka narodeniny? Nájdite obe možnosti.}
\podpis{Eva Patáková}

{%%%%%   Z6-II-1
Pat napísal na tabuľu príklad:
$$
589+544+80=2\,013.
$$
Mat chcel príklad opraviť, aby sa obe strany naozaj rovnali,
a~pátral po neznámom čísle, ktoré potom k~prvému sčítancu na ľavej strane
pripočítal, od druhého sčítanca ho odčítal a~tretieho sčítanca ním
vynásobil. Po prevedení týchto operácií bol príklad vypočítaný správne.
Aké číslo Mat našiel?
}
\podpis{Libuše Hozová}

{%%%%%   Z6-II-2
Lenka si myslí dve dvojciferné čísla. Jedno má obe cifry párne a~druhé obe nepárne.
Keď obe čísla sčíta, dostane opäť dvojciferné číslo, ktoré má prvú cifru
párnu a~druhú nepárnu. Navyše nám Lenka prezradila, že všetky tri dvojciferné čísla
sú násobkami čísla tri a~jedna z~nepárnych cifier je $9$.
Aké čísla si mohla Lenka myslieť? Nájdite všetky možnosti.
}
\podpis{Veronika Hucíková}

{%%%%%   Z6-II-3
Štvoruholník $ABCD$ má nasledujúce vlastnosti:
\begin{itemize}
  \iitem strany $AB$ a~$CD$ sú rovnobežné,
  \iitem pri vrchole~$B$ je pravý uhol,
  \iitem trojuholník $ADB$ je rovnoramenný so základňou~$AB$,
  \iitem strany $BC$ a~$CD$ sú dlhé $10\cm$.
\end{itemize}
Určte obsah tohto štvoruholníka.
}
\podpis{Ján Mazák}

{%%%%%   Z7-II-1
Dedo Vendelín išiel so svojimi dvoma vnukmi Cyrilom a~Metodom kúpiť rybárske prúty a ďalšie potreby. Cena nákupu zaujala Cyrila aj deda.
Išlo o~štvorciferné číslo, v~ktorom prvá cifra bola o~jedna väčšia ako tretia cifra, ale o~jedna menšia ako posledná cifra.
Súčet všetkých štyroch cifier bol~$6$.
Cyril si ešte všimol, že dvojčíslie zložené z~prvých dvoch cifier predstavovalo dvojciferné číslo o~$7$ väčšie ako dvojčíslie zložené z~posledných dvoch cifier.
%Cyril si ešte všimol, že prvé dvojčíslie predstavovalo dvojciferné číslo o~$7$ väčšie ako dvojčíslie druhé.
Deda však zaujalo číslo preto, lebo bolo súčinom jeho veku a~veku oboch vnukov,
pritom každý z~vnukov mal viac ako jeden rok. Koľko rokov mal dedo Vendelín a~koľko jeho vnuci?
}
\podpis{Libuše Hozová}

{%%%%%   Z7-II-2
Petra má rovnostranný trojuholník $ABC$. Najskôr trojuholník prehla tak, aby bod~$A$ splynul s~bodom~$C$.
Potom vzniknutý útvar prehla tak, že bod~$B$ splynul s~bodom~$C$.
Tento útvar potom obkreslila na papier a~zistila, že jeho obsah je $12\cm^2$. Určte obsah pôvodného trojuholníka.
}
\podpis{Erika Novotná}

{%%%%%   Z7-II-3
Do skladu priviezli cement vo vreciach po 25\,kg a~po 40\,kg. Menších vriec bolo dvakrát viac ako tých väčších.
Skladník nahlásil vedúcemu počet všetkých privezených vriec, ale nespomenul, koľko je ktorých.
Vedúci si myslel, že všetky vrecia vážia 25\,kg. Nahlásený počet vriec teda vynásobil číslom~25 a~výsledok zadokumentoval
ako hmotnosť dodávky cementu. Cez noc zlodeji ukradli 60~väčších vriec, a~tak na sklade ostalo presne
toľko kg cementu, koľko vedúci zapísal. Koľko kg cementu ostalo?
}
\podpis{Libor Šimůnek}

{%%%%%   Z8-II-1
Júlia má na papieri napísané štvorciferné číslo.
Keď vymení cifry na mieste stoviek a~jednotiek a~sčíta toto
nové číslo s~číslom pôvodným, dostane výsledok $3\,332$.
Keby však vymenila cifry na mieste tisícok a~desiatok a~sčítala by toto číslo
s~pôvodným, dostala by výsledok $7\,886$.
Zistite, aké číslo mala Júlia napísané na papieri.
}
\podpis{Erika Novotná}

{%%%%%   Z8-II-2
Pán Zeler mal na záhradke dve rovnaké nádrže tvaru štvorbokého hranola
so štvorcovým dnom, v~oboch dokopy mal 300 litrov vody.
V~prvej nádrži tvorila voda presnú kocku a~vyplnila 62{,}5\,\% nádrže, v~druhej
nádrži bolo vody o~50~litrov viac. Aké rozmery mali nádrže pána Zelera?
}
\podpis{Libuše Hozová}

{%%%%%   Z8-II-3
Šifrovacej hry sa zúčastnilo 168 hráčov v~50~tímoch, ktoré mali dva až päť členov.
Najviac bolo štvorčlenných tímov, trojčlenných tímov bolo~20 a~hry sa zúčastnil
aspoň jeden päťčlenný tím. Koľko bolo dvojčlenných, štvorčlenných a~päťčlenných tímov?
}
\podpis{Martin Mach}

{%%%%%   Z9-II-1
Do triedy chodí 33~žiakov. Pred Vianocami boli s~hájnikom v~lese plniť kŕmidlá. Dievčatá si rozobrali balíky sena. Chlapci sa rozdelili na dve skupiny: tí z~prvej skupiny vzali každý 4~vrecká mrkvy a~3~vrecká orechov (teda každý z~nich vzal 7~vreciek) a~tí z~druhej skupiny vzali každý jedno vrecko jabĺk a~jedno vrecko orechov (teda každý z~nich vzal 2~vrecká). Pomer počtu dievčat, chlapcov z~prvej skupiny a~chlapcov z~druhej skupiny bol rovnaký ako pomer celkového počtu vreciek orechov, jabĺk a~mrkvy. Koľko bolo v~triede dievčat, koľko chlapcov nieslo vrecká s~mrkvou a~koľko ich nieslo vrecká s~jablkami?}
\podpis{Martin Mach}

{%%%%%   Z9-II-2
Na čistú tabuľu sme žltou kriedou napísali trojciferné prirodzené číslo tvorené navzájom rôznymi nenulovými ciframi. Potom sme na tabuľu bielou kriedou vypísali všetky ďalšie trojciferné čísla, ktoré možno získať zmenou poradia cifier žltého čísla. Aritmetický priemer všetkých čísel na tabuli bol $370$. Každé číslo menšie ako žlté sme podčiarkli. Podčiarknuté čísla boli práve tri a~ich aritmetický priemer bol $205$. Určte žlté číslo.}
\podpis{Libor Šimůnek}

{%%%%%   Z9-II-3
Vyznačme vo všeobecnom trojuholníku $ABC$ nasledujúce body podľa \ifobrazkyvedla{}obrázka\else\obr{}\fi{}. Body $A_1$ a~$B_1$ sú stredy strán $BC$ a~$AC$, body $C_1$, $C_2$ a~$C_3$ delia stranu $AB$ na štyri rovnaké časti. Spojíme body $A_1$ a~$B_1$ s~bodmi $C_1$ a~$C_3$, takže vznikne mašľa ohraničená týmito spojnicami. Akú časť obsahu celého trojuholníka mašľa zaberá?
\ifobrazkyvedla\else\insp{z62ii.91}\fi%
}
\podpis{Martin Mach}

{%%%%%   Z9-II-4
Nájdite všetky sedemciferné čísla, ktoré obsahujú každú z~cifier $0$ až $6$ práve raz a~pre ktoré platí, že ich prvé aj posledné dvojčíslie je deliteľné 2, prvé aj posledné trojčíslie je deliteľné~3, prvé aj posledné štvorčíslie je deliteľné 4, prvé aj posledné päťčíslie je deliteľné 5 a~prvé aj posledné šesťčíslie je deliteľné 6.}
\podpis{Martin Mach}

{%%%%%   Z9-III-1
V~malom kráľovstve prišli poddaní pozdraviť nového kráľa a~jeho ministra.
Každý priniesol nejaký darček:
60~najchudobnejších prinieslo len vlastnoručne vyrobenú drevenú sošku,
tí bohatší buď 2~zlatky, alebo 3~strieborniaky.
Pritom strieborniakov bolo darovaných viac ako 40, ale menej ako 100.
Všetky sošky dostal minister a~k~tomu ešte sedminu všetkých zlatiek a~tretinu
všetkých strieborniakov.
Kráľ aj jeho minister dostali rovnaký počet predmetov.
Zistite, koľko mohlo byť poddaných, koľko mohlo byť darovaných zlatiek a~koľko strieborniakov.
}
\podpis{Marta Volfová}

{%%%%%   Z9-III-2
Matej mal v~riadku v~zošite napísaných šesť rôznych prirodzených čísel. Druhé z~nich
bolo dvojnásobkom prvého, tretie bolo dvojnásobkom druhého a~podobne každé ďalšie
číslo bolo dvojnásobkom predchádzajúceho. Matej všetky tieto čísla opísal do nasledujúcej
tabuľky, a~to v~ľubovoľnom poradí, do každého políčka jedno:
\insp{62-z9-iii-2}
Súčet oboch čísel v~prvom stĺpci tabuľky bol 136 a súčet čísel v~druhom stĺpci
bol dvojnásobný, teda 272. Určte súčet čísel v~treťom stĺpci tabuľky.}
\podpis{Libor Šimůnek}

{%%%%%   Z9-III-3
Daný je pravidelný osemuholník $ABCDEFGH$ a~bod~$X$ taký, že bod $A$ je ortocentrom
(\tj. priesečníkom výšok) trojuholníka $BDX$. Vypočítajte veľkosti vnútorných uhlov tohto
trojuholníka.}
\podpis{Vojtěch Žádník}

{%%%%%   Z9-III-4
%PAMPELIŠKA
V~slove TESTOVANIE majú byť nahradené rovnaké písmenká rovnakými nenulovými
ciframi a~rôzne písmenká rôznymi nenulovými ciframi.
Pritom súčin cifier výsledného čísla má byť druhou mocninou
nejakého prirodzeného čísla.
Nájdite najväčšie číslo, ktoré možno nahradením písmen pri splnení uvedených podmienok získať.
% SELECT * FROM `slova` WHERE CHAR_LENGTH(slovo)='10' and substring(slovo,1,1)=substring(slovo,4,1) and
% substring(slovo,2,1)=substring(slovo,10,1) and substring(slovo,1,1)!=substring(slovo,2,1) and substring(slovo,1,1)!=substring(slovo,3,1) and
% substring(slovo,1,1)!=substring(slovo,5,1) and substring(slovo,1,1)!=substring(slovo,6,1) and substring(slovo,1,1)!=substring(slovo,7,1) and
% substring(slovo,1,1)!=substring(slovo,8,1) and substring(slovo,1,1)!=substring(slovo,9,1) and substring(slovo,2,1)!=substring(slovo,3,1) and
% substring(slovo,2,1)!=substring(slovo,5,1) and substring(slovo,2,1)!=substring(slovo,6,1) and substring(slovo,2,1)!=substring(slovo,7,1) and
% substring(slovo,2,1)!=substring(slovo,8,1) and substring(slovo,2,1)!=substring(slovo,9,1) and substring(slovo,3,1)!=substring(slovo,5,1) and
% substring(slovo,3,1)!=substring(slovo,6,1) and substring(slovo,3,1)!=substring(slovo,7,1) and substring(slovo,3,1)!=substring(slovo,8,1) and
% substring(slovo,3,1)!=substring(slovo,9,1)  and substring(slovo,5,1)!=substring(slovo,6,1) and substring(slovo,5,1)!=substring(slovo,7,1) and
% substring(slovo,5,1)!=substring(slovo,8,1) and substring(slovo,5,1)!=substring(slovo,9,1) and substring(slovo,6,1)!=substring(slovo,7,1) and
% substring(slovo,6,1)!=substring(slovo,8,1) and substring(slovo,6,1)!=substring(slovo,9,1) and substring(slovo,7,1)!=substring(slovo,8,1) and
% substring(slovo,7,1)!=substring(slovo,9,1) and substring(slovo,8,1)!=substring(slovo,9,1)
}
\podpis{Eva Patáková}

