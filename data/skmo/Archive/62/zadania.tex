{%%%%%   A-I-1
Nájdite všetky dvojice prvočísel $p$, $q$, pre ktoré existuje prirodzené
číslo~$a$ také, že
$$
\frac{pq}{p+q}=\frac{a^2+1}{a+1}.
$$}
\podpis{Ján Mazák, Róbert Tóth}

{%%%%%   A-I-2
Dve kružnice $k_1(S_1,r_1)$ a~$k_2(S_2,r_2)$ sa zvonka dotýkajú
a~ležia vo štvorci $ABCD$ so stranou~$a$
tak, že $k_1$ se dotýka strán $AD$ a~$CD$ a~$k_2$ sa dotýka strán $BC$
a~$CD$. Dokážte, že aspoň jeden z~trojuholníkov
$AS_1S_2$, $ BS_1S_2$ má obsah najviac $\frac3{16}a^2$.}
\podpis{Tomáš Jurík}

{%%%%%   A-I-3
Označme $p(n)$
počet všetkých $n$-ciferných čísel zložených len z~cifier
$1$, $2$, $3$, $4$, $5$, v~ktorých
sa každé dve susedné cifry líšia aspoň o~$2$.
Dokážte, že pre každé prirodzené číslo $n$ platí
$$
5\cdot2{,}4^{n-1}\le p(n)\le 5\cdot2{,}5^{n-1}.
$$}
\podpis{Pavel Novotný}

{%%%%%   A-I-4
Nájdite všetky funkcie $f\colon {\Bbb R}\setminus\{0\}\to {\Bbb R}$ také, že pre všetky
nenulové čísla $x$, $y$ platí
$$
x\cdot f(xy)+f(-y)=x\cdot f(x).
$$}
\podpis{Pavel Calábek}

{%%%%%   A-I-5
Označme $I$ stred kružnice vpísanej trojuholníku $ABC$. Kružnica, ktorá
prechádza vrcholom~$B$
a~dotýka sa priamky~$AI$ v~bode~$I$, pretína strany $AB$, $BC$ postupne
v~bodoch $P$, $Q$. Priesečník priamky~$QI$ so stranou~$AC$ označme~$R$.
Dokážte, že platí
$$
|AR|\cdot|BQ|=|PI|^2.
$$}
\podpis{Jaroslav Švrček}

{%%%%%   A-I-6
V~obore reálnych čísel vyriešte sústavu rovníc
$$
\aligned
\sin^2x+\cos^2y=&\tg^2z,\\
\sin^2y+\cos^2z=&\tg^2x,\\
\sin^2z+\cos^2x=&\tg^2y.
\endaligned
$$}
\podpis{Pavel Calábek}

{%%%%%   B-I-1
Určte všetky trojice $(a,b,c)$ prirodzených čísel, pre ktoré platí
$$
2^a+4^b=8^c.
$$}
\podpis{Jaroslav Švrček}

{%%%%%   B-I-2
V~obore reálnych čísel riešte rovnicu
$$
x^3+(3\sqrt2-2)x^2-(1+\sqrt2)x-14(\sqrt2-1)=0,
$$
ak viete, že má aspoň jeden celočíselný koreň. Prípadné iracionálne korene
zapíšte v~jednoduchom tvare bez odmocnín z~iracionálnych čísel.}
\podpis{Jaromír Šimša}

{%%%%%   B-I-3
Nech $V$ je priesečník výšok ostrouhlého trojuholníka $ABC$. Priamka~$CV$ je
spoločnou dotyčnicou kružníc $k$ a~$l$, ktoré sa zvonka dotýkajú v~bode~$V$
a~pritom každá z~nich prechádza jedným z~vrcholov $A$ a~$B$. Ich priesečníky
s~vnútrami strán $AC$ a~$BC$ označme $P$ a~$Q$.
Dokážte, že polpriamka~$VC$ je osou uhla $PVQ$ a~že body $A$, $B$, $P$,~$Q$ ležia na jednej kružnici.}
\podpis{Jaroslav Švrček}

{%%%%%   B-I-4
Nájdite najmenšiu hodnotu zlomku
$$
V(n)=\dfrac{n^3-10n^2+17n-4}{n^2-10n+18},
$$
pričom $n$ je ľubovoľné prirodzené číslo väčšie ako~$2$.}
\podpis{Vojtech Bálint}

{%%%%%   B-I-5
V~rovine je daná úsečka~$AB$. Pre ľubovoľný bod~$X$ tejto roviny, ktorý je rôzny
od $A$ aj $B$, označme $X_A$, resp.
$X_B$ obraz bodu $A$, resp. $B$  v~osovej súmernosti podľa priamky~$XB$,
resp. $XA$. Nájdite všetky také body~$X$, ktoré spolu s~bodmi $X_A$,
$X_B$ tvoria vrcholy rovnostranného trojuholníka.}
\podpis{Pavel Calábek}

{%%%%%   B-I-6
Je dané prirodzené číslo $k<12$. Vo vrcholoch pravidelného 12-uholníka sú
napísané čísla $1,2,\dots,12$ (ako na ciferníku hodín). V~jednom kroku
môžeme buď vymeniť niektoré dve protiľahlé čísla, alebo
zvoliť ľubovoľných $k$~susedných vrcholov a~v~nich napísané čísla zväčšiť o~$1$.
Označme $T(k)$ nasledovné tvrdenie: "Po konečnom počte krokov možno dostať všetkých
12~čísel rovnakých."
Dokážte, že $T(2)$ neplatí, $T(5)$ platí, a~rozhodnite o~platnosti $T(3)$.}
\podpis{Ján Mazák}

{%%%%%   C-I-1
Štvorcová tabuľka je rozdelená na $16\times16$ políčok. Kobylka sa po nej
pohybuje dvoma smermi: vpravo alebo dole, pričom
strieda skoky o~dve a~o~tri políčka (t.\,j. žiadne dva po sebe idúce
skoky nie sú rovnako dlhé). Začína skokom dĺžky dva
z~ľavého horného políčka. Koľkými rôznymi cestami sa môže
kobylka dostať na pravé dolné políčko? (Pod cestou máme na mysli
postupnosť políčok, na ktoré kobylka doskočí.)}
\podpis{Peter Novotný}

{%%%%%   C-I-2
Pre kladné reálne čísla $a$, $b$, $c$, $d$ platí
$$
a+b=c+d,\qquad ad = bc,\qquad ac+bd = 1.
$$
Akú najväčšiu hodnotu môže mať súčet $a+b+c+d$?}
\podpis{Ján Mazák}

{%%%%%   C-I-3
Daný je obdĺžnik $ABCD$ s~obvodom~$o$. V~jeho rovine nájdite množinu všetkých bodov,
ktorých súčet vzdialeností od priamok $AB$, $BC$, $CD$, $DA$ je
rovný~$\frac23o$.}
\podpis{Tomáš Jurík}

{%%%%%   C-I-4
Rozhodnite, či z~ľubovoľných siedmich vrcholov daného pravidelného
19-uholníka možno vždy vybrať štyri, ktoré sú vrcholmi lichobežníka.}
\podpis{Jaromír Šimša}

{%%%%%   C-I-5
Určte všetky celé čísla~$n$, pre ktoré $2n^3-3n^2+n+3$ je prvočíslo.}
\podpis{Jaroslav Švrček}

{%%%%%   C-I-6
Vnútri pravidelného šesťuholníka $ABCDEF$ s~obsahom $30\cm^2$ je zvolený
bod~$M$. Obsahy trojuholníkov $ABM$ a~$BCM$ sú postupne $3\cm^2$
a~$2\cm^2$. Určte obsahy trojuholníkov $CDM$,  $DEM$,  $EFM$ a~$FAM$.}
\podpis{Pavel Leischner}

{%%%%%   A-S-1
V obdĺžniku $ABCD$ so stranami $|AB|=9$, $|BC|=8$ ležia navzájom sa dotýkajúce kružnice $k_1(S_1,r_1)$ a $k_2(S_2,r_2)$ tak, že $k_1$ sa dotýka strán $AD$ a $CD$, $k_2$ sa dotýka strán $AB$ a $BC$.
\ite a) Dokážte, že $r_1+r_2=5$.
\ite b) Určte najmenšiu a najväčšiu možnú hodnotu obsahu trojuholníka $AS_1S_2$.\endgraf}
\podpis{Pavel Novotný}

{%%%%%   A-S-2
Na každej z~$n+1$ stien $n$-bokého ihlana je napísané číslo~$0$. V~každom kroku zvolíme niektorý vrchol a~čísla na
všetkých stenách obsahujúcich tento vrchol zväčšíme o~$1$ alebo ich všetky zmenšíme o~$1$. Dokážte, že nemôže nastať situácia,
v~ktorej by na všetkých stenách ihlana bolo napísané číslo~$1$.}
\podpis{Peter Novotný}

{%%%%%   A-S-3
Určte všetky trojice reálnych čísel $a$, $b$, $c$, ktoré spĺňajú podmienky
$$
a^2+b^2+c^2=26,\quad a+b=5\quad\text{a}\quad b+c\ge7.
$$
}
\podpis{Pavel Novotný}

{%%%%%   A-II-1
Daných je 21 rôznych celých čísel takých, že súčet ľubovoľných jedenástich
z~nich je väčší ako súčet desiatich zvyšných čísel.
\ite{a)} Dokážte, že každé z~daných čísel je väčšie ako $100$.
\ite{b)} Určte všetky také skupiny 21~rôznych celých čísel,
ktoré obsahujú číslo~$101$.}
\podpis{Jaromír Šimša}

{%%%%%   A-II-2
Nech $A$, $B$ sú množiny celých kladných čísel také, že súčet
ľubovoľných dvoch rôznych čísel z~$A$
patrí do $B$ a~podiel ľubovoľných dvoch rôznych čísel z~$B$ (väčšie delené menším)
patrí do~$A$.
Určte najväčší možný počet prvkov množiny $A\cup B$.}
\podpis{Martin Panák}

{%%%%%   A-II-3
V~pravouhlom trojuholníku $ABC$ s~preponou~$AB$ a~odvesnami dĺžok
$|AC|=4$ a~$|BC|=3$ ležia navzájom sa dotýkajúce kružnice
$k_1(S_1,r_1)$ a~$k_2(S_2,r_2)$ tak, že $k_1$ sa dotýka strán $AB$ a~$AC$
a~$k_2$ sa dotýka strán $AB$ a~$BC$. Určte polomery $r_1$ a~$r_2$, ak platí
$4r_1=9r_2$.}
\podpis{Pavel Novotný}

{%%%%%   A-II-4
Dokážte, že kladné čísla $a$, $b$, $c$ sú dĺžkami strán
trojuholníka práve vtedy, keď sústava rovníc
$$
a(yz+x)=b(xz+y)=c(xy+z),\quad x+y+z=1
$$
s~neznámymi $x$, $y$, $z$ má riešenie v~obore kladných reálnych čísel.}
\podpis{Tomáš Jurík}

{%%%%%   A-III-1
Nájdite všetky dvojice celých čísel $a$, $b$, pre ktoré platí rovnosť
$$
\frac{a^2+1}{2b^2-3}=\frac{a-1}{2b-1}.
$$
}
\podpis{Pavel Novotný}

{%%%%%   A-III-2
Každý zo zbojníkov v~$n$-člennej družine $(n\ge3)$ nazbíjal určitý počet mincí. Všetkých nazbíjaných mincí bolo $100n$. Zbojníci sa rozhodli podeliť korisť nasledujúcim spôsobom: v~každom kroku dá jeden zo zbojníkov po jednej minci iným dvom. Nájdite všetky prirodzené čísla $n\ge3$, pre ktoré po konečnom počte krokov môže mať každý zbojník 100 mincí bez ohľadu na to, koľko mincí jednotliví zbojníci nazbíjali.}
\podpis{Ján Mazák}

{%%%%%   A-III-3
V~rovnobežníku $ABCD$ so stredom~$S$ označme
$O$ stred kružnice vpísanej trojuholníku $ABD$ a~$T$ bod jej
dotyku s~uhlopriečkou~$BD$. Dokážte, že priamky $OS$ a~$CT$ sú
rovnobežné.}
\podpis{Jaromír Šimša}

{%%%%%   A-III-4
Na tabuli je napísané v~desiatkovej sústave celé kladné číslo~$N$. Ak nie je jednociferné, zotrieme jeho poslednú cifru~$c$ a~číslo~$m$, ktoré na tabuli ostane, nahradíme číslom ${|m-3c|}$. (Ak napríklad bolo na tabuli číslo $N=1\,204$, po úprave tam bude $120-{3\cdot4}=108$.) Nájdite všetky prirodzené čísla~$N$, z~ktorých opakovaním opísanej úpravy nakoniec dostaneme číslo~$0$.}
\podpis{Peter Novotný}

{%%%%%   A-III-5
Daný je rovnobežník $ABCD$ taký, že päty $K$, $L$ kolmíc spustených z~bodu~$D$ postupne na strany $AB$, $BC$ sú ich vnútornými bodmi. Dokážte, že $KL\parallel AC$ práve vtedy, keď
$$
|\uhol BCA|+|\uhol ABD|=|\uhol BDA|+|\uhol ACD|.
$$}
\podpis{Ján Mazák}

{%%%%%   A-III-6
Nájdite všetky kladné reálne čísla $p$ také, že
$$
\sqrt{a^2+pb^2}+\sqrt{b^2+pa^2}\ge a+b+(p-1)\sqrt{ab}
$$
platí pre ľubovoľnú dvojicu kladných reálnych čísel $a$, $b$.}
\podpis{Jaromír Šimša}

{%%%%%   B-S-1
Dokážte, že žiadna z~rovníc
$$
3^{2x}+6^y=2013,\quad |3^{2x}-6^y|=2013
$$
nemá riešenie v~obore celých kladných čísel.
}
\podpis{Jaroslav Švrček}

{%%%%%   B-S-2
Do políčok štvorčekovej mriežky $11{\times}11$ sme postupne zľava doprava a~zhora nadol
zapísali čísla $1,2,\dots,121$. Štvorcovou doskou $3{\times}3$ sme všetkými
možnými spôsobmi zakryli presne deväť políčok. V~koľkých prípadoch bol súčet deviatich zakrytých
čísel druhou mocninou celého čísla?
}
\podpis{Vojtech Bálint}

{%%%%%   B-S-3
Uvažujme dve kružnice so stredmi $S_1$ a~$S_2$ také, že ich
spoločné vnútorné dotyčnice pretínajú ich spoločné vonkajšie dotyčnice v~štyroch bodoch. Dokážte, že tieto
štyri priesečníky ležia na Tálesovej kružnici nad priemerom~$S_1S_2$.
}
\podpis{Tomáš Jurík}

{%%%%%   B-II-1
Pre ľubovoľné reálne čísla $k\ne\pm1$, $p\ne0$ a~$q$ dokážte tvrdenie: Rovnica
$$
x^2+px+q=0
$$
má v~obore reálnych čísel dva korene, z~ktorých jeden je $k$-násobkom druhého,
práve vtedy, keď platí $kp^2=(k+1)^2q$.}
\podpis{Jaromír Šimša}

{%%%%%   B-II-2
Obec má 100 obyvateľov. Vieme, že každý z~nich má v~obci práve troch známych. (Známosti sú
vzájomné.)
\ite{a)} Dokážte, že v~obci existuje skupina 25~osôb, medzi ktorými sa žiadne dve
nepoznajú.
\ite{b)} Nájdite najmenšie prirodzené číslo $n$ s~vlastnosťou, že v~ľubovoľnej
skupine $n$~osôb každej takej obce existuje dvojica známych.\endgraf}
\podpis{Ján Mazák}

{%%%%%   B-II-3
Určte všetky trojice $(a,b,c)$ celých kladných čísel, pre ktoré platí
$$
2^{a+2b+1}+4^a+16^b=4^c.
$$}
\podpis{Jaroslav Švrček}

{%%%%%   B-II-4
V~rovine sú dané kružnice $m$, $n$, ktoré sa pretínajú v~bodoch $K$, $L$.
Dotyčnica v~bode~$K$ ku kružnici~$m$ pretína kružnicu~$n$ v~bode $A\ne K$,
dotyčnica v~bode~$L$ ku kružnici~$n$ pretína kružnicu~$m$ v~bode $C\ne K$. Bod
$B\ne L$ je priesečník priamky~$AL$ s~kružnicou~$m$ a~bod $D\ne K$ je priesečník
priamky~$CK$ s~kružnicou~$n$. Dokážte, že štvoruholník $ABCD$ je rovnobežník.}
\podpis{Pavel Leischner}

{%%%%%   C-S-1
Danému rovnostrannému trojuholníku vpíšme a~opíšme kružnicu.
Označme $S$ obsah vzniknutého medzikružia a~$T$ obsah kruhu, ktorého priemer je zhodný s~dĺžkou strany daného
trojuholníka.
Ktorý z~obsahov $S$, $T$ je väčší? Svoju odpoveď zdôvodnite.
}
\podpis{Leo Boček}

{%%%%%   C-S-2
Určte všetky dvojice $a$, $b$ celých kladných čísel, pre ktoré platí
$$
a\cdot [a,b]=4\cdot (a,b),
$$
pričom symbol $[a,b]$ označuje najmenší spoločný násobok a~$(a,b)$
najväčší spoločný deliteľ celých kladných čísel $a$, $b$.
}
\podpis{Jaroslav Švrček}

{%%%%%   C-S-3
Každý vrchol pravidelného devätnásťuholníka je ofarbený jednou
zo šiestich farieb. Dokážte, že niektorý tupouhlý trojuholník má všetky vrcholy ofarbené rovnakou farbou.
}
\podpis{Jaromír Šimša}

{%%%%%   C-II-1
V~tanečnej sa zišla skupina chlapcov a~dievčat. Každý z~prítomných 15~chlapcov pozná práve 4~dievčatá a~každé dievča pozná práve 10~chlapcov. (Známosti sú vzájomné.) Dokážte, že ľubovoľní dvaja chlapci majú aspoň dve spoločné známe.}
\podpis{Ján Mazák}

{%%%%%   C-II-2
Vnútri rovnobežníka $ABCD$ je daný bod~$K$ a~v~páse medzi rovnobežkami $BC$ a~$AD$
v~polrovine opačnej k~$CDA$ je daný bod~$L$. Obsahy trojuholníkov $ABK$,
$BCK$, $DAK$ a~$DCL$ sú $S_{ABK}=18\cm^2$,
$S_{BCK}=8\cm^2$, $S_{DAK}=16\cm^2$, $S_{DCL}=36\cm^2$.
Vypočítajte obsahy trojuholníkov $CDK$ a~$ABL$.}
\podpis{Pavel Novotný}

{%%%%%   C-II-3
Nájdite všetky dvojice celých kladných čísel $a$ a~$b$,
pre ktoré je číslo $a^2+b$ o~$62$ väčšie ako číslo $b^2+a$.}
\podpis{Jaromír Šimša}

{%%%%%   C-II-4
Určte najmenšie celé kladné číslo~$v$, pre ktoré platí: Medzi ľubovoľnými $v$~vrcholmi pravidelného
dvadsaťuholníka možno nájsť tri, ktoré sú vrcholmi
pravouhlého rovnoramenného trojuholníka.}
\podpis{Jaromír Šimša}

{%%%%%   vyberko, den 1, priklad 1
Dokážte, že pre každé celé číslo $n\ge 2$ a~ľubovoľné kladné reálne čísla $x_1, x_2, \dots, x_n$ platí
$$
\sum_{k=1}^n kx_k\le {n\choose 2} + \sum_{k=1}^n x_k^k.
$$}
\podpis{Ján Mazák, Róbert Tóth:http://amc.maa.org/a-activities/a6-mosp/a6-1-mosparchives/2002-ma/02mosphw.html, uloha 5}

{%%%%%   vyberko, den 1, priklad 2
Polpriamky $OA$ a~$OB$ sa dotýkajú kružnice~$k$ v~rôznych bodoch $A$ a~$B$. Nech $K$ je vnútorný bod kratšieho oblúka~$AB$
kružnice~$k$. Priesečník polpriamky~$OB$ s~rovnobežkou s~priamkou~$OA$ prechádzajúcou bodom~$K$ označme $L$.
Priesečník priamky~$AK$ s~kružnicou~$l$ opísanou trojuholníku $KLB$ (rôzny od $K$) označme $M$. Dokážte,
že priamka $OM$ sa dotýka kružnice~$l$.}
\podpis{Ján Mazák, Róbert Tóth:http://amc.maa.org/a-activities/a6-mosp/a6-1-mosparchives/2002-ma/02mosphw.html, uloha 11}

{%%%%%   vyberko, den 1, priklad 3
V~každej z~troch krajín žije $2n$ matematikov. Nájdite najmenšie celé číslo $k$ s~nasledujúcou vlastnosťou:
Ak každý matematik pozná aspoň $k$ kolegov z~iných krajín, tak existujú traja matematici, ktorí sa poznajú navzájom.
(Vzťah "poznať sa" je vzájomný.)}
\podpis{Ján Mazák, Róbert Tóth:http://www.imomath.com/othercomp/Fra/FraTST02.pdf, uloha 4}

{%%%%%   vyberko, den 1, priklad 4
Nájdite všetky usporiadané trojice kladných celých čísel $(a,b,c)$ také,
že $(a,b,c)=1$, $a\le b\le c$ a~číslo $a+b+c$ je deliteľom čísla $a^n+b^n+c^n$ pre každé kladné celé číslo~$n$.}
\podpis{Ján Mazák, Róbert Tóth:http://math.ca/Competitions/CMO/solutions/sol_2005.pdf, uloha 5}

{%%%%%   vyberko, den 2, priklad 1
Nájdite všetky trojice $(x,y,z)$ reálnych čísel, ktoré sú riešením sústavy rovníc
$$\eqalign{
x^3&=3x-12y+50,\cr
y^3&=12y+3z-2,\cr
z^3&=27z+27x.\cr
}$$
}
\podpis{Dominik Csiba, Tomáš Kocák:USA Team selection 2009 #7}

{%%%%%   vyberko, den 2, priklad 2
Nájdite všetky polynómy $P(x)$ s~reálnymi koeficientmi, pre ktoré je
$$
(x+1)P(x-1)-(x-1)P(x)
$$
konštantný polynóm.}
\podpis{Dominik Csiba, Tomáš Kocák:Canada national olympiad}

{%%%%%   vyberko, den 2, priklad 3
Nech $P$, $Q$ a~$R$ sú body na stranách $BC$, $CA$ a~$AB$ ostrouhlého trojuholníka $ABC$ také, že trojuholník $PQR$ je rovnostranný a~má minimálny obsah spomedzi všetkých takých rovnostranných trojuholníkov. Dokážte, že kolmice z~bodov $A$, $B$ a~$C$ postupne na strany $QR$, $RP$ a~$PQ$ sa pretínajú v jednom bode.}
\podpis{Dominik Csiba, Tomáš Kocák:USA Team selection 2008 #2}

{%%%%%   vyberko, den 2, priklad 4
Dokážte, že neexistuje celé číslo $n$ také, že $n^7+7$ je druhou mocninou celého čísla.}
\podpis{Dominik Csiba, Tomáš Kocák:USA Team selection 2008 #4}

{%%%%%   vyberko, den 3, priklad 1
Pre $n = 1, 2, 3$ budeme za číslo $n$-tého typu považovať nulu, ľubovoľný člen geometrickej postupnosti $1,(n+2),(n+2)^2,(n+2)^3,\dots$ a~tiež súčet niekoľkých jej rôznych členov. Dokážte, že každé prirodzené číslo sa dá vyjadriť ako súčet čísla prvého typu, čísla druhého typu a~čísla tretieho typu.}
\podpis{Tomáš Jurík, Martin Kollár:Moskovska matematicka olympiada 2012}

{%%%%%   vyberko, den 3, priklad 2
Je možné zafarbiť štvorčeky nekonečnej štvorčekovej siete dvoma farbami (bielou a~čiernou) tak, že ľubovoľná priamka rovnobežná so stranami štvorčekov prechádza konečne veľa bielymi štvorčekmi a~ľubovoľná priamka nerovnobežná so stranami štvorčekov prechádza konečne veľa čiernymi štvorčekmi? (Priamka prechádza štvorčekom, ak s~ním má spoločný aspoň jeden bod.)}
\podpis{Tomáš Jurík, Martin Kollár:Moskovska matematicka olympiada 2013}

{%%%%%   vyberko, den 3, priklad 3
Kružnice $k_1$ a~$k_2$ so stredmi v~bodoch $O_1$ a~$O_2$ sa pretínajú v~dvoch bodoch $A$ a~$B$. Priamky $O_2B$ a~$O_1B$ pretínajú kružnice $k_1$ a~$k_2$ postupne v~bodoch $E$ a~$F$ (rôznych od $B$). Rovnobežka s~priamkou $EF$ prechádzajúca bodom $B$ pretína kružnice $k_1$ a~$k_2$ v~bodoch $M$ a~$N$ (rôznych od $B$). Dokážte, že ak bod~$B$ leží vnútri úsečky~$MN$, tak $|MN|=|AE|+|AF|$.}
\podpis{Tomáš Jurík, Martin Kollár:vserosivska MO}

{%%%%%   vyberko, den 3, priklad 4
Dokážte, že ak pre nezáporné čísla $x$, $y$ platí nerovnosť $x^2+y^3\ge x^3+y^4$, tak platí aj nerovnosť $x^3+y^3\le 2$.}
\podpis{Tomáš Jurík, Martin Kollár:vserosivska MO}

{%%%%%   vyberko, den 4, priklad 1
Pre každé prirodzené číslo $n \ge 3$ nájdite najmenšie $k$ také, že množinu ľubovoľných $n$ bodov v~rovine, z~ktorých žiadne tri neležia na jednej priamke, možno oddeliť systémom $k$~priamok. (Systém priamok oddeľuje body množiny, ak pre každé dva body množiny existuje priamka, od ktorej ležia na opačných stranách.)}
\podpis{Richard Kollár, Peter Novotný:Kvant M498, riesenie 1/1979}

{%%%%%   vyberko, den 4, priklad 2
Na šachovnici $n \times n$ (pričom $n\ge2$) je položených niekoľko domín rozmerov $2 \times 1$, pričom na šachovnicu
už nie je možné umiestniť ďalšie domino tak, aby sa neprekrývalo s~nejakým už položeným
dominom. Dokážte, že voľných políčok na šachovnici nie je viac ako $n^2/3$.
%Tiež ukážte, že tento odhad už nemožno zlepšiť.
}
\podpis{Richard Kollár, Peter Novotný:Kvant M504, riesenie 3/1979}

{%%%%%   vyberko, den 4, priklad 3
Nech $1\le a_1 < a_2 < \dots < a_n \le 2n$ je konečná postupnosť prirodzených čísel, pričom $n \ge 6$.
\item{a)}
Dokážte, že pre $k = 6$ platí
$$
\min_{1\le i < j \le n} \nsn(a_i, a_j) \le k \left( \left\lfloor \frac{n}{2}\right\rfloor + 1 \right),
\tag1
$$
kde $\nsn(a_i, a_j)$ označuje najmenší spoločný násobok čísel $a_i$, $a_j$.
Ukážte, že koeficient $k=6$ je najlepší možný, \tj.  pre žiadne $n\ge 6$ neexistuje $k < 6$ také, že (1) platí
pre všetky postupnosti $\{a_i\}_{i=1}^{n}$ vyhovujúce zadaniu.
\item{b)}
Dokážte, že
$$
\max_{1\le i < j \le n} \nsd(a_i, a_j) >  \frac{38}{147} \, n -  \frac{310}{21},
$$
kde $\nsd(a_i, a_j)$ označuje najväčší spoločný deliteľ čísel $a_i$, $a_j$.\endgraf
%Ukážte, že  koeficient $k = 38/147$ je najlepší možný,
%t.j. neexistuje $c \in \RR$ také, že (\ref{eq2})
%platí pre $k > 38/147$ a pre všetky postupnosti $\{a_i\}_{i=1}^{n}$ vyhovujúce zadaniu.
}
\podpis{Richard Kollár, Peter Novotný:Kvant M507, riesenie 3/1979}

{%%%%%   vyberko, den 5, priklad 1
%{\it Zadanie bude zverejnené po IMO 2013.}
Neprázdnu množinu $A\subseteq \Bbb Z$ nazveme {\it čarovná}, ak spĺňa podmienku:
$$
\text{Ak $x,y\in A$ (môže byť aj $x=y$), potom aj $x^2+kxy+y^2\in A$ pre všetky $k\in \Bbb Z$.}
$$
Nájdite všetky také dvojice nenulových celých čísel $m$, $n$ (vrátane prípadov $m=n$), že jediná čarovná množina obsahujúca aj $m$ aj $n$ je $\Bbb Z$.
}
\podpis{Michal Kopf, Filip Sládek:N1 ISL 2012}

{%%%%%   vyberko, den 5, priklad 2
%{\it Zadanie bude zverejnené po IMO 2013.}
Daný je trojuholník $ABC$, pričom $|AB|\ne |AC|$. Označme $O$ stred jeho opísanej kružnice. Os uhla $BAC$ pretína stranu~$BC$ v~bode~$D$. Bod~$E$ je obrazom bodu~$D$ v~stredovej súmernosti podľa stredu strany~$BC$. Priamky kolmé na $BC$ prechádzajúce postupne bodmi $D$, $E$ pretínajú $AO$ a~$AD$ postupne v~$X$, $Y$. Dokážte, že body $B$, $X$, $C$ a~$Y$ ležia na jednej kružnici.
}
\podpis{Michal Kopf, Filip Sládek:G4 ISL 2012}

{%%%%%   vyberko, den 5, priklad 3
%{\it Zadanie bude zverejnené po IMO 2013.}
Filip a~Miki hrajú hru s $N > 2013$ účastníkmi výberového sústredenia a~2013 stoličkami umiestnenými na kružnici.
Najprv Filip posadí účastníkov na stoličky tak, aby na každej stoličke sedel aspoň jeden človek. V~ďalšom ťahu ide Miki a~už sa budú stále striedať.
\item{$\bullet$} Miki v~každom svojom ťahu presadí z~každej stoličky práve jedného človeka na susednú.
\item{$\bullet$} Filip vyberie niekoľko ľudí, ktorí sedia navzájom na rôznych stoličkách a~ktorých Miki v~predchádzajúcom ťahu nepresadil, a~každého z~nich presadí na susednú stoličku.

Nájdite najmenšie $N$ také, že Filipovi sa vždy podarí, aby po jeho ťahu boli obsadené všetky stoličky bez ohľadu na Mikiho snaženie a~dĺžku partie.
}
\podpis{Michal Kopf, Filip Sládek:C4 ISL 2012}

{%%%%%   vyberko, den 4, priklad 4
...}
\podpis{...}

{%%%%%   vyberko, den 5, priklad 4
...}
\podpis{...}

{%%%%%   trojstretnutie, priklad 1
Daný je tetivový štvoruholník $ABCD$, pričom $|BC|=|CD|$. Nech
$\omega$ je kružnica so stredom~$C$ dotýkajúca sa uhlopriečky~$BD$.
Označme $I$ stred kružnice vpísanej trojuholníku $ABD$. Dokážte, že priamka
prechádzajúca bodom~$I$, ktorá je rovnobežná s~$AB$, sa dotýka kružnice~$\omega$.}
\podpis{Kamil Duszenko}

{%%%%%   trojstretnutie, priklad 2
Dokážte, že pre každé reálne číslo $x>0$ a~každé celé číslo $n>0$ platí
$$
x^n+\frac1{x^n}-2\ge n^2\Bigl(x+\frac1x-2\Bigr).
$$
}
\podpis{Kamil Duszenko}

{%%%%%   trojstretnutie, priklad 3
Pre každé racionálne číslo~$r$ uvažujme tvrdenie:
Ak $x$ je reálne číslo také, že čísla $x^2-rx$ a~$x^3-rx$ sú obe racionálne, tak $x$ je tiež racionálne.
\ite a) Dokážte tvrdenie pre $r\ge\frac43$ a~pre $r\le 0$.
\ite b) Nech $p$, $q$ sú rôzne nepárne prvočísla také, že $3p<4q$. Dokážte, že tvrdenie pre $r=\frac pq$ neplatí.\endgraf
}
\podpis{Jaromír Šimša}

{%%%%%   trojstretnutie, priklad 4
Nech $a$, $b$ sú celé čísla, pričom $b$ nie je druhou mocninou celého čísla.
Dokážte, že $x^2+ax+b$ môže byť druhou mocninou celého čísla len pre konečne veľa celočíselných hodnôt~$x$.}
\podpis{Martin Panák}

{%%%%%   trojstretnutie, priklad 5
Trojuholníková sieť rozdeľuje rovnostranný trojuholník so stranou dĺžky~$n$ na
$n^2$ trojuholníkových buniek (\obr). Niektoré bunky sú infikované. Bunka, ktorá zatiaľ nie je infikovaná,
sa infikuje, ak susedí (stranou) s~aspoň dvoma už infikovanými bunkami. Určte pre $n=12$ najmenší možný počet na začiatku infikovaných buniek, pri ktorom je možné, že po čase budú infikované všetky bunky pôvodného trojuholníka.\insp{cps.3}
}
\podpis{Radek Horenský}

{%%%%%   trojstretnutie, priklad 6
Daný je trojuholník $ABC$ a~jemu opísaná kružnica. Bod $P$ je stredom oblúka $BAC$.
Kružnica s~priemerom~$CP$ pretína os uhla $BAC$ v~bodoch
$K$, $L$ ($|AK|<|AL|$). Bod~$M$ je obrazom bodu $L$ v~osovej súmernosti podľa
priamky~$BC$. Dokážte, že kružnica opísaná trojuholníku $BKM$ prechádza stredom úsečky~$BC$.}
\podpis{Dominik Burek, Tomasz Cie\'sla}

{%%%%%   IMO, priklad 1
Dokážte, že ku každej dvojici kladných celých čísel $k$ a~$n$ existuje $k$ kladných celých čísel $m_1, m_2, \dots, m_k$ (nie nutne rôznych) takých, že
$$
1+\frac{2^k-1}n = \left(1+\frac1{m_1}\right)\left(1+\frac1{m_2}\right) \cdots \left(1+\frac1{m_k}\right).
$$
}
\podpis{Japonsko}

{%%%%%   IMO, priklad 2
Konfigurácia 4027 bodov v~rovine sa nazýva {\it kolumbijská}, ak pozostáva z~2013 červených a~2014 modrých bodov a~žiadne tri body tejto konfigurácie neležia na jednej priamke. Ak nakreslíme niekoľko priamok, rovina sa rozdelí na niekoľko oblastí. Rozloženie priamok je {\it dobré} pre kolumbijskú konfiguráciu, ak sú splnené dve nasledovné podmienky:
\ite $\bullet$ žiadna priamka neprechádza žiadnym bodom konfigurácie;
\ite $\bullet$žiadna oblasť neobsahuje body oboch farieb.

Nájdite najmenšiu hodnotu $k$ takú, že pre každú kolumbijskú konfiguráciu 4027 bodov existuje dobré rozloženie $k$~priamok.}
\podpis{Austrália}

{%%%%%   IMO, priklad 3
Nech kružnica pripísaná k~strane~$BC$ trojuholníka $ABC$ sa dotýka strany~$BC$ v~bode~$A_1$. Definujme body $B_1$ na $CA$ a~$C_1$
na $AB$ analogicky, použijúc pripísané kružnice k~stranám $CA$ a~$AB$. Predpokladajme, že stred kružnice opísanej trojuholníku $A_1B_1C_1$
leží na kružnici opísanej trojuholníku $ABC$. Dokážte, že
trojuholník $ABC$ je pravouhlý.}
\podpis{Rusko}

{%%%%%   IMO, priklad 4
Nech $ABC$ je ostrouhlý trojuholník s~ortocentrom~$H$ a~nech $W$ je vnútorný bod strany~$BC$.
Body $M$ a~$N$ sú postupne päty výšok z~bodov $B$ a~$C$. Označme $\omega_1$ kružnicu opísanú trojuholníku~$BW\!N$ a~nech $X$ je taký bod na $\omega_1$, že $W\!X$ je jej priemerom. Analogicky označme $\omega_2$ kružnicu opísanú trojuholníku~$CW\!M$
a nech $Y$ je taký bod na $\omega_2$, že $WY$ je jej priemerom. Dokážte, že body $X$, $Y$ a~$H$ ležia na jednej priamke.}
\podpis{Thajsko}

{%%%%%   IMO, priklad 5
Nech $\Bbb Q^{\p}$ je množina kladných racionálnych čísel. Nech $f\colon\ \Bbb Q^{\p}\to\Bbb R$ je funkcia spĺňajúca nasledovné tri podmienky:
\itemitem{(i)}
pre všetky $x, y\in \Bbb Q^{\p}$ platí $f(x)f(y) \ge f(xy)$;
\itemitem{(ii)}
pre všetky $x, y\in \Bbb Q^{\p}$ platí $f(x+y)\ge f(x)+f(y)$;
\itemitem{(iii)}
existuje racionálne číslo $a>1$ také, že $f(a)=a$.
\endgraf\noindent
Dokážte, že $f(x)=x$ pre všetky $x\in \Bbb Q^{\p}$.}
\podpis{Bulharsko}

{%%%%%   IMO, priklad 6
Nech $n\ge3$ je celé číslo. Uvažujme kružnicu a~na nej $n+1$ rovnomerne rozložených bodov. Uvažujme všetky také označenia týchto bodov znakmi $0, 1, \dots, n$, že každý znak je použitý práve raz. Dve takéto označenia sa považujú za zhodné, ak jedno z~nich je možné dostať z~druhého rotáciou kružnice.
Označenie sa nazýva {\it krásne}, ak pre ľubovoľné štyri znaky $a<b<c<d$ také, že $a+d=b+c$, tetiva spájajúca body označené znakmi $a$ a~$d$ nepretína tetivu spájajúcu body označené znakmi $b$ a~$c$.
Nech $M$ je počet krásnych označení a~nech $N$ je počet usporiadaných dvojíc $(x,y)$ nesúdeliteľných kladných celých čísel takých, že $x+y \le n$. Dokážte, že
$$
M=N+1.
$$
}
\podpis{Rusko}

{%%%%%   MEMO, priklad 1
Nech $a$, $b$ a~$c$ sú kladné reálne čísla, pre ktoré platí
$$
a+b+c=\frac{1}{a^2}+\frac{1}{b^2}+\frac{1}{c^2}.
$$
Dokážte, že
$$
2(a+b+c) \ge \root3\of{7a^2b+1}+\root3\of{7b^2c+1}+\root3\of{7c^2a+1}.
$$
Nájdite všetky trojice $(a,b,c)$, pre ktoré nastáva rovnosť.}
\podpis{Slovensko, Patrik Bak}

{%%%%%   MEMO, priklad 2
Nech $n$ je kladné celé číslo. Na šachovnici pozostávajúcej z~$4n \times 4n$  políčok je rozmiestnených $4n$ žetónov. Každý riadok a~každý stĺpec obsahuje práve jeden žetón. Pri ťahu je žetón presunutý na stranou susediace políčko. Na políčkach môže byť aj viac ako jeden žetón. Cieľom je presunúť žetóny tak, že nakoniec budú umiestnené na všetkých políčkach jednej z~dvoch diagonál šachovnice. Určte najmenšie $k(n)$ také, že pre ľubovoľné počiatočné rozloženie žetónov vieme dosiahnuť výsledné rozloženie na najviac $k(n)$ ťahov.}
\podpis{Nemecko, Bernd Mulansky}

{%%%%%   MEMO, priklad 3
Je daný rovnoramenný trojuholník $ABC$ taký, že $|AC|=|BC|$. Nech $N$ je vnútorný bod trojuholníka $ABC$, pre ktorý platí
$$
2|\angle ANB| = 180^\circ +|\angle ACB|.
$$
Nech $D$ je priesečníkom priamky~$BN$ a~priamky rovnobežnej s~$AN$ prechádzajúcej bodom~$C$. Označme $P$ priesečník osí uhlov $CAN$ a~$ABN$. Dokážte, že priamky $DP$ a~$AN$ sú na seba kolmé.}
\podpis{Chorvátsko, Matija Basi\'c}

{%%%%%   MEMO, priklad 4
Nech $a$ a~$b$ sú kladné celé čísla. Dokážte, že existujú kladné celé čísla $x$ a~$y$ také, že
$$
\binom{x+y}{2} = ax + by.$$
}
\podpis{Maďarsko, Bálint Hujter}

{%%%%%   MEMO, priklad t1
Nájdite všetky funkcie $f\colon\Bbb R \to \Bbb R$ také, že pre všetky $x,y\in\Bbb R$ platí
$$
f(xf(x)+2y)=f(x^2)+f(y)+x+y-1.
$$
}
\podpis{Slovensko, Patrik Bak}

{%%%%%   MEMO, priklad t2
Nech $x,y,z,w\in\Bbb R \setminus \{0\}$ sú také, že $x +y \ne 0$, $z+w \ne 0$ a $xy+zw \ge 0$. Ukážte platnosť nerovnosti
$$
\left(\frac{x+y}{z+w}+\frac{z+w}{x+y}\right)^{-1} + \frac{1}{2} \ge \left(\frac{x}{z}+\frac{z}{x}\right)^{-1}+\left(\frac{y}{w}+\frac{w}{y}\right)^{-1}.
$$
}
\podpis{Švajčiarsko, Raphael Steiner}

{%%%%%   MEMO, priklad t3
Na jednej strane ulice sa nachádza $n \ge 2$ domov. Zo západu na východ sú označené číslami od $1$ po $n$. Číslo každého domu je napísané na ceduľke.
Jedného dňa sa obyvatelia ulice rozhodli vystreliť si z~poštára a~pomiešali ceduľky s~číslami domov nasledujúcim spôsobom: každej dvojici susedných domov vymenili počas dňa ceduľky s~ich aktuálnym číslom práve raz. Koľko rôznych usporiadaní ceduliek s~číslami môže na konci dňa nastať?}
\podpis{Maďarsko, Bálint Hujter}

{%%%%%   MEMO, priklad t4
Uvažujme konečne veľa bodov v~rovine takých, že žiadne tri neležia na jednej priamke. Každý z~týchto bodov ofarbíme červenou alebo zelenou farbou tak, že vo vnútri trojuholníka s~vrcholmi jednej farby sa nachádza aspoň jeden bod ofarbený druhou farbou. Aký je maximálny počet bodov s~touto vlastnosťou?}
\podpis{Maďarsko, Bálint Hujter}

{%%%%%   MEMO, priklad t5
Je daný ostrouhlý trojuholník $ABC$. Skonštruujte trojuholník $PQR$, pre ktorý platí $|AB|=2|PQ|$, $|BC|=2|QR|$, $|CA|=2|RP|$ a~priamky $PQ$, $QR$ a~$RP$ prechádzajú postupne bodmi $A$, $B$ a~$C$.% (Body $A$, $B$, $C$, $P$, $Q$ a~$R$ sú rôzne.)
}
\podpis{Rakúsko, Gerd Baron}

{%%%%%   MEMO, priklad t6
Nech $K$ je bod vnútri ostrouhlého trojuholníka $ABC$ taký, že $BC$ je spoločnou dotyčnicou kružníc opísaných trojuholníkom $AKB$ a~$AKC$. Nech $D$ je priesečník priamok $CK$ a~$AB$ a~bod~$E$ je priesečník priamok $BK$ a~$AC$. Označme $F$ priesečník priamky~$BC$ a~osi úsečky~$DE$. Kružnica opísaná trojuholníku $ABC$ a~kružnica~$k$ so stredom~$F$ a~polomerom~$FD$ sa pretínajú v~bodoch $P$ a~$Q$. Dokážte, že úsečka~$PQ$ je priemerom kružnice~$k$.}
\podpis{Slovensko, Patrik Bak}

{%%%%%   MEMO, priklad t7
Do tabuľky pozostávajúcej z~$2013\times2013$ políčok sú po riadkoch napísané čísla od $1$ do $2013^2$. Všetky stĺpce a~všetky riadky obsahujúce aspoň jednu z~druhých mocnín $1,4,9,\dots,2013^2$ naraz odstránime. Koľko políčok tabuľky ostane?}
\podpis{Rakúsko, Gerd Baron}

{%%%%%   MEMO, priklad t8
Na tabuli je napísaný výraz
$$
\pm\, \square \pm \square \pm \square \pm \square \pm \square\pm \square.
$$
Hráči $A$ a~$B$ sa striedajú pri nahrádzaní symbolov $\square$ kladnými celými číslami. Hráč~$A$ začína. Keď sú všetky symboly $\square$ nahradené, hráč~$A$ nahradí každý znak $\pm$ znamienkom $\p$ alebo $\m$, nezávisle na ostatných nahradeniach znakov $\pm$. Hráč~$A$ vyhrá, ak hodnota výrazu na tabuli nie je deliteľná žiadnym z~čísel $11,12,\dots,18$. Inak vyhrá hráč~$B$. Určte, ktorý z~hráčov má víťaznú stratégiu.}
\podpis{Česká republika, Michal Rolínek}

{%%%%%   CPSJ, priklad 1
Určte všetky dvojice celých čísel $(x,y)$ spĺňajúcich rovnosť
$$
\sqrt{x-\sqrt{y}}+\sqrt{x+\sqrt{y}}=\sqrt{xy}.
$$
}
\podpis{...}

{%%%%%   CPSJ, priklad 2
Každé kladné celé číslo chceme natrieť načerveno alebo nazeleno, pričom musia byť splnené nasledujúce podmienky:
\ite{$\bullet$} Nech $n$ je ľubovoľné červené číslo. Súčet ľubovoľných $n$ (nie nutne rôznych) červených čísel je červený.
\ite{$\bullet$} Nech $m$ je ľubovoľné zelené číslo. Súčet ľubovoľných $m$ (nie nutne rôznych) zelených čísel je zelený.\endgraf
\noindent
Určte všetky také ofarbenia.}
\podpis{...}

{%%%%%   CPSJ, priklad 3
Päťuholník $ABCDE$ je vpísaný do kružnice a~platí $|AB|=|BC|=|CD|$. Úsečky $AC$ a~$BE$ sa pretínajú v~bode~$K$, úsečky $AD$ a~$CE$ sa pretínajú v~bode~$L$. Dokážte, že $|AK|=|KL|$.}
\podpis{...}

{%%%%%   CPSJ, priklad 4
Nájdite najväčšie dvojciferné číslo $d$ s~nasledujúcou vlastnosťou: pre každé šesťciferné číslo $\overline{aabbcc}$ platí, že číslo $d$ je deliteľom čísla $\overline{aabbcc}$ práve vtedy, keď $d$ je deliteľom trojciferného čísla $\overline{abc}$.}
\podpis{...}

{%%%%%   CPSJ, priklad 5
Nech $M$ je stred strany~$AB$ v~ostrouhlom trojuholníku $ABC$.
Pre ľubovoľný bod~$P$ vnútri úsečky~$AB$ označíme postupne $S_1$ a~$S_2$ stredy kružníc opísaných trojuholníkom $APC$ a~$BPC$. Dokážte,
že stred úsečky $S_1S_2$ leží na osi úsečky~$CM$.}
\podpis{Ján Mazák}

{%%%%%   CPSJ, priklad t1
Rozhodnite, či existuje nekonečne veľa prvočísel $p$, ktoré majú
násobok v~tvare $n^2+n+1$ pre nejaké prirodzené číslo $n$.
}
\podpis{Ján Mazák}

{%%%%%   CPSJ, priklad t2
Nájdite všetky prirodzené čísla $n$ také, že súčet troch najväčších deliteľov čísla $n$ je $1457$.
}
\podpis{Pavel Novotný}

{%%%%%   CPSJ, priklad t3
W pewnej grupie jest $n \ge 5$ osób, przy czym każde dwie osoby, które si\ę{} nie znaj\ą{}, maj\ą{} dok\l{}adnie jednego wspólnego znajomego oraz żadna osoba nie zna wszystkich pozosta\l{}ych. Udowodnij, że można 5 spośród danych $n$ osób usadzić przy okr\ą{}g\l{}ym stole tak, aby każda z~nich siedzia\l{}a pomi\ę{}dzy swoimi
\ite a) znajomymi,
\ite b) nieznajomymi.
}
\podpis{...}

{%%%%%   CPSJ, priklad t4
W czworok\ą{}cie wypuk\l{}ym $ABCD$
$$
\uhol DAB = \uhol ABC = \uhol BCD > 90^\circ.
$$
Okr\ą{}g opisany na trójk\ą{}cie $ABC$ przecina boki $AD$ i $CD$ odpowiednio w punktach $K$ i $L$, różnych od wierzcho\l{}ków czworok\ą{}ta. Odcinki $AL$ i $CK$ przecinaj\ą{} si\ę{} w punkcie $P$. Udowodnij, że $\angle ADB = \angle PDC$.
}
\podpis{...}

{%%%%%   CPSJ, priklad t5
Nechť $a$, $b$, $c$ jsou kladná reálná čísla, pro něž platí $ab+ac+bc\ge a+b+c$. Dokažte, že
$$
a+b+c\ge3.
$$}
\podpis{...}

{%%%%%   CPSJ, priklad t6
V~rovině je dán čtverec $ABCD$, kde $|AB|=a$. Určete nejmenší možnou hodnotu poloměrů
tří shodných kruhů, pomocí nichž je možno pokrýt daný čtverec.}
\podpis{...}
