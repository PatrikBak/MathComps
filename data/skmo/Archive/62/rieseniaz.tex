{%%%%%   Z4-I-1
...}

{%%%%%   Z4-I-2
...}

{%%%%%   Z4-I-3
...}

{%%%%%   Z4-I-4
...}

{%%%%%   Z4-I-5
...}

{%%%%%   Z4-I-6
...}

{%%%%%   Z5-I-1
\napad
Koľko päťdesiateurových bankoviek môžeme použiť, aby bolo možné doplatiť
zvyšok dvadsaťeurovými?

\riesenie
Je zrejmé, že päťdesiateurových bankoviek musela mamička použiť menej ako 6,
pretože $6\cdot50=300$~(\euro).
Preberieme postupne všetky možnosti, \tj. že použila päť, štyri, atď. až
žiadnu päťdesiateurovú bankovku.
Potom zistíme, koľko euro by takto zaplatila a~koľko by jej ešte zvýšilo
doplatiť.
Nakoniec rozhodneme, či by zvyšnú sumu mohla doplatiť iba
dvadsaťeurovými bankovkami.
Celú diskusiu vyjadríme tabuľkou:
$$
\begintable
počet 50€ bankoviek\hfill\|5|4|3|2|1|0\cr
zaplatené\hfill\|250|200|150|100|\hfill50|0\cr
ostáva doplatiť\hfill\|\hfill20|\hfill70|120|170|220|270\cr
počet 20€ bankoviek\hfill\|1|---|6|---|11|---\endtable
$$
Mamička mohla zaplatiť tromi spôsobmi:
\begin{[itemize]}
\item 5 päťdesiateurových a~1 dvadsaťeurová bankovka,
\item 3 päťdesiateurové a~6 dvadsaťeurových bankoviek,
\item 1 päťdesiateurová a~11 dvadsaťeurových bankoviek.
\end{[itemize]}

\poznamka
Podobne sa dá postupovať vzhľadom na~dvadsaťeurové bankovky, ktorých musí
byť menej ako 14 ($14\cdot20=280$); zodpovedajúca tabuľka bude však v~takom
prípade podstatne dlhšia.
}

{%%%%%   Z5-I-2
\napad
Koľko čísel pripočíta Mat k~ľavej a~koľko k~pravej strane rovnosti?

\riesenie
Súčet štyroch čísel na ľavej strane rovnosti je $550+460+359+340=1\,709$;
to je o~$2\,012-1\,709=303$ menšie číslo ako na pravej strane rovnosti.
K~ľavej strane pripočítame Matovo číslo spolu štyrikrát, k~pravej len raz.
Rozdiel $303$ medzi oboma stranami musí teda byť vyrovnaný tromi Matovými číslami.
Hľadané číslo je preto $303:3=101$.

\poznamka
Pre lepšiu názornosť ešte uvádzame kontrolu predchádzajúceho výsledku:
Po pripočítaní na ľavej strane dostávame
$$
(550+101)+(460+101)+(359+101)+(340+101)=651+561+460+441=2\,113,
$$
na pravej strane vychádza
$$
2\,012+101=2\,113.
$$
Vidíme, že číslo 101 vyhovuje požiadavkám.
Súčasne sa ponúka vhodné znázornenie na rovnoramenných váhach.}

{%%%%%   Z5-I-3
\napad
Koľkokrát stlačil Rudo tlačidlo?

\riesenie
Najskôr zistíme, koľkokrát stlačil Rudo tlačidlo, ktorým sa osvetľuje
ciferník:
Tlačidlo držal zakaždým 4 sekundy a~celkové oneskorenie budíka nakoniec bolo 3 minúty, \tj. 180 sekúnd.
Rudo teda stlačil tlačidlo celkom 45-krát ($180:4=45$).

Každý deň stlačil tlačidlo ráno po zobudení.
Teraz zistíme, ktorý deň ho stlačil po prvý raz, a~to tak, že
spätne odpočítame 45~dní od 11.~decembra:
V~decembri (od 11. do 1.) je to 11~dní, november (od 30. do 1.) má 30~dní,
čo je spolu 41~dní.
Potrebujeme odpočítať ešte 4~dni v~októbri: 31., 30., 29., 28.

Rudo stlačil osvetľujúce tlačidlo prvýkrát 28.~októbra.
Budík dostal o~deň skôr, teda 27.~októbra.
}

{%%%%%   Z5-I-4
\napad
Ako môže vyzerať červík starý dva dni?

\riesenie
Prvý deň svojho života má červík iba jeden biely článok (a~hlavu):
\figure12

Druhý deň mu preto môže dorásť nový článok len prvým z~uvedených spôsobov;
všetky červíky staré dva dni teda vyzerajú rovnako ako na \obrplus\obr.
\figure13

Tretí deň sa môže rozdeliť buď prvý (biely) alebo druhý (sivý) článok,
sú teda dve možnosti ako na \obr.
\figure14

Štvrtý deň sa môže pri oboch typoch červíkov z~predošlého dňa rozdeliť
ktorýkoľvek z~jeho troch článkov, musíme teda preveriť celkom 6 možností znázornených na \obr.
\figure15

Porovnaním všetkých šiestich dospelých červíkov zisťujeme, že medzi nimi sú dve
dvojice rovnakých červíkov, zvyšné dva červíky sú odlišné od všetkých ostatných.
Existujú teda práve štyri rôzne farebné varianty dospelých červíkov tohto
zaujímavého druhu.
}

{%%%%%   Z5-I-5
\napad
Zistite, kde všade môžu byť umiestnené zátvorky.

\riesenie
Zadaný príklad bez akýchkoľvek zátvoriek vychádza
$$
3\cdot15+20:4+1=45+5+1=51.
$$
Teraz budeme vkladať zátvorky a~priebežne porovnávať výsledky.
Snažíme sa vyčerpať všetky možnosti, pritom budeme ignorovať zátvorky, ktoré
sú zbytočné, napr.
$$
[3\cdot(15+20)]:4+1=3\cdot(15+20):4+1
$$
a~pod.
Najskôr uvádzame možnosti s~jednou zátvorkou, následne s~dvoma.
Umiestniť zmysluplne tri či viacej zátvoriek sa už nedá.
\bgroup
\thinsize=0pt
\thicksize=0pt
$$
\begintable
$(3\cdot15+20):4+1=(45+20):4+1=65:4+1$\hfill|nedá sa deliť bezo zvyšku\hfill\cr
$3\cdot(15+20):4+1=3\cdot35:4+1=105:4+1$\hfill|nedá sa deliť bezo zvyšku\hfill\cr
$3\cdot(15+20:4)+1=3\cdot(15+5)+1=3\cdot20+1=61$\hfill|\cr
$3\cdot(15+20:4+1)=3\cdot(15+5+1)=3\cdot21=63$\hfill|\cr
$3\cdot15+20:(4+1)=45+20:5=45+4=49$\hfill|\cr
$3\cdot[(15+20):4+1]=3\cdot[35:4+1]$\hfill|nedá sa deliť bezo zvyšku\hfill\cr
$3\cdot[15+20:(4+1)]=3\cdot[15+4]=3\cdot19=57$\hfill|\hfill\cr
$(3\cdot15+20):(4+1)=(45+20):5=65:5=13$\hfill|\cr
$3\cdot(15+20):(4+1)=3\cdot35:5=105:5=21$\hfill|\hfill%
\endtable
$$
\egroup
Najväčšie číslo sme získali takto:
$$
3\cdot(15+20:4+1)=63,
$$
najmenšie číslo sme získali nasledovne:
$$
(3\cdot15+20):(4+1)=13.
$$

\poznamka
Deti budú zrejme vkladať zátvorky skúšaním.
Nemusia pritom overiť všetky možnosti; je pravdepodobné, že budú vynechávať tie,
kde sa nedá deliť bezo zvyšku,
a~že napr. pri hľadaní najväčšieho čísla vynechajú niektoré prípady s~tým, že
"výsledok by bol moc malý".
Také postupy uznávajte.

Za správne uznávajte aj zdôvodnenie, že na získanie čo najväčšieho celého čísla
chceme čo najmenšieho deliteľa a~súčasne čo najväčší výraz, ktorý násobíme
tromi; pre čo najmenšie celé číslo uvažujeme opačne.

Ak žiaka napadne vložiť začiatok alebo koniec zátvorky doprostred dvojciferného čísla, môže
získať najmenšie číslo takto:
$$
(3\cdot15+2)0:(4+1)=47\cdot0:5=0.
$$
Aj také riešenie uznávajte.
}

{%%%%%   Z5-I-6
\napad
Nakreslite si tvar záhrady, najlepšie na štvorčekový papier.

\riesenie
Nakreslíme celú záhradu do štvorčekovej siete, v~ktorej jeden štvorček
predstavuje jeden meter štvorcový (\obr).
\figure101

Teraz ľahko určíme plochu útvaru.
Najskôr spočítame všetky celé štvorčeky -- tých je vo vyznačenej ploche 21.
Ešte ostáva pripočítať dva trojuholníčky, ktoré dokopy tvoria jeden ďalší
celý štvorček.
Napočítali sme 22 štvorčekov, záhrada má teda $22\,\text{m}^2$.

\ineriesenie
Ak nemáme k~dispozícii štvorčekový papier, môžeme nákres záhrady rozdeliť
ako na \obr; spoločný bod pomocných úsečiek označíme $B$ a~podľa zadania doplníme veľkosti potrebných strán (všetko v~metroch).
\figure110

Odtiaľ ľahko určíme obsahy jednotlivých častí:
\begin{itemize}
  \item  Obdĺžnik "Plaško -- Smejko -- Kýchal -- Spachtoš" má obsah
    $3\cdot4=12\,(\text{m}^2)$.
  \item  Obdĺžnik "Kýblik -- Vedko -- Dudroš -- $B$" má obsah
    $4\cdot2=8\,(\text{m}^2)$.
  \item  Trojuholník "Kýblik -- Spachtoš -- $B$" môžeme chápať ako polovicu štvorca so stranou
    2\,m rozdeleného uhlopriečkou; jeho obsah je teda polovičný vzhľadom
    na pôvodný štvorec, \tj. $(2\cdot2):2=2\,(\text{m}^2)$.
\end{itemize}

Záhrada trpaslíkov má obsah $12+8+2=22\,(\text{m}^2)$.
}

{%%%%%   Z6-I-1
\napad
Zapíšte si čísla pod seba a~uvažujte ako pri písomnom sčítaní.

\riesenie
Cifry mysleného čísla označíme postupne $N_1$, $N_2$, $N_3$. Sčítanie uvedené v~zadaní znázorňuje nasledujúca schéma:
\algg{N_1&N_2&N_3\\4\ &2\ &1\ }{P_1&P_2&P_3}
Pre každé nepárne $N_3$ je súčet $N_3+1$ párny. Avšak pre každé nepárne $N_2$ je súčet $N_2+2$ nepárny. Aby sme napriek tomu dostali vo výsledku v~stĺpci desiatok párnu cifru, musí pri sčítaní v~stĺpci jednotiek nastať prechod cez desiatku. Teda $N_3$ môže byť jedine $9$.

Pre každé nepárne $N_1$ je súčet $N_1+4$ nepárny. Ak má byť napriek tomu vo výsledku v~stĺpci stoviek párna cifra, musí pri sčítaní v~stĺpci desiatok nastať prechod cez desiatku. Teda $N_2$ môže byť iba $9$ a~$7$.

Pri sčítaní na mieste stoviek prechod cez desiatku nastať nesmie, pretože výsledok má byť podľa zadania trojciferný. Teda $N_1$ môže byt iba $1$ alebo $3$.

Zoskupením prípustných cifier dostávame celkom štyri čísla, ktoré si Ľuboš mohol myslieť:
$$
179,\ 199,\ 379\text{ a }399.
$$
}

{%%%%%   Z6-I-2
\napad
Obsah každého trojuholníka sa dá vyjadriť pomocou obsahu (nanajvýš) dvoch
pravouhlých trojuholníkov.

\riesenie
Náhodným skúšaním mrežových bodov veľmi skoro zistíme, že potrebujeme
rozobrať tri kvalitatívne odlišné prípady, keď trojuholník $ABC$ je
a) pravouhlý s~pravým uhlom pri~vrchole $A$ alebo $B$, b) ostrouhlý alebo c) tupouhlý.

a)
Predpokladajme, že trojuholník $ABC$ je pravouhlý s~pravým uhlom pri~vrchole $A$ alebo $B$.
Budeme uvažovať prvú možnosť, riešenie v~druhom prípade bude súmerné.

Taký trojuholník tvorí práve polovicu pravouholníka, ktorého jedna strana je
$AB$ a~druhá $AC$.
Obsah tohto pravouholníka má byť rovný 9 štvorčekom.
Veľkosť $AB$ je 3 jednotky, bod $C$ teda musí byť $9:3=3$ jednotky od $A$.
Tento bod označíme $C_1$, súmerné riešenie vľavo od bodu~$B$ je označené
$C_2$ (\obr).
\figure21


b)
Predpokladajme teraz, že bod $C$ je nejaký mrežový bod v~páse medzi priamkami
$AC_1$ a~$BC_2$.
Trojuholník $ABC$ rozdelíme výškou na stranu~$AB$ na dva menšie pravouhlé
trojuholníky.
Každý z~týchto trojuholníkov je polovicou nejakého pravouholníka, poz. \obr.
Tieto dva pravouholníky tvoria väčší pravouholník, ktorého obsah je dvojnásobkom obsahu
trojuholníka $ABC$.
Jedna strana tohto pravouholníka je $AB$ a~druhá je zhodná s~výškou~$CP$.
Hľadáme teda taký bod~$C$, aby obsah takého pravouholníka bol 9 štvorčekov.
Skúšaním alebo úvahou ako vyššie odhalíme dve riešenia, ktoré sú vyznačené ako
$C_3$ a~$C_4$.
\figure22

c)
Nakoniec musíme rozobrať možnosti, keď bod~$C$ je nejaký mrežový bod
mimo pásu určeného priamkami $AC_1$ a~$BC_2$.
Trojuholník $ABC$ je v~tomto prípade tupouhlý a~výška na stranu~$AB$ ide
mimo neho; pätu tejto výšky označíme opäť~$P$.
Budeme uvažovať len trojuholníky s~tupým uhlom pri vrchole~$A$, zvyšné riešenia
sú súmerné.

Obsah trojuholníka $ABC$ je teraz rozdielom obsahov pravouhlých trojuholníkov
$CPB$ a~$CPA$.
Každý z~týchto trojuholníkov je polovicou vhodného pravouholníka (\obr).
Rozdiel obsahov týchto pravouholníkov je teda dvojnásobkom obsahu trojuholníka
$ABC$ a~je rovnaký ako obsah pravouholníka, ktorý je na \obrr1{} obtiahnutý neprerušovanou čiarou.
Jedna strana tohto pravouholníka je $AB$ a~druhá je zhodná s~výškou~$CP$.
Hľadáme teda taký bod~$C$, aby obsah takého pravouholníka bol 9 štvorčekov.
Skúšaním alebo úvahou ako vyššie odhalíme dve riešenia, ktoré sú označené ako
$C_5$ a~$C_6$.
Súmerné riešenia nad priamkou $BC_2$ sú $C_7$ a~$C_8$.
\figure23

Úloha má spolu 8 riešení, ktoré sme postupne označili $C_1$, \dots,
$C_8$.

\poznamky
Predstava so súčtovým pravouholníkom v~odseku~b),
resp. rozdielovým pravouholníkom v~odseku~c),
nie je nutná na~úspešné doriešenie úlohy.
Stačí, keď si riešiteľ uvedomí, že obsah trojuholníka $ABC$ je súčtom, resp.
rozdielom, obsahov pravouhlých trojuholníkov $CPB$ a~$CPA$.
Zdôvodnenie, že napr. vyznačený bod $C_5$ je riešením, potom môže vyzerať nasledovne:
$$
S_{ABC_5}=\frac{3\cdot 5}2-\frac{3\cdot 2}2=\frac92.
$$

Všimnime si, že pre ľubovoľný bod~$C$ na priamke~$C_1C_2$ vyjde podľa
predošlých úvah obsah trojuholníka $ABC$ tiež 4{,}5 štvorčeka.
Vlastne sme tak dokázali, že obsah ľubovoľného trojuholníka je rovný
polovici pravouholníka, ktorý má jednu stranu spoločnú s~trojuholníkom a~druhú zhodnú s~výškou na túto stranu.

Pokiaľ žiak tento fakt už pozná, je riešenie obzvlášť jednoduché:
stačí nájsť jeden vyhovujúci bod a~všetky ostatné sú mrežové body ležiace
na rovnobežke s~priamkou~$AB$, ktorá prechádza týmto vyhovujúcim bodom.
}

{%%%%%   Z6-I-3
\napad
Pomôžte si prehľadnou tabuľkou, z~ktorej bude jasné, kedy ktorý obor klame a~kedy nie.

\riesenie
Informácie zo zadania kvôli prehľadnosti vpíšeme do tabuľky:
$$
\begintable
\|Koloman|Bartolomej\crthick
pondelok\|$+$|$-$\cr
utorok\|$-$|$-$\cr
streda\|$-$|$+$\cr
štvrtok\|$-$|$+$\cr
piatok\|$+$|$+$\cr
sobota\|$-$|$-$\cr
nedeľa\|$+$|$-$%
\endtable
$$

1.
Koloman určite môže povedať "včera som hovoril pravdu" v~pondelok, pretože v~pondelok hovorí pravdu a~v~nedeľu (včera) tiež.
Iná dvojica po sebe idúcich dní, počas ktorých hovorí pravdu, nie je.
Ak by Koloman povedal zadaný výrok v~deň, keď práve klame,
znamenalo by to, že predošlý deň v~skutočnosti klamal.
To by sa mohlo stať jedine v~stredu alebo vo štvrtok.
Koloman teda môže tvrdiť "včera som hovoril pravdu"
v~pondelok, v~stredu a~vo štvrtok.

2.
Podobne, keď niektorý z~obrov povie "včera som klamal", môže to byť buď
v~deň, keď hovorí pravdu a~súčasne predošlý deň klamal, alebo naopak.
Koloman môže tento výrok prehlásiť v~utorok, v~piatok, v~sobotu a v~nedeľu;
Bartolomej v~stredu a~v~sobotu.
Jediný deň, keď môžu obaja tvrdiť "včera som klamal", je sobota.}

{%%%%%   Z6-I-4
\napad
Všimnite si, že $192=48\cdot4$.

\riesenie
Predpokladajme, že výsledok $48$ dostaneme, keď vynásobíme čísla z~prvého a~druhého
lístočka, výsledok $192$ dostaneme z~prvého a~tretieho lístočka a~výsledok~$36$ z~druhého a~tretieho lístočka.

Keďže $192=48\cdot4$, číslo na treťom lístočku musí byť štvornásobkom čísla
na druhom lístočku.
Na druhom lístočku je teda také číslo, že keď ho vynásobíme s~jeho
štvornásobkom, dostaneme $36$.
To znamená, že keby sme toto číslo vynásobili so sebou samým, dostali by sme
štvrtinu predchádzajúceho výsledku, \tj. $9$.
Na druhom lístočku je teda číslo $3$.
Na treťom lístočku potom musí byť $3\cdot4=12$ a~na prvom $48:3=16$.
Na Eviných lístočkoch sú napísané tieto čísla: $16$, $3$ a~$12$.

\inynapad
Každé prirodzené číslo sa dá napísať ako súčin dvoch prirodzených čísel iba
konečne veľa spôsobmi.

\ineriesenie
Rovnako ako pri predošlom riešení predpokladáme, že
výsledok $48$ dostaneme vynásobením čísel z~prvého a~druhého lístočka, výsledok $192$ dostaneme z~prvého a~tretieho lístočka a~výsledok $36$ z~druhého a~tretieho lístočka.

Číslo $36$ môžeme napísať ako súčin dvoch prirodzených čísel iba piatimi
spôsobmi:
$$
36=1\cdot36=2\cdot18=3\cdot12=4\cdot9=6\cdot6.
$$
Pritom zo zadania vieme, že na treťom lístočku musí byť väčšie číslo ako na
druhom lístočku (súčin čísel z~prvého a~tretieho lístočka je väčší
ako súčin čísel z~prvého a~druhého).
Odtiaľ máme iba štyri možné dvojice čísel na druhom a~treťom lístočku.
Zo známych hodnôt súčinov čísel na lístočkoch dopočítame dvojakým spôsobom
číslo na prvom lístočku, a~ak sa tieto výsledky budú zhodovať, nájdeme riešenie.
$$
\begintable
2.~číslo|3.~číslo\|\multispan{2}\hfil1.~číslo\hfil\nr
($x$)|($y$)\|($48:x$)|($192:y$)\crthick
1|36\|48|$-$\cr
2|18\|24|$-$\cr
3|12\|16|16\cr
4|9\|12|$-$%
\endtable
$$
(Pomlčka označuje, že čísla nemožno deliť bezo zvyšku, \tj. výsledok nie je prirodzené číslo.)
Vidíme jedinú vyhovujúcu možnosť:
na Eviných lístočkoch sú napísané čísla $16$, $3$ a~$12$.}

{%%%%%   Z6-I-5
\napad
Ako sa zmení povrch telesa po odstránení tmavej kocky?

\riesenie
Pôvodne sú na povrchu zostaveného telesa tri steny tmavej kocky, ostatné
tri jej steny sa dotýkajú svetlých kociek.
Po odstránení tmavej kocky sa celkový povrch telesa nezmení
(tri tmavé steny sa nahradia tromi svetlými).

Aby sa povrch telesa nezmenil ani po preložení tmavej kocky, musí sa táto
dotýkať práve troch svetlých kociek
(tri svetlé steny budú nahradené tromi tmavými).
Skúšaním rýchlo zistíme, že tmavú kocku môžeme premiestniť jedine na
tri miesta znázornené na \obr.
\figure31
}

{%%%%%   Z6-I-6
\napad
Určte, aký by bol ich celkový účet, keby aj druhýkrát zaplatil každý
zvlášť.

\riesenie
Keby v~deň, keď priatelia odhalili čašníkov podvod, platil každý zvlášť, vypýtal by si
čašník od každého $1{,}68+0{,}04=1{,}72$~(\euro).
Celkom by tak zaplatili $3\cdot 1{,}72=5{,}16$~(\euro).
Tým, že platil jeden za všetkých, sa cena znížila o~sumu zodpovedajúcu dvojnásobku dátumu.
Táto cena je podľa zadania 4{,}86€.
Dvojnásobku dátumu teda zodpovedá suma $5{,}16-4{,}86=0{,}30$~(\euro).
Číslo v~dátume je $30:2=15$;
priatelia podvod odhalili 15.~septembra.

\poznamka
Svoj výsledok si môžeme overiť záverečným zhrnutím:
Prvýkrát boli priatelia v~bufete 11.~septembra.
Každý mal podľa jedálneho lístka platiť 1{,}57~€, platil však $1{,}57+0{,}11=1{,}68$~(\euro).
Druhýkrát sa v~bufete zišli 15.~septembra.
Jeden za všetkých mal podľa jedálneho lístka zaplatiť $3\cdot 1{,}57=4{,}71$~(\euro),
čašník si však vypýtal $4{,}71+0{,}15=4{,}86$~(\euro).
}

{%%%%%   Z7-I-1
\napad
Všímajte si dvojice trojuholníkov so spoločnou stranou.


\riesenie
Označme čísla v~krúžkoch ako na \obr.
Budeme si postupne všímať dvojice trojuholníkov, ktoré majú spoločnú stranu.
\figure40

Trojuholníky so spoločnou stranou~$DC$:
Keďže súčet čísel v~trojuholníku $DCB$ je $19$ a~súčet čísel
v~trojuholníku $DCF$ je $11$, musí byť číslo~$B$ o~$8$ väčšie ako číslo~$F$.
V~krúžkoch môžu byť len prirodzené čísla od $1$ po $9$, máme teda jedinú
možnosť: $B=9$ a~$F=1$.

Keď porovnáme trojuholníky so spoločnou stranou~$CB$, zistíme, že číslo~$A$
je o~$3$~väčšie ako číslo~$D$.
Túto dvojicu nevieme zatiaľ určiť jednoznačne, preto píšeme $A=D+3$.

Podobne porovnaním trojuholníkov so spoločnou stranou~$BD$ určíme, že $C=E+5$ (\obr).
\figure41

Z~trojuholníka so súčtom čísel $19$ teraz vidíme, že $9+E+5+D=19$, \tj. $E+D=5$.
Keďže číslo $1$ sme už použili, musí byť $E=2$ a~$D=3$, alebo naopak.
Každá z~dvoch uvedených možností vedie na jedno riešenie, ktoré vidíme na \obr.
(V~oboch prípadoch sú čísla v~krúžkoch navzájom rôzne.)
\figure42
}

{%%%%%   Z7-I-2
\napad
Mohlo by byť na záhone napr. 100 kvetov?

\res
Celkový počet kvetov musí byť deliteľný piatimi, pretože jednu pätinu tvoria
tulipány.
Z~podobného dôvodu musí byť počet všetkých kvetov deliteľný deviatimi (kvôli
narcisom) a~pätnástimi (kvôli hyacintom).
Počet všetkých kvetov na záhone je teda nejaký spoločný násobok čísel $5$, $9$ a~$15$.
Najmenší spoločný násobok týchto troch čísel je $45$, teda celkový počet kvetov
je násobkom čísla~$45$.

% \goodbreak
V~nasledujúcej tabuľke prechádzame postupne násobky $45$
a~určujeme zodpovedajúce počty jednotlivých druhov kvetov.
Hľadáme taký riadok, kde sú všetky tieto počty v~rozsahu od 30 do 60
(vrátane).

\noindent$$
\begintable
%\para{celkem\\$c$}\|\para{tulipány\\$t=\frac15c$}|\para{narcisy\\}|\para{hyacinty\\}|\para{macešky\\}\crthick
celkom\|tulipány|narcisy|hyacinty|sirôtky\nr
($c$)\|($t=\frac15c$)|($n=\frac29c$)|($h=\frac4{15}c$)|($s=c-t-n-h$)\crthick
\ 45\|\ 9|10|12|14\cr
\ 90\|18|20|24|28\cr
135\|27|30|36|42\cr
\bf180\|\bf36|\bf40|\bf48|\bf56\cr
225\|45|50|60|70%
\endtable
$$
Jediný vyhovujúci prípad je zvýraznený vo štvrtom riadku.
So zväčšujúcim sa celkovým počtom kvetov sa zväčšujú aj počty kvetov jednotlivých
druhov, takže ďalšie (neuvedené) možnosti zrejme vyššie uvedeným požiadavkám
nevyhovujú.
Na záhone pred školou máme celkom 180 kvetov.

\poznamka
Akonáhle v~riadku nájdeme číslo menšie ako 30 alebo väčšie ako 60, nemusíme
ostatné položky dopočítavať -- táto možnosť totiž určite nevyhovuje.
Predošlá diskusia teda môže byť úplná, aj keď tabuľka je neúplná.

\ineriesenie
Najskôr dopočítame, akú časť medzi všetkými kvetmi tvoria sirôtky:
$$
1-\frac15-\frac29-\frac4{15}=\frac{14}{45}.
$$
Aby bol počet sirôtok prirodzeným číslom, musí byť počet všetkých kvetov
násobkom čísla~$45$.
V~takom prípade sú aj počty ostatných kvetov prirodzené čísla.

Počty všetkých jednotlivých druhov sú v~rozsahu od 30 do 60 (vrátane), odkiaľ
vieme určiť rozsah celkového počtu kvetov:

Tulipány tvoria  $\frac15$ celkového počtu kvetov,
takže počet všetkých kvetov musí byť
$$
\text{od\ }30\cdot5=150\text{\ do\ }60\cdot5=300,
$$
podľa pomerného zastúpenia narcisov musí byť počet všetkých kvetov
$$
\text{od\ }30\cdot\frac92=135\text{\ do\ }60\cdot\frac92=270,
$$
podľa hyacintov musí byť počet všetkých kvetov
$$
\text{od\ }30\cdot\frac{15}4=112,5\text{\ do\ }60\cdot\frac{15}4=225
$$
a~podľa sirôtok
$$
\text{od\ }30\cdot\frac{45}{14}\doteq96{,}4\text{\ do\
}60\cdot\frac{45}{14}\doteq192{,}9.
$$
Predchádzajúce štyri podmienky majú platiť súčasne, teda celkový počet
kvetov sa pohybuje v~rozsahu od 150 do 192 (vrátane).
Medzi týmito číslami je jediným násobkom 45 číslo 180.
Na záhone rastie dokopy 180 kvetov.
}

{%%%%%   Z7-I-3
\napad
Pomôžte si prehľadnou tabuľkou, z ktorej by bolo jasné, kedy ktorý obor klame a~kedy nie.

\res
Informácie zo zadania kvôli prehľadnosti vpíšeme do tabuľky:
$$
\begintable
\|Bobr|Koloděj\crthick
pondelok\|$-$|$+$\cr
utorok\|$-$|$-$\cr
streda\|$-$|$-$\cr
štvrtok\|$-$|$-$\cr
piatok\|$-$|$+$\cr
sobota\|$+$|$-$\cr
nedeľa\|$+$|$+$%
\endtable
$$

Bartolomej môže tvrdiť "včera sme obaja klamali"
buď v~deň, keď hovorí pravdu a~súčasne predošlý deň obaja obri klamali (čo sa
stať nemôže),
alebo v~deň, keď klame a~súčasne predošlý deň aspoň jeden z~obrov hovoril
pravdu (\tj. v~pondelok alebo v~utorok).
Bartolomej teda môže povedať "včera sme obaja klamali" jedine v~pondelok a~v~utorok.

Koloman môže tvrdiť "aspoň jeden z~nás hovoril včera pravdu"
buď v~deň, keď hovorí pravdu a~súčasne predošlý deň aspoň jeden z~obrov
hovoril pravdu (\tj. v~pondelok alebo v~nedeľu),
alebo v~deň, keď klame a~súčasne predošlý deň žiadny z~nich pravdu nehovoril
(\tj. v~stredu alebo vo štvrtok).
Koloman teda môže povedať "aspoň jeden z~nás hovoril včera pravdu"
jedine v~pondelok, v stredu, vo štvrtok a~v~nedeľu.

Teda jediný deň, keď môžu obri viesť uvedený rozhovor, je pondelok.}

{%%%%%   Z7-I-4
\napad
Môže byť medzi doplnenými číslami napr. $5$?

\res
Jeden zo spôsobov, ako čísla doplniť, je samozrejme ten, ktorý pani učiteľka
zotrela (v~tomto prípade je rozdiel susedných čísel rovný~$3$).
Ďalší možný, zrejme najjednoduchší, spôsob je doplniť všetky prirodzené
čísla od $1$ do $43$ (v~tomto prípade je rozdiel rovný~$1$).

Každé doplnenie podľa zadania je jednoznačne určené rozdielom susedných čísel, ktorý
označíme~$d$.
Všetky možné doplnenia možno teda určiť skúšaním všetkých možných
rozdielov~$d$ a~kontrolou, či v~zodpovedajúcej postupnosti (začínajúcej číslom~$1$) sú
obsiahnuté aj čísla $19$ a~$43$.

Avšak aby v~takej postupnosti bolo číslo $19$,
musí byť rozdiel $19-1=18$ nejakým násobkom určujúcej konštanty~$d$.
Podobne, aby v~takej postupnosti bolo aj číslo $43$,
musí byť rozdiel $43-19=24$ nejakým násobkom čísla~$d$.
Inými slovami, $d$ musí byť spoločným deliteľom čísel $18$ a~$24$.
Všetky možné doplnenia teda zodpovedajú všetkým spoločným deliteľom čísel $18$ a~$24$,
čo sú práve čísla $1$, $2$, $3$ a~$6$.
Pani učiteľka mohla doplniť čísla na tabuľu štyrmi spôsobmi.

\poznamka
Predchádzajúce úvahy môžu byť vhodne podporené predstavou na číselnej osi;
na \obr{} je zvýraznené delenie s~najväčším možným rozdielom $d=6$.
\figure102
}

{%%%%%   Z7-I-5
\napad
Je nejaký vzťah medzi dĺžkou uhlopriečky a~dĺžkou kratšej strany obdĺžnika?

\riesenie
Priesečník uhlopriečok označíme~$S$.
Musíme rozhodnúť, či sa zadaná veľkosť $60\st$ vzťahuje na~uhol $ASB$ alebo $ASD$.
Keďže pre strany obdĺžnika platí $|AB|>|AD|$, musí byť uhol $ASB$
tupouhlý a~uhol $ASD$ ostrouhlý.
Veľkosť $60\st$ teda prislúcha uhlu $ASD$ (\obr).
\figure150

V~každom obdĺžniku sú trojuholníky $ASD$ a~$BSC$ rovnoramenné a~navzájom
zhodné, v~našom prípade sú dokonca rovnostranné.
To znamená, že úsečky $AS$, $SC$, $CB$, $BS$, $SD$ a~$DA$ sú zhodné,
ich dĺžku (v~metroch) označíme $s$.
Chceme určiť dĺžku uhlopriečky, ktorá pri súčasnom označení zodpovedá hodnote
$2s$.

Dĺžka veľkého okruhu je teda $6s$ a~celková vzdialenosť, ktorú ubehol Mojmír,
je ${10\cdot 6s}=60s$.
Dĺžka malej dráhy je $2s$ a~celková vzdialenosť, ktorú ubehol Vojtech, je
$15\cdot 2s=30s$.
Obaja dokopy tak ubehli $60s+30s=90s$, čo je podľa zadania rovné
$4{,}5\,\text{km}=4\,500\,\text{m}$.
Platí preto
$$
90s=4\,500,
$$
odkiaľ vyplýva $2s=100$.
Dĺžka uhlopriečky je 100\,m.}

{%%%%%   Z7-I-6
\napad
Skúšajte najskôr také body, aby niektorá strana trojuholníka ležala
na nejakej priamke tvoriacej štvorčekovú sieť.

\res
Ak hľadáme riešenie skúšaním, zväčša začneme skúšať mrežové body tak,
ako naznačuje pomôcka.
Uvažujme najskôr mrežové body na vodorovnej priamke prechádzajúcej bodom~$A$; polohu
bodu~$C$ počítame od $A$ doľava:
\begin{enumerate}
\item trojuholník je pravouhlý a~jeho obsah je zrejme 2 štvorčeky, čo je
málo,
\item výška z~bodu~$B$ rozdeľuje trojuholník na dva (zhodné) pravouhlé
trojuholníky, obsah je $2+2=4$ štvorčeky, čo je stále málo,
\item výška z~bodu~$B$ rozdeľuje trojuholník na dva (nezhodné) pravouhlé
trojuholníky, obsah je $4+2=6$ štvorčekov a~máme prvé vyhovujúce
riešenie, ktoré označíme $C_1$.
\end{enumerate}
Tento bod možno nájsť aj bez skúšania, ak si včas uvedomíme, že obsah
každého uvažovaného trojuholníka je rovný polovici obsahu pravouholníka,
ktorého jedna strana je $AC$ a~druhá je rovná 4 jednotkám (veľkosť výšky z~bodu~$B$, poz. \obr).
Hľadáme preto taký mrežový bod na myslenej priamke, aby obsah zodpovedajúceho
pravouholníka bol rovný 12 štvorčekom.
Bod~$C$ tak musí byť $12:4=3$ jednotky od $A$, a~aby sme neopustili
vyznačenú oblasť, musíme smerovať doľava.
\figure51

Veľmi podobným spôsobom možno zdôvodniť aj ďalšie riešenie na vodorovnej priamke
prechádzajúcej bodom~$B$, ktoré označíme $C_2$.
(Súmerný bod podľa $B$ opäť vychádza mimo vyznačenú oblasť.)
Priamka~$C_1C_2$ je rovnobežná s~$AB$, takže každý trojuholník $ABC$, ktorého
vrchol~$C$ leží na tejto priamke, má tú istú výšku na stranu $AB$, teda aj ten istý
obsah.
Hľadáme teda také mrežové body, ktoré súčasne ležia na priamke~$C_1C_2$.
Takto nájdeme bod, ktorý je označený~$C_3$ (\obr).
\figure52

Analogickou úvahou v~opačnej polrovine ohraničenej priamkou~$AB$ zistíme, že
zvyšné riešenia sú práve tie mrežové body, ktoré súčasne ležia na vyznačenej
priamke na \obr.
Takto nachádzame posledný vyhovujúci bod, ktorý je označený~$C_4$.
\figure53

Úloha má spolu 4 riešenia, ktoré sme postupne označili $C_1$, $C_2$, $C_3$ a~$C_4$.

\poznamky
V~uvedenom riešení predpokladáme znalosť faktu, že obsah ľubovoľného
trojuholníka je rovný polovici pravouholníka, ktorý má jednu stranu
spoločnú s~trojuholníkom a~druhú zhodnú s~výškou na túto stranu.
Jednoduché zdôvodnenie tohto tvrdenia je v~podstate ukázané v~riešení úlohy
Z6--I--2.

Aj bez tohto poznatku sa dá úloha doriešiť skúšaním, ako sme naznačili v~úvode.
Obsah ľubovoľného trojuholníka $ABC$ s~vrcholmi v~mrežových bodoch sa dá vždy
vyjadriť nasledovne:
\begin{itemize}
\item trojuholníku $ABC$ opíšeme pravouholník, ktorého strany ležia na
priamkach tvoriacich štvorčekovú sieť,
\item ak je to nutné, rozdelíme doplnkové plochy k~trojuholníku v~opísanom pravouholníku na pravouhlé trojuholníky, príp. pravouholníky,
\item obsah trojuholníka vyjadríme ako rozdiel obsahu opísaného
pravouholníka a~obsahov jednotlivých doplnkových častí z~predchádzajúceho
kroku.
\end{itemize}
Výpočet obsahov niektorých trojuholníkov by podľa tohto návodu vyzeral
nasledovne (poz. predošlé obrázky):
$$
\aligned
S_{ABC_1}&=3\cdot4-\frac{2\cdot4}2-\frac{1\cdot4}2=12-6=6,\\
S_{ABC_2}&=4\cdot4-\frac{4\cdot4}2-\frac{1\cdot4}2=16-10=6,\\
S_{ABC_3}&=5\cdot8-\frac{5\cdot8}2-\frac{4\cdot4}2-1\cdot4-\frac{1\cdot4}2=40-34=6,\\
S_{ABC_4}&=2\cdot8-\frac{1\cdot4}2-\frac{2\cdot4}2-\frac{1\cdot8}2=16-10=6.
\endaligned
$$
Všimnite si, že ani pri tomto postupe nie je nutné vyčerpávať všetky
možnosti:
ak máme napr. zistené, že bod~$C_2$ je riešením, určite už nemusíme uvažovať
také mrežové body,
keď by zodpovedajúci trojuholník buď obsahoval trojuholník $ABC_2$ alebo bol jeho časťou
(v~prvom prípade by vzniknutý trojuholník mal väčší ako požadovaný obsah, v~druhom prípade menší).}

{%%%%%   Z8-I-1
\napad
Z~každej oznamovacej vety v~zadaní možno priamo určiť jedného činiteľa.

\riesenie
Pracujme najskôr s~druhou vetou:
zmenšením jedného činiteľa o~$10$ sa zmenší súčin o~$400$.
Pritom $400$ sú dve tretiny zo~$600$, teda číslo $10$ musí byť dvoma tretinami zo zmenšovaného činiteľa.
Tým je preto číslo $15$.

Ďalej pracujme s~treťou vetou:
zväčšením jedného činiteľa o~$5$ sa zväčší súčin na dvojnásobok.
Zväčšením o~$5$ sa teda tento činiteľ zväčší tiež na dvojnásobok.
Činiteľ je preto $5$.

Z~prvej vety zadania vieme, že súčin všetkých činiteľov je $600$, dva z~nich uvádzame vyššie, tretí je
$600 : 15 : 5 = 8$.

Informáciám zo zadania vyhovujú čísla $5$, $8$ a~$15$.

\inynapad
Zadanie vedie na tri rovnice o~troch neznámych.

\ineriesenie
Keďže zo zadania nie je jasné, či zmenšujeme/zväčšujeme vždy toho istého
činiteľa alebo zakaždým iného, musíme prebrať obidve možnosti.
V~každom prípade hľadané prirodzené čísla označíme $x$, $y$ a~$z$.

1.
Predpokladajme, že sa v~zadaní hovorí o~dvoch rôznych činiteľoch.
V~takom prípade môžeme informácie zo zadania prepísať nasledovne:
$$
\aligned
xyz&=600,\\
(x-10)yz&=200,\\
x(y+5)z&=1\,200.
\endaligned
$$
Druhá rovnosť po roznásobení je
$xyz-10yz=200$.
Keďže $xyz=600$, po úprave dostávame
$10yz=400$, \tj. $yz=40$.
Z~rovnosti $xyz=600$ teraz vyplýva $x\cdot40=600$, \tj.
$$
x=15.
$$
Podobne, tretia rovnosť po roznásobení je
$xyz+5xz=1\,200$.
Keďže $xyz=600$, po úprave dostávame
$5xz=600$, \tj. $xz=120$.
Keďže už poznáme $x=15$, musí byť
$$
z=8.
$$
Dosadením opäť do rovnosti $xyz=600$ máme $120y=600$, odkiaľ vyplýva
$$
y=5.
$$

2.
Predpokladajme, že sa v~zadaní hovorí dvakrát o~rovnakom činiteľovi.
V~takom prípade môžeme písať:
$$
\aligned
xyz&=600,\\
(x-10)yz&=200,\\
(x+5)yz&=1\,200.
\endaligned
$$
Rovnako ako vyššie roznásobíme druhú, resp. tretiu rovnosť,
dosadíme $xyz=600$ a~po úprave dostaneme
$10yz=400$, \tj. $yz=40$,
resp.
$5yz=600$, \tj. $yz=120$.
Keďže $40\ne120$, vidíme, že východiskový predpoklad nemôže byť splnený.

V~zadaní sa hovorí o~dvoch rôznych činiteľoch;
uvažovanú vlastnosť majú práve tieto tri prirodzené čísla: $5$, $8$ a~$15$.}

{%%%%%   Z8-I-2
\napad
Najskôr zistite, koľko ktorých kociek má Stano pred samotným skladaním
k~dispozícii.

\riesenie
Potrebujeme určiť, z~akých kociek Stano skladal svoju veľkú kocku.
Rohové kocky z~rozložených obielených útvarov
majú jednu dvojicu susedných stien sivú,
zvyšok biely; celkom ich je $7\cdot4=28$.
Ostatné kocky z~týchto útvarov
majú jednu dvojicu protiľahlých stien sivú, zvyšok biely;
celkom ich je tiež $7\cdot4=28$.
K~týmto kockám sa ešte pridalo 8 celých bielych.

Stano mal k~dispozícii celkom 64 kociek, skladal teda
veľkú kocku s~"hranou 4 malé kocky" ($4\cdot4\cdot4=64$).
Teraz určíme počty kociek vo veľkej kocke podľa počtu viditeľných stien:
Rohové kocky majú viditeľné tri steny; tých je celkom 8.
Na hranách sú kocky s~dvoma viditeľnými stenami; celkom ich
je $12\cdot2=24$.
V~stenách sú kocky s~jednou viditeľnou stenou; celkom ich
je $6\cdot4=24$.
Vnútri veľkej kocky je 8~kociek, ktoré nevidno vôbec.

Z~rozrezaných útvarov Stano nezískal žiadne kocky, ktoré by mali
3 sivé steny.
Zrejme teda celú sivú veľkú kocku zostaviť nemohol, napriek tomu, že celková sivá
plocha na kockách je väčšia než povrch veľkej kocky.
My samozrejme nevieme, ako Stano kocku skladal, nič menej, aby čo najväčšia
časť povrchu bola sivá, mohol postupovať napr. takto:
\begin{itemize}
  \item všetkých 8 čisto bielych kociek umiestni doprostred,
  \item 24 kociek so susednými sivými stenami použije na hrany veľkej
    kocky,
  \item zvyšné 4 kocky so susednými sivými stenami umiestni do niektorých
    vrcholov (týmto dostane na povrchu veľkej kocky 4 biele plôšky),
  \item 24 kociek s~protiľahlými sivými stenami použije do stien veľkej
    kocky,
  \item zvyšné 4 kocky s~protiľahlými sivými stenami umiestni do zvyšných
    vrcholov (na povrchu pribudne 8 bielych plôšok).
\end{itemize}
\noindent
Pri tomto postupe by na povrchu veľkej kocky bolo 12 bielych plôšok, \tj.
$12\cm^2$.

Aby bolo zrejmé, že lepšie už kocku zložiť nemožno, všimneme si nasledujúce
skutočnosti:
Žiadnu z~kociek, ktoré majú protiľahlé steny sivé, nikdy nemôžeme vo veľkej
kocke umiestniť tak, aby obe sivé steny bolo vidno -- použitím všetkých týchto
kociek možno teda obsiahnuť nanajvýš $28\cm^2$ sivej plochy na povrchu veľkej
kocky.
Naopak, kocky, ktoré majú susedné steny sivé, sa dajú umiestniť tak, aby obe
sivé steny bolo vidno -- použitím všetkých týchto kociek možno obsiahnuť nanajvýš
$56\cm^2$ sivej plochy na povrchu veľkej kocky.
Na povrchu veľkej kocky môže byť nanajvýš $28+56=84\,(\text{cm}^2)$ sivej plochy.
Povrch celej kocky je $6\cdot4\cdot4=96\,(\text{cm}^2)$; na jej povrchu teda
nemôže byť menej ako $96-84=12\,(\text{cm}^2)$ bielych.

Predchádzajúci postup ukazuje jednu z~možností, ako tento výsledok realizovať.
Nech už Stano postupoval akokoľvek, $12\cm^2$ povrchu kocky bude určite
bielych.
}

{%%%%%   Z8-I-3
\napad
Začnite prostredným dvojčíslím, potom uvažujte ostatné
podmienky zo zadania.

\riesenie
Prostredné dve miesta môžu byť obsadené práve štyrmi spôsobmi:
$$
*44*,\ *77*,\ *47*,\ *74*.
$$
Číslo je deliteľné $15$ práve vtedy, keď je deliteľné tromi a~zároveň piatimi.
Pritom číslo je deliteľné piatimi práve vtedy, keď jeho posledná cifra je buď
$0$ alebo $5$,
a~číslo je deliteľné tromi práve vtedy, keď jeho ciferný súčet je deliteľný
tromi.

Ak je posledná cifra $0$, tak uvažujeme nasledujúce možnosti
$$
*440,\ *770,\ *470,\ *740
$$
a~hľadáme prvú cifru tak, aby ciferný súčet bol deliteľný tromi.
Vo všetkých prípadoch vychádza tá istá možná trojica: $1$, $4$ alebo $7$.

Ak je posledná cifra $5$, tak uvažujeme podobne nasledujúce možnosti
$$
*445,\ *775,\ *475,\ *745.
$$
Vo všetkých prípadoch vychádza tá istá možná trojica: $2$, $5$ alebo $8$.

Podmienkam zo zadania vyhovuje $4\cdot2\cdot3=24$ možností.
Na prvom mieste môže byť ktorákoľvek nenulová cifra okrem $3$, $6$ a~$9$.}

{%%%%%   Z8-I-4
\napad
Hľadajte rovnoramenné trojuholníky, pri ktorých viete určiť veľkosti
vnútorných uhlov.

\res
Hľadaný uhol budeme nazývať $\alpha$.
Ďalej označme vrcholy a~stred hviezdy ako na \obr.
Ak spojíme všetky vrcholy so stredom~$S$, vidíme veľa rovnoramenných
trojuholníkov, z~ktorých viaceré sú navzájom zhodné.
Napr. trojuholníky $ASC$, $BSD$, \dots, $GSB$ sú zhodné:
všetky tieto trojuholníky majú zhodné ramená a~uhol pri vrchole~$S$
(ktorého veľkosť je rovná dvojnásobku veľkosti uhla $ASB$, \tj.
$\gamma=2\cdot\frac{360\st}7$).
Uhol~$\alpha$ je preto rovný dvojnásobku uhla pri vrchole~$A$ v~trojuholníku $ASC$.
Keďže je tento trojuholník rovnoramenný, je uhol $\alpha$
rovnaký ako súčet vnútorných uhlov pri~vrcholoch $A$ a~$C$.
\figure70

Súčet veľkostí vnútorných uhlov v~ľubovoľnom trojuholníku je $180\st$,
preto platí $\alpha+\gamma=180\st$, odkiaľ ľahko dopočítame
veľkosť uhla $\alpha$:
$$
\alpha=180\st-\frac27\cdot 360\st
=\Bigl(1-\frac47\Bigr)\cdot 180\st
=\Bigl(77\frac17\Bigr)\st
\doteq 77\st 8' 34''.
$$

\poznamka
Ak vieme (alebo zdôvodníme), že veľkosť vnútorného uhla v~pravidelnom
sedemuholníku je rovná $\frac57\cdot180\st$, tak môžeme úlohu doriešiť nasledovne (\obr).
\figure71

Vnútorné uhly pri vrcholoch $K$ a~$Q$ v~rovnoramennom trojuholníku $KAQ$ sú
vonkajšími uhlami pravidelného sedemuholníka $KLMNOPQ$; ich veľkosť je preto
$180\st-\frac57180\st=\frac27180\st$.
Súčet veľkostí vnútorných uhlov v~trojuholníku $KAQ$ je
$\alpha+\frac47180\st=180\st$, odkiaľ vyjadríme neznámu:
$\alpha=\left(1-\frac47\right)\cdot 180\st=\dots $}

{%%%%%   Z8-I-5
\napad
Pracujte so súčtom vekov všetkých zamestnaných učiteľov.
Uvažujte, ako sa tento súčet mení vzhľadom na uvedený dátum.

\res
Súčet vekov všetkých siedmich učiteľov školy 1. septembra 2007 označme~$c$.
Súčet vekov týchto siedmich ľudí sa do 1. septembra 2010 zväčšil o~$7\cdot3=21$,
súčet vekov všetkých ôsmich učiteľov pracujúcich v~tento deň na škole bol teda
$$
c+21+25=c+46.
$$
Súčet vekov týchto ôsmich ľudí sa do 1. septembra 2012 zväčšil o~$8\cdot2=16$.
V~tento deň jeden z~nich už na škole nepracoval, jeho vek v~ten deň označíme
$x$.
Súčet vekov siedmich zvyšných učiteľov bol v~tento deň rovný
$$
c+46+16-x=c+62-x.
$$

Keďže má byť priemerný vek učiteľov na škole vo všetkých spomenutých dátumoch rovnaký,
platia rovnosti
$$
\frac{c}7=\frac{c+46}8=\frac{c+62-x}7.
$$
Z~rovnosti medzi prvým a~tretím lomeným výrazom priamo vyplýva, že $x=62$.
Učiteľ, ktorý 1. septembra 2012 už na škole nepracoval, mal teda práve 62~rokov.

Úpravami rovnosti medzi prvými dvoma lomenými výrazmi
odvodíme hodnotu~$c$:
$$
\aligned
\frac{c}7&=\frac{c+46}8,\\
8c&=7c+7\cdot46,\\
c&=322.
\endaligned
$$
Priemerný vek učiteľov vo všetkých troch spomenutých dátumoch bol $322:7=46$ rokov.}

{%%%%%   Z8-I-6
\napad
Najskôr určte dĺžky všetkých úsekov bludiska.

\riesenie
Zo zadania vidíme, že dĺžky jednotlivých úsekov bludiska sú postupne
(počítané v~metroch od stredu): 1, 1, 2, 2, 3, 3, 4, 4, atď.
Najskôr určíme, aké dlhé sú posledné úseky bludiska, aby celková dĺžka
bola práve 210\,m.
Či už skúšaním, alebo nejakým pomocným výpočtom, celkom rýchlo zistíme, že
bludisko pozostáva z~nasledujúcich úsekov:
$$
1,\ 1,\ 2,\ 2,\ 3,\ 3,\ 4,\ 4,\ 5,\ 5,\ 6,\ 6,\ 7,\ 7,\ 8,\ 8,\ 9,\ 9,\
10,\ 10,\ 11,\ 11,\ 12,\ 12,\ 13,\ 13,\ 14,\ 14.
$$
Anička prešla o~24\,m viac ako Hanka; v~uvedenej postupnosti preto hľadáme
niekoľko po sebe idúcich čísel, ktorých súčet je $24$.
Aby bolo zrejmé, že sme nezabudli na žiadnu možnosť, budeme postupovať
systematicky podľa počtu úsekov, ktoré delia Aničku od Hanky.
Pre daný počet úsekov môžeme orientačne vyjadriť priemernú dĺžku jedného
úseku.
V~blízkosti tejto hodnoty potom v~našej postupnosti hľadáme zodpovedajúci počet po
sebe idúcich čísel so súčtom $24$.
Výsledok nášho snaženia zhŕňa nasledujúca tabuľka:
$$
\begintable
počet úsekov\|priemerná dĺžka|riešenie\crthick
1\|24|$-$\cr
2\|12|12, 12\cr
3\|8|$-$\cr
4\|6|5, 6, 6, 7\cr
5\|4,8|4, 4, 5, 5, 6\cr
6\|4|3, 3, 4, 4, 5, 5\cr
7\|3,4|$-$\cr
8\|3|1, 2, 2, 3, 3, 4, 4, 5%
\endtable
$$
Posledné riešenie v~tabuľke predstavuje prípad, keď Hanka prešla len 1~meter,
teda najmenšiu možnú vzdialenosť.
Preto nemá zmysel uvažovať 9 a~viac úsekov.
Úloha má päť riešení; dievčatá mohli stáť v~rohoch ako na \obr.
\figure80
}

{%%%%%   Z9-I-1
\napad
Zistite, či pôvodné číslo obsahuje nulu a~či sa v~ňom opakujú cifry.

\res
Označme použité cifry $a$, $b$, $c$.
Zo zadania je zrejmé, že cifry sa v~trojcifernom čísle neopakujú, \tj.
$a$, $b$ a~$c$ sú navzájom rôzne.
Keby totiž niektoré dve cifry boli rovnaké, zmenou ich poradia by sme
dokopy dostali nanajvýš tri rôzne čísla:
ak by napr. $a=b\ne c$ a~všetky cifry boli nenulové,
tak uvedené čísla by boli
$$
aac,\ aca,\ caa.
$$

Podobne môžeme usúdiť, že medzi použitými ciframi musí byť $0$.
Keby totiž $a$, $b$ a~$c$ boli navzájom rôzne a~nenulové cifry,
zmenou ich poradia by sme dostali práve šesť rôznych čísel:
$$
abc,\ acb,\ bac,\ bca,\ cab,\ cba.
$$

Použité cifry sú teda navzájom rôzne a~obsahujú nulu;
bez akejkoľvek ujmy na všeobecnosti môžeme predpokladať, že $a=0$ a~$b<c$.
Zmenou poradia takých cifier dostaneme práve nasledujúce štyri čísla
(píšeme ich usporiadané od najmenšieho):
$$
b0c,\ bc0,\ c0b,\ cb0.
$$

Zvyšok úlohy riešime ako algebrogram:
\algg{&b&0&c\\&b&c&0}{1&0&8&8}
Zo súčtu jednotiek $c+0=8$ vyplýva, že $c=8$ (to súhlasí aj so stĺpcom desiatok: $0+c=8$).
Zo súčtu stoviek $b+b=10$ vyplýva, že $b=5$.
Pôvodné trojciferné číslo musí obsahovať cifry $0$, $5$ a~$8$.}

{%%%%%   Z9-I-2
\napad
Dĺžku jednej strany označte ako neznámu a~pomocou nej potom vyjadrite dĺžky ostatných strán.
Uvedomte si, koľko možností je potrebné rozobrať.

\res
Dĺžky dvoch strán, ktoré sa líšia o~12\,cm, označíme $s$~a~$s+12$.
Tretia strana sa od niektorej z~týchto dvoch líši o~15\,cm.
Nie je zadané, od ktorej z~nich a~či je o~15\,cm väčšia alebo menšia;
preto musíme rozobrať nasledujúce štyri možnosti:
$$
\begintable
\|\multispan{3}\hfil dĺžky strán trojuholníka\hfil\crthick
1.~možnosť\|\hbox to 1.5cm{\hfil$s$\hfil}|\hbox to 1.5cm{\hfil$s+12$\hfil}|\hbox to 1.5cm{\hfil$s+15$\hfil}\cr
2.~možnosť\|$s$|$s+12$|$s-15$\cr
3.~možnosť\|$s$|$s+12$|$s+12+15$\cr
4.~možnosť\|$s$|$s+12$|$s+12-15$%
\endtable
$$

Obvod trojuholníka má byť 75\,cm.
Odtiaľ pre každú možnosť zostavíme rovnicu a~z~nej vypočítame príslušné
$s$.
Na ukážku uvádzame iba výpočet zodpovedajúci 1.~možnosti:
$$
\aligned
s+(s+12)+(s+15)&=75,\\
3s+27&=75,\\
3s&=48,\\
s&=16.
\endaligned
$$
Výsledné $s$~potom dosadíme do tabuľky:
$$
\begintable
\|\multispan{3}\hfil dĺžky strán trojuholníka\hfil\crthick
1.~možnosť\|\hbox to 1.5cm{\hfil16\hfil}|\hbox to 1.5cm{\hfil28\hfil}|\hbox
to 1.5cm{\hfil31\hfil}\cr
2.~možnosť\|26|38|11\cr
3.~možnosť\|12|24|39\cr
4.~možnosť\|22|34|19%
\endtable
$$

Aby vypočítané hodnoty skutočne zodpovedali stranám nejakého trojuholníka,
musia platiť trojuholníkové nerovnosti.
Preto ešte skontrolujeme, či najväčšie číslo v~každom riadku je menšie ako
súčet zvyšných dvoch.
Táto nerovnosť platí iba pri 1. a~4.~možnosti.
Úloha má teda dve riešenia: dĺžky strán trojuholníka môžu byť
16\,cm, 28\,cm a~31\,cm,
alebo 19\,cm, 22\,cm a~34\,cm.}

{%%%%%   Z9-I-3
\napad
Pripomeňte si vzťahy medzi priemernou rýchlosťou, celkovou vzdialenosťou
a~potrebným časom.

\res
Čas od okamihu, keď nás tréner motivoval pri horskej chate,
do odchodu vlaku označme $t$ (v~hodinách).
Dĺžku trasy z~chaty na stanicu označme $s$ (v~km).

Pri pohodlnom tempe by sme išli $\frac{s}4$ hodín a~prišli by sme
tri štvrte hodiny po odchode vlaku; platí teda
$$
\frac{s}4=t+\frac34.
$$
Chôdza v~lepšej obuvi trvá $\frac{s}6$ hodín, čo je o~pol hodiny menej ako $t$; platí
teda
$$
\frac{s}6=t-\frac12.
$$

Z~oboch rovníc vyjadríme $t$:
$$
t=\frac{s}4-\frac34,\quad t=\frac{s}6+\frac12,
$$
a~tak dostaneme novú rovnicu:
$$
\frac{s}4-\frac34=\frac{s}6+\frac12.
$$
Jej úpravami získame dĺžku trasy $s$:
$$
\aligned
3s-9&=2s+6,\\
s&=15.
\endaligned
$$
Trasa z~chaty na stanicu bola dlhá 15\,km.

\poznamka
K~výsledku môžeme dospieť aj~nasledovne.
Z~prvých dvoch vyššie zostavených rovníc vyjadríme $s$:
$$
s=4t+3,\quad s=6t-3,
$$
a~tak dostaneme novú rovnicu
$$
4t+3=6t-3,
$$
odkiaľ ľahko vyjadríme $t=3$.
Spätným dosadením dostaneme
$$
s=12+3=18-3=15.
$$}

{%%%%%   Z9-I-4
\napad
Pre začiatok nepracujte s~osemuholníkom: skonštruujte ortocentrum~$D$
vo všeobecnom trojuholníku $ABX$ a~snažte sa zrekonštruovať~$X$, keď poznáte $A$,
$B$ a~$D$.

\riesenie
Aby sme si uvedomili súvislosti,
uvažujeme najskôr všeobecný trojuholník $ABX$ a~zostrojíme jeho ortocentrum~$D$.
Na \obr{} je naznačené riešenie pre tupouhlý trojuholník, v~tomto
prípade leží ortocentrum zvonka trojuholníka.
(Všimnite si, že pri ostrouhlom trojuholníku by ortocentrum bolo vnútri
a~pri pravouhlom by splývalo s~vrcholom, pri ktorom je pravý uhol.)
\figure100

Teraz rozoberieme, ako zostrojiť trojuholník $ABX$, ak je daná jeho strana
$AB$ a~ortocentrum~$D$.
Z~úvodnej diskusia vieme, že ak $D$ leží na priamke~$AB$, tak $D=A$ alebo
$D=B$ a~vrchol~$X$ v~tomto prípade nemožno určiť jednoznačne.
Preto predpokladáme, že body $A$, $B$ a~$D$ sú vo všeobecnej polohe,
\tj. neležia na jednej priamke.

Vrchol~$X$ je spoločným bodom priamok obsahujúcich strany $a$, $b$ a~výšku
$v_x$;
keď zostrojíme aspoň dve z~týchto troch priamok, bude $X$ daný ako ich
priesečník.
Výška~$v_x$ leží na priamke, ktorá je kolmá na~$AB$ a~prechádza bodom $D$.
Zvyšné dve výšky v~budúcom trojuholníku ležia na priamkach $AD$, príp. $BD$.
Priamka určená stranou~$a$ je kolmá na $AD$ a~prechádza bodom~$B$ (podobne
strana~$b$ je kolmá na $BD$ a~prechádza cez $A$).
Odtiaľ vyplýva nasledujúca možná konštrukcia trojuholníka
$ABX$, keď je daná jeho strana $AB$ a~ortocentrum $D$:
\begin{enumerate}
\item zostrojiť priamku, ktorá je kolmá na~$AB$ a~prechádza bodom~$D$,
\item zostrojiť priamku, ktorá je kolmá na~$AD$ a~prechádza bodom~$B$,
\item označiť $X$ priesečník priamok z~predošlých dvoch krokov,
\item narysovať trojuholník $ABX$.
\end{enumerate}
\noindent
(Namiesto 1. alebo 2.~kroku konštrukcie možno tiež
uvažovať priamku, ktorá je kolmá na~$BD$ a~prechádza bodom~$A$.)

\smallskip
Teraz rozoberieme, ako vyzerá predchádzajúca konštrukcia bodu~$X$ pre trojicu $A$, $B$
a~$D$ v~takej špeciálnej polohe, ktorá je opísaná v~zadaní pomocou
pravidelného osemuholníka.
\begin{enumerate}
\item Kolmicou na priamku~$AB$ prechádzajúcou bodom~$D$ je práve priamka~$CD$.
(Označme $P$ priesečník priamok $AB$ a~$CD$.
V~trojuholníku $BCP$ majú vnútorné uhly pri vrcholoch $B$ a~$C$ veľkosť
$45\st$, pretože sa jedná o~vonkajšie uhly pravidelného osemuholníka.
Uhol pri vrchole~$P$ je preto pravý.)
\item Kolmicou na priamku~$AD$ prechádzajúcou bodom~$B$ je práve priamka~$BG$.
(Podobne ako vyššie môžeme ukázať, že priamky $AH$ a~$BC$ sú kolmé.
Uhlopriečka $BG$ je rovnobežná s~$AH$, podobne $AD$ je rovnobežná s~$BC$,
takže priamky $AD$ a~$BG$ sú tiež kolmé.)
\item Bod~$X$ je priesečníkom priamok $CD$ a~$BG$.
\item Výsledný trojuholník je zvýraznený na \obr.
\insp{z62.103}%
\end{enumerate}

\poznamka
Všimnite si, že vyššie opísaná konštrukcia je práve konštrukcia
ortocentra trojuholníka $ABD$.
Platí teda, že ak je
bod~$D$ ortocentrom (všeobecného) trojuholníka $ABX$, tak
bod~$X$ je ortocentrom trojuholníka $ABD$.
(Podobne bod~$B$ je ortocentrom trojuholníka $ADX$ a~bod~$A$ je ortocentrom
trojuholníka $BDX$.)}

{%%%%%   Z9-I-5
\napad
Číslo v~nejakom políčku označte ako neznámu a~pomocou nej vyplňte celú schému.

\riesenie
Číslo v~prvom políčku označíme neznámou~$a$ a~pomocou nej vyjadríme čísla vo
všetkých ostatných políčkach ako na \obr{} (zlomky uvádzame v~základnom tvare).
\figure91

Aby všetky zapísané výrazy predstavovali celé čísla, musí byť neznáma~$a$
deliteľná všetkými uvedenými menovateľmi.
Najmenším spoločným násobkom menovateľov $3$, $9$, $27$ a~$6$ je číslo $54$.
Neznáma $a$ tak musí byť násobkom čísla $54$.

Ďalej zohľadníme podmienku, že všetky zapísané čísla majú byť štvorciferné.
Najmenším zapísaným číslom je $\frac{10}{27}a$ a~najväčším je
$\frac83a$.
Preto musí platiť:
\begin{itemize}
\item $\frac{10}{27}a\ge1\,000$, po úprave $a\ge2\,700$,
\item $\frac83a\le9\,999$, po úprave $a\le3\,749\frac58$.
\end{itemize}

Určíme počet násobkov čísla $54$ v~intervale od $2\,700$ po $3\,749$.
Najmenším z~nich je hneď $2\,700$ a~ďalej ich je ešte 19, pretože
$(3\,749-2\,700):54=19$ (zvyšok $23$).
Do prvého políčka môžeme teda doplniť celkom 20 rôznych čísel,
čiže schému môžeme vyplniť 20 spôsobmi.}

{%%%%%   Z9-I-6
\napad
Obsah tohto lichobežníka je rovnaký ako obsah vhodného pravouhlého
trojuholníka.
Najskôr nájdite taký trojuholník, potom určte dĺžky strán lichobežníka a~jeho obvod.

\res
Bodom~$C$ vedieme rovnobežku s~uhlopriečkou~$BD$ a~jej priesečník s~priamkou
$AB$ označíme $B'$ (\obr).
Keďže priamky $AB$ a~$CD$ sú tiež rovnobežné, je $BB'CD$ kosodĺžnik a~platí $|B'C|=|BD|=9\cm$ a~$|B'B|=|CD|$.
\figure104

\noindent
Zmysel tejto konštrukcie spočíva v~pozorovaní, že trojuholníky $ACD$ a~$CB'B$
majú rovnaký obsah
(strany $CD$ a~$B'B$ sú zhodné a~výšky oboch trojuholníkov na tieto strany
sú rovnaké).
Preto je obsah lichobežníka $ABCD$ rovnaký ako obsah trojuholníka $AB'C$ a~tento vieme ľahko určiť:
Z~konštrukcie vyplýva, že trojuholník $AB'C$ je pravouhlý,
a~zo zadania poznáme obe jeho odvesny $|AC|=12\cm$ a~$|B'C|=9\cm$.
Obsah trojuholníka $AB'C$, teda aj zadaného lichobežníka, je rovný
$$
S=\frac12\cdot12\cdot9=54\,(\text{cm}^2).
$$

Aby sme určili obvod lichobežníka, potrebujeme poznať dĺžky všetkých jeho strán.

a)
Strana~$BC$ je výškou na stranu~$AB'$ v~práve spomenutom trojuholníku.
Z~Pytagorovej vety spočítame dĺžku prepony v~trojuholníku $AB'C$:
$$
|AB'|=\sqrt{12^2+9^2}=15\,(\text{cm}).
$$
Zo znalosti obsahu tohto trojuholníka určíme jeho výšku $|BC|$:
$$
\aligned
\frac12\cdot15\cdot \vert BC\vert&=54,\\
\vert BC\vert&=7{,}2\,(\text{cm}).
\endaligned
$$

b)
V~pravouhlom trojuholníku $ABC$ poznáme jeho preponu a~teraz aj jednu
odvesnu; pomocou Pytagorovej vety určíme dĺžku druhej odvesny:
$$
|AB|=\sqrt{12^2-7{,}2^2}=9{,}6\,(\text{cm}).
$$

c)
Zrejme platí $|AB'|=|AB|+|BB'|$ a~$|BB'|=|CD|$, odkiaľ ľahko vyjadríme dĺžku
strany $CD$:
$$
|CD|=|AB'|-|AB|=15-9{,}6=5{,}4\,(\text{cm}).
$$

d)
Stranu~$AD$ môžeme vidieť ako preponu v~pravouhlom trojuholníku $APD$, pričom
$P$ je päta kolmice z~bodu~$D$ na stranu~$AB$.
Dĺžky odvesien v~tomto trojuholníku sú $|PD|=|BC|=7{,}2\cm$ a~$|AP|=|AB|-|CD|=9{,}6-5{,}4=4{,}2$\,(cm).
Podľa Pytagorovej vety spočítame aj~dĺžku prepony:
$$
|AD|=\sqrt{7{,}2^2+4{,}2^2}\doteq 8{,}3\,(\text{cm}).
$$

Obvod zadaného lichobežníka je teda približne rovný
$$
o=|AB|+|BC|+|CD|+|DA| \doteq9{,}6+7{,}2+5{,}4+8{,}3 =30{,}5\,(\text{cm}).
$$

\poznamka
Pre zaujímavosť a~kontrolu uvádzame ešte výpočet obsahu lichobežníka pomocou
zvyčajného vzorca:
$$
S=\frac12(|AB|+|CD|)\cdot|BC|=\frac12(9{,}6+5{,}4)\cdot7{,}2=54\,(\text{cm}^2).
$$
Všimnite si, že úvahy v~úvode nášho riešenia platnosť tohto vzorca
vlastne zdôvodňujú.

Vzťah pre výpočet obsahu zadaného lichobežníka sa dá odvodiť aj pomocou \obr.
Na ňom je čiarkovane znázornený obdĺžnik, ktorého každá strana prechádza niektorým
vrcholom lichobežníka a~je rovnobežná s~niektorou jeho uhlopriečkou.
Obsah obdĺžnika je rovný súčinu $|AC|\cdot|BD|$.
Obsah lichobežníka je evidentne polovičný, teda $S=\frac12|AC|\cdot|BD|$.
\figure140
}

{%%%%%   Z4-II-1
...}

{%%%%%   Z4-II-2
...}

{%%%%%   Z4-II-3
...}

{%%%%%   Z5-II-1
Na Mišovom výsledku 30\,kg sa podieľala jedna chyba váhy,
rovnako aj na~Emilovom výsledku 28\,kg bola zahrnutá tá istá chyba váhy.

V~súčte $30+28=58$\,(kg) teda sú zahrnuté dve chyby váhy.
Keď sa však vážili Mišo s~Emilom dokopy, vo výsledku 56\,kg bola
zahrnutá iba jedna chyba váhy.
Rozdiel $58-56=2$\,(kg) teda predstavuje práve jednu chybu váhy.

Skutočná hmotnosť Miša bola $30-2=28$\,(kg),
Emil vážil $28-2=26$\,(kg).

\hodnotenie
4~body za odvodenie chyby váhy;
po 1~bode za hmotnosti Miša a~Emila.
\endhodnotenie}

{%%%%%   Z5-II-2
Je šesť rôznych smerov, ako sa na stavbu, respektíve na jednotlivé kocôčky
stavby, dá pozerať.
Pre každý smer spočítame, koľko stien kocôčok bude treba ofarbiť:
\begin{itemize}
  \item
    Keď sa na obrázok pozrieme zhora, vidíme práve 7~stien, ktoré bude treba
    ofarbiť.
    Rovnaký počet napočítame pri pohľade zdola.
  \item
    Pri pohľade spredu vidíme celkom 8~stien.
    Rovnaký počet napočítame pri pohľade zozadu.
  \item
    Pri pohľade zľava vidíme 7~stien, ale bude treba ofarbiť ešte
    ďalších 5~stien, ktoré sú zakryté -- jedná sa o~kocôčky v~pravom
    pilieri stavby.
    Celkom sme napočítali 12~stien.
    Pri pohľade sprava je situácia rovnaká.
\end{itemize}
Celkom teda potrebujeme ofarbiť $2\cdot(7+8+12)=2\cdot27=54$ stien,
čo je rovnaké ako ofarbiť 9~celých kocôčok $(9\cdot6=54)$.
Na ofarbenie jednej celej kocôčky treba 10\,ml farby, takže celkom
potrebujeme $9\cdot 10=90$\,(ml) farby.

\ineriesenie
Môžeme postupne prebrať všetky kocôčky v~stavbe a~určiť,
koľko ich stien budeme ofarbovať.
Postupujeme po vrstvách zhora nadol,
v~každej vrstve po radoch spredu dozadu,
v~každom rade zľava doprava:
\begin{itemize}
  \item
    V~1. vrstve je jediná kocôčka, na ktorej budeme ofarbovať 5~stien.
  \item
    V~2. vrstve sú tri kocôčky -- ofarbujeme postupne $4+3+4=11$ stien.
  \item
    V~3. vrstve sú štyri kocôčky -- ofarbujeme postupne $4+4+3+3=14$ stien.
  \item
    Vo 4. vrstve je šesť kocôčok -- ofarbujeme postupne $5+5+3+3+4+4=24$ stien.
\end{itemize}
Celkom teda potrebujeme ofarbiť $5+11+14+24=54$ stien.
Ďalší postup môže byť rovnaký ako v~predchádzajúcom riešení.

\hodnotenie
4~body za počet stien, ktoré treba ofarbiť;
2~body za určenie potrebného množstva farby.
\endhodnotenie
}

{%%%%%   Z5-II-3
Za týždeň Radka zjedla $5\cdot3+2\cdot5=25$ lentiliek.
Za štyri týždne ich zjedla 100 a~zostávalo 11~lentiliek.
Tento počet potrebujeme podľa podmienok v~zadaní
úlohy vyjadriť ako súčet niekoľkých trojok a~pätiek,
a~to sa dá jediným spôsobom: ${11=2\cdot3+5}$.

Radka teda jedla lentilky štyri celé týždne, dva pracovné dni a~jeden víkendový
deň.
To znamená, že Radkino maškrtenie končilo buď trojicou dní
štvrtok -- piatok -- sobota, alebo nedeľa -- pondelok -- utorok.
V~prvom prípade vychádzajú Radkine narodeniny na štvrtok, v~druhom prípade na
nedeľu.

\ineriesenie
Rovnako ako vyššie zistíme, že za štyri týždne Radka zjedla 100 lentiliek.
Postupne preskúšame všetky možnosti, kedy mohla mať Radka narodeniny:
$$
\tablewidth=0.95\hsize
\begintable
narodeniny\hfill|množstvo zjedených lentiliek\hfill|záver\crthick
pondelok\hfill|\dots, 100 (ne), 103 (po), 106 (ut), 109 (st), 112
  (št)\hfill| nevychádza\cr
utorok\hfill|\dots, 100 (po), 103 (ut), 106 (st), 109 (št), 112
  (pi)\hfill|nevychádza\cr
streda\hfill|\dots, 100 (ut), 103 (st), 106 (št), 109 (pi), 114
  (so)\hfill|nevychádza\cr
\bf štvrtok\hfill|\dots, 100 (st), 103 (št), 106 (pi), {\bf 111}
  (so)\hfill|\hfill vychádza\cr
piatok\hfill|\dots, 100 (št), 103 (pi), 108 (so), 113 (ne)\hfill|nevychádza\cr
sobota\hfill|\dots, 100 (pi), 105 (so), 110 (ne), 113
  (po)\hfill|nevychádza\cr
\bf nedeľa\hfill|\dots, 100 (so), 105 (ne), 108 (po), {\bf 111}
  (ut)\hfill|\hfill vychádza%
\endtable
$$
Radka mohla mať narodeniny buď vo štvrtok, alebo v~nedeľu.

\hodnotenie
1~bod za určenie, že za týždeň Radka zjedla 25 lentiliek;
1~bod za určenie, že takto môžeme odpočítať 100 lentiliek;
2~body za kľúčovú myšlienku riešenia (rozdelenie $3+3+5$,
navrhnutie tabuľky a~pod.);
po 1~bode za každé správne riešenie.

Za nájdenie a~zdôvodnenie jedného správneho riešenia udeľte nanajvýš 5~bodov.
Ak riešiteľ jedno z~riešení uhádne, dajte 1~bod.
\endhodnotenie
}

{%%%%%   Z6-II-1
Pripočítaním neznámeho čísla k~prvému sčítancu a~odčítaním toho istého čísla od
druhého sčítanca na ľavej strane sa súčet týchto dvoch čísel nezmení
a~je rovný $589+544=1\,133$.
Tento čiastkový súčet je o~$2\,013-1\,133=880$ menší ako číslo na pravej strane
rovnosti.
Preto súčin $80$ a~neznámeho Matovho čísla má byť rovný $880$.
Číslo, ktoré Mat našiel, bolo $880:80=11$.

\hodnotenie
2~body za zistenie, že prvé dve operácie nemajú na výsledok žiadny vplyv;
2~body za vyjadrenie rozdielu $2\,013-1\,133=880$ a~vysvetlenie jeho významu;
2~body za vyjadrenie neznámeho čísla.
\endhodnotenie
}

{%%%%%   Z6-II-2
Súčet párneho a~nepárneho čísla je vždy číslo nepárne.
Avšak vo výslednom súčte je na mieste desiatok číslo párne, čo je možné
jedine vtedy, keď súčet cifier na mieste jednotiek je väčší ako~$10$.
Súčasne si uvedomujeme, že pre dva sčítance je tento súčet najviac~$18$.

Teraz zistíme, ktoré z~nepárnych cifier môže byť $9$:
\begin{itemize}
  \item Ak by to bola druhá cifra vo výsledku, tak by súčet cifier na
    mieste jednotiek bol $19$, čo je príliš veľa.
  \item Ak by to bola prvá cifra v~jednom zo sčítancov, tak by tento
    sčítanec bol aspoň $91$.
    Pritom sčítanec s~párnymi ciframi je aspoň $20$ (na mieste desiatok nemôže
    byť $0$), takže výsledný súčet by nebol dvojciferný.
\end{itemize}
Ostáva jediná možnosť -- $9$ je druhá cifra v~sčítanci s~nepárnymi
ciframi.
Tu zatiaľ žiadny problém nevidíme, takže skúmame ďalej:

Sčítanec s~nepárnymi ciframi môže byť
$$
19,\ 39,\ 59,\ 79\textrm{\ alebo\ }99.
$$
Z~týchto čísel sú násobkom $3$ iba čísla $39$ a~$99$.
Číslo $99$ je však príliš veľké
(pripočítaním akéhokoľvek čísla by sme dostali trojciferné číslo),
takže Lenkino číslo s~nepárnymi ciframi môže byť jedine $39$.

Odtiaľ vidíme, že druhý sčítanec nesmie byť väčší ako $60$ (aby súčet bol
dvojciferný).
Toto číslo má mať iba párne cifry a~navyše z~úvodného odseku vieme, že na
mieste jednotiek musí byť aspoň $2$.
Sčítanec s~párnymi ciframi teda môže byť
$$
22,\ 24,\ 26,\ 28,\ 42,\ 44,\ 46\text{\ alebo\ }48.
$$
Z~týchto čísel sú násobkom $3$ len čísla $24$, $42$ a~$48$.
Spolu teda dostávame nasledujúce tri možnosti:
$$
24+39=63,\ 42+39=81,\ 48+39=87.
$$
Vo všetkých prípadoch sú splnené všetky podmienky zo zadania, takže Lenka si
mohla myslieť ktorúkoľvek z~uvedených dvojíc sčítancov.

\hodnotenie
1~bod za pozorovanie, že súčet cifier na mieste jednotiek je väčší ako $10$;
2~body za určenie čísla $39$ vrátane zdôvodnenia;
3~body za nájdenie sčítancov $24$, $42$ a~$48$ vrátane zdôvodnenia.
\endhodnotenie
}

{%%%%%   Z6-II-3
Obsah štvoruholníka $ABCD$ skúsime určiť ako súčet obsahov niekoľkých v~ňom
obsiahnutých trojuholníkov.

Keďže uhol $ABC$ je pravý a~priamky $AB$ a~$CD$ sú rovnobežné, je aj
uhol $BCD$ pravý.
Pätu výšky v~trojuholníku $ABD$ z~vrcholu $D$ označíme $E$ (\obr).
\insp{z62ii.3}%

Rovnoramenný trojuholník $ABD$ je výškou $DE$ rozdelený na dva zhodné
trojuholníky.
Navyše štvoruholník $BCDE$ je štvorec (je to pravouholník a~$|BC|=|CD|$)
a~jeho uhlopriečka $BD$ ho rozdeľuje na dva zhodné trojuholníky.
Trojuholníky $BCD$, $BED$ a~$AED$ sú teda navzájom zhodné a~obsah každého
z~nich je rovný polovici obsahu štvorca $BCDE$, \tj.
$$
\frac{10\cdot10}2=50\,(\text{cm}^2).
$$
Obsah štvoruholníka $ABCD$ je rovný súčtu obsahov týchto troch trojuholníkov:
$$
S_{ABCD}=3\cdot50=150\,(\text{cm}^2).
$$

\hodnotenie
2~body za zdôvodnenie, že štvoruholník $BCDE$ je štvorec;
2~body za rozdelenie na zhodné trojuholníky;
2~body za vyjadrenie obsahu.
\endhodnotenie
}

{%%%%%   Z7-II-1
Číslo vyjadrujúce cenu rybárskych potrieb je štvorciferné,
pritom prvá cifra je o~jedna väčšia ako cifra tretia a~o~jedna menšia ako
štvrtá.
Uvedené tri cifry sú teda navzájom rôzne a~súčet všetkých štyroch cifier má
byť $6$.
Doteraz uvedeným podmienkam vyhovujú len čísla $2\,013$ a~$1\,302$.
Iba v~prvom prípade však platí, že prvé dvojčíslie predstavuje číslo
o~$7$ väčšie ako druhé dvojčíslie -- cena potrieb teda bola $2\,013$.

Vyjadriť $2\,013$ ako súčin troch čísel, z~ktorých žiadne nie je rovné jednej, možno
len jediným spôsobom:
$$
2\,013=3\cdot11\cdot61.
$$
Dedo Vendelín mal 61~rokov a~jeho vnuci 3 a~11 rokov.

\hodnotenie
3~body za objavenie čísla $2\,013$ vrátane zdôvodnenia;
3~body za rozklad čísla a~odpoveď.
\endhodnotenie
}

{%%%%%   Z7-II-2
Po prvom prehnutí splynul bod~$A$ s~bodom~$C$.
Priamka, podľa ktorej sa prehýbalo, je osou úsečky~$AC$,
ktorá v~rovnostrannom trojuholníku $ABC$ prechádza bodom~$B$.
Po tomto prehnutí dostala Petra trojuholník $BEC$, ktorý tvorí práve polovicu
trojuholníka $ABC$ (bod $E$ je stredom úsečky $AC$, \obr).
\insp{z62ii.4}%

Po druhom prehnutí splynul aj bod~$B$ s~bodom~$C$.
Priamka, podľa ktorej sa prehýbalo, je osou úsečky~$BC$,
ktorá v~pôvodnom trojuholníku prechádza bodom~$A$.
Po tomto prehnutí dostala Petra štvoruholník $OECD$
($D$ je stredom úsečky $BC$ a~$O$ je priesečník osí, \obr).
\insp{z62ii.5}%

Osi strán v~rovnostrannom trojuholníku $ABC$
sú osami súmernosti tohto trojuholníka.
Odtiaľ vyplýva, že štvoruholníky $OECD$, $ODBF$ a~$OFAE$ sú navzájom zhodné
a~každý z~nich má obsah $12\cm^2$ ($F$ je stredom úsečky~$AB$).
Obsah trojuholníka $ABC$ je rovný súčtu obsahov týchto troch štvoruholníkov:
$$
S_{ABC}=3\cdot12=36\,(\text{cm}^2).
$$

\hodnotenie
2~body za správnu interpretáciu prvého prehýbania (trojuholník $BEC$);
2~body za správnu interpretáciu druhého prehýbania (štvoruholník $OECD$);
2~body za vyjadrenie obsahu trojuholníka $ABC$.
\endhodnotenie
}

{%%%%%   Z7-II-3
Zlodeji ukradli $60\cdot40=2\,400$\,kg.
Pred krádežou chýbalo v~dokumentácii z~každého veľkého vreca 15\,kg.
Celková nezapísaná hmotnosť sa rovná tej ukradnutej, veľkých vriec teda bolo
dodaných
$$
2\,400:15=160.
$$
Malých vriec bolo dvakrát viac, \tj. 320;
celkom dodávka obsahovala $160+320=480$ vriec.
Vedúci zadokumentoval hmotnosť $480\cdot25=12\,000$\,kg.
Z~dodaného cementu zvýšilo množstvo zodpovedajúce dokumentácii, teda
12\,000\,kg.

\hodnotenie
3~body za počet veľkých vriec pred krádežou;
3~body za hmotnosť zvyšného cementu.
\endhodnotenie
}

{%%%%%   Z8-II-1
Cifry v~Júliinom čísle označíme $a$, $b$, $c$, $d$.
Potom informácie zo zadania môžeme zapísať takto:
$$
\alggg{a&b&c&d\\a&d&c&b}{3&3&3&2}
\hskip1cm
\alggg{a&b&c&d\\c&b&a&d}{7&8&8&6}
$$

Z~prvého súčtu vyplýva, že $a=1$
(keby $a=0$, tak by hľadané číslo nebolo štvorciferné,
keby $a\ge2$, tak by súčet bol väčší ako $4\,000$).
Dosadíme do predchádzajúceho vyjadrenia a~podobne budeme postupovať ďalej:
%%Přitom si uvědomujeme, že při naznačeném sčítání se do vedlejšího sloupce
%%přenáší nejvýše jednička.
$$
\alggg{1&b&c&d\\1&d&c&b}{3&3&3&2}
\hskip1cm
\alggg{1&b&c&d\\c&b&1&d}{7&8&8&6}
$$

Z~prvého stĺpca v~druhom súčte vyplýva, že $c$ je $5$ alebo $6$.
Z~tretieho stĺpca toho istého súčtu vyplýva, že $c$ je $6$ alebo $7$.
Preto musí byť $c=6$.
Dosadíme do tretieho stĺpca a~vidíme, že v~tomto stĺpci sa do výsledku
prenáša jednotka z~posledného stĺpca.
To znamená, že $d+d=16$, čiže $d=8$.
Po dosadení dostávame:
$$
\alggg{1&b&6&8\\1&8&6&b}{3&3&3&2}
\hskip1cm
\alggg{1&b&6&8\\6&b&1&8}{7&8&8&6}
$$

Teraz napr. z~posledného stĺpca v~prvom súčte vyvodíme, že $b=4$.
Tým sme určili všetky neznáme, dosadíme za $b$ na ostatných miestach
a~urobíme skúšku správnosti:
$$
\alggg{1&4&6&8\\1&8&6&4}{3&3&3&2}
\hskip1cm
\alggg{1&4&6&8\\6&4&1&8}{7&8&8&6}
$$

Keďže všetko vychádza, vieme, že Júlia mala na papieri číslo
$1\,468$.

\hodnotenie
1~bod za formuláciu problému pomocou neznámych;
po 1~bode za určenie každej neznámej;
1~bod za skúšku.
\endhodnotenie
}

{%%%%%   Z8-II-2
Dĺžku hrany štvorcovej podstavy označíme $a$ a~výšku hranola $v$.
Obe veličiny vyjadrujeme v~dm;
objem nádrže v~litroch je rovný $V=a^2\cdot v$.
Objem vody v~prvej nádrži bol $a^3$, v~druhej nádrži $a^3+50$
a~dokopy $300$.
Platí teda
$$
a^3+(a^3+50)=300,
$$
odkiaľ ľahko vyjadríme $a^3=125$ a~$a=5$\,(dm).

V~prvej nádrži teda bolo 125 litrov vody, čo predstavovalo 62{,}5\,\% celkového
objemu~$V$. Platí preto
$$
125=\frac{62{,}5}{100}\cdot V,
$$
odkiaľ vyjadríme $V=125\cdot\frac{100}{62{,}5}=2\cdot 100=200$\,(dm$^3$).
Zo vzťahu $V=a^2\cdot v$ teraz dopočítame poslednú neznámu:
$$
v=\frac{V}{a^2}=\frac{200}{25}=8\,(\text{dm}).
$$
Rozmery každej z~nádrží boli $5\,\text{dm}\times5\,\text{dm}\times8\,\text{dm}$.

\hodnotenie
2~body za vyjadrenie hrany podstavy;
2~body za vyjadrenie objemu nádrže;
2~body za vyjadrenie výšky.
\endhodnotenie
}

{%%%%%   Z8-II-3
Hry sa zúčastnilo 20 trojčlenných tímov, čo predstavuje 60 hráčov.
Zvyšných 108 hráčov z~celkového počtu chceme rozdeliť do 30 tímov po dvoch,
štyroch a~piatich hráčoch.

Štvorčlenných tímov bolo najviac, \tj. minimálne 21, čo predstavuje minimálne
84 hráčov.
Zvyšných 24 hráčov potrebujeme rozdeliť do 9 tímov po dvoch, štyroch a~piatich hráčoch.

Päťčlenný tím bol aspoň jeden.
Navyše z~24 hráčov možno zostaviť nanajvýš štyri päťčlenné tímy.
Teraz preberieme všetky prípady, ktoré môžu nastať:
\begin{itemize}
  \item Ak by päťčlenný tím bol práve 1, tak ostáva rozdeliť 19 hráčov
    do 8 tímov po dvoch a~štyroch hráčoch.
    To však nie je možné, pretože 19 je nepárne číslo.
  \item Ak by päťčlenné tímy boli 2, tak ostáva rozdeliť 14 hráčov
    do 7 tímov po dvoch a~štyroch hráčoch.
    To sa dá realizovať len jediným spôsobom -- všetci títo hráči budú
    v~dvojčlenných tímoch a~žiadny ďalší štvorčlenný tím sa nevytvorí.
  \item Ak by päťčlenné tímy boli 3, tak ostáva rozdeliť 9 hráčov
    do 6 tímov po dvoch a~štyroch hráčoch.
    To však nie je možné, pretože na 6 tímov treba aspoň 12~hráčov.
  \item Ak by päťčlenné tímy boli 4, tak ostáva rozdeliť 4 hráčov
    do 5 tímov po dvoch a~štyroch hráčoch.
    To však nie je možné, pretože na 5 tímov treba aspoň 10~hráčov.
\end{itemize}

Z~uvedenej diskusie vychádza jediná možnosť --
hry sa zúčastnilo 7 dvojčlenných, 20 trojčlenných, 21 štvorčlenných tímov
a~2 päťčlenné tímy.

\hodnotenie
2~body za úvahu, že stačí rozdeliť 24 hráčov do 9 tímov po 2, 4 a~5 hráčoch;
3~body za zdôvodnenie toho, že päťčlenné tímy mohli byť jedine dva;
1~bod za správny výsledok.
\endhodnotenie
}

{%%%%%   Z9-II-1
Označme písmenom~$d$ počet dievčat, písmenom~$x$ počet chlapcov s~mrkvou
a~písmenom~$y$ počet chlapcov s~jablkami.
Potom počet vreciek orechov je $3x+y$, počet vreciek jabĺk je $y$ a~počet vreciek
mrkvy je~$4x$.
Rovnosť medzi pomermi počtov zo zadania je
$$
d:x:y=(3x+y):y:4x.
$$
Odtiaľ môžeme vyjadriť pomer medzi $x$ a~$y$:
$$
\aligned
\frac{x}{y}&=\frac{y}{4x},\\
4x^2&=y^2,\\
2x&=y,
\endaligned
$$
pričom si uvedomujeme, že všetky neznáme sú kladné čísla.
Dosadením do úvodnej rovnosti dostávame
$$
d:x:y=5x:2x:4x=5:2:4.
$$
Zo zadania ďalej vieme, že $d+x+y=33$.
Práve zistený pomer sa teda snažíme rozšíriť tak, aby jeho členy dávali
súčet~$33$:
$$
d:x:y=15:6:12.
$$
V~triede je 15~dievčat, 6~chlapcov nieslo vrecká s~mrkvou a~12~chlapcov nieslo vrecká
s~jablkami.

\hodnotenie
2~body za zistenie pomeru počtu chlapcov s~mrkvou a~chlapcov s~jablkami;
2~body za získanie pomeru počtu dievčat, chlapcov s~mrkvou a~chlapcov s~jablkami;
2~body za správnu odpoveď.
\endhodnotenie}

{%%%%%   Z9-II-2
Cifry hľadaného čísla označíme $a$, $b$, $c$, pričom budeme predpokladať, že
$$
0 < a~< b < c. \tag1
$$
Čísla na tabuli, zoradené od najmenšieho po najväčšie, potom sú
$$
\overline{abc},\
\overline{acb},\
\overline{bac},\
\overline{bca},\
\overline{cab},\
\overline{cba}.
$$

Aritmetický priemer prvých troch z~nich je 205, teda platí
$$
(100a+10b+c)+(100a+10c+b)+(100b+10a+c)=3\cdot205,
$$
po úprave
$$
210a+111b+12c=615. \tag2
$$
Podobne zostavíme rovnicu na základe znalosti aritmetického priemeru všetkých čísel:
$$
222a+222b+222c=6\cdot370,
$$
po úprave
$$
\align
222\cdot(a+b+c)&=2\,220, \\
a+b+c&=10.\tag3
\endalign
$$

Z~odvodených vzťahov \thetag1, \thetag2 a~\thetag3 sme teraz schopní určiť jednoznačne cifry
$a$, $b$ a~$c$:
Pre $a\ge3$ by bola hodnota na ľavej strane rovnice \thetag2 príliš veľká a~táto rovnosť by nemohla platiť.
Pre $a=2$ by vzhľadom na~podmienku \thetag1 muselo byť $b \ge 3$
a~aj v~takom prípade by bola hodnota na ľavej strane rovnice \thetag2 príliš veľká.
Preto $a=1$.
Dosadením do rovníc \thetag2 a~\thetag3 a~ich úpravou získame sústavu
$$
\aligned
111b+12c&=405,\\
b+c&=9,
\endaligned
$$
ktorá má jediné riešenie $b=3$, $c=6$.
Žlté číslo, nami zapísané ako $\overline{bca}$, je teda~$361$.

\hodnotenie
2~body za zistenie, že ciferný súčet je 10;
2~body za vzťah $210a+111b+12c=615$ alebo jeho obdobu;
1~bod za určenie cifier hľadaného čísla;
1~bod za hľadané číslo.
\endhodnotenie}

{%%%%%   Z9-II-3
Ukážeme, že štvoruholník $C_1C_3A_1B_1$ má polovičný obsah ako trojuholník
$ABC$ a~mašľa zaberá práve polovicu tohto štvoruholníka.
Odtiaľ vyplýva, že mašľa zaberá štvrtinu obsahu celého trojuholníka.

Označme $c$ dĺžku strany~$AB$ a~$v$ veľkosť
výšky trojuholníka $ABC$ na~túto stranu.
Pri tomto označení je obsah trojuholníka rovný
$$
S_{ABC}=\frac12 cv.
$$

Úsečka $B_1A_1$ je stredná priečka trojuholníka $ABC$, je teda rovnobežná so
stranou~$AB$ a~má polovičnú veľkosť.
Body $C_1$, $C_2$ a~$C_3$ delia stranu $AB$ na štyri rovnaké diely, preto je
úsečka~$C_1C_3$ polovicou strany~$AB$, má teda rovnakú veľkosť ako úsečka~$B_1A_1$.
Keďže úsečky $B_1A_1$ a~$C_1C_3$ sú rovnobežné a~rovnako dlhé, je
štvoruholník $C_1C_3A_1B_1$ rovnobežníkom.
Veľkosť výšky rovnobežníka na stranu~$C_1C_3$ zodpovedá vzdialenosti strany~$AB$
a~strednej priečky~$B_1A_1$, čo je práve polovica výšky~$v$.
Obsah rovnobežníka je preto rovný
$$
S_{C_1C_3A_1B_1}=\frac{c}2\cdot\frac{v}2=\frac12 S_{ABC}.
$$

Uhlopriečky v~rovnobežníku $C_1C_3A_1B_1$ ho rozdeľujú na štyri
trojuholníky, ktoré majú rovnaký obsah.
Tento fakt stačí zdôvodniť pre ľubovoľné dva susedné trojuholníky:
Uhlopriečky rovnobežníka sa navzájom rozpoľujú, odkiaľ vyplýva $|SC_1|=|SA_1|$,
pričom $S$ označuje priesečník uhlopriečok.
Ďalej výška trojuholníka $SC_1B_1$ na stranu~$SC_1$ je rovnaká ako
výška trojuholníka $SB_1A_1$ na stranu~$SA_1$.
Odtiaľ vyplýva, že trojuholníky $SC_1B_1$ a~$SB_1A_1$ majú rovnaký obsah.
Rovnakým spôsobom možno zdôvodniť, že všetky trojuholníky $SC_1B_1$,
$SB_1A_1$, $SA_1C_3$ a~$SC_3C_1$ majú rovnaký obsah.

Odtiaľ teda vyplýva, že
$$
S_{\text{mašľa}}=\frac12S_{C_1C_3A_1B_1},
$$
čo spolu s~predchádzajúcou rovnosťou dokazuje,
že mašľa zaberá štvrtinu obsahu trojuholníka $ABC$.

\hodnotenie
2~body za zdôvodnenie, že $C_1C_3A_1B_1$ je rovnobežník a~vyjadrenie jeho
obsahu;
2~body za zdôvodnenie, že uhlopriečky tento rovnobežník rozdeľujú na štyri
trojuholníky s~rovnakým obsahom;
2~body za dopočítanie úlohy a~správny výsledok.

\poznamka
Pri riešení úlohy možno samozrejme využiť rôzne iné poznatky, ktoré
tu nerozvádzame.
Napr. nesusedné dvojice trojuholníkov v~rovnobežníku $C_1C_3A_1B_1$ sú
vlastne zhodné, pritom je ľahké vyjadriť výšku a~obsah trojuholníka
$SC_3C_1$ vzhľadom na trojuholník $ABC$ atď.
Úlohu je možné riešiť aj bez pomocnej úvahy s~rovnobežníkom.
V~každom prípade prispôsobte hodnotenie tak, aby 4~body zodpovedali
zdôvodneniu jednotlivých postrehov a~2~body vyjadreniu výsledku.
\endhodnotenie
}

{%%%%%   Z9-II-4
Posledná cifra musí byť párna, pretože posledné dvojčíslie je deliteľné dvoma.
Aby posledné päťčíslie bolo deliteľné piatimi, musí byť posledná cifra $0$ alebo $5$.
Z~nich párna je~$0$.
Prvé päťčíslie je tiež deliteľné piatimi, pre piate miesto ostáva cifra~$5$.
Zatiaľ máme číslo určené takto:
$$
*\,***\,5*0.
$$
Posledné trojčíslie je deliteľné tromi, preto jeho ciferný súčet musí byť deliteľný tromi.
Na predposlednú pozíciu tak môžeme doplniť~$4$ alebo~$1$.
Aby posledné štvorčíslie bolo deliteľné štyrmi, musí byť posledné dvojčíslie deliteľné štyrmi.
Na predposlednom mieste tak môže byť~$4$, nie však~$1$.
Teraz máme číslo určené takto:
$$
*\,***\,540.
$$

Keďže prvé dvojčíslie je deliteľné dvoma a~prvé štvorčíslie je deliteľné štyrmi,
sú cifry na druhej a~štvrtej pozícii párne. Preto na prvej a~tretej pozícii budú cifry nepárne.
Zo znalosti, že posledné šesťčíslie je deliteľné šiestimi, odvodíme prvú cifru.
Číslo je deliteľné šiestimi práve vtedy, keď je súčasne deliteľné dvoma aj tromi.
Na prvom mieste musí byť cifra zvolená tak, aby ciferný súčet zvyšku čísla bol deliteľný tromi.
Súčet cifier $0$ až $6$ je $21$, takže na prvom mieste môžu byť cifry $3$ a~$6$, z~ktorých nepárna je cifra~$3$.
Na tretej pozícii musí byť zvyšná nepárna cifra, čiže~$1$.
Číslo sme doposiaľ určili takto:
$$
3\,*1*\,540.
$$

Ostáva doplniť cifry $2$ a~$6$.
Aby prvé trojčíslie bolo deliteľné tromi, dáme cifru~$2$ na druhú pozíciu.
Hľadané číslo môže byť jedine
$$
3\,216\,540.
$$

Na záver musíme overiť, že sme na žiadnu požiadavku zo zadania nezabudli.
Ostáva teda overiť, že prvé štvorčíslie je deliteľné štyrmi a~prvé šesťčíslie
je deliteľné šiestimi.
To je splnené, takže hľadané číslo je $3\,216\,540$.

\hodnotenie
Po 1~bode za správne určenie a~zdôvodnenie pozície piatich cifier;
1~bod za výsledné číslo.
Nájdenie čísla bez zdôvodnenia jednotlivých krokov ohodnoťte 2~bodmi.
\endhodnotenie}

{%%%%%   Z9-III-1
Označme počet zlatiek $z$~a~počet strieborniakov~$s$. Zlatky boli
darované po dvoch, teda $z$ je párne číslo, strieborniaky po troch, teda
$s$ je deliteľné tromi. Pritom $s$ je medzi $40$ a~$100$. Minister dostal
60~sošiek a~$\frac17{z}+\frac13{s}$ mincí, kráľ dostal
$\frac67{z}+\frac23{s}$ mincí. Z~toho vyplýva, že $z$ musí byť
deliteľné siedmimi, čo s~predchádzajúcou podmienkou znamená, že $z$ musí byť
deliteľné~$14$. Kráľ aj jeho minister dostali rovnaký počet predmetov, čo
pri našom označení dáva rovnicu
$$
60+\frac{z}7+\frac{s}3=\frac{6z}7+\frac{2s}3,
\tag1
$$
po úprave
$$
\align
\frac{5z}7+\frac{s}3&=60,\\
15z+7s&=7\cdot3\cdot60.\tag2
\endalign
$$

Hľadáme všetky celočíselné riešenia rovnice~\thetag2, ktoré vyhovujú vyššie
uvedeným podmienkam. Keď vyjadríme neznámu~$z$,
$$
z=84-\frac{7s}{15},
\tag3
$$
vidíme, že $s$ musí byť deliteľné $15$.
V~úvode sme si všimli, že $z$ má byť párne, teda aj hodnota výrazu
$\frac{7}{15}s$ musí byť párna, čo znamená, že také $s$ musí byť párne
číslo. Preto $s$~musí byť deliteľné~$30$, čo vzhľadom na podmienku
$40<s<100$ znamená, že $s$ môže byť jedine $60$ a~$90$. Pre obe tieto
možnosti dopočítame $z$ podľa \thetag3, počet poddaných potom určíme ako
$60+\frac13{s}+\frac12{z}$. Úloha má nasledujúce dve riešenia (uvedomte
si, že všetky vyššie spomenuté podmienky sú splnené):
$$
\begintable
strieborniakov|zlatiek|poddaných\crthick
60|56|108\cr
90|42|111%
\endtable
$$


\ineriesenie
Rovnicu \thetag2 možno riešiť aj tak, že vyjadríme neznámu~$s$:
$$
s=180-\frac{15z}7.
$$
Z~úvodu vieme, že $z$ musí byť deliteľné 14, preto
za $z$ postupne dosadzujeme násobky 14 a~sledujeme, či platí
$40<s<100$.
Ak áno, vypočítame počet poddaných ako $60+\frac13{s}+\frac12{z}$
(uvedomte si, že ostatné vyššie uvedené podmienky sú splnené):
$$
\begintable
zlatiek|strieborniakov|poddaných\crthick
14|150|\cr
28|120|\cr
\bf42|\bf90|\bf111\cr
\bf56|\bf60|\bf108\cr
70|30|
\endtable
$$
Ďalej nepokračujeme, pretože $s$ bude vychádzať menšie ako $40$;
úloha má dve riešenia.

\hodnotenie
2~body za zostavenie rovnice \thetag1 a~jej následnú úpravu;
2~body za nájdenie jednej možnosti;
1~bod za nájdenie druhej možnosti;
1~bod za zdôvodnenie, že riešení nie je viac.
\endhodnotenie
}

{%%%%%   Z9-III-2
Najmenšie dopĺňané číslo označíme neznámou~$a$ a~pomocou nej vyjadríme
všetky dopĺňané čísla:
$$
a,\ 2a,\ 4a,\ 8a,\ 16a,\ 32a.
$$
V~prvom stĺpci nemôže byť číslo $32a$, pretože súčet ktorýchkoľvek iných dvoch
dopĺňaných čísel je menší než $32a$, a~teda by súčet čísel v~druhom
stĺpci nemohol byť väčší ako v~prvom.
Možné súčty čísel prvého stĺpca sú:
\bgroup
\thinsize=0pt
\thicksize=0pt
\def\tstrut{\vrule height \baselineskip depth 8pt}
$$
\begintable
$a+2a=3a$|$2a+4a=6a$|$4a+8a=12a$|$8a+16a=24a$\cr
$a+4a=5a$|$2a+8a=10a$|$4a+16a=20a$|\cr
$a+8a=9a$|$2a+16a=18a$||\cr
$a+16a=17a$|||
\endtable
$$\egroup
Prirodzené číslo, ktoré v~týchto súčtoch násobí neznámu~$a$, musí byť
deliteľom čísla $136=2\cdot2\cdot2\cdot17$.
Tomu vyhovuje jedine súčet
$$
17a=a+16a.
$$
Súčet čísel druhého stĺpca je potom $34a$ a~tento možno získať jedine ako
$$
34a=2a+32a.
$$
Na doplnenie do tretieho stĺpca teda ostávajú čísla $4a$ a~$8a$.
Z~rovnice $17a=136$ dostaneme $a=8$;
súčet čísel v~treťom stĺpci je teda
$$
4a+8a=12a=12\cdot8=96.
$$

\hodnotenie
3~body za poznatok, že v~prvom stĺpci je $a$ a~$16a$,
vrátane zdôvodnenia, že sa jedná o~jedinú možnosť;
1~bod za poznatok, že v~treťom stĺpci je $4a$ a~$8a$;
2~body za výsledok $96$.
\endhodnotenie
}

{%%%%%   Z9-III-3
Bod $A$ má byť priesečníkom výšok v~trojuholníku $BDX$.
To znamená, že na spojnici bodu~$A$ s~vrcholom~$B$ leží výška na stranu
$DX$, podobne na spojnici $A$ s~vrcholom~$D$ leží výška na stranu~$BX$.
Strana~$DX$ je teda kolmá na priamku~$AB$, podobne strana~$BX$ je kolmá na
priamku~$AD$ (\obr).
\insp{z62ii.105}%

Vzhľadom na to, že body $A$, $B$ a~$D$ sú vrcholmi pravidelného
osemuholníka, platí:
\begin{enumerate}
  \item Kolmica na $AB$ prechádzajúca bodom~$D$ je priamka~$CD$.
    (Vonkajší uhol pravidelného osemuholníka má veľkosť $45\st$.
    Uhol medzi $AB$ a~$CD$ je uhlom pri vrchole~$P$ v~trojuholníku $BPC$, a~preto je pravý.)
  \item Kolmica na $AD$ predchádzajúca bodom~$B$ je priamka~$BG$.
    ($AD$ a~$BG$ sú uhlopriečky rovnobežné so stranami $BC$ a~$AH$.
    Tieto sú rovnako ako strany $AB$ a~$CD$ kolmé, a~preto sú uvedené
    uhlopriečky kolmé.)
  \item Bod~$X$ je priesečníkom priamok $CD$ a~$BG$.
\end{enumerate}

Teraz už ľahko určíme veľkosti všetkých uhlov v~trojuholníku $BDX$.
Trojuholník $BCD$ je rovnoramenný s~ramenami $BC$ a~$CD$, pričom uhol $BCD$
je vnútorným uhlom pravidelného osemuholníka.
Odtiaľ dopočítame
$$
|\angle CDB|=|\angle CBD|=\frac12(180\st-135\st)=22{,}5\st.
$$
Priamky $AD$ a~$BC$ sú rovnobežné, uhol medzi $AD$ a~$BG$ je pravý, preto
aj uhol $CBX$ je pravý.
Odtiaľ vyjadríme
$$
|\angle DBX|=22{,}5\st+90\st=112{,}5\st.
$$
Posledný neznámy vnútorný uhol v~trojuholníku $BDX$ má veľkosť
$$
|\angle DXB|=180\st-22{,}5\st-112{,}5\st=45\st.
$$

\hodnotenie
3~body za nájdenie bodu~$X$; po 1~bode za hľadané uhly vrátane zdôvodnenia.

\poznamka
Podobnú úlohu poznáme z~domáceho kola ({\bf Z9--I--4}), takže úvodný
rozbor je možné zostručniť: $X$ je priesečníkom výšok v~trojuholníku $ABD$.

V~riešení niekoľkokrát používame zásadný poznatok, že súčet veľkostí
vnútorných uhlov v~ľubovoľnom trojuholníku je $180\st$.
Z~toho vyplýva aj to, že veľkosť vnútorného uhla v~pravidelnom osemuholníku
je $135\st$.
\endhodnotenie
}

{%%%%%   Z9-III-4
V~slove TESTOVANIE je 8 rôznych písmen.
Môžeme používať len nenulové cifry, vyberáme teda osem cifier z~deviatich
možných.
Súčin
$$
T^2\cdot E^2\cdot S\cdot O\cdot V\cdot A\cdot N\cdot I
$$
má byť druhou mocninou prirodzeného čísla.
Cifry $T$ a~$E$ sú v~súčine v~druhej mocnine,
stačí preto zabezpečiť, aby súčin
$$
S\cdot O\cdot V\cdot A\cdot N\cdot I
\tag1
$$
bol druhou mocninou prirodzeného čísla,
\tj. aby v~jeho prvočíselnom rozklade boli všetky prvočísla v~párnej
mocnine.
Uvažujme, ktoré činitele môžeme dosadzovať:
$$
\begintable
1|2|3|4|5|6|7|8|9\cr
1|2|3|$2\cdot2$|5|$2\cdot3$|7|$2\cdot2\cdot2$|$3\cdot3$
\endtable
$$
Cifry $5$ a~$7$ nemôžeme v~súčine \thetag1 použiť, pretože ich nemáme čím doplniť
do druhej mocniny.
Číslo, ktoré vznikne nahradením písmen v~slove
TESTOVANIE, má byť čo najväčšie, preto sa snažíme postupne dopĺňať čo
najväčšie hodnoty za $T$, $E$, $S$ atď.

Ak za $T$ zvolíme $9$, ostane pre súčin \thetag1 šestica cifier
$8$, $6$, $4$, $3$, $2$, $1$.
Po dosadení tento súčin vychádza $2^7\cdot3^2$, čo nevyhovuje vyššie
uvedenej požiadavke.

Ak za $T$ zvolíme $8$, ostane pre súčin \thetag1 šestica
$9$, $6$, $4$, $3$, $2$, $1$,
ktorá po dosadení dáva $2^4\cdot3^4$, čo uvedenej požiadavke vyhovuje.
Týmito ciframi nahradíme $S$, $O$, $V$, $A$, $N$,~$I$ práve v~tomto zostupnom poradí.
Pre $A$ potom ostávajú cifry $5$ alebo $7$~-- vyberieme väčšie z~nich.
Hľadané číslo je
$$
8\,798\,643\,217.
$$

\hodnotenie
2~body za poznatok, že cifry $5$ a~$7$ nemôžu byť v~súčine \thetag1;
2~body za zdôvodnenie, že treba zvoliť $T=8$;
2~body za výsledné číslo.
\endhodnotenie
}

