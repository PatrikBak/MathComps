{%%%%%   A-I-1
Pre dané kladné čísla $x\ne y$ uvažujme priemery
$$
a=\frac{x+y}2,\ g=\sqrt{xy},\ h=\frac{2xy}{x+y},\
k=\sqrt{\frac{x^2+y^2}2}.
$$
(Ide o~aritmetický, geometrický, harmonický a~kvadratický
priemer čísel $x$ a~$y$.)
Zo všetkých rozdelení štvorice $a$, $g$, $h$, $k$ na dve dvojice
$r$, $s$ a~$t$, $u$
vyberte to rozdelenie, pri ktorom má výraz $V=r+s-t-u$  najmenšiu
kladnú hodnotu.
}
\podpis{J. Šimša}

{%%%%%   A-I-2
V~priestore je daná kocka $ABCDEFGH$. Uvažujme
ľubovoľnú rovinu, ktorá prechádza bodom $B$ a~dotýka sa gule
vpísanej danej kocke, a~označme $P$, $Q$ jej priesečníky s~hranami
$EF$, $GF$. Dokážte, že odchýlka rovín $BPH$ a~$BQH$ je $60\st$.
}
\podpis{P. Leischner}

{%%%%%   A-I-3
Zistite, pre  ktoré $b$ je obor hodnôt funkcie
$f(x)=x^4+x^3-2x^2+bx$  interval $\langle b,\infty)$.
}
\podpis{P. Černek}

{%%%%%   A-I-4
V~rovine sú dané kružnice $k_1(S_1,3\,{\text{cm}})$
a~$k_2(S_2,4\,{\text{cm}})$, ktoré  majú vnútorný dotyk v~bode $A$.
Ďalej je daný bod $S$ vnútri kružnice $k_1$. Zostrojte trojuholník
$ABC$ tak, aby jeho strana $BC$ bola tetivou kružnice~$k_2$
a~zároveň dotyčnicou kružnice $k_1$, a~aby bod $S$ bol stredom kružnice
vpísanej trojuholníku $ABC$.
}
\podpis{P. Leischner}

{%%%%%   A-I-5
Uvažujme trojuholník $ABC$ s~ostrými uhlami $\alpha$,  $\gamma$
pri~vrcholoch $A$, $C$ a~s~nasledujúcou vlastnosťou:  ťažnica z~vrcholu $A$
a~výška z~vrcholu $B$ sa pretínajú v~bode, ktorý leží na osi uhla
pri vrchole $C$. Dokážte, že potom platí  $\tg\alpha=\tg^2\gamma\cdot
\tg\frac{\gamma}2$.
}
\podpis{J. Šimša}

{%%%%%   A-I-6
Určte najväčší možný počet $1\,994$-ciferných prirodzených čísel,
ktoré sa navzájom líšia poradím  číslic.
}
\podpis{J. Šimša}

{%%%%%   B-I-1
Určte všetky dvojice reálnych čísel $p$, $q$ takých, že
rovnici
$$
 x^4 + q^2(x+p) = p^2(x+q)^2
$$
vyhovujú
práve tri rôzne reálne čísla, pričom súčet týchto troch čísel
je rovný nule.
}
\podpis{J. Šimša}

{%%%%%   B-I-2
Každému bodu štvorca so~stranou~ $1$ je priradená práve jedna z
troch farieb. Dokážte, že pri ľubovoľnom takomto ofarbení môžeme
vo štvorci nájsť dva body rovnakej farby, ktorých vzdialenosť je
aspoň $1{,}007$.
}
\podpis{M. Čadek}

{%%%%%   B-I-3
Pre dané kladné čísla $x\ne y$ uvažujme priemery
$$
a=\frac{x+y}2,\ g=\sqrt{xy},\ h=\frac{2xy}{x+y},\
k=\sqrt{\frac{x^2+y^2}2}.
$$
Zo všetkých rozdelení štvorice $a$, $g$, $h$, $k$ na dve
dvojice $r$, $s$ a~$t$, $u$ vyberte to rozdelenie, pre ktoré má
výraz $V = rs - tu$ najmenšiu kladnú hodnotu.
}
\podpis{J. Šimša}

{%%%%%   B-I-4
V~rovine sú dané priamky $a$, $b$ zvierajúce uhol veľkosti
$28\st$.  Určte všetky $n$, pre ktoré existuje konvexný
$n$-uholník súmerný  ako podľa priamky $a$, tak podľa priamky $b$.
}
\podpis{P. Černek}

{%%%%%   B-I-5
Nájdite obor hodnôt funkcie
$$
f(x)=x-\sqrt{x^2-3x-10}.
$$
}
\podpis{P. Černek}

{%%%%%   B-I-6
Uvažujme trojboký ihlan $ABCD$ s~hranami  $|AC| = |AD| =6$\,cm,
$|BC| =4 $\,cm,  $|CD| =2\sqrt 6 $\,cm, ktorého podstavou
je pravouhlý trojuholník $ABC$ s~preponou $AB$.  Určte výšku
ihlanu, ak
hrana $BD$ zviera so svojím kolmým priemetom do roviny podstavy uhol
veľkosti~ $45\st$.
}
\podpis{R. Kollár}

{%%%%%   C-I-1
Určte všetky štvorciferné čísla deliteľné~$4$, pre ktoré platí:
Ak v~čísle vymeníme prvé dve cifry, dostaneme číslo
deliteľné~$7$. Ak v~čísle vymeníme prostredné dve cifry,
dostaneme číslo deliteľné~$5$. Ak v~čísle vymeníme posledné
dve cifry, dostaneme číslo deliteľné~$9$.
}
\podpis{P. Černek}

{%%%%%   C-I-2
Daná je polokružnica so stredom~$S$ zostrojená nad priemerom~$AB$. Zostrojte takú jej dotyčnicu~$t$ s~dotykovým bodom~$T$
($A\ne T\ne B$), aby platilo $P_{BCS}=2P_{DAT}$, kde $P_{XYZ}$
označuje obsah trojuholníka $XYZ$, a~kde body $D$, $C$ sú po rade päty
kolmíc spustených z~bodov $A$, $B$ na priamku~$t$.
}
\podpis{J. Švrček}

{%%%%%   C-I-3
Každý bod obvodu štvorca so~stranou $10$\,cm je ofarbený jednou z
dvoch farieb. Dokážte, že pri ľubovoľnom ofarbení môžeme na obvode
štvorca vždy nájsť body rovnakej farby tak, aby trojuholník s~týmito
vrcholmi mal obsah aspoň $25$\,cm$^2$.
}
\podpis{M. Čadek}

{%%%%%   C-I-4
Je daný štvorsten $ABCD$ taký, že $|AC|=5$\,cm, $|BC|=8$\,cm,
$|CD|=5\sqrt2$\,cm a~$|AD|=5\sqrt3$\,cm. Určte veľkosť výšky
prechádzajúcej vrcholom $D$, ak jeho stena $ABC$ je pravouhlý
trojuholník s~preponou $AB$ a~hrana $BD$ zviera so svojím kolmým priemetom
do roviny $ABC$ uhol veľkosti $45\st$.
}
\podpis{P. Leischner}

{%%%%%   C-I-5
Mnohosten nakreslený na \obr{} má $2n+1$ vrcholov. Každému z~nich
je priradené prirodzené číslo tak, že súčty čísel vo vrcholoch
každej z~$2n+1$ stien sú rovnaké. Určte $n$, ak viete, že medzi
použitými číslami sú $7$, $8$, $9$, $216$.
\insp{c44.1}%
}
\podpis{P. Černek}

{%%%%%   C-I-6
V~rovine je narysovaný trujuholník $ABC$. Popíšte postup, ako sa pomocou
kružidla na čo najmenej krokov presvedčit o~tom,
že $|\uh BAC|=40\st$ a~$|\uh ABC|=56\st$.

Pritom za krok považujeme každé zapichnutie alebo priloženie
kružidla. Napríklad na~narysovanie kružnice alebo oblúku stačí
jeden krok, na~porovnanie dĺžok dvoch úsečiek so~spoločným krajným
bodom stačí rovnako jeden krok, zatiaľ čo na~porovnanie dĺžok dvoch
úsečiek bez spoločného krajného bodu sú potrebné dva kroky.
}
\podpis{J. Šimša}

{%%%%%   A-S-1
V~rovine sú dané kružnice
$k_1(S_1,r_1) \text{\ a\ } k_2(S_2,r_2)$, pretínajúce sa v dvoch bodoch
$A$ a $B$, pričom
$ |S_1S_2|> \nomathbreak r_2\geqq \nomathbreak r_1$.
Zostrojte body $X\in k_1$ a
$Y\in k_2$ tak, aby bod $A$ ležal vnútri úsečky $XY$
a aby trojuholník $BXY$ mal čo najväčší obsah.
}
\podpis{J. Šimša}

{%%%%%   A-S-2
Nájdite všetky funkcie $f$ tvaru $f(x)=\sqrt{ax^2+bx+c}\,$
s~reálnymi koeficientami $a$, $b$, $c$ a s nasledujúcou vlastnosťou:
definičný obor a obor hodnôt funkcie $f$ sú dve rovnaké neprázdne
množiny. (Za definičný obor funkcie $f$ považujeme množinu {\it všetkých\/} reálnych čísel $x$, pre ktoré má výraz $\sqrt{ax^2+bx+c}$
zmysel.)
}
\podpis{J. Šimša}

{%%%%%   A-S-3
Určte kladné reálne čísla $x\ne y$ také, že všetky štyri
ich priemery
$$
a=\frac{x+y}2,\quad g=\sqrt{xy},\quad h=\frac{2xy}{x+y},\quad
k=\sqrt{\frac{x^2+y^2}2}
$$
ležia v~množine $M=\left\{\,\dfrac{45}2,\,18\sqrt2,\,30,\,
25\sqrt2,\,40,\,10\sqrt{23}\,\right\}$.
}
\podpis{J. Šimša}

{%%%%%   A-II-1
Koľko pätnásťmiestnych čísel zložených len z~číslic $3$ a $8$
je deliteľných jedenástimi?
}
\podpis{P. Černek}

{%%%%%   A-II-2
Daný je trojuholník $ABC$ s uhlom  veľkosti
$105^o$  pri~vrchole $C$. Určte
veľkosti  zostávajúcich dvoch vnútorných uhlov, ak viete, že ťažnica
vedená z~vrcholu $A$ pretne os uhla pri~vrchole $B$ v~bode, ktorý leží
na osi strany $AB$.
}
\podpis{J. Šimša}

{%%%%%   A-II-3
Daný je štvorsten $ABCD$, pre ktorý platí
$$
|AB|=2a,\ |CD|=2b,\ |AC|=|AD|=|BC|=|BD|=c .
$$
Určte polomer guľovej plochy vpísanej štvorstenu $ABCD$.
}
\podpis{P. Leischner}

{%%%%%   A-II-4
Nájdite všetky mnohočleny $f$ s~reálnymi koeficientami
také, že pre každé reálne číslo $x$  platí nerovnosť
$$
f(x)\cdot x\cdot f(1-x)+x^3+100\geqq 0\,.
$$
}
\podpis{P. Hliněný}

{%%%%%   A-III-1
Daný je štvorsten $ABCD$, pre ktorý platí
$$
|\angle BAC|+|\angle CAD|+|\angle DAB|=
|\angle ABC|+|\angle CBD|+|\angle DBA|= 180{\st}.
$$
Dokážte, že $|CD|\geqq|AB|$.
}
\podpis{P. Leischner}

{%%%%%   A-III-2
Určte kladné reálne čísla $x$ a~$y$, ak viete, že priemery
$$
a=\frac{x+y}2,\quad g=\sqrt{xy},\quad h=\frac{2xy}{x+y},\quad
k=\sqrt{\frac{x^2+y^2}2}
$$
sú prirodzené čísla, ktorých súčet sa rovná~ 66.
}
\podpis{J. Šimša}

{%%%%%   A-III-3
V~rovine je daných päť rôznych bodov a~päť rôznych priamok. Dokážte,
že z~nich možno vybrať dva rôzne body a~dve rôzne priamky tak, aby
žiadny z~vybraných bodov neležal na žiadnej z~vybraných
priamok.
}
\podpis{P. Hliněný}

{%%%%%   A-III-4
Rozhodnite, či existuje 10\,000 desaťmiestnych čísel
deliteľných siedmimi, ktoré sú zapísané rovnakou skupinou desiatich
číslic v~rôznych poradiach.
}
\podpis{J. Šimša}

{%%%%%   A-III-5
Na kružnici $k$ so~stredom $S$ sú dané body $A$ a~$B$
tak, že tetivu $AB$ je z~bodu $S$ vidieť pod uhlom $90{\st}$.
Kružnice $k_1$, $k_2$ sa dotýkajú zvnútra
kružnice $k$ po rade v~bodoch $A$, $B$ a~naviac sa navzájom
zvonku dotýkajú v~bode $Z$. Kružnica $k_3$ ležiaca vnútri uhla $ASB$
sa dotýka (zvnútra) kružnice $k$ v~bode $C$ a~(zvonku) kružníc
$k_1$, $k_2$
po rade v~bodoch $X$, $Y$. Dokážte, že úsečku $XY$ je z~bodu $C$ vidieť pod
uhlom $45{\st}$.
}
\podpis{M. Engliš}

{%%%%%   A-III-6
Pre ktoré reálne čísla $p$ má rovnica
$$
x^3-2p(p+1)x^2+(p^4+4p^3-1)x-3p^3 =0
%\tag1
$$
tri rôzne korene, ktoré sú dĺžkami strán nejakého pravouhlého
trojuholníka?
}
\podpis{J. Šimša}

{%%%%%   B-S-1
Rovnica
$$
 2x^3 - 9x^2 + 7x + m = 0
$$
má tri rôzne reálne korene. Súčin dvoch z nich je rovný
$\m 1$. Určte číslo $m$ a~korene rovnice.}
\podpis{P.Černek, P.Leischner}

{%%%%%   B-S-2
V~trojuholníku $ABC$ označme $M$ stred strany $AB$ a~$S$ stred
úsečky $CM$. Vo vnútri úsečky $MS$ volíme postupne na~rôznych miestach
bod $O$ a~zisťujeme obvod prieniku trojuholníka $ABC$ s~jeho
obrazom v stredovej súmernosti so stredom $O$. Aký musí platiť
vzťah medzi dĺžkami strán trojuholníka $ABC$, aby tento obvod
nezávisel od voľby bodu $O$?
}
\podpis{P. Leischner}

{%%%%%   B-S-3
V~rovine je daných $n$ bodov. Ak ich navzájom pospájame
priamkami, prechádzajú tieto priamky danými bodmi a
vytvárajú aj ďalšie priesečníky. Dokážte, že počet týchto nových
priesečníkov nie je väčší ako
$$
 \frac18n(n-1)(n-2)(n-3) .
$$
}
\podpis{P. Hliněný}

{%%%%%   B-II-1
Určte všetky reálne čísla $a$, pre ktoré existuje práve jedna
usporiadaná dvojica $[x,y]$ reálnych čísel takých, že
$$
x + \dfrac{1}{y} - \dfrac{y}{x} = y + \dfrac{1}{x} - \dfrac{x}{y} = a
$$
}
\podpis{P. Černek}

{%%%%%   B-II-2
Pre dané kladné čísla $x\ne y$ označme
$$
 a~= \frac12(x+y),\quad g = \sqrt{xy},\quad k~=
\sqrt{\frac12(x^2+y^2)}.
$$
Rozhodnite, pri ktorom zo šiestich možných poradí $r$, $s$, $t$
čísel $a$, $g$, $k$ má výraz $V = \frac{r-s}t$ najmenšiu kladnú
hodnotu.
}
\podpis{J. Šimša}

{%%%%%   B-II-3
Každému bodu jednotkovej kocky priradíme jednu z dvoch farieb.
Dokážte, že pri ľubovoľnom takomto ofarbení existujú v~kocke
dva body rovnakej farby, ktorých vzdialenosť je aspoň $d = \frac32$.
Platí toto tvrdenie aj~pre niektoré $d > \frac32 $?
}
\podpis{M. Čadek}

{%%%%%   B-II-4
Uhlopriečky daného tetivového štvoruholníka $ABCD$ sú navzájom
kolmé a~pretínajú sa v~bode~ $E$. Označme $M$ priesečník kolmice
z~bodu $E$ na stranu $AB$ s~protiľahlou stranou $CD$. Porovnajte
obsahy trojuholníkov $CME$ a~$MDE$.}
\podpis{P. Leischner}

{%%%%%   C-S-1
Určte všetky trojice celých nezáporných čísel $a$, $b$, $c$, ktoré
vyhovujú sústave rovníc
$$\align
     a+bc &=3c, \\
     b+ca &=3a, \\
     c+ab &=3b .
 \endalign
$$
}
\podpis{J. Švrček}

{%%%%%   C-S-2
V~rovine je daný štvorec $ABCD$ so stredom $S$.
Vo vnútri úsečiek $SA$ a~$SC$ sú zvolené po rade body $E$ a~$F$
tak, že $|SE|=|SF|$. Zostrojme priesečník $X$ polpriamky $BE$ so
stranou $AD$ a~priesečník $Y$ polpriamky $DF$ s~predĺžením
strany $AB$. Dokážte, že obsah trojuholníka $AXY$ nezávisí od
polohy bodov $E$ a~$F$.
}
\podpis{J. Šimša}

{%%%%%   C-S-3
V~rovine sú dané dve úsečky, ktoré sa navzájom nepretínajú.
Navrhnite postup, ako zistiť, či sú rovnobežné. K~dispozícii
máte len kružidlo, ktorého maximálny polomer je menší ako
vzdialenosť ľubovoľných dvoch bodov, z~ktorých každý patrí inej z~oboch
úsečiek.
}
\podpis{P. Hliněný}

{%%%%%   C-II-1
Určte počet všetkých štvorciferných čísel~$n$ s~vlastnosťou: Ak
k~číslu~$n$ pripočítame štvorciferné číslo~$n'$, ktoré má
v~desiatkovej sústave opačné poradie cifier ako číslo~$n$,
dostaneme číslo, ktoré je deliteľné číslom~$70$.}
\podpis{J. Švrček}

{%%%%%   C-II-2
Určte všetky reálne čísla $a$, pre ktoré má sústava rovníc
$$
x^2 -2y=y^2-2x=a
$$
jediné riešenie. (Riešením rozumieme usporiadanú dvojicu
$[x,y]$ reálnych čísel vyhovujúcu sústave rovníc.)
}
\podpis{L. Boček}

{%%%%%   C-II-3
V~rovine je daný rovnostranný trojuholník $ABC$ a~priamky $p_A$,
$p_B$, ktoré sú kolmé na~$AB$ a~prechádzajú po rade bodmi $A$,
$B$. Zostrojte pravouhlý trojuholník $KLC$ s~preponou $KL$, ktorý
má rovnaký obsah ako trojuholník $ABC$, a~pritom jeho vrcholy $K$, $L$
ležia po rade na priamkach $p_A$, $p_B$.
}
\podpis{J. Švrček}

{%%%%%   C-II-4
Každému bodu jednotkovej kocky je priradená jedna zo štyroch farieb.
Dokážte, že pri ľubovoľnom takomto ofarbení existujú v~kocke
dva body rovnakej farby, ktorých vzdialenosť je aspoň $\frac 12
\sqrt{5}$.
}
\podpis{M. Čadek}

{%%%%%   vyberko, den 1, priklad 1
...}
\podpis{...}

{%%%%%   vyberko, den 1, priklad 2
...}
\podpis{...}

{%%%%%   vyberko, den 1, priklad 3
...}
\podpis{...}

{%%%%%   vyberko, den 1, priklad 4
...}
\podpis{...}

{%%%%%   vyberko, den 2, priklad 1
...}
\podpis{...}

{%%%%%   vyberko, den 2, priklad 2
...}
\podpis{...}

{%%%%%   vyberko, den 2, priklad 3
...}
\podpis{...}

{%%%%%   vyberko, den 2, priklad 4
...}
\podpis{...}

{%%%%%   vyberko, den 3, priklad 1
...}
\podpis{...}

{%%%%%   vyberko, den 3, priklad 2
...}
\podpis{...}

{%%%%%   vyberko, den 3, priklad 3
...}
\podpis{...}

{%%%%%   vyberko, den 3, priklad 4
...}
\podpis{...}

{%%%%%   vyberko, den 4, priklad 1
...}
\podpis{...}

{%%%%%   vyberko, den 4, priklad 2
...}
\podpis{...}

{%%%%%   vyberko, den 4, priklad 3
...}
\podpis{...}

{%%%%%   vyberko, den 4, priklad 4
...}
\podpis{...}

{%%%%%   vyberko, den 5, priklad 1
...}
\podpis{...}

{%%%%%   vyberko, den 5, priklad 2
...}
\podpis{...}

{%%%%%   vyberko, den 5, priklad 3
...}
\podpis{...}

{%%%%%   vyberko, den 5, priklad 4
...}
\podpis{...}

{%%%%%   trojstretnutie, priklad 1
...}
\podpis{...}

{%%%%%   trojstretnutie, priklad 2
...}
\podpis{...}

{%%%%%   trojstretnutie, priklad 3
...}
\podpis{...}

{%%%%%   trojstretnutie, priklad 4
...}
\podpis{...}

{%%%%%   trojstretnutie, priklad 5
...}
\podpis{...}

{%%%%%   trojstretnutie, priklad 6
...}
\podpis{...}

{%%%%%   IMO, priklad 1
...}
\podpis{...}

{%%%%%   IMO, priklad 2
...}
\podpis{...}

{%%%%%   IMO, priklad 3
...}
\podpis{...}

{%%%%%   IMO, priklad 4
...}
\podpis{...}

{%%%%%   IMO, priklad 5
...}
\podpis{...}

{%%%%%   IMO, priklad 6
...}
\podpis{...}

{%%%%%   MEMO, priklad 1
}
\podpis{}

{%%%%%   MEMO, priklad 2
}
\podpis{}

{%%%%%   MEMO, priklad 3
}
\podpis{}

{%%%%%   MEMO, priklad 4
}
\podpis{}

{%%%%%   MEMO, priklad t1
}
\podpis{}

{%%%%%   MEMO, priklad t2
}
\podpis{}

{%%%%%   MEMO, priklad t3
}
\podpis{}

{%%%%%   MEMO, priklad t4
}
\podpis{}

{%%%%%   MEMO, priklad t5
}
\podpis{}

{%%%%%   MEMO, priklad t6
}
\podpis{}

{%%%%%   MEMO, priklad t7
}
\podpis{}

{%%%%%   MEMO, priklad t8
}
\podpis{}
