{%%%%%   Z4-I-1
Na stole so štvorcovou doskou o strane $1\,\text{m}$ bola "trochu nakrivo" umiestnená kruhová dečka.
Od najbližšej strany dosky stola bol jej kraj vzdialený $10\cm$, od susednej strany potom $20\cm$
a od najvzdialenejšej strany $40\cm$.
\begin{itemize}
\itemvar{a)} Ako ďaleko bol okraj dečky od štvrtej strany dosky stola?
\itemvar{b)} Aký polomer mala dečka?
\end{itemize}
}
\podpis{S. Bednářová}

{%%%%%   Z4-I-2
Jožo Nudilsa sa zabával tým, že písal za sebou postupne prirodzené čísla. Začal jednotkou: $1234567891011\dots$ Po čase ho to prestalo baviť a kriticky sa pozrel na svoj výtvor. Zistil, že v postupnosti číslic, ktoré napísal, sa vyskytujú iba raz tri päťky priamo za sebou.
\begin{itemize}
\itemvar{a)} Najmenej koľko za sebou idúcich prirodzených čísel napísal Jožo?
\itemvar{b)} Najmenej koľko číslic napísal Jožo?
\end{itemize}
}
\podpis{S. Bednářová}

{%%%%%   Z4-I-3
Bývam v Tomášovciach, ale pracujem v Rimavskej Sobote. Autobus, ktorým do práce cestujem, má nasledujúce zastávky (v uvedenom poradí): Tomášovce, Bátka, Rokytník, Bátka, Bakta, Vinica, Rimavská Sobota. Z Bátky do Rimavskej Soboty cez Baktu a Vinicu je to po ceste $11\text{ km}$, z Rokytníka cez Bátku a Baktu do Vinice $12\text{ km}$, z Bátky cez Baktu do Vinice $9\text{ km}$. Z Tomášoviec do Bátky je to rovnako ďaleko ako z Vinice do Rimavskej Soboty.
\begin{itemize}
\itemvar{a)} Koľko km prejde autobus z Tomášoviec do Rimavskej Soboty touto trasou?
\itemvar{b)} Koľko km by to bolo z Tomášoviec do Rimavskej Soboty, keby autobus nezachádzal do Rokytníka?
\end{itemize}
}
\podpis{S. Bednářová}

{%%%%%   Z4-I-4
Doplň do prázdnych políčok na \obr{} prirodzené čísla od $1$ do $16$ (každé číslo môžeš použiť len raz)
tak, aby platili matematické vzťahy.
\insp{z58.41}%
}
\podpis{M. Smitková}

{%%%%%   Z4-I-5
Paľko s Radkou si kupujú spolu cukríky. Pri poslednom nákupe platil Paľko 92\,Sk za 5 balení z dvoch druhov cukríkov. Sám si vzal z každého druhu po jednom balení a Radka dostala jedno balenie gumených a dve balenia čokoládových cukríkov. Jej nákup bol tak o 20\,Sk drahší ako Paľkov.
\begin{itemize}
\itemvar{a)} Koľko korún má za nákup dať Radka Paľkovi?
\itemvar{b)} Koľko stojí jedno balenie gumených cukríkov?
\end{itemize}
}
\podpis{M. Dillingerová}

{%%%%%   Z4-I-6
Danko si zo štvorčekovej siete vystrihol útvar ako na \obr{}.
Odstrihni dva štvorčeky siete tak, aby sa
výsledný útvar nerozpadol a aby mal čo
najväčší obvod.
Nájdi dve riešenia.
\insp{z58.42}%
}
\podpis{M. Dillingerová}

{%%%%%   Z5-I-1
Učiteľka Kadrnožková kupovala v pokladni zoologickej záhrady vstupenky pre svojich
žiakov a pre seba. Vstupenka pre dospelého bola drahšia ako pre školáka, ale nie viac ako
dvakrát. Učiteľka Kadrnožková zaplatila 994\,Sk. Učiteľ Hniezdo mal so sebou o troch žiakov
viac ako učiteľka Kadrnožková, a za svojich žiakov a za seba zaplatil 1120\,Sk.
\begin{itemize}
\itemvar{a)} Koľko žiakov mal so sebou učiteľ Hniezdo?
\itemvar{b)} Koľko stála vstupenka pre dospelého?
\end{itemize}
}
\podpis{L. Šimůnek}

{%%%%%   Z5-I-2
Fero Nudilsa sa zabával tým, že písal za sebou idúce prirodzené čísla. Začal
jednotkou: $1234567891011\dots$ Po čase ho to prestalo baviť, dokončil práve rozpísané
číslo a kriticky sa pozrel na svoj výtvor. Zistil, že v postupnosti číslic, ktoré napísal, sa
vyskytuje päť jednotiek za sebou.
\begin{itemize}
\itemvar{a)} Najmenej koľko za sebou idúcich prirodzených čísel napísal Fero?
\itemvar{b)} Najmenej koľko číslic napísal Fero?
\end{itemize}
}
\podpis{S. Bednářová}

{%%%%%   Z5-I-3
Najvyššia známa sopka na zemeguli je Mauna Kea na Havajských ostrovoch. Jej
výška od úpätia po vrchol je dokonca o 358 metrov väčšia, ako je nadmorská výška
najvyššej hory sveta, Mont Everestu. Nedvíha sa však z pevniny, ale z dna Tichého
oceánu, z 5000 metrovej hĺbky. Keby morská hladina v tejto oblasti klesla o 397
metrov, bola by ponorená časť Mauna Key presne rovnako vysoká, ako časť, ktorá by
vyčnievala nad hladinu.
\begin{itemize}
\itemvar{a)} Akú nadmorskú výšku má vrchol sopky?
\itemvar{b)} Koľko meria Mauna Kea od úpätia po vrchol?
\itemvar{c)} Akú nadmorskú výšku má Mont Everest?
\end{itemize}
\noindent(Údaje o nadmorských výškach uvádzané v rôznych literatúrach sa môžu líšiť. Je to
spôsobené jednak nepresnosťami niektorých meraní, jednak pohybmi zemskej kôry --
tieto výšky sa skutočne menia! Pri riešení úlohy preto vychádzaj len z údajov
uvedených v úlohe.)}
\podpis{S. Bednářová}

{%%%%%   Z5-I-4
Klasická hracia kocka sa kotúľala naznačeným smerom po pláne
na \obr. Pri jej pohybe na každom políčku ostali otlačené
bodky zo steny, ktorou sa plánu dotýkala. Súčet všetkých bodiek
otlačených na pláne bol $23$. Koľko bodiek bolo otlačených na
zafarbenom políčku?
(Klasická hracia kocka má na stenách bodky $1,2,\dots,6$
umiestnené tak, že súčet počtu bodiek na protiľahlých stenách je
$7$. Plán pozostáva zo štvorcov, ktoré sú rovnako veľké ako steny
kocky.)
\insp{z58.51}%
}
\podpis{M. Dillingerová}

{%%%%%   Z5-I-5
Digitálne hodiny ukazujú hodiny a minúty, napríklad 14:37. Akú dobu (v minútach)
svieti za 24 hodín na týchto hodinách aspoň jedna päťka?}
\podpis{M. Volfová}

{%%%%%   Z5-I-6
Danko si zo štvorčekovej siete vystrihol útvar ako na \obr{}.
Odstrihni dva štvorčeky siete tak, aby sa
výsledný útvar nerozpadol a aby mal čo
najväčší obvod. Nájdi všetky riešenia.
\insp{z58.52}%
}
\podpis{M. Dillingerová}

{%%%%%   Z6-I-1
...}
\podpis{...}

{%%%%%   Z6-I-2
...}
\podpis{...}

{%%%%%   Z6-I-3
Bé-banka vydáva bankomatové karty so štvormiestnym PIN kódom, ktorý neobsahuje
číslicu 0. Pán Skleróza sa bál, že zabudne PIN kód svojej karty, preto si ho napísal
priamo na kartu. Aby to však prípadný zlodej nemal také ľahké, napísal si ho tam
rímskymi číslicami: IIIVIIIXIV. Svoj nápad prezradil najlepšiemu priateľovi, pánovi
Odkukalovi. Tomu sa tak zapáčil, že spravil so svojím PIN kódom to isté a na kartu si
správne zapísal: IVIIIVI. Na svoje veľké prekvapenie však z rímskeho zápisu nevedel
svoj PIN kód presne určiť!
\begin{itemize}
\itemvar{a)} Aký PIN kód má karta pána Sklerózu?
\itemvar{b)} Aký PIN kód môže mať karta pána Odkukala?
\end{itemize}
}
\podpis{S. Bednářová}

{%%%%%   Z6-I-4
Načrtni všetky možné tvarovo rôzne štvoruholníky, ktoré majú vrcholy vo vrcholoch
daného pravidelného šesťuholníka. Urči, aké by boli ich obsahy, keby šesťuholník
mal obsah $156\cm^2$.}
\podpis{M. Volfová}

{%%%%%   Z6-I-5
Pani Kučerová bola na sedemdennej dovolenke a Katka jej sľúbila v tom čase venčiť
psa a krmiť králiky. Dostala za to veľkú tortu a 700 Sk. Po ďalšej dovolenke podľa
rovnakých pravidiel dostala Katka za štyri dni venčenia a krmenia rovnakú tortu a 340
Sk. Akú cenu mala torta?}
\podpis{M. Volfová}

{%%%%%   Z6-I-6
...}
\podpis{...}

{%%%%%   Z7-I-1
...}
\podpis{...}

{%%%%%   Z7-I-2
...}
\podpis{...}

{%%%%%   Z7-I-3
Turisti plánovali dlhú túru na tri dni tak, že každý deň prejdú tretinu celej trasy. To
dodržali iba prvý deň. Druhý deň prešli iba tretinu zvyšku cesty, ktorý mali ráno pred
sebou. Tretí deň, unavení, iba štvrtinu zvyšku, ktorý ich v ten deň čakal. Posledných
24 km do cieľa ich doviezlo terénne auto. Koľko km mala mať celá túra a koľko km
prešli (pešo) prvý, druhý, tretí deň?}
\podpis{M. Volfová}

{%%%%%   Z7-I-4
Pán Horák je o 3 roky starší ako jeho žena. Ich prvorodený syn je o 4 roky starší ako
ich druhorodený a všetci štyria majú spolu 81 rokov. Pred 5 rokmi bol súčet vekov
členov tejto rodiny 62 rokov. Urči dnešný vek všetkých členov tejto štvorčlennej rodiny
-- rodičov i oboch synov.}
\podpis{M. Volfová}

{%%%%%   Z7-I-5
Zuzka napísala päťciferné číslo. Keď predeň pridala jednotku, dostala číslo
šesťciferné. Potom túto jednotku pripísala na koniec pôvodného päťciferného čísla.
Dostala číslo, ktoré bolo trikrát väčšie ako predchádzajúce šesťciferné číslo. Ktoré
päťciferné číslo Zuzka napísala pôvodne?}
\podpis{L. Hozová}

{%%%%%   Z7-I-6
Daný je obdĺžnik $ABCD$. Bodom $A$ vedieme priamku, ktorá pretne úsečku $CD$ v~bode
$X$ tak, že pre obsahy vzniknutých útvarov platí $S_{AXD}:S_{ABCX}=1:2$. Bodom~$X$ vedieme
priamku, ktorá pretne úsečku $AB$ v bode $Y$ tak, že platí $S_{AXY}:S_{YBCX}=1:2$. Bodom $Y$
opäť vedieme priamku, ktorá pretne úsečku $XC$ v~bode~$Z$ tak, že platí
$S_{XYZ}:S_{YBCZ}=1:2$.
Vypočítaj pomer obsahov $S_{AXD}:S_{AXZY}$.}
\podpis{M. Dillingerová}

{%%%%%   Z8-I-1
Myslím si nezáporné číslo v tvare zlomku s menovateľom 12 a celočíselným
čitateľom. Keď ho napíšem v tvare desatinného čísla, bude mať pred desatinnou
čiarkou aj za desatinnou čiarkou po jednej platnej číslici, obe budú nenulové. Čísel
s takouto vlastnosťou je viac. Ak ich zoradíme od najmenšieho po najväčšie, bude to
"moje" predposledné. Aké číslo si myslím?}
\podpis{S. Bednářová}

{%%%%%   Z8-I-2
...}
\podpis{...}

{%%%%%   Z8-I-3
Grafik v redakcii novín dostal dva obrázky, aby ich umiestnil k článku. Prvý bol $13\cm$
široký a $9\cm$ vysoký, druhý mal na šírku $14\cm$ a na výšku $12\cm$. Grafik sa rozhodol
umiestniť obrázky na stránku vedľa seba tak, aby sa dotýkali a aby oba mali rovnakú
výšku. Po vytlačení novín mali obrázky spolu šírku $18{,}8\cm$. Obrázky teda vhodne
zmenšil bez toho, aby ich akokoľvek orezával. Aká bola výška obrázkov vo
vytlačených novinách?}
\podpis{L. Šimůnek}

{%%%%%   Z8-I-4
Dané sú tri vzájomne rôzne nenulové číslice. Na tabuľu sme napísali všetky rôzne
trojciferné čísla, ktoré sa dajú z uvedených číslic zložiť. Na každé číslo sme pritom
použili všetky tri číslice. Súčet čísel napísaných na tabuli je $1\,776$. Ktoré tri číslice boli
dané?}
\podpis{L. Šimůnek}

{%%%%%   Z8-I-5
Na veži radnice sú hodiny s dvomi ručičkami, ktoré majú blízko stredu
ciferníku dvierka používané pri údržbe. Otvárajú sa von, čo je
nepraktické -- napríklad presne od 12:09 zakrývá dvierka veľká ručička,
a tie sa preto nedajú otvoriť. Najskôr sa dvierka opäť dajú otvoriť presne
o 12:21. Koľko minút denne sa dvierka nedajú otvoriť? Nezabudnite, že
dvierka môže zakryť aj malá ručička hodín.}
\podpis{L. Šimůnek}

{%%%%%   Z8-I-6
...}
\podpis{...}

{%%%%%   Z9-I-1
...}
\podpis{...}

{%%%%%   Z9-I-2
Alena, Barbora, Cyril a Dávid si spoločne kúpili tandem -- bicykel pre dvoch. Na
vychádzky na tandeme vyrážajú vždy v dvojici. Každý bol s každým už aspoň raz a
nikto iný se na tandeme ešte neviezol. Alena bola na vychádzke na tandeme
jedenásťkrát, Barbora dvadsaťkrát, Cyril iba štyrikrát. Určite, koľkokrát minimálne a
koľkokrát maximálne mohol byť na vychádzke na tandeme Dávid.}
\podpis{L. Šimůnek}

{%%%%%   Z9-I-3
...}
\podpis{...}

{%%%%%   Z9-I-4
...}
\podpis{...}

{%%%%%   Z9-I-5
Na stole s kruhovou doskou o priemere $0{,}6\,\text{m}$ je "nakrivo" položený štvorcový obrus
so stranou $1\,\text{m}$ -- jeho stred je vzhľadom na stred dosky posunutý. Jeden cíp obrusu
prečnieva cez hranu dosky stola $0{,}5\,\text{m}$, susedný cíp $0{,}3\,\text{m}$. Zistite dĺžku presahu
zvyšných dvoch cípov.}
\podpis{S. Bednářová}

{%%%%%   Z9-I-6
Štyria otcovia chceli deťom sponzorovať lyžiarsky zájazd.
\begin{itemize}
\itemvar{} Prvý sľúbil: dám 11\,500\,Sk,
\itemvar{} druhý sľúbil: dám tretinu toho, čo vy ostatní,
\itemvar{} tretí sľúbil: ja dám štvrtinu toho, čo vy ostatní,
\itemvar{} štvrtý sľúbil: ja dám pätinu toho, čo vy ostatní.
\end{itemize}
\noindent
Koľko konkrétne korún sľúbil druhý, tretí, štvrtý otecko?}
\podpis{M. Volfová}

{%%%%%   Z4-II-1
Mirko si zo štvorcovej siete s~vpísanými číslami vystrihol útvar na \obr.
Odstrihni dva štvorčeky útvaru tak, aby sa výsledný
útvar nerozpadol, aby po odstrihnutí oboch
štvorčekov mal rovnaký obvod ako pôvodne a~aby
súčet vpísaných čísel bol najmenší možný.
\insp{z58.43}%
}
\podpis{M. Petrová, M. Dillingerová}

{%%%%%   Z4-II-2
Jeden detský lístok na plaváreň stojí 1€. Jeden dospelý lístok stojí 2€. Teta
Eva a~ujo Adam išli na plaváreň s~deťmi. Všetky lístky pri pokladni zaplatil ujo
Adam. Teta Eva mu potom dala za seba a~všetky dievčatá 5€. Koľko platil ujo
Adam pri pokladni ak chlapcov bolo dvakrát toľko ako dievčat?}
\podpis{M. Dillingerová}

{%%%%%   Z4-II-3
Pani Jedináčkovej sa narodili trojičky. Prvá prišla na svet najťažšia Katka, po
nej Lenka a posledná najľahšia Marienka. Keby Katka vážila pri narodení
o~2\,310 gramov viac, vážila by toľko, čo Lenka s~Marienkou spolu. Keby vážila
o~4\,660~g viac, vážila by toľko, čo všetky tri spolu. Zistite pôrodnú hmotnosť
jednotlivých dievčatiek v~gramoch, ak viete, že sa udáva s~presnosťou na
desiatky gramov.}
\podpis{S. Bednářová}

{%%%%%   Z5-II-1
Mirko si zo štvorcovej siete s~vpísanými číslami vystrihol útvar na \obr{}.
Odstrihni dva štvorčeky útvaru tak, aby sa výsledný útvar
nerozpadol, aby po odstrihnutí oboch štvorčekov mal
rovnaký obvod ako pôvodne a aby súčet vpísaných čísel bol
najmenší možný.
\insp{z58.53}%
}
\podpis{M. Petrová, M. Dillingerová}

{%%%%%   Z5-II-2
Pätnásť rovnakých na sebe položených listov papiera som naraz preložil napoly.
Získal som tak "zošit", ktorého stránky som očísloval po poradí číslami 1 až 60. Ktoré
ďalšie tri čísla sú napísané na tom istom liste papiera ako číslo 25?}
\podpis{L. Šimůnek}

{%%%%%   Z5-II-3
Fero Všímavý si opakoval malú násobilku jednotlivých čísel tým, že vypisoval jej
výsledky za sebou bez medzier a čiarok. Napríklad u násobilky čísla 2 by mal
napísané $2468101214161820$. Do jedného riadku takto zapísal násobilku 3, za ňou
ihneď násobilku 5 a nakoniec 9. Potom si tento riadok prezrel a zistil, že sa v ňom
objavujú zrkadlové čísla. (Zrkadlové číslo má aspoň 3 číslice a číta sa zozadu
rovnako ako spredu, napríklad: $272$, $3553$, $98089$.) Napíš 3 najmenšie zrkadlové
čísla a jedno najväčšie zrkadlové číslo z Ferovho riadka.}
\podpis{L. Hozová}

{%%%%%   Z6-II-1
Katka chce obdarovať svoje kamarátky a rozmýšľa: keby som každej kúpila sponku
za 2{,}80 €, ostalo by mi ešte 2{,}90 €, ale keby to bol medvedík za 4{,}20 €, tak by mi
ešte 1{,}30 € chýbalo. Koľko má Katka kamarátok a koľko peňazí na darčeky?}
\podpis{M. Volfová}

{%%%%%   Z6-II-2
...}
\podpis{...}

{%%%%%   Z6-II-3
Traja záhradníci mali veľkú úrodu mrkvy a tak skúsili mrkvu odšťavovať. Po
odšťavení všetku získanú šťavu naliali do 9 pohárov. Všetky boli plné, každý však
mal iný objem: $1\,dl, 2\,dl, 3\,dl, \dots, 9\,dl$. Chceli sa spravodlivo podeliť tak, aby každý
dostal rovnaký počet pohárov i rovnako veľa šťavy. Ako to mohli urobiť?}
\podpis{M. Volfová}

{%%%%%   Z7-II-1
Rado číta zaujímavú knižku. Včera prečítal 15 strán, dnes ďalších 12 strán. S~údivom
si uvedomil, že súčet čísel strán, ktoré prečítal včera, je rovnaký ako súčet čísel
strán, ktoré prečítal dnes. Zisti číslo na stránke, ktorou začne najbližšie čítanie?
(Rado pri čítaní žiadne stránky nepreskakuje ani nečíta žiadnu stránku dva a
viackrát. Každodenné čítanie nikdy neskončí rozčítanou stránkou.)}
\podpis{M. Petrová}

{%%%%%   Z7-II-2
Tajný agent sa snaží rozlúštiť prístupový kód. Zatiaľ získal tieto informácie:
\begin{itemize}
\item je to štvorciferné číslo,
\item nie je deliteľné siedmimi,
\item číslica na mieste desiatok je súčtom číslice na mieste jednotiek a číslice na mieste
stoviek,
\item číslo vytvorené z prvých dvoch číslic kódu (v tomto poradí) je pätnásťnásobkom
poslednej číslice kódu,
\item prvá a posledná číslica kódu (v tomto poradí) tvoria prvočíslo.
\end{itemize}
\noindent
Stačia mu tieto informácie na rozlúštenie kódu? Svoj záver zdôvodni.}
\podpis{M. Petrová}

{%%%%%   Z7-II-3
...}
\podpis{...}

{%%%%%   Z8-II-1
Pri lese, ktorý mal tvar rovnoramenného trojuholníka, sa u jedného z jeho vrcholov
utáborili Ivo a Peter. Uprostred protiľahlej strany bola studnička. Chlapci sa rozhodli,
že k nej nepôjdu lesom, ale po jeho obvode. Každý vyšiel iným smerom, ale obaja
rýchlosťou 4 km/h. Ivo dorazil k studničke za 15 minút, Peter za 12 minút. Zisti dĺžky
strán trojuholníka lesa. (Dĺžky strán zaokrúhlite na celé metre.)}
\podpis{M. Volfová}

{%%%%%   Z8-II-2
Eva písala za sebou idúce prirodzené čísla: $1234567891011\dots$ Ktorá číslica by bola
zapísaná na 2\,009-tom mieste Evinho čísla?}
\podpis{M. Volfová}

{%%%%%   Z8-II-3
Tri dané prirodzené čísla sú zoradené podľa veľkosti. Určite tieto čísla na základe
nasledujúcich informácií:
\begin{itemize}
\item priemer daných troch čísel je rovný prostrednému z nich,
\item rozdiel niektorých dvoch daných čísel je $321$,
\item súčet niektorých dvoch daných čísel je $777$.
\end{itemize}
}
\podpis{L. Šimůnek}

{%%%%%   Z9-II-1
...}
\podpis{...}

{%%%%%   Z9-II-2
Nočný strážnik si na skrátenie času v službe písal postupnosť čísel. Začal istým
prirodzeným číslom. Každý ďalší člen postupnosti vytvoril tak, že
k predchádzajúcemu číslu pričítal určité číslo: k prvému členu pričítal 1, k druhému 3,
k tretiemu 5, k~štvrtému 1, k piatemu 3, k šiestemu 5, k siedmemu 1 a tak ďalej.
Vieme, že sa v jeho postupnosti nachádzajú čísla 40 a 874.
\begin{itemize}
\itemvar{a)} Ktoré číslo nasleduje v postupnosti priamo po čísle 40 a ktoré priamo po čísle
874?
\itemvar{b)} V postupnosti nájdeme dva priamo po sebe idúce členy, ktorých súčet je 491.
Ktoré dve čísla to sú?
\end{itemize}
}
\podpis{L. Šimůnek}

{%%%%%   Z9-II-3
Vojto Vodník sa bavil tým, že prelieval vodu medzi tromi nádobami. Najprv prelial po
jednej tretine vody z druhej nádoby do prvej a tretej. Potom prelial po jednej štvrtine
vody z prvej nádoby do druhej a tretej a nakoniec ešte po jednej pätine vody z tretej
nádoby do prvej a druhej nádoby. Nakoniec zostalo v každej nádobe po 1 litre vody.
Koľko vody mal Vojto pôvodne v jednotlivých nádobách?}
\podpis{M. Petrová}

{%%%%%   Z9-II-4
...}
\podpis{...}

{%%%%%   Z9-III-1
Na našu "zamyšenú" chalupu sme priviezli myšilovca kocúra Viliama. V pondelok
chytil $\frac12$ všetkých myší, v utorok $\frac13$ zvyšku, v stredu $\frac14$ tých, čo zostali po utorňajšom
love, a v štvrtok už len $\frac15$ zvyšku. V piatok sa zostávajúce myši radšej odsťahovali.
Koľko bolo myší na chalupe pôvodne, ak sa v piatok odsťahovalo o 2 myši viac ako
ich Viliam v utorok chytil?}
\podpis{M. Volfová, M. Dillingerová}

{%%%%%   Z9-III-2
Juraj, Vojto a Oto na súťaži získali všetky 3 medaily, zlatú, striebornú i bronzovú.
Nechceli sa chváliť, tak takto žartovali:
\begin{itemize}
\itemvar{} Juraj: "Oto získal zlatú!"
\itemvar{} Vojto: "Ale nie, Oto získal striebornú!"
\itemvar{} Oto: "Nedostal som ani zlatú ani striebornú!"
\end{itemize}
\noindent
Tréner prezradil, že nositeľ zlatej medaily hovoril pravdu a nositeľ bronzovej klamal.
Zistite, kto ktorú medailu získal.}
\podpis{M. Volfová}

{%%%%%   Z9-III-3
...}
\podpis{...}

{%%%%%   Z9-III-4
Adam s Evou hrali šachy o jablko.
Adam vyhral a utešoval Evu: "To vieš, ja hrávam šachy dlho, dvakrát dlhšie ako ty!"
Eva sa hnevala: "Ale minule si hovoril, že ich hrávaš trikrát dlhšie!"
Adam sa divil: "To že som hovoril? A kedy to bolo?"
"Predvlani!"
"No tak to áno, hovoril som pravdu -- a dnes tiež."
Ako dlho (v rokoch) teda hráva Adam šach?}
\podpis{M. Volfová}

