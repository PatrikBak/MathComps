{%%%%%   A-I-1
V~obore reálnych čísel riešte sústavu rovníc
$$
\align
2\sin x\cos(x+y) + \sin y &= 1,\\
2\sin y\cos(y+x) + \sin x &= 1.
\endalign
$$}
\podpis{Jaroslav Švrček}

{%%%%%   A-I-2
Daný je tetivový štvoruholník $ABCD$.
Dokážte, že spojnica priesečníkov výšok trojuholníka $ABC$
s~priesečníkom výšok trojuholníka $ABD$ je rovnobežná s~priamkou~$CD$.}
\podpis{Tomáš Jurík}

{%%%%%   A-I-3
Nájdite všetky dvojice prirodzených čísel $x$, $y$ také, že
$\displaystyle {xy^2\over x+y}$ je prvočíslo.}
\podpis{Ján Mazák}

{%%%%%   A-I-4
Uvažujme nekonečnú aritmetickú postupnosť
$$
a,a+d,a+2d,\dots,
\tag{$\ast$}
$$
kde $a$, $d$ sú prirodzené (\tj.~kladné celé) čísla.
\ite a) Nájdite príklad postupnosti~($\ast$), ktorá obsahuje
        nekonečne veľa $k$-tych mocnín prirodzených čísel pre všetky $k=2,3,\dots$
\ite b) Nájdite príklad postupnosti~($\ast$), ktorá neobsahuje
        žiadnu $k$-tu mocninu prirodzeného čísla pre žiadne $k=2,3,\dots$
\ite c) Nájdite príklad postupnosti~($\ast$), ktorá neobsahuje
        žiadnu druhú mocninu prirodzeného čísla, ale obsahuje nekonečne veľa
        tretích mocnín prirodzených čísel.
\ite d) Dokážte, že pre všetky prirodzené čísla $a$, $d$, $k$ ($k>1$) platí:
        Postupnosť~($\ast$) buď neobsahuje žiadnu $k$-tu mocninu
        prirodzeného čísla, alebo obsahuje nekonečne veľa $k$-tych mocnín
        prirodzených čísel.}
\podpis{Jaroslav Zhouf}

{%%%%%   A-I-5
V~každom vrchole pravidelného $2008$-uholníka leží jedna minca.
Vyberieme dve mince a~premiestnime každú z~nich do susedného vrcholu
tak, že jedna sa posunie v~smere
a~druhá proti smeru chodu hodinových ručičiek.
Rozhodnite, či je možné týmto spôsobom všetky mince postupne
presunúť:
\ite a) na 8~kôpok po 251 minciach,
\ite b) na 251 kôpok po 8~minciach.}
\podpis{Radek Horenský}

{%%%%%   A-I-6
Daný je trojuholník $ABC$. Vnútri strán $AC$, $BC$ sú dané body $E$, $D$ tak, že
$|AE|=|BD|$. Označme $M$ stred strany~$AB$ a~$P$ priesečník priamok $AD$ a~$BE$.
Dokážte, že obraz bodu~$P$ v~stredovej súmernosti so stredom~$M$ leží
na osi uhla $ACB$.}
\podpis{Ján Mazák}

{%%%%%   B-I-1
Na tabuli je napísané štvorciferné číslo deliteľné ôsmimi, ktorého
posledná cifra je~8. Keby sme poslednú cifru nahradili
cifrou~7, získali by sme číslo deliteľné deviatimi. Keby sme
však poslednú cifru nahradili cifrou~9, získali by sme
číslo deliteľné siedmimi. Určte číslo, ktoré je napísané na tabuli.}
\podpis{Peter Novotný}

{%%%%%   B-I-2
Určte všetky trojice $(x,y,z)$ reálnych čísel, pre ktoré platí
$$
\eqalign{
 x^2+xy=y^2+z^2,\cr
 z^2+zy=y^2+x^2.
}
$$}
\podpis{Jaroslav Švrček}

{%%%%%   B-I-3
Na strane~$BC$, resp. $CD$ rovnobežníka $ABCD$ určte body $E$,
resp. $F$ tak, aby úsečky $EF$, $BD$ boli rovnobežné a~trojuholníky $ABE$,
$AEF$ a~$AFD$ mali rovnaké obsahy.}
\podpis{Jaroslav Zhouf}

{%%%%%   B-I-4
Na pláne $7\times7$ hráme hru lode. Nachádza sa na nej jedna loď
$2\times3$. Môžeme sa spýtať na ľubovoľné políčko plánu, a~ak
loď zasiahneme, hra končí. Ak nie, pýtame sa znova. Určte
najmenší počet otázok, ktoré potrebujeme, aby sme s~istotou
loď zasiahli.}
\podpis{Ján Mazák}

{%%%%%   B-I-5
Trojuholníku $ABC$ je opísaná kružnica~$k$. Os strany~$AB$ pretne
kružnicu~$k$ v~bode~$K$, ktorý leží v~polrovine opačnej
k~polrovine $ABC$. Osi strán $AC$ a~$BC$ pretnú priamku~$CK$ postupne
v~bodoch $P$ a~$Q$. Dokážte, že trojuholníky $AKP$ a~$KBQ$
sú zhodné.
}
\podpis{Leo Boček}

{%%%%%   B-I-6
Nájdite všetky dvojice celých čísel $(m,n)$, pre ktoré je hodnota výrazu
$$
\frac{m+3n-1}{mn+2n-m-2}
$$
celé kladné číslo.}
\podpis{Vojtech Bálint}

{%%%%%   C-I-1
Tomáš, Jakub, Martin a~Peter organizovali na námestí zbierku pre dobročinné účely.
Za chvíľu sa pri nich postupne zastavilo päť okoloidúcich. Prvý dal Tomášovi
do jeho pokladničky 3~Sk, Jakubovi 2~Sk, Martinovi 1~Sk a~Petrovi nič. Druhý
dal jednému z~chlapcov 8~Sk a~ostatným trom nedal nič.
Tretí dal dvom chlapcom po 2~Sk a~dvom nič. Štvrtý dal dvom chlapcom
po~4~Sk a~dvom nič. Piaty dal dvom chlapcom po 8~Sk a~dvom nič.
Potom chlapci zistili, že každý z~nich vyzbieral inú čiastku,
pričom tieto tvoria štyri po sebe idúce prirodzené čísla.
Ktorý z~chlapcov vyzbieral najmenej a~ktorý najviac korún?}
\podpis{Peter Novotný}

{%%%%%   C-I-2
Pravouhlému trojuholníku $ABC$ s~preponou~$AB$ je opísaná kružnica.
Päty kolmíc z~bodov $A$, $B$ na dotyčnicu k~tejto kružnici v~bode~$C$ označme $D$, $E$.
Vyjadrite dĺžku úsečky~$DE$ pomocou dĺžok odvesien trojuholníka~$ABC$.}
\podpis{Pavel Leischner}

{%%%%%   C-I-3
Nájdite všetky štvorciferné čísla~$n$, ktoré majú nasledujúce tri vlastnosti:
V~zápise čísla~$n$ sú dve rôzne cifry, každá dvakrát.
Číslo $n$ je deliteľné siedmimi.
Číslo, ktoré vznikne otočením poradia cifier čísla~$n$, je tiež štvorciferné
a~deliteľné siedmimi.}
\podpis{Pavel Novotný}

{%%%%%   C-I-4
Daný je konvexný päťuholník $ABCDE$. Na polpriamke~$BC$ zostrojte taký
bod~$G$, aby obsah trojuholníka $ABG$ bol zhodný s~obsahom daného päťuholníka.}
\podpis{Lucie Růžičková}

{%%%%%   C-I-5
Z~množiny $\{1,2,3,\dots,99\}$ vyberte čo najväčší počet
čísel tak, aby súčet žiadnych dvoch vybraných čísel nebol násobkom
jedenástich. (Vysvetlite, prečo
zvolený výber má požadovanú vlastnosť a~prečo žiadny
výber väčšieho počtu čísel nevyhovuje.)}
\podpis{Jaromír Šimša}

{%%%%%   C-I-6
Dokážte, že pre ľubovoľné rôzne kladné čísla $a$, $b$ platí
$$
\frac{a+b}{2}<\frac{2(a^2+ab+b^2)}{3(a+b)}<\sqrt{\frac{a^2+b^2}{2}}.
$$}
\podpis{Jaromír Šimša}

{%%%%%   A-S-1
Zistite, pre ktoré dvojice kladných celých čísel $m$ a~$n$ platí
$$
\sqrt{m^2-4}<2\sqrt{n}-m<\sqrt{m^2-2}.
$$}
\podpis{Jaromír Šimša}

{%%%%%   A-S-2
Nech $ABC$ je ostrouhlý trojuholník, v~ktorom vnútorný uhol pri vrchole~$A$ má veľkosť~$45^{\circ}$. Označme $D$ pätu výšky z~vrcholu~$C$. Uvažujme ďalej ľubovoľný vnútorný bod~$P$ výšky~$CD$. Dokážte tvrdenie: Priamky $AP$ a~$BC$ sú navzájom kolmé práve vtedy, keď úsečky $AP$ a~$BC$ sú zhodné.}
\podpis{Jaroslav Švrček}

{%%%%%   A-S-3
Určte všetky prirodzené čísla, ktorými možno krátiť niektorý zo zlomkov tvaru
$$
{3p-q\over5p+2q},
$$
kde $p$ a~$q$ sú nesúdeliteľné celé čísla.}
\podpis{Vojtech Bálint}

{%%%%%   A-II-1
%kategorie B, příklad č. 14
Isté štvorciferné prirodzené číslo je deliteľné siedmimi. Ak zapíšeme jeho číslice v~opačnom poradí, dostaneme väčšie
štvorciferné číslo, ktoré je tiež deliteľné siedmimi. Navyše po delení číslom~$37$ dávajú obe spomenuté štvorciferné čísla rovnaký zvyšok. Určte pôvodné štvorciferné číslo.}
\podpis{Jaromír Šimša}

{%%%%%   A-II-2
Na odvesnách dĺžok $a$, $b$ pravouhlého trojuholníka ležia postupne stredy dvoch kružníc $k_a$, $k_b$. Obe kružnice sa dotýkajú prepony a~prechádzajú vrcholom oproti prepone. Polomery uvedených kružníc označme $\rho_a$, $\rho_b$. Určte najväčšie kladné reálne číslo~$p$ také, že nerovnosť
$$
\frac{1}{\rho_a}+\frac{1}{\rho_b} \ge p\Bigl(\frac{1}{a}+\frac{1}{b}\Bigr)
$$
platí pre všetky pravouhlé trojuholníky.}
\podpis{Jaroslav Švrček}

{%%%%%   A-II-3
Určte veľkosti vnútorných uhlov $\alpha$, $\beta$, $\gamma$ trojuholníka, pre ktoré platí
$$
\align
  2\sin \beta \sin(\alpha+\beta)-\cos \alpha &= 1,\\
  2\sin \gamma \sin(\beta+\gamma)-\cos \beta  &= 0.
\endalign
$$}
\podpis{Jaroslav Švrček}

{%%%%%   A-II-4
Vnútri strany~$BC$ ostrouhlého trojuholníka $ABC$ zvoľme bod~$D$ a~na úsečke~$AD$ bod~$P$ tak, aby neležal na ťažnici z~vrcholu~$C$. Priamka tejto ťažnice pretne kružnicu opísanú trojuholníku $CPD$ v~bode, ktorý označíme~$K$ ($K\ne C$).
Dokážte, že kružnica opísaná trojuholníku $AKP$ prechádza okrem bodu~$A$ ďalším pevným bodom, ktorý od výberu bodov $D$ a~$P$ nezávisí.}
\podpis{Tomáš Jurík}

{%%%%%   A-III-1
Dokážte, že ak sú všetky čísla $p$, $3p+2$, $5p+4$, $7p+6$, $9p+8$ a~$11p+10$ prvočísla, tak číslo $6p+11$ je zložené.}
\podpis{Pavel Novotný}

{%%%%%   A-III-2
Na kratšom z~oblúkov~$CD$ kružnice opísanej pravouholníku $ABCD$ zvoľme bod~$P$. Päty kolmíc z~bodu~$P$ na priamky $AB$, $AC$ a~$BD$ označme postupne~$K$, $L$ a~$M$. Ukážte, že uhol $LKM$ má veľkosť $45\st{}$ práve vtedy, keď $ABCD$ je štvorec.}
\podpis{Tomáš Jurík}

{%%%%%   A-III-3
Nájdite najmenšie kladné číslo $x$, pre ktoré platí: Ak $a$, $b$, $c$, $d$ sú ľubovoľné kladné čísla, ktorých súčin je $1$, potom
$$
a^x+b^x+c^x+d^x \ge \frac1a + \frac1b + \frac1c + \frac1d.
$$}
\podpis{Pavel Novotný}

{%%%%%   A-III-4
Skúmajme, pre ktoré prirodzené čísla~$n$ existujú práve štyri prirodzené čísla~$k$ také, že číslo $n+k$ je deliteľom čísla $n+k^2$.
 \ite a) Ukážte, že vyhovuje $n=58$ a~nájdite príslušné štyri $k$.
 \ite b) Dokážte, že párne $n=2p$, kde $p\ge3$, vyhovuje práve vtedy, keď $p$ aj $2p+1$ sú prvočísla.
 
(Nulu medzi prirodzené čísla nepočítame.)}
\podpis{Jaromír Šimša}

{%%%%%   A-III-5
V~každom z~vrcholov pravidelného $n$-uholníka $A_1A_2\dots A_n$ leží určitý počet mincí: vo vrchole~$A_k$ je to práve $k$~mincí, $1\le k\le n$. Vyberieme dve mince a~preložíme každú z~nich do susedného vrcholu tak, že jedna sa posunie v~smere a~druhá proti smeru chodu hodinových ručičiek. Rozhodnite, pre ktoré $n$ možno po konečnom počte takých preložení dosiahnuť, že pre ľubovoľné $k$, $1\le k\le n$, bude vo vrchole~$A_k$ ležať $n+1-k$~mincí.}
\podpis{Radek Horenský}

{%%%%%   A-III-6
V~rovine~$\omega$ sú dané dva rôzne body $O$ a~$T$. Nájdite množinu vrcholov všetkých trojuholníkov, ktoré ležia v~rovine~$\omega$ a~majú ťažisko v~bode~$T$ a~stred opísanej kružnice v~bode~$O$.}
\podpis{Jaromír Šimša}

{%%%%%   B-S-1
V~obore reálnych čísel riešte sústavu rovníc
$$\eqalign{
 ax+y&=2,\cr
  x-y&=2a,\cr
  x+y&=1\cr}
$$
s~neznámymi $x$, $y$ a~reálnym parametrom~$a$.}
\podpis{Jaroslav Švrček}

{%%%%%   B-S-2
Pre vnútorný bod~$P$ strany~$AB$ ostrouhlého trojuholníka $ABC$ označme $K$ a~$L$ päty kolmíc z~bodu~$P$ na priamky $AC$ a~$BC$. Zostrojte taký bod~$P$, pre ktorý priamka~$CP$ rozpoľuje úsečku~$KL$.}
\podpis{Pavel Calábek}

{%%%%%   B-S-3
Číslo nazveme {\it magickým\/} práve vtedy, keď sa dá vyjadriť ako súčet trojciferného čísla~$m$ a~trojciferného čísla~$m'$ zapísaného rovnakými číslicami v~opačnom poradí. Niektoré magické čísla možno takto vyjadriť viacerými spôsobmi; napríklad $1554=579+975=777+777$. Určte všetky magické čísla, ktoré majú takých vyjadrení $m+m'$
čo najviac. (Na poradie $m$ a~$m'$ neberieme ohľad.)}
\podpis{Aleš Kobza}

{%%%%%   B-II-1
V~obore reálnych čísel riešte sústavu rovníc
$$\eqalign{
  x+y&=1,\cr
  x-y&=a,\cr
  -4ax+4y&=z^2+4
}$$
s~neznámymi $x$, $y$, $z$ a~reálnym parametrom~$a$.}
\podpis{Jaroslav Švrček}

{%%%%%   B-II-2
Na pláne $5\times 5$ hráme hru lode. Zo štyroch políčok plánu je vytvorená jedna loď niektorého z~tvarov na \obr.
\insp{b58.5}%
%$$\epsfbox{b58.1}$$
Môžeme sa spýtať na ľubovoľné políčko plánu, a~ak loď zasiahneme, hra končí.
% Pokud ne, zalícíme znovu.
\ite a) Navrhnite osem políčok, na ktoré sa stačí spýtať, aby sme mali
        istotu zásahu lode.
\ite b) Zdôvodnite, prečo žiadnych sedem otázok takú istotu nedáva.}
\podpis{Ján Mazák}

{%%%%%   B-II-3
Je daný ostrouhlý trojuholník $ABC$, ktorý nie je rovnoramenný. Označme $K$ priesečník osi uhla $ACB$ s~osou strany~$AB$. Priamka~$CK$ pretína výšky z~vrcholov~$A$ a~$B$ v~bodoch, ktoré označíme postupne $P$ a~$Q$. Predpokladajme, že trojuholníky $AKP$ a~$BKQ$ majú rovnaký obsah. Určte veľkosť uhla $ACB$.}
\podpis{Ján Mazák}

{%%%%%   B-II-4
K~ľubovoľnému prirodzenému číslu určíme jeho zvyšky po delení každým z~desiatich prirodzených čísel $2, 3, 4,~\dots, 11$ a~týchto desať zvyškov (niektoré môžu byť nulové) sčítame. Určte všetky také čísla menšie ako $25\,000$, ktoré majú uvedený súčet čo najmenší.
(Nulu za prirodzené číslo nepovažujeme).}
\podpis{Jaromír Šimša}

{%%%%%   C-S-1
Dokážte, že pre ľubovoľné nezáporné čísla $a$, $b$, $c$ platí
$$
(a+bc)(b+ac)\ge ab(c+1)^2.
$$
Zistite, kedy nastane rovnosť.}
\podpis{Jaromír Šimša}

{%%%%%   C-S-2
V~pravouhlom trojuholníku $ABC$ označíme $P$ pätu výšky z~vrcholu~$C$ na~preponu~$AB$. Priesečník úsečky~$AB$ s~priamkou, ktorá prechádza vrcholom~$C$ a~stredom kružnice vpísanej trojuholníku $PBC$, označíme~$D$. Dokážte, že úsečky $AD$ a~$AC$ sú zhodné.}
\podpis{Pavel Leischner}

{%%%%%   C-S-3
Keď isté dve prirodzené čísla v~rovnakom poradí sčítame, odčítame, vydelíme a~vynásobíme a~všetky štyri výsledky sčítame, dostaneme $2\,009$. Určte tieto dve čísla.}
\podpis{Vojtech Bálint}

{%%%%%   C-II-1
Uvažujme výraz
$$
V(x)=\frac{5x^{4}-4x^{2}+5}{x^{4}+1}.
$$
\ite a) Dokážte, že pre každé reálne číslo~$x$ platí $V(x)\ge3$.
\ite b) Nájdite najväčšiu hodnotu $V(x)$.}
\podpis{Aleš Kobza}

{%%%%%   C-II-2
V~pravouhlom trojuholníku $ABC$ označíme $P$ pätu výšky z~vrcholu~$C$ na~preponu~$AB$ a~$D$, $E$ stredy kružníc vpísaných postupne trojuholníkom $APC$, $CPB$. Dokážte, že stred kružnice vpísanej trojuholníku $ABC$ je priesečníkom výšok trojuholníka $CDE$.}
\podpis{Pavel Leischner}

{%%%%%   C-II-3
Z~množiny $\{1,2,3,\dots,99\}$ je vybraných niekoľko rôznych čísel tak, že súčet žiadnych troch z~nich nie je násobkom deviatich.
\ite a) Dokážte, že medzi vybranými číslami sú najviac štyri deliteľné tromi.
\ite b) Ukážte, že vybraných čísel môže byť 26.}
\podpis{Jaromír Šimša}

{%%%%%   C-II-4
Pravouhlému trojuholníku $ABC$ s~preponou~$AB$ a~obsahom~$S$ je opísaná kružnica. Dotyčnica k~tejto kružnici v~bode~$C$ pretína dotyčnice vedené bodmi $A$ a~$B$ v~bodoch $D$ a~$E$. Vyjadrite dĺžku úsečky~$DE$ pomocou dĺžky~$c$ prepony a~obsahu~$S$.}
\podpis{Peter Novotný}

{%%%%%   vyberko, den 1, priklad 1
Nech $S\subset\Bbb R$ je podmnožina množiny reálnych čísel. Hovoríme, že dvojica funkcií $f\colon S\to S$, $g\colon S\to S$ tvorí {\it brémsku dvojicu} na $S$, ak sú splnené nasledovné podmienky:
\item{(i)}Obe funkcie sú rýdzo rastúce, teda $f(x)<f(y)$ a~$g(x)<g(y)$ pre ľubovoľné $x,y\in S$ spĺňajúce $x<y$.
\item{(ii)} Pre každé $x\in S$ platí $f(g(g(x)))<g(f(x))$.

Rozhodnite, či existuje brémska dvojica
\item{a)}na množine $S=\Bbb N$, \tj. na množine prirodzených čísel;
\item{b)}na množine $S=\{a-1/b;\ a,b\in\Bbb N\}$.}
\podpis{Peter Novotný:Shortlist 2008, A3}

{%%%%%   vyberko, den 1, priklad 2
Pre každé prirodzené číslo~$n$ určte počet takých permutácií $(a_1,a_2,\dots,a_n)$ množiny $\{1,2,\dots,n\}$, že
$$
k\mid 2(a_1+a_2+\cdots+a_k)\quad\text{pre všetky $k=1,2,\dots,n$.}
$$}
\podpis{Peter Novotný:Shortlist 2008, C2}

{%%%%%   vyberko, den 1, priklad 3
Nech $ABCD$ je konvexný štvoruholník a~$P$, $Q$ sú body v~jeho vnútri, pričom štvoruholníky $PQDA$ a~$QPBC$ sú tetivové. Na úsečke~$PQ$ leží taký bod~$E$, že
$$
|\uhol PAE|=|\uhol QDE|\quad\text{a}\quad |\uhol PBE|=|\uhol QCE|.
$$
Dokážte, že štvoruholník $ABCD$ je tetivový.
}
\podpis{Peter Novotný:Shortlist 2008, G3}

{%%%%%   vyberko, den 2, priklad 1
Postupnosť reálnych čísel ${\{a_n\}}_{n\ge 0}$ spĺňa nasledovnú podmienku:
$$
a_{m+n}+a_{m-n}=\frac12(a_{2m}+a_{2n})\quad\text{pre všetky $m\ge n\ge 0$.}
$$
Vypočítajte hodnotu $a_{2009}$ za podmienky, že $a_1=1$.
}
\podpis{Ondrej Mikuláš, Michal Prusák:Croatian National competition 2003, 4th grade / 4.2}

{%%%%%   vyberko, den 2, priklad 2
Na niektorých políčkach štvorčekovej mriežky $2009\times 2009$ je položený kamienok (na každom políčku najviac jeden). Pre každé prázdne políčko mriežky nachádzajúce sa v~$i$-tom riadku a~$j$-tom stĺpci platí, že súčet počtu kamienkov v~$i$-tom riadku a~počtu kamienkov v~$j$-tom stĺpci je aspoň $2009$. Nájdite najmenší počet kamienkov, ktoré tam musia byť položené.}
\podpis{Ondrej Mikuláš, Michal Prusák:Japonsko 1999, uloha 1}

{%%%%%   vyberko, den 2, priklad 3
Kružnice $k_1$ a~$k_2$ sa dotýkajú zvonka v~bode~$K$. Navyše sa dotýkajú zvnútra kružnice~$m$ v~bode~$A_1$, resp. $A_2$. Nech $P$ je jeden z~priesečníkov kružnice~$m$ so spoločnou dotyčnicou kružníc $k_1$ a~$k_2$ prechádzajúcou bodom~$K$. Priamka~$PA_1$ pretína $k_1$ druhýkrát v~bode~$B_1$, podobne $PA_2$ pretína $k_2$ druhýkrát v~bode~$B_2$. Dokážte, že $B_1B_2$ je spoločná dotyčnica kružníc $k_1$ a~$k_2$.}
\podpis{Ondrej Mikuláš, Michal Prusák:British Mathematical Olympiads 1996, round 2/3}

{%%%%%   vyberko, den 2, priklad 4
Nájdite všetky kladné celé čísla~$n$ také, že ich prvočíselný rozklad obsahuje iba čísla $2$ a~$5$ (nie nutne obe), pričom $n+25$ je druhou mocninou prirodzeného čísla.}
\podpis{Ondrej Mikuláš, Michal Prusák:Austrian-Polish Mathematics Competition 2000, uloha 1}

{%%%%%   vyberko, den 3, priklad 1
Na štvorčekovom papieri s~dĺžkou strany štvorčeka~$1$ uvažujme množinu~$S$
všetkých mrežových bodov. Pre prirodzené číslo~$k$ nazveme dvojicu rôznych
bodov $k$-priateľskou, ak existuje bod~$C$ z~$S$ taký, že obsah
trojuholníka $ABC$ je rovný~$k$. Podmnožinu~$T$ množiny~$S$ nazveme $k$-banda,
ak každá dvojica bodov v~$T$ je $k$-priateľskou. Nájdi najmenšie celé
kladné číslo~$k$, pre ktoré existuje $k$-banda s~viac ako 200 prvkami.}
\podpis{Martin Potočný:Shortlist 2008, C3}

{%%%%%   vyberko, den 3, priklad 2
Je daný lichobežník $ABCD$ s~rovnobežnými stranami $AB$ a~$CD$.
Predpokladajme existenciu bodu~$E$ na priamke~$BC$ mimo úsečky~$BC$ a~bodu~$F$
vnútri úsečky~$AD$ takých, že veľkosti uhlov $DAE$ a~$CBF$ sú rovnaké.
Označme $I$ priesečník $CD$ a~$EF$ a~$J$ priesečník $AB$ a~$EF$. Nech
$K$ je stredom úsečky~$EF$, pričom neleží na priamke~$AB$.
Dokážte, že $I$ patrí kružnici opísanej trojuholníku $ABK$ práve vtedy, keď $K$
patrí kružnici opísanej trojuholníku $CDJ$.}
\podpis{Martin Potočný:Shortlist 2008, G2}

{%%%%%   vyberko, den 3, priklad 3
Nech $k$ a~$n$ sú nezáporné celé čísla, pričom $k$ je menšie ako $n-1$.
Uvažujme množinu~$\Cal L$ obsahujúcu $n$~priamok v~rovine takú, že žiadna
dvojica nie je rovnobežná a~žiadna trojica sa nepretína v~jednom bode.
Označme $\Cal I$ množinu priesečníkov priamok z~$\Cal L$. Nech $O$ je bod v~rovine
neležiaci na žiadnej priamke z~$\Cal L$. Bod~$X$ z~$\Cal I$ je ofarbený na červeno, ak otvorená úsečka~$OX$ (bez koncových bodov) pretína najviac $k$~priamok z~$\Cal L$. Dokážte, že $\Cal I$ obsahuje najmenej
$(k+1)(k+2)/2$ červených bodov.}
\podpis{Martin Potočný:Shortlist 2008, G5}

{%%%%%   vyberko, den 4, priklad 1
Nech $S_a$, $S_b$, $S_c$ sú stredy a~$R_a$, $R_b$, $R_c$ polomery kružníc pripísaných ku stranám $BC$, $CA$, $AB$ trojuholníka $ABC$. Označme postupne $r_a$, $r_b$, $r_c$ polomery kružníc vpísaných trojuholníkom $BCS_a$, $ACS_b$, $ABS_c$. Dokážte, že
$$
\frac{r_a}{R_a} + \frac{r_b}{R_b} + \frac{r_c}{R_c} = 1.
$$}
\podpis{Peter Novotný, Erika Trojáková:American Monthly, 2008}

{%%%%%   vyberko, den 4, priklad 2
Nech $n\ge2$ je dané prirodzené číslo. Nájdite všetky mnohočleny~$P$ stupňa menšieho ako $n$ s~celočíselnými koeficientmi spĺňajúce nasledujúcu podmienku: Existuje postupnosť celých čísel $x_1<x_2<\cdots<x_n$ takých, že
$$
P(x_{k+1})=P(x_k)+7\quad\text{pre $k=1,2,\dots,n-1$.}
$$}
\podpis{Peter Novotný, Erika Trojáková:Poľské sústredenie, 2006}

{%%%%%   vyberko, den 4, priklad 3
Čísla $a$, $b$, $c$ sú dĺžkami strán daného trojuholníka. Dokážte, že
$$
\frac{a}{b+c-a}+\frac{b}{c+a-b}+\frac{c}{a+b-c}\ge\frac{b+c-a}{a}+\frac{c+a-b}{b}+\frac{a+b-c}{c}.
$$}
\podpis{Peter Novotný, Erika Trojáková:Poľské sústredenie, 2006}

{%%%%%   vyberko, den 4, priklad 4
Dané je prirodzené číslo~$k$. Určte najmenšiu hodnotu, akú môže nadobúdať ciferný súčet nejakého násobku čísla $10^k-1$.}
\podpis{Peter Novotný, Erika Trojáková:Poľské sústredenie, 2007}

{%%%%%   vyberko, den 5, priklad 1
V~rovine je daný rovnoramenný trojuholník $ABC$ s~ramenami $AB$ a~$AC$. Bod~$M$ je stredom jeho základne~$BC$. Zvoľme ľubovoľný bod~$X$ vo vnútri menšieho z~oblúkov~$MA$ na kružnici opísanej trojuholníku $ABM$. Označme $T$ taký bod vnútri ostrého uhla $BMA$, že $|\uhol TMX|=90\st$ a~$|TX|=|BX|$. Dokážte, že rozdiel
$$
|\uhol BTM|-|\uhol MTC|
$$
nezávisí od voľby bodu~$X$.}
\podpis{Michal Burger:ISL 2007, G2, AIMO 2008, TST 1, P3, Ukrainian TST 2008 Problem 1}

{%%%%%   vyberko, den 5, priklad 2
Nech pre nepárne celé čísla $a$, $b$, $c$, $d$ platí $0 < a < b < c < d$ a~$ad = cd$.
Dokážte, že ak $a + d = 2m$ a~$b + c = 2k$ pre nejaké prirodzené čísla $m$ a~$k$,
potom $a = 1$.}
\podpis{Michal Burger:Je vybraty z PEN (A 51) a ze vraj bol IMO 1984/6}

{%%%%%   vyberko, den 5, priklad 3
Nájdite všetky funkcie $f\colon\Bbb R\to\Bbb R$ také, že pre každé dve reálne čísla $x$, $y$
platí
$$
f(xf(y)) + y + f(x) = f (x+f(y)) + yf(x).
$$}
\podpis{Michal Burger:Iran Team Selection Test 2008}

{%%%%%   vyberko, den 1, priklad 4
...}
\podpis{...}

{%%%%%   vyberko, den 3, priklad 4
...}
\podpis{...}

{%%%%%   vyberko, den 5, priklad 4
...}
\podpis{...}

{%%%%%   trojstretnutie, priklad 1
Označme $\Bbb R^+$ množinu všetkých kladných reálnych čísel. Nájdite všetky funkcie $f\colon\Bbb R^{\p}\to \Bbb R^{\p}$, ktoré pre ľubovoľné $x,y\in\Bbb R^{\p}$ spĺňajú podmienku
$$
\bigl(1+yf(x)\bigr)\bigl(1-yf(x+y)\bigr)=1.
$$}
\podpis{František Kardoš}

{%%%%%   trojstretnutie, priklad 2
Pre dané kladné celé čísla $a$, $k$ je postupnosť $(a_n)_{n=1}^\infty$ definovaná vzťahmi
$$
a_1=a \qquad\text{a}\qquad a_{n+1}=a_n+k\cdot \varrho(a_n)\quad\text{pre $n=1,2,\dots$,}
$$
pričom $\varrho(m)$ označuje súčin cifier čísla $m$ zapísaného v~desiatkovej sústave (napríklad $\varrho(413)=12$, $\varrho(308)=0$ a pod.). Dokážte, že existujú kladné celé čísla $a$, $k$ také, že postupnosť $(a_n)_{n=1}^\infty$ obsahuje práve 2009 rôznych čísel.}
\podpis{Peter Novotný}

{%%%%%   trojstretnutie, priklad 3
Nech $k$ je kružnica pripísaná k~strane~$BC$ daného trojuholníka $ABC$. Zvoľme priamku~$p$ rovnobežnú so stranou~$BC$ pretínajúcu úsečky $AB$, $AC$ v~bodoch $D$, $E$. Kružnicu vpísanú do trojuholníka $ADE$ označme $l$.
Dotyčnice ku kružnici~$k$ vedené z~bodov $D$, $E$ neprechádzajúce bodom~$A$ sa pretínajú v~bode~$P$. Dotyčnice ku kružnici~$l$ vedené z~bodov $B$, $C$ neprechádzajúce bodom~$A$ sa pretínajú v~bode~$Q$. Dokážte, že priamka~$PQ$ prechádza pevným bodom nezávislým od voľby priamky~$p$.}
\podpis{Tomáš Jurík}

{%%%%%   trojstretnutie, priklad 4
Daná je kružnica~$k$ a~jej tetiva~$AB$, ktorá nie je jej priemerom. Vnútri dlhšieho oblúka~$AB$ kružnice~$k$ zvolíme ľubovoľne bod~$C$. Obrazy bodov $A$ a~$B$ v~osových súmernostiach podľa priamok $BC$ a~$AC$ označíme $K$ a~$L$. Dokážte, že vzdialenosť stredov úsečiek $KL$ a~$AB$ nezávisí od polohy bodu~$C$.}
\podpis{Tomáš Jurík}

{%%%%%   trojstretnutie, priklad 5
Daná je $n$-tica celých čísel $a_1,\dots,a_n$ spĺňajúca nasledujúce podmienky:
\item{($i$)} $1\le a_1<a_2<\cdots<a_n\le 50$;
\item{($ii$)} pre každú $n$-ticu kladných celých čísel $b_1,\dots,b_n$ existuje kladné celé číslo~$m$ a~$n$-tica kladných celých čísel $c_1,\dots,c_n$ taká, že
$$
  m\cdot b_i=c_i^{a_i}\quad\text{pre $i=1,\dots,n$.}
$$

Dokážte, že $n\le16$ a~určte počet rôznych $n$-tíc $a_1,\dots,a_n$ spĺňajúcich dané podmienky pre $n=16$.}
\podpis{Peter Novotný}

{%%%%%   trojstretnutie, priklad 6
Nech $n\ge16$ je prirodzené číslo. Uvažujme množinu
$$
G = \bigl\{(x,y)\ :\ x,y\in\{1,2,\dots,n\}\bigr\}
$$
pozostávajúcu z~$n^2$ bodov roviny. Nech $A$ je ľubovoľná podmnožina množiny~$G$ obsahujúca aspoň $4n\sqrt{n}$ prvkov. Dokážte, že existuje aspoň $n^2$ konvexných štvoruholníkov majúcich vrcholy v~$A$, ktorých všetky uhlopriečky prechádzajú jedným bodom.}
\podpis{Poľsko}

{%%%%%   IMO, priklad 1
Nech $n$ je kladné celé číslo a~$a_1,\dots,a_k$ ($k\ge 2$) sú navzájom rôzne celé čísla z~množiny $\{1,\dots,n\}$ také, že $n$ je deliteľom čísla $a_i(a_{i+1}-1)$ pre $i=1,\dots,k-1$.
Dokážte, že $n$ nie je deliteľom čísla $a_k(a_1-1)$.}
\podpis{Austrália}

{%%%%%   IMO, priklad 2
Daný je trojuholník $ABC$ so stredom opísanej kružnice $O$.
Nech $P$ resp. $Q$ je vnútorný bod strany $CA$ resp. $AB$.
Označme postupne $K$, $L$, $M$ stredy úsečiek $BP$, $CQ$, $PQ$ a~$\Gamma$ kružnicu prechádzajúcu bodmi $K$, $L$, $M$.
Predpokladajme, že priamka $PQ$ sa dotýka kružnice $\Gamma$. Dokážte, že $|OP|=|OQ|$.}
\podpis{Rusko}

{%%%%%   IMO, priklad 3
Predpokladajme, že $s_1,s_2,s_3,\dots$ je rastúca postupnosť kladných celých čísel taká, že obe jej podpostupnosti
$$
s_{s_1},\ s_{s_2},\ s_{s_3},\ \dots\qquad\text{a}\qquad s_{s_1+1},\ s_{s_2+1},\ s_{s_3+1},\ \dots
$$
sú aritmetické.
Dokážte, že potom aj postupnosť $s_1, s_2, s_3, \dots$ je aritmetická.}
\podpis{USA}

{%%%%%   IMO, priklad 4
Daný je trojuholník $ABC$, pričom $|AB|=|AC|$.
Osi uhlov $CAB$ a~$ABC$ pretínajú strany $BC$ a~$CA$ postupne v~bodoch $D$ a~$E$.
Nech $K$ je stred kružnice vpísanej do trojuholníka $ADC$.
Predpokladajme, že $|\angle BEK| = 45^\circ$.
Nájdite všetky možné veľkosti uhla $CAB$.}
\podpis{Belgicko}

{%%%%%   IMO, priklad 5
Určte všetky také funkcie $f$ z~množiny kladných celých čísel do množiny kladných celých čísel, že pre všetky kladné celé čísla $a$, $b$ existuje nedegenerovaný trojuholník so stranami dĺžok
$$
a,\quad f(b),\quad f(b+f(a)-1).
$$
(Trojuholník je {\it nedegenerovaný}, ak jeho vrcholy neležia na jednej priamke.)}
\podpis{Francúzsko}

{%%%%%   IMO, priklad 6
Nech $a_1,a_2,\dots,a_n$ sú navzájom rôzne kladné celé čísla a~$M$ je množina $n-1$ kladných celých čísel neobsahujúca číslo $s=a_1+a_2+\cdots+a_n$.
Lúčny koník skáče pozdĺž číselnej osi, pričom začína v~bode~$0$ a~urobí smerom doprava $n$ skokov s~dĺžkami $a_1,a_2,\dots,a_n$ v nejakom poradí.
Dokážte, že poradie skokov sa dá zvoliť tak, aby lúčny koník nepristál na žiadnom čísle z~množiny~$M$.}
\podpis{Rusko}

{%%%%%   MEMO, priklad 1
Nájdite všetky funkcie $f\colon\Bbb R\to\Bbb R$ také, že
$$
f(xf(y)) + f(f(x) + f(y)) = yf(x) + f(x + f(y))
$$
pre všetky $x, y\in\Bbb R$, pričom $\Bbb R$ označuje množinu všetkých reálnych čísel.}
\podpis{Slovinsko}

{%%%%%   MEMO, priklad 2
Majme $n\ge3$ rôznych farieb. Nech $f(n)$ označuje najväčšie celé číslo s~vlastnosťou, že
každá strana a~každá uhlopriečka konvexného mnohouholníka majúceho $f(n)$ vrcholov sa dá ofarbiť
jednou z~$n$ farieb nasledujúcim spôsobom:
\item{$\bullet$} použité sú aspoň dve farby a
\item{$\bullet$} každé tri vrcholy mnohouholníka určujú buď trojicu úsečiek rovnakej farby, alebo trojicu úsečiek troch rôznych farieb.

Dokážte, že $f(n)\le (n-1)^2$, a~že rovnosť v~tejto nerovnosti nastáva pre nekonečne veľa hodnôt~$n$.}
\podpis{Slovinsko}

{%%%%%   MEMO, priklad 3
Nech $ABCD$ je konvexný štvoruholník, pričom strany $AB$ a~$CD$ nie sú rovnobežné a~$|AB|=|CD|$.
Stredy uhlopriečok $AC$ a~$BD$ označme $E$ a~$F$. Priamka~$EF$ pretína úsečky $AB$, $CD$
postupne v~bodoch $G$, $H$. Dokážte, že $|\angle AGH|=|\angle DHG|$.}
\podpis{Maďarsko}

{%%%%%   MEMO, priklad 4
Nájdite všetky celé čísla $k \ge 2$ také, že číslo $n^{n-1}-m^{m-1}$ nie je deliteľné číslom~$k$ pre žiadnu dvojicu $(m,n)$ rôznych kladných celých čísel menších alebo rovných~$k$.}
\podpis{Švajčiarsko}

{%%%%%   MEMO, priklad t1
Reálne čísla $x$, $y$, $z$ spĺňajú podmienku $x^2+y^2+z^2+9 = 4(x+y+z)$. Dokážte, že
$$
x^4+y^4+z^4+16(x^2+y^2+z^2)\ge 8(x^3+y^3+z^3)+27
$$
a~zistite, kedy v~nerovnosti platí rovnosť.}
\podpis{Slovensko, Ján Mazák}

{%%%%%   MEMO, priklad t2
Dané sú reálne čísla $a$, $b$, $c$, pričom ku každým dvom rovniciam spomedzi
$$
x^2+ax+b=0,\quad x^2+bx+c=0,\quad x^2+cx+a=0
$$
existuje práve jedno reálne číslo, ktoré je riešením obidvoch. Určte všetky možné hodnoty výrazu $a^2+b^2+c^2$.}
\podpis{Slovensko, Pavel Novotný}

{%%%%%   MEMO, priklad t3
Na tabuli sú napísané čísla $0,1,2,\dots,n$, pričom $n\ge 2$. V~každom kroku zotrieme číslo, ktoré je aritmetickým priemerom dvoch rôznych čísel, ktoré ešte na tabuli zostali. Také kroky robíme až do momentu, keď už nemôžeme zotrieť žiadne číslo. Označme $g(n)$ najmenší možný počet čísel, ktoré môžu na konci zostať na tabuli. Určte $g(n)$ pre každé~$n$.}
\podpis{Poľsko}

{%%%%%   MEMO, priklad t4
Každé políčko hracej plochy rozmerov $2009\times 2009$ ofarbíme jednou z~$n$~farieb (nemusíme použiť všetky farby).
Hovoríme, že daná farba je {\it súvislá}, ak má na celej ploche takú farbu iba jedno políčko, alebo ak pre
ľubovoľné dve políčka tejto farby môže šachová dáma prejsť z~jedného na druhé, pričom nikdy nezastaví na políčku inej farby (šachová dáma sa vie pohybovať vodorovne, zvisle a~diagonálne). Nájdite najväčšie také~$n$, že pre ľubovoľné ofarbenie hracej plochy je aspoň jedna {\it použitá} farba súvislá.}
\podpis{Poľsko}

{%%%%%   MEMO, priklad t5
V~rovnobežníku $ABCD$, v~ktorom $|\angle BAD| = 60^\circ$, označme $E$ priesečník uhlopriečok. Kružnica opísaná trojuholníku $ACD$ pretína priamku~$BA$ v~bode $K \ne A$, priamku~$BD$ v~$P \ne D$ a~priamku~$BC$ v~$L \ne C$.
Priamka~$EP$ pretína kružnicu opísanú trojuholníku $CEL$ v~bodoch $E$ a~$M$. Dokážte, že trojuholníky $KLM$ a~$CAP$ sú zhodné.}
\podpis{Slovinsko}

{%%%%%   MEMO, priklad t6
Daný je tetivový štvoruholník $ABCD$, pričom $|CD|=|DA|$. Body $E$, $F$ ležia postupne na stranách $AB$, $BC$, pričom $|\angle ADC| = 2|\angle EDF|$. Úsečky $DK$ a~$DM$ sú postupne výškou a~ťažnicou trojuholníka $DEF$.
Bod~$L$ je obrazom bodu~$K$ v~stredovej súmernosti podľa bodu~$M$. Dokážte, že priamky $DM$ a~$BL$ sú rovnobežné.}
\podpis{Poľsko}

{%%%%%   MEMO, priklad t7
Nájdite všetky dvojice celých čísel $(m,n)$, ktoré sú riešením rovnice
$$
(m+n)^4=m^2n^2+m^2+n^2+6mn.
$$}
\podpis{Chorvátsko}

{%%%%%   MEMO, priklad t8
Nájdite všetky riešenia rovnice
$$
2^x+2009=3^y5^z
$$
v~obore celých nezáporných čísel.}
\podpis{Litva}
