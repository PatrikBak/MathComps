{%%%%%   A-I-1
Použitím známych vzorcov
$$
\cos(x+y)=\cos x\cos y-\sin x\sin y,\quad \sin 2x=2\sin x\cos x,\quad \cos 2x=1-2\sin^2x
$$
dostaneme úpravou ľavej strany prvej rovnice
$$
\align
2\sin x\cos(x+y) + \sin y&=2\sin x(\cos x\cos y-\sin x\sin y) + \sin
y=\\
&=2\sin x\cos x\cos y+(1-2\sin^2x)\sin y=\\
&=\sin2x\cos y+\cos2x\sin y=\\&=\sin(2x+y).
\endalign
$$
Podobne ľavá strana druhej rovnice je rovná $\sin(2y+x)$. Zadaná sústava je teda ekvivalentná so sústavou
$$
\aligned
\sin(2x+y) &= 1,\\
\sin(2y+x) &= 1.
\endaligned
\tag1
$$
Keďže funkcia sínus nadobúda hodnotu~$1$ práve v~bodoch tvaru $\frac\pi2+2k\pi$, kde $k$ je celé číslo, budú riešením sústavy práve tie dvojice $(x,y)$, pre ktoré existujú celé čísla $k$, $l$ také, že
$$
2x+y=\tfrac\pi2+2k\pi,\quad 2y+x=\tfrac\pi2+2l\pi.
\tag2
$$
Odtiaľ buď odčítaním vhodných násobkov rovníc alebo priamym vyjadrením jednej premennej z~prvej rovnice a~dosadením do druhej rovnice po úprave dostaneme
$$
x=\tfrac\pi6+(4k-2l)\tfrac\pi3,\quad y=\tfrac\pi6+(4l-2k)\tfrac\pi3.
$$
Riešením sústavy sú teda dvojice $(\frac\pi6+(4k-2l)\frac\pi3,\frac\pi6+(4l-2k)\frac\pi3)$, kde $k$, $l$ sú ľubovoľné celé čísla. Nie je nutné robiť skúšku, nakoľko z~postupu vyplýva, že takéto dvojice $(x,y)$ spĺňajú vzťahy~\thetag2 a~teda aj sústavu~\thetag1.

\poznamka
Uvedený výsledok možno zapísať aj v~inom tvare. Keďže $x-y=(6k-6l)\frac\pi3=2\pi(k-l)$, možno pri položení $m=k-l$, $n=2k-l$ písať $x=\frac\pi6+n\frac{2\pi}3$, $y=x-2\pi m$, teda riešením sú dvojice $(\frac\pi6+n\frac{2\pi}3,\frac\pi6+n\frac{2\pi}3-2\pi m)$, kde $m$, $n$ sú ľubovoľné celé čísla. (Keď $k$, $l$ prebiehajú všetky možné dvojice celých čísel, tak aj $m$, $n$ prebehnú všetky možné dvojice celých čísel.)

\ineriesenie
Zrejme ak je riešením zadanej sústavy dvojica $(x,y)$, sú vďaka periodickosti funkcií sínus a~kosínus s~periódou $2\pi$ riešením aj všetky dvojice $(x+2k\pi,y+2l\pi)$. Budeme teda sústavu riešiť v~obore $\langle0,2\pi)$ a~na konci nájdené riešenia "posunieme" o~$(2k\pi,2l\pi)$, aby sme získali všeobecné riešenie.

Odčítaním rovníc sústavy získame po rozklade ľavej strany na súčin rovnicu
$$
(\sin x-\sin y)\bigr(2\cos(x+y)-1\bigl)=0.
$$
Rozlíšime dva prípady podľa toho, ktorý z~činiteľov je nulový.

\smallskip
I.
Ak $\sin x=\sin y$, tak vzhľadom na podmienku $x,y\in\langle0,2\pi)$ máme tri možnosti: buď $x=y$, alebo $x+y=\pi$, alebo $x+y=3\pi$ (\obr).
\insp{a58.1}%

Pri prvej možnosti po dosadení do pôvodnej sústavy získame jedinú rovnicu
$$
2\sin x\cos2x+\sin x=1.
$$
Z~nej s~využitím vzorca $\cos 2x=1-2\sin^2x$ a~po substitúcii $\sin x=t$ ekvivalentnými úpravami postupne dostaneme
$$
\align
2\sin x(1-2\sin^2x)+\sin x&=1,\\
2t(1-2t^2)+t&=1,\\
% -4t^3+3t-1&=0,\\
4t^3-3t+1&=0,\\
(t+1)(2t-1)^2&=0.
\endalign
$$
Pri poslednej úprave sme "uhádli" koreň $t=\m1$ a~rozklad na súčin získali vydelením mnohočlena $4t^3-3t+1$ koreňovým činiteľom $t+1$. Vzhľadom na použitú substitúciu $t=\sin x$ sú riešením ostatnej rovnice tie $x\in\langle0,2\pi)$, pre ktoré buď $\sin x=\m1$, alebo $\sin x=\frac12$, čiže $x\in\{\frc\pi6,\frc{5\pi}6,\frc{3\pi}2\}$. V~skúmanom obore teda dostávame ako riešenia zadanej sústavy dvojice $(\frac\pi6,\frac\pi6)$, $(\frac{5\pi}6,\frac{5\pi}6)$ a~$(\frac{3\pi}2,\frac{3\pi}2)$.

Pri druhej a~tretej možnosti, \tj. keď $x+y=\pi$ alebo $x+y=3\pi$, máme $\cos(x+y)=\m1$. Dosadením do pôvodnej sústavy (s~využitím rovnosti $\sin x=\sin y$) získame jedinú rovnicu $2\sin x\cdot(\m1)+\sin x=1$. Preto $\sin x=\m1$, a~teda aj $\sin y=\m1$. Odtiaľ získame v~skúmanom obore jediné riešenie $x=y=\frac{3\pi}2$, ktoré sme našli aj pri prvej možnosti.

\smallskip
II.
Ak $2\cos(x+y)-1=0$, čiže $\cos(x+y)=\frac12$, tak $x+y=\pm\frac\pi3+2k\pi$ pre nejaké celé~$k$ a~niektoré znamienko. Po dosadení do pôvodnej sústavy dostaneme jedinú rovnicu $\sin x+\sin y=1$, ktorá prejde na tvar $\sin x+\sin(\pm\frac\pi3+2k\pi-x)=1$. Vďaka periodickosti funkcie sínus s~periódou $2\pi$ a~použitím známeho vzorca
$$
\sin a+\sin b=2\sin\frac{a+b}2\cos\frac{a-b}2
$$
môžeme ľavú stranu upraviť na tvar
$$
\align
\sin x+\sin(\pm\tfrac\pi3+2k\pi-x)&=\sin x+\sin(\pm\tfrac\pi3-x)=2\sin(\pm\tfrac\pi6)\cos(x\mp\tfrac\pi6)=\\
&=2\cdot(\pm\tfrac12)\cos(x\mp\tfrac\pi6)=\pm\cos(x\mp\tfrac\pi6).
\endalign
$$
Riešená rovnica je teda ekvivalentná s~rovnicou $\pm\cos(x\mp\tfrac\pi6)=1$.

Pre "horné" znamienko dostávame $\cos(x-\frac\pi6)=1$, čomu v~obore $\langle0,2\pi)$ vyhovuje iba $x=\frac\pi6$. Odtiaľ $y=\frac\pi3+2k\pi-x=\frac\pi6+2k\pi$, čomu v~skúmanom obore vyhovuje iba $y=\frac\pi6$. Pre "dolné" znamienko máme $\cos(x+\tfrac\pi6)=\m1$, čomu v~skúmanom obore vyhovuje iba $x=\frac{5\pi}6$. Odtiaľ $y=\m\frac\pi3+2k\pi-x=2k\pi-\frac{7\pi}6$, čomu v~skúmanom obore vyhovuje iba $y=\frac{5\pi}6$. Dostávame tak len riešenia, ktoré sme objavili aj v~prvom prípade.

\zaver
Riešením v~obore $\langle0,2\pi)$ sú dvojice $(\frac\pi6,\frac\pi6)$, $(\frac{5\pi}6,\frac{5\pi}6)$, $(\frac{3\pi}2,\frac{3\pi}2)$. V~obore reálnych čísel sú to potom dvojice
$$
 (\tfrac\pi6+2k\pi,\tfrac\pi6+2l\pi),\quad (\tfrac{5\pi}6+2k\pi,\tfrac{5\pi}6+2l\pi),\quad (\tfrac{3\pi}2+2k\pi,\tfrac{3\pi}2+2l\pi),
$$
kde $k$, $l$ sú ľubovoľné celé čísla.

\návody
Dokážte platnosť známych súčtových vzorcov
$$
\cos(a+b)=\cos a\cos b-\sin a\sin b,\qquad \sin(a+b)=\sin a\cos b+\cos a\sin b.
$$
[Prvý vzorec možno dokázať napr. vhodným použitím kosínusovej vety pre trojuholník, ktorého dva vrcholy ležia na jednotkovej kružnici a~tretí vrchol je jej stredom. Druhý vzorec sa dá ľahko odvodiť z~prvého.]

\D
V~obore reálnych čísel vyriešte rovnicu
$$
1+\sin\frac{x+\pi}{5}\cdot\sin\frac{x-\pi}{11}=0.
$$
[55--A--S--3]

Dokážte, že $\cos 36^\circ=\frac{1+\sqrt5}4$.
[Ak $\alpha=36^\circ$, tak $\sin2\alpha=\sin3\alpha$, lebo $2\alpha+3\alpha=180^\circ$. Keďže $\sin2\alpha=2\sin\alpha\cos\alpha$ a
$$
\align
\sin3\alpha &= \sin(2\alpha+\alpha) = \sin2\alpha\cos\alpha+\cos2\alpha\sin\alpha = \\
 &= 2\sin\alpha\cos^2\alpha+(2\cos^2\alpha-1)\sin\alpha = \sin\alpha(4\cos^2\alpha-1),
\endalign
$$
dostávame $2\sin\alpha\cos\alpha=\sin\alpha(4\cos^2\alpha-1)$. Po vydelení nenulovým $\sin\alpha$ máme $4\cos^2\alpha-2\cos\alpha-1=0$. Číslo $t=\frac{1+\sqrt5}4$ je jediným kladným riešením kvadratickej rovnice $4t^2-2t-1=0$.]
\endnávod
}

{%%%%%   A-I-2
Označme $k$ kružnicu opísanú štvoruholníku $ABCD$. Nech priesečníky výšok trojuholníkov $ABC$ a $ABD$ sú postupne $U$ a $V$ (\obr).
\insp{a58.2}%

\niedorocenky{Niekoľko známych vlastností priesečníka výšok v~súvislosti s~opísanou kružnicou je zachytených v~návodných a~dopĺňajúcich úlohách. V~našej situácii sa nám bude hodiť, že }\dorocenky{Je známe, že obraz priesečníka výšok v~osovej súmernosti podľa strany trojuholníka leží na kružnici opísanej tomuto trojuholníku. V~našej situácii teda }
obraz~$U'$ bodu~$U$ v~osovej súmernosti podľa strany~$AB$ leží na kružnici~$k$, ktorá je opísanou kružnicou trojuholníka $ABC$. (Toto platí aj pre tupouhlý trojuholník $ABC$.) Podobne leží na kružnici~$k$ aj obraz~$V'$ bodu~$V$ v~osovej súmernosti podľa strany~$AB$.

Predpokladajme, že trojuholníky $ABC$ a~$ABD$ sú ostrouhlé. Potom body~$U$ a~$V$ ležia v~polrovine $ABC$. Priamky $CU'$ a~$DV'$ sú rovnobežné, preto štvoruholník $CU'V'D$ je tetivový lichobežník, a~teda je to lichobežník rovnoramenný. Z~tohto a~z~vlastností osovej súmernosti dostávame rovnosti
$$
|\uhol CDV'| = |\uhol U'V'D| = |\uhol UVV'|.
$$
Keďže body $C$ a~$U$ ležia v~rovnakej polrovine vzhľadom na priamku~$V'D$, sú priamky $CD$ a~$UV$ rovnobežné, ako sme mali dokázať. (V~poslednej úvahe sme využili, že body $D$, $V$, $V'$ ležia na priamke v~tomto poradí.)

V~prípade, keď aspoň jeden z~trojuholníkov $ABC$ a~$ABD$ je tupouhlý, je argumentácia veľmi podobná. Body $C$, $D$, $V'$, $U'$ vždy vytvoria rovnoramenný lichobežník, aj keď na jeho obvode môžu ležať v~inom poradí.

\ineriesenie
Nech $U$ je priesečník výšok trojuholníka $ABC$. Ukážeme, že dĺžka úsečky~$CU$ nezávisí od polohy bodu~$C$ na oblúku~$AB$ kružnice~$k$ opísanej trojuholníku $ABC$.

Budeme používať štandardné označenie pre veľkosti strán a~uhlov v~trojuholníku $ABC$. Označme navyše $K$ pätu výšky z~vrcholu~$A$ na stranu~$BC$. Predpokladajme najskôr, že trojuholník $ABC$ je ostrouhlý. Jednoduchým výpočtom z~vhodných trojuholníkov zistíme, že $|\uhol CUK|=\beta$. Využitím goniometrických funkcií v~trojuholníkoch $AKC$ a~$UKC$ dostaneme
$$
|CU|={|CK|\over \sin |\uhol CUK|}={b \cos\gamma\over \sin\beta}={c\cos\gamma\over \sin\gamma},
$$
posledná rovnosť vyplýva zo sínusovej vety v~trojuholníku $ABC$. Analogickým spôsobom možno ukázať, že aj v~prípade, keď $ABC$ je tupouhlý trojuholník, platí $|CU|=c|\cos\gamma|/\sin\gamma$. Dĺžka úsečky~$CU$ teda závisí len od dĺžky úsečky~$AB$ a~od veľkosti obvodového uhla $ACB$. V~našom prípade je úsečka~$AB$ aj oblúk kružnice pevný, preto sa dĺžka úsečky $CU$ nemení.

Body $C$ a~$D$ ležia na tom istom oblúku kružnice~$k$ určenom úsečkou~$AB$. Preto sú úsečky $CU$ a~$DV$ rovnako dlhé. Navyše sú rovnobežné, čiže štvoruholník $CDVU$ je rovnobežník. A~teda priamky $CD$ a~$VU$ sú rovnobežné.

\návody
Nech $ABC$ je ostrouhlý trojuholník s~priesečníkom výšok $V$ a~opísanou kružnicou $k$. Dokážte, že obraz $V'$ bodu $V$ v osovej súmernosti podľa priamky $AB$ leží na kružnici $k$. Majú túto vlastnosť aj tupouhlé trojuholníky?
[Stačí vyjadriť veľkosť uhla $AVB$ z trojuholníka $AVB$, v ktorom zvyšné dva uhly dopočítame z vhodných pravouhlých trojuholníkov. Tento uhol má veľkosť $180^\circ-|\uhol ACB|$, z čoho vyplýva, že štvoruholník $ACBV'$ je tetivový. Obraz priesečníka výšok v osovej súmernosti podľa strany leží na opísanej kružnici aj v prípade, že trojuholník je tupouhlý. Dôkaz sa spraví podobne vypočítaním veľkostí vhodných uhlov.]

Nech $V$ je priesečník výšok ostrouhlého trojuholníka $ABC$. Dokážte, že kružnice opísané trojuholníkom $ABV$, $BCV$, $CAV$ sú zhodné a porovnajte ich polomer s~polomerom kružnice opísanej trojuholníku $ABC$.
[Všetky tri kružnice sú obrazom kružnice opísanej trojuholníku $ABC$ v osovej súmernosti podľa príslušnej strany. Je to priamym dôsledkom predchádzajúcej návodnej úlohy.]

Daný je trojuholník $ABC$ s ortocentrom $H$. Vyjadrite veľkosť úsečky $CH$ pomocou dĺžok strán a veľkostí uhlov trojuholníka $ABC$. Snažte sa, aby vaše vyjadrenie bolo čo najjednoduchšie. [Pozri druhé uvedené riešenie súťažnej úlohy. Možných postupov aj vyjadrení je viacero.]

\D
Nech $ABC$ je ostrouhlý trojuholník s priesečníkom výšok $V$ a opísanou kružnicou $k$. Dokážte, že obraz bodu $V$ v stredovej súmernosti podľa stredu úsečky $AB$ leží na kružnici $k$. Majú túto vlastnosť aj tupouhlé trojuholníky?

V rovine sú dané tri navzájom rôzne zhodné kružnice so spoločným bodom $H$. Druhé priesečníky dvojíc týchto kružníc (rôzne od bodu $H$) označíme $A$, $B$, $C$. Dokážte, že bod $H$ je priesečníkom výšok trojuholníka $ABC$.

Daný je ostrouhlý trojuholník $ABC$ s pätami výšok $D$, $E$, $F$ ležiacimi postupne na stranách $AB$, $BC$, $CA$. Obraz bodu $F$ v stredovej súmernosti podľa stredu strany $AB$ leží na priamke $DE$. Určte veľkosť uhla $BAC$.
[57--A--II--3]

Daný je trojuholník $ABC$. Dokážte, že os uhla $ACB$ a os strany $AB$ sa pretínajú na kružnici opísanej trojuholníku $ABC$.

V tetivovom štvoruholníku $ABCD$ označme $L$, $M$ stredy kružníc vpísaných postupne do trojuholníkov $BCA$, $BCD$. Ďalej označme $R$ priesečník kolmíc vedených z bodov $L$ a $M$ postupne na priamky $AC$ a $BD$. Dokážte, že trojuholník $LMR$ je rovnoramenný.
[56--A--III--2]

Na kružnici s polomerom $r$ leží $5$ rôznych bodov $A$, $B$, $C$, $D$, $E$ v tomto poradí, pričom platí $|AC|=|BD|=|CE|=r$. Dokážte, že trojuholník, ktorého vrcholmi sú ortocentrá trojuholníkov $ACD$, $BCD$ a $BCE$, je pravouhlý.
[C-P-S trojstretnutie 2006/1]

Dokážte, že všetky stredy strán a päty výšok v ľubovoľnom trojuholníku ležia na jednej kružnici. (Táto kružnica je známa pod názvom {\it Feuerbachova kružnica} alebo kružnica deviatich bodov -- okrem spomínaných šiestich bodov na nej totiž ešte ležia stredy úsečiek spájajúcich priesečník výšok s jednotlivými vrcholmi trojuholníka.)

Nech $ABC$ je trojuholník a $P$ bod v jeho rovine. Označme $D$, $E$, $F$ päty kolmíc z~bodu~$P$ na priamky $AB$, $BC$, $CA$. Dokážte, že ak bod $P$ leží na kružnici opísanej trojuholníku $ABC$, tak body $D$, $E$, $F$ ležia na priamke. (Táto priamka sa nazýva {\it Simsonovou priamkou} bodu $P$.) Má takúto vlastnosť aj nejaký bod $P$ ležiaci mimo kružnice opísanej trojuholníku $ABC$?

Nech $P$ je bod na kružnici opísanej trojuholníku $ABC$. Označme $H$ priesečník výšok trojuholníka $ABC$. Nech $X$ je priesečník Simsonovej priamky bodu $P$ s~úsečkou~$PH$. Dokážte, že $X$ je stredom úsečky $PH$ a~leží na Feuerbachovej kružnici trojuholníka $ABC$. (Riešenie tejto náročnej úlohy je možné nájsť na stránke {\tt http://mathforum.org/library/drmath/view/61688.html}.)

Nech $PQ$ je ľubovoľný priemer kružnice opísanej trojuholníku $ABC$. Dokážte, že Simsonove priamky bodov $P$ a~$Q$ sú na seba kolmé a pretínajú sa na Feuerbachovej kružnici trojuholníka $ABC$. (Druhá časť tejto úlohy je naozaj náročná.)
\endnávod
}

{%%%%%   A-I-3
Predpokladajme, že prirodzené čísla $x$, $y$ vyhovujú zadaniu, \tj.
$$
\frac{xy^2}{x+y}=p,
\tag1
$$
pričom $p$ je prvočíslo. Označme $d$ najväčšieho spoločného deliteľa čísel $x$, $y$. Potom $x=da$, $y=db$, pričom prirodzené čísla $a$, $b$ už nemajú žiadneho spoločného deliteľa väčšieho ako $1$, teda sú nesúdeliteľné. Rovnosť \thetag1 tak môžeme po vynásobení menovateľom $x+y$ a~vydelení kladným číslom $d$ zapísať v~tvare
$$
d^2ab^2 = p(a+b).
\tag2
$$
Keďže $b^2$ delí ľavú stranu, musí deliť aj pravú stranu. Avšak čísla $a$, $b$ sú nesúdeliteľné, preto aj čísla $b^2$, $a+b$ sú nesúdeliteľné\niedorocenky{\footnote{Poz. prvú návodnú úlohu.}}. Podľa známeho tvrdenia\niedorocenky{\footnote{Poz. druhú návodnú úlohu.}}\dorocenky{\footnote{Ak $k$, $l$ sú nesúdeliteľné a~$k\mid lm$, tak $k\mid m$.}} potom $b^2\mid p$. Prvočíslo $p$ má iba dva delitele: $1$ a~$p$. Z~nich je druhou mocninou iba číslo $1$, preto nutne $b=1$. Rovnosť \thetag2 teda môžeme prepísať na
$$
d^2a=p(a+1).
\tag3
$$
Zopakujeme teraz podobné úvahy. Keďže $a$ delí ľavú stranu, musí deliť aj pravú stranu. Pritom čísla $a$, $a+1$ sú nesúdeliteľné, takže $a\mid p$. Nutne preto buď $a=1$, alebo $a=p$. Rozlíšime dva prípady.

Ak $a=1$, tak po dosadení do \thetag3 máme $d^2=2p$. Zrejme $2p$ je druhou mocninou prirodzeného čísla jedine v~prípade $p=2$. Potom $d=2$ a dostávame dvojicu $x=2$, $y=2$.

Ak $a=p$, dosadením do \thetag3 a~vydelením kladným $p$ dostaneme $d^2=p+1$, čiže $p=(d+1)(d-1)$. Čísla $d+1$, $d-1$ sú teda dva rôzne (nezáporné) delitele prvočísla $p$, z~čoho nutne $d-1=1$, $d+1=p$. Takže $d=2$, $p=3$ a~dostávame dvojicu $x=6$, $y=2$.

Skúškou sa ľahko presvedčíme, že obe nájdené dvojice vyhovujú zadaniu.

\zaver
Zadaniu vyhovujú dvojice $(2,2)$ a~$(6,2)$.

\návody
Dokážte, že ak sú čísla $a$, $b$ nesúdeliteľné, \tj. $\nsd(a,b)=1$, tak aj a)~$\nsd(b,a+b)=1$; b)~$\nsd(b^2,a+b)=1$. [a)~Ak by $b$ a~$a+b$ mali deliteľa $d>1$, ten by delil aj ich rozdiel, ktorý je rovný $a$, teda $d$ by bol spoločným deliteľom čísel $a$, $b$. b)~Ak by $b^2$ a~$a+b$ mali spoločného deliteľa $d>1$, mali by aj spoločného prvočíselného deliteľa $p$, ktorý by nutne delil aj $b$, teda $b$ a~$a+b$ by mali spoločného deliteľa $p>1$; ďalej rovnako ako v~časti~a).]

Dokážte, že ak $\nsd(k,l)=1$ a~$k\mid lm$, tak $k\mid m$. [Keďže $k\mid lm$, tak $lm=kt$ pre nejaké celé $t$. Keďže $k$, $l$ sú nesúdeliteľné, tak $kx+ly=1$ pre nejaké celé $x$, $y$. Vynásobením číslom $m$ dostaneme $m=kmx+lmy=k(mx+ty)$, teda $k\mid m$.]

\D
Určte všetky dvojice prvočísel $p$, $q$, pre ktoré platí $p+q^2=q+p^3$. [55--B--II--1]

Nájdite všetky dvojice prvočísel $p$, $q$, pre ktoré platí $p+q^{2}=q+145p^{2}$. [55--C--II--4]
\endnávod
}

{%%%%%   A-I-4
a) Položme napríklad $a=1$, $d=1$. Potom postupnosť~\thetag{$\ast$} má tvar
$$
1,2,3,4,\dots,
$$
\tj. obsahuje všetky prirodzené čísla. Medzi nimi je samozrejme nekonečne veľa $k$-tych mocnín pre každé~$k$. (Vyhovujúce $a$, $d$ možno zvoliť aj mnohými inými spôsobmi.)

\smallskip
b) Položme napríklad $a=2$, $d=4$. Postupnosť~\thetag{$\ast$} má vtedy tvar
$$
2,6,10,14,\dots,
$$
\tj. obsahuje čísla $4n+2$, kde $n=0,1,2,\dots$ Táto postupnosť obsahuje len párne čísla, preto určite neobsahuje žiadnu $k$-tu mocninu {\it nepárneho\/} čísla. Avšak $k$-ta mocnina ľubovoľného {\it párneho\/} čísla je deliteľná číslom $2^k$, teda aj číslom $4$ (pre $k\ge2$), a~nemôže byť tvaru $4n+2$. Takže zvolená postupnosť neobsahuje žiadnu $k$-tu mocninu prirodzeného čísla pre žiadne $k=2,3,\dots$ (Podobne možno zdôvodniť, že vyhovuje postupnosť, ktorú dostaneme voľbou $a=p$, $d=p^2$ pre ľubovoľné prvočíslo~$p$.)

\smallskip
c) Položme napríklad $a=8$, $d=16$. Postupnosť~\thetag{$\ast$} má vtedy tvar
$$
8,24,40,56,\dots,
$$
\tj. obsahuje čísla $16n+8$, kde $n=0,1,2,\dots$ Keďže $16n+8=8(2n+1)$, zvolená postupnosť neobsahuje žiadnu druhú mocninu prirodzeného čísla. V~rozklade na súčin prvočísel má totiž každý jej člen činiteľ $2^3$, zatiaľ čo druhé mocniny majú v~rozklade na súčin prvočísel všetky exponenty párne. Na druhej strane, v~danej postupnosti sa zrejme nachádzajú všetky čísla $(2\cdot1)^3,(2\cdot3)^3,(2\cdot5)^3,\dots$, čiže $8$-násobky tretích mocnín nepárnych čísel. Takže postupnosť obsahuje nekonečne veľa tretín mocnín prirodzených čísel. (Opäť sme mohli $a$, $d$ zvoliť aj inak, stačí zobrať $a=p^3$, $d=p^4$, kde $p$ je ľubovoľné prvočíslo.)

d) Ak sa v~postupnosti~\thetag{$\ast$} nenachádza žiadna $k$-ta mocnina, dokazované tvrdenie triviálne platí. Predpokladajme, že sa v~postupnosti nachádza aspoň jedna $k$-ta mocnina. Všeobecný člen v~\thetag{$\ast$} má tvar $a+nd$, pričom $n$ je nezáporné celé číslo. Pre nejaké prirodzené číslo~$m$ teda platí $m^k=a+nd$. Chceme ukázať, že medzi členmi z~\thetag{$\ast$} je nekonečne veľa ďalších $k$-tych mocnín. Všetky členy postupnosti~\thetag{$\ast$} dávajú po delení číslom~$d$ rovnaký zvyšok (taký, ako dáva po delení číslom~$d$ číslo~$a$). Zároveň vieme, že ak dve čísla dávajú po delení $d$ rovnaký zvyšok, dávajú rovnaký zvyšok po delení $d$ aj ich $k$-te mocniny. V~postupnosti~\thetag{$\ast$} teda budú ležať aj $k$-te mocniny čísel $m+td$ pre ľubovoľné prirodzené číslo~$t$. Naozaj, podľa binomickej vety máme
$$
\aligned
(m+td)^k&=m^k+km^{k-1}td+\textstyle\binom k2m^{k-2}t^2d^2+\cdots+kmt^{k-1}d^{k-1}+t^kd^k=\\
&=m^k+d\cdot\left(km^{k-1}t+\textstyle\binom k2m^{k-2}t^2d+\cdots+kmt^{k-1}d^{k-2}+t^kd^{k-1}\right)=\\
&=m^k+d\cdot M=a+nd+dM=a+d(n+M).
\endaligned
$$
Keďže $M$ (výraz vo veľkej zátvorke) je zjavne prirodzené číslo, $(m+td)^k=a+d(n+M)$ je členom postupnosti~\thetag{$\ast$} pre každé prirodzené $t$. Takže \thetag{$\ast$} obsahuje nekonečne veľa $k$-tych mocnín.

\návody
Dokážte, že medzi číslami tvaru $8n+4$ sa nachádza nekonečne veľa druhých mocnín prirodzených čísel.

Dokážte, že medzi číslami tvaru $8n+4$ sa nenachádza žiadna tretia mocnina prirodzeného čísla.

Dokážte tvrdenie z~časti~d) pre prípad $k=2$, \tj. dokážte, že postupnosť~\thetag{$\ast$} buď neobsahuje žiadnu druhú mocninu, alebo obsahuje nekonečne veľa druhých mocnín prirodzených čísel.

Dokážte, že ak dve čísla dávajú po delení číslom~$d$ rovnaký zvyšok, tak aj ich $k$-te mocniny dávajú po delení číslom~$d$ rovnaký zvyšok. [Ak $d\mid a-b$, tak aj $d\mid (a-b)(a^{k-1}+a^{k-2}b+\cdots+b^{k-1})=a^k-b^k$. Inou možnosťou je tvrdenie dokázať pomocou binomickej vety.]

\D
Rozhodnite, či existuje aritmetická postupnosť, ktorá neobsahuje žiadne Fibonacciho číslo (\tj. číslo, ktoré je členom postupnosti $(a_n)_{n=0}^\infty$, pričom $a_1=a_2=1$ a~$a_{n+2}=a_{n+1}+a_n$ pre $n\ge1$).
\endnávod
}

{%%%%%   A-I-5
Očíslujme si zaradom vrcholy daného mnohouholníka číslami od $1$ po $2008$.

\smallskip
a) Po chvíli nájdeme postup, ako presúvať mince, aby sme došli k~zadanému cieľu. Popíšeme jednu z~možností.

Mince z~vrcholov $1, 2, \dots, 251$ postupne zhromaždíme na jednej kôpke vo vrchole s~číslom $251$. Ich pohyb budeme vyvažovať presúvaním mincí z~vrcholov $1758$ až $2008$ do vrcholu s~číslom $1758$. Takto vytvoríme dve kôpky po $251$ minciach. Podobným spôsobom zhromaždíme mince z~vrcholov s~číslami $252$ až $502$ na jednej kôpke vo vrchole s~číslom $502$. Ich pohyb vyvážime vytvorením rovnako početnej kôpky vo vrchole s~číslom $1507$. Takto postupujeme aj ďalej; posledné dve kôpky s~$251$ mincami budú vo vrcholoch s~číslami $1004$ a~$1005$.

\smallskip
b) Postup spĺňajúci pravidlá presunu mincí sa nájsť nedá, čo v~ďalšom dokážeme.

Priraďme každej minci číslo vrcholu, v~ktorom sa nachádza. Všimnime si súčet~$S$ všetkých čísel priradených minciam. Čo sa stane, keď presunieme dvojicu mincí? Ak presun nenastane medzi dvoma vrcholmi s~číslami $1$ a~$2008$, hodnota súčtu~$S$ sa nezmení: jednej z~presúvaných mincí sa priradené číslo o~$1$ zväčší a~druhej sa o~$1$ zmenší. Ak medzi vrcholmi s~číslami $1$ a~$2008$ presun nastane, hodnota $S$ sa buď nezmení (vtedy, keď sa obe mince presúvajú medzi týmito vrcholmi, teda si len navzájom vymenia pozície), alebo sa zmení o~$2008$ (môže vzrásť alebo klesnúť). Celkovo to môžeme zhrnúť tak, že zvyšok súčtu~$S$ po delení číslom $2008$ sa pri presunoch mincí nemení.

Hodnota $S$ je na začiatku $1+2+\cdots+2008=1004\cdot 2009$. Toto číslo dáva po delení číslom~$8$ zvyšok~$4$. Na konci máme $251$ kôpok po $8$~mincí. Každá z~kôpok prispieva do $S$ násobkom čísla~$8$, preto hodnota $S$ by mala byť na konci deliteľná číslom~$8$. Zvyšok hodnoty $S$ po delení číslom~$8$ sa však nemení, lebo $8\mid 2008$. Zvyšky hodnoty $S$ po delení~$8$ sú rôzne pre úvodnú a~cieľovú pozíciu, čiže popísanými presunmi mincí nemôžeme dosiahnuť z~úvodnej pozície pozíciu s~$251$ kôpkami po $8$~mincí.

\návody
Na stole je $n$ pohárov otočených hore dnom. V jednom kroku môžeme otočiť $k$ z~nich naopak. Je možné dosiahnuť, aby po konečnom počte krokov boli všetky poháre otočené dole dnom? Vyriešte túto úlohu pre $n=9$, $k=2$ a $n=9$, $k=5$. Odpovede dôsledne zdôvodnite. [Pre $n=9$ a $k=5$ to je možné. Pre $n=9$ a $k=2$ nie, pretože sa nemení parita počtu pohárov otočených hore dnom.]

V každom vrchole štvorca je jedna minca. Vyberieme dve mince a premiestnime každú z nich do susedného vrcholu tak, že jedna sa posunie v smere a druhá proti smeru chodu hodinových ručičiek. Rozhodnite, či je možné týmto spôsobom postupne premiestniť všetky mince do jedného vrcholu. [Podobne ako v riešení súťažnej úlohy budeme uvažovať zvyšok súčtu čísel priradených minciam po delení štyrmi. Je možné rozdeliť všetky možné pozície do skupín podľa tohto zvyšku, úvodná a cieľová pozícia sú v rôznych skupinách, preto sa z jednej nedá dostať do druhej.]

Vyriešte predchádzajúcu návodnú úlohu s pravidelným osemuholníkom namiesto štvorca.

\D
Okolo ohňa sedí $n+1$ psov ($n\ge 1$). Jeden z~nich je šéf a~má $n$ kostí, ostatní nemajú nič. V~jednom kroku zvolíme dvoch psov $A$ a~$B$ (nie nutne rôznych), z~ktorých každý má aspoň jednu kosť a~spolu majú aspoň dve kosti. Zoberieme jednu kosť psovi~$A$ a~dáme ju jednému zo susedov psa~$B$ a~zoberieme jednu kosť psovi~$B$ a~dáme ju jednému zo susedov psa~$A$. Pre ktoré $n$ sa po sérii vhodných krokov môžeme dostať do situácie, že každý pes okrem šéfa má jednu kosť?
[KMS 2005/6, 3. zimná séria, úloha 7]

Okolo okrúhleho stola sedí $n$ detí. Erika je z~nich najstaršia a~má $n$ cukríkov. Ostatné deti nemajú žiadne cukríky. Erika sa rozhodla, že cukríky rozdelí a stanovila nasledovné pravidlá. V~každom kole zdvihnú ruky všetky deti, ktoré majú pri sebe aspoň dva cukríky. Erika jedného z~prihlásených vyberie a~ten dá každému svojmu susedovi jeden cukrík. (V~prvom kole sa teda prihlási iba Erika a~dá svojim susedom po cukríku.) Zistite, pre ktoré $n\ge 3$ môže delenie po konečnom počte kôl skončiť tak, že každé dieťa bude mať práve jeden cukrík.
[C-P-S trojstretnutie 2006/2]

Čísla $1, 2, \dots, n$ sú v tomto poradí napísané vo vrcholoch pravidelného $n$-uholníka. V~jednom kroku môžeme dve susedné čísla nahradiť ich aritmetickým priemerom. Je možné dosiahnuť, aby boli všetky napísané čísla rovnaké?
[KMS 2006/7, 2. letná séria, úloha 10]
\endnávod
}

{%%%%%   A-I-6
Nech $Q$ je obrazom bodu $P$ v~stredovej súmernosti so stredom $M$.
Bod $Q$ leží na osi uhla $ACB$ práve vtedy, keď je rovnako vzdialený od priamok $AC$ a $BC$.
\insp{a58.3}%

Chceme využiť rovnosť dĺžok úsekov $AE$ a~$BD$ v~súvislosti s~bodom~$Q$. Všimnime si preto trojuholníky $AEQ$ a~$BDQ$ (\obr). Bod~$Q$ je rovnako vzdialený od priamok $AC$ a~$BC$ práve vtedy, keď trojuholníky $AEQ$ a~$BDQ$ majú rovnaký obsah. Dokážeme tvrdenie o~rovnosti obsahov týchto trojuholníkov.

Priamka~$BQ$ je rovnobežná s~priamkou~$AD$, pretože je jej obrazom v~stredovej súmernosti so stredom~$M$.
Preto trojuholníky $QBD$ a~$QBA$ majú rovnaký obsah (majú rovnaké výšky na spoločnú základňu~$QB$).
Podobne aj obsah trojuholníka $QAE$ je rovnaký, ako obsah trojuholníka $QAB$. Takže náš dôkaz je hotový.

%Priamka $BQ$ je rovnobežná s priamkou $PD$, pretože je jej obrazom v stredovej súmernosti so stredom $M$. %Preto trojuholníky $BQD$ a $BQP$ majú rovnaký obsah (majú rovnaké výšky na spoločnú základňu $BQ$). Podobne %majú aj trojuholníky $AQE$ a $AQP$ rovnaký obsah.
%
%Ostáva ukázať, že obsahy trojuholníkov $AQP$ a $BQP$ sú rovnaké. To však ľahko vidíme z toho, že $PQ$ je uhlopriečkou rovnobežníka $AQBP$.

\ineriesenie
V~zadaní sa spomína stredová súmernosť so stredom~$M$. Takáto súmernosť často pomáha v~úlohách týkajúcich sa ťažníc trojuholníka. Poďme ju využiť aj tu. Nech sa v~tejto stredovej súmernosti zobrazí bod~$C$ do bodu~$C'$ a~bod~$P$ do bodu~$Q$. Ďalej označíme $K$ priesečník priamok $C'B$ a~$AD$. Priesečníky priamky~$C'P$ s~priamkami $AB$ a~$AC$ označíme $N$ a~$L$ (\obr).
\insp{a58.4}%

Máme dokázať, že bod~$Q$ leží na osi uhla $ACB$. Vďaka vlastnostiam stredovej súmernosti toto platí práve vtedy, keď bod~$P$ leží na osi uhla $AC'B$ (vnútorného v~trojuholníku $AC'B$). Je známe (pozri druhú návodnú úlohu), že toto je pravda práve vtedy, keď bod~$N$ rozdelí úsečku~$AB$ v~pomere dĺžok úsečiek $AC$ a~$BC$.
Naším cieľom preto bude určiť veľkosť pomeru $AN:BN$. Nech $|BD|=|AE|=x$, $|CD|=y$ a~$|AC|=b$.

Trojuholníky $ADC$ a~$KDB$ sú podobné, preto
$$
|BK|=|AC|\cdot{|BD|\over |CD|}=b\cdot {x\over y}.
$$
Rovnoľahlosť rovnobežných priamok $BC'$ a~$AC$ so stredom v~bode~$P$ zachováva pomery, preto
$$
|LE|=|AE|\cdot{|C'B|\over |KB|}=x\cdot {b\over b\cdot {x\over y}}=y.
$$
Nakoniec, trojuholníky $ANL$ a~$BNC'$ sú podobné, z~čoho dostaneme
$$
{|AN|\over |BN|}={|AL|\over |BC'|}={|AE|+|LE|\over |BC'|}={x+y\over |BC'|}={|AC'|\over |BC'|}.
$$
Platnosť tejto rovnosti znamená, že priamka~$C'N$, a~teda aj priamka~$C'P$, je osou uhla $AC'B$.


\návody
%1
a) Nájdite množinu bodov, ktoré sú rovnako vzdialené od dvoch daných priamok.\hfil\break
b) S využitím vlastnosti z a) dokážte, že osi vnútorných uhlov trojuholníka sa pretínajú v jednom bode.

%2
Dokážte, že os vnútorného uhla v trojuholníku pretína protiľahlú stranu v pomere priľahlých strán. [Nech $K$ je priesečník osi uhla $ACB$ so stranou $AB$. Použijeme sínusovú vetu v trojuholníkoch $CKA$ a $CKB$, využijeme, že uhly $AKC$ a $BKC$ sú doplnkové a uhly $ACK$ a $BCK$ majú rovnakú veľkosť. Iné riešenie: pomer $|AK|\colon |BK|$ je rovnaký, ako pomer obsahov trojuholníkov $AKC$ a $BKC$, pritom tieto dva trojuholníky majú rovnako dlhé výšky z vrcholu $K$.]

%3
Daný je trojuholník $ABC$. Nájdite množinu bodov $X$ takých, že trojuholník $ABX$ má rovnaký obsah ako trojuholník $ABC$. [Dvojica priamok rovnobežných s priamkou $AB$ a rovnako od nej vzdialených.]

%4
Nech $ABCD$ je lichobežník so základňami $AB$ a $CD$. Dokážte, že priesečník priamok $AC$ a $BD$, priesečník priamok $AD$ a $BC$ a stredy základní tohto lichobežníka ležia na priamke. [Uvažujme rovnoľahlosti, ktoré zobrazia úsečku $AB$ na úsečku $CD$. Tieto rovnoľahlosti sú dve a ich stredmi sú tie dva priesečníky zo zadania úlohy. Obe tieto rovnoľahlosti zachovávajú pomery, preto zobrazia stred úsečky $AB$ na stred úsečky $CD$. Takže stredy uvažovaných rovnoľahlostí ležia na spojnici stredov základní.]

%5
Nech $ABCD$ je konvexný štvoruholník. Stredy jeho strán označíme postupne $K$, $L$, $M$, $N$.\hfil\break
a) Dokážte, že $KLMN$ je rovnobežník.\hfil\break
b) Aký je pomer obsahov štvoruholníkov $KLMN$ a $ABCD$?

\D

Dokážte, že ťažnice v trojuholníku sa pretínajú v jednom bode a rozdelia trojuholník na $6$ častí s rovnakým obsahom.

Dokážte, že ak $x$, $y$, $z$ sú dĺžky ťažníc trojuholníka $ABC$, tak existuje trojuholník s~dĺžkami strán rovnými $x$, $y$, $z$. Aký obsah má tento trojuholník, ak obsah trojuholníka $ABC$ je $S$? [Využite stredovú súmernosť podľa stredu strany $BC$.]

Je daný trojuholník $ABC$. Vnútri jeho strán $BC$, $CA$, $AB$ uvažujme postupne body $K$, $L$, $M$ také, že úsečky $AK$, $BL$, $CM$ sa pretínajú v bode $U$.
Ak trojuholníky $AMU$ a $KCU$ majú obsah $P$ a trojuholníky $MBU$ a $CLU$ obsah $Q$, potom $P=Q$. Dokážte.
[49--A--S--2]

Daný je trojuholník $ABC$ a body $K$, $L$, $M$ ležiace postupne vnútri strán $BC$, $CA$, $AB$ tak, že priamky $AK$, $BL$, $CM$ majú spoločný bod $X$.\hfil\break
a) Dokážte, že pomer $|AM|\colon |BM|$ je rovnaký, ako pomer obsahov trojuholníkov $ACX$ a $BCX$.\hfil\break
b) Dokážte, že
$$
{|AM|\over |BM|}\cdot {|BK|\over |CK|}\cdot {|CL| \over |AL|}=1.
$$
(Toto tvrdenie je časťou {\it Cevovej vety}. Porovnajte túto vetu s {\it Menelaovou vetou}. Všimnite si, že často je výhodné vo výpočtoch i~dôkazoch previesť pomer vzdialeností na pomer obsahov. Použite tento prístup v druhej návodnej úlohe.)

Nech $K$, $L$, $M$ sú po rade vnútorné body strán $BC$, $CA$, $AB$ daného trojuholníka
$ABC$ také, že kružnice vpísané dvojiciam trojuholníkov $ABK$ a $CAK$, $BCL$ a $ABL$,
$CAM$ a $BCM$ majú vonkajší dotyk. Potom sa priamky $AK$, $BL$, $CM$ pretínajú v~jednom bode. Dokážte.
[49--A--I--2]

Určte všetky konvexné štvoruholníky $ABCD$ s nasledujúcou vlastnosťou: Vnútri štvoruholníka $ABCD$ existuje bod $E$ taký, že každá priamka, ktorá prechádza týmto bodom a pretína strany $AB$ a $CD$ vo vnútorných bodoch, delí štvoruholník $ABCD$ na dve časti s rovnakým obsahom. Svoju odpoveď zdôvodnite.
[49--A--II--4]
\endnávod
}

{%%%%%   B-I-1
Označme $n$ trojciferné číslo určené prvým trojčíslím (zľava) hľadaného štvorciferného čísla, ktoré je potom rovné $10n+8$. Podľa zadania úlohy platí
$$
\eqalignno{
 8&\mid 10n+8,&(1)\cr
 9&\mid 10n+7,&(2)\cr
 7&\mid 10n+9.&(3)
}$$
Zo vzťahu \thetag{1} vyplýva $8\mid10n$, čiže $4\mid 5n$. Čísla $4$ a~$5$ sú nesúdeliteľné, preto $4\mid n$, čiže $n=4k$, kde $k$ je prirodzené číslo. Dosadením $n=4k$ do vzťahu \thetag{2} dostaneme $9\mid 40k+7$, čiže $9\mid 4k+7$.
Z~tabuľky zvyškov čísel $4k+7$ po delení deviatimi
$$
\tabskip=8pt\vbox{\offinterlineskip
\halign{\strut\hfil$#$~~\vrule&&\hfil $#$\cr
k&0&1&2&3&4&5&6&7&8\cr
\noalign{\hrule}
4k+7&7&2&6&1&5&0&4&8&3\cr
}}
$$
vidíme, že toto číslo je deliteľné deviatimi práve vtedy, keď číslo~$k$ po delení deviatimi dáva zvyšok~$5$. Preto $k=9l+5$, kde $l$ je celé číslo, takže $n=4k=36l+20$. Dosadením takého $n$ do vzťahu \thetag{3} dostaneme $7\mid 360l+209$, čiže $7\mid 3l-1$. Opäť zostavíme tabuľku zvyškov, tentoraz po delení čísla $3l-1$ siedmimi.
$$
\tabskip=8pt\vbox{\offinterlineskip
\halign{\strut\hfil$#$~~\vrule&&\hfil $#$\cr
l      &0&1&2&3&4&5&6\cr
\noalign{\hrule}
3l-1   &6&2&5&1&4&0&3\cr
}}
$$
Vidíme, že $7\mid 3l-1$ práve vtedy, keď $l=7m+5$, kde $m$ je celé číslo. Odtiaľ dostávame, že všetky celočíselné~$n$ spĺňajúce trojicu podmienok \thetag{1}--\thetag{3} sú tvaru $n=36l+20=252m+200$.

Dodajme, že namiesto zostavovania tabuliek sme mohli využiť úpravy
$$
\eqalignno{
40k+7&=36k+4(k-5)+27,\cr
360l+209&=357l+3(l-5)+224,\cr}
$$
z~ktorých by sme ako skôr dostali $9\mid k-5$ a~$7\mid l-5$.

Číslo $n=252m+200$ je trojciferné jedine pre $m\in\{0,1,2,3\}$; hľadané~$n$ je preto z~množiny $\{200,452,704,956\}$ a~na tabuli bolo napísané jedno z~čísel $2\,008$, $4\,528$, $7\,048$, $9\,568$. Skúškou (ktorá však pri našom
postupe nie je nutná) môžeme overiť, že každé z~týchto štyroch čísel vyhovuje zadaniu úlohy.

\ineriesenie
Pri druhom postupe budeme úvahy o~deliteľnosti výhodne zapisovať kongruenciami. Zápis $a\equiv b \pmod{m}$ (čítame "$a$ je kongruentné s~$b$ modulo~$m$") znamená, že čísla $a$, $b$ dávajú po delení číslom~$m$ rovnaké zvyšky, čiže $m\mid a-b$.

Označme $N$ hľadané štvorciferné číslo končiace číslicou 8. Keďže pri jej zámene číslicou~7, resp.~9 dostaneme číslo $N-1$, resp. $N+1$, všetky podmienky zo zadania úlohy možno vyjadriť štyrmi kongruenciami
$$
\eqalignno{
 N&\equiv 8 \pmod{10},&(4)\cr
 N&\equiv 0 \pmod{8},&(5)\cr
 N-1&\equiv 0 \pmod{9},&(6)\cr
 N+1&\equiv 0 \pmod{7}.&(7)
}$$

Zo vzťahu \thetag{5} vyplýva $N=8k$, kde $k$ je celé číslo. Dosadením do vzťahu \thetag{4} dostaneme $8k\equiv 8 \pmod{10}$, čiže $4k\equiv 4\pmod{5}$, čo po delení číslom~$4$ (nesúdeliteľným s~číslom~$5$) vedie k~podmienke $k\equiv 1 \pmod{5}$. Preto $k=5l+1$, kde $l$ je celé číslo. Dosadením $N=8k=40l+8$ do vzťahu \thetag{6} obdržíme podmienku  $40l+7\equiv 0\pmod{9}$. Jej úpravou dostaneme
$$
40l\equiv-7\equiv-7+9\cdot23=200\pmod{9}
$$
a~po vydelení číslom~$40$ (nesúdeliteľným s~číslom~$9$) dôjdeme k~podmienke $l\equiv 5\pmod9$. Existuje teda
celé číslo $m$ také, že $l=9m+5$. Dosadením $N=40l+8=360m+208$ do vzťahu \thetag{7} dostaneme $360m+209\equiv 0\pmod7$, čiže $3m\equiv1\pmod7$. Úpravou
$$
3m\equiv1\equiv1+2\cdot7=15\pmod7
$$
po vydelení číslom $3$ vyjde $m\equiv5\pmod 7$, takže $m=7n+5$, kde $n$ je celé číslo. Hľadané~$N$ je preto tvaru
$N=360m+208=2\,520n+2\,008$. Také $N$ je štvorciferné práve vtedy, keď $n\in\{0,1,2,3\}$. Na tabuli preto mohlo byť
napísané ktorékoľvek číslo z~množiny $\{2\,008, 4\,528, 7\,048, 9\,568\}$ a~žiadne iné.


\návody
Nájdite najmenšie prirodzené číslo~$n$ s~vlastnosťou a) $5\mid n+1$, $6\mid n$, $7\mid n-1$; b) $4\mid n-2$, $5\mid n-3$, $6\mid n-4$.
[a) 204, b) 58]

Určte všetky prirodzené čísla~$n$, ktoré sa nedajú zapísať v~tvare $n=3x+5y$, kde $x$, $y$ sú prirodzené čísla.
[35--C--I--2]

\D
Pre ľubovoľné trojciferné číslo určíme zvyšky po delení číslami 2, 3, 4,~\dots, 10 a~získaných deväť čísel sčítame. Určte najmenšiu možnú hodnotu takého súčtu. [47--C--I--1]
\endnávod
}

{%%%%%   B-I-2
Odčítaním prvej rovnice od druhej dostaneme po úprave
$$
\postdisplaypenalty 10000
(z-x)(2z+2x+y)=0.
$$
Sú preto možné dva prípady, ktoré rozoberieme samostatne.

\smallskip
% \item
{a)}
{\it Prípad $z-x=0$}. Dosadením $z=x$ do prvej rovnice sústavy dostaneme $x^2+xy=y^2+x^2$, čiže $y(x-y)=0$. To znamená,
že platí $y=0$ alebo $x=y$. V~prvom prípade dostávame trojice $(x,y,z)=(x,0,x)$, v~druhom $(x,y,z)=(x,x,x)$; také trojice sú riešeniami danej sústavy pre ľubovoľné reálne číslo~$x$, ako ľahko overíme dosadením (aj keď taká skúška pri našom postupe vlastne nie je nutná).

% \item
{b)}
{\it Prípad $2z+2x+y=0$}. Dosadením $y=\m2x-2z$ do prvej rovnice sústavy dostaneme
$$
x^2+x(-2x-2z)=(-2x-2z)^2+z^2,\quad\text{čiže}\quad5(x+z)^2=0.
$$
Posledná rovnica je splnená práve vtedy, keď $z=\m x$, vtedy však $y=\m2x-2z=0$. Dostávame trojice $(x,y,z)=(x,0,\m x)$, ktoré sú riešeniami danej sústavy  pre každé reálne~$x$, ako overíme dosadením. (O~takej skúške platí to isté čo v~prípade~a).)

\odpoved
Všetky riešenia $(x,y,z)$ danej sústavy sú trojice troch typov:
$$
(x,x,x),\quad(x,0,x),\quad(x,0,\m x),
$$
kde $x$ je ľubovoľné reálne číslo.

\ineriesenie
Obe rovnice sústavy sčítame. Po úprave dostaneme rovnicu
$$
y(x+z-2y)=0
$$
a~opäť rozlíšime dve možnosti.

\smallskip
% \item
{a)} {\it Prípad $y=0$}. Z~prvej rovnice sústavy ihneď vidíme, že $x^2=z^2$, čiže $z={\pm} x$. Skúškou overíme, že každá z~trojíc $(x,0,x)$ a~$(x,0,\m x)$ je pre ľubovoľné reálne~$x$ riešením.

% \item
{b)} {\it Prípad $x+z-2y=0$}. Dosadením $y=\frac12(x+z)$ do prvej rovnice sústavy dostaneme
$$
x^2+\frac{x(x+z)}{2}=\frac{(x+z)^2}{4}+z^2,\quad\text{po úprave}
\quad x^2=z^2.
$$
Platí teda $z=\m x$ alebo $z=x$. Dosadením do rovnosti $x+z-2y=0$ v~prvom prípade dostaneme $y=0$, v~druhom prípade
$y=x$. Zodpovedajúce trojice $(x,0,\m x)$ a~$(x,x,x)$ sú riešeniami pre každé reálne~$x$ (prvé z~nich sme však našli už v~časti~a)).

\návody
V~obore reálnych čísel riešte sústavu rovníc
$$\eqalign{
x^2+1&= 2y,\cr
y^2+1&= 2x.}
$$
[Odčítaním rovníc dostaneme po úprave $(x-y)(x+y+2)=0$. Ak $x=y$, potom $x=y=1$. Pre $x+y=\m2$ sústava nemá riešenie.]

\D
V~obore reálnych čísel riešte sústavu rovníc
$$\eqalign{
x^2 - y&= z^2,\cr
y^2 - z&= x^2,\cr
z^2 - x&= y^2.}$$
[57--A--S--1]

V~obore reálnych čísel riešte sústavu rovníc
$$
\align
x^2+y+z=&2,\cr
x+y^2+z=&2,\cr
x+y+z^2=&2.
\endalign
$$
[Rozdiel prvých dvoch rovníc sústavy možno upraviť na tvar $(x-y)(x+y-1)=0$. Riešeniami sú ľubovoľné permutácie trojíc $(1,1,0)$, $ (2,\m1,\m1)$ a~tiež dve trojice $(a,a,a)$ pre $a=\m1\pm\sqrt3$.]
\endnávod
}

{%%%%%   B-I-3
Označme $a$ veľkosť strán $AB$ a~$CD$ a~$v$ vzdialenosť ich priamok, ktorá je zároveň rovná výške trojuholníka $AFD$ z~vrcholu~$A$ (\obr). Z~podmienky $EF\parallel BD$ podľa vety~$uu$ vyplýva, že trojuholníky $BCD$ a~$ECF$ sú podobné; označme $k\in(0,1)$ koeficient ich podobnosti. Keď ho vypočítame, bude úloha vyriešená.
\insp{b58.1}%

Keďže $|FC|=ka$, $|FD|=(1-k)a$ a~výšky trojuholníkov $ECF$, $ABE$ zo spoločného vrcholu~$E$ majú veľkosti $kv$, resp. $(1-k)v$, pre obsahy trojuholníkov $AFD$ a~$ABE$ platí
$$
S_{AFD}=\frac{(1-k)av}{2}=\frac{a(1-k)v}{2}=S_{ABE},
$$
takže oba obsahy sa rovnajú pre ľubovoľné $k\in(0,1)$. Obsah trojuholníka $ECF$ má hodnotu $S_{ECF}=\frac12 ka\cdot kv=\frac12 k^2av$ a~obsah celého rovnobežníka $ABCD$ je daný vzťahom $S_{ABCD}=av$, preto môžeme obsah trojuholníka $AEF$ vyjadriť takto:
$$\eqalign{
S_{AEF}&=S_{ABCD}-S_{ABE}-S_{ECF}-S_{AFD}=\cr
       &=av\left(1-\tfrac12(1-k)-\tfrac12 k^2 -\tfrac12(1-k)\right)
%        =\cr&
         =av\left(k-\tfrac12k^2\right).}
$$
Obsahy trojuholníkov $ABE$, $AFD$ teda budú zhodné s~obsahom trojuholníka $AEF$ práve vtedy, keď bude platiť
$$
\tfrac12(1-k)=k-\tfrac12k^2,\quad\text{čiže}\quad k^2-3k+1=0.
$$
Táto kvadratická rovnica má korene
$$
k_{1,2}=\frac{3\pm\sqrt5}2,
$$
z~ktorých podmienke $k\in(0,1)$ vyhovuje iba koreň $k=\frac12\bigl(3-\sqrt5\bigr)$. Dodajme, že pre také $k$ platí
$$
1-k=\frac{\sqrt5-1}{2}=\frac{k}{1-k}.
$$

\odpoved
Hľadané body $E$, $F$ sú určené pomermi
$$
|CE|:|EB|=|CF|:|FD|=\bigl(\sqrt5-1\bigr):2.
$$

\poznamka
Rovnosť $(1-k):1=k:(1-k)$ zo záveru riešenia znamená, že body $E$, $F$ delia príslušné strany rovnobežníka v~pomere tzv. {\it zlatého rezu}. Vyjadrujú to rovnosti
$$
|CE|:|EB|=|EB|:|BC|\quad\text{a}\quad
|CF|:|FD|=|FD|:|DC|.
$$

\návody
Základňa~$AB$ lichobežníka $ABCD$ je trikrát dlhšia ako základňa~$CD$. Označme $M$ stred strany $AB$ a~$P$ priesečník úsečky~$DM$ s~uhlopriečkou~$AC$. Vypočítajte pomer obsahov trojuholníka $CDP$ a~štvoruholníka $MBCP$. [55--C--II--1]

Daný je lichobežník $ABCD$ ($AB\parallel CD$) s~jednotkovým obsahom, pre ktorý platí $|AB|=2|CD|$. Označme $K$, $L$ postupne stredy strán $BC$ a~$CD$. Určte obsah trojuholníka $AKL$.
[Obsahy trojuholníkov $ABK$, $CLK$ a~$ADL$ sú postupne $\frac13$, $\frac1{12}$ a~$\frac16$, teda obsah trojuholníka $AKL$ je $\frac5{12}$.]

\D
Daný je rovnobežník $ABCD$. Priamka vedená bodom~$D$ pretína úsečku~$AC$ v~bode~$G$, úsečku~$BC$ v~bode~$F$ a~polpriamku~$AB$ v~bode~$E$ tak, že trojuholníky $BEF$ a~$CGF$ majú rovnaký obsah. Určte pomer $|AG|:|CG|$. [54--B--I--2]
\endnávod
}

{%%%%%   B-I-4
Podľa \obr\ môžeme na plán umiestniť 8 disjunktných obdĺžnikov $2\times3$ (stredné políčko plánu zostane prázdne). Aby sme s~istotou zasiahli loď, musíme sa spýtať na aspoň jedno políčko v~každom z~ôsmich vyznačených~obdĺžnikov, preto je nutný počet otázok aspoň~8.

Na \obr\ je uvedený príklad výberu ôsmich políčok, na ktoré sa stačí spýtať, aby sa už mimo nich nedala na plán umiestniť žiadna loď $2\times3$. Preto týchto 8~otázok k~zasiahnutiu lode vždy stačí.
\instwop{b58.2}{b58.3}{4.7}

Z~oboch uvedených úvah vyplýva nasledujúci záver.

\odpoved
Najmenší počet otázok, ktoré potrebujeme, aby sme s~istotou loď zasiahli, je práve~8.

\návody
Na pláne $n\times n$ hráme hru lode. Nachádza sa na ňom jedna loď $2\times3$. Môžeme sa spýtať na ľubovoľné políčko plánu, a~ak loď zasiahneme, hra končí. Ak nie, pýtame sa znova. Určte najmenší počet otázok, ktoré potrebujeme, aby sme s~istotou loď zasiahli. Úlohu riešte pre $n=3,4,5$. [1, 2, 4]

Predošlú úlohu riešte pre jednu loď $2\times 2$ na pláne $8\times8$ a~na pláne $7\times 7$. [16, 12]

\D
Určte najmenšie prirodzené číslo~$k$ s~vlastnosťou: keď vyberieme $k$ rôznych čísel z~množiny $\{1,2,\dots,1999\}$, potom medzi nimi existujú dve, ktorých súčet je $2000$. [49--C--S--1]

Určte najmenšie prirodzené číslo~$k$ s~vlastnosťou: keď vyberieme $k$ rôznych čísel z~množiny $\{1,2,\dots,2000\}$, potom medzi nimi existujú dve, ktorých súčet alebo rozdiel je $667$. [49--A--S--3]

Nájdite najmenšie prirodzené čísla~$k$, pre ktoré platia jednotlivé tvrdenia a), b) a~c): Ak obsadíme figúrkami ľubovoľných $k$~políčok šachovnice $8\times8$, tak budú obsadené niektoré
a) tri susedné políčka niektorého riadku,
b) tri susedné políčka niektorého šikmého radu,
c) štyri susedné políčka niektorého riadku alebo stĺpca.
Pod šikmým radom rozumieme takú skupinu políčok, ktorých uhlopriečky jedného z~oboch smerov ležia na jednej priamke. [49--C--I--3]

Dokážte, že na šachovnici $8\times8$ nemožno rozmiestniť 7~strelcov tak, aby všetky políčka šachovnice boli ohrozené. Ďalej ukážte, že možno na šachovnici rozmiestniť 8~strelcov tak, aby každé neobsadené políčko šachovnice bolo ohrozené niektorým zo strelcov. [37--B--II--2, 37--B--S--1]

Aký najväčší počet kráľov možno umiestniť na šachovnicu $8\times8$, aby sa žiadni dvaja navzájom neohrozovali? [16. Šachovnicu rozdeľte na 16 častí $2\times2$, v~každej z~nich môže byť najviac jeden kráľ.]

%% Pole tabulky $n\times n$, kde $n\geq3$, jsou střídavě černá
%% a~bílá jako na obyčejné šachovnici, přičemž pole v~levém horním
%% rohu je černé. Bílá pole budeme barvit načerno následujícím
%% postupem. V~jednom kroku vybereme libovolný obdélník $2\times3$
%% nebo $3\times2$, ve kterém jsou ještě tři bílá pole, a~tato tři pole
%% začerníme. Pro která $n$ můžeme po určitém počtu kroků začernit
%% celou tabulku? [57--A--II--3]
\endnávod
}

{%%%%%   B-I-5
Označme $\alpha$, $\beta$, $\gamma$ zvyčajným spôsobom veľkosti vnútorných uhlov trojuholníka $ABC$ (\obr). Bod~$K$ leží na osi úsečky~$AB$, preto $|AK|=|KB|$. Trojuholník $AKB$ je rovnoramenný so základňou~$AB$,
\insp{b58.4}
jeho vnútorné uhly pri vrcholoch $A$ a~$B$ sú teda zhodné. Podľa vety o~obvodových uhloch sú zhodné aj uhly $BCK$ a~$BAK$, resp. $ACK$ a~$ABK$, preto sú zhodné aj uhly $BCK$ a~$ACK$. Polpriamka $CK$ je teda osou uhla $ACB$:
$$
|\uhol ACK|=|\uhol BCK|=\frac\gamma2.
$$
Keďže bod~$P$ leží na osi strany~$AC$, je trojuholník $ACP$ rovnoramenný a~jeho vnútorné uhly pri základni~$AC$ majú veľkosť $\frac12\gamma$, takže jeho vonkajší uhol $APK$ pri vrchole~$P$ má veľkosť $\frac12\gamma+\frac12\gamma=\gamma$. Rovnako z~rovnoramenného trojuholníka $BCQ$ odvodíme, že aj veľkosť uhla $BQK$ je $\gamma$. Podľa vety o~obvodových uhloch sú zhodné uhly $ABC$ a~$AKC$, teda uhol $AKC$ (čiže uhol $AKP$) má veľkosť $\beta$ a~-- celkom analogicky~-- uhol $BKQ$ má veľkosť~$\a$.

V~každom z~trojuholníkov $AKP$ a~$BKQ$ už poznáme veľkosti dvoch vnútorných uhlov ($\beta$,~$\gamma$, resp. $\alpha$, $\gamma$), takže vidíme, že zostávajúce uhly $KAP$ a~$KBQ$ majú veľkosti $\alpha$, resp.~$\beta$.

Z~predošlého vyplýva, že trojuholníky $AKP$ a~$KBQ$ sú zhodné podľa vety $usu$, lebo majú zhodné strany $AK$ a~$KB$
aj obe dvojice k~nim priľahlých vnútorných uhlov.

K~uvedenému postupu dodajme, že výpočet uhlov $KAP$ a~$KBQ$ cez uhly $APK$ a~$BQK$ možno obísť takto: zhodnosť uhlov $KAP$ a~$BAC$ (resp. $KBQ$ a~$ABC$) vyplýva zo zhodnosti uhlov $KAB$ a~$PAC$ (resp. $KBA$ a~$QBC$).


\návody
Nech $ABC$ je ostrouhlý trojuholník. Označme $K$, $L$ päty výšok z~vrcholov $A$, $B$, ďalej $M$ stred strany~$AB$ a~$V$ priesečník výšok trojuholníka $ABC$. Dokážte, že os uhla $KML$ prechádza stredom úsečky~$VC$. [54--B--II--3]

\D
V~tetivovom štvoruholníku $ABCD$ označme $L$, $M$ stredy kružníc vpísaných postupne trojuholníkom $BCA$, $BCD$. Ďalej označme $R$ priesečník kolmíc vedených z~bodov $L$ a~$M$ postupne na priamky $AC$ a~$BD$. Dokážte, že trojuholník $LMR$ je rovnoramenný.
[56--A--III--2]

Označme $S$ stred kružnice vpísanej trojuholníku $ABC$. Dokážte, že stred kružnice opísanej trojuholníku $ABS$ leží na kružnici opísanej trojuholníku $ABC$.
[Pre bod~$K$ z~riešenia súťažnej úlohy platí $|KA|=|KB|=|KS|$, lebo $S\in KC$ a~$|\uhol KAS|=\frac12(\alpha+\gamma)$, takže aj $|\uhol KSA|=\frac12(\alpha+\gamma)$.]
\endnávod
}

{%%%%%   B-I-6
Najskôr si všimnime, že menovateľ zlomku možno postupným vynímaním rozložiť na súčin $(m+2)(n-1)$. Preto bude výhodné položiť $a=m+2$, $b=n-1$ a~pre nové neznáme (nenulové!) celé čísla $a$, $b$ skúmať, kedy je hodnota daného výrazu
$$
V=\frac{m+3n-1}{mn+2n-m-2}=\frac{(a-2)+3(b+1)-1}{ab}=\frac{a+3b}{ab}
$$
(ako vyžaduje zadanie) {\it celé kladné\/} číslo (používajme ďalej zvyčajný termín {\it prirodzené číslo\/}). Uveďme dva možné prístupy k~riešeniu takej otázky.

Pri prvom spôsobe využijeme rozklad
$$
V=\frac{a+3b}{ab}=\frac{1}{b}+\frac{3}{a}
$$
a~zrejmé odhady
$$
0<\Bigl|\frac1{b}\Bigr|\leq1,\qquad
0<\Bigl|\frac3{a}\Bigr|\leq3.
$$
Keby platilo $a<0$, bolo by $\frac3{a}<0$, čiže $V<\frc1{b}\leq1$, teda $V$ by nebolo prirodzené číslo. Preto nutne platí $a>0$.

Pre $a>6$ je $\frc3{a}<\frac12$, a~teda $V<\frc1b+\frac12$, takže nerovnosť $V\ge1$ platí jedine vtedy, keď $\frc1b>\frac12$, čo spĺňa jediné celé číslo $b=1$, pre ktoré máme $1<V<\frac32$. Preto musí platiť $1\le a\le6$. Týchto šesť možností jednotlivo rozoberieme:

{\smallskip
\item{$\triangleright$}
${a=1}$. Číslo $V=3+\frc1{b}$ je celé jedine pre $b=\pm1$, kedy je aj kladné. V~pôvodných neznámych dostávame dve riešenia $(m,n)=(\m1,2)$ a~$(m,n)=(\m1,0)$.

\parskip2pt
\item{$\triangleright$}
${a=2}$. Číslo $V=\frac32+\frc1{b}$ je prirodzené práve vtedy, keď $b=\pm2$; zodpovedajúce riešenia sú $(m,n)=(0,3)$ a~$(m,n)=(0,\m1)$.

\item{$\triangleright$}
${a=3}$. Číslo $V=1+\frc1{b}$ je prirodzené práve vtedy, keď $b=1$, teda $(m,n)=(1,2)$.

\item{$\triangleright$}
${a=4}$. Číslo $V=\frac34+\frc1{b}$ je prirodzené práve vtedy, keď $b=4$, teda $(m,n)=(2,5)$.

\item{$\triangleright$}
${a=5}$. Číslo $V=\frac35+\frc1{b}$ zrejme nie je celé pre žiadne celé $b$.

\item{$\triangleright$}
${a=6}$. Číslo $V=\frac12+\frc1{b}$ je prirodzené práve vtedy, keď $b=2$, teda $(m,n)=(4,3)$.
}

\odpoved
Existuje práve 7~dvojíc celých čísel $(m,n)$, pre ktoré je hodnota daného výrazu~$V$ celým kladným číslom,
sú to dvojice
$$
(m,n)\in\{(-1,2),(-1,0),(0,3),(0,-1),(1,2),(2,5),(4,3)\}.
$$

\ineriesenie
Hľadáme nenulové celé $a$, $b$, pre ktoré $a+3b=kab$ pre vhodné prirodzené~$k$. Označme $d\ge1$ najväčší spoločný deliteľ takých čísel $a$, $b$. Potom $a=xd$ a~$b=yd$ pre celé nesúdeliteľné čísla $x$, $y$, ktoré spĺňajú rovnicu $(x+3y)d=kxyd^2$, čiže $x+3y=kxyd$. Odtiaľ vyplýva, že číslo~$y$ delí nesúdeliteľné číslo~$x$. To je možné jedine vtedy, keď $y=\pm1$.

V~prípade $y=1$ máme rovnicu $x+3=kxd$, čiže $3=x(kd-1)$. Keďže platí $kd\ge1$ (čísla $k$, $d$ sú prirodzené), tak buď $x=1$ a~$kd-1=3$ (potom $kd=4$, a~teda $d\in\{1,2,4\}$, takže $(a,b)=(d,d)$ je jedna z~dvojíc $(1,1)$, $(2,2)$, $(4,4)$), alebo $x=3$ a~$kd-1=1$ (potom $kd=2$, a~teda $d\in\{1,2\}$, takže $(a,b)=(3d,d)$ je jedna z~dvojíc $(3,1)$, $(6,2)$).

V~prípade $y=\m1$ máme rovnicu $x-3=\m kxd$, čiže $3=x(1+kd)$, čo vzhľadom na nerovnosť $1+kd\ge2$ znamená, že $x=1$ a~$1+kd=3$, takže $kd=2$, a~teda $d\in\{1,2\}$, preto $(a,b)=(d,\m d)$ je jedna z~dvojíc $(1,\m1)$, $(2,\m2)$.

Zistili sme, že existuje sedem vyhovujúcich dvojíc $(a,b)$, vypísať zodpovedajúce riešenia $(m,n)=(a-2,b+1)$ je už jednoduché (poz. odpoveď vyššie).

\návody
Nájdite všetky riešenia rovnice $xyz=3(x+y+z)$ v~obore celých kladných čísel. Riešenia, ktoré sa líšia iba poradím, nepovažujeme za rôzne. [36--B--II--3b]

Nech $a$, $b$ sú nesúdeliteľné prirodzené čísla. Potom prirodzené čísla $x$, $y$, $z$, kde $x=a(a+b)$, $y=b(a+b)$, $z=ab$, sú nesúdeliteľné a~platí $\frac1x+\frac1y=\frac1z$. Dokážte.

Nech naopak $x$, $y$, $z$ sú nesúdeliteľné prirodzené čísla, pre ktoré platí $\frac1x+\frac1y=\frac1z$. Ukážte, že potom existujú prirodzené čísla $a$, $b$ také, že $x=a(a+b)$, $y=b(a+b)$, $z=ab$.
[33--C--I--5]
\endnávod
}

{%%%%%   C-I-1
Dokopy chlapci dostali $3+2+1+8+2\cdot2+2\cdot4+2\cdot8=42$~Sk. Toto číslo možno jediným spôsobom vyjadriť ako súčet štyroch po sebe idúcich prirodzených čísel: $42=9+10+11+12$. Štyria chlapci teda (v~nejakom poradí) vyzbierali sumy $9$, $10$, $11$ a~$12$~Sk.

Žiadny chlapec nemohol dostať $8$~Sk zároveň od druhého aj od piateho okoloidúceho (inak by mal aspoň $16$~Sk, najviac však mohol každý z~chlapcov dostať $12$~Sk). Takže od druhého a~piateho majú traja chlapci po $8$~Sk a~jeden od nich nedostal nič. Najviac jeden z~týchto troch chlapcov mohol dostať $4$~Sk od štvrtého okoloidúceho, inak by mali už aspoň dvaja chlapci aspoň $12$~Sk. Štvrtý okoloidúci musel teda dať $4$~Sk práve jednému z~nich a~$4$~Sk zostávajúcemu chlapcovi. Bez peňazí prvého a~tretieho okoloidúceho teda majú chlapci vybraných $12$, $8$, $8$ a~$4$~Sk. Chlapec, ktorý dostal v~súčte od druhého, štvrtého a~piateho okoloidúceho dvanásť korún, už nemohol dostať od prvého a~tretieho okoloidúceho nič, lebo by mal viac ako dvanásť korún.  Ten, ktorý dostal v~súčte od druhého, štvrtého a~piateho okoloidúceho $4$~Sk, musel dostať od prvého a~tretieho v~súčte maximálnu možnú čiastku, \tj. $3+2=5$~Sk, inak by mal dokopy menej ako $9$~Sk (dostal teda práve $9$~Sk a~vyzbieral najmenej). Takže najmenej vyzbieral Tomáš, lebo on dostal od prvého okoloidúceho $3$~Sk, a~najviac Peter, ktorý od prvého okoloidúceho nedostal nič.

\input pstricks
\newrgbcolor{gray}{0 1 0}
Úvahy ľahko dokončíme a~ukážeme, že popísané rozdelenie {\it je\/} skutočne {\it možné}. Ako už vieme, Tomáš vyzbieral $9$~Sk a~Peter $12$~Sk. Jakub, ktorý dostal $2$~Sk od prvého, nemohol dostať od tretieho nič, takže dostal celkom $10$~Sk, a~Martin $11$~Sk. Všetky úvahy môžeme prehľadne usporiadať do tabuľky, ktorú postupne dopĺňame.
$$
\def\hline{\noalign{\hrule}}
\vbox{\moveleft2.5mm
\vbox{\offinterlineskip\medmuskip1mu\let\cdot\times
 \def\P{\rlap{\enspace$\to{}$P}}
 \def\H{\rlap{\enspace $\to{}$T}}
 \def\g#1{{\gray\tenbf#1}}\def\bf#1{{\text{\red\tenbf\fam\bffam#1}}}
 \halign{&\strut\vrule\hbox to 3.8em{\hss#\unskip\hss}\cr
1  &2   & 3  &4     &5    &$\Sigma$\cr\hline \hline
   &\bf8&    & 0    & 0   &\cr        \hline
   &0   &    & 0    &\bf8 &\cr        \hline
0  &0   & 0  &\bf4  &\bf8 &12\P\cr    \hline
\g3&0   &\g2 &\bf4  & 0   &$\le9$\H\cr     \hline
$\bf1+\bf2+\bf3$&$1\cdot\bf8$&$2\cdot\bf2$&$2\cdot\bf4$&$2\cdot\bf8$&\cr
}}}
$$

\návody
Ukážte, že prirodzené číslo~$n$ možno vyjadriť ako súčet štyroch po sebe idúcich čísel práve vtedy, keď $n\ge 10$ a~$n$ dáva zvyšok dva po delení štyrmi. [$(k+1)+(k+2)+(k+3)+(k+4)=4k+10$]

\D
Dokážte, že ľubovoľné prirodzené číslo $n\ge 3$, ktoré nie je mocninou čísla~$2$, možno vyjadriť ako súčet niekoľkých po sebe idúcich prirodzených čísel. [$n=\frac{n-1}2+\frac{n+1}2$ pre $n$ nepárne, $n=(\frac np-\frac{p-1}2)+(\frac
 np-\frac{p-1}2+1)+\dots+(\frac np+\frac{p-1}2)$ pre $n=p\cdot q$, kde $p>1$ je nepárny deliteľ]

V~klobúku je päť loptičiek a~na každej z~nich je napísané jedno prirodzené číslo. Súčet čísel na loptičkách v~klobúku je $27$ a~čísla na ľubovoľných dvoch loptičkách sa líšia aspoň o~dva. Dokážte, že v~klobúku nie je loptička s~číslom~$6$. [V~klobúku môžu byť buď loptičky s~číslami $1$, $3$, $5$, $7$, $11$, alebo $1$, $3$, $5$, $8$, $10$.]
\endnávod
}

{%%%%%   C-I-2
\fontplace
\tpoint A; \tpoint B; \bpoint C;
\bpoint D; \bpoint E; \tpoint O; \tpoint P;
\cpoint\a; \cpoint\a; \cpoint\a; \cpoint\b; \cpoint\xy0,-.5 \b;
\bpoint r; \rBpoint r; \lpoint v;
[1] \hfil\Obr

\fontplace
\tpoint A; \tpoint B; \bpoint\xy1,0 C;
\bpoint\xy-1,1 D; \bpoint E; \tpoint O;
\rpoint F;
\bpoint r; \bpoint r; \rBpoint r;
\lBpoint a; \rBpoint b; \lBpoint a;
\bpoint v_c; \bpoint v_c;
[2] \hfil\Obr

Označme odvesny trojuholníka $ABC$ zvyčajným spôsobom $a$, $b$ a~protiľahlé uhly $\alpha$, $\beta$. Stred prepony~$AB$ (ktorý je súčasne stredom opísanej kružnice) označíme~$O$ (\obr).

Výška $v=CP$ rozdeľuje trojuholník $ABC$ na trojuholníky $ACP$ a~$CBP$ podobné trojuholníku $ABC$ podľa vety $uu$
($\alpha+\beta=90^\circ$), úsečka~$OC$ je kolmá na $DE$ a~navyše $|OC|=|OA|=r$ (polomer opísanej kružnice). Odtiaľ $|\uhol OCA|=|\uhol OAC|=\alpha$ a~$|\uhol DCA|=90^\circ-|\uhol OCA|=\beta$.

Pravouhlé trojuholníky $ACP$ a~$ACD$ so spoločnou preponou~$AC$ sa teda zhodujú aj v~uhloch pri vrchole~$C$. Sú preto zhodné, dokonca súmerne združené podľa priamky~$AC$. Analogicky sú trojuholníky $CBP$ a~$CBE$ súmerne združené podľa~$BC$. Takže $|CD|=|CE|=v$, čiže
$|DE|=2v=2ab/\sqrt{a^2+b^2}$, lebo z~dvojakého vyjadrenia dvojnásobku obsahu trojuholníka $ABC$ vyplýva $v=ab/|AB|$, pričom $|AB|=\sqrt{a^2+b^2}$.

\poznamka
Namiesto dvojakého vyjadrenia obsahu môžeme na výpočet výšky~$CP$ využiť podobnosť trojuholníkov $CBP$ a~$ABC$: $\sin\alpha=|CP|/|AC|=|BC|/|AB|$.

\twocpictures

\ineriesenie
Úsečka~$OC$ je strednou priečkou lichobežníka $DABE$, lebo je rovnobežná so základňami a~prechádza stredom~$O$ ramena~$AB$. Preto $D$ je obrazom bodu~$E$ v~súmernosti podľa stredu~$C$. Obraz~$F$ bodu~$B$ v~tej istej súmernosti leží na polpriamke~$AD$ za bodom~$D$ (\obr). Máme $|CF|=|BC|=a$, uhol $ACF$ je pravý, a~teda trojuholníky $AFC$ a~$ABC$ sú zhodné. Vidíme, že $CD$ je výška v~trojuholníku $AFC$ zhodná s~výškou~$v_c$ trojuholníka $ABC$, a~$DE$ je jej dvojnásobkom. Veľkosť výšky~$v_c$ dopočítame rovnako ako v~predchádzajúcom riešení.

\odpoved
$|DE|=2ab/\sqrt{a^2+b^2}$.
%% {\smc Další řešení (\obr):} Stejně jako v~předchozím
%%  zjistíme, že $OC$ je střední příčka lichoběžníku $DABE$.
%%  Dvojnásobek součtu obsahů trojúhelníků $OCA$ a~$OCB$ je roven
%%  dvojnásobku obsahu trojúhelníku $ABC$, tedy $rv+rv=ab$. Odtud po
%%  úpravě s~využitím Pythagorovy věty: $|DE|=2v=2ab/\sqrt{a^2+b^2}$.

\návody
Vyjadrite výšku~$v_c$ pravouhlého trojuholníka $ABC$ s~pravým uhlom pri vrchole~$C$ pomocou strán $a$, $b$, $c$ tohto trojuholníka.

\D
Nech $k$ je kružnica opísaná pravouhlému trojuholníku $ABC$ s~preponou~$AB$ dĺžky~$c$. Označme $S$ stred strany~$AB$ a~$D$ a~$E$ priesečníky osí strán $BC$ a~$AC$ s~jedným oblúkom~$AB$ kružnice~$k$. Vyjadrite obsah trojuholníka $DSE$ pomocou dĺžky prepony $c$. [$c^2/8$]

Vyjadrite obsah rovnoramenného lichobežníka $ABCD$ so základňami $AB$ a~$CD$ pomocou dĺžok $a$, $c$ jeho základní a~dĺžky~$b$ jeho ramien. [$\frac14(a+c)\sqrt{4b^2-(a-c)^2}$]

V~obdĺžniku $ABCD$ platí $|AB|>|BC|$. Oblúk~$AC$ kružnice, ktorej stred leží na strane~$AB$, pretína stranu~$CD$ v~bode~$M$. Dokážte, že priamky $AM$ a~$BD$ sú navzájom kolmé. [48--C--I--2]
\endnávod
}

{%%%%%   C-I-3
V~riešení budeme označovať číslo, ktoré vznikne otočením poradia cifier čísla~$n$, ako $\bar n$. Rozoberieme tri
prípady.

(i)
Číslo~$n$ má tvar $aabb$, kde $a$, $b$ sú rôzne cifry. Takže $n=1100a+11b$ a~$\bar n=1100b+11a$. Číslo~$7$ má deliť ako $n$, tak $\bar{n}$, teda aj ich rozdiel $n-\bar{n}=1089(a-b)$ a~súčet $n+\bar{n}=1111(a+b)$. Keďže ani číslo $1089$, ani číslo $1111$ nie sú násobkom siedmich a~sedem je prvočíslo, tak $7\mid a-b$ aj $7\mid a+b$. Ak použijeme rovnakú úvahu ešte raz, vidíme, že $7\mid (a-b)+(a+b)=2a$ a~$7\mid (a+b)-(a-b)=2b$, teda $7\mid a$ a~$7\mid b$, čiže $a,b\in\{0,7\}$. Cifry $a$, $b$ sú navzájom rôzne, preto jedna z~nich musí byť~$0$. Ale potom jedno z~čísel $aabb$, $bbaa$ nie je štvorciferné. Hľadané číslo~$n$ teda nemôže mať uvedený tvar.

(ii)
Číslo~$n$ má tvar $abab$. Potom $7\mid n=1010a+101b$ a~tiež $7\mid \bar{n}=1010b+101a$. Podobne ako v~predchádzajúcom prípade odvodíme, že $7\mid n-\bar{n}=909(a-b)$ a~$7\mid n+\bar{n}=1111(a+b)$, a~z~rovnakých dôvodov ako v~predchádzajúcom prípade zisťujeme, že $7\mid a$, $7\mid b$. Niektorá z~cifier by teda musela byť~$0$. Číslo~$n$ tak nemôže mať ani tvar $abab$.

(iii) Číslo~$n$ má tvar $abba$. Potom otočením poradia cifier vznikne to isté číslo, takže máme jedinú podmienku
 $7\mid 1001a+110b$. Keďže $7\mid 1001$ a~$7\nmid 110$, je táto podmienka ekvivalentná s~podmienkou $7\deli b$. Preto $b\in\{0,7\}$, $a\in\{1,2,\dots,9\}$, $a\ne b$. Vyhovuje tak všetkých 17~čísel, ktoré práve uvedené podmienky spĺňajú: $1\,001$, $2\,002$, $3\,003$, $4\,004$,
 $5\,005$, $6\,006$, $7\,007$, $8\,008$, $9\,009$, $1\,771$,
 $2\,772$, $3\,773$, $4\,774$, $5\,775$, $6\,776$, $8\,778$, $9\,779$.

\návody
Určte počet všetkých štvorciferných prirodzených čísel, ktoré sú deliteľné šiestimi a~v~ktorých zápise sa vyskytujú práve dve jednotky. [56--C--S--1]

Určte počet všetkých trojíc dvojciferných prirodzených čísel $a$, $b$, $c$, ktorých súčin $abc$ má zápis, v~ktorom sú všetky cifry rovnaké. Trojice líšiace sa len poradím čísel považujeme za rovnaké, \tj. započítavame ich iba raz. [54--C--I--5]

K~prirodzenému číslu~$m$ zapísanému rovnakými ciframi sme prečítali štvorciferné prirodzené číslo~$n$. Získali sme štvorciferné číslo s~opačným poradím cifier, než má číslo~$n$. Určte všetky také dvojice čísel $m$ a~$n$. [52--C--I--5]
\endnávod
}

{%%%%%   C-I-4
\fontplace
\rpoint A; \tpoint B; \rtpoint\xy1,0 C;
\bpoint D; \bpoint E; \rtpoint F; \tpoint G;
[3] \hfil\Obr

{\it Rozbor}. Najskôr uvažujme bod~$F$, ktorý je priesečníkom priamky~$BC$ a~rovnobežky s~$EC$ prechádzajúcej bodom~$D$ (keďže $E\notin BC$, sú $EC$ a~$BC$ rôznobežné, \obr). Obsahy trojuholníkov $ECD$ a~$ECF$ sú zhodné (majú spoločnú stranu~$EC$ a~zhodnú výšku na túto stranu), obsah päťuholníka $ABCDE$ je teda zhodný s~obsahom štvoruholníka $ABFE$.
\inspicture{}

Ďalej uvažujme bod~$G$, ktorý je priesečníkom priamky~$BC$ a~rovnobežky s~$AF$ prechádzajúcej bodom~$E$. Potom sú opäť obsahy trojuholníkov $AFE$ a~$AFG$ zhodné, a~sú preto zhodné aj obsahy štvoruholníka $ABFE$ a~trojuholníka $ABG$. Bod~$G$ tak má požadovanú vlastnosť.

Hľadaný bod je na polpriamke~$BC$ jediný, lebo pre rôzne body $X$, $Y$ na polpriamke~$BC$ majú trojuholníky $ABX$ a~$ABY$ rôzne výšky na spoločnú stranu~$AB$, majú teda rôzne obsahy.

{\it Popis konštrukcie.}

1. $p$; $p\parallel EC$, $D\in p$;

2. $F$; $F\in p\cap BC$;

3. $q$; $q\parallel AF$, $E\in q$;

4. $G$; $G\in q\cap BC$;

Úloha má jediné riešenie.

\návody
Označme $P$ priesečník uhlopriečok daného konvexného štvoruholníka $ABCD$. Dokážte, že priamky $AB$ a~$CD$ sú rovnobežné práve vtedy, keď trojuholníky $ADP$ a~$BCP$ majú rovnaký obsah. [Rovnosť obsahov trojuholníkov $ADP$ a~$BCP$ je ekvivalentná s~rovnosťou obsahov trojuholníkov $ABC$ a~$ABD$ so spoločnou stranou~$AB$.]

V~kružnici s~polomerom~$2$ je daná tetiva~$AB$ dĺžky~$3$. Určte, aký najväčší obsah môže mať štvoruholník $AXBY$, ak jeho vrcholy $X$, $Y$ ležia na kružnici~$k$. [Najväčší obsah~$6$ má deltoid, ktorého uhlopriečka~$XY$ je priemerom kružnice~$k$.]

\D
Daný je obdĺžnik $ABCD$. Nech priamky $p$ a~$q$, ktoré prechádzajú vrcholom~$A$, pretínajú polkružnice zvonka pripísané stranám $BC$ a~$CD$ daného obdĺžnika postupne v~bodoch $K$ a~$L$ ($B\ne K\ne C\ne L\ne D)$ a~strany $BC$ a~$CD$ postupne v~bodoch $P$ a~$Q$ tak, že trojuholník $ABP$ má rovnaký obsah ako trojuholník $KCP$ a~zároveň trojuholník $AQD$ má rovnaký obsah ako trojuholník $CLQ$. Dokážte, že body $K$, $L$, $C$ ležia na jednej priamke. [53--C--I--2]
\endnávod
}

{%%%%%   C-I-5
Čísla od $1$ do $99$ rozdelíme podľa ich zvyšku po delení číslom $11$ do jedenástich deväťprvkových skupín $T_0$, $T_1$, \dots, $T_{10}$:
$$
\align
T_0=&\{11,22,33,\dots,99\},\\
T_1=&\{\phantom01,12,23,\dots,89\},\\
T_2=&\{\phantom02,13,24,\dots,90\},\\
   \vdots&\\
T_{10}=&\{10,21,32,\dots,98\}.
\endalign
$$

Ak vyberieme jedno číslo z~$T_0$ (viac ich ani vybrať nesmieme) a~všetky čísla z~$T_1$, $T_2$, $T_3$, $T_4$ a~$T_5$, dostaneme vyhovujúci výber $1+5\cdot9=46$ čísel, lebo súčet dvoch čísel z~$0$, $1$, $2$, $3$, $4$, $5$ je deliteľný jedenástimi jedine v~prípade $0+0$, z~množiny $T_0$ sme však vybrali iba jedno číslo.

Na druhej strane, v~ľubovoľnom vyhovujúcom výbere je najviac jedno číslo zo skupiny~$T_0$ a~najviac 9~čísel z~každej zo skupín
$$
T_1\cup T_{10},\ T_2\cup T_9,\ T_3\cup T_8,\ T_4\cup T_7,\ T_5\cup T_6,
$$
lebo pri výbere 10~čísel z~niektorej skupiny $T_i\cup T_{11-i}$ by medzi vybranými bolo niektoré číslo zo skupiny~$T_i$ a~aj niektoré číslo zo skupiny $T_{11-i}$; ich súčet by potom bol deliteľný jedenástimi. Celkom je teda vo výbere najviac $1+5\cdot9=46$ čísel.

\poznamka
Možno uvedené "učesané" riešenie vyzerá príliš trikovo. Avšak počiatočné úvahy každého riešiteľa k~nemu rýchlo vedú:
iste záleží len na zvyškoch vybraných čísel, takže rozdelenie na triedy~$T_i$ a~vyberanie z~nich je prirodzené. Je jasné, že z~$T_0$ môže byť vybrané len jedno číslo a~všetko ďalšie, o~čo sa musíme starať, je požiadavka, aby sme nevybrali zároveň po čísle zo skupiny~$T_i$ aj zo skupiny~$T_{11-i}$. Ak je už vybrané niektoré číslo z~triedy~$T_i$,
kde $i\ne0$, môžeme kľudne vybrať všetky čísla z~$T_i$, to už skúmanú vlastnosť nepokazí. Je preto dokonca jasné, ako {\it všetky\/} možné výbery najväčšieho počtu čísel vyzerajú.


\návody
Ukážte, že z~ľubovoľných $n$ prirodzených čísel možno vybrať niekoľko (napríklad aj jedno) tak, že ich súčet je deliteľný číslom~$n$.
[Uvažujte čísla $a_1$, $a_1+a_2$,~\dots, $a_1+\cdots+a_n$ a~ich zvyšky po delení~$n$.]

Zistite, pre ktoré prirodzené čísla~$n$ ($n\ge 2$) je možné z~množiny $\{1,2,\dots,n-1\}$ vybrať aspoň dve navzájom rôzne párne čísla tak, aby ich súčet bol deliteľný číslom~$n$.
[54--C--I--2]

\D
Určte počet všetkých trojíc navzájom rôznych trojciferných prirodzených čísel, ktorých súčet je deliteľný každým z~troch sčítaných čísel.
[55--C--I--3]
\endnávod
}

{%%%%%   C-I-6
Ľavú nerovnosť dokážeme ekvivalentnými úpravami:
$$
\align
\frac{a+b}{2}<&\frac{2(a^2+ab+b^2)}{3(a+b)},\quad\big|\cdot6(a+b)\\
3(a+b)^2<&4(a^2+ab+b^2),\\
0<&(a-b)^2.
\endalign
$$
Posledná nerovnosť vzhľadom na predpoklad $a\ne b$ platí. Aj pravú nerovnosť zo zadania budeme ekvivalentne upravovať, začneme umocnením každej strany na druhú:
$$
\align
\frac{4(a^2+ab+b^2)^2}{9(a+b)^2}<&\frac{a^2+b^2}{2},\quad \big|\cdot18(a+b)^2\\
8(a^2+ab+b^2)^2<&9(a^2+b^2)(a+b)^2,\\
8(a^4+b^4+2a^3b+2ab^3+3a^2b^2)<& 9(a^4+b^4+2a^3b+2ab^3+2a^2b^2),\\
6a^2b^2<&a^4+b^4+2a^3b+2ab^3.
\endalign
$$
Posledná nerovnosť je súčtom nerovností $2a^2b^2<a^4+b^4$ a~$4a^2b^2<2a^3b+2ab^3$, ktoré obe platia, lebo po presune členov z~ľavých strán na pravé dostaneme po rozklade už zrejmé nerovnosti $0<(a^2-b^2)^2$, resp. $0<2ab(a-b)^2$.

\návody
Pre $a$, $b\in{\ssize\Bbb R}$ dokážte
$$
a^4+b^4\ge a^3b+b^3a.
$$
[Upravte na tvar $(a^3-b^3)(a-b)\ge 0$.]

\D
Dokážte, že pre každé tri reálne čísla $x$, $y$, $z$, ktoré spĺňajú nerovnosti $0<x<y<z<1$, platí nerovnosť
$$
x^2+y^2+z^2<xy+yz+zx+z-x.
$$
[48--C--II--4]

Dokážte, že pre ľubovoľné kladné čísla $a$, $b$ a~$c$ platí nerovnosť
$$
\Bigl(a+\frac 1b\Bigr)\Bigl(b+\frac 1c\Bigr)\Bigl(c+\frac 1a\Bigr)\ge 8.
$$
[55--B--S--1]

Ak reálne čísla $a$, $b$, $c$, $d$ spĺňajú rovnosti
$$
a^2+b^2=b^2+c^2=c^2+d^2=1,
$$
platí nerovnosť
$$
ab+ac+ad+bc+bd+cd\le 3.
$$
Dokážte a~zistite, kedy nastane rovnosť.
[55--C--II--2]
\endnávod
}

{%%%%%   A-S-1
Ak prirodzené čísla $m$, $n$ spĺňajú zadané nerovnosti, tak zrejme
$$
m\ge2 \qquad \text{a} \qquad 2\sqrt{n}-m>0
\tag1
$$
(inak by výraz na ľavej strane zadaných nerovností nebol definovaný, resp. prostredný výraz by nebol väčší ako nezáporný výraz na ľavej strane). Predpokladajme, že podmienky~\thetag1 sú splnené. V~takom prípade môžeme urobiť na každej z~oboch nerovností v~jednom stĺpci nasledujúce {\it ekvivalentné\/} úpravy (pri každom zo štyroch umocňovaní na druhú sú obe strany dobre definované a~nezáporné):
$$
\alignat4
2\sqrt{n}-m&<\sqrt{m^2-2}   &&\bigm|^{\ 2}\qquad&\sqrt{m^2-4}&<2\sqrt{n}-m&&\bigm|^{\ 2}  \\
4n-4m\sqrt{n}+m^2&<m^2-2    &&            &m^2-4&<4n-4m\sqrt{n}+m^2       &&   \\
n+\tfrac12       &<m\sqrt{n}&&\bigm|^{\ 2}&m\sqrt{n}&<n+1                 &&\bigm|^{\ 2}    \\
n^2+n+\tfrac14   &<m^2n     &&\bigm|{}:n  &m^2n&<n^2+2n+1                 &&\bigm|{}:n      \\
n+1+\frac{1}{4n} &< m^2     &\quad&       &m^2&<n+2+\frac{1}{n}           &\quad&
\endalignat
$$
Posledné dve nerovnosti platia práve vtedy, keď sa $m^2$ nachádza v~intervale
$$
\left(n+1+\frac{1}{4n},n+2+\frac{1}{n}\right).
$$
Ten vzhľadom na zrejmé nerovnosti $0<\frac{1}{4n}\le\frac14$ a~$0<\frac{1}{n}\le1$ obsahuje jediné prirodzené číslo $n+2$. Prirodzené čísla $m$, $n$ teda spĺňajú výsledné nerovnosti práve vtedy, keď $m^2=n+2$.

Ešte treba zistiť, kedy pre $n=m^2-2$ platia podmienky \thetag1. Pre $m\ge2$ môžeme urobiť nasledujúce ekvivalentné úpravy:
$$
\aligned
2\sqrt{m^2-2}-m&>0,\\
2\sqrt{m^2-2}&>m,\quad \bigm|^{\,2}\\
4(m^2-2)&>m^2,\\
3m^2&>8.\\
\endaligned
$$
Posledná nerovnosť a~teda aj podmienky~\thetag1 sú pre každé $m\ge2$ splnené.

\odpoved
Hľadané dvojice sú $(m,n)=(m,m^2-2)$, kde $m\ge2$ je ľubovoľné prirodzené číslo.

\nobreak\medskip\petit\noindent
Za úplné riešenie dajte 6~bodov. Za nájdenie intervalu pre hodnotu $m^2$ (či v~inej forme vyriešenú sústavu zadaných nerovníc s~neznámou~$m$ a~parametrom~$n$) správnymi dôsledkovými úpravami dajte 2~body, ďalšie 2~body potom dajte za úvahu o~celočíselnosti vedúcu ku vzťahu $n=m^2-2$; zvyšné 2~body sú za skúšku, pri ktorej je pre nájdené dvojice
nutné zdôvodniť platnosť {\it pôvodných\/} nerovností, a~to vysvetlením, prečo sú urobené úpravy umocnením korektné. Ak pri skúške nie je vylúčená hodnota $m=1$, strhnite 1~bod.

Ak riešiteľ stanoví všeobecné podmienky, pri ktorých sú všetky urobené úpravy daných nerovností ekvivalentné, tak dajte 5~bodov za odvodenie vzťahu $n=m^2-2$ a~1~bod za následné overenie, že pre každé $m\ge2$ dvojica $(m,n)=(m,m^2-2)$
stanovené podmienky spĺňa.
\endpetit
\bigbreak}

{%%%%%   A-S-2
Dokážeme najprv prvú implikáciu. Nech $AP\perp BC$. Potom bod~$P$ je priesečníkom výšok trojuholníka $ABC$. Chceme dokázať, že úsečky $AP$ a~$BC$ sú zhodné, preto nájdeme dva zhodné trojuholníky, v~ktorých sú tieto úsečky zodpovedajúcimi si stranami.

Označme $E$ priesečník priamky~$BP$ so stranou $AC$, \tj. pätu výšky spustenej z~vrcholu~$B$. Z~pravouhlého trojuholníka $ABE$ a~zadanej veľkosti uhla $BAC$ ľahko dopočítame, že $|\uhol PBD|=45^\circ$. Preto trojuholník $PDB$ je pravouhlý a~rovnoramenný, čiže $|DP|=|DB|$. Podobne trojuholník $ADC$ je rovnoramenný a~pravouhlý, teda $|DA|=|DC|$. Podľa vety $sus$ sú potom pravouhlé trojuholníky $APD$ a $CBD$ zhodné a~ich prepony $AP$, $BC$ majú rovnakú dĺžku (\obr).
\insp{a58.5}%

\smallskip
Ostáva dokázať druhú implikáciu. Predpokladajme, že $|AP|=|BC|$. Keďže $ADC$ je rovnoramenný pravouhlý trojuholník, platí $|AD|=|CD|$, z~čoho vyplýva, že trojuholníky $PAD$ a~$BCD$ sú zhodné podľa vety $Ssu$. Odtiaľ máme $|PD|=|BD|$, z~čoho dostávame $|\uhol ABP|=45^\circ$. Označme opäť $E$ priesečník priamky~$BP$ so stranou~$AC$. Z~trojuholníka $ABE$ jednoducho odvodíme, že uhol $BEA$ je pravý, takže priamka~$BP$ je výškou trojuholníka $ABC$ (\obrr1). Preto bod~$P$ je priesečník výšok tohto trojuholníka. Z~toho vyplýva, že $AP$ je výška na stranu~$BC$, čiže je na ňu kolmá.

\ineriesenie
Ak $AP\perp BC$, je bod~$P$ priesečníkom výšok trojuholníka $ABC$. Označme $|\uhol BAC|=\alpha=45^\circ$. Podobne ako v~jednom z~riešení
\niedorocenky{druhej úlohy domáceho kola\footnote{V~uvedenom riešení bola pri štandardnom označení odvodená rovnosť
             $|CU|=\frac{c|\cos\gamma|}{\sin\gamma}$, kde $U$ je priesečník výšok trojuholníka $ABC$.}}%
\dorocenky{úlohy A -- I -- 2}
možno odvodiť
$$
|AP|=\frac{|BC|\cdot\cos\alpha}{\sin\alpha}=|BC|\cdot\cotg\alpha=|BC|\cdot\cotg 45^\circ=|BC|.
$$

\smallskip
Nech naopak $|AP|=|BC|$. Označme $Q$ priesečník výšok trojuholníka $ABC$. Z~dokázanej prvej implikácie máme $|AQ|=|BC|$. Všetky body úsečky~$CD$ majú navzájom rôznu vzdialenosť od bodu~$A$, preto vnútri úsečky~$CD$ môže ležať nanajvýš jeden bod~$P$ s~vlastnosťou $|AP|=|BC|$, a~tento bod musí byť totožný s~bodom~$Q$.

\nobreak\medskip\petit\noindent
Za úplné riešenie dajte 6~bodov. Pritom za úplný dôkaz každej z~oboch implikácií udeľte 3~body. Za dôkaz prvej implikácie využívajúci {\it bez dôkazu\/} vzťah $|AP|=|BC|\cotg\alpha$ odvodený v~domácom kole udeľte 3~body, ak žiak uvedie, že používa vzťah z~riešenia domáceho kola, inak dajte 1~bod.

V~prípade neúplného dôkazu niektorej z~implikácií udeľte body zodpovedajúce tomu, ako ďaleko sa žiak dostal; napríklad za odvodenie rovnosti $|DP|=|DB|$ v~prípade prvej implikácie 1~bod, za odvodenie rovnosti $|AD|=|CD|$ tiež 1~bod.

Za zdôvodnené pozorovanie, že trojuholník $ADC$ je rovnoramenný, udeľte 1~bod aj v prípade, že žiak toto pozorovanie nepoužije na úspešný dôkaz ani jednej z~implikácií.
\endpetit
\bigbreak}

{%%%%%   A-S-3
Aby sa uvedený zlomok dal krátiť prirodzeným číslom~$d$, musí byť $d$ zároveň deliteľom čitateľa aj menovateľa. Predpokladajme teda, že pre nejaké prirodzené číslo~$d$ a~nesúdeliteľné celé čísla $p$, $q$ platí $d\mid 3p-q$ a~súčasne $d\mid 5p+2q$. Vhodným násobením a~sčítaním oboch výrazov dostávame
$$
d\mid 2(3p-q)+(5p+2q)=11p,\qquad\text{a tiež}\qquad d\mid 3(5p+2q)-5(3p-q)=11q.
$$
Takže $d$ je deliteľom oboch čísel $11p$, $11q$ a~musí deliť aj číslo\footnote{Využívame známy poznatok, že každý spoločný deliteľ daných dvoch celých čísel $a$, $b$ je deliteľom ich najväčšieho spoločného deliteľa $\nsd(a,b)$.}
$$
\nsd(11p,11q)=11\cdot\nsd(p,q)=11.
$$
Odtiaľ máme $d\in\{1,11\}$. Uvedený zlomok, ak vôbec, sa teda dá krátiť iba číslom~$11$ (ak $d=1$, sotva možno hovoriť o~"krátení"). Ľahko nájdeme príklad nesúdeliteľných čísel $p$, $q$, pre ktoré sa jedenástimi daný zlomok naozaj dá krátiť. Napr. pre $p=9$, $q=5$ platí
$$
\frac{3p-q}{5p+2q}=\frac{3\cdot9-5}{5\cdot9+2\cdot5}=\frac{11\cdot2}{11\cdot5}=\frac{2}{5}.
$$

\odpoved
Jediné prirodzené číslo, ktorým sa dá krátiť niektorý z~uvedených zlomkov, je 11.

\poznamka
Úvahu, že z~vlastností $d\mid 11p$, $d\mid 11q$ a~$\nsd(p,q)=1$ vyplýva $d\in\{1,11\}$, možno previesť aj inými spôsobmi. Môžeme napríklad rozlíšiť dva prípady: Ak $d$ nie je násobkom jedenástich, tak z~$d\mid 11p$, $d\mid 11q$ máme $d\mid p$, $d\mid q$, čiže $d=1$. Ak $d=11k$, kde $k$ je prirodzené, tak z~$d\mid 11p$, $d\mid 11q$ vyplýva $k\mid p$, $k\mid q$, čiže $k=1$ a~$d=11$.

\nobreak\medskip\petit\noindent
Za úplné riešenie dajte 6~bodov, z~toho 1~bod za prevod na vzťahy $d\mid 3p-q$ a~$d\mid 5p+2q$, ďalšie 2~body za odvodenie vzťahov $d\mid 11p$ a~$d\mid 11q$, ďalší bod za záver $d=11$ a~zostávajúce 2~body za uvedenie (resp. za dôkaz existencie) vyhovujúceho príkladu dvojice nesúdeliteľných čísel $p$ a~$q$.
\endpetit
\bigbreak}

{%%%%%   A-II-1
%%Nechť $a$, $b$, $c$, $d$ jsou číslice hledaného čísla. $\overline{abcd} \equiv a + 2b + 3c + d \pmod7$ a současně %%$\overline{dcba} \equiv d + 2c + 3b + a \pmod7$. Sečtením dostaneme po úpravě $7 \mid (b + c)$ a $7 \mid a + b - d$. %%Z podmínky $37 \mid \overline{abcd} - \overline{dcba} = 27\cdot 37(a - d) + 99(b - c)$ dostaneme $b = c$, vzhledem k %%dilitelnosti sedmi $b = c = 7$ nebo $b = c = 0$, tedy $7 \mid a - d$. Vzhledem k $1 < a \le d$ dostaneme čtyři řešení %%$1008$, $1778$, $2009$, $2779$.
Označme hľadané číslo $n=\overline{abcd}=1000a+100b+10c+d$ a~číslo s~opačným poradím číslic $k=\overline{dcba}=1000d+100c+10b+a$. Obe čísla dávajú rovnaký zvyšok po delení číslom~$37$, preto je ich rozdiel
$$
\aligned
k-n&=(1000d+100c+10b+a)-(1000a+100b+10c+d)=\\
&=999(d-a)+90(c-b)=37\cdot27(d-a)+90(c-b)
\endaligned
\tag1
$$
deliteľný číslom~$37$, čiže $37\mid 90(c-b)$. Keďže $37$ je prvočíslo a~číslo~$90$ nie je jeho násobkom, nutne $37\mid c-b$. To je pre číslice $b$, $c$ možné len v~prípade, že $b=c$. Naopak, ak $b=c$, tak z~vyjadrenia~\thetag1 je zrejmé, že rozdiel $k-n$ je deliteľný číslom $37$, \tj. čísla $n$ a~$k$ dávajú po delení $37$ rovnaký zvyšok. Môžeme teda položiť $n=\overline{abbd}$, $k=\overline{dbba}$ a~ďalej sa už zaoberať len podmienkami o~deliteľnosti siedmimi.

Keďže sú siedmimi deliteľné obe čísla $n$, $k$, je siedmimi deliteľný aj ich rozdiel. Dosadením $c=b$ do vyjadrenia \thetag1 dostávame
$$
7\mid k-n=37\cdot27(d-a).
$$
Rovnakou úvahou ako predtým ($7$ je prvočíslo a~$37\cdot27$ nie je jeho násobkom) dostávame $7\mid d-a$. Navyše zo zadanej podmienky $k>n$ vyplýva $d>a$; nemôže byť ani $d=a$, inak by sme mali $n=\overline{abbd}=\overline{dbba}=k$. Číslice $a$, $d$ preto musia spĺňať vzťah $d-a=7$. Prípustné sú len dve možnosti: $a=1$, $d=8$, alebo $a=2$, $d=9$. (Prípad $a=0$, $d=7$ je vylúčený, lebo $a$ je začiatočnou číslicou čísla~$n$.)

Ak $a=1$, $d=8$, budú čísla
$$
\aligned
n&=\overline{1bb8}=1008+110b=7\cdot144+110b,\\
k&=\overline{8bb1}=8001+110b=7\cdot1143+110b
\endaligned
$$
deliteľné siedmimi vtedy a~len vtedy, keď $7\mid b$, čiže $b=0$ alebo $b=7$. Teda $n=1008$ alebo $n=1778$.

Ak $a=2$, $d=9$, budú čísla
$$
\aligned
n&=\overline{2bb9}=2009+110b=7\cdot287+110b,\\
k&=\overline{9bb2}=9002+110b=7\cdot1286+110b
\endaligned
$$
deliteľné siedmimi vtedy a~len vtedy, keď $7\mid b$, čiže $b=0$ alebo $b=7$. Teda $n=2009$ alebo $n=2779$.

\odpoved
Hľadané štvorciferné číslo je niektoré zo štvorice $1008$, $1778$, $2009$, $2779$.

\nobreak\medskip\petit\noindent
Za úplné riešenie dajte 6~bodov. Z~toho 2~body dajte za zdôvodnenie podmienky $b=c$, 2~body za odvodenie vzťahu $d-a=7$ a~2~body za zdôvodnenie $7\mid b$. Ak riešenie pozostáva z~odvodenia nutných podmienok bez zmienky o~tom, že sú aj postačujúce, je nutná skúška a~v~prípade jej opomenutia dajte len 5~bodov.
\endpetit
\bigbreak}

{%%%%%   A-II-2
Označme vrcholy daného trojuholníka $A$, $B$, $C$ tak, aby vrcholy $A$, $B$ ležali postupne oproti odvesnám dĺžok $a$, $b$.
\insp{a58.7}%

Najprv vypočítame veľkosti polomerov polkružníc $k_a$ a~$k_b$. Označme $A'$ obraz bodu~$A$ v~osovej súmernosti podľa priamky~$BC$. Kružnica~$k_a$ je vpísaná do trojuholníka $ABA'$ (\obr). Rovnoramenný trojuholník $ABA'$ má obvod $o=2(b+c)$ a~obsah $S=ab$, preto polomer kružnice~$k_a$ vypočítame podľa známeho vzťahu
$$
\varrho_a = \frac{2S}o = \frac{ab}{b+c}.
$$
Podobne vypočítame polomer kružnice~$k_b$, dostaneme $\varrho_b=ab/(a+c)$.

Pre $p$ a~pre ľubovoľný pravouhlý trojuholník s~odvesnami $a$, $b$ a~preponou~$c$ má platiť
$$
p\le\frac{\frac1{\varrho_a}+\frac1{\varrho_b}}{\frac1a+\frac1b}=
\frac{\frac{b+c}{ab}+\frac{a+c}{ab}}{\frac{a+b}{ab}}=\frac{a+b+2c}{a+b}=1+\frac{2c}{a+b}=
1+\frac{2\sqrt{a^2+b^2}}{a+b}.
$$
Voľbou $a=b$ dostávame $p\le 1+2\sqrt{2a^2}/2a=1+\sqrt2$. Ukážeme, že pre $p=1+\sqrt2$ je zadaná nerovnosť vždy splnená. Naozaj, z~uvedeného výpočtu a~z~nerovnosti medzi kvadratickým a~aritmetickým priemerom dostávame
$$
\frac{\frac1{\varrho_a}+\frac1{\varrho_b}}{\frac1a+\frac1b}=
1+\frac{2\sqrt{a^2+b^2}}{a+b}=1+\frac{2\sqrt{\frac{a^2+b^2}2}\sqrt2}{a+b}\ge
1+\frac{2\frac{a+b}2\sqrt2}{a+b}
= 1+\sqrt 2.
$$

\odpoved
Najväčšie reálne číslo také, že zadaná nerovnosť platí pre všetky pravouhlé trojuholníky, je $p=1+\sqrt2$.
\insp{a58.8}%

\poznamka
Veľkosť polomerov $\varrho_a$ a~$\varrho_b$ je možné vypočítať aj inými spôsobmi, napríklad takto: Nech $c$ je dĺžka prepony. Stredy $S_a$, $S_b$ polkružníc $k_a$, $k_b$ ležia na osiach uhlov $CAB$ a~$CBA$. Je známe, že os uhla delí v~trojuholníku protiľahlú stranu v~pomere priľahlých strán. V~našom prípade (\obr) dostávame $|S_aC|/|S_aB|=|AC|/|AB|$, \tj.
$$
\frac{\varrho_a}{a-\varrho_a}=\frac bc,
$$
odkiaľ úpravou ľahko vyjadríme $\varrho_a=ab/(b+c)$. Analogicky vypočítame $\varrho_b$.

\nobreak\medskip\petit\noindent
Za úplné riešenie dajte 6~bodov, z toho 2 body za výpočet veľkostí polomerov $\varrho_a$ a $\varrho_b$, 1 bod za nájdenie hodnoty $p=1+\sqrt{2}$ a 3 body za dôkaz nerovnosti zo zadania pre $p=1+\sqrt{2}$.
\endpetit
\bigbreak}

{%%%%%   A-II-3
Podobne ako pri riešení prvej úlohy domáceho kola, využitím známych súčtových vzorcov goniometrických funkcií pre ľubovoľné reálne čísla $x$, $y$ dostávame
$$
\align
2\sin y\sin(x+y) - \cos x&=2\sin y(\sin x\cos y+\cos x\sin y) - \cos x=\\
&=2\sin y\cos y\sin x+(2\sin^2y-1)\cos x=\\
&=\sin2y\sin x-\cos2y\cos x=\\&=-\cos(x+2y).
\endalign
$$
Z~podmienok úlohy potom pre veľkosti vnútorných uhlov $\alpha$, $\beta$, $\gamma$ trojuholníka platí
$$
\align
  \cos\alpha - 2\sin\beta\sin(\alpha+\beta) &= \cos(\alpha+2\beta)= -1,\tag1\\
  \cos\beta - 2\sin\gamma\sin(\beta+\gamma) &= \cos(\beta+2\gamma)=  0.\tag2
\endalign
$$

Vnútorné uhly ľubovoľného trojuholníka ležia v~intervale $(0,\pi)$, z~čoho vyplývajú nerovnosti $0<\alpha+2\beta<3\pi$.\footnote{Platí dokonca $\alpha+2\beta<2\pi$, lebo $\alpha+\beta=\pi-\gamma<\pi$.} Ich spojením s~\thetag1 máme $\alpha+2\beta=\pi$. Odtiaľ
$$
\gamma=\pi-(\alpha+\beta)=\pi-(\alpha+2\beta)+\beta=\pi-\pi+\beta=\beta.
$$
Dosadením do \thetag2 dostávame
$$
\cos3\beta=0.\tag3
$$
Uhol $\beta$ je ostrý, lebo je zhodný s~uhlom $\gamma$ a~trojuholník nemôže mať dva pravé, resp. dva tupé vnútorné uhly. Teda $0<3\beta<\frac32\pi$, a~vzhľadom na \thetag3 máme $3\beta=\frac12\pi$, čiže $\beta=\gamma=\frac16\pi$. Ľahko dopočítame $\alpha=\pi-\beta-\gamma=\frac23\pi$. Skúškou (ktorá však pri uvedenom postupe nie je nutná) ľahko overíme, že táto trojica $\alpha$, $\beta$, $\gamma$ spĺňa všetky podmienky zadania.

\odpoved
Podmienkam úlohy vyhovuje trojuholník, ktorého veľkosti vnútorných uhlov (uvedené v~stupňoch) sú $\alpha=120^{\circ}$, $\beta=\gamma=30^{\circ}$.

\poznamka
Úlohu možno riešiť aj iným postupom. Z~druhej rovnice sústavy sa dá odvodiť vzťah $\tg\alpha\cdot\tg\gamma=\m1$, z~ktorého vyplýva $\alpha-\gamma=\pm\frac12\pi$. Pre každú z~oboch možností znamienka dosadením do prvej rovnice získame kubickú rovnicu v~premennej $t=\sin\gamma$, ktorú možno vyriešiť uhádnutím koreňov.

\nobreak\medskip\petit\noindent
Za úplné riešenie dajte 6~bodov. Za odvodenie vzťahov \thetag1, \thetag2 dajte 2~body, ďalšie 2~body za vzťah $\beta=\gamma$ a~posledné 2~body za dopočítanie veľkostí jednotlivých uhlov. Ak žiakov postup vyžaduje urobenie skúšky, za jej vynechanie strhnite 1~bod. Po jednom bode tiež strhnite, ak je niektorá z~rovností $\alpha+2\beta=\pi$, resp. $3\beta=\frac12\pi$ (vyplývajúca z~hodnôt príslušných kosínusov) odvodená bez spomenutia potrebných nerovností $0<\alpha+2\beta<3\pi$, resp. $0<3\beta<\frac32\pi$.

Ak žiak úlohu rieši postupom naznačeným v~poznámke, za odvodenie vzťahu $\tg\alpha\cdot\tg\gamma=\m1$ dajte 1~bod a~za odvodenie rovnosti $\alpha-\gamma=\pm\frac12\pi$ ďalšie 2~body. Po jednom bode dajte za úspešné vyriešenie každej z~dvoch kubických rovníc a~posledný bod za správne dopočítanie veľkostí uhlov a~urobenie skúšky.
\endpetit
\bigbreak}

{%%%%%   A-II-4
Označme $Q$ priesečník úsečky~$AD$ a~ťažnice z~vrcholu~$C$ (teda $Q$ je "zakázaná" poloha bodu~$P$). Sú dve možnosti, kde môže ležať bod~$P$: vnútri úsečky~$DQ$ alebo vnútri úsečky~$QA$. Ukážeme, že v~oboch prípadoch prechádza kružnica opísaná trojuholníku $AKP$ bodom~$M$ súmerne združeným s~bodom~$C$ podľa stredu strany~$AB$ (ktorého poloha samozrejme od výberu bodov $D$ a~$P$ nezávisí).

\smallskip
Uvažujme ako prvý prípad, keď bod~$P$ leží vnútri úsečky~$DQ$. Dokážme najskôr, že potom bod~$K$ leží vnútri úsečky~$CQ$. Nech $S$ je stred strany~$AB$. Bod~$K$ nemôže ležať vnútri polpriamky opačnej k~polpriamke~$QC$, v~takom prípade by totiž bod~$P$ ležal vnútri trojuholníka $CKD$, \tj. body $C$, $K$, $D$, $P$ by v~žiadnom prípade nemohli ležať na jednej kružnici. Zadanie triviálne vylučuje aj možnosti $K=Q$ a~$K=C$. Ostáva vylúčiť možnosť, že $K$ leží vnútri polpriamky opačnej k~polpriamke~$CQ$ (\obr).
\insp{a58.9}%
Ak by to tak bolo, tak by body $K$ a~$P$ ležali v~opačných polrovinách určených priamkou~$DC$ a~uhly $DKC$ a~$CPA$ by museli mať rovnakú veľkosť, aby bol štvoruholník $DKCP$ tetivový. Pri uvažovanej polohe bodov však zrejme platí
$$
|\angle DKC|<|\angle DCS| \qquad\text{a}\qquad |\angle CBS|=|\angle CBA|<|\angle CPA|.
$$
Z~ostrouhlosti trojuholníka $ABC$ vyplýva $|\angle DCS|<|\angle CBS|$ (lebo $|CS|>|BS|$, keďže $C$ leží zvonka Tálesovej kružnice so stredom~$S$ a~priemerom~$AB$). Spolu máme $|\angle DKC|<|\angle CPA|$.

Bod~$K$ teda musí ležať vnútri úsečky~$CQ$ (\obr). Označme $\varphi$ veľkosť uhla $KCB$. Body $C$ a~$P$ sú protiľahlými vrcholmi tetivového štvoruholníka $CDPK$, preto $|\angle DPK|=180^\circ-\varphi$. Z~toho dostávame $|\angle APK|=\varphi$. Rovnakú veľkosť ako uhol $APK$ má aj uhol $AMC$, pretože priamky $AM$ a~$BC$ sú rovnobežné. Zrejme body $P$ a~$M$ ležia v~rovnakej polrovine vzhľadom na priamku~$AK$ (oba totiž ležia v~polrovine $AKQ$). Z~rovnosti $|\angle APK|=|\angle AMK|$ potom vyplýva, že body $A$, $K$, $P$ a~$M$ ležia na kružnici.
\insp{a58.10}%

\smallskip
V~druhom prípade leží bod~$P$ vnútri úsečky~$QA$. Dokážme, že potom $K$ leží vnútri úsečky~$QM$. Bod~$K$ samozrejme musí ležať na polpriamke opačnej k~polpriamke~$QC$, \tj. na polpriamke~$QM$. Stačí vylúčiť možnosť, že $K$ leží až "za" bodom~$M$, čiže na polpriamke opačnej k~polpriamke~$MQ$ (\obr).
\insp{a58.11}%
Ak by to tak bolo, zrejme by s~využitím ostrosti uhla~$\gamma$ v~trojuholníku $ABC$ platilo
$$
|\angle CDK|>|\angle CBM|=180^\circ-\gamma>90^\circ \quad\text{a}\quad
|\angle CPK|>|\angle CAM|=180^\circ-\gamma>90^\circ.
$$
Odtiaľ $|\angle CDK|+|\angle CPK|>180^\circ$, čo nie je možné vzhľadom na tetivovosť štvoruholníka $CPKD$ (súčet veľkostí protiľahlých uhlov musí byť rovný $180^\circ$).

Bod~$K$ teda musí ležať vnútri úsečky~$QM$ (\obr). Označme opäť $\varphi$ veľkosť uhla $KCB$. Uhly $DCK$ a~$DPK$ sú obvodové uhly nad tetivou~$DK$, čiže $|\angle DPK|=\varphi$ a~$|\angle APK|=180^\circ-\varphi$. Uhol $AMC$ má veľkosť~$\varphi$. Priamka~$AK$ oddeľuje body $Q$ a~$M$, preto body $P$ a~$M$ ležia v~rôznych polrovinách vzhľadom na túto priamku. Takže $APKM$ je tetivový štvoruholník, lebo uhly pri protiľahlých vrcholoch $P$ a~$M$ majú súčet $180^\circ$.
\insp{a58.12}%

\ineriesenie
Dokážeme tvrdenie bez predpokladu ostrouhlosti trojuholníka $ABC$. Označme body $Q$ a~$M$ rovnako ako v~prvom riešení. Nevýhodou predošlého postupu je, že musíme rozoberať veľa možností a~zdôvodňovať, že štvorice bodov ležia na uvažovaných kružniciach v~správnom poradí (pritom pri tupouhlom trojuholníku $ABC$ môže bod~$K$ ležať aj na polpriamke opačnej k~polpriamke $CQ$, resp. $MQ$). Namiesto obvodových uhlov využijeme mocnosť bodu ku kružnici. Z~nej pre bod~$Q$ a~kružnicu opísanú štvorici bodov $C$, $P$, $D$, $K$ dostávame (bez ohľadu na polohu bodu~$P$) $|QK|\cdot|QC|=|QP|\cdot|QD|$, teda $|QK|:|QP|=|QD|:|QC|$. Z~podobnosti trojuholníkov $QDC$ a~$QAM$, ktorá vyplýva z~rovnobežnosti priamok $BC$ a~$AM$, máme $|QD|:|QC|=|QA|:|QM|$. Platí teda
$$
\frac{|QK|}{|QP|}=\frac{|QD|}{|QC|}=\frac{|QA|}{|QM|},
$$
odkiaľ $|QK|\cdot|QM|=|QP|\cdot|QA|$. Z~tejto rovnosti a~zo známeho "obráteného" tvrdenia o~mocnosti bodu ku kružnici už priamo vyplýva, že body $K$, $M$, $P$, $A$ ležia na jednej kružnici, bez ohľadu na to, či bod~$Q$ leží vnútri oboch úsečiek $KM$, $PA$, alebo mimo oboch týchto úsečiek. (Zrejme nie je možné, aby ležal vnútri jednej z~nich a~mimo druhej z~nich; na dôkaz toho stačí rozlíšiť dve možné polohy bodu~$P$ na úsečke~$DA$ podobne ako v~prvom riešení).

\nobreak\medskip\petit\noindent
Za úplné riešenie dajte 6~bodov, z~toho 2~body za objavenie pevného bodu~$M$. Ak žiak spraví dôkaz len pre jeden z~prípadov uvedených v~prvom riešení, udeľte 5~bodov. Ak žiak bez zdôvodnenia predpokladá správnu polohu bodov $P$ a~$M$ vzhľadom na priamku~$AK$, strhnite 1~bod (čiže treba dať 5~bodov za dôkaz pre oba prípady alebo 4~body, ak sa žiak zaoberá len jedným z~uvedených prípadov). Pri riešení využívajúcom mocnosť bodu ku kružnici strhnite 1~bod, ak žiak nezdôvodní, že bod $Q$ leží buď vnútri oboch úsečiek $KM$, $PA$, alebo mimo oboch týchto úsečiek. Riešenia iného typu ako uvedené hodnoťte v~súlade s~touto schémou.
\endpetit
\bigbreak}

{%%%%%   A-III-1
Predpokladajme, že všetky čísla $p$, $3p+2$, $5p+4$, $7p+6$, $9p+8$ a~$11p+10$ sú prvočíslami. Skúmajme, aký zvyšok po delení piatimi môže dávať $p$, teda pre aké $l$ z~množiny $\{0,1,2,3,4\}$ a~nezáporné celé číslo~$k$ môže platiť $p=5k+l$.

\item{$\triangleright$} Ak $p=5k$, tak $11p+10=5(11k+2)$ nie je prvočíslom pre žiadne $k$.
\item{$\triangleright$} Ak $p=5k+1$, tak $3p+2=5(3k+1)$ je prvočíslom jedine pre $k=0$, ale potom $p=1$, čo nie je prvočíslo.
\item{$\triangleright$} Ak $p=5k+2$, tak $7p+6=5(7k+4)$ nie je prvočíslom pre žiadne $k$.
\item{$\triangleright$} Ak $p=5k+3$, tak $9p+8=5(9k+7)$ nie je prvočíslom pre žiadne $k$.

\noindent
Číslo~$p$ preto musí byť tvaru $5k+4$. Potom $6p+11=5(6k+7)$, čo je zložené číslo pre každé nezáporné celé $k$.

\poznamka
Najmenšie $p$, pre ktoré sú $p$, $3p+2$, $5p+4$, $7p+6$, $9p+8$ a~$11p+10$ prvočíslami, je $p=2\,099$.
}

{%%%%%   A-III-2
Ukážeme, že uhol $LKM$ má rovnakú veľkosť ako uhol $CBD$. Odtiaľ zadané tvrdenie triviálne vyplýva (uhol $CBD$ má veľkosť $45^\circ$ práve vtedy, keď $|BC|=|CD|$, čiže keď $ABCD$ je štvorec).
\insp{a58.13}

Body $B$, $K$, $M$, $P$ ležia v~tomto poradí na Tálesovej kružnici nad priemerom~$BP$. Pre veľkosti obvodových uhlov nad tetivou~$PM$ teda platí $|\uhol PKM|=|\uhol PBM|$. Podobne body $A$, $K$, $L$, $P$ ležia v~tomto poradí na Tálesovej kružnici nad priemerom~$AP$ a~pre veľkosti obvodových uhlov nad tetivou~$PL$ máme $|\uhol LKP|=|\uhol LAP|$. Napokon, z~obvodových uhlov nad tetivou~$CP$ kružnice opísanej pravouholníku $ABCD$ dostávame $|\uhol CAP|=|\uhol CBP|$.

Z~uvedených rovností vyplýva (\obr)
$$
\align
|\uhol LKM|&=|\uhol LKP|+|\uhol PKM|=|\uhol LAP|+|\uhol PBM|=|\uhol CAP|+|\uhol PBD|=\\
           &=|\uhol CBP|+|\uhol PBD|=|\uhol CBD|,
\endalign
$$
čo sme chceli dokázať.

\poznamka
Uvedený postup možno použiť aj v~triviálnom prípade, keď $P=C$ alebo $P=D$; vtedy majú niektoré z~uvažovaných uhlov nulovú veľkosť.

\ineriesenie
Opäť dokážeme, že uhly $LKM$ a~$CBD$ majú rovnakú veľkosť. Označme $N$ pätu kolmice z~bodu~$P$ na priamku~$BC$. Body $K$, $L$, $N$ ležia na Simsonovej priamke prislúchajúcej bodu~$P$ a~trojuholníku $ABC$ (\obr). Na Tálesovej kružnici nad priemerom~$PB$ ležia body $K$, $M$ aj $N$. Z~obvodových uhlov nad tetivou $MN$ teda máme
$$
|\uhol LKM|=|\uhol NKM|=|\uhol NBM|=|\uhol CBD|.
$$
\insp{a58.14}
}

{%%%%%   A-III-3
Nech $a$, $b$, $c$, $d$ sú ľubovoľné kladné čísla, ktorých súčin je $1$. Podľa nerovnosti medzi aritmetickým a~geometrickým priemerom trojice čísel $a^x$, $b^x$, $c^x$ máme
$$
\frac{a^x+b^x+c^x}3\ge\root3\of{a^xb^xc^x}=\root3\of{\frac1{d^x}}.
$$
Zvolením $x=3$ v~predošlej nerovnosti dostávame $\frac13(a^3+b^3+c^3)\ge1/d$. Zrejme rovnako platí
$$
\tfrac13(a^3+b^3+d^3)\ge1/c,\quad \tfrac13(a^3+c^3+d^3)\ge1/b,\quad \tfrac13(b^3+c^3+d^3)\ge1/a.
$$
Sčítaním uvedených štyroch nerovností dostávame
$$
a^3+b^3+c^3+d^3\ge\frac1a+\frac1b+\frac1c+\frac1d,
$$
teda pre $x=3$ zadaná nerovnosť vždy platí.

\smallskip
Ukážeme, že $x=3$ je hľadanou najmenšou hodnotou, teda že pre ľubovoľné $x<3$ zadaná nerovnosť nie je vždy splnená. Hľadajme štvoricu, pre ktorú nerovnosť nebude platiť, v~tvare $a=b=c=t$, $d=1/t^3$ pre vhodné $t$ (závislé na danom $x<3$). Pre takéto $a$, $b$, $c$, $d$ vždy platí $abcd=1$, a~tiež
$$
\align
a^x+b^x+c^x+d^x &= 3t^x + \frac{1}{t^{3x}},\\
\frac1a + \frac1b + \frac1c + \frac1d &= \frac3t + t^3 > t^3.
\endalign
$$
Ak $t>1$, tak $1/t^{3x}<t^x$ a~ľavá strana zadanej nerovnosti je menšia ako $4t^x$. Aby nerovnosť neplatila, stačí zvoliť $t>1$ tak, aby platilo $t^3>4t^x$, čo je pre $x<3$ ekvivalentné s~podmienkou
$$
t>\root3-x\of4.
$$
Tú vieme voľbou dostatočne veľkého $t$ triviálne splniť.

\zaver
Hľadané najmenšie kladné číslo je $x=3$.
}

{%%%%%   A-III-4
Z~rovnosti $n+k^2=(k+n)(k-n)+n(n+1)$ vidíme, že $n+k\mid n+k^2$ práve vtedy, keď $n+k\mid n(n+1)$. Počet čísel $k$ s~touto vlastnosťou je teda rovný počtu tých deliteľov čísla $D=n(n+1)$, ktoré sú väčšie ako $n$.

\smallskip
a) V~prípade $n=58$ z~rozkladu na prvočinitele príslušného $D=58\cdot59=2\cdot29\cdot59$ vidíme, že delitele
čísla~$D$ väčšie ako $58$ sú práve štyri: $59$, $2\cdot59=118$, $29\cdot59=1\,711$ a~$2\cdot29\cdot59=3\,422$. To sú hodnoty $58+k$, takže príslušné $k$ sú o~$58$ menšie, teda postupne $k=1$, $k=60$, $k=1\,653$ a~$k=3\,364=58^2$.

\smallskip
b) Pre párne $n=2p$, kde $p\ge3$, platí $D=2p(2p+1)$, takže ľahko vypíšeme štyri delitele čísla~$D$, ktoré sú väčšie ako dané $n=2p$:
$$
2p+1<2(2p+1)<p(2p+1)<2p(2p+1).
\tag1
$$
Ak sú $p$, $2p+1$ prvočísla, žiadne iné také delitele číslo~$D$ zrejme nemá, teda $n=2p$ spĺňa vlastnosť zo zadania.

Dokážeme, že ak aspoň jedno z~čísel $p$, $2p+1$ je zložené, tak $D$ má okrem deliteľov vypísaných v~\thetag1 ešte aspoň jedného deliteľa väčšieho ako $2p$.

Ak je číslo $p\ge3$ zložené, je deliteľné niektorým $q$, $2\le q\le\frac12p$ a~číslo~$D$ má deliteľa $2q(2p+1)$, ktorý s~výnimkou prípadu $q=\frac12p$ leží medzi druhým a~tretím deliteľom vypísaným v~\thetag1:
$$
2(2p+1)<2q(2p+1)<p(2p+1).
$$
Ak číslo~$p$ nemá iného netriviálneho deliteľa okrem $q=\frac12p$, platí $p=4$. Vtedy však ani číslo $2p+1=9$ nie je prvočíslo, takže piateho deliteľa nájdeme podľa nasledujúceho odseku.

Ak (nepárne) číslo $2p+1$ je zložené, tak je deliteľné niektorým $q$, $3\le q<p$, a~číslo~$D$ má deliteľa $2pq$, ktorý leží medzi druhým a~tretím deliteľom vypísaným v~\thetag1:
$$
2(2p+1)<2pq<p(2p+1),\quad\text{lebo}\quad
q>2+\frac{1}{p}\ \text{a}\
q<p+\frac{1}{2}.
$$
Ekvivalencia z~časti~b) je tak dokázaná.
}

{%%%%%   A-III-5
Označme každú mincu číslom vrcholu, na ktorom leží (teda číslom z množiny $\{1,2,\dots,n\}$) a~po každom preložení dvojice mincí označenie upravme. Sledujme, ako sa zmení súčet~$S$ všetkých čísel priradených minciam po jednom preložení.

Ak neprekladáme mince medzi vrcholmi $A_1$ a~$A_n$, súčet sa nezmení, lebo na jednej minci sa číslo zmenší a~na druhej zväčší o~$1$. Rovnako sa súčet nezmení, ak preložíme jednu mincu z~$A_1$ do~$A_n$ a~druhú z~$A_n$ do $A_1$. Ak preložíme jednu mincu z~$A_1$ do $A_n$ a~druhú z~$A_i$ do $A_{i+1}$ (kde $1\le i\le n-1$), súčet $S$ stúpne o~$(n-1)+1=n$. Ak naopak preložíme jednu mincu z~$A_n$ do $A_1$ a~druhú z~$A_{i+1}$ do $A_i$ (kde $1\le i\le n-1$), súčet~$S$ klesne o~$n$. Z~uvedeného vyplýva, že zvyšok súčtu~$S$ po delení číslom~$n$ sa nikdy nezmení.

V~počiatočnej pozícii má súčet~$S$ hodnotu
$$
1\cdot 1+2\cdot 2+\cdots+n\cdot n=\sum_{k=1}^n k^2=\frac16n(n+1)(2n+1),
$$
v~želanej konečnej pozícii má $S$ hodnotu
$$
\align
\sum_{k=1}^n k(n+1-k) &= (n+1)\sum_{k=1}^n k- \sum_{k=1}^n k^2  =\\
&=\frac12n(n+1)^2-\frac16n(n+1)(2n+1)=\frac16n(n+1)(n+2)
\endalign
$$
(využili sme známe vzorce pre súčet prvých a~druhých mocnín čísel od $1$ po $n$). Aby bolo možné presunúť mince z~počiatočnej do želanej konečnej pozície, musia uvedené dve hodnoty dávať rovnaký zvyšok po delení~$n$. To je možné len vtedy, keď ich rozdiel $\frac16n(n+1)(n-1)$ je deliteľný~$n$, čiže keď číslo $\frac16(n+1)(n-1)=\frac16(n^2-1)$ je celé. Dosadením jednotlivých zvyškových tried modulo~6 ľahko overíme, že táto podmienka je splnená práve vtedy, keď $n$ dáva po delení šiestimi zvyšok $1$ alebo $5$.

\smallskip
Ostáva ukázať, že pre takéto $n$ vieme mince poprekladať požadovaným spôsobom. Popíšeme jeden z~možných postupov. Mincu, ktorá je na začiatku vo vrchole~$A_1$, nazvime $M$. Všetky mince budeme prekladať v~rovnakom smere, jedine mincu~$M$ (ktorú preložíme v~každom kroku bez toho, aby by sme to v~nasledujúcom odseku pripomínali) budeme prekladať opačným smerom a~dodržiavať tak zadané pravidlá prekladania.
\insp{a58.15}%

Označme $n=2m+1$ (zrejme uvažované hodnoty $n$ sú nepárne). Mince budeme prekladať tak, ako je znázornené na \obr. Najprv preložíme $n-1$ mincí z~vrcholu~$A_n$ na vrchol~$A_1$ (posun o~$1$), potom $n-3$ mincí z~vrcholu $A_{n-1}$ na vrchol~$A_2$ (posun o~$3$), atď., až napokon preložíme 2~mince z~vrcholu $A_{m+2}$ na vrchol $A_m$ (posun o~$n-2$). Tým dosiahneme, nepočítajúc mincu~$M$, že v~každom vrchole bude požadovaný počet mincí, len vo vrchole $A_1$ bude $n-1$ mincí. Vypočítajme, kde sa po uvedených preloženiach nachádza minca~$M$.

Celkový počet preložení bol
$$
\align
T & = (n-1)\cdot 1+(n-3)\cdot 3+\cdots+2\cdot (n-2)= \sum_{k=1}^m k(n-k) = \\
  & = (2m+1)\sum_{k=1}^m k - \sum_{k=1}^m k^2 =\frac12m(m+1)(2m+1)-\frac16m(m+1)(2m+1)=\\
  & = \frac13m(m+1)(2m+1)=\frac13m(m+1)n.
\endalign
$$
Pritom $m(m+1)=\frac14(n+1)(n-1)$ je číslo deliteľné tromi, lebo $n$ nie je násobkom troch. Takže $T$ je celočíselným násobkom~$n$ a~minca $M$ sa po príslušnom počte okruhov ocitla opäť vo vrchole $A_1$. V~ňom je preto tiež požadovaný počet $n$ mincí.

\odpoved
Požadovaný stav je možné dosiahnuť práve vtedy, keď $n$ dáva po delení šiestimi zvyšok $1$ alebo $5$.
}

{%%%%%   A-III-6
Vezmime nejaký bod~$A$ z~roviny~$\omega$. Aby mohol byť vrcholom trojuholníka opísaného v~zadaní, musí byť rôzny od bodov $O$ a~$T$. Popíšeme všeobecnú konštrukciu trojuholníka $ABC$, v~ktorom máme daný vrchol~$A$, stred opísanej kružnice~$O$ a~ťažisko~$T$ (pre trojicu navzájom rôznych bodov $A$, $O$, $T$). Potom zistíme, pre ktoré body~$A$ takýto trojuholník nie je možné skonštruovať.

Nech $A'$ je stred strany~$BC$. Bod~$A'$ je obrazom bodu~$A$ v~rovnoľahlosti so stredom~$T$ a~koeficientom~$\m\frac12$. Ak $A'\ne O$, body $B$ a~$C$ ležia na kolmici~$p$ na priamku~$OA'$ vedenej bodom~$A'$ a~zároveň na opísanej kružnici~$k$ so stredom~$O$ a~polomerom~$OA$ (\obr).
\insp{a58.16}%

K~danému bodu~$A$ vieme vždy zostrojiť bod~$A'$ ako jeho obraz v~uvedenej rovnoľahlosti. Predpokladajme najprv, že $A'\ne O$. Aby sme dostali dva rôzne body $B$ a~$C$, musí byť priamka~$p$ sečnicou kružnice~$k$. To nastáva práve vtedy, keď $|OA'|<|OA|$. Označme $O'$ obraz bodu~$O$ v~rovnoľahlosti so stredom~$T$ a~koeficientom~$\m2$. Platí $|O'A|=2|OA'|$, preto konštrukčnú podmienku môžeme zapísať v~tvare $|O'A|<2|OA|$. Takže bod~$A$ musí ležať mimo kruhu určeného Apollóniovou kružnicou\footnote{Pre dané dva rôzne body $P$, $Q$ a~kladné číslo $k\ne1$ je Apollóniova kružnica množina bodov~$X$, pre ktoré platí $|PX|=k|QX|$. Stred Apollóniovej kružnice leží na priamke $PQ$, rovnako ako dva body kružnice, ktoré vieme pre dané $k$ jednoducho zostrojiť.} $m(S;|ST|)$, kde $S$ je bod súmerne združený s~bodom~$T$ podľa bodu~$O$ (\obr).

Ak teda $A'\ne O$, čiže $A\ne O'$, dostaneme konštrukciou tri body $A$, $B$, $C$. Tie budú vrcholmi vyhovujúceho trojuholníka, ak neležia na priamke. Na priamke ležia, keď je priamka~$BC$ totožná s~priamkou~$AT$, \tj. keď priamka~$OA'$ je kolmá na $AT$. Bod~$A'$ preto nesmie ležať na Tálesovej kružnici nad priemerom~$OT$ a~(po "zobrazení" tejto podmienky v~rovnoľahlosti so stredom~$T$ a~koeficientom $\m2$) bod~$A$ nesmie ležať na Tálesovej kružnici nad priemerom~$O'T$ (\obr).
\inspinsp{a58.17}{a58.18}%

V~prípade, že bod~$A$ je totožný s~bodom $O'$, \tj. $A'=O$, namiesto kolmice $p$ môžeme zobrať ľubovoľnú priamku (rôznu od $AT$) prechádzajúcu bodom~$O$ (\obr). Dostaneme tak nekonečne veľa rôznych trojuholníkov $ABC$ s~pravým uhlom pri vrchole~$A$, ktoré spĺňajú všetky podmienky zadania.
\insp{a58.19}%

\zaver
Hľadanou množinou bodov je vonkajšia oblasť kružnice~$m$ okrem bodov ležiacich na Tálesovej kružnici nad priemerom~$O'T$, pričom bod~$O'$ do hľadanej množiny tiež patrí (\obr).
\insp{a58.20}%

\poznamka
Hľadaná množina bodov sa dá popísať analyticky bez toho, aby sme využili poznatok o~Apollóniovej kružnici.
}

{%%%%%   B-S-1
Sčítaním druhej a~tretej rovnice dostaneme $2x=2a+1$, odčítaním druhej rovnice od tretej $2y=\m2a+1$. Odtiaľ vyjadríme
$$
% \eqalign{
% x&=\hphantom{-}a+\tfrac12,\cr
% y&=\m a+\tfrac12}
x=a+\tfrac12,\quad y=-a+\tfrac12
\tag1
$$
a~dosadíme do prvej rovnice pôvodnej sústavy. Po úprave dostaneme kvadratickú rovnicu
$$
a^2-\tfrac12a-\tfrac32=0,
\tag2
$$
ktorá má korene $a_1=\m1$ a~$a_2=\tfrac32$. Pre každú z~týchto dvoch (jediných možných) hodnôt parametra~$a$ už ľahko stanovíme neznáme $x$ a~$y$ dosadením do vzťahov~\thetag1.

Daná sústava rovníc má riešenie iba pre dve hodnoty parametra~$a$, jednak pre $a=\m1$, keď je jej jediným riešením
$(x,y)=\left({-\tfrac12},{\tfrac32}\right)$, jednak pre $a=\tfrac32$, keď $(x,y)=\left(2,{-1}\right)$.

Skúška dosadením je jednoduchá, možno ju vynechať takýmto zdôvodnením: Sústava dvoch rovníc, ktorú sme dostali (a~vyriešili) sčítaním a~odčítaním druhej a~tretej rovnice, je s~dvojicou pôvodných rovníc ekvivalentná. Zostávajúca (prvá) rovnica sústavy je potom ekvivalentná s~kvadratickou rovnicou~\thetag2, ktorej riešením sme našli možné hodnoty
parametra~$a$.

\nobreak\medskip\petit\noindent
Za úplné riešenie dajte 6~bodov, z~toho 2~body za správne vyjadrenie $x$ a~$y$ z~druhej a~tretej rovnice, 2~body za vyriešenie kvadratickej rovnice, ktorá vznikne dosadením týchto hodnôt do prvej rovnice, a~po 1~bode za správnu odpoveď a~skúšku. Za numerické chyby pri výpočte strhnite najviac 1~bod.

\endpetit
\bigbreak}

{%%%%%   B-S-2
Označme $S$ stred úsečky~$CP$. Podľa Tálesovej vety ležia body $K$ a~$L$ na kružnici~$p$ zostrojenej nad priemerom~$CP$. Predpokladajme, že bod~$P$ má požadovanú vlastnosť, \tj. že priemer~$CP$ rozpoľuje tetivu~$KL$ (\obr).
\insp{b58.6}%

Priemer ľubovoľnej kružnice rozpoľuje každý iný priemer tejto kružnice a~tiež všetky tetivy naň kolmé. Žiadnu inú tetivu rozpoľovať nemôže: keď totiž prechádza dvoma {\it rôznymi} bodmi jej osi súmernosti (stredom tetivy a~stredom kružnice), musí byť -- rovnako ako táto os -- na danú tetivu kolmý.

Tetiva~$KL$ však nemôže byť priemerom kružnice~$p$, pretože podľa Tálesovej vety by bol uhol $KCL$ (a~teda aj uhol $ACB$) pravý, čo odporuje zadaniu, preto je tetiva~$KL$ na priemer~$CP$ kolmá. V~tomto prípade sú trojuholníky $CKP$ a~$CLP$ súmerne združené podľa priamky~$CP$, odkiaľ už vyplýva, že uhly $KCP$ a~$LCP$ sú zhodné. Polpriamka~$CP$ je teda osou uhla $ACB$.

Ak je naopak polpriamka~$CP$ osou uhla $ACB$, zhodujú sa pravouhlé trojuholníky $CKP$ a~$CLP$ v~spoločnej prepone~$CP$ a~v~dvoch vnútorných uhloch, takže body $K$ a~$L$ sú súmerne združené podľa priamky~$CP$. Preto tetiva~$CP$ rozpoľuje úsečku~$KL$.

\odpoved
Existuje práve jeden vnútorný bod strany~$AB$ ostrouhlého trojuholníka $ABC$, pre ktorý úsečka~$CP$ rozpoľuje úsečku~$KL$. Je to priesečník osi vnútorného uhla pri jeho vrchole~$C$ so stranou~$AB$.

\nobreak\medskip\petit\noindent
Za úplné riešenie dajte 6~bodov, z~toho 4~body za dôkaz skutočnosti, že bod~$P$ musí ležať na osi uhla $ACB$ a~2~body za overenie, že bod ležiaci na osi uhla má požadovanú vlastnosť. Ak riešiteľ bez dôkazu uvedie, že bod~$P$ leží na osi uhla $ACB$, udeľte len 1~bod. Tvrdenie o~tetivách, ktoré priemer kružnice rozpoľujú, možno považovať za zrejmé. Naopak, strhnite 1~bod, ak si riešiteľ neuvedomí, že tetiva~$KL$ nemôže byť priemerom kružnice~$p$.

\endpetit
\bigbreak}

{%%%%%   B-S-3
Každé trojciferné číslo má vyjadrenie $m=100a+10b+c$, kde $a$, $b$, $c$ sú jeho cifry a~$a\ne0$. Trojciferné číslo zapísané rovnakými ciframi v~opačnom poradí má potom vyjadrenie $m'=100c+10b+a$, $c\ne0$.
%% Pak $m'=100c+10b+a$ je vyjádření
%% trojmístného čísla zapsaného stejnými číslicemi v~opačném pořadí.
Keďže na poradie čísel $m$ a~$m'$ neberieme ohľad, pre určitosť predpokladajme, že $m\le m'$, čiže $a\le c$, pričom
$a,c\in\{1,2,3,\dots,9\}$ a~$b\in\{0,1,2,\dots,9\}$.

Pre magické číslo~$x$ podľa zavedeného označenia cifier platí
$$
x=m+m'=101(a+c)+20b.
$$
Vidíme, že hodnota~$x$ nezávisí ani tak od jednotlivých cifier $a$, $c$, ako od ich súčtu $s=a+c$, ktorý môže nadobúdať hodnoty $s\in\{2,3,\dots,18\}$. Ďalej už budeme pracovať iba s~vyjadrením $x=101s+20b$.

Predpokladajme na chvíľu, že sa ako súčet $101s+20b$ dá niektoré magické číslo~$x$ zapísať dvoma rôznymi spôsobmi:
$$
x=101s+20b=101s'+20b'.    \tag1
$$
Z~rovnosti $101(s-s')=20(b-b')$ a~nesúdeliteľnosti čísel $101$ a~$20$ vyplýva, že číslo $101$ musí deliť číslo $b-b'$. Keďže však $b$ a~$b'$ sú cifry, platí $\m9\le b-b'\le9$. V~tomto intervale nájdeme jediné číslo deliteľné číslom $101$,
a~to číslo~$0$. Preto $b-b'=0$, čiže $b=b'$, a~teda aj $s=s'$. To však odporuje predpokladu, že číslo~$x$ má dve rôzne vyjadrenia tvaru~\thetag1. Znamená to, že vo vyjadrení $x=101s+20b$ má každé magické číslo~$x$ jednoznačne určenú cifru~$b$ aj jednoznačne určený súčet~$s$.

Počet spôsobov, ktorými možno magické číslo vyjadriť ako súčet $m+m'$, čiže $101s+20b$, sa preto rovná počtu spôsobov, ktorými možno vyjadriť zodpovedajúcu hodnotu~$s$ ako súčet dvoch cifier~$a$ a~$c$, pričom $1\le a\le c\le 9$. V~množine $\{2,3,\dots,18\}$
%% je taková
má najväčší počet takých vyjadrení
%% hodnota $s$ zřejmě jediná a~je rovna číslu~10,
číslo~$s=10$, ktoré sa dá vyjadriť práve piatimi vyhovujúcimi súčtami:
$$
\postdisplaypenalty 10000
10=1+9=2+8=3+7=4+6=5+5.
$$
Ostatné čísla majú takých vyjadrení menej.

%% Podrobnější zdůvodnění:
Naozaj: v~prípade $s\le9$ z~rovnosti $a+c\le9$ a~predpokladu $a\le c$ vyplýva $a\le4$, takže menšia cifra~$a$ nadobúda nanajvýš štyri hodnoty, rovnako ako väčšia cifra~$c$ v~prípade $s\ge11$, keď zo vzťahov $a+c\ge11$ a~$a\le c$ vyplýva $c\ge6$.

Najväčším počtom súčtov $m+m'$ (piatimi súčtami) sa dajú vyjadriť magické čísla tvaru $101\cdot10+20b$, kde $b\in\{0,1,2,\dots,9\}$, jedná sa teda o~čísla z~desaťprvkovej množiny
$$
\{1\,010,1\,030,1\,050,\dots,1\,190\}.
$$

\nobreak\medskip\petit\noindent
Za úplné riešenie dajte 6~bodov, z~toho 3~body za dôkaz skutočnosti, že magické číslo má jediné vyjadrenie súčtom
$101s+20b$. Poznatok, že najväčší počet vyjadrení $s=a+c$ má číslo $s=10$, je natoľko zrejmý, že môže byť uvedený bez zdôvodnenia. V~prípade správneho postupu s~numericky chybným vyčíslením niektorých z~10 riešení strhnite najviac
1~bod.

\endpetit
\bigbreak}

{%%%%%   B-II-1
Sčítaním prvej a~druhej rovnice danej sústavy dostaneme $2x=1+a$, odčítaním druhej rovnice od prvej $2y=1-a$. Odtiaľ
$$
x=\frac12(1+a),\qquad y=\frac12(1-a).\tag1
$$
Keď dosadíme za $x$ a~$y$ do tretej rovnice pôvodnej sústavy, dostaneme rovnicu
$$
-2a(1+a)+2(1-a)=z^2+4,\quad\text{čiže}\quad z^2+2a^2+4a+2=0,
$$
ktorú upravíme na tvar
$$
z^2+2(a+1)^2=0.\niedorocenky{\tag2}
$$
Oba sčítance na ľavej strane poslednej rovnice sú nezáporné čísla. Ich súčet je $0$ práve vtedy, keď $z=0$, $a=\m1$. Dosadením týchto hodnôt do \thetag1 dostaneme $x=0$, $y=1$.

\zaver
Daná sústava rovníc má riešenie iba pre $a=\m1$, a~to $x=0$, $y=1$, $z=0$. Skúška pri tomto postupe nie je nutná.

\nobreak\medskip\petit\noindent
Za úplné riešenie dajte 6~bodov, z~toho 2~body za správne vyjadrenie $x$ a~$y$ pomocou parametra~$a$ z~prvých dvoch rovníc, 3~body za vyriešenie rovnice~\thetag2, ktorá vznikne dosadením týchto hodnôt do tretej rovnice a~1~bod za uvedenie správnej odpovede.
\endpetit
\bigbreak}

{%%%%%   B-II-2
a) Stačí sa spýtať napríklad na čierne políčka na \obr: v~každom riadku aj stĺpci sú vedľa seba najviac dve biele políčka, zatiaľ čo každá z~lodí zakryje v~jednom z~oboch smerov pravé tri vedľa sebe stojace políčka. Aspoň jedno z~nich teda bude čierne.
\inspinsp{b58.7}{b58.8}

\smallskip
b) Na~zásah lode na pláne s~rozmermi $3\times2$ potrebujeme aspoň dve otázky, pretože žiadne jeho políčko neleží na všetkých lodiach, ktoré na tento plán môžeme umiestniť. Na pláne $5\times5$ môžeme vyznačiť štyri neprekrývajúce sa oblasti $3\times2$ (\obr). Aj keby loď bola umiestnená iba na jednej z~týchto štyroch oblastí, sedem otázok na jej zásah by nestačilo~-- podľa predchádzajúcej úvahy totiž potrebujeme aspoň $4\cdot2=8$ otázok.
%% Pokud
%% bychom položili sedm otázek, potom bychom alespoň na jednu ze
%% čtyř desek $3\times2$ kladli nejvýše jednu otázku a~na takové
%% desce se může nacházet nezasažená loď. Sedm otázek proto obecně nestačí.

\nobreak\medskip\petit\noindent
Za úplné riešenie úlohy dajte 6~bodov, z~toho za časť~a) dajte najviac 2~body (a~to aj za jednoduchý náčrtok bez ďalšieho zdôvodnenia), časť~b) ohodnoťte najviac 4~bodmi.
\endpetit
\bigbreak}

{%%%%%   B-II-3
Označme vnútorné uhly v~trojuholníku $ABC$ zvyčajným spôsobom. Nech $K'$ je druhý priesečník osi uhla $ACB$ s~kružnicou opísanou trojuholníku $ABC$. Zo zhodnosti obvodových uhlov $ACK'$ a~$BCK'$ vyplýva zhodnosť zodpovedajúcich tetív $AK'$ a~$BK'$, takže bod~$K'$ rozpoľuje oblúk~$AB$ ležiaci oproti vrcholu~$C$, a~je preto totožný s~bodom~$K$ (\obr). Podľa vety o~obvodových uhloch sú
\insp{b58.9}%
veľkosti uhlov $AKC$ a~$BKC$ postupne rovné $\beta$ a~$\alpha$. Označme $V_a$, $V_b$ päty výšok prislúchajúcich vrcholom $A$, $B$ trojuholníka $ABC$. Keďže $ABC$ je ostrouhlý trojuholník, sú body $V_a$ a~$V_b$ vnútorné body zodpovedajúcich strán. Veľkosť uhla $APK$ je zhodná s~veľkosťou vnútorného uhla pri vrchole~$P$ v~pravouhlom trojuholníku $CPV_a$, je teda rovná $90^\circ-\frac12\gamma$. Rovnakú veľkosť má analogicky aj uhol $BQK$.

Trojuholníky $AKP$ a~$BKQ$ majú rovnaký obsah, zhodné strany $AK$ a~$BK$, a~teda aj výšky na ne, a~navyše sa zhodujú aj v~uhle oproti nim. Z~konštrukcie trojuholníka podľa danej strany, výšky na túto stranu a~protiľahlého vnútorného uhla a~zo súmernosti zostrojených riešení vyplýva, že trojuholník $AKP$ je zhodný buď s~trojuholníkom $KBQ$, alebo s~trojuholníkom $BKQ$. Keďže trojuholník $ABC$ nie je rovnoramenný (\tj.~$\alpha\ne\beta$), je trojuholník $AKP$ zhodný
s~trojuholníkom $KBQ$. Veľkosť vnútorného uhla pri vrchole~$A$ trojuholníka $PAK$ je
$180^\circ-\beta-(90^\circ-\frac12\gamma)=90^\circ-\beta+\frac12\gamma$,
takže z~uvedenej zhodnosti vyplýva
$$
90^\circ-\beta+\frac\gamma2=\alpha, \qquad \hbox{čiže}\qquad 90^\circ+\frac\gamma2=
\alpha+\beta=180^\circ-\gamma.
$$
Odtiaľ dostávame $\gamma=60^\circ$. Naopak ak $\gamma=60^\circ$, je $|\uhol APK|=|\uhol BQK|=60\st$ a~trojuholníky $AKP$ a~$KBQ$ sú zhodné podľa vety {\it usu}, majú teda rovnaký obsah.

\zaver
Uhol $ACB$ má veľkosť $60^\circ$.

\nobreak\medskip\petit\noindent
Za úplné riešenie dajte 6~bodov, z~toho 1~bod za zistenie, že $K$ je stredom oblúka~$AB$ (dôkaz nie je nutný, ak študent uvedie, že ide o~známu skutočnosť). Ďalej 1~bodom oceňte vyjadrenie veľkostí vnútorných uhlov v~trojuholníkoch $AKP$ a~$BKQ$ pomocou $\alpha$, $\beta$, $\gamma$. Ďalšie 2~body dajte za dôkaz zhodnosti trojuholníkov $AKP$ a~$BKQ$, 1~bod za výpočet uhla $\gamma$ a~1~bod za dôkaz, že z~podmienky $\gamma=60^\circ$ vyplýva zhodnosť (obsahov)
trojuholníkov $AKP$ a~$KBQ$.
\endpetit
\bigbreak}

{%%%%%   B-II-4
Uvažujme prirodzené číslo $n<25\,000$ a~označme $r_2$, $r_3$, \dots, $r_{11}$ jemu prislúchajúce zvyšky po delení číslami $2$, $3$, \dots, $11$. Súčet nezáporných zvyškov
% pro všechna čísla~$k\in \mm M=\{2, 3, 4,\dots, 11\}$ platí $0\leq r_k<k$.
%je príslušný
% Pro žádné z uvažovaných čísel $n$ nemůže být
%súčet
$z=r_2+r_3+\cdots+r_{11}$ je tiež nezáporný. V~danom prípade však nemôže byť rovný~$0$, pretože to by znamenalo, že číslo~$n$ je deliteľné každým z~prvkov množiny~$\mm M=\{2, 3, 4,\dots, 11\}$, ktorých najmenší spoločný násobok je $27\,720>25\,000$.

Ukážeme, že najmenší možný súčet je~$1$, a~zároveň nájdeme aj všetky čísla~$n$ menšie ako $25\,000$ s~touto vlastnosťou.

Ak je %(pro některé přirozené číslo $n<25\,000$)
príslušný súčet rovný~$1$, sú všetky zvyšky~$r_k$ s~výnimkou jedného rovné~$0$, a~existuje teda práve jedno $d\in \mm M$ také, že $r_d=1$.
%a~tento jeden zbytek $r_d$ by byl~1.
Ukážeme, že $d=7$ alebo $d=11$. Zrejme nemôže byť $d\le5$, to by totiž nenulový zvyšok prislúchal aj číslu $2d\in \mm M$. Keby zvyšok~$1$ prislúchal jednému z~čísel $d=6,8,9,10$, prislúchal by nutne aj jednému z~čísel $2$ alebo $3$.

% \item{a)}
Ak $d=7$, musí byť hľadané číslo násobkom všetkých čísel z~množiny $\mm M\setminus\{7\}$, teda násobkom čísla $3\,960$. Toto
číslo dáva po delení~$7$ zvyšok~$5$, zvyšok~$1$ dáva jeho trojnásobok $n={3\cdot3\,960}=11\,880$, ktorý vyhovuje podmienkam úlohy, a~všeobecne každý $(3+7a)$-násobok; avšak ďalší násobok $10\cdot3\,960$ s~vyhovujúcim zvyškom je už väčší ako $25\,000$.

% \item{b)}
Ak $d=11$, musí byť hľadané číslo násobkom všetkých čísel z~množiny $\mm M\setminus\{11\}$, teda násobkom čísla $2\,520$. Keďže toto číslo dáva po delení~$11$ zvyšok~$1$, vyhovuje pre $d=11$ jedine ono (ďalší násobok $(1+11)\cdot2\,520$ s~vyhovujúcim zvyškom je totiž už väčší ako $25\,000$).

\zaver
Hľadané čísla sú dve, a~to $11\,880$ a~$2\,520$.

\nobreak\medskip\petit\noindent
Za úplné riešenie dajte 6~bodov, za dôkaz, že súčet zvyškov je aspoň jedna, dajte 1~bod, za dôkaz, že hľadané číslo je
deliteľné všetkými číslami z~$\mm M$ okrem $7$ alebo $11$, pre ktoré dáva zvyšok~$1$, dajte 3~body, za nájdenie zodpovedajúcich čísel $11\,880$ a~$2\,520$ dajte po 1~bode. Ak riešiteľ ukáže, že súčet zvyškov je aspoň~$1$, a~uvedením jedného z~vyhovujúcich čísel ukáže, že najmenší súčet zvyškov je skutočne~$1$, ohodnoťte jeho riešenie 3~bodmi (lebo úlohou bolo nájsť {\it všetky\/} vyhovujúce čísla).
\endpetit}

{%%%%%   C-S-1
Roznásobením a~ďalšími ekvivalentnými úpravami dostaneme
$$
\align
ab+b^2c+a^2c+abc^2 &\geqq abc^2+2abc+ab,\\
b^2c+a^2c&\geqq2abc,\\
(a-b)^2c&\geqq0.
\endalign
$$
Podľa zadania platí $c\ge0$ a~druhá mocnina reálneho čísla $a-b$ je tiež nezáporná, takže je nezáporná aj ľavá strana upravenej nerovnosti. Rovnosť v~tejto (a~rovnako aj v~pôvodnej nerovnosti) nastane práve vtedy, keď $a-b=0$ alebo $c=0$, teda práve vtedy, keď je splnená aspoň jedna z~podmienok $a=b$, $c=0$.

\nobreak\medskip\petit\noindent
Za úplné riešenie dajte 6~bodov. Za upravenú nerovnosť $(a-b)^2c\geqq0$ dajte 4~body, ďalší 1~bod potom za argument o~nezápornosti činiteľov. Odvodenie (nie však uhádnutie) oboch prípadov rovnosti oceňte tiež 1~bodom. Ten nedajte, ak nie je diskusia o~rovnosti úplná (ak je napríklad vynechaná možnosť $c=0$).
\endpetit
\bigbreak}

{%%%%%   C-S-2
V~pravouhlom trojuholníku $ABC$ s~preponou~$AB$ pre veľkosti $\alpha$, $\beta$ uhlov pri vrcholoch $A$, $B$ platí $\alpha+\beta=90\st$, preto $|\angle ACP|=90\st-\a=\b$ a~$|\angle BCD|=|\angle DCP|=\frac12(90\st-\b)=\frac12\a$,
lebo priamka~$CD$ je osou uhla~$BCP$ (\obr). Pre vonkajší uhol $ADC$ trojuholníka $BCD$ tak zrejme platí $|\angle ADC|=|\angle DBC|+|\angle BCD|=\beta+\frac12\alpha=|\angle DCA|$.

Zistili sme, že trojuholník $ADC$ má pri vrcholoch $C$, $D$ zhodné vnútorné uhly, je teda rovnoramenný, a~preto $|AD|=|AC|$.
\insp{c58.4}

\nobreak\medskip\petit\noindent
Za úplné riešenie dajte 6~bodov. Správne čiastočné poznatky vedúce k~riešeniu oceňte 1~bodom (napríklad výpočet len jedného z~uhlov $ACD$ alebo $ADC$), nanajvýš dvoma bodmi.
\endpetit
\bigbreak}

{%%%%%   C-S-3
Pre hľadané prirodzené čísla $x$ a~$y$ sa dá podmienka zo zadania vyjadriť rovnicou
$$
(x+y)+(x-y)+\Bigl(\frac xy\Bigr)+(x\cdot y)=2\,009,
\tag1
$$
v~ktorej sme čiastočné výsledky jednotlivých operácií dali do zátvoriek.

Vyriešme rovnicu \thetag1 vzhľadom na neznámu~$x$ (v~ktorej je, na rozdiel od neznámej~$y$, rovnica {\it lineárna\/}):
$$
\align
2x+\frac xy+xy&=2\,009,\\
2xy+x+xy^2&=2\,009y,\\
x(y+1)^2&=2\,009y,\\
x&=\frac{2\,009y}{(y+1)^2}.\tag2
\endalign
$$
Hľadáme práve tie prirodzené čísla~$y$, pre ktoré má nájdený zlomok celočíselnú hodnotu, čo možno vyjadriť vzťahom $(y+1)^2\mid 2\,009y$. Keďže čísla $y$ a~$y+1$ sú nesúdeliteľné, sú nesúdeliteľné aj čísla $y$ a~$(y+1)^2$, takže musí platiť $(y+1)^2\mid 2\,009=7^2\cdot 41$. Keďže $y+1$ je celé číslo väčšie ako~$1$ (a~činitele $7$, $41$ sú prvočísla),
poslednej podmienke vyhovuje iba hodnota $y=6$, ktorej po dosadení do \thetag2 zodpovedá $x=246$. (Skúška nie je nutná, lebo rovnice \thetag1 a~\thetag2 sú v~obore prirodzených čísel ekvivalentné.)

%Odpověď
Hľadané čísla v~uvažovanom poradí sú $246$ a~$6$.


\nobreak\medskip\petit\noindent
Za úplné riešenie dajte 6~bodov, z~toho 1~bod za zostavenie rovnice~\thetag1 a~1~bod za vyjadrenie~\thetag2 či ekvivalentnú rovnicu v~súčinovom tvare $x(y+1)^2=2\,009y$. Iba uhádnutie jej riešenia oceňte ďalším bodom. Ak je v~inak úplnom riešení uvedená podmienka $(y+1)^2\mid 2\,009$ bez potrebnej zmienky o~nesúdeliteľnosti čísel $y$ a~$(y+1)^2$, 1~bod strhnite. Vynechanie skúšky (či zmienky o~jej zbytočnosti) stratou bodu nepenalizujte.

\endpetit
\bigbreak}

{%%%%%   C-II-1
Výraz~$V$ je zrejme definovaný pre všetky reálne čísla~$x$.

\smallskip
a)
Keďže $x^4+1>0$ pre každé~$x$, nerovnosť $V(x)\ge 3$ je ekvivalentná s~nerovnosťou $5x^4-4x^{2}+5\ge3(x^4+1)$, čiže
$2x^4-4x^{2}+2\ge0$. Výraz na ľavej strane je rovný $2(x^2-1)^2$, takže je nezáporný pre každé~$x$.

\smallskip
b)
Využime nasledujúcu úpravu:
$$
V(x)=\frac{5x^{4}-4x^{2}+5}{x^{4}+1}=
\frac{5(x^{4}+1)}{x^{4}+1}-\frac{4x^{2}}{x^{4}+1}=
5-\frac{4x^{2}}{x^{4}+1}.
$$
Keďže zlomok
$$
\frac{4x^{2}}{x^{4}+1}
$$
je vďaka párnym mocninám premennej~$x$ pre ľubovoľné reálne číslo~$x$ nezáporný, nadobúda výraz~$V$ svoju najväčšiu hodnotu $V_{\max}$ práve vtedy, keď
$$
\frac{4x^{2}}{x^{4}+1}=0,
$$
teda práve vtedy, keď $x=0$. Dostávame tak $V_{\max}=V(0)=5$.

\nobreak\medskip\petit\noindent
Za úplné riešenie dajte 6~bodov, z~toho 2~body za vyriešenie časti~a), 4~body za úplné riešenie časti~b): 3~body za dôkaz nerovnosti $V(x)\le5$ a~1~bod za určenie rovnosti pre $x=0$. Algebraickú úpravu zlomku $V(x)$ čiastočným vydelením čitateľa menovateľom bez ďalšieho úspešného zhodnotenia oceňte 1~bodom.
\endpetit
\bigbreak}

{%%%%%   C-II-2
V~pravouhlom trojuholníku $ABC$ s~preponou~$AB$ označme $\alpha$ veľkosť vnútorného uhla pri vrchole~$A$, zrejme potom platí $|\uhol ACP|=90\st-\alpha$, $|\uhol PCB|=\alpha$. Stred~$D$ kružnice vpísanej trojuholníku $APC$ leží na osi uhla $PAC$, takže $|\uhol DAC|=\frac12\alpha$, a~podobne aj $|\uhol PCE|=\frac12\alpha$. Odtiaľ pre veľkosť uhla $AUC$ v~trojuholníku $AUC$, pričom $U$ je priesečník polpriamok $AD$ a~$CE$ (\obr), vychádza
$$
|\uhol AUC|=180\st-(90\st-\alpha+\tfrac12\alpha)-\tfrac12\alpha=90\st.
$$
To znamená, že polpriamka~$AD$ je kolmá na $CE$, úsečka~$DU$ je teda výška v~trojuholníku $DEC$. Úplne rovnako zistíme, že aj polpriamka~$BE$ (ktorá je zároveň osou uhla~$ABC$) je kolmá na $CD$. Dostávame tak, že priesečník polpriamok $AD$ a~$BE$, čo je stred kružnice vpísanej trojuholníku $ABC$, je zároveň aj priesečníkom výšok trojuholníka $DEC$.
\insp{c58.5}%

\ineriesenie
Označme $F$ a~$G$ zodpovedajúce priesečníky priamok $CD$ a~$CE$ so stranou~$AB$ (\obr).
\insp{c58.6}%
Podľa tvrdenia 2.\,úlohy školského kola je trojuholník $CAG$ rovnoramenný so základňou~$CG$. Os~$AD$ uhla $CAG$ rovnoramenného trojuholníka $CAG$ je tak aj jeho osou súmernosti a~je preto kolmá na základňu~$CG$, teda aj~na $CE$.
Podobne zistíme, že aj trojuholník $CBF$ je rovnoramenný so základňou~$CF$, takže os~$BE$ uhla $FBC$ je kolmá na $CF$, teda aj na $CD$. Priesečník oboch osí $AD$ a~$BE$ je tak nielen stredom kružnice vpísanej trojuholníku $ABC$, ale aj priesečníkom výšok trojuholníka $CDE$, čo sme mali dokázať.

\nobreak\medskip\petit\noindent
Za úplné riešenie dajte 6~bodov. V~opačnom prípade oceňte 1~bodom jednotlivé čiastočné poznatky vedúce k~riešeniu (napríklad výpočet jedného z~uhlov $ACP$ alebo $PCE$). Za odhalenie rovnoramenného trojuholníka $CAG$ alebo $CBF$ a~odkaz na úlohu školského kola dajte 3~body rovnako ako za iný dôkaz kolmosti $AD$ a~$CE$ či $BE$ a~$CD$.
\endpetit
\bigbreak}

{%%%%%   C-II-3
Podľa zvyškov po delení deviatimi rozdelíme všetkých 99 uvažovaných čísel do deviatich jedenásťprvkových tried $T_0$, $T_1$, \dots, $T_{8}$ (do~triedy $T_i$ patria všetky čísla so zvyškom~$i$):
$$
\align
T_0=&\{9,18,27,\dots,99\},\\
T_1=&\{1,10,19,\dots,91\},\\
T_2=&\{2,11,20,\dots,92\},\\
   \vdots&\\
T_{8}=&\{8,17,26,\dots,98\}.
\endalign
$$

\smallskip
a) Našou úlohou je dokázať, že v~$T_{0}\cup T_{3}\cup T_{6}$ ležia najviac štyri vybrané čísla. Z~každej z~tried $T_{0}$, $T_{3}$, $T_{6}$ môžu pochádzať najviac dve z~vybraných čísel (súčet ľubovoľných troch čísel z~jednej takej triedy už totiž deliteľný deviatimi je). Keďže súčet ľubovoľných troch čísel, ktoré po jednom ležia v~triedach $T_{0}$, $T_{3}$ a~$T_{6}$, je deviatimi deliteľný, aspoň jedna z~týchto tried žiadne vybrané číslo neobsahuje. Z~oboch vyslovených záverov vyplýva dokazované tvrdenie: vybraných čísel deliteľných tromi je totiž najviac $2+2+0=4$.

\smallskip
b) Ukážeme, že vyhovujúci výber môže obsahovať 26~čísel. Vyberieme po dvoch číslach z~$T_{0}$, $T_{3}$ a~po 11~číslach
(teda všetky čísla) z~$T_{1}$ a~$T_{2}$. Dostaneme tak celkom $2\cdot2+2\cdot11=26$ čísel; pritom
súčet ľubovoľných troch z~nich dáva po delení deviatimi zvyšok aspoň $0+0+1=1$, najviac však $2+3+3=8$, takže deviatimi deliteľný byť nemôže.

%{\it Poznámka}. Lze dokázat, že největší možný počet vybraných
%čísel je právě 26. Hlavní myšlenkou je úvaha, že mezi třídami
%$T_1$, $T_2$, $T_4$, $T_5$, $T_7$, $T_8$ mohou existovat nejvýše
%dvě takové, ze kterých je vybráno po nejméně dvou číslech. Taková
%třída je totiž nejvýše jedna v~každé z~trojic $(T_1,T_4,T_7)$ a
%$(T_2,T_5,T_8)$, neboť každý ze součtů
%$$
%1+1+7,\ 4+4+1,\ 7+7+4,\quad\text{resp.}\quad
%2+2+5,\ 5+5+8,\ 8+8+2
%$$
%je dělitelný devíti.

\nobreak\medskip\petit\noindent
Za úplné riešenie dajte 6~bodov, a~to 3~body za časť~a) a~3~body za časť~b). Ak žiaci v~časti~b) iba uvedú množinu 26~čísel, ktorá spĺňa podmienku zo zadania, bez toho, aby tento fakt nejako odôvodnili, dajte za túto časť iba 1~bod.
\endpetit
\bigbreak}

{%%%%%   C-II-4
Označme $O$ stred opísanej kružnice, teda stred prepony~$AB$ daného pravouhlého trojuholníka $ABC$, a~$v$ veľkosť jeho výšky na preponu (\obr).
\insp{c58.7}%
Trojuholník $EDO$ je zrejme tiež pravouhlý, pretože jeho strany $DO$ a~$EO$ sú kolmé na odvesny trojuholníka $ABC$; pritom jeho výškou na preponu je úsečka~$OC$ (s~veľkosťou~$\frac12c$). Vzhľadom na súmernosť úsečky~$AC$ podľa osi~$OD$
platí pre jeho uhol pri vrchole~$D$, že $|\uhol CDO|=90\st-|\uhol COD|=90\st-|\uhol AOD|=\alpha$.
%  $|\uh CEO|=90\st-|\uh COE|=90\st-|\uh BOE|=\b$.
Trojuholníky $EDO$ a~$ABC$ sú teda podobné ({\it uu\/}). Koeficient~$k$ tejto podobnosti je daný pomerom dĺžok %velikostí
zodpovedajúcich výšok na prepony, takže $k=|OC|/v=\frac12c/v$,
a~keďže $vc=2S$, je
$$
k={c^2\over4S}.
$$
V~uvedenej podobnosti zodpovedá prepone~$AB$ prepona~$DE$, preto pre jej veľkosť platí
$$
|DE|=kc={c^3\over4S}.
$$

\ineriesenie
Zo súmernosti dotyčníc z~bodu ku kružnici vyplýva, že oba trojuholníky $ACD$ aj $BCE$ sú rovnoramenné, $|AD|=|DC|$, $|BE|=|CE|$. Rovnoramenné sú aj trojuholníky $ACO$ a~$BCO$, pričom $O$ je stred prepony~$AB$ (ramená oboch trojuholníkov majú veľkosť polomeru kružnice opísanej pravouhlému trojuholníku $ABC$, čo je~$\frac12c$). Ukážeme, že ide o~dve dvojice podobných trojuholníkov $ACD\sim BCO$ a~$ACO\sim BCE$. K~tomu si stačí všimnúť, že v~štvoruholníku $AOCD$, ktorý je zložený z~dvoch zhodných pravouhlých trojuholníkov, platí $|\uhol CDA|=180\st-|\uhol AOC|=|\uhol COB|$. Rovnoramenné trojuholníky $ACD$ a~$BCO$ sú teda podobné podľa vety~{\it uu}. Z~tejto podobnosti vyplýva rovnosť $|CD|:|CA|=|CO|:|CB|$, takže pri zvyčajnom označení odvesien dostávame $|CD|=\frac12cb/a$, a z~podobnosti trojuholníkov $ACO$ a~$BCE$ potom $|CE|=\frac12{ca/b}$. Celkom tak je
$$
|DE|=|DC|+|CE|=\frac{cb}{2a}+\frac{ca}{2b}=\frac{cb^2+ca^2}{2ab}=
\frac{c(a^2+b^2)}{2\cdot2S}=\frac{c^3}{4S}.
$$

\poznamky
Podobnosť spomenutých rovnoramenných trojuholníkov môžeme odvodiť tiež tak, že si všimneme rovnosti zodpovedajúcich uhlov $ACO$ a~$BCE$ pri základniach: oba totiž dopĺňajú uhol $OCB$ do pravého uhla ($ACB$, resp. $OCE$). Preto $ACO\sim BCE$.

Ďalšiu možnosť dáva objavenie rovnosti $|\uhol ADO|=|\uhol BAC|=\alpha$ (ramená jedného uhla sú kolmé na ramená druhého). Z~pravouhlého trojuholníka $ODA$ tak máme $|AO|:|AD|=\tg|\uhol ADO|=\tg\alpha=a:b$, takže
$|CD|=|AD|=\frac12cb/a$, a~analogicky pre pravouhlý trojuholník $OEB$.

\nobreak\medskip\petit\noindent
Za úplné riešenie dajte 6~bodov. Za odhalenie vhodnej rovnosti uhlov dajte 3~body, za výpočet dĺžky úsečky~$DE$ ďalšie 3~body.
\endpetit
\bigbreak}

{%%%%%   vyberko, den 1, priklad 1
...}

{%%%%%   vyberko, den 1, priklad 2
...}

{%%%%%   vyberko, den 1, priklad 3
...}

{%%%%%   vyberko, den 1, priklad 4
...}

{%%%%%   vyberko, den 2, priklad 1
...}

{%%%%%   vyberko, den 2, priklad 2
...}

{%%%%%   vyberko, den 2, priklad 3
...}

{%%%%%   vyberko, den 2, priklad 4
...}

{%%%%%   vyberko, den 3, priklad 1
...}

{%%%%%   vyberko, den 3, priklad 2
...}

{%%%%%   vyberko, den 3, priklad 3
...}

{%%%%%   vyberko, den 4, priklad 1
...}

{%%%%%   vyberko, den 4, priklad 2
...}

{%%%%%   vyberko, den 4, priklad 3
...}

{%%%%%   vyberko, den 4, priklad 4
...}

{%%%%%   vyberko, den 5, priklad 1
...}

{%%%%%   vyberko, den 5, priklad 2
...}

{%%%%%   vyberko, den 5, priklad 3
...}

{%%%%%   vyberko, den 5, priklad 4
...}

{%%%%%   trojstretnutie, priklad 1
Postupnými úpravami zadanej podmienky dostávame
$$
\align
1+yf(x)-yf(x+y)-y^2f(x)f(x+y)&=1,\\
yf(x)-yf(x+y)&=y^2f(x)f(x+y).
\endalign
$$
Poslednú rovnosť môžeme vydeliť hodnotou $y\ne0$. Pre ľubovoľné $x,y\in\Bbb R^+$ tak po ďalších úpravách (všetky výrazy, ktorými budeme deliť, sú evidentne nenulové) dostaneme
$$
\align
f(x)-f(x+y)&=yf(x)f(x+y),\\
f(x+y)&=\frac{f(x)}{1+yf(x)},\\
\frac 1{f(x+y)}&=y+\frac1{f(x)},
\endalign
$$
a teda aj
$$
\frac1{f(y+x)}=x+\frac1{f(y)}.
$$
Odtiaľ
$$
y+\frac1{f(x)}=x+\frac1{f(y)}
$$
pre všetky kladné $x$, $y$. Dosadením $y=1$ dostaneme
$$
\frac1{f(x)}=x+\frac1{f(1)}-1=x+c,\qquad\text{teda}\qquad f(x)=\frac1{x+c},
$$
kde $c$ je konštanta. Keďže $f(x)>0$ pre každé $x>0$, musí byť $x+c>0$ pre každé $x>0$, čiže $c\ge 0$.

Ľahko overíme, že každá funkcia $f(x)=1/(x+c)$, kde $c\ge 0$, vyhovuje:
$$
\left(1+\frac y{x+c}\right)\left(1-\frac y{x+y+c}\right)=\frac{x+c+y}{x+c}\cdot\frac{x+y+c-y}{x+y+c}=1.
$$

\ineriesenie
Dosaďme do zadanej podmienky $x=1$ a~$f(1)=a>0$. Úpravou dostávame
$$
\align
(1+ay)\bigl(1-yf(y+1)\bigr)&=1,\\
ay-yf(y+1)(1+ay)&=0,\\
f(y+1)&=\frac{a}{1+ay}.
\endalign
$$
Keď teraz opäť do zadanej podmienky dosadíme $y=1$ a~$f(x+1)=a/(1+ax)$, máme
$$
\align
\bigl(1+f(x)\bigr)\left(1-\frac{a}{1+ax}\right)&=1,\\
f(x)\cdot\frac{1+ax-a}{1+ax}&=\frac{a}{1+ax},\\
f(x)&=\frac{a}{1+ax-a}=\frac{1}{x+\frac1a-1}=\frac{1}{x+c}.
\endalign
$$
Podobne ako v~prvom riešení musí byť $c\ge0$ a~ľahko overíme, že každá taká funkcia vyhovuje.

\poznamka
Riešenie možno zapísať aj v~tvare
$$
f(x)=\frac{a}{1+(x-1)a},
$$
kde $a=f(1)\in(0,1\rangle$ je reálny parameter. Dá sa očakávať, že viaceré súťažné riešenia budú zapísané práve takto.
}

{%%%%%   trojstretnutie, priklad 2
Postupnosť $(a_n)_{n=1}^\infty$ je evidentne rastúca až po prvý člen, v ktorého zápise sa vyskytne cifra~0 a~počnúc týmto členom je konštantná. Naším cieľom je teda nájsť také hodnoty $a$, $k$, že cifra 0 sa po prvý raz vyskytne v~člene $a_{2009}$. Úlohu vyriešime všeobecnejšie -- uvedieme také hodnoty $a$, $k$, že cifra 0 sa po prvý raz vyskytne v~člene $a_m$, pričom $m>4$ je dané celé číslo.

Zoberme
$$
a=\frac{10^{2m-5}-1}9=\underbrace{11\dots1}_{\text{$2m-5$ jednotiek}},
\qquad
k=10^{m-3}+4=1\underbrace{00\dots0}_{\text{$m-4$ núl}}4.
$$
Postupne máme
$$
\align
a_1&=a=\underbrace{11\dots1}_{2m-5},\\ \varrho(a_1)&=1,\\
a_2&=a_1+k=a_1+1\underbrace{00\dots0}_{m-4}4=
         \underbrace{11\dots1}_{m-3}2\underbrace{11\dots1}_{m-4}5,
         \\ \varrho(a_2)&=10,\\
a_3&=a_2+10k=a_2+1\underbrace{00\dots0}_{m-4}40=
             \underbrace{11\dots1}_{m-4}22\underbrace{11\dots1}_{m-5}55,
    \\ \varrho(a_3)&=100,\\
&~\,\vdots \\
a_i&=a_{i-1}+10^{i-2}k=
    \underbrace{11\dots1}_{m-i-1}\underbrace{22\dots2}_{i-1}\underbrace{11\dots1}_{m-i-2}\underbrace{55\dots5}_{i-1},
    \\ \varrho(a_i)&=10^{i-1},\\
&~\,\vdots \\
a_{m-2}&=a_{m-3}+10^{m-4}k=1\underbrace{22\dots2}_{m-3}\underbrace{55\dots5}_{m-3},
        \\ \varrho(a_{m-2})&=10^{m-3},\\
a_{m-1}&=a_{m-2}+10^{m-3}k=
         a_{m-2}+1\underbrace{00\dots0}_{m-4}4\underbrace{00\dots0}_{m-3}=
         \underbrace{22\dots2}_{m-3}6\underbrace{55\dots5}_{m-3},
        \\ \varrho(a_{m-1})&=6\cdot10^{m-3},\\
a_m&=a_{m-1}+6\cdot10^{m-3}k=
     a_{m-1}+6\underbrace{00\dots0}_{m-5}24\underbrace{00\dots0}_{m-3}=
     8\underbrace{22\dots2}_{m-5}50\underbrace{55\dots5}_{m-3},
     \\ \varrho(a_m)&=0.
\endalign
$$

\zaver
Postupnosť obsahuje práve 2009 rôznych čísel napríklad pre $a=\frac19(10^{4013}-1)$, $k=10^{2006}+4$.

\ineriesenie
Zvoľme
$$
a=6\underbrace{11\dots1}_{2007},\qquad
k=\underbrace{33\dots3}_{2007}4=\frac16\cdot2\underbrace{00\dots0}_{2007}4.
$$
Potom
$$
\alignat3
a_1&=6\underbrace{11\dots1}_{2007},       & \varrho(a_1)&=6,    & k\varrho(a_1)&=2\underbrace{00\dots0}_{2007}4,\\
a_2&=26\underbrace{11\dots1}_{2006}5,     & \varrho(a_2)&=60,   & k\varrho(a_2)&=2\underbrace{00\dots0}_{2007}40,\\
a_3&=226\underbrace{11\dots1}_{2005}55,   & \varrho(a_3)&=600,  & k\varrho(a_3)&=2\underbrace{00\dots0}_{2007}400, \\
a_4&=2226\underbrace{11\dots1}_{2004}555, & \varrho(a_4)&=6000, & k\varrho(a_4)&=2\underbrace{00\dots0}_{2007}4000, \\
&~\,\vdots &&&&\\
a_{i+1}&=\underbrace{22\dots2}_i6\underbrace{11\dots1}_{2007-i}\underbrace{55\dots5}_i,\quad &
  \varrho(a_{i+1})&=6\underbrace{0\dots0}_i, \quad&
  k\varrho(a_{i+1})&=2\underbrace{00\dots0}_{2007}4\underbrace{00\dots0}_i,\\
&~\,\vdots &&&&\\
a_{2007}&=\underbrace{22\dots2}_{2006}61\underbrace{55\dots5}_{2006}, &
  \varrho(a_{2007})&=6\underbrace{0\dots0}_{2006}, &
  k\varrho(a_{2007})&=2\underbrace{00\dots0}_{2007}4\underbrace{00\dots0}_{2006},\\
a_{2008}&=\underbrace{22\dots2}_{2007}6\underbrace{55\dots5}_{2007}, &
  \varrho(a_{2008})&=6\underbrace{0\dots0}_{2007}, &
  k\varrho(a_{2008})&=2\underbrace{00\dots0}_{2007}4\underbrace{00\dots0}_{2007},\\
a_{2009}&=\underbrace{22\dots2}_{2007}30\underbrace{55\dots5}_{2007}, &
  \varrho(a_{2009})&=0, & k\varrho(a_{2009})&=0
\endalignat
$$
a ďalej samozrejme $a_{2009}=a_{2010}=a_{2011}=\dots$

\poznamka
Na vyriešenie úlohy stačí nájsť také hodnoty $a$, $k$, aby postupnosť obsahovala {\it aspoň\/} 2009 rôznych čísel a~zároveň neobsahovala nekonečne veľa rôznych čísel. Ak totiž uvedená postupnosť obsahuje práve $m$ rôznych čísel, pričom $m>2009$, tak postupnosť s~prvým členom $a_{m-2008}$ bude obsahovať želaných 2009 rôznych čísel.
}

{%%%%%   trojstretnutie, priklad 3
Nech $T_k$ je bod, v~ktorom sa kružnica~$k$ dotýka strany~$BC$ a~$T_l$ je bod, v~ktorom sa kružnica~$l$ dotýka strany~$DE$. Ukážeme, že hľadaným pevným bodom je bod $T_k$.

Najskôr dokážeme, že body $T_k$, $T_l$ a~$P$ sú kolineárne. Označme body dotyku kružnice~$k$ s~priamkami $EP$, $DP$ postupne $U$, $V$, priesečníky strany~$BC$ s~týmito priamkami postupne $M$, $N$ a~body dotyku kružnice~$k$ s~polpriamkami $AB$, $AC$ postupne $T_1$, $T_2$.
\insp{cps.1}%

Keďže $BC\parallel DE$, sú trojuholníky $DEP$ a~$NMP$ podobné a~rovnoľahlosť $\Cal H$ so stredom $P$ a~koeficientom $q=|MN|/|ED|$ zobrazí úsečku $DE$ na úsečku $NM$. Na kolineárnosť bodov $T_k$, $T_l$, $P$ stačí dokázať rovnosť
$$
\frac{|MT_k|}{|NT_k|}=\frac{|ET_l|}{|DT_l|};
\tag1
$$
ak je totiž splnená, zobrazí sa v~rovnoľahlosti $\Cal H$ bod $T_l$ do bodu $T_k$.

Označme $a$, $b$, $c$ dĺžky strán trojuholníka $DEP$ tak ako na \obr. Ďalej nech $|AD|=e$, $|AE|=d$. Pripomeňme známe vzťahy pre dĺžku úseku medzi vrcholom trojuholníka a dotykovým bodom vpísanej, resp. pripísanej kružnice: V~trojuholníku $XYZ$ je vzdialenosť vrcholu~$X$ od dotykového bodu vpísanej, resp. pripísanej kružnice (ležiaceho na strane~$XY$) rovná $(|XY|+|XZ|-|YZ|)/2$, resp. $(|XY|+|YZ|-|XZ|)/2$.

Kružnica~$k$ je pripísanou kružnicou k~strane~$NM$ trojuholníka $NMP$. Preto
$$
\frac{|MT_k|}{|NT_k|}=\frac{(|MN|+|NP|-|MP|)/2}{(|MN|+|MP|-|NP|)/2}=\frac{qa+qc-qb}{qa+qb-qc}=\frac{a+c-b}{a+b-c}.
\tag2
$$
Kružnica~$l$ je vpísanou kružnicou do trojuholníka $DEA$. Preto
$$
\frac{|ET_l|}{|DT_l|}=\frac{(|DE|+|AE|-|AD|)/2}{(|DE|+|AD|-|AE|)/2}=\frac{a+d-e}{a+e-d}.
\tag3
$$
Ak z~nejakého bodu vedieme ku kružnici dve dotyčnice, vzdialenosť oboch dotykových bodov od daného bodu je rovnaká. Opakovaným použitím tohto faktu dostávame
$$
e+c+|PU|=e+c+|PV|=e+|DT_1|=|AT_1|=|AT_2|=d+|ET_2|=d+b+|PU|,
$$
čiže $e+c=d+b$. Preto $c-b=d-e$ a~dosadením do \thetag2, \thetag3 okamžite dostávame
$$
\frac{|MT_k|}{|NT_k|}=\frac{a+(c-b)}{a-(c-b)}=\frac{a+(d-e)}{a-(d-e)}=\frac{|ET_l|}{|DT_l|},
$$
čo je požadovaná rovnosť \thetag1. Bod $P$ teda leží na priamke $T_lT_k$.

\smallskip
Podobne dokážeme, že aj body $T_k$, $T_l$ a~$Q$ sú kolineárne. Označme body dotyku kružnice~$l$ s~priamkami $CQ$, $BQ$ postupne $U'$, $V'$, priesečníky strany~$DE$ s~týmito priamkami postupne $M'$, $N'$ a~body dotyku kružnice~$l$ so stranami $AD$, $AE$ postupne $T_1'$, $T_2'$. Ďalej nech $a'$, $b'$, $c'$ sú dĺžky strán trojuholníka $BCQ$, $|AB|=e'$, $|AC|=d'$.
\insp{cps.2}%

Analogickými úvahami ako v~prvej časti dostávame (tentoraz sú obe kružnice pripísané)
$$
\frac{|M'T_l|}{|N'T_l|}=\frac{a'+c'-b'}{a'+b'-c'},\qquad
\frac{|CT_k|}{|BT_k|}=\frac{a'+e'-d'}{a'+d'-e'}.
$$
Porovnávaním dĺžok (\obr) máme
$$
e'-c'-|QU'|=e'-c'-|QV'|=e'-|BT_1'|=|AT_1'|=|AT_2'|=d'-|CT_2'|=d'-b'-|QU'|,
$$
čiže $c'-b'=e'-d'$ a~následne
$$
\frac{|M'T_l|}{|N'T_l|}=\frac{|CT_k|}{|BT_k|}.
$$
Z~rovnoľahlosti trojuholníkov $BCQ$ a~$N'M'Q$ napokon dostávame, že bod~$Q$ leží na priamke $T_lT_k$.

\smallskip
Priamka~$PQ$ (zrejme $P\ne Q$) je teda totožná s~priamkou~$T_lT_k$ a~prechádza bodom~$T_k$, ktorý je nezávislý od polohy priamky~$p$.
}

{%%%%%   trojstretnutie, priklad 4
V~trojuholníku $ABC$ označme $S$ stred strany~$AB$ a~$P$, $Q$ päty výšok z~vrcholov $A$, $B$. Stred úsečky~$KL$ označme $M$. Body $P$, $Q$ sú samozrejme stredmi úsečiek $AK$, $BL$ (\obr). Takže $QS$ je strednou priečkou trojuholníka $LAB$ a~$MP$ strednou priečkou trojuholníka $LAK$. Odtiaľ
$$
|QS|=\tfrac12|LA|=|MP|\qquad\text{a}\qquad QS\parallel LA\parallel MP,
$$
čiže $SPMQ$ je rovnobežník (to zrejme platí aj v~prípade, keď niektorý z~trojuholníkov $LAB$, $LAK$ je "degenerovaný"). \insp{cps.3}%

Body $P$, $Q$ ležia na Tálesovej kružnici nad priemerom~$AB$, preto $|SP|=|SQ|=\frac12|AB|$. Rovnobežník $SPMQ$ je teda kosoštvorec a~dĺžka jeho strany nezávisí od polohy bodu~$C$. Aby sme dokázali, že ani dĺžka jeho uhlopriečky~$SM$ nezávisí od polohy bodu~$C$, stačí dokázať, že veľkosť uhla, ktorý zvierajú jeho strany $SP$, $SQ$, je pre ľubovoľnú polohu bodu~$C$ na kružnici~$k$ rovnaká (všetky možné kosoštvorce $SPMQ$, a~teda aj ich uhlopriečky $SM$, sú potom navzájom zhodné).
\inspnspab{cps.4}{cps.5}{\qquad}%

Ak je uhol~$\alpha$ trojuholníka $ABC$ ostrý, leží bod $Q$ vnútri strany~$AC$ (uhol $\gamma$ je podľa zadania ostrý vždy) a~uhol $PSQ$ je stredovým uhlom k~obvodovému uhlu $PAQ$ nad tetivou~$PQ$ Tálesovej kružnice nad priemerom~$AB$ (\obr{}a). Takže
$$
|\uhol PSQ|=2|\uhol PAQ|=2(90^\circ-\gamma)=180^\circ-2\gamma.
$$
Rovnaké vyjadrenie dostaneme aj v~prípade, že uhol $\alpha$ nie je ostrý, vtedy je totiž ostrý uhol $\beta$ a~môžeme namiesto uhla $PAQ$ použiť obvodový uhol $PBQ$ (\obrr1b):
$$
|\uhol PSQ|=2|\uhol PBQ|=2(90^\circ-\gamma)=180^\circ-2\gamma.
$$
Keďže pri pohybe bodu~$C$ po kružnici~$k$ sa veľkosť uhla~$\gamma$ nemení (je to obvodový uhol nad pevnou tetivou~$AB$), nemení sa ani veľkosť uhla $PSQ$, čo sme chceli dokázať.

\ineriesenie
Označme $\alpha$, $\beta$, $\gamma$ veľkosti vnútorných uhlov trojuholníka $ABC$. Budeme predpokladať, že uhol $\alpha$ je ostrý; prípad, keď $\alpha$ nie je ostrý je analogický (vtedy je $\beta$ ostrý). Nech $S$, $M$, $U$, $V$ sú postupne stredy úsečiek $AB$, $KL$, $AL$, $BK$ a~$H$ je priesečník priamok $AK$ a~$BL$ (\obr{}a,\,b).
\inspnspab{cps.6}{cps.7}{\ }%
Štvoruholník $USV\!M$ je rovnobežník ($SU \parallel BL \parallel MV$, $SV \parallel AK \parallel MU$). Zrejme
$$
|SV| = |AB|\sin\beta, \ |MV|=|SU|=|AB|\sin\alpha, \ |\angle SVM| = |\angle AHL| = |\angle ACB| = \gamma.
$$
Použitím sínusovej vety v~trojuholníku $ABC$ máme
$$
\frac{|AC|}{|SV|}=\frac{1}{|AB|}\cdot\frac{|AC|}{\sin\beta}=\frac{1}{|AB|}\cdot\frac{|BC|}{\sin\alpha}=\frac{|BC|}{|MV|},
$$
teda trojuholníky $SVM$ a~$ACB$ sú podobné (strany zvierajúce rovnaký uhol majú dĺžky v~rovnakom pomere). Keď opäť použijeme sínusovú vetu v~trojuholníku $ABC$, dostaneme
$$
|SM|=|AB|\cdot\frac{|SV|}{|AC|}=\frac{|AB|^2\sin\beta}{|AC|}=\frac{|AB|^2\sin\gamma}{|AB|}=|AB|\sin\gamma,
$$
čo je výraz, ktorý zrejme nezávisí od polohy bodu~$C$.
}

{%%%%%   trojstretnutie, priklad 5
Najskôr dokážeme, že čísla $a_1$, \dots, $a_n$ sú navzájom nesúdeliteľné. Ak by to tak nebolo, mali by sme $(a_i,a_j)=d>1$ pre nejaké $i\ne j$. Nech $a_i=u\cdot d$, $a_j=v\cdot d$. Zvoľme $b_i=1$, $b_j=2$. Podľa podmienky ($ii$) existujú $m$, $c_i$ a~$c_j$ také, že
$$
m\cdot b_i=c_i^{a_i}\quad\text{a}\quad m\cdot b_j=c_j^{a_j},
\qquad\text{teda}\qquad
m=(c_i^u)^d\quad\text{a}\quad 2m=(c_j^v)^d.
$$
Odtiaľ $2(c_i^u)^d=(c_j^v)^d$, čo nie je možné, nakoľko exponent prvočísla~$2$ v~prvočíselnom rozklade pravej strany je násobkom čísla~$d$ a v~prvočíselnom rozklade ľavej strany nie je násobkom čísla~$d$.

Predpokladajme, že čísla $a_1$, \dots, $a_n$ sú navzájom nesúdeliteľné. Ukážeme, že potom je podmienka ($ii$) splnená. Nech $b_1$, \dots, $b_n$ je ľubovoľná $n$-tica prirodzených čísel a~$p_1$, \dots, $p_k$ sú všetky prvočísla nachádzajúce sa v~prvočíselných rozkladoch čísel $b_1$, \dots, $b_n$. Hľadajme $m$ v~tvare
$$
m=p_1^{\alpha_1}\cdot\dots\cdot p_k^{\alpha_k}.
$$
Pre každé $i=1,\dots,n$ označme $\beta_{i,j}$ exponent prvočísla~$p_j$ v~prvočíselnom rozklade $b_i$. Aby číslo $m\cdot b_i$ bolo $a_i$-tou mocninou, stačí, aby pre každé $j=1,\dots,k$ bolo $\alpha_j+\beta_{i,j}$ násobkom $a_i$. Každú hodnotu $\alpha_j$ teda stačí zvoliť tak, aby platilo
$$
\alpha_j\equiv-\beta_{1,j}\pmod{a_1},\quad
\alpha_j\equiv-\beta_{2,j}\pmod{a_2},\quad\dots,\quad
\alpha_j\equiv-\beta_{n,j}\pmod{a_n}.
$$
Existencia takého $\alpha_j$ vyplýva z~čínskej zvyškovej vety (keďže $a_1$, \dots, $a_n$ sú navzájom nesúdeliteľné).

Dokázali sme, že podmienka ($ii$) je splnená práve vtedy, keď sú čísla $a_1$, \dots, $a_n$ navzájom nesúdeliteľné. Medzi číslami $1$, $2$, \dots, $50$ je práve pätnásť prvočísel. Ak by bolo $n\ge17$, medzi číslami $2\le a_2<a_3<\dots<a_n\le50$ by určite existovali aspoň dve čísla majúce v~prvočíselnom rozklade rovnaké prvočíslo, teda by boli súdeliteľné. Preto nutne $n\le16$.

Ak $n=16$, musí byť $a_1=1$ a~každé z~pätnástich čísel $a_2$, $a_3$, \dots, $a_{16}$ musí byť mocninou iného prvočísla. Vypíšme, ktoré mocniny prvočísel môžeme použiť:
$$
\alignat2
p&=2\colon&\qquad&2,4,8,16,32,\\
p&=3\colon&\qquad&3,9,27,\\
p&=5\colon&\qquad&5,25,\\
p&=7\colon&\qquad&7,49,\\
p&\ge11\colon&\qquad&\text{iba $p$}.
\endalignat
$$
Celkový počet vyhovujúcich šestnástic je teda $5\cdot3\cdot2\cdot2=60$.
}

{%%%%%   trojstretnutie, priklad 6
Označme $|A|=m\ge 4n\sqrt{n}$. Nech $\Cal S$ je množina všetkých úsečiek majúcich krajné body v~$A$. Zrejme $|\Cal S|=\binom m2$. Súradnice stredu každej úsečky z~$\Cal S$ sú celé násobky čísla $\frac12$ a~ležia v~konvexnom obale množiny~$G$. Takých bodov je $(2n-1)^2$, teda menej ako $4n^2$. Preto existuje bod $B$, ktorý je stredom aspoň $\binom m2/(4n^2)$ úsečiek z~$\Cal S$. Nech $\Cal P$ je množina všetkých úsečiek z~$\Cal S$, ktorých stredom je $B$. Potom
$$
|\Cal P|\ge \frac{\binom m2}{4n^2}=\frac{m(m-1)}{8n^2}\ge\frac{4n\sqrt{n}(4n\sqrt{n}-1)}{8n^2}=
\frac{16n^3-4n\sqrt{n}}{8n^2}=2n-\frac{1}{2\sqrt n}>2n-1,
$$
takže $|\Cal P|\ge2n$.

Rozdeľme $\Cal P$ na triedy úsečiek ležiacich na jednej priamke. Označme počet týchto tried $k$ a~počet úsečiek v~$i$-tej triede $a_i$ pre $i=1,\dots,k$. Každá úsečka spomedzi $a_i$ úsečiek jednej triedy má krajné body v~$G$ a~všetkých $2a_i$ krajných bodov (ktoré sú zrejme rôzne) úsečiek jednej triedy leží na jednej priamke, preto $2a_i\le n$. Pritom každé dve úsečky z~$\Cal P$ sú uhlopriečkami rovnobežníka práve vtedy, keď neležia na jednej priamke. Pre počet rôznych rovnobežníkov s~uhlopriečkami patriacimi do $\Cal P$ tak dostávame
$$
\align
\sum_{1\le i<j\le k} a_i a_j&=\frac12\left(\left(\sum_{i=1}^k a_i\right)^2-\sum_{i=1}^k a_i^2\right)\ge
                              \frac12\left(\left(\sum_{i=1}^k a_i\right)^2-\sum_{i=1}^k a_i\cdot \frac n2\right)=\\
  &=\frac12\left(|\Cal P|^2-|\Cal P|\cdot\frac n2\right)=
    \frac12|\Cal P|\left(|\Cal P|-\frac n2\right)\ge n\left(2n-\frac n2\right)=\frac32n^2>n^2.
\endalign
$$
Existuje teda viac ako $n^2$ konvexných štvoruholníkov (rovnobežníkov) s~požadovanou vlastnosťou.
}

{%%%%%   IMO, priklad 1
\podla{Michala Hagaru}
Podmienku $n\mid a_i(a_{i+1}-1)$ prepíšeme v~tvare kongruencie a~upravíme:
$$
\align
a_i(a_{i+1}-1) &\equiv 0 \pmod n,\\
a_ia_{i+1}-a_i &\equiv 0 \pmod n,\\
a_ia_{i+1}     &\equiv a_i \pmod n,\qquad i=1,2,\dots,k-1. \tag{1}
\endalign
$$
Využitím \thetag1 postupne pre $i=1,2,\dots,k-1$ dostávame
$$
a_1 \equiv a_1a_2 \equiv a_1a_2a_3 \equiv a_1a_2a_3a_4 \equiv \cdots \equiv a_1a_2\dots a_k \pmod n.
\tag2
$$
Predpokladajme sporom, že $n\mid a_k(a_1-1)$, teda že $a_ka_1\equiv a_k\pmod n$. Potom
$$
a_1a_2\dots a_k=a_2\dots a_ka_1\equiv a_2\dots a_k\pmod n,
$$
čo v~spojení s~\thetag2 dáva
$$
a_1\equiv a_2\dots a_k\pmod n.
\tag3
$$
Rovnako ako pri \thetag2, využitím \thetag1 postupne pre $i=2,\dots,k-1$ máme
$$
a_2 \equiv a_2a_3 \equiv a_2a_3a_4 \equiv \cdots \equiv a_2\dots a_k \pmod n.
$$
Spolu s~\thetag3 odtiaľ
$$
a_1\equiv a_2\dots a_k\equiv a_2\pmod n,
$$
teda $a_1$ a~$a_2$ dávajú rovnaký zvyšok po delení $n$, čo je v~spore s~predpokladom, že sú to navzájom rôzne čísla z~množiny $\{1,\dots,n\}$.
}

{%%%%%   IMO, priklad 2
\podla{Michala Hagaru}
Úsečka~$MK$ je strednou priečkou trojuholníka $QBP$, preto
$$
|KM|=\frac12|QB|\qquad\text{a}\qquad AB\parallel MK.
$$
Zo striedavých uhlov potom $|\uhol AQP|=|\uhol KMQ|$. Keďže $PQ$ je dotyčnicou kružnice~$\Gamma$, z~rovnosti úsekového a~obvodového uhla prislúchajúceho tetive $MK$ máme $|\uhol KMQ|=|\uhol KLM|$. Spolu teda $|\uhol AQP|=|\uhol KLM|$. Analogicky dostaneme (\obr)
$$
|LM|=\frac12|PC|\qquad\text{a}\qquad |\uhol APQ|=|\uhol LMP|=|\uhol LKM|.
$$
\insp{mmo.1}%

Z~uvedených rovností uhlov vyplýva podobnosť trojuholníkov $QAP$ a~$LMK$ (majú rovnaké veľkosti prislúchajúcich vnútorných uhlov). Z~pomerov strán postupne (po dosadení vyjadrených dĺžok $|KM|$ a~$|LM|$) dostávame
$$
\aligned
\frac{|QA|}{|LM|}&=\frac{|PA|}{|KM|},\\
\frac{|QA|}{\frac12|PC|}&=\frac{|PA|}{\frac12|QB|},\\
|QA|\cdot|QB|&=|PA|\cdot|PC|.
\endaligned
$$
Posledná rovnosť znamená, že body $Q$ a~$P$ majú rovnakú mocnosť ku kružnici opísanej trojuholníku $ABC$ (zrejme oba body ležia vnútri tejto kružnice). Ak označíme $r$ jej polomer, zhodnú mocnosť možno zapísať v~tvare
$$
|OQ|^2-r^2=|OP|^2-r^2,
$$
odkiaľ už triviálne $|OP|=|OQ|$.
}

{%%%%%   IMO, priklad 3
Označme $D$ diferenciu aritmetickej postupnosti $s_{s_1}$, $s_{s_2}$, $s_{s_3}$, \dots{} a~pre každé $n=1,2,\dots$ označme $d_n=s_{n+1}-s_n$. Naším cieľom je dokázať, že hodnota~$d_n$ je pre všetky~$n$ rovnaká. Najskôr ukážeme, že množina hodnôt~$d_n$ je ohraničená. Keďže zadaná postupnosť je rastúca, pre každé~$n$ je $d_n\ge1$. Preto\footnote{Ohraničenosť hodnôt~$d_n$ sa dá zdôvodniť aj menej formálne: Každé dva po sebe idúce členy postupnosti $(s_n)_{n\in\Bbb N}$ možno zrejme vložiť medzi niektoré dva členy postupnosti $1$, $s_{s_1}$, $s_{s_2}$, $s_{s_3}$,~\dots, ktorá je od druhého člena aritmetická. Preto triviálne $s_{n+1}-s_n\le\max\{D,s_{s_1}-1\}$.}
$$
d_n=s_{n+1}-s_n=\underbrace{1+1+\cdots+1}_{\text{$(s_{n+1}-s_n)$-krát}}\le d_{s_n}+d_{s_n+1}+\cdots+d_{s_{n+1}-1}=D.
$$

Označme $m$ najmenšiu a~$M$ najväčšiu z~hodnôt~$d_n$ (z~ohraničenosti vyplýva existencia minima aj maxima). Stačí dokázať, že $m=M$. Predpokladajme sporom, že $m<M$.

Nech $k$ je ľubovoľný taký index, že $d_k=m$. Teda $s_{k+1}-s_k=m$, odkiaľ
$$
\align
D&=s_{s_{k+1}}-s_{s_k}=s_{s_k+m}-s_{s_k}=\\
&=d_{s_k}+d_{s_k+1}+\cdots+d_{s_k+m-1}\le \underbrace{M+M+\cdots+M}_{\text{$m$-krát}} = mM.
\tag1
\endalign
$$
Podobne ak $K$ je ľubovoľný taký index, že $d_K=M$, tak $s_{K+1}-s_K=M$, teda
$$
\align
D&=s_{s_{K+1}}-s_{s_K}=s_{s_K+M}-s_{s_K}=\\
&=d_{s_K}+d_{s_K+1}+\cdots+d_{s_K+M-1}\ge \underbrace{m+m+\cdots+m}_{\text{$M$-krát}} = Mm.
\tag2
\endalign
$$

Z~\thetag1 a~\thetag2 vyplýva, že $D=Mm$, a~aby platila rovnosť, nutne $d_{s_k}=d_{s_k+1}=\dots=d_{s_k+m-1}=M$ a~$d_{s_K}=d_{s_K+1}=\dots=d_{s_K+M-1}=m$.
Špeciálne máme
$$
d_{s_k}=M\qquad\text{a}\qquad d_{s_K}=m.
\tag3
$$
Z~rastúcosti postupnosti $(s_n)_{n\in\Bbb N}$ vyplýva $s_k\ge k$. Navyše dokonca $s_k>k$, lebo ak by sme mali $s_k=k$, tak podľa \thetag3 by bolo $m=d_k=d_{s_k}=M$, čo je v~spore s~$m<M$. Rovnako možno ukázať, že $s_K>K$.

Položme $n_1=k$. Teda $d_{n_1}=m$ a podľa \thetag3 platí $d_{s_{n_1}}=M$. Ďalej zvoľme $n_2=s_{n_1}>n_1$. Keďže $d_{n_2}=M$, môže $n_2$ vystupovať v~pozícii~$K$ a~teda podľa \thetag3 máme $d_{s_{n_2}}=m$ a~taktiež $s_{n_2}>n_2$. Následne preto môžeme zvoliť $n_3=s_{n_2}$ a~z~\thetag3 (keďže $n_3$ môže vystupovať v~pozícii~$k$) opäť $d_{s_{n_3}}=M$, atď. Predpisom $n_{i+1}=s_{n_i}$ takto skonštruujeme rastúcu postupnosť $n_1$, $n_2$, $n_3$, \dots{} takú, že
$$
d_{s_{n_1}}=M,\quad d_{s_{n_2}}=m,\quad d_{s_{n_3}}=M,\quad d_{s_{n_4}}=m,\quad  \dots
\tag4
$$
Pritom postupnosť $d_{s_{n_1}}$, $d_{s_{n_2}}$, \dots{} je podpostupnosťou postupnosti $d_{s_1}$, $d_{s_2}$,~\dots{} Tá má členy, ktoré sú rozdielmi členov aritmetických postupností $s_{s_1+1}$, $s_{s_2+1}$,~\dots{} a~$s_{s_1}$, $s_{s_2}$,~\dots, teda aj sama je aritmetickou postupnosťou. Keďže podľa \thetag4 sa v~nej nekonečne veľakrát opakuje hodnota $m$ (aj $M$), nutne to musí byť konštantná aritmetická postupnosť a~$m=M$.
}

{%%%%%   IMO, priklad 4
\podla{Martina Bachratého}
Označme $I$ stred kružnice vpísanej do trojuholníka $ABC$ (je to priesečník priamok $AD$, $BE$, $CK$). Uvažujme kružnice $k_1$, $k_2$ opísané trojuholníkom $IKE$, $IKD$ (\obr). Obe majú nad spoločnou tetivou~$IK$ obvodový uhol veľkosti $45\st$, lebo $K$ leží na osi pravého uhla $ADC$. Preto sú obe kružnice zhodné a~teda osovo súmerné podľa priamky~$CI$. Podľa tejto priamky sú však osovo súmerné aj priamky $AC$, $BC$. Takže v~osovej súmernosti podľa priamky~$CI$ sa priesečníky kružnice~$k_1$ so stranou~$AC$ zobrazia na priesečníky kružnice~$k_2$ so stranou~$BC$.
\insp{mmo.2}%

Keďže jeden zo spoločných bodov kružnice~$k_1$ a~priamky~$AC$ je bod~$E$ a~jeden zo spoločných bodov kružnice~$k_2$ a~priamky~$BC$ je bod~$D$, môžu nastať dva prípady.

\pripad1
Bod~$D$ je obrazom bodu~$E$ v~osovej súmernosti podľa $CI$ (Tento prípad zahŕňa aj možnosť, že kružnice $k_1$, $k_2$ sa dotýkajú priamok $AC$, $BC$ -- vtedy sú $E$ a~$D$ jediné "priesečníky" kružníc s~priamkami a~nutne musia byť navzájom súmerné). Potom sú osovo súmerné podľa $CI$ celé trojuholníky $CID$ a~$CIE$, z~ktorých prvý je pravouhlý. Nutne teda aj uhol $CEI$ je pravý, čo znamená, že v~trojuholníku $ABC$ je os uhla pri vrchole~$B$ totožná s~výškou na stranu~$AC$. To je zrejme možné jedine v~prípade, že trojuholník $ABC$ je rovnoramenný so základňou~$AC$. Avšak podľa zadania je rovnoramenným so základňou~$AB$. Takže musí byť rovnostranný, z~čoho $|\uhol CAB|=60\st$.

\pripad2
Kružnica~$k_2$ pretína priamku~$BC$ v~dvoch rôznych bodoch $D$ a~$E'$, pričom $E'$ a~$E$ sú súmerné podľa priamky~$CI$. Keďže $E'$ leží na $BC$, uhol $IDE'$ je pravý a~teda $IE'$ je priemerom Tálesovej kružnice prechádzajúcej bodom~$D$. Táto kružnica je evidentne totožná s~$k_2$, preto aj uhol $IKE'$ je pravý (\obr).
\insp{mmo.3}%
Vzhľadom na spomenutú súmernosť je potom pravý aj uhol $IKE$ a~tretí uhol v~trojuholníku $IKE$ musí mať veľkosť
$$
|\uhol EIK|=180\st-|\uhol IKE|-|\uhol IEK|=180\st-90\st-45\st=45\st.
$$
Ak označíme $\beta$ veľkosť vnútorného uhla v~rovnoramennom trojuholníku $ABC$ pri vrcholoch $B$ a~$C$, tak rovnoramenný trojuholník $BCI$ má pri základni $BC$ vnútorné uhly veľkosti $\frac12\beta$ a~oproti základni uhol veľkosti $180\st-|\uhol EIK|=180\st-45\st=135\st$. Z~rovnosti
$$
\tfrac12\beta+\tfrac12\beta+135\st=180\st
$$
už triviálne dopočítame $\beta=45\st$, čiže $|\uhol CAB|=180\st-2\beta=90\st$.

\smallskip
Teda jediné prípustné hodnoty pre veľkosť uhla $CAB$ sú $60\st$ a~$90\st$. Pre dokončenie úlohy ešte musíme dokázať, že pre rovnoramenný trojuholník so základňou~$BC$ a~uhlom $CAB$ veľkosti $60\st$ resp. $90\st$ je veľkosť uhla $BEK$ skutočne $45\st$. (Zatiaľ sme dokázali len jeden smer implikácie: ak $|\uhol BEK|=45\st$, tak $|\uhol CAB|\in\{60\st,90\st\}$. Potrebujeme dokázať aj opačný smer, \tj. že hodnoty $60\st$ a~$90\st$ sú v~zadanej situácii naozaj možné veľkosti uhla $CAB$.)

Každý z~oboch prípadov preveríme osobitne. Možných postupov je viacero, v~daných konfiguráciách je trojuholník $ABC$ (až na veľkosť strany~$BC$) jednoznačne daný, takže sa jedná o~štandardnú úlohu. Uvedieme postup bez použitia goniometrických funkcií alebo analytickej geometrie.

Ak $|\uhol CAB|=60\st$, tak $ABC$ je rovnostranný trojuholník (\obr{}a). Zo symetrie potom $|\uhol BEK|=|\uhol ADK|=45\st$ (keďže $K$ leží na osi pravého uhla $ADC$).
\inspnspab{mmo.4}{mmo.5}{\quad}%

Ak $|\uhol CAB|=90\st=\alpha$, tak $\beta=45\st$ a~ľahko vypočítame veľkosti
$$
\aligned
|\uhol EIK|&=|\uhol CBI|+|\uhol BCI|=\tfrac12\beta+\tfrac12\beta=\beta=45\st,\\
|\uhol AEI|&=180\st-\alpha-\tfrac12\beta=180\st-90\st-22{,}5\st=67{,}5\st,\\
|\uhol AIE|&=|\uhol BID|=180\st-90\st-\tfrac12\beta=90\st-22{,}5\st=67{,}5\st.
\endaligned
$$
Teda $AIE$ je rovnoramenný trojuholník so základňou~$IE$ a~body $I$, $E$ sú súmerne združené podľa priamky $AK$, ktorá je osou uhla $CAD$ (\obrr1b). Odtiaľ $|KI|=|KE|$, čiže $KIE$ je rovnoramenný trojuholník a~$|\uhol BEK|=|\uhol EIK|=45\st$.

\ineriesenie
Označme $|\uhol BAC|=\alpha$ a~$|\uhol ABC|=|\uhol BCA|=\beta=90\st-\frac12\alpha$. Bod $I$ nech je rovnako ako v~predošlom riešení stred kružnice vpísanej do trojuholníka $ABC$. Bod~$K$ leží na priesečníkoch osí uhlov trojuholníka $ADC$, preto
$$
|\uhol ECK|=|\uhol KCD|=\tfrac12\beta=45\st-\tfrac14\alpha\qquad\text{a}\qquad
|\uhol CDK|=|\uhol KDA|=45\st.
$$
Z~trojuholníka $DCI$ potom $|\uhol DIC|=45\st+\frac14\alpha$. Pomocou zadaného predpokladu $|\uhol BEK|=45\st$ následne z~trojuholníkov $BCE$ a~$KCE$ odvodíme
$$
\aligned
|\uhol KEC|&=180\st-\tfrac12\beta-\beta-45\st=135\st-\tfrac32\beta=135\st-\tfrac32(90\st-\tfrac12\alpha)=\tfrac34\alpha,\\
|\uhol IKE|&=\tfrac34\alpha+45\st-\tfrac14\alpha=45\st+\tfrac12\alpha.
\endaligned
$$
\insp{mmo.6}%

Zo sínusových viet v~trojuholníkoch $ICE$, $IKE$, $IDK$, $IDC$ teda máme (\obr)
$$
\alignat2
\frac{|IC|}{|IE|}&=\frac{\sin(45\st+\tfrac34\alpha)}{\sin(45\st-\tfrac14\alpha)},
&\qquad
\frac{|IE|}{|IK|}&=\frac{\sin(45\st+\tfrac12\alpha)}{\sin45\st},\\
\frac{|IK|}{|ID|}&=\frac{\sin45\st}{\sin(90\st-\tfrac14\alpha)},
&\qquad
\frac{|ID|}{|IC|}&=\frac{\sin(45\st-\tfrac14\alpha)}{\sin90\st}.
\endalignat
$$
Vynásobením uvedených štyroch rovností dostaneme
$$
1=\frac{\sin(45\st+\tfrac34\alpha)\sin(45\st+\tfrac12\alpha)}{\sin(90\st-\tfrac14\alpha)}.
$$
Použitím známych goniometrických identít
$$
\sin(90\st-x)=\cos x\qquad\text{a}\qquad\sin x\cdot\sin y=\frac12(\cos(x-y)-\cos(x+y))
$$
rovnicu upravíme na
$$
\aligned
\cos\tfrac14\alpha&=\frac12(\cos\tfrac14\alpha-\cos(90\st+\tfrac54\alpha)),\\
\frac12(\cos(90\st+\tfrac54\alpha)+\cos\tfrac14\alpha)&=0.
\endaligned
$$
Použitím ďalšej identity $\cos x+\cos y=2\cos\frac12(x+y)\cos\frac12(x-y)$ napokon získame
$$
\cos(45\st+\tfrac34\alpha)\cdot\cos(45\st+\tfrac12\alpha)=0.
$$
To znamená (keďže $0\st<\alpha<180\st$), že buď $45\st+\frac34\alpha=90\st$, \tj. $\alpha=60\st$, alebo $45\st+\tfrac12\alpha=90\st$, \tj. $\alpha=90\st$. Takže jediné prípustné hodnoty pre veľkosť uhla $CAB$ sú $60\st$ a~$90\st$. Na druhej strane, pri oboch týchto hodnotách platí $|\uhol BEK|=45\st$, čo možno ukázať rovnako ako v~prvom riešení.
}

{%%%%%   IMO, priklad 5
Trojuholníkovú nerovnosť pre trojicu čísel $x$, $y$, $z$ budeme používať aj v~tvare $|x-y|<z$ (táto nerovnosť zahŕňa $x<y+z$ a~zároveň $y<x+z$). Označme $f(1)=m$. Po dosadení $a=1$ dostávame, že čísla $1$, $f(b)$, $f({b+m-1})$ sú pre každé prirodzené~$b$ stranami trojuholníka, spĺňajú teda nerovnosť
$$
|f(b)-f(b+m-1)|<1,
$$
čo je vzhľadom na celočíselnosť hodnôt $f(b)$, $f(b+m-1)$ možné jedine v~prípade, že $f(b)=f(b+m-1)$.

Ak by bolo $m>1$, tak $f$ by bola periodická s~periódou $m-1$. V~takom prípade by (periodicky) nadobúdala iba hodnoty
$$
f(1),\ f(2),\ \dots,\ f(m-1).
$$
Ak označíme $H$ najväčšiu z~týchto hodnôt, po dosadení $a=2H$ dostávame, že čísla $2H$, $f(b)$, $f(b+f(2H)-1)$ spĺňajú trojuholníkovú nerovnosť
$$
2H<f(b)+f(b+f(2H)-1)\le H+H=2H,
$$
čo je očividne spor.

Nutne teda $m=1$, \tj. $f(1)=1$. Dosadením $b=1$ z~toho dostávame, že čísla $a$, $1$, $f(f(a))$ sú pre ľubovoľné prirodzené číslo~$a$ stranami trojuholníka, takže $|a-f(f(a))|<1$, čo opäť vzhľadom na celočíselnosť $a$ a $f(f(a))$ znamená
$$
f(f(a))=a\quad\text{pre všetky prirodzené čísla $a$.}
\tag1
$$

Označme $f(2)=k$. Zrejme $k\ne1$, lebo podľa \thetag1 máme $f(k)=f(f(2))=2$, zatiaľ čo $f(1)=1$. Z~\thetag1 dokonca vyplýva, že funkcia~$f$ je prostá, \tj. žiadnu hodnotu nenadobúda viac než raz. Ak totiž $f(x)=f(y)$, tak nutne $x=f(f(x))=f(f(y))=y$. Tento známy fakt (že z~\thetag1 vyplýva prostosť~$f$) budeme v~riešení využívať.

Položme $a=2$, $b=k$. Potom $2$, $2$ a~$f(k+k-1)=f(2k-1)$ sú stranami trojuholníka, čiže
$$
|f(2k-1)-2|<2,\qquad\text{a odtiaľ}\quad f(2k-1)\in\{1,2,3\}.
$$
Avšak $f$ už nadobúda hodnotu~$1$ v~bode~$1$ a~hodnotu~$2$ v~bode~$k$ a~zároveň $2k-1\ne1$ a~$2k-1\ne k$. Z~prostosti $f$ teda nutne vyplýva $f(2k-1)=3$.

Položme ďalej $a=2$, $b=2k-1$. Z~toho $2$, $3$ a~$f(2k-1+k-1)=f(3k-2)$ sú stranami trojuholníka a~analogicky dostávame
$$
|f(3k-2)-3|<2,\qquad\text{čiže}\quad f(3k-2)\in\{2,3,4\}.
$$
Keďže hodnoty $2$, $3$ už $f$ nadobúda v~bodoch $k$, $2k-1$ a~zároveň $3k-2\ne k$ a~$3k-2\ne2k-1$, nutne $f(3k-2)=4$.

Matematickou indukciou (ktorej prvý krok sme pre hodnoty $n=1,2,3$ práve urobili) ľahko dokážeme, že
$$
f(nk-(n-1))=n+1\qquad\text{pre všetky prirodzené čísla $n$}.
$$
Ak totiž toto tvrdenie platí pre $n$ aj pre $n-1$, po dosadení $a=2$, $b=nk-(n-1)$ dostávame trojicu strán $2$, $n+1$ a~$f(nk-(n-1)+k-1)=f((n+1)k-n)$, teda
$$
|f((n+1)k-n)-(n+1)|<2,\qquad\text{odkiaľ}\quad f((n+1)k-n)\in\{n,n+1,n+2\}.
$$
Hodnoty $n$ a~$n+1$ sú však podľa indukčného predpokladu už "obsadené" bodmi ${(n-1)k}-{(n-2)}$ a~$nk-(n-1)$ (ktoré sú oba rôzne od $(n+1)k-n$), takže jedinou možnosťou je $f((n+1)k-n)=n+2$ a~tvrdenie platí aj pre $n+1$. Tým je dôkaz indukciou ukončený.

Špeciálne potom tvrdenie platí aj pre $n=k-1$, teda
$$
f((k-1)k-(k-2))=k.
$$
Keďže sme už skôr ukázali, že $f(2)=k$, z~prostosti $f$ máme $(k-1)k-(k-2)=2$, z~čoho po triviálnej úprave $k(k-2)=0$. Samozrejme $k\ne0$, dostávame tak $k=2$ a~indukciou dokázané tvrdenie prechádza na tvar
$$
f(n+1)=n+1\qquad\text{pre všetky prirodzené čísla $n$}.
$$
Spolu s~tvrdením $f(1)=1$, ktoré sme dokázali na začiatku, dostávame, že jedinou vyhovujúcou funkciou môže byť identita $f(x)=x$. Ľahko overíme, že táto funkcia vyhovuje, lebo trojica čísel $a$, $b$, $a+b-1$ spĺňa všetky tri trojuholníkové nerovnosti triviálne (pre ľubovoľné $a,b\in\Bbb N$).
}

{%%%%%   IMO, priklad 6
Na začiatku si všimnime, že zo zadaného tvrdenia vyplýva jeho zovšeobecnenie: Predpoklad, že množina~$M$ obsahuje práve $n-1$ kladných celých čísel možno nahradiť podmienkou $|M\cap(0,s-a_{\min}\rangle|\le n-1$, pričom $a_{\min}$ je najmenšie spomedzi $a_1$, $a_2$, \dots, $a_n$. Tento fakt v~dôkaze použijeme.

Postupovať budeme matematickou indukciou vzhľadom na~$n$. Prípad $n=1$ je triviálny. V~druhom kroku indukcie predpokladajme, že $n>1$ a~že tvrdenie je pravdivé pre všetky prirodzené čísla menšie ako~$n$. Ďalej nech $a_1$, $a_2$, \dots, $a_n$, $s$, $M$ sú dané a~spĺňajú predpoklady dokazovaného tvrdenia. Bez ujmy na všeobecnosti predpokladajme, že $a_n<a_{n-1}<\cdots<a_2<a_1$. Označme
$$
T_k=\sum_{i=1}^k a_i\quad\text{pre $k=0,1,\dots,n$}.
$$
Zrejme $0=T_0<T_1<\cdots<T_n=s$. Najskôr dokážeme pomocné tvrdenie.

\smallskip\noindent
{\it Tvrdenie 1}.
Stačí ukázať, že pre nejaké $m\in\{1,2,\dots,n\}$ dokáže urobiť koník $m$~skokov, pričom nikdy nepristane na čísle z~$M$ a~navyše počas týchto $m$~skokov preskočí spolu aspoň ponad $m$ bodov z~$M$.

\smallskip\noindent
{\it Dôkaz}.
Keďže $|M|=n-1$, evidentne $m\ne n$. Položme $n'=n-m$. Potom $1\le n'<n$. Zvyšných $n'$~skokov bez pristátia na ktoromkoľvek zo zvyšných zakázaných nanajvýš $n'-1$ čísel z~$M$ dokáže koník urobiť vďaka indukčnému predpokladu (stačí posunúť začiatok z~čísla $0$ do čísla, kde sa koník nachádza po $m$~skokoch).\hfill$\square$

\smallskip
Číslo $k\in\{1,2,\dots,n\}$ budeme nazývať {\it slušné}, ak koník dokáže urobiť $k$~skokov s~dĺžkami $a_1$, $a_2$, \dots, $a_k$ (v~nejakom poradí) tak, že s~výnimkou posledného skoku nikdy nepristane na čísle z~$M$ (po $k$-tom skoku môže a~nemusí skončiť na čísle z~$M$).

Číslo~$1$ je očividne slušné. Preto existuje najväčšie číslo $k^*$ také, že všetky čísla $1$, $2$, \dots, $k^*$ sú slušné. Ak $k^*=n$, niet čo dokazovať. Zaoberajme sa ďalej len prípadom $k^*\le n-1$ (teda $k^*+1$ nie je slušné). Dokážeme ďalšie pomocné tvrdenie.

\smallskip\noindent
{\it Tvrdenie 2}.
Platí
$$
T_{k^*}\in M \qquad\text{a}\qquad |M\cap(0,T_{k^*})|\ge k^*.
$$

\smallskip\noindent
{\it Dôkaz}.
Ak $T_{k^*}\notin M$, postupnosť skokov, vďaka ktorej je číslo $k^*$ slušné, možno predĺžiť pridaním skoku dĺžky $a_{k^*+1}$, čo je v~spore s~tým, že $k^*+1$ nie je slušné. Takže nutne $T_{k^*}\in M$.

Ak $|M\cap(0,T_{k^*})|<k^*$, tak existuje $l\in\{1,2,\dots,k^*\}$ také, že $T_{k^*+1}-a_l\notin M$. Potom podľa indukčného predpokladu (s~hodnotou $k^*$ namiesto $n$) dokáže koník doskákať do čísla $T_{k^*+1}-a_l$ pomocou $k^*$~skokov s~dĺžkami z~množiny $\{a_1,a_2,\dots,{a_{k^*}+1}\}\setminus\{a_l\}$, pričom ani raz nepristane na čísle z~$M$. Takže aj $k^*+1$ je slušné, čo je spor.\hfill$\square$

\smallskip
Podľa práve dokázaného tvrdenia existuje najmenšie číslo $\tilde k\in\{1,2,\dots,k^*\}$ také, že
$$
T_{\tilde k}\in M\qquad\text{a}\qquad |M\cap(0,T_{\tilde k})|\ge\tilde k.
$$

\smallskip\noindent
{\it Tvrdenie 3}.
Stačí uvažovať prípad
$$
|M\cap(0,T_{\tilde k-1}\rangle|\le \tilde k-1.
\tag1
$$

\smallskip\noindent
{\it Dôkaz}.
Ak $\tilde k=1$, tak \thetag1 platí triviálne. Ďalej nech $\tilde k>1$. Ak $T_{\tilde k-1}\in M$, tak \thetag1 vyplýva z~minimálnosti $\tilde k$. Ak $T_{\tilde k-1}\notin M$ a~platilo by $|M\cap(0,T_{\tilde k-1}\rangle|\ge \tilde k-1$, zo slušnosti čísla $\tilde k-1$ by sme dostali situáciu ako v~Tvrdení~1 pre $m=\tilde k-1$. V~tomto prípade teda dokonca stačí uvažovať prípad $|M\cap(0,T_{\tilde k-1}\rangle|\le \tilde k-2$.\hfill$\square$

\smallskip
Označme $v\ge0$ číslo, pre ktoré $|M\cap(0,T_{\tilde k})|=\tilde k+v$. Nech $r_1>r_2>\cdots>r_p$ sú všetky také indexy $r\in\{\tilde k+1,\tilde k+2,\dots,n\}$, že $T_{\tilde k}+a_r\notin M$. Potom
$$
  n-1=|M|=|M\cap(0,T_{\tilde k})|+1+|M\cap(T_{\tilde k},s)|\ge\tilde k+v+1+(n-\tilde k-p)
$$
a~odtiaľ $p\ge v+2$. Platí
$$
%\gathered
T_{\tilde k}+a_{r_1}-a_1<T_{\tilde k}+a_{r_1}-a_2<\cdots<T_{\tilde k}+a_{r_1}-a_{\tilde k}%<\\
<T_{\tilde k}+a_{r_2}-a_{\tilde k}<\cdots<T_{\tilde k}+a_{r_{v+2}}-a_{\tilde k}%<T_{\tilde k}
%\endgathered
$$
a~všetkých $\tilde k+v+1$ čísel porovnaných v~predošlých nerovnostiach leží v~intervale $(0,T_{\tilde k})$. Preto existujú $r\in\{\tilde k+1,\tilde k+2,\dots,n\}$ a~$q\in\{1,2,\dots,\tilde k\}$ také, že $T_{\tilde k}+a_r\notin M$ a~$T_{\tilde k}+a_r-a_q\notin M$. Zoberme množinu skokových dĺžok $B=\{a_1,a_2,\dots,a_{\tilde k},a_r\}\setminus\{a_q\}$. Máme
$$
\sum_{x\in B}x=T_{\tilde k}+a_r-a_q
$$
a
$$
T_{\tilde k}+a_r-a_q-\min(B)=T_{\tilde k}-a_q\le T_{\tilde k-1}.
$$
Podľa \thetag1, indukčného predpokladu a~zovšeobecnenia spomenutého úplne na začiatku s~hodnotou $\tilde k$ namiesto $n$ sa koník dokáže dostať na číslo $T_{\tilde k}+a_r-a_q$ pomocou $\tilde k$~skokov s~dĺžkami z~množiny~$B$ bez pristátia na čísle z~$M$. Odtiaľ vie skočiť na $T_{\tilde k}+a_r$, čím dosiahneme situáciu ako v~Tvrdení~1 s~hodnotou $m=\tilde k+1$. Tým je tvrdenie dokázané.
}

{%%%%%   MEMO, priklad 1
Konštantná funkcia $f(x)=0$ zrejme vyhovuje. Predpokladajme ďalej, že existuje $a\in\Bbb R$ také, že $f(a)=0$. Dokážeme, že potom je $f$ prostá.

Nech $f(y_1)=f(y_2)$ pre nejaké $y_1,y_2\in\Bbb R$. Po postupnom dosadení $y=y_1$, $y=y_2$ do zadanej rovnice a~odčítaní vzniknutých rovností dostaneme $y_1f(x)=y_2f(x)$ pre všetky $x\in\Bbb R$. Po zvolení $x=a$ môžeme $f(a)$ vykrátiť, teda $y_1=y_2$ a~$f$ je naozaj prostá.

Dosadením $x=0$, $y=1$ do pôvodnej rovnice získame
$$
f(0)+f(f(0)+f(1))=f(0)+f(f(1)),
$$
teda $f(f(0)+f(1))=f(f(1))$ a~vzhľadom na prostosť $f$ máme $f(0)+f(1)=f(1)$, odkiaľ $f(0)=0$.

Zvolením $y=0$ následne zo zadanej rovnice dostaneme $f(f(x))=f(x)$, z~čoho opäť vďaka prostosti vyplýva $f(x)=x$ pre všetky $x\in\Bbb R$. Aj táto funkcia vyhovuje, o~čom sa dosadením ľahko presvedčíme. Jedinými riešeniami sú teda funkcie $f(x)=0$ a~$f(x)=x$.
}

{%%%%%   MEMO, priklad 2
V úlohe ide o mnohouholník s ofarbenými všetkými uhlopriečkami aj stranami,
môžeme teda o ňom uvažovať ako o kompletnom grafe.
Farbíme všetky hrany s~použitím $n$ farieb tak,
že každý trojuholník resp. kompletný trojbodový podgraf, tzv. 3-klika, bude buď jednofarebný alebo trojfarebný.
Celý graf ale nemôže byť jednofarebný.

Vzácny čitateľ isto rýchlo pochopí, že podmienka pre trojuholníky
je v skutočnosti ekvivalentná s nasledovnou podmienkou. Graf pozostáva výlučne
z kompletných jednofarebných podgrafov, tzv. klík,
pričom dve kliky jednej farby sú vrcholovo disjunktné.
Otázne už je len ako veľké môžu takéto kliky byť a koľko ich bude najviac z jednej farby.

Ak by jedna z týchto jednofarebných klík (povedzme modrá) mala $n$ alebo viac vrcholov,
musel by existovať vrchol $v$ mimo nej, ktorý by bol s každým vrcholom modrej kliky
spojený inou ako modrou farbou. Všetky trojuholníky tvorené dvoma vrcholmi
z~modrej kliky a vrcholom $v$ obsahujú jednu modrú a jednu inú hranu, sú teda trojfarebné.
Čiže $n$ hrán spájajúcich vrchol $v$ s modrou klikou by muselo byť navzájom rôznofarebných, čo nie je možné,
lebo okrem modrej farby máme k dispozícii už len $n-1$ farieb.
Preto každá jednofarebná klika má maximálne $n-1$ vrcholov.

Jeden vrchol leží v maximálne $n$ rôznych klikách (rôznej farby) a každá z nich má najviac $n-2$ ďalších vrcholov.
Tento vrchol má teda najviac $n(n-2)$ susedov. V grafe je teda nanajvýš $f(n)=n(n-2)+1=(n-1)^2$ vrcholov.

\smallskip
V druhej časti úlohy sa budeme zaoberať konštrukciou $n$-farebného ofarbenia hrán
kompletného grafu s $(n-1)^2$ vrcholmi pre čo najviac rôznych hodnôt čísla $n$.
Všimnime si, že dve kliky rôznych farieb majú v tomto maximálnom prípade práve jeden spoločný vrchol a dve kliky
rovnakej farby ani jeden. Natíska sa tu idea nakresliť si všetky body jednej kliky na priamku jednej farby.
Body kliky inej farby budú na priamke nielen inej farby ale aj iného smeru
tak, aby priesečník týchto dvoch priamok reprezentoval spoločný vrchol príslušných dvoch klík.
Dve kliky rovnakej farby by sa dali reprezentovať rovnobežkami.

Skutočnosť, že vrcholov je dokopy $(n-1)^2$, nám našepkáva umiestniť ich do štvorca (vrcholy teda budeme reprezentovať súradnicami $(i,j)$, pričom $i,j\in\{0,1,\dots,n-2\}$) a~pospájať ich pomocou $n-1$ vodorovných priamok prvej farby.
Potom ich pospájame $n-1$ zvislými priamkami druhej farby.
Prvá priamka tretej farby pôjde po hlavnej diagonále bodmi tvaru $(k,k)$ kde $k\in\{0,\dots,n-2\}$.
Ostatné budú rovnobežne s~ňou spájať všetky body tvaru $((a+k) \bmod (n-1),k)$.
Budú to vlastne všetky priamky so smernicou $1$, ktoré keď "vybehnú" vpravo zo štvorca,
vrátia sa v~príslušnej výške zľava. Dá sa to predstaviť aj tak, že by sme pravý okraj štvorca (s~prvou súradnicou $n-1$) stotožnili s~ľavým (s~prvou súradnicou $0$). Vznikol by tak valec.

Podobne smernica všetkých $n-1$ priamok štvrtej farby bude $2$ a~princíp kongruencie sa tu bude uplatňovať nielen v~pravo-ľavom smere ale aj zdola nahor. Dalo by sa to predstaviť tak,
že v~spomínanom valci navyše stotožníme spodný okraj s~horným (vznikol by tzv. torus).
Podobne smernice ostatných $(n-1)$-tíc priamok budú vždy $3$, $4$, \dots, $(n-2)$.

Ak $n-1$ je prvočíslo, dá sa poľahky ukázať, že táto konštrukcia zaručí, že prienik dvoch klík rôznej farby bude práve jeden vrchol. Keďže prvočísel je nekonečne veľa, máme nekonečne veľa rôznych $n$, pre ktoré $f(n)=(n-1)^2$. Tým je úloha vyriešená.

\poznamka
Skúste sami, či by ste vedeli vymyslieť konštrukciu, ktorá by niečo podobné zaručovala pre širšiu množinu čísel ako je množina tých $n$, že $n-1$ je prvočíslo.
}

{%%%%%   MEMO, priklad 3
Zvoľme za počiatok súradnicovej sústavy priesečník $P$ priamok $AB$ a~$CD$.
Polohové vektory bodov $A$, $B$, $C$, $D$ označíme postupne $\vec a$, $\vec b$, $\vec c$, $\vec d$.
\insp{memo.1}%
Potom smerové vektory priamok $AB$, $CD$ sú $\vec a-\vec b$, $\vec d-\vec c$ a keďže podľa zadania majú rovnakú veľkosť, smerový vektor osi uhla nimi tvoreného je
$$
\vec o=\frac{(\vec a-\vec b)+(\vec d-\vec c\,)}2.
$$
Stredy $E$, $F$ uhlopriečok $AC$, $BD$ majú polohové vektory $\vec e=\frac12(\vec a+\vec c\,)$, $\vec f=\frac12(\vec b+\vec d)$.
Z~rovnosti $|AB|=|CD|$ vyplýva
$$
\align
0 &= (\vec a-\vec b)^2-(\vec d-\vec c\,)^2 =
    \left((\vec a-\vec b)-(\vec d-\vec c\,)\right)\cdot\left((\vec a-\vec b)+(\vec d-\vec c\,)\right) = \\
  &= 4\left(\frac{\vec a+\vec c}2-\frac{\vec b+\vec d}2\,\right)\cdot\frac{(\vec a-\vec b)+(\vec d-\vec c\,)}2=4(\vec e-\vec f\,)\cdot\vec o
\endalign
$$
(všetky uvedené násobenia a~druhé mocniny vektorov sú {\it skalárne\/} súčiny).
Teda os uhla zovretého priamkami $AB$ a~$CD$ je kolmá na priamku~$EF$ a~trojuholník $GHP$ je rovnoramenný so základňou~$GH$ (os uhla je totožná s~výškou jedine v~rovnoramennom trojuholníku, \obr).
Keďže body $A$ a~$D$ ležia v~tej istej polrovine určenej priamkou $GH$,
uhly $AGH$ a~$DHG$ sú zhodné (buď sú oba vnútornými uhlami pri základni rovnoramenného trojuholníka, alebo sú ich doplnkami do $180\st$).
}

{%%%%%   MEMO, priklad 4
Podmienka v~zadaní je ekvivalentná s~injektívnosťou funkcie
$$
f(m)= m^{m-1} \bmod k
$$
na obore zvyškových tried po delení číslom $k$.

Skúsime za $k$ dosadiť najskôr malé hodnoty a~všimneme si, že pre $k=2$ a~$k=3$ je táto funkcia injektívnou.
Navyše pre každé $k$ platí $f(1)=1$ a~$f(k)=0$. Pre párne $k$ väčšie ako $2$ dostávame
$$
f(k-1) \equiv (k-1)^{k-2} \equiv 1=f(1) \pmod k,
$$
teda funkcia $f$ nie je injektívna. Ak sa v~rozklade čísla~$k$ na súčin prvočísel nachádza niektoré prvočíslo, nazvime ho $p$, viac ako raz, potom
$$
f(k/p)=0=f(k).
$$
Ďalšie vyhovujúce $k$ môžu teda mať v~rozklade len rôzne nepárne prvočísla.
Uvažujme teraz vyhovujúce nepárne $k$ väčšie ako $4$.
Hodnoty $f(k)$, $f(k-2)$, \dots, $f(3)$, $f(1)$ a~$f(4)\equiv4^3=8^2\pmod k$
sú všetko zvyšky druhých mocnín prirodzeného čísla po delení číslom $k$ a~je ich spolu $\frac12(k+1)+1$.
Ale rôznych kvadratických zvyškov je nanajvýš $\frac12(k+1)$, lebo $a^2\equiv(k-a)^2\pmod k$. Čiže aspoň dve zo spomenutých hodnôt funkcie~$f$ budú rovnaké a~funkcia nebude injektívna.
Okrem hodnôt $2$ a~$3$ už ďalšie vyhovujúce $k$ neexistuje.
}

{%%%%%   MEMO, priklad t1
Danú podmienku možno prepísať na tvar
$$
x(x-4)+y(y-4)+z(z-4)=-9
$$
a~skúmanú nerovnosť zase na nápadne podobný tvar
$$
x^2(x-4)^2+y^2(y-4)^2+z^2(z-4)^2\ge 27.
$$
Teraz už len vhodne použijeme Cauchyho nerovnosť pre dvojicu vektorov $(1,1,1)$ a~$({x(x-4)},{y(y-4)},{z(z-4)})$:
$$
\align
81&=(-9)^2=\bigl(1\cdot x(x-4)+1\cdot y(y-4)+1\cdot z(z-4)\bigr)^2\le\\
  &\le(1^2+1^2+1^2)\bigl(x^2(x-4)^2+y^2(y-4)^2+z^2(z-4)^2\bigr).
\endalign
$$
Rovnosť v~Cauchyho nerovnosti nastáva práve vtedy, keď
$$
x(x-4)=y(y-4)=z(z-4)=-3.
$$
Riešením je 8 usporiadaných trojíc
$$
(1,1,1),\ (1,1,3),\ (1,3,1),\ (3,1,1),\ (1,3,3),\ (3,1,3),\ (3,3,1),\ (3,3,3).
$$

\ineriesenie
S~využitím väzby $x^2+y^2+z^2+9 = 4(x+y+z)$ možno zadanú nerovnosť ekvivalentne upraviť na tvar
$$
(x-2)^4 + (y-2)^4 + (z-2)^4 \ge 3,
$$
zatiaľ čo samotnú väzbu priamo prepíšeme na $(x-2)^2 + (y-2)^2 + (z-2)^2 = 3$.
Pre trojicu reálnych čísel $u$, $v$, $w$ spĺňajúcich $u^2+v^2+w^2=3$ však platí
$$
\align
(u^2-1)^2+(v^2-1)^2+(w^2-1)^2 &\ge 0,\\
u^4+v^4+w^4+3 &\ge 2(u^2+v^2+w^2)=6,\\
u^4+v^4+w^4 &\ge 3,
\endalign
$$
čo je po substitúcii $u = x - 2$, $v = y - 2$, $w = z - 2$ ekvivalentné s~dokazovaným tvrdením. Rovnosť zrejme nastáva, keď $u^2=v^2=w^2=1$, z~čoho dostaneme rovnaké trojice ako v~prvom riešení.
}

{%%%%%   MEMO, priklad t2
Vyšetríme dva prípady.

\smallskip
Ak všetky tri rovnice majú jeden spoločný koreň $x_1$, tak
$$
\align
x_1^2+ax_1+b&=0,\tag1\\
x_1^2+bx_1+c&=0,\tag2\\
x_1^2+cx_1+a&=0.\tag3
\endalign
$$
Potom kombináciou rovností $(\thetag1-\thetag2)+(x_1-1)(\thetag1-\thetag3)$ a~úpravou dostaneme
$$
(x_1^2-x_1+1)(a-c)=0,
$$
čo vzhľadom na nenulovosť kvadratického výrazu $x_1^2-x_1+1$ (so záporným diskriminantom) znamená ${a=c}$. Pomocou cyklickej zámeny dostaneme $c=b=a$, a~aby mali každé dve z~troch totožných kvadratických rovníc $x^2+ax+a$ práve jedno spoločné riešenie, musia mať dvojnásobný koreň. Odtiaľ $a^2-4a=0$, teda $a=b=c=0$ alebo $a=b=c=4$, z~čoho $a^2+b^2+c^2=0$, resp. $48$.

\smallskip
Druhý prípad, kedy jedna rovnica má korene $x_1$, $x_2$, druhá rovnica má korene $x_2$, $x_3$ a~tretia $x_1$, $x_3$ (pričom všetky tri korene sú rôzne, inak by sme mali predošlý prípad) sa vyšetruje už o~niečo komplikovanejšie. Treba sa prirodzene zaoberať koeficientmi polynómu
$$
(x^2+ax+b)(x^2+bx+c)(x^2+cx+a),
\tag4
$$
o~ktorom vieme, že má tri dvojnásobné nulové body a~teda tvar
$$
(x^3-px^2+qx-r)^2,
\tag5
$$
pričom navyše z~Vi\`etovych vzťahov pre pôvodné polynómy vyplýva
$$
2p=2(x_1+x_2+x_3)=-(a+b+c)=-(x_1x_2+x_2x_3+x_1x_3)=-q,
\tag6
$$
Ak označíme $a+b+c=e$, $ab+bc+ca=f$, $abc=g$, tak roznásobením totožných polynómov \thetag4 a~\thetag5 a~porovnaním ich koeficientov pri mocninách $x^5$, $x^4$, $x^3$ a $x^0$ dostaneme postupne
$$
e=-2p,\qquad e+f=p^2-4p,\qquad e^2-f+g=4p^2-2r,\qquad g=r^2.
\tag7
$$
Koeficienty pri $x^2$ a~$x$ síce nie sú symetrické v~premenných $a$, $b$, $c$ (a~nedajú sa teda vyjadriť pomocou $e$, $f$, $g$), ale ich súčet symetrický je, teda porovnaním po úprave dostaneme
$$
f+ef-3g=4p^2+6pr.
\tag8
$$
Spojením \thetag6, \thetag7 a~\thetag8 dostávame sústavu šiestich rovníc o~šiestich neznámych $p$, $q$, $r$, $e$, $f$, $g$. Tú môžeme riešiť viacerými rôznymi spôsobmi. Napríklad z~prvých dvoch rovníc máme $q=\m2p$, $e=\m2p$, teda po dosadení do tretej získame $f=p^2-2p$. Z~piatej rovnice máme priamo $g=r^2$, teda za všetky premenné $q$, $e$, $f$, $g$ vieme do štvrtej a~šiestej rovnice dosadiť výrazy zapísané len premennými $p$, $r$. Po úprave dostaneme rovnice
$$
(p-1)^2=(r+1)^2,\qquad -2p^3+p^2-2p-6pr-3r^2=0.
$$
Ak $p-1=r+1$, dosadením $r=p-2$ do poslednej rovnice dostaneme rovnicu $(p+6)(p-1)^2=0$. Ak naopak $p-1=-(r+1)$, dosadením $r=-p$ dostaneme $p(p-1)^2=0$. Ak $p=\m6$, dostaneme $r=\m8$, $q=12$, teda polynóm v~\thetag5 by bol rovný $(x+2)^6$ a~nemal by tri {\it rôzne\/} korene. Rovnako nevyhovuje $p=0$, kedy by uvedený polynóm vyšiel $x^6$. Jedinou možnosťou je $p=1$, odkiaľ $r=\m1$, $q=\m2$, $e=\m2$, $f=\m1$, $g=1$ a~polynóm v~\thetag5 má tvar
$$
(x^3-x^2-2x+1)^2.
\tag9
$$
Ľahko možno nahliadnuť, že tento polynóm má tri rôzne reálne dvojnásobné korene (ležiace postupne vnútri intervalov $(\m2,\m1)$, $(0,1)$, $(1,2)$).
Už len dopočítame $a^2+b^2+c^2=e^2-2f=6$.

\odpoved
Uvedený výraz môže nadobúdať hodnoty $0$, $6$ a~$48$.

\poznamka
Pre správne riešenie treba ešte zdôvodniť, že k~trojici koreňov $(x_1,x_2,x_3)$ polynómu \thetag9 skutočne prislúchajú tri kvadratické rovnice s~koeficientmi ako v~zadaní. Tento technický krok vynecháme.
%%Jednou z~možností, ako to urobiť, je ukázať, že príslušné $a$, $b$, $c$ (ktoré sú podľa Vi\`etových vzťahov koreňmi polynómu $x^3-ex^2+fx-g$) po dosadení do \thetag4 naozaj dajú polynóm identický s~\thetag5. Keďže koeficient pri $x^6$ je očividne rovnaký a~rovnosť koeficientov pri $x^5$, $x^4$, $x^3$, $x^0$ vyplýva z~toho, že $p$, $q$, $r$, $e$, $f$, $g$ spĺňajú \thetag7, stačí dokázať rovnosť koeficientov pri $x^2$ a~$x$. Na základe platnosti \thetag8 vieme, že súčet koeficientov pri $x^2$ a~$x$ je rovnaký.
}

{%%%%%   MEMO, priklad t3
Krajné hodnoty $0$ a~$n$ sa nedajú nijako zotrieť a~hravo vyskúšame, že
$$
g(1)=g(2)=2,\qquad g(3)=3,\qquad g(4)=2.
$$
Technika zotierania použitá pre $n=4$ sa dá rozšíriť na všetky $n=2^a$, preto tiež $g(2^a)=2$ pre všetky prirodzené čísla~$a$.

Majme teraz $n$, ktoré nie je mocninou dvojky a~teda $2^a<n<2^{a+1}$ pre nejaké prirodzené~$a$. Predstavme si, že by na konci ostali tiež len dve čísla. Pri spätnej rekonštrukcii (\tj. pridávaním aritmetického priemeru niektorých dvoch čísel na tabuli) vieme v~ľubovoľnom kroku skonštruovať iba číslo tvaru $tn/2^k$, pričom $t$ a~$k\ge1$ sú prirodzené čísla. Teda nikdy by takto nevzniklo číslo $1$ (keď $n$ nie je mocninou dvoch, ani žiadny jeho násobok ňou nemôže byť), čo je spor\footnote{Formálne možno použiť na dôkaz matematickú indukciu.}.

Naopak, existuje postup zotierania najskôr čísel $n-1$, $n-2$ a tak ďalej až po $2^a+1$ (číslo $n-i$ je aritmetickým priemerom čísel $n$ a $n-2i$),
na ktorý sa dá nadviazať skôr spomenutou technikou zotierania všetkých čísel medzi $0$ a $2^a$. Na tabuli potom zostanú len tri čísla $0$, $2^a$, $n$.
Preto ak $n$ nie je mocninou dvojky, tak $g(n)=3$.
}

{%%%%%   MEMO, priklad t4
Predvedieme jeden príklad ofarbenia plochy $k\times k$ (pre veľké nepárne~$k$) troma nesúvislými farbami $1$, $2$, $3$.
Do všetkých políčok napíšeme pre začiatok $0$. Vyberieme si dva susediace stĺpce, napríklad druhý a~tretí.
Do políčka $(2,k)$ pripočítame~$1$, do políčka $(3,1)$ zase $2$. Teraz do všetkých políčok
dosiahnuteľných ťahom dámy z~políčka $(2,k)$, okrem tohto políčka samotného, pripočítame $2$.
Podobne do všetkých políčok dosiahnuteľných ťahom dámy z~políčka $(3, 1)$, okrem tohto políčka samotného, pripočítame $1$. V~tabuľke je znázornená situácia pre $k=9$.
$$
\vbox{\offinterlineskip \everycr{\noalign{\hrule}}
       \halign{\strut\vrule\hbox to 1.2em{\hfill #\hfil}\vrule&&\hbox to 1.2em{\hfill#\hfil}\vrule\cr
     2  &  {\bf1}  &  3  &  2  &  2  &  2  &  2  &  2  &   2  \cr
     2  &  2  &  3  &     &     &     &     &     &      \cr
        &  2  &  1  &  2  &     &     &     &     &   1  \cr
        &  2  &  1  &     &  2  &     &     &  1  &      \cr
        &  2  &  1  &     &     &  2  &  1  &     &      \cr
        &  2  &  1  &     &     &  1  &  2  &     &      \cr
     1  &  2  &  1  &     &  1  &     &     &  2  &      \cr
        &  3  &  1  &  1  &     &     &     &     &   2  \cr
     1  &  3  &  {\bf2}  &  1  &  1  &  1  &  1  &  1  &   1  \cr
}}
$$

Všimnime si, že pre nepárne $k$ majú políčka $(2,k)$ a~$(3,1)$ rôzne pôvodné "šachovnicové" farby.
Preto len 4 políčka vo vybraných dvoch stĺpcoch majú farbu $1 + 2 = 3$.
Sú to $(2,1)$, $(2,2)$ a~$(3, k)$, $(3,k-1)$. Pre dostatočne veľké nepárne $k$ sú tieto dva súvislé páry vzájomne nesúvislé.
Do políčok, kde ešte stále ostala $0$, môžeme teraz vpísať ľubovoľné z~čísel $1$ a~$2$. Políčka $(2,k)$ a~$(3,1)$ ostanú "izolované" od ostatných políčok svojej farby.
Táto technika ofarbovania troma farbami sa dokonca dá poľahky rozšíriť na párne dostatočne veľké $k$
zvolením "izolovaných" políčok $(2,k)$ a~$(4,1)$. Preto $n$ je menšie ako $3$.

\smallskip
V~druhej časti riešenia naznačíme dôkaz, že pre každú plochu $k\times m$ je v~ľubovoľnom jej dvojfarebnom ofarbení aspoň jedna z~farieb súvislá.
Použijeme matematickú indukciu vzhľadom na súčet rozmerov $k+m$. Prvý indukčný krok je triviálny.

V~druhom kroku predpokladajme, že tvrdenie je pravdivé pre ľubovoľnú plochu $k'\times m'$ spĺňajúcu $k'+m'<k+m$. Uvažujme ľubovoľné ofarbenie plochy~$S$ s~rozmermi $k\times m$ dvoma farbami, napr. červenou a~modrou. Označme $S_1$, $S_2$, resp. $S_3$ plochy, ktoré dostaneme z~$S$ odstránením  prvého stĺpca, posledného stĺpca, resp. posledného riadka. Podľa indukčného predpokladu a Dirichletovho princípu niektoré dve plochy $S_i$, $S_j$ majú súvislú rovnakú farbu, napr. červenú. Označme $A_0=S_i\cap S_j$, $A_1=S_j\setminus S_i$ a~$A_2=S_i\setminus S_j$.

Ľahko možno dokázať pravdivosť nasledujúcich troch pozorovaní:
\itemitem{(1)}Ak celá množina $A_0$ je jednofarebná, napr. modrá, tak modrá farba je súvislá v~celej $S$.
\itemitem{(2)}Ak v~$A_0$ existuje červené políčko, tak červená je súvislá v~celej množine $S_i\cup S_j$.
\itemitem{(3)}Ak množina $A_1\cup A_2$ je jednofarebná, napr. modrá, tak modrá farba je súvislá v~celej $S$.

Predpokladajme sporom, že v~$S$ nie je ani jedna farba súvislá. Potom podľa (1) existuje v~$A_0$ červené políčko. Môžu nastať dve možnosti. Ak $S=S_i\cup S_j$ (\tj. ak $\{i,j\}=\{1,2\}$), tak podľa (2) je v~$S$ súvislá červená farba, čo je spor. V~opačnom prípade tvorí množina $S_i\cup S_j$ celú $S$ okrem jedného rohového políčka. Jedinou možnosťou, ako nájsť nesúvislé červené políčka v~$S$, je ofarbiť červenou uvedené rohové políčko a~celý zvyšok stĺpca a~riadka obsahujúceho toto políčko ofarbiť modrou. Potom je však podľa (5) súvislá modrá farba, čo je opäť spor.

\odpoved
Najväčšie hľadané číslo je $n=2$.
}

{%%%%%   MEMO, priklad t5
Ak vezmeme bod $F$ na polpriamke $AB$ taký, že $|BC| = |BF| = |FC|$, dostaneme rovnoramenný lichobežník,
ktorého opísaná kružnica je zároveň opísanou kružnicou trojuholníka $ADC$.
Preto tento bod $F$ bude zhodný s bodom $K$.
Analogicky ak vezmeme bod $G$ na polpriamke $BC$ taký, že $|GB| = |GA| = |BA|$,
zistíme, že bude zhodný s bodom $L$.
Preto $|AC| = |KL|$, keďže sú symetrické podľa osi uhla $CBK$ (alebo uhla $ABL$).
\insp{memo.2}%

Z~tetivového štvoruholníka $APCD$ máme
$$
|\uhol APC| = 180^\circ - |\uhol ADC| = 180^\circ - 120^\circ = 60^\circ.
$$

Uvažujme zobrazenie $\Cal Z$, ktoré vznikne zložením kružnicovej inverzie podľa kružnice so stredom $B$ a~polomerom $\sqrt{|BA|\cdot|BK|}=\sqrt{|BL|\cdot|BC|}$ a~stredovej súmernosti so stredom~$B$. V~tomto zobrazení sa body $K$, $L$ zobrazia postupne na $A$, $C$. Navyše z~mocností bodu $B$ máme $|BA|\cdot|BK|=|BD|\cdot|BP|$ a~tiež $|BL|\cdot|BC|=|BE|\cdot|BM|$, čiže ${\Cal Z}(D)=P$ a~${\Cal Z}(E)=M$.
Keďže obrazy bodov $A$, $C$, $E$ sú $K$, $L$, $M$ a~z~vlastností inverzie obraz každej priamky neprechádzajúcej cez $B$ je kružnica prechádzajúca cez $B$, štvoruholník $BLMK$ je tetivový. Preto
$$
|\uhol KML| = 180^\circ - |\uhol KBL| = 180^\circ - 120^\circ = 60^\circ.
$$
Ďalej máme (\obr)
$$
|\uhol LKM| =|\uhol LBM| =|\uhol DBC| = |\uhol ADP| = |\uhol ACP|.
$$
Preto trojuholníky $APC$, $LMK$ sú podobné a~vzhľadom na fakt $|AC| = |LK|$ dokázaný v~úvode musia byť zhodné.
}

{%%%%%   MEMO, priklad t6
Ak $N$ je obraz bodu $A$ v osovej súmernosti podľa $DE$, potom
$$
|\uhol CDN| = |\uhol CDA| - |\uhol ADN| = 2 |\uhol EDF| - 2 |\uhol EDN| = 2 |\uhol FDN|
$$
a $|ND| = |AD| = |CD|$. Takže $N$ je zároveň obrazom bodu $C$ v osovej súmernosti podľa $DF$.
Teda trojuholníky $ADE$ a $NDE$ sú zhodné, rovnako aj trojuholníky $CDF$ a $NDF$. Preto
$$
|\uhol END| + |\uhol DNF| = |\uhol EAD| + |\uhol DCF| = |\uhol BAD| +|\uhol DCB| = 180^\circ,
$$
čiže $N$ leží na úsečke $EF$ (\obr).
\insp{memo.3}%

Použime označenie $d(X, YZ)$ pre vzdialenosť bodu $X$ od priamky $YZ$. Potom
$$
d(D, BE) = d(D, AE) = d(D, NE) = d(D, FN) = d(D, FC) = d(D, FB),
$$
pričom druhá a predposledná rovnosť sú dôsledkom spomenutých zhodností trojuholníkov.
Všimnime si, že bod $D$ leží v rovnakej vzdialenosti od troch priamok obsahujúcich
strany trojuholníka $BEF$ a je teda stredom pripísanej kružnice trojuholníka $BEF$.
Preto pripísaná kružnica sa dotýka úsečky $EF$ v bode $K$.
\insp{memo.4}%

Z vlastností vpísanej a~pripísanej kružnice vyplýva,
že vpísaná kružnica trojuholníka $BEF$ sa dotýka úsečky $EF$ v~bode $L$ (stačí si spomenúť na vyjadrenie dĺžok úsekov od vrcholov trojuholníka k~dotykovým bodom vpísanej, resp. pripísanej kružnice).
Nech $P$ je obraz bodu $K$ v stredovej súmernosti so stredom v bode $D$ (\obr).
Rovnoľahlosť so stredom v bode $B$,
ktorá zobrazí vpísanú kružnicu trojuholníka $BEF$
na jeho pripísanú kružnicu, zobrazí $L$ na $P$, a teda body $B$, $L$, $P$ ležia na jednej priamke.
Teraz už dokazované tvrdenie vyplýva z rovnobežnosti $DM$ a $PL$,
ktorá je triviálnym dôsledkom rovnosti pomerov $|KL|:|KM| = |KP|:|DP|=2$.
}

{%%%%%   MEMO, priklad t7
Ak je dvojica $(m,n)$ riešením, sú ním aj dvojice $(n,m)$, $(\m m,\m n)$ a~$(\m n,\m m)$. Pre $n=0$ dostaneme rovnicu $m^4=m^2$ s~riešeniami
$m\in\{-1,0,1\}$. Zo symetrickosti riešenia vyplýva, že riešením je
päť dvojíc
$$
(n,m)\in\{(0,-1),(-1,0),(0,0),(0,1),(1,0)\}.
$$
Treba si uvedomiť, že ak by boli $m$ aj $n$ kladné, ľavá strana by bola
rádovo väčšia ako pravá, čo dáva silné tušenie neexistencie
takéhoto riešenia. Formálny dôkaz správnosti tohto tušenia môže
vyzerať napríklad takto: Bez ujmy na všeobecnosti nech $0<n\le m$, potom na základe
rovnice platia aj nerovnosti
$$
\align
(m+1)^4 &\le (m+n)^4 = m^2n^2+(m+n)^2+4mn\le m^4+8m^2, \\
4m^3+6m^2+4m+1 &\le 8m^2, \\
2m^2(2m-1) &< 0.
\endalign
$$
Pre ľubovoľné prirodzené číslo~$m$ je toto spor, preto riešením nie je žiadna
dvojica kladných čísel. Navyše analogicky ani žiadna dvojica
záporných čísel.

Preto bez ujmy na všeobecnosti nech $n=\m k$ je záporné a~$0<k\le m$. Potom
$$
\align
(m-k)^4 &= m^2k^2+m^2+k^2-6mk, \\
m^4-4m^3k+5m^2k^2-4mk^3+k^4 &= m^2 -6mk + k^2, \\
(m^2 -mk + k^2)(m^2 -3mk + k^2) &= m^2 -6mk + k^2, \\
(m^2 -mk + k^2)(m^2 -3mk + k^2 -1) &= -5mk\\
\text{a zároveň}\qquad
(m^2 -mk + k^2 -1)(m^2 -3mk + k^2) &= -3mk.
\endalign
$$
Odtiaľ
$$
(m^2 -mk + k^2) \mid 5mk
\qquad \text{a} \qquad
(m^2 -3mk + k^2) \mid 3mk.
$$
Nech $\nsd(m,k)=d$ a~$k=da$, $m=db$ (pričom $a,b>0$), potom
$$
(b^2 -ba + a^2) \mid 5ba
\qquad \text{a} \qquad
(b^2 -3ba + a^2) \mid 3ba.
$$
Ale
$$
\nsd(b^2 -ba + a^2,b) =1
\qquad \text{a} \qquad
\nsd(b^2 -ba + a^2,a) =1,
$$
čo je ekvivalentné s~$\nsd(b^2 -ba + a^2,ba) =1$, preto
$$
b^2 -ba + a^2 \in \{1,5\}.
$$
Ak $b^2 -ba + a^2=1$ a~pozrieme sa na to ako na kvadratickú rovnicu
s~neznámou~$a$, diskriminant $4-3b^2$ je štvorcom prirodzeného čísla
len pre $b=1$. Odtiaľ sa poľahky dopočítajú riešenia
$$
(n,m)\in\{(2,-2),(-2,2)\}.
$$
Ak $b^2 -ba + a^2=5$, diskriminant $20-3b^2$ nie je štvorcom
prirodzeného čísla nikdy. Celočíselným riešením rovnice je teda
sedmica už spomenutých dvojíc.

\poznamka
Druhú "vetvu" s~výrazom $3mk$ na pravej strane sme v~riešení nepotrebovali. Ak by sme ju použili v~analogickom postupe, museli by sme pre výraz $b^2 -3ba + a^2$ preverovať až štyri hodnoty $\pm1$, $\pm3$. Pri výraze $b^2 -ba + a^2$ sme hodnoty $\m1$, $\m5$ preverovať nemuseli, lebo $b^2+a^2\ge2ab$, \tj. $b^2-ab+a^2>0$. Okrem toho, diskriminant by pre niektorú rovnicu $b^2 -3ba + a^2=\pm1$, resp. $\pm3$, mohol byť štvorcom pre nekonečne veľa hodnôt~$b$.

Inou možnosťou je skombinovať obe vetvy a~riešiť sústavy $b^2 -ba + a^2=p$, $b^2 -3ba + a^2=q$ pre $p\in\{1,5\}$, $q\in\{\pm1,\pm3\}$. Tým sa možno vyhnúť úvahám o~diskriminantoch, keďže každú zo sústav je ľahké v~obore celých čísel vyriešiť (napr. po odčítaní rovníc priamo dostaneme hodnotu súčinu $ab$).
}

{%%%%%   MEMO, priklad t8
Ľahko nahliadneme, že $x\ge 2$ a $y+z\ge 1$.
Uvažujúc nad danou rovnicou v~zvyškových triedach dostaneme $1 \equiv (-1)^y \pmod 4$,
preto $y$ musí byť párne. Podobne ak $y > 0$, tak $(-1)^x + 2 \equiv 0 \pmod 3$,
čiže $x$ je párne. Ak $z > 0$, máme $2^x - 1 \equiv 0 \pmod 5$, čo pre $x$ znamená deliteľnosť štyrmi.
Ale aspoň jedno z dvojice $y$, $z$ je kladné, a teda $x$ musí byť tak či onak párne. Nech
$x = 2t$ a $y = 2u$. Potom
$$
\align
2^x + 2009 &\equiv 2^x \in\{1,2,4\} \pmod 7,\\
3^y 5^z &\equiv 9^u (-2)^z \in\{3,5,6\} \pmod 7\quad\text{pre nepárne  $z$.}
\endalign
$$
Preto $z$ je párne a $z = 2v$.
Rovnicu teraz možno prepísať na tvar
$$
2009 = (3^u 5^v - 2^t)(3^u 5^v + 2^t).
$$
Existuje však len jedna dvojica čísel, ktorých súčin je 2009 a~ich rozdiel je mocninou dvoch (\tj. $2^{t+1}$): $41$ a $49$.
Z toho už poľahky dopočítame jediné riešenie
$$
x = 4,\quad y = 4,\quad z = 2.
$$
}
