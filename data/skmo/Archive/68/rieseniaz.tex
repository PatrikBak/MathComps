{%%%%%   Z4-I-1
...}

{%%%%%   Z4-I-2
...}

{%%%%%   Z4-I-3
...}

{%%%%%   Z4-I-4
...}

{%%%%%   Z4-I-5
...}

{%%%%%   Z4-I-6
...}

{%%%%%   Z5-I-1
\napad
Koľko najmenej a~koľko najviac pasteliek môžu mať dokopy?

\riesenie
Vojto má menej pasteliek ako Miška, teda môže mať
$$0,\,1,\,2,\,3,\ \text{alebo}\ 4$$
pasteliek.
Vendelín má o~päť viac pasteliek ako Vojto, teda môže mať postupne
$$5,\,6,\,7,\,8,\ \text{alebo}\ 9$$
pasteliek.
Všetci traja dokopy majú dvojnásobok toho, čo má Vendelín, teda môžu mať postupne
$$10,\,12,\,14,\,16,\ \text{alebo}\ 18$$
pasteliek.

Medzi týmito číslami je jedine 14 sedemnásobkom celého čísla, $14=7\cdot 2$.
Teda Vojto má 2~pastelky a~Vendelín $2+5=7$ pasteliek.
}

{%%%%%   Z5-I-2
\napad
Vyrobte si tiež také trojuholníky.

\riesenie
Obvod jedného trojuholníka je $3+4+5=12$\,(cm).
Po priložení trojuholníka celou stranou k~už zloženému útvaru môže byť obvod nového útvaru väčší buď o~$3+4-5=2$, alebo o~$3+5-4=4$, alebo o~$4+5-3=6$\,(cm).
Tieto poznatky poskytujú určitú pomôcku k~tomu, ako prikladaním trojuholníkov realizovať ten-ktorý obvod.
Hlavná podmienka, ktorú je nutné pri experimentovaní strážiť, je, aby výsledný útvar bol štvoruholníkom.

Pre jednotlivé obvody uvádzame všetky riešenia, pre ktoré sú výsledné štvoruholníky navzájom nezhodné (v~niektorých prípadoch je možné trojuholníky, z~ktorých je štvoruholník zložený, ešte poprekladať).
Pre vyhovujúce riešenie úlohy stačí pri každom obvode uviesť dve možnosti.

Štvoruholník s~obvodom 14\,cm možno zložiť iba z~dvoch trojuholníkov:
\figure{z5-I-2a}%


Štvoruholník s~obvodom 18\,cm možno zložiť buď z~dvoch, alebo z~troch trojuholníkov:
\figure{z5-I-2b}%


Štvoruholník s~obvodom 22\,cm možno zložiť len buď z~troch, alebo zo štyroch trojuholníkov:
\figure{z5-I-2c}%


Štvoruholník s~obvodom 26\,cm možno zložiť len zo štyroch trojuholníkov:
\figure{z5-I-2d}%


\poznamka
To, že uvedené útvary sú štvoruholníky, vyplýva vo väčšine prípadov z~toho, že v~použitých trojuholníkoch sú pravé uhly oproti najdlhšej, päťcentimetrovej strane, príp. z~toho, že z~dvoch takých trojuholníkov sa dá zložiť obdĺžnik.
Až na niekoľko výnimiek možno na problém pozerať aj ako na vymedzovanie oblastí v~pomocnej obdĺžnikovej sieti (s~uhlopriečkami).
Úplné zdôvodnenie sa nedá po riešiteľoch v~tejto kategórii vyžadovať.

}

{%%%%%   Z5-I-3
\napad
Ktorý deň v~mesiaci musí mať človek narodeniny, aby mohol oslavovať antinarodeniny?

\riesenie
Kto sa narodil 1.\,1., ten má antinarodeniny v~ten istý deň ako narodeniny.
Kto sa narodil v~období od 2.\,1. do 12.\,1., ten má antinarodeniny v~iný deň ako narodeniny.
Kto sa narodil v~období od 13.\,1. do 31.\,1., ten antinarodeniny oslavovať nemôže, keďže mesiacov v~roku je len~12.
V~mesiaci január je teda 11~dní, ktoré vyhovujú požiadavke zo zadania.

Podobne je to v~každom ďalšom mesiaci: prvých dvanásť dní dáva zmysluplný dátum antinarodenín, z~toho práve v~jednom prípade sa jedná o~ten istý dátum. Vyhovujúcich dní v~každom mesiaci je~11, mesiacov je~12, vyhovujúcich dní v~roku je preto $11\cdot12=132$.
}

{%%%%%   Z5-I-4
\napad
Koľko nôh prislúcha obsadenej stoličke?

\riesenie
Stôl a~dve neobsadené stoličky majú celkom $3+8=11$ nôh.
Na všetky obsadené stoličky tak pripadá $101-11=90$ nôh.
Na každej takej stoličke sedel jeden skaut s~dvoma nohami.
V~klubovni teda bolo $90:6=15$ obsadených stoličiek a~$15+2=17$ stoličiek celkom.

}

{%%%%%   Z5-I-5
\napad
Kam mohol Tomáš umiestniť zátvorky?

\riesenie
Tomáš mal 4 kartičky s~číslami a~5 kartičiek so symbolmi.
V~ním zostavených úlohách sa tieto dva typy kartičiek striedali, preto každá začínala a~končila symbolom:
$$\text{S\ Č\ S\ Č\ S\ Č\ S\ Č\ S}$$
Pred ľavou zátvorkou a~za pravou zátvorkou by mal byť symbol s~nejakou matematickou operáciou.
Aby súčasne bola splnená predchádzajúca podmienka a~aby kartičky tvorili zmysluplné úlohy, musel Tomáš umiestniť zátvorky na kraje takto:
$$\text{(\ Č\ S\ Č\ S\ Č\ S\ Č\ )}$$

Násobenie a~pričítanie výsledok zväčšuje, pričom väčší vplyv má násobenie, odčítanie výsledok zmenšuje.
Najväčší možný výsledok mohol Tomáš získať násobením najväčších možných čísel, odčítaním najmenšieho možného čísla a~pričítaním zvyšného čísla, teda napr. takto:
$$\text{(\ 20\ $\times$\ 20\ $-$\ 18\ $+$\ 19\ )},$$
čo je rovné 401.

\poznamka
V~uvedenom riešení sme neuvažovali skrátený zápis násobenia zátvorky číslom.
S~týmto nápadom by najväčší výsledok bolo možné získať takto:
$$\text{$+$\ 19\ (\ 20\ $\times$\ 20\ $-$\ 18\ )},$$
čo je rovné 7\,258.
Aj také riešenie hodnoťte ako správne.
}

{%%%%%   Z5-I-6
Pri postupnom pootáčaní plánika dostávame:
\figure{z5-I-6a}%


Cesta v~jednotlivých prípadoch prechádza siedmimi, ôsmimi, resp. štyrmi sivými políčkami.
}

{%%%%%   Z6-I-1
\napad
Koľko hrušiek si vzala Mirka druhý raz?

\riesenie
Ivan si bral trikrát po dvoch hruškách, nakoniec tak mal 6~hrušiek.
Aby sme určili, koľko nakoniec mala Mirka, postupne odzadu zistíme, ako sa počty hrušiek vyvíjali.
Na to si stačí uvedomiť, že pred každým Ivanovým odoberaním bolo v~mise o~dve hrušky viac a~pred každým Mirkiným odoberaním bol v~mise dvojnásobný počet hrušiek.

Ivan si pri svojom treťom odoberaní vzal posledné 2~hrušky.

Mirka si pri svojom druhom odoberaní vzala 2~hrušky, pred tým v~mise boli 4~hrušky.

Ivan si pri svojom druhom odoberaní vzal 2~hrušky, pred tým v~mise bolo 6~hrušiek.

Mirka si pri svojom prvom odoberaní vzala 6~hrušiek, pred tým v~mise bolo 12~hrušiek.

Ivan si pri svojom prvom odoberaní vzal 2~hrušky, pôvodne v~mise bolo 14~hrušiek.

Mirka si celkom vzala 8~hrušiek, nakoniec teda mala o~dve hrušky viac ako Ivan.
}

{%%%%%   Z6-I-2
\napad
Všimnite si určité súmernosti.

\riesenie
Jedno z~najjednoduchších možných rozdelení je naznačené na nasledujúcom obrázku:
\figure{z6-I-2a}%


Novo vzniknuté časti sú -- rovnako ako pôvodný útvar -- súmerné podľa stredu ohraničujúceho štvorca.
Predchádzajúce rozdelenie je možné postupne modifikovať tak, aby stredovo súmerné štvorčeky patrili do rôznych častí.
Pridávaním, resp. odoberaním štvorčekov v~prvom stĺpci dostávame nasledujúce možné rozdelenia:
\figure{z6-I-2b}%


\poznamka
Akákoľvek ďalšia modifikácia vedie buď k~rozdeleniu zhodnému s~niektorým z~predchádzajúcich, alebo k~rozdeleniu, ktoré pozostáva z~viac nesúvislých častí.
Uvedené štyri riešenia teda predstavujú všetky rôzne spôsoby rozdelenia.
}

{%%%%%   Z6-I-3
\napad
Ktoré čísla môžete dopĺňať?

\riesenie
Prvočíselný rozklad čísla 2018 je $2\cdot1009$.
Číslo 2018 je teda možné zapísať ako súčin troch kladných čísel iba dvoma spôsobmi (až na zámenu poradia činiteľov):
$$
1\cdot1\cdot2018,\quad 1\cdot2\cdot1009.
$$
Do prázdnych políčok je teda možné doplniť iba niektoré z~čísel 1, 2, 1009 a~2018.
Kvôli jednoduchšiemu vyjadrovaniu si neznáme čísla v~prázdnych políčkach označíme:
\figure{z6-I-3a}%


Aby platilo $1\cdot A\cdot B=A\cdot B\cdot C$, musí byť $C=1$.
Aby platilo $A\cdot B\cdot C=B\cdot C\cdot 2$, musí byť $A=2$.
Aby platilo $B\cdot C\cdot 2=C\cdot 2\cdot D$, musí byť $D=B$.
Takto postupne zisťujeme
$$
1=C=E,\quad A=2=F,\quad B=D=G\quad\text{atď.}
$$
Čísla v~políčkach sa teda pravidelne opakujú podľa nasledujúceho vzoru:
\figure{z6-I-3b}%


Aby teraz súčin ľubovoľných troch navzájom susediacich políčok bol 2018, musí byť $B=1009$.
V~hornom riadku sa teda pravidelne strieda trojica čísel 2, 1, 1009.
Keďže $2019=3\cdot673$, je 2019.~políčko tretím políčkom v~673.~trojici v~hornom riadku.
Preto je v~tomto políčku číslo 1009.

\poznamka
Akonáhle vieme, ktoré čísla sa môžu v~políčkach vyskytovať,
môžeme ich začať postupne dopĺňať do niektorého z~prázdnych políčok a~následne skúmať, či a~prípadne ako pokračovať ďalej.
Tak možno vylúčiť všetky možnosti až na tú uvedenú vyššie.
(Keby sme napr. doplnili $A=1$, tak z~požiadavky $1\cdot A\cdot B=2018$ vyplýva, že $B=2018$.
Aby ďalej platilo $A\cdot B\cdot C=2018$, muselo by byť $C=1$, a~teda $B\cdot C\cdot 2=2018\cdot1\cdot2$.
Tento súčin však nie je 2018, preto $A$ nemôže byť 1.)

Riešenie, z~ktorého nie je zrejmé, prečo vyššie uvedené doplnenie je jediné možné, nemôže byť hodnotené najlepším stupňom.
}

{%%%%%   Z6-I-4
\napad
Znázornite si všetky tri situácie.

\riesenie
Všetky tri situácie si znázorníme takto:
\figure{z6-I-4a}%


Keďže Jarko s~Fankom na hlave merajú rovnako ako Maško, môžeme v~druhej a~tretej situácii Maška touto dvojicou nahradiť:
\figure{z6-I-4b}%


Teraz je všetko jasné:
Dvaja Fankovia na sebe merajú 34\,cm, jeden Fanko tak meria 17\,cm.
Dvaja Jarkovia na sebe merajú 72\,cm, jeden Jarko preto meria 36\,cm.
Maško potom meria ako Fanko a~Jarko dokopy, teda $17+36=53$\,(cm).

}

{%%%%%   Z6-I-5
\napad
Začnite s~ciframi schovanými za písmenami $A$ a~$L$.

\riesenie
Najskôr si všimnime, že v~príklade sa vyskytuje desať rôznych písmen, teda pri nahrádzaní budú použité všetky cifry.

Ďalej vidíme, že sčítame päťciferné číslo s~trojciferným a~že na mieste tisícok sa mení cifra.
To je možné jedine v~prípade, že dochádza k~prenosu zo stĺpca stoviek (\uv{prechod cez desiatku}).
Vzhľadom na to, že príslušný súčet je vždy menší ako 20, musí byť $A=9$, $L=0$ a~$U=R+1$.

Teraz oba sčítance na mieste desiatok sú 9, teda by mohlo byť $H=8$ alebo $H=9$ podľa toho, či dochádza k~prenosu z~posledného stĺpca.
Vzhľadom na to, že $H$ a~$A$ majú byť rôzne, musí byť $H=8$ a~k~prenosu z~posledného stĺpca nedochádza.
Pre cifry v~tomto stĺpci preto platí $Y=M+D$.

V~stĺpci na mieste stoviek prispieva prenos zo stĺpca desiatok a~súčasne sám prispieva do stĺpca tisícok.
Pre cifry v~tomto stĺpci preto platí $10+O={T+R+1}$.
Kvôli prehľadnosti stanovené poznatky zhrnieme:
$$
\alggg{
R&9&T&9&M\\
&&R&9&D}
{U&0&O&8&Y},
\quad\text{pričom}\quad
U=R+1,\quad O=T+R-9,\quad Y=M+D.
$$

Neznáme písmená treba nahradiť ciframi od 1 do 7 tak, aby platili všetky uvedené vzťahy.
Tu sa nedá vyhnúť skúšaniu, ktoré môže byť značne zdĺhavé.
Vzhľadom na to, že písmeno~$R$ sa opakuje, je vhodné začať skúšať odtiaľ.
Ukazuje sa, že pre všetky vyhovujúce riešenia je $R=6$, a~teda $U=7$.
Zvyšné cifry od 1 do 5 môžu byť dosadené nasledovne:

$$
\alggg{
6&9&5&9&1\\
&&6&9&3}
{7&0&2&8&4}
\quad
\alggg{
6&9&5&9&3\\
&&6&9&1}
{7&0&2&8&4}
\qquad
\alggg{
6&9&4&9&2\\
&&6&9&3}
{7&0&1&8&5}
\quad
\alggg{
6&9&4&9&3\\
&&6&9&2}
{7&0&1&8&5}
$$

Pre uspokojivé vyriešenie úlohy stačí nájsť dve z~uvedených nahradení.
Zámena $M$ a~$D$ je zrejmou možnosťou, ako z~jedného riešenia odvodiť druhé.

\poznamka
Pred samotným skúšaním si možno všimnúť, že aby platilo $O\ge1$, musí byť $T+R\ge10$.
Vzhľadom na to, že žiadna z~týchto cifier nemôže byť väčšia ako 7, musí byť ako $T$, tak $R$ aspoň 3.
Z~uvedených vzťahov ďalej vyplýva, že $Y$ musí byť tiež aspoň~3 a~$U$ aspoň~4.
Cifry 1 a~2 preto musia byť niektoré dve písmená z~trojice $M$,~$D$~a~$O$.

}

{%%%%%   Z6-I-6
\napad
Rozdeľte obrázok na menšie navzájom zhodné útvary.

\riesenie
Zo zadania vieme, že dvanásťuholník $ABCDEFGHIJKL$ pozostáva zo šiestich navzájom zhodných štvorcov.
Každý z~týchto štvorcov môže byť ďalej rozdelený uhlopriečkami na štyri navzájom zhodné trojuholníčky.
Dvanásťuholník teda pozostáva z~24 navzájom zhodných trojuholníčkov, a~to tak, že každý z~vyššie menovaných útvarov je tvorený niekoľkými takýmito trojuholníčkami:
\figure{z6-I-6a}%


Štvoruholník $EFGM$ pozostáva zo siedmich týchto trojuholníčkov a~má obsah 7\,cm$^2$.
Jeden trojuholníček má preto obsah 1\,cm$^2$ a~obsahy všetkých útvarov sú
$$
\gather
S_{IJM}=1\cm^2,\quad
S_{IGH}=S_{JKL}=2\cm^2,\\
S_{CDEMB}=5\cm^2,\quad
S_{ABML}=S_{EFGM}=7\cm^2.
\endgather
$$
}

{%%%%%   Z7-I-1
\napad
Ktoré cifry nemôžu byť na kartičkách?

\riesenie
Číslo je deliteľné šiestimi práve vtedy, keď je deliteľné dvoma a~súčasne troma, \tj. práve vtedy, keď je párne a~jeho ciferný súčet je deliteľný troma.
Na všetkých kartičkách preto musia byť párne cifry a~ich súčet musí byť deliteľný troma.
Trojice cifier vyhovujúce týmto dvom požiadavkám sú (až na poradie):
$$
2,\ 2,\ 2,\quad
2,\ 2,\ 8,\quad
2,\ 4,\ 6,\quad
2,\ 8,\ 8,\quad
4,\ 4,\ 4,\quad
4,\ 6,\ 8,\quad
6,\ 6,\ 6,\quad
8,\ 8,\ 8.
$$

Teraz je potrebné pre každú trojicu vyskúšať, či možno pri niektorom usporiadaní cifier dostať trojciferné číslo deliteľné~11.
To sa okrem tretej trojice nestane a~pre túto trojicu dve zo šiestich možných usporiadaní dávajú číslo deliteľné 11:
$$
462:11=42,\quad
264:11=24.
$$

Na kartičkách môžu byť jedine cifry 2, 4 a~6.

\poznamka
Pri overovaní deliteľnosti 11 možno výhodne využiť nasledujúce kritérium:
číslo je deliteľné 11 práve vtedy, keď rozdiel súčtu cifier na párnych a~na nepárnych miestach je deliteľný~11.
(V~našom prípade $4-6+2=0$ a~$2-6+4=0$ sú deliteľné~11, ale napr. $2-4+6=4$ nie je.)
}

{%%%%%   Z7-I-2
\napad
Rozdeľte obrázok na menšie navzájom zhodné útvary.

\riesenie
Zo zadania vieme, že dvanásťuholník $ABCDEFGHIJKL$ pozostáva zo šiestich navzájom zhodných štvorcov.
Každý z~týchto štvorcov môže byť ďalej rozdelený uhlopriečkami na štyri navzájom zhodné trojuholníčky.
Dvanásťuholník teda pozostáva z~24 navzájom zhodných trojuholníčkov, a~to tak, že každý z~vyššie menovaných útvarov je tvorený niekoľkými takýmito trojuholníčkami:
\figure{z7-I-2a}%


Teraz je zrejmé, že obsahy štvoruholníka $ABMJ$ a~štvoruholníka $EFGM$ sú v~pomere $3:7$.
Keďže obsah štvoruholníka $ABMJ$ je 1,8\,cm$^2$, je obsah štvoruholníka $EFGM$ rovný 4,2\,cm$^2$.
}

{%%%%%   Z7-I-3
\napad
Koľko orieškov si vzal Edo?

\riesenie
Budeme postupovať odzadu:

Keď si Edo vzal polovicu orieškov, ktoré na neho zvýšili po Danovi, a~jeden navyše, neostalo nič.
Túto polovicu teda tvoril jeden oriešok.
Po Danovi zvýšili dva oriešky.

Dano si tiež bral polovicu orieškov, ktoré na neho zvýšili po Cyrilovi, a~jeden navyše.
Túto polovicu teda tvorili oné dva oriešky plus jeden navyše.
Po Cyrilovi zvýšilo šesť orieškov.

Podobne je to s~ostatnými.
Polovicu orieškov, ktoré zvýšili na Cyrila po Bobovi, tvorilo predchádzajúcich šesť orieškov plus jeden navyše.
Po Bobovi zvýšilo 14~orieškov.

Polovicu orieškov, ktoré zvýšili na Boba po Adamovi, tvorilo predchádzajúcich 14~orieškov plus jeden navyše.
Po Adamovi zvýšilo 30~orieškov.

Týchto 30~orieškov plus jeden navyše tvorilo polovicu všetkých orieškov, ktoré boli pripravené na rozobranie.
Pôvodne bolo na kôpke 62~orieškov.

\poznamka
Ak $z$ označuje počet orieškov, ktoré zvýšili po odoberaní niektorého z~chlapcov, tak pred tým na kôpke bolo $2(z+1)$ orieškov.
Toto je skrátený zápis opakujúcej sa myšlienky predchádzajúceho riešenia.

Naopak, ak $n$ označuje počet orieškov na kôpke pred tým, ako niektorý chlapec začal odoberať, tak si vzal $\frac{n}2+1$ orieškov a~po ňom zvýšilo $n-\left( \frac{n}2+1 \right)=\frac{n}2-1$ orieškov.
Opakovaním tohto kroku dostávame postupnosť zvyškov
$$
n,\quad
\frac{n}2-1,\quad
\frac{n}4-\frac32,\quad
\frac{n}8-\frac74,\quad
\frac{n}{16}-\frac{15}8,\quad
\frac{n}{32}-\frac{31}{16},\quad\dots
$$
Keby $n$ bol pôvodný počet orieškov, tak zvyšok po Edovi by bol $\frac{n}{32}-\frac{31}{16}$.
Tento výraz je rovný nule práve vtedy, keď $n=62$.
}

{%%%%%   Z7-I-4
\napad
Koľkokrát zuby prvého kolesa zapadnú medzi zuby druhého kolesa, ak sa prvé koleso otočí štyrikrát?

\riesenie
Na všetkých kolesách je počet zubov použitých pri otáčaní rovnaký (každý zub počítame toľkokrát, koľkokrát bol v~kontakte s~iným zubom na inom kolese). Podľa zadania vieme o~tomto počte povedať nasledujúce.

Prvé koleso malo 32~zubov a~urobilo štyri otáčky a~časť piatej, teda bolo použitých viac ako $32\cdot4=128$ a~menej ako $32\cdot5=160$ zubov.
Druhé koleso malo 24~zubov a~urobilo päť otáčok a~časť šiestej, teda bolo použitých viac ako $24\cdot5=120$ a~menej ako $24\cdot6=144$ zubov.
Tretie koleso sa otočilo presne osemkrát, teda počet použitých zubov je deliteľný ôsmimi.

Dohromady, počet použitých zubov je číslo, ktoré je násobkom ôsmich, je väčšie ako 128 a~menšie ako 144.
Také číslo je jediné, konkrétne 136.
Tretie koleso malo ${136:8}=17$ zubov.
}

{%%%%%   Z7-I-5
\napad
Predstavte si jednotlivé objednávky v~rámci celkového počtu dodaných ruží.

\riesenie
Ak sčítame počty ruží dodaných do všetkých troch predajní, dostaneme 270 kusov.
V~tomto súčte sú dvakrát zahrnuté počty ruží od každej farby.

V~prvej predajni boli ruže červené a~žlté v~celkovom počte 120 kusov.
Dvojnásobok tohto počtu je 240, bielych ruží dodaných do zvyšných dvoch predajní teda bolo $270-240=30$.

V~druhej predajni bolo 105 ruží červených a~bielych.
Dvojnásobok tohto počtu je 210, žltých ruží dodaných do zvyšných dvoch predajní teda bolo $270-210=60$.

V~tretej predajni bolo 45~ruží žltých a~bielych.
Dvojnásobok tohto počtu je 90, červených ruží dodaných do zvyšných dvoch predajní teda bolo $270-90=180$.

\poznamka
V~jednotlivých predajniach boli počty ruží každej farby polovičné, teda bielych 15, žltých 30 a~červených 90.

Ak by sme tieto pôvodne neznáme počty označili $b$, $\check{z}$ a~$\check{c}$, tak informácie zo zadania možno zapísať ako
$$
\check{c}+\check{z}=120,\quad
\check{c}+b=105,\quad
\check{z}+b=45.
$$
Úvahy vo vyššie uvedenom riešení tak zodpovedajú nasledujúcim úpravám:
$$
2\check{c}+2\check{z}+2b=270,\quad
2\check{c}+2\check{z}=240,
\quad\text{teda}\quad
2b=30,
$$
a~podobne vo zvyšných dvoch prípadoch.

}

{%%%%%   Z7-I-6
\napad
Čo je rovnoramenný trojuholník?

\riesenie
Strana~$AB$ rovnoramenného trojuholníka $ABC$ môže byť buď jeho základňou, alebo ramenom.
Podľa toho rozdelíme riešenie na dve časti.

a) Strana~$AB$ je základňou rovnoramenného trojuholníka $ABC$.
V~tomto prípade je $C$ hlavným vrcholom trojuholníka $ABC$ a~leží na jeho osi súmernosti.
Os súmernosti trojuholníka $ABC$ je osou úsečky~$AB$.
Táto priamka pretína zadanú kružnicu v~dvoch bodoch, čo sú dve možné riešenia úlohy, ktoré označíme $C_1$ a~$C_2$.

b) Strana~$AB$ je ramenom rovnoramenného trojuholníka $ABC$.
V~tomto prípade môže byť hlavným vrcholom trojuholníka $ABC$ buď bod~$A$, alebo bod~$B$.
Ak by hlavným vrcholom bol bod~$A$, tak by strana~$AC$ bola ramenom a~bod~$C$ by bol od bodu~$A$ rovnako vzdialený ako bod~$B$.
Bod~$C$ by teda ležal na kružnici so stredom v~bode~$A$ prechádzajúcej bodom~$B$.
Táto kružnica pretína zadanú kružnicu v~jednom ďalšom bode, ktorý označíme~$C_3$.

Podobne, ak by hlavným vrcholom bol bod~$B$, tak by zvyšný vrchol trojuholníka ležal na kružnici so stredom v~bode~$B$ prechádzajúcej bodom~$A$.
Zodpovedajúci priesečník so zadanou kružnicou označíme~$C_4$.

\smallskip
Na zadanej kružnici ležia štyri body $C_1$, $C_2$, $C_3$, $C_4$ vyhovujúce podmienkam zo zadania.
Konštrukcia všetkých bodov vyplýva z~predchádzajúceho opisu:
body $C_3$ a~$C_4$ sú priesečníky danej kružnice s~dvoma pomocnými kružnicami;
os úsečky~$AB$, a~teda body $C_1$ a~$C_2$, je určená spoločnými bodmi týchto dvoch pomocných kružníc.
\figure{z7-I-6}%


\poznamka
Z~pravouhlosti trojuholníka $ABS$ vyplýva, že body $A$, $B$, $C_3$, $C_4$ tvoria vrcholy štvorca.
Z~toho možno vyvodiť alternatívne konštrukcie prislúchajúcich bodov.

}

{%%%%%   Z8-I-1
\napad
Môže mať Fero (či Dávid) 8, 15 alebo 47 rokov?

\riesenie
Číslo 238 možno rozložiť na súčin dvoch čísel nasledujúcimi spôsobmi:
$$
238=1\cdot238=2\cdot119=7\cdot34=14\cdot17.
$$
Medzi týmito dvojicami sú súčasné veky Fera a~Dávida.
Po pričítaní 4 ku každému z~nich máme dostať súčin 378.
Preberme všetky možnosti:
$$
\align
(1+4)\cdot(238+4)&=1210, \\
(2+4)\cdot(119+4)&=738, \\
(7+4)\cdot(34+4)&=418, \\
(14+4)\cdot(17+4)&=378.
\endalign
$$
Jediná vyhovujúca možnosť je tá posledná -- jeden z~chlapcov má 14~rokov, druhý~17.
Súčet súčasných vekov Fera a~Dávida je 31~rokov.

\poznamka
Alternatívne možno vyhovujúcu dvojicu nájsť pomocou rozkladov 378, ktoré sú:
$$
378=1\cdot378=2\cdot189=3\cdot126=6\cdot63=7\cdot54=9\cdot42=14\cdot27=18\cdot21.
$$
Jediné dvojice v~rozkladoch 238 a~378, v~ktorých sú oba činitele druhého rozkladu o~4 väčšie ako pri prvom, sú 14 a~17, resp. 18 a~21.

Extrémne hodnoty v~uvedených rozkladoch možno vylúčiť ako nereálne.
Ak riešiteľ s~odkazom na túto skutočnosť nepreberie všetky možnosti, považujte jeho postup za správny.

\ineriesenie
Ak súčasný vek Fera označíme $f$ a~súčasný vek Dávida označíme $d$, tak informácie zo zadania znamenajú
$$
f\cdot d=238
\quad\text{a}\quad
(f+4)\cdot(d+4)=378.
$$
Po roznásobení ľavej strany v~druhej podmienke a~dosadením prvej podmienky dostávame
$$
\align
238+4f+4d+16&=378, \\
4(f+d)&=124, \\
f+d&=31.
\endalign
$$
Súčet súčasných vekov Fera a~Dávida je 31~rokov.
}

{%%%%%   Z8-I-2
\napad
Zaoberajte sa každým jazykom zvlášť.

\riesenie
Keby oným jazykom, ktorý nový žiak ovládal, bola francúzština, tak by všetci traja spolužiaci hádali nesprávne.

Keby oným jazykom bola španielčina, tak by všetci traja spolužiaci hádali správne.

Keby oným jazykom bola nemčina, tak by prvý dvaja spolužiaci hádali správne a~tretí nesprávne.
Toto je jediný prípad, keď aspoň jeden spolužiak háda správne a~aspoň jeden nesprávne.
Nový žiak preto ovláda nemčinu.
}

{%%%%%   Z8-I-3
\napad
Hľadajte (a~zdôvodnite) navzájom zhodné časti obrázka.

\riesenie
Všetky kružnice na predchádzajúcom obrázku sú navzájom zhodné.
Preto aj vyznačené úsečky na nasledujúcom obrázku sú navzájom zhodné a~rovnako tak im zodpovedajúce oblúky.
\figure{z8-I-3a}%


Dva oblúky tvoriace okraj každého lupeňa majú dvojnásobnú dĺžku ako príslušný oblúk pôvodnej kružnice, nad ktorým bol vytvorený.
Obvod kvietka je preto dvojnásobkom dĺžky pôvodnej kružnice, \tj. 32\,cm.
}

{%%%%%   Z8-I-4
\napad
Vyjadrite výsledok všeobecne pomocou neznámych cifier na kartičkách.

\riesenie
Označme tri nenulové cifry na kartičkách $a$, $b$, $c$ a~predpokladajme, že platí $a<b<c$.
Vojtove čísla postupne boli $\overline{cba0}$, $\overline{cba}$, $\overline{cb}$ a~$c$,
Martinove čísla boli $\overline{a0bc}$, $\overline{a0b}$, $\overline{a0}$ a~$a$.
Adamov súčet rozdielov dvojíc týchto čísel možno v~rozvinutom tvare, po úprave, vyjadriť ako
$1110c+100b-1100a$.
Podľa zadania je tento súčet rovný $9090$, čo je ekvivalentné rovnici
$$
111c+10b-110a=909, \tag{1}
$$
pričom $0<a<b<c<10$.
Keďže na mieste jednotiek prispieva iba prvý sčítanec, musí byť $c=9$.
Po dosadení a~úprave dostávame
$$
11a-b=9,
$$
pričom $0<a<b<9$.
Jediným riešením tejto úlohy je dvojica $a=1$ a~$b=2$.
(Keby $a\ge2$, muselo by byť $b\ge13$, čo nie je možné.)

Na kartičkách boli cifry 0, 1, 2 a~9.

\poznamka
Keby sme sa pri Adamovom súčte sústredili na hodnoty na mieste jednotiek, desiatok atď., dostaneme namiesto rovnice (1) ekvivalentnú rovnicu
$$
100(c-a)+10(b+c-a)+c=909.
$$
Z~toho vyplýva, že $c=9$ a~ďalej buď $b+c-a=0$, $c-a=9$, alebo $b+c-a=10$, $c-a=8$.
Keďže $a$ a~$c$ sú rôzne nenulové čísla, je rozdiel $c-a$ rôzny od $c$.
Preto musí nastať druhá možnosť a~postupným dosadzovaním určíme $a=1$ a~$b=2$.
}

{%%%%%   Z8-I-5
\napad
Môže Václav rozvrhnúť prestávky tak, aby plne využil svoju oddýchnutosť?

\riesenie
Václav musí spraviť práve šesť polhodinových prestávok.
Tým stratí $6\cdot 1/2=3$ hodiny.
Prestávky môže rozvrhnúť tak, že na začiatku práce a~po každej prestávke môže pracovať celú hodinu zvýšeným tempom (napr. pri rovnomernom rozdelení).
Tým získa $7\cdot 1/4$ hodín, \tj. 1 hodinu a~45 minút.

Bez prestávok normálnym tempom by Václav pracoval 26~hodín.
So všetkými povinnými prestávkami a~všetkými dovolenými zvýšeniami tempa by Václav pracoval $26+3-7/4$ hodín, \tj. 27~hodín a~15~minút.
To je najkratšia doba, za ktorú môže svoju úlohu splniť.
}

{%%%%%   Z8-I-6
\napad
Pozmeňte útvary tak, aby sa vám ľahšie porovnávali ich obsahy.

\riesenie
Úsečka~$NP$ rozdeľuje lichobežník $KLMN$ na trojuholník $KPN$ a~lichobežník $PLMN$.
Lichobežník $KLMN$ má rovnaký obsah ako trojuholník $KON$, pričom bod~$O$ na priamke~$KL$ je obrazom bodu~$N$ vzhľadom na stredovú súmernosť podľa stredu úsečky~$LM$.
Podobne, lichobežník $PLMN$ má rovnaký obsah ako trojuholník $PON$.
\figure{z8-I-6}%


Úsečka~$NP$ rozdeľuje lichobežník $KLMN$ na dve časti s~rovnakým obsahom práve vtedy, keď $P$ je stredom úsečky~$KO$.
Vzhľadom na to, že $|KO|=|KL|+|LO|$ a~$|LO|=|MN|$, platí
$$
|KP| =\frac{40+16}2=28\,(\Cm).
$$

\poznamka
V~úvodnej časti predchádzajúceho riešenia je naznačené odvodenie známeho vzorca $S=\frac{(a+c)v}2$ pre obsah lichobežníka so základňami $a$ a~$c$ a~výškou~$v$.
Dosadením možno požiadavku zo zadania vyjadriť pomocou rovnice
$$
\frac{x\cdot v}2=\frac{(40+16-x)v}2,
$$
pričom $x=|KP|$.


}

{%%%%%   Z9-I-1
\napad
Môžu byť obe neznáme súčasne väčšie ako napr. 14?

\riesenie
Pre rovnaké neznáme je riešením zrejme $x=y=8$.

Rôzne neznáme nemôžu byť obe súčasne menšie, resp. väčšie ako 8 (potom by totiž ľavá strana rovnice bola väčšia, resp. menšia ako $\frac14$).
Jedna neznáma teda musí byť menšia ako 8 a~druhá väčšia ako 8.
Vzhľadom na~symetrickosť rovnice stačí ďalej uvažovať prípad, keď $x<y$.
Za tohto predpokladu je $x<8$ a~$y>8$,
takže $x$ môže nadobúdať iba hodnoty od 1 do 7.

Ak z~rovnice vyjadríme $y$ všeobecne pomocou $x$, dostaneme
$$
y=\frac{4x}{x-4}. \tag{1}
$$
Z toho je zrejmé, že $y$ je kladné práve vtedy, keď $x>4$.
Teda $x$ môže nadobúdať iba hodnoty od 5 do 7.
Pre tieto tri možnosti stačí preveriť, či $y$ vychádza celé:
\begin{itemize}
\item pre $x=5$ vychádza $y=20$,
\item pre $x=6$ vychádza $y=12$,
\item pre $x=7$ vychádza $y=\frac{28}{3}$.
\end{itemize}

Celkom tak vidíme, že všetky dvojice $(x,y)$, ktoré sú riešením úlohy, sú $(5,20)$, $(6,12)$, $(8,8)$, $(12,6)$ a~$(20,5)$.

\ineriesenie
Ak z~rovnice vyjadríme $y$ všeobecne pomocou $x$, dostaneme (1).
Tento výraz môžeme ďalej upraviť na \uv{celú časť plus zvyšok}:
$$
y=\frac{4x}{x-4}=\frac{4(x-4)+16}{x-4}=4+\frac{16}{x-4}. \tag{2}
$$
Z toho je zrejmé, že $y$ je kladné celé číslo práve vtedy, keď $x-4$ je kladným celým deliteľom čísla 16, a~tých je práve päť:
\begin{itemize}
\item pre $x-4=1$ vychádza $x=5$ a~$y=20$,
\item pre $x-4=2$ vychádza $x=6$ a~$y=12$,
\item pre $x-4=4$ vychádza $x=8$ a~$y=8$,
\item pre $x-4=8$ vychádza $x=12$ a~$y=6$,
\item pre $x-4=16$ vychádza $x=20$ a~$y=5$.
\end{itemize}
Tento výpis zahŕňa všetky riešenia úlohy.

\poznamka
Pri úprave zadanej rovnice na tvar (1) možno naraziť na ekvivalentnú rovnicu
$$
xy-4x-4y=0. \tag{3}
$$
Ľavá strana súhlasí s~tromi sčítancami v~roznásobení výrazu $(x-4)(y-4)$, chýba iba~16.
Pričítaním 16 k~obom stranám rovnice (3) dostávame
$$
(x-4)(y-4)=16.
$$
Všetky riešenia tak možno nájsť pomocou všetkých možných rozkladov čísla 16 na súčin dvoch kladných celých čísel.
Tieto nápady sú len bezzlomkovou interpretáciou úpravy~(2) a~následného postupu.
}

{%%%%%   Z9-I-2
\napad
Zadané body vymedzujú ďalšie trojuholníky.
Zamyslite sa nad obsahmi niektorých z~nich.

\riesenie
Obsah trojuholníka $KLM$ vyjadríme ako rozdiel obsahu trojuholníka $ABC$ a~obsahov trojuholníkov $AKM$, $KBL$ a~$MLC$.
\figure{z9-I-2a}%


Pomer vzdialeností bodov $M$ a~$C$ od bodu~$A$ je rovnaký ako pomer vzdialeností týchto bodov od priamky~$AB$.
Výška trojuholníka $AKM$ idúca vrcholom~$M$ je teda tretinová vzhľadom k~výške trojuholníka $ABC$ idúcej vrcholom~$C$.
Súčasne strana~$AK$ prvého trojuholníka protiľahlá vrcholu~$M$ je polovičná vzhľadom k~strane~$AB$ druhého trojuholníka protiľahlej vrcholu~$C$.
Trojuholník $AKM$ preto zaberá $\frac13\cdot\frac12=\frac16$ obsahu trojuholníka $ABC$.

Podobne, výška trojuholníka $KBL$ idúca vrcholom~$L$ je dvojtretinová vzhľadom k~výške trojuholníka $ABC$ idúcej vrcholom~$C$ a~príslušná strana~$KB$ prvého trojuholníka je polovičná vzhľadom k~strane~$AB$ druhého trojuholníka.
Trojuholník $KBL$ preto zaberá $\frac23\cdot\frac12=\frac13$ obsahu trojuholníka $ABC$.

Do tretice, výška trojuholníka $CML$ idúca vrcholom~$L$ je tretinová vzhľadom k~výške trojuholníka $ABC$ idúcej vrcholom~$B$ a~príslušná strana~$CM$ prvého trojuholníka je dvojtretinová vzhľadom k~strane~$AC$ druhého trojuholníka.
Trojuholník $KBL$ preto zaberá $\frac13\cdot\frac23=\frac29$ obsahu trojuholníka $ABC$.

Dohromady, trojuholník $KLM$ zaberá
$$
1-\frac16-\frac13-\frac29=\frac{5}{18}
$$
obsahu trojuholníka $ABC$.

\poznamka
Všimnite si, že uvedené riešenie je platné pre všeobecný trojuholník $ABC$.

S~predpokladom rovnostrannosti sú strany a~výšky jednotlivých trojuholníkov naznačené v~nasledujúcom obrázku ($a$ označuje stranu a~$v$ výšku trojuholníka $ABC$).
\figure{z9-I-2b}%


S~týmto označením je $S_{ABC}=\frac12av$ a~$S_{KLM}=\frac{5}{36}av$, teda $S_{KLM}=\frac5{18}S_{ABC}$.

}

{%%%%%   Z9-I-3
\napad
Uvažujte všetky možnosti otvorenosti, resp. zatvorenosti kín bez obmedzujúcich podmienok.

\riesenie
Zaujímame sa o~situáciu, keď je južné kino zatvorené.
Bez ďalších informácií o~otváracích hodinách by mohli nastať nasledujúce štyri prípady, ktoré postupne porovnáme s~podmienkami zo zadania:
\begin{enumerate}\alphatrue
\item Ak by severné aj východné kino bolo otvorené, tak by sme boli v~rozpore s~treťou podmienkou.
Táto situácia nenastane.
\item Ak by severné kino bolo otvorené a~východné zatvorené, tak nie sme v~rozpore so~žiadnou z~podmienok.
Táto situácia je možná.
\item Ak by severné kino bolo zatvorené a~východné otvorené, tak by sme boli v~rozpore so štvrtou podmienkou.
Táto situácia nenastane.
\item Ak by severné aj východné kino bolo zatvorené, tak by sme boli v~rozpore s~prvou podmienkou.
Táto situácia nenastane.
\end{enumerate}

Jediná situácia vyhovujúca všetkým podmienkam zo zadania je b):
keď je južné kino zatvorené, tak je severné kino určite otvorené.

\napadd
Aké možnosti vyhovujú tretej podmienke?

\ineriesenie
Vzhľadom na tretiu podmienku môžeme štvrtú podmienku nahradiť nasledujúcou:

\item{4'.} ak je otvorené východné kino, tak je otvorené aj južné kino.

\smallskip\noindent
Podľa tretej podmienky môžeme rozlíšiť tri prípady:
\begin{enumerate}\alphatrue
\item Severné kino je otvorené a~východné zatvorené.
Potom podľa druhej podmienky je južné kino zatvorené.
\item Severné kino je zatvorené a~východné otvorené.
Potom podľa štvrtej podmienky (resp. jej práve uvedenej náhrady) je južné kino otvorené.
\item Severné aj východné kino je zatvorené.
Potom podľa prvej podmienky je južné kino otvorené.
\end{enumerate}

V~žiadnom z~týchto prípadov nie sme v~rozpore s~podmienkami, ktoré sme pri vyvodzovaní nepoužili.
Všetky možné situácie otvorenosti (1), resp. zatvorenosti (0) kín uvádzame kvôli prehľadnosti v~tabuľke:
$$\begintable
S\|1|0|0\cr
V\|0|1|0\cr
J\|0|1|1\endtable
$$

Ak je južné kino zatvorené, tak je severné kino určite otvorené (a~východné určite zatvorené).

\poznamka
Diskusia v~riešení úlohy môže byť vedená rôznymi spôsobmi.
Pre tieto účely znázorníme možnosti, ktoré pripúšťajú podmienky zo zadania samostatne -- prázdne políčka môžu obsahovať ako 1, tak~0:

Prvá podmienka pripúšťa možnosti:
$$\begintable
S\|1||\cr
V\||1|\cr
J\|||1\endtable
$$

Druhá podmienka pripúšťa možnosti:
$$\begintable
S\|0|\cr
V\||\cr
J\|1|0\endtable
$$

Tretia podmienka pripúšťa možnosti:
$$\begintable
S\|1|0|0\cr
V\|0|1|0\cr
J\|||\endtable
$$

Štvrtá podmienka pripúšťa možnosti:
$$\begintable
S\||1|1|\cr
V\|1|1|1|0\cr
J\|1||1|\endtable
$$

Výber možností, ktoré vyhovujú všetkým štyrom podmienkam súčasne (teda \uv{prienik} týchto štyroch tabuliek), vedie k~tabuľke na konci predchádzajúceho riešenia.
}

{%%%%%   Z9-I-4
\napad
Najskôr riešte úlohu bez informácie, že za pôvodne plánovanú sumu možno pokryť o~deväť stoličiek viac.

\riesenie
V~akcii bola každá štvrtá stolička za polovičnú cenu.
Ak mohol takto hotelier ušetriť za 7,5 stoličky, objednával 15~štvoríc stoličiek a~nanajvýš tri ďalšie stoličky, \tj. objednával najmenej 60 a~nanajvýš 63 stoličiek.

Oproti pôvodnému plánu si v~akcii mohol dopriať o~9 stoličiek viac, \tj. najmenej 69 a~nanajvýš 72 stoličiek.
V~prvom prípade je $69=17\cdot4+1$, teda úspora by predstavovala len $17\cdot\frac12=8{,}5$ stoličiek, čo nezodpovedá predpokladu.
Rovnaký záver platí aj pre ďalšie dve možnosti $70=17\cdot4+2$ a~$71=17\cdot4+3$.
Jedine pre $72=18\cdot4$ zodpovedá úspora práve 9 stoličkám.

Zo štyroch zvažovaných možností je iba posledná riešením úlohy.
Hotelier chcel pôvodne kúpiť 63~stoličiek.

\ineriesenie
Rovnako ako v~predchádzajúcom riešení určíme, že úspora za 7,5 stoličky zodpovedá objednávke najmenej 60 a~nanajvýš 63 stoličiek
($7{,}5=\frac12\cdot15$ a~$15\cdot4=60$).

Podobne určíme, že úspora za 9~stoličiek zodpovedá objednávke najmenej 72 a~nanajvýš 75 stoličiek
($9=\frac12\cdot18$ a~$18\cdot4=72$).
Tento počet má byť zároveň o~9 väčší ako počet stoličiek v~predchádzajúcej objednávke, teda úspora za 9 stoličiek má zodpovedať objednávke najmenej 69 a~nanajvýš 72 stoličiek.
Tieto dve podmienky sú splnené iba v~jedinom prípade, ktorý zodpovedá počtu 72 v~druhej, resp. 63 v~prvej objednávke.

Hotelier chcel pôvodne kúpiť 63 stoličiek.
}

{%%%%%   Z9-I-5
\napad
V~akom vzťahu sú stredy dvoch dotýkajúcich sa kružníc a~príslušný dotykový bod?

\riesenie
V~nasledujúcom budeme opakovane používať poznatok, že spoločný bod dvoch dotýkajúcich sa kružníc leží na spojnici ich stredov.
Polomer bielych kruhov budeme označovať~$r$.

Stredy Adamových bielych kruhov tvoria vrcholy štvorca a~stred červeného kruhu leží v~strede tohto štvorca, teda v~priesečníku jeho uhlopriečok.
Priemer červeného kruhu je rovný rozdielu uhlopriečky uvedeného štvorca a~dvoch polomerov~$r$.
\figure{z9-I-5a}%


Pomocný štvorec má stranu~$2r$, jeho uhlopriečka má podľa Pytagorovej vety veľkosť
$$
\sqrt{4r^2+4r^2}=2\sqrt2r.
$$
Polomer červeného kruhu je teda rovný
$$
\frac12(2\sqrt2r-2r)=(\sqrt2-1)r.
$$

Stredy Eviných bielych kruhov tvoria vrcholy rovnostranného trojuholníka a~stred zeleného kruhu leží v~strede tohto trojuholníka, teda v~priesečníku jeho výšok, resp. ťažníc.
Polomer zeleného kruhu je rovný rozdielu vzdialenosti stredu, \tj. ťažiska uvedeného trojuholníka, od jeho vrcholu a~polomeru~$r$.
\figure{z9-I-5b}%


Pomocný rovnostranný trojuholník má stranu~$2r$
a~každá výška ho delí na pravouhlé trojuholníky s~preponou~$2r$ a~odvesnou~$r$.
Druhá odvesna, teda táto výška, má podľa Pytagorovej vety veľkosť
$$
\sqrt{4r^2-r^2}=\sqrt3r.
$$
Vzdialenosť ťažiska rovnostranného trojuholníka od vrcholu je rovná dvom tretinám dĺžky ťažnice, \tj. práve vyjadrenej výšky.
Polomer zeleného kruhu je teda rovný
$$
\frac23\sqrt3r-r=\frac{2\sqrt3-3}{3}r.
$$

}

{%%%%%   Z9-I-6
Všetky dvojciferné prvočísla sú vypísané v~prvom riadku nasledujúcej tabuľky.
V~druhom riadku sú uvedené ciferné súčiny jednotlivých čísel.
V~treťom riadku sú najmenšie prirodzené čísla so zodpovedajúcimi cifernými súčinmi
(tieto čísla možno určiť porovnaním rozkladov so všetkými deliteľmi menšími ako 10).
Dvojciferných bombastických prvočísel je sedem a~v~tabuľke sú vyznačené tučne.
\bgroup
\def\ctr#1{\hfil\ #1\ }
$$
\begintable
11|13|17|19|23|\bf29|31|\bf37|41|43|\bf47|53|\bf59|61|\bf67|71|73|\bf79|83|\bf89|97\crthick
1| 3| 7| 9| 6|18| 3|21| 4|12|28|15|45| 6|42| 7|21|63|24|72|63\cr
1| 3| 7| 9| 6|29| 3|37| 4|26|47|35|59| 6|67| 7|37|79|38|89|79\endtable
$$
\egroup

V~predchádzajúcom výpise si môžeme všimnúť niekoľko vecí súvisiacich s~druhou časťou úlohy.
Číslo~23 nie je bombastické, pretože 6 je menšie číslo s~rovnakým ciferným súčinom.
Všeobecnejšie, žiadne číslo obsahujúce cifry 2 a~3 nemôže byť bombastické, lebo vynechaním týchto dvoch cifier a~doplnením 6 na ľubovoľné miesto dostaneme menšie číslo s~rovnakým ciferným súčinom
(napr. pre $\underline27\underline37$ je jedno z~takých čísel $\underline677$).

Podobne, číslo~34 nie je bombastické, pretože 26 je menšie číslo s~rovnakým ciferným súčinom.
Teda ani žiadne číslo obsahujúce cifry 3 a~4 nemôže byť bombastické (pozri napr. čísla $\underline38\underline4$ a~$\underline28\underline6$).
Do tretice, číslo 36 nie je bombastické, pretože 29 je menšie číslo s~rovnakým ciferným súčinom.
Teda ani žiadne číslo obsahujúce cifry 3 a~6 nemôže byť bombastické (pozri napr. čísla $2\underline34\underline6$ a~$2\underline{29}4$).

Zato vidíme, že číslo~38 bombastické je.
Jediná párna cifra, ktorá môže byť s~3 v~Karolovom bombastickom čísle je teda 8.



\poznamka
V~tabuľke v~prvej časti úlohy nebolo nutné uvažovať čísla obsahujúce cifru~1 ani čísla, ktoré majú na mieste desiatok väčšiu cifru ako na mieste jednotiek -- také čísla nikdy nie sú bombastické.
S~týmto postrehom stačilo testovať iba osem z~uvedených čísel.
}

{%%%%%   Z4-II-1
...}

{%%%%%   Z4-II-2
...}

{%%%%%   Z4-II-3
...}

{%%%%%   Z5-II-1
Všetkých chrtov bez Dunča bolo $36-1=35$.
Pätina z~tohto počtu je ${35:5}=7$; pred Dunčom dobehlo 7~psov,
za Dunčom dobehlo $7\cdot4=28$ psov.
Dunčo dobehol ôsmy.

\hodnotenie
3~body za rozdelenie 35 psov bez Dunča na pätiny;
3~body za rozdelenie psov pred/za Dunčom a~umiestnenie Dunča.
\endhodnotenie
}

{%%%%%   Z5-II-2
V~každej štvorici cifier, ktorú mohol Filip vybrať z~Agátinho čísla, sú navzájom rôzne cifry.
Výsledné dve čísla preto nemohli mať rovnaké cifry na mieste desiatok (a~líšiť sa o~1 na mieste jednotiek).
Tieto čísla preto museli byť tvaru $*9$ a~$*0$,
pričom cifry na mieste desiatok sa líšili o~1.

V~Agátinom čísle sú všetky možné štvorice po sebe idúcich cifier, medzi ktorými je 9 a~0, tieto:
$$
7890,\quad 8901,\quad 9012.
$$

Z~prvej štvorice mohol Filip zostaviť čísla 79 a~80, z~druhej štvorice nemohol zostaviť nič,
z tretej štvorice mohol zostaviť čísla 19 a~20.

\hodnotenie
Po 2~bodoch za každú vyhovujúcu dvojicu čísel; 2~body za kvalitu komentára.
\endhodnotenie
}

{%%%%%   Z5-II-3
Podľa zadania postupne znázorníme trasu Aničky, trasu Vojta a~ich vzájomný vzťah
(každý vyznačený dielik predstavuje 1~km):
\insp{Z5-II-3b.eps}%

Teraz vidíme, že Anička to má z~hotela k~rybníku najmenej 4~km, a~to napr. 2~km na sever a~2~km na západ.
Vojto to má z~kempu k~rybníku najmenej 3~km, a~to napr. 2~km na juh a~1~km na východ.

\hodnotenie
Po 1~bode za znázornenie trás Aničky a~Vojta;
2~body za ich vzájomný vzťah;
po 1~bode za najkratšie cesty Aničky a~Vojta k~rybníku.

\odst{Poznámka k~druhej časti}
Trasy rovnakej dĺžky medzi dvoma miestami nie sú podľa uvedených pravidiel jednoznačné a~možno ich vymyslieť mnohými spôsobmi
(napr. Anička mohla svoje 4~km z~hotela k~rybníku ísť aj takto: 1,5~km na západ, 2~km na sever a~0,5~km na západ).
Taký rozbor od riešiteľov neočakávame, súčasťou úplného riešenia ale musí byť nejaká konkrétna realizácia.
\endhodnotenie
}

{%%%%%   Z6-II-1
Pri riešení budeme opakovane odkazovať na vlastnosti súčtov a~súčinov celých čísel vzhľadom na ich paritu,
ktoré zhŕňame v~nasledujúcej schéme:
$$
\thinsize=0pt
\thicksize=0pt
%\def\ctr#1{\hfil\quad#1\quad}
\begintable
párne $+$ párne $=$ párne,\quad
párne $+$ nepárne $=$ nepárne,\quad
nepárne $+$ nepárne $=$ párne,\cr
párne $\times$ párne $=$ párne,\quad
párne $\times$ nepárne $=$ párne,\quad
nepárne $\times$ nepárne $=$ nepárne.\endtable
$$

Z~toho a~z~tretej podmienky zo zadania sa dozvedáme, že súčin Danovho a~Haninho čísla je nepárny, a~teda obe tieto čísla sú nepárne.

Z~druhej podmienky vyplýva, že rozdiel Anninho a~Haninho čísla je nepárny.
Keďže Hanino číslo je nepárne, musí byť Annino číslo párne.

Keďže Danovo číslo je nepárne, vyplýva z~prvej podmienky, že trojnásobok Janovho čísla je párny, a~teda Janovo číslo je tiež párne.

Danovo a~Hanino obľúbené číslo je nepárne, Annino a~Janovo číslo je párne.

\hodnotenie
Po 1~bode za určenie parity každého čísla;
2~body za kvalitu komentára.
\endhodnotenie
}

{%%%%%   Z6-II-2
Vyjdeme z~požiadavky, že štvrtina dvojeurových mincí dáva rovnakú sumu ako tretina jednoeurových mincí.
Z~toho jednak vyplýva, že počet dvojeurových mincí je násobkom štyroch a~počet jednoeurových mincí je násobkom troch, a~jednak, že tieto počty sú v~pomere $4:6$.
Teda počet dvojeuroviek je rovný~$4k$ a~počet jednoeuroviek~$6k$, pričom $k$ je nejaké kladné celé číslo ($2\cdot\frac{4k}4=\frac{6k}3$).

Pri tomto označení je celkový počet mincí rovný $10k=290$.
Z~toho vyplýva, že $k=29$, teda že Anička má $4\cdot 29=116$ dvojeurových mincí a~$6\cdot 29=174$ jednoeurových mincí.
Celková hodnota jej úspor teda je
$$
2\cdot 116+174 =406\ \text{eur}.
$$

\hodnotenie
4~body za určenie vzťahov medzi počtami jednoeurových a~dvojeurových mincí (v~našom prípade pomocou neznámej~$k$);
2~body za určenie~$k$ a~výsledok.

\poznamka
Pomocou neznámej~$k$ je celková hodnota Aničkiných úspor vyjadrená ako $({2\cdot4}+6)\cdot k=14\cdot k$.
\endhodnotenie
}

{%%%%%   Z6-II-3
Sivú časť si môžeme predstaviť ako pravouhlý trojuholník s~odvesnami rovnými dvom a~trom stranám štvorca, z~ktorého je odobraný jeden štvorec.
Obsah tejto časti je preto rovnaký ako obsah 2~štvorcov ($\frac12\cdot2\cdot3-1=2$).

Bielu časť tvorí trojuholník, ktorého jedna strana je rovná strane štvorca a~výška na túto stranu je rovná trom stranám štvorca.
Obsah tejto časti je teda rovnaký ako obsah $\frac32$ štvorca ($\frac12\cdot1\cdot3=\frac32$).

Rozdiel obsahov sivej a~bielej časti zodpovedá polovici štvorca, čo je podľa zadania 0,6~cm$^2$.
Takže jeden štvorec má obsah 1,2 cm$^2$ a~obsah celého útvaru je 6 cm$^2$ ($5\cdot1{,}2=6$).

\hodnotenie
Po 2~bodoch za vyjadrenie obsahov sivej a~bielej časti;
2~body za dopočítanie obsahu v~cm$^2$.

\endhodnotenie
}

{%%%%%   Z7-II-1
Zuzka vypočítala dvakrát viac úloh ako Majka.
Teda počet úloh, ktoré celkom vypočítali Zuzka a~Vašo, je rovnaký ako počet úloh, ktoré celkom vypočítali Majka a~Vašo, zväčšený o~počet úloh, ktoré vypočítala Majka.

Zuzka a~Vašo vypočítali celkom 32~úloh, Majka a~Vašo vypočítali celkom 25~úloh, teda Majka vypočítala 7~úloh ($32-25=7$).
Z~toho dostávame, že Vašo vypočítal 18~úloh ($25-7=18$).

\hodnotenie
3~body za úvodnú úvahu;
3~body za dopočítanie a~záver.

\poznamka
Ak $m$, $v$, resp. $z$ sú počty úloh, ktoré vypočítali Majka, Vašo, resp. Zuzka, tak zo zadania máme
$m+v=25$, $z+v=32$ a~$z=2m$.
Predchádzajúce myšlienky tak možno stručne zapísať nasledovne:
$$
32=2m+v=m+m+v=m+25,
$$
teda $m=7$ a~$v=18$.
\endhodnotenie
}

{%%%%%   Z7-II-2
Aby nové číslo pozostávalo zo šiestich rôznych cifier, môžeme vkladať dve rôzne cifry
z~cifier
$$
3,\ 4,\ 5,\ 6,\ 7,\ 8.
$$

Súčet cifier čísla 2019 je 12, teda číslo deliteľné tromi.
Aby bolo aj~novo vzniknuté číslo deliteľné tromi, môžeme vkladať len také cifry, ktorých súčet je deliteľný tromi.
Posledná podmienka zo zadania navyše vyžaduje, aby tento súčet bol nepárny.
Zo všetkých možných dvojíc čísel tak môžeme použiť len nasledujúce (v~ľubovoľnom poradí):
$$
3, 6,\quad 4, 5,\quad 7, 8.
$$

Aby nové číslo začínalo 2 a~končilo 9, môžeme cifry vkladať len do miest vyznačených hviezdičkou:
$$
2{*}{*}019,\quad 2{*}0{*}19,\quad 2{*}01{*}9,\quad 20{*}{*}19,\quad 20{*}1{*}9,\quad 201{*}{*}9.
$$

Aby prvé trojčíslie bolo deliteľné tromi, nemôžeme v~prvom prípade doplniť žiadnu z~uvedených dvojíc,
v~druhom až piatom prípade môžeme doplniť dvojicu 4, 5 alebo 7, 8 (v~tomto poradí) a~v~poslednom prípade môžeme doplniť ktorúkoľvek z~uvedených dvojíc (v~ľubovoľnom poradí):
$$
\gather
2\underline40\underline519,\quad 2\underline70\underline819,\quad
2\underline401\underline59,\quad 2\underline701\underline89,\quad
20\underline41\underline59,\quad 20\underline71\underline89,\\
201\underline3\underline69,\quad 201\underline6\underline39,\quad
201\underline4\underline59,\quad 201\underline5\underline49,\quad
201\underline7\underline89,\quad 201\underline8\underline79.
\endgather
$$

Aby prvé štvorčíslie bolo deliteľné štyrmi, musí byť druhé dvojčíslie deliteľné štyrmi.
Z~uvedených možností tak nakoniec ostávajú iba dve
$$
2\underline70\underline819,\quad 201\underline6\underline39.
$$

Využili sme všetky požiadavky zo zadania; opytovaný rozdiel je rovný 69180.

\hodnotenie
3~body za vyhovujúce možnosti a~konečný rozdiel;
3~body za kvalitu, resp. úplnosť komentára.
\endhodnotenie}

{%%%%%   Z7-II-3
V~obrázku zvýrazníme osem zhodných štvorcov a~pomocou nich vyjadríme obsahy jednotlivých častí:
\insp{z7-II-3a.eps}%

Biela časť tvorí polovicu štvorca.

Tmavosivú časť môžeme rozdeliť na štvorec a~pravouhlý trojuholník $ZUL$, ktorého odvesny sú rovné jednej a~dvom stranám štvorca.
Tento trojuholník má rovnaký obsah ako štvorec ($\frac12\cdot1\cdot2=1$), teda obsah tmavosivej časti je rovnaký ako obsah 2~štvorcov.

Čiernu časť môžeme rozdeliť na tri pravouhlé trojuholníky:
trojuholníky $ZEL$ a~$EKX$ sú zhodné s~vyššie spomínaným trojuholníkom $ZUL$,
trojuholník $ZDC$ tvorí polovicu štvorca.
Obsah čiernej časti je teda rovnaký ako obsah $\frac52$ štvorca ($1+1+\frac12=\frac52$).

Svetlosivú časť môžeme rozdeliť na štvorec a~trojuholník $GXY$.
Body $X$ a~$E$ sú vrcholy štvorcov a~bod~$Y$ je stredom úsečky~$EX$, ktorú interpretujeme ako uhlopriečku obdĺžnika $KEVX$.
Úsečka~$HY$ je preto zhodná s~polovicou strany štvorca.
Táto úsečka je výškou trojuholníka $GXY$ na stranu~$GX$, a~tá je rovná trom stranám štvorca.
Trojuholník~$GXY$ má rovnaký obsah ako $\frac34$ štvorca ($\frac12\cdot\frac12\cdot3=\frac34$), teda obsah svetlosivej časti je rovnaký ako obsah $\frac74$ štvorca ($\frac34+1=\frac74$).

Zvyšnú časť môžeme vyjadriť ako rozdiel vyššie uvedených častí,
$$
8-2-\frac12-\frac52-\frac74=3-\frac74=\frac54,
$$
teda obsah šrafovanej časti je rovnaký ako obsah $\frac54$ štvorca.

Teraz využijeme poznatok, že obsah čiernej časti je 7,5 cm$^2$.
To podľa predchádzajúceho zodpovedá $\frac52$ štvorca, teda obsah jedného štvorca je 3 cm$^2$ ($\frac52\cdot3=7{,}5$).
Z~toho uzatvárame, že
obsah bielej časti je 1,5 cm$^2$ ($\frac12\cdot3=1{,}5$),
obsah tmavosivej časti je 6~cm$^2$ ($2\cdot3=6$),
obsah svetlosivej časti je 5,25 cm$^2$ ($\frac74\cdot3=5{,}25$),
a~obsah šrafovanej časti je 3,75 cm$^2$ ($\frac54\cdot3=3{,}75$).

\hodnotenie
4 body za vyjadrenie obsahov všetkých častí pomocou štvorcov;
2 body za dopočítanie obsahov v~cm$^2$.
\endhodnotenie}

{%%%%%   Z8-II-1
Ľudia, ktorí bývajú nad niekým, sú obyvateľmi 2. a~1. poschodia.
Ľudia, ktorí bývajú pod niekým, sú obyvateľmi 1. poschodia a~prízemia.
V~súčte $35+45=80$ sú tak obyvatelia 1.~poschodia započítaní dvakrát.

Ak počet obyvateľov 1. poschodia označíme~$p$, tak počet všetkých obyvateľov v~dome môžeme vyjadriť jednak $80-p$, jednak $3p$.
Z~toho dostávame rovnicu, ktorú ľahko vyriešime:
$$
\aligned
3p&=80-p, \\
4p&=80, \\
p&=20.
\endaligned
$$
V~dome býva celkom 60 ľudí.

\hodnotenie
2~body za postreh, že obyvatelia 1. poschodia sú v~súčte $35+45$ započítaní dvakrát;
2~body za zostavenie a~vyriešenie rovnice;
2~body za počet osôb v~dome.

\poznamka
Ak $d$, $p$, resp. $z$ označuje počty obyvateľov na 2. poschodí, na~1. poschodí, resp. na prízemí, tak zo zadania máme
$$
d+p=35,\quad p+z=45,\quad d+p+z=3p.
$$
Z~toho možno rozličnými spôsobmi vyjadriť všetky neznáme: $d=15$, $p=20$ a~$z=25$.
Také riešenia hodnoťte
po 2~bodoch za zostavenie rovníc, za ich vyriešenie a~za záver.
\endhodnotenie

}

{%%%%%   Z8-II-2
Keď číslo nie je deliteľné 2, tak nie je deliteľné ani 4, 6 a~8.
Keď číslo nie je deliteľné 3, tak nie je deliteľné ani 6 a~9.
Keď číslo nie je deliteľné 4, tak nie je deliteľné ani 8.
Keď číslo nie je deliteľné 6, tak nie je deliteľné ani 2 a~3.
Žiadne z~čísel 2, 3, 4 a~6 teda nemôže byť oným jediným číslom z~uvedeného zoznamu, ktoré nie je deliteľom hľadaného čísla.

Číslo deliteľné všetkými číslami z~uvedeného zoznamu okrem 5 musí byť násobkom $7\cdot8\cdot9=504$, čo je najmenší spoločný násobok zvyšných čísel.
Kladné číslo menšie ako 1000 s~touto vlastnosťou je jediné, a~to 504.

Číslo deliteľné všetkými číslami z~uvedeného zoznamu okrem 7 musí byť násobkom $5\cdot8\cdot9=360$.
Kladné čísla menšie ako 1000 s~touto vlastnosťou sú dve, a~to 360 a~720.

Číslo deliteľné všetkými číslami z~uvedeného zoznamu okrem 8 musí byť násobkom $4\cdot5\cdot{7\cdot9}=1260$.
Kladné číslo menšie ako 1000 s~touto vlastnosťou nie je žiadne.

Číslo deliteľné všetkými číslami z~uvedeného zoznamu okrem 9 musí byť násobkom $3\cdot5\cdot{7\cdot8}=840$.
Kladné číslo menšie ako 1000 s~touto vlastnosťou je jediné, a~to 840.

Čísla s~uvedenými vlastnosťami sú práve štyri.

\hodnotenie
2~body za vyhovujúce štyri možnosti;
4~body za vylúčenie ostatných možností a~kvalitu komentára.
\endhodnotenie}

{%%%%%   Z8-II-3
Označme $S$ stred základne~$AB$.
Zo zadania vyplýva, že úsečky $AS$, $SB$ a~$CD$ sú zhodné, teda štvoruholníky $ASCD$ a~$SBCD$ sú rovnobežníky.
\insp{z8-II-3.eps}%

Uhlopriečka~$SD$ delí rovnobežník $ASCD$ na dva zhodné trojuholníky.
Aj uhlopriečka~$SC$ delí rovnobežník $SBCD$ na dva zhodné trojuholníky.
Obsah lichobežníka $ABCD$ je teda rovný trojnásobku obsahu trojuholníka $ASD$.

Zo zadania navyše vieme, že úsečky $AD$ a~$CD$ sú zhodné, teda rovnobežník $ASCD$ je kosoštvorcom a~jeho uhlopriečky $SD$ a~$AC$ sa pretínajú kolmo.
Pritom $|SD|=|BC|=24$~cm a~$|AC|=10$~cm, obsah trojuholníka $ASD$ je preto rovný $\frac12\cdot24\cdot5=60\ (\Cm^2)$.

Obsah lichobežníka $ABCD$ je rovný $3\cdot 60=180\ (\Cm^2)$.

\hodnotenie
2~body za vzťah medzi obsahom lichobežníka $ABCD$ a~obsahom trojuholníka $ASD$;
2~body za kolmosť úsečiek $SD$ a~$AC$;
2~body za dopočítanie obsahu a~kvalitu komentára.
\endhodnotenie}

{%%%%%   Z9-II-1
Čísla na tabuli označíme $x$ a~$y$.
Na jednej strane kartičky tak bolo napísané číslo $x+y$, na druhej strane číslo $x\cdot y$ a~platilo
$$
x+y+xy=97. \tag{1}
$$
Ľavá strana súhlasí s~tromi sčítancami v~roznásobení výrazu $(x+1)(y+1)$, chýba iba~1.
Pričítaním~1 k~obom stranám rovnice~(1) dostávame
$$
(x+1)(y+1)=98. \tag{2}
$$

Keďže $x$ a~$y$ sú prirodzené čísla, činitele na ľavej strane tejto rovnice musia byť aspoň~2.
Číslo 98 možno vyjadriť ako súčin dvoch prirodzených čísel väčších ako 1 iba týmito spôsobmi (poradie činiteľov ignorujeme):
$$
2\cdot 49=98,\quad 7\cdot 14=98. \tag{3}
$$

Prvej možnosti zodpovedajú čísla $x=1$ a~$y=48$, pre ktoré je $x+y=49$ a~$xy=48$.
Ani jedno z~týchto dvoch čísel nie je prvočíslom, preto táto možnosť nevyhovuje zadaniu úlohy.

Druhej možnosti zodpovedajú čísla $x=6$ a~$y=13$, pre ktoré je $x+y=19$ a~$xy=78$.
Číslo~19 je a~číslo~78 nie je prvočíslom, čo vyhovuje zadaniu úlohy.
Marienka napísala na tabuľu čísla 6 a~13.

\hodnotenie
Po 1~bode za vyjadrenie (1) a~úpravu (2);
2~body za rozklady (3);
2~body za rozbor možností a~záver.
\endhodnotenie

\ineriesenie
Pri rovnakom označení ako vyššie
môžeme z~rovnice (1) vyjadriť $y$ pomocou $x$:
$$
y=\frac{97-x}{1+x}. \tag{4}
$$
Postupným dosadzovaním prirodzených čísel za $x$ môžeme vyjadriť~$y$ a~overiť,
či sa jedná o~prirodzené číslo a~či jedno z~čísel $x+y$ a~$xy$ je prvočíslom:

$$
\begintable
$x$\|1|2|3|4|5|6|7|8|\dots\cr
$y$\|48|$\frac{95}{3}$|$\frac{47}{2}$|$\frac{93}{5}$|$\frac{46}{3}$|13|$\frac{45}{4}$|$\frac{89}{9}$|\dots\crthick
$x+y$\|49|||||\bf19|||\cr
$xy$\|48|||||78|||\endtable
$$

Medzi uvedenými možnosťami je jediná vyhovujúca dvojica čísel 6 a~13.
Teraz by sa mali preveriť ešte ostatné dosadenia za $x$ až po 96 (aby $y$ bolo kladné).
To však nie je nutné, keďže s~rastúcim~$x$ hodnota~$y$ stále klesá a~je určite menšia ako 9
(keby $x$ aj $y$ boli súčasne väčšie alebo rovné~9, tak by hodnota $x+y+xy$ bola väčšia alebo rovná $9+9+81=99$).
Vzhľadom na symetrickosť úlohy sa preto všetky riešenia musia vyskytovať v~uvedenej tabuľke.

Marienka napísala na tabuľu čísla 6 a~13.

\hodnotenie
Po 1~bode za vyjadrenia (1) a~(4);
2~body za dosadzovanie a~overovanie ako v~tabuľke;
2~body za úplnosť diskusie a~záver.
\endhodnotenie

\poznamka
Lomený výraz (4) môžeme vyjadriť v tvare \uv{celá časť plus zvyšok} takto:
$$
y=\frac{97-x}{1+x}
=\frac{-(1+x)+98}{1+x} =-1+\frac{98}{1+x}.
$$
Keďže $x$ a~$y$ majú byť prirodzené čísla, musí byť $1+x$ kladným vlastným deliteľom čísla~98,
a~také delitele máme práve štyri: 2, 7, 14 a~49.
Z~toho dostaneme dve možnosti (až na poradie $x$ a~$y$), z~ktorých jedna vyhovuje všetkým požiadavkám.
Tento postup možno chápať ako inú interpretáciu rovnice (2) a~následných úvah.
}

{%%%%%   Z9-II-2
Objem kvádra je rovný súčinu obsahu podstavy a~veľkosti výšky.
Zmena výšky o~2\,cm, spôsobuje zmenu objemu o~90\,cm$^3$; zodpovedajúca podstava má preto obsah $90:2=45\,(\Cm^2)$.

Pri ďalšej zmene výšky sa podstava nemení, preto zmena objemu kvádra na tri pätiny zodpovedá zmene výšky tiež na tri pätiny.
Ak označíme výšku pôvodného kvádra v~centimetroch $v$, platí
$$
\frac{v+2}{2}=\frac{3}{5}v.
$$
Z~toho sa ľahko vyjadrí $v=10$\,cm.

Poslednou nevyužitou informáciou zo zadania je, že jedna hrana kvádra je päťkrát dlhšia ako iná jeho hrana.
Môžu nastať tri možnosti:
\begin{enumerate}\alphatrue
\item Jedna hrana podstavy je päťkrát dlhšia ako výška, \tj. 50\,cm.
Odtiaľ druhá hrana podstavy meria $45:50=0{,}9$\,(cm).
\item Jedna hrana podstavy je päťkrát kratšia ako výška, \tj. 2\,cm.
Odtiaľ druhá hrana podstavy meria $45:2=22{,}5$\,(cm).
\item Jedna hrana podstavy je päťkrát dlhšia ako druhá jej hrana.
Ak označíme veľkosť kratšej z~týchto hrán v~centimetroch~$a$, platí
$a(5a)=45$, teda $a^2=9$ a~$a=3$\,(cm).
Odtiaľ dlhšia hrana podstavy meria $5\cdot 3=15$\,(cm).
\end{enumerate}

Rozmery pôvodného kvádra v~centimetroch môžu byť
$0{,}9\times 50\times10$, alebo ${2\times 22{,}5\times 10}$, alebo $3\times 15\times 10$.

\hodnotenie
1~bod za obsah podstavy;
2~body za výšku pôvodného kvádra;
po 1~bode za každé správne riešenie.
\endhodnotenie
}

{%%%%%   Z9-II-3
Pri tvorení trojciferných čísel s~uvedenými vlastnosťami je možné postupovať nasledovne:

Prvé miesto možno obsadiť ľubovoľnou z~piatich uvedených cifier, \tj. 5~možností.
Pre každé z~týchto obsadení možno druhé miesto obsadiť ľubovoľnou zo štyroch zvyšných cifier, \tj. celkom $5\cdot4=20$ možností.
Pre každé z~týchto obsadení možno tretie miesto obsadiť ľubovoľnou z~troch zvyšných cifier, \tj. celkom $5\cdot4\cdot3=60$ možností.
Takto vzniknutých trojciferných čísel je teda~60.

Na určenie celkového súčtu všetkých týchto čísel si stačí uvedomiť, že na každom mieste sa každá z~piatich cifier vyskytuje dvanásťkrát ($5\cdot 12=60$).
Súčet cifier na vybranom mieste vo všetkých uvažovaných číslach je rovný
$$
12\cdot(3+4+5+7+9)=12\cdot 28=336.
$$
Podľa príslušného miesta sa táto hodnota v~celkovom súčte vyskytuje jedenkrát, desaťkrát a~stokrát.
Súčet všetkých uvažovaných čísel je preto rovný
$$
336\cdot(1+10+100)=336\cdot 111=37\,296.
$$

\hodnotenie
2~body za počet čísel;
2~body za celkový súčet;
2~body za kvalitu, resp. úplnosť komentára.
\endhodnotenie}

{%%%%%   Z9-II-4
Stred kružnice opísanej pravouhlému trojuholníku je v~strede jeho prepony; tento bod je na obrázku označený~$O$.
Stred kružnice vpísanej je spoločným priesečníkom osí vnútorných uhlov trojuholníka; tento bod je označený~$V$ a~body dotyku kružnice s~jednotlivými stranami trojuholníka sú označené $X$, $Y$ a~$Z$:
\insp{Z9-II-4.eps}%

Polomer kružnice opísanej je 14,5\,cm, prepona~$KM$ je priemerom tejto kružnice, teda veľkosť $KM$ je 29\,cm.

Polomer kružnice vpísanej je 6\,cm, a~to je aj veľkosť úsečiek $VX$, $VY$ a~$VZ$.
Zhodné úsečky $VX$ a~$VZ$ sú susednými stranami v~pravouholníku $VXLZ$, tento pravouholník je preto štvorcom a~veľkosti úsečiek $LX$ a~$LZ$ sú tiež 6\,cm.

Trojuholníky $KYV$ a~$KZV$ sú oba pravouhlé, majú zhodné uhly pri vrchole~$K$ a~spoločnú stranu~$KV$.
Preto sú tieto trojuholníky zhodné, teda úsečky $KY$ a~$KZ$ sú zhodné; ich veľkosť na chvíľu označme~$a$.
Z~obdobného dôvodu sú zhodné aj úsečky $MX$ a~$MY$; ich veľkosť označíme~$b$.

Celkom tak dostávame $a+b=|KM|=29$\,cm a~obvod trojuholníka $KLM$ vieme vyjadriť ako
$$
2a+2b+2\cdot 6=2(a+b)+12=70\,(\Cm).
$$

\hodnotenie
1~bod za veľkosť prepony;
3~body za vyjadrenie obvodu;
2~body za kvalitu komentára.
\endhodnotenie

\poznamka
Ak označíme polomer kružnice opísanej~$R$ a~veľkosť prepony $|KM|=l$, tak prvé pozorovanie v~uvedenom riešení môžeme všeobecne zapísať $l=2R$, teda
$$R=\frac{l}{2}. \tag{1}$$
Ak ďalej označíme polomer kružnice vpísanej~$r$ a~veľkosti odvesien $|LM|=k$ a~$|KL|=m$, tak ďalšie závery z~predchádzajúceho riešenia sú $m=r+a$, $k=r+b$ a~$l=a+b$:
\insp{Z9-II-4a.eps}%

Z~toho vyplýva, že $k+m-l=2r$, teda
$$r=\frac{k+m-l}2, \tag{2}$$
a~obvod trojuholníka možno všeobecne vyjadriť ako
$$
k+m+l=2r+4R. \tag{3}
$$

Vzorce (1) a~(2) možno považovať za známe a~vyskytujú sa v~niektorých tabuľkách.
Na nich založené riešenie (3) by teda malo byť posudzované ako správne.
}

{%%%%%   Z9-III-1
Čísla napísané na tabuli označíme $a$ a~$b$, pričom budeme predpokladať, že $a\ge b$.
Zo zadania postupne vyplýva
$$
\aligned
(a-b)^2&=a^2-b^2-4\,038, \\
a^2-2ab+b^2&=a^2-b^2-4\,038, \\
2ab-2b^2&=4\,038, \\
b(a-b)&=2\,019,
\endaligned
$$
pričom čísla $b$ a~$a-b$ sú podľa predpokladov kladné.

Rozklad čísla 2\,019 na súčin prvočísel je $3\cdot 673$.
Číslo 2\,019 možno preto vyjadriť ako súčin dvoch kladných celých čísel nasledujúcimi spôsobmi:
$$
2\,019 =1\cdot2\,019 =2\,019\cdot1 =3\cdot673 =673\cdot3.
$$
Poradie činiteľov zdôrazňujeme kvôli všetkým možným priradeniam $b$ a~$a-b$.
Zodpovedajúce dvojice čísel $(a,b)$ sú
$(2\,020,1)$, $(2\,020, 2\,019)$, $(676, 3)$ a~$(676,673)$.

\hodnotenie
1~bod za zostavenie východiskovej rovnice;
2~body za úpravu na tvar súčinu;
1~bod za rozklad na súčin prvočísel;
2~body za určenie vyhovujúcich dvojíc.

Pri inom postupe riešenia dajte 2~body za určenie vyhovujúcich dvojíc
a~4~body za kvalitu komentára, najmä zdôvodnenie, že viac riešení neexistuje.

\endhodnotenie
}

{%%%%%   Z9-III-2
Najskôr predpokladajme, že Alan je Poctivec.
V~takom prípade by jeho výrok bol pravdivý a~Bruno by bol Klamár.
Brunov výrok by teda nebol pravdivý a~to by znamenalo, že Alan a~Ctibor by patrili do rôznych skupín.
Keďže Alan je Poctivec, Ctibor by bol Klamár.

Teraz predpokladajme, že Alan je Klamár.
V~takom prípade by jeho výrok nebol pravdivý a~Bruno by bol Poctivec.
Brunov výrok by teda bol pravdivý a~to by znamenalo, že Alan a~Ctibor by patrili do rovnakej skupiny.
Keďže Alan je Klamár, Ctibor by bol tiež Klamár.

S~istotou možno určiť, že Ctibor je Klamár.

\ineriesenie
Uvažujme všetky možné prípady, keď pri každom z~troch domorodcov (A, B, C) uvažujeme každý z~dvoch prípadov (P, K).
Celkom dostávame osem možností, ktoré postupne porovnáme s~výrokmi Alana a~Bruna.
Prípadný spor s~niektorým z~týchto výrokov je vyznačený v~poslednom riadku tabuľky:
$$
\begintable
A\|P|P|P|P|K|K|K|K\cr
B\|P|P|K|K|P|P|K|K\cr
C\|P|K|P|K|P|K|P|K\crthick
spor s\|A|A\ B|B||B||A|A\ B\endtable
$$

S~istotou možno určiť, že Ctibor je Klamár.

\hodnotenie
2~body za správny záver; 4~body za kvalitu komentára.
\endhodnotenie

}

{%%%%%   Z9-III-3
Zo zadania máme dve rovnosti:
$$
X=Y\cdot Z+27=1{,}1\cdot Y\cdot Z+3.
$$
Úpravami druhej rovnosti dostávame $0{,}1\cdot Y\cdot Z=24$, teda $Y\cdot Z=240$.
Dosadením späť zisťujeme, že $X=267$.

Delenie so zvyškom sa týka celých čísel.
Všetky čísla $X$, $Y$, $Z$ a~$1{,}1\cdot Y$ preto musia byť celé.
Z~toho vyplýva, že číslo~$Y$ musí byť násobkom~10.
Zvyšok po delení je menší ako deliteľ.
Preto musí byť $Z\ge 4$ a~$Y\ge28$, čo spolu s~predchádzajúcim záverom dáva $Y\ge 30$.

Z~rovnosti $Y\cdot Z=240$ a~požiadavky $Z\ge 4$ dostávame $Y\le 240:4=60$.
Podobne z~požiadavky $Y\ge 30$ dostávame $Z\le 240:30=8$.
Takto sme odhalili dve vyhovujúce dvojice čísel $Y$ a~$Z$.
Všetky riešenia dostaneme systematickým rozborom možností v~rámci uvedených obmedzení:
$$
\begintable
$Y$\|30|40|50|60\cr
$Z$\|8|6||4\endtable
$$
Vyhovujúce trojice čísel $(X,Y,Z)$ sú $(267,30,8)$, $(267,40,6)$ a~$(267,60,4)$.

\hodnotenie
Po 1~bode za vyjadrenie $Y\cdot Z=240$ a~$X=267$;
po 1~bode za každé z~troch riešení;
1~bod za kvalitu komentára.
\endhodnotenie

\poznamka
V~uvedenom riešení možno výhodne využiť prvočíselný rozklad $240={2^4\cdot3\cdot5}$.

}

{%%%%%   Z9-III-4
Veľkosť strany~$AC$ je menšia ako priemer kružnice, teda táto strana neobsahuje stred~$S$.
Výška trojuholníka $ABC$ z~vrcholu~$B$ môže byť ľubovoľne malá, ale nemôže byť ľubovoľne veľká: táto výška je najväčšia práve vtedy, keď obsahuje stred~$S$.
V~takom prípade má úloha jediné riešenie (zodpovedajúci trojuholník je rovnoramenný) a~veľkosť výšky je rovná veľkosti úsečky~$PB_0$ ako na obrázku:
\insp{z9-III-4.eps}%

Veľkosť $PB_0$ je rovná súčtu veľkostí úsečiek $PS$ a~$SB_0$.
Veľkosť úsečky~$SB_0$ je rovná polomeru kružnice, \tj.~39\,mm.
Veľkosť úsečky~$PS$ určíme pomocou Pytagorovej vety v~pravouhlom trojuholníku $APS$:
$$
|PS|=\sqrt{|AS|^2-|AP|^2} =\sqrt{39^2-36^2}=\sqrt{225}=15\,(\text{mm}).
$$

Teda hraničná veľkosť výšky je $39+15=54$\,(mm).
Úloha má dve riešenia, ak je výška z~vrcholu~$B$ menšia ako 54\,mm (a~väčšia ako 0\,mm).

\hodnotenie
3~body za vyjadrenie hraničnej hodnoty;
3~body za rozbor možností a~kvalitu komentára.

Odpovede založené len na odhade alebo meraní z~narysovaného obrázka hodnoťte 0~bodmi.

\endhodnotenie}

