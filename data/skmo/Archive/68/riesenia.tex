{%%%%%   A-I-1
\def\a{(a_n)_{n=1}^\infty}%
\def\b{(b_n)_{n=1}^\infty}%
a) Predpokladajme, že postupnosť $\a$ je konštantná. Potom
musí platiť $a_2 = a_1$, čo môžeme použitím vzťahu zo zadania
zapísať ako
$$
a_1 = \frac {a_1^2}{a_1^2-4a_1+6}.
$$
Túto rovnicu ľahko ekvivalentne upravíme na $a_1(a_1-2)(a_1-3) = 0$.
Z~toho dostávame, že $a_1 \in \{0, 2, 3\}$. Je vidno, že pre tieto
hodnoty~$a_1$ je postupnosť $\a$ naozaj konštantná a~všetky jej
členy sú rovné~$a_1$. Formálne by sme to dokázali matematickou
indukciou.

b) Nech $a_1 = 5$. Postupne vypočítavame niekoľko ďalších
členov postupnosti $(a_n)$. Dostaneme $a_2 \approx 2{,}27$, $a_3 \approx
2{,}49$, $a_4\approx 2{,}77$, atď. Z~toho môžeme nadobudnúť dojem,
že pre všetky $n \ge 2$ platí $2 <a_n <3$. Túto hypotézu
dokážeme matematickou indukciou.

Pre $n = 2$ tvrdenie zjavne platí. Predpokladajme, že tvrdenie platí pre dané
$n\ge 2$. Potom
$$
\aligned
3-a_{n+1} &= 3- \frac {a_n^2}{a_n^2-4a_n+6} =\frac {2 (a_n-3)^2}{(a_n-2)^2+2}> 0, \cr
a_{n+1}-2 &= \frac {a_n^2}{a_n^2-4a_n+6}-2 =\frac {(6-a_n) (a_n-2)}{(a_n-2)^2+2}> 0.
\endaligned
$$
To dokazuje obe nerovnosti $2 <a_{n+1} <3$, takže dôkaz matematickou
indukciou je hotový. Platí teda aj $2 <a_{2018} <3$, z~čoho
vyplýva, že najväčšie celé číslo neprevyšujúce $a_{2018}$ je~2.


\návody
O~postupnosti $\b$ vieme, že pre všetky prirodzené
čísla~$n$ platí $b_{n+1} = b_n^2-2$. Nájdite všetky hodnoty $b_1$,
pre ktoré sú všetky členy $\b$ rovné~$b_1$.
[Zo vzťahu pre $n = 1$
dostávame $b_1 = b_1^2-2$, z~čoho $b_1\in \{- 1, 2\}$. Následne overíme,
že tieto hodnoty vyhovujú.]

O~postupnosti $\b$ vieme, že $b_1 = 1$ a~že pre všetky
prirodzené čísla~$n$ platí $b_{n+1} = 3b_n / (b_n+1)$. Dokážte, že
všetky členy postupnosti sú z~intervalu $\langle1, 2)$.
[Matematickou indukciou overte, že pre všetky prirodzené~$n$ platia
nerovnosti $1\le b_n <2$.]

\D
O~postupnosti $\b$ je známe, že pre všetky prirodzené
čísla~$n$ platí $b_{n+1} = 2b_n^2 / (b_n^2-3)$. Nájdite všetky hodnoty
$b_1$ také, že postupnosť $b_2, b_3, b_4, \dots$ je konštantná.
[Odvoďte, že $b_{n+1} = b_n$ práve vtedy, keď $b_n \in \{\m1, 0, 3\}$, takže
$b_2$ musí byť rovné jednej z~týchto hodnôt. Následne
vypočítavame zodpovedajúce hodnoty~$b_1$. Výsledok je $b_1 \in
\{{-3},\allowbreak{-1}, 0, 1, 3\}$.]

Odvoďte explicitné vyjadrenie postupnosti $\b$ z~úlohy~N2.
[Zadaný vzťah upravíme na $3 \cdot 1 / b_{n+1} = 1+1 / b_n$.
Postupnosť $c_n = 1 / b_n$ teda spĺňa rovnosť $3c_{n+1} = c_n+1$. Substitúciou
$c_n = d_n+1/2$ sa zbavíme konštantného člena a~dostaneme
$3d_{n+1} = d_n$. Postupnosť~$d_n$ je teda geometrická, takže dokážeme
určiť jej explicitný tvar. Spätným dosadzovaním postupne
nájdeme vyjadrenie pre~$b_n$. Výsledok je $b_n = 2 \cdot 3^{n-1} /
(3^{n-1}+1)$.]

Najznámejšia rekurentne definovaná postupnosť je
Fibonacciho postupnosť. Tá je daná vzťahmi $f_1 = 1$, $f_2 = 1$
a~rekurentným vzťahom $f_n = f_{n-1}+f_{n-2}$ platiacim pre každé $n \ge 3$.
Dokážte, že $\sum_{i~= 1}^{n} f_i = f_{n+2}-1$ a~$\sum_{i~= 1}^{n}
f_i^2 = f_n \cdot f_{n+1}$.
[Postupujte matematickou indukciou podľa~$n$.]
\endnávod}

{%%%%%   A-I-2
Ako dokážeme, uvedené štyri body ležia na jednej kružnici
v~takom poradí, že tvoria vrcholy tetivového štvoruholníka $E_1E_2D_2D_1$.
Najskôr však vysvetlíme, prečo je tento štvoruholník konvexný.

Keďže trojuholník $ABC$ má podľa predpokladu ostré vnútorné uhly
pri~vrcholoch $B$ a~$C$,\footnote{Predpoklad, že aj uhol pri vrchole~$A$
je ostrý, v~riešení potrebovať nebudeme.} leží bod~$D$ vnútri
strany~$BC$ a~výšky v~pravouhlých trojuholníkoch $ABD$, $ACD$
vedené z~vrcholu~$D$ majú svoje päty vnútri strán $AB$, resp. $AC$.
Tieto päty spolu s~vrcholmi $B$, $C$ tak tvoria vrcholy
konvexného štvoruholníka, s~ktorým je skúmaný štvoruholník
$E_1E_2D_2D_1$ rovnoľahlý s~koeficientom~2 podľa stredu~$D$
(\obr). Je to teda naozaj štvoruholník konvexný.
\insp{a68.1}%

Všimnime si, že vďaka osovým súmernostiam platí
$|AD_1|=|AD|=|AD_2|$. Bod~$A$ je teda stredom kružnice opísanej
trojuholníku~$D_1DD_2$. Keďže priamka~$AD$ body~$D_1$ a~$D_2$ oddeľuje,
obvodový uhol $D_1D_2D$ v~dotyčnej kružnici je rovný polovici
konvexného stredového uhla $D_1AD$, ktorého osou je práve
polpriamka~$AB$. Preto sú zhodné tri ostré uhly $D_1D_2D$, $D_1AB$
a~$DAB$ (vyznačené na \obrr1{} oblúčikmi). Ich veľkosť je
z~pravouhlého trojuholníka~$ABD$ rovná $90\st-\be$, pričom ako zvyčajne
$\be=|\uhel ABC|$. Keďže uhol $DD_2E_2$ je zrejme pravý, má
vnútorný uhol pri vrchole~$D_2$ konvexného štvoruholníka $E_1E_2D_2D_1$
veľkosť
$$
|\uhel D_1D_2E_2|=|\uhel D_1D_2D|+|\uhel DD_2E_2|=(90\st-\be)+90\st=180\st-\be,
$$
zatiaľ čo vnútorný uhol pri protiľahlom vrchole~$E_1$ má zrejme
veľkosť~$\be$. Súčet oboch protiľahlých uhlov je tak $180\st$, čiže
$E_1E_2D_2D_1$ je skutočne tetivový štvoruholník.

Prejdime k~druhému tvrdeniu o~tom, kde stred kružnice opísanej štvoruholníku
$E_1E_2D_2D_1$ leží. Vieme, že je priesečníkom osí jeho strán, z~ktorých
pre naše úvahy vyberieme strany $E_1D_1$ a~$E_2D_2$.

Trojuholník $DE_2D_2$ je pravouhlý s~pravým
uhlom pri vrchole~$D_2$. Pritom bod~$C$ leží na osi strany~$DD_2$
a~na jeho prepone~$DE_2$~-- nutne teda musí byť jej stredom, a~teda
aj~stredom kružnice tomuto trojuholníku opísanej. Os úsečky~$D_2E_2$ tým
pádom prechádza bodom~$C$ (\obr) a~navyše je zrejme kolmá na~$AC$.
Analogicky os úsečky~$D_1E_1$ je priamka kolmá na~$AB$
prechádzajúca bodom~$B$. Tieto dve priamky sa zrejme pretínajú na
kružnici opísanej trojuholníku $ABC$, lebo podľa Tálesovej vety
musí spojnica ich priesečníkov s~bodom~$A$ tvoriť priemer tejto kružnice.
Tým je dôkaz celého tvrdenia ukončený.
\insp{a68.2}%


\ineres
Ukážeme iný argument, prečo $|\angle DD_2D_1| = 90^\circ- \beta$, čo
stačí na~dôkaz, že $E_1E_2D_2D_1$ je tetivový.
Body $E_1$, $E_2$ sú zrejme postupne obrazy bodov $B$,~$C$
v~rovnoľahlosti so stredom~$D$ a~koeficientom~$2$. Ak označíme $A'$
obraz bodu~$A$ v~tejto rovnoľahlosti (\obr), budú trojice bodov
\insp{a68.3}%
$A'$, $D_1$, $E_1$ a~$A'$, $D_2$, $E_2$ zrejme kolineárne. Štvoruholník
$A'D_1DD_2$ je pritom tetivový, lebo oba uhly $A'D_1D$ a~$A'D_2D$
sú pravé. Z~toho dostávame $|\angle D_1D_2D| = |\angle D_1A'D|$.
A~keďže $D_1A'\parallel BA$, je aj $|\angle D_1A'D| = |\angle
BAD| = 90^\circ- \beta$, takže dokopy naozaj máme $|\angle
DD_2D_1| = 90^\circ- \beta$.

\ineres
Ešte ďalším poučným spôsobom dokážeme prvú časť tvrdenia úlohy.
Definujme bod~$A'$ ako v~predošlom riešení (\obrr1).
Potom použitím Euklidových viet v~pravouhlých trojuholníkoch
$A'E_1D$ a~$A'E_2D$ dostávame $|A'D|^2 = |A'D_1| \cdot |A'E_1|$
a~$|A'D|^2 = |A'D_2| \cdot |A'E_2|$. Spolu tak máme rovnosť
$$
|A'D_1|\cdot |A'E_1| = |A'D_2| \cdot |A'E_2|.
$$
Keďže bod~$A'$ leží zvonka oboch úsečiek $D_1E_1$ a~$D_2E_2$,
vyplýva z~mocnosti\footnote{{\tt https://kms.sk/248/plugin/attachments/download/393/},
strana 37, sekcia 2.2.} bodu~$A'$ ku kružnici opísanej trojuholníku $E_1D_1D_2$, že na tej
kružnici leží aj bod~$E_2$, \tj. že štvoruholník $D_1E_1E_2D_2$ je naozaj tetivový.

\poznamka
Existujú ďalšie alternatívne
riešenia tejto úlohy pomocou situácie, ktorá vznikne po aplikovaní
rovnoľahlosti so stredom v~bode~$D$ a~koeficientom~$1/2$. Pomocou nej môže
byť napríklad prvá časť úlohy formulovaná tak, že máme dokázať, že
kolmé priemety päty~$D$ na strany $AB$ a~$AC$ ležia spolu s~bodmi $B$ a~$C$ na
kružnici. Táto formulácia je pomerne prirodzená.
Druhá časť tvrdenia sa ale najlepšie dokazuje podľa \obrr2.



\návody
Dokážte vetu o~obvodovom a~stredovom uhle. [Majme
kružnicu~$k$ so stredom~$O$ a~jej tetivu~$AB$. Nech $X$ je bod na~$k$.
Tvrdenie dokážeme v~prípade, keď je $O$ vnútorným bodom
trojuholníka $ABX$ (v~ostatných prípadoch je dôkaz podobný).
Označme $|\angle AXO| = \alpha$ a~$|\angle BXO| = \beta$.
Z~rovnoramenných trojuholníkov $AOX$ a~$BOX$ spočítame, že $|\angle
XOA| = 180^\circ-2\alpha$ a~$|\angle XOB| = 180^\circ-2\beta$.
Z~toho ľahko máme $|\angle AOB| = 2(\alpha+\beta)$, čo sme mali
dokázať.]

Dokážte, že konvexný štvoruholník $ABCD$ je tetivový
práve vtedy, keď je súčet jeho protiľahlých uhlov rovný~$180^\circ$.
[Ak je $ABCD$ tetivový, označme $O$ stred kružnice jemu
opísanej. Konvexný a~nekonvexný uhol $AOC$ dávajú dokopy
$360^\circ$. Z~vety o~obvodovom a~stredovom uhle máme, že tento súčet je
rovný dvojnásobku súčtu veľkostí uhlov $ABC$ a~$ADC$~--
tento súčet je teda rovný $180^\circ$. Predpokladajme
naopak, že súčet protiľahlých uhlov je rovný $180^\circ$.
Bez ujmy na všeobecnosti predpokladajme, že $|\angle
ADC| \ge 90^\circ$. Potom $|\angle ABC| \leq 90^\circ$. Označme $O$
stred kružnice opísanej trojuholníku $ADC$. Bod~$O$ zrejme leží
v~polrovine $ACB$. Navyše ľahko vypočítame, že konvexný uhol $AOC$
má veľkosť rovnú dvojnásobku uhla pri vrchole~$B$. Spolu
s~$|OA| = |OC|$ tak máme, že $O$ musí nutne byť stredom kružnice opísanej
trojuholníku $ABC$, takže je stredom kružnice opísanej celému
štvoruholníku $ABCD$.]

Dokážte, že konvexný štvoruholník $ABCD$ je tetivový
práve vtedy, keď $|\angle ACB| = |\angle ADB|$.
[Ak je štvoruholník $ABCD$ tetivový, tak oba tieto uhly sú rovné
polovici prislúchajúceho stredového uhla $AOB$. Ak naopak platí
uvedená rovnosť, označme $O$ stred kružnice opísanej
trojuholníku $ABC$. Pozrime sa na bod~$O$ z~pohľadu trojuholníka
$ABD$. Máme $|OA| = |OB|$. Ďalej veľkosť uhla $AOB$ (konvexného
či nekonvexného) je rovná dvojnásobku veľkosti uhla pri $D$.
Keďže leží v~"správnej" polrovine vzhľadom na~priamku~$AB$,
musí sa jednať o~stred kružnice opísanej $ABD$, teda aj celému
štvoruholníku.]

Daný je ostrý uhol $XAY$ a~vnútri neho bod~$P$. Nech
$P_1$, $P_2$ sú obrazy bodu~$P$ v~osovej súmernosti podľa jednotlivých
ramien $AX$, $AY$ uhla $XAY$. Dokážte, že $|\angle P_1AP_2| = 2|\angle
XAY|$. [Označme $|\angle XAP| = \alpha$ a~$|\angle PAY| = \beta$. Potom
$|\angle XAY| = \alpha+\beta$. Vďaka osovej súmernosti platí $|\angle
P_1AX| = \alpha$ a~$|\angle P_2AY| = \beta$. Je teda naozaj $|\angle
P_1AP_2| = 2(\alpha+\beta)$.]

\D
Daný je ostrouhlý trojuholník $ABC$. Nech $D$, $E$, $F$ sú postupne
päty výšok na strany $BC$, $CA$, $AB$. Dokážte, že sa priamky $AD$, $BE$, $CF$
pretínajú v~jednom bode. [Nech $H$ označuje priesečník priamok $BE$
a~$CF$. Stačí dokázať, že priamka~$AH$ je kolmá na $BC$. Všimnime
si, že body $A$, $F$, $H$, $E$ ležia na kružnici vďaka pravým uhlom pri
vrcholoch $E$,~$F$. To isté platí aj pre body $B$, $C$, $E$, $F$. Je teda
$|\angle HAE| = |\angle HFE| = |\angle CBE| = 90^\circ-\gamma$. Z~toho už vyplýva
dokazovaná kolmosť.]

Označme $H$ priesečník priamok z~úlohy~D1. Dokážte,
že $H$ je stredom kružnice vpísanej trojuholníku $DEF$. [Vďaka
symetrii stačí dokázať, že $DH$ je os vnútorného uhla $EDF$.
Keďže $DH$ a~$DB$ sú kolmé, stačí dokázať, že $DB$ je os
vonkajšieho uhla $EDF$, teda že $|\angle FDB| = |\angle EDC|$. Oba
tieto uhly sú ale rovné~$\alpha$, čo je vidno z~tetivových
štvoruholníkov $AFDC$, resp. $AEDB$ (tie sú tetivové vďaka
pravému uhlu nad tetivami $AC$, resp. $AB$).]

Daný je ostrouhlý trojuholník $ABC$. Označme $D$ pätu jeho
výšky na stranu~$BC$. Dokážte, že päty kolmíc z~$D$ na zvyšné
strany a~zvyšné výšky ležia na jednej priamke.
[Nech $P$, $Q$,~$R$ sú
postupne kolmé priemety bodu~$D$ na $AB$, $AC$ a~výšku z~vrcholu~$B$,
ktorej pätu ešte označíme~$S$. Štvoruholníky
$BDRP$, $DQSR$, $ASDB$ sú vďaka pravým uhlom tetivové. Z~nich
postupne vypočítame $|\angle DRP| = 180^\circ- \beta$ a~$|\angle
DRQ| = |\angle DSC| = \beta$. Body $P$, $Q$, $R$ sú teda kolineárne.
Vďaka symetrii sme hotoví.]

Daný je ostrouhlý trojuholník $ABC$ s~priesečníkom výšok~$H$
a~stredom kružnice opísanej~$O$. Dokážte, že priamky $AH$, $AO$ sú
súmerne združené podľa osi vnútorného uhla $BAC$ (také dvojice
priamok so spoločným bodom~$A$ sa nazývajú {\it izogonálne\/} vzhľadom na daný uhol $BAC$).
[Ľahko vypočítame, že $|\angle BAH| = 90^\circ- \beta$. Z~rovnoramenného
trojuholníka $AOC$ a~vety o~obvodovom a~stredovom uhle potom určíme, že
$|\angle CAO| = 90^\circ- \beta$, takže $|\angle BAH| = |\angle CAO|$,
odkiaľ už vyplýva dokazované tvrdenie.]

Daný je ostrouhlý trojuholník $ABC$. Dokážte, že
polpriamky izogonálne vzhľadom na uhol~$BAC$ (pozri definíciu izogonálnosti
v~predošlej úlohe) pretínajú kružnicu opísanú trojuholníku~$ABC$ v~bodoch
rôznych od~$A$, ktoré sú súmerne združené podľa osi úsečky~$BC$.
[Označme $K$ a~$L$ priesečníky týchto priamok s~kružnicou
opísanou. Z~definície izogonálnosti máme $|\angle BAK| = |\angle CAL|$,
takže tetivy $BK$ a~$CL$ majú rovnakú veľkosť, a~tak body $B$, $C$, $K$, $L$
tvoria rovnoramenný lichobežník, z~čoho už vyplýva dokazované tvrdenie.]
\endnávod}

{%%%%%   A-I-3
Vzhľadom na to, že člen s~$n$ rastie oveľa rýchlejšie ako člen s~$m$,
má zmysel preskúmať najskôr malé hodnoty~$m$.

Keď $m = 0$, je skúmaný vzťah ekvivalentný s~nerovnosťou $n^{n+1} \le 3$. Tomu
zrejme vyhovujú $n = 0$ a~$n = 1$, zatiaľ čo pre $n \ge 2$ platí $n^{n+1}
\ge 8$. Dostávame tak dve riešenia $(m, n)=(0, 0)$ a~$(m, n)=(0, 1)$.

Pre $m = 1$ riešime nerovnicu $|4-n^{n+1}| \leq 3$. Vidíme, že $n = 0$
nevyhovuje, $n = 1$ vyhovuje a~pre $n \ge 2$ platí
$|4-n^{n+1}|= n^{n+1}-4 \ge 4$. Máme teda riešenie $(m, n) = (1, 1)$.

Ďalej už budeme predpokladať, že $m\ge2$, danú nerovnicu prepíšeme
v~tvare rovnice $4m^2=n^{n+1}+a$, pričom $a$ je (neznáme) celé číslo,
ktorého absolútna hodnota neprevyšuje~3, a~rozlíšime prípady $a=0$,
$|a|\in\{1, 3\}$ a~$|a|=2$.

Rozoberme najskôr prípad, keď $4m^2 = n^{n+1}$. Na ľavej strane rovnice
je kladné párne číslo, ktoré je navyše druhou mocninou celého
čísla. To isté musí platiť aj pre pravú stranu rovnice, takže
číslo~$n$ musí byť kladné a~párne: položme $n = 2k$. Potom $n^{n+1} = (2k)^{2k+1}$,
a~keďže exponent $2k+1$ je nepárny, bude toto číslo druhou mocninou
celého čísla práve vtedy, keď jeho základ~$2k$ bude druhou mocninou celého
čísla, čiže $2k = r^2$, pričom $r$ je kladné celé číslo, ktoré zjavne
musí byť párne. Je preto $r = 2l$, a~teda $k= 2l^2$ a~$n=2k = 4l^2$, pričom $l$ je kladné
celé číslo. Keďže $m$ je
kladné, z~rovnice $4m^2=n^{n+1}$ po vydelení štyrmi a~odmocnení dostaneme
$$
m = \sqrt {\frac {n^{n+1}}{4}} = \frac {\big(\sqrt {4l^2} \big)^{4l^2+1}}{2}
= \frac {(2l)^{4l^2+1}}{2}= l\cdot (2l)^{4l^2}.
$$
Pre každé kladné celé číslo~$l$ tak vychádza, že dvojica
$(m, n)=(l (2l)^{4l^2}, 4l^2)$ je riešením úlohy.

Uvažujme teraz prípady $4m^2 = n^{n+1} \pm a$, pričom $a\in
\{1, 3\}$. Z~faktu, že pravá strana rovnice musí byť párne číslo,
vyplýva, že $n^{n+1}$ musí byť nepárne číslo, čo znamená, že samo~$n$
je nepárne číslo. Položme $n = 2k+1$, pričom $k$ je nezáporné celé
číslo. Potom dostávame
$$
\align
4m^2&=n^{2k+2} \pm a,\\
(2m+n^{k+1}) (2m-n^{k+1})&=\pm a.
\endalign
$$

Číslo $\pm a$ potrebujeme rozložiť na súčin dvoch celých
čísel, ktoré v~súčte dávajú $(2m+n^{k+1})+(2m-n^{k+1}) = 4m\ge 8$
(použili sme predpoklad $m \ge 2$). Aspoň jedno z~nich je teda
väčšie ako $3$, čo nemôže nastať, pretože medzi deliteľmi
čísla $\pm a$ môžu byť iba čísla ${-3}$, ${-1}$, $1$, $3$.

Ostáva rozobrať poslednú možnosť $4m^2 = n^{n+1} \pm 2$.
Z~tejto rovnice vyplýva, že číslo~$n$ nie je nula a~je párne, takže jeho
mocnina $n^{n+1}$ s~exponentom väčším ako~1 je deliteľná štyrmi
rovnako ako $4m^2$ na ľavej strane rovnice. Vidíme, že požadovaná rovnosť nemôže
platiť, preto v~tomto prípade žiadne riešenie nedostaneme.

\zaver
Všetky riešenia úlohy sú $(0, 0)$, $(0, 1)$, $(1, 1)$ a~nekonečne veľa
dvojíc tvaru $(m, n) = (l (2l)^{4l^2}, 4l^2)$, pričom $l$ je
ľubovoľné prirodzené číslo.


\návody
Nájdite všetky prirodzené čísla~$n$ také, že
$n^{n+1}$ je druhou mocninou prirodzeného čísla.
[Zjavne vyhovujú
všetky nepárne čísla~$n$, lebo vtedy je exponent $n+1$ párny.
Ak je $n$ párne, je exponent $n+1$ nepárny, takže aj samo~$n$
musí byť druhou mocninou celého čísla, \tj. $n = 4k^2$
pre nejaké celé číslo~$k$.]

Nájdite všetky riešenia nerovnice $|x^2-y^2| \le 2$, pričom $x$, $y$
sú celé čísla. [Máme rovnicu $(x-y)(x+y)=a$, pričom $|a|\le2$.
Je preto buď $(x-y)(x+y)=0$ a~úlohe vyhovujú
ľubovoľné dvojice $(x,x)$ a~$(x,{-x})$, pričom $x$ je celé, alebo
$|x+y|=|x-y|=1$, keďže obe čísla $x-y$, $x+y$ majú zrejme rovnakú paritu.
Ľahko tak nájdeme ďalšie štyri vyhovujúce dvojice:
$(0, {\pm 1})$, $({\pm 1}, 0)$.]

\D
Dokážte, že žiadne číslo tvaru $4k+2$, pričom $k$ je
celé číslo, sa nedá zapísať ako rozdiel dvoch druhých mocnín
celých čísel. [Druhé mocniny celých čísel dávajú po delení 4
zvyšky 0 a~1, rozdiel dvoch z~nich teda môže dať iba zvyšok 0, 1
alebo $3\equiv{-1}$.]

Nájdite všetky prirodzené čísla~$x$ také, že $x^2+x-2$ je mocnina dvoch.
[Platí $x^2+x-2 = (x-1)(x+2)$, takže obe
čísla $x-1$ a~$x+2$ musia byť mocniny dvoch. Ich rozdiel je rovný~3, teda
jedna z~týchto mocnín musí byť nepárna. Jediná možnosť je $x-1 = 2^0=1$, čiže
$x = 2$, ktorá naozaj vyhovuje, lebo $x^2+x-2 = 4$.]

Nájdite všetky prirodzené~$x$ také, že $x^2+x+1$ je
druhou mocninou celého čísla.
[Žiadne také číslo~$x$ neexistuje, lebo $x^2 <x^2+x+1 <(x+1)^2$.]
\endnávod}

{%%%%%   A-I-4
Označme $x_1 <x_2 <\dots <x_n$ prvky
množiny~$\mm M$. Zvoľme pevný index~$i$. Ak je prvok~$x_i$
aritmetickým priemerom prvkov $x_j <x_k$, musí zrejme platiť
$x_j <x_i <x_k$. Pre hodnotu~$j$ tak máme možnosti $1, \dots, i-1$ (tých
je $i-1$), zatiaľ čo pre hodnotu~$k$ máme možnosti
$i+1, i+2, \ldots, n$ (tých je $n-i$). Navyše ak je $x_i$
aritmetickým priemerom rôznych dvojíc $x_{j_1} <x_{k_1}$ a~$x_{j_2}
<x_{k_2}$, tak $j_1 \ne j_2$ a~$k_1 \ne k_2$ (ak by napr. bolo
$j_1 = j_2$, ľahko by sme dostali $k_1 = k_2$, čo je v~spore s~tým, že
sa jedná o~rôzne dvojice). Každú z~možných hodnôt $j$ či $k$ preto
môžeme vybrať nanajvýš raz, teda počet neusporiadaných dvojíc $p$, $q$
rôznych čísel z~$\mm M$, pre ktoré platí $x_i=\frac12(p+q)$ s~daným indexom~$i$,
je nanajvýš $\min \{i-1, n-i\}$. S~prihliadnutím na~ich požadované usporiadanie
je to potom dvojnásobok, $2 \min \{i-1, n-i\}$.

Podľa zadania je $n = 2k+1$, pričom $k$ je prirodzené číslo. Sčítaním našich
odhadov pre všetky indexy $i= 1, 2, \dots, n$ dostávame, že hľadaný
počet dvojíc je nanajvýš
$$
\align
2 (\min \{0, 2k\}+\min \{1, 2 k-1\}+\dots+\min\{k,k\}+\dots+\min \{2k, 0\})&=\\
=2 (0+1+\ldots+(k-1)+k+(k-1)+(k-2)+\ldots+1+0)&=\\
=2(1+(k-1))+2(2+(k-2))+\dots+2((k-1)+1)+2k&=k\cdot 2k.
\endalign
$$
Tým pádom sme hotoví, pretože $\frac {1}{2} (n-1)^2 = 2k^2$.

\poznamky
Ak $n = 2k$, zistíme, že počet skúmaných dvojíc je nanajvýš
$$
2 (\min \{0, 2k-1\}+\min \{1, 2 k-2\}+\dots+\min \{2k-1, 0\}) = 2k (k-1).
$$
Ľahko overíme, že jednotný vzorec pre zápis oboch prípadov je ${n
\choose 2}-\left \lfloor \frac {n}{2} \right \rfloor$, pričom $\left \lfloor x
\right \rfloor$ označuje dolnú celú časť čísla~$x$.

%\poznamka 2.
Skúmaný odhad sa nedá všeobecne
zlepšiť, ako dosvedčuje príklad množiny $\mm M=\{1, 2, \ldots, n\}$.
Rovnaká množina funguje aj pre párne~$n$ (pozri poznámku~1).

%\poznámka 3.
V~riešení sme nikde
nepoužili, že v~uvažovanej množine~$\mm M$ sú prirodzené čísla. Uvedené
riešenie funguje aj pre $n$-prvkovú množinu reálnych čísel.

%\poznámka 4.
Odhad počtu dvojíc z~tvrdenia úlohy platí pre akúkoľvek množinu~$\mm M$
kladných reálnych čísel aj v~prípade, keď aritmetický priemer dvoch čísel
$p$, $q$ zameníme iným zo známych priemerov (napr. geometrickým $\sqrt{pq}$
alebo harmonickým $2pq/(p+q)$). Je možné dokazovať rovnaký odhad
aj pre nesymetrické priemery, akým je napr. vážený priemer $\frac23 p+\frac 13q$,
len počet usporiadaných dvojíc sa potom nedá počítať ako dvojnásobok
počtu dvojíc neusporiadaných.

\návody
Dokážte, že ak $p <q$, leží aritmetický priemer čísel
$p$ a~$q$ v~intervale $(p, q)$. [Treba overiť nerovnosti
$p <(p+q) / 2$ a~$(p+q) / 2 <q$. Obe sú ekvivalentné s~$p <q$.]

Sú dané štyri prirodzené čísla $x_1 <x_2 <x_3 <x_4$.
Môže byť číslo~$x_2$ aritmetickým priemerom dvoch rôznych
(neusporiadaných)
dvojíc z~týchto čísel? [Nie. Aby bolo $x_2$
priemerom nejakých dvoch čísel, muselo by jedno z~nich byť menšie ako
$x_2$, zatiaľ čo druhé by bolo väčšie. Jediný kandidát na menšie
číslo je $x_1$. Ak by ale platilo $x_2 = (x_1+x_3) / 2$
a~$x_2 = (x_1+x_4) / 2$, mali by sme $x_3 = x_4$, čo je v~spore s~predpokladom.]

Daná je množina čísel $\{1, 2, 3, 4, 5\}$. Koľko dvojíc
čísel $p <q$ z~tejto množiny spĺňa, že číslo $(p+q) / 2$ je jedným
z~prvkov množiny?
[Čísla $1$ a~$5$ zrejme nie sú priemerom
žiadnych dvoch čísel. Čísla $2$ a~$4$ sú priemermi dvojíc $(1, 3)$
resp. $(3, 5)$. Napokon číslo~$3$ je priemerom dvojíc $(1, 5)$
a~$(2, 4)$. Spolu máme štyri dvojice.]

\D
Dokážte, že pre párne~$n$ je maximálny počet
dvojíc skúmaných úlohou rovný $\frac{1}{2} n(n-2)$.

Pre každé prirodzené~$n$ (párne aj nepárne) nájdite
príklad množiny, pre ktorú je počet skúmaných dvojíc maximálny.

Zostane tvrdenie úlohy v~platnosti, aj keď namiesto množiny
prirodzených čísel uvažujeme množinu reálnych čísel?

Zostane tvrdenie v~platnosti, aj keď nahradíme
aritmetický priemer geometrickým?
\endnávod}

{%%%%%   A-I-5
Označme $P$ bod ležiaci na polpriamke
opačnej k~polpriamke~$BC$ taký, že $|BP| = |BA|$. Analogicky označme $Q$ bod
ležiaci na polpriamke opačnej k~polpriamke~$CB$ taký, že $|CQ| = |CA|$.
Úsečka~$PQ$ má potom dĺžku rovnú veľkosti obvodu~$o$ trojuholníka~$ABC$.

Ďalej označme $I_a$ stred kružnice pripísanej k~strane~$BC$
trojuholníka $ABC$ (\obr). Priamka~$I_aB$ je potom osou uhla $ABP$. A~keďže
trojuholník $ABP$ je rovnoramenný, je to zároveň os úsečky~$AP$.
Bod~$I_a$ teda leží na osi úsečky~$AP$ a~podobne aj na
osi úsečky~$AQ$, čo znamená, že bod~$I_a$ je stredom kružnice
opísanej trojuholníku $APQ$, a~leží tak aj na osi jeho tretej strany~$PQ$.
Jej stred~$M$ je preto kolmým priemetom bodu~$I_a$ na~$PQ$, a~teda je
zároveň aj dotykovým bodom kružnice pripísanej k~strane~$BC$.
\insp{a68.4}%

Z~predošlého {\it rozboru\/} už vyplýva {\it konštrukcia}. Najskôr
zostrojíme úsečku~$PQ$ dĺžky~$o$ a~jej stred~$M$. Následne môžeme
zostrojiť bod~$I_a$, lebo $|MI_a| = \rho$ a~$MI_a
\perp PQ$. Potom nájdeme bod~$A$, ktorý leží jednak na kružnici
$k(I_a, |I_aP|)$ (lebo $I_a$ je stredom kružnice opísanej trojuholníku~$APQ$)
a~jednak na priamke~$l$ rovnobežnej s~$PQ$ vo vzdialenosti~$v$, ktorá
leží v~polrovine opačnej k~$PQI_a$. Pomocou nájdených bodov~$A$
dokážeme zostrojiť body $B$ a~$C$ rôznymi spôsobmi~-- napríklad ako
priesečníky osí úsečiek $AP$ a~$AQ$ s~úsečkou~$PQ$. Tieto
priesečníky budú pre každý zostrojený bod~$A$ existovať a~budú ležať
na úsečke~$PQ$ v~\uv{správnom} poradí, keďže trojuholník $APQ$ je
tupouhlý s~tupým uhlom pri vrchole~$A$ (stred~$I_a$
jeho kružnice opísanej leží v~polrovine opačnej k~polrovine~$PQA$).

Teraz dokážeme, že takto zostrojené body $A$, $B$, $C$ sú vrcholmi trojuholníka,
ktorý má všetky požadované vlastnosti.
Keďže body $B$ a~$C$ ležia na osiach úsečiek $AP$
a~$AQ$, platí $|AB| = |PB|$ a~$|CA| = |CQ|$, takže
$|AB|+|BC|+|CA| = |PB|+|BC|+|CQ| = |PQ| = o$. Bod~$A$ bol nájdený na
priamke~$l$, preto z~jej definície vyplýva, že vzdialenosť $A$ od~$BC$
je naozaj~$v$. Napokon bod~$I_a$ je stredom kružnice~$k$
opísanej trojuholníku~$APQ$, takže leží na osiach úsečiek $AP$
a~$AQ$, ktoré sú ale z~rovnoramennosti $ABP$ a~$ACQ$ totožné s~osami
vonkajších uhlov $ABC$ a~$ACB$, preto $I_a$ je naozaj stredom
kružnice pripísanej k~strane~$BC$ trojuholníka $ABC$. Navyše je
$I_a$ zostrojený tak, že jeho vzdialenosť od~$PQ$ je rovná~$\rho$.
Zostrojený trojuholník $ABC$ tak spĺňa podmienky zo zadania.

Ostáva spraviť {\it diskusiu\/} o počte riešení. Zaujíma nás
počet vyhovujúcich trojuholníkov $ABC$, z~ktorých žiadny nie je obrazom
iného v~zhodnosti, v~ktorej by si ich vrcholy zodpovedali
podľa písmen, ktorými sú označené.
Po zostrojení bodov $P$, $Q$ máme dve možné polohy pre bod~$I_a$. Tým
však budú zodpovedať zhodné riešenia súmerne združené podľa~$PQ$,
preto uvažujme iba jednu z~týchto polôh.
Následne určíme bod~$A$ ako priesečník kružnice~$k$ a~priamky~$l$.

Označme $r$ polomer kružnice~$k$ ($r=|I_aP| = |I_aQ|$). Vzdialenosť bodu~$I_a$
od priamky~$l$ je rovná $\rho+v$. Preto ak $r>\rho+v$, dostaneme dva rôzne
priesečníky $A_1$ a~$A_2$. Tým zrejme budú zodpovedať dva rôzne
trojuholníky $A_1B_1C_1$, $A_2B_2C_2$ (\obr), ktoré budú navzájom
súmerne združené podľa osi úsečky~$PQ$, ktorou je priamka~$MI_a$.
V~tom prípade má úloha
dve riešenia.\footnote{Oba trojuholníky sú síce nepriamo zhodné, ale s~odlišným poradím vrcholov!}
\insp{a68.5}%

Ak $r = \rho+v$, bude priamka~$l$ dotyčnicou kružnice~$k$, takže
dostaneme jediný bod~$A$ a~jediný vyhovujúci trojuholník $ABC$
(ktorý zo symetrie bude navyše rovnoramenný).

Napokon ak $r <\rho+v$, tak
žiadny priesečník nedostaneme. Hodnotu~$r$ možno z~pravouhlého
trojuholníka~$PMI_a$ vyjadriť pomocou $o$ a~$\rho$ ako
$\sqrt{|MI_a|^2+|PM|^2} = \sqrt{\rho^2+\frac14 {o^2}}$. Výsledky
môžeme zhrnúť do tabuľky:
$$
\vbox{\let\\=\cr\halign{\hss#\unskip:\hss&\quad\hss#\hss\cr
$\sqrt {\rho^2+\frac1{4}{o^2} }>\rho+v$ &2 riešenia \\
$\sqrt {\rho^2+\frac1{4}{o^2} }=\rho+v$ &1 riešenie \\
$\sqrt {\rho^2+\frac1{4}{o^2} }<\rho+v$ &0 riešení \\
}}
$$

\poznamka
Záverečná tabuľka ukazuje, že
v~každom trojuholníku $ABC$ platí nerovnosť $\sqrt {\rho^2+\frac14{o^2}} \ge \rho+v$.
Tá sa dá po umocnení napísať do
krajšieho ekvivalentného tvaru $o^2 \ge 4v (2 \rho+v)$. Táto nerovnosť
sa dá tiež dokázať pomocou známych vzťahov $v= \frc {2S} {a}$,
$\rho = \frc {2S} {(b+c-a)}$ a~Herónovho vzorca
$16S^2 = (a+b+c) (a+b-c) (b+c-a) (c+a-b)$, pričom~$S$~označuje obsah trojuholníka~$ABC$.
Po sérii ekvivalentných úprav totiž vyjde nerovnosť $(b-c)^2 \ge 0$.


\ineres
Označme $I_a$ stred
kružnice~$k$ pripísanej ku~strane $BC$ a~$P$, $Q$, $R$ postupne jej dotykové
body s~priamkami $BC$, $CA$, $AB$.
Potom zrejme platí $|AQ|=|AR|$
a~$|AR|+|AQ|=|AB|+|BR|+|AC|+|CQ|=|AB|+|BP|+|AC|+|CP|=|AB|+|BC|+|CA|=o$,
takže obe úsečky $AQ$, $AR$ majú dĺžku $\frac12 o$.
Uvažujme ďalej kružnicu $l (A, v)$, ktorá sa zrejme dotýka priamky~$BC$ v~päte
výšky z~vrcholu~$A$. Priamka~$BC$ je tak spoločná (vnútorná) dotyčnica kružníc
$k$ a~$l$ (\obr).
Po tomto rozbore sa môžeme hneď pustiť do opisu konštrukcie.
\insp{a68.6}%

Najskôr zostrojíme pravouhlý trojuholník $AQI_a$ s~pravým uhlom pri
vrchole~$Q$, v~ktorom poznáme dĺžky odvesien $|AQ|= \frac12 {o}$
a~$|QI_a|= \rho$. Analogické vlastnosti má aj pravouhlý trojuholník
$ARI_a$, takže ľahko zostrojíme aj bod~$R$, napríklad ako
obraz bodu~$Q$ v~osovej súmernosti podľa priamky~$AI_a$.
Ďalej zostrojíme kružnice $k$ a~$l$.
Body $B$ a~$C$ potom nájdeme ako priesečníky úsečiek $AQ$
a~$AR$ s~ľubovoľnou zo spoločných vnútorných dotyčníc kružníc $k$ a~$l$
(ich konštrukcia je dobre známa školská úloha).

Overme, že takto zostrojený trojuholník $ABC$ spĺňa podmienky zo
zadania. Keďže sa priamka~$BC$ dotýka kružnice~$l$ (tá má stred v~$A$),
je vzdialenosť bodu~$A$ od priamky~$BC$ rovná~$v$. Kružnica~$k$ sa zrejme
dotýka všetkých troch priamok $AB$, $BC$, $CA$, a~keďže body
$I_a$ a~$A$ ležia v~opačných polrovinách vzhľadom na priamku~$BC$,
musí sa jednať o~kružnicu pripísanú k~strane~$BC$ (a~jej polomer je
tak naozaj~$\rho$). Napokon z~úvodného výpočtu vyplýva,
že obvod trojuholníka~$ABC$ je rovný dvojnásobku dĺžky úsečky~$AQ$, teda~$o$.

Už stačí len spraviť diskusiu o~počte riešení. Trojuholník
$AQI_a$ môžeme zostrojiť vždy,
bod~$R$ je potom určený jednoznačne a~to isté
samozrejme platí aj o~kružniciach $k$ a~$l$. Označme $r =|AI_a|$. Ak
$r>\rho+v$, kružnice $k$ a~$l$ nemajú spoločný bod (a~zrejme
žiadna z~nich \uv{neobsahuje druhú}), takže existujú práve dve
spoločné vnútorné dotyčnice, ktorým zodpovedajú dvojice bodov
$B_1$,~$C_1$ a~$B_2$,~$C_2$ (\obr). Trojuholníky $AB_1C_1$ a~$AB_2C_2$
(pri takom poradí vrcholov)
zrejme nie sú zhodné, takže v~tom prípade máme dve riešenia. Ak
$r = \rho+v$, tak sa kružnice $k$ a~$l$ dotýkajú, teda existuje
práve jedna vnútorná spoločná dotyčnica oboch kružníc, ktorej
zodpovedá jediný vyhovujúci trojuholník $ABC$. Napokon ak
$r <\rho+v$, tak spoločné vnútorné dotyčnice neexistujú, a~tak úloha
nemá riešenie. Hodnota~$r$ je pritom z~pravouhlého
trojuholníka $AQI_a$ rovná
$\sqrt {|QI_a|^ 2+|AQ|^2} = \sqrt {\rho^2+\frac14 {o^2}}$, takže
dostávame rovnaký výsledok diskusie ako v~predošlom riešení.
\insp{a68.7}%

\ineres
Stručne ukážeme ešte
iný spôsob, ako dokončiť predošlé riešenie.
Označme $T$ priesečník $AI_a$ a~$BC$ a~ďalej $D$
pätu výšky z~vrcholu~$A$. Trojuholníky $TAD$, $TI_aP$ sú
podobné (podľa vety $uu$, lebo sú pravouhlé a~zdieľajú
vrcholový uhol, \obr). Platí teda $|I_aT|:|TA|=|I_aP|:|AD|= \rho: v$.
\insp{a68.8}%
Po zostrojení štvoruholníka $AQI_aR$ preto môžeme zostrojiť bod~$T$
ako (jediný) bod úsečky~$I_aA$, ktorý ju delí v~danom pomere
z~predchádzajúcej vety. Body
$B$ a~$C$ následne nájdeme ako priesečníky ktorejkoľvek z~dotyčníc z~bodu~$T$
ku~kružnici~$k$ s~úsečkami $AR$ a~$AQ$. Pri dôkaze správnosti
konštrukcie spätne využijeme podobnosť trojuholníkov $TAD\sim TI_aP$
na dôkaz toho, že $|AD|= v$.

Diskusia o~počte riešení potom závisí od polohy bodu~$T$ vzhľadom
ku kružnici~$k$. Ak označíme $r =|I_aA|$,
nie je ťažké vypočítať, že $|I_aT|= \frc{\rho \cdot r}{(\rho+v)}$.
Ak ${|I_aT|> \rho}$, leží bod~$T$ vo vonkajšej oblasti kružnice~$k$, takže
existujú dve dotyčnice z~bodu~$T$ ku~$k$, ktoré zodpovedajú dvom
riešeniam. V~prípade $|I_aT|= \rho$ je táto dotyčnica jediná,
zatiaľ čo v~prípade $|I_aT|<\rho$ leží bod~$T$ vo vnútornej oblasti
kružnice~$k$, takže sa z~neho dotyčnica ku~$k$ viesť nedá. Ľahko sa
presvedčíme, že nájdené podmienky zodpovedajú podmienkam
z~predošlých riešení.

\poznamka
Bod~$T$ opísaný v~tomto
riešení je zrejme stredom rovnoľahlosti so záporným koeficientom, ktorá
zobrazuje $k$ na $l$ (pričom $l$ je kružnica z~predošlého riešenia). Zostrojiť
tento bod je prvým krokom jednej z~možných konštrukcií spoločných
vnútorných dotyčníc kružníc $k$ a~$l$.


\návody
Daný je trojuholník $ABC$. Uvažujme kružnicu
pripísanú k~strane~$BC$ tohto trojuholníka a~označme $P$, $Q$, $R$
dotykové body tejto kružnice s~priamkami $BC$, $CA$, $AB$. Dokážte,
že veľkosť úsečky $|AQ|$ je rovná polovici obvodu trojuholníka~$ABC$.
[Zrejme $|AQ|=|AR|$ a~platí
$|AR|+|AQ|=|AB|+|BR|+|AC|+|CQ|=|AB|+|BP|+|AC|+|CP|=|AB|+|BC|+|CA|$,
odkiaľ vyplýva dokazované tvrdenie.]

Dané sú dve kružnice $k_1$, $ k_2$ so stredmi v~bodoch
$O_1$, $ O_2$, ktoré nemajú spoločný bod a~žiadna neleží vo vnútornej oblasti druhej.
Pripomeňte si, ako zostrojiť spoločné vnútorné dotyčnice týchto
kružníc. [Tieto dotyčnice sa pretínajú v~strede rovnoľahlosti so
záporným koeficientom, ktorá prevádza jednu kružnicu na druhú.
Stačí nájsť tento stred a~zostrojiť z~neho dotyčnice. Stred
nájdeme napríklad takto: Vezmeme body $P \in k_1$, $Q \in k_2$
také, že $PO_1 \parallel QO_2$, pričom $P$, $ Q$ ležia v~navzájom
opačných polrovinách vzhľadom na~$O_1O_2$. Hľadaný stred
rovnoľahlosti je potom priesečníkom priamok $O_1O_2$ a~$PQ$.
Iný postup konštrukcie spoločných vnútorných dotyčníc daných kružníc
$k_1(O_1,r_1)$ a~$k_2(O_2,r_2)$ vyplýva z~toho, že tieto priamky
sú zrejme rovnobežné s~dotyčnicami z~bodu~$O_2$ k~pomocnej kružnici
$k_1'(O_1,r_1+r_2)$, ktorú ľahko zostrojíme použitím Tálesovej kružnice
s~priemerom~$O_1O_2$.]

Zostrojte trojuholník $ABC$, ak poznáte jeho obvod,
výšku na stranu~$BC$ a~veľkosť uhla~$\alpha$. [Označme $P$ a~$Q$
body na polpriamkach opačných k~$BC$ a~$CB$ také, že $|BP|=|BA|$
a~$|CQ|=|CA|$. Z~rovnoramenných trojuholníkov $BAP$ a~$CAQ$ ľahko
spočítame, že $|\angle BAP|= \frac12 {\beta}$ a~$|\angle
CAQ|= \frac12 {\gamma}$. Tým pádom $|\angle PAQ|=
\frac12 {\beta}+\frac12 {\gamma}+\alpha = 90^\circ+\frac12 {\alpha}$.
V~trojuholníku $APQ$ poznáme stranu~$PQ$, uhol pri vrchole~$A$ a~tiež
výšku na stranu~$A$, takže ho vieme zostrojiť. Body $B$, $C$ následne
nájdeme ako priesečníky strany~$PQ$ a~osí úsečiek $AP$ a~$AQ$.]

\D
Daný je trojuholník $ABC$. Kružnica jemu vpísaná sa
dotýka strán $BC$, $CA$, $AB$ v~bodoch $P$, $ Q$, $ R$. Dokážte, že
$|AQ|= s-a$, pričom $a=|BC|$ a~$s$ je polovica obvodu trojuholníka $ABC$
(ďalšie dve rovnosti analogicky).
[Postupujeme podobne ako v~úlohe~N1. Platí $|AQ|=|AR|$
a~$|AQ|+|AR|=|AB|-|BR|+|AC|-|CQ|=|AB|-|BP|+|AC|-|CP|=|AB|+|AC|-|BC|$,
z~čoho už vyplýva dokazované tvrdenie.]

Dokážte, že v~pravouhlom trojuholníku $ABC$ s~pravým
uhlom pri vrchole~$A$ je polomer kružnice vpísanej rovný $s-a$, pričom
$a=|BC|$ a~$s$ je polovica obvodu $ABC$. [Označme $I$ stred jeho vpísanej kružnice
a~$P$, $ Q$ dotykové body s~odvesnami $AB$, $AC$. Štvoruholník
$APIQ$ je štvorec, takže $|IQ|=|AP|$. Z~výsledku úlohy~D1 vyplýva, že
$|AP|= s-a$.]

Daný je trojuholník $ABC$. Kružnica jemu vpísaná sa
dotýka strany~$BC$ v~bode~$D$. Kružnica pripísaná k~jeho strane~$BC$ sa
jej dotýka v~bode~$E$. Dokážte, že $D$, $ E$ sú súmerne združené
podľa stredu~$BC$. [Z~úlohy~D1 vieme, že $|BD|= s-b$. Ostáva
dokázať, že $|CE|= s-b$. Ak označíme $F$ dotykový bod pripísanej
kružnice s~priamkou~$AC$, tak podľa úlohy~N1 máme $|AF|= s$.
Teda $|CE|=|CF|=|AF|-|AC|= s-b$.]

Dokážte, že v~dotyčnicovom štvoruholníku $ABCD$ je
súčet veľkostí jeho protiľahlých strán rovnaký.
[Predpokladajme, že štvoruholník je dotyčnicový, a~označme postupne
$P$, $Q$, $R$, $S$ dotykové body vpísanej kružnice
so stranami $AB$, $BC$, $CD$, $DA$. Potom
$|AB|+|CD|=|AP|+|PB|+|CR|+|RD|=|AS|+|BQ|+|CQ|+|DS|=|BC|+|AD|$.]

Daný je trojuholník $ABC$. Kružnica~$k$ jemu vpísaná
sa dotýka strany~$BC$ v~bode~$D$. Nech $E$ je obraz bodu~$D$
v~stredovej súmernosti podľa stredu~$BC$. Úsečka~$AE$ pretína
kružnicu~$k$ v~dvoch bodoch, označme $F$ ten, ktorý je bližšie
k~bodu~$A$. Dokážte, že $DF \perp BC$. [Podľa úlohy~D3 je bod~$E$
dotykový bod kružnice~$l$ pripísanej k~strane~$BC$. Nech $I_a$ je
stred tejto kružnice a~$I$ stred $k$. Bod~$A$ je stredom
rovnoľahlosti, ktorá zobrazuje $l$ na $k$. V~tejto rovnoľahlosti sa
$E$ zobrazuje na $F$, takže $IF \parallel I_aE$. Lenže $I_aE$ je
priamka kolmá na $BC$.]
\endnávod}

{%%%%%   A-I-6
Najskôr očíslujme vrcholy postupne v~jednom smere
$0, 1, \ldots, n-1$ tak, aby vo vrchole~$0$ bola pasca. Ak stojí figúrka na
vrchole~$a$ a~Tom povie číslo~$b$, tak vrcholy, v~ktorých po tomto ťahu
môže byť figúrka, zodpovedajú zvyškom čísel $a-b$ a~$a+b$ po
delení číslom~$n$~-- tieto zvyšky budeme označovať ako $(a-b)_n$, resp.
$(a+b)_n$.

Predpokladajme najskôr, že $n$ má nepárneho deliteľa~$d$ väčšieho ako~1.
Dokážeme, že potom môže Jerry urobiť ťah figúrkou tak, aby nikdy
neskončila v~pasci.
Jeho stratégia bude nasledujúca:

Na začiatku položí figúrku na ľubovoľný vrchol, ktorý nie je
deliteľný~$d$,\footnote{Takú vlastnosť má pre ľubovoľné $d>1$
napríklad vrchol s~číslom~1.}
a~ďalej bude vyberať smer posunu figúrky tak,
aby nikdy neskončila na vrchole s~číslom deliteľným~$d$. Dokážeme, že
takú možnosť má Jerry vždy, teda
jeho figúrka nikdy neskončí v~pasci, lebo jej číslo~0 je deliteľné~$d$.

Ak je figúrka na vrchole s~číslom~$a$ nedeliteľným $d$ a~Tom povie číslo~$b$,
tak aspoň jedno z~čísel $p = (a+b)_n$ alebo $q = (a-b)_n$ nie je deliteľné~$d$.
Pre vhodné celé čísla $p'$,~$q'$ je totiž
$a+b = p'n+p$ a~$a-b = q'n+q$ a~sčítaním oboch rovností dostaneme
$2a = n ({p'+ q'})+(p+q)$. Ak by obe čísla $p$ a~$q$ boli deliteľné~$d$,
bola by pravá strana poslednej rovnosti deliteľná~$d$ ($d$ delí~$n$),
zatiaľ čo číslo~$2a$ na ľavej strane deliteľné číslom~$d$ nie je
($d$ je podľa predpokladu nepárne číslo, ktoré nedelí~$a$).
Aspoň jeden z~dvoch možných
Jerryho ťahov je teda taký, že sa figúrka posunie na vrchol s~číslom
nedeliteľným~$d$, a~tým pádom nikdy neskončí v~pasci.

Ostáva teda vyšetriť prípad, keď $n$ nemá nepárneho deliteľa, teda
$n = 2^k$ pre nejaké prirodzené číslo $k\ge2$.
Dokážeme, že v~tomto prípade Tom po konečnom počte
ťahov dokáže dostať figúrku do pasce.

Pre jednoduchšie vyjadrovanie budeme hovoriť, že vrchol $n$-uholníka, ktorý nie je pascou,
má stupeň~$b$, ak bude $2^b$ najvyššia mocnina čísla~2, ktorá ešte
delí jeho nenulové číslo.

Teraz opíšeme Tomovu stratégiu pre $n = 2^k$, keď vidí
pozíciu figúrky. Predpokladajme, že figúrka je aktuálne na
vrchole stupňa~$p$. Taký vrchol má číslo $2^p q$, pričom $q$ je
nepárne. V~takom prípade Tom povie $2^p$. Dokážeme, že
vďaka tomu dostane figúrku buď do pasce, alebo na vrchol, ktorého
stupeň bude vyšší ako~$p$. To už Tomovi zaručí, že po
konečnom počte krokov vyhrá, lebo všetky vrcholy majú čísla menšie
ako~$n = 2^k$, teda ich stupne sú menšie ako~$k$, a~tak sa
stupeň vrcholu, kam sa figúrka posunie, nemôže stále iba zväčšovať.

Vďaka Tomovmu ťahu $2^p$ figúrka skončí na vrchole s~číslom
$r = (2^p (q \pm 1))_n$, pričom znamienko zodpovedá smeru,
ktorým Jerry figúrkou ťahal. Ak $r = 0$, je figúrka
v~pasci. Ak $r \ne 0$, môžeme písať $2^p (q \pm
1) = nr'+ r$. Číslo $q \pm 1$ je v~každom prípade párne, takže
ľavá strana uvedeného vzťahu je deliteľná číslom $2^{p+1}$.
A~keďže $p \le k-1$, delí mocnina $2^{p+1}$ aj číslo $n = 2^k$,
teda delí aj druhý sčítanec~$r$ z~pravej strany.
Keďže $r \ne0$, znamená to, že stupeň vrcholu s~číslom~$r$ je
vyšší ako~$p$. To sme potrebovali zdôvodniť.

Teraz dokážeme, že pre $n = 2^k$ dokáže Tom po konečnom počte krokov
dostať figúrku do pasce bez toho, aby jej pozíciu videl.
Jeho stratégia sa v~predchádzajúcom prípade opierala
o~znalosť stupňa jej vrcholu. Aby sa tohto problému
zbavil, potrebuje predovšetkým mať pre každú možnú hodnotu~$p$
stupňa jej vrcholu pripravenú stratégiu (postupnosť ťahov), ktorá
mu zaručí úspech~-- takú postupnosť ťahov označme~$S(p)$.
Dohodnime sa ešte, že "ťahom~$b$" v~týchto stratégiách nazveme
Tomov krok, keď povie práve dotyčné číslo~$b$.

Zamyslime sa teraz nad tým, ako majú vyzerať jednotlivé stratégie~$S(p)$.
Použijeme pritom poučenie z~predošlej časti, pričom sme
dokázali, že ak pri figúrke na vrchole stupňa~$p$ Tom použije ťah~$2^p$,
prejde následne figúrka buď do pasce, alebo na vrchol
stupňa vyššieho ako~$p$.

Je jasné, že stratégia $S(k-1)$ spočíva v~jedinom ťahu $2^{k-1}$.
Vrchol stupňa $k-1$ je totiž jediný a~má číslo $\frac12n=2^{k-1}$,
takže po ťahu $2^{k-1}$ sa z~neho figúrka (ľubovoľným smerom) premiestni do pasce.

Podobne stratégia $S(k-2)$ začína ťahom $2^{k-2}$. To Tomovi buď
zaručí výhru, alebo sa figúrka dostane na vrchol s~vyšším
stupňom, teda nutne na vrchol stupňa $k-1$. Preto ak po ťahu
$2^{k-2}$ Tom použije stratégiu~$S(k-1)$, určite vyhrá,
keďže $S(k-1)$ je z~definície víťazná stratégia pre
vrchol stupňa $k-1$.

Analogicky stratégia $S(k-3)$ začne ťahom $2^{k-3}$. Tento ťah
Tomovi buď prinesie výhru, alebo dostane figúrku na vrchol stupňa $k-1$, alebo
na vrchol stupňa $k-2$. Pre obe tieto možnosti už síce má Tom stratégiu,
nevie ale, ktorá z~nich nastala. Kľúčové pozorovanie je však toto:

Stratégia $S(k-1)$ buď zafunguje, alebo nezmení stupeň vrcholu.
Naozaj, ak je figúrka na vrchole stupňa $k-2$, \tj. na vrchole s~číslom
tvaru $2^{k-2} q$ (pričom $q$ je nepárne), potom vrchol po
prevedení stratégie $S(k-1)$ (obsahujúcej iba ťah
$2^{k-1}$) bude mať číslo $r = (2^{k-2} (q \pm 2))_n$. Z~vyjadrenia
$2^{k-2} (q \pm 2) = nr'+ r$ totiž vidíme, že $2^{k-2}$
delí~$r$ (keďže delí ľavú stranu uvedeného vzťahu aj $n = 2^k)$,
zatiaľ čo $2^{k-1}$ už~$r$ nedelí (lebo $q\pm2$ je nepárne),
a~to znamená, že vrchol s~číslom~$r$ má stupeň $k-2$.

Vďaka tomu vidíme, že po ťahu $2^{k-3}$, ktorý neviedol k~výhre, má vždy
zmysel použiť stratégiu $S(k-1)$. Ak je figúrka na vrchole stupňa $k-1$,
je víťazná. V~opačnom prípade je na vrchole stupňa $k-2$, a~to
aj po prevedení stratégie $S(k-1)$, takže následné prevedenie
stratégie $S(k-2)$ zaručí, že Tom figúrku dostane do pasce.

Uvedomme si, že naše kľúčové pozorovanie vychádza zo
všeobecného faktu, že ak je figúrka na vrchole stupňa~$p$, tak po
prevedení ťahu $2^{p'}$, pričom $p'> p$, zostane figúrka na vrchole stupňa~$p$.
To dokážeme analogicky: Ak je figúrka na vrchole s~číslom $2^p q$,
pričom $q$ je nepárne, tak po ťahu $2^{p'}$ je na vrchole s~číslom
$r = (2^p (q+2^{p'-p}))_n$, platí teda $2^p(q+2^{p'-p}) = nr'+ r$,
takže $2^p$ delí ľavú stranu a~aj $n$, z~čoho vyplýva,
že delí aj~$r$, avšak $2^{p+1}$ nedelí ľavú
stranu a~delí~$n$, preto nemôže deliť~$r$, z~čoho už vyplýva, že
$r$ je číslo vrcholu stupňa~$p$.

Teraz sme pripravení formálne opísať stratégiu $S(p)$, pričom $p$ je
dané celé číslo spĺňajúce $0 \le p \le k-1$. Túto stratégiu
definujeme zostupnou matematickou indukciou:
\smallskip
\ite (i) $S(k-1)$ pozostáva z~ťahu $2^{k-1}$.
\ite (ii) Ak celé $l$ spĺňa $0 \le l \le k-2$, definujme,
že $S(l)$ pozostáva z~ťahu $2^l$ nasledovaného ťahmi zo
stratégií $S(k-1), S(k-2), \dots, S(l+1)$.
\smallskip

Všimnime si, že z~tejto definície vyplýva, že stratégia $S(p)$
pozostáva iba z~ťahov tvaru~$2^t$, pričom $t \ge p$. To môžeme
formálne dokázať zostupnou matematickou indukciou:

Ak $p = k-1$, je tvrdenie zrejmé. Predpokladajme, že je dané $p$
také, že tvrdenie platí pre čísla $k-1, k-2, \ldots, p+1$.
Dokážeme, že potom platí aj pre $p$. Stratégia $S(p)$ pozostáva
z~ťahu $2^p$ (pre ktorý dokazované tvrdenie zrejme platí), a~z~ťahov
zo stratégií $S(k-1), S(k-2), \ldots, S(p+1)$. Z~indukčného
predpokladu máme, že pre každé~$l$ spĺňajúce $k-1 \ge l \ge p+1$
stratégia $S(l)$ obsahuje iba ťahy tvaru $2^t$, pričom $t \ge l$.
Keďže $l \ge p+1$, je $t \ge p+1 \ge p$. Všetky tieto ťahy sú
teda požadovaného tvaru, čo dokazuje tvrdenie pre~$p$, čím je
dôkaz indukciou ukončený.

Pripomeňme naše pozorovanie, ktoré hovorí, že ak je
figúrka na vrchole stupňa $p$ a~my urobíme ťah $2^{p'}$, pričom $p'> p$,
bude figúrka opäť na vrchole stupňa~$p$. Vďaka nemu dostávame, že
aplikovaním celej stratégie $S(p')$, ktorá pozostáva iba
z~požadovaných ťahov, sa táto skutočnosť nezmení.

Vďaka tomu vidíme, že stratégia $S(p)$ naozaj dostane
figúrku do pasce, ak na začiatku stojí na
vrchole stupňa~$p$. Formálne to dokážeme zostupnou matematickou indukciou:

Pre $p = k-1$ sme to zdôvodnili už predtým. Predpokladajme, že
číslo~$p$ je také, že tvrdenie platí pre čísla $k-1, k-2, \ldots, p+1$.
Dokážeme, že potom platí aj pre~$p$. Predpokladajme, že sme na
vrchole stupňa~$p$. Po aplikovaní prvého ťahu $2^p$ je figúrka buď
v~pasci, alebo sa dostane na vrchol stupňa~$p'$, pričom $p'> p$. V~druhom prípade
stratégie $S(k-1), S(k-2), \ldots, S(p'+ 1)$ nezmenia fakt, že sme na
vrchole stupňa~$p'$, a~následne stratégia $S(p')$ z~indukčného predpokladu
zafunguje a~dostane figúrku do pasce. Dôkaz indukciou je teda
ukončený.

Posledný krok, ktorý ostáva, je definovať Tomovu finálnu
stratégiu. Tá bude pozostávať z~postupného prevedenia stratégií
$S(k-1), S(k-2), \ldots, S(0)$. Teraz už ľahko zdôvodníme,
že táto postupnosť stratégií bude vždy úspešná~-- ak je totiž
figúrka na začiatku na vrchole stupňa~$h$, tak stratégie
$S(k-1), S(k-2), \ldots, S(h+1)$ tento fakt nezmenia a~následne zafunguje
stratégia~$S(h)$. Tým je ukončená posledná časť úlohy.

\zaver
Pre $n$, ktoré je mocninou dvoch, má vyhrávajúcu stratégiu Tom,
pričom nemusí vidieť pozíciu figúrky. Pre všetky ostatné~$n$ má
vyhrávajúcu stratégiu Jerry.

\poznamka
Keby Tom okrem toho, že nevidí
figúrku, nepoznal ani číslo~$n$, tak by v~prípade, že $n$ je
mocnina dvoch, mohol aj~tak vyhrať v~konečnom čase. Stačilo by mu
skúšať postupne svoju stratégiu pre všetky mocniny dvoch.

\poznamka
Vráťme sa ešte k~prípadu, keď
$n$ nie je mocnina dvoch, \tj. je tvaru $2^kl$, pričom $l>1$ je
nepárne. Z~nášho riešenia vyplýva, že všetky
počiatočné pozície, pre ktoré Jerryho stratégia nefunguje, sú
násobky~$l$. V~týchto pozíciách naozaj vyhráva Tom, pričom nemusí
vidieť pozíciu figúrky~-- stačí už opísanú stratégiu
modifikovať tak, že všade, kde je ťah $2^a$, bude ťah $2^a~l$.


\ineres
Ukážeme iné riešenie
v~prípadoch, keď Tom vidí figúrku. Úlohu si môžeme preformulovať
tak, že hráme na celočíselnej osi, pričom pasce sú čísla
deliteľné~$n$ (iné čísla ako celé v~našom riešení neuvažujeme).
Číslo nazveme {\it zlé}, ak existuje postupnosť
ťahov, ktorá dostane na ňom stojacu figúrku do pasce, nech hrá Jerry akokoľvek.
Všetky ostatné čísla nazveme {\it dobré}.
Z~definície sú všetky čísla deliteľné~$n$ zlé.

Uvedomme si, že Jerry má vyhrávajúcu stratégiu práve vtedy, keď
existuje aspoň jedno dobré číslo~$a$. V~tom prípade mu stačí
položiť figúrku naň. Nech už Tom povie akékoľvek číslo, je
aspoň jedno z~oboch čísel, na ktoré sa figúrka môže premiestniť,
dobré, keďže v~opačnom prípade by z~definície zlého čísla
platilo, že číslo~$a$ je zlé, čo je spor. Ak sú naopak všetky
čísla zlé, tak z~ich definície sú všetky možné pozície
figúrky pre Jerryho prehrávajúce.

Ďalej platí, že ak sú dve rôzne čísla $a$ a~$b$ zlé a~číslo
$(a+b) / 2$ je celé, je zlé. Naozaj, ak je $(a+b) / 2$ celé,
majú čísla $a$ a~$b$ rovnakú paritu, takže aj číslo $(b-a) / 2$ je
celé. Ťah $(b-a) / 2$ dostane figúrku z~čísla $(a+b) / 2$ na jedno
z~čísel $a$ alebo~$b$, takže číslo $(a+b) / 2$ je naozaj zlé. Naopak
je ľahko vidno, že ak je číslo zlé, je to preto, že je
buď deliteľné číslom~$n$, alebo je aritmetickým priemerom dvoch zlých
čísel. S~týmito pozorovaniami už dokážeme opísať všetky zlé
čísla.

Predpokladajme, že $n$ má nepárneho deliteľa~$d$ väčšieho ako~1.
Na začiatku sú teda všetky zlé čísla deliteľné~$d$ (keďže sa jedná
o~čísla deliteľné~$n$). Ak máme dve zlé čísla $a$ a~$b$
deliteľné~$d$ a~ich aritmetický priemer je celé číslo, je
aj tento priemer deliteľný~$d$ (keďže $d$ je nepárne). Všetky
zlé čísla teda musia byť deliteľné~$d$, preto existuje nejaké
dobré číslo, teda v~tomto prípade má Jerry vyhrávajúcu stratégiu.

V~druhom prípade je $n = 2^k$ pre nejaké prirodzené číslo~$k$.
Dokážeme, že v~tomto prípade sú všetky celé čísla zlé.
Použijeme na to nasledujúcu úvahu: Pre všetky celé čísla~$a$
a~prirodzené čísla~$l$ platí, že ak sú krajné body intervalu
$\langle a, {a+2^l}\rangle$ zlé, sú aj všetky čísla tohto intervalu
zlé. Toto tvrdenie dokážeme matematickou indukciou podľa~$l$:

Pre $l = 1$ máme interval $\langle a, a+2 \rangle$. Jeho jediný
vnútorný bod $a+1$ je zrejme priemerom krajných bodov, ktoré sú
zlé, takže aj číslo $a+1$ je zlé. Predpokladajme, že tvrdenie
platí pre $l$ a~všetky celé~$a$. Dokážeme, že potom platí aj pre
$l+1$. Máme dokázať, že všetky čísla intervalu $\langle a,
a+2^{l+1} \rangle$ sú zlé. Keďže podľa predpokladu sú krajné
body zlé, je aj stred tohto intervalu rovný $a+2^l$ zlý.
Interval sa rozdelil na dva podintervaly $\langle a, a+2^l \rangle$
a~$\langle a+2^l, a+2^{l+1} \rangle$. Na oba tieto intervaly však môžeme
použiť indukčný predpoklad.
Všetky vnútorné
body týchto intervalov sú teda zlé, takže aj celý interval
$\langle a, a+2^{l+1} \rangle$ je zlý. Tým je dôkaz indukciou
ukončený.

Teraz už stačí len aplikovať toto pomocné tvrdenie na intervaly
${\langle 2^k~l, 2^k(l+1) \rangle}$, pričom $l$ je celé
číslo. Ich krajné body sú zlé, keďže to sú násobky $n = 2^k$.
Dĺžka týchto intervalov je mocnina dvoch, takže nutne všetky
čísla týchto intervalov sú zlé. Týmito intervalmi ale pokrývame
všetky celé čísla, teda dôkaz, že všetky celé čísla sú zlé, je hotový.

\poznamka
Stratégiu sme v~tomto riešení
opísali implicitne. Dá sa však spätne odvodiť. V~prvom prípade
stačí každej dvojici $(a, b)$, pričom $a$ je dobré číslo a~$b$ je
ľubovoľné, priradiť taký Jerryho ťah, pri ktorom figúrka
skončí na dobrom čísle. V~druhom prípade treba pre každé
zlé číslo zaznačiť ťah, ktorý ho dostane na zlé číslo, kvôli
ktorému je zlé (uvedený ťah $(b-a) / 2$ v~riešení). Je
vidno, že takto explicitne opísané stratégie sú rovnaké ako
stratégie z~predchádzajúceho riešenia.

\poznamka
Podobne ako v~predošlom riešení
možno aj tu pomocou predvedených úvah opísať, ako vyzerá situácia
pre všetky možné počiatočné pozície figúrky. V~kontexte tohto
riešenia to znamená pre všeobecné~$n$ opísať všetky dobré
čísla. Skombinovaním úvah z~riešení sa dá dokázať, že ak
$n = 2^k~l$, pričom $l$ je nepárne číslo, sú dobré
{\it práve\/} tie čísla, ktoré nie sú násobkami~$l$.



\návody
Vyriešte úlohu pre nepárne čísla $n$. [V~tomto
prípade vždy vyhráva Jerry, keďže sa nikdy neocitne na políčku,
ktoré je z~oboch strán rovnako vzdialené od pasce, takže vždy aspoň
jeden z~jeho dvoch ťahov nevedie do pasce.]

Dokážte, že bez ujmy na všeobecnosti môžeme
predpokladať, že Tom volí iba čísla nanajvýš rovné $n / 2$.
[V~prvom rade Tomovi stačí voliť čísla nanajvýš rovné~$n$,
keďže čísla $a$, $a+n$ zrejme majú rovnaký efekt. Tiež je
vidno, že aj čísla $a$, $n-a$ majú rovnaký efekt. Tomovi teda
stačí voliť vždy menšie číslo z~dvojice $a$, $n-a$. To
je zrejme nanajvýš rovné~$n / 2$.]

Vyriešte úlohu pre $n = 6$. [Očíslujme vrcholy
$0, 1, \ldots, 5$ tak, že $0$ je pasca. Figúrka nikdy
nemôže byť na vrchole s~číslom~3, keďže ťahom~3 by nutne
skončila v~pasci. Podľa úlohy~N2 stačí predpokladať, že
Tom hovorí iba čísla $1$, $2$, $3$. Uvedomte si, že pre každý
zo zvyšných vrcholov $1$, $2$, $4$, $5$ a~pre každé Tomom zadané číslo
$1$, $2$,~$3$ môže Jerry spraviť ťah, že figúrka neskončí na vrcholoch
s~číslami 3 a~6, takže nemôže prehrať.]

Vyriešte úlohu pre $n = 4$. [Očíslujme vrcholy ako
v~úlohe~N3. Rozdeľme si vrcholy, ktoré nie sú pascami, do dvoch skupín $A= \{2\}$
a~$B = \{1, 3\}$. Ak je figúrka v~skupine~$A$, Tom ťahom~$2$ vyhrá.
Ak je figúrka v~skupine~$B$, vyhrá buď ťahom~$1$, alebo ju
dostane do skupiny~$A$, pričom vyhrá ťahom~$2$. Ak by nevidel pozíciu
figúrky, vyskúša najskôr ťah~2. Týmto ťahom buď vyhrá, alebo
bude figúrka v~skupine~$B$, pričom zostane aj po tomto ťahu. Vtedy mu
stačí použiť stratégiu pre skupinu~$B$, \tj. povedať čísla
1, 2. Finálna stratégia v~prípade, že nevidí figúrku, je teda
$2$, $1$, $2$.]

Tom zvolí k~danej hre (pri ktorej na hrací plán s~$n$-uholníkom vidí)
jednoduchú stratégiu: v~každom kroku povie také číslo menšie ako~$n$,
pri ktorom sa následne figúrka ocitne v~pasci, ak ju posunie Jerry proti
smeru chodu hodinových ručičiek. Dokážte, že Tom zaručene vyhrá v~prípade $n=8$.
[Vypisujte postupne polohy figúrky
od víťazného konca,
a~dokážte, že takto dostanete všetky možné východiskové polohy figúrky.
Dokážete objaviť, ktoré ďalšie $n$ budú mať rovnakú vlastnosť?]

\D
Ako sa zmení odpoveď v~úlohe, ak má Tom zakázanú
konečnú množinu čísel, \tj. nesmie ich povedať? [Odpoveď sa
nezmení. Ak Tom používa vo svojej stratégii zakázané číslo~$a$,
stačí ho nahradiť číslom $a+kn$ pre nejaké prirodzené~$k$.
Keďže je zakázaných konečne veľa čísel, nejaké číslo
tohto tvaru zakázané nebude. Takú výmenu urobíme so všetkými
zakázanými číslami, ktoré sa môžu v~jeho stratégii vyskytnúť.]

Ako sa zmení odpoveď, ak Tom nevidí figúrku
a~zároveň nepozná číslo~$n$? [Odpoveď sa nezmení.]

Ako sa zmení odpoveď v~prípade, že figúrka je na
začiatku položená na niektorom konkrétnom políčku?

Štyri poháre sú umiestnené do rohov štvorcovej tácky.
Každý je umiestnený dnom nahor alebo dnom nadol. Slepá osoba je
posadená pred tácku a~má za úlohu prevracať poháre tak, aby boli
všetky otočené rovnakým smerom. Poháre sú prevracané nasledujúcim
spôsobom: V~jednom ťahu môžu byť uchopené ľubovoľné dva poháre,
pričom osoba cíti ich
orientáciu a~môže buď niektorý z~nich otočiť opačným smerom,
alebo oba z~nich, alebo žiadny. Následne je tácka otočená
o~celočíselný násobok uhla $90^\circ$. Ako má osoba postupovať pri
prevracaní? [\hbox{\tt https://en.wikipedia.org/wiki/Four\_glasses\_puzzle}]
\endnávod}

{%%%%%   B-I-1
V~prvej časti riešenia budeme predpokladať, že sme v~hľadanom
osemcifernom čísle~$N$ vyškrtli prvé štyri cifry. Tie tvoria
štvorciferné číslo, ktoré označíme~$A$. Zvyšné štyri cifry čísla~$N$
tvoria štvorciferné číslo~$B$, ktoré je spomenuté v~zadaní, pritom
zrejme platí $N=10^4A+B$. Požadovaná vlastnosť je preto vyjadrená
rovnicou $10^4A+B=2\,019B$, čiže $5\,000A=1\,009B$. Keďže čísla
$5\,000$ a~$1\,009$ sú nesúdeliteľné, musí byť štvorciferné číslo~$B$
násobkom (tiež štvorciferného) čísla $5\,000$, ktorého dvojnásobok
je už však päťciferný. Podmienke deliteľnosti tak vyhovuje jediné
číslo $B=5\,000$. Preň z~rovnice $5\,000A=1\,009B$ vychádza $A=1\,009$,
takže celkom $N=10\,095\,000$ $(=2\,019\times5\,000)$.

V~druhej časti riešenia budeme uvažovať súhrnne všetky ďalšie
prípady povoleného škrtania štyroch cifier osemciferného čísla~$N$.
Vtedy medzi nimi nie je jeho prvá cifra, ktorú označíme ako~$a$.
Ukážeme, že po takom škrtaní sa číslo~$N$ zmenší viac ako $5\,000$\spojovnik{}krát (takže nemôže byť riešením
úlohy).

Naozaj, číslo~$N$ s~dekadickým zápisom $N=10^7a+\dots$
sa zmení (po uvažovanom škrtnutí 4~cifier)
na číslo so zápisom $10^3a+\dots$,\footnote{Tri bodky
znamenajú vklady nižších mocnín základu~10, v~oboch
prípadoch vo všeobecnosti rozdielne. Pre ďalšie úvahy nie je ich zloženie
podstatné.}
ktoré je určite menšie ako $10^3(a+1)$,
pre avizované zmenšenie tak stačí dokázať
nerovnosť $10^7a\ge5\,000\cdot10^3(a+1)$. Tá však platí, lebo
po vydelení oboch strán číslom $5\cdot10^{6}$ prejde na nerovnosť
$2a\ge a+1$, čiže $a\ge1$. Stačí dodať, že prvá cifra~$a$
zápisu čísla~$N$ je vždy rôzna od nuly.

\odpoved
Hľadané osemciferné číslo je jediné, a~to $10\,095\,000$.


\návody
Určte všetky štvorciferné čísla, z~ktorých po
vyškrtnutí prvého dvojčíslia dostaneme dvojciferné číslo, ktoré je
69-krát menšie. [1725, 3450 a~5175. Hľadané číslo je tvaru
$10^2A+B$, pričom dvojciferné čísla $A$, $B$ spĺňajú rovnicu
$10^2A+B=69B$, čiže $25A=17B$. Vďaka nesúdeliteľnosti čísel $17$
a~$25$ z~toho vyplýva, že $B$ je násobkom čísla 25.]

Dokážte, že ak v~štvorcifernom čísle vyškrtneme dve
z~jeho posledných troch cifier, dostaneme číslo viac ako 50-krát
menšie. [Pôvodné číslo je tvaru $\overline{c{\star}{\star}{\star}}$,
zmenšené je tvaru $\overline{c{\star}}$, takže prvé z~nich je aspoň
$10^3c$, zatiaľ čo druhé číslo je menšie ako $10(c+1)$. Preto stačí
overiť nerovnosť $10^3c\ge50\cdot10(c+1)$, tá je ale splnená
pre každé $c\ge1$.]

\D
Nájdite všetky štvorciferné čísla $\overline{abcd}$, pre ktoré platí
$\overline{abcd}=20\cdot\overline{ab}+16\cdot\overline{cd}$. [65--C--S--1]

Určte najväčšie dvojciferné číslo~$k$ s~nasledujúcou vlastnosťou:
existuje prirodzené číslo~$N$, z~ktorého po škrtnutí prvej číslice
zľava dostaneme číslo $k$-krát menšie. (Po škrtnutí číslice môže
zápis čísla začínať jednou či niekoľkými nulami.) K~určenému
číslu~$k$ potom nájdite najmenšie vyhovujúce číslo~$N$.
[56--C--II--4]

Pre ktoré racionálne číslo $r>1$ existuje najviac
tých štvorciferných čísel, z~ktorých vyškrtnutím prvého dvojčíslia
dostaneme dvojciferné číslo, ktoré je $r$-krát menšie? [Pre
$r=101$. Podľa označenia z~úlohy N1 hľadáme také $r$, pre ktoré
existuje najväčší počet dvojíc $A$, $B$ dvojciferných čísel
spĺňajúcich rovnicu $10^2A+B=rB$, čiže $10^2A=(r-1)B$. Číslo~$B$
je pri danom~$r$ číslom~$A$ jednoznačne určené, pritom
$A\in\{10,11,\dots,99\}$. Z~toho vyplýva, že pre každé $r$ je
vhodných čísel nanajvýš 90, pritom tento počet
získame práve vtedy, keď pre každé dvojciferné~$A$ bude
$B=10^2A/(r-1)$ tiež dvojciferné celé číslo. To je splnené iba pre
$r=101$, keď $B=A$, takže všetkých 90 vyhovujúcich štvorciferných
čísel je tvaru $\overline{abab}$.]
\endnávod}

{%%%%%   B-I-2
Úloha je o~známom pravouhlom trojuholníku s~odvesnami dĺžok 3, 4 a~preponou (označenou ako $BC$) dĺžky~5,
ktorého obsah je $\frac12\cdot3\cdot4=6$. Označme ešte $P$ pätu
kolmice z~bodu~$N$ na priamku~$BC$ (\obr). Keďže bod~$N$
leží podľa zadania na osi uhla $ABC$, platí $|NA|=|NP|$.
\insp{b68.1}%

V~celom riešení budeme hľadať vzťahy medzi obsahmi rôznych častí
trojuholníka~$ABC$. Keďže $M$ je stred strany~$BC$, platia rovnosti
$$
S_{ABM}=S_{ACM}=\dfrac12 S_{ABC}=3\quad\text{a}\quad
S_{BMK}=S_{CMK},
$$
lebo v~oboch prípadoch sa jedná o~dvojicu trojuholníkov so zhodnou výškou
na zhodné základne. Porovnanie základní a~výšok dvoch trojuholníkov vedie
aj k~úmerám
$$
S_{ABN}:S_{CBN}=S_{AKN}:S_{CKN}=|AN|:|CN|=4:5,
$$
pretože v~oboch dvojiciach sú zapísané trojuholníky so zhodnými výškami
na svoje základne $AN$ a~$CN$, pritom trojuholníky v~prvej dvojici
$ABN$, $CBN$ majú navyše zhodné výšky
$NA$, resp. $NP$ zo spoločného vrcholu~$N$, takže pomer ich obsahov je
naozaj rovný pomeru $4:5$ dĺžok ich základní $AB$ a~$CB$.
Zo známej hodnoty $S_{ABN}+S_{CBN}=S_{ABC}=6$ teda vyplýva
$$
S_{ABN}=\frac83\quad\text{a}\quad
S_{CBN}=\frac{10}{3}.
$$

Naše riešenie založíme na výpočte neznámeho obsahu trojuholníka~$AKN$, ktorý
označíme $S$ ako na obrázku a~pre ktorý zo skôr odvodených
vzťahov teraz zostavíme rovnicu. Za tým účelom postupne vyjadríme
$$\align
S_{ABK}&=S_{ABN}-S_{AKN}=\frac83-S,\\
S_{BMK}&=S_{ABM}-S_{ABK}=3-\Bigl(\frac83-S\Bigr)=S+\frac13,\\
S_{BCK}&=2S_{BMK}=2S+\frac23,\\
S_{CKN}&=S_{CBN}-S_{BCK}=\frac{10}{3}-\Bigl(2S+\frac23\Bigr)=
\frac83-2S.
\endalign
$$
Dosadením do úmery $S_{AKN}:S_{CKN}=4:5$ tak dostaneme rovnicu
$$
\frac{S}{\frac83-2S}=\frac45,
$$
z~ktorej vychádza $S=\frac{32}{39}$. Obsahy, ktoré máme porovnať, tak
majú hodnoty
$$
\align
S_{ABK}&=S_{ABN}-S_{AKN}=\frac83-S=\frac{72}{39},\\
S_{CNKM}&=S_{ACM}-S_{AKN}=3-S=\frac{85}{39}.
\endalign
$$

\odpoved
Hľadaný pomer je $72:85$.


\návody
Dokážte, že ľubovoľný trojuholník je každou svojou ťažnicou rozdelený na dva
menšie trojuholníky s~rovnakým obsahom. [Dotyčné trojuholníky majú zhodnú výšku
na zhodné základne.]

Dokážte, že ľubovoľný trojuholník je svojimi tromi ťažnicami rozdelený na
šesť menších trojuholníkov s~rovnakým obsahom. [Pre trojuholník $ABC$ s~ťažnicami
$AA_1$, $BB_1$, $CC_1$ a~ťažiskom~$T$ využite rovnosti typu $S_{ABA_1}=S_{ACA_1}$
a~$S_{BTA_1}=S_{CTA_1}$, platné podľa výsledku úlohy N1.]

Odvoďte pravidlo o~pomere, v~akom delí os vnútorného uhla
daného trojuholníka jeho protiľahlú stranu: Os uhla $BAC$ pretína stranu~$BC$ trojuholníka~$ABC$ v~bode~$P$ určenom úmerou $|PB|:|PC|=|AB|:|AC|$.
[Dokážte, že oba pomery majú rovnakú hodnotu ako pomer obsahov
trojuholníkov $ABP$ a~$ACP$.]

\D
Vnútri strán $AB$, $AC$ daného trojuholníka $ABC$ sú zvolené postupne
body $E$, $F$, pričom $EF \parallel BC$. Úsečka~$EF$
je potom rozdelená bodom~$D$ tak, že platí
$p = |ED|:|DF| = |BE|:|EA|$.
\item{a)} Ukážte, že pomer obsahov trojuholníkov $ABC$ a~$ABD$ je pre $p=2:3$ rovnaký
ako pre $p=3:2$.
\item{b)} Zdôvodnite, prečo pomer obsahov trojuholníkov $ABC$ a~$ABD$ má hodnotu
aspoň~$4$. [65--C--I--4]

V~pravouhlom lichobežníku $ABCD$ s~pravým uhlom pri vrchole~$A$ základne~$AB$ je
bod~$K$ priesečníkom výšky~$CP$ lichobežníka s~jeho uhlopriečkou~$BD$.
Obsah štvoruholníka $APCD$ je polovicou obsahu lichobežníka
$ABCD$. Určte, akú časť obsahu trojuholníka $ABC$ zaberá trojuholník $BCK$.
[65--C--II--3]

Daný je lichobežník $ABCD$ so základňami $AB$, $CD$, pričom $2|AB|=3|CD|$.
\item{a)}
Nájdite bod~$P$ vnútri lichobežníka tak, aby obsahy trojuholníkov $ABP$
a~$CDP$ boli v~pomere $3:1$ a~aj obsahy trojuholníkov $BCP$ a~$DAP$ boli v~pomere $3:1$.
\item{b)}
Pre nájdený bod~$P$ určte postupný pomer obsahov trojuholníkov $ABP$, $BCP$, $CDP$ a~$DAP$.
[64--C--II--3]
\endnávod}

{%%%%%   B-I-3
Pre ilustráciu najskôr vypíšme, ako vyzerajú postupné úpravy napríklad
čísla~19:
$$
19\to20\to10\to5\to6\to3\to4\to2\to1.
$$
Podľa zadania sa každé párne číslo~$n$ zmení úpravou $\boxed{:2}$
na číslo $\frac12n$, zatiaľ čo každé nepárne číslo $n$ sa zmení úpravou
$\boxed{+1}$ na~číslo $n+1$, ktoré je samo párne, takže po druhej úprave
$\boxed{:2}$ z~pôvodného nepárneho čísla~$n$
dostaneme číslo $\frac12(n+1)$. Všimnime si, že prvá
z~nerovností
$$
n>\tfrac12n,\quad \text{resp.}\quad n>\tfrac12(n+1)
$$
platí pre každé párne $n\ge2$ a~že druhá nerovnosť platí pre
každé nepárne $n\ge3$. Znamená to, že ak začneme opakovane upravovať
ľubovoľne dané prirodzené číslo väčšie ako~1, budeme postupne (vždy
po jednej alebo dvoch úpravách) dostávať menšie a~menšie prirodzené
čísla, takže sa po konečnom počte krokov nutne dostaneme k~číslu~1.
(Keby sme dostali nekonečnú postupnosť čísel napospol väčších
ako 1, pre najmenšie z~nich by sme sa dostali do sporu s~jednou
z~vyššie uvedených nerovností.)

\smallskip
b)
Pre vyriešenie úlohy s~konkrétnym číslom $10^{6}$ bude výhodné
zistiť všeobecnejšie, ktoré čísla budú potrebovať (s~ohľadom na svoju
veľkosť) relatívne veľký počet úprav, kým sa zmenia na číslo~1.
Budú to určite čísla nepárne, v~ktorých sa navyše úpravy $\boxed{:2}$
nebudú pokiaľ možno opakovať za sebou (aby nedošlo k~rýchlemu
zníženiu hodnôt ako $20\to10\to5$ v~ilustračnom príklade),
budú sa teda s~úpravami $\boxed{+1}$ striedať.
Kandidátov s~relatívne veľkým počtom
potrebných úprav tak budeme nachádzať v~opačne zapísanom reťazci
úprav
$$
1\gets2\gets4\gets3\gets6\gets5\gets10\gets9\gets18\gets17\gets34
\gets33\gets\dots
$$
(aj predposledná úprava, rovnako ako tá posledná, musí byť
$\boxed{:2}$, inak by sme dostali $1\gets2\gets1$). Ako sa zdá,
každé z~nepárnych čísel 3, 5, 9, 17, 33, \dots, ktoré sa v~takom reťazci
objavilo, je číslo tvaru $2^k+1$ a~číslo~1 z~neho dostaneme
po $2k+1$ úpravách. Pre $k=3$ sa jedná o~číslo~$9$, z~ktorého
číslo~1 dostaneme po 7~úpravách, čo je, ako sa ľahko
numericky preverí, najväčší počet potrebných úprav pre všetky
východiskové čísla, ktoré neprevyšujú číslo $2^4=16$.
To nás vedie k~domnienke, ktorú teraz vyslovíme
a~potom dokážeme matematickou indukciou.

{\sl Pre každé prirodzené číslo~$k$ platí:
Z~ľubovoľného čísla $n\in\{1,2,\dots,2^{k+1}\}$ sa dostaneme
k~číslu~$1$ po nanajvýš $2k+1$ úpravách, pritom $2k+1$ úprav
budeme potrebovať pre jediné z~uvedených čísel, konkrétne pre číslo $n=2^k+1$.}

Označme $\mm M_k=\{1,2,\dots,2^{k+1}\}$ množinu hodnôt $n$ z~uvedeného
tvrdenia. Pre $k=1$ toto tvrdenie vyplýva z~reťazca $1\gets2\gets4\gets3$,
lebo $\mm M_1=\{1,2,3,4\}$. Ak platí dané tvrdenie
pre určitú hodnotu~$k$, bude platiť aj pre hodnotu $k+1$, keď
pre novo uvažované čísla~$n$ tvoriace množinu
$$
\mm N_k=\mm M_{k+1}\setminus \mm M_k,\quad\text{čiže}\quad
\mm N_k=\{2^{k+1}+1,2^{k+1}+2,\dots,2^{k+2}\},
$$
dokážeme, že každé z~nich po nanajvýš dvoch úpravách padne do
množiny~$\mm M_k$, pritom to z~nich, ktoré sa zmení
na číslo $2^k+1$ až po dvoch úpravách, je jedine číslo $2^{k+1}+1$.

Naozaj, ľubovoľné párne číslo $n\in\mm N_k$ sa zmení jedinou úpravou na číslo
$\frac12n$, ktoré patrí do~$\mm M_k$, keďže z~nerovnosti $n\le 2^{k+2}$ vyplýva
$\frac12 n\le 2^{k+1}$.
Ľubovoľné nepárne číslo $n\in\mm N_k$ sa zmení po dvoch úpravách na číslo $\frac12(n+1)$,
a~to patrí do~$\mm M_k$, keďže z~nerovnosti $n\le 2^{k+2}-1$ vyplýva
$\frac12 (n+1)\le 2^{k+1}$.

Číslo, ktorého úprava vedie k~(nepárnemu) číslu $2^k+1$, je jediné,
a~to párne číslo $2^{k+1}+2$. To potom možno dostať dvoma spôsobmi: jednak
z~čísla $2^{k+1}+1\in\mm N_k$ úpravou $\boxed{+1}$, alebo úpravou
$\boxed{:2}$ z~čísla $2^{k+2}+4$, ktoré ale do množiny~$\mm N_k$ nepatrí.
Tým je dôkaz indukciou ukončený.

Teraz už ľahko získame odpoveď na stanovenú úlohu. Keďže
$$
2^{19}+1=524\,289<10^6<2^{20},
$$
je podľa dokázaného tvrdenia hľadané číslo rovné $524\,289$.

\ineriesenie
Uvedieme ešte iné riešenie oboch častí, ktoré výhodne používa dvojkovú sústavu. Nech $n$ je ľubovoľné prirodzené číslo väčšie ako 1.

Uvažujme najprv prípad, keď $n$ je mocnina dvoch, \tj. má zápis $(100\ldots0)_2$ (takto označujeme zápis čísla v~dvojkovej sústave). V~každom kroku zrejme delíme dvoma, čo znamená, že odstránime poslednú nulu. To robíme, až kým nedostaneme cifru 1. Ak má číslo na začiatku $k$~núl, v~tomto prípade nastane práve $k$~úprav.

Ak $n$ nie je mocnina dvoch, tak dokážeme, že pre potrebný počet úprav $p(n)$ platí vzorec $p(n)=c(n)+v(n)+1$, kde $c(n)$ je počet cifier čísla~$n$ zapísaného v~dvojkovej sústave a~$v(n)$ je počet "vnútorných" núl v~tomto zápise.

Tento vzťah najprv dokážeme pre prípad, keď máme číslo tvaru $n=(11\ldots 1)_2$ s~$k$~jednotkami. Jeho úpravou dostaneme $m=(100\ldots 0)_2$ s~$k$ nulami, preto $p(n)=p(m)+1$. Pritom už vieme, že $p(m)=k$, takže $p(n)=k+1$. Keďže $c(n)=k$ a~$v(n)=0$, naozaj $p(n)=c(n)+v(n)+1$.

Zvyšné prípady vyriešime matematickou indukciou. Prípady $n=2 =(10)_2$, $n=3=(11)_2$ a~$n=4=(100)_2$ sme už vybavili. Ak $n=5 =(101)_2$, tak ľahko overíme, že potrebujeme práve 5~krokov, a keďže $c(n)=3$, $v(n)=1$, naozaj $p(n)=c(n)+v(n)+1$.

Predpokladajme, že je dané číslo $n_0 \ge 5$ také, že dokazované tvrdenie platí pre všetky $n<n_0$, ktoré nie sú mocninami dvoch. Dokážeme, že potom platí aj pre $n$. Rozoberieme dva prípady:

\smallskip
\item{$\triangleright$}Nech je $n$ párne. Napíšme ho v~tvare $(1a_1a_2 \ldots a_k 0)_2$. Po úprave dostaneme menšie číslo $m=(1a_1a_2\dots a_k)_2$. Keďže $n$ nie je mocnina dvoch, ani $m$ nie je mocnina dvoch. Čísla $n$ a~$m$ majú zrejme rovnaký počet vnútorných núl, \tj. $v(n)=v(m)$. Taktiež platí $c(n)=k+2$ a $c(m)=k+1$. Platí teda $p(n)=1+p(m)=1+c(m)+v(m)+1=1+(k+1)+v(n)+1=c(n)+v(n)+1$.

\smallskip
\item{$\triangleright$}Nech je $n$ nepárne. Ak obsahuje samé jednotky, tak je to prípad, v~ktorom sme už tvrdenie dokázali. Predpokladajme teda, že obsahuje aspoň jednu (vnútornú) nulu a napíšme ho v~tvare $(1a_1a_2\ldots a_l 011 \ldots 1)_2$, pričom jednotiek na konci je $k \ge 1$. Po dvoch úpravách dostaneme
     $$
     (1a_1a_2\ldots a_l 0\underbrace{11 \ldots 1}_k)_2 \to (1a_1a_2\ldots a_l 1 \underbrace{0 \ldots 0}_{k})_2 \to (1a_1a_2\ldots a_l1\underbrace{0 \ldots 0}_{k-1}) = m.
     $$
     Zrejme $c(n)=k+l+2$, $c(m)=k+l+1$ a $v(n)=1+v(m)$. Keďže $m<n$ a~nie je mocninou dvoch, môžeme naň použiť indukčný predpoklad. Platí teda $p(n)=2+p(m)=2+c(m)+v(m)+1=(k+l+2)+v(n)+1=c(n)+v(n)+1$. Tvrdenie je tým dokázané.

\smallskip\noindent
Z~daného vzorca okamžite vyplýva časť~a). Dokonca sme našli požadovaný počet operácií. V~časti~b) je cieľom tento počet maximalizovať. Vezmime pevnú dĺžku~$k$ zápisu čísla~$n$ v~dvojkovej sústave. Ak je $n$ mocninou dvoch, tak potrebujeme $k-1$ operácií. Ak nie, tak počet operácii je rovný $1+k+v(n)$. Pritom počet vnútorných núl $v(n)$ je zrejme nanajvýš rovný $k-2$, vtedy ide o~číslo tvaru
$$
(1\underbrace{0\ldots 0}_{k-2}1)_2=2^k+1.
$$
Celkový počet operácií je v~takom prípade rovný $2k-1$. Potrebujeme teda nájsť najväčšie $k$ také, že $2^k+1 \le 10^6$. Ľahko sa presvedčíme, že je ním $k=19$, lebo $2^{19}+1 = 524\,289 < 10^6 < 2^{20}$. Hľadané číslo v~časti~b) je preto rovné $524\,289$.

\poznamka
Nájdený vzťah sa dá napísať viacerými spôsobmi, ktoré zahŕňajú aj prípad, keď je $n$ mocninou dvoch. Dá sa overiť, že ekvivalentná formulácia je $p(n)={c(n-1)}+o(n-1)$, kde $o(n-1)$ je počet núl v~binárnom zápise čísla~$n$. Iný spôsob je $p(n) = j(n-1) + 2 \cdot o(n-1)$, kde $j(n-1)$ je počet núl v~binárnom zápise $n-1$. Na dôkaz si stačí uvedomiť vzťah medzi zápisom $n$ a~$n-1$ v~dvojkovej sústave.

%\poznamka
%Počty $p(n)$ úprav zo súťažnej úlohy, po ktorých sa zmení dané východiskové číslo~$n$ po prvý raz na~číslo~1, spĺňajú zrejme pre každé $n>1$ rekurentné vzťahy
%$$
%p(2n)=p(n)+1 \quad\text{a}\quad
%p(2n+1)=p(n+1)+2.
%$$
%Vďaka nim sú všetky hodnoty $p(n)$ určené prvými dvoma hodnotami $p(2)=1$ a~$p(3)=3$. Na~určenie a~zápis explicitného vzorca pre všeobecnú hodnotu $p(n)$ je výhodné využiť dvojkovú sústavu: Pre každé $n>1$ platí vzorec
%$$
%p(n)=j(n-1)+2\cdot o(n-1),
%$$
%pričom $j(k)$ a~$o(k)$ označuje počet jednotiek, resp. núl v~dvojkovom zápise čísla~$k$. Čitateľovi prenechávame dôkaz tohto vzorca matematickou indukciou (využite pri ňom vyššie uvedené rekurentné vzťahy), rovnako ako overenie faktu, že tvrdenie, ktoré sme dokázali v~riešení časti~b) súťažnej úlohy, je okamžitým dôsledkom takto zapísaného explicitného vzorca.

\návody
{\everypar={}
Vo všetkých návodných úlohách sa zaoberáme {\it úpravou\/}
zo súťažnej úlohy.\par}

Pre ktoré jednociferné číslo je nutné spraviť najväčší
počet jeho úprav, kým získame číslo~1? O~koľko úprav pôjde? [Číslo 9,
sedem úprav.]

Nájdite najmenšie prirodzené číslo, pre ktoré je nutné
spraviť 8, resp. 9 jeho úprav, kým získame číslo 1. [Číslo 18,
resp. číslo 17.]

Určte všetky prirodzené čísla, s~ktorými je nutné spraviť
päť úprav, kým získame číslo~1. [Čísla 5, 12, 14, 15 a~32.
Vypisujte všetky možné postupy úprav \uv{odzadu}, \tj. od
konečného čísla 1 k~východiskovému číslu. Pre daný počet piatich úprav
sú tieto možnosti:
$1\gets2\gets4\gets3\gets6\gets5$,
$1\gets2\gets4\gets3\gets6\gets12$,
$1\gets2\gets4\gets8\gets7\gets14$,
$1\gets2\gets4\gets8\gets16\gets15$,
$1\gets2\gets4\gets8\gets16\gets32$.]

Pre ľubovoľné prirodzené čísla $k$ a~$p$ dokážte: Ak
sa všetky prirodzené čísla neprevyšujúce hodnotu~$k$ zmenia po
nanajvýš $p$ úpravách na~číslo~1, tak na to isté pre čísla
neprevyšujúce hodnotu $2k-1$ stačí nanajvýš $p+2$ úprav.
[Rozlíšte, či je dané číslo $n$, $n\le2k-1$, párne alebo
nepárne -- v~prvom prípade stačí nanajvýš $p+1$, v~druhom nanajvýš $p+2$ úprav.]

\D
Nech $l$ je pevné nepárne číslo väčšie ako~1.
Úpravou prirodzeného čísla nazveme nasledujúcu operáciu:
ak je číslo párne, vydelíme ho dvoma; ak je nepárne, pričítame
k~nemu dané číslo~$l$. Dokážte, že z~ľubovoľného prirodzeného čísla
dostaneme po niekoľkých úpravách číslo, ktoré neprevyšuje číslo~$l$,
pritom všetky ďalšie čísla neprevyšujú číslo~$2l$ a~periodicky sa
opakujú. Na príklade $l=7$ sa presvedčte, že zrejmé opakovanie
$l\to 2l\to l\to\dots$ nie je jediné možné. [Využite
nasledujúce poznatky: Párne číslo sa po jednej úprave
zmenší. Nepárne číslo väčšie ako $l$ sa zmenší po dvoch úpravách.
Nepárne číslo, ktoré neprevyšuje~$l$, sa po jednej úprave zväčší
na párne číslo neprevyšujúce~$2l$, takže po druhej úprave sa opäť zmenší na
číslo neprevyšujúce~$l$. Čísla sa periodicky zacyklia,
akonáhle sa v~priebehu úprav objaví niektoré číslo po druhý raz. Pre
$l=7$ existujú okrem $7\to14\to7$ ešte dva ďalšie cykly
$1\to8\to4\to2\to1$, $3\to10\to5\to12\to6\to3$
(keďže sme vypísali všetky čísla od 1 do 7, žiadne ďalšie cykly
neexistujú).]
\endnávod}

{%%%%%   B-I-4
Všimnime si na úvod, že výraz~$V$ má za daných podmienok na
čísla~$a$,~$b$ vždy zmysel, lebo rovnosť $a^2+b^2=1$ vylučuje možnosť
$a=b=0$, a~tak je súčet nezáporných čísel $a$, $b$ rôzny od nuly
(čiže kladný).

{\it Určenie minima}. Zo všeobecne platnej nerovnosti
$(a-b)^2\ge0$ v~našej situácii máme $2ab\le a^2+b^2=1$, a~teda
$0\le ab\le1/2$. Okrem toho nerovnosť $2ab\le1$ spolu
s~rovnosťou $(a+b)^2=1+2ab$ vedie k~odhadom $1\le(a+b)^2\le2$,
z~ktorých po odmocnení dostaneme $1\le a+b\le\sqrt2$. Preto platí
$$
V=\frac{(a^2+b^2)^2-2a^2b^2+ab+1}{a+b}=\frac{2+ab(1-2ab)}{a+b}\ge
\frac{2}{a+b}\ge\frac{2}{\sqrt2}=\sqrt2,
$$
pritom rovnosť $V=\sqrt2$ ako vidno nastane práve vtedy, keď $2ab=1$
(čiže $a=b$, ako vyplýva z~úvodnej vety tohto odseku)
a~zároveň $a+b=\sqrt2$, teda jedine pre $a=b=\frac12\sqrt2$.

{\it Určenie maxima}. Vďaka predpokladu $a^2+b^2=1$ platí
$$
\align
V=&\frac{(a+b)(a^3+b^3)-ab(a^2+b^2)+ab+1}{a+b}=\frac{(a+b)(a^3+b^3)+1}{a+b}=\\
=&a^3+b^3+\frac{1}{a+b}\le 1+\frac{1}{1}=2,
\endalign
$$
lebo kvôli rovnosti $a^2+b^2=1$ musí platiť $0\le a\le 1$ a~$0\le b\le
1$, odkiaľ vyplývajú odhady $a^3\le a^2\le a$, resp. $b^3\le b^2\le b$,
po ktorých sčítaní dostaneme
$$
a^3+b^3\le a^2+b^2\le a+b,\quad\text{čiže}\quad
a^3+b^3\le 1\le a+b.
$$
Rovnosť $V=2$ nastane práve vtedy, keď oba súčty $a^3+b^3$, $a+b$
sú (rovnako ako súčet $a^2+b^2$) rovné 1, teda v~prípadoch, keď
$\{a,b\}=\{0,1\}$.

\odpoved
Minimálna hodnota výrazu $V$ je~$\sqrt2$,
jeho maximálna hodnota je~2.

\ineriesenie
Ukážeme iný postup, ako dokázať nerovnosti
$\sqrt2\le V\le 2$ (že sú obe rovnosti dosiahnuteľné, už
preverovať nebudeme).
Tak ako v~prvom riešení využijeme rovnosť $a^2+b^2=1$
na úpravu čitateľa zadaného zlomku, konkrétne
$$
a^4+b^4+ab+1=(a^2+b^2)^2-2a^2b^2+ab+1=2+ab(1-2ab),
$$
a~dvojicu nerovností, ktoré chceme dokázať, ekvivalentne upravíme:
$$
\gather
\sqrt2\le\frac{2+ab(1-2ab)}{a+b}\le 2,\quad|\cdot(a+b),\\
\sqrt2(a+b)\le2+ab(1-2ab)\le2(a+b) ,\quad|^{2}\\
2(a+b)^2\le\bigl(2+ab(1-2ab)\bigr)^2\le4(a+b)^2,\\
2(1+2ab)\le\bigl(2+ab(1-2ab)\bigr)^2\le4(1+2ab) .
\endgather
$$
Máme teda dokázať odhady $2+4ab\le W\le 4+8ab$, pričom
$$
W=\bigl(2+ab(1-2ab)\bigr)^2=4+4ab(1-2ab)+a^2b^2(1-2ab)^2.
$$
Dolný odhad $2+4ab\le W$ dostaneme, keď rovnako ako v~prvom
riešení odvodíme nerovnosť $2ab\le1$; vďaka nej potom totiž platí
$2+4ab\le 4\le W$.

Horný odhad $W\le 4+8ab$ po dosadení rozvoja pre $W$
ďalej ekvivalentne upravíme (za predpokladu, že obe čísla $a$, $b$
sú {\it kladné}, inak je totiž nutne
$\{a,b\}=\{0,1\}$ a~potom, ako vieme, nastáva rovnosť $V=2$, a~teda aj $W=4$):
$$
\align
4+4ab(1-2ab)+a^2b^2(1-2ab)^2&\le 4+8ab, \quad |-4\\
4ab(1-2ab)+a^2b^2(1-2ab)^2&\le 8ab ,\quad |:ab\\
4(1-2ab)+ab(1-2ab)^2&\le 8.
\endalign
$$
Ľavá strana poslednej nerovnosti je však vďaka odhadom $0<ab\le\frac12$
dokonca menšia ako číslo $4+\frac12$. Tým je dôkaz oboch nerovností
$\sqrt2\le V\le 2$ ukončený.


\návody
{\everypar={}
V~úlohách N1--N4 aj~D1 sú $a$, $b$ nezáporné čísla, pre ktoré
platí $a^2+b^2=1$.\par}

Nájdite najmenšiu aj najväčšiu možnú hodnotu ako súčinu
$ab$, tak súčtu $a+b$. [$\min(ab)=0$, $\max(ab)=1/2$,
$\min(a+b)=1$, $\max(a+b)=\sqrt2$. Využite nerovnosť $(a-b)^2\ge0$
a~rovnosť $(a+b)^2=1+2ab$.]

Dokážte, že súčet $a^4+b^4+ab+1$ závisí iba od súčinu
$ab$. [Súčet možno upraviť na tvar $2+ab(1-2ab)$.]

Vyjadrite, o~čo sa výraz $V=(a^4+b^4+ab+1)/(a+b)$
zo súťažnej úlohy líši od súčtu $a^3+b^3$. [$V-(a^3+b^3)=1/(a+b)$]

Dokážte rovnosť $\max(a^3+b^3)=1$. [Využite nerovnosti
$a^3\le a^2$ a~$b^3\le b^2$.]

\D
Nájdite najväčšiu možnú hodnotu podielu $P=ab/(a+b)$.
[$\max P=\sqrt2/4$. Využite rovnosť $2P=(a+b)-1/(a+b)$ a~fakt,
že $\max(a+b)=\sqrt2$ (úloha N1).]

Dokážte, že pre každé kladné reálne číslo~$t$ platia nerovnosti
$$
0\le\frac{t^2+1}{t+1} -\sqrt{t}\le|t-1|.
\eqno\hbox{[67--B--I--2]}
$$

Nájdite všetky kladné reálne čísla~$t$ také, že pre
ľubovoľné nezáporné reálne číslo~$x$ platí nerovnosť
$$
\frac t{x+2}+\frac x{t(x+1)}\le 1.
\eqno\hbox{[67--B--II--2]}
$$

Určte všetky reálne čísla $r$ také, že nerovnosť $a^3 + ab + b^3\ge a^2 + b^2$
platí pre všetky dvojice reálnych čísel $a$, $b$, ktoré sú väčšie alebo rovné~$r$. [66--B--I--6]

Dokážte, že pre všetky kladné reálne čísla $ a\le b\le c$ platí
$$
(-a+b+c)\Big(\frac{1}{a}+\frac{1}{b}+\frac{1}{c}\Big)\ge3.
\eqno\hbox{[66--C--II--4]}
$$

Pre kladné reálne čísla $a$, $b$, $c$ platí $c^2 + ab = a^2 + b^2$. Dokážte, že potom platí aj $c^2+ab\le ac+bc$. [63--C--II--3]

Nech $a$, $b$, $c$ sú reálne čísla, ktorých súčet je~$6$.
Dokážte, že aspoň jedno z~čísel $ab+bc$, $bc+ca$, $ca+ab$ nie je väčšie ako $8$. [60--B--I--3]
\endnávod}

{%%%%%   B-I-5
Poznamenajme najskôr, že vďaka pravouhlému trojuholníku~$ADM$ je uhol $BMA$
tupý. Uhly $BAM$, $ABM$ sú teda ostré, takže kolmý priemet~$P$
bodu~$M$ padne dovnútra strany~$AB$, zatiaľ čo
kolmý priemet~$Q$ bodu~$B$ padne na polpriamku
opačnú k~polpriamke~$MA$ (\obr, v~ktorom vystačíme iba
s~naznačenou uhlopriečkou~$AC$, lebo vrcholu~$C$ sa dokazované
tvrdenie netýka~-- môže ním dokonca byť aj vnútorný bod úsečky~$MQ$).
V~takto upresnenej situácii objavíme vďaka Tálesovej vete
hneď tri tetivové štvoruholníky: prvým je štvoruholník $ABQD$,
ktorému možno opísať Tálesovu kružnicu s~priemerom~$AB$
a~vnútri ktorého nutne leží aj bod~$M$, a~ďalšími dvoma sú
$APMD$ a~$BPMQ$.
V~obrázku sú zároveň vyznačené zhodné obvodové uhly.
Jedným oblúčikom zhodné uhly nad tetivou~$DQ$ v~štvoruholníku $ABQD$ a~s~nimi
zhodný nad tetivou~$DM$ v~štvoruholníku $APMD$, ako aj druhý nad tetivou~$MQ$
v~štvoruholníku $BPMQ$. Dvoma oblúčikmi potom zhodné uhly nad tetivou~$BQ$
opäť v~štvoruholníku $ABQD$ a~s~nimi zhodný nad tetivou~$MP$ v~štvoruholníku $APMD$.
\insp{b68.2}%

Výsledkom je, že v~trojuholníku~$PQD$ platí
$|\uhel DPM|=|\uhel QPM|$ a~$|\uhel PDM|=|\uhel QDM|$, teda bod~$M$
je priesečníkom dvoch osí vnútorných uhlov trojuholníka~$PQD$,
čo je vlastnosť, ktorú má práve stred jeho vpísanej kružnice.


\návody
Pripomeňte si Tálesovu vetu a~všeobecnejší poznatok
o~obvodových a~stredových uhloch v~danej kružnici.

V~súťažnej úlohe objavte tri štvorice pomenovaných bodov,
ktoré ležia vždy na jednej kružnici.

\D
V~rovine je daný rovnobežník $ABCD$, ktorého uhlopriečka~$BD$ je kolmá na stranu~$AD$. Označme $M$ $(M\ne A)$ priesečník priamky~$AC$ s~kružnicou majúcou priemer~$AD$. Dokážte, že os úsečky~$BM$ prechádza stredom strany~$CD$. [57--B--II--3]

% 65-A-I-3-D1
Daný je štvorec $ABCD$. Na kratšom oblúku~$AB$ jemu opísanej kružnice
zvolíme bod~$X$. Priesečník úsečky~$XC$ so stranou~$AB$ označme
$Y$ a~priesečník úsečky~$XD$ s~uhlopriečkou~$AC$ označme $Z$. Dokážte, že
$YZ \perp AC$. [Nájdite skrytú štvoricu bodov, ktoré ležia na jednej
kružnici.]

% 66-A-I-5-N1
Pomocou počítania veľkostí uhlov dokážte, že výšky v~ostrouhlom
trojuholníku $ABC$ sa pretínajú v~jednom bode. [Označme postupne $D$ a~$E$
päty výšok z~vrcholov $A$ a~$B$, ďalej $P$ priesečník úsečiek $AD$
a~$BE$ a~$X$ priesečník $CP$ a~$AB$. Dokážeme, že priamka~$CP$ je kolmá
na $AB$. Štvoruholníky $ABDE$ a~$CDPE$ sú tetivové, pretože ich vrcholy
ležia na Tálesových kružniciach s~priemermi $AB$ a~$CP$. Preto uhly
$BAD$, $BED$, $PCD$ majú všetky rovnakú veľkosť $90^\circ-|\uhol ABC|$. Uhol
$CXB$, ktorý dopočítame zo známych veľkostí zvyšných uhlov
v~trojuholníku $CXB$, je teda pravý.]
\endnávod}

{%%%%%   B-I-6
Množinu $\mm M=\{1,2,\dots,2\,018\}$ zo zadania rozdelíme na dve časti
$$
\mm A=\{1,2,\dots,9,10\},\quad \mm B=\{11,12,\dots,2\,018\}
$$
a~ukážeme, že hľadaný počet všetkých pekných podmnožín množiny~$\mm M$
je rovný počtu {\it všetkých\/} podmnožín množiny $\mm B$ (vrátane
prázdnej množiny). Keďže množina $\mm B$ má ${2\,018-10}=2\,008$
prvkov, bude tento počet rovný číslu $2^{2\,008}$, lebo všeobecne každá
$n$\spojovnik{}prvková množina má práve $2^n$ podmnožín (stačí uplatniť pravidlo
súčinu na skutočnosť, že pre každý z~$n$ prvkov máme dve
možnosti: buď ho do konštruovanej podmnožiny zahrnúť, alebo
nie).

Úvodné tvrdenie overíme, keď ukážeme, že každá pekná podmnožina~$\mm P$
danej množiny~$\mm M$ je jednoznačne určená svojou časťou ležiacou
v~$\mm B$, teda množinou $\mm Q=\mm P\cap \mm B$, a~že naopak ku každej množine
$\mm Q\subseteq \mm B$ sa nájde pekná množina $\mm P\subseteq \mm M$, pre ktorú
$\mm Q=\mm P\cap \mm B$. Bude to znamenať, že medzi množinou všetkých pekných
podmnožín~$\mm M$ a~množinou všetkých podmnožín~$\mm B$ existuje bijekcia (navzájom
jednoznačné zobrazenie), takže obe množiny majú rovnaký počet
prvkov.

Opíšme teda, ako možno ktorúkoľvek pevne zvolenú
peknú množinu $\mm P\subseteq \mm M$ jednoznačne rekonštruovať
podľa jej časti $\mm Q=\mm P\cap \mm B$, \tj. určiť jej druhú časť
$\mm P\cap \mm A$. Inými slovami, podľa známej množiny~$\mm Q$ máme rozhodnúť,
ktoré z~najmenších čísel tvoriacich
množinu~$\mm A$ do peknej množiny~$\mm P$ patrí a~ktoré nie.
Jednoducho to zistíme pre každú z~cifier
$c\in\{2,3,\dots,9\}$: keďže počet výskytov cifry~$c$ v~zápise
všetkých čísel z~$\mm P$ je párny, číslo~$c$ do~$\mm P$ patrí práve vtedy, keď
je počet cifier~$c$ v~zápise všetkých čísel z~časti~$\mm Q$ nepárny
(lebo uvedené dva počty sa môžu líšiť nanajvýš o~1). Rovnakou
úvahou o~počte núl môžeme rozhodnúť, či do množiny~$\mm P$ patrí
číslo~10. Majúc túto informáciu o~čísle~10, môžeme učiniť (podľa počtu
jednotiek) aj posledné rozhodnutie, a~to či v~$\mm P$ leží číslo~1.
Tým je rekonštrukcia celej množiny~$\mm P$ hotová.

Postup z~predchádzajúceho odseku môžeme zrejme celý úspešne uplatniť
na~{\it ľubovoľnej\/} podmnožine~$\mm Q$ množiny~$\mm B$ (bez toho, aby sme dopredu vedeli,
že dotyčná pekná množina~$\mm P$ existuje, takže namiesto o~rekonštrukciu
pôjde o~jej konštrukciu).
Zostavíme tak peknú množinu $\mm P\subseteq \mm M$,
ktorá vznikne z~množiny~$\mm Q$ pridaním niektorých (vyššie presne
určených) čísel z~$\mm A$ a~pre ktorú bude platiť rovnosť
$\mm Q=\mm P\cap \mm B$, ako sme si želali. Tým je
riešenie celej úlohy ukončené.

\odpoved
Hľadaný počet pekných podmnožín je $2^{2008}$.


\návody
{\everypar={}
Vo všetkých úlohách sa jedná o~pekné množiny v~zmysle súťažnej
úlohy.\par}

Neznáma pekná množina~$\mm P$ obsahuje práve tieto viacciferné
čísla: 13, 21, 34, 55, 89. Nájdite všetky jednociferné čísla,
ktoré patria do $\mm P$.
[Práve čísla 2, 4, 8 a~9, každá z~ostatných cifier je totiž v~daných piatich
dvojciferných číslach zastúpená v~párnom počte exemplárov.]

Koľko pekných podmnožín má množina
$$
\{1, 2, 3, 11, 22, 33, 111, 222, 333\} ?
$$
[$4^3=64$. Každá z~troch daných trojíc čísel $\overline{c},\overline{cc},\overline{ccc}$
má v~peknej množine štyri možné zastúpenia: $\emptyset$, $\{\overline{cc}\}$,
$\{\overline{c},\overline{ccc}\}$ alebo $\{\overline{c},\overline{cc},\overline{ccc}\}$. Iné
riešenie: každá z~dotyčných pekných množín je určená svojimi
viaccifernými číslami, ktoré môžu spolu vytvoriť ľubovoľnú
z~$2^6$~podmnožín množiny $\{11, 22, 33, 111, 222, 333\}$.]

Ktorými konečnými množinami~$\mm Q$ viacciferných čísel môžeme
nahradiť množinu $\{13, 21,\allowbreak 34, 55, 89\}$ v~zadaní N1 tak,
aby úloha mala riešenie? [V~zápisoch všetkých čísel z~$\mm Q$ musí byť
cifra~0 zastúpená v~párnom počte, a~to je jediná podmienka na
$\mm Q$, lebo podľa parít počtov zastúpení
ostatných cifier 1 až 9 potom určíme, ktoré z~čísel 1 až 9 do
uvedenej peknej množiny budú patriť a~ktoré nie.]

Koľko pekných podmnožín má množina
$$
\{1, 3, 5, 7, 9, 11, 13, 15, 17, 19\} ?
$$
[32. Dvojciferné čísla z~každej dotyčnej peknej množiny môžu tvoriť
ľubovoľnú z~$2^5$~podmnožín päťprvkovej množiny
$\{11, 13, 15, 17, 19\}$.]

Koľko pekných podmnožín má množina
$$
\{1, 2, 3, 10, 11, 12, 13, 20, 21, 22, 23, 30\} ?
$$
[256. Čísla {\it väčšie ako\/} 10 z~každej dotyčnej peknej množiny môžu
tvoriť ľubovoľnú z~$2^8$~podmnožín osemprvkovej množiny
$$
\{11, 12, 13, 20, 21, 22, 23, 30\}.
$$
(Číslo 10 bude patriť do peknej
podmnožiny vtedy a~len vtedy, keď tam bude patriť
práve jedno z~čísel 20 a~30.)]
\endnávod}

{%%%%%   C-I-1
Ak by medzi hľadanými štyrmi číslami boli súčasne čísla 20 aj~21, bolo
by neznáme číslo násobkom troch, štyroch, piatich a~siedmich, a~teda
by bolo deliteľné aj~číslami 6, 15, 70, čo odporuje požiadavkám úlohy.
Do hľadanej štvorice tak jedno z~čísel 20, 21 nepatrí, a~preto tam patria
všetky tri čísla 6, 15, 70. Každý spoločný násobok čísel 6 a~70 je deliteľný ako tromi, tak
siedmimi, a~teda aj~číslom~21. Preto je štvrtým hľadaným číslom číslo~21.

Hľadanými štyrmi číslami sú čísla 6, 15, 21, 70.
Ich najmenší spoločný násobok je~210, čo je číslo, ktoré
nie je deliteľné zvyšným číslom~20.
(Neznámym číslom tak môže byť aj ľubovoľný násobok $210l$, pričom $l$~je nepárne.)
Tým je úloha vyriešená.

\poznamka
Úlohu možno riešiť systematicky tak, že najskôr vypíšeme všetkých päť
štvorprvkových podmnožín, teda množiny
$$
\gather
\mm A=\{15,20,21,70\},\quad
\mm B=\{6,20,21,70\},\quad
\mm C=\{6,15,21,70\},\\
\mm D=\{6,15,20,70\},\quad
\mm E=\{6,15,20,21\},
\endgather
$$
a~zistíme, ktorá z~nich môže spĺňať podmienky úlohy, teda že existuje
číslo~$N$, medzi ktorého delitele patria všetky štyri prvky uvažovanej množiny,
zatiaľ čo vynechané piate číslo jeho deliteľom nie je.
Tak všetky množiny okrem $\mm C$ postupne vylúčime, napríklad množinu~$\mm A$:
ak je $N$ deliteľné oboma číslami $15=3\cdot5$
a~$20=2\cdot10$, je deliteľné aj~číslom $2\cdot3=6$, teda $6\in\mm A$,
a~to je spor. Podobne dôjdeme k~sporom $15\in\mm B$, $21\in\mm D$
a~$70\in\mm E$.

\návody
Určte päť prirodzených čísel, ktorých súčet je 20 a~ktorých súčin je 420.
[Keďže $420=2^2\cdot 3\cdot 5\cdot 7$, $1+4+3+5+7=20$
a~hľadané čísla, ktoré sú násobkami 5 a~7, musia byť jednociferné
(súčet ostatných troch čísel je aspoň~6),
sú dve z~hľadaných čísel priamo čísla 5 a~7 a~zvyšné tri (nie nutne rôzne)
ležia v~množine $\{1,2,3,4,6\}$, pritom ich súčet je 8 a~súčin 12,
takže sa jedná o~trojicu 1,~3,~4.]

Isté prirodzené číslo má práve štyri delitele, ktorých
súčet je 176. Určte toto číslo, ak viete, že súčet všetkých jeho cifier
je 12.
[Hľadané číslo~$n$ je deliteľné tromi a~väčšie ako~9, preto jeho štyrmi
deliteľmi sú práve čísla $1<3<n/3<n$. Z~toho vychádza $n=129$.]

\D
Dané celé číslo je deliteľné aspoň štyrmi číslami
z~množiny $\{2,3,5,6,10,15\}$. Dokážte, že je deliteľné každým
z~nich.
[Všimnime si, že $6=2\cdot3$, $10=2\cdot5$ a~$15=3\cdot5$.
Teda ak je dotyčné číslo deliteľné všetkými tromi číslami 2, 3 a~5, je
deliteľné aj všetkými zvyšnými číslami 6, 10 a~15. V~opačnom prípade by bolo
deliteľné nanajvýš dvoma z~čísel 2, 3 a~5, a~preto aspoň dvoma
z~čísel 6, 10 a~15, a~teda aj každým z~prvočísel 2, 3 a~5, a~to je spor.]
\endnávod

}

{%%%%%   C-I-2
Rovnosť $|AD|=|DE|$ znamená, že bod~$D$ je stred úsečky~$AE$.
Rovnako je bod~$E$ stredom~$BD$, a~keďže body $D$ a~$E$ ležia vnútri úsečky~$AB$,
delia ju na tretiny (\obr).
\insp{c68.1}%

Vzhľadom na to, že bod~$D$ delí ťažnicu~$BA$ trojuholníka $BCF$ v~dvoch tretinách,
je jeho ťažiskom. Úsečka~$CI$ je teda ťažnicou trojuholníka~$BCF$ a~bod~$I$ je stredom
jeho strany~$BF$. Úsečka~$AI$ je teda strednou priečkou trojuholníka~$BCF$, čiže
$AI\parallel BC$, a~preto priamka $AI$ prechádza stredom strany~$FG$
trojuholníka $CFG$.

Podobne vidíme, že bod~$E$ je ťažiskom trojuholníka $C\!AG$, takže $CJ$ je jeho ťažnicou
a~$J$ je stredom jeho strany~$AG$, a~preto aj priamka~$BJ$ rovnobežná s~$FC$
prechádza stredom úsečky~$FG$. Tým je tvrdenie úlohy dokázané.


\návody
Nech $K$, $L$, $M$, $N$ sú postupne stredy strán $AB$,
$BC$, $CD$, $DA$ štvoruholníka $ABCD$. Dokážte, že $KLMN$ je
rovnobežník (tzv. {\it Varignonov\/} rovnobežník).
[Využite vlastnosti strednej priečky v~trojuholníku: úsečky
$KL$ a~$MN$ sú strednými priečkami postupne v~trojuholníkoch $ABC$
a~$CDA$.]

Nech $D$ je stred strany~$AB$ trojuholníka $ABC$ a~$E$ bod
jeho strany $AC$, pre ktorý platí $|AE|=2|CE|$. Označme $F$ priesečník
priamok $BE$ a~$CD$. Dokážte, že platí $|BE|=4|EF|$.
[Označme $M$ stred úsečky~$AE$. Úsečka~$EF$ je strednou priečkou
v~trojuholníku $CMD$ a~úsečka~$MD$ je strednou priečkou
v~trojuholníku $ABE$.]

\D
Nech $E$, $F$ sú postupne stredy strán $AB$, $CD$
konvexného štvoruholníka $ABCD$. Dokážte, že platí $|BC|+|AD|\ge
2|EF|$.
[Uvažujte stred~$M$ uhlopriečky~$AC$ a~úsečky $EM$ a~$FM$,
ktoré sú strednými priečkami v~trojuholníkoch $ABC$ a~$ACD$.]
\endnávod
}

{%%%%%   C-I-3
Pre uvažované kladné reálne čísla $a$, $b$, $c$
vzhľadom na predpoklady úlohy platí
$$
a^2+b^2+c^2+2ab+2bc+2ca=(a+b+c)^2=3^2=9.
$$
Stačí teda dokázať, že je splnená nerovnosť $2ab+2bc+2ca>3abc$. Tá je
však ekvivalentná s~nerovnosťou
$$
ab(2-c)+bc(2-a)+ca(2-b)>0,
$$
ktorá evidentne platí pre všetky kladné reálne čísla $a$, $b$, $c$,
ktoré nie sú väčšie ako~2 a~zároveň nemôžu byť (vzhľadom na podmienku
$a+b+c=3$) všetky rovné~2. Tým je dôkaz ukončený.

\poznamka
Keďže žiadne z~daných kladných čísel $a$, $b$, $c$ neprevyšuje
číslo~2, platia zrejme nerovnosti $a^2\le 2a$, $b^2\le 2b$
a~$c^2\le 2c$. Ich sčítaním dostaneme nerovnosť
$$
a^2+b^2+c^2\le 2(a+b+c)=2\cdot3=6,
$$
pritom rovnosť je vylúčená, lebo by muselo platiť $a=b=c=2$,
čo odporuje predpokladu $a+b+c=3$. Preto na dôkaz nerovnosti zo
záveru súťažnej úlohy stačí
overiť odhad $3abc\le9-6$, čiže $abc\le1$,
ktorý je však okamžitým dôsledkom nerovnosti medzi
aritmetickým a~geometrickým priemerom
$$
\root{3}\of{abc}\le \frac{a+b+c}{3}.
$$
Táto nerovnosť, ako je známe, platí všeobecne pre ľubovoľnú trojicu nezáporných
čísel $a$, $b$,~$c$, pritom rovnosť nastane jedine v~prípade
$a=b=c$. (V~zadanej úlohe je aritmetický priemer čísel $a$,
$b$, $c$ rovný~1, takže platí $\root{3}\of{abc}\le1$, čiže
$abc\le1$.)

\návody
Pre reálne čísla $a$, $b$, $c$ so súčtom~3 platí navyše $a^2+b^2+
c^2 =5$. Aké hodnoty môže nadobúdať výraz $ab+bc+ca$?
[Keďže $(a+b+c)^2 = a^2+b^2+c^2+2 (ab+bc+ca)$, je nutne
$ab+bc+ca = 2$. Hodnota je dosiahnuteľná vďaka trojici $(2, 1, 0)$.]

Nezáporné reálne čísla $a$, $b$, $c$ sú všetky nanajvýš rovné~1. Dokážte,
že $3abc \le a+b+c$. Kedy nastane rovnosť?
[Upravíme na $a~(1-bc)+b (1-ac)+c (1-ab)\ge 0$,
výrazy v~zátvorkách sú nezáporné. Rovnosť nastane
práve vtedy, keď buď $a=b=c=0$, alebo $a=b=c=1$.]

\D
Dokážte, že pre reálne čísla $a$, $b$, $c$ platí $a^2+b^2+c^2\ge
ab+bc+ca$. Kedy nastane rovnosť?
[Nerovnosť je ekvivalentná s~$(a-b)^2 + (b-c)^2 + (c-a)^2 \ge 0$,
ktorá určite platí. Rovnosť nastane jedine v~prípade $a=b=c$.]

Reálne čísla $a$, $b$, $c$ majú súčet 3. Dokážte, že $3\ge ab+bc+ca$.
Kedy nastane rovnosť?
[Vyplýva z~rovnosti $9 = (a+b+c)^2 = a^2+b^2+c^2+2 (ab+bc+ca)$
a~z~predošlej úlohy. Rovnosť nastane jedine v~prípade $a=b=c=1$.]

Dokážte, že pre ľubovoľné reálne čísla $x$, $y$, $z$ platí
nerovnosť
$$x^2+5y^2+4z^2\ge 4y(x+z)$$
a~zistite, kedy nastane rovnosť.
[Anulujte pravú stranu danej nerovnosti a~upravte ju následne
na tvar $(x^2-4xy+4y^2)+(y^2-4yz+4z^2)\ge 0$, pričom na ľavej strane
je nezáporný súčet $(x-2y)^2+(y-2z)^2$. Rovnosť tu nastane
práve vtedy, keď platí $(x,y,z)=(4c,2c,c)$, pričom $c$ je ľubovoľné reálne
číslo.]

Nech $a$, $b$, $c$ sú dĺžky strán trojuholníka. Dokážte,
že platí nerovnosť
$$3a^2+2bc>2ab+2ac.$$
[Danú nerovnosť upravte na tvar
$a^2-(b-c)^2+(a-b)^2+(a-c)^2>0$
a~rozdiel prvých dvoch druhých mocnín nahraďte príslušným súčinom.]
\endnávod
}

{%%%%%   C-I-4
Políčka prvého stĺpca tabuľky môžeme za daných podmienok
ofarbiť práve $4\cdot 3=12$ spôsobmi.
Susedný stĺpec potom možno podľa zadania úlohy ofarbiť
práve $3^2-2$ spôsobmi, lebo každé políčko v~ňom môžeme vzhľadom
na susedný stĺpec ofarbiť jednou z~troch zvyšných farieb,
od výsledného počtu $3^2$ možností však musíme odčítať
oba prípady, keď by sme v~tomto stĺpci dostali pod sebou dve rovnako
ofarbené políčka (jedná sa práve o~tie dve farby, ktoré neboli použité v~prvom stĺpci).

Analogicky môžeme pokračovať v~postupnom ofarbovaní ďalších stĺpcov,
pričom pri každom z~nich máme opäť vždy $3^2-2$ možností.
Počty možností pre jednotlivé stĺpce nakoniec medzi sebou vynásobíme,
aby sme dostali hľadaný počet spôsobov vyhovujúcich ofarbení celej tabuľky.

\zaver
Celkovo existuje $12\cdot (3^2-2)^{13-1}=12\cdot
7^{12}$ možností pre ofarbenie políčok tabuľky $2\times 13$
požadovaným spôsobom.

\návody
Koľkými spôsobmi možno ofarbiť políčka tabuľky $2\times 3$ dvoma
rôznymi farbami tak, že každé políčko je ofarbené jednou farbou?
[Pre každé políčko tabuľky existujú vždy 2~možnosti ofarbenia, pre celú
tabuľku máme teda $2^6=64$ možností.]

Každé políčko tabuľky $13\times 13$ ofarbíme práve jednou z~dvoch farieb. Koľkými spôsobmi to možno spraviť tak, aby žiadne dve
susedné políčka neboli ofarbené rovnakou farbou?
[2~možnosti.]

Každé políčko tabuľky $2\times 2$ ofarbíme práve jednou z~troch
farieb. Koľkými spôsobmi to možno spraviť tak, aby žiadne dve
susedné políčka neboli ofarbené rovnakou farbou?
[Pre ofarbenie prvých dvoch susedných políčok máme 6~možností,
pre zvyšné políčka potom už iba tri možnosti, celkom $6\cdot3=18$ možností.]

\D
Určte, koľkými spôsobmi možno ofarbiť políčka tabuľky $3 \times 3$ tromi rôznymi farbami
(každé políčko práve jednou farbou) tak, aby v~každom riadku a~každom stĺpci boli použité
všetky tri farby.
[Existuje 12~možností. Prvý stĺpec ofarbíme celkom $3\cdot 2=6$ spôsobmi, pre druhý
stĺpec máme vzhľadom na podmienky úlohy iba 2~možnosti. Ofarbenie políčok posledného,
tretieho stĺpca je tým už jednoznačne určené. Celkovo tak máme $3\cdot 2\cdot 2=12$ možností.]
\endnávod}

{%%%%%   C-I-5
Súčet veľkostí všetkých vnútorných uhlov konvexného osemuholníka
je ${6\cdot 180^{\circ}}=1\,080^{\circ}$, takže každý zo súčtov štyroch
dvojíc susedných uhlov zo zadania úlohy je rovný $1\,080^{\circ}:4=270^{\circ}$,
čo zároveň znamená, že všetkých osem vnútorných uhlov je tupých.
Označme postupne $P$, $Q$, $R$ a~$S$ priesečníky dvojíc priamok strán susediacich
so stranami $AB$, $CD$, $EF$ a~$GH$ (\obr). Potom platí
$|\uh APB|=180\st-(180\st-\alpha)-(180\st-\beta)=(\alpha+\beta)-180\st=90^{\circ}$,
takže priamky $BC$ a~$AH$ sú na seba kolmé. Podobne dokážeme, že
$BC\perp DE$, ${ED\perp FG}$ a~aj $FG\perp AH$. Priesečníky priamok, na
ktorých ležia strany $AH$, $BC$, $DE$ a~$FG$, teda tvoria vrcholy pravouholníka~$PQRS$.
\insp{c68.2}%

Body $K$ a~$M$ sú teda stredmi priečok $AD$ a~$EH$ ($K\ne M$)
pásu ohraničeného rovnobežkami $PS$ a~$QR$, a~preto ležia na jeho osi.
Podobne body $L$ a~$N$ ($L\ne N$) sú stredmi priečok $CF$ a~$GB$, a~ležia
tak na osi pásu ohraničeného rovnobežkami $PQ$ a~$RS$. Vzhľadom na to, že osi oboch
pásov sú na seba kolmé, sú na seba kolmé aj priamky $KM$ a~$LN$, čo
sme mali dokázať.

\návody
Pre ľubovoľné prirodzené $n\ge3$ odvoďte vzorec pre
súčet $s_n$ veľkostí všetkých vnútorných uhlov konvexného $n$-uholníka.
[$s_n=(n-2)\cdot 180^\circ$]

V~rovine je daný konvexný šesťuholník $ABCDEF$ so zhodnými
vnútornými uhlami. Dokážte, že osi jeho strán $AB$, $CD$ a~$EF$ sa
pretínajú v~jednom spoločnom bode.
[Priesečníky priamok $BC$, $DE$ a~$FA$ tvoria vrcholy
rovnostranného trojuholníka. Osi jeho vnútorných uhlov sú totožné
s~osami úsečiek $AB$, $CD$ a~$EF$.]

V~rovine je daný konvexný päťuholník $ABCDE$ so zhodnými
vnútornými uhlami. Dokážte, že osi jeho strán $BC$, $EA$ a~os vnútorného
uhla pri vrchole~$D$ sa pretínajú v~jednom spoločnom bode.
[Priesečníky priamok $AB$, $CD$ a~$DE$ tvoria vrcholy
rovnoramenného trojuholníka s~hlavným vrcholom $D$. Osi vnútorných uhlov
pri jeho základni sú totožné s~osami úsečiek $BC$ a~$EA$.]
\endnávod}

{%%%%%   C-I-6
Hľadáme trojciferné čísla $n=\overline{abc}=100a+10b+c$,
ktoré sú deliteľné súčtom
$$
\overline{bc}+\overline{ac}+\overline{ab}=(10b+c)+(10a+c)+(10a+b)=20a+11b+2c,
$$
pričom $\{a,b,c\}$ je trojprvková podmnožina množiny $\{1,2,\dots,9\}$. To
nastane práve vtedy, keď existuje prirodzené číslo~$k$ také, že
$$
100a+10b+c=k(20a+11b+2c),
$$
čiže
$$
(100-20k)a=(11k-10)b+(2k-1)c. \eqno{(1)}
$$
Keďže pravá strana poslednej rovnice je kladná, musí (vzhľadom na
tvar ľavej strany) platiť $k\le4$. Možné hodnoty~$k$ teraz preskúmame
jednotlivo.

\bulet
$k=1$. Z~(1) máme rovnicu $80a=b+c$, ktorá nemá riešenie, lebo
$80a\ge80$ a~${b+c}\le17$.
\bulet
$k=2$. Z~(1) dostaneme rovnicu $60a=12b+3c$, čiže $20a=4b+c$.
Keďže $4b+c\le 44$, je $a\le2$. Navyše z~rovnice vyplýva, že číslo~$c$
musí byť násobkom štyroch, teda $c\in\{4,8\}$, čo v~oboch prípadoch
$a=1$, $a=2$ výhodne využijeme.
\hfil\break
Pre $a=1$ máme $20=4b+c$ s~jedinou
vyhovujúcou dvojicou rôznych cifier $b=3$, $c=8$, ktorej zodpovedá $n=138$.
\hfil\break
Pre $a=2$ máme $40=4b+c$
s~jedinou vyhovujúcou dvojicou rôznych cifier $b=9$, $c=4$, ktorej zodpovedá $n=294$.
\bulet
$k=3$. Z~(1) máme rovnicu $40a=23b+5c$. Podľa deliteľnosti piatimi môže byť
jedine $b=5$. Dostaneme tak $40a=23\cdot 5+5c$, čiže $8a=23+c$ s~dvoma
vyhovujúcimi dvojicami $(a,c)$ rovnými $(3,1)$ a~$(4,9)$, ktorým
zodpovedajú dve hľadané $n$, a~to $n=351$ a~$n=459$.
\bulet
$k=4$. Z~(1) dostaneme rovnicu $20a=34b+7c$, čiže
$20(a-b)=7(2b+c)$. Keďže čísla 20 a~7 sú nesúdeliteľné a~pravá strana
je kladná, musí byť $a-b$ kladný násobok siedmich, takže
$a-b=7$ a~$2b+c=20$. Taká sústava nemá riešenie, lebo podľa
prvej rovnice je $b\le2$, a~teda $2b+c\le4+9=13$.

\zaver
Danej úlohe vyhovujú štyri trojciferné čísla, a~to 138, 294,
351 a~459.

\návody
Určte všetky trojciferné čísla, ktoré sú jedenásťkrát
väčšie ako ich ciferný súčet.
[198, z~rovnosti $100a+10b+c=11(a+b+c)$ upravenej na tvar $89a=b+10c$
vyplýva $a=1$, takže $b=9$ a~$c=8$.]

Určte všetky trojciferné čísla, ktoré sú sedemkrát väčšie
ako dvojciferné číslo, ktoré vznikne z~daného čísla vyškrtnutím jeho
prostrednej cifry.
[105, z~rovnosti $100a+10b+c=7(10a+c)$ upravenej na tvar $5(3a+b)=3c$ vyplýva $c=5$.]

Nájdite všetky štvorciferné čísla $\overline{abcd}$, pre ktoré platí
$\overline{abcd}=20\cdot\overline{ab}+16\cdot\overline{cd}$.
  [65--C--S--1]

\D
Nájdite najväčšie trojciferné číslo, z~ktorého po vyškrtnutí
ľubovoľnej cifry dostaneme prvočíslo.
  [67--C--S--1]

Nájdite najmenšie štvorciferné číslo $\overline{abcd}$ také, že rozdiel
$(\overline{ab})^2-(\overline{cd})^2$ je trojciferné číslo zapísané tromi rovnakými ciframi.
  [67--C--I--1]
\endnávod}

{%%%%%   A-S-1
\def\*{\item{$\triangleright$}}
Označme $x_1$, $x_2$ nie nutne rôzne
korene danej rovnice. Podľa Vi\`etových vzorcov platí $x_1+x_2 ={-p}$
a~$x_1x_2 = q$. Z~prvého z~týchto vzťahov vidíme, že ak je jeden
z~koreňov rovnice celočíselný, musí byť celočíselný aj ten druhý.
Obe čísla $x_1$, $x_2$ sú teda celé a~ich súčin je rovný prvočíslu~$q$.
To sa dá rozložiť na súčin
dvoch celých čísel iba ako $q \cdot 1$ alebo $({-q}) \cdot ({-1})$.
V~prvom prípade je vzťah $x_1+x_2 ={-p}$ ekvivalentný s~rovnicou $p+q+1 = 0$,
v~druhom s~rovnicou $p-q = 1$. Prvá rovnica nemá riešenie, pretože na jej ľavej strane je
kladné číslo. Druhá rovnica má jediné riešenie $(p, q) = (3, 2)$,
lebo $p> q$ a~prvočísla $p$, $q$ majú nutne rôznu paritu, z~čoho hneď vyplýva
$q = 2$, a~teda $p = 3$. Tejto dvojici zodpovedá rovnica $x^2+3x+2 = 0$,
ktorá má celočíselné korene ${-1}$ a~${-2}$. Tým je úloha vyriešená.


\poznamka
Skúška v~tomto riešení nie je
nutná. Ak totiž čísla $x_1$, $x_2$ spĺňajú vzťahy $x_1+x_2 ={-p}$
a~$x_1x_2 = q$, je dobre známe, že sa jedná o~korene
rovnice $x^2+px+q = 0$ (vyplýva to z~rozkladu $x^2+px+q=(x-x_1)(x-x_2)$).

\ineresc{č.\,2}
Označme $a$ celočíselný
koreň danej rovnice. Platí teda $a^2+pa+q = 0$. Vidíme, že $a$~delí
prvé dva sčítance na ľavej strane, preto nutne delí aj~$q$. Keďže
$q$ je prvočíslo, je $a~\in \{{\pm 1},\pm q\}$. Tieto štyri
možnosti postupne rozoberieme:

\* Pre $a~=-1$ máme $p-q = 1$, takže $(p, q) = (3, 2)$.
\* Pre $a~= 1$ máme $p+q+1 = 0$, čo nemôže platiť.
\* Pre $a~= q$ máme $q^2+pq+q = 0$, čo po vydelení kladným~$q$ vedie
na $q+p+1 = 0$. Rovnakú rovnicu sme už dostali predtým.
\* Pre $a~=-q$ máme $q^2-pq+q = 0$ a~po vydelení~$q$
dostávame $p-q = 1$, čo je rovnica, ktorú sme už riešili.

Jediné možné riešenie je $(p, q) = (3, 2)$. Ľahko sa presvedčíme,
že táto dvojica naozaj vyhovuje zadaniu.

\ineresc{č.\,3}
Aby mala rovnica $x^2+px+q = 0$
celočíselný koreň, musí byť jej diskriminant $D = p^2-4q$ nutne
druhou mocninou nezáporného celého čísla~-- označme ho~$a$. Potom
platí $p^2-4q = a^2$, čo môžeme napísať ako $(p-a)(p+a) = 4q$.
Keďže $4q>0$ a~$p+a>0$, je nutne $p-a> 0$. Číslo $4q$ je teda
rozložené na súčin dvoch kladných celých čísel $p-a\le p+a$.
Tieto dve čísla majú pritom rovnakú paritu, pretože ich súčet~$2p$
je párny. Keďže aj ich súčin~$4q$ je párny,
sú nutne obe párne. Ak označíme $p-a= 2k$ a~$p+a= 2l$, bude
$0 <k\le l$ a~$kl = q$, takže nutne $k= 1$ a~$l = q$. Platí teda $p-a=2$
a~$p+a= 2q$, z~čoho odvodíme rovnicu $q = p-1$. Ďalej postupujeme ako
v~predošlých riešeniach.

\poznamka
Po napísaní rovnice
$(p-a) (p+a) = 4q$ je možné postupovať aj~priamočiarejšie: Najskôr
vypíšeme všetky rozklady čísla~$4q$ na súčin dvoch kladných
čísel, a~síce $1 \cdot 4q$, $2 \cdot 2q$, $4 \cdot q$ (v~prípade
$q = 2$ sú posledné dva rozklady totožné), a~rozoberieme jednotlivé
prípady:

\*
Pre rozklad $1 \cdot 4q$ máme vďaka nerovnosti $1 <4q$ sústavu $p-a~= 1$,
$p+a~= 4q$. Po sčítaní oboch rovníc dostávame $2p = 4q+1$, čo
nemôže nastať, lebo ľavá strana rovnice je párna, zatiaľ čo pravá je nepárna.
\*
Podobne pre rozklad $2 \cdot 2q$ máme z~$2 <2q$ sústavu $p-a~= 2$,
$p+a~= 2q$, čo po sčítaní rovníc a~delení dvoma vedie na známu
rovnicu $p = q+1$.
\*
V~prípade rozkladu $4 \cdot q$ nemusíme riešiť prípad $q = 2$,
keďže ten je zahrnutý v~predchádzajúcom prípade. Pre $q = 3$ máme
$p-a~= 3$, $p+a~= 4$, čo nemá celočíselné riešenie. Pre $q \ge 5$
potom máme $p-a~= 4$ a~$p+a~= q$, čo po sčítaní vedie na $2p = 4+q$.
Ľavá strana rovnice je párna, a~teda aj pravá, takže
aj prvočíslo~$q$ je párne, čo nemôže nastať, pretože sme predpokladali $q\ge 5$.


\ineresc{č.\,4}
Označme $a$ celočíselný
koreň danej rovnice. Platí teda $a^2+pa+q = 0$. Rozoberieme prípady
podľa parity prvočísel $p$ a~$q$:

\*
Ak sú obe $p$, $q$ nepárne, nemôže byť číslo~$a$ nepárne, inak
by bolo $a^2+pa+q$ nepárne. Preto je $a$ párne, a~teda $a^2+pa$
je párne, takže nutne aj $q$ je párne, čo je spor.
\*
Ak $p = 2$, dostávame rovnicu $a^2+2a+q = 0$, ktorú napíšeme ako
$(a+1)^2+{(q-1)} = 0$. Vidíme, že ľavá strana rovnice je kladná,
preto v~tomto prípade nemáme riešenie.
\*
Ak $q = 2$, máme $a^2+pa+2 = 0$. Vidíme, že $a$ je nutne
deliteľom čísla~$2$, čo vedie na prípady $a~\in \{{-1},{-2}, 1, 2\}$.
Preskúmaním jednotlivých prípadov nájdeme jedinú možnosť $p = 3$.
Skúškou sa presvedčíme, že dvojica $(p, q)=(3, 2)$ naozaj
vyhovuje.

\poznamka
Toto riešenie je podobné
druhému riešeniu, avšak nedostáva sa explicitne k~rovniciam $p-q = 1$
resp. $p+q+1 = 0$. Niektoré jeho časti sa dajú modifikovať tak, že
sa priblíži k~druhému riešeniu: Napríklad v~druhom prípade možno
namiesto úpravy na štvorec použiť úvahu o~deliteľnosti pre
nájdenie $a~\in \{{-1}, 1, q,{-q}\}$, čo vedie na štyri rovnice o~jednej
neznámej~$q$.


\schemaABC
Za úplné riešenie dajte 6 bodov.

{Všeobecné poznámky:}
\ite1. Pri riešení rovnice $p-q = 1$ akceptujte aj konštatovanie, že
jediné dve prvočísla líšiace sa o~1 sú $2$ a~$3$, prípadne iné
ekvivalentné formulácie, ako napríklad že každé iné dve
prvočísla sa líšia aspoň o~2. Neakceptujte však, ak niekto bez
akéhokoľvek zdôvodnenia napíše, že z~tejto rovnice nutne vyplýva
$p = 3$ a~$q = 2$, bez toho, že pripomenie, že sa jedná o~dve prvočísla.
\ite2. Ak sa riešiteľ nijako nezmieni o~skúške, resp. spätne
nevyrieši rovnicu $x^2+3x+2 = 0$, strhnite bod. Ak korektným
spôsobom zdôvodní, že skúška nie je nutná (čo sa dá v~prvom
riešení), tak bod nestrhávajte.
\ite3. Za uhádnutie riešenia dajte 1~bod.
\ite4. Ak riešiteľ preskúma konečne veľa dvojíc $(p, q)$,
môže dostať maximálne 1~bod (za tú dvojicu, ktorá je riešením).
\ite5. Čiastočné body za jednotlivé riešenia sa nesčítajú.

\smallskip
{Riešenie 1.}
\* [1 bod] Zapísanie oboch Vi\`etových vzorcov.
\* [1 bod] Zdôvodnenie, že oba nie nutne rôzne korene sú
celočíselné.
\* [1 bod] Vypísanie možností pre $\{x_1, x_2\}$ na základe
vzťahu $x_1x_2 = q$.
\* [1 bod] Vyšetrenie prípadu $\{x_1, x_2\} = \{q, 1\}$
vedúceho na rovnicu $p+q+1 = 0$.
\* [1 bod] Vyšetrenie prípadu $\{x_1, x_2\} = \{{-q},{-1}\}$
vedúceho na rovnicu $p-q = 1$ (pozri prvú všeobecnú poznámku).
\* [1 bod] Overenie, že dvojica $p = 3$, $q = 2$ naozaj vyhovuje
(pozri druhú všeobecnú poznámku).

\item{}Neúplné riešenie:
V~prípade, že riešiteľ zabudne na jeden z~možných rozkladov
$x_1x_2$, dajte nanajvýš 4~body.

\smallskip
{Riešenie 2.}
\* [2 body] Odvodenie, že daný celočíselný koreň je deliteľom~$q$.
\* [1 bod] Vypísanie všetkých možností pre
hodnoty tohto deliteľa.
\* [1 bod] Vyšetrenie prípadov $a~={-1}$ resp. $a={-q}$
(vedúcich na rovnakú rovnicu $p-q = 1$, pozri prvú všeobecnú poznámku).
\* [1 bod] Vyšetrenie prípadov $a~= 1$ resp. $a~= q$ (vedúcich
na rovnakú rovnicu $p+q+1 = 0$).
\* [1 bod] Overenie, že dvojica $p = 3$, $q = 2$ naozaj vyhovuje
(pozri druhú všeobecnú poznámku).

\item{}Neúplné riešenie:
V~prípade, že riešiteľ zabudne vyšetriť niektoré delitele
čísla~$q$ (napríklad záporné), dajte nanajvýš 4~body.

\smallskip
{Riešenie 3.}
\* [1 bod] Uvedomenie si, že diskriminant musí byť druhá
mocnina celého čísla (stačí konštatovanie, alebo napísanie
rovnice $p^2-4q = a^2$).
\* [1 bod] Prepísanie skúmanej rovnice na tvar $(p-a) (p+a)= 4q$.
\* [1 bod] Vypísanie všetkých možných rozkladov obmedzených prípadnými úvahami ako
rozbor, ktorý z~činiteľov je väčší; uvedomenie si, že stačí uvažovať kladné
činitele\dots{} (pozri poznámku).
\* [1 bod] Vyriešenie rozkladu $p-a~= 2$, $p+a~= 2q$, ktorý
vedie na rovnicu $p-q = 1$ (pozri prvú všeobecnú poznámku).
\* [1 bod] Vylúčenie všetkých ostatných rozkladov. Tento
bod je implicitne udelený aj v~prípade, keď sa riešiteľ šikovne
vyhne akýmkoľvek ďalším rozkladom podobne ako vo vzorovom
riešení.
\* [1 bod] Overenie, že dvojica $p = 3$, $q = 2$ naozaj vyhovuje
(pozri druhú všeobecnú poznámku).

\item{}Neúplné riešenie.
V~prípade, že riešiteľ vypíše všetky rozklady $4q$, ale
nezdôvodní (alebo nekorektne vyšetrí) poradie činiteľov,
strhnite nanajvýš 1~bod. Ak však neuvedie všetky rozklady $4q$,
dajte nanajvýš 4~body.

\smallskip
{Riešenie 4.}
\* [2 body] Úplné vyriešenie prípadu, keď $p$, $q$ sú
nepárne. Pri menšej chybe strhnite 1~bod.
\* [1 bod] Vyriešenie prípadu $p = 2$, napr. pomocou kladnosti
ako vo vzorovom riešení alebo deliteľnosti (pozri poznámku).
\* [2 body] Vyriešenie prípadu $q = 2$. Za menšie chyby
(napr. zabudnutie deliteľov) strhnite 1~bod.
\* [1 bod] Overenie, že dvojica $p = 3$, $q = 2$ naozaj vyhovuje
(pozri druhú všeobecnú poznámku).

\endpetit\bigskip
}

{%%%%%   A-S-2
Z~predpokladu $|AB|<|AC|$ vyplýva, že bod~$D$
leží na polpriamke opačnej k~$BA$, zatiaľ čo bod~$E$ leží vnútri
úsečky~$AC$. Trojuholníky $ABC$, $AED$ sú zhodné podľa
vety $sus$, lebo sa zhodujú v~uhle pri vrchole~$A$
a~$|AB| = |AE|$ a~$|AC| = |AD|$, ako vyplýva
z~definície bodov $D$ a~$E$. Z~tejto zhodnosti máme
$|\angle AED| = |\angle ABC| =\beta<90\st$, pretože trojuholník $ABC$
je podľa predpokladu ostrouhlý. To vďaka rovnosti ${|\angle
AEF| = 90^\circ}$ znamená, že $|\angle DEF| = 90^\circ- \beta$ (\obr).
Štvoruholník $DFEA$ je tetivový, pretože oba jeho uhly pri vrcholoch
$D$ a~$E$ sú pravé. Pre obvodové uhly nad tetivou~$DF$ tak máme
$|\angle DAF|=|\angle DEF|= 90^\circ-\beta$. Ak označíme $P$
priesečník $AF$ a~$BC$, bude v~trojuholníku~$ABP$ platiť $|\angle
PBA| = \beta$ a~$|\angle BAP| = |\angle DAF| = 90^\circ- \beta$, takže
$|\angle APB| = 90^\circ$ čiže $AF \perp BC$.

\poznamka
Riešenie sa dá rôznym spôsobom
modifikovať redefinovaním bodov, napr. tak, že bod~$F'$ definujeme
ako priesečník kolmice z~$A$ na $BC$ a~$D$ na $AD$, a~pomocou uhlov
dokážeme, že $AE \perp EF'$. Také redefinovanie je pri podobných úlohách často
nevyhnutné~-- uvedené riešenie však ukazuje, že v~tomto prípade to tak nie je.
\inspinsp{a68.9}{a68.10}%

\ineresc{č.\,2}
Označme $P$ kolmý priemet bodu~$F$ na $BC$.
Konvexné štvoruholníky $BDFP$ a~$CEPF$ sú tetivové vďaka pravým
uhlom pri vrcholoch $D$, $P$ a~$E$. Tetivový je aj štvoruholník
$BDCE$ (\obr), pretože sa jedná o~rovnoramenný lichobežník, a~to
z~dôvodu, že osi jeho strán $BE$ a~$CD$ splývajú s~osou uhla $BAC$.
Chordály dvojíc kružníc
opísaných štvoruholníkom $BDFP$, $CEPF$ a~$BDCE$ sú priamky
$BD$, $CE$, $PF$, takže prechádzajú jedným bodom, čo znamená, že
body $A$, $P$, $F$ sú kolineárne, odkiaľ už vyplýva $AF \perp BC$.

\poznamka
Aby sme sa vyhli pojmu {\it chordála}, môžeme uvedený postup
modifikovať nasledovne: Označme
$P'$ kolmý priemet $A$ na $BC$ a~$F'$ priesečník $AP'$ a~kolmice
z~$D$ na $AD$. Body $B$, $D$, $F'$, $P'$ ležia vďaka pravým uhlom na kružnici
a~na kružnici, ako už vieme, ležia aj body $B$, $C$, $D$, $E$. Pre mocnosť
bodu~$A$ tak máme $|AP'| \cdot |AF'| = {|AB| \cdot |AD|} = |AE| \cdot
|AC|$, čo znamená, že aj body $P'$, $F'$, $E$, $C$ ležia na kružnici,
takže z~$|\angle F'P'C| = 90^\circ$ dostávame $|\angle F'EC| = 90^\circ$,
odkiaľ už vyplýva $F = F'$ a~${P'=P}$.

\ineresc{č.\,3}
Štvoruholník $ADFE$ má pravé
uhly pri vrcholoch $D$, $E$,
takže jeho vrcholy ležia na kružnici s~priemerom~$AF$, ktorý teda
prechádza stredom kružnice opísanej trojuholníku $ADE$. Označme~$t$
kolmicu na priamku~$DE$ prechádzajúcu bodom~$A$ (\obr).
Podľa úlohy~D4 k~2.~úlohe domáceho kola dostávame, že
priamky $AF$ a~$t$ sú súmerne združené
podľa osi~$o$ uhla~$BAC$ (sú vzhľadom k~tomuto uhlu izogonálne).
Priamka~$DE$ sa pritom v~tejto osovej súmernosti zobrazuje na priamku~$CB$,
takže z~${t \perp DE}$ hneď dostávame $AF \perp BC$.

\ineresc{č.\,4}
Označme $P$ kolmý priemet vrcholu~$A$ na stranu~$BC$
a~$F'$ priesečník polpriamky~$AP$ s~kolmicou z~bodu~$D$ na $AD$ (\obr).
V~štvoruholníku $DBPF'$ sú uhly pri vrcholoch $D$ a~$P$
pravé a~$|\angle DBP| = 180^\circ- \beta$, takže $|\angle PF'D| = {|\angle
AF'D| = \beta}$. Z~pravouhlého trojuholníka $AF'D$ tak máme
$|AF'| = \frc {|AD|} {\sin \beta} = \frc {b} {\sin \beta}$. Ak
definujeme~$F''$ ako priesečník polpriamky~$AP$
s~kolmicou z~$E$ na~$AE$, dostaneme podobne $|AF''| = \frc {c} {\sin \gamma}$.
Zo sínusovej vety pre trojuholník $ABC$ však máme $\frc {b} {\sin \beta} =
\frc {c} {\sin \gamma}$, takže $|AF'| = |AF''|$. Keďže polpriamky
$AF'$ a~$AF''$ sú totožné, je nutne ${F' = F'' = F}$.
\inspinsp{a68.11}{a68.12}%


\schemaABC
Za úplné riešenie dajte 6 bodov.

{Všeobecné poznámky}: \par\nobreak
\ite1. Ak chýba slovný popis polohy bodov $D$ a~$E$, body nestrhávajte.
\ite2. V~prípade riešenia používajúceho analytickú geometriu
dajte 6~bodov, ak je správne, a~0~bodov v~prípade, že je chybné alebo nedokončené.
\ite3. Čiastočné body za jednotlivé riešenia sa nesčítajú.

\smallskip
{Riešenie 1.}
\* [1 bod] Za dôkaz zhodnosti trojuholníkov $ABC$ a~$AED$.
\* [1 bod] Za dôkaz tetivovosti štvoruholníka $ADFE$.
\* [1 bod] Za vyjadrenie veľkosti uhla $DAF$ (či $DFA$) pomocou uhla $\beta$,
alebo uhla $EAF$ (či $EFA$) pomocou uhla~$\gamma$.
\* [2 body] Za využitie rovnosti obvodových uhlov v~kružnici nad
priemerom~$AF$.
\* [1 bod] Za dokončenie riešenia, \tj. výpočet potvrdzujúci pravý uhol pri bode~$P$.

\item{}V~prípade, že riešiteľ nanovo vymedzí bod~$F$ (pozri poznámku k~1.
riešeniu), je nutné uspôsobiť schému inej definícii.

\item{}Neúplné riešenie:
Dajte nanajvýš 3~body (prvé body uvedené v~bodovacej schéme).

\smallskip
{Riešenie 2.}
\* [1 bod] Za dôkaz, že body $B$, $C$, $D$, $E$ ležia na kružnici
(napr. konštatovaním, že tvoria rovnoramenný lichobežník).
\* [2 body] Za dôkaz, že body $D$, $F$, $P$, $B$ resp. $C$, $F$, $P$, $E$
ležia na kružnici. Tieto body dajte, iba keď je jasné, že bod~$P$
je definovaný ako kolmý priemet bodu~$F$ na $BC$. V~prípade
zmienky iba jednej z~týchto štvoríc dajte 1~bod.
\* [3 body] Za dokončenie riešenia, \tj. použitie chordál troch kružníc.

\item{}V~prípade, že riešiteľ nanovo vymedzí bod~$F$ (pozri poznámku k~2.~riešeniu),
je nutné uspôsobiť schému tomuto redefinovaniu.

\item{}Neúplné riešenie:
Dajte nanajvýš 3~body (prvé body uvedené v~bodovacej schéme).

\smallskip
{Riešenie 3.}
\* [1 bod] Za tvrdenie, že body $D$, $E$ sú obrazy bodov $C$, $B$ v~osovej súmernosti podľa~$o$.
\* [2 body] Za explicitné uvažovanie priamky $t$, \tj. kolmice z~$A$ na $DE$.
\* [3 body] Za dokončenie riešenia, \tj. použitie tvrdenia
o~izogonálnosti $AF$ a~$t$ v~kombinácii so súmernou združenosťou priamok $DE$ a~$BC$.

\item{}Neúplné riešenie:
Dajte nanajvýš 3~body (prvé body uvedené v~bodovacej schéme).

\smallskip
{Riešenie 4.}
\* [2 body] Za vyjadrenie veľkosti úsečky~$AF'$ iba pomocou
prvkov trojuholníka $ABC$, pričom musí byť jasné, že bod~$F$ bol
redefinovaný na $F'$ a~akým spôsobom.\nopagebreak
\* [4 body] Za dokončenie riešenia, \tj. ďalšie výpočty
vedúce k~dôkazu, že $F'= F$ (pozri vzorové riešenie).

\nopagebreak
\item{}Neúplné riešenie:
Dajte nanajvýš 2~body uvedené v~bodovacej schéme.
\endpetit\bigskip}

{%%%%%   A-S-3
Nech $a>1$ je prirodzené číslo. Ak je
párne, dostaneme prevedením úpravy~$\frac12 {a}$. Ak je nepárne,
dostaneme po úprave $a+3$, čo je párne číslo, takže
v~ďalšom kroku dostaneme $\frac12 (a+3)$. Vďaka predpokladu $a>1$ platí
$\frac 12{a}<a$, a~ak je $a>3$, bude aj~$\frac12 (a+3)<a$. To dokazuje, že
každé číslo $a>1$ okrem $a=3$ sa po nanajvýš dvoch úpravách
zmenší. Z~každého čísla väčšieho ako~1 teda po konečnom
počte krokov dostaneme 1 alebo~3.
Z~čísla~1 dostaneme postupne 4, 2 a~opäť 1, z~čísla~3 najskôr~6 a~potom opäť~3.
Zhrnutím prichádzame k~záveru, že k~jednému z dvoch "zacyklení" z~predchádzajúcej vety
nakoniec dôjde pre každé východiskové číslo $a\ge1$.

Ďalej si všimnime, že naša úprava zachováva deliteľnosť
číslom~3. Keďže z~čísla~$a$ dostávame buď $a+3$, alebo $\frac12 {a}$,
je číslo po úprave deliteľné tromi práve vtedy, keď je tromi deliteľné
číslo pred úpravou.
Z~toho hneď vyplýva, že ak sme na začiatku mali
číslo deliteľné tromi, tak sa táto deliteľnosť tromi zachová,
takže nikdy nemôžeme dostať~1. V skutočnosti sa (podľa prvého
odseku) dostaneme k~číslu~3.

Ak naopak na začiatku máme číslo, ktoré tromi deliteľné nie je, tak
nemôžeme dôjsť k~číslu~3, preto dôjdeme k~číslu~1.

\zaver
Vyhovujú všetky prirodzené čísla, ktoré nie sú
deliteľné tromi.

\poznamkac 1.
Riešenie môžeme formalizovať pomocou matematickej indukcie.
Chceme dokázať, že 1 možno dostať práve z~čísel nedeliteľných tromi.
Najskôr tvrdenie overíme pre 2 a~3. Nech $n>3$, budeme predpokladať,
že tvrdenie platí pre všetky čísla menšie ako~$n$.
Po nanajvýš dvoch krokoch dostaneme číslo menšie ako~$n$, ktoré je
deliteľné tromi práve vtedy, keď je tromi deliteľné~$n$, takže použitím
indukčného predpokladu dôkaz dokončíme.

\poznamkac 2.
Záujemcom odporúčame pozrieť sa na 1. úlohu z~MMO 2017\footnote{\pdfklink{skmo.sk/dokument.php?id=2513}{https://skmo.sk/dokument.php?id=2513}},
ktorá sa tejto úlohe veľmi podobá.



\schemaABC
Za úplné riešenie dajte 6~bodov.

Úplné riešenie:
\* [2 body] Dôkaz, že každé číslo $a> 1$ okrem $a= 3$ sa po nanajvýš dvoch krokoch zmenší. (1)
\* [1 bod] Konštatovanie, že v~dôsledku (1) sa každé číslo po konečnom počte krokov zmení na 1 alebo~3. (2)
\* [1 bod] Dôkaz, že úprava čísla zachováva deliteľnosť tromi, \tj. že
číslo pred úpravou je deliteľné tromi práve vtedy, keď je tromi deliteľné
číslo po úprave. (3)\\
Tolerujte, ak je táto ekvivalencia zapísaná len ako implikácia.
\* [1 bod] Zmienka, že v~dôsledku (3) sme číslo 3 mohli získať iba z~čísla deliteľného tromi. (4)
\* [1 bod] Dôkaz pomocou (2) a~(4), že z~čísel nedeliteľných tromi vždy dostaneme~1.

Neúplné riešenie:
\* Za preverenie konečného počtu čísel nedávajte žiadny bod.
\* Ak riešiteľ napíše argument typu \uv{čísla sa postupne
zmenšujú}, dajte 2~body iba v~prípade, keď jeho vyjadrenie
explicitne obsahuje, že sa tak deje po nanajvýš dvoch krokoch --
inak dajte iba 1~bod.
\* Ak riešiteľ neoverí, že pre $a=1$ dostane po dvoch krokoch opäť číslo~1, body nestrhávajte.
\* Za správnu odpoveď ako nezdôvodnenú hypotézu dajte 1~bod.\nopagebreak
\* Čiastočným riešeniam, ktoré nijako nevyužívajú deliteľnosť
tromi, dajte nanajvýš 3~body (ako je spomenuté v~predošlých bodoch).
\endpetit
}

{%%%%%   A-II-1
Dokážeme, že hľadané najmenšie $n$ je 1010. Predpokladajme,
že $n \le 1009$. Potom má Jerry nasledujúcu jednoduchú stratégiu:
Položí figúrku na akékoľvek políčko a~potom vždy ťahá tak, aby
neprehral. Taký ťah by nemohol spraviť, len ak by naľavo
aj napravo od zvoleného políčka bolo nanajvýš $n-1$ políčok. To by sme potom ale mali
dokopy nanajvýš len $2 (n-1)+1 = 2n-1 \le 2017$ políčok, čo je spor.

Predpokladajme, že $n = 1010$.
Očíslujme políčka zľava $1,2, \ldots, 2018$. Ak je figúrka na políčku
1009 alebo 1010, stačí Tomovi povedať 1010 a~Jerry nemôže urobiť ťah. Ak je
figúrka na políčku $k~<1009$, povie Tom $1009-k$. Vtedy Jerry nemôže
spraviť ťah doprava, lebo by sa ocitol na prehrávajúcom políčku~1009. Nutne
teda musí spraviť ťah doľava. Ak tento ťah nemôže spraviť, prehráva. Ak
môže, priblíži sa k~ľavému okraju. Lenže doľava sa nemôže posúvať
donekonečna, takže ak Tom opakuje túto stratégiu, tak po konečnom počte
krokov vyhrá. Analogicky, ak je figúrka na políčku $k> 1010$, povie Tom
číslo $k-1010$, čím Jerryho prinúti urobiť ťah doprava, a~takto pokračuje,
kým nevyhrá.

\ineres
Ukážeme inú stratégiu pre Toma pre $n = 1010$. Ako sme už objasnili
v~predošlom riešení, ak je figúrka na políčku 1009 alebo 1010, stačí Tomovi
povedať 1010. Ak je figúrka na políčku $k~\le 504$, povie Tom
$1009-k$. Keďže $k- {(1009-k)} \le {-1}$, nemôže Jerry spraviť ťah doľava,
takže nutne musí spraviť ťah doprava na prehrávajúce políčko~1009.
Symetricky, ak $k~\ge 1515$, povie Tom $k-1010$ a~donúti Jerryho
urobiť ťah na prehrávajúce políčko~1010, keďže $k+(k-1010) \ge 2020$. Ďalej
ak je figúrka na políčku $505 \le k~\le 1008$, Tom povie 1010,
a~Jerry musí nutne spraviť ťah doprava, čím sa figúrka ocitne na políčku $1515
\le l \le 2018$, o~ktorom už vieme, že na ňom Jerry prehrá. Napokon
ak je figúrka na políčku $1011 \le k\le 1514$, bude po ťahu 1010
na políčku $1 \le l \le 504$, ktoré je pre Jerryho prehrávajúce.

\poznamka
Táto stratégia je zaujímavá tým, že dokazuje, že
Tomovi na výhru stačia nanajvýš tri ťahy. Z~praktického hľadiska je teda
táto stratégia pre neho výhodná.

%\poznamkac 2.
%V~prípade $506 \leq k~\leq 1009$, resp. $1011 \leq k\leq 1514$ sme použili ťah~1010. V~skutočnosti by stačil aj ťah~1009. V~tom prípade sa Jerry ocitne na políčku $1515 \le l \le 2018$, resp. $2 \le l \le 505$, ktoré sú už vyriešené ako prehrávajúce.

\ineres
Pre $n = 1010$ ukážeme ešte jednu jednoduchú Tomovu stratégiu,
pri ktorej dokonca ani nemusí poznať polohu
figúrky. Stratégia je nasledujúca: Nech je figúrka na akomkoľvek
políčku~$k$, použije Tom dvojicu ťahov 1010 a~1009. Vysvetlíme, prečo
je taká stratégia vyhrávajúca.

Ak $k=1009$ alebo $k=1010$, je už ťah 1010 víťazný. Ak $k\le 1008$, spôsobí dvojica
ťahov 1010 a~1009, že sa figúrka nutne posunie na políčko $k+1$.
Analogicky ak $k\ge 1011$, tieto ťahy spôsobia, že sa figúrka posunie
na políčko $k-1$. Uvedená dvojica ťahov teda v~každom kroku figúrku
približuje k~políčku 1009, resp. 1010. Po konečnom počte krokov sa tak figúrka
ocitne na jednom z~týchto políčok, a~na ňom následne prehrá.


\nobreak\medskip\petit
\BeginSchema

\Podnadpis Za úplné riešenie dajte 6~bodov:

\Body1 Správny výsledok $n = 1010$ (tento bod dajte
v~neúplných riešeniach iba v~prípade, že je explicitne uvedené, že sa jedná
o~hypotézu o~výsledku).
\Body2 Jerryho stratégia pre $n \le 1009$ (rozobraná nižšie).
\Body3 Tomova stratégia pre $n = 1010$ (rozobraná nižšie).

\Podnadpis Jerryho stratégia vo všeobecnosti:

\Item V~prípade, že zdôvodnenie správnosti nie je
dostatočné, alebo stratégia obsahuje opraviteľnú chybu, dajte nanajvýš
1~bod.
\Item Tiež dajte nanajvýš 1~bod, ak je stratégia správne
opísaná a~zdôvodnená pre nejaké $n <1009$, pričom sa dá jednoducho
upraviť, aby fungovala pre všetky $n \le 1009$.
\Item Za stratégiu, ktorá nefunguje pre všetky $n \le
1009$, a~ani sa nedá jednoducho upraviť, aby fungovala, neudeľujte žiadny bod.
\Item Ak riešiteľ opíše Jerryho stratégiu pre $n = 1009$,
a~tá funguje aj pre $n <1009$, ale explicitne neuvedie, že funguje
aj pre menšie~$n$, tak bod strhnite práve vtedy, keď nie je evidentné, že
naozaj funguje aj pre $n <1009$.

\Podnadpis Tomova stratégia vo všeobecnosti:

\Item V~prípade, že stratégia alebo dôkaz jej správnosti
obsahuje malú opraviteľnú chybu, strhnite 1~bod.
\Item Ak je stratégia nesprávna, tak sa pri udeľovaní
čiastočných bodov riaďte podľa jednej z~nasledujúcich troch schém.
\Item Čiastočné body za {\it rôzne\/} Tomove stratégie sa nesčítajú.

\Podnadpis Tomova stratégia z~prvého riešenia:

\Body1 Stratégia pre políčka 1009 a~1010.
\Body1 Definovanie stratégie pre zvyšné políčka.
\Body1 Zdôvodnenie, že taká stratégia naozaj funguje v~konečnom počte krokov.

\Podnadpis Tomova stratégia z~druhého riešenia:

\Body1 Stratégia pre políčka 1009 a~1010.
\Body1 Stratégia pre \uv {krajné} políčka $k~\leq 504$ a~$k\ge 1515$.
\Body1 Stratégia pre zvyšné $505 \leq k~\leq 1008$ a~$1011\leq k~\leq 1514$.

\Podnadpis Tomova stratégia z~tretieho riešenia:

\Body1 Explicitné definovanie stratégie.
\Body1 Zdôvodnenie, že figúrka sa po každej dvojici ťahov
ocitne na políčku $k+1$ alebo $k-1$.
\Body1 Záver, že po konečnom počte krokov sa bude nachádzať
na prehrávajúcom políčku 1009 alebo 1010.

\EndSchema

\endpetit
\bigbreak}

{%%%%%   A-II-2
Z~danej rovnosti vyplýva, že číslo $n^{n-1}$ je celé. Toto číslo má
zrejme rovnakú paritu ako číslo~$n$. Číslo $4m^2+2m+3$ je však vždy nepárne,
takže aj~$n$ musí byť nepárne. Tým pádom je $n-1$ párne, takže
$n^{n-1}$ je druhá mocnina nepárneho čísla (z~dvoch možných základov
ďalej vezmeme ten kladný).

Položme $n^{n-1} = k^2$, pričom $k$ je kladné nepárne číslo. Dostávame tak rovnicu
$$
k^2 = 4m^2+2m+3. \eqno {(1)}
$$
Jej pravú stranu doplníme štandardným spôsobom na štvorec, čím
dostávame
$$
k^2 = \left (2m+\frac12 \right)^{\!2}+\frac {11} {4}.
$$

Aby sme mali na oboch stranách celé čísla, vynásobíme získanú rovnicu
číslom~4 a~následne ju upravíme na súčinový tvar:
$$
\displaylines{
4k^2 = (4m+1)^2+11 ,\cr
(2k-4m-1) (2k+4m+1) = 11.
}
$$
Súčin celých čísel $a=2k-4m-1$ a~$b=2k+4m+1$ je teda rovný~11, pritom ich
súčet $a+b=4k$ je číslo kladné, takže kladné sú aj obe čísla $a$ a~$b$. Keďže
11~je prvočíslo, musí byť $\{a,b\}=\{1,11\}$, a~teda $4k=12$, čiže $k=3$.
Z~rovnosti $a=5-4m$ potom pre $a=1$ máme $m=1$, zatiaľ čo pre $a=11$ celočíselné~$m$
neexistuje. A~rovnicu $n^{n-1}=k^2=9$ zrejme spĺňa jediné celé $n=3$.

Jediná dvojica celých čísel $(m, n)$ vyhovujúca zadanej rovnici je $(1,3)$.

\poznamka
Riešenie rovnice $ab=11$ sa samozrejme dá nájsť aj bez ďalších úvah rozobraním štyroch
možností $a=\pm1,\pm11$.
A~namiesto použitia rovnosti $a={5-4m}$ sme mohli hodnotu $k=3$ dosadiť do~(1) a~vyriešiť
kvadratickú rovnicu s~koreňmi $m_1=1$ a~$m_2={-3/2}$.

\ineres
Po tom, ako dokážeme, že $4m^2+2m+3$ je druhá mocnina celého
čísla, môžeme postupovať aj inak. Pre všetky $m\ge2$ totiž platí
$$
\eqalignno{
(2m)^2&<4m^2+2m+3 <(2m+1)^2 \cr
\noalign {\hbox {a~naopak pre $m \le-2$ platí}}
(-2 m-1)^2&<4m^2+2m+3 <(-2 m)^2. \cr
}
$$
Tým pádom ostáva preveriť iba $m \in \{-1,0,1\}$. Ľahko zistíme, že
k~riešeniu vedie iba $m = 1$, z~čoho vyplýva $n = 3$.

\poznamka
Argument so \uv{zovrením} čísla medzi dva po sebe idúce
štvorce možno použiť aj inak. Pre $m \ge 0$ platí $(2m)^2 <
4m^2+2m+3$. Keďže $4m^2+2m+3$ je štvorec, tak to nutne znamená
$(2m+1)^2 \leq 4m^2+2m+3$, z~čoho dostaneme $m \leq 1$.
Pre $m\le{-1}$ možno podobne využiť nerovnosť $({-2}m-1)^2<4m^2+2m+3$.

\ineres
Ukážeme ešte jeden spôsob, ako vyriešiť rovnicu~(1), keď už vieme,
že $k$~je kladné nepárne číslo. Rovnicu upravíme na tvar
$$
(k-2m) (k+2m) = 2m+3. \eqno {(2)}
$$

Ak $m \ge 0$, je pravá strana rovnice~(2) kladná. Keďže $k+2m$ je
kladné, je aj činiteľ $k-2m$ kladný, takže $k-2m \ge 1$.
Vynásobením tohto odhadu číslom~$k+2m$ dostaneme $2m+3\ge k+2m$,
takže $3 \ge k$. Z~dvoch možných hodnôt $k=1$ alebo $k=3$
dôjdeme k~riešeniu iba pre $k=3$, keď máme $(m, n) = (1,3)$.

Ak $m <0$, môžeme dokonca predpokladať, že $m\le{-2}$, lebo pre $m={-1}$
vychádza $k^2=3$, čo nie je možné. Položme $m'={-m}$, potom $m'\ge 2$. Upravme~(2) na
$$
(k+2m')(2m'-k) = 2m'-3.
$$
Keďže čísla
$k+2m'$, $2m'-3$ sú kladné, je kladné aj číslo $2m'-k$, platí teda $2m'-k\ge1$.
Vynásobením tohto odhadu číslom~$k+2m'$ dostaneme $2m'-3 \ge k+2m'$, teda
$-3\ge k$, čo odporuje predpokladu $k>0$.

\poznamka
V~prípade $m \ge 0$ sme mohli odhad $k-2m \ge 1$ využiť
aj inak. Je totiž ekvivalentný s~nerovnosťou $k+2m\ge 4m+1$, takže
$2m+3 = (k-2m)(k+2m)\ge {4m+1}$, odkiaľ hneď vyplýva $m \leq 1$.


\nobreak\medskip\petit
\BeginSchema

\Podnadpis Za úplné riešenie dajte 6 bodov:

\Body1 Zdôvodnenie, že číslo $n$ musí byť nepárne.
\Body1 Zdôvodnenie, že číslo $n^{n-1}$ musí byť druhá mocnina celého čísla.
\Body4 Dokončenie riešenia (rozobrané nižšie).

\Podnadpis Všeobecné poznámky:

\Item V~prípade uhádnutia výsledku $(m, n) = (1,3)$ dajte 1 bod.
\Item V~prípade preskúmania konečného počtu možností dajte nanajvýš 1~bod, a~síce za správny výsledok.
\Item Čiastočné body z~{\it rôznych\/} postupov sa nesčítajú.

\Podnadpis Dokončenie riešenia ako v~1. riešení:

\Body2 Úprava rovnice na súčinový tvar.
\Body1 Rozbor všetkých možností rozkladu $ab=11$ ako v~riešení či ako v~poznámke.
\Body1 Nájdenie výsledku $(m, n) = (1, 3)$.

Neúplné riešenie: V~prípade menšej chyby pri úprave na
súčinový tvar strhnite 1~bod. V~prípade zabudnutia nejakého rozkladu
tiež strhnite 1~bod.

\Podnadpis Dokončenie riešenia ako v~2. riešení:

\Body1 Formulácia úvahy o~tom, že číslo $4m^2+2n+3$ nemôže
ležať medzi dvoma po sebe idúcimi štvorcami (alebo jej ekvivalentná
formulácia, pozri poznámku).
\Body1 Vylúčenie prípadu $m\ge2$.
\Body1 Vylúčenie prípadu $m \le-2$.
\Body1 Vyriešenie zvyšných prípadov a~nájdenie výsledku $(m, n) = (1,3)$.

Neúplné riešenie: V~prípade chýb pri úpravách nerovností
strhnite 1~až~2 body, podľa počtu a~závažnosti chýb.

\Podnadpis Dokončenie riešenia ako v 3. riešení:

\Body1 Napísanie rovnice v tvare $(k-2m) (k+2m) = 2m+3$.
\Body1 Vylúčenie prípadu $m <0$.
\Body1 Rozobranie prípadu $m \ge 0$ vedúceho na $k~\leq 3$ alebo $m \leq 1$ ako v~poznámke.
\Body1 Rozobranie zvyšných prípadov a~nájdenie výsledku $(m, n) = (1,3)$.

Neúplné riešenie: V~prípade chýb pri úpravách nerovností
strhnite 1~až~2 body, podľa počtu a~závažnosti chýb.

\EndSchema

\endpetit
\bigbreak}

{%%%%%   A-II-3
Najskôr dokážeme, že $F$ je stredom kružnice opísanej trojuholníku $ADE$.
Z~$|BA| = |BE|$ vyplýva, že trojuholník $BAE$ je rovnoramenný, takže
$|\angle BAE| = {90^\circ- \frac12\beta}$, a~preto $|\angle CAE| = \frac12\beta $.
Podobne z~toho, že trojuholník $CAD$ je rovnoramenný,
dostaneme $|\angle BAD| = \frac12\gamma$. Tým pádom
$|\angle DAE| = 90^\circ- \frac12\beta - \frac12\gamma = 45^\circ$ (\obr,
body $D$ a~$E$ ležia vnútri strany~$BC$
v~uvedenom poradí, lebo $|CD|+|BE|=|CA|+|CB|>|BC|$).

Na kružnici so stredom~$F$ a~polomerom $|FD|=|FE|$ ležia vďaka pravému
stredovému uhlu $DFE$ všetky tie body polroviny $DEF$, z ktorých úsečku~$DE$
vidno pod uhlom~45\st, teda aj bod~$A$. Bod~$F$ je preto stredom kružnice opísanej
trojuholníku~$ADE$, ako sme na úvod sľúbili dokázať.
\insp{a68ii.1}%

Z~dokázanej rovnosti $|FA|=|FE|$ vyplýva, že bod~$F$ leží
na osi úsečky~$AE$, ktorá je vďaka rovnosti $|BA| = |BE|$ zároveň aj osou uhla
$ABC$, a~podobne priamka~$CF$ je osou uhla $ACB$ (to navyše
znamená, že $F$~je aj stredom kružnice vpísanej trojuholníku $ABC$). Tým
pádom $|\angle CBF| = \frac12\beta$ a~$|\angle BCF| = \frac12\gamma$, takže
z~trojuholníka~$BFC$ dopočítame, že jeho tretí uhol je rovný
$180^\circ- \frac12\beta- \frac12\gamma = 135^\circ$.

\poznamka
Po zistení, že $F$ je stredom kružnice opísanej trojuholníku~$ADE$, sme
mohli postupovať napríklad aj takto: Z~vety o~obvodovom a~stredovom
uhle vyplýva $|\angle AFD| = 2 |\angle AEB| = 180^\circ-\beta$.
Štvoruholník $AFDB$ je teda tetivový. Podobne aj štvoruholník $AFEC$ je
tetivový. Pomocou toho vypočítame
$$
\align
|\angle BFC| =& 90^\circ+|\angle BFD|+|\angle EFC| =\\
=& 90^\circ+|\angle BAD|+ |\angle EAC|
= 90^\circ+\frac {\gamma} {2}+\frac {\beta} {2} = 135^\circ.
\endalign
$$

Ďalšia možnosť je vypočítať súčet veľkostí uhlov $AFB$ a~$AFC$, ktoré
sa pomocou spomenutých tetivových štvoruholníkov prenesú na uhly $ADB$
a~$AEC$, ktorých veľkosti sú postupne $90^\circ+\frac12\gamma$
a~$90^\circ+\frac12\beta$.

\ineres
Pri zvyčajnom označení platí $|BE| = c$ a~$|CD| = b$, takže ľahko vypočítame $|BD| = a-b$
a~$|CE| = a-c$. Preto $|DE| = a- (a-b)-(a-c) = b+c-a$. Označme $M$ stred
úsečky~$DE$. Keďže $DEF$ je rovnoramenný pravouhlý trojuholník, je
$|MF| =|MD|=\frac12|DE| = \frac12(b+c-a)$. Ďalej tak máme
$$
|BM| = |BD|+|DM| = (a-b)+\tfrac12(b+c-a)= \tfrac12(a+c-b) = s-b,
$$
pričom $s$ označuje polovicu obvodu trojuholníka~$ABC$. Platí teda (pozri
dopĺňajúcu úlohu~D1 k~5.~úlohe domáceho kola), že $M$ je dotykový bod
kružnice vpísanej trojuholníku $ABC$. Keďže trojuholník $ABC$ je pravouhlý, zrejme platí
(pozri aj nasledujúcu dopĺňajúcu úlohu), že táto kružnica má polomer rovný
${s-a}= \frac12({b+c-a}) = |MF|$. A~keďže $MF \perp BC$, dostávame,
že $F$ je stredom kružnice vpísanej trojuholníku $ABC$. Vďaka tomu
$|\angle CBF| = \frac12\beta$ a~$|\angle BCF| = \frac12\gamma$, z~čoho už dopočítame
$|\angle BFC|$ ako v~predošlom riešení.

\poznamka
Na rozdiel od predošlého riešenia sme vôbec nepotrebovali
odhaliť, že $F$ je stred kružnice opísanej trojuholníku $ADE$.
\insp{a68ii.2}%

\ineres
Podobne ako v~predošlom riešení definujeme bod~$M$ a~vypočítame
$|BM| = \frac12(a+c-b)$ a~$|MF| = \frac12(b+c-a)$. Ďalej označme $|\angle
MBF| = \delta$ a~$|\angle MCF| = \epsilon$ (\obr). Z~pravouhlého trojuholníka
$MFB$ máme
$$
\tan \delta = \frac {b+c-a} {a+c-b}\quad \hbox {a~analogicky}\quad
\tan \epsilon = \frac {b+c-a} {a+b-c}.
$$
Ďalej použijeme známy vzorec $\tan (180^\circ-x) ={-\tan x}$, následne
súčtový vzorec pre tangens a~nakoniec Pytagorovu vetu $a^2 = b^2+c^2$ na
výpočet
$$
\align
\tan |\angle BFC|&
= \tan (180^\circ- \delta-\epsilon) =-\tan (\delta+\epsilon) =-\frac {\tan
\delta+\tan \epsilon} {1- \tan \delta \tan \epsilon} = \cr
&=-\frac {\frac {b+c-a} {a+c-b}+\frac {b+c-a} {a+b-c}} {1- \frac {b+c-a} {a+c-b} \cdot \frac {b+c-a} {a+b-c}} =-\frac {2a (b+c-a)} {(a+c-b) (a+b-c)-(b+c-a)^2} = \cr
&=-\frac {2a (b+c)-2a^2} {2a (b+c)-2 (b^2+c^2)} =-\frac {2a (b+c)-2 (b^2+c^2)} {2a (b+c)-2 (b^2+c^2)} =-1.
\endalign
$$
Rovnica $\tan |\angle BFC| ={-1}$ má na intervale
$(0^\circ, 180^\circ)$ jediné riešenie $|\angle BFC| = 135^\circ$.

\poznamka
Aj v~tomto riešení sme našli odpoveď bez toho, aby sme
postrehli, že $F$ je stredom kružnice opísanej trojuholníku $ADE$. Nepotrebovali sme ani
to, že $F$ je stredom kružnice vpísanej trojuholníku~$ABC$. Ak by sme
však túto hypotézu mali, vedeli by sme ju dokázať aj výpočtom. Stačí
totiž dokázať, že $\delta = \frac12\beta$. Pritom podľa vzorca pre tangens
polovičného argumentu platí
$$
\tan \frac \beta2 = \sqrt {\frac {1- \cos \beta} {1+ \cos\beta}}
= \sqrt {\frac {1- \frc {c}{a}} {1+ \frc {c}{a}}} = \sqrt {\frac {a-c}{a+c}}.
$$
Ekvivalentnými úpravami ľahko overíme, že $\tan \delta = \tan \frac12\beta$.
A~keďže funkcia tangens je na intervale $(0^\circ, 90^\circ)$ prostá
(je tam rastúca), je nutne $\delta = \frac12\beta$. Analogicky
$\epsilon = \frac12\gamma$, čo už znamená, že $F$ je stredom kružnice
vpísanej trojuholníku $ABC$.

\ineres
Iným spôsobom ukážeme, že bod~$F$ zo zadania úlohy je zároveň
stredom~$I$ kružnice vpísanej trojuholníku $ABC$, ktorej polomer
označíme~$\rho$. Keďže stred~$I$ leží na osiach súmerností
oboch rovnoramenných trojuholníkov $BAE$ a~$CAD$, platí $|IE|=|IA|=|ID|$,
pritom vďaka pravému uhlu $BAC$ je zrejme $|IA|=\rho\sqrt2$. Bod~$I$
ležiaci vo vzdialenosti $\rho$ od priamky~$BC$ tak má od dvoch jej
rôznych bodov $D$ a~$E$ tú istú vzdialenosť~$\rho\sqrt2$, a~preto jeho
kolmý priemet na~$BC$ je podľa Pytagorovej vety stredom základne~$DE$ pravouhlého rovnoramenného trojuholníka $DEI$. Je teda $I=F$, ako
sme sľúbili dokázať.

\nobreak\medskip
\petit
\BeginSchema

Za úplné riešenie dajte 6 bodov.

\Podnadpis Všeobecné poznámky:

\Item V~prípade riešenia používajúceho analytickú geometriu
dajte 6~bodov, ak je správne, a~0~bodov v~prípade, keď je vadné alebo
nedokončené.
\Item Za hypotézu, že $F$ je stredom kružnice vpísanej $\triangle ABC$, dajte 1~bod.
\Item Za hypotézu o~tetivovosti štvoruholníkov $AFDB$ a~$AFEC$ však žiadny bod neudeľujte.
\Item Za hypotézu o~výsledku neudeľujte žiadny bod.
\Item Za absenciu zmienky o~poradí bodov $D$ a~$E$ na prepone~$BC$ body nestrhávajte.
\Item Čiastočné body z~{\it rôznych\/} postupov sa nesčítajú.

\Podnadpis Prvé riešenie:

\Body2 Zistenie, že $|\angle DAE| = 45^\circ$, s~dôkazom.
\Body1 Dôkaz, že $F$ je stredom kružnice opísanej trojuholníku $ADE$.
\Body2 Ďalšie pozorovania umožňujúce vypočítať $|\angle BFC|$, ako napríklad zdôvodnenie,
že $BF$ a~$CF$ sú osi vnútorných uhlov $ABC$ ako v~riešení, alebo dôkaz,
že štvoruholníky $AFDB$ a~$AFEC$ sú tetivové, ako v~poznámke.
\Body1 Samotný výpočet vedúci k~správnemu výsledku $|\angle BFC| = 135^\circ$.

Neúplné riešenie: Za hypotézu, že $F$ je stredom
kružnice~$ADE$, dajte 1~bod. Tento bod možno pripočítať k~všeobecne
udeľovaným bodom za hypotézu, že $F$ je stredom kružnice
vpísanej $\triangle ABC$.

\Podnadpis Riešenia používajúce bod~$M$ všeobecne (druhé a~tretie riešenie):

\Body1 Definovanie bodu~$M$.
\Body1 Vyjadrenie $|MB|$ (alebo $|MC|$) iba pomocou strán $\triangle ABC$.
\Body1 Vyjadrenie $|MF|$ iba pomocou strán $\triangle ABC$.
\Body3 Dokončenie riešenia (rozobrané nižšie).

\Podnadpis Ak riešiteľ dokazuje, že $F$ je stredom kružnice
vpísanej $\triangle ABC$:

\Body2 Dôkaz, že $F$ je stredom kružnice vpísanej
$\triangle ABC$, buď pomocou odvolania sa na známe tvrdenie ako
v~druhom riešení, alebo výpočtom ako v~poznámke k~tretiemu riešeniu.
\Body1 Výpočet $|\angle BFC| = 135^\circ$.

Neúplné riešenie: Za nedokončený výpočtový dôkaz toho, že
$F$ je stredom vpísanej kružnice $\triangle ABC$, neudeľujte žiadny bod navyše
(k~bodom súvisiacim s~definíciou bodu~$M$).

\Podnadpis Ak riešiteľ priamo počíta $\tan |\angle BFC|$:

\Body2 Samotný výpočet $\tan |\angle BFC| =-1$.
\Body1 Vyriešenie rovnice $\tan |\angle BFC| =-1$.

Neúplné riešenie: Za nedokončený výpočtový dôkaz
$\tan |\angle BFC| ={-1}$ neudeľujte žiadny bod navyše (k~bodom súvisiacim
s~definíciou bodu~$M$).

\Podnadpis Štvrté riešenie:

\Body1 Rozhodnutie uvažovať trojuholník $DEI$, pričom $I$ je stred kružnice vpísanej
trojuholníku~$ABC$. Jedná sa o~bod udeľovaný všeobecne za hypotézu, že $F=I$.
\Body1 Dôkaz, že $I$ je stredom kružnice opísanej $\triangle EAD$.
\Body1 Dôkaz, že $|ID|=|IE|=\rho\sqrt2$.
\Body2 Zdôvodnenie, že $DEI$ je rovnoramenný pravouhlý trojuholník, odkiaľ $I=F$.
\Body1 Výpočet $|\uhel BFC|=135\st$.

\EndSchema

\endpetit
\bigbreak
}

{%%%%%   A-II-4
Vzhľadom na symetriu zadania môžeme predpokladať, že
$a\le b \le c$. Dokážeme, že platí odhad $a^2 \leq b^2$. Ten je totiž
ekvivalentný s~$(b-a)(b+a) \ge 0$, čo platí vďaka $b \ge a$ a~$b+a~\ge
0$. Podobne platí odhad $c^2 \leq (1-b)^2$, keďže ten je ekvivalentný
s~$(1-b-c) (1-b+c) \ge 0$, pričom $1-b-c \ge 0$ a~$1-b+c>-b+c \ge 0$.
Použitím oboch odhadov máme
$$
a^2+b^2+c^2 \leq b^2+b^2+(1-b)^2. \eqno {(1)}
$$
Ďalej platí $b+b \ge a+b \ge 0$ a~$b+b \leq b+c \leq 1$, takže $b \in
\langle 0, \frac12 \rangle$. Ostáva nájsť maximum pravej strany~(1) na
tomto intervale.

Pre $b = 0$ platí $b^2+b^2+(1-b)^2 = 1$. Dokážeme, že 1 je hľadané maximum.
Na to je nutné dokázať nerovnosť $b^2+b^2+(1-b)^2 \leq 1$. Tá platí práve
vtedy, keď ${b (3b-2)} \le 0$, čo je splnené pre všetky $b \in \langle 0,
\frac23 \rangle$, a~teda aj pre všetky $b \in \langle 0, \frac12\rangle$.
Keďže pre $(a, b, c) = (0,0,1)$ platí $a^2+b^2+c^2 = 1$,
je hľadané maximum naozaj rovné~1.

\poznamka
Výraz $b^2+b^2+(1-b)^2$ sme na intervale $\langle 0,\frac12\rangle$
mohli maximalizovať aj takto: Keďže sa jedná o~kvadratickú funkciu
v~premennej~$b$, ktorej grafom je parabola otvorená nahor, môže svoje
maximum nadobúdať iba v~krajných bodoch intervalu $\langle 0,\frac12\rangle$.
Stačí teda preveriť obe tieto hodnoty.

\ineres
Zaveďme substitúciu $a+b = x$, $b+c = y$ a~$c+a~= z$.
Čísla $x$, $y$, $z$ potom ležia v~intervale $\langle 0, 1 \rangle$ a~platí
$a= \frac12(x-y+z)$, $b = \frac12(y-z+x)$, $c = \frac12(z-x+y)$. Hodnota $a^2+b^2+c^2$ je
po úprave rovná
$$
\frac {1} {4} (3x^2+3y^2+3z^2-2xy-2yz-2zx). \eqno {(2)}
$$

Pozrime sa na výraz (2) ako na kvadratickú funkciu premennej~$x$.
Vieme, že $x \in \langle 0,1 \rangle$. Koeficient pri~$x^2$ je kladný,
takže grafom tejto funkcie je parabola otvorená nahor. Táto funkcia
preto nadobúda maximum jedine v~niektorom z~krajných bodov intervalu
$\langle 0,1 \rangle$ (je možné, že v~oboch). Rovnakú úvahu však
môžeme použiť aj pre premenné $y$ a~$z$.
To znamená, že pre ľubovoľné $x,y,z\in\<0,1\>$ symetrický výraz
$V=V(x,y,z)$ z~(2) spĺňa nerovnosti
$$
\let\{=( \let\}=)
\align
V(x,y,z)\le\max\{&V(0,y,z),V(1,y,z)\}\le\\
\le\max\{&V(0,0,z),V(0,1,z),V(1,0,z),V(1,1,z)\}\le\\
\le\max\{&V(0,0,0),V(0,0,1),V(0,1,0),V(0,1,1),\\
&V(1,0,0),V(1,0,1),V(1,1,0),V(1,1,1)\}=\\
=\max\{&V(0,0,0),V(0,0,1),V(0,1,1),V(1,1,1)\}=\max\{0,\tfrac34,1,\tfrac34\}=1.
\endalign
$$

Vidíme, že výraz nadobúda maximálnu hodnotu~1
práve vtedy, keď sú práve dve z~premenných $x$, $y$, $z$ rovné~1
a~tretia je rovná~0. To zodpovedá tomu, že práve dve z~premenných $a$, $b$, $c$
sú rovné~0 a~tretia je rovná~1.

\ineres
Po zavedení substitúcie ako v~predchádzajúcom riešení teraz iným
spôsobom ukážeme, že hodnota výrazu~(2) je nanajvýš~1, nech sú čísla
$x$, $y$, $z$ z~intervalu $\<0,1\>$ akékoľvek.

Bez ujmy na všeobecnosti budeme predpokladať, že platí $z=\max(x,y,z)$,
a~dotyčnú nerovnosť zbavenú zlomku
$$
3x^2+3y^2+3z^2-2xy-2yz-2zx\le4
$$
upravíme na tvar
$$
2x(x-z)+2y(y-z)+(x-y)^2+3z^2\le4.
$$
Túto nerovnosť však získame sčítaním štyroch nerovností
$$
2x(x-z)\le0,\quad 2y(y-z)\le0,\quad (x-y)^2\le1,\quad 3z^2\le3,
$$
ktorých platnosť je okamžitým dôsledkom nerovností
$$
0\le x\le z,\quad 0\le y\le z,\quad -1\le x-y\le 1, \quad 0\le z~\le1.
$$

\poznamka
Toto riešenie možno zapísať aj bez premenných $x$, $y$, $z$.
Jednotlivé nerovnosti totiž zodpovedajú vzťahom
$2(a+b)(b-c)\le0$,
$2(b+c)(b-a)\le0$,
$(a-c)^2 \le1$,
$3(a+c)^2 \le3$,
ktoré platia za predpokladu $b=\min(a,b,c)$ (tretí vďaka tomu, že ${-1}\le (a+b)-(b+c)\le 1$).
Súčet ľavých strán týchto nerovností je pritom $4(a^2+b^2+c^2)$.

\nobreak\medskip\petit
\BeginSchema

Za úplné riešenie dajte 6 bodov.

\Podnadpis Všeobecné poznámky:

\Item Za uhádnutie maxima dajte 1 bod práve vtedy, keď je
uvedená aj trojica $(a, b, c)$, pre ktorú sa táto hodnota nadobúda.
\Item Riešenia, ktoré sa opierajú o~úvahy typu \uv{ak
nejakú premennú zväčšíme, tak sa výraz zväčší}, hodnoťte ako prvé
riešenie po prevode týchto úvah na nerovnosti.
\Item Čiastočné body z~{\it rôznych\/} postupov sa nesčítajú.

\Podnadpis 1. riešenie:

\Body1 Dôkaz nerovnosti $a^2 \le b^2$.
\Body1 Dôkaz nerovnosti $c^2 \le (1-b)^2$.
\Body1 Dôkaz $b \in \langle 0, \frac12 \rangle$.
\Body1 Odhad skúmaného výrazu $a^2+b^2+c^2$ zhora výrazom $b^2+b^2+(1-b)^2$.
\Body1 Dôkaz $b^2+b^2+(1-b)^2 \leq 1$ na intervale $\langle 0,\frac12 \rangle$.
\Body1 Maximalizácia tohto výrazu a~uvedenie aspoň jednej
trojice $(a, b, c)$, pre ktorú sa maximum nadobúda.

\Podnadpis Neúplné riešenie:

\Item Za samotný predpoklad $a~\le b \le c$ ani za z~toho vyplývajúci dôsledok $0\le b\le c$
nedávajte žiadny bod.
\Item Ak riešiteľ chybne uvedie, že $a^2 \le b^2$ vyplýva priamo z~$a\le b$, strhnite 1~bod.
\Item Ak riešiteľ chybne uvedie, že $c^2 \le (1-b)^2$ vyplýva priamo z~$c\leq 1-b$, strhnite 1~bod.
\Item Bod za odhadnutie výrazu $a^2+b^2+c^2$ zhora výrazom
$b^2+b^2+(1-b)^2$ dajte ako čiastočný aj v~prípade, keď nie sú
potrebné nerovnosti $a^2 \leq b^2$ a~$c^2 \le (1-b)^2$ dokázané.
\Item Tiež strhnite bod za algebrické chyby pri maximalizácii výrazu $b^2+b^2+(1-b)^2$.
\Item Ak riešiteľ rozoberie iba prípad, keď sú čísla $a$, $b$ a~$c$
všetky nezáporné (jedno z~nich totiž môže byť záporné!), dajte nanajvýš 3~body.

\Podnadpis Riešenie používajúce substitúciu $a+b = x$, $b+c = y$, $c+a~= z$:

\Body1 Samotná Substitúcia.
\Body1 Prevedenie skúmaného výrazu do nových premenných $x$, $y$, $z$.
\Body4 Dokončenie riešenia (rozobrané nižšie).

\Podnadpis Neúplné riešenie:

\Item Za chybu pri prevode do premenných $x$, $y$, $z$ strhnite 1~bod.
\Item Za ďalšie chyby pri algebrických úpravách strhnite 1 až 2 dva body, podľa počtu a~závažnosti chýb.

\Podnadpis Dokončenie riešenia so substitúciou v~štýle druhého riešenia:

\Body1 Zdôvodnenie, že maximum kvadratickej funkcie
definované na $\langle 0,1 \rangle$ s~kladným koeficientom
pri kvadratickom člene sa nadobúda iba v~krajnom bode tohto intervalu.
\Body2 Použitie tohto tvrdenia pre každú z~premenných $x$, $y$, $z$ a~preskúmanie možných kombinácií.
\Body1 Nájdenie maximálnej hodnoty spolu s~trojicou $(a, b, c)$, pre ktorú sa nadobúda.

\Podnadpis Dokončenie riešenia so substitúciou v~štýle tretieho riešenia:

\Body3 Uhádnutie maximálnej hodnoty a~dôkaz potrebnej nerovnosti.
\Body1 Uvedenie aspoň jednej trojice $(a, b, c)$, pre ktorú sa maximálna hodnota nadobúda.

\EndSchema


\endpetit
\bigbreak
}

{%%%%%   A-III-1
Daná sústava rovníc je symetrická, takže stačí
hľadať iba tie riešenia, ktoré spĺňajú nerovnosti $x \ge y \ge z$. Za tohto
predpokladu môžeme odstrániť absolútne hodnoty, a~tak dostaneme
$$
\abovedisplayskip\abovedisplayshortskip
\eqalignno
{
x^2-yz &= y-z+1, &(1)\cr
y^2-zx &= x-z+1, &(2)\cr
z^2-xy &= x-y+1. &(3)
}
$$
Postupným odčítaním rovníc (1) a~(2), resp. (2) a~(3) dostávame
po jednoduchých úpravách rovnice
$$
\eqalign
{
(x-y)(x+y+z+1) &= 0,\cr
(y-z)(x+y+z-1) &= 0.\cr
}
$$
Z~toho vyplýva, že všetky tri čísla $x$, $y$, $z$ nemôžu byť navzájom rôzne,
nemôžu však byť ani všetky rovnaké, keďže to by sme
v~pôvodných rovniciach dostali $0 = 1$. Práve dve z~nich
sú teda rôzne, takže platí buď $x = y > z$ a~$x+y+z~= 1$, alebo
$x>y=z$ a~$x+y+z~=-1$.

Všimnime si, že trojica $(x, y, z)$ vyhovuje pôvodnej sústave práve
vtedy, keď jej vyhovuje \uv{opačná} trojica $({-z},{-y},{-x})$.
Prechod k~opačnej trojici nemení zavedené usporiadanie čísel
v~trojici a~prevádza druhý prípad z~predchádzajúceho odseku na prvý z~nich.

Stačí preto vyriešiť prvý prípad, keď $x=y>z$ a~$x+y+z~= 1$. To dáva
$z~= {1-2x}$. Dosadením napríklad do rovnice~(1) dostaneme po úprave
$x (3x-4) = 0$, takže $x = 0$ alebo $x = \frac43$. Tomu zodpovedajú trojice
$(x, y, z)$ rovné $(0,0,1)$ a~$\left (\frac43, \frac43,-\frac53 \right)$.
Prvá trojica zrejme nespĺňa predpoklad $x \ge y \ge z$, a~tak
pôvodnej sústave vyhovuje iba druhá trojica (skúška pri uvedenom postupe nie je nutná).

S~prihliadnutím na zmeny poradia neznámych a~prechody na opačné trojice
má sústava 6~riešení:
$$
\displaylines{%
\Bigl(\frac43,\frac43,-\frac53 \Bigr),\
\Bigl(-\frac53,\frac43,\frac43 \Bigr),\
\Bigl(\frac43,-\frac53,\frac43 \Bigr),\cr
\Bigl(\frac53,-\frac43,-\frac43\Bigr),\
\Bigl(-\frac43,-\frac43,\frac53\Bigr),\
\Bigl(-\frac43,\frac53,-\frac43\Bigr).
}
$$}

{%%%%%   A-III-2
Všimnime si najskôr, že sa ani zadanie úlohy bez podmienky $a\ge b$,
ani množina dvojíc bodov $(P,Q)$, ktoré sú jej riešeniami,
nezmení, keď navzájom vymeníme označenie vrcholov $B$ a~$D$.
Podmienku $a\ge b$ preto uplatníme až pre jednoduchší zápis
záverečnej diskusie o~počte riešení.

Označme $A'$ a~$C'$ kolmé priemety bodov $A$ a~$C$
na priamku~$BD$. Zrejme oba priemety padnú dovnútra úsečky~$BD$
a~platí $|AA'| = |CC'|$.
Predpokladajme, že body $P$ a~$Q$ majú požadované vlastnosti.
Keďže $|AP| = |CQ|$, je
$P = A'$ práve vtedy, keď $Q = C'$.
Ak nastane táto situácia, budú body $P$, $Q$
(ako body $A'$, $C'$) súmerne združené
podľa stredu~$S$ úsečky~$BD$. To isté bude
o~bodoch $P$, $Q$ platiť aj v~prípade, keď $A'=C'$ $(=S)$~-- vtedy
sú pravouhlé trojuholníky $APS$, $CQS$
zhodné podľa vety $Ssu$, takže $|PS|=|QS|$, pričom nemôže byť $P=Q$.

Tieto "symetrické" prípady $P=A'$, $Q=C'$, $A'=C'$ teda
z~ďalšieho {\it rozboru\/} vylúčime.
Práve tak vylúčime aj možnosť, že by bod~$Q$ ležal na úsečke~$A'P$,
keďže by potom podľa \obr{} platilo $|AP|>|A'P|\ge|PQ|$, čo odporuje požiadavkám úlohy.
A~symetricky ani bod~$P$ nemôže ležať na úsečke~$QC'$.
V~rozbore opakovane využijeme rovnosť
$|A'P|=|C'Q|$, ktorá vyplýva z~trojuholníkov $A'PA$ a~$C'QC$, ktoré sa zhodujú
podľa vety~$Ssu$.%\vadjust{\vskip-.9\baselineskip}
\vadjust{\vskip-\baselineskip}
\insp{a68iii.1}%

Za predpokladov $P\ne A'$, $Q\ne C'$, $A'\ne C'$, $Q\notin A'P$ a~$P\notin C'Q$
rozlíšime možné polohy bodov $P$, $Q$ na priamke~$A'C'$ nasledovne:

\item {1.} $P$ leží na polpriamke opačnej k~polpriamke~$A'C'$:


\itemitem {a)} $Q$ leží na úsečke~$A'C'$
(\obr). Potom z~$|A'P| = |C'Q|$ máme $|PQ| = |A'C'|$,
takže $|AP| = |A'C'|$.

\itemitem {b)} $Q$ leží na polpriamke opačnej k~polpriamke~$C'P$
(\obr). Vtedy z~$|A'P| = |C'Q|$ máme, že body $P$, $Q$ sú
súmerne združené podľa stredu~$S$ úsečky~$A'C'$, čiže stredu
uhlopriečky~$BD$.%\vadjust{\vskip-.5\baselineskip}
\vadjust{\vskip-\baselineskip}
\inspinsp{a68iii.3}{a68iii.4}%

\item {2.} $P$ leží na úsečke $A'C'$:

\nobreak

\itemitem {a)} $Q$ leží na úsečke~$C'P$
(\obr). Vtedy podobne ako v~prípade~1b máme, že body $P$, $Q$
sú súmerne združené podľa stredu~$S$ uhlopriečky~$BD$.

\itemitem {b)} $Q$ leží na polpriamke opačnej k~polpriamke~$C'P$
(\obr). Vtedy podobne ako v~prípade~1a máme $|AP| = |A'C'|$.
%\goodbreak
\vskip-.4\baselineskip
~\inspinspinsp{a68iii.7}{a68iii.8}{a68iii.9}{\qquad}%

\item {3.} $P$ leží na polpriamke opačnej k~polpriamke~$C'A'$:
\hfil\break
Bod~$Q$ potom musí ležať na~polpriamke opačnej k~polpriamke~$A'C'$
(\obr). Podobne ako v~prípade~1b tak máme, že body $P$, $Q$
sú súmerne združené podľa stredu~$S$ uhlopriečky~$BD$.


Dokázali sme, že buď sú body~$P$, $Q$ súmerne združené podľa stredu~$S$,
alebo platí $|AP| = |A'C'|$. Opíšeme najskôr {\it konštrukciu\/} v~prvom prípade,
ktorý zahŕňa aj situácie, keď $P=A'$, $Q=C'$ alebo $A'=C'$, ktoré sme na úvod z~nášho
rozboru vylúčili.
Vtedy platí $|AP| = |PQ| = 2 |PS|$. Naopak, ak niektorý bod~$P$ priamky~$BD$
bude spĺňať rovnosť $|AP|=2|PS|$,
tak k~nemu jednoznačne zostrojíme bod~$Q$ súmerne združený
podľa stredu~$S$, pričom bude platiť $|AP| = |PQ| = |QC|$. Zameriame sa teda
na konštrukciu bodu~$P$.

Označme $D'$ priesečník priamky~$AC$ a~rovnobežky s~$AP$ bodom~$D$.
Trojuholníky $SPA$ a~$SDD'$ sú rovnoľahlé so stredom~$S$. Keďže
$|AP| = 2 |PS|$, je $|DD'| = 2 |DS|$. Z~toho už vyplýva konštrukcia,
v~ktorej najskôr zostrojíme bod~$D'$. Také body sú vždy dva, pričom
jeden leží na polpriamke~$SA$ (\obr) a~druhý
na polpriamke~$SC$ (\obr). Následne zostrojíme bod~$P$ ako
priesečník rovnobežky bodom~$A$ s~priamkou~$DD'$. Keďže máme dve
možné polohy bodu~$D'$, budeme mať dve možné polohy bodu~$P$,
a~tak v~tomto prípade budú existovať vždy práve dve riešenia.
\vadjust{\vskip-\baselineskip}
\inspinsp{a68iii.13}{a68iii.14}%

V~druhom prípade $|AP|=|A'C'|$ môžeme už predpokladať, že $A'\ne C'$.
Podľa rovnosti $|AP|=|A'C'|$ ľahko nájdeme bod~$P$ ako priesečník
priamky~$BD$ s~kružnicou~$k(A, |A'C'|)$. V~prípade $|A'C'|> |AA'|$ dostaneme
dva priesečníky, čiže dve riešenia, v~prípade $|A'C'| = |AA'|$ jedno riešenie
a~napokon žiadne riešenie v~prípade $|A'C'|<|AA'|$. Možné riešenia zodpovedajú
prípadu~1a (\obr), resp.~2b (\obr).
V~oboch z~nich je poloha bodu~$Q$ určená jednoznačne a~ľahko spätne
ukážeme, že platí $|AP| = |PQ| = |QC|$.
\vadjust{\vskip-\baselineskip}
\inspinsp{a68iii.15}{a68iii.16}%

Posúdime teraz, kedy niektoré riešenia z~prvého a~druhého prípadu
splývajú. Ak majú body $P$, $Q$ súmerne združené podľa stredu~$S$
navyše spĺňať aj podmienku $|PQ|=|AP|=|A'C'|$, stane sa tak jedine
v~situácii, keď $\{P,Q\}=\{A',C'\}$. Tá je zrejme možná, iba keď
$P=A'$ a~$Q=C'$, čo vedie k~rovnosti $|A'C'|=|AA'|$. Tá však
znamená práve to, že kružnica~$k$ sa dotýka priamky~$BD$ v~bode~$A'$, a~preto (jediné) riešenie z~druhého prípadu splýva s~tým
z~dvoch riešení prvého prípadu, pri ktorom $P=A'$. Úloha potom
má celkom dve riešenia (\obr\ a~\obrnum).
\inspinsp{a68iii.17}{a68iii.18}%

Zistili sme, že počet riešení z~druhého prípadu, rovnako ako
splynutie jediného jeho riešenia s~jedným riešením prvého prípadu,
závisí od pomeru $|A'C'|:|AA'|$, ktorý stačí vyjadriť pomocou
pomeru $p=a:b$ iba v~prípade, keď $a\ge b$, čiže $p\ge 1$.
Podľa Euklidovej vety o~odvesne platí $|DA'|=b^2/|BD|\le a^2/|BD|=|DC'|$,
odkiaľ $|A'C'| = (a^2-b^2) / |BD|$.
Ďalej $|AA'| = 2S_{ABD} / |BD| = ab / |BD|$, takže
$$
|A'C'|: |AA'| = \frac {a^2-b^2} {ab} = p- \frac {1} {p}.
$$
Pritom $p- \frc {1} {p} \ge 1$ práve vtedy, keď $p \ge\frac1{2} (1+ \sqrt {5})$,
čo je hodnota takzvaného {\it zlatého rezu\/} označovaného~$\phi$.
Výsledky možno zhrnúť takto:
$$
\frac ab> \phi\:\ \hbox {4 riešenia,} \qquad
1\le\frac ab\le \phi\:\ \hbox {2 riešenia.}
$$

\poznamkac 1.
Namiesto skúmania dĺžok na priamke~$A'C'$ sme
mohli skúmať uhly. Vďaka zhodnosti trojuholníkov $APA'$, $ CQC'$
totiž platí $|\uhel APQ| = |\uhel CQP|$ alebo $|\uhel APQ|+|\uhel CQP|=180\st$.
Pomocou toho sa dá vyšetriť, že v~prípadoch 1b, 2a, 3 tvoria body $A$, $C$, $P$, $Q$ rovnobežník
s~uhlopriečkami $AC$, $PQ$, takže $P$, $Q$ sú súmerne združené
podľa stredu~$S$. Ďalej v~prípadoch 1a~(\obr) a~2b~(\obr)
môžeme definovať~$A''$ ako obraz bodu~$A$
v~osovej súmernosti podľa priamky~$BD$ a~dokázať, že body $A''$, $P$, $Q$, $C$ tvoria
kosoštvorec, v~ktorom $|A''P| = |A''C|$, čo už dáva návod, ako
zostrojiť bod~$P$.
\vadjust{\vskip-1.5\baselineskip}
\inspinsp{a68iii.19}{a68iii.20}%
\vadjust{\vskip-\baselineskip}

\poznamkac 2.
Konštrukciu v~symetrickom prípade sme mohli spraviť
aj pomocou {\it Apollóniovej kružnice}. Hľadáme množinu bodov~$P$
takých, že $|AP| = 2 |PS|$. Zostrojíme teda bod~$R$ na úsečke~$AS$
taký, že $|AR| = 2 |RS|$. Ďalej platí $|AC| = 2 |SC|$. Je známe, že
hľadaná množina bodov je kružnica nad priemerom~$RC$. Táto
kružnica pretína polpriamku~$SD$ v~jednom bode (\obr)
a~polpriamku~$SB$ v~druhom bode (\obr).
\vadjust{\vskip-.5\baselineskip}
\inspinsp{a68iii.21}{a68iii.22}%
}

{%%%%%   A-III-3
Dokážeme, že čísla $A~= (a+b+c) (ab+bc+ca)$
a~$B = (a+b) (b+c) (c+a)$ sú nesúdeliteľné. Predpokladajme, že to tak
nie je. Potom existuje prvočíslo~$p$, ktoré delí obe čísla $A$, $B$.
Keďže~$p \mid B$, tak~$p$ delí aspoň jedno z~čísel~$a+b$, $b+c$,
$c+a$. Bez ujmy na všeobecnosti nech je to $a+b$. Potom ale nemôže
platiť $p \mid a+b+c$, lebo by bolo $p \mid c$, čo je v~spore s~predpokladom~(i).
Nutne teda $p \mid ab+bc+ca = ab+c (a+b)$, z~čoho vyplýva $p \mid ab$, takže $p$ delí
aspoň jedno z~čísel $a$, $b$, čo spolu s~reláciou $p \mid a+b$ znamená, že
delí obe čísla $a$, $b$, čo je opäť v~spore s~(i).

Čísla $A$, $B$ sú teda naozaj nesúdeliteľné
a~navyše ich súčin~$AB$ je podľa predpokladu~(ii)
$n$-tá mocnina celého čísla. Preto musí aj každé z~čísel $A$, $B$ byť $n$-tou
mocninou celého čísla. Lenže $abc = A-B$, takže je to naozaj rozdiel
dvoch $n$-tých mocnín celých čísel. Tým je tvrdenie úlohy dokázané.

\poznamka
Trojica $(a, b, c) = (341, 447, 1235)$ vyhovuje zadaniu pre $n = 2$.
}

{%%%%%   A-III-4
Keďže body $A$, $D$, $E$, $J$ ležia na kružnici s~priemerom~$AJ$, stačí
na dôkaz kolmosti $AF \perp FJ$ overiť, že
na tejto kružnici leží aj bod~$F$ (\obr). Nech $G$ je dotykový bod
uvažovanej pripísanej kružnice so stranou~$BC$. Potom z~rovností $|AB| = |BP|$
a~$|BD| = |BG|$
vyplýva zhodnosť trojuholníkov $ABG\cong PBD$ ($sus$) a~podobne z~rovností $|AC| = |CQ|$
a~$|CE| = |CG|$ zhodnosť $ACG\cong QCE$.
Postupne tak dostávame
$|\uhel AEF| = 180\st-|\uhel CEQ|= 180\st-|\uhel CGA|
=|\uhel BGA|=|\uhel BDP|=|\uhel ADP|$.
Odtiaľ $|\uhel AEF|+|\uhel ADF|=180\st$, čo už znamená, že bod~$F$ leží
s~bodmi $A$, $E$, $D$ na jednej kružnici, ako sme potrebovali dokázať.
\insp{a68iii.23}%

\poznamka
Potrebný záver, že štvoruholník $ADFE$ je tetivový, možno odvodiť, ako teraz ukážeme,
aj pomocou druhej dvojice jeho protiľahlých vnútorných uhlov pri vrcholoch $A$ a~$F$.
Zo zhodnosti trojuholníkov $ABG\cong PBD$ a~$ACG\cong QCE$
vyplýva, že $|\uhel DAE|=|\uhel BAC|=|\uhel BAG|+|\uhel GAC|=
|\uhel BPD|+|\uhel CQE| =|\uhel QPF|+|\uhel PQF|= {180^\circ- |\uhel PFQ|}
=180\st-|\uhel DFE|$, čiže $|\uhel DAE|+|\uhel DFE|=180\st$.
}

{%%%%%   A-III-5
Dokážeme, že daný výraz po delení~12 nikdy nedáva zvyšok~7.
Čísla $2^a$, $3^b$ a~$-5^c$ dávajú po delení~12
postupne zvyšky z~množín $\{1,2,4,8\}$, $\{1,3,9\}$, a~$\{{-1},{-5}\}$.
Pre každý súčet~$s$ troch čísel z~týchto množín platí $1+1-5
\le s~\le 8+9-1$, teda ${-3} \le s~\le 16$. Jediná možná
hodnota~$s$ dávajúca zvyšok~7 po delení~12 je teda $s~= 7$. Ak ale
z~tretej množiny vyberieme~${-1}$, musíme z~prvých dvoch vybrať
čísla so súčtom~8, čo sa zjavne nedá. Podobne dopadneme, keď
z~tretej množiny vyberieme~${-5}$. Číslo~7 teda nikdy nevyjadríme ako
súčet troch čísel po jednom z~každej z~týchto troch množín,
čo dokazuje, že skúmaný výraz
nemôže byť rovný číslu tvaru $12k+7$, pričom~$k$ je celé číslo.
A~takých čísel je nekonečne veľa.

\poznamka
Dá sa overiť, že každý iný zvyšok po delení~12
skúmaný výraz nadobúdať môže. Tiež platí, že pre každé
prirodzené $n <12$ skúmaný výraz môže nadobúdať všetky možné
zvyšky po delení~$n$. Naopak najmenšie $n> 12$, pre ktoré sa
nejaký zvyšok znova nenadobúda, je $n=20$, pričom vtedy sú
nedosiahnuteľné dokonca tri zvyšky 11, 13~a~15.

\ineres
Skúmajme zvyšky po delení~20. Dokážeme, že niektorý
nepárny zvyšok sa nedá získať. Čísla $3^b$ a~$-5^c$
dávajú po delení~20 zvyšky z~množín $\{{1, 3, 7, 9}\}$
a~$\{{-1},{-5}\}$. Súčet dvoch čísel po jednom z~týchto dvoch množín nadobúda
nanajvýš~$4 \cdot 2 = 8$ hodnôt, napospol párnych. Číslo~$2^a$ dáva
nepárny zvyšok iba pre $a= 0$, preto pre celý výraz ${2^a+3^b-5^c}$
máme nanajvýš 8~možných nepárnych zvyškov, takže po delení~20
sa nedajú dostať najmenej dva nepárne zvyšky.

\poznamka
Také zvyšky sú dokonca tri, a~síce 11, 13 a~15. Na druhej
strane sa dá overiť, že všetky párne zvyšky sú dosiahnuteľné.
(Nedosiahnuteľné zvyšky pre $n\le30$ ukazuje nasledujúca schéma.)
\inspno{a68iii.0}%
}

{%%%%%   A-III-6
Potrebujeme dosiahnuť to, že súčet čísel v~každom riadku
aj stĺpci bude deliteľný~$n$. Obmedzíme sa preto na vypĺňanie
tabuľky číslami $0,1, \dots, n-1$, pričom každé z~nich bude použité
práve $n$-krát (lebo tak dostaneme zvyšky všetkých čísel od~$1$ do~$n^2$).
Rozlíšime nasledujúce prípady:

\item{$\triangleright$}
$n$ je nepárne číslo.
Uvažujme tabuľku:
\CentrovanaTabulka {|ccccc|}
{
\crl
0&1&2&\dots&$n-1$ \cr
0&1&2&\dots&$n-1$ \cr
\vdots&\vdots&\vdots&\ddots&\vdots \cr
0&1&2&\dots&$n-1$ \crl
}

V~každom stĺpci máme $n$ rovnakých čísel, takže ich súčet je
deliteľný~$n$. Súčty v~riadkoch sú rovné $0+1+\dots+(n-1) =
\frac1{2} n(n-1)$, čo je tiež násobok~$n$, lebo $n$ je nepárne.

\item{$\triangleright$}
$n = 4k$ pre nejaké prirodzené~$k$. V~takom prípade rozdeľme
tabuľku $4k \times 4k$ na~$4k^2$ štvorčekov $2\times 2$ a~uvažujme
nasledujúci vzor:
\CentrovanaTabulka {|cc|}
{
\crl
$x$&$4k-x$ \cr
$4k-x$&$x$ \crl
}

Takých štvorčekov potrebujeme umiestniť práve~$2k$ pre každé
$x = 1,2, \dots, 2k-1$ a~práve~$k$ pre $x = 2k$. Ostáva
umiestniť čísla~0, ktoré dáme do zvyšných $k$~štvorčekov~${2 \times 2}$. Také
vyplnenie tabuľky bude zrejme vyhovovať zadaniu, keďže v~každom
takomto štvorčeku je súčet čísel v~riadkoch aj stĺpcoch
deliteľný~$4k$ a~ich rozmiestnenie v~celej tabuľke túto deliteľnosť
neovplyvní.

\item{$\triangleright$}
$n = 4k+2$ pre nejaké nezáporné celé~$k$. Pre $k~= 0$ (teda
$n = 2$) ľahko zistíme, že požadované vyplnenie neexistuje. Nech
$k\ge 1$. Vtedy vyplňme podtabuľku $4\times 4$ v~ľavom hornom rohu
nasledovne:
\tskip.7em
\CentrovanaTabulka {|cccc|}
{
\crl
1&$4k+1$&0&0 \cr
$2k$&$2k+2$&0&0 \cr
$2k+1$&0& $2k$&1 \cr
0&$2k+1$&$2k+2$&$4k+1$ \crl
}

Vidíme, že v~každom riadku aj stĺpci je zatiaľ súčet čísel deliteľný
$4k+2$. Zvyšok celej tabuľky sa pritom dá rozdeliť
na~štvorčeky~$2 \times 2$ a~podobne ako v~predošlom prípade
vyplniť pomocou tohto vzoru:
\CentrovanaTabulka {|cc|}
{
\crl
$x$&$4k+2-x$ \cr
$4k+2-x$&$x$ \crl
}

Takých štvorčekov potrebujeme práve $2k$ pre $x = 1$ a~$x = 2k$,
práve~$k$ pre $x = {2k+1}$ a~práve~$2k+1$ pre každé $x = 2, \dots, 2k-1$.
Týmto vyplnením zrejme zabezpečíme, že v~každom riadku aj stĺpci bude
súčet čísel deliteľný~$4k+2$.
Ostáva doplniť čísla~0 do zvyšných $k-1$ štvorčekov $2\times 2$,
čím doterajšie vyhovujúce hodnoty riadkových ani stĺpcových súčtov nezmeníme.

Úlohe teda vyhovujú všetky prirodzené čísla~$n$ rôzne od~2.

\ineres
Ukážeme alternatívne riešenie pre prípady $n = 4k$
a~$n = 4k+2$, pričom $k$ je prirodzené číslo.

\item{$\triangleright$} $n = 4k$:
\CentrovanaTabulka {|ccccccccc|}
{
\crl
1&$4k-1$&2&$4k-2$&\dots&$2k-1$&$2k+1$&$2k$& $2k$ \cr
1&$4k-1$&2&$4k-2$&\dots&$2k-1$&$2k+1$&$2k$& $2k$ \cr
\vdots&\vdots&\vdots&\vdots&\ddots&\vdots&\vdots&\vdots&\vdots \cr
1&$4k-1$&2&$4k-2$&\dots&$2k-1$&$2k+1$&$2k$& $2k$ \cr
1&$4k-1$&2&$4k-2$&\dots&$2k-1$&$2k+1$&0&0 \cr
1&$4k-1$&2&$4k-2$&\dots&$2k-1$&$2k+1$&0&0 \cr
\vdots&\vdots&\vdots&\vdots&\ddots&\vdots&\vdots&\vdots&\vdots \cr
1&$4k-1$&2&$4k-2$&\dots&$2k-1$&$2k+1$&0&0 \cr
1&$4k-1$&2&$4k-2$&\dots&$2k-1$&$2k+1$&0&0
\crl
}

Vidíme, že čísla v~riadkoch sa dajú spárovať do dvojíc
so súčtom deliteľným~$4k$, takže ich celkový súčet je
deliteľný~$4k$. V~prvých $4k-2$ stĺpcoch máme po~$4k$ rovnakých
čísel, takže ich súčet je tiež násobok čísla~$4k$. Súčty
v~posledných dvoch stĺpcoch sú rovné $2k \cdot 2k = 4k^2$, čo je
tiež číslo deliteľné~$4k$.

\item{$\triangleright$} $n = 4k+2$.

Všimnime si najskôr, že ľubovoľné čísla
$a_1, a_2,\dots, a_l$, pričom každé z~nich máme k~dispozícii $l$-krát,
dokážeme umiestniť do~tabuľky~$l \times l$ tak, že súčet čísel v~každom riadku
aj stĺpci bude rovnaký:
$$
\abovedisplayskip\abovedisplayshortskip
t(a_1, a_2,\dots, a_l)=\vcenter{\table
{\crl
$a_1$&$a_2$&\dots&$a_{l-1}$&$a_l$ \cr
$a_2$&$a_3$&\dots&$a_l$&$a_1$ \cr
\vdots&\vdots&\ddots&\vdots&\vdots \cr
$a_{l-1}$&$a_l$&\dots&$a_{l-3}$&$a_{l-2}$ \cr
$a_l$&$a_1$&\dots&$a_{l-2}$&$a_{l-1}$ \crl}}
$$

Túto konštrukciu využijeme takto: Rozdeľme množinu
$\{0,1, \dots, 4k+1\} \setminus \{0,{2k+1}\}$ ľubovoľným
spôsobom na dve disjunktné skupiny~$A$, $B$ také, že $|A| = 2k-1$ (teda
$|B| = 2k+1$). Teraz vytvoríme štyri postupnosti $2k+1$ čísel
$$
0,0, A, \quad 2k+1,2k+1, A, \quad B, \quad B,
$$
pričom pre každú z~nich vytvoríme opísaným
spôsobom tabuľku $(2k+1) \times (2k+1)$. Tieto štyri tabuľky potom dáme
k~sebe do konečnej tabuľky $(4k+2)\times(4k+2)$ takto:
\CentrovanaTabulka {|cc|}
{
\crl
$t(0,0, A)$&$t(B)$ \cr
$t(B)$ &$t(2k+1,2k+1, A)$ \crl
}

V~takej tabuľke je každé z~čísel~$0,1, \dots, 4k+1$ použité
práve $(4k+2)$-krát. Ostáva overiť, že súčty vo všetkých
riadkoch a~stĺpcoch sú deliteľné číslom~$4k+2$. Súčet čísel
v~postupnostiach $A$, $B$ je dokopy rovný
$s=1+\dots+(4k+1)-(2k+1) = 2k(4k+2)$,
takže je deliteľný~$4k+2$. Jednotlivé riadky a~stĺpce výslednej
tabuľky majú súčty~$s$ (prvých $2k+1$ riadkov a~stĺpcov)
a~$s+4k+2$ (zvyšné riadky a~stĺpce), takže aj tie sú deliteľné
číslom~$4k+2$, preto vytvorená tabuľka naozaj vyhovuje zadaniu.
}

{%%%%%   B-S-1
V~jednom kroku aktuálne prirodzené číslo~$k$ zväčšíme buď na párne číslo
$m=2k$, alebo na nepárne číslo $m=2k+1$. Podľa parity nového čísla~$m$
tak môžeme rekonštruovať predchádzajúce číslo~$k$: buď $k=m/2$, alebo
$k=(m-1)/2$~-- podľa toho, či $m$ je párne alebo nepárne.

Nepárne číslo $2\,019$ sa teda na tabuli objaví jedine
po čísle $(2\,019-1)/2=1\,009$. Keďže je to opäť číslo nepárne,
po dvoch krokoch sa dostaneme k~cieľovému číslu $2\,019$
iba z~čísla $(1\,009-1)/2=504$. To je číslo párne, takže
po troch krokoch dostaneme $2\,019$ iba z~čísla $504:2=252$,
atď. Celý postup určovania všetkých vyhovujúcich čísel od konečného $2\,019$
vedie k~nasledujúcemu výsledku:
$$
2\,019\gets1\,009\gets504\gets252\gets126\gets63\gets
31\gets15\gets7\gets3\gets1.
$$
(Číslo~1 je najmenšie prirodzené číslo, takže ďalej nepokračujeme.)

\odpoved
Takých počiatočných hodnôt čísla~$n$ je desať
(sú to 1, 3, 7, 15, 31, 63, 126, 252, 504 a~1\,009).

\ineres
Ak budeme čísla na tabuli zapisovať v~dvojkovej sústave, spočíva každá
úprava aktuálneho čísla~$n$ v~tom, že ho vôbec nemusíme zotierať, stačí len
k~jeho zápisu pripísať sprava buď nulu (zmena~$n$ na~$2n$),
alebo jednotku (zmena~$n$ na~${2n+1}$). Číslo~2\,019 teda dostaneme
po určitom počte krokov práve z~takých čísel, ktorých binárny zápis
je tvorený skupinou niekoľkých prvých cifier binárneho zápisu čísla 2\,019.
Keďže $2^{10}=1\,024<2\,019<2\,048=2^{11}$, má binárny zápis čísla 2\,019
práve 11~cifier, takže existuje celkom 10~počiatočných čísel~$n$, z~ktorých
po jednom či viac krokoch dostaneme číslo~2\,019.
(Najmenšie z~týchto čísel je číslo~1, ktoré zodpovedá prvej cifre binárneho
zápisu čísla 2\,019, na ostatných deväť čísel sa zadanie úlohy nepýta.)

\poznamka
Samotný zápis čísla 2019 v~dvojkovej sústave nás nezaujíma, inak ho
zvyčajne hľadáme práve postupnosťou úprav opísanou v~prvom riešení.
Tak binárny zápis 11111100011 čísla 2\,019 dostaneme, keď v~získanej
skupine čísel $1, 3, 7,\dots, 1\,009,\allowbreak 2\,019$ zameníme každé nepárne číslo
jednotkou a~každé párne číslo nulou.


\nobreak\medskip\petit\noindent
Za úplné riešenie dajte 6~bodov, z~toho 4~body za všeobecné zdôvodnenie, že
každé predchádzajúce číslo je určené číslom nasledujúcim,
1~bod za nájdenie všetkých čísel až po jednotku
a~1~bod za formuláciu správneho záveru.

Všeobecný opis z~prvého odseku riešenia nie je nutný (rovnako ako
záverečná poznámka o~najmenšom prirodzenom čísle~1),
riešiteľ môže spätný postup rovno začať určením čísla 1\,009 z~čísla
2\,009 a~pokračovať ďalej.
Ak však pri postupných výpočtoch
čísel urobí numerickú chybu, viac ako 4~body neudeľujte.
Ak urobí viac chýb, dajte nanajvýš 3~body.
Za riešenia, ktoré našli 1009, ale nepokračovali v~hľadaní ďalších čísel, dajte 1~bod.

V~prípade druhého riešenia dajte po 1~bode za opis správania operácie $\times2$
a~za opis správania operácie $\times2 + 1$ v~dvojkovej sústave. Ďalšie
2~body za vysvetlenie, ktoré čísla zodpovedajú hľadaným, 1~bod za nájdenie
počtu cifier čísla 2019 v~dvojkovej sústave (buď odhadom alebo ručným prevodom)
a~napokon 1~bod za samotný záver.
\endpetit
\bigbreak
}

{%%%%%   B-S-2
Trojciferné číslo $n=\overline{abc}$ má požadovanú vlastnosť práve
vtedy, keď jeho cifry $a$, $b$, $c$ spĺňajú rovnicu
$$
100a+10b+c=(10a+c)b^2.
\tag1
$$
Keďže cifra~$b$ v~nej má najzložitejšie zastúpenie, rozoberieme
postupne jej možné hodnoty.

Najskôr si všimneme, že nemôže byť $b\in\{0,1\}$
(pre také cifry by pravá strana~(1) nebola
trojciferným číslom). Keďže $a\ge1$,
nemôže byť ani $b\in\{5, 6, 7, 8, 9\}$~-- pre také cifry by pravá
strana bola aspoň $250a$, zatiaľ čo ľavá strana~(1) je
vždy menšia ako~$100a+100\le200a$. Ostáva nám preto prebrať hodnoty
$b\in\{2, 3, 4\}$.

Pre $b=2$ sa zmení (1) na rovnicu $100a+20+c=40a+4c$, čiže
$20=3(c-20a)$, čo je vylúčené, lebo 20 nie je násobkom troch
(navyše $c-20a<0$ vďaka tomu, že $a\ge1$ a~$c\le9$).

Pre $b=3$ sa zmení (1) na rovnicu $100a+30+c=90a+9c$, čiže
$5(a+3)=4c$, čo znamená, že $c$ je nenulová cifra deliteľná
piatimi, teda $c=5$, a~tak $a=1$. Našli sme riešenie $n=135$ (skúška
nie je nutná).

Pre $b=4$ sa zmení (1) na rovnicu $100a+40+c=160a+16c$, čiže
$40=3(20a+5c)$, čo je vylúčené, pretože 40 nie je násobkom troch
(navyše je $3(20a+5c)\ge60$).

\odpoved
Vyhovuje jediné číslo 135.

\nobreak\medskip\petit\noindent
Za úplné riešenie dajte 6~bodov, z~toho
1~bod za zostavenie rovnice,
1~bod za vylúčenie $b\le 1$,
1~bod za vylúčenie $b\ge 5$,
1~bod za vylúčenie $b = 2$,
1~bod za vylúčenie $b = 4$,
1~bod za vyriešenie $b = 3$ a~nájdenie čísla~135.
V~prípade neúplného riešenia dajte 1~bod za uhádnutie riešenia.

\endpetit
\bigbreak
}

{%%%%%   B-S-3
Keďže druhý z~trojuholníkov $AEC$ a~$CBD$ je podľa zadania rovnoramenný
s~hlavným vrcholom~$B$, ukážeme v~prvej časti riešenia,
že aj prvý trojuholník $AEC$ je rovnoramenný s~hlavným vrcholom~$E$.

Keďže bod~$E$ leží na osi uhla $ABD$, majú jednak oba uhly $ABE$ a~$DBE$
rovnakú veľkosť, ktorú označíme $\al$ (\obr), jednak platí $|EC|=|ED|$.
Úsečka~$DE$ je však zhodná aj s~úsečkou~$AE$, lebo
im obom ako tetivám kružnice~$k$ zodpovedajú
zhodné obvodové uhly~$\al$ s~vrcholom~$B$.\footnote{Možno sa tiež
odvolať na známy výsledok, že bod~$E$ je stredom oblúka~$AD$
kružnice opísanej trojuholníku~$ABD$~-- ten sa však dokazuje práve použitím
poučky, že zhodné obvodové uhly v~jednej kružnici prislúchajú iba
jej zhodným oblúkom.}
Spolu dostávame, že aj úsečky $AE$ a~$CE$ sú zhodné.
Teda $AEC$ je naozaj rovnoramenný trojuholník s~hlavným vrcholom~$E$.
\insp{b68.3}%

Rovnoramenné trojuholníky $AEC$ a~$CBD$ budú (ako máme dokázať)
podobné, a~to podľa vety $uu$, keď ukážeme, že majú zhodné uhly
$EAC$ a~$BCD$ pri svojich základniach $AC$, resp. $CD$.
To je však na obrázku
už vyznačené ako dôsledok rovnobežnosti úsečiek $AE$ a~$CD$,
ktoré sú totiž obe kolmé na priamku~$BE$ (uhol $AEB$ je pravý
podľa Tálesovej vety, kolmosť $BE\perp CD$ vyplýva z~osovej
súmernosti trojuholníka $BCD$).

Ak si nevšimneme
rovnobežnosť úsečiek $CD$ a~$AE$, možno z~trojuholníkov $BCD$ a~$ABE$
ľahko vypočítať, že oba uhly $BCD$ a~$BAE$ (a~teda aj $CAE$)
majú veľkosť $90\st-\al$.

\poznamka
Rovnoramennosť trojuholníka $AEC$ sa dá zdôvodniť aj tak, že $|\uhel CAE| = 180\st -|\uhel EDB|
\hbox{ (kružnica)} = 180\st -|\uhel ECB| \hbox{ (osová súmernosť)} =|\uhel ACE|$ (vedľajší uhol).

\ineres
Kružnica so stredom~$B$ a~polomerom $|BC|$ pretína kružnicu~$k$
nielen v~bode~$D$, ale tiež v~bode~$D'$, ktorý je s~bodom~$D$
súmerne združený podľa priemeru~$AB$ kružnice~$k$. Označme ešte
$F$ druhý priesečník priamky~$CD'$ s~kružnicou~$k$.
Súmerne združené rovnoramenné trojuholníky $CBD$ a~$CBD'$ majú
pri svojich základniach $CD$ a~$CD'$ štyri zhodné vnútorné uhly,
ktoré sú na \obr\ označené písmenami~$\ga$ rovnako ako piaty uhol
$ACF$ (vrcholový k~uhlu $BCD'$) a~šiesty uhol $FAB$ (zhodný s~obvodovým
uhlom~$FD'B$).
\insp{b68.4}%
Podľa vety $uu$ sú trojuholníky $AFC$ a~$CBD$ podobné,
takže naše riešenie bude hotové, keď ukážeme, že bod~$F$ leží na osi uhla
$CBD$ (a~teda platí $E=F$). To je však jednoduché: jednak vďaka zhodným
súhlasným uhlom $BAF$, $BCD$ platí $AF\parallel CD$, jednak vďaka tomu,
že uhol $AFB$ je podľa Tálesovej vety pravý, platí $BF\perp AF$;
spolu máme $BF\perp CD$, a~tak je priamka~$BF$ naozaj osou
súmernosti rovnoramenného trojuholníka $BCD$ s~hlavným vrcholom~$B$.

\nobreak\medskip\petit\noindent
Za úplné riešenie dajte 6~bodov. Pri postupe z~prvého riešenia dajte
celkom 4~body za dôkaz, že $AEC$ je rovnoramenný trojuholník (z~toho
1~bod za odvodenie rovnosti $|CE|=|DE|$,
2~body za zdôvodnenie rovnosti $|AE|=|DE|$, ďalší 1~bod
za ich dôsledok $|AE|=|CE|$) a~napokon 2~body za porovnanie
vnútorných uhlov oboch rovnoramenných trojuholníkov. Za nezdôvodnené
konštatovanie rovnosti $|AE|=|CE|$ žiadny bod neudeľujte,
za prípadné pokračovanie v~podobe overovania
zhodnosti vnútorných uhlov oboch dotyčných trojuholníkov potom dajte nanajvýš
1~bod. Ten dajte aj v~prípade, keď jediným odvodeným poznatkom je
fakt $AE\parallel CD$ a~jeho dôsledok v~podobe zhodnosti uhlov
$CAE$ a~$BCD$.

Pri druhom postupe dajte 2~body za konštrukciu bodov $D'$ a~$F$,
2~body za dôkaz podobnosti trojuholníkov $AFC$, $BCD$ a~2~body za
zdôvodnenie rovnosti $E=F$.

\endpetit
}

{%%%%%   B-II-1
Výraz $V$ upravme nasledujúcim spôsobom:
$$
V=\frac{a^2+b^2}{ab+1}=\frac{(a+b)^2-2ab}{ab+1}=
\frac{4-2ab}{ab+1}=1+\frac{3(1-ab)}{ab+1}.
\tag1$$
Vďaka tomu, že $ab\ge0$, z~predposledného vyjadrenia výrazu~$V$ vyplýva odhad
$$
V\le\frac{4-0}{0+1}=4,
$$
pritom rovnosť $V=4$ je dosiahnutá práve vtedy, keď $ab=0$, čo spĺňajú
dve prípustné dvojice $(a,b)=(2,0)$ a~$(a,b)=(0,2)$.

Zo známej nerovnosti $\sqrt{ab}\le\frac12(a+b)$
medzi aritmetickým a~geometrickým priemerom dvoch nezáporných
čísel $a$, $b$ a~z~podmienky $a+b=2$ vyplýva, že
$\sqrt{ab}\le1$,
a~teda tiež $ab\le1$. Preto je zlomok v~poslednom vyjadrení
(1) nezáporný, a~teda platí $V\ge1$.
Rovnosť $V=1$ pritom nastáva práve vtedy, keď platí
$ab=1$, čo je podľa nášho odvodenia nerovnosti $ab\le1$
splnené pre jedinú prípustnú dvojicu $(a,b)=(1,1)$, lebo iba
v~prípade $a=b$ sa oba využité priemery rovnajú.

\odpoved
Za daných podmienok má výraz~$V$ najmenšiu hodnotu~1
a~najväčšiu hodnotu~4.


\ineres
Tentoraz pred úpravou výrazu~$V$ doňho dosadíme $b=2-a$:
$$
V=\frac{a^2+(2-a)^2}{a(2-a)+1}=\frac{2a^2-4a+4}{\m a^2+2a+1}=
\m2+\frac{6}{\m a^2+2a+1}=\m2+\frac{6}{2-(a-1)^2}.
\tag2$$
Keďže zo zadania vyplýva $0\le a\le 2$ čiže ${-1}\le
a-1\le 1$, platí $0\le(a-1)^2\le1$. Z~toho pre menovateľ
posledného zlomku v~(2) vyplývajú odhady $1\le2-(a-1)^2\le2$,
a~teda sú splnené nerovnosti
$$
V\le\m2+\frac{6}{1}=4\quad\text{a}\quad
V\ge\m2+\frac{6}{2}=1.
$$
Podľa nášho postupu pritom rovnosť $V=4$, resp. $V=1$ nastane
práve vtedy, keď hodnota $(a-1)^2$ bude rovná jednej, resp. nule, čo vedie na
rovnaké dvojice $(a,b)$ ako v~prvom riešení.


\ineres
Ak nájdeme dosiahnuteľné hodnoty $V=1$, $V=4$ a~napadne nás,
že to budú hľadané extrémy výrazu~$V$ za daných podmienok
na čísla $a$ a~$b$, môžeme potrebné nerovnosti $V\ge1$
a~$V\le4$ overiť pomerne ľahko ich ekvivalentnými úpravami,
napríklad tak, že po odstránení zlomku v~oboch prípadoch
využijeme rovnosť $4=({a+b})^2$:
$$
\xalignat2
\frac{a^2+b^2}{ab+1}&\ge1,& \frac{a^2+b^2}{ab+1}&\le4,\\
a^2+b^2&\ge ab+1,& a^2+b^2&\le 4ab+4,\\
4a^2+4b^2&\ge 4ab+(a+b)^2,& a^2+b^2&\le 4ab+(a+b)^2,\\
3(a-b)^2&\ge0,& 0&\le 6ab.
\endxalignat
$$
Keďže obe konečné nerovnosti platia, je celé riešenie hotové.
(Zároveň sme zistili, že rovnosť $V=1$, resp. $V=4$ nastane
práve v~prípade, keď prípustné čísla $a$, $b$ spĺňajú podmienku $a=b$,
resp. $ab=0$.)

\nobreak\medskip\petit\noindent
Za úplné riešenie dajte 6~bodov, z~toho po 2~bodoch za dôkazy
nerovností $V\ge1$, $V\le4$ a~po 1~bode za príklady dvojíc
$(a,b)$, pre ktoré v~dokázaných nerovnostiach nastanú rovnosti, pričom
nepožadujeme odvodenie príkladov dvojíc uvedeným rozborom, stačí ich uviesť.
Ak však sú obe správne hodnoty 1 a~4 len uhádnuté, dajte
dokopy iba 1~bod, a~to iba za predpokladu, že obe hodnoty
sú doložené príkladmi dvojíc $(a,b)$.

\endpetit
\bigbreak
}

{%%%%%   B-II-2
V~ľubovoľnom vyhovujúcom osemcifernom čísle~$N$ označme $A$ jeho prvé
dvojčíslie, $B$ nasledujúce štvorčíslie a~$C$ posledné dvojčíslie.
Ak budeme $A$, $B$, $C$ chápať ako čísla zapísané v~desiatkovej sústave,
platí $10\le A\le 99$, $1\,000\le B\le 9\,999$ (podľa
zadania je číslo~$B$ štvorciferné, takže jeho zápis nezačína
nulou), $0\le C\le 99$. %(zápis čísla~$C$ nulou začínať môže).
Pôvodné číslo~$N$ potom má vyjadrenie $N=10^6A+10^2B+C$, ktoré má
podľa zadania spĺňať rovnicu
$$
10^6A+10^2B+C=2\,019B,\quad\text{čiže}\quad
10^6A+C=1\,919B.
$$
Zapíšme ju vzhľadom na rozklad $1\,919=19\cdot101$ ako rovnicu
$$
(101-1)^3A+C=19\cdot101\cdot B,
$$
z~ktorej vyplýva, že číslo 101 je deliteľom celého čísla $C-A$,
pretože
$$
(101-1)^3A+C=\bigl(101^3A-3\cdot101^2A+3\cdot101A-A\bigr)+C=101k+(C-A),
$$
pričom $k$ je vhodné celé číslo. Keďže však z~uvedených odhadov pre
čísla $A$ a~$C$ vyplývajú nerovnosti $\m99\le C-A\le89$
a~v~intervale $\langle\m99,89\rangle$ leží jediný násobok čísla 101,
konkrétne číslo 0, musí platiť $C-A=0$. Po dosadení $C=A$ dostaneme
zjednodušenú rovnicu
$$
\bigl(10^6+1\bigr)A=19\cdot101\cdot B,
$$
ktorej obe strany už môžeme vydeliť číslom 101, lebo
$10^6+1=101\cdot9\,901$, a~dospieť tak ku konečnej rovnici
$$
9\,901A=19B.
$$
Z~toho vďaka nesúdeliteľnosti čísel 9\,901 a~19 (platí totiž
$9\,901=521\cdot19+2$) vyplýva, že štvorciferné číslo~$B$ je násobkom čísla
9\,901, a~preto posledná rovnica je v~našej situácii splnená jedine
tak, že je $B=9\,901$ a~$A=19$ (a~teda tiež $C=19$).

\odpoved
Úlohe vyhovuje jediné osemciferné číslo
$19\,990\,119$.

\nobreak\medskip\petit\noindent
Za úplné riešenie dajte 6 bodov.
Za vhodné označenie čísel $A$, $B$,
$C$ a~zostavenie rovnice dajte 1 bod, 2~body
za pozorovanie, že 101 musí deliť $A\cdot10^6 + 1$,
1~bod za odvodenie $101 \mid C-A$,
1~bod za odvodenie rovnosti $C = A$ a
1~bod za doriešenie rovnice $9\,901 = 19B$ (nesúdeliteľnosť čísel 9\,901 a~19
je nutné aspoň uviesť, jej dôkaz však nevyžadujeme).

Ak riešiteľ deliteľnosť číslom 101 neuplatní a~dospeje iba
k~záveru $1\,919\mid A\cdot10^6 + 1$, dajte 1 bod.
\endpetit
\bigbreak
}

{%%%%%   B-II-3
Bod $C$ musí ležať na kružnicovom oblúku $ASK$ medzi bodmi $S$ a~$K$
(\obr). Oblúku~$CK$ tak prislúchajú zhodné obvodové uhly $CSK$
a~$CAK$. Druhý z~nich je pritom ostrým obvodovým uhlom nad
oblúkom~$BC$ pôvodnej kružnice~$k$ (lebo ostrý je aj väčší uhol
$SAB$ pri základni~$AB$ rovnoramenného trojuholníka $SAB$),
takže sa rovná polovici konvexného stredového uhla $BSC$. Tej
sa teda rovná aj uhol $CSK$, takže priamka~$KS$ rozpoľuje
uhol $BSC$. Preto je aj osou základne~$BC$ rovnoramenného
trojuholníka $SBC$ a~dôkaz tvrdenia je ukončený.
\insp{b68.5}%

\ineres
Vďaka polohe bodu~$C$ na kružnicovom oblúku $ASK$ (\obr) je
štvoruholník $AKCS$ tetivový, a~tak sa jeho vnútorné uhly $KCS$
a~$KAS$ dopĺňajú do~$180^{\circ}$. Avšak druhý z~týchto uhlov,
totožný s~uhlom $BAS$, je
vďaka rovnosti $|AS|=|BS|$ zhodný s~uhlom $ABS$, ktorý sa
zasa dopĺňa do $180^{\circ}$ s~vedľajším uhlom~$KBS$. Spolu
dostávame zhodnosť uhlov $KBS$ a~$KCS$. Sú to vnútorné uhly
trojuholníkov $BKS$ a~$CKS$, oba protiľahlé k~ich spoločnej strane~$KS$. Keďže vďaka zadaniu úlohy navyše platí $|BS|=|CS|<|KS|$,
sú trojuholníky $BKS$ a~$CKS$ zhodné podľa vety $Ssu$, a~teda
ich vrcholy $B$ a~$C$ sú naozaj
súmerne združené podľa priamky~$KS$, ako sme mali dokázať.
\insp{b68.6}%

\nobreak\medskip\petit\noindent
Za úplné riešenie dajte 6 bodov, absenciu zmienky o~polohe bodu~$C$
na oblúku $ASK$ pritom nepenalizujte.

Pri prvom postupe dajte 2~body za rovnosť uhlov $CSK$ a~$CAK$,
2~body za vzťah medzi uhlami $BAC$ a~$BSC$,
zvyšné 2~body za záverečnú úvahu o~rovnoramennom trojuholníku~$SBC$.

Pri druhom postupe dajte 4~body za
dôkaz zhodnosti uhlov $KBS$ a~$KCS$ (z~toho 2~body za vzťah medzi
uhlami $KCS$ a~$KAS$~-- len za zmienku o~tetivovom štvoruholníku $AKCS$
žiadny bod neudeľujte) a~2~body za uplatnenie vety $Ssu$ (ak pritom chýba porovnanie dĺžok $Ss$, strhnite 1~bod).

\endpetit
\bigbreak
}

{%%%%%   B-II-4
Vieme, že prirodzené číslo je druhou mocninou práve vtedy,
keď v~jeho prvočíselnom rozklade má každé prvočíslo párny
počet výskytov. Prvočísla zo zadanej množiny $\mm Z=\{1,2,3,\dots,20\}$
tvoria množinu $\mm P=\{2,3,5,7,11,13,17,19\}$, zvyšných 12~čísel
(jednotka a~tie zložené) potom množinu
$$
\mm Q=\{1,4,6,8,9,10,12,14,15,16,18,20\}.
$$
Keďže žiadna zo štvorcových podmnožín množiny~$\mm Z$ zrejme nemôže mať
svoje prvky napospol z~$\mm P$, musí obsahovať aspoň jedno číslo z~$\mm Q$.

Vysvetlime, prečo naopak {\it každú\/}
z~$2^{12}-1$ neprázdnych podmnožín~$\mm X$ množiny~$\mm Q$ možno jediným spôsobom
doplniť prvočíslami z~$\mm P$ tak, aby tým vznikla štvorcová
množina (prípadne nie je nutné ani možné doplniť žiadne prvočíslo z~$\mm P$,
ak je už samotná množina~$\mm X$ štvorcová).
Vyplýva to z~toho, že pre danú neprázdnu množinu $\mm X\subseteq \mm Q$
dopĺňajúcimi prvočíslami z~$\mm P$ musia byť práve tie, ktoré sa v~prvočíselnom
rozklade čísla rovného súčinu prvkov z~$\mm X$ vyskytujú v~nepárnom počte.
Takto vytvorené štvorcové množiny v~počte $2^{12}-1$ $(=4\,095)$
sú zrejme navzájom rôzne (lebo
každá má s~množinou~$\mm Q$ iný prienik),
preto je počet všetkých
štvorcových podmnožín množiny~$\mm Z$ naozaj rovný $2^{12}-1$, ako
sme mali dokázať.

\ineres
Ukážeme, že rovnakú myšlienku možno uplatniť aj pri konštrukcii
ľubovoľnej štvorcovej množiny
$\mm X\subseteq\{1,2,3,\dots,20\}$. Budeme
postupne rozhodovať, ktoré z~aktuálnych čísel $20,19,\dots,1$ (radených zostupne)
do~$\mm X$ zaradiť a~ktoré nie, sledujúc pritom rozklad súčinu všetkých
doposiaľ vybraných čísel na prvočinitele (tento [prázdny] súčin považujeme za
rovný~1, kým nevyberieme prvé číslo).

\item{$\triangleright$}
Ak je aktuálne číslo~$n$ zložené alebo rovné jednej, môžeme sa rozhodnúť
ľubovoľne (vybrať ho, alebo nevybrať), pretože ak je $n>1$, sú všetky jeho
prvočinitele menšie a~budú aktuálne neskôr.
\item{$\triangleright$}
Ak je aktuálne číslo~$n$ rovné prvočíslu~$p$,
máme pri rozhodovaní o~jeho výbere poslednú možnosť,
ako ovplyvniť paritu počtu jeho výskytov v~rozklade aktuálneho súčinu
doposiaľ vybraných čísel. Potrebujeme, aby sa
tento počet buď zmenil z~nepárneho čísla na párne~-- vtedy~$p$ vyberieme,
alebo aby zostal párny~-- vtedy~$p$ nevyberieme (ako to bude vždy
pri aktuálnych prvočíslach 19, 17, 13 a~11).

\noindent
Opísané jednoznačné rozhodnutia o~každom aktuálnom prvočísle nám
zaručia, že množina~$\mm X$ všetkých vybraných čísel
po ukončení celej konštrukcie bude štvorcová.
Keďže pritom dve možnosti (vybrať, či nevybrať) budeme mať pre
práve 12~aktuálnych čísel, a~to 20, 18, 16, 15, 14, 12, 10, 9, 8, 6, 4 a~1,
bude celkový počet možných výberov $2^{12}$,
v~jednom prípade ale dostaneme prázdnu množinu~$\mm X$, ktorá je zadaním
úlohy vylúčená.

\ineres
Ukážeme, že úlohu možno (aj keď komplikovanejšie) riešiť istým rozborom
možností, pri ktorom budeme dbať na zastúpenie jednotlivých
prvočísel v~súčine čísel, ktoré budeme do štvorcovej množiny po
etapách vyberať.

Prvočísla 11, 13, 17 a~19 sa zrejme v~žiadnej zo započítaných štvorcových množín
nemôžu vyskytovať. Čísla z~množiny $\mm C=\{1,4,9,16\}$ sú
sami druhými mocninami, preto ich doplnenie či naopak odstránenie
nemá na štvorcovosť takto upravovanej množiny vplyv
(aby to bolo korektné vyjadrenie, považujme dočasne za štvorcovú
aj {\it prázdnu\/} množinu). Odhliadnime teda od doposiaľ spomenutých
čísel a~určme najskôr, koľko štvorcových podmnožín
má množina zvyšných čísel
$$
\mm M=\{2,3,5,6,7,8,10,12,14,15,18,20\}.
$$
V~rozklade každého čísla z~$\mm M$ má niektoré z~prvočísel
$p\in\{2,3,5,7\}$ {\it nepárny\/} počet výskytov, pritom do štvorcovej podmnožiny
musíme vybrať čísla z~$\mm M$ práve tak, aby pre každé prvočíslo~$p$ bol vybraný
{\it párny\/} počet čísel, ktoré majú vo svojom rozklade {\it nepárny\/}
počet výskytov~$p$. Máme teda vybrať párny počet čísel z~každej zo skupín
$$
\{7,14\},\ \{5,10,15,20\},\ \{3,6,12,15\},\ \{2,6,8,10,14,18\}.
\tag
S~$$
Tieto skupiny ale nie sú po dvoch disjunktné, čo komplikuje postup
výberov čísel z~týchto skupín pre ľubovoľnú štvorcovú podmnožinu~$\mm M$,
na ktorý by sme chceli uplatniť
kombinatorické pravidlo súčinu. Rozdeľme preto množinu~$\mm M$ na
(disjunktné) skupiny násobkov jednotlivých prvočísel podľa {\it
najväčšieho\/} z~prvočísel, ktoré majú v~rozklade daného čísla
{\it nepárny\/} počet výskytov:
$$
\mm N_{7}=\{7,14\},\
\mm N_{5}=\{5,10,15,20\},\
\mm N_{3}=\{3,6,12\},\
\mm N_{2}=\{2,8,18\}.
$$
Z~týchto užších skupín (v~uvedenom poradí) budeme vyberať čísla do ľubovoľnej
štvorcovej podmnožiny~$\mm M$, a~to tak, aby sme dodržali podmienku
na zastúpenie čísel z~pôvodných skupín uvedených v~(S).
V~prvých dvoch krokoch musíme
vybrať párny počet (0 alebo~2) čísel z~$\mm N_{7}$ aj párny počet
(0, 2 alebo 4) čísel z~$\mm N_{5}$. Na tieto dva výbery tak máme $2\times
8=2^{4}$ možností. Potom vzhľadom na to, či bolo, resp. nebolo
vybrané číslo $15\in \mm N_5$, musíme z~$\mm N_3$ vybrať nepárny (1 alebo~3),
resp. párny (0 alebo~2) počet čísel. V~oboch prípadoch tak
budeme mať na výber rovnako $4=2^2$ možností, prvé tri výbery
tak možno spraviť $2^{4}\times2^{2}=2^{6}$ spôsobmi. Posledný,
štvrtý výber čísel z~$\mm N_{2}$ bude závisieť na tom,
či sme z~čísel $14\in \mm N_7$, $10\in \mm N_5$ a~$6\in \mm N_3$
vybrali nepárny, resp. párny počet~-- podľa toho musíme z~$\mm N_{2}$
vybrať rovnako tak nepárny (1 alebo~3), resp. párny (0~alebo~2) počet čísel.
Aj teraz máme v~oboch rozlíšených prípadoch pre výber čísel z~$\mm N_2$
rovnako $4=2^2$ možností. Možno teda naposledy uplatniť
pravidlo súčinu a~dôjsť tak k~záveru, že hľadaný počet
všetkých štvorcových podmnožín~$\mm M$ je rovný
$2^{6}\times2^{2}=2^{8}$. Keďže pre každú z~nich máme
$2^4$~možností pre doplnenie
o~čísla z~$\mm C$, je celkový počet štvorcových
podmnožín pôvodnej zadanej množiny rovný
$2^{8+4}$, teda $2^{12}$~-- vrátane tej prázdnej, za ktorú je teraz
ešte nutné odčítať jednotku.

\nobreak\medskip\petit\noindent
Za úplné riešenie dajte 6 bodov, za drobné formálne chyby strhnite
1~bod.

Pri postupe z~prvého riešenia dajte 1~bod za rozdelenie zadanej
množiny na množinu~$\mm P$ zastúpených prvočísel (z~nej je možné
prvočísla 11, 13, 17 a~19 rovno eliminovať) a~na množinu
zvyšných čísel~$\mm Q$, 4~body za vysvetlenie, prečo každú skupinu
čísel z~$\mm Q$ možno jediným spôsobom doplniť niektorými prvočíslami z~$\mm P$
na štvorcovú množinu a~1~bod za dokončenie dôkazu.

Pri postupe z~druhého riešenia dajte 1~bod za
rozhodnutie konštruovať štvorcovú množinu postupným rozhodovaním
o~zastúpení čísel v~klesajúcom poradí, \tj. od čísla 20 k~číslu~1,
2~body za rozhodnutie o~všetkých aktuálnych číslach zložených
a~3~body za rozhodnutie o~všetkých aktuálnych prvočíslach.
Ak je však správne rozhodnuté iba o~prvočíslach 11, 13, 17, 19
a~štvorcových číslach 1, 4, 9 a~16, dajte za taký počiatočný pokus
o~konštrukciu iba 1~bod.

Pri neúplnom postupe z~tretieho riešenia dajte nanajvýš 3~body,
z~toho 1~bod za elimináciu prvočísel 11, 13, 17, 19 spolu
s~konštatovaním o~indiferentnosti čísel 1, 4, 9 a~16.
Druhý bod potom dajte pri ďalších drobných úvahách o~zastúpení čísel
z~vhodne vybraných množín, ako sú napr. $\{7,14\}$ alebo
$\{5,10,15,20\}$. Tretí bod je možné udeliť iba pri výraznejšom
pokroku, akým je napr. úvaha o~tom, že štvorcové množiny sú práve tie,
ktoré obsahujú párny počet čísel z~každej zo štyroch skupín uvedených v~(S).

\endpetit
\bigbreak
}

{%%%%%   C-S-1
Medzi vybranými číslami nesmie byť žiadny násobok deviatich a~pritom medzi nimi
môže byť nanajvýš jedno číslo, ktoré je deliteľné tromi, nie však deviatimi.
Môžeme teda vybrať všetky čísla, ktoré nie sú deliteľné tromi, a~pridať
k~nim akékoľvek číslo, ktoré je deliteľné tromi, nie však deviatimi.

\zaver
Najväčší možný počet čísel, ktoré môžeme požadovaným
spôsobom vybrať z~množiny $\{1, 2,\ldots, 2019\}$, je teda rovný
$\frac23\cdot 2\,019+1=1\,347$.

\smallskip
Úlohe vyhovuje napr. množina $\{1,2,3,4,5,7,8,10,11,\dots,2015,2017,2018\}$,
ktorá má práve $1\,347$~prvkov.
Táto množina obsahuje číslo~3 ako jediné číslo deliteľné tromi, ktoré
však deviatimi deliteľné nie je.

\nobreak\medskip\petit\noindent
Za úplné riešenie dajte 6 bodov, z~toho
1~bod za zdôvodnenie, že tam nemôže byť číslo deliteľné 9,
1~bod za zdôvodnenie, že tam môže byť nanajvýš jedno číslo deliteľné 3 (nedeliteľné 9),
3~body za odhad maximálneho počtu prvkov na základe týchto pozorovaní,
1~bod za príklad vyhovujúcej množiny.

\endpetit
\bigbreak
}

{%%%%%   C-S-2
Ukážeme, že vyhrávajúcu stratégiu má Pavol, ktorý môže vyhrať už
svojim druhým ťahom.

Vrcholy uvažovaného štvorstena označme písmenami $A$, $B$, $C$, $D$.
V~prvom ťahu Pavol zväčší o~2 číslo pri niektorom vrchole štvorstena
$ABCD$, napr. pri~$A$. Michal potom zvolí buď niektorú hranu
vychádzajúcu z~toho istého vrcholu (napr.~$AB$),
alebo vyberie niektorú hranu, ktorá z~$A$ nevychádza, napr.~$BC$.

V~prvom prípade Pavol (vo svojom druhom ťahu) zväčší o~2 číslo napísané
pri jednom z~vrcholov $C$, $D$, napr. pri vrchole~$C$. Pri jednotlivých
vrcholoch $A$, $B$, $C$,~$D$ štvorstena $ABCD$ potom budú {\it po
rade\/} napísané navzájom rôzne čísla $3$, $1$, $2$, $0$. V~druhom prípade Pavol
zväčší o~2 hodnotu pri niektorom z~vrcholov $B$ alebo $C$, napr. pri~$B$.
Pri jednotlivých vrcholoch $A$, $B$, $C$,~$D$ uvažovaného štvorstena $ABCD$
sú potom aj v~tomto prípade napísané {\it po rade\/} navzájom rôzne
čísla $2$, $3$, $1$, $0$.

Tým je úloha vyriešená.

\poznamka
Úlohu možno riešiť bez označenia vrcholov iba úvahami o~neusporiadaných
štvoriciach k~nim pripísaných čísel, keďže každé dva vrcholy štvorstena
sú spojené hranou. So zápismi štvoríc čísel v~poradí od najväčšieho
po najmenšie potom celé riešenie vyzerá nasledovne:
Po prvom Pavlovom ťahu vznikne štvorica $2$, $0$, $0$, $0$, ktorú Michal
môže zmeniť
buď na štvoricu $3$, $1$, $0$, $0$, alebo na štvoricu $2$, $1$, $1$, $0$.
Každú z~oboch štvoríc zrejme Pavol dokáže svojim druhým ťahom zmeniť
na štvoricu $3$,~$2$,~$1$,~$0$, a~to buď zväčšením jednej z~dvoch núl na dvojku,
alebo zväčšením jednej z~dvoch jednotiek na trojku.

\nobreak\medskip\petit\noindent
Za úplné riešenie dajte 6 bodov, z~toho
1~bod za vysvetlenie, že po dvoch ťahoch sú možnosti 3,~1,~0,~0 resp. 2, 1, 1, 0,
následne 2~body za prvý prípad,
2~body za druhý prípad,
1~bod za záver.
Za holé konštatovanie (bez hlbšieho zdôvodnenia), že vyhrávajúcu stratégiu má Pavol,
dajte 1~bod.

\endpetit
\bigbreak
}

{%%%%%   C-S-3
Nech $G$ označuje stred úsečky~$BD$, čo znamená, že bod~$D$ je nielen stredom
strany~$AB$, ale aj stredom úsečky~$FG$ (\obr). Úsečka~$EG$ je potom strednou
priečkou v~trojuholníku $BCD$, a~je teda rovnobežná s~$CD$.
Bod~$C$ tak leží na rovnobežke so stranou~$EG$ trojuholníka $GEF$ idúcej stredom~$D$
jeho strany~$FG$, a~preto táto rovnobežka~$CD$ nutne prechádza aj stredom~$H$
tretej strany~$EF$, ako sme mali dokázať.
\inspinsp{c68.3}{c68.4}%

\ineres
Označme $S$ stred ťažnice~$CD$ trojuholníka $ABC$ (\obr).
Úsečka~$DE$ je stredná priečka v~trojuholníku $ABC$, takže $|DE|=\frac12|AC|$,
a~úsečka~$FS$ je stredná priečka v~trojuholníku $ADC$, takže $|FS|=\frac12|AC|$.
Úsečky $DE$ a~$FS$ sú teda zhodné a~rovnobežné
(so stranou~$AC$). Štvoruholník $DESF$ je preto rovnobežník, a~ako je známe,
jeho uhlopriečky sa navzájom rozpoľujú. Tým je dôkaz ukončený.
\insp{c68.5}%

\ineres
Pre bod~$D$ platí $|FD|:|DB| = 1:2$.
Ak zostrojíme bod~$C'$ ako obraz bodu~$C$ v~stredovej súmernosti
podľa stredu~$F$ (\obr), bude $BF$ ťažnica trojuholníka $BCC'$ a~bod~$D$
jeho ťažisko. Priamka~$CD$ teda obsahuje ťažnicu trojuholníka $BCC'$, a~preto rozpoľuje
jeho stranu~$BC'$ rovnako ako jeho strednú priečku~$EF$ s~ňou rovnobežnú.

\ineres
Máme dokázať, že na priamke~$CD$ leží ťažnica trojuholníka $CEF$, čo je ekvivalentné s~tým,
že obsahy trojuholníkov $CDF$ a~$CED$ (\obrr3)
sú rovnaké.\footnote{To vyplýva zo známeho faktu, že pre ľubovoľný bod~$X$ vnútri
uhla $QPR$ majú trojuholníky $PXQ$, $PXR$ rovnaký obsah práve vtedy, keď bod~$X$ leží na priamke
ťažnice z~vrcholu~$P$ trojuholníka $PQR$.}
Pritom zrejme pre obsahy jednotlivých trojuholníkov platí
$$
S(CDE) = \tfrac12 S(DBC) = \tfrac12 S(CAD)= S(DFC),
$$
lebo $E$ je stred $BC$, $D$ je stred $AB$ a~$F$ je stred $AD$.

\poznamka
Rovnosť obsahov sa dá dokázať aj~postupným výpočtom obsahov vzhľadom
na~obsah trojuholníka~$ABC$: Bez ujmy na všeobecnosti predpokladajme, že ${S(ABC) =1}$.
Keďže $|FB|:|AB| = \frac34$, je $S(BFC) = \frac34$.
Keďže ${|FD|:|DB| = \frac12}$, je $S(CDF) = \frac14$. Ďalej $S(CDB) = \frac12$,
a~keďže $E$ je stred~$CB$,
je $S(CED) = {\frac12 S(CDB) = \frac14}$, takže naozaj $S(CDF) = S(CDB)$.

\nobreak\medskip\petit\noindent
Za úplné riešenie dajte 6~bodov.
Pri prvom postupe dajte 3~body za zavedenie bodu~$G$,
1~bod za dôkaz $EG\parallel CD$,
2~body za zdôvodnenie, že stredná priečka~$DH$ trojuholníka $GEF$ musí ležať na priamke~$CD$.

Pri druhom postupe dajte 3~body za zavedenie bodu~$S$,
1~bod za objav oboch stredných priečok $ED$, $FS$, 1~bod za zdôvodnenie,
že sa jedná o~dve rovnobežné a~zhodné úsečky,
a~1~bod za záver, že $DESF$ je rovnobežník, a~teda sa jeho uhlopriečky rozpoľujú
(namiesto toho možno uplatniť vetu $usu$ na~dôkaz zhodnosti trojuholníkov $DEH$ a~$SFH$,
pričom $H$ označuje priesečník uhlopriečok, z~ktorej potrebná rovnosť $|EH|=|FH|$~vyplýva).

Pri treťom postupe dajte 3~body za zavedenie bodu~$C'$,
1~bod za zdôvodnenie, že $D$ je ťažisko trojuholníka $BCC'$ (a~teda $CD$ je priamka jeho ťažnice),
ďalej 1~bod za objav strednej priečky~$EF$ a~1~bod za záver, že ťažnica rozpoľuje strednú priečku.

Napokon za tvrdenie, že stačí dokázať rovnosť obsahov trojuholníkov $CDF$, $CED$
(alebo ekvivalentne, že obsah štvoruholníka $CFDE$ je dvakrát väčší ako jeden z~nich)
dajte 3~body, za dôkaz rovnosti spomenutých obsahov ďalšie 3~body.
Za nedokončený výpočet však dajte nanajvýš 1~bod.

\endpetit
}

{%%%%%   C-II-1
Akonáhle určíme farby niektorých dvoch susedných políčok
jedného riadka alebo stĺpca tabuľky, ofarbenie všetkých jeho ďalších políčok
je už požiadavkami úlohy určené jednoznačne.
Farby v~každom riadku aj stĺpci sa tak pravidelne striedajú
s~periódou~3.

Povedzme, že $abcabc\dots$ je ofarbenie políčok prvého riadka.
Zo šiestich možných ofarbení políčok ľubovoľného riadka
$$
\aligned
&abcabc\dots,\\
&bcabca\dots,\\
&cabcab\dots,\\
&acbacb\dots,\\
&bacbac\dots,\\
&cbacba\dots
\endaligned
$$
zrejme pre druhý riadok prichádzajú do úvahy iba
druhé a~tretie. Akonáhle jedno z~nich
pre druhý riadok zvolíme, druhé z~nich musí byť ofarbením políčok tretieho riadka.
Štvrtý riadok potom bude ofarbený rovnako ako prvý, piaty ako druhý atď.
s~periódou~3.

Zhrňme naše úvahy. Pre ofarbenie $abcabc\dots$ prvého riadka
máme $3\cdot2=6$ možností (3~pre voľbu~$a$, 2 pre voľbu~$b$).
Pre ofarbenie druhého riadka, ako sme zistili, potom máme dve možnosti. Ofarbenie ďalších
riadkov je už určené jednoznačne.
Počet všetkých vyhovujúcich ofarbení celej tabuľky je teda rovný $6\cdot2=12$.


\nobreak\medskip\petit\noindent
Za úplné riešenie dajte 6 bodov.
Za konštatovanie, že sa farby v~každom riadku
pravidelne striedajú s~periódou~3 (aj bez podrobnejšieho vysvetlenia), dajte 2~body
a~ďalší bod za zdôvodnenie, že počet možností ofarbení jedného riadka
je potom rovný~6.
Ďalšie 2~body dajte za zdôvodnenie, prečo po ofarbení jedného riadka možno ďalší riadok
ofarbiť dvoma spôsobmi, a~posledný bod za záver, že počet ofarbení
je rovný~12. Len za uhádnutie správneho výsledku dajte 1~bod.
(Rovnaké úvahy samozrejme platia aj pre stĺpce.)
\endpetit
\bigbreak}

{%%%%%   C-II-2
Dajme tomu, že máme štyri čísla s~požadovanými vlastnosťami.
Keďže práve tri dvojice majú párneho najväčšieho spoločného deliteľa,
sú práve tri z~nich párne a~jedno nepárne (nemôžu byť
všetky párne a~dve párne sú na tri párne spoločné delitele málo).
Označme tri párne čísla $a$, $b$ a~$c$
tak, že pre ich najväčšie spoločné delitele s~nepárnym číslom~$d$ platí
$(d, a) = 3$, $(d, b) = 5$, $(d, c) = 9$. Z~toho potom vychádza,
že párne čísla $a$, $b$, $c$ sú postupne násobky čísel 6, 10, 18
a~číslo~$d$ je nutne násobok~45.

Čísla $a$, $c$ majú spoločného deliteľa~6, takže nutne platí
$(a, c) = 6$. Hodnoty $(a, b)$ a~$(b, c)$ sú preto v~niektorom poradí
čísla 2 a~4.
Máme tak dve možnosti:

Čísla $a$, $b$ sú násobky 4. Potom čísla $a$, $b$, $c$, $d$ sú postupne násobky
čísel 12, 20, 18,~45. Takáto vyhovujúca štvorica má najmenší súčet
${12 + 20 + 18 + 45 = 95}$.

Čísla $b$, $c$ sú násobky 4. Potom čísla $a$, $b$, $c$, $d$ sú postupne násobky
čísel 6, 20, 36,~45. Takáto vyhovujúca štvorica má najmenší súčet
$6 + 20 + 36 + 45 = 107$.

Najmenší možný súčet je teda 95, čomu zodpovedá štvorica 12, 20, 18, 45.




\nobreak\medskip\petit\noindent
Za úplné riešenie dajte 6 bodov.
Z~toho 1~bod dajte za uvedenie skutočnosti, že tri zo štyroch hľadaných čísel sú párne
a~jedno je nepárne,
a~ďalší bod za zistenie, že jednotlivé čísla sú násobky 6, 10, 18 a~45.
2~body potom dajte za využitie deliteľnosti štyrmi a~napokon 2~body za záver
a~určenie správnej štvorice.
Len za uhádnutie správnej štvorice dajte 2~body.
\endpetit
\bigbreak
}

{%%%%%   C-II-3
Trojuholníky $KLX$ a~$PDX$ sú zrejme podobné (podľa vety~$uu$).
Ich pomer podobnosti je rovný
$|KL|:|PD|=\frac16:\frac12=1:3$. V~rovnakom pomere teda bod~$X$ delí
úsečku~$KP$, takže $|KX|=\frac14 b$, pričom $b=|BC|$. Podľa vety $uu$ sú podobné
aj trojuholníky $BLY$ a~$DQY$, pričom ich pomer podobnosti je
$|BL|:|DQ|=\frac13:\frac23={1:2}$. Je teda $|LY|=\frac13b$.

Označme $Z$ stred strany~$BC$ uvažovaného rovnobežníka $ABCD$
a~veďme bodom~$X$ rovnobežku so stranou~$AB$ (\obr).
Jej priesečníky s~úsečkami $LQ$ a~$BC$ označme postupne $M$ a~$N$,
takže $|LM|=|BN|=|KX|=\frac14 b$,
$|MY|=|LY|-|LM|={\frac13b-\frac14 b}=\frac1{12} b$
a~$|NZ|=|BZ|-|BN|=\frac12b-\frac14 b=\frac14 b$.
Z~toho vyplýva, že
$$
\frac{|MY|}{|NZ|}=\frac{\frac1{12} b}{\frac14 b}=\frac13
$$
a~zároveň
$$
\frac{|XM|}{|XN|}=\frac{|KL|}{|KB|}=
\frac{\frac16}{\frac12}=\frac13,
$$
čo spolu s~rovnosťou súhlasných uhlov $XMY$ a~$XNZ$ znamená, že
trojuholníky $XMY$ a~$XNZ$ sú podobné (podľa vety $sus$).
Ich vnútorné uhly pri spoločnom vrchole~$X$ sú teda zhodné,
a~preto stred~$Z$ strany~$BC$ leží na priamke~$XY$.
\insp{c68.6}%

\poznamka
Úlohu možno riešiť aj využitím poznatku, že priamka~$XY$ prechádza vrcholom~$A$,
čo možno tiež odvodiť úvahami o~podobných trojuholníkoch.
Z~rovností $|AK|=\frac12$, $|AL|=\frac23$, $|KX|=\frac14b$, $|LY|=\frac13b$ totiž vyplýva
$$
\frac{|AK|}{|AL|} = \frac{\frac12} {\frac23} = \frac34, \qquad
\frac{|KX|}{|LY|} = \frac{\frac14b}{\frac13b} =\frac 34,
$$
takže podobne ako vo vzorovom riešení dostávame podobnosť
trojuholníkov $AKX$, $ALY$, z~ktorej už vyplýva, že body $A$, $X$, $Y$ ležia na jednej priamke.
Analogicky dokážeme, že napríklad aj trojice bodov $A$, $X$, $Z$ ležia
na jednej priamke, lebo
$$
\frac{|AK|}{|AB|} = \frac{\frac12} {1} = \frac12, \qquad
\frac{|KX|}{|BZ|} = \frac{\frac14b}{\frac12b} =\frac 12.
$$

\nobreak\medskip\petit\noindent
Za úplné riešenie dajte 6~bodov, z~toho 2~body
za vyjadrenie dĺžok $KX$ a~$LY$ pomocou dĺžky strany~$BC$,
2~body za uvažovanie rovnobežky bodom~$X$ a~2~body za dôkaz
kolineárnosti pomocou podobnosti.

Pri postupe naznačenom v~poznámke dajte 2~body za vyjadrenie
dĺžok $KX$ a~$LY$, ďalšie
2~body za dôkaz kolineárnosti bodov $A$, $X$, $Y$
a~2~body za dôkaz kolineárnosti bodov $A$, $X$, $Z$ (prípadne $A$, $Y$, $Z$).
Len za uhádnutie, že priamka~$XY$ prechádza vrcholom~$A$,
dajte 1~bod a~závery z~toho vyplývajúce už nehodnoťte.
\endpetit
\bigbreak
}

{%%%%%   C-II-4
Z~predpokladanej rovnosti $ab+bc+ca=\frac54$ a~z~podmienok
$b>\frac12$, $c>\frac12$ vyplýva $\frac54>\frac14+\frac12a+\frac12a$,
\tj. $1>a$, a~teda $a>a^2$, lebo číslo~$a$ je podľa predpokladu kladné.
Analogicky dostaneme aj $b>b^2$ a~$c>c^2$.
Sčítaním posledných troch nerovností už dostávame vytúženú nerovnosť
$$
a+b+c>a^2+b^2+c^2.
$$

\nobreak\medskip\petit\noindent
Za úplné riešenie dajte 6~bodov. Z~toho 3~body
dajte za dôkaz, že uvažované reálne čísla $a$, $b$, $c$ sú menšie
ako jedna, a~ďalší 1~bod potom dajte za uvedenie odtiaľ vyplývajúcich
nerovností $a<a^2$, $b<b^2$, $c<c^2$. Za dokončenie dôkazu napokon udeľte
posledné 2~body. Len za uvedenie skutočnosti, že všetky tri čísla $a$, $b$,
$c$ sú menšie ako jedna (bez dôkazu), neudeľujte žiadny bod.
\endpetit
\bigbreak
}

{%%%%%   vyberko, den 1, priklad 1
...}

{%%%%%   vyberko, den 1, priklad 2
...}

{%%%%%   vyberko, den 1, priklad 3
...}

{%%%%%   vyberko, den 1, priklad 4
...}

{%%%%%   vyberko, den 2, priklad 1
...}

{%%%%%   vyberko, den 2, priklad 2
...}

{%%%%%   vyberko, den 2, priklad 3
...}

{%%%%%   vyberko, den 2, priklad 4
...}

{%%%%%   vyberko, den 3, priklad 1
...}

{%%%%%   vyberko, den 3, priklad 2
...}

{%%%%%   vyberko, den 3, priklad 3
...}

{%%%%%   vyberko, den 3, priklad 4
...}

{%%%%%   vyberko, den 4, priklad 1
...}

{%%%%%   vyberko, den 4, priklad 2
...}

{%%%%%   vyberko, den 4, priklad 3
...}

{%%%%%   vyberko, den 4, priklad 4
...}

{%%%%%   vyberko, den 5, priklad 1
...}

{%%%%%   vyberko, den 5, priklad 2
...}

{%%%%%   vyberko, den 5, priklad 3
...}

{%%%%%   vyberko, den 5, priklad 4
...}

{%%%%%   trojstretnutie, priklad 1
...}

{%%%%%   trojstretnutie, priklad 2
...}

{%%%%%   trojstretnutie, priklad 3
...}

{%%%%%   trojstretnutie, priklad 4
...}

{%%%%%   trojstretnutie, priklad 5
...}

{%%%%%   trojstretnutie, priklad 6
...}

{%%%%%   IMO, priklad 1
...}

{%%%%%   IMO, priklad 2
...}

{%%%%%   IMO, priklad 3
...}

{%%%%%   IMO, priklad 4
...}

{%%%%%   IMO, priklad 5
...}

{%%%%%   IMO, priklad 6
...}

{%%%%%   MEMO, priklad 1
...}

{%%%%%   MEMO, priklad 2
...}

{%%%%%   MEMO, priklad 3
...}

{%%%%%   MEMO, priklad 4
...}

{%%%%%   MEMO, priklad t1
...}

{%%%%%   MEMO, priklad t2
...}

{%%%%%   MEMO, priklad t3
...}

{%%%%%   MEMO, priklad t4
...}

{%%%%%   MEMO, priklad t5
...}

{%%%%%   MEMO, priklad t6
...}

{%%%%%   MEMO, priklad t7
...}

{%%%%%   MEMO, priklad t8
...} 