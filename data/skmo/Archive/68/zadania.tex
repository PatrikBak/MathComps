{%%%%% A-I-1
O~postupnosti $(a_n)_{n=1}^{\infty}$ vieme, že pre
všetky prirodzené čísla $n$ platí
$$
a_{n+1}=\frac{a_n^2}{a_n^2-4a_n+6}.
$$
\ite a)
Nájdite všetky hodnoty $a_1$, pre ktoré je táto postupnosť
konštantná.
\ite b)
Nech $a_1=5$. Určte najväčšie celé číslo neprevyšujúce $a_{2018}$.}
\podpis{Vojtech Bálint}

{%%%%% A-I-2
Daný je ostrouhlý trojuholník $ABC$. Označme $D$ pätu
výšky z~vrcholu~$A$ a~$D_1$,~$D_2$ obrazy bodu~$D$ v~osových
súmernostiach postupne podľa priamok $AB$, $AC$. Ďalej označme $E_1$ a~$E_2$ body
na priamke~$BC$ také, že $D_1E_1 \parallel AB$ a~$D_2E_2 \parallel AC$.
Dokážte, že body $D_1$, $D_2$, $E_1$, $E_2$ ležia na jednej kružnici,
ktorej stred leží na kružnici opísanej trojuholníku $ABC$.}
\podpis{Patrik Bak}

{%%%%% A-I-3
Nájdite všetky nezáporné celé čísla $m$, $n$, pre ktoré platí
$|4m^2-n^{n+1}|\le 3$.}
\podpis{Tomáš Jurík}

{%%%%% A-I-4
Daná je množina~$\mn M$ prirodzených čísel s~$n$~prvkami, pričom
$n$~je nepárne číslo väčšie ako jedna.
Dokážte, že počet usporiadaných dvojíc $(p,q)$ rôznych prvkov
z~$\mn M$ takých, že aritmetický priemer čísel $p$, $q$ je prvkom~$\mn M$, je nanajvýš
$\frac12(n-1)^2$.}
\podpis{Martin Panák, Patrik Bak}

{%%%%% A-I-5
Zostrojte trojuholník $ABC$, ak poznáte jeho obvod~$o$,
polomer~$\rho$ kružnice pripísanej ku strane~$BC$ a~veľkosť výšky~$v$ na túto
stranu. Uveďte diskusiu v~závislosti od daných dĺžok.}
\podpis{Patrik Bak}

{%%%%% A-I-6
Na hracom pláne je nakreslený pravidelný $n$-uholník s~jedným
vrcholom vyznačeným ako pasca. Tom a~Jerry hrajú nasledujúcu hru.
Na začiatku Jerry postaví figúrku na niektorý vrchol $n$-uholníka.
V~každom kroku potom Tom povie nejaké prirodzené číslo a~Jerry
posunie figúrku o~tento počet vrcholov podľa svojej voľby buď v~smere, alebo proti smeru chodu hodinových ručičiek. Nájdite
všetky $n\ge3$, pri ktorých môže Jerry
ťahať figúrkou tak, aby nikdy neskončila v~pasci. Ako
sa zmení odpoveď, keď je Tom k~plánu otočený chrbtom, pozná iba dané~$n$
a~nevidí, kam Jerry figúrku na začiatku postaví ani kam s~ňou
v~jednotlivých krokoch ťahá?}
\podpis{Pavel Calábek}

{%%%%% B-I-1
Nájdite všetky osemciferné čísla také, z~ktorých po vyškrtnutí niektorej
štvorice susedných cifier dostaneme štvorciferné číslo,
ktoré je 2019-krát menšie.}
\podpis{Pavel Calábek}

{%%%%% B-I-2
V~trojuholníku~$ABC$ s~pravým uhlom pri vrchole~$A$ platí $|AB|=4$
a~${|AC|=3}$. Označme $M$ stred prepony~$BC$ a~$N$ priesečník osi
vnútorného uhla pri vrchole~$B$ s~odvesnou~$AC$. Úsečky $AM$ a~$BN$
sa pretínajú v~bode, ktorý označme~$K$. Vypočítajte pomer obsahov
trojuholníka~$BAK$ a~štvoruholníka $CNKM$.}
\podpis{Patrik Bak}

{%%%%% B-I-3
{\it Úpravou\/} prirodzeného čísla nazveme nasledujúcu operáciu:
ak je číslo párne, vydelíme ho dvoma; ak je nepárne, pripočítame
k~nemu číslo~1.
\ite a) Dokážte, že z~ľubovoľného prirodzeného čísla dostaneme po
niekoľkých úpravách číslo~1.
\ite b) Pre ktoré z~čísel $1, 2, \dots, 10^6$ budeme potrebovať
najväčší počet úprav, kým získame číslo~1?\endgraf}
\podpis{Ján Mazák}

{%%%%% B-I-4
Pre nezáporné reálne čísla $a$, $b$ platí $a^2+b^2=1$. Určte
najmenšiu aj najväčšiu možnú hodnotu výrazu
$$
V=\frac{a^4+b^4+ab+1}{a+b}.
$$}
\podpis{Patrik Bak, Jaromír Šimša}

{%%%%% B-I-5
Nech $ABCD$ je konvexný štvoruholník, v~ktorom $AD\perp BD$. Označme $M$
priesečník jeho uhlopriečok a~zostrojme kolmý priemet~$P$ bodu~$M$ na
priamku~$AB$ a~kolmý priemet~$Q$ bodu~$B$ na priamku~$AC$.
Dokážte, že bod~$M$ je stredom kružnice vpísanej trojuholníku~$PQD$.}
\podpis{Jaroslav Švrček, Jaromír Šimša}

{%%%%% B-I-6
Konečnú množinu prirodzených čísel nazveme {\it pekná},
ak na výpis týchto čísel v~desiatkovej sústave potrebujeme
párny počet každej zo zastúpených cifier. Peknými množinami sú
napríklad $\{11,13,31\}$, $\{10,100,110\}$ a~tiež prázdna
množina. Určte, koľko je všetkých pekných podmnožín množiny
$\{1,2,\dots,2\,018\}$.}
\podpis{Patrik Bak}

{%%%%% C-I-1
Neznáme číslo je deliteľné práve štyrmi číslami z~množiny $\{6,15,20,21,70\}$.
Určte, ktorými.}
\podpis{Michal Rolínek}

{%%%%% C-I-2
Na strane~$AB$ trojuholníka $ABC$ sú dané body $D$ a~$E$ tak, že $|AD|=|DE|=|EB|$.
Body $A$ a~$B$ sú postupne stredmi úsečiek $CF$ a~$CG$. Priamka~$CD$ pretína priamku~$FB$
v~bode~$I$ a~priamka~$CE$ pretína priamku~$AG$ v~bode~$J$. Dokážte, že priesečník
priamok $AI$ a~$BJ$ leží na priamke~$FG$.}
\podpis{Pavel Calábek}

{%%%%% C-I-3
Nech $a$, $b$, $c$ sú kladné reálne čísla, ktorých súčet je~3, a~každé z~nich
je nanajvýš~2. Dokážte, že platí nerovnosť
$$
\belowdisplayskip 0pt
a^2+b^2+c^2+3abc<9.
$$}
\podpis{Patrik Bak}

{%%%%% C-I-4
Každé políčko tabuľky $2\times 13$ ofarbíme práve jednou zo štyroch farieb.
Koľkými spôsobmi to možno spraviť tak, aby žiadne dve susedné políčka
neboli ofarbené rovnakou farbou? (Za susedné považujeme práve tie políčka tabuľky,
ktoré majú spoločnú stranu.)}
\podpis{Jaroslav Švrček}

{%%%%% C-I-5
Nech $\alpha$, $\beta$, $\gamma$, $\delta$, $\varepsilon$, $\varphi$, $\psi$,
$\omega$ sú postupne veľkosti vnútorných uhlov
pri vrcholoch $A$, $B$, $C$, $D$, $E$, $F$, $G$, $H$ konvexného osemuholníka $ABCDEFGH$,
v~ktorom platí
$$
\alpha +\beta =\gamma +\delta =\varepsilon +\varphi = \psi +\omega.
$$
Označme ďalej $K$, $L$, $M$, $N$ postupne stredy uhlopriečok $AD$, $CF$, $EH$, $GB$.
Dokážte, že priamky $KM$ a~$LN$ sú navzájom kolmé.}
\podpis{Josef Tkadlec}

{%%%%% C-I-6
Nájdite všetky trojciferné čísla~$n$ s~tromi rôznymi nenulovými ciframi,
ktoré sú deliteľné súčtom všetkých troch dvojciferných čísel,
ktoré dostaneme, keď v~pôvodnom čísle vyškrtneme vždy jednu cifru.}
\podpis{Jaromír Šimša}

{%%%%% A-S-1
Nájdite všetky prvočísla $p$, $q$ také, že rovnica
$x^2+px+q = 0$ má aspoň jeden celočíselný koreň.}
\podpis{Patrik Bak}

{%%%%% A-S-2
Daný je ostrouhlý trojuholník $ABC$, v~ktorom $|AB|<|AC|$.
Na polpriamkach $AB$, $AC$ ležia postupne body $D$, $E$ také, že $|AD| = |AC|$
a~$|AE| = |AB|$. Zostrojme v~bode~$D$ kolmicu na~$AD$, v~bode~$E$ kolmicu na~$AE$
a~ich priesečník označme~$F$. Dokážte, že $AF \perp BC$.}
\podpis{Patrik Bak}

{%%%%% A-S-3
{\it Úpravou\/} prirodzeného čísla nazveme nasledujúcu operáciu:
ak je číslo párne, vydelíme ho dvoma; ak je nepárne, pripočítame
k~nemu číslo~3. Určte všetky prirodzené čísla, z~ktorých
dostaneme po niekoľkých úpravách za sebou číslo~1.}
\podpis{Ján Mazák}

{%%%%% A-II-1
Dané je prirodzené číslo~$n$. Tom a~Jerry hrajú
proti sebe hru na pláne pozostávajúcom z~radu
2\,018~políčok. Na začiatku Jerry položí figúrku na nejaké políčko.
V~každom kroku potom Tom povie celé číslo z~intervalu $\langle 1, n\rangle$
a~Jerry posunie figúrku o~vyslovený počet políčok podľa svojej voľby buď doľava,
alebo doprava. Tom vyhráva, akonáhle Jerry nemá kam
spraviť ťah. Nájdite najmenšie~$n$, pre
ktoré Tom vždy dokáže voliť čísla tak, aby po konečnom počte krokov vyhral.}
\podpis{Josef Tkadlec}

{%%%%% A-II-2
Nájdite všetky celé čísla $m$ a~$n$, pre ktoré
platí $n^{n-1} = 4m^2+2m+3$.}
\podpis{Tomáš Jurík}

{%%%%% A-II-3
Daný je pravouhlý trojuholník $ABC$. Na jeho prepone~$BC$
ležia body $D$, $E$ také, že $|CD| = |CA|$, $|BE| = |BA|$. Nech $F$ je taký
vnútorný bod trojuholníka $ABC$, že $DEF$ je pravouhlý
rovnoramenný trojuholník s~preponou~$DE$. Aká je veľkosť uhla~$BFC$?}
\podpis{Patrik Bak}

{%%%%% A-II-4
Nájdite maximálnu hodnotu výrazu $a^2+b^2+c^2$ pre
reálne čísla $a$, $b$, $c$ také, že všetky tri čísla $a+b$,
$b+c$, $c+a$ sú z~intervalu $\langle 0,1 \rangle$.}
\podpis{Ján Mazák}

{%%%%% A-III-1
V~obore reálnych čísel vyriešte sústavu
$$
\belowdisplayskip0pt
\align
x^2-yz&=|y-z|+1, \\
y^2-zx&=|z-x|+1, \\
z^2-xy&=|x-y|+1.
\endalign
$$
}
\podpis{Tomáš Jurík}

{%%%%% A-III-2
Daný je pravouholník $ABCD$, pričom $|AB| = a~\ge b = |BC|$.
Na priamke~$BD$ zostrojte body $P$ a~$Q$ tak, aby platilo
$|AP| = |PQ| = |QC|$. Uveďte diskusiu riešiteľnosti vzhľadom na dĺžky
$a$, $b$.
}
\podpis{Jaroslav Švrček}

{%%%%% A-III-3
Nech $a$, $b$, $c$, $n$ sú kladné celé čísla také, že sú splnené
nasledujúce podmienky:
\ite(i) čísla $a$, $b$, $c$, $a+b+c$ sú po dvoch nesúdeliteľné;
\ite(ii) číslo $(a+b+c)(a+b)(b+c)(c+a)(ab+bc+ca)$ je $n$-tou
mocninou celého čísla.

Dokážte, že súčin $abc$ sa dá zapísať ako rozdiel dvoch $n$-tých
mocnín celých čísel.
}
\podpis{Patrik Bak}

{%%%%% A-III-4
Daný je ostrouhlý trojuholník $ABC$. Na polpriamke
opačnej k~polpriamke~$BC$ leží bod~$P$ taký, že $|AB| = |BP|$. Analogicky na
polpriamke opačnej k~polpriamke~$CB$ leží bod~$Q$ taký, že $|AC| = |CQ|$.
Označme~$J$ stred kružnice pripísanej strane~$BC$ daného trojuholníka
a~$D$, $E$ postupne jej body dotyku s~priamkami $AB$ a~$AC$. Predpokladajme, že polpriamky
opačné k~polpriamkam $DP$ a~$EQ$ sa pretínajú v~bode~$F$ rôznom od~$J$.
Dokážte, že $AF \perp FJ$.
}
\podpis{Patrik Bak}

{%%%%% A-III-5
Dokážte, že existuje nekonečne veľa celých
čísel, ktoré sa nedajú vyjadriť v~tvare ${2^a+3^b-5^c}$, pričom $a$, $b$, $c$
sú nezáporné celé čísla.}
\podpis{Ján Mazák, Tomáš Bárta}

{%%%%% A-III-6
Pre ktoré prirodzené čísla~$n$ možno do tabuľky $n \times n$ vpísať
všetky celé čísla od $1$ po $n^2$
tak, aby aritmetický priemer čísel v~každom riadku
aj stĺpci tabuľky bol celým číslom?}
\podpis{Laura Vištanová}

{%%%%% B-S-1
Na tabuli je napísané kladné celé číslo~$n$. V~jednom kroku smieme
číslo z~tabule zotrieť a~napísať namiesto neho buď jeho dvojnásobok,
alebo jeho dvojnásobok zväčšený o~1. Pre koľko počiatočných
čísel~$n$ rôznych od~2\,019 môžeme dosiahnuť to, že sa po
konečne veľa krokoch číslo 2\,019 na tabuli objaví?}
\podpis{Josef Tkadlec}

{%%%%% B-S-2
Nájdite všetky trojciferné čísla s~touto vlastnosťou:
ak vyškrtneme v~čísle jeho prostrednú cifru a~vzniknuté
dvojciferné číslo vynásobíme druhou mocninou vyškrtnutej cifry,
dostaneme opäť pôvodné trojciferné číslo.}
\podpis{Tomáš Jurík}

{%%%%% B-S-3
Daná je kružnica~$k$ a~jej priemer~$AB$. Vnútri úsečky~$AB$
zvolíme ľubovoľný bod~$C$ a~potom na kružnici~$k$ vyberieme bod~$D$
tak, aby platilo $|BC|=|BD|$. Os uhla $ABD$ pretína
kružnicu~$k$ v~bode~$E$ (rôznom od bodu~$B$).
Dokážte, že trojuholníky $AEC$ a~$CBD$ sú podobné.}
\podpis{Šárka Gergelitsová}

{%%%%% B-II-1
Pre nezáporné reálne čísla $a$, $b$ platí $a+b=2$. Určte
najmenšiu a~najväčšiu možnú hodnotu výrazu
$$
V=\frac{a^2+b^2}{ab+1}.
$$}
\podpis{Patrik Bak}

{%%%%% B-II-2
Nájdite všetky osemciferné čísla s~touto vlastnosťou:
keď vyškrtneme v~čísle jeho prvé dve a~jeho posledné dve cifry,
dostaneme štvorciferné číslo, ktoré je 2\,019-krát menšie ako číslo
pôvodné.}
\podpis{Pavel Calábek}

{%%%%% B-II-3
Daná je kružnica $k$ so stredom $S$ a~tetivou~$AB$, ktorá
nie je jej priemerom. Na polpriamke opačnej k~polpriamke~$BA$ je
vybraný ľubovoľný bod~$K$ rôzny od~$B$. Dokážte, že kružnica
opísaná trojuholníku $AKS$ pretína kružnicu~$k$ v~takom bode~$C$, ktorý
je súmerne združený s~bodom~$B$ podľa priamky~$SK$.}
\podpis{Šárka Gergelitsová}

{%%%%% B-II-4
Hovoríme, že množina kladných celých čísel je {\it štvorcová},
ak je neprázdna, konečná a~ak súčin všetkých jej prvkov
je druhou mocninou celého čísla. Dokážte, že množina
$\{1,2,3,\dots,20\}$ má práve $2^{12}-1$ štvorcových podmnožín.}
\podpis{Josef Tkadlec}

{%%%%% C-S-1
Aký je najväčší možný počet čísel, ktoré sa dajú vybrať z~množiny
$\{1,2,\dots,2019\}$ tak, aby súčin žiadnych troch z~vybraných čísel
nebol deliteľný deviatimi? Uveďte príklad vyhovujúcej podmnožiny
a~zdôvodnite, prečo nemôže mať väčší počet prvkov.}
\podpis{Aleš Kobza}

{%%%%% C-S-2
Pavol a~Michal hrajú nasledujúcu hru: Pri vrcholoch štvorstena
(trojbokého ihlana) sú zozačiatku napísané nuly. Pavol najskôr vyberie niektorý
vrchol štvorstena a~zväčší číslo pri ňom napísané o~2. Potom Michal vyberie
niektorú hranu tohto štvorstena a~zväčší čísla v~oboch jej krajných
bodoch o~1. Ich \uv{ťahy} sa pravidelne striedajú. Vyhráva ten, po
ktorého ťahu budú vo všetkých vrcholoch štvorstena navzájom rôzne
čísla. Ktorý z~hráčov si dokáže zabezpečiť výhru?}
\podpis{Tomáš Jurík}

{%%%%% C-S-3
Nech $D$, $E$ sú postupne stredy strán $AB$, $BC$ trojuholníka $ABC$
a~$F$ je stred úsečky~$AD$. Dokážte, že priamka~$CD$ rozpoľuje úsečku~$EF$.}
\podpis{Jaroslav Švrček}

{%%%%% C-II-1
Každé políčko tabuľky $68 \times 68$ máme ofarbiť jednou z~troch farieb
(červená, modrá, biela). Koľkými spôsobmi to možno spraviť tak,
aby každá trojica susedných políčok v~každom riadku a~v~každom stĺpci
obsahovala políčka všetkých troch farieb?}
\podpis{Josef Tkadlec}

{%%%%% C-II-2
Aký je najmenší možný súčet štyroch prirodzených čísel takých, že
dvojice vytvorené z~týchto čísel majú najväčšie
spoločné delitele 2, 3, 4, 5, 6 a~9?
Uveďte príklad vyhovujúcej štvorice s~takým súčtom a~zdôvodnite, prečo
neexistuje vyhovujúca štvorica s~menším súčtom.}
\podpis{Tomáš Jurík}

{%%%%% C-II-3
Na strane $AB$ rovnobežníka $ABCD$, v~ktorom $|AB|=1$, sú
zvolené body $K$ a~$L$ tak,~že $|BK|=\frac12$,
$|BL|=\frac13$. Na strane~$CD$ sú
zvolené body $P$ a~$Q$ tak, že ${|CP|=\frac12}$ a~$|CQ|=\frac13$.
Priesečník priamok $LD$ a~$KP$ označme $X$, priesečník
priamok $BD$ a~$LQ$ označme $Y$.
Dokážte, že priamka~$XY$ rozpoľuje stranu~$BC$.}
\podpis{Jaroslav Zhouf}

{%%%%% C-II-4
Reálne čísla $a$, $b$, $c$, všetky väčšie ako $\frac12$, spĺňajú
podmienku $ab+bc+ca=\frac54$. Dokážte, že platí
$$
\belowdisplayshortskip0pt
a+b+c>a^2+b^2+c^2.
$$}
\podpis{Patrik Bak}

{%%%%%   vyberko, den 1, priklad 1
[A0] Sú dané reálne čísla $a_1,a_2,a_3,a_4,a_5$ také, že každé dve sa líšia nanajvýš o~1. Nech $s$ je súčet týchto čísel. Predpokladajme, že súčet druhých mocnín týchto čísel je rovný~$2s^2$. Dokážte, že $s^2 \ge \frac{25}{3}$.
}
\podpis{...}

{%%%%%   vyberko, den 1, priklad 2
[C0] Budova má 7~výťahov a~každý z~nich zastavuje na~nejakých 6~poschodiach. Pre každé dve poschodia existuje výťah, ktorý ich priamo spája. Koľko najviac poschodí môže mať táto budova?
}
\podpis{...}

{%%%%%   vyberko, den 1, priklad 3
[G0] V~trojuholníku~$ABC$ platí $|AB|=|AC|$. Kružnica~$\Gamma$ leží zvonku trojuholníka $ABC$ a~dotýka sa priamky~$AC$ v~bode~$C$. Bod~$D$ leží na~$\Gamma$ tak, že kružnice $ABD$ a~$\Gamma$ majú vnútorný dotyk. Úsečka~$AD$ pretína~$\Gamma$ v~bode~$E \ne D$. Dokážte, že priamka~$BE$ sa dotýka~$\Gamma$.}
\podpis{...}

{%%%%%   vyberko, den 1, priklad 4
[N0] Nájdite všetky dvojice $(p,q)$ prvočísel, pre ktoré je číslo $p^{q-1}+q^{p-1}$ druhá mocnina celého čísla.
}
\podpis{...}

{%%%%%   vyberko, den 2, priklad 1
[N1] Nájdite všetky dvojice $(n,k)$ kladných celých čísel, pre ktoré existuje kladné celé číslo~$s$ také, že čísla $sn$ a~$sk$ majú rovnaký počet deliteľov.
}
\podpis{...}

{%%%%%   vyberko, den 2, priklad 2
[A2] Dokážte, že ľubovoľný polynóm stupňa~$n$ s~vedúcim koeficientom~1 s~reálnymi koeficientami sa dá napísať ako aritmetický priemer dvoch polynómov stupňa $n$ s~vedúcim koeficientom~1 takých, že každý z~nich má $n$ reálnych koreňov.
~}
\podpis{...}

{%%%%%   vyberko, den 2, priklad 3
[C3] Nech $a,b$ sú rôzne kladné celé čísla. Na~spočiatku prázdnej tabuli prebieha nasledovný nekonečný proces.
\ite{(i)} Ak je na~tabuli aspoň jedna dvojica rovnakých čísel, tak si zvolíme nejakú takú dvojicu a~zvýšime jeden z~jej prvkov o~$a$ a~druhý o~$b$.
\ite{(ii)} Ak taká dvojica neexistuje, napíšeme na tabuľu dve nuly. 

Dokážte, že bez ohľadu na to, ako robíme ťahy~(i), operácia (ii) bude vykonaná iba konečne veľa krát.
}
\podpis{...}

{%%%%%   vyberko, den 3, priklad 1
[A1] Nech $\Bbb Q^{+}$ označuje množinu kladných racionálnych čísel. Nájdite všetky funkcie ${f\colon \Bbb Q^{+} \to \Bbb Q^{+}}$ spĺňajúce
$$
f\left(x^2f^2(y)\right)=f^2(x) f(y)
$$
pre všetky $x,y \in \Bbb Q^{+}$.
}
\podpis{...}

{%%%%%   vyberko, den 3, priklad 2
[C2] Je dané kladné celé číslo $n$ a~tabuľka $n \times n$. Každé políčko tabuľky je buď prázdne, alebo sa na ňom nachádza celé číslo. Platí, že v~každom riadku a~v~každom stĺpci sú navzájom rôzne čísla. Dokážte, že políčka s~číslami vieme ofarbiť namodro tak, že:
\ite{(1)} V~každom riadku a~v~každom stĺpci sa nachádza nanajvýš jedno modré políčko.
\ite{(2)} Pre každé nezafarbené políčko s~číslom existuje modré políčko v~rovnakom riadku s~väčším číslom alebo modré políčko v~rovnakom stĺpci s~menším číslom.\endgraf
}
\podpis{...}

{%%%%%   vyberko, den 3, priklad 3
[G3] Je daný trojuholník $ABC$ a~nejaký jeho vnútorný bod~$T$. Nech $A_1$, $B_1$, $C_1$ sú obrazy bodu~$T$ v~osových súmernostiach postupne podľa priamok $BC$, $CA$, $AB$. Priamky $A_1T$, $B_1T$, $C_1T$ pretínajú kružnicu~$\Gamma$ opísanú trojuholníku $A_1B_1C_1$ postupne v~bodoch $A_2$, $B_2,$ $C_2$. Dokážte, že priamky $AA_2$, $BB_2$, $CC_2$ sa pretínajú na~$\Gamma$.
}
\podpis{...}

{%%%%%   vyberko, den 4, priklad 1
[G1] V~trojuholníku~$ABC$ platí $|AB|=|AC|$. Označme $M$ stred~$BC$. Nech $P$ je taký bod, že $|PB|<|PC|$ a~$PA \parallel BC$. Body $X,Y$ ležia postupne na~priamkach $PB$ a~$PC$ tak, že $B$ leží na~úsečke~$PX$, $C$ leží na~úsečke~$PY$, a~$|\angle PXM|=|\angle PYM|$. Dokážte, že štvoruholník~$APXY$ je tetivový.
}
\podpis{...}

{%%%%%   vyberko, den 4, priklad 2
[N2] Nájdite všetky celé čísla $n \ge 2$ také, že pre všetky dvojice $(i,j)$ celých čísel spĺňajúcich $0 \leq i,j \leq n$ platí, že číslo $i+j$ má rovnakú paritu ako
$$
{n \choose i} + {n \choose j}.
$$
}
\podpis{...}

{%%%%%   vyberko, den 4, priklad 3
[A3] Nech $a_0,a_1,a_2,\ldots$ je postupnosť reálnych čísel takých, že $a_0=0$, $a_1=1$, a~pre každé $n \ge 2$ existuje $1 \leq k \leq n$ spĺňajúce
$$
a_n = \frac{a_{n-1}+\cdots+a_{n-k}}{k}.
$$
Určte najväčšiu možnú hodnotu rozdielu $a_{2018}-a_{2017}$.
}
\podpis{...}

{%%%%%   vyberko, den 5, priklad 1
[C1] Je dané prirodzené číslo $n \ge 3$. Dokážte, že existuje $2n$-prvková množina~$\mn S$ kladných celých čísel s~nasledovnou vlastnosťou: Pre každé $m \in \{2,3,\ldots,n\}$ sa množina~$\mn S$ dá rozdeliť na dve~podmnožiny s~rovnakým súčtom prvkov, pričom počet prvkov jednej z~nich je rovný~$m$.
}
\podpis{...}

{%%%%%   vyberko, den 5, priklad 2
[G2] Je daný rovnobežník $ABCD$ taký, že $|\angle DAB|>90^\circ$. Nech $H$ je bod priamky~$AD$ taký, že $BH \perp AD$. Označme $M$ stred~$AB$. Ďalej nech $K \ne D$ je priesečník priamky~$DM$ s~kružnicou~$ADB$. Dokážte, že štvoruholník~$HKCD$ je tetivový.
}
\podpis{...}

{%%%%%   vyberko, den 5, priklad 3
[N3] Nech $f:\{1,2,3,\ldots\} \to \{2,3,\ldots\}$ je funkcia taká, že $f(m+n)\mid f(m)+f(n)$ pre všetky dvojice $(m,n)$ kladných celých čísel. Dokážte, že existuje kladné celé číslo $c>1$, ktoré delí všetky hodnoty~$f$.
}
\podpis{...}

{%%%%%   vyberko, den , priklad
...}
\podpis{...}

{%%%%%   vyberko, den , priklad
...}
\podpis{...}

{%%%%%   vyberko, den , priklad
...}
\podpis{...}

{%%%%%   vyberko, den , priklad
...}
\podpis{...}

{%%%%%   trojstretnutie, priklad 1
Je daná kružnica $\omega$. Body $A$, $B$, $C$, $X$, $D$, $Y$ ležia na $\omega$ v~tomto poradí tak, že $BD$ je jej priemer a~$|DX|=|DY|=|DP|$, kde $P$ je priesečník $AC$ a~$BD$.
Označme $E$, $F$ postupne priesečníky priamky $XP$ s~priamkami $AB$, $BC$. Dokážte, že body $B$, $E$, $F$,~$Y$ ležia na jednej kružnici.}
\podpis{Patrik Bak, Slovensko}

{%%%%%   trojstretnutie, priklad 2
Uvažujme kladné celé čísla $n$, ktoré majú aspoň šesť kladných deliteľov. Usporiadajme kladné delitele $n$ do postupnosti $(d_i)_{1\le i\le k}$ tak, že
$$
1=d_1 <d_2 <\dots <d_k =n \quad (k\ge 6).
$$
Nájdite všetky kladné celé čísla $n$ spĺňajúce
$$
n = d_5^2+d_6^2.
$$}
\podpis{Walther Janous, Rakúsko}

{%%%%%   trojstretnutie, priklad 3
Povieme, že rozdelenie konvexného mnohouholníka na konečne veľa trojuholníkov pomocou úsečiek je {\it trilaterácia}, ak žiadne tri vrcholy týchto trojuholníkov neležia na jednej priamke (vrcholy niektorých týchto trojuholníkov môžu ležať vnútri mnohouholníka). Povieme, že trilaterácia je {\it dobrá}, ak sa jej úsečky dajú nahradiť jednosmernými šípkami tak, aby šípky pozdĺž každého trojuholníka tvorili cyklus a~šípky pozdĺž celého konvexného mnohouholníka taktiež tvorili cyklus. Nájdite všetky $n \ge 3$, pre ktoré má pravidelný $n$-uholník dobrú trilateráciu.}
\podpis{Josef Greilhuber, Rakúsko}

{%%%%%   trojstretnutie, priklad 4
Je dané reálne číslo $\alpha$. Nájdite všetky dvojice $(f,g)$ funkcií $f,g\colon\Bbb{R}\to\Bbb{R}$ spĺňajúcich
$$
xf(x + y) + \alpha\cdot yf(x - y) = g(x) + g(y)
$$
pre všetky $x,y\in \Bbb{R}$.}
\podpis{Walther Janous, Rakúsko}

{%%%%%   trojstretnutie, priklad 5
Rozhodnite, či sa dá v~rovine rozmiestniť 100 kruhov $D_2, D_3,\dots,D_{101}$ tak, že nasledujúce podmienky sú splnené pre všetky dvojice indexov $(a,b)$ spĺňajúcich ${2\le a<b\le 101}$:
\ite{(i)} Ak $a\mid b$, potom $D_a$ je obsiahnutý v $D_b$.
\ite{(ii)} Ak $\nsd(a,b)=1$, potom $D_a$ a $D_b$ sú disjunktné.

(Kruh $D(O,r)$ je množina bodov v rovine, ktorých vzdialenosť od daného bodu $O$ neprevyšuje dané kladné reálne číslo $r$.)}
\podpis{Josef Greilhuber, Rakúsko, Josef Tkadlec, ČR}

{%%%%%   trojstretnutie, priklad 6
Je daný ostrouhlý trojuholník $ABC$ spĺňajúci $|AB|<|AC|$ a $|\angle BAC|=60^\circ$.
Označme $AD$, $BE$, $CF$ jeho výšky a $H$ jeho priesečník výšok.
Ďalej označme $K$, $L$, $M$ postupne stredy strán $BC$, $CA$, $AB$.
Dokážte, že stredy úsečiek $AH$, $DK$, $EL$, $FM$ ležia na jednej kružnici.}
\podpis{Dominik Burek, Poľsko}

{%%%%%   IMO, priklad 1
Nájdite všetky funkcie $f$ také,
že $f:\Bbb{Z}\to\Bbb{Z}$
a pre každé celé čísla $a$ a $b$ platí
$$f(2a)+2f(b)=f(f(a+b)).$$}
\podpis{Južná Afrika}

{%%%%%   IMO, priklad 2
Nech v trojuholníku $ABC$
je $A_1$ bod na strane $BC$
a $B_1$ bod na strane $AC$.
Nech $P$ a~$Q$ sú body postupne na úsečkách $AA_1$ a $BB_1$ také,
že $PQ$ je rovnobežná s $AB$.
Nech $P_1$ je bod na priamke $PB_1$ taký,
že $B_1$ leží vnútri úsečky $PP_1$
a platí
$|\uhol PP_1C|=|\uhol BAC|$.
Nech $Q_1$ je bod na priamke $QA_1$ taký,
že $A_1$ leží vnútri úsečky $QQ_1$
a platí
$|\uhol QQ_1C|=|\uhol ABC|$.
Dokážte,
že body $P$, $Q$, $P_1$ a $Q_1$
ležia na tej istej kružnici.}
\podpis{Ukrajina}

{%%%%%   IMO, priklad 3
Sociálna sieť má 2019 používateľov.
Niektoré dvojice z nich sú priatelia,
pričom ak používateľ~$A$ je priateľ používateľa $B$,
tak používateľ $B$ je priateľ používateľa $A$.
Opakovane môže nastať takáto udalosť:

Traja (rôzni) používatelia $A$, $B$ a $C$ sú takí,
že $A$ sa priatelí s $B$ aj s $C$,
ale $B$ a $C$ nie sú priatelia.
V istom okamihu sa $B$ a $C$ spriatelia
a zároveň sa $A$ s $B$ aj s $C$ priateliť prestane.
Všetky ostatné vzťahy pritom zostanú zachované.

Na začiatku je 1010 používateľov s 1009 priateľmi
a 1009 používateľov s 1010 priateľmi.
Dokážte, že existuje postupnosť takýchto udalostí,
po ktorých sa každý používateľ bude priateliť najviac s~jedným iným používateľom.
}
\podpis{Chorvátsko}

{%%%%%   IMO, priklad 4
Nájdite všetky dvojice $(k,n)$ kladných celých čísel
takých, že
$$k!=(2^n-1)(2^n-2)(2^n-4)\cdots(2^n-2^{n-1}).$$
}
\podpis{Salvádor}

{%%%%%   IMO, priklad 5
Banka v Bathe používa mince,
ktoré majú na jednej strane písmeno~$\text{\tt H}$
a na druhej písmeno~$\text{\tt T}$.
Harry má $n$ takýchto mincí usporiadaných vedľa seba zľava doprava.
Opakovane vykonáva takýto krok:
Ak $k>0$
a existuje práve $k$ mincí, na ktorých vidno $\text{\tt H}$,
tak obráti $k$.\,mincu zľava,
a ak na všetkých minciach vidno $\text{\tt T}$,
tak tento postup ukončí.
(Napríklad ak $n=3$
a počiatočná konfigurácia je
$\text{\tt T}\text{\tt H}\text{\tt T}$,
Harryho postup je
$\text{\tt T}\text{\tt H}\text{\tt T}
\to
\text{\tt H}\text{\tt H}\text{\tt T}
\to
\text{\tt H}\text{\tt T}\text{\tt T}
\to
\text{\tt T}\text{\tt T}\text{\tt T}$,
skončí sa teda po troch krokoch.)
\item{a)} Ukážte, že pre každú počiatočnú konfiguráciu
sa Harryho proces skončí po konečnom počte krokov.
\item{b)} Nech pre každú počiatočnú konfiguráciu $C$
označuje $L(C)$ počet krokov potrebných na to,
aby sa Harryho proces skončil. (Napríklad $L(\text{\tt L}\text{\tt H}\text{\tt L})=3$
a $L(\text{\tt L}\text{\tt L}\text{\tt L})=0$.) 
Určte aritmetický priemer hodnôt $L(C)$ pre všetkých $2^n$ počiatočných konfigurácií $C$.\endgraf
}
\podpis{USA}

{%%%%%   IMO, priklad 6
Nech v ostrouhlom trojuholníku $ABC$ platí
$|AB|\ne|AC|$.
Kružnica $\omega$ vpísaná do $ABC$ má stred $I$
a dotýka sa jeho strán $BC$, $CA$, $AB$ postupne v bodoch $D$, $E$, $F$.
Priamka kolmá na $EF$ prechádzajúca bodom $D$ pretína $\omega$ v bode $R$ rôznom od $D$.
Priamka $AR$ pretína $\omega$ v~bode~$P$ rôznom od~$R$.
Kružnice opísané trojuholníkom $PCE$ a $PBF$ sa pretínajú v~bode~$Q$ rôznom od~$P$.
Dokážte, že priesečník priamok $DI$ a $PQ$
leží na priamke prechádzajúcej bodom~$A$ a~kolmej na $AI$.}
\podpis{India}

{%%%%%   MEMO, priklad 1
Nájdite všetky funkcie $f\colon \Bbb R\to\Bbb R$, spĺňajúce
$$
f\bigl(xf(y)+2y\bigr) = f(xy) + xf(y) + f\bigl(f(y)\bigr)
$$
pre všetky reálne $x$ a $y$.}
\podpis{Patrik Bak, Slovensko}

{%%%%%   MEMO, priklad 2
Nech $n\ge 3$ je prirodzené číslo. Vrchol $A_i$ ($1\le i\le n$) konvexného $n$-uholníka $A_1A_2\ldots A_n$ nazývame {\it bohémsky}, ak jeho obraz v~stredovej súmernosti podľa stredu úsečky $A_{i-1}A_{i+1}$ ($A_0=A_n$ a~$A_{n+1}=A_1$) leží vo vnútri alebo na hranici $n$-uholníka $A_1A_2\ldots A_n$. Určte najmenší možný počet bohémskych vrcholov konvexného $n$-uholníka (v~závislosti od~$n$).}
\podpis{Maďarsko}

{%%%%%   MEMO, priklad 3
Označme $ABC$ ostrouhlý trojuholník, pre ktorý platí $|AC|>|BC|$ a~ktorému opísaná kružnica je označená~$\omega$. Nech bod~$P$ leží na kratšom oblúku~$BC$ kružnice~$\omega$, pričom platí $|AP|=|AC|$. Priesečník priamok $AP$ a~$BC$ označme $Q$. Ďalej predpokladajme, že $R$ je bod ležiaci na kratšom oblúku~$AC$ kružnice~$\omega$, pre ktorý platí $|QA|=|QR|$. Napokon priesečník priamky~$BC$ s~osou úsečky~$AB$ označme~$S$.
Dokážte, že body $P$, $Q$, $R$ a~$S$ ležia na jednej kružnici.}
\podpis{Patrik Bak, Slovensko}

{%%%%%   MEMO, priklad 4
Nájdite najmenšie prirodzené $n$ také, že z~ľubovoľných $n$ po sebe idúcich celých čísel je možné vybrať neprázdnu množinu po sebe idúcich čísel, ktorých súčet je deliteľný číslom~2019.
}
\podpis{Poľsko}

{%%%%%   MEMO, priklad t1
Určte najmenšiu a~najväčšiu možnú hodnotu výrazu
$$
\left(\frac{1}{a^2+1}+\frac{1}{b^2+1}+\frac{1}{c^2+1}\right)\left(\frac{a^2}{a^2+1}+\frac{b^2}{b^2+1}+\frac{c^2}{c^2+1}\right),
$$
kde $a$, $b$ a~$c$ sú nezáporné reálne čísla spĺňajúce podmienku $ab+bc+ca=1$.
}
\podpis{Rakúsko}

{%%%%%   MEMO, priklad t2
Nech $\alpha$ je ľubovoľné reálne číslo. Nájdite všetky polynómy~$P$ s~reálnymi koeficientmi také, že pre všetky reálne čísla~$x$ platí
$$
P(2x+\alpha)\le \left(x^{20}+x^{19}\right)P(x).
$$}
\podpis{Rakúsko}

{%%%%%   MEMO, priklad t3
V~triede je $n$~chlapcov a~$n$~dievčat, kde $n$ je prirodzené číslo. Výšky všetkých žiakov v~triede sú po dvoch rôzne. Každé dievča odčíta od počtu chlapcov vyšších ako ona počet dievčat
vyšších ako ona a~výsledok napíše na kúsok papiera. Každý chlapec odčíta od počtu dievčat nižších ako on počet chlapcov nižších ako on a~výsledok napíše na kúsok papiera. Dokážte,
že čísla, ktoré napísali dievčatá, sú (až na poradie) tie isté ako čísla, ktoré napísali chlapci.}
\podpis{Rakúsko}

{%%%%%   MEMO, priklad t4
Dokážte, že každé celé číslo od $1$ po $2019$ sa dá zapísať ako aritmetický výraz obsahujúci najviac 17 symbolov $2$ a~ľubovoľný počet sčítaní, odčítaní, násobení, delení a~zátvoriek.
Symboly $2$ nemôžu byť použité na nijakú inú operáciu, napríklad skladanie viacciferných čísel (ako $222$) alebo mocnín (ako $2^2$).

Príklady povolených výrazov:
$$\left((2 \times 2+2) \times 2-\frac{2}{2}\right) \times 2=22,\quad (2 \times 2 \times 2-2) \times \left(2 \times 2+\frac{2+2+2}{2}\right)=42.$$}
\podpis{Rakúsko}

{%%%%%   MEMO, priklad t5
Nech $ABC$ je ostrouhlý trojuholník spĺňajúci $|AB|<|AC|$. Nech $D$ je bod prieniku osi úsečky $BC$ a~strany~$AC$.
Nech $P$ je bod na kratšom oblúku~$AC$ kružnice opísanej trojuholníku $ABC$ taký, že $DP\parallel BC$. Nech je ďalej $M$ stred strany~$AB$. Dokážte, že $|\angle APD|=|\angle MPB|$.}
\podpis{Poľsko}

{%%%%%   MEMO, priklad t6
Nech $ABC$ je pravouhlý trojuholník s~pravým uhlom pri vrchole~$B$ a~opísanou kružnicou~$c$. Označme $D$ stredný bod kratšieho oblúka~$AB$ kružnice~$c$. Nech $P$ je bod na strane~$AB$ taký, že platí $|CP|=|CD|$ a~nech sú $X$ a~$Y$ dva rozličné body na $c$ spĺňajúce $|AX|=|AY|=|PD|$. Dokážte, že body $X$, $Y$ a~$P$ ležia na jednej priamke.}
\podpis{Poľsko}

{%%%%%   MEMO, priklad t7
Nech $a$, $b$, $c$ sú kladné celé čísla spĺňajúce $a<b<c<a+b$. Dokážte, že
$c(a-1)+b$ nedelí $c(b-1)+a$.}
\podpis{Švajčiarsko}

{%%%%%   MEMO, priklad t8
Nech $N$ je kladné celé číslo také, že súčet druhých mocnín jeho kladných deliteľov sa rovná číslu $N(N+3)$.
Dokážte, že potom existujú indexy $i$ a~$j$ také, že
$N=F_i\cdot F_j$, kde $(F_n)_{n=1}^{\infty}$ je Fibonacciho postupnosť
definovaná vzťahom $F_1 = F_2 = 1$ a $F_n = F_{n-1} + F_{n-2}$ pre všetky
$n\ge3$.}
\podpis{Poľsko}

{%%%%%   CPSJ, priklad 1
Nájdite všetky dvojice kladných celých čísel $a$, $b$, pre ktoré platí $$\sqrt{a + 2\sqrt{b}} = \sqrt{a-2\sqrt{b}} + \sqrt{b}.$$}
\podpis{Patrik Bak, Slovensko}

{%%%%%   CPSJ, priklad 2
Daný je trojuholník~$ABC$ s~ťažiskom~$T$. Označme $M$ stred strany~$BC$. Na polpriamke opačnej k~$BA$ leží bod~$D$ taký, že $|AB|=|BD|$. Podobne na polpriamke opačnej k~$CA$ leží bod~$E$ taký, že $|AC|=|CE|$. Úsečky $TD$, $TE$ pretínajú stranu~$BC$ postupne v~bodoch $P$, $Q$. Dokážte, že body $P$, $Q$, $M$ rozdeľujú úsečku~$BC$ na~štyri rovnako dlhé časti.}
\podpis{Patrik Bak, Slovensko}

{%%%%%   CPSJ, priklad 3
Pre ktoré prirodzené čísla $n$ sa dá tabuľka $n\times n$ vyplniť číslami $1$, $2$ a~$-3$ tak, aby súčet čísel v~každom riadku aj v~každom stĺpci bol $0$?}
\podpis{Ján Mazák, Slovensko}

{%%%%%   CPSJ, priklad 4
Daná je kružnica $k$ a~jej priemer $AB$. Vnútri úsečky~$AB$
zvolíme ľubovoľný bod $C$ a~potom na kružnici~$k$ vyberieme bod~$D$
tak, aby vznikol ostrouhlý trojuholník $BCD$, ktorého opísaná kružnica má
stred v~bode, ktorý označíme $O$. Nech ešte $E$ označuje priesečník
kružnice $k$ s~priamkou $BO$ (rôzny od bodu~$B$).
Dokážte, že trojuholníky $BCD$ a~$ECA$ sú podobné.}
\podpis{Jaromír Šimša, Česká rep.}

{%%%%%   CPSJ, priklad 5
V~skupine osôb má každý práve $d$ známych a~každí dvaja, ktorí sa nepoznajú, majú práve jedného spoločného priateľa. Dokážte, že počet osôb v~tejto skupine nie je väčší ako $d^2+1$.}
\podpis{\L{}ukasz Bożyk, Poľsko}

{%%%%%   CPSJ, priklad t1
Jsou dána racionální čísla $a$, $b$ taková, že čísla $a+b$ a $a^2+b^2$ jsou celá. Dokažte, že čísla $a$, $b$ jsou celá.}
\podpis{\L{}ukasz Bożyk, Poľsko}

{%%%%%   CPSJ, priklad t2
Šachová figurka {\it nemocná věž} se pohybuje po řádcích a sloupcích jako obyčejná věž, ale nejvýše o 2 políčka. Nemocné věže budeme
na čtvercovou šachovnici rozmísťovat tak, aby se žádné dvě z nich navzájem neohrožovaly a aby neexistovalo žádné pole ohrožené více než jednou věží.
\ite a) Dokažte, že na šachovnici $30\times 30$ takto nelze rozmístit více než 100 nemocných věží.
\ite b) Určete největší možný počet nemocných věží, které lze takto rozmístit na šachovnici~$8\times 8$.
\ite c) Dokažte, že na šachovnici $32\times 32$ takto nelze rozmístit více než 120 nemocných věží.\endgraf\nopagebreak}
\podpis{Ján Mazák, Slovensko}

{%%%%%   CPSJ, priklad t3
Niech $ABCD$ b\ę{}dzie czworok\ą{}tem wypuk\l{}ym o~prostopad\l{}ych przek\ą{}tnych, w~którym $\angle BAC=\angle ADB$, $\angle CBD=\angle DCA$, $AB=15$, $CD=8$. Wykaż, że na czworok\ą{}cie $ABCD$ można opisać okr\ą{}g oraz wyznacz odleg\l{}ość mi\ę{}dzy środkiem tego okr\ę{}gu a~punktem przeci\ę{}cia przek\ą{}tnych czworok\ą{}ta. }
\podpis{Ján Mazák, Slovensko}

{%%%%%   CPSJ, priklad t4
Wyznacz wszystkie możliwe wartości wyrażenia $xy+yz+zx$, gdzie liczby rzeczywiste $x$, $y$, $z$ spe\l{}niaj\ą{} warunki $$x^2-yz=y^2-zx=z^2-xy=2.$$}
\podpis{Jaroslav Švrček, Česká rep.}

{%%%%%   CPSJ, priklad t5
Daný je pravidelný 360-uholník $A_1A_2\dots A_{360}$ so stredom~$S$. Pre každý z~trojuholníkov $A_1A_{50}A_{68}$ a $A_1A_{50}A_{69}$
rozhodnite, či jeho obrazy v~120 vhodných otočeniach so stredom~$S$ môžu mať (ako trojuholníky) za vrcholy všetkých 360 bodov~$A_1,A_2,\dots,A_{360}$.}
\podpis{Jaromír Šimša, Česká rep.}

{%%%%%   CPSJ, priklad t6
Daný je tetivový štvoruholník $ABCD$. Na stranách $AB$, $BC$, $CD$, $DA$ ležia postupne body $K$, $L$, $M$, $N$, pričom
$$
\begin{alignat*}{2}
|\measuredangle ADK|&=|\measuredangle BCK|,&\qquad|\measuredangle BAL|&=|\measuredangle CDL|,\\
|\measuredangle CBM|&=|\measuredangle DAM|,&\qquad|\measuredangle DCN|&=|\measuredangle ABN|.
\end{alignat*}
$$
Dokážte, že priamky $KM$ a~$LN$ sú na seba kolmé.}
\podpis{\L{}ukasz Bożyk, Tomasz Przyby\l{}owski, Poľsko}

{%%%%%   EGMO, priklad 1
Nájdite všetky trojice $(a, b, c)$ reálnych čísel také, že platí $ab + bc + ca = 1$ a~súčasne
$$
a^2b + c = b^2c + a = c^2a + b.
\belowdisplayskip0pt
$$}
\podpis{Holandsko}

{%%%%%   EGMO, priklad 2
Je dané kladné celé číslo~$n$. Na štvorcovú tabuľku~$2n\times 2n$ sú umiestnené dominá tak,
že každé políčko tejto tabuľky je susedné s práve jedným políčkom pokrytým dominom. Pre každé~$n$
určte najväčší počet domín, ktoré môžeme takto umiestniť na túto tabuľku.

\poznamka
\emph{Dominom} rozumieme obdĺžnik~$2 \times 1$ alebo $1 \times 2$. Dominá sú umiestňované na tabuľku tak, že
každé domino pokrýva práve dve políčka tabuľky a jednotlivé dominá sa neprekrývajú. Dve políčka
tabuľky sú \emph{susedné} práve vtedy, keď sú rôzne a majú spoločnú stranu.}
\podpis{Luxembursko}

{%%%%%   EGMO, priklad 3
Je daný trojuholník~$ABC$ taký, že $|\angle CAB| > |\angle ABC|$. Označme~$I$ stred kružnice jemu
vpísanej. Nech~$D$ je bod úsečky~$BC$, pre ktorý platí $|\angle CAD|= |\angle ABC|$. Označme~$\omega$ kružnicu, ktorá sa dotýka priamky~$AC$ v~bode~$A$ a~prechádza bodom~$I$. Nech~$X$ ($X \ne A$) je priesečník kružnice~$\omega$ a~kružnice opísanej trojuholníku~$ABC$. Dokážte, že osi uhlov~$DAB$ a~$CXB$ sa pretínajú na~priamke~$BC$.}
\podpis{Poľsko}

{%%%%%   EGMO, priklad 4
Nech $I$ je stred kružnice $\omega$ vpísanej trojuholníku $ABC$. Kružnica prechádzajúca bodom~$B$,
ktorá sa dotýka $AI$ v bode $I$, pretína stranu $AB$ v bode $P$ ($P \ne B$). Analogicky, kružnica prechádzajúca bodom $C$, ktorá sa dotýka $AI$ v bode $I$, pretína stranu $AC$ v bode $Q$ ($Q \ne C$). Dokážte, že
priamka $PQ$ je dotyčnica kružnice~$\omega$.}
\podpis{Poľsko}

{%%%%%   EGMO, priklad 5
Nech $n \geq 2$ je celé číslo a $a_1, a_2, \ldots , a_n$ sú kladné celé čísla. Dokážte, že existujú kladné
celé čísla $b_1, b_2, \ldots , b_n$ spĺňajúce nasledujúce tri podmienky:
\item{(A)} $a_i \leq b_i$ pre $i = 1, 2, \ldots , n$;
\item{(B)} zvyšky po delení čísel $b_1, b_2, \ldots , b_n$ číslom $n$ sú po dvoch rôzne; a
\item{(C)} $\displaystyle b_1 + \ldots + b_n \leq n\left(\frac{n-1}2 + \left\lfloor \frac{a_1+\ldots+a_n}{n}\right\rfloor\right).$
}
\podpis{Holandsko}

{%%%%%   EGMO, priklad 6
Alena nakreslila na kružnici 2019 tetív s navzájom rôznymi krajnými bodmi. Bod nazveme
označený, keď ide o
\item{(i)} jeden z 4038 krajných bodov týchto tetív; alebo
\item{(ii)} priesečník aspoň dvoch z týchto tetív.

Alena potom každý z označených bodov číselne ohodnotí. Zo všetkých 4038 krajných bodov spĺňajúcich (i) ohodnotí nejakých 2019 bodov číslom 0 a zvyšných 2019 bodov číslom 1. Následne ohodnotí
všetky body spĺňajúce (ii) ľubovoľným celým číslom (nie nutne kladným).
Pozdĺž každej tetivy uvažuje úsečky spájajúce po sebe idúce označené body. (Tetiva s $k$ označenými
bodmi obsahuje $k-1$ takýchto úsečiek.) Ďalej ohodnotí každú takúto úsečku dvoma číslami, z ktorých
jedno je žlté a druhé modré, pričom žlté čísla predstavujú súčet číselných ohodnotení koncových bodov
na ohodnocovanej úsečke a modré čísla absolútnu hodnotu ich rozdielu.
Následne Alena zistila, že medzi všetkými $N +1$ žltými číslami sa každé z čísel $0, 1, \ldots , N$ vyskytuje
práve raz. Dokážte, že aspoň jedno modré číslo je násobkom 3.}
\podpis{Spojené kráľovstvo}
