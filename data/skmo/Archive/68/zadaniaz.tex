{%%%%%   Z4-I-1
...}
\podpis{...}

{%%%%%   Z4-I-2
...}
\podpis{...}

{%%%%%   Z4-I-3
...}
\podpis{...}

{%%%%%   Z4-I-4
...}
\podpis{...}

{%%%%%   Z4-I-5
...}
\podpis{...}

{%%%%%   Z4-I-6
...}
\podpis{...}

{%%%%% Z5-I-1
Miška má päť pasteliek. Vojto ich má menej ako Miška. Vendelín ich má toľko, koľko Miška a~Vojto spolu. Všetci traja spolu majú sedemkrát viac pasteliek, ako má Vojto. Koľko pasteliek má Vendelín?}
\podpis{Libuše Hozová}

{%%%%% Z5-I-2
Tereza dostala štyri zhodné pravouhlé trojuholníky so stranami dĺžok 3\,cm, 4\,cm a~5\,cm. Z~týchto trojuholníkov (nie nutne zo všetkých štyroch) skúšala skladať nové útvary. Postupne sa jej podarilo zložiť štvoruholníky s~obvodom 14\,cm, 18\,cm, 22\,cm a~26\,cm, a~to zakaždým dvoma rôznymi spôsobmi (t.\,j. tak, že žiadne dva štvoruholníky neboli zhodné). Nakreslite, aké štvoruholníky mohla Tereza zložiť.}
\podpis{Lucie Růžičková}

{%%%%% Z5-I-3
Štefka rada oslavuje, takže okrem narodenín vymyslela ešte {\it antinarodeniny\/}: dátum antinarodenín vznikne tak, že sa vymení číslo dňa a~číslo mesiaca v~dátume narodenia. Sama sa narodila 8.\,11., takže antinarodeniny má 11.\,8. Jej mamička antinarodeniny oslavovať nemôže: narodila sa 23.\,7., jej antinarodeniny by mali byť 7.\,23., čo ale nie je dátum žiadneho dňa v~roku. Jej brat síce antinarodeniny oslavovať môže, ale má ich v~ten istý deň ako narodeniny: narodil sa 3.\,3. Koľko dní v~roku je takých, že človek, ktorý sa toho dňa narodil, môže oslavovať svoje antinarodeniny, a~to v~iný deň ako svoje narodeniny?}
\podpis{Veronika Hucíková}

{%%%%% Z5-I-4
V~novej klubovni boli iba stoličky a~stôl. Každá stolička mala štyri nohy, stôl bol trojnohý. Do klubovne prišli skauti. Každý si sadol na svoju stoličku, dve stoličky zostali neobsadené a~počet nôh v~miestnosti bol 101. Určte, koľko stoličiek bolo v~klubovni.}
\podpis{Libuše Hozová}

{%%%%% Z5-I-5
Tomáš dostal deväť kartičiek, na ktorých boli nasledujúce čísla a~matematické symboly:
$$
18,\ 19,\ 20,\ 20,\ +,\ -,\ \times,\ (,\ )
$$
Kartičky ukladal tak, že vedľa seba nikdy neležali dve kartičky s~číslami, t.\,j. striedali sa kartičky s~číslami a~kartičky so symbolmi.
Takto vzniknuté úlohy vypočítal a~výsledok si zapísal. Určte, aký najväčší výsledok mohol Tomáš získať.}
\podpis{Karel Pazourek}

{%%%%% Z5-I-6
Na obrázku je hrací plánik a~cesta, ktorú Juro zamýšľal prejsť z~pravého dolného rohu do ľavého horného. Potom zistil, že má plánik chybne pootočený, teda že by nezačínal v~pravom dolnom rohu. Tvar zamýšľanej cesty už ale nemohol zmeniť a~musel ju prejsť pri správnom natočení plánika. Pre každé z~troch možných natočení prekreslite uvedenú cestu a~určte, koľkými sivými políčkami táto cesta prechádza.
\insp{z5-I-6.eps}%
}
\podpis{Eva Semerádová}

{%%%%% Z6-I-1
Ivan a~Mirka sa delili o~hrušky v~mise. Ivan si vždy berie dve hrušky a~Mirka polovicu toho, čo v~mise ostáva. Takto postupne odoberali Ivan, Mirka, Ivan, Mirka a~nakoniec Ivan, ktorý vzal posledné dve hrušky. Určte, kto mal nakoniec viac hrušiek a~o~koľko.}
\podpis{Monika Dillingerová}

{%%%%% Z6-I-2
Ernest si zo štvorčekového papiera vystrihol štvorec $4\times 4$. Kristián v~ňom vystrihol dve diery, pozri dva čierne štvorčeky na obrázku. Tento útvar skúšal Ernest rozstrihnúť podľa vyznačených čiar na dve zhodné časti. Nájdite aspoň štyri rôzne spôsoby, ako to mohol Ernest spraviť. (Pritom dve strihania považujte za rôzne, ak časti vzniknuté jedným strihaním nie sú zhodné s~časťami vzniknutými druhým strihaním.)
\insp{z6-I-2.eps}%
}
\podpis{Alžbeta Bohiniková}

{%%%%% Z6-I-3
Na obrázku sú naznačené dva rady šesťuholníkových políčok, ktoré doprava pokračujú bez obmedzenia. Do každého políčka doplňte jedno kladné celé číslo tak, aby súčin čísel v~ľubovoľných troch navzájom susediacich políčkach bol 2018. Určte číslo, ktoré bude v~2019. políčku v~hornom rade.
\insp{z6-I-3.eps}%
}
\podpis{Lucie Růžičková}

{%%%%% Z6-I-4
Pán Ticháček mal na záhrade troch sadrových trpaslíkov: najväčšieho volal Maško, prostredného Jarko a~najmenšieho Fanko. Keďže sa s~nimi rád hrával, časom zistil, že keď postaví Fanka na Jarka, sú rovnako vysokí ako Maško. Keď naopak postaví Fanka na Maška, merajú spolu o~34\,cm viac ako Jarko. A~keď postaví na Maška Jarka, sú o~72\,cm vyšší ako Fanko. Ako vysokí sú trpaslíci pána Ticháčka?}
\podpis{Michaela Petrová}

{%%%%% Z6-I-5
V~nasledujúcom príklade na sčítanie predstavujú rovnaké písmená rovnaké cifry, rôzne písmená rôzne cifry:
$$
\begin{array}{ccccc}
R & A & T & A & M \\
 & & R & A & D \\
\hline
U & L & O & H & Y \\
\end{array}
$$
Nahraďte písmená ciframi tak, aby bol príklad správne. Nájdite dve rôzne nahradenia.}
\podpis{Erika Novotná}

{%%%%% Z6-I-6
V~dvanásťuholníku~$ABCDEFGHIJKL$ sú každé dve susedné strany navzájom kolmé a~všetky strany s~výnimkou strán $AL$ a~$GF$ sú navzájom zhodné. Strany $AL$ a~$GF$ sú oproti ostatným stranám dvojnásobne dlhé. Úsečky $BG$ a~$EL$ sa pretínajú v~bode~$M$ a~rozdeľujú dvanásťuholník na šesť útvarov (tri trojuholníky, dva štvoruholníky a~jeden päťuholník). Štvoruholník $EFGM$ má obsah 7\,cm$^2$. Určte obsahy ostatných piatich útvarov.
\insp{z6-I-6.eps}%
}
\podpis{Eva Semerádová}

{%%%%% Z7-I-1
Na každej z~troch kartičiek je napísaná jedna cifra rôzna od nuly (na rôznych kartičkách nie sú nutne rôzne cifry). Vieme, že akékoľvek trojciferné číslo zložené z~týchto kartičiek je deliteľné šiestimi. Navyše možno z~týchto kartičiek zložiť trojciferné číslo deliteľné jedenástimi. Aké cifry môžu byť na kartičkách? Určte všetky možnosti.}
\podpis{Veronika Hucíková}

{%%%%% Z7-I-2
V~dvanásťuholníku $ABCDEFGHIJKL$ sú každé dve susedné strany navzájom kolmé a~všetky strany s~výnimkou strán $AL$ a~$GF$ sú navzájom zhodné. Strany $AL$ a~$GF$ sú oproti ostatným stranám dvojnásobne dlhé. Úsečky $BG$ a~$EL$ sa pretínajú v~bode~$M$. Štvoruholník $ABMJ$ má obsah 1,8\,cm$^2$. Určte obsah štvoruholníka $EFGM$.
\insp{z7-I-2.eps}%
}
\podpis{Eva Semerádová}

{%%%%% Z7-I-3
Dedo pripravil pre svojich šesť vnúčat kôpku lieskových orieškov s~tým, nech si ich nejako rozoberú. Prvý prišiel Adam, odpočítal si polovicu, pribral si ešte jeden oriešok a~odišiel. Rovnako sa zachoval druhý Bob, tretí Cyril, štvrtý Dano aj~piaty Edo. Iba Ferko smutne hľadel na prázdny stôl; už pre neho žiadny oriešok nezvýšil. Koľko orieškov bolo pôvodne na kôpke?}
\podpis{Marta Volfová}

{%%%%% Z7-I-4
Betka sa hrala s~ozubenými kolesami, ktoré ukladala tak, ako je naznačené na obrázku. Keď potom zatočila jedným okolo, točili sa všetky ostatné. Nakoniec bola spokojná so súkolesím, pričom prvé koleso malo 32 a~druhé 24~zubov. Keď sa tretie koleso otočilo presne osemkrát, druhé koleso urobilo päť otáčok a~časť šiestej a~prvé koleso urobilo štyri otáčky a~časť piatej. Zistite, koľko zubov malo tretie koleso.
\insp{z7-I-4.eps}%
}
\podpis{Erika Novotná}

{%%%%% Z7-I-5
V~záhradníctve Rose si jedna predajňa objednala celkom 120 ruží vo farbe červenej a~žltej, druhá predajňa celkom 105 ruží vo farbe červenej a~bielej a~tretia predajňa celkom 45~ruží vo~farbe žltej a~bielej. Záhradníctvo zákazku splnilo, a~to tak, že ruží rovnakej farby dodalo do každého obchodu rovnako. Koľko celkovo červených, koľko bielych a~koľko žltých ruží dodalo záhradníctvo do týchto troch predajní?}
\podpis{Marta Volfová}

{%%%%% Z7-I-6
Daný je rovnoramenný pravouhlý trojuholník $ABS$ so základňou~$AB$. Na kružnici, ktorá má stred v~bode~$S$ a~prechádza bodmi $A$ a~$B$, leží bod~$C$ tak, že trojuholník $ABC$ je rovnoramenný. Určte, koľko bodov~$C$ vyhovuje uvedeným podmienkam, a~všetky také body zostrojte.}
\podpis{Karel Pazourek}

{%%%%% Z8-I-1
Fero a~Dávid sa denne stretávajú vo výťahu. Raz ráno zistili, že keď vynásobia svoje súčasné veky, dostanú 238. Keby to isté urobili za štyri roky, bol by tento súčin 378. Určte súčet súčasných vekov Fera a~Dávida.}
\podpis{Michaela Petrová}

{%%%%% Z8-I-2
Do triedy pribudol nový žiak, o~ktorom sa vedelo, že okrem angličtiny vie výborne ešte jeden cudzí jazyk. Traja spolužiaci sa dohadovali, ktorý jazyk to je.

Prvý súdil: \uv{Francúzština to nie je.}

Druhý hádal: \uv{Je to španielčina alebo nemčina.}

Tretí usudzoval: \uv{Je to španielčina.}

Vzápätí sa dozvedeli, že aspoň jeden z~nich hádal správne a~aspoň jeden nesprávne. Určte, ktorý z~menovaných jazykov nový žiak ovládal.}
\podpis{Marta Volfová}

{%%%%% Z8-I-3
Peter narysoval pravidelný šesťuholník, ktorého vrcholy ležali na kružnici dĺžky 16\,cm. Potom pre každý vrchol tohto šesťuholníka narysoval kružnicu so stredom v~tomto vrchole, ktorá prechádzala jeho dvoma susednými vrcholmi. Vznikol tak útvar ako na obrázku. Určte obvod vyznačeného kvietka.
\insp{z8-I-3.eps}%
}
\podpis{Erika Novotná}

{%%%%% Z8-I-4
Na štyroch kartičkách boli štyri rôzne cifry, z~ktorých jedna bola nula. Vojto z~kartičiek zložil čo najväčšie štvorciferné číslo, Martin potom čo najmenšie štvorciferné číslo. Adam zapísal na tabuľu rozdiel Vojtovho a~Martinovho čísla. Potom Vojto z~kartičiek zložil čo najväčšie trojciferné číslo a~Martin čo najmenšie trojciferné číslo. Adam opäť zapísal na tabuľu rozdiel Vojtovho a~Martinovho čísla. Potom Vojto s~Martinom podobne zložili dvojciferné čísla a~Adam zapísal na tabuľu ich rozdiel. Nakoniec Vojto vybral čo najväčšie jednociferné číslo a~Martin čo najmenšie nenulové jednociferné číslo a~Adam zapísal ich rozdiel. Keď Adam sčítal všetky štyri rozdiely na tabuli, vyšlo mu 9090. Určte štyri cifry na kartičkách.}
\podpis{Lucie Růžičková}

{%%%%% Z8-I-5
Kráľ dal murárovi Václavovi za úlohu postaviť múr hrubý 25\,cm, dlhý 50\,m a~vysoký 2\,m. Ak by Václav pracoval bez prestávky a~rovnakým tempom, postavil by múr za 26~hodín. Podľa platných kráľovských nariadení však musí Václav dodržiavať nasledujúce podmienky:
%\begin{itemize}
\itemitem{$\bullet$}
Počas práce musí spraviť práve šesť polhodinových prestávok.
\itemitem{$\bullet$}
Na začiatku práce a~po každej polhodinovej prestávke, keď je dostatočne oddýchnutý, môže pracovať o~štvrtinu rýchlejšie ako normálnym tempom, ale nie dlhšie ako jednu hodinu.
\itemitem{$\bullet$}
Medzi prestávkami musí pracovať najmenej 3/4 hodiny.
%\end{itemize}

Za aký najkratší čas môže Václav splniť zadanú úlohu?
}
\podpis{Jakub Norek}

{%%%%% Z8-I-6
V~lichobežníku $KLMN$ má základňa~$KL$ veľkosť 40\,cm a~základňa~$MN$ má veľkosť 16\,cm. Bod~$P$ leží na úsečke~$KL$ tak, že úsečka~$NP$ rozdeľuje lichobežník na dve časti s~rovnakými obsahmi. Určte veľkosť úsečky~$KP$.
}
\podpis{Libuše Hozová}

{%%%%% Z9-I-1
Nájdite všetky kladné celé čísla $x$ a~$y$, pre ktoré platí
$$
\frac1x+\frac1y=\frac14.
$$
}
\podpis{Alžbeta Bohiniková}

{%%%%% Z9-I-2
V~rovnostrannom trojuholníku $ABC$ je $K$ stredom strany~$AB$, bod~$L$ leží v~tretine strany~$BC$ bližšie bodu~$C$ a~bod~$M$ leží v~tretine strany~$AC$ bližšie bodu~$A$. Určte, akú časť obsahu trojuholníka $ABC$ zaberá trojuholník $KLM$.}
\podpis{Lucie Růžičková}

{%%%%% Z9-I-3
V~našom meste sú tri kiná, ktorým sa hovorí podľa svetových strán. O~ich otváracích hodinách je známe, že:
%\begin{itemize}
\itemitem{$\bullet$}
každý deň je otvorené aspoň jedno kino,
\itemitem{$\bullet$}
ak je otvorené južné kino, tak nie je otvorené severné kino,
\itemitem{$\bullet$}
nikdy nie je otvorené súčasne severné a~východné kino,
\itemitem{$\bullet$}
ak je otvorené východné kino, tak je otvorené aj južné alebo severné kino.\endgraf
%\end{itemize}
\noindent
%\item{}
Vydali sme sa do južného kina a~zistili sme, že je zatvorené. Ktoré zo zvyšných kín je určite otvorené?
}
\podpis{Monika Dillingerová}

{%%%%% Z9-I-4
Hotelier chcel vybaviť jedáleň novými stoličkami. V~katalógu si vybral typ stoličky. Až pri zadávaní objednávky sa od výrobcu dozvedel, že v~rámci zľavovej akcie ponúkajú každú štvrtú stoličku za polovičnú cenu a~že teda oproti plánu môže ušetriť za sedem a~pol stoličky. Hotelier si spočítal, že za pôvodne plánovanú čiastku môže zaobstarať o~deväť stoličiek viac, ako zamýšľal. Koľko stoličiek chcel hotelier pôvodne kúpiť?}
\podpis{Libor Šimůnek}

{%%%%% Z9-I-5
Adam a~Eva vytvárali dekorácie z~navzájom zhodných bielych kruhov. Adam použil štyri kruhy, ktoré položil tak, že sa každý dotýkal dvoch iných kruhov. Medzi ne potom vložil iný kruh, ktorý sa dotýkal všetkých štyroch bielych kruhov, a~ten vyfarbil červenou. Eva použila tri kruhy, ktoré položila tak, že sa dotýkali navzájom. Medzi ne potom vložila iný kruh, ktorý sa dotýkal všetkých troch bielych kruhov, a~ten vyfarbila zelenou. Eva si všimla, že jej zelený kruh a~Adamov červený kruh sú rôzne veľké, a~začali spolu zisťovať, ako sa líšia. Vyjadrite polomery červeného a~zeleného kruhu všeobecne pomocou polomeru bielych kruhov.
\insp{z9-I-5.eps}%
}
\podpis{Marie Krejčová}

{%%%%% Z9-I-6
Prirodzené číslo~$N$ nazveme {\it bombastické}, ak neobsahuje vo svojom zápise žiadnu nulu a~ak žiadne menšie prirodzené číslo nemá rovnaký súčin cifier ako číslo~$N$. Karol sa najskôr zaujímal o~bombastické prvočísla a~tvrdil, že ich nie je veľa. Vypíšte všetky dvojciferné bombastické prvočísla. Potom Karol zvolil jedno bombastické číslo a~prezradil nám, že obsahuje cifru~3 a~že iba jedna z~jeho ďalších cifier je párna. O~ktorú párnu cifru sa mohlo jednať?}
\podpis{Michal Rolínek}

{%%%%%   Z4-II-1
...}
\podpis{...}

{%%%%%   Z4-II-2
...}
\podpis{...}

{%%%%%   Z4-II-3
...}
\podpis{...}

{%%%%% Z5-II-1
Pretekov chrtov sa zúčastnilo 36~psov.
Počet psov, ktoré do cieľa dobehli pred Dunčom, bol štyrikrát menší ako počet tých, ktoré dobehli za ním.
Koľký bol Dunčo?
}
\podpis{Libuše Hozová}

{%%%%% Z5-II-2
Agáta napísala dvadsaťciferné číslo 12345678901234567890.
Filip si z~Agátinho čísla vybral štyri cifry, ktoré boli zapísané bezprostredne za sebou.
Tieto štyri cifry skúšal zapísať v~rôznom poradí a~nakoniec sa mu z~nich podarilo zostaviť
dvojicu dvojciferných čísel, z~ktorých prvé bolo o~1 väčšie ako druhé.
Ktoré dvojice čísel mohol Filip zostaviť?
Určte všetky možnosti.}
\podpis{Lucie Růžičková}

{%%%%% Z5-II-3
Anička vyrazila z~hotela na prechádzku, išla 5~km na sever, potom 2~km na východ, 3~km na juh a~nakoniec 4~km na západ.
Tak došla k~rybníku, kde sa okúpala.
Vojto vyšiel z~kempu, išiel 3~km na juh, 4~km na západ a~1~km na sever.
Tak došiel ku skale, ktorá bola 5~km západne od rybníka, v~ktorom sa kúpala Anička.
Inokedy vyšla Anička zo svojho hotela a~Vojto zo svojho kempu, obaja chceli dôjsť k~rybníku, v~ktorom sa predtým kúpala Anička, a~obaja postupovali iba v~smeroch štyroch svetových strán.
Určte, koľko najmenej kilometrov musela prejsť Anička a~koľko Vojto.
\ifobrazkyvedla\else\insp{z5-II-3.eps}\fi%
}
\podpis{Eva Semerádová}

{%%%%% Z6-II-1
Jano, Dano, Anna a~Hana majú každý svoje obľúbené číslo. Vieme, že
\begin{itemize}
\itemitem{$\bullet$} keď sčítame Danovo číslo a~trojnásobok Janovho čísla, vyjde nepárne číslo,
\itemitem{$\bullet$} keď odčítame od seba Annino a~Hanino číslo a~tento výsledok vynásobíme piatimi, vyjde nepárne číslo,
\itemitem{$\bullet$} keď vynásobíme Danovo číslo Haniným číslom a~k~výsledku pripočítame~17, vyjde párne číslo.
\end{itemize}
Určte, čie obľúbené čísla sú nepárne a~čie párne.
}
\podpis{Eva Semerádová}

{%%%%% Z6-II-2
Anička má v kasičke našporených 290 mincí, a~to jednoeurovky a~dvojeurovky.
Keď použije štvrtinu všetkých dvojeuroviek, zloží rovnakú sumu, ako keď použije tretinu všetkých jednoeuroviek.
Akú sumu má Anička našporenú?
}
\podpis{Lucie Růžičková}

{%%%%% Z6-II-3
Útvar na obrázku je zložený z~piatich zhodných štvorcov a~je rozdelený úsečkami na tri farebne odlíšené časti.
Obsah sivej časti je o~0,6\,cm$^2$ väčší ako obsah bielej časti.
Určte obsah celého útvaru.
\ifobrazkyvedla\else\insp{z6-II-3.eps}\fi%
}
\podpis{Eva Semerádová}

{%%%%% Z7-II-1
Majka, Vašo a~Zuzka počítali cez víkend úlohy.
Majka a~Vašo vypočítali dokopy 25~úloh.
Zuzka a~Vašo vypočítali dokopy 32~úloh.
Pritom Zuzka vypočítala dvakrát viac úloh ako Majka.
Koľko úloh vypočítal Vašo?
}
\podpis{Monika Dillingerová}

{%%%%% Z7-II-2
Medzi cifry čísla 2019 vložte dve cifry tak, aby vzniknuté šesťciferné číslo
\begin{itemize}
\itemitem{$\bullet$} začínalo 2 a~končilo 9,
\itemitem{$\bullet$} bolo zložené zo šiestich rôznych cifier,
\itemitem{$\bullet$} bolo deliteľné tromi,
\itemitem{$\bullet$} jeho prvé trojčíslie bolo deliteľné tromi,
\itemitem{$\bullet$} jeho prvé štvorčíslie bolo deliteľné štyrmi,
\itemitem{$\bullet$} súčet vložených cifier bol nepárny.
\end{itemize}
Určte rozdiel najväčšieho a~najmenšieho šesťciferného čísla s~uvedenými vlastnosťami.}
\podpis{Lucie Růžičková}

{%%%%% Z7-II-3
Útvar na obrázku je zložený z~ôsmich zhodných štvorcov a~je rozdelený úsečkami na päť farebne odlíšených častí.
Pritom bod~$X$ je stredom úsečky~$KJ$, bod~$Y$ je stredom úsečky~$EX$ a~úsečka~$BZ$ je zhodná s~$BC$.
Obsah čiernej časti útvaru je 7,5\,cm$^2$.
Určte obsahy zvyšných štyroch častí.
\ifobrazkyvedla\else\insp{z7-II-3.eps}\fi%
}
\podpis{Eva Semerádová, Monika Dillingerová}

{%%%%% Z8-II-1
V~dvojposchodovom dome, ktorý má obytnú časť okrem 1. a~2. poschodia aj na prízemí, býva 35~ľudí nad niekým a~45 ľudí býva pod niekým.
Pritom na 1.~poschodí býva jedna tretina všetkých osôb žijúcich v~dome.
Koľko osôb býva v~dome celkom?}
\podpis{Libuše Hozová}

{%%%%% Z8-II-2
Pre koľko kladných celých čísel menších ako 1000 platí, že medzi číslami 2, 3, 4, 5, 6, 7, 8 a~9 je práve jedno, ktoré nie je jeho deliteľom?
}
\podpis{Eva Semerádová}

{%%%%% Z8-II-3
V~lichobežníku $ABCD$ so základňami $AB$ a~$CD$ platí, že $|AD|=|CD|$, $|AB|=2|CD|$, $|BC|= 24$\,cm a~$|AC|= 10$\,cm.
Vypočítajte obsah lichobežníka $ABCD$.}
\podpis{Lucie Růžičková}

{%%%%% Z9-II-1
Marienka napísala na tabuľu dve rôzne prirodzené čísla.
Marta si vzala kartičku, na ktorej jednu stranu napísala súčet Marienkiných čísel a~na druhú stranu ich súčin.
Jedno z~čísel na kartičke bolo prvočíslo a~súčet čísel z~oboch strán kartičky bol~97.
Ktoré čísla napísala Marienka na tabuľu?}
\podpis{Libuše Hozová}

{%%%%% Z9-II-2
Máme kváder, ktorého jedna hrana je päťkrát dlhšia ako iná jeho hrana.
Keby sme výšku kvádra zväčšili o~2\,cm, zväčšil by sa jeho objem o~90\,cm$^3$.
Keby sme výšku takto zväčšeného kvádra zmenili na polovicu, bol by objem nového kvádra rovný trom pätinám objemu pôvodného kvádra.
Aké môžu byť rozmery pôvodného kvádra?
Určte všetky možnosti.}
\podpis{Eva Semerádová}

{%%%%% Z9-II-3
Z~cifier 3, 4, 5, 7 a~9 boli vytvorené všetky možné trojciferné čísla tak, aby sa každá cifra vyskytovala v~každom čísle nanajvýš raz.
Určte počet takto vzniknutých čísel a~ich celkový súčet.}
\podpis{Marta Volfová}

{%%%%% Z9-II-4
V~pravouhlom trojuholníku je polomer jemu opísanej kružnice 14,5\,cm a~polomer jemu vpísanej kružnice 6\,cm.
Určte obvod tohto trojuholníka.}
\podpis{Libuše Hozová}

{%%%%% Z9-III-1
Adela napísala na tabuľu dve kladné celé čísla a~dala Lukášovi a~Petrovi za úlohu určiť kladný rozdiel druhých mocnín týchto dvoch čísel.
Lukáš namiesto toho určil druhú mocninu rozdielu daných dvoch čísel.
Vyšlo mu tak číslo o~4\,038 menšie ako Petrovi, ktorý výpočet vykonal správne.
Ktoré dve čísla mohla Adela napísať na tabuľu?
Určte všetky možnosti.
}
\podpis{Lucie Růžičková}

{%%%%% Z9-III-2
Na ostrove žijú dva druhy domorodcov: Poctivci, ktorí vždy hovoria pravdu, a~Klamári, ktorí vždy klamú.
Keď cudzinec stretol troch domorodcov, Alana, Bruna a~Ctibora, spýtal sa ich, do ktorej skupiny patria.
\item Alan oznámil: \uv{Bruno je Klamár.}
\item Bruno povedal: \uv{Alan a~Ctibor sú buď obaja Klamári, alebo obaja Poctivci.}
\item Ctibor sa nevyjadril.

Mohol cudzinec pri niektorom z~týchto domorodcov s~istotou určiť, či je Poctivec, alebo Klamár?
}
\podpis{Marta Volfová}

{%%%%% Z9-III-3
Keď číslo $X$ vydelím číslom $Y$, dostanem číslo $Z$ a~zvyšok 27.
Keď číslo $X$ vydelím číslom $Z$, dostanem číslo $1{,}1\cdot Y$ a~zvyšok 3.
Ktoré čísla $X$, $Y$, $Z$ vyhovujú uvedeným podmienkam?
Určte všetky možnosti.
}
\podpis{Libuše Hozová}

{%%%%% Z9-III-4
Daná je kružnica so stredom $S$ a~polomerom 39\,mm.
Do kružnice máme vpísať trojuholník $ABC$ tak, aby veľkosť strany $AC$ bola 72\,mm a~bod $B$ ležal v~polrovine určenej priamkou $AC$ a~bodom $S$.
Zo zadaných údajov vypočítajte, akú veľkosť má mať výška trojuholníka $ABC$ z~vrcholu $B$, aby úloha mala dve riešenia.
}
\podpis{Marie Krejčová}
