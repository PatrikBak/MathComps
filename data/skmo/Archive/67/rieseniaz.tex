{%%%%%   Z4-I-1
...}

{%%%%%   Z4-I-2
...}

{%%%%%   Z4-I-3
...}

{%%%%%   Z4-I-4
...}

{%%%%%   Z4-I-5
...}

{%%%%%   Z4-I-6
...}

{%%%%%   Z5-I-1
\napad
Koľko stojí jeden koláč?

\riesenie
Jankovo vreckové možno vyjadriť tromi spôsobmi, a~to ako
\begin{itemize}
\item súčet ceny 4 koláčov plus 0,50€,
\item súčet ceny 5 koláčov mínus 0,60€,
\item súčet cien 2 koláčov a~3 šišiek.
\end{itemize}
Z~prvých dvoch vyjadrení vyplýva, že jeden koláč stojí $0{,}50+0{,}60=1{,}10$~€.
Z~toho tiež zisťujeme, že Jankovo vreckové bolo $4\cdot 1{,}10+0{,}50=5\cdot 1{,}10-0{,}60=4{,}90$~€.
Z~tretieho vyjadrenia vyplýva, že za tri šišky by Janko zaplatil $4{,}90-2\cdot 1{,}10=2{,}70$~€.
Jedna šiška teda stojí $2{,}70:3=0{,}90$~€.}

{%%%%%   Z5-I-2
\napad
Aké bolo poradie klietok?

\riesenie
Z~posledných dvoch informácií vyplýva, že strieborná klietka nestála ani úplne vľavo, ani úplne vpravo, teda stála uprostred.
Zlatá klietka stála naľavo od čiernej klietky, teda poradie klietok bolo: zlatá, strieborná, čierna.

Potkan bol v~klietke napravo od striebornej klietky, teda bol v~čiernej klietke.
Strieborná klietka stála napravo od klietky s~morčaťom, teda morča bolo v~zlatej klietke.
Jano mal zvieratá v~klietkach rozmiestnené takto:
$$
\begintable
zlatá|strieborná|čierna\crthick
morča|tchor|potkan\endtable
$$
}

{%%%%%   Z5-I-3
\napad
Určte najskôr počty hviezdičiek v~políčkach $A$, $B$, $C$.

\riesenie
Z~druhej a~piatej podmienky vyplýva, že v~políčku~$A$ je aspoň 1 hviezdička a~v~políčku~$B$ sú aspoň 2 hviezdičky.
Teda v~políčkach $A$ a~$B$ sú spolu aspoň 3 hviezdičky.
Z~tretej a~štvrtej podmienky vyplýva, že v~týchto dvoch políčkach nie sú spolu viac ako 3 hviezdičky.
Preto sú v~políčkach $A$ a~$B$ spolu práve 3 hviezdičky a~v~políčku~$C$ je 5~hviezdičiek:
$$
A=1,\quad B=2,\quad C=5.
$$

Aj ostatné políčka označíme písmenami ako v~nasledujúcom obrázku:
\figure{z5-I-3a}%


Každé z~políčok $A$, $B$ a~$C$ je spoločné dvom z~troch útvarov uvedených v~šiestej podmienke (napr. políčko~$A$ patrí kruhu a~trojuholníku).
Políčko~$D$ je spoločné všetkým trom útvarom.
Zvyšné políčka $E$, $F$ a~$G$ patria do navzájom rôznych útvarov.
Súčet hviezdičiek v~kruhu, trojuholníku a~obdĺžniku je
$15+12+14=41$
a~v~tomto súčte sú hviezdičky z~políčok $A$, $B$, $C$ započítané dvakrát, hviezdičky z~políčka~$D$ trikrát a~hviezdičky z~políčok $E$, $F$, $G$ jedenkrát.
Pritom podľa prvej podmienky je hviezdičiek spolu 21 a~v~tomto súčte sú hviezdičky z~každého políčka počítané jedenkrát.
Rozdiel 20~hviezdičiek preto zodpovedá súčtu hviezdičiek v~políčkach $A$, $B$, $C$ (ktorých je celkom~8) a~dvojnásobku počtu hviezdičiek v~políčku $D$.
V~políčku~$D$ preto musí byť 6~hviezdičiek:
$$
D=(20-8):2=6.
$$

Počty hviezdičiek vo zvyšných políčkach možno teraz dopočítať
podľa informácií v~šiestej podmienke:
$$
\aligned
15&=A+C+D+E,\quad\text{teda}\quad E=15-1-5-6=3, \\
12&=A+B+D+F,\quad\text{teda}\quad F=12-1-5-2=3, \\
14&=B+C+D+G,\quad\text{teda}\quad G=14-2-5-6=1.
\endaligned
$$
\figure{z5-I-3b}%



\ineriesenie
Rovnako ako v~predchádzajúcom riešení odvodíme počty hviezdičiek
v~políčkach $A$, $B$ a~$C$:
$$
A=1,\quad B=2,\quad C=5.
$$
Podľa informácií v~šiestej podmienke zisťujeme, že
$$
\aligned
15&=A+C+D+E,\quad\text{teda}\quad D+E=15-1-5=9, \\
12&=A+B+D+F,\quad\text{teda}\quad D+F=12-1-2=9, \\
14&=B+C+D+G,\quad\text{teda}\quad D+G=14-2-5=7.
\endaligned
$$
Z~toho vidíme, že v~políčkach $E$ a~$F$ je rovnaký počet hviezdičiek, a~ten
je o~2 väčší ako v~políčku~$G$. Teraz môžeme postupne dosadzovať počty
hviezdičiek v~ktoromkoľvek z~políčok $D$, $E$, $F$, $G$, z~predchádzajúceho
vyjadriť počty vo~zvyšných troch políčkach a~overiť, či je celkový
súčet $A+B+C+D+E+F+G$ rovný~21. Dosadzujeme za~$G$, pričom máme na
pamäti, že v~každom políčku má byť aspoň jedna hviezdička:
$$
\begintable
$G$\|$E=F$|$D$|súčet\crthick
\bf 1\|\bf 3|\bf 6|\bf 21\cr
2\|4|5|23\cr
3\|5|4|25\cr
4\|6|3|27\cr
5\|7|2|29\cr
6\|8|1|31\endtable
$$
Jediná vyhovujúca možnosť je zvýraznená v~prvom riadku.
}

{%%%%%   Z5-I-4
\napad
Koľko výmen mohli Eva s~Marekom odohrať, keby nakoniec bol nakreslený
iba jeden krúžok?

\riesenie
Každý krúžok nahrádza 10~krížikov, predchádzajúci zápis teda zodpovedá 37~krížikom.
Každý krížik predstavuje 10 odohraných výmen, Eva s~Marekom teda zohrala najmenej 370 a~najviac 379 výmen.
}

{%%%%% Z5-I-5
\napad
Čo viete o~uhlopriečkach v~obdĺžniku a~vo štvorci?

\riesenie
1.
Obdĺžnik je štvoruholník, ktorý má všetky vnútorné uhly pravé.
Uhlopriečky každého obdĺžnika sú rovnako dlhé a~pretínajú sa vo svojich stredoch.
Z~toho vyplýva, že kružnica so stredom v~priesečníku uhlopriečok, ktorá prechádza jedným vrcholom obdĺžnika, prechádza aj všetkými ostatnými vrcholmi.

Z~týchto vlastností možno odvodiť niekoľko riešení úlohy, napr.:
\begin{itemize}
\item na kružnici~$k$ zvolíme ľubovoľne bod~$E$,
\item bod~$F$ zostrojíme ako priesečník kružnice~$k$ s~kolmicou na priamku~$AE$ idúcou bodom~$E$,
\item bod~$G$ zostrojíme ako priesečník kružnice~$k$ s~kolmicou na priamku~$EF$ idúcou bodom~$F$.
\end{itemize}
Iné riešenie tej istej úlohy je toto:
\begin{itemize}
\item na kružnici~$k$ zvolíme ľubovoľne bod~$E$,
\item bod~$F$ zostrojíme ako priesečník kružnice~$k$ s~priamkou~$AS$,
\item bod~$G$ zostrojíme ako priesečník kružnice~$k$ s~priamkou~$ES$.\figure{z5-I-5a}%
\end{itemize}



2.
Štvorec je štvoruholník, ktorý má všetky vnútorné uhly pravé a~všetky strany rovnako dlhé.
Okrem všetkých vlastností menovaných v~predchádzajúcom prípade navyše platí, že uhlopriečky každého štvorca sú navzájom kolmé.

Úlohu možno riešiť napr. takto:
\begin{itemize}
\item bod~$C$ zostrojíme ako priesečník kružnice~$k$ s~priamkou~$AS$,
\item body $B$ a~$D$ zostrojíme ako priesečníky kružnice~$k$ s~kolmicou na priamku~$AC$ idúcou bodom~$S$.\figure{z5-I-5b}%
\end{itemize}



}

{%%%%%   Z5-I-6
\napad
Aký je súčet čísel na všetkých kartičkách?

\riesenie
Súčet čísel na všetkých ôsmich kartičkách je
$$
2+3+5+7+11+13+17+19=77,
$$
a~to je rovné $39+38$.
Fero si vybral tri kartičky so súčtom čísel 39.
Postupným skúšaním od najväčších čísel nájdeme všetky vyhovujúce možnosti:
$$
\begintable
v~ruke\|na stole\crthick
19, 17, 3\|13, 11, 7, 5, 2\cr
19, 13, 7\|17, 11, 5, 3, 2\endtable
$$
}

{%%%%%   Z6-I-l
\napad
Vyskúšajte opísaný postup s~niekoľkými konkrétnymi číslami.

\riesenie
Dvojciferné číslo začínajúce sedmičkou je tvaru $7*$,
pričom namiesto hviezdičky môže byť ľubovoľná cifra.
Vložením nuly dostávame trojciferné číslo tvaru $70*$.
Bez ohľadu na to, akú cifru zastupuje hviezdička na mieste jednotiek,
rozdiel vychádza stále rovnako:
$$
\alggg{&7&0&*\\-&&7&*}{&6&3&0}
$$
Výsledky Aničky a~Blanky sa nijako nelíšili, obom vyšlo 630.
}

{%%%%%   Z6-I-2
\napad
Viete porovnať jednotlivé kúsky bez počítania?

\riesenie
Najskôr označíme niekoľko pomocných vrcholov ako na obrázku:
\figure{z6-I-2a}%


Úsečka~$AC$ je uhlopriečkou obdĺžnika $ABCD$, a~tá delí obdĺžnik
na dve rovnaké časti.
Jedna časť je tvorená trojuholníkmi $AEC$ a~$EBC$,
druhá časť je tvorená mnohouholníkmi $AHGFD$ a~$CFGH$.

Trojuholník $ABC$ je polovicou obdĺžnika $ABCD$.
Trojuholník $EBC$ je polovicou obdĺžnika $EBCI$,
a~ten je polovicou obdĺžnika $ABCD$.
Preto má trojuholník $EBC$ polovičný obsah v~porovnaní
s~trojuholníkom $ABC$ a~trojuholníky $EBC$ a~$AEC$ tak majú rovnaký obsah.

Mnohouholníky $AHGFD$ a~$CFGH$ možno rozdeliť na menšie časti, ktoré sú po dvojiciach zhodné, pozri prerušované čiary na nasledujúcom obrázku.
Preto majú aj tieto dva mnohouholníky rovnaký obsah.
\figure{z6-I-2b}%


Všetky štyri mnohouholníky teda majú rovnaký obsah, čiže všetky
štyri kúsky čokolády sú rovnako veľké.

\poznamka
Vyjadrenie obsahov jednotlivých kúskov pomocou vyznačených štvorčekov vyzerá
takto: celý obdĺžnik má obsah $6\times 4=24$ štvorčekov, každý
z~trojuholníkov $AEC$ a~$EBC$ má obsah $\frac12 (3\times 4)=6$ štvorčekov,
každý z~mnohouholníkov $AHGFD$ a~$CFGH$ má obsah $3+2+1=6$ štvorčekov
(odvodené z~predchádzajúceho delenia).
}

{%%%%%   Z6-I-3
\napad
Ako sa líšia počty fliaš v~susedných poschodiach?

\riesenie
1.
Fľaše budeme počítať po poschodiach zhora.
Zo zadania a~návodných obrázkov vieme, že v~piatom (najvyššom) poschodí je 1~fľaša,
vo štvrtom poschodí sú 3~fľaše, v~treťom poschodí je 6~fliaš.
Každé ďalšie (nižšie) poschodie si možno predstaviť tak, že sa k~predchádzajúcemu (vyššiemu) poschodiu pridá jeden rad fliaš:
\figure{z6-I-3a}%


Na päťposchodovú pyramídu Jano potreboval
$$
1+\underbrace{1+2}_3+\underbrace{1+2+3}_6+\underbrace{1+2+3+4}_{10}+\underbrace{1+2+3+4+5}_{15} =35\ \text{fliaš}.
$$

2.
S~rovnakým nápadom ako v~predchádzajúcom odseku budeme pracovať ďalej, kým nevyčerpáme maximum zo sto použiteľných fliaš:
na šesťposchodovú pyramídu treba
$$
35+\underbrace{15+6}_{21}=56\ \text{fliaš},
$$
na sedemposchodovú pyramídu treba
$$
56+\underbrace{21+7}_{28}=84\ \text{fliaš},
$$
na osemposchodovú pyramídu treba
$$
84+\underbrace{28+8}_{36}=120\ \text{fliaš}.
$$
So sto fľašami sa dá postaviť nanajvýš sedemposchodová pyramída.
}

{%%%%%   Z6-I-4
\napad
Označte si postupne políčka, na ktoré možno jazdca premiestniť po prvom ťahu, po druhom ťahu atď.

\riesenie
I~keď v~zadaní sú použité pre označenie stĺpcov veľké písmená, v~riešení budeme používať malé.

1.
Po chvíli skúšania zistíme, že doskákať s~jazdcom z~políčka~b1 na políčko~h1 v~štyroch ťahoch sa dá napr. takto: c3, e2, g3, h1, pozri obrázok.
\figure{z6-I-4a}%


Aby sme doplnili všetky postupnosti ťahov medzi týmito políčkami a~na žiadnu možnosť nezabudli, budeme postupovať nasledovne.
Určíme všetky políčka, na ktoré sa dá jazdec z~b1 premiestniť po prvom a~po druhom ťahu:
\figure{z6-I-4b}%


\noindent
Určíme všetky políčka, na ktorých musí jazdec stať po treťom a~po druhom ťahu, aby po štvrtom ťahu skončil na h1:
\figure{z6-I-4c}%


\noindent
Určíme prienik predchádzajúcich dvoch situácií po druhom ťahu:
\figure{z6-I-4d}%


Možnosť s~jazdcom po druhom ťahu na e2 je jedna,
a~to je práve vyššie uvedené riešenie.
Možnosti s~jazdcom po druhom ťahu na e4 sú štyri:
\figure{z6-I-4e}%


\noindent
Možnosť s~jazdcom po druhom ťahu na d1 je jedna, rovnako ako možnosť s~jazdcom po druhom ťahu na f1:
\figure{z6-I-4f}%



2.
Ľahko možno nájsť tiež cestu z~políčka~b1 na políčko~e6 v~štyroch ťahoch, ale v~piatich už nie.
Dôvodom je to, že farba políčka, na ktorom jazdec stojí, sa po každom jeho ťahu mení (jeden ťah jazdca má dve časti: dlhšia časť je o~dve políčka, a~pri tom sa farba zachováva, kratšia časť je o~jedno políčko, a~pri tom sa farba mení):

Východiskové políčko b1 je biele, po prvom ťahu bude jazdec stať na čiernom políčku, po~druhom ťahu bude opäť na bielom atď.~-- po nepárnom počte ťahov bude na čiernom políčku, po párnom počte ťahov bude na bielom políčku.
Políčko e6 je biele a~5~je nepárne číslo, preto sa nedá premiestniť jazdec z~b1 na e6 v~piatich ťahoch.

\poznamka
Sedem možných riešení v~prvej časti úlohy možno nájsť náhodným skúšaním a~následne sa zamyslieť nad zdôvodnením, že sú tieto riešenia všetky.
Pri hodnotení buďte zhovievaví, aj nie celkom úplné zdôvodnenia možno hodnotiť stupňom~1.
Avšak komentáre neobsahujúce žiadne vysvetlenie hodnoťte nanajvýš stupňom~2.
}

{%%%%%   Z6-I-5
\napad
Koľko cukríkov tej-ktorej farby mohlo, resp. nemohlo byť pôvodne v~plechovke?

\riesenie
Ako Cyril, tak Zuzka zjedli niekoľko pätín cukríkov prislúchajúcej farby.
Preto musí byť pôvodný počet ako červených, tak zelených cukríkov deliteľný piatimi.
Budeme ako pôvodný počet červených cukríkov uvažovať čo najmenšie čísla deliteľné piatimi a~skúsime vyjadriť počet zelených cukríkov:

\begin{itemize}
\item Ak by červených cukríkov bolo pôvodne~5, zvýšili by z~nich po odjedení~3.
Tieto 3~cukríky by mali tvoriť $\frac38$ všetkých zvyšných cukríkov, teda všetkých zvyšných cukríkov by bolo~8 a~zvyšných zelených by tak bolo~5.
Týchto 5~cukríkov by malo tvoriť zvyšné $\frac25$~všetkých zelených, čo sa nedá.
\item Ak by červených cukríkov bolo pôvodne~10, zvýšilo by z~nich po odjedení~6.
Týchto 6~cukríkov by malo tvoriť $\frac38$~všetkých zvyšných cukríkov, teda všetkých zvyšných cukríkov by bolo~16 a~zvyšných zelených by tak bolo~10.
Týchto 10~cukríkov by malo tvoriť zvyšné $\frac25$~všetkých zelených, takže všetkých zelených by pôvodne bolo~25.
\end{itemize}

Najmenší počet cukríkov, ktoré mohli byť pôvodne v~plechovke, je $10+25=35$.

\napadd
Akú časť zvyšných cukríkov tvorili zelené cukríky?

\ineriesenie
Červené cukríky tvorili po odjedení $\frac38$ všetkých cukríkov,
zelené cukríky tak tvorili $\frac58$ všetkých zvyšných cukríkov, preto počet zvyšných zelených cukríkov musí byť deliteľný piatimi.

Zuzka zjedla $\frac35$ zelených cukríkov, v~plechovke tak zvýšili $\frac25$ pôvodného počtu zelených cukríkov, preto počet zvyšných zelených cukríkov musí byť deliteľný aj dvoma.
Celkom dostávame, že počet zvyšných zelených cukríkov musí byť deliteľný desiatimi.

Najmenší možný počet zvyšných zelených cukríkov je~10.
V~takom prípade by pôvodný počet zelených cukríkov bol~25, počet zvyšných červených cukríkov~6 a~pôvodný počet červených cukríkov~10.

Najmenší počet cukríkov, ktoré mohli byť pôvodne v~plechovke, je $10+25=35$.

\poznamka
Predchádzajúce úvahy je možné graficky znázorniť takto:
\figure{z6-I-5}%


Pomocou neznámych $c$, resp. $z$, ktoré označujú pôvodné počty červených, resp. zelených cukríkov, je možné úlohu sformulovať takto:
$$
\frac35c=\frac38\Bigl(\frac35c+\frac25z\Bigr),
$$
pričom $c$ a~$z$ sú čísla deliteľné piatimi a~$\frac35c+\frac25z$ je deliteľné ôsmimi.
Predchádzajúce vyjadrenie možno upraviť na
$$
\postdisplaypenalty 10000
8c=3c+2z,\quad\text{čiže}\quad 5c=2z.
$$
Najmenšie $c$ a~$z$ vyhovujúce všetkým uvedeným požiadavkám sú $c=2\cdot5=10$ a~$z=5\cdot5=25$.
}

{%%%%%   Z6-I-6
\napad
Čo všetko viete o~rovnostranných trojuholníkoch?

\riesenie
1.
Úsečky $AS$ a~$AD$ majú byť zhodné s~danou úsečkou~$DS$.
Teda
\begin{itemize}
\item bod~$A$ zostrojíme ako priesečník kružnice~$k$ a~kružnice so stredom~$D$ a~polomerom~$DS$.
\end{itemize}
Také body sú dva.
\figure{z6-I-6a}%


2.
Pre trojuholník $ABC$ s~vrcholmi na kružnici~$k$ platí, že druhé priesečníky priamok $AS$, $BS$ a~$CS$ s~kružnicou~$k$ sú stredovo súmerné s~bodmi $A$, $B$ a~$C$ podľa stredu~$S$.
Ak je trojuholník $ABC$ rovnostranný, je rovnostranný aj tento stredovo súmerný trojuholník.
Všetkých šesť bodov na kružnici~$k$ potom tvorí vrcholy pravidelného šesťuholníka.
Pravidelný šesťuholník je tvorený šiestimi zhodnými rovnostrannými trojuholníčkami, z~ktorých dva sú zostrojené v~prvej časti úlohy.
Úsečka~$A_1A_2$ na predchádzajúcom obrázku je preto jednou zo strán hľadaného trojuholníka:
\begin{itemize}
\item jeden z~bodov $A_1$, $A_2$ v~prvej časti úlohy označíme~$A$,
druhý označíme~$B$,
\item bod~$C$ zostrojíme ako priesečník kružnice~$k$ s~priamkou~$DS$.
\end{itemize}

Alternatívne možno bod~$C$ zostrojiť ako priesečník kružnice~$k$ s~kružnicou so stredom v~bode~$A$, príp. $B$ a~polomerom~$AB$.
Na nasledujúcom obrázku je naznačené ešte iné riešenie založené na doplnení pravidelného šesťuholníka opakovaním konštrukcie z~prvej časti úlohy.
\figure{z6-I-6b}%

}

{%%%%%   Z7-I-l
\napad
Aký je rozdiel cifier Pavlovho čísla?

\riesenie
Úlohu môžeme riešiť ako algebrogram:
$$
\alggg{&a&b\\-&b&a}{&6&3}
$$

Keďže rozdiel je kladný, musí byť $a>b$.
Keďže navyše v~rozdiele na mieste jednotiek je~3, musí sa počítať s~prechodom cez desiatku.
Keďže v~rozdiele na mieste desiatok je~6, musí byť $a-b=7$.
Keďže ďalej obe čísla sú dvojciferné, musí byť $b>0$.
Celkom tak dostávame dve možnosti:
$$
\alggg{&8&1\\-&1&8}{&6&3}
\hskip1cm
\alggg{&9&2\\-&2&9}{&6&3}
$$
Číslo, ktoré mohol Pavol napísať, bolo 81 alebo 92.

\poznamka
Dvojciferné číslo zapísané $\overline{ab}$ možno vyjadriť ako $10a+b$.
Predchádzajúci zápis je preto ekvivalentný s~rovnosťou
$$
\begin{aligned}
(10a+b)-(10b+a)&=63,\\
\end{aligned}
$$
čo po úprave vedie na~$a-b=7$.
}

{%%%%%   Z7-I-2
\napad
Porovnajte obsahy trojuholníkov $ACE$ a~$ABC$.

\riesenie
Obsah trojuholníka závisí od dĺžky jeho strany a~veľkosti výšky na túto stranu.
Keďže priamky $AC$ a~$BD$ sú rovnobežné a~bod~$E$ leží na priamke~$BD$, obsah
trojuholníka $ACE$ je stále rovnaký pre akokoľvek zvolený bod~$E$.
Preto obsah trojuholníka $ACE$ je rovnaký ako obsah trojuholníka $ACD$.
Z~rovnakého dôvodu je aj obsah trojuholníka $ACD$ rovnaký ako obsah trojuholníka $BCD$.
Spolu teda
$$
S_{ACE}=S_{ACD}=S_{BCD}=20\cm^2.
$$

Teraz porovnáme obsahy trojuholníkov $BCD$ a~$DFG$:
\figure{z7-I-2}%


Trojuholníky $DFG$ a~$FBG$ majú spoločnú výšku z~vrcholu~$G$ a~bod~$F$ je
v~strede strany~$BD$, preto majú tieto trojuholníky rovnaký obsah.
Trojuholníky $DFG$ a~$FBG$ dokopy tvoria trojuholník $DBG$, a~preto platí
$
S_{DFG}=\frac12 S_{DBG}.
$
Z~obdobného dôvodu tiež platí
$
S_{DBG}=\frac12 S_{DBC}.
$
Spolu teda platí
$$
S_{DFG}=\frac14 S_{DBC}=\frac14\cdot 20=5\,(\Cm^2).
$$


\poznamka
Predchádzajúce vyjadrenie pomeru obsahov trojuholníkov $DFG$ a~$DBC$
skryto odkazuje na ich podobnosť, čo možno v~zdôvodnení tiež použiť ($FG$ je
strednou priečkou trojuholníka $DBC$, preto sú všetky zodpovedajúce si strany úmerné
v~pomere $1:2$).
Bez odkazu na pojem podobnosti je možné priamo porovnať napr. základne $DF$ a~$DB$
a~zodpovedajúce výšky (oboje v~pomere $1:2$).
Takto možno uvažovať aj pre trojuholníky $DFG$ a~$ACE$ s~ľubovoľným $E\in BD$
(\tj.~bez vyššie použitých transformácií).
}

{%%%%%   Z7-I-3
\napad
Koľko vstupeniek treba žiadať, aby boli práve štyri z~nich zdarma?

\riesenie
Ak by sa pri kúpe vstupeniek pre deti zo~6.A vďaka uvedenej výhode ušetrilo za 4~vstupenky, muselo ísť do zoo aspoň $4\cdot5=20$, avšak menej ako $5\cdot5=25$ detí z~tejto triedy.
Pri počte detí od 20 do 24 by sa muselo zaplatiť vždy o~4 vstupenky menej, teda 16 až 20.
Zaplatená čiastka (v~centoch) je deliteľná 19, nie však 16, 17, 18 či 20 (vidno to z~prvočíselného rozkladu $1\,995=3\cdot5\cdot7\cdot19$).
Pre deti z~6.A~by teda bolo treba zaplatiť 19 vstupeniek a~každá by tak stála $19{,}95:19=1{,}05$€.
Počet detí z~6.A bol o~4 väčší, teda $19+4=23$.

Pri spoločnej kúpe vstupného pre deti z~oboch tried by sa uhradilo 44{,}10€, teda zaplatených vstupeniek by bolo $44{,}10:1{,}05=42$.
V~rámci výhody bola každá štvorica zaplatených vstupeniek doplnená o~jednu vstupenku zdarma, teda pri zaplatení 42~vstupeniek ($10\cdot4+2$) by ich dostali 52 ($10\cdot5+2$).
Počet detí z~6.B bol $52-23=29$.

Do zoo išlo 23~detí z~6.A a~29~detí z~6.B.
}

{%%%%%   Z7-I-4
\napad
Čo môžete povedať o~jednotlivých cifrách hľadaného čísla?

\riesenie
Hľadané šesťciferné číslo označíme $\overline{abcdef}$.
Zo zadania postupne odvodíme niekoľko poznatkov o~tomto čísle:
\begin{enumerate}
\item Celé šesťciferné číslo je deliteľné šiestimi, teda je deliteľné zároveň dvoma a~troma.
Deliteľnosť troma je zaručená tým, že ciferný súčet je (až na poradie sčítancov) rovný $1+2+3+4+5+6=21$, čo je číslo deliteľné troma.
Deliteľnosť dvoma znamená, že $f$ je niektorá z~cifier 2, 4, 6.
\item Päťciferné číslo $\overline{abcde}$ je deliteľné piatimi, preto $e=5$.
\item Štvorciferné číslo $\overline{abcd}$ je deliteľné štyrmi, preto aj číslo $\overline{cd}$ je deliteľné štyrmi.
Preto $d$ je niektorá z~cifier 2, 4, 6.
\item Trojciferné číslo $\overline{abc}$ je deliteľné troma, preto ciferný súčet $a+b+c$ je deliteľný troma.
\item Dvojciferné číslo~$\overline{ab}$ je deliteľné dvoma, preto $b$ je niektorá z~cifier 2, 4, 6.
\end{enumerate}
Jednoznačne je určené $e=5$ a~cifry $b$, $d$, $f$ sú v~niektorom poradí 2, 4, 6.
Na cifry $a$ a~$c$ teda ostáva 1 a~3.
Z~tretej podmienky potom vyplýva, že dvojciferné číslo~$\overline{cd}$ môže byť niektoré z~čísel
$$
12,\quad 16,\quad 32,\quad 36.
$$
Pre každú z~týchto možností je $a$ určené jednoznačne:
v~prvých dvoch prípadoch je $a=3$, vo zvyšných dvoch prípadoch je $a=1$, súčet $a+c$ je však vždy rovný~4.
Aby bola splnená aj štvrtá podmienka, musí byť $b=2$.
Ostávajú teda iba dve možnosti:
Anežka mohla zložiť 321654 alebo 123654.
}

{%%%%%   Z7-I-5
\napad
Nájdite vzťahy medzi vnútornými uhlami uvedených trojuholníkov.

\riesenie
Veľkosti vnútorných uhlov v~trojuholníku $ABC$ označíme postupne $\alpha$, $\beta$, $\gamma$.
V~rovnostrannom trojuholníku $ABD$ majú všetky vnútorné uhly veľkosť~$60\st$.
\figure{z7-I-5}%


Zhodné uhly pri základni rovnoramenného trojuholníka $BCD$ majú veľkosť
$$
|\angle BCD|=|\angle CBD|=|\angle ABD|-|\angle ABC|=60\st-\beta.
$$
Zhodné uhly pri základni rovnoramenného trojuholníka $ACD$ majú veľkosť
$$
|\angle ACD|=|\angle CAD|=|\angle CAB|-|\angle DAB|=\alpha-60\st.
$$
Veľkosť neznámeho uhla $ACB$ môžeme vyjadriť ako
$$
\gamma=|\angle ACD|-|\angle BCD|=
(\alpha-60\st)-(60\st-\beta)=\alpha+\beta-120\st.
$$
Súčet veľkostí vnútorných uhlov v~trojuholníku $ABC$ je 180\st, teda
$$
\alpha+\beta+(\alpha+\beta-120\st)=180\st,
$$
z~čoho vyplýva $\alpha+\beta=150\st$.
Uhol $ACB$ má veľkosť $\gamma=150\st-120\st=30\st$.

\poznamka
Zadaným podmienkam zodpovedá nekonečne veľa situácií;
$\gamma$~je vždy~$30\st$, zvyšných $150\st$ môže byť medzi $\alpha$ a~$\beta$ rozdelených ľubovoľne.

Všetky body $A$, $B$, $C$ ležia na jednej kružnici so stredom v~bode~$D$.
V~takých prípadoch všeobecne platí, že veľkosť uhla $ACB$ je polovicou uhla $ADB$ (pozri vetu o~obvodovom a~stredovom uhle).
}

{%%%%%   Z7-I-6
\napad
Aký objem mala tridsiata nádoba?

\riesenie
S~tromi sedemlitrovými odmerkami by vodník rozlial 21~litrov hmly do $11+12+7=30$ nádob.
So štyrmi päťlitrovými odmerkami by rozlial 20 litrov hmly do $8+10+7+4=29$ nádob.
Posledná, tridsiata nádoba teda mala objem 1~liter.

Hmla z~prvej sedemlitrovej odmerky by naplnila presne 11~nádob, pritom prvých päť litrov by naplnilo presne 8~nádob (prvá päťlitrová odmerka) a~zvyšné dva litre presne 3~nádoby ($11-8=3$).
Táto časť tiež zodpovedá prvým dvom litrom z~druhej päťlitrovej odmerky.
Tá by však vystačila na 10~nádob, teda zvyšné tri litre by naplnili presne 7~nádob ($10-3=7$).
Podobne môžeme doplniť ďalšie podrobnosti o~skupinách nádob a~ich objemoch, ktoré schematicky znázorníme takto:
\figure{z7-I-6}%


Nádoby 1 až 8 pojmú dokopy presne 5~litrov,
nádoby 9 až 11 pojmú dokopy 2~litre,
nádoby 12 až 18 pojmú 3~litre,
nádoby 19 až 23 pojmú 4~litre,
nádoby 24 až 25 pojmú 1~liter, atď.

Posledné dve uvedené nádoby pojmú dokopy to isté čo samotná tridsiata nádoba,
preto má tridsiata nádoba väčší objem ako dvadsiatapiata.
}

{%%%%%   Z8-I-1
\napad
Vyjadrite uvedeným spôsobom čo najviac malých prirodzených čísel, ktoré by sa mohli ďalej hodiť.

\riesenie
Prirodzené čísla obsahujúce iba cifry 9 sú 9, 99, 999, 9\,999 atď.
Náhodné operácie s~týmito číslami vychádzajú všelijako, ale môžeme si všimnúť napr. nasledujúce výsledky:
$$
\frac99=1,\quad
\sqrt9=3,\quad
9+9=18,\quad
99-9=90\quad
\text{a pod.}
$$
V~ďalšom kroku vieme vyjadriť číslo 10, a~to napr. takto:
$$
10=9+\frac99=\frac{9\cdot9+9}9=\frac{99-9}9.
$$
Podobne možno vyjadriť 100, 1\,000 atď., teda aj milión:
$$
1\,000\,000
=999\,999+\frac99
=\frac{999\,999\cdot9+9}9
=\frac{9\,999\,999-999\,999}9.
$$

Z~ďalších nápadov z~prvého kroku môžeme vyjadriť napr.
$$
2=\frac99+\frac99=\frac{9+9}9,\quad
6=9-\sqrt9=\frac{9+9}{\sqrt9}\quad
\text{a pod.}
$$
Z~toho možno vyjadriť milión mnohými ďalšími spôsobmi, napr. takto:
$$
1\,000\,000=\left( 9+\frac99 \right)^{9-\sqrt9}.
$$

\poznamka
Pomocou $\frac99=1$ možno vyjadriť milión aj ako súčet milióna týchto zlomkov.
Tento a~podobné nápady však nie je možné hodnotiť, ak nie sú realizované vyššie opísaným spôsobom (teda sú bez slov alebo bodiek naznačujúcich pokračovanie istej myšlienky).
}

{%%%%%   Z8-I-2
\napad
Uvažujte osovú súmernosť podľa výšky na stranu~$KM$.

\riesenie
Na nasledujúcom obrázku sú znázornené údaje zo zadania, navyše päty všetkých výšok (body $P$, $Q$,~$R$) a~priesečníky osi uhla $PV\!M$ so stranami trojuholníka (body $A$,~$B$):
\figure{z8-I-2a}%

Os uhla~$PVM$ je rovnobežná so stranou~$KM$, preto obe tieto priamky sú kolmé na výšku~$LR$.
Teda pri osovej súmernosti podľa priamky~$LR$ sa ako priamka~$KM$, tak priamka~$AB$ zobrazuje sama na seba.
Uhly $PV\!M$ a~$QV\!K$ sú zhodné (vrcholové uhly) a~os uhla $PV\!M$ je tiež osou uhla $QV\!K$,
preto uhly $AV\!M$ a~$BV\!K$ sú zhodné.
Pri osovej súmernosti podľa priamky~$LR$ sa tak priamka~$PK$ zobrazuje na priamku~$QM$, teda bod~$K$ sa zobrazuje na bod~$M$.
Spolu zisťujeme, že trojuholník $KLM$ je súmerný podľa výšky~$LR$.
Preto sú uhly $MKL$ a~$LMK$ zhodné.

\napadd
Porovnajte uhly, ktoré určuje os uhla $PV\!M$ so stranami $KL$ a~$LM$.

\ineriesenie
Uhly $PV\!M$ a~$QV\!K$ sú zhodné (vrcholové uhly).
Os uhla $PVM$ je tiež osou uhla $QVK$, preto uhly $PV\!A$ a~$QVB$ sú zhodné.
Trojuholníky $PVA$ a~$QVB$ sú oba pravouhlé a~majú zhodné vnútorné uhly
pri vrchole~$V$, preto aj uhly $PAV$ a~$QBV$ sú zhodné.

Os~$AB$ je rovnobežná so stranou~$KM$, preto sú dvojice uhlov $PAV$,
$LMK$ a~$QBV$, $LKM$ zhodné (súhlasné uhly).
Keďže uhly $PAV$ a~$QBV$ sú zhodné, aj uhly $LMK$ a~$LKM$ sú zhodné.

\poznamka
Podľa zadania možno postupne určiť veľkosti rozličných uhlov a~takto nakoniec overiť, že uhly $MKL$ a~$LMK$ sú zhodné.
Veľkosti vybraných uhlov sú vyznačené na nasledujúcom obrázku:
\figure{z8-I-2b}%


}

{%%%%%   Z8-I-3
\napad
Aký je vzťah medzi najmenším spoločným násobkom a~najväčším spoločným deliteľom dvoch čísel?

\riesenie
Všetky štyri čísla boli navzájom rôzne, preto pôvodné dve čísla boli rôzne,
ich najväčší spoločný deliteľ bol menší ako každé z~týchto čísel a~najmenší spoločný násobok väčší ako každé z~týchto čísel.
Ak najväčšieho spoločného deliteľa označíme~$d$, môžeme pôvodné dve čísla zapísať ako $d\cdot x$ a~$d\cdot y$, pričom $x<y$ sú nesúdeliteľné čísla väčšie ako 1.
Najmenší spoločný násobok je potom rovný $d\cdot x\cdot y$.
Spolu teda máme
$$
d<d\cdot x<d\cdot y<d\cdot x\cdot y<100.
$$
Vlastnosť, ktorá Adelku uviedla do úžasu, znamená, že podiel $d\cdot x\cdot y$ a~$d$ je rovný $d$, čiže
$$
d=x\cdot y.
$$
Hľadáme teda nesúdeliteľné čísla $x<y$ väčšie ako 1 také, že $(x\cdot y)^2<100$, čiže $x\cdot y<10$.
Taká dvojica čísel je jediná:
\begin{itemize}
\item pre $x=2$ a~$y=3$ je $x\cdot y=6<10$,
\item pre $x=2$ a~$y=5$ je $x\cdot y=10$,
\item pre $x=3$ a~$y=4$ je $x\cdot y=12>10$,
\item atď.
\end{itemize}
Adelka mala napísané čísla $6\cdot2=12$ a~$6\cdot 3=18$, ku ktorým neskôr pripísala 6 a~$6\cdot 6=36$.
}

{%%%%%   Z8-I-4
\napad
Koľko mlynčekov pribudne za hodinu v~každej zo zmien?

\riesenie
V~dopoludňajšej trojhodinovej zmene pribudlo 27~mlynčekov, čo zodpovedá ${27:3}=9$ mlynčekom za hodinu.
Keďže Róbert rozoberá štyrikrát pomalšie, ako Hubert skladá, Hubert sám by zložil 9~mlynčekov za $\frac34$ hodiny, \tj. 45~minút.
Hubert teda zloží jeden mlynček za $45:9=5$ minút.

V~odpoludňajšej šesťhodinovej zmene pribudlo 120 mlynčekov, čo zodpovedá ${120:6}=20$ mlynčekom za hodinu.
Keďže tentoraz Róbert skladá a~Hubert rozoberá, Róbert sám by zložil 20~mlynčekov za $\frac34$ hodiny, \tj. 45~minút.
Róbert teda zloží jeden mlynček za $45:20=2{,}25$ minút, \tj. 2~minúty a~15~sekúnd.

\ineriesenie
Ak $h$ označuje počet mlynčekov, ktoré zloží Hubert za hodinu, a~$r$ počet mlynčekov, ktoré za hodinu zloží Róbert, tak za hodinu rozloží Hubert $\frac14r$ mlynčekov a~Róbert $\frac14h$ mlynčekov.
Informácie zo zadania vedú na rovnice
$$
\aligned
3\Big( h-\frac14h \Big) &=27, \\
6\Big( r-\frac14r \Big) &=120.
\endaligned
$$
Riešením prvej rovnice je $h=12$, teda Hubert zloží 12~mlynčekov za hodinu,
\tj. 60~minút.
Hubert zloží jeden mlynček za $60:12=5$ minút.
Riešením druhej rovnice je $r=\frac{80}3$, teda Róbert zloží
80~mlynčekov za 3~hodiny, \tj. 180 minút.
Róbert zloží jeden mlynček za $180:80=2{,}25$ minút.
}

{%%%%%   Z8-I-5
\napad
Aké sú obsahy ďalších obdĺžnikov?

\riesenie
Obsah obdĺžnika $HNCK$ je $63-12=51$\,(cm$^2$), obsah obdĺžnika $BLGN$ je $28-12=16$\,(cm$^2$).
Pomer obsahov obdĺžnikov $NGJC$ a~$HNCK$ je rovnaký ako pomer dĺžok úsečiek $NG$ a~$HN$, a~ten je rovnaký ako pomer obsahov obdĺžnikov $BLGN$ a~$MBNH$.
Teda
$$
S_{NGJC}:51=16:12=4:3,
$$
a~preto je obsah obdĺžnika $NGJC$ rovný
$51\cdot4:3=68\,(\Cm^2)$.

Keďže obdĺžniky $ABCD$ a~$EFGH$ sú zhodné a~posunuté, sú napr. úsečky $IE$ a~$CJ$ zhodné a~podobne je to s~ďalšími dvojicami.
Preto sú napr. obdĺžniky $IEMA$ a~$NGJC$ zhodné, a~teda majú rovnaký obsah.
Podobne je to s~ďalšími dvojicami:
\figure{z8-I-5a}%


Obsah obdĺžnika $IFJD$ je rovný $12+2\cdot51+2\cdot16+4\cdot68=418$\,(cm$^2$).
}

{%%%%%   Z8-I-6
\napad
Môže byť číslo $\m a$ väčšie ako $a$?

\riesenie
Číslo $a+1$ je o~1 väčšie ako číslo~$a$, leží teda na číselnej osi vpravo od čísla~$a$ a~vzdialenosť týchto dvoch čísel je rovnaká ako vzdialenosť hľadaných čísel 0 a~1.

O~vzájomnej polohe čísel $a$ a~$\m a$ nič nevieme; záleží na tom, či je číslo~$a$ kladné alebo záporné.
Keďže tiež nevieme nič o~absolútnej hodnote $|a|$ (\tj. vzdialenosti od nuly), nemôžeme porovnať ani čísla $\m a$ a~$a+1$.
Čísla $a$ a~$\m a$ však majú rovnakú absolútnu hodnotu, preto 0 leží na číselnej osi uprostred medzi týmito číslami.

Musíme teda uvažovať nasledujúce tri možnosti usporiadania čísel na číselnej osi:
\begin{itemize}
\item $\m a<a<a+1$,
\item $a<\m a<a+1$,
\item $a<a+1<\m a$.
\end{itemize}

Vo všetkých troch prípadoch možno zostrojiť 0 a~1 takto:
\begin{itemize}
\item bod predstavujúci 0 je stredom úsečky s~krajnými bodmi $a$ a~$\m a$,
\item bod predstavujúci 1 je vpravo od 0 v~rovnakej vzdialenosti ako $a+1$ od $a$.\figure{z8-I-6a}%
\figure{z8-I-6b}%
\figure{z8-I-6c}%
\end{itemize}


\poznamka
Pomer vzdialeností zadaných bodov na číselnej osi určuje hodnotu čísla~$a$ pre každé z~troch možných usporiadaní.
Ak by napr. tento pomer bol $1:2$ (ako na obrázku v~zadaní), tak by zodpovedajúce $a$ bolo v~prvom prípade $\frac12$, v~druhom prípade $\m\frac16$ a~v~treťom prípade $\m\frac32$.
}

{%%%%%   Z9-I-1
\napad
Aký je vzťah medzi počtom prítomných a~súčtom ich vekov?

\riesenie
Vekový priemer všetkých ľudí, ktorí sa zišli na oslave, je rovný podielu súčtu vekov všetkých prítomných (ozn.~$s$) a~ich počtu (ozn.~$n$).
Podľa zadania platí
$$
\frac{s}n=n,
\quad\text{čiže}\quad
s=n^2.
$$
Po odchode tety Bety sa počet prítomných zmenšil o~1 a~súčet ich vekov o~29.
Podľa zadania platí
$$
\frac{s-29}{n-1}=n-1,
\quad\text{čiže}\quad
s-29=(n-1)^2.
$$
Keď do poslednej rovnice dosadíme $s=n^2$, roznásobíme pravú stranu a~ďalej upravíme, dostaneme
$$
\begin{aligned}
n^2-29&=n^2-2n+1,\\
2n&=30,\\
n&=15.
\end{aligned}
$$
Na rodinnú oslavu sa pôvodne dostavilo 15~ľudí.}

{%%%%%   Z9-I-2
\napad
Čo viete povedať o~ďalších trojuholníkoch obsiahnutých v~lichobežníku?

\riesenie
Keďže $VO$ je dlhšou základňou lichobežníka $VODY$, bod~$K$ na uhlopriečke $V\!D$ je bližšie k~vrcholu~$D$.
\figure{z9-I-2}%


Trojuholníky $KOV$ a~$KDO$ majú spoločnú výšku z~vrcholu~$O$ a~dĺžky strán $VK$ a~$KD$ prislúchajúcich k~tejto výške sú v~pomere $3:2$.
Preto aj obsahy týchto trojuholníkov sú v~rovnakom pomere, teda
$$
S_{KDO}=\frac23 S_{KOV}.
$$
Trojuholníky $VOD$ a~$VOY$ majú spoločnú stranu~$VO$ a~rovnakú výšku na túto stranu, preto majú rovnaké obsahy.
Trojuholník $KOV$ je časťou oboch týchto trojuholníkov, preto majú rovnaké obsahy aj trojuholníky $KDO$ a~$KYV$,
$$
S_{KYV}=S_{KDO}=\frac23S_{KOV}.
$$
Trojuholníky $KYV$ a~$KDY$ majú spoločnú výšku z~vrcholu~$Y$ a~zodpovedajúce strany $VK$ a~$KD$ sú v~pomere $3:2$.
Preto aj obsahy týchto trojuholníkov sú v~rovnakom pomere,
$$
S_{KDY}=\frac23 S_{KYV}=\frac49S_{KOV}.
$$
Obsah celého lichobežníka je súčtom obsahov uvedených štyroch trojuholníkov, teda
$$
\aligned
S_{VODY}&=S_{KOV}+S_{KDO}+S_{KYV}+S_{KDY} =\\
&=\Bigl( 1+\frac23+\frac23+\frac49 \Bigr) S_{KOV}
=\frac{25}9\cdot13{,}5
=37{,}5\,(\Cm^2).
\endaligned
$$

\poznamka
Pri postupnom vyčísľovaní obsahov vyššie menovaných trojuholníkov dostávame
$$
S_{KDO}=S_{KYV}=9\cm^2,\quad
S_{KDY}=6\cm^2.
$$

Rovnosť $S_{KYV}=\frac23S_{KOV}$, resp. $S_{KDY}=\frac49S_{KOV}$ možno zdôvodniť priamo pomocou podobnosti trojuholníkov $KOV$ a~$KYD$
(koeficient podobnosti je $3:2$).
}

{%%%%%   Z9-I-3
\napad
Koľko mlynčekov pribudne za hodinu v~každej zo zmien?

\riesenie
V~dopoludňajšej päťhodinovej zmene pribudlo 70~mlynčekov, čo zodpovedá ${70:5}=14$ mlynčekom za hodinu.
V~odpoludňajšej deväťhodinovej zmene pribudlo 36~mlynčekov, čo zodpovedá $36:9=4$ mlynčekom za hodinu.
Ak by roboti pracovali jednu hodinu dopoludňajším spôsobom a~jednu hodinu odpoludňajším spôsobom,
vyrobili by $14+4=18$ mlynčekov a~pritom by vyrobili rovnaké množstvo mlynčekov,
ako keby spolu skladali (a~nič nerozoberali) $\frac34$~hodiny.
Roboti by spolu zložili 360 mlynčekov za $\frac34\cdot 20=15$ hodín, lebo $360=18\cdot 20$.

\ineriesenie
Ak $h$ označuje počet mlynčekov, ktoré zloží Hubert za hodinu, a~$r$ počet mlynčekov,
ktoré za hodinu zloží Róbert, tak za hodinu rozloží Hubert $\frac14h$ mlynčekov
a~Róbert $\frac14r$ mlynčekov.
Informácie zo zadania vedú na sústavu dvoch rovníc
$$
\aligned
5\Big( h-\frac14r \Big) &=70, \\
9\Big( r-\frac14h \Big) &=36.
\endaligned
$$
Riešením tejto sústavy dostaneme $r=8$ a~$h=16$.
Za hodinu by tak obaja roboti spolu zložili $r+h=24$ mlynčekov.
Teda 360 mlynčekov by spolu skladali $360:24=15$ hodín.
}

{%%%%%   Z9-I-4
\napad
Uvažujte paritu (párnosť, nepárnosť) čísel a~ich súčtov.

\riesenie
Ak by sa čísla dali rozsadiť do vagónov podľa ich želaní, tak by súčet troch najväčších čísel z~každého vagóna bol rovnaký ako súčet zvyšných šiestich čísel.
To by znamenalo, že súčet všetkých deviatich čísel by bol párny.
Avšak tento súčet je
$$
1+2+3+4+5+6+7+8+9=45,
$$
čo je nepárne číslo.
Preto čísla požadovaným spôsobom rozsadiť nemožno, pravdu mal výpravca.

\ineriesenie
Medzi číslami 1 až 9 je päť nepárnych ($N$) a~štyri párne ($P$).
Párne číslo možno vyjadriť ako súčet dvoch nepárnych čísel alebo ako súčet dvoch párnych čísel.
Nepárne číslo možno vyjadriť jedine ako súčet nepárneho a~párneho čísla.
V~každom vagóne teda môžu podľa uvedených požiadaviek sedieť iba nasledujúce skupiny čísel:
$$
\text{buď}\ \{N, N, P\},\ \text{alebo}\ \{P, P, P\}.
$$
Akýmkoľvek priradením týchto možností do troch vagónov dostaneme vždy spolu párny počet nepárnych čísel a~nepárny počet párnych čísel.
V~našom prípade je to však naopak, preto sa čísla nedajú rozsadiť podľa ich želaní.
Pravdu mal teda výpravca.

\poznamka
Pri akomkoľvek rozmiestnení piatich nepárnych čísel do troch vagónov bude vždy v~niektorom vagóne práve jedno alebo práve tri nepárne čísla.
A~v~takom vagóne určite nebude platiť požiadavka o~súčte.

\napadd
Ktoré čísla mohli cestovať s~číslom~9?

\ineriesenie
Môžeme postupne po vagónoch rozsadzovať čísla od najväčšieho tak, aby platila požiadavka o~ich súčte.
Číslo~9 môže cestovať s~niektorou z~nasledujúcich dvojíc:
%\begin{itemize}
\item{$\bullet$}
8 a~1: v~ďalšom vagóne môže 7 cestovať s~niektorou z~nasledujúcich dvojíc:
\itemitem{$\triangleright$} 5 a~2: na ďalší vagón ostáva 6, 4 a~3, ale $6\ne 4+3$,
\itemitem{$\triangleright$} 4 a~3: na ďalší vagón ostáva 6, 5 a~2, ale $6\ne 5+2$,
\smallskip
\item{$\bullet$}
7 a~2: v~ďalšom vagóne môže 8 cestovať jedine s~dvojicou
\itemitem{$\triangleright$} 5 a~3: na ďalší vagón ostáva 6, 4 a~1, ale $6\ne 4+1$,
\smallskip
\item{$\bullet$}
6 a~3: v~ďalšom vagóne môže 8 cestovať jedine s~dvojicou
\itemitem{$\triangleright$} 7 a~1: na ďalší vagón ostáva 5, 4 a~2, ale $5\ne 4+2$,
\smallskip
\item{$\bullet$}
5 a~4: v~ďalšom vagóne môže 8 cestovať s~niektorou z~nasledujúcich dvojíc:
\itemitem{$\triangleright$} 7 a~1: na ďalší vagón ostáva 6, 3 a~2, ale $6\ne 3+2$,
\itemitem{$\triangleright$} 6 a~2: na ďalší vagón ostáva 7, 3 a~1, ale $7\ne 3+1$.
%\end{itemize}
\smallskip

Zistili sme, že čísla sa nedajú rozsadiť tak, aby požiadavka o~súčte platila vo všetkých vagónoch.
Pravdu mal teda výpravca.
}

{%%%%%   Z9-I-5
\napad
Kde leží stred kružnice opísanej trojuholníku $AFD$?

\riesenie
Bod~$E$ je rovnako ďaleko od bodov $A$ a~$D$, bod~$F$ je rovnako ďaleko od bodov $B$ a~$C$ a~úsečky $AD$ a~$BC$ sú protiľahlými stranami obdĺžnika.
Preto body $E$ a~$F$ ležia na spoločnej osi úsečiek $AD$ a~$BC$.
Bod~$E$ má rovnakú vzdialenosť od všetkých vrcholov trojuholníka $AFD$, preto je stredom jemu opísanej kružnice.
Znázornenie bodov zo zadania vyzerá nasledovne (body $P$ a~$R$ sú priesečníky osi~$EF$ so stranami $AD$ a~$BC$, \tj. stredy týchto strán):
\figure{z9-I-5}%


Trojuholníky $APE$, $DPE$, $BRF$ a~$CRF$ sú navzájom zhodné pravouhlé trojuholníky, ktorých jedna odvesna je polovicou hľadanej strany obdĺžnika a~veľkosti zvyšných dvoch strán vieme odvodiť zo zadania.
Napr. v~trojuholníku $APE$ má prepona~$AE$ veľkosť 10\,cm a~odvesna~$PE$ má veľkosť $(22-10):2=6$\,cm.
Podľa Pytagorovej vety platí
$$
|PA|^2+6^2=10^2,
$$
teda $|PA|^2=64$ a~$|PA|=8$\,cm.
Dĺžka strany $BC$ je 16\,cm.

\poznamka
Na uvedenom obrázku mlčky naznačujeme, že strana~$AB$ je dlhšia ako $BC$.
To je síce potvrdené nasledujúcim výpočtom, ale možno si to všimnúť priamo:
Prepona~$AE$ v~pravouhlom trojuholníku $APE$ je dlhšia ako odvesna~$PA$, čo je polovica strany~$BC$.
Ak by strana~$BC$ bola dlhšia ako strana~$AB$, bola by úsečka~$AE$ tiež dlhšia ako polovica strany~$AB$.
Z~predchádzajúceho však vieme, že $|AE|=10$\,cm a~$\frac12|AB|=11$\,cm, takže táto situácia nastať nemôže.}

{%%%%%   Z9-I-6
\napad
Čo môžete povedať o~usporiadaní trojice čísel $a$, $2a$, $3a$?

\riesenie
Pred samotným rozborom možností si všimnime niekoľko užitočných faktov.
Keďže čísla majú byť navzájom rôzne, musí byť $a\ne0$.
Vzdialenosti medzi susednými číslami vo štvorici $0$, $a$, $2a$, $3a$ sú rovnaké, a~to $|a|$, pritom usporiadanie tejto štvorice závisí na znamienku $a$:
číslo $a$ je kladné práve vtedy, keď platí
$$
0<a<2a<3a, \tag{1}
$$
číslo $a$ je záporné práve vtedy, keď platí
$$
3a<2a<a<0. \tag{2}
$$
Bez ohľadu na znamienko $a$ ďalej platí
$$
3a<3a+1. \tag{3}
$$

Všetky možné usporiadania troch navzájom rôznych čísel sú tieto:
\begin{enumerate}\alphatrue
\item $a<2a<3a+1$,
\item $a<3a+1<2a$,
\item $3a+1<a<2a$,
\item $3a+1<2a<a$,
\item $2a<3a+1<a$,
\item $2a<a<3a+1$.
\end{enumerate}

Pre kladné $a$ podľa podmienok (1) a~(3) platí $a<2a<3a+1$, čo zodpovedá možnosti~a) a~súčasne vylučuje možnosti b) a~c).
Všeobecný vzťah medzi trojicou čísel vyhovujúcej možnosti~a) a~jej ukotvením na číselnej osi (\tj. číslami 0 a~1) je schematicky znázornený na nasledujúcom obrázku.
Konkrétny príklad usporiadania~a) je daný dosadením napr. $a=\frac23$, teda $\frac23<\frac43<3$.
\figure{z9-I-6a}%


Pre záporné $a$ nemôžeme z~podmienok (2) a~(3) o~vzťahu $3a+1$ vzhľadom k~$a$ a~$2a$ povedať nič bližšie.
Postupne ukážeme, že každý zo zvyšných prípadov je možný:

Všeobecné vzťahy medzi trojicami čísel vyhovujúcich možnostiam d), e), resp. f) a~číslami 0 a~1 sú schematicky znázornené na nasledujúcich obrázkoch.
Konkrétny príklad usporiadania~d) je daný dosadením napr. $a=\m2$, teda $\m5<\m4<\m2$.
\figure{z9-I-6b}%


\noindent
Konkrétny príklad usporiadania~e) je daný dosadením napr. $a=\m\frac23$, teda $\m\frac43<\m1<\m\frac23$.
\figure{z9-I-6c}%


\noindent
Konkrétny príklad usporiadania~f) je daný dosadením napr. $a=\m\frac14$, teda $\m\frac18<\m\frac14<\frac14$.
\figure{z9-I-6d}%


\poznamka
V~prípadoch e) a~f) môžeme voliť trojice bodov predstavujúcich čísla $a$, $2a$ a~$3a+1$ úplne ľubovoľne; naznačeným spôsobom odvodíme umiestnenie 0 a~1, a~tým vlastne určíme hodnotu~$a$.
V~prípadoch a) a~d) to tak nie je;
napr. pre usporiadanie~d) a~bod $3a+1$ zvolený príliš vľavo od bodu $2a$ sa môže stať, že $3a$ vyjde niekde medzi, čo by bolo v~rozpore s~podmienkou (3).

\napadd
Pre jednotlivé usporiadania odvoďte možné hodnoty~$a$.

\ineriesenie
Všetky možné usporiadania troch navzájom rôznych čísel sú uvedené v~zozname a) až f) v~predchádzajúcom riešení.
Riešením nerovností v~jednotlivých prípadoch zisťujeme, že prípad
\begin{enumerate}\alphatrue
\item je splnený pre ľubovoľné $a>0$,
\item nie je splnený pre žiadne $a$,
\item nie je splnený pre žiadne $a$,
\item je splnený pre ľubovoľné $a<\m1$,
\item je splnený pre ľubovoľné $\m1<a<\m\frac12$,
\item je splnený pre ľubovoľné $\m\frac12<a<0$.
\end{enumerate}

Pre ilustráciu uvádzame podrobnosti k~prípadu~b):
zodpovedajúce nerovnosti sú splnené práve vtedy, keď platí
$$
a<3a+1\quad\text{a}\quad 3a+1<2a,
$$
čo je ekvivalentné s~dvojicou nerovností
$$
-\frac12<a\quad\text{a}\quad a<-1.
$$
Tieto dve podmienky súčasne nespĺňa žiadne reálne číslo, preto prípad~b) nastať nemôže.

Diskusia v~ostatných prípadoch je obdobná.
Vo všetkých prípadoch, ktoré majú riešenie, stačí pre konkrétnu odpoveď zvoliť ľubovoľné~$a$ z~určeného intervalu.}

{%%%%%   Z4-II-1
...}

{%%%%%   Z4-II-2
...}

{%%%%%   Z4-II-3
...}

{%%%%%   Z5-II-1
Označme jednotlivé výbehy ciframi ako na obrázku:
\insp{z5-II-1a.eps}%

Podľa prvej informácie sú žirafy buď vo výbehu 3, alebo 4.
Podľa druhej informácie vieme, že výbeh žiráf nemá susediť s~výbehom opíc.
Výbeh 4 však susedí so všetkými ostatnými výbehmi, preto musia byť žirafy
vo výbehu~3.

Jediný výbeh, ktorý s~výbehom žiráf nesusedí, je výbeh~1.
Podľa druhej informácie musia byť vo výbehu 1 opice.

Okrem výbehu žiráf nesusedí s~výbehom opíc už len výbeh~5.
Podľa druhej informácie musia byť vo výbehu 5 nosorožce.

Jediný výbeh, ktorý má rovnaký počet strán ako výbeh opíc, je výbeh~2.
Podľa tretej informácie musia byť vo výbehu 2 levy.

Ostáva jediný neobsadený výbeh, a~to výbeh 4.
Tulene sú teda vo výbehu~4.

\hodnotenie
Po 1~bode za umiestnenie zvierat do výbehov, 1~bod za kvalitu komentára.
Čiastkové odpovede typu \uv{žirafy sú vo výbehu 3 alebo~4} hodnoťte po 1~bode.
Za riešenie bez zdôvodnenia dajte nanajvýš 2 body.
\endhodnotenie}

{%%%%%   Z5-II-2
Číslo 12 možno ako súčin dvoch kladných celých čísel, z~ktorých prvé je menšie ako druhé, získať tromi spôsobmi:
$$
1\cdot12=2\cdot6=3\cdot4.
$$

Ak by najmenšie dve čísla boli 1 a~12, najväčšie z~čísel by malo byť $1+8=9$.
To však nie je možné, keďže 12 nie je menšie ako~9.

Ak by najmenšie dve čísla boli 2 a~6, najväčšie číslo by bolo $2+8=10$.
Zvyšné dve čísla potom majú byť rôzne čísla väčšie ako~6 a~menšie ako 10, ktoré spolu s~číslom~10 dávajú súčet 25.
To spĺňajú iba čísla 7 a~8.

Ak by najmenšie dve čísla boli 3 a~4, najväčšie číslo by bolo $3+8=11$.
Zvyšné dve čísla potom majú byť rôzne čísla väčšie ako~4 a~menšie ako~11, ktoré spolu s~číslom~11 dávajú súčet~25.
To spĺňajú dvojice čísel 5 a~9 a~aj 6 a~8.

Tri vyhovujúce pätice čísel teda sú
$$
(2, 6, 7, 8, 10),\quad (3, 4, 5, 9, 11),\quad (3, 4, 6, 8, 11).
$$

\hodnotenie
2 body za každú päticu čísel: vždy 1 bod za nájdenie danej pätice a~1 bod za odvodenie alebo overenie, že vyhovuje zadaniu.
\endhodnotenie
}

{%%%%%   Z5-II-3
Konštrukcia:
\begin{itemize}
\item Štvorec $ABCD$ so stranou dĺžky 6\,cm,
\item bod $S$ ako priesečník úsečiek $AC$ a~$BD$,
\item bod $K$ ako priesečník kolmice na~$SB$ v~bode~$B$ a~kolmice na~$SC$ v~bode~$C$,
\item bod $L$ ako priesečník kolmice na~$SA$ v~bode~$A$ a~kolmice na~$SD$ v~bode~$D$,
\item body $X$ a~$Y$ ako priesečníky úsečky~$KL$ so stranami $AD$ a~$BC$.
\end{itemize}

Výpočet:
Dĺžka lomenej čiary $KYBAXL$ je súčtom dĺžok úsečiek $KY$, $YB$, $BA$, $AX$ a~$XL$.
Pritom dĺžka úsečky~$AB$ je 6\,cm.

Bod~$Y$ je priesečníkom uhlopriečok štvorca $BKCS$, preto je stredom každej z~týchto úsečiek.
Úsečka~$BC$ je stranou zadaného štvorca, a~tá meria 6\,cm.
Každá z~úsečiek $KY$ a~$YB$ teda meria 3\,cm.

Z~obdobného dôvodu aj každá z~úsečiek $AX$ a~$XL$ meria 3\,cm.

Dĺžka lomenej čiary $KYBAXL$ je
$3+3+6+3+3= 18\,(\Cm)$.
\insp{z5-II-3a.eps}%

\hodnotenie
2 body za prevedenie konštrukcie.
4 body za výpočet, z~toho
2~body za určenie a~zdôvodnenie, že body $X$ a~$Y$ sú stredmi prislúchajúcich štvorcov a~2 body za doriešenie a~výsledok.
Len za sčítanie dĺžok úsečiek bez zdôvodnenia (alebo s~odkazom na meranie zostrojených dĺžok v~obrázku) dajte nanajvýš 1~bod.
\endhodnotenie
}

{%%%%%   Z6-II-1
Keďže 78 strán zodpovedá tretine knihy, má celá kniha $3\cdot 78=234$ strán.
Jankovi ostáva prečítať ešte $234-78=156$ strán.

Od Vianoc do 31.~januára uplynulo celkom $8+31=39$ dní.
Preto Janko denne prečítal $78:39=2$ strany.
Ak by počínajúc 1.~februárom denne prečítal o~štyri strany viac
ako doposiaľ, teda šesť strán, dočítal by knihu za $156:6=26$~dní.
Janko má narodeniny 26.~februára.

\hodnotenie
Po 1~bode za každý z~čiastočných výsledkov
(kniha má 234 strán, ostáva dočítať 156 strán, do konca januára čítal 39~dní, čítal 2~strany denne, knihu dočíta za 26~dní);
1~bod za deň narodenín.
\endhodnotenie
}

{%%%%%   Z6-II-2
Medzi uvedenými číslami je päť nepárnych čísel a~štyri párne čísla.
Po strate štyroch nepárnych čísel tak Evičke zostalo jedno nepárne číslo a~všetky štyri párne (2, 4, 6, 8).
Keďže 24 nie je násobkom 5, 7, ani 9, avšak je násobkom 1 a~3, mohlo byť zvyšné nepárne číslo buď~1, alebo 3.

Po strate dielikov so súčinom 24 zvýšili len dve párne čísla.
Tento súčin preto musel byť vyjadrený pomocou jedného nepárneho čísla (a~to buď 1, alebo 3) a~dvoch párnych čísel (vybraných z~2, 4, 6 a~8).
To je možné buď ako $24=1\cdot4\cdot 6$, alebo ako $24=3\cdot2\cdot 4$.
V~prvom prípade by na posledných dvoch dielikoch boli čísla 2 a~8, v~druhom prípade by to boli čísla 6 a~8.

\hodnotenie
2~body za zistenie, že zvyšné nepárne číslo bolo buď~1, alebo 3;
2~body za zistenie, že 24 má byť súčinom jedného nepárneho a~dvoch párnych čísel;
2~body za rozbor možností a~záver.

Ak riešiteľ najskôr uvažuje všemožné súčiny ako $24=3\cdot8=1\cdot3\cdot8$ a~pod. a~z toho vyberá vyhovujúce možnosti, musí byť z~komentára zrejmé, prečo vyhovujú.
Za riešenie bez zdôvodnenia dajte nanajvýš 2~body.
\endhodnotenie
}

{%%%%%   Z6-II-3
Konštrukcia:
\begin{itemize}
\item Štvorec $ABCD$ so stranou dĺžky 6\,cm,
\item priamka~$p$ ako rovnobežka s~priamkou~$AC$ (resp. kolmica na priamku~$BD$) idúca bodom~$D$,
\item bod~$E$ ako päta kolmice na priamku~$p$ idúcej bodom~$C$,
\item bod~$F$ ako päta kolmice na priamku~$p$ idúcej bodom~$A$.\insp{z6-II-3.eps}%
\end{itemize}

Výpočet:
Uhlopriečky vo štvorci sú zhodné, navzájom kolmé a~ich priesečník je stredom oboch uhlopriečok.
Priesečník uhlopriečok vo štvorci $ABCD$ označíme~$S$.

Z~uvedeného vyplýva, že štvorec $ABCD$ je uhlopriečkami rozdelený na štyri navzájom zhodné trojuholníky $ABS$, $BCS$, $CDS$ a~$DAS$.
Ďalej obdĺžnik $ACEF$ je úsečkou~$SD$ rozdelený na dva zhodné štvorce, pričom každý z~nich je ďalej rozdelený uhlopriečkou ($CD$, resp. $DA$) na dva zhodné trojuholníky.
Ako štvorec $ABCD$, tak obdĺžnik $ACEF$ teda pozostáva zo štyroch navzájom zhodných trojuholníkov, z~ktorých dva sú obom útvarom spoločné.
Preto má obdĺžnik $ACEF$ rovnaký obsah ako štvorec $ABCD$, a~ten je
$$
6\cdot6=36\,(\Cm^2).
$$

\hodnotenie
2~body za prevedenie konštrukcie;
4~body za výpočet, z~toho 3~body za rozdelenie na zhodné trojuholníky a~1~bod za určenie obsahu.
Výpočet založený len na meraní v~obrázku, resp. správny výsledok bez zdôvodnenia nehodnoťte.
\endhodnotenie
}

{%%%%%   Z7-II-1
Zo zadania vyplýva, že 5~klasických gerber stojí rovnako ako 7~minigerber, teda že cena klasickej gerbery a~cena minigerbery sú v~pomere $7:5$.
Ak cenu klasickej gerbery znázorníme 7~rovnakými dielmi, cena minigerbery bude zodpovedať 5~takým dielom.

Cena nákupu 2~minigerber a~1~klasickej gerbery predstavuje $2\cdot5+7=17$ týchto dielov, 1~diel preto zodpovedá sume $10{,}20:17=0{,}60$ eur.
Jedna minigerbera stojí $5\cdot0{,}60 = 3$ eurá.

Overíme ešte, že cena stuhy vychádza nezáporné číslo: Päť klasických gerber, resp. sedem minigerber stojí $5\cdot7\cdot0{,}60=21$ eur.
Cena stuhy je $29{,}50-21=8{,}50$ eur.

\hodnotenie
2~body za zistenie, že ceny klasickej gerbery a~minigerbery sú v~pomere $7:5$;
2~body za zistenie zodpovedajúce rovnici $17d=10,20$;
2~body za odvodenie ceny minigerbery ($d=0{,}60$, ${5\cdot d}=3$).
Ak aj žiak neoverí, že cena stuhy vychádza nezáporné číslo, body nestŕhajte.
\endhodnotenie
}

{%%%%%   Z7-II-2
Filipovo číslo má byť deliteľné piatimi, preto z~ponúkaných cifier musí na miesto jednotiek použiť~5.
V~Kamilinom čísle sa teda 5 vyskytovať nemôže.
Zároveň má jej číslo byť väčšie ako~500, musí teda mať na mieste stoviek cifru 6.

Filipovo číslo má byť deliteľné tromi, tzn. jeho ciferný súčet má byť deliteľný tromi,
teda súčet prvých dvoch cifier má po delení tromi dávať zvyšok 1.
Z~ponúkaných cifier tak Filip mohol zložiť čísla
$$
135,\quad 315,\quad 345,\quad 435.
$$
Kamilino číslo má byť deliteľné štyrmi, tzn. jeho posledné dvojčíslie má byť deliteľné štyrmi.
Z~ponúkaných cifier tak Kamila mohla zložiť čísla
$$
612,\quad 624,\quad 632.
$$
Zároveň má platiť, že čísla použité vo Filipovom čísle nemôžu byť v~Kamilinom čísle, a~naopak.
Filip a~Kamila teda mohli zložiť nasledujúce dvojice čísel:
$$
\begintable
Filip\|135|315|345|\hfill435\cr
Kamila\|624|624|612|\hfill612\crthick
Súčet\|759|939|957|1\,047\endtable
$$

Jediný súčet, ktorý sa číta zľava rovnako ako sprava, je 939.
Filip zložil číslo 315, Kamila zložila číslo 624.

\hodnotenie
Celkom 1~bod za zistenie, že Filipovo číslo končí cifrou 5 a~Kamilino číslo začína cifrou 6;
po 2~bodoch za určenie možných Filipových, resp. Kamiliných čísel vyhovujúcich podmienkam deliteľnosti;
1~bod za výber možných dvojíc, určenie ich súčtov a~doriešenie.

Ak riešiteľ prehliadne niektorú z~podmienok a~okrem jediného možného riešenia uvedie ďalšie (napr. $632+145=777$), dajte nanajvýš 4~body.
Za správne riešenie bez zdôvodnenia dajte nanajvýš 2~body.

\poznamka
Súčet všetkých šiestich daných cifier je deliteľný tromi a~ciferný súčet Filipovho čísla má byť deliteľný tromi, preto aj ciferný súčet Kamilinho čísla musí byť deliteľný tromi.
Tento poznatok vo vyššie uvedenom riešení vylučuje možnosť 632 medzi Kamilinými číslami.
\endhodnotenie
}

{%%%%%   Z7-II-3
Rozbor:

Úsečky $AC$ a~$BD$ sú zhodné, navzájom kolmé a~stred každej z~nich je totožný s~ich priesečníkom.
Preto body $A,B,C,D$ tvoria vrcholy štvorca.
Trojuholník $ABK$ je rovnoramenný so základňou~$AB$, preto jeho vrchol~$K$ leží na osi úsečky~$AB$.
Os úsečky~$AB$ je totožná s~osou úsečky~$CD$ a~na tejto priamke leží aj stred~$S$ štvorca $ABCD$ a~vrchol~$M$ rovnoramenného trojuholníka $CDM$.
Obdobné pozorovanie platí pre trojicu bodov $L$, $S$ a~$N$.
\insp{z7-II-3.eps}%

\medskip

Konštrukcia:
\begin{itemize}
\item Kružnica so stredom~$S$ a~polomerom 3\,cm,
\item navzájom kolmé priamky prechádzajúce bodom~$S$, priesečníky s~kružnicou označené $A,C$ a~$B,D$,
\item štvoruholník (štvorec) $ABCD$,
\item kolmica na priamku~$AB$ (totožná s~kolmicou na priamku~$CD$) idúca bodom~$S$, päta kolmice označená~$O$,
\item bod~$K$ na polpriamke~$SO$ vo vzdialenosti $|AB|$ od bodu~$O$,
\item trojuholník $ABK$,
\item ostatné body a~trojuholníky analogicky.
\end{itemize}

\medskip
Výpočet:

Mnohouholník $AKBLCMDN$ je zložený zo štvorca $ABCD$ a~štyroch navzájom zhodných rovnoramenných trojuholníkov $ABK$, $BCL$, $CDM$, $DAN$.

Trojuholník $ABK$ má ako základňu stranu štvorca $ABCD$ a~zodpovedajúca výška je s~ňou zhodná.
Preto je obsah trojuholníka $ABK$ rovný polovici obsahu štvorca $ABCD$ a~obsah celého mnohouholníka $AKBLCMDN$ je rovný trojnásobku obsahu štvorca $ABCD$.

Štvorec $ABCD$ je uhlopriečkami rozdelený na štyri navzájom zhodné trojuholníky $ABS$, $BCS$, $CDS$ a~$DAS$.
Každý z~týchto trojuholníkov je pravouhlý a~rovnoramenný s~ramenami dĺžky 3\,cm.
Obsah štvorca je teda rovný
$$
\frac{4\cdot3\cdot3}2=18\,(\Cm^2).
$$
Obsah mnohouholníka $AKBLCMDN$ je teda rovný
$
3\cdot 18=54\,(\Cm^2).
$

\hodnotenie
2~body za rozbor a~prevedenie konštrukcie;
4~body za výpočet, z~toho 2~body za určenie vzťahu medzi obsahmi trojuholníka $ABK$ a~štvorca $ABCD$, 2~body za vyčíslenie obsahov štvorca a~celého mnohouholníka.
Výpočet založený len na meraní v~obrázku, resp. správny výsledok bez zdôvodnenia nehodnoťte.
\endhodnotenie
}

{%%%%%   Z8-II-1
V~Jožkovom výraze je práve deväť rôznych písmen, pričom písmená $M$ a~$A$ sa opakujú štyrikrát a~písmeno~$T$ dvakrát.
Pri ľubovoľnom nahradení písmen ciframi podľa uvedených požiadaviek platí, že súčet $M+A+R+D+T+E+I+K+U$ je rovný súčtu všetkých cifier od~1 po~9, čo je~45.
Jožkov súčet teda môžeme vyjadriť ako
$$
4M+4A+R+D+2T+E+I+K+U=3M+3A+T+45.
$$

Najväčší súčet možno dostať tak, že najčastejšie sa opakujúce písmená sú nahradené najväčšími možnými ciframi.
Stačí teda písmená $M$ a~$A$ nahradiť ciframi 9 a~8 (v~ľubovoľnom poradí) a~písmeno~$T$ cifrou~7.
Najväčší súčet, ktorý mohol Jožko dostať, je
$$
3\cdot9+3\cdot8+7+45=103.
$$

Súčet~50 zodpovedá takému nahradeniu písmen, že $3M+3A+T+45=50$, teda $3M+3A+T=5$.
To však nie je možné, lebo pri akomkoľvek nahradení je posledný uvedený súčet určite väčší ako $3+3+1=7$.
Súčet~50 Jožko dostať nemohol.

Súčet~59 zodpovedá takému nahradeniu písmen, že $3M+3A+T+45=59$, teda $3(M+A)+T=14$.
Pritom súčet $M+A$ je najmenej 3 ($=1+2$) a~nanajvýš 4 (ak $M+A>4$, tak $3(M+A)+T>14$):
\begin{itemize}
\item ak $M+A=1+2=3$, tak $T=14-3\cdot3=5$,
\item ak $M+A=1+3=4$, tak $T=14-3\cdot4=2$.
\end{itemize}
V~oboch prípadoch písmená $M$, $A$ a~$T$ zodpovedajú navzájom rôznym cifrám.
Pri súčte~59 mohlo byť písmeno~$T$ nahradené buď cifrou 5, alebo 2.

\hodnotenie
Po 1~bode za odpoveď na každú z~jednotlivých otázok;
3~body za úplnosť a~kvalitu komentára.
\endhodnotenie
}

{%%%%%   Z8-II-2
Ak sú hrany kocôčok $n$-krát menšie ako hrana pôvodnej kocky, tak
táto kocka bola rozdelená na $n\cdot n\cdot n=n^3$ kocôčok.
Povrch kocky je $6\cdot12^2\,(\Cm^2)$.
Súčet povrchov všetkých kocôčok je
$$
n^3\cdot 6\cdot\left(\frac{12}n\right)^{\!2} =n\cdot 6\cdot12^2\,(\Cm^2).
$$
Aby bol tento súčet osemkrát väčší ako povrch pôvodnej kocky, musí byť $n=8$.
Kocka bola rozdelená na $8^3=512$ kocôčok a~hrana každej z~nich merala $12:8=1{,}5$\,(cm).

\hodnotenie
2~body za vzťah medzi pomerom dĺžok hrán kocky a~kocôčok ($1:n$) a~počtom všetkých kocôčok ($n^3$);
2~body za vyjadrenie a~porovnanie povrchov;
2~body za doriešenie.

\poznamka
K~tomu istému výsledku možno dôjsť aj postupným skúšaním delenia kocky, ktoré vedie k~výpočtom zodpovedajúcim dosadzovaniu $n=2,3,\dots$ do predchádzajúcich výrazov.
V~takom prípade prispôsobte hodnotenie vzhľadom na kvalitu komentára.
\endhodnotenie
}

{%%%%%   Z8-II-3
Ako štvoruholník $ABDE$, tak trojuholník $ABF$ možno rozdeliť na dva trojuholníky, z~ktorých $ABD$ je spoločný obom.
Zvyšné časti, \tj. trojuholníky $ADE$ a~$ADF$, preto musia mať rovnaký obsah.
Tieto dva trojuholníky majú spoločnú stranu~$AD$, preto musia mať aj rovnakú výšku na túto stranu.
To znamená, že body $E$ a~$F$ ležia na rovnobežke s~priamkou~$AD$.

Keďže je bod~$E$ stredom strany~$AC$, je aj bod~$F$ stredom úsečky~$CD$.
Pritom bod~$D$ je stredom úsečky~$BC$, bod~$F$ je preto v~troch štvrtinách úsečky~$BC$.
Hľadaná dĺžka úsečky~$BF$ je $4+2=6$\,cm.
\insp{z8-II-3a.eps}%

\ineriesenie
Rovnostranný trojuholník $ABC$ je svojimi strednými priečkami rozdelený na štyri zhodné trojuholníky, z~ktorých tri tvoria štvoruholník $ABDE$.
Preto aj obsah trojuholníka $ABF$ je rovný trom štvrtinám obsahu trojuholníka $ABC$.
Tieto dva trojuholníky majú rovnakú výšku zo spoločného vrcholu~$A$, preto dĺžka strany~$BF$ je rovná trom štvrtinám dĺžky strany~$BC$.
Hľadaná dĺžka úsečky~$BF$ je 6\,cm.
\insp{z8-II-3b.eps}%

\ineriesenie
Úsečka~$ED$ je strednou priečkou trojuholníka $ABC$, preto je rovnobežná s~$AB$ a~má dĺžku $8:2=4$\,(cm).
Štvoruholník $ABDE$ je lichobežníkom so základňami dĺžok 8\,cm a~4\,cm a~výškou, ktorá je rovná polovici výšky trojuholníka $ABC$.

Ak veľkosť výšky trojuholníka $ABC$ označíme~$v$, tak obsah lichobežníka $ABDE$, resp.
obsah trojuholníka $ABF$ je
$$
\frac{(8+4)\cdot\frac12{v}}2=3 v,
\quad\text{resp.}\quad
\frac{|BF|\cdot v}2.
$$
Podľa zadania sú tieto obsahy rovnaké, preto $|BF|=6$\,cm.

\hodnotenie
2~body za akýkoľvek poznatok vedúci k~jednoznačnému vymedzeniu bodu~$F$;
2~body za doriešenie;
2~body podľa kvality komentára.
\endhodnotenie
}

{%%%%%   Z9-II-1
Prvý deň Terezka zjedla $\frac25$ toho, čo zjedla Štefka.
Počet cukríkov zjedených ten deň Štefkou preto musel byť deliteľný 5.
Druhý deň zjedla Štefka $\frac34$ toho, čo zjedla Terezka.
Počet cukríkov zjedených ten deň Terezkou preto musel byť deliteľný 4.
Počty cukríkov zjedených dievčatami v~jednotlivých dňoch uvádzame v~nasledujúcej tabuľke, pričom $x$ a~$y$ sú neznáme prirodzené čísla:
$$
\begintable
\|1. deň|2. deň\crthick
Štefka\|$5x$|$3y$\cr
Terezka\|$2x$|$4y$\endtable
$$

Spolu sa za oba dni zjedlo 35~cukríkov, teda
$$
7x+7y=35,
\quad\text{čiže}\quad
x+y=5.
\tag{1}
$$
Rozdiel medzi počtom cukríkov zjedených Terezkou a~počtom cukríkov zjedených Štefkou je
$$
|(2x+4y)-(5x+3y)|=|y-3x|.
\tag{2}
$$
Ak z~(1) vyjadríme $y=5-x$ a~dosadíme do (2), dostaneme rozdiel $|5-4x|$.
Tento rozdiel je najmenší možný práve vtedy, keď $x=1$.
Z~toho vyplýva, že $y=4$ a~že Terezka celkom zjedla $2\cdot1+4\cdot4=18$ cukríkov.

\poznamka
Dvojice prirodzených čísel $(x,y)$ vyhovujúce rovnici (1) sú štyri:
$$
(1, 4),\quad (2, 3),\quad (3, 2),\quad (4, 1).
$$
Dosadením každej z~týchto dvojíc do úvodnej tabuľky dostaneme konkrétne hodnoty, medzi ktorými možno vybrať tú s~najmenším rozdielom medzi počtami cukríkov zjedených Terezkou a~Štefkou.
Toto riešenie zodpovedá prvej uvedenej dvojici:
$$
\begintable
\|1. deň|2. deň|celkom\crthick
Štefka\|$5$|$12$|17\cr
Terezka\|$2$|$16$|18\endtable
$$

\ineriesenie
Keďže celkový počet cukríkov bol 35, čo je nepárne číslo, najmenší možný rozdiel medzi cukríkmi zjedenými Štefkou a~Terezkou by mohol byť~1.
V~takom prípade by buď Štefka zjedla 18~cukríkov a~Terezka~17, alebo naopak.
Pri rovnakom označení ako v~úvode predchádzajúceho riešenia by prvá možnosť viedla na sústavu rovníc
$$
\aligned
5x+3y&=18,\\
2x+4y&=17.
\endaligned
$$
Po úpravách dostávame riešenie $x=\frac32$ a~$y=\frac72$, ktoré však nie je vyhovujúce, keďže nie je celočíselné.
Druhá uvedená možnosť by viedla na sústavu rovníc
$$
\aligned
5x+3y&=17,\\
2x+4y&=18,
\endaligned
$$
ktorá po úpravách dáva riešenie $x=1$ a~$y=4$.
Terezka celkom zjedla 18~cukríkov.

\hodnotenie
2 body za vzťahy medzi zjedenými cukríkmi v~jednotlivých dňoch (napr. ako v~úvodnej tabuľke);
2 body za zostavenie rovníc;
2 body za doriešenie a~kvalitu komentára.
\endhodnotenie
}

{%%%%%   Z9-II-2
Zo zadania vieme, že (1) trojuholníky $ABS$ a~$CDS$ sú rovnostranné a~(2)~dĺžky ich strán sú v~pomere $2:1$.
Z~prvého poznatku vyplýva, že všetky vnútorné uhly v~týchto trojuholníkoch majú veľkosť~60\st.
Preto je veľkosť uhla $ESC$ rovná ${180\st-60\st}=120\st$.
Keďže veľkosť uhla $SCE$ je~30\st, na veľkosť uhla $SEC$ v~trojuholníku $SEC$ ostáva tiež~30\st.
Trojuholník $SEC$ je teda rovnoramenný so základňou~$EC$, a~preto sú úsečky $SE$ a~$SC$ zhodné.
Z~druhého poznatku vyplýva, že úsečka~$SB$ má dvojnásobnú dĺžku v~porovnaní s~úsečkou~$SC$.
Spolu teda zisťujeme, že bod~$E$ leží v~strede úsečky~$SB$, resp. úsečka~$DB$ je bodmi $S$ a~$E$ rozdelená na tretiny.
\insp{z9-II-2a.eps}%

Trojuholníky $DBC$ a~$EBC$ majú spoločný vrchol~$C$ a~dĺžky jemu protiľahlých strán $DB$ a~$EB$ sú v~pomere $3:1$.
V~rovnakom pomere sú preto aj ich obsahy,
$$
S_{DBC}:S_{EBC}=3:1.
$$
Lichobežník $ABCD$ je uhlopriečkou $DB$ rozdelený na dva trojuholníky $DBA$ a~$DBC$, ktoré majú spoločnú výšku s~lichobežníkom a~dĺžky prislúchajúcich strán $AB$ a~$CD$ sú v~pomere $2:1$.
V~rovnakom pomere sú preto aj obsahy týchto trojuholníkov, tzn.
$$
S_{ABCD}:S_{DCB}=3:1.
$$
Spolu teda zisťujeme, že hľadaný pomer obsahov je
$$
S_{ABCD}:S_{ECB}=9:1.
$$

\poznamka
Trojuholníky $ABS$ a~$CDS$ sú podobné s~koeficientom podobnosti $2:1$, ich obsahy sú preto v~pomere $4:1$.
Názorne možno tento pomer ukázať rozdelením trojuholníka $ASB$ pomocou stredných priečok na štyri trojuholníky zhodné s~$CDS$, pozri obrázok.
Doplnením úsečky~$DF$ je lichobežník $ABCD$ rozdelený na deväť trojuholníkov majúcich rovnaký obsah ako trojuholník $EBC$.
\insp{z9-II-2b.eps}%

\hodnotenie
3~body za určenie, že bod~$E$ je v~strede úsečky~$SB$;
po 1~bode za každý zo zvýraznených pomerov (alebo zodpovedajúce čiastočné kroky pri inom riešení).
\endhodnotenie}

{%%%%%   Z9-II-3
Uvedený výraz môžeme vyjadriť ako
$$
\sqrt{a\cdot b\cdot c}=\sqrt{a\cdot b}\cdot\sqrt{c}=99\sqrt2\cdot\sqrt{c}.
\tag{1}
$$

Pre $c=\sqrt2$ je
$$\sqrt{a\cdot b\cdot c} =99\sqrt2\cdot\sqrt{\sqrt{2}},
$$
čo nie je prirodzené číslo.
Anna teda má pravdu.

Pre $c=98$ je
$$\sqrt{a\cdot b\cdot c} =99\sqrt2\cdot\sqrt{98} =99\sqrt{2\cdot2\cdot7\cdot7} =99\cdot2\cdot7,
$$
čo je prirodzené číslo.
Dana teda má pravdu.

Napr. pre $c=4$ je
$$\sqrt{a\cdot b\cdot c} =99\sqrt2\cdot2,
$$
čo nie je prirodzené číslo.
Hana teda nemá pravdu.

Pre $c=98$ sme už ukázali, že $\sqrt{a\cdot b\cdot c}$ je prirodzené číslo.
Ďalej napr. pre $c=2$ je
$$\sqrt{a\cdot b\cdot c} =99\sqrt2\cdot\sqrt2 =99\cdot2,
$$
čo je prirodzené číslo.
Jana teda nemá pravdu.

\ineriesenie
Aby výraz (1) predstavoval prirodzené číslo, muselo by byť
$$
\sqrt{c}=\sqrt2\cdot k,
\quad\text{čiže}\quad
c=2\cdot k^2
\tag{2}
$$
pre nejaké prirodzené číslo~$k$.

Keďže $c=\sqrt2$ nie je tvaru (2), Anna má pravdu.

Keďže $c=98=2\cdot 7^2$ je tvaru (2), Dana má pravdu.

Keďže nie každé párne číslo je tvaru (2),
Hana nemá pravdu.

Keďže čísel tvaru (2) je nekonečne veľa, Jana nemá pravdu.

\hodnotenie
Po 1~bode za zhodnotenie výrokov Anny a~Dany;
po 2~bodoch za zhodnotenie výrokov Hany a~Jany.
Odpovede bez zdôvodnenia nehodnoťte, aj keby boli správne.
\endhodnotenie
}

{%%%%%   Z9-II-4
Objem vody v~kvádri možno určiť ako rozdiel objemu kvádra a~objemu jeho prázdnej časti.
Objem kvádra je $2\cdot3\cdot4\ \text{dm}^3$, \tj. 24~litrov.
Prázdnu časť kvádra tvorí trojboký hranol, ktorého podstavou je pravouhlý trojuholník a~ktorého výška meria 2\,dm.
Stačí určiť obsah pravouhlého trojuholníka $ABH$:
\insp{z9-II-4a.eps}%

Hornú stenu kvádra omáča voda z~jednej štvrtiny.
To znamená, že hrana~$DH$ dĺžky 4\,dm je rozdelená vodnou hladinou v~bode~$A$ vzdialenom 1\,dm od bodu~$D$, čiže 3\,dm od bodu~$H$.
Vodná hladina je rovnobežná s~rovinou stola a~protiľahlé hrany kvádra sú navzájom rovnobežné, preto má uhol $BAH$ veľkosť 30\st.
Pravý uhol je pri vrchole~$H$, na zvyšný uhol pri vrchole~$B$ tak ostáva 60\st.

Trojuholník $ABH$ môžeme chápať ako polovicu rovnostranného trojuholníka rozdeleného výškou~$AH$.
Dĺžka strany~$BH$ je preto polovičná vzhľadom na dĺžku~$AB$.
Ak veľkosť strany~$BH$ v~dm označíme~$x$, tak podľa Pytagorovej vety v~trojuholníku $ABH$ platí
$$
3^2+x^2=(2x)^2.
$$
Z~toho vyplýva $3x^2=9$, čiže $x=\sqrt3$.
Obsah pravouhlého trojuholníka $ABH$ je teda rovný $\frac{3\sqrt3}2\,\text{dm}^2$.

Objem trojbokého hranola predstavujúceho prázdnu časť kvádra je $2\cdot\frac{3\sqrt3}2=3\sqrt3\,\text{dm}^3$, čo je približne 5,2 litrov.
Objem vody v~kvádri je $24-3\sqrt3\ \text{dm}^3$, čo je približne 18,8~litrov.

\hodnotenie
2~body za úvodnú úvahu a~dĺžku úsečky~$AH$, resp. $AD$;
3~body za dĺžku úsečky~$BH$ a~obsah trojuholníka $ABH$;
1~bod za dopočítanie a~prípadné zaokrúhlenie.
\endhodnotenie
}

{%%%%%   Z9-III-1
Priemer známok každého páru je rovný súčtu všetkých jemu pridelených známok vydelenému číslom~25.
Súčty známok jednotlivých párov by tak mali byť:
$$
4{,}68\cdot25=117,\quad
3{,}86\cdot25=96{,}5,\quad
3{,}36\cdot25=84,\quad
1{,}44\cdot25=36,\quad
1{,}60\cdot25=40.
$$
Vidíme, že súčet známok pri páre~B nie je prirodzené číslo, výsledok tohto páru bol teda chybný.
Teraz ho opravíme.
Súčet všetkých známok udelených v~súťaži bol
$$
25\cdot(1+2+3+4+5)=25\cdot15=375.
$$
Odčítame súčty známok udelených párom A, C, D, E a~získame súčet známok
udelených páru~B:
$$
375-117-84-36-40=375-277=98.
$$
Správny priemer známok udelených páru B bol teda
$$
98:25=3{,}92.
$$

Ak by chcel spomenutý porotca čo najviac prospieť páru~E, mohol namiesto známky~2 udeliť~1.
Tým by sa súčet známok páru~E zmenšil z~40 na~39.
Aby páru~D čo najviac uškodil, mohol namiesto známky~1 udeliť~5 a~súčet jeho známok tak zväčšiť z~36 na~40.
V~takom prípade by výsledok páru~D bol horší ako výsledok páru~E.
Spomenutý porotca teda mohol nepoctivým oznámkovaním posunúť pár~E na prvé miesto.

\hodnotenie
2 body za určenie chybného priemeru;
2 body za jeho opravu;
2 body za riešenie zvyšnej časti úlohy.
\endhodnotenie

\poznamka
Násobenie 25 je to isté ako násobenie 100 nasledované delením~4.
Chybný priemer možno teda odhaliť kontrolou, či dvojčíslie za desatinnou čiarkou je deliteľné štyrmi.
Správny priemer páru~B možno (bez roznásobovania~25 a~následného delenia tým istým číslom) určiť takto:
$$
15-4{,}68-3{,}36-1{,}44-1{,}60=3{,}92.
$$
}

{%%%%%   Z9-III-2
Prvočísla menšie ako 10 sú 2, 3, 5 a~7.
Zo zadania vyplýva, že číslo~$a$ je deliteľné~15 a~číslo~$b$ je deliteľné~4.
Prvočíselné rozklady týchto čísel sú preto tvaru
$$
a=3\cdot 5\cdot p,\qquad b=2\cdot2\cdot q,
$$
pričom $p$ a~$q$ sú prvočísla z~vyššie uvedeného zoznamu.

Keďže najväčší spoločný deliteľ čísel $a$ a~$b$ je rovný dvojnásobku najväčšieho spoločného deliteľa čísel $a$ a~$\frac{b}4$, musí byť $p=2$ a~$q\ne2$.
Teda $$a=3\cdot5\cdot2=30.$$
Keďže najväčší spoločný deliteľ čísel $a$ a~$b$ je rovný najväčšiemu spoločnému deliteľovi čísel $\frac{a}{15}$ a~$b$, nemôže byť $q$ ani~3, ani~5.
Z~predchádzajúceho vieme, že $q$ nemôže byť ani~2.
Teda $q=7$ a~$$b=2\cdot2\cdot7=28.$$

Každé z~uvedených prvočísel je prítomné v~rozklade aspoň jedného z~čísel $a$ a~$b$.
Neznáme čísla sú $a=30$ a~$b=28$.

\hodnotenie
1~bod za deliteľnosť čísla $a$ pätnástimi;
1~bod za deliteľnosť čísla $b$ štyrmi;
3~body za diskusiu možností pre $p$ a~$q$;
1~bod za určenie vyhovujúcej možnosti a~vyjadrenie $a$ a~$b$.
\endhodnotenie
}

{%%%%%   Z9-III-3
Keby v~prvej skupine boli samí poctivci, dostali by výskumníci ako odpoveď čísla 4, 4, 4, 4.
Keby v~skupine boli traja, dvaja, resp. jeden poctivec, nemohli by výskumníci dostať ako odpoveď štyri rovnaké čísla.
Keby v~skupine boli len klamári, mohli by výskumníci ako odpoveď dostať akúkoľvek štvoricu čísel neobsahujúcu~0.
V~prvej skupine preto mohli byť buď štyria poctivci, alebo žiadny.

Keďže všetci poctivci v~skupine musia (na rozdiel od klamárov) zodpovedať rovnako, číslo~$k$ môže predstavovať odpoveď poctivca len vtedy, keď sa opakuje $k$-krát.
Preto počet poctivcov v~druhej skupine mohol byť buď jedna, alebo tri.
(Keby v~skupine boli samí klamári, tak by odpoveď~0 bola pravdivá, čo u~klamárov nie je možné.)

\hodnotenie
Po 2~bodoch za určenie možností pre každú skupinu;
2~body za úplnosť a~kvalitu komentára.
\endhodnotenie
}

{%%%%%   Z9-III-4
Podľa zadania je kosoštvorec $ABCD$ tvorený dvoma zhodnými rovnostrannými trojuholníkmi $ACD$ a~$ABC$.
Bod~$K$ je stredom strany~$BC$, úsečka~$AK$ je výškou v~trojuholníku $ABC$ a~osou uhla $CAB$.
Vnútorné uhly v~rovnostranných trojuholníkoch majú veľkosť~$60\st$, preto je veľkosť uhla $CAK$ rovná $30\st$ a~veľkosť uhla $DAK$ je $60\st+30\st=90\st$.
Trojuholník $DAK$ je teda pravouhlý, pričom jeho prepona je stranou hľadaného štvorca.

Strana $DA$ má veľkosť 4\,cm.
Strana $AK$ je odvesnou pravouhlého trojuholníka $AKC$, ktorého zvyšné strany merajú 4\,cm a~2\,cm.
Podľa Pytagorovej vety platí
$$
|AK|^2 =|AC|^2-|CK|^2 =4^2-2^2 =12\,(\Cm^2).
$$
Teraz podľa Pytagorovej vety v~trojuholníku $DAK$ dostávame
$$
|DK|^2 =|DA|^2+|AK|^2 =4^2+12 =28\,(\Cm^2).
$$
Obsah štvorca $KLMD$ je 28\,cm$^2$.
\insp{z9-III-4.eps}%

\hodnotenie
2~body za určenie pravého uhla $DAK$;
2~body za určenie $|AK|^2$;
2~body za určenie $|DK|^2$.
\endhodnotenie
}

