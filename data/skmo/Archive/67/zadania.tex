{%%%%% A-I-1
Pavol striedavo vpisuje krížiky a~krúžky do políčok tabuľky
(začína krížikom). Keď je tabuľka celá vyplnená, výsledné skóre spočíta ako rozdiel $O-X$, pričom $O$ je celkový počet riadkov a~stĺpcov obsahujúcich viac krúžkov ako krížikov a~$X$ je celkový počet riadkov a~stĺpcov obsahujúcich viac krížikov ako krúžkov.
\ite{a)} Dokážte, že pre tabuľku $2\times n$ bude výsledné skóre vždy 0.
\ite{b)} Určte najvyššie možné skóre dosiahnuteľné pre tabuľku ${(2n+1)}\times {(2n+1)}$ v~závislosti od $n$.\endgraf
}
\podpis{Josef Tkadlec}

{%%%%% A-I-2
Dokážte, že ak je súčet dvoch daných reálnych čísel $a$, $b$ väčší
ako~2, má sústava nerovníc
$$
\postdisplaypenalty=10000
(a-1)x+b<x^2<ax+(b-1)
$$
nekonečne veľa riešení $x$ v~obore reálnych čísel.}
\podpis{Jaromír Šimša}

{%%%%% A-I-3
V~rovine sú dané dve zhodné kružnice s~polomerom~$1$, ktoré majú vonkajší dotyk. Uvažujme pravouholník obsahujúci obe kružnice, ktorého každá strana sa dotýka aspoň jednej z~nich. Určte najväčší a~najmenší možný obsah takého pravouholníka.}
\podpis{Jaroslav Švrček}

{%%%%% A-I-4
Nájdite najväčšie prirodzené číslo~$n$ také, že hodnota súčtu
$$\lfloor \sqrt{1} \rfloor + \lfloor \sqrt{2} \rfloor +
\lfloor \sqrt{3} \rfloor
+ \ldots + \lfloor \sqrt{n} \rfloor$$
je prvočíslo.
Zápis $\lfloor x \rfloor$ označuje najväčšie celé číslo, ktoré nie je väčšie ako $x$.}
\podpis{Patrik Bak}

{%%%%% A-I-5
V~konvexnom štvoruholníku $ABCD$ platí $|\measuredangle ABC| = |\measuredangle ACD|$ a $|\measuredangle ACB| = |\measuredangle ADC|$. Predpokladajme, že stred~$O$ kružnice opísanej trojuholníku $BCD$ je rôzny od bodu~$A$. Dokážte, že uhol $OAC$ je pravý.}
\podpis{Patrik Bak}

{%%%%% A-I-6
Nájdite najväčší možný počet prvkov množiny~$\mm M$ celých čísel, ktorá má nasledujúcu vlastnosť: Z~každej trojice rôznych čísel z~$\mm M$ možno vybrať niektoré dve, ktorých súčet je mocninou čísla 2 s~celočíselným exponentom.}
\podpis{Ján Mazák}

{%%%%% B-I-1
Nájdite všetky mnohočleny tvaru $ax^3+bx^2+cx+d$, ktoré po delení dvojčlenom $2x^2+1$ dávajú zvyšok $x+2$ a~po delení dvojčlenom $x^2+2$ dávajú zvyšok $2x+1$.}
\podpis{Pavel Calábek}

{%%%%% B-I-2
Dokážte, že pre každé kladné reálne číslo~$t$ platia nerovnosti
$$0\le \frac{t^2+1}{t+1}-\sqrt t \le |t-1|.$$}
\podpis{Tomáš Jurík}

{%%%%% B-I-3
Nech $ABCD$ je kosoštvorec s~kratšou uhlopriečkou~$BD$ a~$E$ vnútorný bod jeho strany~$CD$, ktorý leží na kružnici opísanej trojuholníku $ABD$. Určte veľkosť jeho vnútorného uhla pri vrchole~$A$, ak majú kružnice opísané trojuholníkom $ACD$ a~$BCE$ práve jeden spoločný bod.}
\podpis{Jaroslav Švrček}

{%%%%% B-I-4
Určte počet všetkých trojíc prirodzených čísel $a$, $b$, $c$, pre ktoré platí
$$a+ab+abc+ac+c=2017.$$}
\podpis{Patrik Bak}

{%%%%% B-I-5
Daný je lichobežník $ABCD$ ($AB\parallel CD$). Uvažujme obe priamky, z~ktorých každá delí daný lichobežník na dve časti s~rovnakým obsahom a~je pritom rovnobežná s~jeho uhlopriečkou $AC$,
resp. $BD$. Dokážte, že priesečník týchto dvoch priamok leží na úsečke, ktorá spája stredy oboch základní $AB$ a~$CD$.}
\podpis{Jaromír Šimša}

{%%%%% B-I-6
Nájdite najväčší možný počet čísel, ktoré možno vybrať z~množiny $\{1,2,3,\dots,100\}$ tak, aby medzi nimi neboli žiadne dve, ktoré sa líšia o~2 alebo o~5.}
\podpis{Pavel Calábek}

{%%%%% C-I-1
Nájdite najmenšie štvorciferné číslo $\overline{abcd}$ také, že rozdiel
$${\bigl(\hskip 1pt\overline{ab}\hskip 1pt\bigr)^2-\bigl(\hskip 1pt\overline{cd}\hskip 1pt\bigr)^2}$$ je trojciferné číslo zapísané tromi rovnakými ciframi.}
\podpis{Patrik Bak, Mária Dományová}

{%%%%% C-I-2
Určte najväčší možný počet neprázdnych po dvoch disjunktných množín s~rovnakými súčtami prvkov, na ktoré možno rozdeliť množinu
\ite{a)} $\{1,2,...,2017\}$,
\ite{b)} $\{1,2,...,2018\}$.

Ak je množina tvorená jedným číslom, považujeme ho za súčet jej prvkov.
}
\podpis{Patrik Bak}

{%%%%% C-I-3
Daný je pravouhlý trojuholník $ABC$ s~preponou~$AB$, v~ktorom $D$ označuje pätu výšky z~vrcholu~$C$. V~polrovine s~hraničnou priamkou~$AB$ a~vnútorným bodom~$C$ uvažujme body $E$, $F$ také, že uhly $EBA$, $FAB$ sú pravé, $|BE|=|BD|$ a~$|AF|=|AD|$. Dokážte, že priamky $AE$ a~$BF$ sa~pretínajú na úsečke~$CD$.}
\podpis{Jaroslav Švrček}

{%%%%% C-I-4
Určte najväčšie celé číslo~$n$, pri ktorom možno štvorcovú tabuľku $n\times n$ zaplniť prirodzenými číslami od 1 po $n^2$ tak, aby v~každej jej štvorcovej časti $3\times3$ bola zapísaná aspoň jedna druhá mocnina celého čísla.}
\podpis{Jaromír Šimša}

{%%%%% C-I-5
Daná je kružnica $k(O,r)$ a~bod~$A$, pričom $|AO| = d > r$. Dotyčnice z~bodu~$A$ sa dotýkajú kružnice~$k$ v~bodoch $B$, $C$. Trojuholníku $ABC$ je vpísaná kružnica. Vyjadrite jej polomer~$\rho$ pomocou daných dĺžok $d$ a~$r$.}
\podpis{Šárka Gergelitsová}

{%%%%% C-I-6
Na kruhovom opevnení hradu je niekoľko veží. Do nich sa rozmiestni päť čiernych a~päť červených rytierov (v~každej veži ich môže byť viac a~môžu mať rôzne farby) a~začnú strážiť. Po uplynutí každej hodiny prejdú všetci čierni rytieri do susednej veže v~smere chodu hodinových ručičiek a~všetci červení rytieri prejdú do susednej veže v~opačnom smere. Dokážte nasledujúce tvrdenie:
\ite{a)} Ak je veží osem, môžu sa rytieri na začiatku rozmiestniť tak, že počas každej hodiny bude v~každej veži aspoň jeden rytier.
\ite{b)} Ak je veží sedem, niektorú hodinu ostane aspoň jedna veža neobsadená, nech už sa na začiatku rytieri rozmiestnia akokoľvek.\endgraf
}
\podpis{Pavel Calábek}

{%%%%% A-S-1
Určte všetky reálne čísla~$p$, pre ktoré má sústava
nerovníc
$$
\align
x^2+(p-1)x+p&\le0,\\
x^2-(p-1)x+p&\le0
\endalign
$$
aspoň jedno riešenie v~obore reálnych čísel.
}
\podpis{Jaromír Šimša}

{%%%%% A-S-2
V~trojuholníku $ABC$ označme postupne $S_b$, $S_c$ stredy
strán $AC$, $AB$. Dokážte, že ak $|AB| < |AC|$, tak $|\angle BS_cC| < |\angle BS_bC|$.}
\podpis{Patrik Bak}

{%%%%% A-S-3
Pavol striedavo vpisuje krížiky a~krúžky do políčok
tabuľky (začína krížikom). Keď je tabuľka celá vyplnená, výsledné skóre
spočíta ako súčet $X+O$, pričom $X$ je počet riadkov obsahujúcich viac
krížikov ako krúžkov a~$O$ je počet stĺpcov obsahujúcich viac krúžkov ako
krížikov. Určte najvyššie možné skóre dosiahnuteľné pre tabuľku $(2n + 1)
\times (2n + 1)$ v~závislosti od prirodzeného čísla~$n$.}
\podpis{Josef Tkadlec}

{%%%%% A-II-1
Pavol striedavo vpisuje krížiky a~krúžky do políčok
tabuľky (začína krížikom). Keď je tabuľka celá vyplnená, výsledné skóre
spočíta ako rozdiel $X-O$, pričom $X$ je súčet druhých mocnín počtov
krížikov v~jednotlivých riadkoch a~stĺpcoch a~$O$ je súčet druhých
mocnín počtov krúžkov v~jednotlivých riadkoch a~stĺpcoch. Určte všetky
možné hodnoty skóre dosiahnuteľné pre tabuľku $67\times 67$.
}
\podpis{Josef Tkadlec}

{%%%%% A-II-2
Daná je polkružnica~$k$ nad priemerom~$PQ$. Na nej
je zostrojená tetiva~$BC$ pevnej dĺžky~$d$, ktorej krajné body
sú rôzne od bodov $P$, $Q$. Lúč vyslaný z~bodu~$B$ sa
od priemeru~$PQ$ odrazí do bodu~$C$ v~takom bode~$A$, že
$|\uhol PAB|=|\uhol QAC|$. Dokážte, že veľkosť uhla~$BAC$
nezávisí od polohy tetivy~$BC$ na polkružnici.}
\podpis{Šárka Gergelitsová}

{%%%%% A-II-3
Sú dané dve rôzne kladné reálne čísla $a$, $b$. Uvažujme
rovnicu
$$
\lfloor ax+b \rfloor=\lfloor bx+a\rfloor,
$$
pričom $\lfloor y \rfloor$ označuje najväčšie celé číslo, ktoré neprevyšuje~$y$.
Dokážte, že existuje interval dĺžky aspoň
$$
\frac{1}{\max{\{a,b\}}},
$$
ktorého všetky body sú riešeniami danej rovnice.}
\podpis{Patrik Bak}

{%%%%% A-II-4
Rozhodnite, či existujú kladné celé čísla $n$ a~$k$ také, že
$$
\frac{n}{11^k-n}
$$
je druhou mocninou celého čísla.}
\podpis{Ján Mazák}

{%%%%% A-III-1
V~spoločnosti ľudí sú niektoré dvojice spriatelené. Pre
kladné celé číslo $k\ge 3$ hovoríme, že spoločnosť je $k$-dobrá, ak
možno každú $k$-ticu ľudí zo spoločnosti rozsadiť okolo okrúhleho stola
tak, že sa každí dvaja susedia priatelia. Dokážte, že ak je spoločnosť
6-dobrá, tak je aj~7-dobrá.}
\podpis{Josef Tkadlec}

{%%%%% A-III-2
Reálne čísla $x$, $y$, $z$ sú zvolené tak, že čísla
$$
\frac{1}{|x^2+2yz|}, \quad \frac{1}{|y^2+2zx|}, \quad \frac{1}{|z^2+2xy|}
$$
sú dĺžkami strán (nedegenerovaného) trojuholníka. Určte všetky
možné hodnoty výrazu $xy+yz+zx$.}
\podpis{Michal Rolínek}

{%%%%% A-III-3
Daný je trojuholník $ABC$. Os uhla pri vrchole~$A$
pretína stranu~$BC$ v~bode~$D$. Označme $E$, $F$ stredy kružníc opísaných
trojuholníkom $ABD$, $ACD$. Akú veľkosť môže mať uhol $BAC$,
ak stred kružnice opísanej trojuholníku~$AEF$ leží na priamke~$BC$?}
\podpis{Patrik Bak}

{%%%%% A-III-4
Uvažujme ľubovoľnú trojicu celých čísel
$a$, $b$ a~$c$,
ktoré sú dĺžkami strán trojuholníka, nemajú spoločného
deliteľa väčšieho ako~1 a~pre ktoré sú hodnoty všetkých troch
zlomkov
$$
\frac{a^2+b^2-c^2}{a+b-c},\quad
\frac{b^2+c^2-a^2}{b+c-a},\quad
\frac{c^2+a^2-b^2}{c+a-b}
$$
celočíselné. Dokážte, že súčin menovateľov týchto troch zlomkov
alebo jeho dvojnásobok je druhou mocninou celého čísla.}
\podpis{Jaromír Šimša}

{%%%%% A-III-5
Daný je rovnoramenný lichobežník $ABCD$ s~dlhšou základňou~$AB$.
Označme $I$ stred kružnice vpísanej do trojuholníka~$ABC$ a~$J$
stred kružnice pripísanej k~strane~$AD$ trojuholníka~$ACD$. Dokážte, že
priamky $IJ$ a~$AB$ sú rovnobežné.}
\podpis{Patrik Bak}

{%%%%% A-III-6
Nájdite najmenšie prirodzené číslo~$n$ také, že
pre ľubovoľné ofarbenie čísel $1, 2, 3, \dots, n$
tromi farbami existujú medzi uvedenými číslami dve čísla
rovnakej farby, ktorých rozdiel je druhá mocnina prirodzeného čísla.}
\podpis{Vojtech Bálint, Michal Rolínek, Josef Tkadlec}

{%%%%% B-S-1
Uveďte príklad mnohočlena najnižšieho možného kladného stupňa,
ktorý po delení ako mnohočlenom $x^2-4$, tak
mnohočlenom $x^3-8x+8$ dáva zvyšok~$1$.}
\podpis{Ján Mazák}

{%%%%% B-S-2
Daný je trojuholník $ABC$ s~vpísanou kružnicou~$k$. Jej dotykové body
na stranách $AB$, $BC$, $CA$ označme postupne $K$, $L$,
$M$. Nech $E$, $F$ sú postupne obrazy bodu~$K$ v~stredových súmernostiach
podľa stredov $A$ a~$B$. Priesečník priamok $EM$ a~$FL$ označme~$U$.
Dokážte, že bod~$U$ leží na kružnici~$k$ a~že priamky $UK$ a~$AB$
sú navzájom kolmé.}
\podpis{Jaroslav Švrček, Jaroslav Zhouf}

{%%%%% B-S-3
Číslo $2\,018$ sme rozložili na súčet niekoľkých prirodzených čísel
a~ich tretie mocniny sčítali.
Aké zvyšky môže tento súčet dávať po delení šiestimi?}
\podpis{Vojtech Bálint}

{%%%%% B-II-1
V ostrouhlom trojuholníku $ABC$ označme $O$ stred
kružnice opísanej, $S_a$, $S_b$ postupne stredy strán $BC$, $AC$
a~$P$ pätu výšky na stranu~$AB$. Vyjadrite podiel $\frac{|OS_a|}{|OS_b|}$
pomocou $a=|BC|$, $ b=|CA|$ a~$k=\frac{|AP|}{|BP|}$.}
\podpis{Šárka Gergelitsová}

{%%%%% B-II-2
Nájdite všetky kladné reálne čísla~$t$ také, že pre
ľubovoľné nezáporné reálne číslo~$x$ platí nerovnosť
$$
\frac t{x+2}+\frac x{t(x+1)}\le 1.
$$}
\podpis{Tomáš Jurík}

{%%%%% B-II-3
Pre ktoré $n$ je možné vyplniť tabuľku $n\times n$ číslami od~$1$ po~$n^2$
tak, aby súčet čísel v~každom riadku aj v~každom stĺpci bol deliteľný siedmimi?}
\podpis{Ján Mazák}

{%%%%% B-II-4
Nanajvýš koľko čísel možno vybrať z~množiny
$\mn M=\{1,2,\dots,2\,018\}$ tak, aby rozdiel žiadnych dvoch vybraných čísel
nebol rovný prvočíslu?}
\podpis{Josef Tkadlec}

{%%%%% C-S-1
Nájdite najväčšie trojciferné číslo, z~ktorého po vyškrtnutí
ľubovoľnej cifry dostaneme prvočíslo.}
\podpis{Ján Mazák}

{%%%%% C-S-2
Skúmajme, či sa dá štvorcová tabuľka
$n\times n$ vyplniť prirodzenými číslami od~1 do~$n^2$ tak, aby v~každej
štvorcovej časti $2\times2$ bol zapísaný aspoň jeden násobok piatich.
\ite a) Dokážte, že pre žiadne párne~$n$ sa to nedá.
\ite b) Nájdite najväčšie nepárne~$n$, pre ktoré sa to dá.}
\podpis{Jaromír Šimša}

{%%%%% C-S-3
Daný je trojuholník $ABC$ s~tupým uhlom pri vrchole~$A$, v~ktorom $D$
označuje pätu výšky z~vrcholu~$C$. Na kolmiciach na~$AB$, ktoré prechádzajú
bodmi $A$ a~$B$, zostrojme v~polrovine~$ABC$
postupne body $E$ a~$F$, pre ktoré platí ${|AE|=|AD|}$
a~$|BF|=|BD|$. Označme napokon $P$ a~$Q$ priesečníky priamok $AF$ a~$BE$
s~priamkou~$CD$. Dokážte, že $D$ je stredom úsečky~$PQ$.}
\podpis{Jaroslav Švrček, Jaroslav Zhouf}

{%%%%% C-II-1
Nájdite najmenšie prirodzené číslo končiace štvorčíslím 2018, ktoré je
násobkom čísla 2017.}
\podpis{Tomáš Jurík}

{%%%%% C-II-2
Pre celé čísla $x$, $y$, $z$ platí $x^2+y-z=10$, $x^2-y+z=22$.
Nájdite najmenšiu možnú hodnotu výrazu $x^2+y^2+z^2$.}
\podpis{Ján Mazák}

{%%%%% C-II-3
Daný je trojuholník $ABC$. Nech $P$, $Q$ sú postupne stredy strán $AB$, $AC$ a~nech
$R$, $S$ sú vnútorné body úsečky~$BC$, pre ktoré $|BR|=|RS|=|SC|$. Označme~$T$
priesečník priamok $PR$ a~$QS$. Dokážte, že $ABTC$ je rovnobežník.}
\podpis{Patrik Bak}

{%%%%% C-II-4
Určte najmenšie prirodzené číslo~$n$, pre ktoré platí: Keď vyplníme
štvorcovú tabuľku $n\times n$ ľubovoľnými navzájom rôznymi
prirodzenými číslami, vždy sa nájde políčko s~číslom, ktoré po delení tromi
dáva rovnaký zvyšok ako iné číslo z~toho istého riadka aj ako iné číslo
z~toho istého stĺpca.}
\podpis{Jaromír Šimša}

{%%%%%   vyberko, den 1, priklad 1
Dokážte, že pre ľubovoľné prirodzené číslo $n$ platí, že posledná cifra každého deliteľa čísla $\underbrace{111\dots1}_{5^n}$ je 1.}
\podpis{Tomáš Kekeňák, Laura Vištanová:}

{%%%%%   vyberko, den 1, priklad 2
Nech $x$, $y$, $z$ sú kladné reálne čísla, pre ktoré $xyz\ge1$. Dokážte, že platí
$$
\frac{x}{x^3+y^2+z} + \frac{y}{y^3+z^2+x} + \frac{z}{z^3+x^2+y} \le 1.
$$
}
\podpis{Tomáš Kekeňák, Laura Vištanová:}

{%%%%%   vyberko, den 1, priklad 3
Daný je trojuholník $ABC$. Kružnica prechádzajúca cez body $A$, $B$ pretína strany $AC$ a~$BC$ postupne v bodoch $D$, $E$. Priamky $AB$ a $DE$ sa pretínajú v~bode~$F$ a~priamky $BD$ a~$CF$ v~bode~$M$. Dokážte, že $|MF| = |MC|$ práve vtedy, keď $|MB| \cdot |MD| = |MC|^2$.}
\podpis{Tomáš Kekeňák, Laura Vištanová:}

{%%%%%   vyberko, den 1, priklad 4
Alena a Boris hrajú hru v tabuľke s~rozmermi $6\times6$. Každý hráč vo svojom ťahu zvolí racionálne číslo, ktoré sa ešte v~tabuľke nenachádza a napíše ho do niektorého prázdneho políčka. Alena začína a~potom sa obaja v~ťahoch striedajú. Keď je celá tabuľka vyplnená, zafarbia v~každom riadku políčko s~najväčším číslom načierno. Alena vyhrá, ak je možné nakresliť lomenú čiaru spájajúcu vrchnú a~spodnú stranu tabuľky, ktorá celá leží v~čiernych políčkach, inak vyhrá Boris. (Na prechod medzi čiernymi políčkami stačí, aby mali spoločný vrchol.) Ktorý z~hráčov má vyhrávajúcu stratégiu?}
\podpis{Tomáš Kekeňák, Laura Vištanová:https://artofproblemsolving.com/community/c6h5393p17438}

{%%%%%   vyberko, den 2, priklad 1
Dokážte, že existuje nekonečne veľa prirodzených čísel~$n$ s~nasledujúcou vlastnosťou: Existuje $n$ po dvoch rôznych prirodzených čísel $a_1,\dots,a_n$ takých, že $$\sum_{i=1}^n a_i=\sum_{1\le i< j\le n} \nsd(a_i,a_j).$$}
\podpis{Martin Vodička, Slavomír Hanzely:}

{%%%%%   vyberko, den 2, priklad 2
Nech $f:\Bbb R\setminus\{0\}\rightarrow \Bbb R$ je nekonštantná funkcia. Dokážte, že existujú nenulové reálne $x,y$ také, že $x+y\ne 0$ a $f(x+y)<f(xy)$.}
\podpis{Martin Vodička, Slavomír Hanzely:}

{%%%%%   vyberko, den 2, priklad 3
Nech $\mn V$ je konečná neprázdna množina bodov v~rovine, pričom žiadne tri z~nich neležia na jednej priamke. Každému bodu z~$\mn V$ je priradené nezáporné reálne číslo a~súčet všetkých týchto čísel je~1. Medzi niektorými bodmi nakreslíme úsečku tak, aby sa žiadne dve úsečky nepretínali. Potom každej úsečke priradíme číslo, ktoré je súčinom čísel priradených jej koncom. Dokážte, že súčet čísel priradených všetkým úsečkám je najviac~$\frac 38$.

\poznamka
Úsečky sa pretínajú, ak majú spoločný vnútorný bod.}
\podpis{Martin Vodička, Slavomír Hanzely:}

{%%%%%   vyberko, den 2, priklad 4
Štvorsten $ABCD$, ktorého každá stena je ostrouhlý trojuholník, je vpísaný do sféry so stredom v~bode~$O$. Priamka prechádzajúca bodom~$O$ kolmá na rovinu $ABC$ pretína túto sféru v~bode~$D'$, ktorý leží na opačnej strane roviny $ABC$ ako bod~$D$. Priamka~$DD'$ pretína rovinu $ABC$ v~bode~$P$, ktorý leží vnútri trojuholníka $ABC$. Dokážte, že ak $|\uhol APB|=2|\uhol ACB|$, tak $|\uhol ADD'|=|\uhol BDD'|$.}
\podpis{Martin Vodička, Slavomír Hanzely:}

{%%%%%   vyberko, den 3, priklad 1
Nech $n$ je dané kladné celé číslo. Slovo s~dĺžkou~$3n$, ktoré obsahuje každé z~písmen $\mathsf{A}$, $\mathsf{B}$ a~$\mathsf{C}$ práve $n$-krát, nazývame \textit{chameleónom}. V~jednom kroku môžeme navzájom vymeniť ľubovoľné dve susedné písmená. Dokážte, že pre každého chameleóna~$X$ existuje taký chameleón~$Y$, že $X$ nemožno zmeniť na $Y$ pomocou menej ako $3n^2/2$ krokov.}
\podpis{Peter Novotný, Jozef Rajník:IMO Shortlist 2017, C2}

{%%%%%   vyberko, den 3, priklad 2
Daný je ostrouhlý trojuholník $ABC$, ktorého žiadne dve strany nemajú rovnakú dĺžku. Označme $O$ stred jeho opísanej kružnice a~$H$ priesečník výšok. Priamka~$OA$ pretína výšky spustené z~vrcholov $B$ a~$C$ postupne v~bodoch $P$ a~$Q$. Dokážte, že stred kružnice opísanej trojuholníku $PQH$ leží na niektorej ťažnici trojuholníka $ABC$.}
\podpis{Peter Novotný, Jozef Rajník:IMO Shortlist 2017, G3}

{%%%%%   vyberko, den 3, priklad 3
Postupnosť reálnych čísel $(a_n)_{n=1}^\infty$ spĺňa
$$
a_n = -\max_{1\leqq i<n}(a_i + a_{n-i})\qquad\text{pre všetky $n>2018$.}
$$
Dokážte, že táto postupnosť je zdola aj zhora ohraničená.}
\podpis{Peter Novotný, Jozef Rajník:IMO Shortlist 2017, A4}

{%%%%%   vyberko, den 4, priklad 1
Dokážte, že pre každé kladné celé číslo~$a$ existuje kladné celé číslo $b>a$, pre ktoré platí, že
$1+2^a+3^a$ delí $1+2^b+3^b$.}
\podpis{Matej Králik, Peter Súkeník:2015 Polish finals 6, https://artofproblemsolving.com/community/c6h1279442p6724813}

{%%%%%   vyberko, den 4, priklad 2
Nech $n$ je nepárne kladné celé číslo a $x_1, x_2, \dots, x_n, y_1, y_2, \dots, y_n$
sú nezáporné reálne čísla, pre ktoré platí
$x_1 + x_2 + \dots + x_n = y_1 + y_2 + \dots + y_n$.
Dokážte, že môžeme vybrať vlastnú podmnožinu~$\mn S$ množiny $\{1,2,\dots n\}$ takú, že
$$\frac{n-1}{n+1}\sum_{s\in\mn S}x_s \leq \sum_{s\in\mn S}y_s \leq \frac{n+1}{n-1}\sum_{s\in\mn S}x_s.$$}
\podpis{Matej Králik, Peter Súkeník:https://artofproblemsolving.com/community/c6t241f6h1533460_interesting_and_tricky_pigeonhole_sum}

{%%%%%   vyberko, den 4, priklad 3
Nájdite všetky funkcie $f:\Bbb{R} \to \Bbb{R} $, pre ktoré platí
$$\forall x,y\in \Bbb{R}\quad f(x^2+yf(y)+yf(x))=f(x)^2+y^2+xf(y).$$}
\podpis{Matej Králik, Peter Súkeník:https://artofproblemsolving.com/community/c6t300f6h1620411_functional_equation}

{%%%%%   vyberko, den 4, priklad 4
Daný je trojuholník $ABC$ so stredom opísanej kružnice~$O$.
Nech $P$ je taký bod na strane~$AB$, pre ktorý platí $|\angle BOP| = |\angle ABC|$
a nech $Q$ je taký bod na strane~$AC$, pre ktorý platí $|\angle COQ| = |\angle ACB|$.
Dokážte, že ak uhly $BPQ$ a~$CQP$ sú tupé,
tak obraz priamky~$BC$ podľa priamky~$PQ$ tvorí dotyčnicu ku kružnici
opísanej trojuholníku $APQ$.}
\podpis{Matej Králik, Peter Súkeník:2013 Canada National 5, https://artofproblemsolving.com/community/c6h527557p2996617}

{%%%%%   vyberko, den 5, priklad 1
Nech $ABCDE$ je konvexný päťuholník taký, že $|AB|=|BC|=|CD|$, $|\measuredangle EAB| = |\measuredangle BCD|$ a $|\measuredangle EDC| = |\measuredangle CBA|$. Dokážte, že kolmica z bodu $E$ na $BC$ a úsečky $AC$ a $BD$ sa pretínajú v jednom bode.}
\podpis{Patrik Bak, Samuel Sládek:IMO Shortlist 2017, G1}

{%%%%%   vyberko, den 5, priklad 2
Nech $p \ge 2$ je prvočíslo. Patrik a Samo hrajú spolu hru, v~ktorej sa striedajú v~ťahoch. V~každom ťahu aktuálny hráč vyberie index~$i$ z~množiny $\{0,1,\dots,p-1\}$, ktorý nebol predtým zvolený žiadnym z~hráčov, a~potom zvolí číslo $a_i$ z~množiny $\{0,1,2,3,4,5,6,7,8,9\}$. Patrik má prvý ťah. Hra sa končí, keď všetky indexy $\{0,1,\dots,p-1\}$ boli zvolené. Následne je spočítané číslo
$$
M=a_0+10\cdot a_1+\cdots+10^{p-1}\cdot a_{p-1}=\sum^{p-1}_{j=0}a_j \cdot 10^j.
$$
Patrikovou úlohou je zabezpečiť, aby číslo~$M$ bolo deliteľné prvočíslom~$p$. Samovou úlohou je tomu zabrániť. Rozhodnite, ktorý z~hráčov má víťaznú stratégiu.}
\podpis{Patrik Bak, Samuel Sládek:IMO Shortlist 2017, N2}

{%%%%%   vyberko, den 5, priklad 3
Obdĺžnik $\Cal{R}$ s~nepárnymi celočíselnými dĺžkami strán je rozdelený na menšie obdĺžniky s~celočíselnými dĺžkami strán. Dokážte, že sa medzi týmito obdĺžnikmi dá nájsť taký, že vzdialenosti jeho strán od štyroch strán $\Cal{R}$ sú všetky párne alebo všetky nepárne.}
\podpis{Patrik Bak, Samuel Sládek:IMO Shortlist 2017, C1}

{%%%%%   vyberko,
...}
\podpis{...}

{%%%%%   vyberko,
...}
\podpis{...}

{%%%%%   trojstretnutie, priklad 1
Určte všetky funkcie $f\colon \Bbb{R} \to\Bbb{R}$ také, že pre všetky reálne čísla $x$, $y$ platí
$$
%\belowdisplayskip 0pt
f(x^2 + xy) = f(x)f(y) + yf(x) + xf(x+y).
$$
}
\podpis{Walther Janous, Rakúsko}

{%%%%%   trojstretnutie, priklad 2
Daný je ostrouhlý trojuholník $ABC$, ktorého žiadne dve strany nemajú rovnakú dĺžku. Vnútri strán $AB$ a~$AC$ sú postupne zvolené body $D$ a~$E$, pričom $|BD|=|CE|$. Označme $O_1$, $O_2$ postupne stredy kružníc opísaných trojuholníkom $ABE$, $ACD$. Dokážte, že kružnice opísané trojuholníkom  $ABC$, $ADE$ a~$AO_1O_2$ majú spoločný bod rôzny od~$A$.}
\podpis{Patrik Bak}

{%%%%%   trojstretnutie, priklad 3
Okolo okrúhleho stola sedí 2018 hráčov. Na začiatku hry rozdáme hráčom ľubovoľne všetky karty z~balíčka, ktorý obsahuje $K$~kariet (niektorí hráči nemusia dostať žiadne karty). Následne v~každom ťahu zvolíme jedného hráča, ktorý si od každého zo~svojich dvoch susedov zoberie jednu kartu. Je dovolené zvoliť len takého hráča, ktorého obaja susedia majú nenulový počet kariet. Hra skončí, keď taký hráč neexistuje. Určte najväčšiu takú hodnotu~$K$, že bez ohľadu na rozdanie kariet a~následnú voľbu hráčov hra vždy skončí po konečnom počte ťahov.}
\podpis{Peter Novotný}

{%%%%%   trojstretnutie, priklad 4
Daný je ostrouhlý trojuholník $ABC$ s~obvodom~$2s$ a~tri navzájom disjunktné kruhy so stredmi v~bodoch $A$, $B$, $C$. Dokážte, že existuje kruh s~polomerom~$s$, ktorý obsahuje všetky tri dané kruhy.}
\podpis{Josef Tkadlec}

{%%%%%   trojstretnutie, priklad 5
V~obdĺžniku s~rozmermi $2\times 3$ leží lomená čiara s~celkovou dĺžkou~36, ktorá môže pretínať sama seba. Dokážte, že existuje priamka, ktorá je rovnobežná s~dvoma stranami obdĺžnika, pretína zvyšné dve strany v~ich vnútorných bodoch a~pretína danú lomenú čiaru v~menej ako 10~bodoch.}
\podpis{Josef Tkadlec, Vojtech Bálint}

{%%%%%   trojstretnutie, priklad 6
Kladné celé číslo~$n$ nazývame {\it fantastické}, ak existujú kladné racionálne čísla $a$, $b$ také, že
$$
n=a+\frac{1}{a}+b+\frac{1}{b}.
$$
\ite{a)} Dokážte, že existuje nekonečne veľa prvočísel $p$ takých, že žiadny násobok $p$ nie je fantastický.
\ite{b)} Dokážte, že existuje nekonečne veľa prvočísel $p$ takých, že aspoň jeden násobok~$p$ je fantastický.\endgraf
}
\podpis{Walther Janous, Rakúsko}

{%%%%%   IMO, priklad 1
Daný je ostrouhlý trojuholník $ABC$ s~opísanou kružnicou~$\Gamma$. Vnútri strán $AB$ a~$AC$ ležia postupne body $D$ a~$E$, pričom $|AD| = |AE|$. Osi úsečiek $BD$ a~$CE$ pretínajú kratšie oblúky $AB$ a~$AC$ kružnice $\Gamma$ postupne v~bodoch $F$ a~$G$. Dokážte, že priamky $DE$ a~$FG$ sú rovnobežné (alebo totožné).}
\podpis{Grécko}

{%%%%%   IMO, priklad 2
Nájdite všetky celé čísla $n \ge 3$, pre ktoré existujú reálne čísla $a_1$, $a_2, \dots, a_{n+2}$ také, že $a_{n+1}=a_1$, $a_{n+2}=a_2$ a
$$
a_i a_{i+1} +1 = a_{i+2}
$$
pre $i=1,2, \dots ,n$.}
\podpis{Patrik Bak, Slovensko}

{%%%%%   IMO, priklad 3
Rovnostrannú trojuholníkovú tabuľku čísel nazývame {\it antipascalov trojuholník}, keď každé číslo okrem čísel v~spodnom riadku je rovné absolútnej hodnote rozdielu dvoch čísel nachádzajúcich sa bezprostredne pod ním. Napríklad nasledujúca tabuľka je antipascalovým trojuholníkom so štyrmi riadkami, ktorý obsahuje každé celé číslo od $1$ do~$10$:
$$
\gather
4\\
2\quad 6\\
5\quad 7\quad 1\\
8\quad 3\quad\! 10\quad\! 9
\endgather
$$
Rozhodnite, či existuje antipascalov trojuholník s~2018 riadkami, ktorý obsahuje každé celé číslo od $1$ do $1+2+\dots +2018$.}
\podpis{Irán}

{%%%%%   IMO, priklad 4
Nazývajme {\it pozíciou\/} každý bod roviny so súradnicami $(x,y)$ takými, že $x$ aj $y$ sú kladné celé čísla menšie alebo rovné $20$. Na začiatku je každá zo 400 pozícií voľná. Anna a~Boris sa striedajú v~ukladaní kameňov, pričom začína Anna.
Anna vo svojom ťahu položí nový červený kameň na voľnú pozíciu vždy tak, aby vzdialenosť medzi každými dvoma pozíciami obsadenými červenými kameňmi bola rôzna od $\sqrt{5}$. Boris vo svojom ťahu položí nový modrý kameň na ľubovoľnú voľnú pozíciu.
(Pozícia obsadená modrým kameňom môže mať ľubovoľné vzdialenosti od ostatných obsadených pozícií.)
Skončia vtedy, keď niektorý z~nich už nemôže položiť ďalší kameň.
Nájdite najväčšie $K$ také, že Anna dokáže položiť aspoň $K$ červených kameňov bez ohľadu na to, ako ukladá kamene Boris.
}
\podpis{Arménsko}

{%%%%%   IMO, priklad 5
Nech $a_1, a_2, \dots$ je nekonečná postupnosť kladných celých čísel. Predpokladajme, že existuje celé číslo $N>1$ také, že pre každé $n \ge N$ je číslo
$$
\frac{a_1}{a_2} + \frac{a_2}{a_3} + \cdots + \frac{a_{n-1}}{a_n} + \frac{a_n}{a_1}$$
celé.
Dokážte, že existuje kladné celé číslo~$M$ také, že $a_m = a_{m+1}$ pre všetky $m \ge M$.}
\podpis{Mongolsko}

{%%%%%   IMO, priklad 6
Daný je konvexný štvoruholník $ABCD$, pričom $|AB| \cdot |CD| = |BC| \cdot |DA|$. Vnútri neho leží bod~$X$ taký, že
$$
|\measuredangle XAB| = |\measuredangle XCD| \qquad \text{a} \qquad |\measuredangle XBC|=|\measuredangle XDA|.
$$
Dokážte, že $|\measuredangle BXA| + |\measuredangle DXC| = 180^\circ$.}
\podpis{Poľsko}

{%%%%%   MEMO, priklad 1
Označme $\Bbb Q^+$ množinu všetkých kladných racionálnych čísel a nech $\alpha\in \Bbb Q^+$.
Nájdite všetky funkcie $f\colon \Bbb Q^{+} \to (\alpha,\p\infty)$ spĺňajúce
$$
f\left(\frac{x+y}{\alpha}\right) = \frac{f(x)+f(y)}{\alpha}\qquad\text{pre všetky $x,y \in \Bbb{Q}^{+}$.}
$$
}
\podpis{Rakúsko}

{%%%%%   MEMO, priklad 2
Útvary znázornené na obrázkoch pozostávajúce zo 6, respektíve 10 jednotkových štvorcov nazývame {\it schodíky}.
\insp{67-memo-2}
Majme tabuľku s~rozmermi $2018\times 2018$ pozostávajúcu z~$2018^2$ políčok s~rozmermi jednotkového štvorca. Z~tabuľky odstránime ľubovoľné dve políčka z~rovnakého riadku. Dokážte, že zvyšok tabuľky nie je možné (pri rezaní po hraniciach políčok) rozrezať na schodíky (ľubovoľne otočené).
}
\podpis{Ukrajina}

{%%%%%   MEMO, priklad 3
Nech $ABC$ je ostrouhlý trojuholník s~$|AB|<|AC|$ a~nech $D$ je päta výšky z~vrcholu~$A$. Nech $R$ a~$Q$ sú postupne ťažiská trojuholníkov $ABD$ a~$ACD$. Nech $P$ je bod na úsečke~$BC$ taký, že $P \ne D$ a~body $P$, $Q$, $R$ a~$D$ ležia na jednej kružnici. Dokážte, že priamky $AP$, $BQ$ a~$CR$ sa pretínajú v~jednom bode.}
\podpis{Patrik Bak, Slovensko}

{%%%%%   MEMO, priklad 4
\ite a) Dokážte, že pre každé kladné celé číslo~$m$ existuje celé číslo $n\ge m$ také, že
$$
\left\lfloor\frac{n}{1}\right\rfloor\cdot\left\lfloor\frac{n}{2}\right\rfloor\cdot\dots\cdot\left\lfloor\frac{n}{m}\right\rfloor = {n \choose m}. \tag{$*$}
$$
\ite b) Označme $p(m)$ najmenšie celé číslo $n\ge m$ také, že platí $(*)$. Dokážte, že $p(2018)=p(2019)$.

Pre reálne číslo $x$ označujeme $\lfloor x\rfloor$ najväčšie celé číslo neprevyšujúce $x$.
}
\podpis{Patrik Bak, Slovensko}

{%%%%%   MEMO, priklad t1
Nech $a$, $b$ a $c$ sú kladné reálne čísla spĺňajúce $abc=1$. Dokážte, že
$$\frac{a^2-b^2}{a+bc}+\frac{b^2-c^2}{b+ca}+\frac{c^2-a^2}{c+ab}\leq a+b+c-3.$$
}
\podpis{Poľsko}

{%%%%%   MEMO, priklad t2
Nech $P(x)$ je polynóm stupňa $n\ge 2$ s~racionálnymi koeficientmi taký, že $P(x)$ má
$n$ po dvoch rôznych reálnych koreňov, ktoré tvoria aritmetickú postupnosť. Dokážte, že medzi koreňmi $P(x)$ sú dva, ktoré sú aj koreňmi nejakého polynómu stupňa 2 s~racionálnymi koeficientmi.
}
\podpis{Rakúsko}

{%%%%%   MEMO, priklad t3
Skupina pirátov sa pohádala a~teraz každý z~nich mieri na iných dvoch pirátov. Všetci piráti sú postupne vyvolávaní v~nejakom poradí. Ak je vyvolaný pirát stále nažive, vystrelí na oboch pirátov, na ktorých mieri (niektorí z~nich môžu byť už mŕtvi). Všetky strely sú hneď smrteľné. Po tom, čo všetci piráti boli vyvolaní, sa zistilo, že 28 pirátov bolo zabitých. Dokážte, že nech by boli piráti vyvolaní v~akomkoľvek inom poradí, aspoň 10 z~nich by bolo zabitých aj tak.
}
\podpis{Josef Tkadlec, Česko}

{%%%%%   MEMO, priklad t4
Nech $n$ je kladné celé číslo a~$u_1,u_2,\dots,u_n$ sú kladné celé čísla neprevyšujúce $2^k$ pre nejaké celé číslo $k\ge 3$. {\it Reprezentácia\/} celého čísla~$t$ je postupnosť nezáporných celých čísel $a_1,a_2,\dots,a_n$ taká, že
$$
t=a_1u_1+a_2u_2+\dots+a_nu_n.
$$
Dokážte, že ak celé číslo~$t$ má reprezentáciu, tak má aj reprezentáciu, kde menej ako $2k$ z~čísel $a_1,a_2,\dots,a_n$ je nenulových.
}
\podpis{Poľsko}

{%%%%%   MEMO, priklad t5
Nech $ABC$ je ostrouhlý trojuholník s~$|AB|<|AC|$ a~nech $D$ je päta výšky z~vrcholu~$A$. Body $B'$ a~$C'$ ležia na polpriamkach $AB$ a~$AC$ tak, že body $B'$, $C'$ a~$D$ ležia na jednej priamke a~body $B$, $C$, $B'$ a~$C'$ ležia na jednej kružnici so stredom~$O$. Dokážte, že ak $M$ je stred~$BC$ a~$H$ je ortocentrum trojuholníka $ABC$, tak $DHMO$ je rovnobežník.
}
\podpis{Patrik Bak, Slovensko}

{%%%%%   MEMO, priklad t6
Nech $ABC$ je trojuholník. Os uhla $ABC$ pretína stranu~$AC$ v~bode~$L$ a~kružnicu opísanú trojuholníku $ABC$ znova v~bode $W\ne B$. Nech $K$ je kolmý priemet bodu~$L$ na priamku~$AW$. Kružnica opísaná trojuholníku $BLC$ pretína priamku~$CK$ znova v~bode $P\ne C$. Priamky $BP$ a~$AW$ sa pretínajú v~bode~$T$. Dokážte, že $|AW| = |WT|$.}
\podpis{Ukrajina}

{%%%%%   MEMO, priklad t7
Nech $a_1,a_2,a_3,\dots$ je postupnosť kladných celých čísel taká, že
$$
a_1=1\qquad\text{a}\qquad a_{k+1}=a_k^3+1\quad\text{pre všetky kladné celé čísla $k$.}
$$
Dokážte, že pre každé prvočíslo~$p$ tvaru $3l+2$, kde $l$ je nezáporné celé číslo, existuje kladné celé číslo~$n$ také, že $a_n$ je deliteľné~$p$.
}
\podpis{Poľsko}

{%%%%%   MEMO, priklad t8
Celé číslo $n$ nazývame {\it sliezske}, ak existujú kladné celé čísla $a$, $b$ a~$c$ také, že
$$
n=\frac {a^2 + b^2 + c^2}{ab + bc + ca}.
$$
\ite a) Dokážte, že existuje nekonečne veľa sliezskych čísel.
\ite b) Dokážte, že nie všetky kladné celé čísla sú sliezske.
}
\podpis{Nemecko}

{%%%%%   CPSJ, priklad 1
Dané sú tri kladné reálne čísla $x$, $y$, $z$. Rozhodnite, či vždy existuje štvorica $(a,b,c,d)$ reálnych čísel taká, že
$$
ad+bc=x,\qquad ac+bd=y,\qquad ab+cd=z
$$
a~jedno z~čísel $a$, $b$, $c$, $d$ je rovné súčtu zvyšných troch.}
\podpis{\L{}ukasz Bożyk}

{%%%%%   CPSJ, priklad 2
Daný je konvexný šesťuholník $ABCDEF$, ktorého strany $AB$ a~$DE$ sú rovnobežné. Každá z~uhlopriečok $AD$, $BE$, $CF$ rozdeľuje šesťuholník na dva štvoruholníky s~rovnakým obvodom. Dokážte, že tieto tri uhlopriečky sa pretínajú v~jednom bode.
}
\podpis{Jaroslav Švrček}

{%%%%%   CPSJ, priklad 3
Pani učiteľka dala každému zo svojich 37 žiakov 36 pasteliek rôznych farieb. Pritom každí dvaja žiaci dostali práve jednu pastelku rovnakej farby. Určte najmenší možný počet farieb, ktoré môžu mať všetky rozdané pastelky.
}
\podpis{Patrik Bak}

{%%%%%   CPSJ, priklad 4
Nájdite najmenšie prirodzené číslo~$A$ s~nepárnym počtom cifier také, že $A$ je deliteľné číslom~2\,018 a číslo~$B$, ktoré vznikne tak, že v~$A$ vynecháme prostrednú cifru, je tiež deliteľné číslom 2\,018.
}
\podpis{Jaromír Šimša}

{%%%%%   CPSJ, priklad 5
Daný je ostrouhlý trojuholník $ABC$, pričom $|AB| < |AC|$. Na jeho strane~$AC$ uvažujme bod~$E$ taký, že $|AB| = |AE|$. Nech $AD$ je priemer kružnice opísanej trojuholníku $ABC$ a~$S$ je stred oblúka $BDC$ tejto kružnice. Označme $F$ obraz bodu~$D$ v~stredovej súmernosti podľa stredu~$S$. Dokážte, že priamky $FE$ a~$AC$ sú navzájom kolmé.
}
\podpis{Patrik Bak}

{%%%%%   CPSJ, priklad t1
Pro přirozená čísla $a$, $b$ $c$ platí
$$
(a+b+c)^2\mid ab(a+b)+bc(b+c)+ca(c+a)+3abc.
$$
Dokažte, že
$$
(a+b+c)\mid (a-b)^2+(b-c)^2+(c-a)^2.
$$
}
\podpis{Patrik Bak}

{%%%%%   CPSJ, priklad t2
Daný je pravouhlý trojuholník $ABC$ s~preponou~$AB$. Nech $K$ je ľubovoľný vnútorný bod trojuholníka $ABC$ a~body $L$, $M$ sú obrazy bodu~$K$ v~osových súmernostiach postupne podľa priamok $BC$, $AC$. Určte všetky možné hodnoty výrazu $S_{ABLM}/S_{ABC}$, pričom $S_{XY\!\dots{}Z}$ označuje obsah mnohouholníka $XY\!\dots{}Z$.
}
\podpis{Jaroslav Švrček}

{%%%%%   CPSJ, priklad t3
Wyznacz wszystkie liczby rzeczywiste $r$ o~nast\ę{}puj\ą{}cej w\l{}asności: Jeżeli liczby rzeczywiste $a$, $b$, $c$ spe\l{}niaj\ą{} nierówność $|ax^2+bx+c|\le 1$ dla każdego $x\in\langle-1,1\rangle$, to spe\l{}niaj\ą{} również nierówność $|cx^2+bx+a|\leq r$ dla każdego $x\in\langle-1,1\rangle$.
}
\podpis{Jaromír Šimša}

{%%%%%   CPSJ, priklad t4
Přímka procházející středem $M$ rovnostranného trojúhelníku $ABC$ protíná jeho strany $BC$ a $CA$ po řadě v bodech $D$ a $E$. Kružnice opsané trojúhelníkům $AEM$ a $BDM$ se kromě bodu~$M$ dále protínají v~bodě $P$. Dokažte, že střed kružnice opsané trojúhelníku $DEP$ leží na ose úsečky $AB$.
}
\podpis{Tomasz Przyby\l{}owski}

{%%%%%   CPSJ, priklad t5
Okolo okrúhleho stola sedí $2n$ ľudí ($n\ge2$), pričom každý človek sa pozná s~oboma svojimi susedmi a~presne oproti nemu sedí človek, s~ktorým sa nepozná. Dokážte, že ľudí možno presadiť tak, že každý sa bude poznať práve s~jedným zo svojich dvoch susedov.
}
\podpis{\L{}ukasz Bozyk}

{%%%%%   CPSJ, priklad t6
Dodatnie liczby rzeczywiste $a$, $b$ s\ą{} takie, że $a^3+b^3=2$. Wykaż, że zachodzi nierówność
$$
\frac1a+\frac1b\ge 2(a^2-a+1)(b^2-b+1).
$$
}
\podpis{Patrik Bak}
