{%%%%%   A-I-1
a) V~celej časti~a) budeme uvažovať ľubovoľné opísané vyplnenie tabuľky
${2\times n}$, ktoré je (vzhľadom na párny počet $2n$ všetkých políčok)
tvorené $n$~krížikmi a~$n$~krúžkami. Najskôr dokážeme, že v~jednom
z~oboch jej riadkov prevažujú krížiky práve vtedy, keď v~tom druhom
prevažujú krúžky. Oba riadky dokopy tak do výsledného skóre
$O-X$ prispejú nulou.

Označme $x_1$, $x_2$ počty krížikov v~prvom, resp. druhom riadku.
Podľa prvej vety nášho riešenia platí $x_1+x_2=n$. Predpokladajme, že
$x_1>n/2$, že teda v~prvom riadku prevažujú krížiky. Potom
$x_2=n-x_1<n-n/2=n/2$, takže v~druhom riadku naopak prevažujú
krúžky. Podobne z~$x_2>n/2$ vyplýva $x_1<n/2$. Dokázali sme, že
ak v~jednom z~riadkov prevažujú krížiky, potom v~tom druhom prevažujú
krúžky. Analogicky možno ukázať, že ak v~jednom z~riadkov prevažujú
krúžky, tak v~tom druhom prevažujú krížiky.

Teraz dokážeme, že všetkých $n$~stĺpcov tabuľky (každý o~dvoch políčkach)
súhrnne do výsledného skóre $O-X$ tiež prispieva nulou. Na to
označme $x$, $r$, $o$ počty stĺpcov, ktoré obsahujú dva krížiky
(a~žiadny krúžok), resp. jeden krížik (a~jeden krúžok), resp. dva
krúžky (a~žiadny krížik). Ako vieme, tabuľka obsahuje celkom $n$~krížikov aj $n$~krúžkov, takže platí $n=2 x +r$ a~súčasne
$n=r+2 o$. Porovnaním dostaneme rovnosť $x=o$, ktorá nám hovorí,
že počet stĺpcov, v~ktorých prevažujú krížiky, je rovný počtu stĺpcov,
v~ktorých prevažujú krúžky. Aj stĺpce tak do výsledného skóre $O-X$
tabuľky $2\times n$ dokopy prispievajú nulou, a~riešenie časti~a)
je tak ukončené.

\smallskip
b) Pre dané~$n$ uvažujme ľubovoľné opísané vyplnenie tabuľky $(2n+1)\times (2n+1)$.
Keďže počet $2n+1$ políčok v~každom riadku a~stĺpci je nepárne číslo,
v~každom z~nich bude jedna z~oboch značiek prevažovať.
Aj počet $(2n+1)^2$ políčok v~celej tabuľke je nepárne číslo,
takže všetkých krúžkov je o~1 menej ako všetkých krížikov, je ich teda
$\frac12\bigl((2n+1)^2-1\bigr)=2n(n+1)$. Preto je tých riadkov, v~ktorých krúžky prevažujú (\tj.~sú tam v~počte aspoň $n+1$),
nanajvýš $2n(n+1):(n+1)=2n$, teda naopak krížiky prevažujú aspoň
v~jednom riadku. Rovnako tak vysvetlíme, že krúžky prevažujú v~nanajvýš
$2n$ stĺpcoch a~krížiky naopak aspoň v~jednom. Dokopy to
znamená, že $O\le 2n+2n=4n$ a~$X\ge 1+1=2$, takže výsledné skóre
$O-X$ nemôže byť väčšie ako $4n-2$.

Teraz dokážeme, že skóre $4n-2$ možno (pre každé $n$) dosiahnuť.
Z~úvahy predchádzajúceho odseku vyplýva, kedy $2n(n+1)$-prvková podmnožina~
$\mn M$ políčok tabuľky ${(2n+1)}\times ({2n+1})$ s~vpísanými krúžkami bude
príkladom želaného vyplnenia: v~$2n$~riadkoch a~$2n$~stĺpcoch bude po
$n+1$ políčkach z~$\mn M$, zvyšný riadok a~zvyšný stĺpec budú bez
políčok z~$\mn M$. Zrejme stačí jednu takú množinu~$\mn M$ nájsť, lebo
vhodným vpisovaním krížikov a~krúžkov možno dosiahnuť to, aby políčka
s~krúžkami vytvorili vopred danú $2n(n+1)$-prvkovú množinu $\mn M$~--
stačí písať v~akomkoľvek poradí \uv{krížiky mimo $\mn M$ a~krúžky do~$\mn M$}.

Vyhovujúci príklad pre tabuľku $7\times 7$ (keď $n=3$) vidíme na
\obr. Pre všeobecné~$n$ vyzerá obdobná konštrukcia $2n(n+1)$-prvkovej
množiny $\mn M$ takto: V~poslednom riadku a~poslednom stĺpci tabuľky
$(2n+1)\times (2n+1)$ nevyberieme do~$\mn M$ žiadne políčko; v~riadkoch
a~stĺpcoch zvyšného štvorca $2n\times2n$ v~ľavom hornom \uv{rohu}
pôvodnej tabuľky vyberieme po $n+1$ políčkach nasledovne: v~prvom
riadku skupinu prvých ${n+1}$ políčok, ktorú v~každom nasledujúcom
riadku posunieme oproti predošlému riadku o~1~miesto doprava,
pritom políčka z~posledného stĺpca tejto podtabuľky presúvame do stĺpca
prvého. Toto pravidlo bude platiť aj pre prechod z~posledného riadka
k~riadku prvému, takže vybraných bude $n+1$ políčok nielen v~každom riadku, ale
aj~v~každom stĺpci (všeobecnejšia procedúra je opísaná v~návodnej úlohe~N1).
\inspdf{67ai1r.pdf}%

Nakoniec poznamenajme, že dokázaná rovnosť $\max(O-X)=4n-2$ platí
aj~pre $n=0$, pretože tabuľka $1\times 1$ má jediné vyplnenie (jedným
krížikom) s~hodnotami $O=0$ a~$X=2$.


\návody
Dokážte, že pre každé dve kladné celé čísla $k\le n$ možno
všetky políčka tabuľky ${n\times n}$ ofarbiť čierno a~bielo tak, aby sa
v~každom riadku a~každom stĺpci nachádzalo presne $k$ čiernych políčok.
[Skupinu $k$ čiernych políčok vyberme v~prvom riadku tabuľky ľubovoľne.
V~každom ďalšom riadku posunieme čierne políčka o~1~miesto doprava v~porovnaní s~aktuálnym riadkom, pritom políčka z~posledného stĺpca
presúvame do stĺpca prvého. Dostaneme tak vyhovujúce ofarbenie, pretože
ak označíme čierne políčka v~prvom riadku číslami 1 až~$k$ a~ak ich zachováme pri posunoch políčok v~ďalších riadkoch, z~úvahy o~počte posunov medzi
ľubovoľnými dvoma riadkami nám vyplynie, že v~každom stĺpci tabuľky bude
nakoniec práve po jednom políčku s~číslami 1 až~$k$.]

\D
Určte najvyššie možné skóre $O-X$ zo súťažnej úlohy dosiahnuteľné pre tabuľku $2n\times
2n$ v~závislosti na $n$. [Pre $n=1$ vyjde 0, pre $n=2$ vyjde 2, pre $n\ge 3$ vyjde $4n-8$.]
\endnávod
}

{%%%%%   A-I-2
Danú sústavu nerovníc zapíšeme ako
$$
F(x)>0\ \land\  G(x)<0, \tag1
$$
pričom $F(x)=x^2-(a-1)x-b$ a~$G(x)=x^2-ax-b+1$. Všimnime si, že pre
každé~$x$ platí $F(x)-G(x)=x-1$.

Podmienka úlohy $a+b>2$ znamená, že
$$
F(1)=G(1)=2-a-b<0,
$$
takže číslo 1 nie je riešením danej sústavy. Nerovnosť $G(1)<0$ však
znamená, že kvadratická rovnica $G(x)=0$ má jeden koreň
väčší ako~1, ten označíme $x_0$: $G(x_0)=0$ a~$x_0>1$. Potom
$$
F(x_0)=F(x_0)-0=F(x_0)-G(x_0)=x_0-1>0.
$$
Teraz z~nerovností $F(1)<0$ a~$F(x_0)>0$ vyplýva existencia koreňa~$x_1$
rovnice ${F(x)=0}$ v~otvorenom intervale $(1,x_0)$: $F(x_1)=0$
a~$1<x_1<x_0$. Zo~všeobecne známych vlastností kvadratickej funkcie
(s~kladným koeficientom pri~$x^2$) potom vyplýva, že
vďaka nerovnostiam
$$
F(1)<0\ \land\ F(x_1)=0,\quad\text{resp.}\quad
G(1)<0\ \land\ G(x_0)=0
$$
je sústava (1) splnená pre každé $x$ z~intervalu $(x_1,x_0)$,
pritom druhá nerovnosť $G(x)<0$ platí dokonca na väčšom intervale
$(1,x_0)$. Dôkaz je hotový.

\ineres
Trojčleny $F$ a~$G$ z~prvého riešenia majú diskriminanty
$$%\label{simsa:1}
D_F=(a-1)^2+4b=a^2-2a+4b+1\quad\text{a}\quad
D_G=a^2-4(1-b)=a^2+4b-4, \tag2
$$
ktoré sú vďaka podmienke $a+b>2$ (použitej v~tvare $b>2-a$) kladné:
$$
\align
D_F&=a^2-2a+4b+1>a^2-2a+4(2-a)+1=(a-3)^2\ge0,\\
D_G&=a^2+4b-4>a^2+4(2-a)-4=(a-2)^2\ge0.
\endalign
$$
Navyše z~toho vyplývajú nerovnosti
$$%\label{simsa:2}
\sqrt{D_F}>|a-3|\quad\text{a}\quad\sqrt{D_G}>|a-2|. \tag3
$$

Záver úlohy o~sústave nerovníc vyplynie z~poradia koreňov oboch
prislúchajúcich rovníc
$$
\frac{(a-1)-\sqrt{D_F}}{2}<
\frac{a-\sqrt{D_G}}{2}<
\frac{(a-1)+\sqrt{D_F}}{2}<
\frac{a+\sqrt{D_G}}{2},
$$
ktoré teraz dokážeme. (V~skutočnosti prvú nerovnosť zľava
nepotrebujeme, ale vyjde nám ako vedľajší produkt pri dôkaze
ostatných troch nerovností.)
Je zrejmé, že zapísaná štvorica nerovností je ekvivalentná
s~dvojicou
$$%\label{simsa:3}
\sqrt{D_F}+\sqrt{D_G}>1\quad\text{a}\quad
\bigl|\sqrt{D_F}-\sqrt{D_G}\bigr|<1. \tag4
$$

Prvá z~týchto nerovností (na rozdiel od druhej) vyplýva z~(3)
okamžite:
$$
\sqrt{D_F}+\sqrt{D_G}>|a-3|+|a-2|\ge|(a-3)-(a-2)|=1
$$
vďaka známej trojuholníkovej nerovnosti $|u\pm v|\le|u|+|v|$.
Pre
dôkaz druhej nerovnosti v~(4) vynásobíme obe jej strany kladným
výrazom $\sqrt{D_F}+\sqrt{D_G}$ a~dostaneme ekvivalentnú nerovnosť
$$
|D_F-D_G|<\sqrt{D_F}+\sqrt{D_G}.
$$

Keďže $D_F-D_G=5-2a$ podľa~(2), je našou poslednom úlohou
dokázať nerovnosť
$$
|5-2a|<\sqrt{D_F}+\sqrt{D_G}.
$$
To je jednoduché:
$$
|5-2a|=\bigl|(3-a)+(2-a)\bigr|\le|3-a|+|2-a|<\sqrt{D_F}+\sqrt{D_G},
$$
pritom sme opäť využili trojuholníkovú nerovnosť a~odhady~(3).



\návody
Nech $a>0$, $b$ a~$c<0$ sú reálne čísla. Dokážte, že
rovnica $ax^2+bx+c=0$ má jeden kladný a~jeden záporný koreň.
[Buď využite iba nerovnosti
$a>0$ a~$c<0$, pričom $c$ je hodnota uvažovaného trojčlena v~nule,
alebo zapíšte známe vzorce pre korene
a~prihliadnite na nerovnosť $D=b^2-4ac>b^2$.]

\D
Dokážte, že ak reálne čísla $a$, $b$, $c$ spĺňajú
$c(a+b+c)<0$, tak rovnica $ax^2+bx+c=0$ má riešenie $x$ v~intervale
$(0,1)$. [Ak označíme $f(x)=ax^2+bx+c$, tak $f(0)\cdot f(1)<0$, teda
jedna z~hodnôt $f(0)$, $f(1)$ je kladná a~druhá je záporná.]

Predpokladajme, že pre trojčlen $f(x)=ax^2+bx+c$ s~reálnymi koeficientmi
$a$, $b$, $c$ nemá rovnica $f(x)=x$ žiadne riešenie v~obore reálnych čísel.
Dokážte, že ho nemá ani rovnica $f(f(x))=x$.
[Platí buď $f(x)>x$ pre všetky~$x$, alebo $f(x)<x$ pre všetky~$x$,
preto tiež platí buď $f(f(x))>f(x)>x$ pre všetky $x$, alebo $f(f(x))<f(x)<x$
pre všetky~$x$.]
\endnávod}

{%%%%%   A-I-3
Nech $k_1$, $k_2$ sú dve zhodné kružnice s~polomerom
$r=1$ so stredmi postupne $O_1$, $O_2$, ktoré majú vonkajší dotyk. Nech
$ABCD$ je pravouholník vyhovujúci podmienkam úlohy (strany $AB$ a~$BC$
uvažovaného pravouholníka sú dotyčnicami kružnice~$k_1$ a~strany $CD$
a~$DA$ sú dotyčnicami kružnice~$k_2$). Zo zadania je zrejmé, že stredy
oboch kružníc ležia vnútri hľadaného pravouholníka. Označme ďalej $P$
priesečník priamok vedených stredom~$O_1$ rovnobežne
so stranou~$AB$ a~stredom~$O_2$ rovnobežne so stranou~$BC$ (\obr).
Označme ešte $\varphi$ veľkosť uhla $PO_1O_2$. Vzhľadom na symetriu
zadania stačí uvažovať iba hodnoty
$\varphi \in \langle 0;\frac14 \pi \rangle $.
\insp{a67.1}%

Pre skúmaný obsah~$S$ pravouholníka $ABCD$ podľa \obrr1{} platí
$$
S=|AB|\cdot |BC|=(2r+2r\cos\varphi)(2r+2r\sin\varphi)=
4(1+\sin \varphi)(1+\cos\varphi)
$$
(dosadili sme hodnotu $r=1$).
Teraz určíme najmenšiu a~najväčšiu hodnotu obsahu $S$ s~premenným
uhlom $\varphi \in \langle 0;\frac14 \pi \rangle $. Na to stačí určiť
najmenšiu a~najväčšiu hodnotu výrazu (funkcie)
$$
V(\varphi)=(1+\sin \varphi)(1+\cos \varphi) \qquad \text{pre}
\ \varphi \textstyle \in \langle 0;\frac14 \pi \rangle.
$$

Tento predpis funkcie najskôr upravíme nasledujúcim spôsobom:
$$
\textstyle V(\varphi)=1+\sin \varphi +\cos \varphi +\sin \varphi
\cos\varphi = \frac12 +(\sin \varphi +\cos \varphi)+
\frac12(\sin \varphi +\cos \varphi)^2.
$$
Všetko teda závisí od hodnoty $u=\sin \varphi +\cos \varphi$. Ukážeme,
že pre ľubovoľné $\varphi \in \langle 0;\frac14 \pi \rangle $ je
$1\leq u\leq \sqrt{2}$. Keďže pre také $\varphi$ je súčet
$\sin \varphi +\cos \varphi$ zrejme kladný,
môžeme namiesto hľadania minimálnej a~maximálnej hodnoty výrazu~$u$ hľadať
minimálnu a~maximálnu hodnotu výrazu~$u^2$ (na rovnakom intervale).
Platí
$$
u^2=\sin^2 \varphi +2\sin \varphi \cos\varphi +\cos^2
\varphi=1+\sin 2\varphi.
$$
Z~toho vzhľadom na nerovnosť $0\le\sin2\varphi\le1$ vyplýva
$$
1\leq u^2 \leq 2, \qquad \text{teda} \qquad 1\leq u\le\sqrt{2},
$$
pričom $u=1$ práve vtedy, keď $\varphi=0$, a~$u=\sqrt{2}$ práve vtedy, keď
$\varphi=\frac14\pi$.

Keďže kvadratická funkcia $V(u)=\frac12+u+\frac12u^2=\frac12(u+1)^2$
je na intervale $\langle 1;\sqrt{2} \rangle$ rastúca, platia pre každé
dotyčné~$u$ nerovnosti
$$
2=V(1)\leq V(u)\le V(\sqrt2)=\frac{3+2\sqrt{2}}2,
$$
ktoré už vedú k~potrebným odhadom obsahu $S=4V(\phi)$:
$$
8\leq S\leq 6+4\sqrt{2}. \eqno{(1)}
$$
Najmenšia hodnota~$S$ vo vzťahu (1) je pritom dosiahnutá práve vtedy, keď
$\varphi=0$, \tj. v~prípade, keď strany $AB$ a~$CD$ hľadaného
pravouholníka $ABCD$ sú dotyčnicami oboch daných kružníc $k_1$ a~$k_2$
a~súčasne strany $BC$ a~$DA$ sú postupne dotyčnicami iba kružníc $k_1$
a~$k_2$. Jedná sa teda o~prípad, keď hľadaným \uv{opísaným}
pravouholníkom (podľa podmienok úlohy) je obdĺžnik so stranami $4$ a~$2$,
ktorého obsah je naozaj~$8$. Podobne najväčšia hodnota~$S$ vo vzťahu~(1)
je dosiahnutá práve vtedy, keď
$\varphi=\frac14\pi$, \tj. v~prípade, keď hľadaným \uv{opísaným}
pravouholníkom je štvorec $ABCD$ s~obsahom $6+4\sqrt{2}$, na ktorého
uhlopriečke~$BD$ budú ležať stredy oboch daných kružníc.

\odpoved
Najmenší možný obsah je 8, najväčší $6+4\sqrt{2}$.

\ineres
Označme $a=|PO_1|$, $b=|PO_2|$ dĺžky odvesien pravouhlého trojuholníka~$O_1O_2P$
(ktorý môže prípadne degenerovať na jednu z~úsečiek $PO_1$ či $PO_2$).
Zároveň z~Pytagorovej vety máme $a^2+b^2=4r^2=4$ (čo opäť platí, aj keď
je jedna z~hodnôt $a$,~$b$ nulová).
Obsah~$S$ opísaného pravouholníka $ABCD$ potom môžeme vyjadriť (vďaka rozdeleniu úsečkami
na deväť menších pravouholníkov v~\obrr1) ako
$$
S=4r^2+2r(a+b)+ab=4+2(a+b)+ab. \tag2
$$
Ak teraz využijeme nerovnosti
$$
\sqrt{a^2+b^2}\leqq a+b\le \sqrt{2(a^2+b^2)}\quad\text{a}\quad
0\le ab\le\frac{a^2+b^2}2,
$$
ktoré zrejme platia pre ľubovoľné nezáporné čísla $a$ a~$b$, získame
pre obsah~$S$ podľa~(2) odhady
$$
4+2\cdot2+0=8\le S\le 4+2\cdot2\sqrt2+2=6+4\sqrt2.
$$

Minimálnu hodnotu $S=8$ tak dostaneme pre $ab=0$, teda pre obdĺžnik,
ktorého dve strany budú rovnobežné s~úsečkou~$O_1O_2$, a~maximálnu
hodnotu $S=6+4\sqrt2$ pre $a=b=\sqrt2$, teda pre štvorec, ktorého
uhlopriečka bude obsahovať body $O_1$, $O_2$.



\návody
Do rovnostranného trojuholníka so stranou 1 sú vpísané tri zhodné kružnice,
z~ktorých každá sa zvonka dotýka zvyšných dvoch kružníc a~súčasne sa dotýka
práve dvoch strán tohto trojuholníka. Určte polomer týchto troch kružníc.
[$\frac14(\sqrt3-1)$. Hľadaný polomer $r$ spĺňa $2r + 2r \cotg 30^{\circ}=1$.]

Dokážte, že pre každé $x\in\bb R$ platí ${-\sqrt2}\le \sin x + \cos x
\le \sqrt 2$. [Buď dokazujte ekvivalentnú nerovnosť
$(\sin x + \cos x)^2\le2$, alebo využite úpravu na výraz
s~hodnotou $\sin(x+\frac14\pi)$.]

Pre dané kladné čísla $x\ne y$ uvažujme priemery
$$
a=\frac{x+y}2,\ g=\sqrt{xy},\ h=\frac{2xy}{x+y},\
k=\sqrt{\frac{x^2+y^2}2}.
$$
(Jedná sa o~aritmetický, geometrický, harmonický a~kvadratický
priemer čísel $x$ a~$y$.)
Zo všetkých rozdelení štvorice $a$, $g$, $h$, $k$ na dve
dvojice $r$, $s$ a~$t$, $u$ vyberte to rozdelenie, pre ktoré má
výraz $V~= rs - tu$ najmenšiu kladnú hodnotu. [44--B--I--3]


\D
V~obdĺžniku $ABCD$ so~stranami $|AB|=9$, $|BC|=8$ ležia navzájom sa dotýkajúce
kružnice $k_1(S_1,r_1)$ a~$k_2(S_2,r_2)$ tak, že $k_1$ sa dotýka strán
$AD$ a~$CD$, $k_2$ sa dotýka strán $AB$ a~$BC$.
\item{a)} Dokážte, že $r_1+r_2=5$.
\item{b)} Určte najmenšiu a~najväčšiu možnú hodnotu obsahu trojuholníka~$AS_1S_2$.
\endgraf[62--A--S--1]
\endnávod
}

{%%%%%   A-I-4
Uvažujme nekonečnú postupnosť $(a_n)_{n=1}^\infty$
prirodzených čísel $a_n=\lfloor\sqrt n\rfloor$. Táto postupnosť je
zrejme neklesajúca, a~keďže
$$
k=\sqrt{k^2}<\sqrt{k^2+1}<\dots<\sqrt{k^2+2k}<\sqrt{k^2+2k+1}=k+1,
$$
obsahuje každé prirodzené číslo~$k$ presne $(2k+1)$-krát. Taký
popis uvažovanej postupnosti nám už postačí na určenie
zadaných súčtov $s_n=\sum_{i=1}^n a_i$ nasledujúcim spôsobom.

Pre ľubovoľné prirodzené~$n$ označme $k=\lfloor\sqrt n\rfloor$, čiže
$n=k^2+l$ pre vhodné $l\in\{0, 1,\dots, 2k\}$. Potom podľa
predchádzajúceho popisu platí
$$
\align
s_n&=\sum_{i=1}^{k-1} i(2i+1) + k~(l+1)
= 2\sum_{i=1}^{k-1}i^2 + \sum_{i=1}^{k-1} i+ k(l+1)=\\
&= 2 \frac{(k-1)(k-1+1)(2(k-1)+1)}{6} + \frac{(k-1)(k-1+1)}{2} +k(l+1)=\\
&=\frac{(k-1)k(2k-1)}3+\frac{k(k-1)}2+k(l+1)=
\frac{(k-1)k(4k+1)}6+k(l+1),
\endalign
$$
pričom sme využili vzťahy $1+2+\dots+n=\frac12n(n+1)$
a~$1^2+2^2+\dots+n^2={\frac16n(n+1)(2n+1)}$.

Pre ľubovoľné $k>6$ zostane hodnota zlomku
$$
\frac{(k-1)k(4k+1)}6
$$
aj po skrátení šiestimi prirodzeným číslom deliteľným
niektorým prvočiniteľom~$p$ čísla~$k$, ktorý samozrejme delí
aj ďalšieho sčítanca v~odvodenom vyjadrení súčtu~$s_n$.
Prvočíslo~$p$ tak bude deliteľom aj čísla~$s_n$, pričom $p\le
k<s_n$, takže číslo~$s_n$ prvočíslom nebude. Riešenie danej úlohy
preto stačí hľadať iba medzi takými prirodzenými číslami~$n$,
pre ktoré $k=\lfloor\sqrt n\rfloor\leq 6$, teda v~hre zostávajú
iba čísla $n<(6+1)^2=49$. Pre $n=48$ dosadením do odvodeného
vzorca pre $s_n$ spočítame
$s_{48}=203$, čo je číslo deliteľné siedmimi.
Pre $n=47$ vyjde $s_{47}=197$, čo je prvočíslo.

\odpoved
Hľadané číslo je $n=47$.



\návody
Dokážte, že $1+3+5+\dots+(2n-1)=n^2$.

Dokážte, že $1^2+2^2+\dots+n^2=\frac16n(n+1)(2n+1)$.

Zistite, koľko celočíselných riešení má rovnica
$$
\bigl\lfloor\!\root 1\,989 \of n\bigr\rfloor +
\biggl\lfloor\!\root 1\,989 \of {\frac {n+1}2} \biggr\rfloor + \dots +
\biggl\lfloor\!\root 1\,989 \of {\frac {n+1\,988}{1\,989}} \biggr\rfloor =1\,990.
$$
[38--B--II--4]

\D
Dokážte, že pre každé prirodzené číslo $n\ge 2$ platí
$$
\lfloor\sqrt n \rfloor +\lfloor\!\root3 \of n \rfloor
+\dots+\lfloor\!\root n \of n \rfloor =
\lfloor\log_2 n \rfloor +\lfloor\log_3 n \rfloor +\dots+\lfloor\log_n n
\rfloor.
$$
[Tabuľku $n\times n$ vyplníme číslami tak, že do políčka
v~$a$-tom riadku a~$b$-tom stĺpci vpíšeme číslo $a^b$
a~počet políčok, v~ktorých je číslo neprevyšujúce~$n$, spočítame
dvoma spôsobmi: po stĺpcoch a~po riadkoch.]
\endnávod
}

{%%%%%   A-I-5
Označme $k$ kružnicu opísanú trojuholníku~$BCD$.
Podľa zadania sa trojuholníky $ABC$ a~$ACD$ zhodujú v~dvoch vnútorných
uhloch (\obr), takže platí aj tretia rovnosť $|\uhel BAC|=|\uhel CAD|$.
Z~toho predovšetkým vyplýva, že súčet protiľahlých uhlov pri vrcholoch $A$ a~$C$
v~danom štvoruholníku $ABCD$ je väčší ako $180\st$, preto vrchol~$A$
musí ležať vo vnútornej oblasti kružnice~$k$.
\insp{a67.2}%

Predĺžme ako na obrázku úsečky $CA$ a~$DA$ na tetivy $CC'$
a~$DD'$ kružnice~$k$.
Zo zhodnosti obvodových uhlov nad jej oblúkom $D'BC$ máme
$$
|\uhel D'C'C| =|\uhel D'DC|=|\uhel ADC| = |\uhel ACB|,
$$
takže $BCC'D'$ je rovnoramenný lichobežník alebo pravouholník\niedorocenky{ (návodná úloha~N1)}. Podobné trojuholníky $ABC$ a~$AD'C'$
(pripomeňme, že je tiež $|\uhel C'AD'|=|\uhel CAD|=|\uhel BAC|$)
sú tak dokonca zhodné, teda $A$ je stredom základne~$CC'$
rovnoramenného trojuholníka~$OCC'$, čiže $|\uhel OAC|=90^\circ$,
čo sme mali dokázať.

\ineres
Skúmajme úlohu z~pohľadu trojuholníka~$ABD$.
Keďže polpriamka~$AC$ je os jeho vnútorného uhla pri vrchole~$A$,
ako už vieme z~úvodu prvého riešenia,
je $|\uhel BAC|=|\uhel CAD|<90\st$.
Preto uhol pri vrchole~$C$ konvexného štvoruholníka $ABCD$ je tupý,
takže stred~$O$ kružnice~$k$ opísanej trojuholníku~$BCD$ leží
rovnako ako bod~$A$ v~polrovine opačnej k~polrovine $BDC$ (\obr).
\insp{a67.3}%

Z~vlastností obvodového a~stredového uhla tak vyplýva, že veľkosť nekonvexného
uhla~$BOD$ je dvojnásobkom veľkosti konvexného uhla~$BCD$, takže
vzhľadom na rovnosť
$$
|\uhel BCD|=|\uhel ACB|+|\uhel ACD|=|\uhel ADC|+|\uhel ABC|
$$
a~pravidlo o~súčte vnútorných uhlov štvoruholníka platí
$$
|\uhel BOD|=360^\circ -2|\uhel BCD|=360^\circ-|\uhel BCD|
-|\uhel ADC|-|\uhel ABC|=|\uhel BAD|.
$$

Uhly $BOD$ a~$BAD$ sú teda zhodné, takže bod~$O$ leží na oblúku
$BAD$ kružnice opísanej trojuholníku $ABD$ a~ako stred kružnice
opísanej trojuholníku~$BCD$ má od krajných bodov tohto
oblúka rovnakú vzdialenosť. Je preto jeho stredom, takže ako je známe\niedorocenky{ (pozri úlohu~N2)}, leží
na osi vonkajšieho uhla pri vrchole~$A$ trojuholníka $ABD$.
Tvrdenie úlohy tak vyplýva z~toho, že polpriamky $AC$ a~$AO$ sú
osami dvojice susedných uhlov, ktoré sú vždy navzájom kolmé.


\ineres
Označme $M$ stred základne~$BC$ rovnoramenného trojuholníka~$OBC$ (\obr). Keďže
$|\uhel OMC|=90^\circ$, na dôkaz rovnosti $|\uhel OAC|=90^\circ$ stačí
ukázať, že body $O$, $A$, $M$, $C$ ležia na kružnici
(Tálesovej kružnici nad priemerom~$OC$).
Z~vlastností stredových a~obvodových uhlov v~kružnici opísanej
trojuholníku $BCD$ vyplýva $|\uhel MOC|=\frac12
|\uhel BOC|=|\uhel BDC|$, takže stačí dokázať zhodnosť
uhlov $BDC$ a~$MAC$,
pretože to dokopy zaručí, že body $A$ a~$O$ ležia na tom istom
kružnicovom oblúku nad tetivou~$CM$.
(Že body $A$ a~$O$ ležia v~tej istej polrovine určenej priamkou~$BD$,
už vieme z~predchádzajúceho riešenia, kde sme ukázali, že uhol $BCD$ je tupý.)
\insp{a67.4}%

Označme $A'$ obraz bodu~$A$ v~stredovej súmernosti podľa bodu~$M$. Potom
je $ABA'C$ rovnobežník a~želaná zhodnosť uhlov vyplynie
z~podobnosti trojuholníkov $AA'C$ a~$DBC$, ktorú teraz dokážeme
použitím vety {\it sus\/} so zameraním na spoločný vrchol~$C$.
Platí $|\uhel A'CB|=|\uhel ABC|$, takže $|\uhel A'CA|=|\uhel BCD|$.
K~tomu z~podobnosti trojuholníkov $ABC$ a~$ACD$ vyplýva
$|AC|/|CA'|=|AC|/|AB|=|DC|/|CB|$, čím je dôkaz hotový.


\návody
Ak v~tetivovom štvoruholníku $ABCD$ platí $|\uhel DAB|=|\uhel
ABC|$, je $AB\parallel CD$ a~$|BC|=|AD|$. Dokážte. [Dokážte, že
trojuholníky $ABC$ a~$BAD$ sú súmerne združené podľa osi tetivy~$AB$
opísanej kružnice.]

Daný je trojuholník $ABC$ vpísaný do kružnice~$k$. Označme $D$
stred oblúka~$BC$ kružnice~$k$ obsahujúceho bod~$A$. Dokážte, že
v~prípade $A\ne D$ je priamka~$AD$ osou vonkajšieho uhla trojuholníka
$ABC$ pri vrchole~$A$. [Nech napr. $|AB|<|AC|$. Potom polpriamka~$AD$ delí uhol $CAX$, pričom $X$ je bod na predĺžení strany~$AB$
za vrchol~$A$, na dva uhly: jeden je uhol $DAC$ zhodný s~uhlom $DBC$,
druhý je vonkajší uhol pri vrchole~$A$ tetivového štvoruholníka $ABCD$,
zhodný s~protiľahlým vnútorným uhlom $BCD$.]

\D
Na stranách $AB$, $AC$ rôznostranného trojuholníka $ABC$ sú
dané postupne body $X$,~$Y$ tak, že $|BX|=|CY|=d>0$. Dokážte, že os
úsečky~$XY$ prechádza pevným bodom nezávislým na~$d$. [Je to stred~$M$
oblúka $BAC$: trojuholníky $XBM$ a~$YCM$ sú zhodné podľa vety~{\it sus}.]
\endnávod
}

{%%%%%   A-I-6
Poznamenajme na úvod, že spomenutá mocnina ako
celočíselný súčet musí mať nezáporný exponent. Iba také
mocniny čísla~2 budeme ďalej uvažovať.

Dokážeme, že množina $\mm M$ môže mať nanajvýš 6~prvkov, ako má
napríklad vyhovujúca (ako vzápätí overíme) množina
$$
\{-1, 3, 5, -2, 6, 10\}.
$$
Súčet ľubovoľných dvoch čísel z~trojice $\m1$, $3$, $5$ je totiž
mocninou dvoch a~to isté platí pre trojicu $\m2$, $6$, $10$. Akokoľvek
z~uvedenej množiny vyberieme tri čísla, budú niektoré dve z~nich patriť
jednej z~oboch spomenutých trojíc, a~tieto dve čísla
tak budú mať súčet rovný mocnine dvoch.

Teraz pripusťme, že existuje vyhovujúca množina s~viac ako šiestimi prvkami.
Keby množina~$\mm M$ obsahovala tri nekladné čísla, bol by súčet každých
dvoch z~nich záporný, každá mocnina dvoch je však kladná.
Preto množina~$\mm M$ obsahuje nanajvýš dve nekladné čísla, a~teda aspoň päť
kladných čísel. Označme $x$ najväčšie z~nich a~$a$, $b$,
$c$, $d$ nejaké štyri ďalšie kladné čísla z~$\mm M$.

Uvažujme štyri súčty $x+a$, $x+b$, $x+c$, $x+d$. Všetky sú väčšie ako~$x$
a~menšie ako~$2x$. V~intervale $(x,2x)$ však môže ležať nanajvýš jedna
mocnina dvoch, takže nanajvýš jeden z~týchto štyroch súčtov je rovný mocnine
dvoch a~zvyšné tri (bez ujmy na všeobecnosti $x+a$, $x+b$, $x+c$)
mocninami dvoch nie sú.
Použitím podmienky úlohy na trojice $(a,b,x)$, $(a,c,x)$ a~$(b,c,x)$ tak
zistíme, že všetky tri súčty $a+b$, $a+c$ a~$b+c$ musia byť mocniny dvoch.
Bez ujmy na všeobecnosti nech $a=\max\{a,b,c\}$. Podobne ako vyššie
sa zamerajme na otvorený interval $(a,2a)$. V~ňom leží nanajvýš jedna
mocnina dvoch, súčasne v~ňom však ležia obe (rôzne) čísla $a+b$ aj~$a+c$,
o~ktorých sme už usúdili, že musia byť mocninami dvoch.
Dospeli sme tak k~želanému sporu.

\odpoved
Najväčší možný počet prvkov množiny~$\mm M$ je rovný~6.

\ineres
Uvedieme iný dôkaz, že neexistuje vyhovujúca množina~$\mm M$ s~aspoň siedmimi prvkami,
ktorý využíva základné pojmy z~teórie grafov
\footnote{Pozri napr. Petr Hliněný, Základy teorie grafů, Elportál MU, Brno,\newline
URL: {\tt is.muni.cz/elportal/?id=878389}}.

Zrejme stačí ukázať, že žiadna sedemprvková množina~$\mm M$
celých čísel požadované vlastnosti nemá.
Pripusťme,
že taká množina~$\mm M$ existuje, a~jej prvky si predstavme
ako vrcholy istého grafu~$\mm G$. V~ňom dva vrcholy spojíme hranou,
ak je súčet čísel, ktoré predstavujú, mocninou dvoch.
Podmienka zo zadania úlohy hovorí, že v~takto zostavenom grafe sú
medzi každými tromi vrcholmi niektoré dva spojené hranou.
O~grafe~$\mm G$ so siedmimi vrcholmi uvedieme najskôr niekoľko
pozorovaní.

\medskip
\ite(1) Graf~$\mm G$ neobsahuje kružnicu dĺžky~4.

{\it Dôkaz}.
Pripusťme, že graf~$\mm G$ takú kružnicu $a_1a_2a_3a_4$ obsahuje,
takže pre vhodné celé nezáporné čísla $\al_i$ platí
$$
a_1+a_2=2^{\al_1},\
a_2+a_3=2^{\al_2},\
a_3+a_4=2^{\al_3},\
a_4+a_1=2^{\al_4}.
$$
Bez ujmy na všeobecnosti možno predpokladať, že
$a_1=\min\{a_1,a_2,a_3,a_4\}$ a~že $a_2<a_4$. Potom číslo
$\al_1$ je menšie ako ktorékoľvek z~troch (nie nutne rôznych) čísel
$\al_2$, $\al_3$ a~$\al_4$,
takže najvyššia mocnina dvoch, ktorá delí súčet
$2^{\al_1}+2^{\al_3}$, je menšia ako najvyššia mocnina dvoch,
ktorá delí súčet $2^{\al_2}+2^{\al_4}$.
To je však v~spore s~tým, že oba súčty sú rovné tomu istému číslu
$a_1+a_2+a_3+a_4$.
Graf~$\mm G$ teda naozaj neobsahuje kružnicu dĺžky~4.

\ite(2) Každý vrchol grafu~$\mm G$ má stupeň aspoň~3.

{\it Dôkaz}. Pre spor predpokladajme, že existuje vrchol~$v$
stupňa nanajvýš~2. Potom niektoré štyri zo siedmich vrcholov~$\mm G$ nie sú
s~$v$ spojené hranou; označme ich $a$, $b$, $c$, $d$.
Použitie podmienky úlohy na trojice $(v,a,b)$, $(v,b,c)$, $(v,c,d)$, $(v,d,a)$
implikuje existenciu hrán $ab$, $bc$, $cd$ a~$da$, ktoré
však tvoria kružnicu dĺžky~4, čo je v~spore s~(1).

\ite(3) Niektorý vrchol grafu~$\mm G$ má stupeň aspoň~4.

{\it Dôkaz}. V~každom grafe určite platí, že
súčet stupňov všetkých vrcholov je rovný dvojnásobku počtu hrán.
Špeciálne to znamená, že toto číslo je párne. Pri grafe so siedmimi vrcholmi
tak nie je možné, aby každý vrchol mal stupeň
presne~3 (stupeň menší ako tri je vylúčený podľa~(2)).

\medskip
Dokázané pozorovania o~našom grafe~$\mm G$ využijeme nasledovne.
Označme $v$ jeden jeho vrchol stupňa aspoň 4 a~$a$, $b$, $c$, $d$
niektoré štyri jeho susedné vrcholy. Keďže tie
majú stupne aspoň 3, z~každého z~nich vychádzajú aspoň dve hrany,
ktoré nesmerujú do vrcholu~$v$.
Pozdĺž každej takej hrany nakreslíme šípku. Tak získame
aspoň $4\cdot 2=8$ šípok, teda aspoň dve z~nich smerujú
do toho istého vrcholu rôzneho od vrcholu~$v$ (zopakujme, že do neho
nesmeruje žiadna z~nakreslených šípok). Koncové vrcholy takých
dvoch šípok spolu s~vrcholom~$v$ tvoria kružnicu dĺžky~4,
čo je želaný spor.

\návody
Rozhodnite, či existujú tri prirodzené čísla $x>y>z$
také, že $x+y$ aj~$x+z$ sú mocniny dvoch.
[Uvážte, koľko rôznych mocnín dvoch môže ležať v~intervale medzi
číslami $x$ a~$2x$ pre dané prirodzené~$x$.]

Na večierku má každý človek nepárny počet známych (\uv{poznanie sa}
je vzájomné). Dokážte, že počet ľudí na večierku je párny. [Súčet
všetkých počtov známych jednotlivých osôb je rovný dvojnásobku počtu
všetkých dvojíc osôb, ktoré sa poznajú, takže to je číslo párne.]

\D
Každú stranu a~uhlopriečku pravidelného šesťuholníka sme
ofarbili červenou alebo modrou. Dokážte, že existuje trojuholník s~vrcholmi
vo vrcholoch pôvodného šesťuholníka, ktorého všetky tri strany majú
rovnakú farbu. [Dôkaz tohto prvotného výsledku tzv. Ramseyho
teórie nájdete v~prehľadovom článku J.~Šimša, Ramseyova čísla
a~jeich uplatnění v~geometrii, URL:
{\tt mfi.upol.cz/old/MFI\_17\_pdf/Mat\_17\_10.pdf}.]
\endnávod

}

{%%%%%   B-I-1
Pri delení (so zvyškom) mnohočlena tretieho stupňa mnohočlenom druhého stupňa je
podiel rovný mnohočlenu prvého stupňa. Pritom jeho koeficient pri prvej mocnine
je rovný podielu koeficientov pri najvyšších mocninách delenca a~deliteľa, zatiaľ čo
absolútny člen podielu je neznáma, ktorú pri daných dvoch deleniach označíme $e$, resp.~$g$.
Pre hľadaný mnohočlen tak má platiť
$$
\align
ax^3+bx^2+cx+d=&(2x^2+1)\Big(\frac{a}{2}x+e\Big)+x+2
=ax^3+2ex^2+\Big(\frac{a}{2}+1\Big)x+e+2,\\
ax^3+bx^2+cx+d=&(x^2+2)(ax+g)+2x+1=ax^3+gx^2+(2a+2)x+2g+1.
\endalign
$$

Porovnaním koeficientov pravých strán predchádzajúcich dvoch rovností
dostaneme
$$
\align
2e=&g,\\
\frac{a}{2}+1=&2a+2,\\
e+2=&2g+1.
\endalign
$$
Riešením tejto sústavy je $a=\m\frac{2}{3}$, $g=\frac{2}{3}$,
$e=\frac{1}{3}$.
Úlohe preto vyhovuje jediný mnohočlen
$$
ax^3+bx^2+cx+d=-\frac{2}{3}x^3+\frac{2}{3}x^2+\frac{2}{3}x+\frac{7}{3}.
$$


\návody
Určte podiel a~zvyšok po delení mnohočlena $2x^3+x^2-3x+5$
dvojčlenom $x^2-x$. Túto skutočnosť zapíšte vo forme rovnosti bez
použitia zlomkov.
[$2x^3+x^2-3x+5 = (x^2-x)\*(2x+3)+5$]

Určte všetky kvadratické trojčleny $ax^2+bx+c$, ktoré sú bezo
zvyšku deliteľné ako dvojčlenom $x-2$, tak dvojčlenom $x+1$.
[$ax^2-ax-2a$, $a\ne 0$]

Nájdite všetky trojčleny $p(x)=ax^2+bx+c$, ktoré dávajú po delení
dvojčlenom $x+1$ zvyšok~$2$ a~po delení dvojčlenom $x+2$ zvyšok~$1$, pričom $p(1)=61$.
[61--C--I--1]

Nájdite všetky dvojice reálnych čísel $(p,q)$ také,
že mnohočlen $x^2+px+q$ je deliteľom mnohočlena $x^4+px^2+q$.
[56--B--I--5]
\endnávod
}

{%%%%%   B-I-2
Obe nerovnosti vynásobíme kladným číslom $t+1$, aby sme sa zbavili zlomkov,
$$
0\le t^2+1-(t+1)\sqrt t \le |t-1|(t+1),
$$
a~pre jednoduchšie úpravy použijeme substitúciu $u=\sqrt t>0$:
$$
0\le u^4+1-(u^2+1)u~\le |u^2-1|(u^2+1).
$$
Zrejme platí $u^4+1-(u^2+1)u=u^4-u^3-u+1=(u^3-1)(u-1)$
a~$|u^2-1|(u^2+1)=|u^4-1|$, máme teda pre ľubovoľné
kladné~$u$ dokázať nerovnosti
$$
0\le (u^3-1)(u-1) \le |u^4-1|.
$$

Ľavá nerovnosť teraz vyplýva zo známeho rozkladu $u^3-1=(u-1)(u^2+u+1)$, vďaka
ktorému $(u^3-1)(u-1)=(u-1)^2(u^2+u+1)$ je súčinom dvoch nezáporných čísel.
Pravú nerovnosť potom dostaneme pomocou dvoch zrejmých odhadov:
$$
(u^3-1)(u-1) \le|u-1|(u^3+1)\le |u-1|(u^3+u^2+u+1)=|u^4-1|.
$$
Tým sú obe nerovnosti dokázané.

\ineres
Na dôkaz oboch nerovností niekoľkokrát využijeme zrejmú nerovnosť
$2\sqrt t\le t+1$, ktorá platí pre ľubovoľné kladné~$t$.
Pre ľavú nerovnosť tak máme
$$
\frac{t^2+1}{t+1}-\sqrt t \ge\frac{t^2+1}{t+1}-\frac{t+1}{2}
=\frac{t^2-2t+1}{2(t+1)}=\frac{(t-1)^2}{2(t+1)} \ge 0
$$
a~pre pravú nerovnosť
$$
\frac{t^2+1}{t+1}-\sqrt t
=\frac{t^2+1-(t+1)\sqrt t}{t+1}
\le\frac{t^2+1-2\sqrt t\sqrt t}{t+1}
=\frac{|t-1|^2}{t+1}\le|t-1|,
$$
pretože $|t-1| \le t+1$ podľa trojuholníkovej nerovnosti.


\návody
Dokážte, že pre každé kladné reálne číslo~$u$ platí
$$
\frac{u}{\sqrt{2}}+\frac{2}{\sqrt{\vphantom2u}} \ge\sqrt{\vphantom2u}+\sqrt{2}.
$$
[Využite to, že dvojčlen $\sqrt{\vphantom2u}-\sqrt{2}$ možno vyňať ako
z~rozdielu prvých, tak aj druhých sčítancov oboch strán.
Úpravy na súčinový tvar môže uľahčiť, keď najskôr odstránime zlomky
alebo použijeme vhodnú substitúciu, napr. $s=\sqrt{\vphantom2u}$ či $t=\sqrt{2/u}$.]

Dokážte, že pre každé kladné reálne číslo~$a$ platí
$$\frac{1}{\sqrt{a}} > 2(\sqrt{a+1}-\sqrt{a}).$$
[Upravte nerovnosť na tvar $\frac12(a+a+1)>\sqrt{a(a+1)}$.]

Dokážte, že pre každé reálne číslo~$x$ platí
$$1+|x| \ge \sqrt{|x^2-1|}.$$
[Využite odhad $1+|x|\ge\frac12(|x-1|+|x+1|)$ a~potom AG nerovnosť.]

Dokážte, že pre každé dve reálne čísla $x$, $y$ platí
$$1+|x|+|y|+|xy| \ge\sqrt{|x^2-1||y^2-1|}.$$
[Vysvetlite, prečo hodnota $1+x^2+y^2+x^2y^2$ leží medzi hodnotami
druhých mocnín ľavej a~pravej strany dokazovanej nerovnosti.]
\endnávod
}

{%%%%%   B-I-3
Hľadanú veľkosť vnútorného uhla pri vrchole~$A$
uvažovaného kosoštvorca označme~$\alpha$.
Ďalej označme~$k$ kružnicu opísanú trojuholníku~$ACD$
a~$l$ kružnicu opísanú trojuholníku $BCE$.

Z~podmienok úlohy vyplýva, že $ABED$ je rovnoramenný lichobežník
s~kratšou základňou~$ED$. Je teda $|EB|=|DA|=|CB|$, takže trojuholník
$BCE$ je rovnoramenný so základňou~$CE$.

Bod~$C$ je podľa zadania jediným spoločným bodom kružníc $k$ a~$l$,
preto v~tomto bode existuje spoločná dotyčnica~$t$ oboch kružníc (\obr).
Vzhľadom na to, že $E$ je vnútorným bodom strany~$CD$ (tetivy kružnice~$k$),
majú vo vrchole~$C$ obe kružnice vnútorný dotyk. Pritom dotyčnica~$t$
zviera ako s~tetivou~$CD$ kružnice~$k$, tak s~tetivou~$CE$ kružnice~$l$
ten istý úsekový uhol vyznačený na obrázku. To znamená, že zodpovedajúce obvodové
uhly prislúchajúce uvedeným tetivám kružníc $k$ a~$l$ sú zhodné, čiže
$|\angle CAD|=|\angle CBE|$.
\insp{b67.1}%

Keďže uhlopriečka~$AC$ je osou vnútorného uhla pri vrchole~$A$
v~kosoštvorci $ABCD$, je $|\angle CAD|=\frac{1}{2}\alpha$, zatiaľ čo
rovnoramenný trojuholník $CEB$ má pri základni uhly veľkosti
$|\angle BEC|=|\angle BCE|=|\angle BAD|=\alpha$, takže
$|\angle CBE|=180^\circ-2\alpha$.
Spolu tak dostávame pre veľkosť uhla~$\alpha$ rovnicu
$$
180^{\circ}-2\alpha=\frac12\,\alpha,
$$
ktorej riešením je $\alpha = 72^\circ$.



\návody
Trojuholníku $ABC$ s~vnútornými uhlami $\alpha$, $\beta$, $\gamma$ je
opísaná kružnica. K~tejto kružnici sú vedené dotyčnice v~bodoch $A$, $B$, $C$.
Určte veľkosti vnútorných uhlov trojuholníka ohraničeného týmito dotyčnicami.
Rozoberte rôzne tvary trojuholníka $ABC$. [Napr. pre ostrouhlý
trojuholník s~rôzne veľkými vnútornými uhlami majú uhly nového
trojuholníka veľkosti $180^\circ - 2\alpha$, $180^\circ - 2\beta$,
$180^\circ - 2\gamma$.]

Dokážte, že rovnoramennému lichobežníku možno opísať kružnicu a~žiadnemu
inému lichobežníku kružnicu opísať nemožno.

V~rovnoramennom lichobežníku $ABCD$ platí $|BC|=|CD|=|DA|$
a~$|\uhol DAB|=|\uhol ABC|=36\st$. Na základni~$AB$ je daný bod~$K$
tak, že $|AK|=|AD|$. Dokážte, že kružnice opísané trojuholníkom $AKD$
a~$KBC$ majú vonkajší dotyk.
[53--B--I--2]

Nech $ABCD$ je lichobežník s~ostrými uhlami pri základni~$AB$. Nech $E$ je
taký bod základne~$AB$, že kružnice opísané trojuholníkom $AED$ a~$EBC$ sa dotýkajú
zvonku. Dokážte, že bod~$E$ leží na kružnici opísanej trojuholníku $CDV$,
kde $V$ je priesečník priamok $AD$ a~$BC$.
[53--B--S--3]

Daný je rovnobežník $ABCD$ s~tupým uhlom $ABC$. Na jeho uhlopriečke~$AC$
v~polrovine $BDC$ zvoľme bod~$P$ tak, aby platilo $|\angle
BPD|=|\angle ABC|$. Dokážte, že priamka~$CD$ je dotyčnicou ku kružnici
opísanej trojuholníku $BCP$ práve vtedy, keď úsečky $AB$ a~$BD$ sú zhodné.
[59--A--II--2]
\endnávod
}

{%%%%%   B-I-4
Ľavú stranu danej rovnice najskôr upravíme
$$
\align
a+ab+abc+ac+c=&a(1+b)+ac(1+b)+c =a(1+b)(1+c) + c =\\
=&a(1+b)(1+c) + (1+c)-1=(1+c)(a(1+b)+1)-1,
\endalign
$$
vďaka čomu dostaneme ekvivalentnú rovnicu
$$
(1+c)(a(1+b)+1) = 2\,018.
$$

Číslo $2\,018$ sa dá napísať ako súčin dvoch čísel dvoma spôsobmi:
$1\cdot2\,008$ alebo $2 \cdot 1\,009$. Keďže je $1+c\ge2$ a~${a(1+b)}+1\ge3$,
môže byť jedine $1+c=2$ a~${a(1+b)+1}=1\,009$, čiže $c=1$
a~$a(1+b)=1\,008$.
V~každej vyhovujúcej trojici $(a,b,c)$ je tak $c=1$, a~preto vlastne
hľadáme počet dvojíc $(a,b)$ spĺňajúcich rovnicu z~konca predchádzajúcej vety.

Číslo $1\,008 = 2^4 \cdot 3^2 \cdot 7$, ktoré sa má rovnať $a(1+b)$,
má $5 \cdot 3 \cdot 2 = 30$
rôznych deliteľov vrátane jednotky a~seba. Keďže $1+b \ge2$, nemôže byť
$a=1\,008$. Pre každý iný z~29~deliteľov~$a$ čísla~$1\,008$ dostaneme
jednu dvojicu riešenia $(a,b)$.

\odpoved
Hľadaný počet trojíc je rovný 29.


\návody
Určte počet deliteľov čísla $2\,016$. [$6 \cdot 3 \cdot 2 = 36$
deliteľov]

Určte všetky dvojice prirodzených čísel $a$, $b$, pre ktoré platí
$a+ab+b=2\,017.$
[$(a, b)=(1, 1\,008)$, $(a, b)=(1\,008, 1)$]

Nájdite všetky dvojice nezáporných celých čísel $a$, $b$, pre ktoré platí $a^2+b+2=a+b^2$.
[59--C--S--3, $(a, b)=(1, 2)$, $(a, b)=(0, 2)$]

Nájdite všetky trojice celých čísel $x$, $y$, $z$, pre ktoré platí
$$
x+yz=2\,005, \quad
y+xz=2\,006.
$$
[54--C--S--1, $(x, y, z)=(669, 668, 2)$, $ (x, y, z)=(2\,005, 2\,006, 0)$]
\endnávod
}

{%%%%%   B-I-5
Najskôr ukážeme, že priesečník~$S$ uhlopriečok $AC$, $BD$ leží
na spojnici stredu~$K$ základne~$AB$ a~stredu~$L$ základne~$CD$.
Zo zhodnosti vyznačených striedavých uhlov na \obr\
vyplýva podobnosť trojuholníkov $ABS\sim CDS$ (veta $uu$), čo zaručuje
podobnosť aj ich \uv{polovíc}, trojuholníkov $AKS$ a~$CLS$ (podľa vety~$sus$,
pretože v~prv uvedenej podobnosti si zodpovedajú nielen strany $AS$ a~$CS$,
ale aj prislúchajúce polovice strán $AB$ a~$CD$). Z~toho dostávame zhodnosť
uhlov $ASK$ a~$CSL$, na jednej priamke teda ležia nielen ich prvé, ale
aj druhé ramená, čiže $S\in KL$, ako sme sľúbili ukázať.
\insp{b67.2}%

Predpokladajme, že $|AB|>|CD|$ (inak zmeníme označenie vrcholov). %Potom
Z~porovnania strán a~výšok trojuholníkov pre ich obsahy dostávame
$$
S_{ABC}=S_{ABD}>S_{CDA}=S_{CDB}.
$$
Preto uvažovaná rovnobežka s~$AC$ pretína trojuholník $ABC$
(a~nie $ACD$), teda jeho strany $AB$ a~$BC$ v~bodoch, ktoré označíme postupne $E$ a~$F$. Keďže
$$
S_{EBF}=\frac12 S_{ABCD}>\frac12 S_{ABC},
$$
leží bod~$E$ na strane~$AB$ zaručene v~jej \uv{prvej
polovici}~$AK$. Podobne skúmaná rovnobežka s~$BD$ pretína strany $AB$
a~$AD$ v~bodoch, ktoré označíme $G$ a~$H$, pričom bod~$G$ leží na
strane~$AB$ medzi bodmi $K$ a~$B$ (\obr). Priesečník~$P$ priamok $EF$ a~$GH$ teda leží
v~trojuholníku $ABS$ a~podľa vety~$uu$ platí $\triangle ABS\sim\triangle EGP$.
\insp{b67.3}%

Ešte dokážeme, že bod~$K$ je stredom strany~$EG$ trojuholníka $EGP$. Potom už
bude totiž jednoduché ukázať, že bod~$P$ leží na úsečke~$KS$.

Všimnime si, že v~podobnostiach $\triangle ABC\sim\triangle EBF$
a~$\triangle ABD\sim\triangle AGH$ majú oba trojuholníky $ABC$ a~$ABD$ rovnaký obsah.
Obsahy trojuholníkov $EBF$ a~$AGH$ sú tiež rovnaké (rovné polovici
obsahu daného lichobežníka $ABCD$), takže obe podobnosti majú rovnaký koeficient,
čiže $|EB|/|AB|=|AG|/|AB|$. Platí tak
$|EB|=|AG|$, čiže $|EK|+|KB|=|AK|+|KG|$, odkiaľ vďaka $|KB|=|AK|$
vyplýva $|EK|=|KG|$.

Bod~$K$ je teda spoločným stredom úsečiek $AB$ a~$EG$. Z~podobnosti trojuholníkov
$ABS$ a~$EGP$ tak vyplýva aj podobnosť trojuholníkov $SAK$ a~$PEK$ (rovnakú
úvahu sme urobili pre inú dvojicu trojuholníkov už na začiatku riešenia).
Jej dôsledkom je zhodnosť uhlov $ASK$ a~$EPK$,
z~ktorej ale vyplýva $SK\parallel PK$, takže bod~$P$ musí nutne ležať
na úsečke~$SK$. Leží teda aj na spojnici~$KL$, čo sme mali dokázať.

\poznamka
Riešitelia budú koeficienty podobností z~posledného
odseku riešenia zrejme vyjadrovať cez obsahy trojuholníkov vyjadrené
vzorcami
$$
S_{ABC}=S_{ABD}=\frac{av}{2},\quad
S_{EBF}=S_{AGH}=\frac12 S_{ABCD}=\frac{(a+c)v}{4},
$$
pričom $a$, $c$ a~$v$ sú dĺžky základní a~výšky daného lichobežníka.

Alebo sa dá pracovať so vzdialenosťami bodov od priamok a~ukázať, že platí
$$
\frac{d(P,AS)}{d(P,BS)}=\frac{d(K,AS)}{d(K,BS)}.
$$
(Polpriamka~$SK$ je množinou všetkých tých bodov uhla $ASB$,
ktoré majú od jeho ramien rovnaký pomer vzdialeností ako bod~$K$.)
Ak označíme $k$ spoločný koeficient oboch podobností $\triangle
ABC\sim\triangle EBF$ a~$\triangle ABD\sim\triangle AGH$, dostaneme
$$
\frac{d(P,AS)}{d(P,BS)}=\frac{d(B,AC)-d(B,EF)}{d(A,BD)-d(A,GH)}=
\frac{(1-k)d(B,AC)}{(1-k)d(A,BD)}=\frac{d(B,AC)}{d(A,BD)},
$$
avšak $d(B,AC)=2d(K,AS)$ a~$d(A,BD)=2d(K,BS)$, čím je dôkaz ukončený.


\návody
Daný je lichobežník $ABCD$ ($AB\parallel CD$). Dokážte, že štyri
body, a~to stredy základní lichobežníka, priesečník jeho uhlopriečok
a~priesečník priamok oboch ramien lichobežníka, ležia na jednej
priamke.

Daný je lichobežník $ABCD$ ($AB\parallel CD$), $S$ označuje priesečník
jeho uhlopriečok. Dokážte, že obsahy trojuholníkov $BCS$ a~$ADS$ sú
rovnaké.

Je daný lichobežník $ABCD$. Stred základne~$AB$ označme~$P$.
Uvažujme rovnobežku so základňou~$AB$, ktorá pretína úsečky $AD$, $PD$, $PC$, $BC$ postupne
v~bodoch $K$, $L$, $M$,~$N$.
\item{a)} Dokážte, že $|KL| =|MN|$.
\item{b)} Určte polohu priamky~$KL$ tak, aby platilo aj $|KL|=|LM|$.\endgraf
[60--C--I--6]

Vnútri obdĺžnika $ABCD$ ležia body $X$ a~$Y$ tak, že celý obdĺžnik je
rozdelený na dva trojuholníky $ADX$, $BCY$ s~rovnakým obsahom a~dva konvexné
štvoruholníky $ABYX$ a~$CDXY$ taktiež s~rovnakým obsahom. Dokážte, že potom
úsečka $XY$ prechádza stredom obdĺžnika. [43--C--II--3]

Na strane~$BC$, resp. $CD$ rovnobežníka $ABCD$ určte body $E$,
resp. $F$ tak, aby úsečky $EF$, $BD$ boli rovnobežné a~trojuholníky $ABE$,
$AEF$ a~$AFD$ mali rovnaké obsahy.
[\hbox{58--B--I--3}]
\endnávod
}

{%%%%%   B-I-6
Najskôr ukážeme, že z~ľubovoľných siedmich po sebe idúcich čísel,
označme ich
$$
a,\,a+1,\,a+2,\,a+3,\,a+4,\,a+5,\,a+6,
$$
možno požadovaným spôsobom vybrať nanajvýš tri čísla. Za tým účelom všetkých
sedem čísel rozdelíme do troch množín
$$
\mm A=\{a,a+2,a+5\},\quad
\mm B=\{a+1,a+3\},\quad
\mm C=\{a+4,a+6\}.
$$
Keďže z~množiny~$\mm A$ môžeme vybrať nanajvýš dve čísla
a~z~každej z~množín $\mm B$ a~$\mm C$ nanajvýš po jednom čísle, stačí len vylúčiť
prípad, že z~množiny~$\mm A$ sú vybrané dve čísla (nutne $a+2$ a~$a+5$)
a~súčasne z~množín $\mm B$ a~$\mm C$ je vybrané po jednom čísle.
Vtedy kvôli číslu $a+2$
je z~$\mm C$ nutne vybrané číslo $a+6$ a~kvôli číslu $a+5$ je z~$\mm B$
vybrané číslo $a+1$. Súčasný výber čísel $a+1$, $a+6$ však možný
nie je.

Pomocou dokázanej vlastnosti teraz ľahko vysvetlíme, prečo zo všetkých
100~daných čísel od~1 po~100, ktoré máme podľa zadania k~dispozícii, nemožno
požadovaným spôsobom vybrať viac ako 44~čísel.
Za tým účelom zo všetkých 100~čísel zostavíme 14~disjunktných sedmíc po sebe
idúcich čísel
$$
\let\quad\enspace
\{1,2,3,4,5,6,7\},\quad
\{8,9,10,11,12,13,14\},\quad\dots,\quad
\{92,93,94,95,96,97,98\},
\tag1
$$
do ktorých sme nezahrnuli iba dve najväčšie čísla 99 a~100. Ako
sme ukázali, z~každej vypísanej sedmice možno vybrať nanajvýš tri
čísla, preto počet všetkých vybraných čísel naozaj nikdy
neprevýši číslo $14\cdot3+2=44$.

V~poslednej časti riešenia ukážeme, že výber 44~čísel je možný a~že
je pritom jediný (aj keď to dokázať zadanie úlohy nevyžaduje).
Predpokladajme teda, že máme ľubovoľný
vyhovujúci výber 44~čísel. Podľa predchádzajúceho odseku vieme, že
medzi vybranými musia byť dvojice čísel 99 a~100. Ak pozmeníme
zostavu 14~disjunktných sedmíc na
$$
\let\quad\enspace
\{3,4,5,6,7,8,9\},\quad
\{10,11,12,13,14,15,16\},\quad\dots,\quad
\{94,95,96,97,98,99,100\}, \tag2
$$
dôjdeme k~záveru, že medzi vybranými musia tiež byť čísla 1 a~2.
Je jasné, že použitím iných zostáv 14~sedmíc, zložených vždy z~niekoľkých
prvých sedmíc z~(1) a~niekoľkých posledných sedmíc z~(2), odvodíme, že medzi vybranými
číslami musia byť aj dvojice 8 a~9, 15 a~16,~\dots, 92 a~93,
spolu teda všetkých 15~dvojíc susedných čísel, ktoré po
delení siedmimi dávajú zvyšky 1 a~2.

Čo možno ďalej usúdiť
o~zvyšných $44-30=14$ vybraných číslach? Vďaka dokázanej
vlastnosti vieme, že to musia byť čísla po jednom vybrané
zo 14~sedmíc~(1); keďže v~každej z~týchto sedmíc sú určite
vybrané dve najmenšie čísla, tretie vybrané číslo zrejme musí byť to,
ktoré je v~sedmici piate najmenšie.

Dokázali sme, že každý vyhovujúci výber 44~čísel musí vyzerať takto:
$$
\{1,2,5\}\cup\{8,9,12\}\cup\{15,16,19\}\cup\dots\cup
\{92,93,96\}\cup\{99,100\}.
$$
Že tento výber, zložený zo 14~trojíc a~jednej dvojice, je naozaj
vyhovujúci, vysvetlíme ľahko: dve čísla z~rovnakej skupiny sa líšia
o~1, 3 alebo~4, rozdiel každých dvoch čísel ležiacich
v~susedných skupinách je jedno z~čísel 3, 4, 6, 7, 8, 10 či~11;
čísla z~nesusedných skupín majú rozdiely aspoň~10.


\návody
Daná je množina čísel $\{1,2,3,4,5,6,7,8,9,10\}$. Nájdite všetky
jej podmnožiny s~najväčším možným počtom prvkov, aby medzi nimi neboli
žiadne dve čísla, ktoré sa líšia o~2 alebo o~5.
[$\{1,2,5,8,9\}$, $\{2,3,6,9,10\}$]

Zistite, pre ktoré prirodzené čísla $n\ge2$ je možné z~množiny
$\{1,2,\dots,n-1\}$ vybrať navzájom rôzne párne čísla
tak, aby ich súčet bol deliteľný číslom~$n$.
[54--C--I--2]

Pre ktoré prirodzené čísla~$n$ možno z~množiny
$\{n, n + 1, n + 2,\dots, n^2\}$ vybrať štyri navzájom rôzne
čísla $a$, $b$, $c$, $d$ tak, aby platilo  $ab = cd$?
[54--C--S--2]

Z~množiny $\{1,2,3,\dots,99\}$ vyberte čo najväčší počet
čísel tak, aby súčet žiadnych dvoch vybraných čísel nebol násobkom
jedenástich.
[58--C--I--5]

Z~množiny $\{1,2,3,\dots,99\}$ je vybraných niekoľko rôznych čísel tak, že súčet žiadnych troch z~nich nie je násobkom deviatich.
\item{a)} Dokážte, že medzi vybranými číslami sú najviac štyri deliteľné tromi.
\item{b)} Ukážte, že vybraných čísel môže byť 26.\endgraf
[58--C--II--3]
\endnávod
}

{%%%%%   C-I-1
Hľadajme najskôr riešenie rovnice pre najmenšie trojciferné číslo
s~rovnakými ciframi, ktoré rovno rozložíme na súčin prvočísel:
$$
\overline {ab}^2- \overline {cd}^2
= \left (\overline {ab}+\overline {cd} \right) \left (\overline {ab}-\overline {cd} \right)
= 111= 37 \cdot3 .
\tag1
$$
Prvá zátvorka je kladná a~súčin oboch zátvoriek je kladné číslo, preto
sú obe zátvorky kladné. Číslo $\overline {abcd}$ je štvorciferné, teda
$a\ge 1$, takže $\overline {ab} \ge 10$, a~teda
$\overline {ab}+\overline {cd} \ge 10$. Keďže 37 a~3 sú prvočísla,
máme iba dve možnosti v~rozklade rovnice (1): buď
$\overline {ab}+\overline {cd} = 37$ a~$\overline {ab}-\overline {cd} = 3$,
alebo $\overline {ab}+\overline {cd} = 111$
a~$\overline {ab}-\overline {cd} = 1$.

V~prvom prípade dostávame riešením sústavy $\overline {ab}+\overline {cd} = 37$,
$\overline {ab}-\overline {cd} = 3$ hodnoty $\overline {ab} = 20$
a~$\overline {cd} = 17$, takže $\overline {abcd} = 2\,017$. V~druhom prípade
dostaneme obdobným spôsobom $\overline {abcd} = 5\,655$.

Teraz ukážeme, že žiadne číslo menšie ako $2\,017$ nemá požadovanú
vlastnosť. Pripusťme, že také číslo $\overline {abcd} <2\,017$
existuje. V~tom prípade musí byť $\overline {ab} \le 20$, a~teda
$\overline {ab}^2- \overline {cd}^2 \le 20^2= 400 <444$, takže
rozdiel $\overline {ab}^2- \overline {cd}^2$ je rovný
jednému z~čísel 111, 222 alebo 333. Prvú možnosť sme už rozobrali úplným
riešením rovnice~(1), rozoberme zvyšné dve možnosti.

Rovnica
$$
\left (\overline {ab}+\overline {cd} \right) \left (\overline {ab}-\overline {cd} \right)
= 222 = 2 \cdot3 \cdot 37
$$
vedie (s~prihliadnutím na $\overline {ab} \ge 10$) k~možnostiam\footnote{Možnosť
$(\overline {ab}+\overline {cd}) (\overline {ab}-\overline {cd}) = 222$
možno vylúčiť hneď pozorovaním, že obe zátvorky majú rovnakú paritu
(rozdiel medzi nimi je párne číslo), a~preto ich súčin nemôže byť párne
číslo~222, ktoré totiž nie je deliteľné štyrmi.}
$\left (\overline {ab}+\overline {cd}, \overline {ab}-\overline {cd} \right)
\in \{(37,6), (74,3),\allowbreak (111,2), (222,1)\}$. Vyriešením štyroch
prislúchajúcich sústav o~dvoch neznámych dostaneme
$$
\left (\overline {ab}, \overline {cd} \right)\in
\bigl\{
\big(\tfrac {43} 2, \tfrac {31} 2 \big),
\big(\tfrac {77} 2, \tfrac {71} 2 \big),
\big(\tfrac {113}2, \tfrac {109}2 \big),
\big(\tfrac {223}2, \tfrac {221}2 \big) \bigr\},
$$
čo nedáva žiadne celočíselné riešenie.

Nakoniec rovnica
$$
\left (\overline {ab}+\overline {cd} \right)
\left (\overline {ab}-\overline {cd} \right) = 333 = 3 \cdot3 \cdot 37
$$
vedie na možnosti
$\left (\overline {ab}+\overline {cd}, \overline {ab}-\overline {cd} \right)
\in \{(37,9), (111,3), (333,1)\}$.
Vyriešením troch prislúchajúcich sústav dostaneme
$$
\left (\overline {ab}, \overline {cd} \right) \in\{(23, 14), (57, 54), (166, 165)\},
$$
pričom ani v~jednom prípade nevychádza $\overline {ab}\le 20$.

\odpoved
Riešením úlohy je číslo $2\,017$.

\ineres
Budeme postupovať podobne ako v~predchádzajúcom riešení, len lepšie využijeme
to, že 111 je deliteľné prvočíslom~37. Riešme teda rovnicu
$$
(\overline {ab}-\overline {cd}) (\overline {ab}+\overline {cd}) = 3 \cdot 37 \cdot k,
$$
pričom $k$ je nejaká nenulová cifra.

Hľadáme prirodzené čísla $l$ a~$m$ také, že
$0<l<m$ a~$l \cdot m=3 \cdot 37 \cdot k$, a~pre ne potom vyriešime sústavu
$\overline {ab}-\overline {cd}=l$, $\overline {ab}+\overline {cd}=m$.
Pritom zrejme platí $\overline {ab} = \frac12(l+m)$, a~keďže
už hľadáme iba číslo $\overline {abcd}$ menšie ako 2\,017, môžeme predpokladať,
že $\overline {ab} \leq 20$, čiže $l+m \leq 40$.

Keďže súčin $lm$ je deliteľný prvočíslom~37, musí byť jedno z~čísel $l$, $m$
deliteľné~$37$. Keďže je však $l+m \leq 40$ a~$l<m$, musí byť $m=37$ a~$l\le3$,
a~teda $k= 1$, čo je prípad, ktorý sme už rozobrali.
Žiadne vyhovujúce číslo menšie ako $2\,017$ tak neexistuje.

\návody
Nájdite všetky trojciferné čísla $\overline {abc}$ také, že
$\overline {ab} \cdot c$ je trojciferné číslo s~tromi rovnakými
ciframi. [373, 376, 379, 743, 746, 749]

Nájdite najväčšie štvorciferné číslo $\overline {abcd}$ také, že
$\overline {ab} \cdot \overline {cd}$ je trojciferné číslo s~tromi
rovnakými ciframi. [7\,412]

\D
Nájdite všetky štvorciferné čísla $\overline {abcd}$, také, že
$\overline {ab}^2- \overline {cd}^2$ je štvorciferné číslo so štyrmi
rovnakými ciframi. [5\,645, 6\,734, 7\,823, 8\,912]

Nájdite všetky štvorciferné čísla $\overline {abcd}$, také že
$\overline {ab}^2- \overline {cd}^2$ je trojciferné číslo s~tromi
rovnakými ciframi. [2\,017, 2\,314, 2\,611, 2\,908, 3\,205, 4\,034, 4\,331, 5\,655,
5\,754, 5\,853, 5\,952, 6\,051, 7\,771, 9\,491]
\endnávod
}

{%%%%%   C-I-2
Pozrime sa najskôr na množiny s~najmenším možným počtom prvkov;
jednoprvkovú množinu môžeme zrejme v~rozdelení mať nanajvýš jednu~--
v~opačnom prípade by tie dve jednoprvkové množiny nemali rovnaký súčet
prvkov. Ostatné množiny sú teda aspoň dvojprvkové.

V~časti~a) vezmime ako jednoprvkovú množinu tú s~najväčším číslom, teda
$\{2\,017\}$. Všetky ostatné množiny potom budú aspoň dvojprvkové
a~mali by mať súčet prvkov 2\,017: zvyšné čísla $1, 2, \dots, 2\,016$
už ľahko rozdelíme na~1\,008 vyhovujúcich množín $\{1, 2\,016\}$,
$\{2,2\,015\}$,~\dots, $\{1\,008,1\,009\}$ so žiadaným
súčtom prvkov~2\,017.

Podľa práve uvedeného postupu dokážeme rozdeliť aj množinu
$\{1,2, \dots, 2\,018\}$ na 1\,009 množín $\{1,2\,018\}$, $\{2,2\,017\}$,~\dots,
$\{1\,009,1\,010\}$ s~rovnakým súčtom prvkov rovným~2\,019.

Teraz ešte ukážeme, prečo ani v~jednej z~častí a) či b) nemôžeme
vytvoriť viac ako 1\,009 množín. Predpokladajme, že by tých množín bolo
aspoň 1\,010. Keďže nanajvýš jedna z~nich môže byť jednoprvková
a~ostatné sú aspoň dvojprvkové, bol by počet všetkých ich prvkov
aspoň $1+1\,009 \cdot 2 = 2\,019$, zatiaľ čo my máme k~dispozícii iba 2\,017~prvkov
v~časti~a) a~2\,018 prvkov v~časti~b).

\poznamka
Súčet všetkých čísel v~časti~a) je $1\,009 \cdot 2\,017$ a~v~časti~b)
$1\,009 \cdot 2\,019$. Ak už uhádneme, že množín má byť 1\,009, tak zrejme
súčet prvkov v~jednotlivých množinách musí byť 2\,017 v~časti~a) a~2\,019
v~časti~b).

\návody
Nájdite vzorec pre súčet čísel $1+2+\cdots+n$. [$n(n+1) / 2$]

Zdôvodnite, že ak rozdelíme 2\,000 holubov do~1\,001 klietok, bude nejaká
klietka buď prázdna, alebo v~nej bude iba jeden holub.
[Ak by v~každej klietke boli aspoň dva holuby,
bolo by holubov spolu aspoň $2 \cdot 1\,001 = 2\,002$, čo dáva spor.]

\D
Na koľko najviac neprázdnych množín s~rovnakým súčtom je možné rozdeliť
množinu $\{1,2, \dots, n\}$? [$\lfloor (n+1) / 2 \rfloor$, \tj. $(n+1) / 2$, ak
je $n$ nepárne, a~$n / 2$, ak je $n$ párne]

Na koľko najviac neprázdnych množín so súčtami prvkov deliteľnými tromi
je možné rozdeliť množinu $\{1,2, \dots, 3m\}$?
[Príkladom rozdelenia na $2m$ množín je $m$ dvojprvkových množín $\{3k-2,3k-1\}$
a~$m$ jednoprvkových množín $\{3k\}$.
Vysvetlíme, prečo rozdelenie na viac ako $2m$ množín neexistuje.
Pri ľubovoľnom vyhovujúcom rozdelení môžeme od každej viacprvkovej množiny
obsahujúcej násobok troch toto číslo \uv{odtrhnúť} ako jednoprvkovú množinu,
a~vytvoriť tak početnejšie rozdelenie. Preto pri hľadaní dotyčného maxima môžeme
uvažovať iba rozdelenia, v~ktorých všetkých $m$ násobkov troch tvorí jednoprvkové množiny.
Každé z~ostatných $2m$ čísel potom leží v~nejakej množine s~najmenej dvoma prvkami,
takže viacprvkových množín je nanajvýš $(3m-m):2=m$.]

Nájdite vzorec pre súčet čísel $1+3+5+\cdots+(2n-1)$. [$n^2$]
\endnávod
}

{%%%%%   C-I-3
Označme $c_a = |AF| = |AD|$ a~$c_b = |BE| = |BD|$. Priesečník priamky~$AE$
s~$CD$ označme~$P$. Z~podobnosti pravouhlých trojuholníkov $ABE$
a~$ADP$ vyplýva (\obr)
$$
% \frac {c_a} {c_a+c_b} = \frac {|PD|} {c_b}
\frac {|DP|}{|BE|} = \frac {|AD|}{|AB|},
\quad\text{čiže}\quad
|DP| = \frac {|AD|\cdot|BE|}{|AB|} =\frac{c_ac_b} {c_a+c_b} .
$$

Ak označíme analogicky $Q$ priesečník priamky~$BF$ s~$CD$, vyjde z~podobnosti
pravouhlých trojuholníkov $ABF$ a~$DBQ$
$$
|DQ| = \frac {c_ac_b}{c_a+c_b},
$$
čo znamená, že $|DP| = |DQ|$, a~teda $P = Q$, lebo oba body ležia na
polpriamke~$DC$.
\insp{c67.1}%

Ešte dokážeme, že bod~$P$ leží {\it vnútri\/} úsečky~$CD$, \tj.~$|PD| <|CD|$.
Z~konštrukcie bodu~$P$ vyplýva, že $|PD|<|EB|=c_b$ a~$|PD|<|FA|=c_a$.
Navyše pre odvesny oboch podobných pravouhlých trojuholníkov $ADC$ a~$CDB$
zrejme platí buď $c_a\le |CD|\le c_b$ (keď pre odvesny
trojuholníka $ADC$ platí $c_a=|AD|\le|CD|$, tak platí aj $|CD|\le|DB|=c_b$
v~trojuholníku $CDB$), alebo $c_b\le |CD|\le c_a$ (v~opačnom prípade).
V~každom prípade však je $|PD|<\min(c_a,c_b)\le |CD|$.

\ineres
Uvažovaný štvoruholník $ABEF$ je zrejme pravouhlý lichobežník alebo pravouholník.
Označme $P$ priesečník uhlopriečok $AE$ a~$BF$ a~$P_1$, $P_2$
jeho kolmé priemety na strany $AF$, resp. $BE$ (\obr).
Ukážeme, že bod~$P$ leží na kolmici na stranu~$AB$ vedenej bodom~$D$
(pätou výšky pôvodného trojuholníka $ABC$).
\insp{c67.2}%

Keďže $AE$ je priečka rovnobežiek $AF\parallel BE$, sú trojuholníky $AFP$ a~$EBP$
podobné podľa vety~$uu$, pričom pre ich výšky zo spoločného vrcholu~$P$
vďaka tomu platí
$$
\frac{|PP_1|}{|PP_2|}=\frac{|AF|}{|BE|}=\frac{|AD|}{|BD|}.
$$
Bod~$P$ teda delí úsečku~$P_1P_2$ rovnobežnú s~$AB$ v~rovnakom pomere, v~akom
bod~$D$ delí úsečku~$AB$. Preto $PD\parallel AF\parallel BE$, a~teda $PD\perp AB$,
ako sme sľúbili ukázať.

Ešte si uvedomme, že vďaka rovnostiam $|BE|=|BD|$ a~$|AF|=|AD|$ platia
pre odvesny pravouhlých trojuholníkov $ABF$ a~$ABE$ nerovnosti
$|BE|<|AB|$ a~$|AF|<|AB|$, takže každý z~uhlov $BAE$ i~$ABF$ je menší ako~$45^\circ$.

Ak sa vrátime k~zadaniu úlohy, vidíme, že priamky $AE$ a~$BF$
sa pretínajú na polpriamke~$DC$. A~keďže daný trojuholník $ABC$ je pravouhlý,
má jeden z~jeho uhlov $BAC$, $ABC$
veľkosť aspoň~$45^\circ$ (ich súčet je $90^\circ$). Bod~$P$ teda musí
určite ležať vnútri jedného z~uhlov $BAC$ či $ABC$, teda dokonca {\it vnútri\/}
úsečky~$DC$, čo sme chceli dokázať.



\návody
V~pravouhlom trojuholníku $ABC$ s~preponou~$AB$ označme $D$ pätu výšky
z~vrcholu~$C$. Dokážte, že platia tzv. Euklidove vety
$|CD|^2 = |AD|\cdot|BD|$, $|AC|^2 = |AB|\cdot|AD|$ a~$|BC|^2 = |AB|\cdot|BD|$.

Daný je trojuholník $ABC$, v~ktorom $D$ označuje pätu výšky
z~vrcholu~$C$. V~polrovine $ABC$
uvažujme body $F$, $E$ také, že uhly $EBA$, $FAB$ sú pravé, $|BE| = |BD|$
a~$|AF| = |AD|$. Dokážte, že priamky $AE$ a~$BF$ sa pretínajú na priamke~$CD$. [Využite podobnosť pravouhlých trojuholníkov.]

\D
Daný je pravouhlý trojuholník $ABC$ s~preponou $AB$, v~ktorom $D$
označuje pätu výšky z~vrcholu~$C$. V~polrovine
$ABC$ uvažujme body $F$, $E$ také, že uhly $EBA$,
$FAB$ sú pravé, $|BE| = |BD|$ a~$|AF| = |AD|$. Dokážte, že
sa priamky $AE$ a~$BF$ pretínajú
na~úsečke~$DG$, pričom $G$ je stred úsečky~$CD$.
[Uvedomte si, že dĺžka jedného z~úsekov $c_a$, $c_b$ je nanajvýš rovná polovici
dĺžky prepony~$AB$.]

Daný je pravouhlý trojuholník $ABC$ s~preponou~$AB$, v~ktorom $D$
označuje pätu výšky z~vrcholu~$C$. V~polrovine
$ABC$ uvažujme body $F$, $E$ také, že uhly $EBA$,
$FAB$ sú pravé, $|BE| = |BD|$ a~$|AF| = |AD|$. Dokážte, že priamka~$EF$
pretína úsečku~$CD$. [Priamka~$EF$ pretína polpriamku~$DC$ vo
vzdialenosti $2c_ac_b / (c_a+c_b)$ od bodu~$D$.
Alebo si uvedomte, že priesečník uhlopriečok lichobežníka či pravouholníka
$BEFA$ rozpoľuje zodpovedajúcu priečku, a~použite~D1.]
\endnávod
}

{%%%%%   C-I-4
Ľahko sa nám podarí požadovaným spôsobom zaplniť tabuľku $11 \times11$~--
stačí z~11~druhých mocnín celých čísel $1^2, 2^2, \dots, 11^2$ vybrať deväť
a~umiestniť ich na~deväť políčok tabuľky so súradnicami
$$
(3,3), \ (3,6), \ (3,9), \quad
(6,3), \ (6,6), \ (6,9), \quad
(9,3), \ (9,6), \ (9,9)
$$
a~ostatnými číslami zaplniť zvyšné políčka akokoľvek. Keďže
v~každej časti $3 \times3$ danej tabuľky je jedno z~uvedených deviatich
políčok, číslo $n = 11$ má požadovanú vlastnosť.

Ukážeme, že úlohe nevyhovuje žiadne $n\ge12$.
Pre ľubovoľné $n \ge 12$ označme $k$ celočíselný podiel čísla~$n$
po delení tromi so zvyškom, čiže $3k\le n\le3k+2$ a~${k\ge4}$.
V~tabuľke~$n\times n$
tak nájdeme $k^2$ neprekrývajúcich sa (disjunktných) štvorcov
$3 \times3$, k~dispozícii však máme iba $n$~druhých mocnín celých čísel
$1^2, 2^2, \dots, n^2$, pričom zrejme platí
$$
n\le3k+2<4k\le k^2.
$$
Pre $n\ge12$ tak danú tabuľku nedokážeme požadovaným spôsobom vyplniť.

\odpoved
Hľadané najväčšie $n$ je rovné 11.

\návody
Nájdite všetky prirodzené čísla~$n$, pre ktoré možno štvorcovú
tabuľku $n \times n$ zaplniť prirodzenými číslami od 1 do~$n^2$ tak, aby
v~každom riadku aj v~každom stĺpci bola zapísaná aspoň jedna druhá
mocnina celého čísla. [Ide to pre každé $n$~-- štvorce $1^2, 2^2,
\dots, n^2$ zapíšeme na uhlopriečku.]

Určte najväčšie celé číslo~$n$, pri ktorom je možné štvorcovú tabuľku
$n\times n$ zaplniť prirodzenými číslami od 1 do~$n^2$ tak, aby
v~každej jej štvorcovej časti $2 \times2$ bola zapísaná aspoň jedna
druhá mocnina celého čísla. [$n = 5$]

\D
Do štvorcovej tabuľky $11 \times 11$ sme vpísali prirodzené čísla
$1, 2, \dots, 121$ postupne po riadkoch zľava doprava a~zhora nadol. Štvorcovou
doštičkou $4 \times 4$ sme všetkými možnými spôsobmi zakryli práve 16~políčok.
Koľkokrát bol súčet zakrytých 16~čísel druhou mocninou celého čísla?
[65--B--I--2]

Štvorcovú tabuľku $6\times 6$ zaplníme všetkými celými číslami od
1 do 36.
\item{a)} Uveďte príklad takého zaplnenia tabuľky, že súčet každých
dvoch čísel v~rovnakom riadku či v~rovnakom stĺpci je väčší ako~11.
\item{b)} Dokážte, že pri ľubovoľnom zaplnení tabuľky sa v~niektorom riadku alebo stĺpci nájdu
dve čísla, ktorých súčet neprevyšuje~12.\endgraf
[66--C--II--2]
\endnávod
}

{%%%%%   C-I-5
Ukážeme najskôr, že stred~$D$ kružnice vpísanej trojuholníku~$ABC$
leží na danej kružnici~$k$.

Zo súmernosti dotyčníc $AB$, $AC$ kružnice~$k$ podľa osi~$OA$ jej tetivy~$BC$ vyplýva, že
trojuholník $ABC$ je rovnoramenný a~stred~$D$ jeho vpísanej kružnice leží na spojnici hlavného
vrcholu~$A$ a~stredu~$E$ základne~$BC$, pričom $DE$ je jej polomer (\obr).
Keďže stred~$D$ leží aj na osi uhla $ABC$, sú
vyznačené uhly $DBA$ a~$DBE$ zhodné. Zhodné sú aj vyznačené uhly $DAB$ a~$EBO$,
lebo oba dopĺňajú ten istý uhol $AOB$ do $90\st$. Pre vonkajší uhol $ODB$
trojuholníka $DAB$ tak platí
$$
|\uhel ODB| = |\uhel DBA|+|\uhel DAB|=|\uhel DBE|+|\uhel EBO|=|\uhel DBO|,
$$
takže trojuholník $OBD$ je rovnoramenný so základňou~$BD$,
a~preto $|OD|=|OB|=r$. Bod~$D$ tak naozaj leží na kružnici~$k$,
takže $|OE|=|OD|-|DE|=r-\rho$.
\insp{c67.3}%

Teraz z~podobnosti trojuholníkov $BOE$ a~$AOB$ zaručenej vetou $uu$ dostaneme
$$
\frac {r- \rho} {r} = \frac {|OE|} {|OB|}=\frac {|OB|} {|OA|}= \frac {r} {d},
\quad\text{čiže}\quad
\frac {r^2} d=r- \rho,
$$
teda
$$
\rho=r- \frac {r^2} d = \frac {rd-r^2} {d} = \frac {r (d-r)} {d}.
$$


\ineres
Podobne ako v~prvom riešení si uvedomíme, že body $A$, $D$, $O$ ležia
na jednej priamke, a~to, že bod~$E$ je pätou výšky z~vrcholu~$B$ v~trojuholníku
$ABO$. Označme navyše $T$ pätu kolmice z~bodu $D$ na~$AB$, čo je
\insp{c67.4}%
zároveň bod dotyku kružnice vpísanej trojuholníku~$ABC$ (\obr),
a~keďže priamky $BA$ a~$BC$ sú jej dotyčnice,
je $|BT| = |BE|$. Dĺžku~$|BE|$ určíme z~dvojakého
vyjadrenia obsahu trojuholníka $ABO$:
$$
2 \cdot S_{ABO} = |AO| \cdot |BE|=|AB| \cdot |OB|,
$$
takže
$$
|BT| = |BE|=\frac {|AB| r} {d}. \tag1
$$

Na výpočet polomeru~$\rho$ nakoniec použijeme podobnosť pravouhlých
trojuholníkov $ABO$ a~$ATD$ a~vyjadrenie (1):
$$
\align
\frac {\rho} {r}&=\frac {|DT|} {|OB|} = \frac {|AT|} {|AB|} = \frac {|AB|-|BT|}{|AB|}
= 1- \frac {|BT|} {|AB|} = 1- \frac {r} {d}, \\
\intext{odkiaľ}
\rho&=r\Bigl(1-\frac{r}{d}\Bigr)=\frac {r (d-r)} {d}.
\endalign
$$

\návody
Daná je kružnica $k(O, r)$ a~bod~$A$, pričom $|AO| = d> r$. Dotyčnice z~bodu~$A$
sa dotýkajú kružnice~$k$ v~bodoch $B$, $C$. Prechádza kružnica
opísaná trojuholníku $BCO$ bodom~$A$? [Áno, je to Tálesova kružnica nad
úsečkou~$AO$.]

Daná je kružnica $k(O, r)$ a~bod~$A$, pričom $|AO| = d> r$. Dotyčnice z~bodu~$A$
sa dotýkajú kružnice~$k$ v~bodoch $B$, $C$. Trojuholníku $ABC$ je
pripísaná kružnica ku strane~$BC$. Leží jej stred na kružnici~$k$?
[Áno, označme $|\angle BAC| = \alpha$ a~dopočítajme veľkosti uhlov pri
osiach vonkajších uhlov trojuholníka $ABC$ pri vrcholoch $B$ a~$C$.]

\D
Daný je trojuholník $ABC$ so stredom~$I$ vpísanej kružnice.
Priesečník osi strany~$BC$ s~oblúkom jemu opísanej kružnice, ktorý
neobsahuje vrchol~$A$, označme~$O$.
Dokážte, že kružnica $k(O, |OB|)$ prechádza bodom~$I$. [Najskôr
zdôvodnite, prečo bod~$O$ leží na polpriamke~$AI$ (ktorá je osou uhla
$BAC$) a~potom vyjadrite pomocou veľkostí uhlov trojuholníka $ABC$
veľkosť uhlov $IBO$ a~$BOI$.]

V~rovine sú dané kružnice $k$ a~$l$, ktoré sa pretínajú v~bodoch
$E$ a~$F$. Dotyčnica ku kružnici~$l$ zostrojená v~bode~$E$ pretína kružnicu~$k$
v~bode~$H$ ($H\ne E$). Na oblúku~$EH$ kružnice~$k$, ktorý neobsahuje bod~$F$,
zvoľme bod~$C$ ($E\ne C\ne H$) a~priesečník priamky~$CE$ s~kružnicou~$l$
označme~$D$ ($D\ne E$). Dokážte, že trojuholníky $DEF$ a~$CHF$ sú podobné.
[66--B--II--3]
\endnávod
}

{%%%%%   C-I-6
V~prípade ôsmich veží ich očíslujme v~smere otáčania hodinových ručičiek
číslami $1, 2, \dots, 8$. Jedno z~možných riešení prvej časti úlohy je, že
čierni rytieri obsadia na začiatku všetky veže s~párnymi číslami
(v~jednej z~nich budú dvaja rytieri a~vo zvyšných troch bude po jednom rytierovi)
a~podobným spôsobom obsadia červení rytieri veže s~nepárnymi číslami.
Po každej hodine sa situácia zmení len tak, že rytieri z~veží
s~párnymi číslami obsadia veže s~nepárnymi číslami a~naopak.
Vždy teda zostanú všetky veže strážené.

Prípad so siedmimi vežami je trochu náročnejší. Máme iba päť rytierov
každej farby, takže aspoň dve veže nebudú obsadené čiernymi a~aspoň dve
veže nebudú obsadené červenými rytiermi. Z~pohľadu čiernych rytierov sa
dve veže neobsadené červenými rytiermi (nazvime ich biele) každú hodinu
posunú o~dve pozície proti smeru otáčania hodinových ručičiek. Čísla 7 a~2 sú
nesúdeliteľné, takže z~tohto pohľadu sa každá biela veža vráti na svoju počiatočnú
pozíciu až po siedmich hodinách. Počas piatich hodín tak dve biele veže zaujmú
aspoň šesť rôznych pozícií (každá z~nich zaujme päť rôznych pozícií,
avšak zrejme nemôže ísť o~dve rovnaké pätice, to by sa veže dostali
na pôvodnú pozíciu šiestym posunom\footnote{Priebeh posunov jednej veže možno
znázorniť aj šípkami medzi vrcholmi načrtnutého sedemuholníka;
tým sa stane očividným ako tvrdenie o~perióde 7~hodín celého pohybu,
tak tvrdenie o~rôznosti pätíc po sebe nasledujúcich pozícií dvoch veží
s~rôznymi počiatočnými pozíciami.}), a~teda sa v~niektorej hodine aspoň jedna
dostane do aspoň jedného z~dvoch miest neobsadených čiernymi rytiermi.

\návody
Na kruhovom opevnení hradu sú štyri veže. Do nich sa rozmiestnia dvaja
čierni a~dvaja červení rytieri a~začnú strážiť. Po uplynutí každej
hodiny prejdú všetci čierni rytieri do susednej veže v~smere chodu
hodinových ručičiek a~všetci červení rytieri prejdú do~susednej veže
v~opačnom smere. Rozmiestnite rytierov tak, aby počas každej hodiny bola
každá veža strážená. [Stačí rozmiestniť rytierov tak, aby v~susedných
vežiach neboli rytieri rovnakej farby.]

Na kruhovom opevnení hradu sú tri veže. Do nich sa rozmiestnia dvaja
čierni a~dvaja červení rytieri a~začnú strážiť. Po uplynutí každej
hodiny prejdú všetci čierni rytieri do susednej veže v~smere chodu
hodinových ručičiek a~všetci červení rytieri prejdú do susednej veže
v~opačnom smere. Viete rozmiestniť rytierov tak, aby počas každej hodiny
bola každá veža strážená? [Nie. Na začiatku stráženia vyberte jednu vežu~$X$,
ktorú nestrážia čierni rytieri, a~jednu vežu~$Y$, ktorú nestrážia červení
rytieri. Podobne možno nestrážené veže v~ďalších hodinách určiť posunom
$Y$ a~$X$ v~smere, resp. v~protismere chodu hodinových ručičiek.
Raz za tri hodiny tak bude platiť $X=Y$.]

\D
~\vskip-\baselineskip\item{a)}
Marienka rozmiestni do vrcholov pravidelného osemuholníka rôzne počty
od jedného po osem cukríkov. Peter si potom môže vybrať, ktoré tri kôpky
cukríkov dá Marienke, ostatné si ponechá. Jedinou podmienkou je, že tieto tri
kôpky ležia vo vrcholoch rovnoramenného trojuholníka.
Marienka chce rozmiestniť cukríky tak, aby ich dostala čo najviac, nech už Peter
trojicu vrcholov vyberie akokoľvek. Koľko ich tak Marienka zaručene získa?
\item{b)}
Rovnakú úlohu vyriešte aj pre pravidelný deväťuholník, do ktorého
vrcholov rozmiestni Marienka 1 až~9~cukríkov. (Medzi rovnoramenné trojuholníky
zaraďujeme aj trojuholníky rovnostranné.)\endgraf
[66--C--I--6]

Každému vrcholu pravidelného 66-uholníka priradíme jedno z~čísel~1 alebo~${-1}$.
Ku každej úsečke spájajúcej dva jeho vrcholy (strane či uhlopriečke) potom
pripíšeme súčin čísel v~jej krajných bodoch a~všetky čísla pri jednotlivých
úsečkách sčítame. Určte najmenšiu možnú a~najmenšiu nezápornú hodnotu takéhoto
súčtu.
[66--B--I--1]
\endnávod
}

{%%%%%   A-S-1
Vyhovuje každé $p\le0$ (lebo $x=0$ je vtedy riešením
danej sústavy),
zatiaľ čo žiadne $p>0$ nevyhovuje, pretože sčítaním oboch nerovníc
dostaneme vzťah $2x^2+2p\le0$, ktorý v~prípade $p>0$
zrejme nespĺňa žiadne reálne číslo~$x$.

\ineres
Grafy funkcií $f(x)=x^2+ux+v$ a~$g(x)=x^2-ux+v$
(paraboly obrátené nahor) sú súmerne združené podľa osi~$y$,
lebo $f(x)=g(\m x)$ pre každé~$x$. Preto prípadné množiny riešení
jednotlivých nerovníc $f(x)\le0$, resp. $g(x)\le0$, ktorými
sú, ako je známe, uzavreté intervaly, ktoré môžu degenerovať
na jednobodové množiny, sú na číselnej osi~$x$ dve súvislé
množiny súmerne združené
podľa počiatku. Ich prienik je tak neprázdny práve vtedy, keď majú
spoločný bod $x=0$, čo nastane práve vtedy, keď $v=f(0)=g(0)\le0$.
Pre danú sústavu to je nerovnosť $p\le0$.
(Ako je jasné aj~z~predchádzajúceho riešenia, od tvaru
koeficientu $u=p-1$ výsledok vôbec nezáleží.)

\ineres
Uvažujme množiny $\langle x_2,x_1\rangle$ a $\langle x_4,x_3\rangle$
všetkých riešení prvej, resp. druhej nerovnice, pričom
$$
x_{1,2}=\frac{1-p\pm\sqrt{D}}{2}\quad\text{a}\quad
x_{3,4}=\frac{p-1\pm\sqrt{D}}{2},
$$
pritom $D=(p-1)^2-4p$ a~oba intervaly (aspoň ako jednoprvkové
množiny) existujú práve vtedy, keď $D\ge0$, čiže
$p\in({\m\infty},3-2\sqrt2\rangle\cup\langle3+2\sqrt2,\infty)$
(také~$p$ ďalej nazývame prípustné).
Tieto intervaly majú neprázdny prienik práve vtedy,
keď platí $x_1\ge x_4$ a~zároveň $x_3\ge x_2$.
Nutnosť tejto podmienky vyplýva z~nerovností $x_1\ge x_0\ge x_2$ a~$x_3\ge x_0\ge x_4$
pre bod~$x_0$ z~prieniku oboch intervalov. Naopak z~nerovností $x_1\ge x_4$ a~$x_3\ge x_2$
vyplýva, že pre čísla $a=\max(x_2,x_4)$ a $b=\min(x_1,x_3)$ platí $a\le b$,
takže prienikom skúmaných intervalov je neprázdna množina všetkých~$x$, pre ktoré $a\le x\le b$.

Nerovnosť $x_1\ge x_4$ platí práve vtedy, keď
$\sqrt{D}\ge p-1$, čo spĺňajú práve všetky prípustné $p\le1$,
zatiaľ čo druhá nerovnosť $x_3\ge x_2$ platí práve vtedy, keď
$\sqrt{D}\ge 1-p$, čo spĺňajú práve všetky prípustné
$p\notin(0,1)$. Obe podmienky tak spĺňajú práve všetky $p\le0$
(každé z~nich je prípustné). Riešenie je hotové.

Dodajme, že kľúčové nerovnosti $\sqrt{D}\ge p-1$ a~$\sqrt{D}\ge 1-p$
môžeme zapísať jedinou nerovnosťou $\sqrt{D}\ge|p-1|$. Z~toho
vzhľadom na $D=(p-1)^2-4p$ hneď vidíme, že hľadané~$p$ sú práve
všetky nekladné čísla (pre ne zrejme platí $D\ge0$, takže
určovanie množiny všetkých prípustných~$p$ bolo vlastne zbytočné).



\nobreak\medskip\petit\noindent
Za úplné riešenie dajte 6~bodov, z~toho:
4~body za dôkaz, že žiadne $p>0$ nevyhovuje;
2~body za dôkaz, že každé $p\le 0$ vyhovuje.
Neúplné riešenia:
2~body za pozorovanie, že grafy dvoch kvadratických funkcií na ľavých
stranách sú súmerne združené podľa osi~$y$; 1~bod za správnu
odpoveď (bez zdôvodnenia).
Pri postupe z~tretieho riešenia dajte 1~bod za výpis intervalov riešenia každej z~oboch nerovníc
a~za určenie hodnôt~$p$, pre ktoré sa jedná o~neprázdne množiny. Ďalej potom dajte po 2~bodoch
za vyriešenie každej z~podmienok $x_1\ge x_4$, $x_3\ge x_2$ a~1~bod za dokončenie.
\endpetit
\bigbreak
}

{%%%%%   A-S-2
Najskôr pripomeňme známy dôsledok tvrdenia o~kružnicovom oblúku
ako množine bodov významnej vlastnosti:
{\sl Ak ležia body $X$ a~$Y$ vnútri jednej polroviny
s~hraničnou priamkou~$BC$, tak nerovnosť $|\uhol BXC|<|\uhol BYC|$ platí
práve vtedy, keď bod~$Y$ leží vnútri kruhového odseku určeného úsečkou~$BC$
a~kružnicovým oblúkom~$BXC$.}

Našou úlohou je teda ukázať, že za predpokladu $|AB|<|AC|$ leží
bod~$S_b$ vnútri odseku určeného kružnicovým oblúkom $BS_cC$.

Ako je známe, stredná priečka~$S_bS_c$ trojuholníka~$ABC$ leží na priamke, ktorá
je rovnobežná so stranou~$BC$, a~tá preto vytne na kružnicovom
oblúku $BS_cC$ takú tetivu~$S_cP$, ktorá má so stranou~$BC$
spoločnú os (\obr). Ukážeme, že bod~$S_b$ je vnútorným bodom
tetivy~$S_cP$ (a~nie bodom jej predĺženia za krajný bod~$P$). Na to
treba (a~stačí) overiť nerovnosť $|\uhol BCA|<|\uhol BCP|$.
Tú môžeme vďaka zhodnosti súmerne združených
uhlov $BCP$ a~$CBA$ prepísať ako nerovnosť $|\uhol BCA|<|\uhol CBA|$,
ktorá je, ako vieme, dokonca ekvivalentná
s~predpokladom $|AB|<|AC|$ zo zadania úlohy.

\poznamka
Poznatok, že bod~$S_b$ je vnútorným bodom tetivy~$S_cP$, možno tiež dokázať zistením, že bod~$S_b$ má od osi strany~$BC$
menšiu vzdialenosť ako bod~$S_c$. Tieto dve vzdialenosti sú
dĺžkami prvých odvesien dvoch zrejmých pravouhlých trojuholníkov so spoločnou druhou
odvesnou, ktorá má krajný bod v~strede~$S_a$ úsečky~$BC$,
takže vďaka Pytagorovej vete máme vlastne dokázať nerovnosť
$|S_aS_b|<|S_aS_c|$ pre dĺžky oboch prepôn. Tie však sú strednými
priečkami trojuholníka~$ABC$, takže platí $|S_aS_b|=\frac12|AB|$
a~$|S_aS_c|=\frac12|AC|$, a~tak sa opäť dostávame k~podmienke
$|AB|<|AC|$ zo zadania.
\inspinsp{a67.5}{a67.6}%

\ineres
Bod~$A$ zrejme leží vo vonkajšej oblasti kružnice~$k$ opísanej trojuholníku~$BS_cC$.
Má teda ku kružnici~$k$ kladnú mocnosť~$m$ danú vzťahom
$$
m=|AS_c|\cdot |AB|=\frac{|AB|^2}{2}.
$$
Vďaka tomu vieme, že na kružnici~$k$ taktiež leží taký
bod~$Q$ polpriamky~$AC$, ktorý má od
jej počiatku~$A$ vzdialenosť určenú rovnosťou $|AQ|\cdot|AC|=m$
(vo~všeobecnom prípade nie je vylúčené, že $|AQ|\ge|AC|$). Za nášho
predpokladu $|AB|<|AC|$ však platí
$$
|AQ|=\frac{m}{|AC|}=\frac{|AB|^2}{2|AC|}<\frac{|AC|^2}{2|AC|}=
\frac{|AC|}{2}=|AS_b|<|AC|,
$$
takže body $A$, $Q$, $S_b$, $C$ ležia na priamke v~tomto poradí
(\obr). Bod~$S_b$ je tak vnútorným bodom tetivy~$CQ$ kružnice~$k$, čo podľa poznatku (pripomenutého v~úvode predchádzajúceho riešenia)
už znamená, že $|\uhol BS_cC|<|\uhol BS_bC|$.


\ineres
Označme $S_a$ stred úsečky~$BC$ a~$o$ jej os.
Akonáhle si uvedomíme, že vďaka predpokladu $|AB|<|AC|$
má bod~$S_b$ menšiu vzdialenosť od osi~$o$ ako bod~$S_c$,
vyplýva tvrdenie úlohy z~nasledujúceho pozorovania: Pre ľubovoľný bod~$X$
na priamke~$S_bS_c$ (kolmej na os~$o$) klesá veľkosť uhla $BXC$
so vzdialenosťou~$d$ bodu~$X$ od osi~$o$. Namiesto už využitej geometrickej argumentácie
to overíme trigonometrickým výpočtom.

Označme $v$ vzdialenosť priamky~$S_bS_c$ od strany~$BC$, $d$
vzdialenosť bodu~$X$ od osi~$o$ a~$a=|BC|$. Z~kosínusovej vety
pre trojuholník $BCX$ tak máme
$$
a^2=|BX|^2+|CX|^2-2|BX||CX|\cos|\uhel BXC|.
$$
Ak využijeme navyše to, že trojuholník $BCX$ má od vzdialenosti~$d$
nezávislý obsah $S=\frac12av=\frac12|BX||CX|\sin|\uhel BXC|$, dostaneme
po jednoduchej úprave
$$
\align
\cotg|\uhel BXC|=&{|BX|^2+|CX|^2-a^2 \over 4S}=\\
=&{(d+\frac12a)^2+v^2+(d-\frac12a)^2+v^2-a^2 \over 4S}=
{4d^2+4v^2-a^2 \over 4av}.
\endalign
$$
To je zjavne rastúca funkcia parametra~$d$, zatiaľ čo funkcia cotg je
na intervale $(0\st,180\st)$ klesajúca.

\nobreak\medskip\petit\noindent
Za úplné riešenie dajte 6~bodov, z~toho
3~body za zavedenie tetivy~$S_cP$ alebo $CQ$, ďalšie 2~body
za zdôvodnenie, prečo je bod~$S_b$ vnútorným bodom zavedenej tetivy
a~1~bod za dokončenie dôkazu odkazom na kružnicový oblúk (alebo kruhový odsek)
ako množinu bodov potrebnej vlastnosti.
\endgraf
Neúplné riešenie:
Len za zdôvodnenie, že bod~$S_b$ je bližšie k~osi strany~$BC$ ako bod~$S_c$, dajte 3~body.
Stroho zdôvodnené postupy pozostávajúce len z~platných tvrdení
hodnoťte benevolentne.
Naproti tomu za postup využívajúci (všeobecne neplatné) tvrdenie "V~lichobežníku
$BCS_bS_c$ platí $|BS_c|<|CS_b|$, takže $|\uhol BS_cC| < |\uhol BS_bC|$" dajte nanajvýš 1~bod.
\endpetit
\bigbreak
}

{%%%%%   A-S-3
Chceme, aby krížiky prevažovali v~čo najviac riadkoch
a~krúžky v~čo najviac stĺpcoch.

Všetkých políčok v~tabuľke je nepárny počet $(2n+1)^2=4n^2+4n+1$,
krížikov vo vyplnenej tabuľke je o~jeden viac ako krúžkov, takže ich
je $2n^2+2n+1$, zatiaľ čo krúžkov iba $2n^2+2n$. Je tiež jasné,
že tabuľka akokoľvek vyplnená uvedenými počtami krížikov a~krúžkov je výsledkom,
ktorý Pavol môže svojim postupom dosiahnuť.

Najskôr dokážeme, že krížiky nemôžu prevažovať vo všetkých riadkoch. Keby
krížiky prevažovali v~každom riadku, muselo by ich byť aspoň
$(2n+1)(n+1)=2n^2+3n+1$. Krížikov v~tabuľke je však iba
$2n^2+2n+1=2n(2n+1)+1$, môžu teda prevažovať nanajvýš v~$2n$~riadkoch.

Rovnakým argumentom dokážeme, že krúžky môžu prevažovať nanajvýš v~$2n$~stĺpcoch.
Najvyššie dosiahnuteľné skóre je preto $2n+2n=4n$.

V~druhej časti riešenia dokážeme, že skóre~$4n$ sa dá pre každé prirodzené číslo~$n$
dosiahnuť. Stačí rozmiestniť krížiky do ľavých $n+1$ stĺpcov
prvých $n$~riadkov, pravých $n+1$ stĺpcov ďalších $n$~riadkov
a~prostredného políčka spodného riadku (krúžky potom umiestnime do všetkých
ostatných políčok). Situáciu pre $n=3$ znázorňuje \obr. Ľahko overíme, že
krížikov je celkom $(n+1) n + (n+1) n+1=2n^2+2n+1$, teda správny počet
(takže aj na krúžky ostáva správny počet políčok), a~že krížiky
prevažujú vo všetkých riadkoch okrem posledného, zatiaľ čo krúžky prevažujú
vo všetkých stĺpcoch okrem prostredného.
\inspdf{67as3.pdf}%

\nobreak\medskip\petit\noindent
Za úplné riešenie dajte 6~bodov, z~toho
3~body za dôkaz, že skóre nemôže byť vyššie ako~$4n$;
2~body za opis vyplnenia, ktoré pre každé (všeobecné)~$n$ vedie na skóre~$4n$;
1~bod za uvedenie, že opísané vyplnenie obsahuje správny celkový počet
krížikov a~krúžkov.
Neúplné riešenie:
1~bod za správnu odpoveď (bez zdôvodnenia).
\endpetit
\bigbreak
}

{%%%%%   A-II-1
Označme $n=67$ rozmer tabuľky. Keďže číslo~$n$ je
nepárne, je krížikov v~tabuľke celkom $k=\frac12(n^2+1)$. Príspevok riadku
obsahujúceho $a$~krížikov a~$n-a$ krúžkov do celkového skóre je
$a^2-(n-a)^2= 2na - n^2$.

Keďže súčet všetkých hodnôt~$a$ pre jednotlivé riadky je rovný vyššie
určenému číslu~$k$, sčítaním cez všetkých $n$~riadkov získavame,
že ich celkový príspevok je
$$
2nk-n\cdot n^2= \frac{n^2+1}2\cdot 2n -n^3=n.
$$
To isté platí aj pre stĺpce, takže výsledné skóre je vždy rovné
$n+n=2n=134$.

\ineres
Uvažujme tabuľku $n\times n$
vyplnenú úplne ľubovoľne krúžkami a~krížikmi. Dokážeme, že prepísaním
ľubovoľného krúžku na krížik sa skóre zvýši o~$4n$, a~keďže pre
tabuľku vyplnenú samými krúžkami je skóre rovné~${-2}n^3$ a~Pavlom
vyplnená tabuľka obsahuje $\frac12(n^2+1)$ krížikov, bude výsledné skóre vždy
rovné ${-2}n^3+4n\cdot \frac12(n^2+1)=2n$.

Označme prepisované políčko~$P$ a~predpokladajme, že v~stĺpci a~riadku
obsahujúcom~$P$ sa nachádza $s$, resp. $r$~krížikov. Príspevok tohto
riadku a~stĺpca sa tak zmení z~pôvodného
$$
A=r^2-(n-r)^2+s^2-(n-s)^2=2n(r+s)-2n^2
$$
na
$$
B=(r+1)^2-(n-r-1)^2+(s+1)^2-(n-s-1)^2=2n(r+1+s+1)-2n^2,
$$
zatiaľ čo príspevok ostatných riadkov a~stĺpcov sa nezmení. Keďže $B-A=4n$,
sme hotoví.

\ineres
Ak vyplní Pavol tabuľku $67\times 67$ tak, že sa krížiky
nachádzajú práve vo všetkých políčkach prvých 33~riadkov
a~v~prvých 34~políčkach 34.~riadku, ľahko vyjadríme
$$
X= 33\cdot 67^2+34^2 +34\cdot 34^2+33\cdot 33^2
$$
a
$$
O= 33^2+33\cdot 67^2+ 34\cdot 33^2+33\cdot 34^2.
$$
Pre takto vyplnenú tabuľku nám vyjde $X-O= 2\cdot (34^2-33^2)=134$.

Teraz dokážeme, že hodnota skóre nezávisí od toho, v~ktorých
% $k=\frac12(n^2+1)$
políčkach krížiky sú. Na to stačí dokázať, že hodnota
skóre sa nezmení, keď prehodíme krížik s~krúžkom ležiacim v~tom istom
riadku alebo v~tom istom stĺpci. Opakovaným prevádzaním takých operácií možno
totiž z~každej Pavlom vyplnenej tabuľky získať
tabuľku vyplnenú ako vo vyššie opísanom prípade.

Bez ujmy na všeobecnosti predpokladajme, že prehadzované znaky ležia v~tom istom
riadku, a~označme $s_x$, $s_o$ stĺpce, v~ktorých leží prehadzovaný krížik, resp.
krúžok. Napokon označme $a$, $b$ počet (ostatných) krížikov
v~stĺpcoch $s_x$, $s_o$. Príspevok stĺpca~$s_x$ do výsledného skóre
sa zmení z~pôvodného $A_1=(a+1)^2-(n-a-1)^2$ na nové $A_2=a^2-(n-a)^2$,
príspevok stĺpca~$s_o$ sa zmení z~pôvodného $B_1=b^2-(n-b)^2$ na nové
$B_2=(b+1)^2-(n-b-1)^2$ a~príspevky ostatných stĺpcov a~riadkov sa
nezmenia. Výsledné skóre sa tak zmení o~hodnotu
$$
\postdisplaypenalty10000
A_2 - A_1+B_2-B_1 =-2a-1-2(n-a)+1 + 2b+1 +2(n-b)-1=-2n+2n=0.
$$
Tým sme hotoví.

\nobreak\medskip\petit\noindent
Za úplné riešenie dajte 6~bodov.
Neúplné riešenie: Po 1~bode dajte za tvrdenie, že skóre bude vždy 134
(bez zdôvodnenia), a~za opísaný výpočet skóre pre nejaké konkrétne
vyplnenie tabuľky. Pri treťom riešení nestrhávajte body za absenciu
formálneho dôkazu, že z~jednej vyplnenej tabuľky možno postupným
prehadzovaním krížikov a~krúžkov dostať každú inú vyplnenú tabuľku.

\endpetit
\bigbreak}

{%%%%%   A-II-2
Označme $l$ polkružnicu, ktorá je obrazom polkružnice~$k$
v~osovej súmernosti podľa priemeru~$PQ$,
a~zostrojme bod~$C'$ ako obraz bodu~$C$ v~tejto osovej súmernosti.
Bod~$C'$ zrejme leží na~$l$. Z~vlastností osovej
súmernosti ďalej vyplýva $|\uhel QAC'|=|\uhel QAC|$, a~keďže
$|\uhel QAC|=|\uhel PAB|$, ležia body $B$, $A$,
$C'$ na jednej priamke (\obr). Zároveň je trojuholník $C'CA$ rovnoramenný, takže
pre jeho vonkajší uhol $BAC$ platí
$$
|\uhel BAC|=|\uhel AC'C|+|\uhel ACC'|=2|\uhel BC'C|.
% |\uhel BAC|=180^\circ-|\uhel CAC'|=|\uhel AC'C|+|\uhel C'CA|=2\cdot|\uhel BC'C|.
$$

Tetiva $BC$ kružnice~$k\cup l$ má pevnú dĺžku, preto má aj príslušný obvodový
uhol~$BC'C$ konštantnú veľkosť. Hodnota $|\uhel BAC|=2 |\uhel BC'C|$
tak nezávisí od polohy tetivy~$BC$.
\inspinsp{a67.7}{a67.8}%

\ineres
Označme $O$ stred priemeru~$PQ$ polkružnice~$k$.
Dokážeme, že bod~$O$ vždy leží na kružnici~$s$ opísanej trojuholníku~$ABC$ (\obr).
A~keďže celý priemer~$PQ$ leží v~polrovine určenej tetivou~$BC$,
vyplynie z~rovnosti obvodových uhlov nad tetivou~$BC$ rovnosť
$|\uhel BAC|=|\uhel BOC|$, čo je hodnota, ktorá od polohy tetivy~$BC$
nezávisí. Tým bude tvrdenie úlohy dokázané.

Uvedomme si, že bod~$O$ leží na osi~$o$ úsečky~$BC$, na ktorej leží
aj stred~$S$ toho oblúka~$BC$ kružnice~$s$, ktorý neprechádza bodom~$A$.
Priamka~$OS$ tak obsahuje priemer kružnice~$s$ s~jedným
krajným bodom~$S$. Podľa Tálesovej vety bude jeho druhým krajným
bodom práve bod~$O$ priamky~$PQ$ (a~náš dôkaz tak bude ukončený),
keď overíme, že $SA\perp PQ$.

Podľa zadania sú zhodné uhly $BAP$ a~$CAQ$, vďaka zhodnosti tetív
$SB$ a~$SC$ kružnice~$s$ sú však zhodné aj uhly $BAS$ a~$CAS$,
takže sú zhodné aj uhly $SAP$ a~$SAQ$, sú to teda naozaj dva
pravé uhly, ako sme sľúbili overiť.


\poznamka
Dodajme, že bod~$A$ na $PQ$ je rovnosťou zo zadania jednoznačne určený.
Stačí preto ukázať, že bod~$A$ možno nájsť ako priesečník kružnice~$s$
opísanej (rovnoramennému) trojuholníku~$OBC$ s~priamkou~$PQ$: Ak je $BC$ rovnobežné s~$PQ$,
je zrejme $A=O$; v~opačnom prípade existuje ďalší priesečník
$X\ne O$ kružnice~$s$ s~priamkou~$PQ$ a~pre ten platí,
že v~ňom vztýčená kolmica na~$PQ$ prechádza stredom~$S$ oblúka~$BC$,
je teda osou uhla~$BXC$, takže naozaj $X=A$.


\nobreak\medskip\petit\noindent
Za úplné riešenie dajte 6~bodov, z~toho pri prvom postupe dajte
2~body za zdôvodnenie, že bod~$C'$ leží na priamke~$AB$ (alebo $B'$ na $AC$),
2~body za vyjadrenie $|\uhel BAC|$ pomocou $|\uhel BC'C|$ alebo $|\uhel BOC|$,
2~body za dokončenie dôkazu.
Pri druhom postupe dajte
4~body za dôkaz toho, že bod~$O$ vždy leží na
kružnici opísanej trojuholníku $ABC$ či bod~$A$ na
kružnici opísanej trojuholníku~$OBC$
a~2~body za dokončenie dôkazu.

\endpetit
\bigbreak}

{%%%%%   A-II-3
Uvažujme lineárne funkcie $f(x)=ax+b$ a~$g(x)=bx+a$. Keďže čísla $a$, $b$ sú
kladné a~rôzne, sú ich grafy dve rôzne priamky s~kladnou smernicou a~obe funkcie
$f$ a~$g$ sú rastúce. A~keďže $f(1)=g(1)=a+b$, je bod $P[1,a+b]$
priesečníkom oboch priamok (\obr).
\insp{a67.9}%

Bez ujmy na všeobecnosti predpokladajme, že $b>a$, takže
priamka určená funkciou~$g$ je "strmšia" ako priamka určená funkciou~$f$.
Inými slovami, pre $x<1$ je $f(x)>g(x)$, zatiaľ čo pre $x>1$ je $f(x)<g(x)$.
To inak vyplýva aj z~algebraického vyjadrenia
$$
f(x)-g(x)=(ax+b)-(bx+a)=(b-a)(1-x).
$$

Označme $t=\lfloor a+b \rfloor$ a~nájdime čísla $x_1\le 1< x_2$
také, že $g(x_1)=t$ a~$g(x_2)=t+1$ (\tj. $x_1=(t-a)/b$
a~$x_2=(t+1-a)/b$). Tvrdíme, že interval $\langle x_1,x_2)$,
ktorý obsahuje číslo~1, má všetky požadované vlastnosti.

Najskôr ukážeme, že pre každé $x\in\langle x_1,x_2)$ platí
$$
t\le f(x)<t+1\quad\text{a}\quad t\le g(x)<t+1,
\tag1
$$
čo povedie k~záveru, že $x$ je riešením zadanej rovnice,
pretože obe jej strany potom budú rovné číslu~$t$.

Naozaj, pre každé také $x$
platí buď $x_1\le x\le 1$, alebo $1<x<x_2$. Vďaka nerovnostiam
medzi hodnotami $f$ a~$g$ v~prvom prípade platí
$$
t=g(x_1)\le g(x)\le f(x)\le f(1)=a+b<t+1,
$$
v~druhom prípade potom je
$$
t\le a+b=f(1)<f(x)<g(x)<g(x_2)=t+1,
$$
takže (1) platí v~oboch prípadoch. Všetky čísla
z~intervalu $\langle x_1,x_2)$ sú tak riešením zadanej rovnice.
Navyše z~rovností
$$
1=t+1-t = bx_2+a-(bx_1+a)=b(x_2-x_1)
$$
vyplýva
$$
\postdisplaypenalty 10000
x_2-x_1=\frac1b= \frac1{\max\{a,b\}},
$$
teda interval $\langle x_1,x_2)$ má aj požadovanú dĺžku.

Z~nášho výsledku vyplýva, že podmienkam úlohy vyhovuje aj interval
$(x_1,x_2)$.

\ineres
Pre ľubovoľne zvolené celé číslo $t$ má rovnica $\lfloor ax+b\rfloor=t$
za riešenia práve tie~$x$, pre ktoré platí $t\le ax+b<t+1$. Také
$x$ vďaka podmienke $a>0$ tvoria interval
$$
I_1=\Big\langle\frac{t-b}{a},\frac{t-b+1}{a}\Big),
$$
ktorý má dĺžku $1/a$. Podobne všetky riešenia rovnice
$\lfloor bx+a\rfloor=t$ tvoria interval
$$
I_2=\Big\langle\frac{t-a}{b},\frac{t-a+1}{b}\Big)
$$
dĺžky $1/b$. V~prípade $a<b$ tak stačí dokázať
existenciu celého~$t$, pre ktoré má interval $I_1\cap I_2$ rovnakú dĺžku
$1/b$ ako kratší interval~$I_2$. To nastane práve
vtedy, keď bude platiť $I_2\subset I_1$, čo možno vyjadriť dvojicou
nerovností medzi krajnými bodmi oboch intervalov v~tvare
$$
\frac{t-b}{a}\le\frac{t-a}{b}\quad\text{a}\quad
\frac{t-a+1}{b}\le\frac{t-b+1}{a}.
$$

Ľahko sa presvedčíme, že za nášho predpokladu $0<a<b$
obe nerovnosti platia práve vtedy, keď
$$
a+b-1\le t\le a+b.
$$

Vždy teda vyhovuje $t=\lfloor a+b\rfloor$; ak je číslo $a+b$ celé,
vyhovuje aj $t=a+b-1$. Vzhľadom na symetriu platí predchádzajúca veta
aj v~opačnej situácii, keď $a>b$.



\nobreak\medskip\petit\noindent
Za úplné riešenie dajte 6~bodov, v~prípade drobných nedostatkov strhnite 1~bod.
Pri hodnotení čiastočných riešení dajte nasledujúce body (čiastkové body podľa
prvého a~druhého riešenia sa nesčítajú!).

Pri postupe podľa prvého riešenia 1~bod za konštatovanie, že vyhovuje $x=1$, čiže
že sa uvažované priamky pretínajú v~bode $[1,a+b]$, 1~bod za rozhodnutie,
že budeme hľadať iba tie~$x$, pre ktoré sa obe strany rovnice rovnajú hodnote
$t=\lfloor a+b\rfloor$, 1~bod za výpočet krajných bodov $x_1$, $x_2$ pri strmšej
funkcii pomocou $a$, $b$ a~$t$ (postačuje pre jeden z~prípadov $a<b$, $a>b$),
ďalší 1~bod za dôkaz, že interval s~týmito krajnými bodmi má požadovanú dĺžku $1/\max(a,b)$
a~2~body za algebraické alebo grafické overenie, že na tomto (polouzavretom) intervale
sa obe strany rovnice rovnajú.
V~prípade postupu podľa druhého riešenia dajte 2~body za nájdenie intervalov,
v~ktorých sú funkcie $\lfloor ax+b\rfloor$ a~$\lfloor bx+a\rfloor$ rovné
konštante~$t$, 1~bod za dôkaz, že kratšie z~nich majú požadovanú dĺžku $1/\max(a,b)$,
1~bod za správny zápis inklúzie a~2~body za dopočítanie $t=\lfloor a+b\rfloor$.

Ak riešiteľ nebude postupovať v~dostatočnej všeobecnosti vzhľadom na reálne
parametre $a$ a~$b$, dajte nanajvýš 1~bod za (úplný) dôkaz pre jednu
konkrétnu dvojicu $(a,b)$ a~nanajvýš 3~body za (úplný) dôkaz pre nekonečne veľa
dvojíc $(a,b)$ (napríklad pre dvojice $a=1$, $b=m>1$, pričom $m$ je ľubovoľné prirodzené číslo).
\endpetit
\bigbreak
}

{%%%%%   A-II-4
Také čísla neexistujú.
Predpokladajme naopak, že $n$ a~$k$ sú kladné celé čísla také, že
$$
\frac{n}{11^k-n}=a^2
$$
pre nejaké kladné celé číslo~$a$. Po jednoduchej úprave dostávame
$$
n(a^2+1)=a^2\cdot 11^k.
$$

Keďže čísla $a^2$ a~$a^2+1$ sú pre každé $a\ge 1$ nesúdeliteľné,
musí byť $a^2+1$ deliteľom čísla~$11^k$, a~to deliteľom netriviálnym,
pretože $a^2+1>1$. To znamená, že $a^2+1=11^t$ pre nejaké $1\le t\le k$,
čiže číslo~$a^2$ musí po delení~11 dávať zvyšok~10.
Ľahko však overíme, že žiadna druhá mocnina celého čísla zvyšok~10 po
delení~11 nedáva:
to samozrejme stačí overiť iba pre čísla $0, 1, \dots,10$.
Ich druhé mocniny dávajú postupne zvyšky $0,1,4,9,5,3,3,5,9,4,1$,
čím sme dospeli k sľúbenému sporu.

Žiadne také čísla $n$ a~$k$ neexistujú.

\poznamka
Keďže $(11-r)^2=11\cdot(11-2r)+r^2$,
dávajú druhé mocniny čísel $r$ a~${11-r}$ po delení~11 rovnaké zvyšky,
takže posledné tvrdenie stačilo overiť iba pre čísla $0,1,2,3,4,5$.



\nobreak\medskip\petit\noindent
Za úplné riešenie dajte 6~bodov, z~toho
3~body za dôkaz, že stačí riešiť rovnicu $a^2+1=11^t$ pre $t\ge 1$;
2~body za dôkaz, že žiadna druhá mocnina celého čísla nedáva zvyšok~10
po delení~11;
1~bod za dokončenie dôkazu.
Len za uvedenie správnej odpovedi (bez akéhokoľvek zdôvodnenia) neudeľujte žiadny bod.

\endpetit
}

{%%%%%   A-III-1
Uvažujme ľubovoľnú sedmicu ľudí zo skupiny, ktorá je 6-dobrá, a~označme ich
$A$ až $G$. Stačí dokázať, že túto sedmicu možno požadovaným spôsobom
rozsadiť okolo okrúhleho stola. Berme do úvahy len vzťahy medzi $A,\dots,G$.
Najskôr dokážeme, že každý z~nich má (medzi nimi) aspoň troch priateľov.
Bez ujmy na všeobecnosti to ukážeme pre~$G$.

Podľa predpokladu možno okolo okrúhleho stola rozsadiť šesticu
$B,\dots,G$, takže $G$ má určite aspoň dvoch priateľov. Bez ujmy na
všeobecnosti je jedným z~nich $F$. Podľa predpokladu možno ale okolo stola
rozsadiť aj šesticu $A,\dots,E,G$ (bez~$F$), takže aj v~nej má $G$ aspoň
dvoch priateľov, teda spolu s~$F$ má $G$ aspoň troch priateľov.

To, že každý člen sedmice má aspoň troch priateľov, ale
znamená, že aspoň jeden člen má najmenej štyroch priateľov, pretože
keby každý zo siedmich členov mal práve troch priateľov, existovalo by
v~sedmici spolu presne $\frac12\cdot 7\cdot 3$ spriatelených
dvojíc, čo zrejme nie je možné.

Teraz (opäť bez ujmy na všeobecnosti) predpokladajme, že člen s~aspoň štyrmi
priateľmi je~$G$. Podľa predpokladu možno okolo okrúhleho stola rozsadiť
šesticu $A,\dots,F$. V~takom rozsadení musia niektorí dvaja zo štyroch
priateľov $G$ sedieť vedľa seba. Člena~$G$ potom môžeme posadiť medzi nich
a~sme hotoví.

\poznamka
Tvrdenie, že ak je spoločnosť $k$-dobrá, tak je
aj~$(k+1)$-dobrá, platí práve pre
$k\in\{3,4,5,6,7,8,10,11,13,16\}$.\footnote{Pozri Wikipédia:
Hypohamiltonian graph.} Kontrapríkladom pre $k=9$ je napríklad takzvaný
{\it Petersenov graf\/} (\obr).
\insp{a67.10}%
}

{%%%%%   A-III-2
Pri voľbe $x=y=z=t>0$ sú spomenuté čísla dĺžkami strán
rovnostranného trojuholníka a~$xy+yz+zx = 3t^2$, takže výraz $xy+yz+zx$
môže nadobúdať všetky kladné hodnoty. Podobne pre $x=y=t>0$ a~$z=\m2t$
majú tri zlomky postupne hodnoty $\frac13t^{-2}$, $\frac13t^{-2}$, $\frac16t^{-2}$,
čo sú kladné čísla zodpovedajúce dĺžkam strán rovnoramenného
trojuholníka (platí $\frac 16 < \frac13+\frac13$). Pritom $xy+yz+zx=
\m3t^2$, takže výraz $xy+yz+zx$ môže nadobúdať aj všetky záporné hodnoty.

Ďalej dokážeme, že nulu výraz $xy+yz+zx$ nadobúdať nemôže. Predpokladajme opak.
Čísla $x$, $y$, $z$ sú nutne po dvoch rôzne: ak by
platilo napríklad $x=y$, bol by menovateľ prvého zlomku rovný
$|x^2+2yz|=|xy +(yz+xz)|=0$, čo nie je možné.

Skúmajme zlomky bez absolútnych hodnôt. Odčítaním $xy+yz+zx=0$ od
každého menovateľa s~následnou úpravou na súčin dostaneme
$$
\align
\frac{1}{x^2+2yz}&+\frac{1}{y^2+2zx}+\frac{1}{z^2+2xy} = \\
=&\frac{1}{(x-y)(x-z)}+\frac{1}{(y-z)(y-x)}+\frac{1}{(z-x)(z-y)} = \\
=&\frac{(z-y)+(x-z)+(y-x)}{(x-y)(y-z)(z-x)} = 0.
% =&\frac{z-y+x-z+y-x}{(x-y)(y-z)(z-x)} = 0.
\endalign
$$
Z~toho ale vyplýva, že v~pôvodnej trojici zlomkov (s~absolútnymi
hodnotami) bola hodnota jedného z~nich súčtom hodnôt zvyšných dvoch. To
je v~spore s~predpokladom, že tieto hodnoty sú dĺžkami strán
nedegenerovaného trojuholníka (nemôžu totiž spĺňať trojuholníkovú nerovnosť!).

\odpoved
Možnými hodnotami výrazu sú všetky reálne čísla okrem~0.
}

{%%%%%   A-III-3
Označme $\alpha$ veľkosť skúmaného uhla $BAC$ a~$O$
stred kružnice opísanej trojuholníku~$AEF$.
Keďže uhly $BAD$ a~$CAD$ sú ostré, ležia oba body $E$ a~$F$
v~polrovine~$BCA$, a~preto
pre zodpovedajúce stredové a~obvodové uhly prislúchajúce tetivám $BD$ a~$CD$
kružníc opísaných trojuholníkom $ABD$ a~$ACD$ (\obr) platí
$$
|\uhol BED|=2 |\uhol BAD|=\alpha=2 |\uhol DAC|=|\uhol DFC|.
$$
Rovnoramenné trojuholníky $BED$ a~$DFC$ sú teda podobné ($sus$), takže
$|\uhol EDB|=|\uhol FDC|$. A~keďže ich základne ležia na jednej priamke,
je dokonca $|\uhol EDF|=\alpha$.
Priamka~$BC$ je preto osou vonkajšieho uhla pri vrchole~$D$
v~trojuholníku $EDF$. Tá, ako je známe, prechádza stredom oblúka~$EDF$ kružnice opísanej
trojuholníku $EDF$, teda bodom, ktorý leží na osi jej tetivy~$EF$. Tým bodom
je však bod~$O$, ktorý ako stred kružnice opísanej trojuholníku~$AEF$ leží
na osi strany~$EF$ (a~podľa predpokladu aj na $BC$).
Špeciálne tak platí $|\uhol FOE|=|\uhol FDE|=\alpha$.
\insp{a67.11}%

Keďže $AEDF$ je deltoid, je aj $|\uhol EAF|=\alpha$.
Zo súmernosti je zrejmé, že priamka~$EF$ oddeľuje body $A$ a~$O$
(ako už vieme, bod~$O$ leží na oblúku $EDF$, ktorý je súmerne združený
s~oblúkom~$EAF$).
Podľa vety
o~obvodovom a~stredovom uhle je preto veľkosť nekonvexného uhla $EOF$
rovná dvojnásobku veľkosti konvexného uhla $EAF$. Tým pádom
$360^\circ-\alpha =2 |\uhol EAF|=2 \alpha$, z~čoho okamžite
vyplýva $\alpha=120^\circ$.

Naopak sa ľahko presvedčíme, že aspoň jeden taký trojuholník existuje (\obr):
napríklad pre $|AB|=|AC|$ a~$\alpha=120^\circ$ sú $E$, $F$ stredmi strán $AB$, $AC$,
trojuholníky $DAE$ a~$DAF$ sú rovnostranné a~stred kružnice opísanej
trojuholníku $AEF$ naozaj leží na strane~$BC$ (je totiž totožný so
stredom~$D$ strany~$BC$).
\insp{a67.12}%

\odpoved
Jediná možná veľkosť uhla $BAC$ je $120^\circ$.
}

{%%%%%   A-III-4
Označme $z=a+b-c$, $x=b+c-a$, $y=c+a-b$ kladné menovatele
jednotlivých zlomkov. Potom $a=\frac12(y+z)$, $b=\frac12(z+x)$, $c=\frac12(x+y)$
a
$$
a^2+b^2-c^2=\tfrac14\big((y+z)^2+(z+x)^2-(x+y)^2\big)=\tfrac12 \big(z(z+x+y)-xy\big),
$$
takže podľa predpokladu musí platiť $z\mid xy$ a~podobne $y\mid xz$ a~$x\mid yz$.

Teraz stačí dokázať, že pre každé nepárne prvočíslo~$p$ je exponent jeho najvyššej mocniny,
ktorá ešte delí súčin~$xyz$, párny. Ak bude aj exponent najvyššej mocniny
dvojky, ktorá ešte delí súčin~$xyz$, párny,
bude $xyz$ druhou mocninou celého čísla. V~opačnom prípade bude druhou
mocninou jeho dvojnásobok~$2xyz$.

Pre nepárne prvočíslo~$p$ označme najvyššie mocniny, v~ktorých $p$ delí čísla
$x$, $y$, $z$, ako $p^\alpha$, $p^\beta$, $p^\gamma$. Bez ujmy na
všeobecnosti môžeme predpokladať, že $\min\{\alpha,\beta,\gamma\}=\gamma$.
Keby bolo $\gamma>0$, delilo by $p$ každé z~čísel $x$, $y$, $z$, a~teda
aj každé z~čísel $a$, $b$, $c$, čo je v~spore s~ich predpokladanou
nesúdeliteľnosťou. Tým pádom $\gamma=0$.

Z~deliteľnosti $x\mid yz$ potom vyplýva $\alpha\le \beta$ a~podobne z~$y\mid xz$ vyplýva
$\beta\le \alpha$, teda $\beta=\alpha$, takže v~súčine $xyz$ sa $p$
naozaj vyskytuje v~párnej mocnine $\alpha+\beta+\gamma=2\alpha$.

Tým je tvrdenie úlohy dokázané.}

{%%%%%   A-III-5
Označme $K$ stred kružnice vpísanej trojuholníku $ABD$.
Keďže zrejme platí $IK \parallel AB$, stačí dokázať $JK \parallel
AB$. Označme $|\uhol ABD|=|\uhol ACD|=\varphi$. Potom $|\uhol
AKD|=90^\circ+\frac12\varphi$ a~$|\uhol DJA|=90^\circ-\frac12\varphi$,
takže štvoruholník $AKDJ$ je tetivový (\obr).
\insp{a67.13}%

Keďže priamky $AK$, $DJ$ sú osi striedavých uhlov, sú rovnobežné,
čo spolu s~objavenou kružnicou dáva $|\uhol AKJ|=|\uhol ADJ|=|\uhol DAK|=|\uhol KAB|$.
Priamky $AB$ a~$JK$ sú teda rovnobežné.

\poznamka
Stred~$M$ oblúka~$DA$ spoločnej kružnice opísanej trojuholníkom $ACD$ a~$ABD$ má,
ako je všeobecne známe, od vrcholov $A$ a~$D$ rovnakú vzdialenosť ako od stredov $L$ a~$K$
kružníc postupne týmto trojuholníkom vpísaným. Keďže oba uhly $LAJ$ a~$LDJ$ sú navyše
pravé, ležia body $A$, $D$, $L$ a~$K$ na kružnici s~priemerom~$LJ$ (\obr).
Trojuholník $JKM$ je teda rovnoramenný a~platí
$|\uh MJK|=\frac12|\uh CMK|=\frac12|\uh CMB|=\frac12|\uh CAB|=\frac12|\uh ACD|=|\uh JCD|$, čo
dáva potrebnú rovnobežnosť ${JK\parallel CD\parallel AB}$.
\insp{a67.14}%

\ineriesenie
Označme $J'$ pätu výšky z~bodu~$J$ na priamku~$CD$
a~$I'$ pätu výšky z~bodu~$I$ na priamku~$AB$ (\obr).
\insp{a67.15}%

Stačí dokázať $|II'|+|JJ'|=v$, pričom $v$ je výška lichobežníka.
Označme $|AB|=a$, $|BC|=|AD|=b$, $|CD|=c$, $|AC|=|BD|=u$.
Podľa známych vzorcov potom platí
$$
|JJ'|=\frac{2S_{ACD}}{|AC|+|CD|-|AD|}=\frac{c\cdot v}{u+c-b},\quad
|II'|=\frac{2S_{ABC}}{|AB|+|BC|+|AC|}=\frac{a\cdot v}{a+b+u}.
$$

Po dosadení a~roznásobení ostane dokázať rovnosť $u^2=ac+b^2$, ktorú dostaneme
spojením dvoch pytagorejských rovností
$$
u^2=\Big(\frac{a+c}{2}\Big)^{\!2}+v^2,\quad
b^2=\Big(\frac{a-c}{2}\Big)^{\!2}+v^2
$$
alebo z~Ptolemaiovej vety pre (tetivový) rovnoramenný lichobežník $ABCD$.}

{%%%%%   A-III-6
Dokážeme, že hľadané prirodzené číslo je $n=29$. Najskôr ukážeme, že nech
ofarbíme prvých 29~prirodzených čísel akokoľvek, vždy medzi nimi budú
nejaké dve čísla rovnakej farby, ktoré sa budú líšiť o~druhú mocninu prirodzeného čísla.
A~potom uvedieme príklad vhodného ofarbenia 28~čísel, ktoré ukáže,
že požadovanú vlastnosť nemá žiadne $n\le28$.

Pripusťme naopak, že prvých 29~prirodzených čísel možno ofarbiť
tromi farbami $A$, $B$, $C$ tak,
že rozdiel žiadnych dvoch čísel tej istej farby nie je druhá mocnina,
a~označme~$f(i)$ farbu čísla~$i$ pre
$i\in\{1, 2, \dots, 29\}$.

Keďže 9, 16 a~25 sú druhé mocniny, musia mať každé dve z~čísel
$1$, $10$, $26$ rôznu farbu. To isté platí aj pre každé dve z~čísel $1$, $17$, $26$,
teda čísla 10 a~17 musia mať rovnakú farbu.
Rovnakú úvahu uplatníme aj pre ďalšie trojice tvaru $a,a+9,a+25$ a~$a,a+16,a+25$,
$a\in\{2, 3, 4\}$, teda rovnakú farbu musia mať aj čísla 11 a~18, 12 a~19,
13 a~20, čiže $f(11)=f(18)$, $f(12)=f(19)$ a~$f(13)=f(20)$.

Označme~$A$ farbu čísel 10 a~17, \tj.~$f(10)=f(17)=A$.
Keďže čísla 10 a~11 sa líšia
o~$1=1^2$, musia mať dvojice 11, 18 inú farbu ako~$A$, označme
ich farbu ako~$B$.
Keďže
$19=18+1^2=10+3^2$, musí byť $f(19)$ rôzne od $f(18)=B$ aj~$f(10)=A$,
takže $f(12)=f(19)=C$. Z~rovností $20=19+1^2=11+3^2$ však podobne vyplýva, že
$f(20)\ne f(19)=C$ a~$f(20)\ne f(11)=B$, musí preto byť
$f(13)=f(20)=A$. Odvodili sme $f(13)=A=f(17)$, čo je želaný spor,
pretože $17-13=4=2^2$. Pre prvých 29~čísel teda také ofarbenie neexistuje.

Kontrapríklad pre $n=28$ uvádzame v~nasledujúcej tabuľke.
\inspdftab{67Aiii6b.pdf}%

Ľahko overíme, že každé dve čísla, ktoré sa líšia o~1, 4, 9, 16 či~25
sú ofarbené rôznou farbou.
}

{%%%%%   B-S-1
Označme $P$ hľadaný mnohočlen. Výsledky jeho delenia oboma danými
mnohočlenmi znamenajú, že pre vhodné mnohočleny $A$ a~$B$ platia
rovnosti
$$
P(x)=A(x)(x^2-4)+1\quad\text{a}\quad
P(x)=B(x)(x^3-8x+8)+1.
$$
Pritom $A$ a~$B$ nie sú nulové mnohočleny, lebo $P$ musí mať
podľa zadania kladný stupeň, ktorý je tak aspoň~3.

Porovnaním pravých strán dostaneme rovnosť mnohočlenov
$$
A(x)(x^2-4)=B(x)(x^3-8x+8).
$$
Medzi korene mnohočlena na ľavej strane určite patria čísla $x=2$ a~$x={-2}$,
z~ktorých iba prvé je koreňom kubického činiteľa $x^3-8x+8$ na
pravej strane (to ľahko zistíme dosadením).
Preto $x={-2}$ musí byť koreňom mnohočlena~$B$, ktorý tak má
stupeň aspoň jedna, teda hľadaný mnohočlen~$P$ má stupeň aspoň~4.
Navyše $P$ bude mať stupeň štyri práve vtedy, keď
zodpovedajúce $B$ bude tvaru $B(x)=a(x+2)$ pre vhodné číslo
$a\ne0$. (Zodpovedajúce $A(x)=a(x^3-8x+8)/(x-2)$ nie je
nutné počítať, napriek tomu ho uveďme: $A(x)=a(x^2+2x-4)$.)

Všetky mnohočleny~$P$ vyhovujúce podmienkam úlohy
tak sú tvaru
$$
P(x)=\underbrace{a(x+2)}_{B(x)}(x^3-8x+8)+1=a(x^4+2x^3-8x^2-8x+16)+1.
$$
(Skúška pri uvedenom postupe nie je nutná, pre úplné riešenie stačilo
uviesť jeden príklad, napríklad voľbou $a=1$.)



\nobreak\medskip\petit\noindent
Za úplné riešenie dajte 6~bodov.
Za vytvorenie potrebnej rovnosti
dajte 2~body. Za dôkaz, že
hľadaný mnohočlen má stupeň aspoň~4, dajte 3~body. Za dokončenie riešenia
dajte 1~bod.
\endpetit
\bigbreak
}

{%%%%%   B-S-2
Z~rovnosti prislúchajúcich úsekov dotyčníc ku kružnici~$k$ vyplýva $|AM|=|AK|$
a~$|BL|=|BK|$, preto $|AM|=|AK|=|AE|$ a~$|BL|=|BK|=|BF|$ (\obr).
To ale znamená, že bod~$A$ je stredom kružnice opísanej trojuholníku $EKM$
a~podobne bod~$B$ stredom kružnice opísanej trojuholníku $KFL$.
\insp{b67.4}%

Podľa Tálesovej vety z~toho vyplýva,
že trojuholníky $EKM$ a~$KFL$ sú oba pravouhlé s~pravými uhlami pri
vrcholoch $M$ a~$L$. Preto
$$
|\uh KMU|+|\uh KLU|=90^{\circ}+90^{\circ}=180^{\circ},
$$
čo znamená, že $MKLU$ je tetivový štvoruholník.
Bod~$U$ teda leží na kružnici~$k$ opísanej trojuholníku $MKL$.

Navyše je úsečka~$UK$ priemerom kružnice~$k$, a~keďže strana~$AB$ sa jej
v~bode~$K$ dotýka, je priemer~$UK$ kolmý na $AB$.



\nobreak\medskip\petit\noindent
Za úplné riešenie dajte 6~bodov.
Za dôkaz, že uhly $EKM$ a~$KFL$ sú pravé, dajte 3~body a~ďalší bod
za odvodenie,
že bod~$U$ leží na kružnici~$k$. Za dôkaz, že priamky
$UK$ a~$AB$ sú navzájom kolmé, dajte 2~body.
\endpetit
\bigbreak
}

{%%%%%   B-S-3
Pre každé prirodzené číslo~$n$ platí $n^3-n =(n-1)n(n+1)$.
Keďže súčin troch po sebe idúcich prirodzených čísel je
deliteľný šiestimi, je číslo $n^3-n$ deliteľné šiestimi pre každé prirodzené
číslo~$n$.

Ak je číslo 2\,018 vyjadrené ako súčet niekoľkých prirodzených čísel,
$$
2\,018 = x_1 + \dots + x_k,
$$
môžeme pre zodpovedajúci súčet~$T$ tretích mocnín písať
$$
\align
T =& x_1^3+ \dots + x_k^3=\\
=&(x_1^3 - x_1) + \dots + (x_k^3 - x_k)+(x_1+ \dots + x_k)=\\
=&(x_1^3 - x_1) + \dots + (x_k^3 - x_k)+2\,018=\\
=&6t+6\cdot336+2,
\endalign
$$
takže číslo~$T$ dáva bez ohľadu na spôsob rozkladu čísla~2\,018
po delení šiestimi vždy zvyšok~2.

\poznamka
Riešenie možno podať stručnejšie, ak ovládame základy číselných kongruencií.
Vieme potom, že tvrdenie o~rovnakom zvyšku čísel $n^3$ a~$n$ po delení šiestimi
možno dokázať pre všeobecné~$n$ tak, že ho numericky overíme iba pre
$n\in\{0, 1, 2, 3, 4, 5\}$. Potom už je jasné, že aj čísla
$$
2\,018 = x_1 + \dots + x_k\quad\hbox{a}\quad x_1^3+ \dots + x_k^3
$$
dávajú po delení šiestimi rovnaký zvyšok. A~číslo 2\,018
dáva po delení šiestimi zvyšok~2.


\nobreak\medskip\petit\noindent
Za úplné riešenie dajte 6~bodov.
Za akékoľvek zdôvodnenie, že prirodzené číslo a~jeho tretia mocnina dávajú po delení
šiestimi rovnaký zvyšok, dajte 2~body. Za zdôvodnenie, že čísla $x_1 +
\dots + x_k$ a~$x_1^3+ \dots + x_k^3$ dávajú po delení šiestimi rovnaký
zvyšok, dajte ďalšie 2~body. Za dokončenie riešenia dajte zvyšné 2~body.

\endpetit
\bigbreak
}

{%%%%%   B-II-1
Trojuholníky $S_bOC$ a~$PBC$ sú podobné, lebo sú oba pravouhlé
a~veľkosť obvodového uhla $PBC$ prislúchajúceho tetive~$AC$ je rovná
polovici veľkosti stredového uhla $AOC$ prislúchajúceho tej istej tetive (\obr).
\insp{b67.5}%

Z~rovnakého dôvodu sú podobné aj trojuholníky $S_aOC$ a~$PAC$.
Platí preto
$$
\frac{|OS_a|}{|OC|} = \frac{|AP|}{|AC|}= \frac{|AP|}{b},\qquad
\frac{|OS_b|}{|OC|} = \frac{|BP|}{|BC|}= \frac{|BP|}{a},
$$
odkiaľ
$$
\frac{|OS_a|}{|OS_b|}
= \frac{|AP| \cdot |OC|}{b} \cdot \frac{a}{|BP| \cdot |OC|}
= \frac{|AP|}{|BP|}\cdot \frac{a}{b}.
= k\cdot \frac{a}{b}.
$$


\nobreak\medskip\petit\noindent
Za úplné riešenie dajte 6~bodov,
pritom každý z~nasledujúcich krokov oceňte jedným bodom:
1.~jedna podobnosť (s~dôkazom),
2.~druhá podobnosť (s~dôkazom),
3.~správny pomer v~prvej podobnosti,
4.~správny pomer v~druhej podobnosti,
5.~spojenie do rovnosti (eliminácia $|OC|$),
6.~správny záver (vyjadrenie pomocou $a$, $b$,~$k$).

\endpetit
\bigbreak
}

{%%%%%   B-II-2
Vďaka podmienkam $x\ge 0$ a~$t>0$ môžeme danú nerovnosť ekvivalentne upraviť
na kvadratickú nerovnosť vzhľadom na~$x$:
$$
\align
\frac {t}{x+2}+\frac {x}{t(x+1)}\leq & 1,\\
t^2(x+1)+x(x+2) \leq & t(x+1)(x+2) ,\\
x^2+(t^2+2)x+t^2\leq & tx^2+3tx+2t ,\\
0\leq &(t-1)x^2+(3t-t^2-2)x+2t-t^2,\\
0\leq &(t-1)x^2+(2-t)(t-1)x+t(2-t) . \tag1
\endalign
$$

Dosadením $x=0$ dostaneme $t(2-t)\ge0$, a~tak pre hľadané $t>0$ nutne $t\le 2$.

Ak má nerovnosť~(1) platiť pre všetky
nezáporné čísla~$x$, nemôže byť koeficient pri $x^2$ záporný, inak by sme
pre dostatočne veľké nezáporné~$x$ určite získali na pravej strane~(1) zápornú hodnotu.
Musí preto byť $t-1\ge0$, čiže $t\ge 1$.

Naopak pre ľubovoľné $t\in\left<1,2\right>$ budú zrejme koeficienty všetkých
troch členov kvadratického trojčlena (v~premennej~$x$) na pravej strane
nerovnosti~(1) nezáporné,
takže daná nerovnosť bude platiť pre ľubovoľné nezáporné reálne číslo~$x$.



\nobreak\medskip\petit\noindent
Za úplné riešenie dajte 6~bodov.

Za dôkaz, že $t\ge1$, dajte 2~body (možno nahliadnuť napríklad sporom: pre $t<1$
po úprave na $x^2 \le(t-2) x + t (t-2) / (1-t)$ je vidno, že parabola
nemôže ležať pod žiadnou priamkou).

2~body za dôkaz, že $t\le2$. To možno dokázať opäť sporom aj tak, že
za~predpokladu $t>2$ upravíme na $x ^ 2\ge(t-2) x + t (t-2) / (t-1)$,
takže graf priamky, ktorú predstavuje výraz na pravej strane,
pretína os~$y$ v~kladnom bode, čo odporuje polohe vrcholu paraboly $y=x^2$.

2~body za dôkaz, že nerovnici naozaj vyhovuje interval $\left<1,2\right>$
a~formuláciu záveru.

1~bod strhnite, ak je vynechaná zmienka o~ekvivalentných úpravách, pokiaľ je nutná.

Len za overenie, že tvrdenie funguje pre konečne veľa $t$ z~intervalu
$\left<1,2\right>$, žiadne body nedávajte.
\endpetit
\bigbreak
}

{%%%%%   B-II-3
Súčet všetkých čísel v~tabuľke je $\frac12n^2(n^2+1)$.
Ak má byť tento súčet deliteľný siedmimi, musí byť siedmimi deliteľné
buď samo~$n$, alebo číslo $n^2+1$. Ako sa však ľahko presvedčíme,
číslo $n^2+1$ nie je násobkom siedmich pre žiadne prirodzené~$n$. (To
samozrejme stačí overiť len pre~$n$ rovné všetkým možným zvyškom
$0, 1,\dots, 6$ po delení siedmimi.)

Ak je naopak $n$ násobok siedmich, možno tabuľku naozaj vyplniť tak, aby
požiadavky úlohy boli splnené. Stačí do nej vpísať čísla od~$1$ do~$n^2$
postupne podľa veľkosti po jednotlivých riadkoch.
V~$i$-tom riadku ($1\le i\le n$) tak budú čísla
$$
7(i-1)+1,\ 7(i-1)+2,\ \dots,\ 7(i-1)+n,
$$
ktorých súčet je deliteľný siedmimi, pretože súčet $1+2+\dots+n=\frac12n(n+1)$
je zrejme násobkom siedmich.

Podobne v~$j$-tom stĺpci ($1\le j\le n$) budú čísla
$$
j,\ n+j,\ \dots,\ (n-1) n+j,
$$
ktorých súčet je $nj+n(1+2+\dots+n-1)$, čo je násobok čísla~$n$, a~teda aj siedmich.

\nobreak\medskip\petit\noindent
Za úplné riešenie dajte 6~bodov.
Za nájdenie nutnej podmienky $7\mid n$ dajte 3~body, za overenie,
že pre tieto~$n$ možno tabuľku vyplniť, dajte 3~body.
Podrobnejšie:
1~bod za pozorovanie, že súčet čísel tabuľky musí byť deliteľný~7,
1~bod za výpočet tohto súčtu ako $\frac12n ^ 2 (n ^ 2 + 1)$,
1~bod za korektný dôkaz, že $7 \mid n$,
1~bod za opis vyhovujúceho vyplnenia,
1~bod za dôkaz, že riadky majú súčet deliteľný~7,
1~bod za dôkaz, že stĺpce majú súčet deliteľný~7.

A~čiastočné body:
1~bod dajte za hypotézu, že odpoveďou je $7 \mid n$.
Za dôkaz, že pre konečne veľa $n$ sa to spraviť nedá, ani
za nájdenie vyplnenia pre $n = 7$ (a~nič viac) žiadne body nedávajte.
\endpetit
\bigbreak
}

{%%%%%   B-II-4
Výber 505 čísel $1, 5, 9, 13, 17,\dots, 2\,017\in\mn M$,
ktoré dávajú po delení štyrmi zvyšok~1, má zrejme požadovanú vlastnosť,
lebo rozdiel každých dvoch vybraných čísel je násobkom štyroch, teda číslo zložené.
(Keďže 4 je najmenšie zložené číslo, nie je možný podobne pravidelný výber
s~väčším počtom čísel.)

Ukážeme, že viac čísel s~požadovanou vlastnosťou z~danej množiny
vybrať nemožno, nech už postupujeme akokoľvek. Na to najskôr overíme kľúčové tvrdenie,
že totiž z~každej osmice po sebe idúcich čísel možno vybrať nanajvýš
dve čísla.

Dokážeme najskôr, že ak vyberieme nejaké číslo~$n$, zo siedmich nasledujúcich čísel
$n+1,n+2,\dots,n+7$ možno vybrať nanajvýš jedno. Naozaj, potom už sa nedá
vybrať žiadne z~čísel $n+2$, $n+3$, $n+5$, $n+7$.
A~zo zvyšnej trojice $n+1$, $n+4$, $n+6$ možno
vybrať iba jedno číslo, lebo ich rozdiely sú prvočísla 2, 3 a~5.
Tým je tvrdenie o~sedmici čísel, ktoré nasledujú za ktorýmkoľvek vybraným číslom~$n$,
dokázané. Z~neho je už zrejmé, že {\it žiadna\/} osmica po sebe idúcich čísel nemôže
obsahovať viac ako dve vybrané čísla, ako sme sľúbili overiť.

Danú množinu $\mn M$ môžeme rozdeliť na množinu $\{1, 2, \dots, 10\}$
a~251~nasledujúcich osmíc. Pritom z~množiny $\{1, 2, \dots, 10\}$
možno vybrať nanajvýš tri vyhovujúce čísla.
Keby sme z~prvej desiatky
$$
\{1, 2,\dots, 10\}=\{1, 2\}\cup\{3, 4,\dots, 10\}=\{1, 2,\dots, 8\}\cup\{9, 10\}
$$
vybrali vyhovujúcim spôsobom štyri čísla, podľa dokázanej vlastnosti
každej osmice po sebe idúcich čísel by sme museli vybrať ako obe čísla 1 a~2,
tak obe čísla 9 a~10, avšak $9-2$ je prvočíslo.
Z~množiny~$\mn M$ tak možno vybrať nanajvýš $2\cdot 251+3=505$~čísel.

\nobreak\medskip\petit\noindent
Za úplné riešenie dajte 6~bodov.
1~bod za myšlienku, že z~osmice po sebe idúcich čísel sa dajú vybrať
nanajvýš dve a~1~bod za korektný dôkaz tejto myšlienky.
1~bod za rozdelenie čísel správnym spôsobom na disjunktné osmice a~"zvyšok".
1~bod za správne odvodenie, že z~takéhoto rozdelenia vyplýva, že môžeme
vybrať nanajvýš 505 čísel.
1~bod za konštrukciu samotnej vyhovujúcej množiny
a~1~bod za zdôvodnenie, že zostrojená množina úlohe vyhovuje.

V~prípade čiastočného riešenia dajte 1~bod za konštrukciu príkladu 505 čísel,
ktorá vyhovujú a~ešte bod navyše za dôkaz, že
k~ním nemožno pridať žiadne ďalšie číslo.
\endpetit
\bigbreak
}

{%%%%%   C-S-1
Hľadáme najväčšie medzi číslami $\overline{abc}$, pre ktoré všetky tri
čísla $\overline {ab}$, $\overline{ac}$ aj $\overline{bc}$ sú
prvočísla. Dvojciferné prvočísla $\overline {ab}$, $\overline{ac}$
sú nepárne, preto ich posledné
cifry musia byť nepárne. Navyše medzi nimi nemôže byť ani
cifra~5, inak by príslušné dvojciferné číslo bolo deliteľné piatimi.
Stačí teda ďalej skúmať iba cifry $b$ a~$c$ z~množiny $\{1,3,7,9\}$.

Cifru~$a$ chceme čo najväčšiu, budeme teda postupne prechádzať možnosti
počínajúc hodnotou $a=9$, kým nenájdeme riešenie.

Pre $a=9$ sú čísla $91 = 7 \cdot 13$, $93 = 3 \cdot 31$, $99 = 3 \cdot 33$
zložené, jedine 97 je prvočíslo, a~tak ostáva jediná možnosť $b = c = 7$.
V~tom prípade je však číslo $\overline {bc} = 77= 7 \cdot 11$ zložené.
Cifrou~9 teda hľadané číslo začínať nemôže.

Podobne pre $a= 8$ vylúčime
možnosti $b, c \in \{1,7\}$, pretože $81 = 3 \cdot 27$ a~$87 = 3 \cdot 29$.
Ostáva tak iba možnosť $b, c \in \{3, 9\}$. V~tom prípade je však číslo
$\overline {bc}$ deliteľné tromi.

Ak hľadané číslo začína cifrou $a=7$, nemôže byť $b=7$ ani $c=7$,
zato čísla 71, 73 aj~79 sú napospol prvočísla. Zo zvyšných kandidátov 1, 3, 9
na cifry $b$, $c$ možno vytvoriť štyri dvojciferné prvočísla
31, 19, 13, 11 a~najväčšie z~nich je
číslo~31. Vidíme teda, že číslo 731 spĺňa podmienky úlohy
a~žiadne väčšie také trojciferné číslo neexistuje.

Hľadané najväčšie trojciferné číslo je 731.


\nobreak\medskip\petit\noindent
Za úplné riešenie dajte 6~bodov.
Za vylúčenie možnosti $b, c \in \{2, 4, 6, 8\}$ dajte 1~bod, za vylúčenie
$b = 5$ a~$c = 5$ dajte ďalší bod. Za vylúčenie každej z~možností $a~= 9$
a~$a~= 8$ dajte po 1~bode. Posledné 2~body dajte za rozobranie možnosti
$a~= 7$ a~nájdenie správneho riešenia. Systematické prehľadávanie
možností od najväčšieho trojciferného čísla bez nájdenia správneho
riešenia ohodnoťte nanajvýš 4~bodmi.
\endpetit
\bigbreak
}

{%%%%%   C-S-2
Pre párne $n = 2k$ rozdeľme tabuľku na neprekrývajúce sa
štvorcové časti $2\times 2$~-- tých bude presne $k^2$ a~v~každej z~nich
má byť zapísaný iný násobok piatich. Na to ale potrebujeme
$k^2$ násobkov piatich, z~ktorých najmenšie sú čísla 5, 10,~\dots, $5k^2$,
prinajmenšom posledné uvedené však
medzi prirodzenými číslami od~1 do $n^2 =(2k)^2 = 4k^2$ nenájdeme.
Tým je časť~a) dokázaná.

Pre nepárne $n = 2k+1$ vyberieme v~tabuľke podobným spôsobom
$k^2$ neprekrývajúcich sa štvorcových častí $2\times 2$~--
napríklad tak, že vynecháme jej posledný riadok a~posledný stĺpec.
Aby sme mohli tabuľku vyplniť požadovaným
spôsobom, musí opäť platiť $5k^2\le n^2=(2k+1)^2=4k^2+4k+1$, čiže
$k^2\le 4k+1$. Posledná nerovnosť ale nemôže platiť pre $k\ge 5$,
lebo pre také~$k$ je naopak $k^2\ge 5k> 4k+1$.

Najväčšie nepárne~$n$, pre ktoré máme medzi číslami od~1 do $n^2$ dostatok
násobkov piatich, je teda $n=9$. Všetkých 16~násobkov piatich vpíšeme do
jednotlivých častí $2\times 2$,
pričom zvyšné čísla potom vpíšeme do prázdnych políčok ľubovoľne:
$$
\tabulkacsLXVII

&&&&&&&&&

&& 5&& 10&& 15&& 20&

&&&&&&&&&

&& 25&& 30&& 35&& 40&

&&&&&&&&&

&& 45&& 50&& 55&& 60&

&&&&&&&&&

&& 65&& 70&& 75&& 80&

&&&&&&&&&


$$

Hľadané najväčšie nepárne~$n$, pre ktoré možno tabuľku vyplniť požadovaným
spôsobom, je rovné deviatim.


\nobreak\medskip\petit\noindent
Za úplné riešenie dajte 6~bodov.
Za myšlienku
rozdeliť tabuľku na~$k^2$ neprekrývajúcich sa častí $2 \times 2$
dajte 3~body, za dokončenie dôkazu časti~a) potom 1~bod, za odvodenie
podmienky $n\le9$ v~časti~b) ďalší bod a~bod za uvedenie príkladu vyplnenia.

\endpetit
\bigbreak
}

{%%%%%   C-S-3
Keďže trojuholník $ABC$ má pri vrchole~$A$ tupý uhol, leží bod~$D$ zvonka
úsečky~$AB$. Z~konštrukcie jednotlivých bodov tak vyplýva, že
body $P$ a~$Q$ ležia v~opačných polrovinách s~hraničnou priamkou~$AB$ (\obr).
\insp{c67.5}%

Z~podobnosti pravouhlých trojuholníkov $ADP\sim ABF$ vyplýva
$$
\frac {|DP|} {|AD|} = \frac {|BF|} {|AB|} = \frac {|BD|} {|AB|} \quad
\hbox {a~odtiaľ} \quad |DP| = \frac {|AD| \cdot |BD|} {|AB|},
$$
pričom sme využili rovnosť $|BF| = |BD|$ zo zadania.

Podobne z~podobnosti pravouhlých trojuholníkov $DBQ \sim ABE$ vyplýva
$$
\frac {|DQ|} {|DB|} = \frac {|AE|} {|AB|} = \frac {|AD|} {|AB|}
\quad\hbox {a~odtiaľ}\quad
|DQ| = \frac {|AD| \cdot |BD|} {|AB|},
$$
pričom sme využili druhú rovnosť $|AE| = |AD|$ zo zadania.

Vidíme tak, že dĺžky úsečiek $DP$ a~$DQ$ sa rovnajú, a~keďže
bod~$D$ leží vnútri úsečky~$PQ$, je jej stredom, čo sme
chceli dokázať.


\nobreak\medskip\petit\noindent
Za úplné riešenie dajte 6~bodov.
Zapísanie podobnosti trojuholníkov $ADP \sim ABE$ a~$DBQ \sim ABF$
ohodnoťte po jednom bode, ďalšie 2~body dajte za vyjadrenie dĺžok
úsečiek $|DP| = |AD| \cdot |BF| / |AB|$ a~$|DQ| = |BD| \cdot |AE| / |AB|$
a~posledné 2~body dajte za správne využitie rovností zo zadania.

\endpetit
}

{%%%%%   C-II-1
Využijeme to, že každý súčin čísla~7 s~jednociferným číslom končí inou cifrou.
Postupne budeme odzadu dopĺňať cifry hľadaného činiteľa (k~danému činiteľu~2\,017)
tak, aby sme vo výsledku dostali číslo končiace štvorčíslím 2\,018.

\def\hline{\noalign{\hrule}}
V~prvom kroku hľadáme cifru, ktorej 7-násobok končí cifrou~8:
$$
\vbox{\let\\=\cr
\halign{\strut\ $#$ \hss&&\hbox to1em{\hss$#$\hss}\\
& 2&0&1&7 \\
\times& *&*&*&? \\
\hline
\cdots&*&*&*&*&\\
\cdots&*&*&*&& \\
\cdots&*&*&&& \\
\hline
\cdots&2&0&1&8\\
}}
$$
Cifrou~8 končí jediný zo súčinov čísla~7 s~ciframi 0 až 9, a~to číslo $28=4\cdot7$.
Na mieste otáznika preto musí byť cifra~4. Doplníme
ju a~v~prvom riadku pod čiarou tak bude $4 \cdot 2\,017 = 8\,068$, čo
naozaj končí cifrou~8. Vo výslednom súčine potrebujeme mať cifru~1
na mieste desiatok; tú spolu s~cifrou~6 na mieste desiatok v~čísle
8\,068 dostaneme iba tak, že k~nej pričítame cifru~5~-- tentoraz teda
hľadáme taký násobok čísla~2\,017, ktorý končí cifrou~5:
$$
\vbox{\let\\=\cr
\halign{\strut\ $#$ \hss&&\hbox to1em{\hss$#$\hss}\\
& 2&0&1&7 \\
\times& *&*&?&4 \\
\hline
& 8&0&6&8 \\
\cdots&*&*&5&\\
\cdots&*&*&& \\
\cdots&*&&& \\
\hline
\cdots&2&0&1&8\\
}}
$$
Tomu vyhovuje iba násobok $5 \cdot 7 = 35$.
Podobným postupom doplníme aj zvyšné dve cifry~-- na mieste stoviek preto bude
cifra~3 a~na mieste tisícok cifra~4:
$$
\vbox{\let\\=\cr
\halign{\strut\ $#$ \hss&&\hbox to1em{\hss$#$\hss}\\
& 2&0&1&7 \\
\times& *&?&5&4 \\
\hline
& 8&0&6&8 \\
\cdots& 0&8&5&\\
\cdots& *&1&& \\
\cdots& *&&& \\
\hline
\cdots&2&0&1&8\\
}}
\qquad
\vbox{\let\\=\cr
\halign{\strut\ $#$ \hss&&\hbox to1em{\hss$#$\hss}\\
& 2&0&1&7 \\
\times& ?&3&5&4 \\
\hline
& 8&0&6&8 \\
\cdots& 0&8&5&\\
\cdots& 5&1&& \\
\cdots& 8&&& \\
\hline
\cdots&2&0&1&8\\
}}
\qquad
\vbox{\let\\=\cr
\halign{\strut\ $#$ \hss&&\hbox to1em{\hss$#$\hss}\\
& 2&0&1&7 \\
\times& 4&3&5&4 \\
\hline
& 8&0&6&8 \\
\cdots& 0&8&5&\\
\cdots& 5&1&& \\
\cdots& 8&&& \\
\hline
\cdots&2&0&1&8\\
}}
$$

Každé číslo, ktoré po vynásobení číslom 2\,017 končí
štvorčíslím 2\,018, musí teda končiť štvorčíslím 4\,354, a~tak najmenší
vyhovujúci násobok je $2\,017 \cdot 4\,354 = 8\,782\,018$.


\nobreak\medskip\petit\noindent
Za úplné riešenie dajte 6~bodov.
Za správnu voľbu postupu zisťovania cifier sprava dajte 1~bod. Každú
cifru správneho riešenia odmeňte 1~bodom a~zdôvodnenie, že nájdené
číslo je najmenšie, odmeňte posledným bodom. Ak je
postup správny, ale nájdené číslo nie je správne, dajte nanajvýš 4~body.

\endpetit
\bigbreak
}

{%%%%%   C-II-2
Sčítaním daných rovností dostaneme
$$
(x^2+y-z)+(x^2-y+z) = 2x^2 = 10+22 = 32,
$$
preto $x^2 = 16$, čiže $x={\pm4}$. Keď tento výsledok dosadíme späť
do prvej či druhej z~daných rovností, vyjde v~oboch prípadoch $z=y+6$.

Dosaďme teraz obe získané rovnosti do výrazu, ktorý máme minimalizovať:
$$
x^2+y^2+z^2 = 16+y^2+(y+6)^2 = 2y^2+12y+52 = 2(y^2+6y+26)
= 2\big((y+3)^2+17 \big).
$$
Keďže $(y+3)^2 \ge 0$, je najmenšia možná hodnota daného výrazu
rovná $2 \cdot 17 = 34$. Túto hodnotu upravený výraz dosahuje pre
$y = {-3}$, z~čoho vychádza $z=y+6=3$, nájdené čísla $x$, $y$ a~$z$
sú teda celé, ako vyžaduje zadanie.

\nobreak\medskip\petit\noindent
Za úplné riešenie dajte 6~bodov.
Za odvodenie oboch vzťahov $x^2 = 16$ a~$z=y+6$ dajte po 1~bode, 1~bod za~prepis
výrazu v~zadaní na výraz s~jednou neznámou,
2~body za jeho úpravu na výraz s~druhou mocninou a~1~bod za nájdenie minima 34.
Ak však chýba konštatovanie, že hodnotu 34 výraz skutočne nadobúda, strhnite 1~bod.
\endpetit
\bigbreak
}

{%%%%%   C-II-3
Dokážeme, že uhlopriečky štvoruholníka $ABTC$ sa navzájom
rozpoľujú, čo bude znamenať, že $ABTC$ je naozaj rovnobežník.

Označme $X$ stred úsečky~$PQ$ a~$Y$ stred úsečky~$RS$ (\obr). Z~rovnosti
$|BR| = |SC|$ vyplýva, že bod~$Y$ je zároveň stredom úsečky~$BC$.
Keďže ako je známe $PQ\parallel BC$, sú trojuholníky $APQ$ a~$ABC$
\insp{c67.6}%
podobné podľa vety $uu$, takže ich \uv{polovice} $APX$ a~$ABY$
sú podobné podľa vety $sus$, teda bod~$X$ leží vnútri úsečky~$AY$.
Analogicky z~rovnobežnosti $PQ\parallel RS$ vyplýva podobnosť trojuholníkov $TRS$ a~$TPQ$
($uu$), takže sú podobné aj trojuholníky $TRY$ a~$TPX$ ($sus$),
a~bod~$Y$ tak leží vnútri úsečky~$TX$ (lebo $|RS|<|PQ|$). Spolu dostávame,
že bod~$Y$ leží na úsečke~$AT$.
Ostáva dokázať, že~
$Y$~je nielen stredom úsečky~$BC$, ale aj stredom úsečky~$AT$.

Keďže úsečka~$PQ$ je strednou priečkou trojuholníka $ABC$, je
$|PQ| = \frac12|BC|$, čiže $|PX| = \frac12|BY|$,
takže z~prvej podobnosti $APX\sim ABY$ máme
$$
|AX| = \frac12|AY|,\quad\text{čiže}\quad|AY| = 2|XY|. \tag1
$$

Keďže $|RS| = \frac13|BC|=\frac13\cdot 2|PQ|$, je $|RY| = \frac23|PX|$,
a~preto z~druhej podobnosti $TRY\sim TPX$ vychádza %$|TY| = \frac 23|TX|$
$$
|TY| = \frac 23|TX| = \frac 23 (|TY|+|XY|),
\quad \text{čiže} \quad |TY| = 2|XY|.
$$
To spolu s~druhou rovnosťou v~(1) dáva $|TY| = |AY|$. Tým je dôkaz ukončený.


\ineries
Doplňme trojuholník $ABC$ na rovnobežník $ABT'C$. Jeho
uhlopriečka~$AT'$ potom prechádza stredom~$Y$ úsečky~$BC$ (\obr),
\insp{c67.7}%
a~zo zadania teda vyplýva, že
$|RY| = \frac12|RS| = \frac12\cdot\frac13|BC|= \frac13\cdot\frac12|BC|= \frac13|BY|$,
čiže $|BR|: |RY| = 2:1$.
Bod~$R$ tak leží na ťažnici~$BY$ trojuholníka $ABT'$ a~delí ju v~pomere $2:1$,
preto je ťažiskom. Priamka~$T'R$ je teda ťažnicou
trojuholníka $ABT'$, a~prechádza preto stredom~$P$ strany~$AB$. Podobne dostaneme,
že priamka~$T'S$ prechádza bodom~$Q$, a~tak je bod~$T'$ priesečníkom
priamok $PR$ a~$QS$, čiže $T' = T$, čo sme chceli dokázať.


\nobreak\medskip\petit\noindent
Za úplné riešenie dajte 6~bodov.
Za dôkaz, že úsečka~$AT$ rozpoľuje úsečku~$BC$,
dajte celkom 3~body, z~toho po jednom bode za obe
podobnosti a~tretí bod za dôkaz kolineárnosti bodov $A$, $Y$ a~$T$. Ďalšie 2~body dajte za vyjadrenie pomeru
$|TY|: |TX| = 2: 3$ a~za odvodenie rovnosti $|TY| = |AY|$.
A~posledný bod za tvrdenie, že štvoruholník $ABTC$ je rovnobežník,
pretože sa jeho uhlopriečky rozpoľujú.

Ak riešiteľ postupuje ako v~druhom riešení, oceňte 2~bodmi doplnenie
na rovnobežník $ABT'C$. Tri body dajte
za tvrdenie, že bod~$R$ (resp.~$S$) je ťažiskom trojuholníka $ABT'$
(resp. $ACT'$), pričom 2~body dajte za odvodenie pomeru $|BR|: |RY| = 2: 1$ (resp.
$|SC|: |SY| = 2: 1$). Posledný bod dajte za zistenie, že bod~$T'$ leží
rovnako ako bod~$T$ v~priesečníku priamok $PR$ a~$QS$.

Ak riešiteľ postupuje inak ako v~uvedených riešeniach, hodnoťte
jednotlivé zistenia v~súlade s~uvedenými schémami.

\endpetit
\bigbreak
}

{%%%%%   C-II-4
Pre čísla $n = 1,2,3$ nie je ťažké uviesť príklad
tabuliek, v~ktorých želané políčko neexistuje. Keďže nás zaujímajú iba zvyšky čísel
po delení tromi, zostavíme najskôr prislúchajúce tabuľky tak, aby sme
v~každom riadku aj stĺpci mali rôzne zvyšky, a~potom uvedené zvyšky nahradíme
rôznymi číslami s~tým istým zvyškom (tabuľka pre $n=1$ môže obsahovať
ľubovoľné prirodzené číslo):
$$
\tabulkackLXVII

&1


\qquad
\tabulkackLXVII

&2&1

&1&2


\qquad
\tabulkackLXVII

&0&1&2

&2&0&1

&1&2&0


\qquad
\longrightarrow
\qquad
\tabulkackLXVII

&2&1

&4&5


\qquad
\tabulkackLXVII

&3&1&2

&5&6&4

&7&8&9


$$

Teraz ukážeme, že pre $n = 4$ už dané tvrdenie platí. Čísla v~ľubovoľne
vyplnenej tabuľke~$4\times4$ nahradíme ich zvyškami 0, 1, 2 po delení
tromi. Zvoľme ľubovoľný riadok~$r_1$ vyplnenej tabuľky. Sú v~ňom zapísané štyri
zvyšky, takže aspoň jeden z~nich~-- označme ho~$z_1$~-- sa v~riadku
opakuje, je teda zapísaný
v~dvoch rôznych stĺpcoch, ktoré označíme $s_1$ a~$s_2$.
Ak je v~stĺpci~$s_1$
zvyšok~$z_1$ zapísaný dvakrát, sme s~hľadaním požadovaného políčka
hotoví (číslo v~riadku~$r_1$ a~stĺpci~$s_1$ má vo svojom riadku aj stĺpci
ešte ďalšie číslo s~rovnakým zvyškom~$z_1$ po delení tromi).

V~opačnom prípade je v~stĺpci~$s_1$ nejaký iný
zvyšok~-- označme ho~$z_2\ne z_1$~--~zapísaný aspoň dvakrát; prislúchajúce
riadky označme $r_2$ a~$r_3$:
$$
\vbox{\let\\=\cr
\halign{\strut#\hss\enspace\vrule&&\hss\enspace#\enspace\hss\\
& $s_1$&$s_2$ \\
\noalign{\hrule}
$r_1$&$z_1$&$z_1$ \\
$r_2$&$z_2$&? \\
$r_3$&$z_2$&? \\
}}
$$

Ak je na niektorom z~dvoch políčok uvažovanej tabuľky označených zatiaľ otáznikom
zvyšok~$z_1$, sme hotoví,
keďže požadovanú vlastnosť má políčko v~riadku~$r_1$ a~v~stĺpci~$s_2$.
Podobne sme hotoví, ak je na mieste jedného z~otáznikov zvyšok~$z_2$,
keďže v~tom prípade vyhovuje políčko so
zvyškom~$z_2$ v~tom istom riadku a~stĺpci~$s_1$. Ostáva ešte tá
možnosť, že na miestach oboch otáznikov sú zapísané dva rovnaké
zvyšky~$z_3$ ($z_2\ne z_3\ne z_1$).
Označme zvyšné stĺpce ako $s_3$ a~$s_4$ (na ich skutočnom poradí
v~tabuľke samozrejme nezáleží):
$$
\vbox{\let\\=\cr
\halign{\strut#\hss\enspace\vrule&&\hss\enspace#\enspace\hss\\
& $s_1$& $s_2$& $s_3$&$s_4$ \\
\noalign{\hrule}
$r_1$& $z_1$& $z_1$& & \\
$r_2$& $z_2$& $z_3$& ?&? \\
$r_3$& $z_2$& $z_3$& ?&? \\
}}
$$

Ak je teraz na mieste jedného zo štyroch otáznikov zvyšok~$z_2$, má
požadovanú vlastnosť políčko v~tom istom riadku a~v~stĺpci~$s_1$.
Podobne ak je na mieste jedného zo štyroch otáznikov
zvyšok~$z_3$, vyhovuje políčko v~tom istom riadku a~v~stĺpci~$s_2$.
Napokon ak na mieste všetkých štyroch otáznikov sú napísané iba
zvyšky~$z_1$, tak každé z~týchto štyroch políčok
má požadovanú vlastnosť.

Práve sme ukázali, že v~každej tabuľke
$4 \times 4$ nájdeme vždy číslo, ktoré dáva rovnaký zvyšok po delení
tromi ako iné číslo z~toho istého riadka aj ako iné číslo z~toho istého
stĺpca. Hľadané najmenšie~$n$ je teda rovné štyrom.


\nobreak\medskip\petit\noindent
Za úplné riešenie dajte 6~bodov.
Body rozdeľte zhruba nasledovne:
1~bod za tabuľku $3\times3$,
1~bod za ideu, že v~každom riadku a~stĺpci tabuľky $4\times4$ sa nejaký zvyšok opakuje,
1~bod za prepojenie riadkov a~stĺpcov (zmysluplné použitie myšlienky),
1~bod za následné doplnenie zvyškov~$z_3$ (dvojica otáznikov),
1~bod za doplnenie ďalších zvyškov (štvorica otáznikov), ktoré vedú k~nájdeniu
vhodného políčka,
a~1~bod za správny záver, že najmenšie hľadané číslo je~4.
\endpetit
\bigbreak
}

{%%%%%   vyberko, den 1, priklad 1
...}

{%%%%%   vyberko, den 1, priklad 2
...}

{%%%%%   vyberko, den 1, priklad 3
...}

{%%%%%   vyberko, den 1, priklad 4
...}

{%%%%%   vyberko, den 2, priklad 1
...}

{%%%%%   vyberko, den 2, priklad 2
...}

{%%%%%   vyberko, den 2, priklad 3
...}

{%%%%%   vyberko, den 2, priklad 4
...}

{%%%%%   vyberko, den 3, priklad 1
...}

{%%%%%   vyberko, den 3, priklad 2
...}

{%%%%%   vyberko, den 3, priklad 3
...}

{%%%%%   vyberko, den 3, priklad 4
...}

{%%%%%   vyberko, den 4, priklad 1
...}

{%%%%%   vyberko, den 4, priklad 2
...}

{%%%%%   vyberko, den 4, priklad 3
...}

{%%%%%   vyberko, den 4, priklad 4
...}

{%%%%%   vyberko, den 5, priklad 1
...}

{%%%%%   vyberko, den 5, priklad 2
...}

{%%%%%   vyberko, den 5, priklad 3
...}

{%%%%%   vyberko, den 5, priklad 4
...}

{%%%%%   trojstretnutie, priklad 1
...}

{%%%%%   trojstretnutie, priklad 2
...}

{%%%%%   trojstretnutie, priklad 3
...}

{%%%%%   trojstretnutie, priklad 4
...}

{%%%%%   trojstretnutie, priklad 5
...}

{%%%%%   trojstretnutie, priklad 6
...}

{%%%%%   IMO, priklad 1
Keďže~$F$ leží na osi~$BD$, trojuholník $BFD$ je rovnoramenný. Označme~$K$ druhý priesečník priamky~$FD$ s~$\Gamma$. Z~obvodového uhla nad tetivou~$AF$ dostaneme $|\angle AKF| = |\angle ABF|$, a~keďže $|\angle ABF| = |\angle BDF|$, trojuholník~$AKD$ je rovnoramenný (\obr). Z~toho máme $|AD|=|AK|$, a~keďže tiež $|AD|=|AE|$, $A$ je stred kružnice opísanej trojuholníku $DEK$.

Označme~$|\angle KDE|=\phi$. Úloha bude dokončená, ak dokážeme, že $|\angle KFG|=\phi$, čo je ekvivalentné s~$|\angle KAG|=\phi$. Z~vety o~obvodom a~stredovom uhle platí $|\angle EAK|=2 \phi$, takže stačí dokázať, že $AG$ je os $\angle EAK$, teda os úsečky~$EK$. Keďže $G$ z~definície leží na osi úsečky~$EC$, stačí dokázať, že $G$ je stredom kružnice opísanej trojuholníku $KEC$.
\inspinspmedzera{imo1.1}{imo1.2}{\quad}%

Z~obvodového uhla nad tetivou~$CK$ máme $|\angle KGC|=180^\circ-2\phi$. Ďalej z~rovnoramennosti $AKE$ máme $|\angle KEA|=90^\circ-\phi$, takže $|\angle KEC|=90^\circ+\phi$. Označme $G'$ stred kružnice opísanej trojuholníku $KEC$. Z~vety o~obvodom a~stredovom uhle platí $|\angle KG'C|=2\cdot(180^\circ-|\angle KEC|)=2\cdot(180^\circ-(90^\circ+\phi))=180^\circ-2\phi$, takže~$G'$ leží na $\Gamma$ na oblúku~$KC$ neobsahujúcom~$B$. Keďže $G'$ leží aj na osi~$EC$, nutne $G=G'$, čo stačilo dokázať.

\poznamka
Po definovaní~$K$ a~dokázaní, že $|AK|=|AD|$ ($=|AE|$) je možné úlohu dokončiť aj inak. Ak definujeme~$L$ ako druhý priesečník $GE$ a~$\Gamma$ (\obr), tak analogicky ako $K$ aj $L$ leží na kružnici so stredom~$A$ a~polomerom~$|AD|$. Štvoruholník $KEDL$ je teda tetivový, takže spolu máme
$$
|\angle GFK| = |\angle GLK| = |\angle ELK| = |\angle EDK|,
$$
čo je ekvivalentné s~dokazovanou rovnobežnosťou.

\ineriesenie
Označme $O$ stred $\Gamma$ a~$M$, $N$ postupne stredy oblúkov $AB$ a~$AC$ neobsahujúcich $C$ a~$B$ (\obr). Vzdialenosť~$d$ medzi $OM$ a~osou~$o_1$ úsečky~$BD$ je rovná $\frac12 |AB| - \frac12 |BD|=\frac12 |AD|$, takže je rovnaká ako vzdialenosť $ON$ a~osi~$o_2$ úsečky~$CE$. To znamená, že rovnoramenný lichobežník s~vrcholmi na $\Gamma$ určený priamkami $OM$ a~$o_1$ je zhodný s~rovnoramenným lichobežníkom s~vrcholmi na $\Gamma$ určeným priamkami $ON$ a~$o_2$. Z~toho máme, že oblúky $MF$ a~$NG$ sú rovnako dlhé. Z~toho vyplýva $FG \parallel MN$, takže stačí dokázať $MN \parallel DE$.

Označme~$P$ stred oblúka~$BC$ neobsahujúceho~$A$. Potom je $AP$ os uhla $\angle DAE$ a~z~rovnoramennosti $ADE$ platí $AP \perp DE$. Stačí teda dokázať $AP \perp MN$. Uhol medzi tetivami $AP$~a~$MN$ je však rovný
%
\footnote{$(XY)$ označuje veľkosť obvodového uhla prislúchajúceho oblúku~$XY$ zapísaného proti smeru hodinových ručičiek. Používame tvrdenie, že~ak máme body $X$, $X'$, $Y$, $Y'$ na kružnici v~tomto poradí, tak uhol medzi tetivami $XY$ a~$XX'$ je rovný $(XX') + (YY')$.}
%
$$
(MP)+ (NA) = (MB) + (BP) + (NA) = \frac{(AB)+(BC)+(CA)}2=90^\circ,
$$
čo bolo treba dokázať.
\inspinspmedzera{imo1.3}{imo1.4}{\quad}%

\ineriesenie
Nech~$P$, $Q$ sú postupne priesečníky priamky~$FG$ s~priamkami $AB$, $AC$. Označme $F'$ a~$G'$ také body, že $AF'FB$ a~$AG'GC$ sú rovnoramenné lichobežníky (\obr). Body $F'$ a~$G'$ zrejme ležia na~$\Gamma$. Tiež platí
$$
|\angle F'AD| = |\angle ABF|=|\angle DBF|=|\angle FDB|,
$$
z~čoho vyplýva $AF' \parallel DF$. Štvoruholník $AF'FD$ je teda rovnobežník. Analogicky $AG'GE$ je rovnobežník. Platí teda $|F'F|=|AD|=|AE|=|G'G|$. Štvoruholník $F'FGG'$ je teda rovnoramenný lichobežník, takže $|\angle F'FG| = |\angle G'GF|$. Vďaka rovnobežnostiam $F'F \parallel AP$ a~$G'G \parallel AQ$ to znamená $|\angle APQ| = |\angle AQP|$. Trojuholníky $APQ$, $ADE$ sú teda oba rovnoramenné, z~čoho už zrejme vyplýva $PQ \parallel DE$. }

{%%%%%   IMO, priklad 2
Pre pohodlnejšie vyjadrovanie rozšírme postupnosť~$a_1,\dots,a_{n+2}$ na nekonečnú periodickú postupnosť s~periódou~$n$ (nie nutne najkratšou).

Ak postupnosť obsahuje dva kladné členy $a_i$, $a_{i+1}$, tak~$a_{i+2} = a_i a_{i+1} + 1 > 1$, takže ďalší člen je taktiež kladný, a~navyše väčší~ako~1. Z~indukcie teda vyplýva, že všetky členy sú kladné a~väčšie ako~1. Avšak potom $a_{i+2} = a_i a_{i+1} + 1 \ge 1 \cdot a_{i+1} + 1 > a_{i+1}$ pre každý index~$i$, čo je nemožné, keďže naša postupnosť je periodická.

Ak postupnosť obsahuje číslo~0, teda ak~$a_i=0$ pre nejaký index~$i$, tak čísla $a_{i+1}=a_{i-1}a_i+1$ a~$a_{i+2}=a_i a_{i+1}+1$ sú obe rovné~1, takže ide o~dva po sebe idúce kladné členy postupnosti, čo sme už vylúčili.

Všimnime si, že po dvoch záporných členoch musí nasledovať kladný člen: ak platí $a_i < 0$ a~$a_{i+1}<0$, tak $a_{i+2}=a_i a_{i+1}+1 > 1 > 0$. Zhrnutím tohto pozorovania a~predošlých úvah máme, že každý kladný člen je nasledovaný jedným alebo dvoma zápornými členmi a~následne kladným členom.

Uvážme najprv prípad, kedy žiaden kladný člen nie je nasledovaný dvoma zápornými. V~tom prípade sa teda kladné a~záporné členy striedajú. Takže, ak $a_i < 0$, tak $a_{i+1} > 0$, $a_{i+2} < 0$, $a_{i+3} > 0$ (a~tak ďalej). Všimnime si, že $a_i a_{i+1} + 1 = a_{i+2} < 0 < a_{i+3} = a_{i+1}a_{i+2}+1$. Keďže $a_{i+1}>0$, máme $a_i < a_{i+2}$. Z~toho máme, že záporné členy tvoria rastúcu postupnosť $a_i < a_{i+2} < a_{i+4} < \dots$, čo nie je možné, keďže postupnosť je periodická.

Jediný zostávajúci prípad je, že postupnosť obsahuje dva po sebe idúce záporné členy. Predpokladajme, že $a_i$ a~$a_{i+1}$ sú záporné; potom $a_{i+2}=a_i a_{i+1} + 1 > 1 > 0$. Číslo~$a_{i+3}$ teda musí byť záporné. Platí $a_{i+4}=a_{i+2}a_{i+3}+1 < 1 < a_i a_{i+1} + 1 = a_{i+2}$. S~týmto pozorovaním ďalej máme
$$
a_{i+5}-a_{i+4}=(a_{i+3}a_{i+4}+1)-(a_{i+2}a_{i+3}+1)=a_{i+3}(a_{i+4}-a_{i+2}) > 0,
$$
takže $a_{i+5}>a_{i+4}$. Keďže najviac jedno z~čísel $a_{i+4}$, $a_{i+5}$ môže byť kladné, nutne číslo~$a_{i+4}$ musí byť záporné.

Dokázali sme, že čísla $a_i$, $a_{i+1}$, $a_{i+2}$, ktoré sú postupne záporné, záporné a~kladné, sú nutne nasledované číslami $a_{i+3}$, $a_{i+4}$, $a_{i+5}$, ktoré sú taktiež postupne záporné, záporné a~kladné. V~ďalších troch členoch sa nutne musí zopakovať tento vzor. Z~toho už vidieť, že~$n$ je nutne deliteľné~3.

Na druhej strane, pre $n$ deliteľné~3 je $(a_1,a_2,a_3,\dots)=(-1,-1,2,-1-1,2,\dots)$ zrejmým riešením. Platí teda, že všetky vyhovujúce čísla~$n$ sú čísla deliteľné tromi.

\ineriesenie
Dokážeme, že najkratšia perióda postupnosti $a_1,a_2,a_3,\dots$ je~rovná~3. Čísla $a_1,a_2\dots,a_n$ nemôžu byť všetky navzájom rovné, keďže rovnica $x^2+1=x$ nemá reálne riešenie. Najkratšia perióda teda nemôže byť~1.

Aplikovaním rekurentného vzťahu pre $i$ a~$i+1$ dostávame
$$
\displaylines
{
(a_{i+2}-1)a_{i+2} = a_i a_{i+1} a_{i+2} = a_i(a_{i+3}-1),\quad \hbox{takže} \cr
a_{i+2}^2 - a_i a_{i+3} = a_{i+2} - a_i.
}
$$
Sčítaním týchto vzťahov pre $i=1,2,\dots,n$ a~násobením dvomi dostávame
$$
\sum_{i=1}^n (a_i - a_{i+3})^2 = 0.
$$

To dokazuje, že $a_i = a_{i+3}$ pre každý index~$i$, takže postupnosť $a_1,a_2,\dots$ je naozaj periodická s~periódou~3. Keďže najkratšia perióda nemôže byť~1, je nutne~3, z~čoho už plynie, že~$n$ je nutne deliteľné tromi.

\poznamka
Vyriešením sústavy $ab+1=c$, $bc+1=a$, $ca+1=b$ zistíme, že riešenie $(-1,-1,2)$ sa opakuje vo všetkých postupnostiach spĺňajúcich zadanie.
}

{%%%%%   IMO, priklad 3
Dokážeme, že taký antipascalov trojuholník neexistuje. Nech~$T$ je antipascalov trojuholník obsahujúci každé číslo od $1$ do $1+2+\dots+n$ a~nech $a_1$ je najvyššie položené číslo v~$T$ (\obr). Dve čísla pod $a_1$ označme $a_2$ a~$b_2=a_1+a_2$, dve čísla pod $b_2$ označme $a_3$ a~$b_3=a_1+a_2+a_3$, a~tak ďalej až po posledný riadok, kde máme nejaké $a_n$ a~$b_n=a_1+a_2+\ldots+a_n$, ktoré sú susedmi pod číslom $b_{n-1}=a_1+a_2+\dots+a_{n-1}$. Keďže $a_1,a_2,\dots,a_n$ je $n$ navzájom rôznych prirodzených čísel ktorých súčet (rovný~$b_n$) neprevyšuje najväčšie číslo v~$T$ rovné $1+2+\dots+n$, nutne ide o~nejaké poradie čísel $1,2,\dots,n$.
\inspinsp{imo3.1}{imo3.2}%

Uvážme teraz dva \uv{rovnostranné} podtrojuholníky~$T$, ktorých spodný riadok obsahuje čísla naľavo, respektíve napravo od dvojice $a_n$, $b_n$ (\obr{}). (Jeden z nich môže byť prázdny.) Aspoň jeden z~nich, napríklad~$T'$, má stranu dĺžky~$l \ge \lceil(n-2)/2\rceil$. Keďže~$T'$ má tiež antipascalovskú vlastnosť, obsahuje $l$ navzájom rôznych prirodzených čísel~$a_1', a_2', \ldots, a_l'$, kde $a_1'$ je vrchol a~čísla $a'_k$ a~$b'_k=a_1'+a_2'+\dots+a_k'$ sú dvaja susedia pod $b'_{k-1}$ pre každé $k=2,3,\dots,l$. Keďže všetky $a_k$ ležia mimo $T'$ a~tvoria poradie $1,2,\dots,n$, všetky $a_k'$ sú väčšie ako~$n$. Z~toho vyplýva
$$
\align
b_l' &\ge (n+1)+(n+2)+\dots+(n+l) = \frac{l(2n+l+1)}2\ge\\
&\ge \frac12 \cdot \frac{n-2}2 \left(2n+\frac{n-2}2+1\right)=\frac{5n(n-2)}8,
\endalign
$$
čo je viac ako~$1+2+\dots+n=n(n+1)/2$ pre $n=2018$. Požadovaný antipascalov trojuholník teda naozaj neexistuje.

\poznamka
Horný odhad sa dá vylepšiť vďaka pozorovaniu, že~$b_l' \ne b_n$. To znamená $n(n+1)/2 = b_n > b_l' \ge\lceil (n-2)/2 \rceil (2n + \lceil (n-2)/2 \rceil + 1)/2$, z~čoho sa dá odvodiť $n \le 7$ pre nepárne~$n$ a~$n \leq 12$ pre párne~$n$. Zdá sa, že najväčší antipascalov trojuholník, ktorého prvky sú v~nejakom poradí prirodzené čísla od $1$ do $1+2+\dots+n$, má 5~riadkov.}

{%%%%%   IMO, priklad 4
Dokážeme, že hľadané~$K$ je rovné~100. Políčka $(x,y)$ a~$(x',y')$ majú vzdialenosť~$\sqrt5$ práve vtedy, keď sú čísla $|x-x'|$ a~$|y-y'|$ rovné $1$ a~$2$ v~nejakom poradí. Inak povedané, keď platí, že medzi týmito políčkami existuje ťah šachového koňa.

Uvažujme ofarbenie všetkých políčok šachovnicovým spôsobom tak, že ľavý horný roh je čierny. Platí, že všetky dvojice políčok vzdialených ťahom jazdca majú rôznu farbu. Z~toho vyplýva, že Anna môže ukladať svoje kamene na políčka tej istej farby, a~nikdy sa nestane, že urobí zakázaný ťah. Takýchto políčok je~$(20 \cdot 20)/2 = 200$ (z~každej farby), a~keďže sa s~Borisom striedajú v~ťahoch, tak jej môže modrými kameňmi obsadiť najviac 100~z~týchto políčok, takže v~každom prípade vie Anna položiť aspoň 100~červených kameňov.
\inspinspinsp{imo4.3}{imo4.1}{imo4.2}{\qquad}%

V~ďalšom kroku dokážeme, že Boris vie zabezpečiť, že Anna obsadí prinajlepšom 100 políčok svojimi kameňmi. Jeho stratégia spočíva v~tom, že si všetky políčka šachovnice rozdelí do štvoríc políčok $A$, $B$, $C$, $D$ takých, že existujú ťahy jazdcom v~cykle $A{-}B{-}C{-}D{-}A$. Akonáhle Anna položí svoj kameň na nejaké políčko, napríklad~$A$ (\obr), Borisovi stačí položiť kameň na zodpovedajúce \uv{protiľahlé} políčko, v~našom príklade~$C$. Tým zabezpečí, že z~každej takejto štvorice bude Anninými kameňmi obsadené najviac jedno políčko. Z~toho už bude vyplývať, že Anna celkovo položí najviac 100~svojich kameňov.

Požadované rozdelenie na štvorice pritom naozaj existuje. Najprv rozdelíme štvorec $4 \times 4$ na tieto štvorice (\obr). Uvažovaný štvorec~$20 \times 20$ následne rozdelíme na $5 \cdot 5$ štvorcov s~rozmermi $4 \times 4$, v~ktorých zopakujeme tento vzor (\obr). Tým je teda úloha dokončená a~výsledok je~$K=100$.}

{%%%%%   IMO, priklad 5
\podla{Michala Staníka}
Vezmime ľubovoľné prirodzené číslo $i \ge N$. Uvažovaný výraz je celé číslo pre~$i$, ale aj pre $i+1 > N$. Rozdiel týchto výrazov je rovný
$$
\align
D_i&=\left(\frac{a_1}{a_2} + \frac{a_2}{a_3} + \cdots+ \frac{a_{i-1}}{a_i} + \frac{a_i}{a_{i+1}} + \frac{a_{i+1}}{a_1}\right) - \left(\frac{a_1}{a_2} + \frac{a_2}{a_3} + \cdots + \frac{a_{i-1}}{a_i} + \frac{a_i}{a_1}\right) =\\
&=\frac{a_i}{a_{i+1}} + \frac{a_{i+1}}{a_1}-\frac{a_i}{a_1} = \frac{a_1(a_i-a_{i+1}) + a_{i+1}^2}{a_1a_{i+1}}.
\endalign
$$
Rozdiely~$D_i$ sú všetky celočíselné. Dokážeme, že táto podmienka stačí na to, aby bola postupnosť~$a_i$ od istého člena konštantná.

Keďže $D_i$ je celé číslo, nutne platí $a_1 \mid a_{i+1}^2$. Z~toho máme, že každé prvočíslo~$p$ deliace~$a_1$ delí aj $a_{i+1}^2$, takže aj $a_{i+1}$. Keďže to platí pre ľubovoľné $i \ge N$, všetky čísla $a_{N+1},a_{N+2},\dots$ sú deliteľné~$p$. Vydeľme teda všetky čísla $a_1,a_{N+1},a_{N+2},\dots$ číslom~$p$. Týmto krokom sa zrejme nezmenia hodnoty rozdielov $D_{N+1}, D_{N+2},\dots$. Taktiež sa nezmení ani dokazované tvrdenie hovoriace, že postupnosť je od istého člena konštantná.

V~ďalšom kroku vezmeme ďalšie prvočíslo~$p'$ deliace~$a_1$ (nemusí byť nutne rôzne od~$p$), a~znova odvodíme $p' \mid a_{i+1}$, tentoraz pre~$i \ge N+1$. V~tomto kroku číslom~$p$ delíme teda čísla $a_1, a_{N+2}, a_{N+3}, \dots$ a~zachováme hodnoty rozdielov $D_{N+2},D_{N+3},\dots$

Tento postup opakujeme, až kým sa nedostaneme k~$a_1=1$ a~nejakému indexu~$N_0$ takému, že pre všetky $i \ge N_0$ hodnota~$D_i$ ostala zachovaná (a~teda je celočíselná). Presnejšie, ak na začiatku bolo $a_1=p_1^{\alpha_1} \cdot p_2^{\alpha_2} \dots p_k^{\alpha_k}$, tak~$N_0 = N+\alpha_1+\alpha_2+\dots+\alpha_n$.

Predpokladajme teda, že $a_1=1$, a~pre $i \ge N_0$ sú čísla~$D_i$ celočíselné. Vidíme, že to znamená, že $a_{i+1}\mid a_i$, z~čoho vyplýva $a_{i+1} \le a_i$. Čísla $a_{N_0}, a_{N_0+1}, a_{N_0+2},\dots$ teda tvoria nerastúcu postupnosť prirodzených čísel. Táto postupnosť teda musí byť od určitého člena konštantná, takže tvrdenie je dokázané.}

{%%%%%   IMO, priklad 6
Nech~$E$ je taký bod, že trojuholníky $XAB$ a~$XEC$ sú priamo podobné. Podľa známeho tvrdenia sú potom aj trojuholníky $XAE$ a~$XBC$ podobné. Keďže $|\angle AXB|=|\angle EXC|$, dokazované tvrdenie je ekvivalentné s~tým, že body $E$, $X$, $D$ sú kolineárne (\obr).

Platí $|\angle XCE| + |\angle XCD| = |\angle XBA|+|\angle XAB|<180^\circ$ a~$|\angle XAE| + |\angle XAD| = |\angle XBC|+|\angle XAD|=|\angle XDA|+|\angle XAD|<180^\circ$, čo dokazuje, že~$X$ leží vnútri uhlov $ECD$ a~$EAD$ štvoruholníka $EADC$.
\insp{imo6.1}%

Z~podobností $\triangle XAE \sim \triangle XBC$ a~$\triangle XAB \sim \triangle XEC$ máme $|XA| \cdot |BC| = {|XB| \cdot |AE|}$ a~$|XB| \cdot |CE| = |XC| \cdot |AB|$. Násobenie týchto rovníc spolu s~predpokladom zo zadania $|AB|\cdot|CD|=|BC|\cdot|AD|$ dáva $|XA| \cdot |CE| \cdot |CD| = |AE| \cdot |XC| \cdot |AD|$, alebo ekvivalentne
$$
\frac{|XA| \cdot |DE|}{|AD|\cdot |AE|}=\frac{|XC| \cdot |DE|}{|CD| \cdot |CE|}.
\tag1
$$

Tvrdenie dokončíme na základe nasledujúceho pomocného tvrdenia: {\sl Nech $PQR$ je trojuholník a~$X$ je bod ležiaci v~jeho vnútornom uhle $QPR$ taký, že $|\angle QPX|=|\angle PRX|$. Potom
$$
\frac{|PX| \cdot |QR|}{|PQ| \cdot |PR|}<1
$$
práve vtedy, keď~$X$ leží vnútri trojuholníka $PQR$.}

Predtým, ako toto tvrdenie dokážeme, si všimnime, že s~týmto tvrdením sme už hotoví. Ak by totiž $X$ neležal na $ED$, tak keďže leží vnútri štvoruholníka $EADC$, nutne leží zvonka jedného z~trojuholníkov $EDA$, $EDC$ a~vnútri toho druhého, takže podľa pomocného tvrdenia nemôže platiť rovnosť (1).
\inspinspblizko{imo6.2}{imo6.3}

Teraz dokážeme pomocné tvrdenie: Množina bodov~$X$ ležiacich vnútri uhla $QPR$ takých, že $|\angle QPX|=|\angle PRX|$, je oblúk~$\alpha$ kružnice~$\gamma$ prechádzajúcej~$R$ a~dotýkajúcej sa $PQ$ v~$P$. Nech $\gamma$ pretína priamku~$QR$ znova v~$Y$ (ak sa $\gamma$ dotýka $QR$, tak položme $Y=R$). Podobnosť $\triangle QPY \sim \triangle QRP$ dáva
$$
|PY|=\frac{|PQ|\cdot |PR|}{|QR|}.
$$
Zostáva teda dokázať, že $|PX|<|PY|$ práve vtedy, keď~$X$ leží vnútri trojuholníka $PQR$. Rozoberme 2 prípady:

Ak $Y$ leží vnútri úsečky~$QR$ (\obr), tak zrejme $|\angle RYP|>|\angle QPY|=|\angle PRY|$, takže $|PR|>|PY|$. Kružnica~$\gamma'$ so stredom~$P$ a~polomerom~$|PY|$ teda neobsahuje bod~$R$. Všetky vnútorné body~$X$ podoblúka~$PY$ oblúka~$\alpha$ ležia vnútri $PQR$ a~zároveň vnútri $\gamma'$, takže pre ne platí $|PX|<|PY|$. Zvyšné body~$\alpha$ ležia mimo $PQR$ a~zároveň mimo $\gamma'$, takže pre ne platí opačná nerovnosť $|PX|>|PY|$.

Ak $Y$ leží na polpriamke opačnej k~$RQ$ (\obr), tak zrejme $|\angle PRY|>|\angle QPR|=|\angle PYR|$, takže $|PY|>|PR|$. Kružnica~$\gamma''$ so stredom~$P$ a~polomerom~$|PR|$ teda neobsahuje bod~$Y$. Celý oblúk~$\alpha$ leží vnútri $PQR$ a~zároveň vnútri $\gamma''$, takže pre všetky jeho body~$X$ platí $|PX|<|PR|<|PY|$.
}

{%%%%%   MEMO, priklad 1
...}

{%%%%%   MEMO, priklad 2
...}

{%%%%%   MEMO, priklad 3
...}

{%%%%%   MEMO, priklad 4
...}

{%%%%%   MEMO, priklad t1
...}

{%%%%%   MEMO, priklad t2
...}

{%%%%%   MEMO, priklad t3
...}

{%%%%%   MEMO, priklad t4
...}

{%%%%%   MEMO, priklad t5
...}

{%%%%%   MEMO, priklad t6
...}

{%%%%%   MEMO, priklad t7
...}

{%%%%%   MEMO, priklad t8
...} 