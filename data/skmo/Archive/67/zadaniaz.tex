{%%%%%   Z4-I-1
...}
\podpis{...}

{%%%%%   Z4-I-2
...}
\podpis{...}

{%%%%%   Z4-I-3
...}
\podpis{...}

{%%%%%   Z4-I-4
...}
\podpis{...}

{%%%%%   Z4-I-5
...}
\podpis{...}

{%%%%%   Z4-I-6
...}
\podpis{...}

{%%%%% Z5-I-1
Janko dostal vreckové a~chce si zaň kúpiť niečo dobré.
Keby si kúpil štyri koláče, zvýšilo by mu 0,50€.
Keby si chcel kúpiť päť koláčov, chýbalo by mu 0,60€.
Keby si kúpil dva koláče a~tri šišky, utratil by celé vreckové bezo zvyšku.
Koľko stojí jedna šiška?}
\podpis{Lenka Dedková}

{%%%%% Z5-I-2
Jano mal tri klietky (čiernu, striebornú, zlatú) a~tri zvieratá (morča, potkana a~tchora).
V~každej klietke bolo jedno zviera.
Zlatá klietka stála naľavo od čiernej klietky.
Strieborná klietka stála napravo od klietky s~morčaťom.
Potkan bol v~klietke napravo od striebornej klietky.
Určte, v~ktorej klietke bolo ktoré zviera.}
\podpis{Libuše Hozová}

{%%%%% Z5-I-3
Na obrázku je diagram so siedmimi políčkami.
Nakreslite do neho hviezdičky tak, aby boli splnené všetky nasledujúce podmienky:
\ite{$\bullet$} Hviezdičiek je celkom~21.
\ite{$\bullet$} V~každom políčku je aspoň jedna hviezdička.
\ite{$\bullet$} V~políčkach označených $A$, $B$, $C$ je dokopy 8~hviezdičiek.
\ite{$\bullet$} V~políčkach označených $A$ a~$B$ je dokopy menej hviezdičiek ako v~políčku označenom~$C$.
\ite{$\bullet$} V~políčku označenom $B$ je viac hviezdičiek ako v~políčku označenom~$A$.
\ite{$\bullet$} V~kruhu je celkom 15~hviezdičiek, v~trojuholníku celkom 12~hviezdičiek
a~v~obdĺžniku celkom 14~hviezdičiek.\endgraf%
~\insp{z5-I-3.eps}
}
\podpis{Eva Semerádová}

{%%%%% Z5-I-4
Eva s~Marekom hrali bedminton a~Viktor im počítal výmeny.
Po každých 10~výmenách nakreslil Viktor krížik (X).
Potom namiesto každých 10~krížikov nakreslil krúžok (O) a~prislúchajúcich 10~krížikov zmazal.
Keď Eva a~Marek hru ukončili, mal Viktor nakreslené toto:
$$
\text{OOOXXXXXXX}
$$
Určte, koľko najmenej a~koľko najviac výmen Eva s~Marekom mohli zohrať.}
\podpis{Miroslava Farkas Smitková}

{%%%%% Z5-I-5
Zostrojte ľubovoľnú úsečku~$AS$, potom zostrojte kružnicu~$k$ so stredom v~bode~$S$, ktorá prechádza bodom~$A$.
\ite{1.} Zostrojte na kružnici~$k$ body $E$, $F$, $G$ tak, aby spolu s~bodom~$A$ určovali obdĺžnik $AEFG$.
Nájdite aspoň dve riešenia.
\ite{2.} Zostrojte na kružnici~$k$ body $B$, $C$, $D$ tak, aby spolu s~bodom~$A$ určovali štvorec $ABCD$.\endgraf
}
\podpis{Lucie Růžičková}

{%%%%% Z5-I-6
Na stole ležalo osem kartičiek s~číslami 2, 3, 5, 7, 11, 13, 17, 19.
Fero si vybral tri kartičky.
Sčítal na nich napísané čísla a~zistil, že ich súčet je o~1 väčší ako súčet čísel na zvyšných kartičkách.
Ktoré kartičky mohli zostať na stole?
Určte všetky možnosti.}
\podpis{Libuše Hozová}

{%%%%% Z6-I-1
Anička a~Blanka si napísali každá jedno dvojciferné číslo, ktoré začínalo sedmičkou.
Dievčatá si zvolili rôzne čísla.
Potom každá medzi obe cifry vložila nulu, takže im vzniklo trojciferné číslo.
Od neho každá odčítala svoje pôvodné dvojciferné číslo.
Výsledok ich prekvapil.
Určte, ako sa ich výsledky líšili.}
\podpis{Libuše Hozová}

{%%%%% Z6-I-2
Erika chcela ponúknuť čokoládu svojim trom kamarátkam.
Keď ju vytiahla z~batohu, zistila, že je polámaná ako na obrázku.
(Vyznačené štvorčeky sú navzájom zhodné.)
Dievčatá sa dohodli, že čokoládu ďalej lámať nebudú a~lósom určia, aký veľký kúsok ktorá dostane.
Zoraďte štyri kúsky čokolády od najmenšieho po najväčší.
\insp{z6-I-2.eps}%
}
\podpis{Katarína Jasenčáková}

{%%%%% Z6-I-3
Jano mal 100 rovnakých zaváracích fliaš, z~ktorých si staval trojboké pyramídy.
Najvyššie poschodie pyramídy má vždy jednu fľašu, druhé poschodie zhora predstavuje rovnostranný trojuholník, ktorého strana pozostáva z~dvoch fliaš, atď.
Príklad konštrukcie trojposchodovej pyramídy je na obrázku.
\insp{z6-I-3.eps}%
\ite{1.} Koľko fliaš Jano potreboval na päťposchodovú pyramídu?
\ite{2.} Koľko poschodí mala pyramída, na ktorú bolo použitých čo najviac Janových fliaš?
\endgraf
}
\podpis{Katarína Jasenčáková}

{%%%%% Z6-I-4
Veronika má klasickú šachovnicu s~$8\times 8$ políčkami.
Riadky sú označené ciframi 1 až~8, stĺpce písmenami A až~H.
Veronika položila na políčko~B1 jazdca, s~ktorým možno pohybovať iba tak ako v~šachu.
\ite{1.} Je možné premiestniť jazdca štyrmi ťahmi na políčko~H1?
\ite{2.} Je možné premiestniť jazdca piatimi ťahmi na políčko~E6?
\endgraf\noindent
Ak áno, popíšte všetky možné postupnosti ťahov.
Ak nie, zdôvodnite, prečo to možné nie je.}
\podpis{Katarína Jasenčáková}

{%%%%% Z6-I-5
V~plechovke boli červené a~zelené cukríky.
Cyril zjedol $\frac25$ všetkých červených cukríkov a~Zuzka zjedla $\frac35$~všetkých zelených cukríkov.
Teraz tvoria červené cukríky $\frac38$ všetkých cukríkov v~plechovke.
Koľko najmenej cukríkov mohlo byť pôvodne v~plechovke?
}
\podpis{Lucie Růžičková}

{%%%%% Z6-I-6
Zostrojte ľubovoľnú úsečku~$DS$, potom zostrojte kružnicu~$k$ so stredom v~bode~$S$, ktorá prechádza bodom~$D$.
\ite{1.} Zostrojte rovnostranný trojuholník $DAS$, ktorého vrchol~$A$ leží
na kružnici~$k$.
\ite{2.} Zostrojte rovnostranný trojuholník $ABC$, ktorého vrcholy $B$ a~$C$ tiež ležia na kružnici~$k$.
\endgraf
}
\podpis{Lucie Růžičková}

{%%%%% Z7-I-1
Peter povedal Pavlovi:
\uv{Napíš dvojciferné prirodzené číslo, ktoré má tú vlastnosť, že keď od neho odčítaš dvojciferné prirodzené číslo s~tými istými ciframi napísanými v~opačnom poradí, dostaneš rozdiel~63.}
Ktoré číslo mohol Pavol napísať? Určte všetky možnosti.
}
\podpis{Libuše Hozová}

{%%%%% Z7-I-2
Dané sú dve dvojice rovnobežných priamok $AB\parallel CD$ a~${AC\parallel BD}$.
Bod~$E$ leží na priamke~$BD$, bod~$F$ je stredom úsečky~$BD$, bod~$G$ je stredom úseč\-ky~$CD$ a~obsah trojuholníka $ACE$ je 20\,cm$^2$.
Určte obsah trojuholníka $DFG$.}
\podpis{Vladimíra Semeráková}

{%%%%% Z7-I-3
Zoologická záhrada ponúkala školským skupinám výhodné vstupné: každý piaty žiak dostáva vstupenku zdarma.
Pán učiteľ~6.A spočítal, že ak kúpi vstupné deťom zo svojej triedy, ušetrí za štyri vstupenky a~zaplatí 19{,}95€.
Pani učiteľka~6.B mu navrhla, nech kúpi vstupenky deťom oboch tried naraz, a~tak budú platiť 44{,}10€.
Koľko detí z~6.A~a~koľko detí z~6.B išlo do zoo? (Cena vstupenky v~centoch je celočíselná.)
}
\podpis{Libor Šimůnek}

{%%%%% Z7-I-4
Na stole ležalo šesť kartičiek s~ciframi 1, 2, 3, 4, 5, 6.
Anežka z~týchto kartičiek zložila šesťciferné číslo, ktoré bolo deliteľné šiestimi.
Potom postupne odoberala kartičky sprava.
Keď odobrala prvú kartičku, zostalo na stole päťciferné číslo deliteľné piatimi.
Keď odobrala ďalšiu kartičku, zostalo štvorciferné číslo deliteľné štyrmi.
Keď odoberala ďalej, získala postupne trojciferné číslo deliteľné tromi a~dvojciferné číslo deliteľné dvoma.
Ktoré šesťciferné číslo mohla Anežka pôvodne zložiť?
Určte všetky možnosti.
}
\podpis{Lucie Růžičková}

{%%%%% Z7-I-5
Prokop zostrojil trojuholník $ABC$, ktorého vnútorný uhol pri vrchole~$A$ bol väčší ako $60^\circ$ a~vnútorný uhol pri vrchole~$B$ bol menší ako~$60^\circ$. Juraj narysoval v~polrovine určenej priamkou~$AB$ a~bodom~$C$ bod~$D$, a~to tak, že trojuholník $ABD$ bol rovnostranný.
Potom chlapci zistili, že trojuholníky $ACD$ a~$BCD$ sú rovnoramenné s~hlavným vrcholom~$D$.
Určte veľkosť uhla $ACB$.}
\podpis{Eva Semerádová}

{%%%%% Z7-I-6
Vodník Chaluha nalieval hmlu do rozmanitých, rôzne veľkých nádob, ktoré si starostlivo zoradil na polici.
Pri nalievaní postupoval postupne z~jednej strany, žiadnu nádobu nepreskakoval.
Do každej nádoby sa vojde aspoň deciliter hmly.
Keby nalieval hmlu sedemlitrovou odmerkou, hmla z~prvej odmerky by naplnila presne 11~nádob, hmla z~druhej odmerky by naplnila presne ďalších 12~nádob a~hmla z~tretej odmerky by naplnila presne 7~nádob.
Ak by použil päťlitrovú odmerku, tak hmla z~prvej odmerky by naplnila presne 8~nádob, z druhej presne 10~nádob, z~tretej presne 7~nádob a~zo štvrtej odmerky presne 4~nádoby.
Rozhodnite, či je tridsiata nádoba v~poradí väčšia ako dvadsiata piata.}
\podpis{Karel Pazourek}

{%%%%% Z8-I-1
Vyjadrite číslo milión pomocou čísel obsahujúcich iba cifry 9 a~algebrických operácií plus, mínus, krát, delené, mocnina a~odmocnina.
Určte aspoň tri rôzne riešenia.}
\podpis{Lenka Dedková}

{%%%%% Z8-I-2
V~ostrouhlom trojuholníku $KLM$ má uhol $KLM$ veľkosť~$68^\circ$.
Bod~$V$ je priesečníkom výšok a~$P$ je pätou výšky na stranu~$LM$.
Os uhla $PVM$ je rovnobežná so stranou~$KM$.
Porovnajte veľkosti uhlov $MKL$ a~$LMK$.
}
\podpis{Libuše Hozová}

{%%%%% Z8-I-3
Adelka mala na papieri napísané dve čísla.
Keď k~nim pripísala ešte ich najväčší spoločný deliteľ a~najmenší spoločný násobok, dostala štyri rôzne čísla menšie ako 100.
S~úžasom zistila, že keď vydelí najväčšie z~týchto štyroch čísel najmenším, dostane najväčší spoločný deliteľ všetkých štyroch čísel.
Ktoré čísla mala Adelka napísané na papieri?}
\podpis{Michaela Petrová}

{%%%%% Z8-I-4
Roboti Róbert a~Hubert skladajú a~rozoberajú mlynčeky na kávu.
Pritom každý z~nich mlynček zloží štyrikrát rýchlejšie, ako ho ten druhý rozoberie.
Keď ráno prišli do dielne, niekoľko mlynčekov už tam bolo zložených.
O~9:00 začal Hubert skladať a~Róbert rozoberať, presne o~12:00 Hubert dokončil skladanie mlynčeka a~Róbert rozoberanie iného.
Spolu za túto zmenu pribudlo 27~mlynčekov.
O~13:00 začal Róbert skladať a~Hubert rozoberať, presne o~19:00 dokončil Róbert skladanie posledného mlynčeka a~Hubert rozoberanie iného.
Spolu za túto zmenu pribudlo 120 mlynčekov.
Za ako dlho zloží mlynček Hubert?
Za ako dlho ho zloží Róbert?
}
\podpis{Karel Pazourek}

{%%%%% Z8-I-5
Zhodné obdĺžniky $ABCD$ a~$EFGH$ sú umiestnené tak, že ich zhodné strany sú rovnobežné.
Body $I$, $J$, $K$, $L$, $M$ a~$N$ sú priesečníky predĺžených strán ako na obrázku.
Obsah obdĺžnika $BNHM$ je 12\,cm$^2$, obsah obdĺžnika $MBCK$ je 63\,cm$^2$ a~obsah obdĺžnika $MLGH$ je 28\,cm$^2$.
Určte obsah obdĺžnika $IFJD$.
\insp{z8-I-5.eps}%
}
\podpis{Eva Semerádová}

{%%%%% Z8-I-6
Priamka predstavuje číselnú os a~vyznačené body zodpovedajú číslam $a$, ${-a}$, $a+1$, avšak nie nutne v~tomto poradí.
Zostrojte body, ktoré zodpovedajú číslam 0 a~1.
Preberte všetky možnosti.
\insp{z8-I-6.eps}%
}
\podpis{Michaela Petrová}

{%%%%% Z9-I-1
Vekový priemer všetkých ľudí, ktorí sa zišli na rodinnej oslave, bol rovný počtu prítomných.
Teta Beta, ktorá mala 29~rokov, sa vzápätí ospravedlnila a~odišla.
Aj~po odchode tety Bety bol vekový priemer všetkých prítomných ľudí rovný ich počtu.
Koľko ľudí bolo pôvodne na oslave?
}
\podpis{Libuše Hozová}

{%%%%% Z9-I-2
V~lichobežníku $VODY$ platí, že $VO$ je dlhšou základňou, priesečník uhlopriečok~$K$ delí úsečku~$VD$ v~pomere $3:2$ a~obsah trojuholníka $KOV$ je rovný 13{,}5\,cm$^2$.
Určte obsah celého lichobežníka.
}
\podpis{Michaela Petrová}

{%%%%% Z9-I-3
Roboti Róbert a~Hubert skladajú a~rozoberajú mlynčeky na kávu.
Pritom každý z~nich mlynček zloží štyrikrát rýchlejšie, ako ho sám rozoberie.
Keď ráno prišli do dielne, niekoľko mlynčekov už tam bolo zložených.
O~7:00 začal Hubert skladať a~Róbert rozoberať, presne o~12:00 Hubert dokončil skladanie mlynčeka a~Róbert rozoberanie iného.
Celkom za túto zmenu pribudlo 70~mlynčekov.
O~13:00 začal Róbert skladať a~Hubert rozoberať, presne o~22:00 dokončil Róbert skladanie posledného mlynčeka a~Hubert rozoberanie iného.
Celkom za túto zmenu pribudlo 36~mlynčekov.
Za ako dlho by zložili 360~mlynčekov, keby Róbert aj Hubert skladali spoločne?
}
\podpis{Karel Pazourek}

{%%%%% Z9-I-4
Čísla 1, 2, 3, 4, 5, 6, 7, 8 a~9 sa chystali na cestu vlakom s~tromi vagónmi.
Chceli sa rozsadiť tak, aby v~každom vagóne sedeli tri čísla a~najväčšie z~každej trojice bolo rovné súčtu zvyšných dvoch.
Sprievodca tvrdil, že to nie je problém, a~snažil sa číslam pomôcť.
Naopak výpravca tvrdil, že to nie je možné.
Rozhodnite, kto z~nich mal pravdu.
}
\podpis{Erika Novotná}

{%%%%% Z9-I-5
Vnútri obdĺžnika $ABCD$ ležia body $E$ a~$F$ tak, že úsečky $EA$, $ED$, $EF$, $FB$, $FC$ sú navzájom zhodné.
Strana~$AB$ je dlhá 22\,cm a~kružnica opísaná trojuholníku $AFD$ má polomer 10\,cm.
Určte dĺžku strany~$BC$.
}
\podpis{Lucie Růžičková}

{%%%%% Z9-I-6
Na priamke predstavujúcej číselnú os uvážte navzájom rôzne body zodpovedajúce číslam $a$, $2a$, $3a+1$ vo všetkých možných poradiach.
Pri každej možnosti rozhodnite, či je také usporiadanie možné.
Ak áno, uveďte konkrétny príklad, ak nie, zdôvodnite prečo.
\insp{z9-I-6.eps}%
}
\podpis{Michaela Petrová}

{%%%%%   Z4-II-1
...}
\podpis{...}

{%%%%%   Z4-II-2
...}
\podpis{...}

{%%%%%   Z4-II-3
...}
\podpis{...}

{%%%%% Z5-II-1
Na obrázku je znázornených päť výbehov časti zoo.
Každý výbeh obýva jeden z~piatich druhov zvierat.
Pritom vieme, že
\itemitem{$\bullet$} výbeh žiráf má päť strán,
\itemitem{$\bullet$} výbeh opíc nesusedí ani s~výbehom nosorožcov, ani s~výbehom žiráf,
\itemitem{$\bullet$} výbeh levov má rovnaký počet strán ako výbeh opíc,
\itemitem{$\bullet$} tulene majú vo výbehu jazierko.

Určte, ktoré zvieratá sú v~ktorom výbehu.
\ifobrazkyvedla\else\insp{z5-II-1.eps}\fi%
}
\podpis{Erika Novotná}

{%%%%% Z5-II-2
Na stole ležalo päť kartičiek s~navzájom rôznymi kladnými celými číslami.
Marek spočítal, že najväčšie z~čísel na kartičkách je o~8 väčšie ako najmenšie číslo.
Adam vzal zo stola kartičky s~najmenšími dvoma číslami a~spočítal, že súčin týchto dvoch čísel je rovný~12.
Dominik potom spočítal, že súčet čísel na zvyšných troch kartičkách je rovný~25.
Ktoré čísla mohli byť napísané na kartičkách?
Nájdite tri riešenia.
}
\podpis{Lucie Růžičková}

{%%%%% Z5-II-3
Zostrojte štvorec $ABCD$ so stranou dĺžky 6\,cm a~priesečník jeho uhlopriečok označte~$S$.
Zostrojte bod~$K$ tak, aby spolu s~bodmi $S$, $B$, $C$ tvoril štvorec $BKCS$.
Zostrojte bod~$L$ tak, aby spolu s~bodmi $S$, $A$, $D$ tvoril štvorec $ASDL$.
Zostrojte úsečku~$KL$, priesečník úsečiek $KL$ a~$AD$ označte~$X$, priesečník úsečiek $KL$ a~$BC$ označte~$Y$.
Zo zadaných údajov vypočítajte dĺžku lomenej čiary $KYBAXL$.}
\podpis{Lucie Růžičková}

{%%%%% Z6-II-1
Janko dostal na Vianoce knihu, ktorú hneď na Štedrý deň 24.~decembra začal čítať. Čítal denne rovnaký počet strán, a~to až do 31.~januára v~novom roku. Toho dňa zistil, že zatiaľ prečítal 78~strán, čo bola práve tretina knihy. Súčasne zistil, že ak by chcel knihu dočítať práve v~deň svojich narodenín, musel by počínajúc zajtrajškom každý deň prečítať o~štyri strany viac ako doposiaľ. Určte, kedy má Janko narodeniny.}
\podpis{Miroslava Farkas Smitková}

{%%%%% Z6-II-2
Evička mala stavebnicu s~deviatimi dielikmi, ktoré boli označené 1, 2, 3, 4, 5, 6, 7, 8 a~9. Časom sa jej podarilo všetky dieliky stratiť, a~to nasledujúcim spôsobom:
\itemitem{$\bullet$} najskôr stratila štyri dieliky označené nepárnymi číslami,
\itemitem{$\bullet$} potom stratila dieliky so súčinom čísel 24,
\itemitem{$\bullet$} nakoniec stratila posledné dva dieliky, na ktorých boli párne čísla.
\endgraf\noindent
Zistite, ktoré čísla mohli byť napísané na posledných dvoch dielikoch. Nájdite dve riešenia.}
\podpis{Erika Novotná}

{%%%%% Z6-II-3
Zostrojte štvorec $ABCD$ so stranou dĺžky 6\,cm. Zostrojte priamku~$p$ rovnobežnú s~uhlopriečkou~$AC$ a~prechádzajúcu bodom~$D$. Zostrojte obdĺžnik $ACEF$ tak, aby vrcholy $E$ a~$F$ ležali na priamke~$p$. Zo zadaných údajov vypočítajte obsah obdĺžnika $ACEF$.}
\podpis{Lucie Růžičková}

{%%%%% Z7-II-1
Florián premýšľal, akú kyticu nechá mamičke uviazať na Deň matiek. V~kvetinárstve podľa cenníka spočítal, že či kúpi 5~klasických gerber alebo 7~minigerber, kytica bude stáť po doplnení ozdobnej stuhy rovnako, a~to 29,50 eur. Ak by však kúpil iba 2~minigerbery a~1~klasickú gerberu bez ďalších doplnkov, zaplatil by 10,20~eur. Koľko stojí 1~minigerbera?}
\podpis{Libor Šimůnek}

{%%%%% Z7-II-2
Na stole ležalo šesť kartičiek s~ciframi 1, 2, 3, 4, 5 a~6. Kamila z~troch kartičiek zložila trojciferné číslo, ktoré bolo väčšie ako 500 a~bolo deliteľné štyrmi. Filip zo zvyšných troch kartičiek zložil trojciferné číslo deliteľné tromi aj piatimi. Kamila potom obe trojciferné čísla sčítala a~Filip si všimol, že tento súčet je trojciferné číslo, ktoré sa číta rovnako zľava ako sprava. Ktoré čísla mohli Kamila s~Filipom zložiť? Určte všetky možnosti.}
\podpis{Lucie Růžičková}

{%%%%% Z7-II-3
Zostrojte kružnicu so stredom~$S$ a~polomerom 3\,cm. Zostrojte dva navzájom kolmé priemery $AC$ a~$BD$ tejto kružnice. Zostrojte rovnoramenné trojuholníky $ABK$, $BCL$, $CDM$, $DAN$ tak, aby:
\itemitem{$\bullet$} základňou každého trojuholníka bola strana štvoruholníka $ABCD$,
\itemitem{$\bullet$} základňa každého trojuholníka bola zhodná s~výškou na túto stranu,
\itemitem{$\bullet$} žiadny trojuholník neprekrýval štvoruholník $ABCD$.
\endgraf\noindent
Zo zadaných údajov vypočítajte obsah mnohouholníka $AKBLCMDN$.}
\podpis{Marie Krejčová}

{%%%%% Z8-II-1
Jožko si do zošita napísal nasledujúcu úlohu:
$$
M+A+M+R+A+D+M+A+T+E+M+A+T+I+K+U=
$$
Potom nahradzoval písmená ciframi od 1 do 9, a~to tak, že rôzne písmená nahradzoval rôznymi ciframi a~rovnaké rovnakými. Aký najväčší súčet mohol Jožko dostať? A~mohol Jožko dostať súčet 50?

Časom sa Jožkovi podarilo vytvoriť súčet~59. Ktorá cifra mohla v~takom prípade zodpovedať písmenu~$T$? Určte všetky možnosti.
}
\podpis{Erika Novotná}

{%%%%% Z8-II-2
Kocka s~hranou 12\,cm bola rozdelená na menšie navzájom zhodné kocôčky tak, že súčet všetkých ich povrchov bol osemkrát väčší ako povrch pôvodnej kocky. Určte, koľko bolo malých kocôčok a~aké dlhé boli ich hrany.}
\podpis{Marta Volfová}

{%%%%% Z8-II-3
V~rovnostrannom trojuholníku $ABC$ so stranou dĺžky 8\,cm je bod~$D$ stred strany~$BC$ a~bod~$E$ je stred strany~$AC$. Bod~$F$ leží na úsečke~$BC$ tak, že obsah trojuholníka $ABF$ je rovnaký ako obsah štvoruholníka $ABDE$. Vypočítajte dĺžku úsečky~$BF$.}
\podpis{Lucie Růžičková}

{%%%%% Z9-II-1
Štefka a~Terezka dostali bonboniéru, v~ktorej bolo 35~čokoládových cukríkov. Prvý deň zjedla Terezka $\frac25$ toho, čo zjedla v~tento deň Štefka.
Druhý deň zjedla Štefka $\frac34$~toho, čo zjedla v~tento deň Terezka. Na konci druhého dňa bola bonboniéra prázdna. Koľko cukríkov celkom zjedla Terezka, keď vieme,
%že rozdiel medzi počtom cukríkov zjedených Terezkou a~počtom cukríkov zjedených Štefkou bol najmenší možný?
že počet cukríkov zjedených Terezkou a~počet cukríkov zjedených Štefkou sa líšia o~najmenšiu možnú hodnotu?
Každý deň zjedlo každé z~dievčat aspoň jeden cukrík a žiadny cukrík nebol delený na časti.
}
\podpis{Libuše Hozová}

{%%%%% Z9-II-2
Vo štvoruholníku $ABCD$ sú strany $AB$ a~$CD$ rovnobežné, pričom strana~$AB$ je dvakrát dlhšia ako strana~$CD$.
Bod~$S$ je priesečníkom uhlopriečok štvoruholníka $ABCD$ a~trojuholníky $ABS$ a~$CDS$ sú oba rovnostranné.
Bod~$E$ je taký bod úsečky~$BS$, že veľkosť uhla $ACE$ je~$30\st$.
Určte pomer obsahov štvoruholníka $ABCD$ a~trojuholníka $EBC$.}
\podpis{Eva Semerádová}

{%%%%% Z9-II-3
Štyri kamarátky našli v~učebnici nasledujúcu poznámku:
$$
\text{Vieme, že $\sqrt{a\cdot b}=99\sqrt2$ a~že $\sqrt{a\cdot b\cdot c}$ je prirodzené číslo.}
$$
Teraz premýšľajú a~dohadujú sa, čo môžu povedať o~čísle~$c$:
\item Anna: \uv{Určite to nemôže byť $\sqrt2$.}
\item Dana: \uv{Môže to byť napr. 98.}
\item Hana: \uv{Môže to byť akékoľvek párne číslo.}
\item Jana: \uv{Také číslo je určite len jedno.}

Rozhodnite, ktoré z~dievčat má (majú) pravdu, a~vysvetlite prečo.}
\podpis{Eva Semerádová}

{%%%%% Z9-II-4
Kváder s~rozmermi $20\cm\times 30\cm\times 40\cm$ je položený tak, že hrana dĺžky $20\cm$ leží na stole a~hrana dĺžky $40\cm$ zviera so stolom uhol~$30\st$.
Kváder je čiastočne naplnený vodou, ktorá omáča hornú stenu s~rozmermi $20\cm\times 40\cm$ z~jednej štvrtiny.
Určte objem vody v~kvádri.
\ifobrazkyvedla\else\insp{z9-II-4.eps}\fi%
}
\podpis{Marie Krejčová}

{%%%%% Z9-III-1
Vo finále súťaže spoločenských tancov mal každý z~dvadsiatich piatich porotcov ohodnotiť päť tanečných párov známkami 1 až 5 ako v~škole, pričom každú známku musel použiť práve raz. Po súťaži bola zverejnená tabuľka s~priemermi známok pre každý pár:
\insp{67-z9-iii-1}
Vzápätí sa zistilo, že práve jeden z~priemerov je v~tabuľke uvedený zle. Zistite, ktorý priemer je chybný, a~opravte ho.

Jedným z~porotcov bol strýko tanečníka páru~E. Páry však hodnotil čestne a~bez zaujatosti nasledujúcim spôsobom: 1 pre D, 2 pre E, 3 pre B, 4 pre C a~5 pre A. Napriek tomu mu po vyhlásení opravených výsledkov vŕtalo v~hlave, či by býval mohol nepoctivým oznámkovaním posunúť pár E na prvé miesto. Zistite, či to mohol dokázať.}
\podpis{Libor Šimůnek}

{%%%%% Z9-III-2
Každé z~čísel $a$ a~$b$ sa dá vyjadriť ako súčin troch prvočísel menších ako 10. Každé prvočíslo menšie ako 10 je prítomné v~rozklade aspoň jedného z~čísel $a$ a~$b$. Najväčší spoločný deliteľ čísel $a$ a~$b$ je rovný najväčšiemu spoločnému deliteľovi čísel $\frac{a}{15}$ a~$b$ a~súčasne
dvojnásobku najväčšieho spoločného deliteľa čísel $a$ a~$\frac{b}4$. Určte čísla $a$ a~$b$.}
\podpis{Eva Semerádová}

{%%%%% Z9-III-3
Na tajomnom ostrove žijú dva druhy domorodcov: jedni hovoria výlučne pravdu (poctivci), druhí stále klamú (klamári). Výskumníci tam stretli niekoľko skupín domorodcov a~vždy sa každého zo skupiny spýtali, koľko je v~ich skupine poctivcov.
\itemitem{$\bullet$} Ako odpovede od jednej štvorčlennej skupiny dostali všetky čísla rovnaké. %Ako odpovede od jednej skupiny dostali štyri rovnaké čísla.
\itemitem{$\bullet$} Ako odpovede od druhej skupiny dostali čísla 0, 1, 3, 3, 3, 4.

Koľko poctivcov mohlo byť v~jednej a~koľko v~druhej skupine? Určte všetky možnosti.}
\podpis{Marta Volfová}

{%%%%% Z9-III-4
V~kosoštvorci $ABCD$ so stranou dĺžky 4\,cm a~uhlopriečkou~$AC$ dĺžky 4\,cm je bod~$K$ stredom strany~$BC$. Ďalej sú zostrojené body $L$ a~$M$, ktoré tvoria spolu s~bodmi $D$ a~$K$ štvorec $KLMD$. Vypočítajte obsah štvorca $KLMD$.}
\podpis{Lucie Růžičková}

