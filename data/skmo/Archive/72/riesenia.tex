{%%%%%   A-I-1
Keďže $\floor{y}$ a $2022$ sú celé čísla,
z~rovnice $2x+\floor{y}=2022$ vyplýva, že číslo~$2x$ je tiež
celé, teda platí $\floor{2x}=2x$. Tým pádom zo zadanej sústavy
eliminujeme neznámu $x$, keď odpočítame prvú rovnicu od druhej.
Dostaneme
$$
3y-\floor{y}=1.
\tag1
$$
Vďaka (1) je $3y$ celé číslo, takže má (podľa svojho zvyšku po delení
tromi) jedno z~vyjadrení $3k$, $3k+1$ alebo $3k+2$, kde $k$ je celé
číslo. Odtiaľ (po vydelení tromi) vychádza, že pre číslo $y$ platí
buď $y=k$, alebo $y=k+\frac13$, alebo $y=k+\frac23$,
pričom $k=\floor{y}$. Tieto tri možnosti teraz rozoberieme.
\item{$\triangleright$} V~prípade $y=k$ je (1) rovnica $3k-k=1$ s~neceločíselným riešením
$k=\frac12$.
\item{$\triangleright$} V~prípade $y=k+\frac13$ je (1) rovnica $(3k+1)-k=1$ s~riešením
$k=0$, ktorému zodpovedá $y=\frac13$. Pôvodná sústava rovníc je potom
zrejme splnená práve vtedy, keď $2x=2022$, čiže $x=1011$.
\item{$\triangleright$} V~prípade $y=k+\frac23$ je (1) rovnica $(3k+2)-k=1$
s~neceločíselným riešením $k=\m\frac12$.

   \Zav
Jediným riešením zadanej sústavy rovníc je dvojica $(x,y)=(1011,\frac13)$.

   \Pozn
Odvodenú rovnicu (1) možno riešiť aj tak, že všeobecné reálne číslo
$y$ zapíšeme v tvare $y=k+r$, kde $k=\floor{y}$ a číslo
$r\in\langle0,1)$ je tzv. zlomková časť čísla $y$. Dosadením do
(1) dostaneme rovnicu
$$
3(k+r)-k=1,\quad\hbox{čiže}\quad 2k=1-3r.
$$
Keďže $2k$ je celé číslo deliteľné dvoma a číslo $1-3r$ zrejme
leží v~intervale $({-2},1\rangle$, rovnosť týchto čísel nastane
v~jedinom prípade, keď $2k=1-3r=0$, čiže $k=0$ a $r=\frac13$, t.\,j.
$y=\frac13$.

  \Jres
Slovná definícia, ktorá je pripojená k~zadaniu úlohy, nám
hovorí, že $\floor{a}$ je celé číslo, pre ktoré platí $\floor{a}\leqq
a$ a zároveň $\floor{a}+1>a$. Celé číslo $\floor{a}$ tak spĺňa
odhady $a-1<\floor{a}\leq a$, platné pre každé reálne číslo $a$.
Podľa nich z~prvej rovnice danej sústavy dostávame
$$
{2022\leq 2x+y<2023},
\tag2
$$
podobne z druhej rovnice vychádza
$$
{2023\leq 3y+2x< 2024}.
\tag3
$$

Získané nerovnosti môžeme skombinovať dvoma spôsobmi.
Spojením druhej časti~(2) s~prvou časťou~(3) dostaneme
${2x+y <2023\leq 3y+2x}$, odkiaľ z~porovnania krajných výrazov
vyplýva $y>0$. Ak upravíme prvú časť~(2)
na ${2024\leq 2x+y+2}$, tak v spojení s~druhou časťou~(3)
dostaneme ${3y+2x<2024\leq 2x+y+2}$, takže tentoraz
z~porovnania krajných výrazov vyplýva $y<1$.

Dokopy nám vyšlo $0<y<1$, takže platí $\floor{y}=0$.
Vďaka tomu sa prvá rovnica zo zadania redukuje na tvar $2x=2022$,
ktorému vyhovuje jediné $x=1011$. Jeho dosadením do druhej
zadanej rovnice dostaneme rovnicu $3y+2022=2023$
s~jediným riešením $y=\frac13$, ktoré skutočne
spĺňa podmienku $\floor{y}=0$ použitú v~prvej rovnici.
Dvojica $(x,y) = (1011,\frac13)$ je preto jediným riešením
zadanej sústavy rovníc.

   \Pozn
Odvodenie rovnosti $\floor{y}=0$ je možné urýchliť tak, že
prvú zadanú rovnicu odpočítame od druhej a výsledok toho odčítania
zapíšeme v tvare
$$
2y=1+\bigl(2x-\floor{2x}\bigr)-\bigl(y-\floor{y}\bigr).
$$
Keďže v oboch okrúhlych zátvorkách napravo sú čísla z intervalu
$\langle0,1)$, má zrejme celá pravá strana hodnotu z~intervalu $(0,2)$.
Platí teda $0<2y<2$, čiže $0<y<1$, odkiaľ už vyplýva $\floor{y}=0$.

\návody
V~obore reálnych čísel riešte rovnicu $\floor{3x+5} = 10$.
[$x\in\left\langle\frac53, 2\right)$. Reálne číslo $x$ spĺňa danú
rovnicu práve vtedy, keď $10\leq 3x+5 < 11$. Všetky riešenia tejto sústavy
dvoch nerovníc tvoria interval $\left\langle\frac53, 2\right)$.]

V~obore reálnych čísel riešte sústavu rovníc $x + \floor{2y} =
8$, $\floor{3x}-y = 3$.
[$(x, y) = (2,3)$. Podľa prvej rovnice je číslo $x$ celé,
podľa druhej je aj číslo~$y$ celé. Tým pádom $\floor{2y}=2y$
a $\floor{3x}=3x$, takže máme sústavu $x+2y=8$, $3x-y=3$.
Tá má jediné riešenie $(x, y) = (2,3)$, čo je aj riešenie pôvodnej sústavy (lebo obe čísla $x$,~$y$ vyšli celé).]

\D
V~obore reálnych čísel riešte sústavu rovníc $3x + \floor{y} =
10$, $\floor{4x} + x + y = 17$.
[Dve riešenia $(x,y)=\left(\frac{10}{3},\frac23\right)$ a
$(x,y)=\left(\frac{11}{3}, -\frac23\right)$.
Podľa prvej rovnice je číslo~$3x$ celé, podľa druhej je
aj číslo $x+y$ celé. Tým pádom nastane jeden z troch prípadov:
a)~Číslo $x$ je celé. Potom aj čísla $y$ a $4x$ sú celé. Dostávame sústavu
$3x+y=10$, $5x+y=17$, ktorá však nemá riešenie v~oboru celých čísel.
b)~Číslo $x$ je tvar $x = x'+\frac13$, kde $x'$ je celé číslo. Potom
$y=y'+\frac23$ pre nejaké celé číslo $y'$ a $\floor{4x}=4x'+1$.
Dostaneme tak sústavu $3x'+y'=9$, $5x'+y'=15$ s~jediným riešením
$(x',y')=(3,0)$, ktorému zodpovedá riešenie
$(x, y)=\left(\frac{10}{3},\frac23\right)$
pôvodnej sústavy.
c)~Číslo $x$ má tvar $x = x'+\frac23$, kde $x'$ je celé číslo. Potom
$y=y'+\frac13$ pre nejaké celé číslo $y'$ a $\floor{4x}=4x'+2$. Tentoraz
nám vyjde sústava $3x'+y'=8$, $5x'+y'=14$ s~jediným riešením $(x', y')=(3, {-1})$,
ktorému zodpovedá riešenie $(x, y)=\left(\frac{11}{3}, {-\frac23}\right)$
pôvodnej sústavy.]

V~obore reálnych čísel riešte sústavu rovníc $\floor{x+y} = x-y$,
$\floor{5y+x} = 5y-x$.
[Dve riešenia $(x,y)=(0,0)$ a $(x,y)=\left(\frac14,
\frac14\right)$. Keďže obe čísla $x-y$ a $5y-x$ sú celé,
ich súčet rovný $4y$ je tiež celé číslo. Preto zo zadaných
rovníc vyplýva $5y-x=\floor{5y+x}=\floor{4y+(y+x)}=4y+\floor{y+x}=
4y+(x-y)=3y+x$, t.\,j. $5y-x=3y+x$, odkiaľ nutne $y=x$. Pôvodnú sústavu
potom možno zapísať ako dvojicu rovníc $\floor{2x} = 0$ a~$\floor{6x}=4x$.
Tejto sústave vyhovujú práve tie reálne $x$, pre ktoré súčasne platí $0\leqq 2x<1$,
$4x\leqq 6x<4x+1$ a pritom číslo $4x$ je celé. Pretože všetky
nerovnice z~poslednej vety sú splnené len pre
$x\in\langle 0,\frac12)$, vyhovujú práve hodnoty $x=0$ a $x=\frac14$.]

\endnávod

}

{%%%%%   A-I-2
Keďže úsečky $AB'$, $AC'$ majú podľa zadania rovnakú dĺžku $|AB|+|AC|$,
je $AB'C'$ rovnoramenný trojuholník so základňou $B'C'$. Znamená to, že
os úsečky~$B'C'$ splýva s~osou uhla $BAC$ (\obr).
Nech $S\ne A$ je priesečník tejto osi
s~kružnicou opísanou trojuholníku~$ABC$. Ak dokážeme, že $S$ je stred
kružnice opísanej trojuholníku~$AB'C'$, budeme s~riešením hotoví. Keďže
bod $S$ leží na osi strany $B'C'$, máme $|SB'|=|SC'|$.
Zostáva preto dokázať, že aj $|SA|=|SC'|$.
\insp{a72i.21}%

Zo zhodnosti obvodových uhlov $SAB$ a $SAC$ vyplýva, že bod $S$ je
stredom oblúka~$BC$, a~teda $|BS|=|CS|$. Okrem toho
z~tetivového štvoruholníka $ABSC$ máme
$|\angle ACS| = 180^\circ - |\angle SBA| =|\angle C'BS|$.
Dokopy s~rovnosťou $|CA| = |BC'|$ dostávame, že trojuholníky
$SAC$ a $SC'B$ sú zhodné podľa vety $sus$, a preto skutočne
platí $|SA| = |SC'|$.

  \Jres
Definujme bod $S$ ako v~prvom riešení. Tentoraz potrebnú
rovnosť $|SA| = |SC'|$ overíme, keď ukážeme, že bod $S$
leží na osi úsečky $AC'$.

V špeciálnom prípade, keď platí $|AB|=|AC|$, je stredom úsečky $AC'$
bod~$B$ (podľa konštrukcie bodu $C'$); stačí teda overiť, že
uhol $ABS$ je potom pravý. To však vyplýva z~toho, že tetivový štvoruholník
$ABSC$ je vtedy zložený z dvoch zhodných trojuholníkov $ABS$ a $ACS$, má teda uhly
pri protiľahlých vrcholoch $B$ a $C$ zhodné, a teda pravé.

V prípade, keď $|AB|\ne|AC|$, môžeme bez ujmy na všeobecnosti
predpokladať, že platí $|AB|>|AC|$ ako v~\obr.
V~ňom $P$ a $Q$ sú kolmé priemety bodu~$S$ postupne na priamky
$AB$, resp. $AC$. Vďaka nášmu predpokladu $|AB|>|AC|$ leží bod~$P$
vnútri úsečky~$AB$, zatiaľ čo bod~$Q$ leží na polpriamke opačnej
k~polpriamke~$CA$. V~súlade s~úvodným odsekom budeme dokazovať,
že bod~$P$ je stredom úsečky~$AC'$.
\insp{a72i.22}%


Z~tetivového štvoruholníka $ABSC$ vyplýva
$|\angle SBP| =|\angle SBA| =180^\circ - |\angle SCA| = |\angle
SCQ|$, t.\,j. vyznačené uhly $SBP$ a $SCQ$ sú zhodné. Vďaka pravým
uhlom $BPS$ a~$CQS$ sú zhodné aj uhly $PSB$ a $QSC$.
Okrem toho máme $|PS| = |QS|$, pretože $S$ leží na osi uhla $C'AB'$.
Dostávame tak, že trojuholníky $PBS$ a $QCS$ sú zhodné podľa
vety $usu$. Odtiaľ vyplýva rovnosť $|BP|=|CQ|$.
Navyše zo zhodných pravouhlých trojuholníkov $ASP$ a $ASQ$ ešte máme $|AP|=|AQ|$,
takže spolu vychádza
$$
|AP|=|AQ|=|AC|+|CQ|=|C'B|+|BP|=|C'P|.
$$
Bod $P$ je teda skutočne stredom úsečky $AC'$, ako sme mali
dokázať.

   \Jres
Tentoraz označíme $S$ stred kružnice opísanej trojuholníku~$AB'C'$ (\obr)
a budeme dokazovať, že body $A$, $B$, $S$, $C$ ležia na
jednej kružnici. Podľa úvodu prvého riešenia vieme, že $AB'C'$
je rovnoramenný trojuholník so základňou~$B'C'$, teda stred $S$
kružnice jemu opísanej leží na osi uhla $C'AB'$, čiže $BAC$.
Body $B$ a $C$ preto ležia v~opačných polrovinách s~hraničnou
priamkou $AS$, teda nám
stačí overiť rovnosť $|\angle ABS| = 180^\circ - |\angle ACS|$.
\insp{a72i.23}%

Z~rovností $|AC'| = |AB'|$ a $|AS| = |C'S| = |B'S|$ vyplýva, že
trojuholníky $C'AS$ a $AB'S$ sú rovnoramenné a zhodné podľa vety $sss$.
Trojuholník $C'AS$ sa tak v~otočení so stredom $S$ o~orientovaný
uhol $C'SA$ zobrazí na trojuholník $AB'S$. Keďže $B$ leží na strane $C'A$,
$C$ leží na strane $AB'$ a pritom podľa zadania platí $|C'B| = |AC|$,
spomínané otočenie zobrazuje $B$ na $C$, a teda uhol $C'BS$ na uhol $ACS$.
Preto platí $|\angle C'BS| = |\angle ACS|$, odkiaľ už vyplýva
$|\angle ABS| = 180^\circ - |\angle C'BS| = 180^\circ - |\angle
ACS|$, ako sme sľúbili ukázať.

\návody
Dokážte, že konvexný štvoruholník $ABCD$ je tetivový (t.\,j. jeho
vrcholy ležia na jednej kružnici) práve vtedy, keď platí $|\angle
ABD| = |\angle ACD|$.
[a)~Nech vrcholy $A$, $B$, $C$, $D$ ležia na kružnici so
stredom $S$. Podľa vety o~obvodovom a~stredovom uhle potom platí
$|\angle ABD| = \frac12|\angle ASD| = |\angle ACD|$. b) ~Nech
naopak platí $|\angle ABD| = |\angle ACD|$. Označme $S_1$, $S_2$
stredy kružníc opísaných postupne trojuholníkom $ABD$, $ACD$.
Oba tieto stredy ležia na osi úsečky $AD$ a v~rovnakej polrovine
s~hraničnou priamkou $AD$. Podľa vety o~obvodovom a stredovom uhle
navyše platí $|\angle AS_1D| = 2|\angle ABD| = 2|\angle
ACD| = |\angle AS_2D|$. Dokopy už dostávame, že $S_1=S_2$, takže
kružnice opísané trojuholníkom $ABD$, $ACD$ splývajú.]

Dokážte, že konvexný štvoruholník $ABCD$ je tetivový
práve vtedy, keď súčet veľkostí uhlov $ABC$ a~$ADC$ je $180^\circ$.
[a)~Nech vrcholy $A$, $B$, $C$, $D$ ležia na kružnici so stredom $S$.
Konvexný a nekonvexný uhol $ASC$ sa dopĺňajú do uhla $360^\circ$.
Súčet veľkostí týchto dvoch stredových uhlov je rovný
dvojnásobku súčtu veľkostí obvodových uhlov $ABC$ a~$ADC$,
ktorý sám je teda rovný $180^\circ$. b) Nech naopak
platí $|\angle ABC| + |\angle ADC| = 180^\circ$. V~prípade, keď
oba uhly $ABC$, $ADC$ sú pravé, vyplýva potrebný záver
z~Tálesovej vety. V~opačnom prípade môžeme predpokladať, že napr.
uhol $ABC$ je ostrý a uhol $ADC$ je tupý. Potom vo vnútri polroviny
$ACB$ leží ako stred $S_1$ kružnice opísanej trojuholníku $ABC$,
tak aj stred~$S_2$ kružnice opísanej trojuholníku $ADC$, pritom oba konvexné uhly
$AS_1C$ a $AS_2C$ majú postupne veľkosti $2|\angle ABC|$ a
$360^\circ-2|\angle ADC|$, ktoré sú vďaka predpokladu rovnaké.
Navyše oba body $S_1$, $S_2$ ležia na osi úsečky $AC$, takže spolu
dostávame $S_1 = S_2$, a~teda kružnice opísané trojuholníkom $ABC$, $ADC$
splývajú.]

Dokážte {\it tvrdenie \uv{o~Švrčkovom bode}\/}:
V~ľubovoľnom trojuholníku $ABC$ prechádza os vnútorného uhla $BAC$ stredom
toho oblúka $BC$ kružnice opísanej trojuholníku~$ABC$, na ktorom neleží vrchol $A$.
[Označme $S\ne A$ druhý priesečník osi uhla~$BAC$ s~kružnicou
opísanou trojuholníku $ABC$. V~tetivovom štvoruholníku $ABSC$
platí $|\angle CBS| = |\angle CAS| = |\angle BAS| = |\angle BCS|$.
To znamená, že $BSC$ je rovnoramenný trojuholník so základňou $BC$,
teda~$S$ je stred príslušného oblúka $BC$. Inak je možné využiť
všeobecnejšie tvrdenie: dva obvodové uhly v~tej istej kružnici sú zhodné
práve vtedy, keď sú zhodné oblúky, ktorým tieto obvodové uhly
zodpovedajú.]

\D
Dokážte {\it tvrdenie \uv{o troch prstoch}\/}:
V~danom trojuholníku $ABC$ označme~$I$
stred kružnice vpísanej a $S$ stred toho oblúka $BC$
kružnice opísanej trojuholníku~$ABC$, na ktorom neleží vrchol $A$.
Potom platí $|SB|=|SI|=|SC|$.
[Stačí zrejme dokázať len jednu rovnosť $|SB|=|SI|$.
Pri štandardnom označení veľkostí vnútorných uhlov
trojuholníka $ABC$ platí
$$
|\uhol SBI| = |\uhol SBC| + |\uhol CBI| = |\uhol SAC| +
\beta/2 = \alpha/2 + \beta/2.
$$
Keďže $SIB$ je vonkajší uhol trojuholníka $ABI$, platí
tiež
$$
|\uhol SIB| = |\uhol IAB|+|\uhol ABI| = \alpha/2+\beta/2.
$$
Trojuholník $SIB$ tak skutočne má rovnaké ramená $SB$ a $SI$.]

Dokážte, že os vonkajšieho uhla pri vrchole $A$ ľubovoľného
trojuholníka $ABC$ prechádza stredom toho oblúka $BC$ kružnice
opísanej trojuholníku $ABC$, na ktorom leží vrchol $A$.
[Označme $N\ne A$ druhý priesečník osi vonkajšieho uhla pri vrchole $A$
s~kružnicou opísanou (v~prípade $|AB|=|AC|$, keď sa táto os
kružnice opísanej iba dotýka, je tvrdenie úlohy zrejmé).
Uvažujme tiež bod $S$ z~úlohy~N3. Keďže $S$ leží na osi
vnútorného uhla, $N$ na osi vonkajšieho uhla a tieto dve osi sú
navzájom kolmé, platí $|\angle SAN| = 90^\circ$. Podľa
Tálesovej vety je potom $SN$ priemerom kružnice opísanej.
Keďže $S$ je pritom stred jej oblúka $BC$
neobsahujúceho bod~$A$, $N$ je stred druhého oblúka $BC$.]

Je daný ostrouhlý trojuholník $ABC$. Vnútri strany $AB$ leží
bod $D$ a na polpriamke opačnej k~$CA$ leží bod $E$ tak, že
$|BD| = |CE|$. Dokážte, že kružnice opísané trojuholníkom $ABC$
a $ADE$ majú okrem bodu $A$ ešte ďalší spoločný bod na osi uhla $BAC$.
[Na polpriamky opačné k~$CA$ a $BA$ dokreslíme postupne body
$B'$ a $C'$ určené rovnosťami $|B'C| = |AB|$ a $|C'B| =|AC|$.
Podľa výsledku súťažnej úlohy stred kružnice opísanej
$\triangle AB'C'$ leží na kružnici opísanej
$\triangle ABC$.
Tento výsledok môžeme uplatniť na $\triangle AB'C'$ ešte raz, keď za východiskový
vezmeme trojuholník $ADE$ a~prihliadneme na rovnosti
$|B'E|={|B'C|-|CE|}={|AB|-|BD|}=|AD|$ a~$|C'D|=|C'B|+|BD|=|AC|+|CE|=|AE|$. Stred kružnice opísanej
$\triangle AB'C'$ leží preto tiež na kružnici opísanej
$\triangle ADE$. Našli sme tak priesečník kružníc opísaných
$\triangle ABC$ a $\triangle ADE$, ktorý je rôzny od bodu $A$
a ktorý leží na osi uhla~$BAC$ -- ide totiž o~stred kružnice opísanej
trojuholníku $AB'C'$ a ten je podľa osi uhla $BAC$ súmerný,
lebo obe jeho strany $AB'$, $AC'$ majú dĺžku $|AB|+|AC|$.
{Iný postup: Označme $S$ stred kratšieho oblúka $BC$ kružnice
opísanej $\triangle ABC$. Z~tetivového štvoruholníka $ABSC$ máme
$|\uhol DBS|=|\uhol ECS|$, a preto trojuholníky $DBS$ a $ECS$ sú
zhodné podľa vety $sus$. Odtiaľ
$|\uhol ADS|={180^{\circ}-|\uhol SDB|}={180^{\circ}-|\uhol SEC|}=
{180^{\circ}-|\uhol SEA|}$, takže podľa úlohy N2 je
aj štvoruholník $ADSE$ tetivový.}]

\endnávod
}

{%%%%%   A-I-3
Ukážeme, že najmenší potrebný počet ťahov je rovný $2n^2 + 4n -2$.

Budeme rozlišovať ťahy {\it vodorovné\/} a ťahy {\it zvislé}
-- podľa toho, či je žetón posunutý v~riadku, resp. v stĺpci.
Potrebné počty vodorovných a zvislých ťahov odhadneme oddelene.

Začneme s~vodorovnými ťahmi. Rovnako ako žetóny označme aj stĺpce
hracieho plánu zľava doprava číslami 1 až $2n$. Žetón 1 je na začiatku
v~stĺpci~1 a na konci má byť v~stĺpci~$2n$, musíme
s~ním preto vykonať aspoň $2n-1$ ťahov doprava. Všeobecne žetón~$k$
sa dostane zo stĺpca $k$ nakoniec do stĺpca $2n+1-k$, a tak v~prípade
$k \leq n$ je na to potrebných aspoň $2n+1-2k$ ťahov doprava,
zatiaľ čo v~prípade $k > n$ aspoň $2k-2n-1$ ťahov doľava.
Celkový počet vodorovných ťahov preto nemôže byť menší ako súčet
$$
\displaylines{
\underbrace{(2n-1) + (2n-3) + \dots + 1}_{\hbox{\sevenrm za žetóny 1 až $\scriptstyle n$}}
+\underbrace{1 + \dots + (2n-3) +(2n-1)}_{\hbox{\sevenrm za žetóny $\scriptstyle n+1$ až $\scriptstyle 2n$}}
   =\cr
=2\bigl(1 + 3 + \dots + (2n-1)\bigr)= 2n^2.}
$$
Vodorovných ťahov je preto aspoň $2n^2$.

Teraz sa zamerajme na zvislé ťahy. Žetón nazveme {\it lenivým},
ak zostane po celý čas v~dolnom riadku; ostatným žetónom hovorme
{\it akčné}. Všimnime si, že najviac jeden žetón môže byť
lenivý -- každé dva žetóny sú totiž v dolnom riadku nakoniec
v~opačnom poradí, ako boli na začiatku; keby teda oba boli lenivé,
museli by niekedy stáť na rovnakom políčku, čo nie je možné. Akčných
žetónov je teda aspoň $2n-1$ a s~každým z~nich boli vykonané
aspoň 2 zvislé ťahy -- najskôr nahor a neskôr nadol.
Zvislých ťahov celkom preto musí byť aspoň $2\cdot(2n-1)=4n-2$.

Dokopy dostávame, že na splnenie úlohy potrebujeme aspoň
$2n^2+4n-2$ ťahov. V~druhej časti riešenia ukážeme, že tento počet
ťahov skutočne stačí.
\vadjust{\bigskip
\centerline{{\raise16.7pt\hbox{$\phantom\rightarrow$}\quad}\epsfbox{a72i.31}{\quad\raise16.7pt\hbox{$\rightarrow$}\quad}\epsfbox{a72i.32}{\quad\raise16.7pt\hbox{$\rightarrow$}}}
\nopagebreak\medskip
\centerline{{\raise16.7pt\hbox{$\rightarrow$}\quad}\epsfbox{a72i.33}{\quad\raise16.7pt\hbox{$\rightarrow$}\quad}\epsfbox{a72i.34}{\quad\raise16.7pt\hbox{$\phantom\rightarrow$}}}
\ifobrazkycisla\nopagebreak\medskip\centerline\Obr\fi\bigskip}


Jeden možný postup pre všeobecné $n$ ilustrujeme na \obr{} pre $n=4$.
Najskôr všetkých $2n$ žetónov okrem prvého presunieme do horného riadka.
Potom presunieme žetón~1 po dolnom riadku z~prvého stĺpca do
posledného. Následne presunieme postupne žetóny $2$ až $n$;
každý z nich najprv do dolného riadka a vzápätí doprava na
posledné voľné políčko (ktoré je jeho cieľové). Na záver
presunieme postupne žetóny $n+1$ až $2n$ -- každý najskôr doľava
na políčko jeho cieľového stĺpca a vzápätí do spodného riadka.

Pri práve opísanom postupe zodpovedajú počty zvislých aj
vodorovných ťahov presne tým odhadom, ktoré sme odvodili v~prvej
časti riešenia: zvislých ťahov sme previedli práve~$4n-2$ a
ani vodorovných ťahov sme so žiadnym žetónom nevykonali viac,
než bolo skôr udané za nutné.
Celkový počet ťahov pri opísanom postupe je teda skutočne $2n^2+4n-2$.


\návody
Uvažujme situáciu súťažnej úlohy pre $n=3$. {\it Vodorovným
\/} nazveme každý ťah, pri ktorom je žetón posunutý v~riadku.
Udajte príklad postupnosti ťahov, ktorou splníme cieľ úlohy
a ktorá pritom obsahuje najmenší možný počet vodorovných ťahov.
[Tabuľka má šesť stĺpcov, preto so žetónom 1 musíme vykonať
aspoň 5 ťahov doprava, so žetónom 2 aspoň 3 doprava,
so žetónom 3 aspoň 1 doprava, so žetónom 4 aspoň 1 doľava,
so žetónom 5 aspoň 3 doľava a so žetónom 6 aspoň 5 doľava. Celkom tak
potrebujeme aspoň 18 vodorovných ťahov. Vyhovujúci príklad
s~18 vodorovnými ťahmi: Najprv presunieme žetóny 2 až 6 hore,
potom žetón 1 do jeho cieľa, následne žetóny 2 až 5 dole a potom
žetón 6 do jeho cieľa. Takto sme vodorovnými ťahmi
iba s~žetónmi 1, 6 dosiahli to, že sa prehodili.
Podobne potom prehodíme žetóny 2, 5 a nakoniec žetóny 3, 4.]

Rovnosť $1+2+4+5+\dots+(3n-2)+(3n-1)=3n^2$ dokážte pre každé
prirodzené číslo~$n$.
[Použijeme indukciu vzhľadom na číslo $n$. Pre $n=1$ rovnosť platí
($1+2=3\cdot1^2$). Ak platí pre nejaké $n=k$, potom pre $n=k+1$
ju odvodíme takto: $1+2+\dots+(3k+1)+(3k+2)=3k^2+(3k+1)+(3k+2)=
3k^2+6k+3=3(k+1)^2$. Iný postup: Sčítajte $2n$ rovností
$i+(3n-i)=3n$ pre $i\in\{1,2,\dots,3n-2,3n-1\}$ a
výslednú rovnosť vydeľte dvoma.]

Zdôvodnite, že v~priebehu ťahov vedúcich k~cieľu súťažnej úlohy sa
z~každých dvoch žetónov musí aspoň jeden niekedy dostať do horného
riadku hracieho plánu.
[Uvážme žetóny $i$ a $j$, kde $i<j$. Na začiatku je $i$
naľavo od $j$, ale na konci je $i$ napravo od~$j$.
Takto by sa ich poradie v~dolnom riadku nemohlo vymeniť, keby oba
žetóny boli v~tomto riadku stále.]

\D
Uvažujme rovnaké počiatočné rozostavenie $2n$ žetónov
ako v~súťažnej úlohe. Koľkými najmenej ťahmi možno získať
rozostavenie, keď opäť všetky žetóny budú v~dolnom riadku,
avšak žetón~1 sa ocitne v poslednom stĺpci?
[$4n$ ťahov. V~prvom ťahu musíme posunúť nejaký žetón hore a niekedy
neskôr ho posunúť dole. S~žetónom 1 musíme vykonať aspoň
$2n-1$ ťahov doprava. Aby sme vysvetlili, že celkový počet ťahov
doľava je tiež aspoň $2n-1$, označme stĺpce zľava doprava
číslami 1 až $2n$ a uvažujme premennú veličinu, ktorá je rovná
súčtu $2n$ čísel tých stĺpcov, v ktorých sa jednotlivé z~$2n$ žetónov
aktuálne nachádzajú. Táto veličina má v počiatočnom aj koncovom
rozostavení rovnakú hodnotu (rovnú $1+2+\cdots+2n$),
s~každým ťahom doprava vzrastie o~1, s~každým ťahom doľava klesne o~1
a pri ťahoch v stĺpcoch sa nemení --
preto musia byť celkové počty ťahov doprava a ťahov doľava
dokonca rovnaké. Dokázali sme tak, že ťahov všetkými smermi musí
byť aspoň $2+(2n-1)+(2n-1)=4n$. Počet $4n$ ťahov stačí:
žetón 1 posunieme nahor, potom všetky ostatné o~1 políčko doľava
a nakoniec žetón 1 do posledného stĺpca a nadol.]


V~jednom rade stojí $n$ žetónov postupne s~číslami od 1 do $n$.
V~každom ťahu môžeme navzájom vymeniť dva susedné žetóny. Koľkým
najmenej ťahmi možno pôvodné poradie žetónov zmeniť na opačné, t.\,j.
s~číslami od $n$ do 1?
[$\frac{n(n-1)}{2}$ ťahov. Každú dvojicu žetónov musíme niekedy
(ako susedné dva žetóny) prehodiť.
Keďže všetkých dvojíc je $\frac{n(n-1)}{2}$, potrebujeme aspoň
$\frac{n(n-1)}{2}$ ťahov. Toľko ťahov skutočne stačí --
presunieme napríklad najprv žetón~1 na posledné miesto ($n-1$ ťahov), potom
žetón~2 na predposledné miesto ($n-2$~ťahov) atď., až nakoniec
žetón $n-1$ na druhé miesto (1 ťah). Tak vykonáme práve
$(n-1)+(n-2)+\dots+2+1 = \frac{n(n-1)}{2}$ ťahov.]

V~situácii zo súťažnej úlohy je tentoraz v~dolnom riadku
rozmiestnených $2n$ žetónov s~číslami $1$, $2$,\dots $2n$
{\it v~ľubovoľnom poradí}. Koľkými najmenej ťahmi možno vždy dosiahnuť to,
aby všetkých $2n$ žetónov bolo v~dolnom riadku rozmiestnených {\it vzostupne},
t.\,j. v~poradí ako na začiatku pôvodnej úlohy?
[Tento počet je rovnaký ako počet ťahov v~súťažnej úlohe (ktorý tu
prezrádzať nebudeme). Najprv dokážeme matematickou indukciou
nasledujúce tvrdenia. {\sl Nech $k$ je prirodzené číslo.
Uvažujme hrací plán $k\times2$, na ktorom je (iba) v~hornom riadku
nejaký počet žetónov s~určitými navzájom rôznymi číslami vybranými
z~množiny $\{1,2,\dots,k\}$. Potom existuje taká postupnosť ťahov,
ktorá pre každé $i$ premiestni žetón s~číslom $i$ (ak na pláne je)
na dolné políčko $i$-teho stĺpca a ktorá na to pre každý žetón
využije najmenší možný počet ťahov.} Pre $k=1$ tvrdenie zjavne platí.
Nech je teraz $k \geq 2$ a
nech pre všetky hracie plány $k'\times2$, kde $k'<k$, tvrdenie platí.
Na zadanom pláne $k\times 2$ (ktorý spĺňa predpoklady tvrdenia)
vezmime žetón s najväčším číslom, označme ho~$i$,
tento žetón posuňme nadol a potom ho
presuňme do $i$-teho stĺpca. Následne vďaka indukčnému predpokladu
presunieme na správne miesta všetky žetóny,
ktoré sa nachádzajú v~prvých $i-1$ stĺpcoch. Potom budeme po jednom
zľava presúvať ešte žetóny, ktoré v~zadanom pláne $k\times 2$
prípadne zostali od $i$-teho stĺpca napravo: Každý z~nich,
ak má číslo $j$, presunieme najskôr doľava do $j$-teho stĺpca a potom
nadol. Zostavili sme tak pre zadaný plán $2\times k$ postupnosť
ťahov, ktorá má zrejme všetky potrebné vlastnosti; dôkaz indukciou
je tak ukončený.\hfil\break
%
\indent Prejdime k~vlastnej úlohe D3. Pri ľubovoľnej východiskovej situácii s~ťahmi
začneme tak, že všetky žetóny -- okrem toho s~číslom $2n$ --
posunieme nahor a potom žetón $2n$ presunieme na posledné miesto
(ak už tam nestál). Následne na žetóny z~prvých $2n-1$ stĺpcov
uplatníme postupnosť ťahov z dokázaného tvrdenia. Nakoniec potom
na správne miesto presunieme prípadný žetón z horného políčka
posledného stĺpca. Pri takej konštrukcii bude počet ťahov
najväčší, ak budú na začiatku žetóny
usporiadané zostupne. V~tomto prípade optimálnosť konštrukcie
vyplýva z~riešenia pôvodnej úlohy.]
\endnávod
}

{%%%%%   A-I-4
Ukážeme, že hľadaný medián má hodnotu $q =\dfrac{k+1}{n+1}$.
V~celom riešení bude $q$ označovať práve toto číslo.

Keďže dané čísla $n$ a $k$ sú nepárne, zlomok s hodnotou~$q$
je na tabuli skutočne zapísaný -- napríklad to je zlomok
$\dfrac{\frac12(k+1)}{\frac12(n+1)}$.

Podľa porovnania s~číslom $q$ povieme, že nejaký zlomok je
\item{$\triangleright$} {\it malý}, ak je jeho hodnota menšia ako $q$,
\item{$\triangleright$} {\it stredný}, ak je jeho hodnota rovná $q$,
\item{$\triangleright$} {\it veľký}, ak je jeho hodnota väčšia ako $q$.

\smallskip
Počet $k\cdot n$ všetkých zapísaných zlomkov je nepárny; aby sme
ukázali, že pri ich usporiadaní podľa veľkosti bude mať prostredný zlomok
hodnotu $q$, stačí dokázať, že malých zlomkov je na tabuli
práve toľko ako tých veľkých. (Posledné bude tiež znamenať, že
počet stredných zlomkov je nepárny, čo znovu potvrdí ich
existenciu.)

Zlomky zapísané na tabuli popárujeme -- každý zlomok $i/j$ dáme do
dvojice so zlomkom~$i'/j'$ (a naopak) práve vtedy, keď bude
$i'=k+1-i$ a $j'=n+1-j$, čo možno skutočne prepísať symetricky ako
$i+i'=k+1$ a $j+j'=n+1$. Uvedomme si, že nerovnosti $1 \leq aj \leq k$ a~$1 \leq j \leq n$ zrejme platia práve vtedy, keď platí
$1 \leq i'\leq k$ a~$1 \leq j' \leq n$. (Práve tieto
celočíselné nerovnosti určujú množinu zlomkov zapísaných ako $i/j$,
resp. $i'/j'$.)

Je zrejmé, že iba zlomok z~konca druhého odseku
nášho riešenia je takto \uv{spárovaný} sám so sebou a že všetky ostatné
zapísané zlomky sú skutočne rozdelené do dvojíc.
Ak ďalej ukážeme, že v~každej takej
dvojici je buď jeden malý a jeden veľký zlomok, alebo v~nej sú
dva stredné zlomky, vyplynie z toho už potrebný záver, že
počet malých zlomkov je rovnaký ako počet veľkých zlomkov.

Vďaka spomínanej symetrii stačí overiť, že pri
zavedenom označení je zlomok $i'/j'$ malý práve vtedy, keď je zlomok $i/j$ veľký. Overenie cestou ekvivalentných
úprav je rutinné:
\def\ekv{\ \Leftrightarrow\ }
$$\eqalign{
\frac{i'}{j'}<\frac{k+1}{n+1}&\ekv
\frac{k+1-i}{n+1-j}<\frac{k+1}{n+1}%\ekv\cr
\ekv(k+1-i)(n+1)<(k+1)(n+1-j)\ekv\cr
&\ekv(k+1)(n+1)-i(n+1)<(k+1)(n+1)-(k+1)j\ekv\cr
&\ekv i(n+1)>(k+1)j\ekv \frac{i}{j}>\frac{k+1}{n+1}.
}$$
Tým je celé riešenie hotové.

\poznamky
Namiesto úprav nerovností v závere riešenia sme mohli vykonať túto úvahu:
Zoberme $j$~rovnakých zlomkov $i/j$ a $j'$ rovnakých zlomkov $i'/j'$ --
celkom to je $j+j'$ zlomkov so súčtom $i+i'$, takže ich aritmetický priemer
je rovný $(i+i')/(j+j')=(k+1)/(n+1)=q$.
Keďže sme však priemerovali najviac dve rôzne hodnoty,
išlo buď o~jedinú hodnotu $q$, alebo o jednu hodnotu menšiu ako~$q$
a~jednu hodnotu väčšiu ako $q$.

Motiváciu pre zvolené párovanie zlomkov $i/j$ a~$i'/j'$
poskytuje nasledujúce užitočné pravidlo (z~návodnej úlohy~N4):
{\sl Pre ľubovoľnú štvoricu reálnych čísel $a$, $b$, $c$ a $d$, pričom $b>0$ a $d>0$, platí implikácia
}
\def\impl{\ \Rightarrow\ }
$$
\frac{a}{b}<\frac{c}{d}\impl
\frac{a}{b}<\frac{a+c}{b+d}<\frac{c}{d}.
$$
Pri označení z nášho riešenia totiž stačí len rozlíšiť, ktorý z dvoch zlomkov $i/j$ a~$i'/j'$ má menšiu hodnotu, a podľa toho
uplatniť uvedenú implikáciu. Dostaneme tak, že zlomok s menšou
hodnotou je skutočne malý a že zlomok s väčšou hodnotou
je skutočne veľký.

   \Jres
Na úlohu sa pozrieme geometricky -- využijeme na to rovinu
s~karteziánskou sústavou súradníc $Oxy$. V~nej každý
zlomok $i/j$, ktorý Martin zapísal na tabuľu, znázorníme ako bod $B$
s~dvojicou súradníc $[j,i]$.\fnote{Dôvodom k~takejto zmene poradia
čísel $i$ a $j$ je, že hodnota dotyčného zlomku $i/j$ sa rovná
smernici priamky, ktorá spája začiatok $O[0,0]$ práve
s~bodom $B[j,i]$.}
Dostaneme tak práve tie body $B[j,i]$ našej roviny,
pre ktoré $j\in\{1,2,\dots,n\}$ a $i\in\{1,2,\dots,k\}$.
Množinu týchto bodov, ktorú sme na obrázku vykreslili pre $n=11$ a
$k=5$,
označíme~$M$ a budeme jej ďalej hovoriť \uv{mriežka}. Má tvar obdĺžnika
s~vrcholmi $[1,1]$, $[n,1]$, $[n,k]$ a $[1,k]$. Keďže čísla
$n$, $k$ sú nepárne, má stred $S$ tohto obdĺžnika celočíselné
súradnice $j_0=\frac12(n+1)$ a $i_0=\frac12(k+1)$.
Stred $S[j_0,i_0]$ je tak sám bodom mriežky $M$. Dodajme ešte,
že priamka $OS$ má smernicu $i_0/j_0$ a že hodnotu tohto zlomku
rovnako ako v~prvom riešení označíme $q$ aj v závere tohto riešenia.
\insp{a72i.41}%

Všimnime si, že podľa stredu $S$ je súmerný nielen spomínaný
obdĺžnik, ale súmerná je aj samotná mriežka $M$ (na \obr{} sme
vyznačili jej dva súmerne združené body $B$ a~$B'$).\fnote{K
tomuto tvrdeniu, ktoré možno považovať za zrejmé, sa ešte vrátime
v~poznámke za riešením.} Ak teda zostrojíme priamku $OS$,
bude vo vnútri každej z oboch vyťatých polrovín ležať rovnaký počet
bodov z~$M$. Objasnime, čím sa tieto rovnako početné skupiny bodov
\uv{pod priamkou~$OS$} a \uv{nad priamkou $OS$} líšia.

Bod $B[j,i]$ mriežky $M$ leží pod priamkou $OS$ práve vtedy, keď
má priamka $OB$ menšiu smernicu ako priamka $OS$, t.\,j. práve vtedy, keď
platí $i/j<i_0/j_0=q$. Pod priamkou $OS$ teda ležia práve tie body
$B[j,i]$ mriežky $M$, ktoré zodpovedajú {\it malým\/} zlomkom
$i/j$, ako sme im hovorili v~prvom riešení.
Podobne body mriežky $M$
nad priamkou $OS$ zodpovedajú {\it veľkým\/} zlomkom. Tak sme
znovu overili potrebné tvrdenie, že totiž malých aj veľkých zlomkov
je rovnaký počet.

   \Pozn
V druhom riešení sme využili súmernosť so stredom $S[j_0,i_0]$.
Tá zobrazí každý bod $B[j,i]$ na bod $B'[j',i']$, kde (ako je známe
z~analytickej geometrie) platí $j'=2j_0-j$ a $i'=2i_0-i$. Po
dosadení $j_0=\frac12(n+1)$ a $i_0=\frac12(k+1)$ dostaneme
symetrické rovnosti $j+j'=n+1$ a $i+i'=k+1$. Vidíme, že párovanie zlomkov
z~prvého riešenia je vyjadrením súmernej združenosti bodov mriežky
$M$ z~druhého riešenia. Tieto dve riešenia tak sú založené
vlastne na rovnakom nápade.

\návody
Martin napísal na tabuľu hodnotu rozdielu $i/j -j/i$
pre každú dvojicu prirodzených čísel $i\leq 5$, $j\leq 5$.
Určte medián všetkých čísel na tabuli.
[Medián je 0, pretože vždy, keď je rozdiel $i/j -j/i$ kladný,
je opačný rozdiel $j/i -i/j$ záporný a naopak.
Podrobnejšie: označme $f(i, j) =i/j -j/i$, potom na tabuli je 25 čísel,
z~nich 5 -- $f(1, 1)$, $f(2, 2)$, $f(3, 3)$, $f(4, 4)$ a $f(5, 5)$ -- je
rovných 0. Zvyšných 20 čísel rozdelíme do 10 dvojíc: každé číslo
$f(i, j)$, kde $i\ne j$, spárujeme s~číslom $f(j, i)$.
V~každej dvojici je zrejme jedno číslo kladné a jedno číslo záporné.
Na tabuli je tak 10 čísel kladných, 10 záporných a 5 núl.
Medián je preto 0.]

Riešte súťažnú úlohu pre prípad $k=n$.
[V~tomto prípade je medián 1. Podobne ako v~úlohe N1
vyčleníme zvlášť zlomky $i/i$ s~hodnotou 1 -- tých je $k$, teda
nepárny počet.
Ostatné zlomky zase rozdelíme do dvojíc:
každý zlomok $i/j$ spárujeme s prevráteným zlomkom $j/i$.
V~každej dvojici je zrejme jeden zlomok menší ako~1 a jeden zlomok väčší
ako~1. Zlomkov menších ako~1 je teda rovnaký počet
ako zlomkov väčších ako~1, takže 1 je medián.]

Riešte súťažnú úlohu pre prípad $n=3$.
[V~tomto prípade je medián $(k+1)/4$. Na tabuli máme
pre každé $i\in\{1,2,\dots,k\}$ napísané tri zlomky $i/1$,
$i/2$, $i/3$. Zamerajme sa najprv na zlomky
$i/2$. Keďže $k$ je nepárne, vo vzostupnom poradí týchto zlomkov
stojí uprostred zlomok $\frac12(1/2+k/2)=(k+1)/4$, ktorý tak je
ich mediánom. Porovnávame s ním teraz ostatných $2k$ zlomkov
$i/1$ a $i/3$ s~čitateľmi~$i$ od 1 do $k$.
Využijeme na to jednak ekvivalenciu
$$\let\ekv=\Leftrightarrow
\frac{i}{1}<\frac{k+1}{4}\ekv(k+1)-i>\frac{3(k+1)}{4}\ekv
\frac{k+1-i}{3}>\frac{k+1}{4},
$$
jednak tie isté ekvivalencie s~opačnými znakmi ostrých nerovností.
Vyplýva z nich, že počet tých zlomkov $i/1$, ktoré sú menšie (resp.
väčšie) ako $(k+1)/4$, je rovnaký ako počet tých zlomkov $i/3$, ktoré sú
väčšie (resp. menšie) ako $(k+1)/4$. Odtiaľ už vyplýva, že $(k+1)/4$
je hľadaný medián všetkých $3k$ zlomkov.]

Dokážte, že pre ľubovoľnú štvoricu reálnych čísel
$a$, $b$, $c$ a $d$, pričom $b>0$ a $d>0$, platí implikácia
$$
\frac{a}{b}<\frac{c}{d}\Rightarrow
\frac{a}{b}<\frac{a+c}{b+d}<\frac{c}{d}.
$$
[Nerovnosti z pravej strany implikácie sú
ekvivalentné s nerovnosťami $a(b+d)<(a+c)b$, resp. $(a+c)d<c(b+d)$.
Tie sú zrejme obe ekvivalentné s nerovnosťou $ad<bc$,
ktorá je dôsledkom nerovnosti z ľavej strany implikácie.]

\D
Napíšme na tabuľu súčet $a+b+c+d+e$ pre každú päticu
$(a,b,c,d,e)$ prirodzených čísel menších ako 6. Určte medián všetkých
$5^5$ čísel na tabuli.
[Medián je 15. Päticu $(3, 3, 3, 3, 3)$ so súčtom 15 dajme
bokom a ostatné pätice rozdeľme do dvojíc tak, že každú
päticu $(a, b, c, d, e)$ spárujeme s~päticou
$(6-a, 6-b, 6-c, 6-d, 6-e)$. V~každej dvojici buď
obe pätice majú súčet 15, alebo jedna pätica má súčet menší ako
15 a~druhá pätica má súčet väčší ako 15.]


Nech $k$, $n$ sú nepárne prirodzené čísla.
Pre každé dve prirodzené čísla $i\leq k$, $j\leq n$ napíšme na tabuľu
zlomok $(i-j)/(i+j)$. Určite medián všetkých týchto zlomkov.
Využite na to výsledok súťažnej úlohy.
[Ukážte, že pre kladné čísla $a$, $b$, $c$, $d$ platí
$(a-b)/(a+b)<(c-d)/(c+d)$ práve vtedy, keď $a/b<c/d$. Z hľadiska
usporiadania hodnôt zlomkov je teda situácia na tabuli rovnaká ako v
súťažnej úlohe. Ak preto označíme $x/y$ medián zo súťažnej úlohy,
bude hľadaný medián rovný $(x-y)/(x+y)$.]

Uvažujme množinu $\{1,2,4,5,8,10,16,20,32,40,80,160\}$
a~všetky jej trojprvkové podmnožiny. Rozhodnite, či je viac tých,
ktoré majú súčin svojich prvkov väčší ako 2006,
alebo tých, ktoré majú súčin svojich prvkov menší ako 2006.
[\pdfklink{56-A-S-2}{https://skmo.sk/dokument.php?id=223}]

Martin pre každú neprázdnu podmnožinu $M$ množiny
$\{0, 1, \dots, 16\}$ napísal
na tabuľu zvyšok súčtu všetkých prvkov z $M$ po delení
číslom~17. Určte, ktorý zvyšok má na tabuli
najväčší počet výskytov.
[Zvyšok 0. Ukážeme, že
ak by sme na tabuľu nezapísali zvyšok súčtu prvkov
celej množiny $\{0,1,\dots,16\}$,
tak každý zvyšok od 0 do~16 by mal na tabuli rovnaký počet výskytov.
Pre dôkaz uvážme všetky
$k$-prvkové podmnožiny $\{0,1,\dots,16\}$ s~pevným $k$ od 1 do 16.
Rozdeľme tieto podmnožiny do 17-prvkových skupín tak, že v každej
skupine budeme mať s každou množinou $\{x_1, \dots, x_k\}$
nasledujúcich 16 množín (sčítanie ďalej chápeme ako operáciu so
zvyškami modulo 17):
$\{1+x_1, \dots,1+x_k\}$,
$\{2+x_1, \dots,2+x_k\},\dots,$
$\{16+x_1, \dots,16+x_k\}$.
Zvyšky súčtov prvkov jednotlivých 17 množín
v každej skupine teda budú zvyšky
$\sum x_i$, $k+\sum x_i$, $2k+\sum x_i$,\dots, $16k+\sum x_i$.
V tomto zozname zvyškov bude každý zo 17 možných zvyškov
zastúpený práve raz, a to vďaka nesúdeliteľnosti čísel~$k$ a~17.
Keďže to platí pre každú vytvorenú skupinu,
každý zvyšok bude zvyškom súčtov prvkov rovnakého počtu
$k$-prvkových podmnožín, a to pre každé $k$ od 1 do 16. Tým je
sľúbený dôkaz hotový. Keďže nezahrnutá 17-prvková množina
$\{0,1,\dots,16\}$ má súčet prvkov 136 so zvyškom 0,
je tento zvyšok zapísaný na tabuli v počte o 1 väčšom
ako každý iný zo 16~zvyškov.]


\leavevmode\emph{Zlomkovou časťou} $\{x\}$ reálneho čísla $x$ nazývame
číslo $\{x\} = x - \floor{x}$, kde $\floor{x}$ označuje {\it celú
časť\/} čísla $x$ (pozri súťažnú úlohu 1). Uvažujme jednak medián
čísel $\{\sqrt{1}\}$, $\{\sqrt{2}\}, \dots$, $\{\sqrt{999\,999}\}$,
jednak medián čísel $\{\root3\of{1}\}$, $\{\root3\of{2}\}, \dots$,
$\{\root3\of{999\,999}\}$. Ktorý z týchto mediánov je väčší?
[Väčší je druhý medián. Nech $n$ je prirodzené číslo. Všimnime si,
že pre celé čísla $i$ od $n^2$ do $n^2+n$, ktorých je $n+1$, platí
$n\leq\sqrt{i}<n+\frac12$, takže $\{\sqrt{i}\}<\frac12$.
Podobne pre celé čísla $i$ od $n^2+n+1$ do $n^2+2n=(n+1)^2-1$, ktorých
je~$n$, platí $\{\sqrt{i}\}>\frac12$. Porovnaním oboch počtov $i$
zisťujeme, že pre {\it najtesnejšiu väčšinu} celých čísel
$i$ z~intervalu $\langle n^2, (n+1)^2-1 \rangle$ je
hodnota $\{\sqrt{i}\}$ menšia ako~$\frac12$. Ak necháme $n$
prebiehať hodnoty od 1 do 999, uvažované intervaly disjunktne
pokryjú celé čísla práve v rozpätí od 1 do 999\,999.
Preto je v prvej zadanej postupnosti väčšina čísel
menších ako $\frac12$, teda je taký aj ich medián. Podobne
sa teraz pre prirodzené $n$ pozrime na hodnoty $\{\root3\of{i}\}$
pre celé čísla $i$ z intervalu $\langle n^3, (n+1)^3-1 \rangle$.
Pre uvažované $i$ platí $\{\root3\of{i}\}<\frac12$ práve vtedy, keď
$i<\left(n+\frac12\right)^3 = n^3 + \frac32n^2 + \frac34n +
\frac18$. Počet dotyčných $i$, ktoré spĺňajú poslednú podmienku,
je teda práve $\floor{\frac32n^2 + \frac34n + \frac18} + 1$, čo
neprevyšuje hodnotu $\frac32n^2+\frac34n+\frac98$, ktorá je vďaka $n\geq1$
menšia ako {$\frac12\bigl[(n+1)^3-n^3\bigr]$}.
Nerovnosť $\{\root3\of{i}\}<\frac12$ tak spĺňa {\it menšina\/}
celých čísel $i$ zo skúmaného intervalu. Ak vezmeme teraz
$n$ od 1 do 99, tieto intervaly disjunktne pokryjú celé čísla
práve v rozpätí od 1 do~999\,999. Preto je v druhej zadanej postupnosti
väčšina čísel väčších ako $\frac12$, teda je taký aj ich medián.
Dokopy dostávame, že druhý medián je väčší ako prvý.]

\endnávod
}

{%%%%%   A-I-5
Nech v danom trojuholníku $ABC$ je
$M$ stred strany~$AB$, $N$ stred strany~$AC$, $K$ a $L$
priesečníky osi uhla $CAB$ postupne s osami strán $AB$ a $AC$,
ktorých priesečník je označený $O$. Trojuholník $KLO$ je teda tým trojuholníkom,
o ktorom je reč v~zadaní úlohy. Priesečník jeho výšok ešte
označíme $H$.
Všetky pomenované body sú vyznačené na \obr{}
nakreslenom pre prípad $|AB|<|AC|$. (Prípad $|AB|>|AC|$ vyzerá
analogicky, prípad $|AB|=|AC|$ je zadaním vylúčený -- body $K$, $L$,~$O$
vtedy splývajú v jeden bod.)
\inspdf{a72i.51.pdf}%

Podľa zadania máme dokázať, že bod $H$ leží na ťažnici
z~vrcholu $A$ trojuholníka~$ABC$. Využijeme na to výsledok návodnej
úlohy~N1, podľa ktorého stačí ukázať, že trojuholníky $ABH$ a $ACH$
majú rovnaký obsah.

Vďaka tomu, že $HL\perp OK\perp AB$, platí $HL \parallel AB$. Bod
$H$ tak má od priamky~$AB$ rovnakú vzdialenosť ako bod~$L$. Tá je
však rovná dĺžke úsečky $LN$, pretože bod~$L$ leží na osi uhla
$CAB$ a $N$ je kolmý priemet~$L$ na~$AC$. Dokopy dostávame, že
obsah prvého trojuholníka $ABH$ je rovný $\frac12|AB|\cdot|LN|$.
Analogicky zistíme, že obsah druhého trojuholníka $ACH$
je rovný $\frac12|AC|\cdot|KM|$.
Zostáva tak dokázať rovnosť ${|AB|\cdot|LN|}={|AC|\cdot|KM|}$.

Pre body $K$ a $L$ ležiace na osi uhla $CAB$ platí $|\angle
MAK| = |\angle NAL|$.
Pravouhlé trojuholníky $AKM$ a $ALN$ sú teda podobné podľa vety~\emph{uu},
a preto platí úmera $|KM|:|AM|=|LN|:|AN|$. Odtiaľ vzhľadom na $|AM|=\frac12|AB|$ a $|AN| =\frac12|AC|$ dostávame
$|KM|:|AB|=|LN|:|AC|$, čiže $|AB|\cdot|LN|=|AC|\cdot|KM|$, ako
sme potrebovali dokázať.

    \Jres
Okrem bodov z~prvého riešenia uvážime ešte priesečník $E$ priamok
$AB$, $KH$ a $F$ priesečník priamok $AC$, $LH$. Opäť si
všimneme, že platí $KH\parallel AC$ a $LH\parallel AB$,
takže štvoruholník $AEHF$ je rovnobežník (pozri \obr{}).
\inspdf{a72i.52.pdf}%

Znovu využijeme aj podobnosť trojuholníkov $AKM$ a $ANL$, podľa ktorej
platí $|KM|:|LN|=|AM|:|AN|=|AB|:|AC|$. Zhodnosť vonkajších uhlov pri
vrcholoch $E$, $F$ rovnobežníka $AEHF$ znamená, že $|\uhol KEM|=|\uhol
LFN|$. Podobné sú tak aj pravouhlé trojuholníky $EKM$ a $FLN$,
odkiaľ vyplýva, že pomer $|EM|:|FN|$ je rovný pomeru
$|KM|:|LN|$, teda aj pomeru $|AB|:|AC|$. Preto platí
$$
\frac{|AE|}{|AF|} =\frac{|AM| - |EM|}{|AN| - |FN|}=
\frac{\frac{|AB|}{|AC|}\cdot|AN|-\frac{|AB|}{|AC|}\cdot|FN|}{|AN|-|FN|}
= \frac{|AB|}{|AC|},
$$
teda trojuholníky $AEF$ a $ABC$ sú podobné podľa vety $sus$
(zhodujú sa v~uhle pri vrchole $A$ a v~pomere priľahlých strán).
Dostávame tak rozhodujúcu reláciu $EF \parallel{BC}$.


Keďže v~rovnobežníku $AEHF$ priamka $AH$ rozpoľuje uhlopriečku $EF$,
rozpoľuje táto priamka aj úsečku $BC$, ktorá je totiž s~úsečkou $EF$
rovnoľahlá podľa stredu $A$. Inak povedané, bod~$H$ leží na
ťažnici z~vrcholu $A$ trojuholníka $ABC$, ako sme mali dokázať.

\návody
{\everypar{}
\smallskip
\emph{Dohovor}. V~riešeniach úloh budeme obsah trojuholníka $XYZ$ označovať {$[XYZ]$}.
\smallskip
}

Nech $X$ je vnútorný bod trojuholníka $ABC$. Dokážte, že
$X$ leží na jeho ťažnici z~vrcholu~$A$ práve vtedy, keď trojuholníky
$ABX$ a $ACX$ majú rovnaký obsah.
[Označme $D$ priesečník polpriamky $AX$ so stranou $BC$.
Trojuholníky $ADB$ a $ADC$ majú spoločnú výšku z~vrcholu~$A$,
preto $[ADB]:[ADC]=|DB|:|DC|$. Podobne tiež
$[XDB]:[XDC] = |DB|:|DC|$. Dokopy dostávame
$$
\frac{[ABX]}{[ACX]} = \frac{[ADB]-[XDB]}{[ADC]-[XDC]} =
\frac{\frac{|DB|}{|DC|}\cdot[ADC] -
\frac{|DB|}{|DC|}\cdot[XDC]}{[ADC]-[XDC]} = \frac{|DB|}{|DC|}.
$$
Vidíme, že trojuholníky $ABX$ a $ACX$ majú rovnaký obsah práve vtedy, keď
$|DB| = |DC|$, t.\,j. práve vtedy, keď $D$ je stred $BC$, čiže
$AD$ je ťažnica trojuholníka $ABC$. Iný postup: Spojme vnútorný bod $X$
s vrcholmi $A$, $B$, $C$ a stredom $D$ strany $BC$. Keďže
$[XBD]=[XCD]$, rovnosť $[ABX]=[ACX]$ nastane práve vtedy, keď
$[ABX]+[XBD]=[ACX]+[XCD]$, teda práve vtedy, keď dvojica úsečiek
$AX$, $XD$ rozpoľuje obsah trojuholníka $ABC$. To zrejme platí, ak $X$
leží na úsečke $AD$, a navyše to zrejme neplatí, ak leží~$X$
vo vnútri jedného z trojuholníkov $ABD$, $ACD$.]

{\everypar{}
\smallskip
\emph{Dohovor}. V~úlohách N2--N4 budeme skúmať situáciu
zo súťažnej úlohy. Nech teda v~ostrouhlom
rôznostrannom trojuholníku $ABC$ je $M$ stred strany~$AB$,
$K$ a $L$ priesečníky osi uhla pri vrchole $A$ postupne
s~osami strán $AB$ a $AC$, ktorých priesečník je označený~$O$;
napokon $H$ je priesečník výšok trojuholníka $KLO$.
\smallskip
}

Ukážte, že vzdialenosť bodu $H$ od priamky $AC$ sa rovná $|KM|$.
[Z $KH\perp LN\perp AC$ máme $KH\parallel AC$. Tým pádom vzdialenosť
$H$ od $AC$ je rovnaká ako vzdialenosť~$K$ od $AC$. Keďže $K$
leží na osi uhla $BAC$, má od $AC$ rovnakú vzdialenosť ako od
$AB$, teda $|KM|$.]

Nech priamka $HK$ pretína stranu $AB$ v~bode $E$ a priamka
$HL$ stranu $AC$ v~bode~$F$. Dokážte, že priamka
$AH$ delí úsečku $EF$ na dva zhodné úseky.
[Platí $HE\parallel AC$ a $HF\parallel AB$, takže $AEHF$ je rovnobežník.
Jeho uhlopriečky $EF$ a $AH$ sa preto navzájom rozpoľujú.]

Pri označení z~úlohy N3 dokážte, že trojuholníky $EMK$ a $FNL$ sú podobné.
[Vyplýva to z vety \emph{uu}, pretože trojuholníky majú pri vrcholoch $M$, $N$
pravé uhly a aj ich uhly pri vrcholoch $E$, $F$ sú zhodné (vďaka
rovnobežníku $AEHF$, pozri riešenie N3).]

\D
Použitím výsledku úlohy N1 dokážte známe tvrdenie, že ťažnice
ľubovoľného trojuholníka sa pretínajú v~jednom bode.
[V trojuholníku $ABC$ označme $T$ priesečník ťažníc z~vrcholov $B$ a $C$.
Z~úlohy N1 vieme, že {$[ABT] = [BCT]$ a $[BCT] = [ACT]$,
odkiaľ $[ABT] = [ACT]$}, teda opäť podľa úlohy N1 bod~$T$
leží na ťažnici z~vrcholu~$A$.]


V trojuholníku $ABC$ označme $D$ priesečník osi uhla $BAC$ so stranou $BC$.
Ukážte, že $|BD|:|DC|=|AB|:|AC|$.
[Trojuholníky $ABD$ a $ACD$ majú spoločnú výšku z~vrcholu~$A$,
preto $[ABD]:[ACD] =|BD|:|DC|$.
Zároveň však majú zhodné výšky z~vrcholu~$D$, takže
$[ABD]:[ACD]=|AB|:|AC|$. Z oboch rovností už vyplýva potrebný záver.]

Os uhla $BCA$ trojuholníka $ABC$ pretne jemu
opísanú kružnicu v~bode $R$ rôznom od bodu $C$, os strany $BC$
pretne v~bode $P$ a os strany $AC$ v~bode $Q$. Stred strany $BC$
označíme $K$ a stred strany $AC$ označíme~$L$. Dokážte, že
trojuholníky $RPK$ a $RQL$ majú rovnaký obsah.
[IMO 2007, úloha 4, rieš. \pdfklink{IMO Shortlist 2007, Problem G1}{https://www.imo-official.org/problems/IMO2007SL.pdf\#page=40}.]
\endnávod
}

{%%%%%   A-I-6
a)
Matematickou indukciou najskôr dokážeme, že
$a_n \geq 2$ pre
 každé $n$. Pre $n=1$ a $n=2$ to platí, pretože
$a_1 = 3$ a $a_2 = 2$. Predpokladajme teraz, že pre niektoré $n\geq 3$
nerovnosť $a_k \geq 2$ platí pre každé $k<n$. Podľa zadania potom
máme $a_n = a_1 a_2 a_3 \dots a_{n-1} - 1\geq a_1 a_2-1=5$,
takže skutočne $a_n \geq 2$.

Ukážme teraz, že všetky členy $a_n$ sú navzájom
nesúdeliteľné čísla. Pre ľubovoľné dva indexy $k < n$ totiž platí
$a_k \mid a_1 a_2 \dots a_{n-1} = a_n+1$, odkiaľ pre najväčší
spoločný deliteľ~$D$ čísel $a_n$ a $a_k$ dostávame $D \mid a_n$ a
zároveň $D \mid a_n+1$ (pretože $D \mid a_k$ a $a_k \mid a_n+1$),
takže nutne $D = 1$, teda $a_n$ a $a_k$ sú nesúdeliteľné čísla.

Vzhľadom na $a_n \geq 2$ nájdeme pre každý index $n$ prvočíslo,
označme ho $p_n$, pre ktoré $p_n \mid a_n$. Vďaka tomu,
že všetky $a_n$ sú navzájom nesúdeliteľné, všetky nájdené
prvočísla $p_n$ sú navzájom rôzne. Tvrdenie z~časti a) je tak dokázané.

\smallskip
b)
Ak je $n\geq 2$, tak $a_{n+1}=a_1 a_2 a_3 \dots a_{n-1}a_n - 1
=(a_n + 1)a_n - 1 = a_n^2 + a_n - 1$. Ďalej budeme pracovať s týmto
vyjadrením.

Predpokladajme, že platí $p \mid a_n$ pre nejaké $n\geq2$
a pre nejaké prvočíslo $p$. Potom
$a_{n+1} = a_n^2 + a_n - 1 \equiv \m1\pmod p$.
Odtiaľto pre ďalší člen $a_{n+2}$ dostaneme
$a_{n+2} = a_{n+1}^2 + a_{n+1} - 1 \equiv (-1)^2 + (-1) - 1
\equiv \m1 \pmod p$, a tak teraz matematickou indukciou dochádzame
k~záveru, že všetky členy $a_k$ s~indexmi $k\geq n+1$ dávajú po
delení $p$ rovnaký zvyšok ${p-1}$. Ak uvedený
predpoklad $p \mid a_n$ bude pre nejaké $n \geq 2$ splnený,
dané prvočíslo $p$ nazveme {\it zlé}. Našou úlohou je vlastne
nájsť nekonečne veľa prvočísel $p\geq5$, ktoré nie sú zlé
(podmienku $p \geq 5$ kladieme, aby neplatilo $p\mid a_1=3$).

Majme teraz prvočíslo $p$ s~vlastnosťou, že pre nejaké $n\geq2$
platí $a_n \equiv 1 \pmod p$. Potom $a_{n+1} = a_n^2 +
a_n - 1\equiv 1^2 + 1 - 1 \equiv 1 \pmod p$, takže použitím
matematickej indukcie dostávame, že všetky členy $a_k$ s~indexmi
$k\geq n$ dávajú po delení $p$ zvyšok~1. Vtedy dané $p$
nazveme {\it dobré}. Všimnime si, že žiadne prvočíslo $p\geq 5$
nemôže byť dobré aj zlé zároveň -- nie je totiž možné,
aby pre dostatočne veľké $k$ platili obe relácie
$a_k\equiv1\pmod p$ aj $a_k\equiv\m1\pmod p$. Preto nám stačí
dokázať, že existuje nekonečne veľa dobrých prvočísel.

Na hľadanie dobrých prvočísel využijeme postupnosť
$(b_n)_{n=1}^\infty$ zadanú predpisom $b_n = a_n-1$ pre každé
$n\geq1$. Zrejme $b_1 = 2$,
$b_2 = 1$ a pre každé $n\geq2$ platí
$$\eqalign{
b_{n+1} &= a_{n+1} - 1 = (a_n^2 + a_n - 1) - 1 = ((b_n+1)^2 +
(b_n+1) - 1) - 1=\cr
&= b_n^2 + 3b_n = b_n(b_n+3).
}$$
Prvočíslo $p$ je potom dobré práve vtedy, keď $p \mid b_n$
pre nejaké $n \geq 2$. Dostali sme sa tak k situácii podobnej tej,
ktorú sme riešili v~časti a) -- potrebujeme dokázať
existenciu nekonečne veľa prvočísel deliacich aspoň
jeden člen novej postupnosti
$(b_n)_{n=2}^\infty$ určenej prvým členom $b_2=1$ a
vzťahom $b_{n+1}=b_n(b_n+3)$ pre každé $n\geq2$.

Začneme povšimnutím, že $b_k \mid b_n$, ak $2\leq k\leq n$.
Skutočne z~$b_{k+1} = b_k(b_k+3)$ máme $b_k \mid b_{k+1}$ a
odtiaľ už indukciou vyplýva $b_k \mid b_n$ pre každé $n\geq k$.

Teraz dokážeme, že za predpokladu $2\leq k<n$ sú čísla $b_k+3$ a $b_n+3$
nesúdeliteľné. Ich najväčší spoločný deliteľ $D$ totiž spĺňa
$D \mid b_k+3 \mid b_{k+1}\mid b_n$ a zároveň $D \mid b_n+3$,
takže spolu $D \mid (b_n+3)-b_n=3$, a preto buď $D = 1$, alebo
$D = 3$. Zostáva vylúčiť hodnotu $D=3$: Vzhľadom na $b_2=1$
vyplýva zo vzťahu
$b_{n+1} = b_n(b_n+3)$ jednoduchou indukciou relácia $b_n \equiv 1 \pmod 3$
pre každé $n \geq 2$, takže $3 \nmid b_n$, a preto tiež
$3 \nmid b_n+3$, a teda $D\ne3$.

Napokon vieme, že $b_n \geq 1$ pre každé $n$ (lebo $a_n \geq 2$),
a teda $b_n+3 \geq 4$. Pre každé $n$ tak nájdeme prvočíslo $p_n$
s~vlastnosťou $p_n \mid b_n+3$. Všetky tieto prvočísla~$p_n$ sú
podľa predchádzajúceho odseku navzájom rôzne, navyše
z~$b_n+3 \mid b_{n+1}$ vyplýva (pre nás kľúčová) relácia
$p_n \mid b_{n+1}$ pre každé $n \geq 2$.
Našli sme teda potrebnú nekonečnú postupnosť
prvočísel deliacich aspoň jeden člen postupnosti
$(b_n)_{n=2}^\infty$. Dôkaz tvrdenia z~časti b) je ukončený.

  \Pozn
Ukážeme, že tvrdenie z časti a) možno tiež dokázať sporom.
Pripusťme teda, že všetkých prvočísel, ktoré delia niektorý člen
postupnosti $(a_n)_{n=1}^\infty$, je konečný počet~-- označme ich
$p_1,\dots,p_k$. Iste nájdeme taký veľký index $r$, že
medzi deliteľmi prvých $r$~členov $a_1,\dots,a_r$ sú všetky
prvočísla $p_1,\dots,p_k$. Potom však nasledujúci člen
$a_{r+1}=a_1a_2\dots a_r-1$ je celé číslo, nie je deliteľné žiadnym
z~týchto prvočísel, a to je (vzhľadom na nerovnosť $a_{r+1}\geqq2$
z úvodu riešenia) spor.


\návody
{\everypar{}
\smallskip
V~úlohách N1--N4 a D1 je $(a_n)_{n=1}^\infty$ postupnosť zo
zadania súťažnej úlohy.
\smallskip
}

Dokážte, že každé dva členy $a_m$, $a_n$ sú v~prípade $m\ne n$
dve nesúdeliteľné čísla.
[Ak je $m>n$ alebo $m-1\geq n$, tak z~rovnosti
$a_m=a_1a_2a_3\dots a_{m-1}-1$
vyplýva $a_n \mid a_m+1$. Pre najväčší spoločný deliteľ $D$
čísel $a_m$, $a_n$ tak platí $D\mid a_m$ a zároveň
$D\mid a_n \mid a_m+1$, takže aj $D\mid (a_m+1)-a_m=1$, t.\,j.
$D=1$.]

Pre každé $n \geq 3$ vyjadrite $a_n$ iba pomocou $a_{n-1}$.
[$a_n = (a_1 a_2 \dots a_{n-2}) a_{n-1} - 1 = (a_{n-1}
+ 1)a_{n-1} - 1 = a_{n-1}^2 + a_{n-1} - 1.$]


Nech $p\geq3$ je prvočíslo a nech $p\mid a_n - 1$ pre nejaké
$n$. Ukážte, že $p\nmid a_m$ platí pre každé $m\geq n$.
[Keďže $p\geq3$ a $a_1-1=2$, tak $n\ne1$. Z~predpokladu
$a_n\equiv 1 \pmod p$ podľa výsledku N2 dostávame
$a_{n+1} \equiv a_n^2 + a_n - 1 \equiv 1^2+1-1 \equiv 1 \pmod
p$. Matematickou indukciou potom získavame $a_m \equiv 1 \pmod
p$ pre každé $m\geqq n$. Relácia $p \nmid a_m$ je toho dôsledkom.]

Nech $p\geq3$ je prvočíslo a nech $p\mid a_n - 1$ pre nejaké $n$.
Ukážte, že $p\nmid a_m$ platí pre každé $m<n$.
[Pripusťme, že naopak $p \mid a_m$ pre nejaké $m<n$. To spolu s rovnosťou
$a_n=a_1a_2a_3\dots a_{n-1}-1$ znamená, že
$p \mid a_m\mid a_1a_2a_3\dots a_{n-1}=a_n+1$. Spolu máme
$p\mid a_n - 1$ a $p\mid a_n+1$, a teda $p\mid(a_n+1)-(a_n -
1)=2$, čo odporuje predpokladu~$p\geqq3$.]

\D
Dokážte, že čísla $a_m^2 + a_m + 1$ a $a_n^2 + a_n + 1$ sú
v~prípadne $m>n\geq2$ nesúdeliteľná.
[Nech $p$ je prvočíslo a $p \mid a_n^2 + a_n + 1$. Keďže
$3=a_1 \mid a_1a_2\dots a_{n-1}=a_n+1$, tak
$a_n \equiv 2 \pmod 3$, a preto $a_n^2+a_n+1 \equiv
2^2+2+1 \equiv {1}\pmod 3$, takže $p\ne3$. Podľa výsledku N2
postupne dostávame
$a_{n+1}=a_n^2 + a_n - 1=(a_n^2+a_n+1)-2 \equiv \m2 \pmod p$,
$a_{n+2}=a_{n+1}^2 + a_{n+1} - 1 \equiv (-2)^2 + (-2) - 1
\equiv 1 \pmod p$,
$a_{n+3} \equiv 1^2+1-1 \equiv 1 \pmod p$, ďalej už matematickou
indukciou získavame $a_{k} \equiv 1 \pmod p$ pre každé
$k \geq n+2$. Preto číslo $a_m$ (s~indexom $m>n$)
dáva po delení $p$ zvyšok $\m2$ alebo~1,
teda číslo $a_m^2+a_m+1$ dáva zvyšok 1 alebo 3. Z~toho už vyplýva
potrebný záver $p \nmid a_m^2+a_m+1$, lebo (ako už vieme)
$p \ne 3$.]

Dokážte, že existuje nekonečne veľa prvočísel, z ktorých každé
je deliteľom súčtu $2^{2^n} + 1$ pre nejaké prirodzené číslo $n$.
[Opakovaným uplatnením vzorca $2^{2k}-1=(2^k-1)(2^k+1)$ dôjdeme
k~rozkladu $2^{2^n}-1 =
(2^{2^0}+1)(2^{2^1}+1)\dots(2^{2^{n-1}}+1)$. V~prípade $0\leq n<m$
tak platí $2^{2^n}+1 \mid 2^{2^m}-1$. Najväčší spoločný deliteľ
dvoch nepárnych čísel $2^{2^n}+1$ a~$2^{2^m}+1$ je preto tiež deliteľom
čísla $2^{2^m}-1$ a teda aj čísla $\left(2^{2^m}+1\right)-
\left(2^{2^m}-1\right)=2$, takže je to nutne číslo 1. Postupnosť
$\left(2^{2^n}+1\right)_{n=0}^\infty$ je teda zložená
s~navzájom nesúdeliteľných čísel. Ak teda priradíme každému $n$
akýkoľvek prvočiniteľ čísla $2^{2^n}+1$, dostaneme nekonečnú
postupnosť navzájom rôznych prvočísel vyhovujúcich zadaniu úlohy.]

Dokážte, že existuje nekonečne veľa prvočísel, z ktorých každé
je deliteľom rozdielu $2^{2n-1} - 1$ pre nejaké prirodzené číslo $n$.
[Najprv použitím Euklidovho algoritmu dokážte:
{\sl Ak je $d$ najväčší spoločný deliteľ prirodzených čísel $a$
a $b$, tak najväčším spoločným deliteľom čísel $2^a-1$ a $2^b-1$
je číslo $2^d-1$.} V~dôsledku toho platí: Ak sú $p$ a $q$ dve
rôzne prvočísla, tak čísla $2^p-1$ a $2^q-1$ sú nesúdeliteľné.
Ak preto vyberieme ku každému nepárnemu prvočíslu $p$ nejaký
prvočiniteľ čísla $2^p-1$, dostaneme výber nekonečne veľa prvočísel
vyhovujúcich zadaniu úlohy.]

Dokážte, že existuje nekonečne veľa prvočísel, ktoré nie sú
deliteľmi súčtu $2^{2^n} + 1$ pre žiadne prirodzené číslo $n$.
[Vyhovujú všetky prvočísla s~vlastnosťou zo zadania úlohy D3
(ktorých je nekonečne veľa). Zoberme ľubovoľné z~nich, povedzme~$p$,
a vyberme k~nemu dotyčné
$n$ s~vlastnosťou $p\mid 2^{2n-1} - 1$. Pripusťme, že pre nejaké
prirodzené $m$ platí aj $p\mid 2^{2^m} + 1$. Keďže čísla
$2n-1$ a $2^{m+1}$ sú nesúdeliteľné, vďaka tvrdeniu uvedenému v~riešení
D3 sú tiež nesúdeliteľné aj čísla $2^{2n-1} - 1$ a $2^{2^{m+1}}-1$.
Keďže však $2^{2^m} + 1\mid 2^{2^{m+1}}-1$, z~predpokladu
$p\mid 2^{2^m} + 1$ dostávame $p\mid 2^{2^{m+1}}-1$, zároveň však
$p\mid 2^{2n-1} - 1$, teda $p$ je spoločný deliteľ dvoch
nesúdeliteľných čísel, a to je spor.]

Dokážte, že existuje nekonečne veľa prvočísel, ktoré nie sú
deliteľmi rozdielu $2^{2n-1} - 1$ pre žiadne prirodzené číslo $n$.
[Vyhovujú všetky prvočísla s~vlastnosťou zo zadania úlohy D2
(ktorých je nekonečne veľa). Dôkaz sporom je rovnaký ako
v~riešení D4, pretože vychádza z tých istých predpokladov: pre niektoré
prvočíslo $p$ sa nájdu prirodzené čísla $m$, $n$ také, že
$p\mid 2^{2^m} + 1$ a $p\mid 2^{2n-1} - 1$.]

\endnávod
}

{%%%%%   B-I-1
a) Všetky čísla na tabuli sú podľa zadania kladné a rôzne.
V~každom kroku preto určite podčiarkneme najmenšie číslo na tabuli, a~-- pokiaľ nie je na tabuli jediné~-- tiež druhé najmenšie číslo,
lebo ani ono sa nemôže rovnať súčtu dvoch rôznych čísel na
tabuli. Keďže v~každom kroku tak nejaké číslo zotrieme,
po konečnom počte krokov bude tabuľa prázdna,
ako sme mali dokázať.

b) Najskôr poznamenajme, že tabuľa zostane prázdna po
najviac 5 krokoch. To je podľa časti a) zrejmé, ak v žiadnom
kroku nebude zotreté iba jedno číslo. Krok s~jedným zotretým
číslom však môže byť iba jeden, totiž ten posledný;
predchádzajúcich krokov vtedy však nemôže byť viac ako 4.

Teraz uvedieme príklad desiatich východiskových čísel, pre ktoré tabuľa
po štvrtom kroku ešte prázdna nebude.
To určite nastane, ak v každom zo štyroch krokov
budú zotreté len dve čísla (tie
najmenšie).\fnote{V jednom z nich by mohli byť zotreté aj tri
čísla, ale túto možnosť pri našej konštrukcii neuvažujeme.}
Na to musí byť každé
väčšie číslo (\tj. od tretieho najmenšieho) vždy rovné súčtu dvoch rôznych
(menších) čísel, ktoré sa na tabuli doposiaľ nachádzajú.
To nás motivuje uvážiť postupnosť prirodzených čísel,
v~ktorej {\it každé nasledujúce číslo je rovné súčtu dvoch predchádzajúcich
čísel}. Ak zvolíme napríklad za prvé dve čísla 1 a 2, bude
tretie číslo $1+2=3$, štvrté $2+3=5$, atď. Vypíšme prvých desať
týchto čísel (ktoré sa nazývajú {\it Fibonacciho} a hrajú
významnú úlohu v~rôznych oblastiach matematiky aj jej aplikáciách):
$$
1, 2, 3, 5, 8, 13, 21, 34, 55, 89.
$$
Ak budú na začiatku práve tieto čísla napísané na tabuli, potom
v~každom zo štyroch krokov zrejme zotrieme iba dve čísla.
Potvrďme to aj zápisom týchto krokov:
$$\eqalign{
(\underline{1}, \underline{2}, 3, 5, 8, 13, 21, 34, 55, 89 )&\to
(\underline{3}, \underline{5}, 8, 13, 21, 34, 55, 89 )\to\cr
&\to
(\underline{8}, \underline{13}, 21, 34, 55, 89 )\to
(\underline{21}, \underline{34}, 55, 89)\to
(\underline{55},\underline{89}).
}$$
Tým máme vyriešenú aj časť b): Hľadaný najväčší počet krokov je rovný 4.

\návody
{\everypar{}
\smallskip
Budeme sa zaoberať iba námetom zo súťažnej úlohy.
\smallskip
}

Po koľkých krokoch bude tabuľa prázdna, ak je na nej na
začiatku napísaná pätica najmenších prirodzených čísel?
[Po dvoch krokoch. V~prvom kroku zotrieme čísla 1 a~2,
v~druhom kroku zvyšné čísla 3, 4, 5.]

Koľko najmenej prirodzených čísel môže byť na tabuli napísaných,
ak chceme, aby tabuľa zostala prázdna až po dvoch krokoch?
[Ak budú na tabuli najviac dve čísla, bude tabuľa zrejme prázdna
hneď po prvom kroku. Tri napísané čísla niekedy k dvom krokom
vedú~-- všeobecne je to trojica $(a,b,c)$, kde $c=a+b$. Podrobnejšie:
Ak chceme, aby tabuľa po prvom kroku ešte nebola prázdna,
musí byť niektoré z napísaných čísel $c$ súčtom niektorých ďalších
čísel $a$ a $b$ spĺňajúcich $a\ne b$. Keďže v~rovnosti
$c=a+b$ sú všetky čísla kladné, máme okrem $a\ne b$ tiež $c>a$
a $c>b$. Na tabuli teda musia byť napísané aspoň tri čísla, ako
napr. $(1, 2, 3)$.]

Vyriešte variant súťažnej úlohy, v ktorom budeme podčiarkovať a
následne zotierať práve tie čísla, ktoré nie sú {\it súčinom}
žiadnych dvoch rôznych čísel napísaných na tabuli.
[Ak je na tabuli číslo 1, bude v prvom kroku zotreté ako jediné.
Inak v~každom kroku budú medzi zotretými dve najmenšie čísla --
výnimkou môže byť len posledný krok, ak bude pri ňom na tabuli jediné
číslo. Z uvedených poznatkov už vyplýva, že tabuľa bude prázdna
po najviac 6 krokoch. Po 5~krokoch ešte prázdna byť nemusí,
ako ukazuje príklad východiskových čísel 1, 2, 3, 6, 18, \dots,
kde každé číslo počnúc štvrtým je rovné súčinu
dvoch čísel predchádzajúcich. Hľadaný najväčší počet krokov
je teda rovný 5.]

\D
Na začiatku môžeme na tabuľu napísať ľubovoľnú sedmicu
rôznych prirodzených čísel obsahujúcich čísla 1 a 2.
Nájdite všetky také sedmice, pre ktoré
po troch krokoch tabuľa ešte nebude prázdna.
[Takéto sedmice sú štyri: $(1,2,3,4,7,10,17)$, $(1,2,3,4,7,11,18)$,
$(1,2,3,5,8,11,19)$ a $(1,2,3,5,8,13,21)$.
V každom kroku musíme zotrieť iba dve najmenšie čísla.
Usporiadajme čísla vzostupne. Tretie číslo tak musí byť $1+2=3$,
štvrté $1+3=4$ alebo $2+3=5$. Podobne po číslach $3,4$ musia
nasledovať čísla $7,10$ alebo $7,11$ a po číslach $3,5$ čísla
$8,11$ alebo $8,13$. Posledné siedme číslo musí byť súčtom piateho
a~šiesteho čísla.]

Ako by sa zmenili závery súťažnej úlohy, ak by na začiatku
bolo napísaných na tabuli akýchkoľvek desať navzájom rôznych
{\it celých} čísel?
[Už tvrdenie z~časti a) by prestalo platiť -- uvážte napríklad
desaticu $(-3, -2, -1, 0, 1, 2, 3, 4, 5, 6)$, v ktorej
žiadne číslo nepodčiarkneme, a teda ani nezotrieme.]

Majme nejakú východiskovú desaticu rôznych kladných celých čísel,
pre ktorú po štyroch krokoch tabuľa ešte nebude prázdna.
Môže sa najväčšie číslo z~takej desatice rovnať číslu~35?
[Môže:
$$\eqalign{
(\underline{1},\underline{2},3,4,7,10,11,17,18,35)&\to
(\underline{3},\underline{4},7,10,11,17,18,35)\to\cr
&\to(\underline{7},\underline{10},\underline{11},17,18,35)\to
(\underline{17},\underline{18},35)\to(\underline{35}).~]
}$$

\endnávod

}

{%%%%%   B-I-2
Počet $M$ všetkých možných vyplnení tabuľky $3\times 3$ číslami od 1 do 9
má hodnotu $M=9!$ (pozri návodnú úlohu N2).
\def\l{{\mn n}}
\def\s{{\mn p}}


Aby sme určili hodnotu $N$, zaoberajme sa najskôr otázkou,
ako v~tabuľke~$3\times3$ vybrať pozície
pre párne a nepárne čísla, aby vyplnená tabuľka mala súčet
troch čísel v~každom riadku aj stĺpci nepárny. Oplatí sa nám na to
používať $\l$ a $\s$ ako znaky nepárnych, resp. párnych čísel.

Uvedomme si, že súčet troch čísel je $\l$ práve vtedy, keď medzi sčítancami
je buď jedno $\l$, alebo tri $\l$. Preto je nejaké vyplnenie
tabuľky vyhovujúce práve vtedy, keď v~ľubovoľnom riadku aj stĺpci
je buď jedno, alebo tri $\l$. Keďže medzi číslami od 1 do 9 je päť $\l$
(a štyri $\s$), sú počty $\l$ v~jednotlivých riadkoch zrejme rovné
1,\,1,\,3 (v~akomkoľvek poradí); to isté platí pre počty $\l$ v~jednotlivých
stĺpcoch. Oboje nastane práve vtedy, keď všetkých päť $\l$ vypĺňa
zjednotenie jedného riadka a jedného stĺpca (všetky štyri políčka
mimo toto zjednotenie potom vypĺňajú~$\s$).

Keďže riadok pre tri $\l$ je možné vybrať 3~spôsobmi a rovnako tak
stĺpec pre tri $\l$ je možné vybrať 3~spôsobmi, existuje práve
$3\cdot3=9$ vyhovujúcich vyplnení tabuľky $3\times3$
piatimi znakmi~$\l$ a štyrmi znakmi $\s$. Pri každom takom vyplnení
potom (podľa N2) päť znakov~$\l$ môžeme nahradiť číslami 1, 3, 5, 7, 9 práve
$5!$ spôsobmi, štyri~$\s$ potom číslami 2, 4, 6, 8
práve $4!$ spôsobmi. Na nahradenie všetkých znakov v~jednej tabuľke
tak máme $5!\cdot4!$ možností, pritom východiskových tabuliek je~9.
Odtiaľ už vychádza $N=9\cdot5!\cdot4!$.

Pre určené hodnoty $M$ a $N$ platí
$$
\frac{N}{M}=\frac{9\cdot5!\cdot4!}{9!}=
\frac{9\cdot5!\cdot24}{5!\cdot6\cdot7\cdot8\cdot9}=\frac{1}{14},
$$
je teda $N:M=1:14$.

\poznamka
V~našom riešení sme vlastne postupovali tak, že sme každé vyplnenie
tabuľky $3\times3$ číslami od~1 do~9 \uv{zakódovali} ako tabuľku
$3\times3$ vyplnenú piatimi $\l$ a štyrmi~$\s$, podľa ktorej je totiž možné
rozhodnúť, či je východiskové vyplnenie vyhovujúce (alebo nie).
Je zrejmé, že jedna taká tabuľka s~piatimi $\l$ a štyrmi $\s$
zodpovedá rovnakému počtu východiskových vyplnení (v~riešení sme ten počet
$5!\cdot4!$ určili). Hľadaný pomer $N:M$ by sme preto mohli hľadať
ako pomer menších čísel $N':M'$, kde $M'$ je
počet všetkých rozmiestnení piatich znakov $\l$ do políčok tabuľky $3\times3$
a $N'$ je počet ich vyhovujúcich rozmiestnení. Z~kombinatoriky
je známe, že $M'=\binom{9}{5} =
\frac{9\cdot 8 \cdot 7 \cdot 6 \cdot 5}{5\cdot 4 \cdot 3 \cdot 2
\cdot 1} = 9 \cdot 14$ (pozri doplňujúcu úlohu D1)
a z nášho riešenia vieme, že $N'=9$. Tým pádom
$$
\frac{N}{M}=\frac{N'}{M'}=\frac{9}{9 \cdot 14}=\frac{1}{14}.
$$

\návody
Tabuľka $3\times3$ je vyplnená rôznymi celými číslami tak, že
súčty troch čísel v~každom riadku aj stĺpci sú nepárne čísla.
Môžu byť v~tabuľke a) čísla od 1 do 9, b) čísla od 2 do 10?
[a) áno, b) nie. V~prípade a) napríklad môžeme nepárnymi číslami,
ktorých je 5, zaplniť prvý riadok a prvý stĺpec. Keďže
$2+3+{\dots}+10=54$ je párne číslo, musí byť v~prípade b) párny súčet
čísel v aspoň jednom riadku (aj v aspoň jednom stĺpci).]

Nech $k>1$ je celé číslo.
Majme danú tabuľku s~$k$ políčkami, ktorú máme vyplniť $k$~danými
a navzájom rôznymi číslami (tak, aby každé z~nich bolo použité).
Dokážte, že počet všetkých takýchto vyplnení je rovný súčinu
$1\cdot2\cdot\dots\cdot(k-1)\cdot k$. Tento súčin nazývame faktoriálom
čísla $k$ a označujeme symbolom $k!$, ktorý čítame \uv{$k$ faktoriál}.
Kladieme tiež $1!=1$ a $0!=1$.
[Označme políčka číslami od 1 do $k$ a vyberajme pre ne čísla
postupne takto: najskôr pre políčko 1, potom pre políčko 2 atď., až
nakoniec pre políčko~$k$. Počty možností týchto výberov budú
postupne $k$, $k-1$ atď., až 1. Celkový počet vyplnení dostaneme,
keď uvedené počty možností medzi sebou vynásobíme.]

Riešte súťažnú úlohu pre tabuľku $2 \times 2$ a čísla od 1 do 4.
[$1:3$. Medzi číslami od 1 do~4 sú dve nepárne a dve párne, takže všetky
riadkové a stĺpcové súčty budú nepárne, keď nepárne čísla $1$ a $3$
nebudú ležať ani v rovnakom riadku ani stĺpci, \tj. budú na
jednej z oboch diagonál. Vybrať diagonálu pre čísla 1, 3 môžeme
dvoma spôsobmi, umiestniť na ňu čísla 1, 3 dvoma spôsobmi a na
druhú diagonálu čísla 2, 4 tiež dvoma spôsobmi.
Celkom tak existuje práve $2\cdot2\cdot2=8$ vyplnení,
ktoré vyhovujú zadaniu. Keďže podľa N2 je počet
všetkých vyplnení rovný $4!=24$, hľadaný pomer je rovný
$8:24$, čiže $1:3$.]

\D
Majme celé čísla $0 \leq k\leq n$. Koľkými spôsobmi možno z~$n$
guľôčok rôznych farieb vybrať niektorých $k$?
[Vyberajme týchto $k$ guľôčok jednu po druhej. V~prvom kroku máme
$n$~možností, v~druhom $n-1$ možností atď., až v~$k$-tom
kroku máme ${n-k+1}$ možností. Uvedomme si, že ak vyberieme
tie isté guľôčky v inom poradí, dostaneme rovnaký výsledný výber.
Možných poradí $k$ guľôčok je podľa N2 práve~$k!$.
Hľadaný počet $k$-tic je preto rovný
$\frac{n\cdot(n-1)\cdots(n-k+1)}{k!} =
\frac{n!}{(n-k)!\cdot k!}$. Výsledok sa nazýva {\it
kombinačné číslo} a označuje sa symbolom $\binom{n}{k}$, ktorý
čítame \uv{$n$ nad $k$}.]

Riešte súťažnú úlohu pre tabuľku $4 \times 4$ a celé čísla od 1 do 16.
[$N:M=8:715$. Pre počet $M$ všetkých vyplnení platí $M=16!$. Ukážme ďalej,
že pre počet $N$ tých vyplnení, kde sú súčty všetkých čísel v~každom
riadku aj stĺpci nepárne čísla, platí $N=144\cdot8!\cdot8!$. Odtiaľ
už po rutinnom skrátení zlomku $(144\cdot8!\cdot8!)/16!$ vyplynie
uvedený výsledok.
  \hfill\break\hglue2em
Medzi číslami od 1 do 16 je práve osem nepárnych čísel. V~každom
riadku musí byť buď jedno, alebo tri nepárne čísla. Celkom ich je 8,
preto ľahko zistíme, že v dvoch riadkoch $r_1$, $r_2$
musia byť tri nepárne čísla a vo zvyšných dvoch riadkoch $r_3$, $r_4$
jedno nepárne číslo. Rovnaký záver platí aj pre stĺpce --
v dvoch stĺpcoch $s_1$, $s_2$ musia byť tri nepárne čísla
a vo zvyšných dvoch stĺpcoch $s_3$, $s_4$ jedno nepárne číslo.
  \hfill\break\hglue2em
Ukážme, že vo štvorici políčok daných prienikom riadkov $r_1$, $r_2$
so stĺpcami $s_1$, $s_2$ sú len
nepárne čísla. Označme ich počet $x$. Keďže
v~riadkoch $r_1$ a $r_2$ je dokopy $6$ nepárnych čísel, čo platí
aj pre stĺpce $s_1$ a $s_2$, máme $2\cdot 6 - x \leq 8$, odkiaľ
$x\geq 4$, \tj. naozaj $x=4$. Spomínaná štvorica políčok
tak obsahuje štyri nepárne čísla. Ostatné dve nepárne čísla z riadkov
$r_1$ a $r_2$ sa potom nachádzajú po jednom v stĺpcoch $s_3$
a~$s_4$, pre výber ich pozícií tak máme 2~možnosti.
To isté platí pre ostatné dve nepárne čísla zo stĺpcov $s_1$ a $s_2$: pre
výber ich pozícií v~riadkoch $r_3$ a~$r_4$ máme taktiež
2~možnosti. Tým máme popísané možné vyhovujúce
pozície všetkých ôsmich nepárnych čísel.
  \hfill\break\hglue2em
Celkový počet vyhovujúcich výberov pozícií nepárnych a párnych čísel
preto spočítame takto: najskôr zvolíme ľubovoľne dvojicu riadkov
$s_1$, $s_2$ (6 možností) a~dvojicu stĺpcov $r_1$, $r_2$ (6~možností), potom vykonáme výbery pre pozície
nepárnych čísel vo dvojiciach stĺpcov $s_3$,~$s_4$ a $r_3$, $r_4$
(pre každý z~oboch výberov máme ako vieme 2~možnosti).
Počet vyhovujúcich výberov pozícií pre nepárne a párne čísla
je teda ${6\cdot6\cdot2\cdot2}=144$. Odtiaľ už vyplýva vyššie uvedená
hodnota $N=144\cdot8!\cdot8!$, pretože 8 pozícií pre nepárne čísla
rovnako ako 8 pozícií pre párne čísla môžeme vyplniť 8! spôsobmi.]

Do každého políčka štvorcovej tabuľky $n\times n$ vpíšeme jedno
z~čísel $1,2,\dots,n$ tak, aby v~každom riadku aj v~každom stĺpci boli
buď všetky čísla rovnaké, alebo všetky navzájom rôzne. Príkladom pre $n=5$
je nasledujúca tabuľka:
$$
\vbox{%\let\par\cr\catcode`\ =4
% \obeylines\halign{&\hbox{\hss\enspace#\enspace\hss}
%\obeylines
\def\STRUT{\vrule width 0pt height 10.2pt depth 4.2pt}
\halign{\STRUT\vrule#&&\hbox to1.5em{\hss#\hss}\vrule\cr
\noalign{\hrule}
&$5$&$4$&$1$&$2$&$3$\cr
\noalign{\hrule}
&$3$&$3$&$3$&$3$&$3$\cr
\noalign{\hrule}
&$4$&$1$&$2$&$5$&$3$\cr
\noalign{\hrule}
&$1$&$2$&$5$&$4$&$3$\cr
\noalign{\hrule}
&$2$&$5$&$4$&$1$&$3$\cr
\noalign{\hrule}
}}
$$
Označme $S$ súčet všetkých čísel tabuľky. Koľko rôznych hodnôt~$S$ pre
dané $n$ existuje?
[\pdfklink{50-B-I-3}{http://www.matematickaolympiada.cz/media/440634/B50i.pdf}]

Je dané celé číslo $n\geq2$. Koľkými spôsobmi je možné vyfarbiť políčka
tabuľky $n \times n$ štyrmi farbami tak,
aby v~každom štvorčeku $2\times 2$ bola každá farba použitá
práve raz? [$2^3 \cdot 3^{2n-3}$ spôsobmi.
Vyfarbime najprv prvý riadok a prvý stĺpec tak, aby v~žiadnom
štvorci $2\times 2$ nebola žiadna farba použitá viackrát.
Začnime políčkom v ľavom hornom rohu -- pre jeho vyfarbenie máme 4 možnosti. Potom postupne vyfarbujme prvý stĺpec zhora nadol~-- v~každom kroku máme na výber z troch farieb. Nakoniec vyfarbíme
prvý riadok zľava doprava~-- v~prvom kroku máme iba 2~možnosti zafarbenia,
vo zvyšných krokoch máme 3~možnosti. Celkový počet vyhovujúcich
zafarbení prvého riadku a~prvého stĺpca je preto rovný
$4 \cdot3^{n-1}\cdot2\cdot 3^{n-2}$, \tj. $2^3 \cdot 3^{2n-3}$. To
je výsledok úlohy, pretože každé také zafarbenie je možné rozšíriť
na vyhovujúce zafarbenie celej tabuľky práve jedným spôsobom.
Naozaj, akonáhle poznáme farby troch políčok niektorého štvorčeka $2\times 2$,
farba štvrtého políčka je jednoznačne určená; také dofarbovanie môžeme vykonať
napríklad tak, že zľava doprava dofarbíme $n-1$ štvorčekov
najprv v druhom riadku, potom v treťom riadku atď. až nakoniec v~$n$-tom
riadku. Využijeme pritom všetkých $(n-1)^2$ štvorčekov $2\times2$ v~danej
šachovnici, takže získané zafarbenie je vyhovujúce.]

\endnávod
}

{%%%%%   B-I-3
Obe zadané rovnice po dosadení trojčlenov $P$ a $Q$
$$
a(x^2+ax+b)+b(x^2+bx+a)=0,\quad\text{resp.}\quad
a(x^2+bx+a)+b(x^2+ax+b)=0
$$
upravíme na štandardný tvar
$$
(a+b)x^2+(a^2+b^2)x+2ab=0,\quad\text{resp.}\quad
(a+b)x^2+2abx+(a^2+b^2)=0.
\tag1
$$
Vidíme, že v prípade $a+b=0$ nie sú tieto rovnice
kvadratické, čo odporuje zadaniu. Hľadané čísla $a$, $b$ tak
nutne vyhovujú podmienke $a+b\ne0$, o ktorej budeme ďalej
predpokladať, že je splnená.

Ako dobre vieme, kvadratické rovnice majú dvojnásobné korene práve vtedy,
keď sú ich diskriminanty nulové. V prípade rovníc \thetag1 tak
vypísaním ich diskriminantov dostaneme pre neznáme $a$, $b$
sústavu rovníc
$$\eqalign{
(a^2+b^2)^2-4(a+b)\cdot2ab&=0,\cr
(2ab)^2-4(a+b)(a^2+b^2)&=0,
}$$
ktorú prepíšeme na tvar
$$\eqalign{
(a^2+b^2)^2&=8(a+b)ab,\cr
(a+b)(a^2+b^2)&=a^2b^2.
}\tag2
$$
Vynásobme prvú rovnicu výrazom $a^2+b^2$ (nenulovým vďaka
podmienke $a+b\ne0$). Dostaneme rovnicu
$$
(a^2+b^2)^3=8(a^2+b^2)(a+b)ab.
$$
Sem do pravej strany môžeme za $(a^2+b^2)(a+b)$ dosadiť $a^2b^2$
podľa druhej rovnice z~\thetag2. Po tomto zjednodušení tak predchádzajúca rovnica
získa tvar
$$
(a^2+b^2)^3=8a^3b^3,\quad\text{čiže}\quad(a^2+b^2)^3=(2ab)^3.
\tag3
$$
Keďže tretie mocniny dvoch reálnych čísel sa ako je známe rovnajú
práve vtedy, keď sa rovnajú ich základy, z odvodenej rovnice \thetag3
vyplýva $a^2+b^2=2ab$, čiže $(a-b)^2=0$,
čo nastane len v prípade $a=b$. Po dosadení do prvej
z rovníc \thetag2 dostaneme
$$
(a^2+a^2)^2=8(a+a)a^2,\quad\text{po úprave}\quad a^3(a-4)=0.
$$
Vychádzajú tak len dve možnosti -- dvojica $(a,b)$ sa rovná buď $(0,0)$,
alebo $(4,4)$. Keďže dvojica $(0,0)$ odporuje odvodenej podmienke
$a+b\ne0$, do úvahy prichádza iba dvojica $(a,b)=(4,4)$. Tá je
naozaj riešením našej úlohy, pretože je riešením sústavy~\thetag2, ktorá
vyjadruje nulovosť diskriminantov oboch rovníc.\fnote{Aj keď to nie je
nevyhnutné, poznamenajme, že pre $a=b=4$ majú obe rovnice zo zadania
po úprave na~\thetag1 ten istý tvar $8x^2+32x+32=0$,
čo je rovnica s dvojnásobným koreňom~$\m2$, ktorý taktiež
určíme priamo v priebehu druhého riešenia.}

\poznamka
Kľúčovým krokom uvedeného riešenia bolo odvodenie dôsledku \thetag3
sústavy rovníc \thetag2~-- pozri tiež návodnú úlohu N2. Inak je možné
dôsledok \thetag3 získať tiež tak, že rovnice z \thetag2 medzi sebou
vynásobíme a potom výslednú rovnosť vydelíme výrazom $a+b$, o~ktorom už vieme, že sa nerovná nule.

\ineriesenie
Vyjadríme opäť zadané rovnice v štandardnom tvare
$$
(a+b)x^2+(a^2+b^2)x+2ab=0,\quad\text{resp.}\quad
(a+b)x^2+2abx+(a^2+b^2)=0
$$
a poznamenáme, že musí platiť $a+b\ne0$, aby išlo o kvadratické
rovnice. Namiesto počítania s diskriminantmi teraz využijeme iný známy
poznatok o tom, že kvadratická rovnica
$px^2+qx+r=0$ má dvojnásobný koreň $x_0$ práve vtedy, keď
zastúpený trojčlen má rozklad $px^2+qx+r=p(x-x_0)^2$. Našou úlohou
je tak nájsť
práve tie dvojice~$(a,b)$, kde $a+b\ne0$, pre ktoré platia
rozklady
$$\eqalign{
(a+b)x^2+(a^2+b^2)x+2ab&=(a+b)(x+u)^2,\cr
(a+b)x^2+2abx+(a^2+b^2)&=(a+b)(x+v)^2
}\tag4
$$
s~vhodnými reálnymi číslami $u$, $v$ (dvojnásobnými koreňmi rovníc
potom budú čísla $\m u$, resp.~$\m v$). Keďže rovnosti koeficientov
pri mocninách
$x^2$ sú v každom z rozkladov~\thetag4 splnené automaticky, stačí vypísať a ďalej
uvažovať iba rovnosti koeficientov pri mocninách $x^{1}$ a $x^0$:
$$
x^{1}\colon~
\eqalign{
a^2+b^2&=2(a+b)u,\cr
    2ab&=2(a+b)v,
}
\qquad
x^{0}\colon~
\eqalign{
     2ab&=(a+b)u^2,\cr
a^2+b^2&=(a+b)v^2.
}
\tag5
$$
Zdôraznime, že rozklady \thetag4 budú platiť práve vtedy, keď bude
splnená sústava štyroch rovníc~\thetag5. Vyriešime ju vcelku ľahko.

Najprv si uvedomíme, že z $a+b\ne0$ vyplýva $a^2+b^2\ne0$,
teda podľa \thetag5 taktiež platí $u,v,a,b\ne0$. Ak preto porovnáme
dvojice rovníc v~\thetag5 zobratých \uv{do kríža},
dostaneme rovnice $2u=v^2$ a $u^2=2v$. Ich vynásobením
dostaneme $2u^3=2v^3$, čiže $u=v$, a~preto
z~$2u=v^2$ vzhľadom na $u\ne0$ máme $u=v=2$.
Zostáva určiť zodpovedajúce čísla $a$ a~$b$.

Pre odvodené hodnoty $u=v=2$ sa sústava \thetag5 redukuje na
dvojicu rovníc
$$
a^2+b^2=4(a+b)=2ab.
$$
Z rovnosti krajných výrazov však máme
$(a-b)^2=0$, \tj. $a=b$, takže $4(a+b)=2ab$ znamená $8a=2a^2$,
odkiaľ už vzhľadom na $a\ne0$ dostávame jediné vyhovujúce hodnoty
$a=b=4$.\fnote{Ani pri druhom riešení nie je skúška nutná.}

\návody
Rozhodnite, pre ktoré reálne hodnoty parametra $p$ je
$$
(p+2)x^2+2(p+1)x+(p-1)=0
$$
kvadratickou rovnicou s dvojnásobným koreňom.
[Jediná hodnota $p=\m3$.
Ak $p=\m2$, nejde o kvadratickú rovnicu. Ak je
$p\ne\m2$, má táto kvadratická rovnica dvojnásobný koreň
práve vtedy, keď je jej diskriminant nulový. Ten je pritom rovný
${(2(p+1))^2} - {4(p+2)(p-1)}$, po úprave $4(p + 3)$, čo je rovné
nule iba pre $p=\m3$.]

Nech reálne čísla $r$, $s$, $t$ spĺňajú sústavu
dvoch rovníc $r^2=8st$ a $s^2=rt$. Ukážte, že $r=2s$. Rozmyslite si, ako tento výsledok využiť na riešenie súťažnej
úlohy, keď za $r$, $s$, $t$ zvolíte vhodné výrazy.
[Ak vynásobíme prvú rovnicu $r$, dostávame $r^3 = 8rst$.
Na pravej strane potom môžeme nahradiť $rt$ za $s^2$ vďaka druhej
rovnici. Dostaneme $r^3 = 8s^3$, čiže $r^3=(2s)^3$, čo naozaj dáva
$r=2s$, pretože funkcia $y=x^3$ je ako je známe rastúca, a teda prostá. Použitie na riešenie súťažnej úlohy tu prezrádzať nebudeme.]

\D
Nájdite všetky kvadratické trojčleny
$ax^2  + bx + c$
také, že ak ľubovoľný z~koeficientov $a$, $b$, $c$ zväčšíme
o~$1$, dostaneme nový kvadratický trojčlen, ktorý bude mať
dvojnásobný koreň.
[\pdfklink{53-B-II-2}{https://skmo.sk/dokument.php?id=258}]

V obore reálnych čísel $r$, $s$, $t$ riešte sústavu dvoch rovníc
z úlohy N2.
[Riešeniami sú práve trojice $(r,s,t)=(4t,2t,t)$ a
$(r,s,t)=(0,0,t)$, pričom $t$ je v oboch prípadoch ľubovoľné reálne
číslo. Podľa N2 platí nutne $r=2s$; po dosadení takého $r$
získajú rovnice tvar $4s^2=8st$ a $s^2=2st$. Vidíme, že v prípade
$s=0$ je $r=2s=0$ a $t$ je ľubovoľné; v prípade $s\ne0$ sa
obe rovnice $4s^2=8st$ a $s^2=2st$ zjednodušia na $s=2t$, takže
$r=2s=4t$, a teda $(r,s,t)=(4t,2t,t)$, kde $t$ je ľubovoľné.]

Navzájom rôzne nenulové reálne čísla $a$, $b$, $c$ sa dajú šiestimi spôsobmi
doplniť ako koeficienty kvadratickej rovnice
$$
\square x^2+\square x+\square=0.
$$
a) Rozhodnite, či existuje trojica $(a, b, c)$ taká, že
všetky zostavené rovnice majú aspoň jeden reálny koreň.
\hfill\break
b) Rozhodnite, či existuje trojica ($a$, $b$, $c$) taká, že
práve päť zo šiestich zostavených rovníc má aspoň jeden reálny koreň.
[\pdfklink{69-B-II-1}{http://www.matematickaolympiada.cz/media/6510496/b69ii.pdf}]

a) Dokážte nerovnosť $4(a^2+b^2)>(a+b)^2+ab$ pre všetky dvojice
kladných reálnych čísel $a$, $b$.
\hfill\break
b) Nájdite najmenšie reálne číslo $k$ také, aby nerovnosť
$k(a^2+b^2)\geqq(a+b)^2+ab$ platila pre všetky dvojice kladných
reálnych čísel $a$, $b$.
[\pdfklink{70-B-II-1}{https://skmo.sk/dokument.php?id=3604}]

Nájdite všetky reálne riešenia sústavy rovníc
$$
\frac{1}{x+y} + z = 1, \quad \frac{1}{y+z} + x = 1 \quad
\frac{1}{z+x} + y = 1.
$$
[\pdfklink{69-A-II-1}{https://skmo.sk/dokument.php?id=3386}]

\endnávod
}

{%%%%%   B-I-4
Keďže priamky $BC$, $CD$ sú rôznobežné a podľa zadania platí
$CD \parallel AE$, sú priamky $BC$ a $AE$ rôznobežné. Označme $P$
ich priesečník. Zo zadaných podmienok rovnobežnosti vyplýva, že
$PCDE$ je rovnobežník, v ktorom podľa návodnej úlohy~N1 je $A$
vnútorný bod strany $PE$ a $B$ je vnútorný bod strany $PC$ (\obr).
\inspsc{b72i.41}{0.8333}%

Našou úlohou je dokázať rovnosť $|CD|=|DE|$, \tj. ukázať, že
zostrojený rovnobežník $PCDE$ je kosoštvorec (prípadne štvorec).
Využijeme na to známy vzorec $S=zv$ pre obsah rovnobežníka,
podľa ktorého stačí overiť, že výšky z~vrcholu~$D$ na strany $PC$
a $PE$ sú zhodné (potom sú totiž zhodné aj tieto susedné
strany rovnobežníka). Inak povedané, máme overiť, že bod $D$ má
rovnakú vzdialenosť od priamok $BC$ a $AE$. Z~rovnosti
$|\angle BAD|=|\angle DAE|$ však vyplýva, že bod $D$ leží na osi
konvexného uhla $BAE$, takže má bod $D$ rovnakú vzdialenosť od priamok
$AE$ a $AB$ (pozri návodnú úlohu N2). Podobne z~rovnosti
$|\angle CBD|=|\angle DBA|$ vyplýva, že bod~$D$ má rovnakú
vzdialenosť od priamok $AB$ a~$BC$. Dokopy dostávame, že
bod~$D$ má rovnakú vzdialenosť od priamok $BC$ a $AE$, ako sme
chceli ukázať.

\ineriesenie
Rovnako ako v~prvom riešení uvážime rovnobežník $PCDE$ a iným
spôsobom overíme, že to je kosoštvorec (prípadne štvorec). Namiesto
vzorca pre jeho obsah využijeme poznatky
o~kružniciach pripísaných stranám všeobecného trojuholníka (ktoré sú menej
známe, a preto sa im venujeme v~návodnej úlohe~N3~-- na tú sa
už v~ďalšom odseku nebudeme odvolávať).
\inspsc{b72i.42}{0.8333}%

V~prvom riešení sme vyšli z~toho, že bod $D$ leží
na osiach konvexných uhlov $BAE$ a~$ABC$. Inak vyjadrené, bod $D$
leží na osiach vonkajších uhlov pri vrcholoch $A$ a~$B$ trojuholníka $APB$. Kružnica
pripísaná jeho strane $AB$ má teda stred práve v~bode~$D$, ??takže $D$ leží tiež na osi (vnútorného) uhla pri vrchole $P$
tohto trojuholníka (pozri \obr{}).
Tento uhol $APB$ je však totožný s~uhlom $EPC$,
takže vrchol $D$ rovnobežníka $PCDE$ leží na osi jeho
vnútorného uhla pri vrchole $P$, jedná sa preto naozaj
o~kosoštvorec (prípadne štvorec).

\ineriesenie
Obe predchádzajúce riešenia boli založené na poznatku, že bod $D$ má
rovnaké vzdialenosti od troch priamok $AB$, $BC$ a $AE$ (v druhom
riešení to bol polomer pripísanej kružnice). Ak poznáme sínusovú vetu
pre všeobecný trojuholník, môžeme podať tretí variant riešenia
skryto využívajúci rovnaký poznatok
(pozri poznámku za riešením), pri ktorom navyše nevyužijeme
ani pomocný bod $P$. Tento postup zapíšeme nasledovne.

Na \obr{} sú vyznačené tri uhly $\alpha$, $\beta$ a $\gamma$
určené rovnosťami
$$
\alpha=|\angle BAD|=|\angle DAE|,\quad
\beta=|\angle CBD|=|\angle DBA|,\quad
\gamma=|\angle DCB|=|\angle AED|
$$
(zhodnosť dvojice uhlov $DCB$ a $AED$ platí vďaka rovnobežnostiam zo
zadania). Použitím sínusovej vety v~trojuholníkoch $BCD$, $ADE$ a $ABD$
dostaneme postupne rovnosti
$$
\frac{|BD|}{|CD|}=\frac{\sin\gamma}{\sin\beta},\qquad
\frac{|DE|}{|AD|}=\frac{\sin\alpha}{\sin\gamma},\qquad
\frac{|AD|}{|BD|}=\frac{\sin\beta}{\sin\alpha}.
$$
Vynásobením týchto troch rovností dostaneme $|DE|/|CD|=1$, \tj.
$|CD|=|DE|$, ako sme mali dokázať.
\inspsc{b72i.43}{0.8333}%

\poznamka
Tri rovnosti zo sínusovej vety, ktoré sme využili,
vyplývajú z dvoch vyjadrení výšok dotyčných troch trojuholníkov
$$
|CD|\sin\gamma=|BD|\sin\beta,\
|AD|\sin\alpha=|DE|\sin\gamma,\
|BD|\sin\beta=|AD|\sin\alpha,
$$
ktoré sú spomínanými vzdialenosťami bodu $D$ postupne od priamok
$BC$, $AE$, $AB$.

\ineriesenie
Obraz $C'$ bodu $C$ v osovej súmernosti podľa priamky~$BD$ leží na
polpriamke $BA$ tak, že $|C'D|=|CD|$ a $|\uhol DC'B|=|\uhol
DCB|$ (\obr). Podobne obraz~$E'$ bodu~$E$ v osovej súmernosti podľa priamky~$AD$ leží na polpriamke $AB$ tak, že $|E'D|=|ED|$ a
$|\uhol DE'A|=|\uhol DEA|$. Ak je $C'=E'$, sme hotoví:
$|CD|=|C'D|=|E'D|=|ED|$. Rovnaká séria rovností platí aj
v prípade $C'\ne E'$, pretože trojuholník $DC'E'$ má zhodné uhly pri vrcholoch
$C'$ a $E'$.
Z rovnobežností $BC\parallel DE$ a $CD\parallel AE$ totiž
vyplýva zhodnosť uhlov $DCB$ a $DEA$, a teda aj uhlov $DC'B$ a $DE'A$,
ktoré sú oba vnútorné alebo oba vonkajšie uhly trojuholníka $DC'E$~--keby jeden z~ich bol vnútorný uhol a druhý vonkajší uhol,
mal by ten prvý menší rozmer.\fnote{Upozornime, že
body $A$, $B$, $C'$, $E'$ môžu ležať na priamke v iných poradiach
ako na obrázku; preto sa nestačí odkázať na to, čo je z~(jedného)
obrázku vidno.}
\inspsc{b72i.45}{0.8333}%


\návody
Rozmyslite si a zdôvodnite, že päťuholník $ABCDE$ zo súťažnej úlohy
je možné získať z~istého rovnobežníka \uv{odstrihnutím} jedného jeho
\uv{rohu}.
[Zadaný konvexný päťuholník leží ako v~páse medzi rovnobežkami
$BC$ a $DE$, tak v~páse medzi rovnobežkami $CD$ a $AE$. Preto leží
aj v~prieniku týchto dvoch pásov, ktorým je rovnobežník $PCDE$, pričom
$P$ je priesečník (rôznobežných) priamok $BC$ a $AE$. Keďže
zvyšné vrcholy $A$, $B$ sú vnútorné body strán~$PE$, resp. $PC$, od
rovnobežníka $PCDE$ oddelíme trojuholník $APB$.]

Pripomeňme, že osou ľubovoľného (konvexného aj nekonvexného) uhla
s~vrcholom~$V$ nazývame tú polpriamku s~počiatočným bodom $P$,
ktorá daný uhol rozdeľuje na dva zhodné uhly (\tj. uhly rovnakej
veľkosti). Pripomeňte si tiež a dokážte \uv{vetu o~osi uhla}:
{\sl\it Os uhla~$AVB$, ktorý má veľkosť menšiu ako $180^{\circ}$,
je tvorená práve tými jeho bodmi, ktoré majú od oboch priamok $VA$, $VB$
rovnakú vzdialenosť.}
[Označme $\alpha=|\uhol AVB|$. Stačí uvažovať len vnútorné
body uhla $AVB$, nech $X$ je ľubovoľný z~ich. Pri označení
$\alpha_1=|\uhol AVX|$ a~$\alpha_2=|\uhol XVB|$ platí
$\alpha_1+\alpha_2=\alpha<180^{\circ}$ a bod~$X$ má od priamok
$VA$, $VB$ vzdialenosti $|VX|\sin\alpha_1$, resp. $|VX|\sin\alpha_2$.
Tie sa preto rovnajú práve vtedy, keď platí $\sin\alpha_1=\sin\alpha_2$, čo pre
konvexné uhly nastane len v dvoch prípadoch: $\alpha_1=\alpha_2$ alebo
$\alpha_1+\alpha_2=180^{\circ}$. Druhý z nich je však vyššie vylúčený;
prvý prípad znamená práve to, že $X$ je vnútorným bodom
osi uhla~$AVB$.]

Kružnicou pripísanou (ku) strane $KL$ všeobecného trojuholníka $KLM$ rozumieme tú
kružnicu, ktorá sa dotýka strany $KL$ v jej vnútornom bode
a priamok $KM$, $LM$ v~bodoch ležiacich vo vnútri polroviny
opačnej k~polrovine $KLM$.
Dokážte, že stred takej kružnice je priesečníkom osí vonkajších
uhlov pri vrcholoch $K$, $L$ trojuholníka~$KLM$ a~že ním prechádza tiež
os jeho vnútorného uhla pri vrchole~$M$.
[Priesečník $S$ spomínaných dvoch osí vonkajších uhlov je vnútorným bodom
uhla $KML$ ležiacim v~polrovine opačnej k~polrovine $KLM$ a
má podľa N2 rovnakú vzdialenosť $v$ od všetkých troch priamok $KM$, $KL$
a $LM$, teda (opäť vďaka N2) leží aj na osi uhla~$KML$.
Kružnica so stredom $S$ a polomerom $v$ je potom zrejme pripísaná
strane $KL$ trojuholníka $KLM$.]

\D
Dokážte, že pre päťuholník $ABCDE$ zo súťažnej úlohy
platí $|\uhol CDE|>60^{\circ}$.
[Nech $P$ je priesečník priamok $BC$ a $AE$. Potom $PCDE$ je
rovnobežník s~bodmi $A$, $B$ vo vnútri strán~$PE$, resp. $PC$.
Označme $\alpha=|\angle BAD|=|\angle DAE|$, $\beta=|\angle CBD|=|\angle DBA|$
a $\delta=|\angle CDE|=|\angle EPC|$. Zo súčtu vnútorných uhlov
$\triangle APB$, ktorý má vyjadrenie
${(180^{\circ}-2\alpha)}+{(180^{\circ}-2\beta)}+\delta=180^{\circ}$,
vyplýva rovnosť $\alpha+\beta=90^{\circ}+\frac12\delta$. Ak porovnáme
v~$\triangle ADE$ vnútorný uhol pri vrchole $A$ s~vonkajším uhlom pri
vrchole~$E$, dostaneme $\alpha<\delta$. Rovnako tak z~$\triangle BCD$
dostaneme $\beta<\delta$. Sčítaním dvoch odvodených nerovností
vychádza $\alpha+\beta<2\delta$, takže z~rovnosti $\alpha+\beta=90^{\circ}+\frac12\delta$
vyplýva $90^{\circ}+\frac12\delta<2\delta$, odkiaľ $\delta>60^{\circ}$, ako
sme mali dokázať.]

Nech $ABC$ je ostrouhlý trojuholník s najdlhšou stranou $BC$.
Vnútri strán $AB$ a $AC$
ležia postupne body $D$ a $E$ tak, že $|CD| = |CA|$ a $|BE| =
|BA|$. Označme $F$ taký bod,
že $ABFC$ je rovnobežník. Ukážte, že $|FD| = |FE|$.
[\pdfklink{71-B-I-2}{https://skmo.sk/dokument.php?id=3924}]

V~trojuholníku $ABC$ označme $I$ stred kružnice vpísanej.
Priamky $BI$, $CI$ pretnú kružnicu opísanú trojuholníku $ABC$
postupne v~bodoch $S\ne B$, $T\ne C$. Úsečka~$ST$ pretína strany
$AB$, $AC$ v~bodoch $K$, $L$. Dokážte, že štvoruholník $AKIL$ je
kosoštvorec (prípadne štvorec).
[\pdfklink{71-A-S-2}{https://skmo.sk/dokument.php?id=3918}]

V~ostrouhlom trojuholníku $ABC$ sú $D$ a $E$ vnútorné body strany~$BC$,
pritom~$D$ leží medzi $B$ a~$E$, $|AD|=|CD|$ a~$|AE|=|BE|$.
Predpokladajme, že os uhla $DAE$ má s~osou úsečky~$BC$ jediný
spoločný bod, ktorý označíme~$F$. Dokážte rovnosť
$|\uhel BAC|+|\uhel DFE|=180\st$.
[\pdfklink{70-A-II-3}{https://skmo.sk/dokument.php?id=3471}]

\endnávod
}

{%%%%%   B-I-5
a) Skúšaním malých hodnôt $a$, $b$, $c$ môžeme nájsť napríklad
trojicu $(a,b,c)=(4,1,2)$, ktorá vyhovuje podmienke $ab=c^2$ a
zároveň pre ňu platí rovnosť $a+b-2c=1$. Ak teraz vynásobíme každé
z~týchto čísel $a$, $b$, $c$ daným prvočíslom~$p$, podmienka $ab=c^2$
zostane zachovaná a hodnota výrazu $a+b-2c$ sa zmení
z~$1$ na $p$. Dostávame tak príklad trojice $(a,b,c)=(4p,p,2p)$,
ktorá má požadované vlastnosti.
\smallskip

b) Označme $d$ najväčší spoločný deliteľ čísel $a$, $b$.
Potom $a = a'd$ a~$b=b'd$, pričom $a'$ a $b'$ sú nesúdeliteľné kladné celé čísla.
Keďže súčin $ab$ má byť druhou mocninou kladného celého čísla $c$,
z~rovnosti $c^2=ab=(a'b')d^2$ podľa návodnej úlohy N2 vyplýva,
že aj súčin $a'b'$ musí byť druhou mocninou kladného celého čísla. To
vďaka návodnej úlohe N3 znamená, že samotné čísla $a'$, $b'$ musia byť
druhými mocninami nejakých kladných celých čísel $u$ a $v$,
teda $a'=u^2$ a $b'=v^2$. Čísla $a$, $b$ tak sú tvaru $a=u^2d$
a~$b=v^2d$. Po ich dosadení do rovnosti $ab=c^2$ dostaneme $(uvd)^2=c^2$,
odkiaľ $c=uvd$. Celkom tak každé riešenie rovnice $ab=c^2$ v~obore
kladných celých čísel má tvar
$$
a = u^2d, \quad b = v^2d, \quad c = uvd.
$$
Zadaný výraz $a+b+2c$ preto má hodnotu
$$
a+b+2c =u^2d + v^2d + 2uvd = (u+v)^2d.
\tag1
$$
Keďže $u,v\geq1$, platí $u+v\geq 2$. Číslo $(u+v)^2$ je tak
zložené. Vďaka \thetag1 je preto zložené aj číslo $a+b+2c$,
ako sme mali dokázať.

\poznamka
Možno dokázať (pozri doplňujúcu úlohu D1), že každá trojica $(a,b,c)$
vyhovujúca zadaniu časti a) je daná vzťahmi
$$
\{a,b\}=\{(n+1)^2 p,n^2 p\}\quad\text{a}\quad c=n(n+1)p,
$$
pričom $n$ je ľubovoľné kladné celé číslo. Napríklad pre $n=1$
dostávame za podmienky $a>b$ trojicu $(a,b,c)=(4p,p,2p)$.

\ineriesenie
Najprv dokážeme tvrdenie z~časti b), a to sporom. Uvedomme si, že
číslo $a+b+2c$ je aspoň 4, takže nie je zložené práve vtedy, keď je prvočíslo.
Pripusťme teda, že existuje niektorá trojica $(a,b,c)$ a prvočíslo $p$ také,
že $a+b+2c=p$. Potom $a+b=p-2c$, odkiaľ
umocnením a využitím rovnosti $ab=c^2$ dostaneme
$$
(a+b)^2=p^2-4pc+4c^2=p^2-4pc+4ab.
$$
Túto rovnosť upravíme na tvar $(a-b)^2=p(p-4c)$. Z~neho vidíme,
že prvočíslo~$p$ je deliteľom rozdielu $a-b$. Ak by sa však
kladné čísla $a$, $b$ navzájom líšili o~nenulový násobok čísla $p$,
mali by sme $a+b+2c>a+b>p$, a to je spor. Preto platí $a=b$, takže z~rovnosti
$ab=c^2$ vyplýva $a=b=c$. Potom však $p=a+b+2c=4c$, teda prvočíslo $p$
je násobkom štyroch, a to je spor.

Podobným postupom môžeme aj pre časť a) hľadať k~danému prvočíslu $p$
trojicu $(a,b,c)$ spĺňajúcu rovnosť $a+b-2c=p$. Jej úpravou
tentoraz dostaneme $(a-b)^2=p(p+4c)$. Odtiaľ opäť vyplýva, že čísla
$a$, $b$ sa líšia o~násobok čísla $p$, \tj. $a=b+kp$ pre vhodné
celé číslo. Po dosadení za $a$ do rovnosti $a+b-2c=p$ dostaneme
po úprave $2(c-b)=(k-1)p$. Vidíme, že voľbou $k=3$ bude
rovnosť $2(c-b)=(k-1)p$ splnená, pokiaľ bude $c-b=p$,
\tj. $c=b+p$. Ak dosadíme také~$c$ spolu s~$a=b+3p$ do rovnosti
$ab=c^2$, dostaneme $(b+3p)b=(b+p)^2$, po úprave $bp=p^2$, čiže
$b=p$. Opäť tak dostávame trojicu $(a,b,c)=(4p,p,2p)$.

\návody
Existuje nejaká trojica $(a,b,c)$ prirodzených čísel
spĺňajúce podmienky $ab=c^2$ a~${a+b-2c} = 1$? Ak áno, ako by sa
dala využiť na riešenie časti a) súťažnej úlohy pre ľubovoľné
prvočíslo $p$?
[Áno, ale prezrádzať ju nebudeme, nieto ešte spôsob, ako by sa dala
využiť. Skúste vypísať všetky trojice $(a,b,c)$ prirodzených čísel
spĺňajúcich rovnicu $ab=c^2$ pre malé hodnoty $c$.
Vyhovuje niektorá z~ich aj druhej podmienke?]

Dokážte, že ak pre prirodzené čísla $u$, $v$, $w$ platí
$u^2=v^2w$, je číslo $w$ druhou mocninou prirodzeného čísla.
[Uvážme ľubovoľné prvočíslo $p$ deliace číslo~$w$.
Z~rovnosti $u^2=v^2w$ vyplýva, že $p$ je aj deliteľ čísla $u$,
takže číslo $u^2$ má vo svojom rozklade na prvočinitele
párny počet výskytov $p$,
ktorý však musí byť väčší ako prípadný párny počet výskytov $p$
v~rozklade čísla $v^2$. Odtiaľ už vyplýva, že $p$ má tiež párny
počet výskytov v~rozklade čísla $w$, ktoré je teda druhou
mocninou prirodzeného čísla (platí to aj v~prípade $w=1$).]

Dokážte, že ak súčin dvoch nesúdeliteľných prirodzených čísel
$u$, $v$ je rovný druhej mocnine celého čísla, sú
obe čísla $u, v$ tiež druhými mocninami celých čísel.
[Uvážme rozklady čísel $u$, $v$ a $uv$ na prvočinitele.
V~rozklade druhej mocniny rovnej číslu $uv$
má každé prvočíslo párny počet výskytov.
Čísla $u$, $v$ však nemajú žiadne spoločné
prvočinitele, preto aj v ich rozkladoch má každé prvočíslo
párny počet výskytov (v~jednom z~rozkladov $u$, $v$ je to nula, v druhom
rovnaký počet ako v~rozklade $uv$).]

\D
Pre dané prvočíslo $p$ nájdite {\it všetky\/} trojice
$(a,b,c)$ kladných celých čísel spĺňajúcich obe rovnosti
$ab=c^2$ a $a+b-2c=p$. [$\{a,b\}=\{(n+1)p,np\}$ a $c=n(n+1)p$,
kde $n$ je ľubovoľné prirodzené číslo.
Označme $d$ najväčší spoločný deliteľ čísel $a$, $b$. Keďže
súčin $ab$ má byť druhou mocninou celého čísla, podľa N2 a N3
platí $a = u^2d$, $b = v^2d$ pre vhodné prirodzené $u$, $v$, takže
$c=uvd$. Dosadením do $a+b-2c=p$ po jednoduchej úprave dostávame
$d(u-v)^2 = p$. Keďže $p$ je prvočíslo, musí byť nutne
$(u-v) =\pm 1$ a~$d=p$. V~prípade $u-v=1$ máme $a=(v+1)^2p$, $b=v^2p$
a $c=v(v+1)p$; v~prípade $u-v=\m1$ podobne $a=u^2p$, $b=(u+1)^2p$ a
$c=u(u+1)p$.]

Pravouhlý trojuholník má celočíselné dĺžky strán a obvod 11\,990. Navyše vieme, že jedna
jeho odvesna má prvočíselnú dĺžku. Určte ju.
[\pdfklink{71-B-I-1}{https://skmo.sk/dokument.php?id=3924}]

Nájdite všetky dvojice prvočísel $p$, $q$, pre ktoré platí
$p+q^{2}=q+145p^{2}$.
[\pdfklink{55-C-II-4}{https://skmo.sk/dokument.php?id=241}]

Určte všetky dvojice prvočísel $p$, $q$, pre ktoré platí
$p+q^2=q+p^3$.
[\pdfklink{55-B-II-1}{https://skmo.sk/dokument.php?id=238}]


Kladné celé čísla $a$, $b$ spĺňajú rovnosť $b^2 = a^2 + ab + b$.
Dokážte, že $b$ je druhou mocninou kladného celého čísla.
[\pdfklink{69-A-III-4}{https://skmo.sk/dokument.php?id=3448}]

\endnávod
}

{%%%%%   B-I-6
Keďže trojuholník $ABX$ je pravouhlý s pravým uhlom pri vrchole
$B$ a~my máme dokázať, že je navyše rovnoramenný, je naším cieľom
odvodiť rovnosť $|AB|=|BX|$.

Označme $k$ kružnicu, na ktorej podľa zadania ležia body $B$, $I$, $M$
a $C$, a to zrejme v~tomto poradí. Podľa vety o~obvodových uhloch
sú uhly $BIC$ a $BMC$ nad tetivou~$BC$ kružnice~$k$ zhodné.
Ich veľkosti teraz vyjadríme pomocou veľkosti $\alpha$ vnútorného
uhla pri vrchole $A$ trojuholníka $ABC$, aby sme potom ich porovnaním
uhol $\alpha$ určili.

Podľa Tálesovej vety je bod $M$ stredom kružnice opísanej
pravouhlému trojuholníku~$ABC$, teda vďaka vete o~obvodovom a stredovom uhle
platí $|\uhol BMC|=2\alpha$. Dodajme
ešte jeden dôsledok Tálesovej vety, ktorý využijeme za chvíľu:
Rovnosti $|MA|=|MB|=|MC|$ znamenajú, že oba trojuholníky $ABM$ a $BCM$
sú rovnoramenné.

Na určenie veľkosti uhla $BIC$ využijeme to, že polpriamky
$BI$ a $CI$ sú osami vnútorných uhlov trojuholníka $ABC$ (\obr). Preto pri
štandardnom označení ich veľkostí dostávame\fnote{Nie je podstatné,
že $\beta=90^{\circ}$ -- odvodený vzorec pre
$|\uhol BIC|$ platí vo všeobecnom trojuholníku~$ABC$,
ako uvádzame v návodnej úlohe N3.}
$$
|\uhol BIC|=180^{\circ}-\tfrac12\beta-\tfrac12\gamma=
90^{\circ}+\tfrac12\alpha.
$$
\inspsc{b72i.61}{0.8333}%

Zhodnosť uhlov $BIC$ a $BMC$ teda znamená, že platí
$2\alpha=90^{\circ}+\frac12\alpha$, čiže
$\alpha=60^{\circ}$. Rovnoramenný trojuholník $ABM$ je teda
rovnostranný:
$$
|AB|=|BM|=|MA|,
\tag1$$
zatiaľ čo druhý rovnoramenný trojuholník $BCM$ má pri základni $BC$
vnútorné uhly veľkosti $\gamma=90\st-\alpha=30\st$.

Určené uhly teraz využijeme na výpočet niektorých ďalších
uhlov v~tetivovom štvoruholníku~$BCMI$ s~opísanou kružnicou~$k$.
Z rovností
$|\uhol BCM|=30\st$ a $|\uhol IBC|=45\st$ dostávame
$$\eqalign{
|\uhol BMI|&=|\uhol BCI|=\tfrac12|\uhol BCM|=15\st,\cr
|\uhol CMI|&=180\st-|\uhol IBC|=135\st.
}$$
Tým pádom $|\uhol BCM|+|\uhol CMI|=30\st+135\st=165\st<180\st$,
odkiaľ vyplýva, že bod~$X$ zo zadania úlohy je spoločným bodom
polpriamok $CB$, $MI$ a že v~trojuholníku~$CMX$
platí $|\uhol MXC|=180\st-165\st=15\st$.
Všimnime si ešte, že bod $B$ je vnútorným bodom úsečky~$CX$
(lebo leží v polrovine $MIC$ vďaka konvexnosti $BCMI$).
Preto platí
$$
|\uhol MXB|=|\uhol MXC|=15\st\quad\text{a}\quad
|\uhol BMX|=|\uhol BMI|=15\st.
$$
Trojuholník $BMX$ je teda rovnoramenný so základňou $MX$, takže
$|BM|=|BX|$. To spolu s~\thetag1 už vedie k vytúženému záveru, že
totiž $|AB|=|BX|$.


\návody
Na kružnici so stredom $O$ sú dané body $B$ a $C$ také, že
$|\uhol BOC|=120^\circ$. Zvoľme bod $A$ na
dlhšom oblúku $BC$ a označme $|\uhol AOB|=\delta$. a)~Zistite veľkosť uhla $BAC$, keď
$\delta=140^\circ$. b)~Zistite, ako máme voliť uhol $\delta$, aby bol uhol $BAC$ čo najväčší. c)~Na kratšom oblúku $BC$ zvolíme bod $A'$.
Zistite, ako máme voliť polohy bodov $A$, $A'$
(oba ležia na danej kružnici), aby súčet $|\uhol BAC|+|\uhol BA'C|$
bol čo najväčší.
[V~rovnoramenných trojuholníkoch $BOC$, $COA$ a $AOB$ spočítajte
uhly, alebo ich vyjadrite v~závislosti na uhle~$\delta$. V~časti a)
vyjde $|\uhol BAC|=60^\circ$, rovnako ako v~časti b), a to nezávisle na
voľbe $\delta$.\fnote{Oba fakty sú dôsledkom známej {\it vety
o obvodových a stredových uhloch} v ľubovoľnej kružnici.}
V~časti~c) vyjde súčet $180^\circ$ nezávisle od
polohy bodu $A$ alebo $A'$.\fnote{Výsledok časti c) má známe zovšeobecnenie:
Konvexný štvoruholník je tetivový (\tj. jeho vrcholy ležia na jednej
kružnici) práve vtedy, keď súčet veľkostí jeho protiľahlých vnútorných
uhlov je $180^\circ$.}]

Majme daný konvexný štvoruholník $PQRS$. Dokážte, že jeho vrcholy
ležia na jednej kružnici práve vtedy, keď $|\uhol PRQ|=|\uhol PSQ|$.
[Na úvod konštatujme, že vďaka konvexnosti $PQRS$ ležia
vrcholy $R$, $S$ posudzovaných uhlov $PRQ$, $PSQ$ vo vnútri
tej istej polroviny s hraničnou priamkou $PQ$.
Predpokladajme najprv, že štvoruholník $PQRS$ má všetky štyri
vrcholy na jednej kružnici. Rovnosť $|\uhol PRQ|=|\uhol PSQ|$
je potom rovnosťou dvoch obvodových uhlov tejto kružnice, ktoré
prislúchajú tomu istému oblúku $PQ$.
  \hfill\break\hglue2em
Predpokladajme naopak, že uhly $PRQ$ a $PSQ$ sú zhodné a
označme $\varphi$ ich veľkosť. Dokážeme, že kružnice opísané
trojuholníkom $PRQ$ a $PSQ$ majú rovnaký stred, tým pádom aj rovnaký polomer.
V prípade $\varphi=90\st$ je to dôsledok Tálesovej vety. Posúďme
teraz prípad $\varphi<90\st$. Stredy kružníc opísaných
trojuholníkom~$PRQ$ a $PSQ$ potom ležia vo vnútri polroviny~$PQR=PQS$ a
každý z nich tvorí s bodmi $P$ a~$Q$ rovnoramenný trojuholník
so základňou $PQ$, ktorý má podľa vety o~obvodovom a stredovom uhle
pri hlavnom vrchole uhol $2\varphi$. Preto tieto dva stredy
splývajú. V~prípade $\varphi>90\st$ potom stredy oboch opísaných kružníc
ležia v~polrovine opačnej k~polrovine~$PQR=PQS$ a
zodpovedajúce rovnoramenné trojuholníky vtedy majú pri hlavnom vrchole uhol
$360\st-2\varphi$.\fnote{Dokázané
tvrdenie je okamžitým dôsledkom tzv. {\it vety o ekvigonále úsečky\/}:
Množina všetkých bodov, z ktorých danú úsečku $PQ$
vidno pod daným uhlom $\alpha$, pričom
$0\st<\alpha<180\st$, je tvorená vnútornými bodmi dvoch kružnicových
oblúkov s krajnými bodmi $P$ a $Q$, ktoré sú súmerne združené
podľa priamky $PQ$.}]

V~trojuholníku $ABC$ označme $I$ stred kružnice vpísanej a
$\alpha$ veľkosť vnútorného uhla pri vrchole $A$. Vyjadrite
veľkosť uhla $BIC$ pomocou $\alpha$.
[$90^\circ+\frac12\alpha$. Označme postupne $\beta$,~$\gamma$
veľkosti vnútorných uhlov pri vrcholoch $B$ a $C$. Keďže bod $I$ leží na osiach oboch týchto
uhlov, zo súčtu vnútorných uhlov v trojuholníku $BIC$ dostávame
$|\uhol BIC| = 180^\circ - \frac12\beta - \frac12\gamma=
90^\circ+\frac12\alpha$.]

V~pravouhlom trojuholníku $ABC$ označíme $M$ stred prepony $AC$
a $\alpha$ veľkosť vnútorného uhla pri vrchole~$A$.
Vyjadrite pomocou $\alpha$
veľkosti všetkých vnútorných uhlov v~trojuholníkoch $ABM$ a~$BCM$.
[Trojice $(\alpha,\alpha,180\st-2\alpha)$ a $(90\st-\alpha,90\st-\alpha,2\alpha)$.
Využite to, že vďaka Tálesovej vete sú oba trojuholníky $ABM$ a $BCM$
rovnoramenné s~hlavným vrcholom $M$.]

\D
Daný je pravouhlý trojuholník $ABC$ s pravým uhlom pri vrchole $C$.
Nech $D$ je ľubovoľný vnútorný bod odvesny $AC$ a $p$ kolmica z bodu $D$
na preponu $AB$. Označme $E\ne D$ bod priamky $p$ taký, že
body $A$, $B$, $D$, $E$ ležia na kružnici. Označme ešte~$F$
priesečník priamok $p$ a $BC$. Dokážte, že $|AE|=|AF|$.
[\pdfklink{70-B-II-3}{https://skmo.sk/dokument.php?id=3604}]

Nech $ABCD$ je konvexný štvoruholník, v~ktorom $AD\perp BD$. Označme $M$
priesečník jeho uhlopriečok a~zostrojme kolmý priemet~$P$ bodu~$M$ na
priamku~$AB$ a~kolmý priemet~$Q$ bodu~$B$ na priamku~$AC$.
Dokážte, že bod~$M$ je stredom kružnice vpísanej trojuholníku~$PQD$.
[\pdfklink{68-B-I-5}{https://skmo.sk/dokument.php?id=3042}]

Dokážte, že stredy kružníc zvonka pripísaných jednotlivým stranám
ľubovoľného konvexného štvoruholníka ležia na jednej kružnici.
\hfill\vadjust{\smallskip}\break
(Kružnicou pripísanou napríklad strane~$AB$ konvexného štvoruholníka $ABCD$ rozumieme
kružnicu, ktorá sa dotýka strany~$AB$ a~polpriamok opačných
k~polpriamkam $AD$ a~$BC$.)
[\pdfklink{69-B-S-2}{https://skmo.sk/dokument.php?id=3391}]

Daná je kružnica~$k$ a~jej priemer~$AB$. Vnútri úsečky~$AB$
zvolíme ľubovoľný bod~$C$ a~potom na kružnici~$k$ vyberieme bod~$D$
tak, aby platilo $|BC|=|BD|$. Os uhla $ABD$ pretína
kružnicu~$k$ v~bode~$E$ (rôznom od bodu~$B$).
Dokážte, že trojuholníky $AEC$ a~$CBD$ sú podobné.
[\pdfklink{68-B-S-3}{https://skmo.sk/dokument.php?id=3046}]

Označme~$I$ stred kružnice vpísanej pravouhlému trojuholníku $ABC$
s~pravým uhlom pri vrchole~$A$. Ďalej označme $M$ a~$N$ stredy
úsečiek $AB$ a~$BI$. Dokážte, že priamka~$CI$ je dotyčnicou kružnice
opísanej trojuholníku $BMN$.
[\pdfklink{70-A-III-2}{https://skmo.sk/dokument.php?id=3576}]

\endnávod
}

{%%%%%   C-I-1
Menovatele zadaných zlomkov sú prirodzené čísla od 1 do
2022. Pritom zlomok s~daným menovateľom $j$ má čitateľ $c$ určený
rovnosťou $c+j=2022$, teda $c=2022-j$. Naše zlomky preto majú
vyjadrenie, ktoré rovno ešte upravíme:
$$
\frac{2022-j}{j}=\frac{2022}{j}-1,\quad\hbox{pričom}\quad j=2022,2021,\dots,1.
$$
(Hodnoty menovateľov $j$ sme vypísali zostupne, ako je to
pri zlomkoch v~zadaní.)

Podľa zadania je našou úlohou určiť,
koľko zo všetkých menovateľov~$j$ delí príslušného čitateľa $2022-j$.
Vďaka vykonanej úprave je naša úloha zjednodušená: hľadáme počet tých~$j$
od~1 do~2022, ktoré sú deliteľmi čísla~2022.

Keďže číslo 2022 má prvočíselný rozklad $2022=2\cdot3\cdot337$,
jeho delitele sú práve čísla 1, 2, 3, 6, 337, 674, 1011 a 2022,
ktorých je celkom osem. To sú teda menovatele~$j$ všetkých uvažovaných
zlomkov s~celočíselnými hodnotami. Ich čitatele $c=2022-j$ sú
postupne čísla $2021$, $2020$, $2019$, $2016$, $1685$, $1348$,
$1011$ a $0$.

\zaver
Celočíselné hodnoty nadobúda {\it práve osem\/} zo zadaných
zlomkov, konkrétne
$$
\frac{0}{2022}, \,\frac{1011}{1011}, \,\frac{1348}{674},
\,\frac{1685}{337}, \,\frac{2016}{6}, \,\frac{2019}{3},
\,\frac{2020}{2}, \,\frac{2021}{1}.
$$


\poznamky
Poznatok o tom, že 337 je prvočíslo, je možné overiť nahliadnutím do
tabuliek prvočísel. Ak ich nemáme k~dispozícii, stačí otestovať, či
číslo 337 nie je deliteľné niektorým prvočíslom, ktoré neprevyšuje
hodnotu $\sqrt{337}$ (ktorá leží medzi číslami 18 a 19, pretože
$18^2=324$ a $19^2=361$). Zo školy totiž vieme
poznatok, že každé zložené prirodzené číslo $n$ je deliteľné
niektorým prvočíslom $p$ s~vlastnosťou $p\leq\sqrt{n}$.

Zdôraznime, že v~úplnom riešení súťažnej úlohy nie je nutné
vypisovať osem zlomkov požadovanej vlastnosti. Namiesto toho je možné
využiť fakt, že každé $n$ tvaru $n=pqr$, kde $p$, $q$, $r$ sú
navzájom rôzne prvočísla, má práve osem deliteľov. Ich konkrétny
výpis je možné nahradiť kombinatorickou úvahou: Pre číslo $n$ uvedeného
tvaru zostavíme jeho ľubovoľný deliteľ $d$ tak, že sa pre každé z troch
prvočísel $p$,~$q$,~$r$ rozhodneme, či ho do rozkladu $d$ na
prvočinitele zaradiť alebo nie. (Ak napríklad nezaradíme žiadne z~nich,
dostaneme $d=1$.) Celkový počet možností, ako deliteľ $d$
zostaviť, je preto rovný $2\cdot2\cdot2=8$ -- ako sme totiž uviedli,
pre každé z troch prvočísel $p$, $q$, $r$ máme 2 možnosti:
zaradiť alebo nie.

\návody
Rozdiel dvoch prirodzených čísel je rovný $4$, pričom jedno
z~čísel je násobkom druhého. O~aké čísla sa jedná?
[$(5,1), (6,2), (8,4)$. Nech $a>b$ sú hľadané čísla. Potom
menšie číslo~$b$ je deliteľom väčšieho čísla $a$,
a teda aj deliteľom čísla $a-b$, ktoré sa podľa zadania rovná~4.
Preto $b\in\{1,2,4\}$. Tento poznatok vyplýva aj
z~vyjadrenia podielu oboch čísel v tvare
$$\frac{a}{b}=\frac{b+4}{b}=1+\frac{4}{b}.]
$$

Číslo $73$ rozložte na súčet dvoch prirodzených čísel tak,
aby ich podiel bol tiež prirodzené číslo.
[Jediné riešenie $72+1$. Postupujte podobne ako v~riešenie N1: využite
napríklad vyjadrenie $$\frac{a}{b}=\frac{73-b}{b}=\frac{73}{b}-1
$$
a poznatok, že 73 je prvočíslo.]

Rozhodnite, pre ktoré prirodzené čísla $n$ nadobúda zlomok
$$
\frac{4n+1}{2n-3}
$$
celočíselné hodnoty.
[$n\in\{1,2,5\}$. Z~úpravy
$$
\frac{4n+1}{2n-3}=\frac{2(2n-3)+7}{2n-3}=2+\frac{7}{2n-3}
$$
vidíme, že hľadáme práve tie $n$, pre ktoré je
celé (napríklad aj záporné) číslo $2n-3$ deliteľom čísla $7$,
t.\,j. jedným z~čísel $\pm 1$, $\pm 7$. Niektorej z~rovníc $2n-3=\pm1$, $2n-3=\pm7$
vyhovujú práve hodnoty $n\in \{-2,1,2,5\}$, z ktorých záporné $n=\m2$
musíme kvôli zadaniu vylúčiť.]

\D
Rozhodnite, pre ktoré prirodzené čísla $n$ nadobúda zlomok
$$
\frac{n+72}{2n}
$$
celočíselnú hodnotu.
[$n\in\{8,24,72\}$. Daný zlomok má celočíselnú hodnotu $k$
práve vtedy, keď platí $n+72=2nk$, čiže $72=n(2k-1)$. Odtiaľ vidíme, že
celé číslo $2k-1$ je kladné a že je to nepárny deliteľ čísla 72. Preto
$2k-1\in\{1,3,9\}$ a rovnosť $72=n(2k-1)$ je potom splnená pre tri
vyššie uvedené $n$.]

Každý zlomok zo zadania súťažnej úlohy, ktorý {\it nenadobúda\/}
celočíselnú hodnotu, skrátime na zlomok v~základnom tvare. Určte
všetky tie pôvodné zlomky, ktoré po skrátení budú mať
menovateľ rovný 2.
[$\frac{2018}{4}$, $\frac{2010}{12}$ a $\frac{674}{1348}$.
Budú to práve zlomky s~menovateľom $2k$
pre vhodné $k$ od 1 do 1011, ktorých čitateľ $2022-2k$ je
deliteľný číslom $k$, nie však číslom $2k$. Ekvivalentne vyjadrené:
Číslo~$2022$ je deliteľné číslom $k$, nie však číslom $2k$.
Hľadáme teda tie $k$ od 1 do 1011, ktoré delia číslo
$2022$, nedelia však číslo $1011$. Sú to zrejme iba {\it
párne\/} čísla $2$, $6$ a $674$, ktorým zodpovedajú
tri na úvod vypísané zlomky.]
\endnávod
}

{%%%%%   C-I-2
Najskôr vysvetlíme, prečo viac ako polovica známok Šebestovej sú
jednotky. Vyjdeme zo zrejmého poznatku, že ak niektorú známku
nahradíme lepšou známkou, priemer známok sa tiež zlepší. Keby preto
jednotiek bola najviac polovica, tak by sme po prípadnom nahradení
známok 3, 4, 5 (ak nejaké existujú) lepšími dvojkami dostali
priemer aspoň $1{,}5$, teda horší ako $1{,}12$. Jednotiek teda
skutočne musí byť viac ako polovica zo všetkých známok.

Označme teraz $s$ súčet všetkých známok Šebestovej a $p$ ich
počet. Z~rovnosti $s/p=1{,}12=28/25$ vzhľadom
na nesúdeliteľnosť čísel 25 a 28 vyplýva, že $p$ je násobok čísla~25.
Keby teda platilo $p>25$, mali by sme $p\geqq50$, a tak by jednotiek
bolo (podľa prvého odseku) viac ako 25. Zostáva preto posúdiť
prípad, keď $p=25$ a $s=28$. Keby pritom jednotiek bolo najviac
21, bol by počet horších známok aspoň $25-21=4$, a tak by
platilo $s\geqq21+4\cdot2=29$, a to je spor. Preto aj v~prípade
$p=25$ muselo jednotiek byť aspoň 22.

\poznamka
Aj keď to zadanie úlohy nevyžaduje, ukážme, že
Šebestová mohla mať z~päťminútoviek {\it práve\/} 22 jednotiek, a
to v~jedinom prípade, keď okrem nich mala už len 3~dvojky.
Zachovajme označenie nášho riešenia. Podľa neho v~prípade 22
jednotiek muselo byť $p=25$ a $s=28$, takže známky horšie ako 1 boli
práve tri a ich súčet bol $28-22=6$, \tj. išlo o~tri dvojky.

\ineriesenie
Označme počet jednotiek, dvojok, \dots, pätiek postupne
$a$, $b$, $c$, $d$, $e$. Potom platí
$$
\frac{1a+2b+3c+4d+5e}{a+b+c+d+e}=1{,}12=\frac{112}{100}=\frac{28}{25}.
\tag1
$$
Posledný zlomok v~\thetag1 je v~základnom tvare, takže čísla
z~prvého zlomku majú nutne vyjadrenie
$$\eqalign{
a+2b+3c+4d+5e&=28k,\cr
a+b+c+d+e&=25k}
\tag2
$$
pre vhodné prirodzené číslo $k$.
Z týchto dvoch rovností vylúčime
hodnotu $b$, a to tak, že od dvojnásobku druhej rovnice
odpočítame prvú rovnicu. Dostaneme
$$
2(a+b+c+d+e)-(a+2b+3c+4d+5e)=2\cdot25k-28k,
$$
odkiaľ po úprave vychádza
$$
a=22k+c+2d+3e.
\tag3
$$
Sme prakticky hotoví, pretože vďaka zrejmej nerovnosti $c+2d+3e\geq0$
(čísla $c$, $d$, $e$ sú totiž nezáporné) z~odvodenej rovnosti \thetag3
už vyplýva $a\geqq22k\geqq22$. Opäť tiež vidíme, že rovnosť
$a=22$ nastane práve vtedy, keď bude platiť $k=1$ a $c+2d+3e=0$, \tj.
$c=d=e=0$; po dosadení do ktorejkoľvek z~rovníc v~\thetag2 potom vyjde $b=3$.

\návody
Pažout dostal z desaťminútoviek osemkrát známku $5$, šesťkrát
známku $4$, štyrikrát známku~$3$ a dvakrát známku $2$.
Koľko by k tomu ešte musel dostať jednotiek,
aby sa priemer jeho známok zlepšil presne o~1 stupeň?
[10. Potrebný počet jednotiek označme~$n$. Keďže pôvodný priemer má
hodnotu
$$
\frac{8\cdot 5+6\cdot 4+4\cdot 3+2\cdot
2}{8+6+4+2}=\frac{80}{20}=4,
$$
po pridaní $n$ jednotiek má byť rovný 3, \tj. má platiť
$$
\frac{8\cdot 5+6\cdot 4+4\cdot 3+2\cdot 2+n\cdot1}{8+6+4+2+n}
=\frac{80+n}{20+n}=3.
$$
Riešením rovnice $80+n=3(20+n)$ dostaneme $n=10$.]

Horáček dostal z desaťminútoviek najskôr trikrát
známku $2$, ďalšie jeho známky už boli iba päťky.
Koľko ich dostal, ak bol priemer jeho známok horší ako $4{,}2$?
[Aspoň 9. Počet pätiek označme $n$. Má platiť
$$\frac{3\cdot 2+n\cdot
5}{3+n}=\frac{5n+6}{n+3}>4{,}2=\frac{21}{5}.$$
Úpravou nerovnice $(5n+6)\cdot5>21\cdot(n+3)$ dostaneme
$4n>33$, takže $n\geqq9$.]

Čermáková mala z desaťminútoviek, ktorých bolo menej ako 15,
priemer známok presne $1{,}75$. O~koľko známok sa mohlo jednať?
[4, 8 alebo 12. Označme $p$ počet známok a $s$ ich súčet. Platí
$$
\frac{s}{p}=1{,}75=\frac{175}{100}=\frac{7}{4}.
$$
Keďže posledný zlomok je v~základnom tvare, musí byť $s=7k$ a $p=4k$
pre vhodné prirodzené číslo $k$. Podľa zadania pripadajú do úvahy iba
hodnoty $k$ rovné 1, 2 a 3.]

Mach tvrdí, že keby z ďalšej desaťminútovky dostal
známku $1$, vylepšil by si tak priemer z~presne $1{,}15$ na
presne~$1{,}12$. Je to možné?
[Nie je. Pri priemere $1{,}15=\frac{23}{20}$ by bol
počet známok $p$ násobkom čísla 20, pri novom priemere
$1{,}12=\frac{28}{25}$ by bol počet známok $p+1$ násobkom čísla
25. Obe čísla $p$ a $p+1$ však nemôžu byť súčasne násobky
piatich. Iný postup: Pri počte známok $p$ s~priemerom $1{,}15$ by
bol ich súčet $1{,}15p$, po obdržaní novej jednotky by potom
malo platiť
$$
\frac{1{,}15p+1}{p+1}=1{,}12.
$$
Táto rovnica má síce riešenie $p=4$, zodpovedá mu však
pôvodný súčet známok $1{,}15p=4{,}6$, čo nie je celé číslo.]

\D
Kropáček mal z niekoľkých desaťminútoviek priemer známok približne
$3{,}14$ (zaokrúhlené na stotiny). Mohlo sa jednať o~osem známok?
[Nie. Označme $p$ počet známok a~$s$ ich súčet. Hodnota podielu
$s/p$ leží v~intervale $\langle3{,}135;3{,}145)$, takže celé číslo~$s$ leží
v~intervale $\langle3{,}135p;3{,}145p)$. Pre $p=8$ však
ide o interval $\langle25{,}08;25{,}16)$.]
\endnávod
}

{%%%%%   C-I-3
Označme ešte $Q$ priesečník priamky $AP$ s~úsečkou $BC$.
Obr.\,\obrnum{} nám napovedá, že $P$ je stred úsečky $AQ$ a
$Q$ je stred úsečky $BC$. Odložme na chvíľu dôkaz týchto dvoch faktov
a ukážme najskôr, ako z~nich vyplýva tvrdenie úlohy.
\inspdf{c72i_p31.pdf}%
Predpokladajme teda, že platí rovnosť
$|MN|=|BQ|=|CQ|$ a $|AP|=|PQ|$.
Podľa zadania úlohy platí však aj rovnosť $|MN|=|AP|$. Dokopy tak
dostávame, že úsečky $BQ$, $CQ$ a $PQ$ sú zhodné. Bod $P$ preto
leží na tej kružnici so stredom v~bode~$Q$, ktorej priemerom
je úsečka $BC$. Podľa Tálesovej vety to už znamená, že
uhol~$BPC$ je skutočne pravý.

Ako sme sľúbili, ukážeme teraz, že skutočne $P$ je stred úsečky $AQ$ a
$Q$ je stred úsečky~$BC$. Úsečka $MN$ je stredná priečka trojuholníka $ABC$,
teda $MN\parallel BC$ a~$|MN|=\frac12|BC|$ (pozri návodnú úlohu N1).
Vďaka $MN\parallel BC$
je trojuholník $ABQ$ podobný trojuholníku $AMP$ podľa vety \emph{uu}. Pre dĺžky
strán týchto podobných trojuholníkov tak z~rovnosti $|AB|=2\cdot|AM|$
vyplývajú ďalšie dve rovnosti $|AQ|=2\cdot|AP|$ a~$|BQ|={2\cdot|MP|}$.
Podľa prvej z nich je $P$ stred $AQ$; z~druhej rovnosti
$|BQ|={2\cdot|MP|}$, v ktorej $2\cdot|MP|=|MN|=\frac12|BC|$,
vyplýva $|BQ|=\frac12|BC|$, čo znamená práve to, že $Q$ je stred
$BC$, ako sme chceli ukázať. (O~využití podobných trojuholníkov
vo všeobecnejšej situácii pojednáva návodná úloha N2.)

\poznamka
Popíšme kratší spôsob ako ukázať, že $P$ je stred úsečky $AQ$ a
$Q$ je stred úsečky~$BC$. Ak označíme $Q'$ stred strany $BC$, tak
z~vlastností stredných priečok $MQ'$ a~$NQ'$ trojuholníka $ABC$ vyplýva, že
štvoruholník $AMQ'N$ je rovnobežník. Jeho uhlopriečky $MN$ a $AQ'$ sa
teda navzájom rozpoľujú, takže sa pretínajú v~bode $P$ (ktorý je
totiž zadaný ako stred úsečky $MN$). Odtiaľ vyplýva, že úsečka $AQ'$
prechádza bodom~$P$, a tak je nutne $Q'=Q$, \tj. $Q$ je stred
strany $BC$ a $P$ ako stred uhlopriečky $AQ'$ je stredom
úsečky $AQ$. (Dodajme, že potrebné vlastnosti bodov $P$ a $Q$ je možné
okamžite získať použitím geometrického zobrazenia, ktorému hovoríme
{\it rovnoľahlosť\/} so stredom $A$ a koeficientom $2$.)

\návody
Použitím podobných trojuholníkov odvoďte známu vlastnosť stredných priečok
všeobecného trojuholníka $ABC$: {\sl Ak je $M$ stred strany $AB$ a~$N$ stred strany $AC$, tak $MN\parallel BC$ a $|MN|=\frac12|BC|$.}
[Keďže $|AM|:|AB|=|AN|:|AC|=1:2$, sú trojuholníky $ABC$ a $AMN$ podobné
podľa vety $sus$. Zo zhodnosti ich uhlov $ABC$ a~$AMN$ potom vyplýva
$MN\parallel BC$ a vďaka podobnostnému pomeru $1:2$ platí tiež
$|MN|=\frac12|BC|$.]

Sú dané rovnobežky $p$, $q$ a bod $S$, ktorý na nich
neleží. Na priamke $p$ sú dané tri rôzne body $A$, $B$, $C$.
Priesečníky priamky $q$ s~priamkami
$SA$, $SB$, $SC$ sú označené postupne $D$, $E$, $F$.
Dokážte rovnosti
$$
\frac{|AB|}{|DE|}=\frac{|AC|}{|DF|}=\frac{|BC|}{|EF|}=
\frac{|SA|}{|SD|}=\frac{|SB|}{|SE|}=\frac{|SC|}{|SF|}.
$$
[Vďaka zhodnosti vrcholových a súhlasných či striedavých uhlov
sú podľa vety~\emph{uu} navzájom podobné trojuholníky $SAB$ a~$SDE$,
rovnako ako trojuholníky $SAC$ a~$SDF$, ako aj~trojuholníky $SBC$ a~$SEF$. Vďaka stranám týchto trojuholníkov so spoločným krajným bodom $S$
majú všetky tri podobnosti rovnaký koeficient rovný posledným
trom zlomkom v~dokazovanej sérii rovností; prvé tri zlomky
vyjadrujú tento koeficient pre strany protiľahlé k~vrcholu $S$
dotyčných trojuholníkov.]

Pripomeňte si Tálesovu vetu a použite ju na dôkaz tvrdenia: {\sl Os
pravého uhla v~rôznostrannom pravouhlom trojuholníku rozpoľuje uhol medzi jeho
výškou k prepone a~ťažnicou k prepone.}
[Nech v~trojuholníku $ABC$ platí $\gamma=90^{\circ}$ a $\beta<\alpha$, \tj.
$\beta<45^{\circ}$. Označme $C_1$ stred prepony $AB$
a $C_0$ pätu výšky z~vrcholu $C$. Podľa Tálesovej vety v~trojuholníku $C_1CB$
platí $|C_1C|=|C_1B|$, teda $|\uhol BCC_1|=|\uhol C_1CB|=\beta$.
V~pravouhlom trojuholníku $ACC_0$ zase máme $|\uhol ACC_0|=90^{\circ}-|\uhol
C_0CA|={90^{\circ}-\alpha}=\beta$. Uhly $BCC_1$ a $ACC_0$, ktoré ležia
v~pravom uhle $ACB$, tak majú rovnakú veľkosť $\beta<45^{\circ}$, a preto
sa neprekrývajú, a tak os celého uhla $ACB$ je súčasne
aj osou súmernosti jeho \uv{zvyšnej} časti medzi uhlami $BCC_1$ a $ACC_0$,
\tj. osou uhla $C_1CC_0$.]

\D
Vrchol~$C$ štvorcov $ABCD$ a~$CJKL$ je vnútorným bodom
úsečky~$AK$ aj úsečky~$DJ$. Body $E$, $F$, $G$ a~$H$ sú postupne stredy
úsečiek $BC$, $BK$, $DK$ a~$DC$. Vyjadrite obsah štvoruholníka $EFGH$ pomocou obsahov~$S$ a~$T$ štvorcov $ABCD$ a~$CJKL$.
[\pdfklink{C-55-S-2}{https://skmo.sk/dokument.php?id=240}]

V rovine je daný pravouhlý trojuholník~$ABC$ taký, že kružnica $k(A;|AC|)$ pretína
preponu~$AB$ v~jej strede~$S$. Dokážte, že kružnica opísaná trojuholníku~$BCS$
je zhodná s~kružnicou~$k$.
[\pdfklink{C-51-S-2}{https://skmo.sk/dokument.php?id=280}]

V~lichobežníku $ABCD$ so základňami $AB$, $CD$ označíme $P$
priesečník vnútorných uhlov pri vrcholoch $A$, $D$ a $Q$ priesečník vnútorných
uhlov pri vrcholoch $B$, $C$. Dokážte, že body~$P$ a~$Q$ ležia na jednej
rovnobežke so základňami lichobežníka.
[Stred $M$ ramena~$AD$ leží na osi $o$ pásu medzi rovnobežkami $AB$ a
$CD$. Keďže súčet uhlov $BAD$ a $ADC$ je vďaka $AB\parallel CD$
rovný $180^{\circ}$, súčet polovičných uhlov $PAD$ a $ADP$ je
rovný $90^{\circ}$, teda $PAD$ je pravouhlý trojuholník s~preponou
$AD$ so stredom $M$. Podľa Tálesovej vety je $PAM$ rovnoramenný
trojuholník so základňou $PA$, teda uhol $MPA$ je zhodný s~uhlom $PAM$,
a teda aj s~uhlom~$PAB$. Zo zhodnosti (striedavých) uhlov $MPA$ a $PAB$
vyplýva $MP\parallel AB$, teda $M$ leží na osi $o$. Analogickou úvahou
o~strede $N$ ramena $BC$ zistíme, že na osi $o$ leží aj bod $Q$.]
\endnávod
}

{%%%%%   C-I-4
Označme $P$ celkový počet žetónov na kôpkach na stole a
vyjadrime, ako sa aktuálna hodnota $P$ po každom ťahu zmení.
Podľa zadania Mach v~každom ťahu odstráni z dvoch
kôpok rovnaký počet žetónov, ktorý označíme $p$ (hodnota $p$
pritom závisí na danom ťahu). Celkový počet $P$ žetónov sa tak
týmto ťahom zníži o~hodnotu $2p$, teda na počet $P-2p$.
Keďže číslo $2p$ je párne pre
každé celé $p$, v~prípade párneho~$P$ bude aj $P-2p$ párne, v~prípade
nepárneho $P$ bude aj $P-2p$ nepárne. Počet žetónov na stole
tak po ľubovoľnom počte ťahov nezmení svoju {\it paritu\/}
(týmto termínom označujeme \uv{párnosť} alebo \uv{nepárnosť} daného
čísla).\fnote{Hovoríme, že parita celkového počtu žetónov je {\it
invariantom\/} vykonávaných úprav.}

Určme teraz paritu $P$ v oboch zadaných prípadoch.
\item{a)}
Pre $k=10$ máme $P=1+2+\dots+10=55$, čo je nepárne
číslo, takže naša predchádzajúca úvaha nevylučuje možnosť, že po
určitom počte ťahov zostane na stole jediný žetón.
To môže Mach ľahko dosiahnuť napríklad takto: Najskôr
kôpky spáruje do 5 dvojíc s~počtami žetónov $(1,2)$,
$(3,4),\dots,$ $(9,10)$ a vykoná 5 ťahov, keď
z~kôpok $(j,j+1)$ odstráni po $j$ žetónoch pre $j=1,3,\dots,9$.
Zostane mu potom iba päť \uv{kôpok} po 1 žetóne, potom už
štyri z nich zrejme odstráni na dva ťahy.

\item{b)}
Pre $k=11$ máme $P=1+2+\dots+10=66$, čo je číslo párne.
Podľa našej úvahy bude počet žetónov po ľubovoľnom počte ťahov
párny, takže nikdy na stole nezostane jediný žetón.

\zaver
V~prípade a) sa to Machovi môže podariť, v~prípade
b) nie.

\návody
Šebestová roztrhla list papiera na tri kúsky, potom niektoré z týchto
kúskov opäť roztrhla každý na tri kúsky, atď.
Rozhodnite, ktoré počty kúskov 4, 5, $6,\dots,$ 20
mohla týmto postupom získať. [Nepárne počty 5, $7,\dots,$ 19.
Celkový počet kúskov sa každým roztrhnutím jedného z nich zväčší o 2,
takže zostáva nepárny ako na začiatku.]

Na tabuli je napísané a) 5 písmen R a 5 písmen S,
b) 25 písmen R a 30 písmen S. V~každom kroku
zotrieme dve napísané písmená a nahradíme ich písmenom~R, resp.~S,
ak boli zotreté písmená rôzne, resp. rovnaké. Ktoré písmeno
zostane na tabuli posledné? [Písmeno R v~oboch prípadoch a)
a b). Počet písmen R sa po každom kroku buď
nezmení (ak zotrieme dve S~či po jednom R a~S), alebo sa
zmenší o~2 (ak zotrieme dve R), takže zostane po každom
počte krokov nepárny, a~preto nikdy neklesne na nulu.
Keďže sa po jednom kroku celkový počet písmen na tabuli zníži o~1,
po konečnom počte krokov zostane na tabuli ako posledné písmeno R.]

Na tabuli sú napísané 3 jednotky, 3 dvojky a 3 trojky.
V~každom kroku je povolené zotrieť ľubovoľné dve rôzne cifry a
pripísať namiesto nich zostávajúcu tretiu cifru. Po sérii takýchto úprav
sa nám podarilo dôjsť k situácii, keď na tabuli zostala jediná cifra,
a to dvojka. Mohlo sa stať, že pri inom priebehu úprav by sme došli
k~inej jedinej cifre, \tj. k~jednotke alebo trojke?
Zmení sa odpoveď pri iných východiskových počtoch cifier?
[Nemohlo sa to stať, ani pri iných východiskových počtoch.
Skúmajme aktuálny súčet $S$ všetkých cifier na tabuli.
Pri zámene $(1,2)\to3$ sa $S$ nezmení, pri zámene $(1,3)\to2$
sa $S$ zmenší o~2, napokon pri zámene $(2,3)\to1$ sa $S$ zmenší o~4.
Vidíme, že $S$ nemení svoju paritu. Nemôžeme teda z toho istého východiskového stavu
niekedy dôjsť k~jedinej párnej cifre, inokedy k~jedinej nepárnej cifre.]

\D
Na tabuli sú napísané prirodzené čísla od 1 do 100.
V~každom kroku zotrieme trojicu po sebe idúcich čísel (ak existuje
taká trojica). Môžu na tabuli zostať nakoniec čísla,
ktorých celkový súčet bude $111$?
[Nie. Súčet troch po sebe idúcich čísel $(n-1)+n+(n+1)=3n$ je
deliteľný tromi, takže súčet čísel na tabuli po každom kroku
klesne o~násobok troch; jeho zvyšok pri delení tromi sa teda
nezmení. Na začiatku máme súčet $1+2+\dots+100=5050$ so
zvyškom~1 po delení tromi, číslo 111 však má zvyšok 0.]

Vráťme sa k situácii z úlohy N3 so všeobecnými východiskovými počtami cifier.
Rozhodnite, či je možné, aby sme dvoma odlišnými postupmi úprav
došli raz k~jedinej cifre~1 a druhýkrát k~jedinej cifre 3.
[Možné to nie je. Preznačme cifry 1, 2, 3 za písmená A, B, C
v~akomkoľvek poradí~-- povolené úpravy to
nijako neovplyvní. Preto negatívna odpoveď k~D2 vyplýva z~výsledku~N3.
Inak je možné spoločné riešenie N3 a D2 podať takto: označiť počet jednotiek,
dvojok a trojok na tabuli po rade $j$, $d$, $t$ a ukázať, že pri
úpravách nemení paritu žiadny z troch súčtov $j+d$, $j+t$ a $d+t$.
Dodajme, že v~riešení N3 sme využili paritu súčtu $j+2d+3t$,
ktorá je rovnaká ako parita súčtu $j+t$.]
\endnávod
}

{%%%%%   C-I-5
Podľa návodnej úlohy N1 uvažujme súmernosť $ABCDE$ podľa tej jeho osi,
ktorá prechádza vrcholom $D$ (pozri \obr{}).
Návodná úloha N2 nám pritom hovorí,
že je to os súmernosti rovnoramenného lichobežníka $ABCE$
so základňami $AB$ a $CE$. Priesečník jeho uhlopriečok $AC$ a $BE$
preto leží na osi základne $AB$, je teda bodom $F$ zo zadania úlohy.
\inspinsp{c72i.51}{c72i.52}%

Vieme teda, že bod $F$ leží na uhlopriečke $BE$. Tá je však
rovnobežná so stranou~$CD$, a~to vďaka súmernosti $ABCDE$
podľa jeho osi, ktorá tentoraz prechádza vrcholom~$A$ (pozri \obr{}). Body $F$ a $E$ tak majú od priamky
$CD$ rovnakú vzdialenosť, a~preto trojuholník $CDF$ má rovnaký obsah ako
trojuholník $CDE$, ktorý je ale zhodný s~trojuholníkom $ABC$. Tým je rovnosť
obsahov trojuholníkov $ABC$ a $CDF$ dokázaná.

\poznamka
V druhom odseku riešenia sme mohli postupovať inak: Podobne
ako $BE\parallel CD$ platí tiež $AC\parallel ED$ (zo
súmernosti $ABCDE$ podľa jeho osi idúcej vrcholom $B$), takže $CDEF$
je rovnobežník (dokonca kosoštvorec, pretože $|CD|=|DE|$), teda obsahy oboch trojuholníkov $CDF$ a $CDE$ sú rovnaké~-- rovnajú sa totiž polovici obsahu spomínaného rovnobežníka.


\návody
{\everypar{}
\smallskip
Pripomeňme, že pravidelný päťuholník je konvexný päťuholník,
ktorý má zhodné všetky strany aj všetky vnútorné uhly.
\smallskip
}

V~pravidelnom päťuholníku $ABCDE$ narysujeme osi všetkých jeho
strán a osi všetkých jeho uhlopriečok. Koľko rôznych priamok to bude?
Vysvetlite, prečo každá z nich je osou súmernosti celého
päťuholníka a prechádza jedným jeho vrcholom.
[Päť priamok. Stačí ukázať, že napríklad strana $AB$ a uhlopriečka $CE$
majú spoločnú os, ktorá prechádza zvyšným piatym vrcholom $D$.
Vyjdeme z~toho, že $BCD$, $CDE$ a~$DEA$ sú zhodné
rovnoramenné trojuholníky s~hlavnými vrcholmi postupne $C$,~$D$,~$E$.
Odvodíme, že os uhla~$CDE$ je spoločnou osou úsečiek~$CE$ a~$AB$:
Pre prvú z nich to vyplýva z~rovnoramenného trojuholníka $CDE$,
pre druhú z~trojuholníka $BDA$, ktorý je rovnako rovnoramenný, lebo
vďaka zhodným trojuholníkom $BCD$ a $DEA$ platí $|BD|=|DA|$ a navyše
$|\uhol CDB|=|\uhol EDA|$.]

Dokážte, že každé štyri vrcholy pravidelného
päťuholníka tvoria vrcholy rovnoramenného lichobežníka.
[Vyplýva to z~riešenia N1: Ukázali sme tam, že os súmernosti
celého päťuholníka prechádzajúca vrcholom $D$ je spoločnou osou
úsečiek $AB$ a $CE$, takže to sú základne rovnoramenného
lichobežníka $ABCE$~-- druhé dve protiľahlé strany $BC$ a~$EA$ sú totiž
zhodné, nie však rovnobežné (vďaka tupým uhlom $ABC$ a~$BAE$).]

Rovnobežné úsečky $KL$ a $MN$ neležia na jednej priamke.
Dokážte, že trojuholníky~$KLM$ a~$KLN$ majú rovnaký obsah.
[Z~podmienky $KL\parallel MN$ vyplýva, že výšky na spoločnú
stranu~$KL$ oboch trojuholníkov $KLM$ a $KLN$ sú zhodné.
Pre ne tak do vzorca $S=\frac12 zv$ pre obsah všeobecného trojuholníka
dosadíme rovnaké hodnoty $z$ a~$v$.]

\D
V~pravidelnom päťuholníku $ABCDE$ označme $G$ priesečník
uhlopriečky $AC$ a $BD$. Ukážte, že štvoruholník $AGDE$ je
kosoštvorec.
[Z~lichobežníkov $ACDE$ a $BDEA$ (pozri výsledok N2) vyplýva
$AG\parallel DE$ a $GD\parallel EA$, takže $AGDE$ je rovnobežník;
vďaka $|DE|=|EA|$ sa jedná skutočne o~kosoštvorec.]

Dokážte, že dve uhlopriečky pravidelného päťuholníka, ktoré
vychádzajú z~jedného jeho vrcholu, rozdeľujú príslušný vnútorný uhol
na tretiny.
[Stačí ukázať, že v~pravidelnom päťuholníku $ABCDE$ sú zhodné
tri uhly s~vrcholom $A$, teda $BAC$, $CAD$ a $DAE$.
Vnútorné uhly pravidelného päťuholníka majú veľkosť
$3\cdot180^{\circ}:5=108^{\circ}$. Preto uhly pri základniach
rovnoramenných trojuholníkov $ABC$ a $DEA$ majú veľkosť
${\bigl(180^{\circ}-108^{\circ}\bigr):2}=36^\circ$. Vidíme, že oba uhly
$BAC$ a $DAE$ majú v porovnaní s~uhlom $BAE$ tretinovú veľkosť
(lebo $36:108=1:3$), takže tretinovú veľkosť má i~tretí uhol
$CAD$. (Dodajme, že z~vlastností tzv. {\it stredových\/} a
{\it obvodových uhlov\/} v~kružnici vyplýva nasledujúce tvrdenie
pre ľubovoľné $n\geqq4$:
Všetky uhlopriečky pravidelného $n$-uholníka
vychádzajúce z jedného jeho vrcholu delia jemu príslušný vnútorný uhol
na $n-2$ zhodných častí.)]

Označme $a$ dĺžku strany a $u$ dĺžku uhlopriečky daného
pravidelného päťuholníka. Dokážte rovnosť $a^2+au=u^2$.
[V~pravidelnom päťuholníku $ABCD$ označme~$G$ priesečník
uhlopriečky $AC$ a $BD$. Podľa úlohy N2 je $DABC$ rovnoramenný
lichobežník ($DA\parallel BC$), takže trojuholníky $DAG$ a $BCG$ sú
podľa vety $uu$ podobné. Platí preto $|DA|:|BC|={|DG|:|GB|}$, čiže
$u:a=|DG|:|GB|$. Podľa úlohy D1 je $AGDE$ kosoštvorec o~strane~$a$, takže platí $|DG|=a$ a $|GB|=|BD|-|DG|=u-a$. Dosadením do
$u:a=|DG|:|GB|$ dostaneme $u:a=a:(u-a)$, odkiaľ už ľahko vyplýva
rovnosť $a^2+au=u^2$. (Úmera $u:a=a:(u-a)$ znamená, že bod $G$ delí
každú z~uhlopriečok $AC$ a $BD$ v~tzv. {\it zlatom reze}.)]

\endnávod
}

{%%%%%   C-I-6
Ukážeme, že hľadané najväčšie $n$ je rovné 21.

Najskôr overíme, že požadovanú vlastnosť nemá žiadne
prirodzené číslo $n\geqq22$. Pre každé takéto $n$ sú v~dotyčnej
množine $\{1,2,\dots, n\}$ zastúpené párne čísla 4, 6, $8, \dots,$ 22,
ktoré nie sú prvočísla (kvôli tomu sme vynechali najmenšie párne
číslo~2) a~ktorých je požadovaný počet $(22-4):2+1=10$. Keďže ani
súčet žiadnych dvoch z~10 vypísaných čísel nebude prvočíslom
(opäť pôjde o~párne číslo väčšie ako 2), tvrdenie úlohy pre týchto
10~čísel neplatí. Preto skutočne nevyhovuje žiadne $n\geqq22$.

Teraz dokážeme, že $n=21$ požadovanú vlastnosť má. Ak vyberieme
z~množiny $\{1,2,\dots, 21\}$ ľubovoľných (ďalej pevných)
10 rôznych čísel, medzi
ktorými nie je žiadne prvočíslo (inak nie je čo dokazovať), vyberieme
vlastne 10 rôznych čísel z~množiny $M=\{1,4,6,8,9,10,12,14,15,16,18,20,21\}$.
Rozdelíme ju pre ďalšie účely na skupinu nepárnych čísel a skupinu párnych čísel:
$$
L=\{1,9,15,21\}\qquad\hbox{a}\qquad
S=\{4,6,8,10,12,14,16,18,20\}.
$$
Vidíme, že $L$ má 4 prvky a $S$ má 9 prvkov. Z~toho vyplýva, že medzi
10 vybranými číslami je aspoň jedno číslo z~$L$ ($10-9=1$) a aspoň 6 čísel
z~$S$ ($10-4=6$), teda z~$S$ nie sú vybrané najviac 3 čísla ($9-6=3$).
Inak povedané, z~ľubovoľných štyroch čísel z~$S$ je aspoň jedno vybrané.

Rozlíšime teraz, ktoré z~nepárnych čísel 1, 9, 15, 21
je vybrané (pôjde o~diskusiu štyroch prípadov, ktoré sa navzájom
nemusia nutne vylučovať).
\item{$\triangleright$}
Vybrané je číslo 1. Keďže $1+4$, $1+6$, $1+10$, $1+16$
sú prvočísla a {\it aspoň jedno zo štyroch\/} párnych čísel 4, 6, 10, 16
musí byť vybrané (podľa záveru predchádzajúceho odseku),
tvrdenie úlohy pre vybraných 10 čísel platí.
\item{$\triangleright$}
Vybrané je číslo 9. Podobne ako vyššie využijeme štyri súčty
$9+4$, $9+8$, $9+10$, $9+14$.
\item{$\triangleright$}
Vybrané je číslo 15. Tentoraz uplatníme štyri súčty $15+4$, $15+8$,
$15+14$, $15+16$.
\item{$\triangleright$}
Vybrané je číslo 21. V~tomto prípade vhodné štyri súčty sú
$21+8$, $21+10$, $21+16$, $21+20$.

\poznamka
Existuje veľa iných spôsobov ako ukázať, že pri výbere
ľubovoľných 10 rôznych čísel z~množiny
$$
M=\{1,4,6,8,9,10,12,14,15,16,18,20,21\}
$$
bude súčet niektorých dvoch vybraných čísel rovný prvočíslu.
Obvykle je pri nich nutné rozobrať viac prípadov ako v našom riešení,
avšak jeden rafinovaný postup žiadny rozbor v~podstate nevyžaduje,
a preto ho teraz popíšeme.

Rozdeľme celú množinu $M$ na 9 navzájom disjunktných podmnožín
$$
\{1,4\},~\{6\},~\{8,9\},~\{10\},~\{12\},~\{14,15\},~
\{16\},~\{18\},~\{20,21\}.
$$
Keďže podmnožín je 9, pri výbere ľubovoľných 10 rôznych čísel
z~množiny $M$ musia byť zrejme vždy vybrané obe čísla z~niektorej
dvojprvkovej podmnožiny. Pre každú z nich je však súčtom jej
prvkov prvočíslo.

\návody
Ukážte, že z~množiny $\{1,2,3,\dots,10\}$ je možné vybrať $4$ rôzne
čísla tak, aby medzi nimi nebolo žiadne prvočíslo ani dve čísla,
ktorých súčet je prvočíslom. Nájdite tiež všetky také výbery.
[Vyhovujúci výber 4, 6, 8, 10 je jediný. Musí sa jednať o~4 čísla
z~množiny $\{1, 4, 6, 8, 9, 10\}$, ktorá má 6 prvkov.
Číslo $1$ nemôže byť
vo výbere s~tromi číslami 4, 6 a 10, preto je \uv{nepoužiteľné}.
To isté platí aj pre číslo 9 kvôli podobnej \uv{kolízii}
s tromi číslami 4, 8 a 10. Do úvahy tak prichádzajú iba štyri párne
čísla 4, 6, 8 a 10. Ich výber vyhovuje, pretože súčet každých
dvoch z~nich je tiež párne číslo rôzne od~2.]


Ukážte, že pre každé prirodzené číslo $n\geqq2$ je možné
z~množiny $\{1,2,3,\dots,2n\}$ vybrať $n-1$~čísel tak, aby medzi nimi
nebolo žiadne prvočíslo ani dve čísla, ktorých súčet je prvočíslom.
[Výber bude mať požadovanú vlastnosť, ak bude napríklad zostavený
z~párnych zložených čísel. Spĺňa to výber $n-1$ čísel
4, 6, $8, \dots,$ $2n$.]

\D
Ukážte, že počet všetkých šesťciferných prvočísel neprevyšuje $300\,000$.
[Šesťciferné sú čísla od 100\,000 do 999\,999, je ich celkom
900\,000. Stačí teda ukázať, že aspoň 600\,000 z nich je
deliteľných dvoma alebo tromi. Deliteľných dvoma je ich 450\,000,
deliteľných tromi 300\,000. V~súčte $450\,000+300\,000=750\,000$
sú však započítané dvakrát čísla, ktoré sú deliteľné dvoma
aj tromi, \tj. čísla deliteľné šiestimi. Tých je 150\,000, takže dvoma alebo
tromi je deliteľných práve $750\,000-150\,000=600\,000$ šesťciferných
čísel. Poznámka: Podobne zistíme, že existuje 660\,000
šesťciferných zložených čísel, ktoré sú deliteľné 2, 3 alebo 5,
teda počet šesťciferných prvočísel neprevyšuje $240\,000$.
Aj tento odhad je však veľmi hrubý~-- presný počet
šesťciferných prvočísel je 68\,906.]

Nájdite najväčšie trojciferné číslo, z~ktorého po vyškrtnutí
ľubovoľnej cifry dostaneme prvočíslo. [Číslo 731, pozri \pdfklink{67-C-S-1}{https://skmo.sk/dokument.php?id=2618}]

Nanajvýš koľko čísel možno vybrať z~množiny
$M=\{1,2,\dots,2\,018\}$ tak, aby rozdiel žiadnych dvoch vybraných čísel
nebol rovný prvočíslu?
[505 čísel, pozri \pdfklink{67-B-II-4}{https://skmo.sk/dokument.php?id=2753}]
\endnávod
}

{%%%%%   A-S-1
Prvá rovnica zadanej sústavy je splnená práve vtedy, keď sú
splnené nasledujúce dve podmienky:
\item{$\triangleright$} číslo $7z$ je celé,
\item{$\triangleright$} platí $7z\leq 3x+5y+7z< 7z+1$, čiže $3x+5y\in\langle0,1)$.

\noindent Podobne druhá a tretia rovnica sú splnené práve vtedy, keď čísla
$7x$ a $7y$ sú celé a~platí $3y+5z,3z+5x\in\langle0, 1)$.

Uvažujme teraz ľubovoľnú trojicu podľa zadania nezáporných čísel $(x,y,z)$,
ktorá je riešením úlohy. Z~nerovností $z\geqq0$ a $3z+5x<1$ vyplýva
$5x<1$, odkiaľ $7x<\frac75<2$. To znamená, že \emph{nezáporné celé}
číslo $7x$ sa rovná jednému z~čísel~0 alebo 1,
t.\,j. platí $x\in\{0,\frac17\}$. Podobne aj $y,z\in\{0,\frac17\}$.

V tejto chvíli máme už len $2^3=8$ trojíc $(x,y,z)$, ktoré sú
adeptami na riešenie úlohy, takže by sme ich mohli jednotlivo
testovať. Tomu sa vyhneme, keď si všimneme, že ak by
niektoré dve z~čísel $x$, $y$, $z$ boli rovné $\frac17$, jeden
z~výrazov $3x+5y$, $3y+5z$, $3z+5x$ by mal hodnotu $\frac87$,
ktorá je väčšia ako 1, a to je spor. Vieme teda, že {\it najviac jedno\/}
z~čísel $x$, $y$, $z$ je rovné $\frac17$ a ostatné sú rovné nule.
Potom ale každý z troch (nezáporných) výrazov $3x+5y$, $3y+5z$, $3z+5x$
je rovný nanajvýš $\frac57$, takže podmienky, ktoré sme uviedli
v~úvode riešenia ako ekvivalencie zadaných rovníc, sú splnené,
a teda všetky také trojice sú riešeniami.

\zaver
Úloha má práve 4 riešenia
$$\textstyle(x,y,z)\in \bigl\{(0,0,0), (\frac17,0,0), (0,\frac17,0), (0,0,\frac17 )\bigr\}.$$


\schemaABC
Za úplné riešenie dajte 6 bodov. V~neúplných riešeniach oceňte
čiastočné kroky nasledovne:
\item{A0.} Čísla $7x$, $7y$, $7z$ sú celé: 0 bodov.
\item{A1.} Správna odpoveď (aj bez dôkazu a skúšky): 1 bod.
\item{A2.} $3x+5y,3y+5z,3z+5x<1$: 1 bod za jednu alebo dve nerovnosti, 2 body za
všetky tri nerovnosti.
\item{A3.} $5x,5y,5z<1$ (alebo vo forme $x,y,z<\frac15$): 1 bod len za všetky
tri nerovnosti.
\item{B1.} Prevedenie úlohy na otestovanie určitého počtu opísaných trojíc
$(x,y,z)$: 4~body v~prípade jednociferného počtu, 3~body v~prípade
dvojciferného počtu.
\item{B2.} Úplné otestovanie všetkých trojíc z~B1: 1 bod v~prípade jednociferného počtu,
2~body v~prípade dvojciferného počtu.

\noindent
Celkom potom udeľte $\rm A1+\max(A2+A3,B1+B2)$ bodov.
\endschema

}

{%%%%%   A-S-2
Keďže podľa zadania $|BC|=|AP|=|EQ|$, $|BP|=|AQ|=|ED|$ a $|\angle CBP| = |\angle PAQ| = |\angle QED|$, sú podľa vety \emph{sus}
trojuholníky $PBC$, $QAP$ a $DEQ$ navzájom zhodné.
\inspsc{a72s.1}{0.8333}%

Z~toho vyplýva $|CP| =|PQ|=|QD|$ a tiež
$$
|\angle CPQ| =180^\circ -|\angle BPC|-|\angle APQ|=180^\circ - |\angle PQA| - |\angle EQD| = |\angle PQD|.
$$
To znamená, že podľa vety \emph{sus} sú tiež zhodné rovnoramenné
trojuholníky $CPQ$ a~$DQP$. Z toho vyplýva tiež zhodnosť ich výšok
z~vrcholov $C$ a $D$ na spoločnú protiľahlú stranu~$PQ$, a teda $CD\parallel PQ$.

\poznamka
Poznatok, že trojuholníky $CPQ$ a $DQP$ sú rovnoramenné a zhodné, je možné
získať tiež úvahou, že ide o dve (v~\obr{} nevyfarbené)
zodpovedajúce si časti zhodných štvoruholníkov~$QABC$ a $DEAP$.
Zhodnosť týchto štvoruholníkov je
dôsledkom toho, že podľa vety $sus$ platia zhodnosti
$\triangle QAB\cong \triangle DEA$ a
$\triangle ABC\cong \triangle EAP$.\fnote{Dalo by sa povedať, že
sme použili \uv{vetu \emph{susus}} o~zhodnosti dvoch konvexných
štvoruholníkov.}

V~nasledujúcom riešení upresníme, že zhodné zobrazenie
štvoruholníka $QABC$ na štvoruholník $DEAP$ je určité otočenie. Vďaka
tomu nové riešenie ukončíme inak (bez použitia výšok zhodných
trojuholníkov).


\ineriesenie
Označme $S$ stred kružnice, ktorá prechádza vrcholmi $B$,~$A$,~$E$.
V dôsledku zadania bodov $P$ a $Q$ platí $|BA|=|AE|$.
Preto v~otočení so stredom $S$ o~orientovaný uhol~$BSA$
platí $B\to A\to E$, a teda aj $P\to Q$ (\obr).
\inspsc{a72s.2}{0.8333}%

Ďalším dôsledkom vzťahov $B\to A\to E$ je zhodnosť štyroch uhlov
$SBA$, $SAB$, $SAE$ a~$SEA$. Odtiaľ vyplýva, že osami zhodných uhlov
$CBA$, $BAE$ a $AED$ sú postupne polpriamky $BS$, $AS$ a $ES$.
V~našom otočení je tak obrazom orientovaného uhla $CBS$
orientovaný uhol $BAS$, teda vzhľadom na $|BC|=|AP|$ platí $C\to P$.
To isté platí o~orientovaných uhloch $SAE$ a $SED$, rovnosť
$|AQ|=|ED|$ potom vedie ku $Q\to D$. Dokopy máme $C\to P\to Q\to D$,
odkiaľ vyplýva, že úsečky~$CD$ a $PQ$ majú spoločnú os~--
os úsečky~$CD$ totiž rozpoľuje uhol $CSD$, a teda
rozpoľuje aj uhol $PSQ$, a preto je tiež osou úsečky~$PQ$. Vďaka spoločnej
osi tak sú úsečky~$CD$ a~$PQ$ rovnobežné.


\schemaABC
Za úplné riešenie dajte 6 bodov. V~neúplných riešeniach oceňte
čiastočné výsledky nasledovne:
\item{A0.} Hypotéza o~zhodnosti trojuholníkov $CPQ$ a $DQP$ (bez dôkazu): 0 bodov.
\item{A1.} Zhodnosť troch trojuholníkov $DEQ$, $QAP$ a $PBC$ alebo zhodnosť dvoch
štvoruholníkov $CBAQ$ a $PAED$: 2 body.
\item{A2.} Zhodnosť trojuholníkov $CPQ$ a $DQP$: 2 body, z~toho 1 bod za rovnosti
$|DQ|=|QP|=|PC|$ a~1~bod za rovnosť $|\angle DQP| = |\angle QPC|$.
\item{B1.} Existencia otočenia, v ktorom $B \to A\to E$ a $P\to Q$: 2 body.
\item{B2.} Vzťahy $C\to P$ a $Q\to D$: 3 body (2 body za iba jeden vzťah).

\noindent
Celkom potom dajte $\rm \max(A1+A2,B1+B2)$ bodov.
\endschema
}

{%%%%%   A-S-3
Vzhľadom na rozklad $720=2^4\cdot3^2\cdot5$ má číslo 720 práve tri
prvočinitele 2, 3 a~5, takže každý jeho deliteľ má tvar
$2^\alpha\cdot3^\beta\cdot5^\gamma$, kde $\alpha$, $\beta$, $\gamma$ sú
nezáporné celé čísla spĺňajúce nerovnosti $\alpha\leq4$, $\beta\leq2$ a
$\gamma\leq1$ (ktoré ďalej potrebovať nebudeme). Určite aj súčin
ľubovoľných troch deliteľov čísla 720 má tvar $2^\alpha\cdot3^\beta\cdot5^\gamma$
s~nezápornými celými číslami $\alpha$, $\beta$ a~$\gamma$. Z~dvoch čísel
takého tvaru je prvé deliteľom druhého práve vtedy, keď hodnoty
$\alpha$, $\beta$, $\gamma$ pre prvé číslo neprevyšujú rovnomenné
hodnoty pre druhé číslo.

Tvrdenie úlohy dokážeme sporom.
Pripusťme, že niektoré štyri delitele čísla~720
majú tú vlastnosť, že žiadny z nich nedelí súčin troch ostatných deliteľov.
Potom každý z nich vo svojom rozklade musí mať
niektoré z~prvočísel 2, 3, 5 vo vyššej mocnine, než ju má vo svojom rozklade
súčin ostatných troch deliteľov, a teda aj ktorýkoľvek z nich.
Delitele však sú štyri a zastúpené prvočísla len tri, a to je
zrejmý spor.

\ineriesenie
Priamy dôkaz tvrdenia úlohy vyložíme jednou z niekoľkých možných
obmien (pozri poznámku za riešením). Rovnako ako v~prvom riešení
využijeme všetko, čo je obsiahnuté v~jeho prvom odseku.

Zvoľme teda ľubovoľné štyri delitele čísla~720. Bude
výhodné ďalej hovoriť o~prvočiniteli~$p$ daného deliteľa
aj v~prípade, keď v~jeho rozklade je $p$ \uv{zastúpené} ako činiteľ~$p^0$.
Najprv zo štyroch deliteľov vyberieme tri,
ktoré obsahujú prvočíslo 2 v~mocninách, ktoré neprevyšujú
túto mocninu u~štvrtého deliteľa (ak je možností takého výberu viac,
zvolíme jednu z~nich). Z týchto troch
deliteľov potom vyberieme dva, ktoré obsahujú prvočíslo 3 v~mocninách,
ktoré neprevyšujú túto mocninu u~tretieho deliteľa. Z~týchto dvoch
deliteľov nakoniec vyberieme ten, ktorý obsahuje prvočíslo 5
v~mocnine neprevyšujúcej túto mocninu u druhého deliteľa.
V~poslednom vybranom deliteľovi je potom každé $p\in\{2,3,5\}$
zastúpené v~mocnine, ktorá neprevyšuje aspoň jednu z~mocnín $p$
zastúpených v~ostatných troch deliteľoch. To zaručuje, že
posledný vybraný deliteľ má vlastnosť požadovanú zadaním úlohy.

\poznamka
Opíšme jednu z možných obmien podaného výkladu.
Z~ľubovoľne zvolených štyroch deliteľov najprv preškrtneme ten,
ktorý obsahuje prvočíslo 2 v~najvyššej
mocnine (ak je adeptov na preškrtnutie viac, preškrtneme len
jedného -- ktoréhokoľvek z~ich).\fnote{Tri nepreškrtnuté delitele
sú vlastne trojicou z~prvého výberu pôvodného postupu.}
Preškrtnutý deliteľ môžeme ponechať \uv{v hre} a zopakovať procedúru
škrtania jedného deliteľa ešte dvakrát:
raz pre prvočíslo 3 a druhýkrát pre prvočíslo 5.
Niektoré zo štyroch deliteľov tak môžu byť preškrtnuté aj viackrát; keďže
sme však škrtali iba trikrát, aspoň jeden deliteľ
zostane nepreškrtnutý, má teda zrejme požadovanú vlastnosť.


\schemaABC
Za úplné riešenie dajte 6 bodov. V~neúplných riešeniach oceňte
čiastočné závery nasledovne:
\item{A1.} Delitele čísla $720$ sú v tvare $2^\alpha\cdot3^\beta\cdot5^\gamma$
(stačí vymenovať všetky tri možné prvočinitele): 1~bod,
prípadne 2~body, len keď sú uvedené všetky poznatky z úvodného
odseku prvého riešenia a~pritom za A2 nie je udelený žiadny bod.
\item{A2.} Úvahy o~porovnaní rovnomenných stupňov $\alpha$, $\beta$, $\gamma$
(t.\,j. počtov výskytov jednotlivých prvočísel 2, 3, 5)
pre dané štyri delitele: 0--4 body podľa stupňa priblíženia
k~tvrdeniu úlohy.

\noindent
Celkom potom dajte $\rm A1+A2$ bodov.
\endschema

}

{%%%%%   A-II-1
V prvej časti dokážeme, že na splnenie úlohy je vždy
potrebných aspoň 64~ťahov vo zvislom smere a aspoň 8~ťahov vo
vodorovnom smere, celkom teda aspoň $64+8=72$ ťahov.

Je zrejmé, že s každým z 8 bielych žetónov musíme ťahať aspoň
štyrikrát smerom nadol a s~každým z~8~čiernych žetónov aspoň štyrikrát
smerom nahor. Celkom tak musíme naozaj urobiť aspoň
$8\times4 + 8\times4 = 64$ ťahov vo zvislom smere.

V každom stĺpci herného plánu na začiatku leží jeden biely žetón
nad jedným čiernym žetónom, na konci je to naopak.
Aspoň jeden z dvoch pôvodných žetónov musí teda svoj stĺpec
v niektorom ťahu opustiť, \tj. posunúť sa vo vodorovnom smere.
Keďže to platí pre každý z~8~stĺpcov, musíme naozaj
vykonať aspoň 8~ťahov vo vodorovnom smere.


V druhej časti riešenia ukážeme, že 72 ťahov na splnenie úlohy
stačí.\fnote{V poznámke za riešením uvedieme poznatok, podľa
ktorého možno \emph{všetky} riešenia 72 ťahmi zostrojiť, ak poznáme
všetky riešenia 18 ťahmi pre hrací plán $2\times5$.}
Rozdelíme za tým účelom daný hrací plán $8\times5$ na 4 časti $2\times5$
a v~každej z~nich presunieme žetóny
použitím $2+5+4+5+2=18$ ťahov v piatich etapách znázornených na
\obr. Celkovo to potom bude naozaj $4\cdot18=72$ ťahov.
{
\def\O#1{$\vcenter{\epsfboxsc{a72ii.1#1}{.8333}}$}
\def\P#1{\hfil$\mathbin{\mathop{\longrightarrow}\limits^{\textstyle #1}}$\hfil}
\vadjust{\bigskip\centerline{\O1 \P2 \O2 \P5 \O3 \P4 \O4 \P5 \O5 \P2 \O6}\medskip\centerline{\Obr}\bigskip}
}

\zaver
Najmenší možný počet ťahov je rovný 72.

\poznamka
Dokážeme, že každé riešenie zadanej úlohy 72 ťahmi má
nasledujúcu vlastnosť: {\sl Všetky vykonané ťahy možno rozdeliť do
štyroch skupín po 18 ťahoch tak, že ťahy z~tej istej skupiny (v pôvodnom
poradí) riešia \uv{redukciu} zadanej úlohy na jednu
zo štyroch častí $2\times5$, na ktoré je celý plán $8\times5$ rozdelený.}
Na to je nutné a súčasne stačí
ukázať, že žiadny žetón v priebehu daných 72 ťahov neopustí tú zo
spomínaných častí $2\times5$, v ktorej sa pôvodne nachádzal. Určite
si pritom stačí všímať len ťahy vo vodorovnom smere~-- hovorme im
$v$-ťahy.

Označme stĺpce hracieho plánu číslami 1 až 8 zľava doprava.
Nech $i\to i+1$, resp. $i\to i-1$ označuje $v$-ťah zo stĺpca $i$
doprava, resp. doľava. So sľúbeným dôkazom budeme hotoví, keď
ukážeme, že osem jednotlivých $v$-ťahov z každého riešenia 72 ťahmi
má (v~niektorom poradí) tvar
$$
1\to2,\ 2\to1,\ 3\to4,\ 4\to 3,\dots,7\to8,\ 8\to7.
$$
To je však dôsledok tej našej úvahy, podľa ktorej pri
\emph{ľubovoľnom riešení} pre každý stĺpec~$i$ existuje
aspoň jeden $v$-ťah $i\to{\ast}$; keďže pri
riešení $72$ ťahmi je $v$-ťahov práve osem, je
v~ňom po jednom ťahu $i\to{\ast}$ pre každé $i$, a teda
aj po jednom ťahu ${\ast}\to i$ pre každé $i$, pretože
počet ťahov \emph{zo} stĺpca $i$ sa musí rovnať počtu ťahov
\emph{do} stĺpca $i$.
Pre $i=1$ sa jedná nutne o~ťahy $1\to2$ a $2\to1$, a teda pre
$i=3$ o~ťahy $3\to4$ a $4\to3$, atď. až pre $i=7$ o ťahy $7\to8$ a
$8\to7$.


\schemaABC
Za úplné riešenie dajte 6 bodov. V~neúplných riešeniach oceňte
čiastočné kroky nasledovne:
\item{A1.} Dôkaz, že je potrebných aspoň 64 zvislých ťahov: 1 bod.
\item{A2.} Dôkaz, že je potrebných aspoň 8 vodorovných ťahov: 1 bod.
\item{A3.} Dôkaz, že je potrebných aspoň 72 ťahov: 3 body.
\item{A4.} Popis ľubovoľného minimálneho riešenia (\tj. postupnosti 72 ťahov, ktorá prevedie pôvodné rozostavenie na cieľové): 2 body.
\item{A5.} Vysvetlenie, prečo je riešenie z A4 skutočne zložené zo 72 ťahov: 1 bod.
\item{A6.} Popis a určenie počtu ťahov niektorého riešenia, ktoré nie je minimálne, \tj. má viac ako 72 ťahov: 1 bod.

\noindent
Celkom potom dajte $\rm\max(A1+A2,A3)+\max(A4+A5,A6)$ bodov.
\endschema
}

{%%%%%   A-II-2
Nech $(x,y)$ je ľubovoľné riešenie danej sústavy. Keďže
hodnota $\sqrt{\sqrt{x} + 2}$ je zrejme kladná,
podľa prvej rovnice je nutne $y>2$. Podobne z~druhej rovnice vyplýva
nutná podmienka $x>2$.

Teraz dokážeme, že čísla $x$ a $y$ musia byť rovnaké.\fnote{Zdôraznime,
že len zo symetrie sústavy rovníc rovnosť $x=y$ nevyplýva.}
Využijeme na to poznatok, že
funkcia \emph{druhá odmocnina} je rastúca. V~prípade $x>y$ by
teda platilo
$$
\sqrt{\sqrt{x}+2} > \sqrt{\sqrt{y}+2},
$$
čo je možné vďaka zadaniu prepísať ako $y-2>x-2$, čiže $y>x$, a to je spor.
Prípad $x<y$ sa vylúči analogicky. Rovnosť $x=y$ je tak dokázaná.

Zaoberajme sa preto ďalej (jediným možným) prípadom $x=y$.
Pôvodná sústava dvoch rovníc sa vtedy zrejme redukuje na jednu rovnicu
$$
\sqrt{\sqrt{x}+2}=x-2.
\tag1
$$
Po substitúcii $s=\sqrt x$, keď $x=s^2$, prejde rovnica (1) na rovnicu
$\sqrt{s+2}=s^2-2$, pritom zrejme $s>\sqrt2$.
Pre každé takéto $s$ môžeme rovnicu ekvivalentne
umocniť na druhú a získať tak rovnicu $s + 2 = (s^2 - 2)^2$,
ktorú ešte prepíšeme na zvyčajný tvar
$s^4 - 4s^2 -s+2 = 0$.
Všimnime si, že táto rovnica má koreň $s=2$.
Potvrdzuje to aj rozklad
$$
s^4 - 4s^2 -s+2=s^2(s^2-4)-(s-2)=s^2(s-2)(s+2)-(s-2)=
(s-2)(s^3 +2s^2 - 1),
$$
podľa ktorého teraz ukážeme, že $s=2$ je jediný koreň odvodenej rovnice,
ktorý spĺňa našu podmienku $s>\sqrt2$.
Naozaj, pre každé $s>\sqrt2$ totiž platí $s^3 +2s^2-1>0$
(platí to dokonca už pre $s=1$). Jediná vyhovujúca hodnota $s=2$
zodpovedá jedinému riešeniu $x=s^2=4$ rovnice (1),
a teda i~jedinému riešeniu $x=y=4$
zadanej úlohy.\fnote{Skúška pri našom postupe nie je nevyhnutná.}

\zaver
Zadaná sústava rovníc má jediné riešenie $(x,y)=(4,4)$.

\poznamka
Uveďme druhé možné odvodenie rovnosti $x=y$; iný postup
riešenia rovnice~(1) uvádzame v~poznámke za druhým riešením.

Nový dôkaz rovnosti $x=y$ začneme tak, že obe rovnice zo
zadania umocníme na druhú:
$$\eqalign{
\sqrt{x} + 2&=(y-2)^2,\cr
\sqrt{y} + 2&= (x-2)^2.}
$$
Odčítaním druhej umocnenej rovnice od prvej a následnými úpravami
rozdielu pravých strán dostaneme
$$ \setbox0=\hbox{$(y-2)^2 - (x-2)^2$}
\eqalign{
\sqrt{x} - \sqrt{y}&=(y-2)^2 - (x-2)^2=
\bigl((y-2) - (x-2)\bigr)\bigl((y-2) + (x-2)\bigr)=\cr
&=\hbox to \wd0{$(y - x)(x+y-4)$}=(\sqrt{y} - \sqrt{x})(\sqrt{y} + \sqrt{x}) (x+y-4).}
$$
Ak pripustíme, že $x\ne y$, po vydelení oboch krajných výrazov
hodnotou $\sqrt{y}-\sqrt{x}\ne0$ dostaneme
$$
-1=(\sqrt{y} + \sqrt{x})(x+y-4).
$$
Ako však vieme, obe čísla $x$ a $y$ sú väčšie ako 2, teda pravá strana
poslednej rovnosti je kladná, a to je želaný spor.

\ineriesenie
Využijeme opäť poznatok, že obe čísla $x$ a $y$ sú väčšie ako 2,
a~zavedieme funkciu $f\colon(2,\infty)\to(2,\infty)$ predpisom
$f(t)=\sqrt{t}+2$ pre každé~$t>2$. Rovnice zo zadania prepísané na tvar
$$
\eqalign{\sqrt{\sqrt{x}+2}+2&=y,\cr
\sqrt{\sqrt{y}+2}+2&=x}
$$
potom môžeme pomocou funkcie $f$ zapísať ako sústavu rovníc
$$
\eqalign{f\bigl(f(x)\bigr)&=y,\cr
f\bigl(f(y)\bigr)&=x.}
$$
Vidíme, že jej riešením sú práve dvojice tvaru
$(x, y)=\Bigl(x, f\bigl(f(x)\bigr)\Bigr)$, kde číslo~$x$ spĺňa vzťah $f\biggl(f\Bigl(f\bigl(f (x)\bigr)\Bigr)\biggr)=x$.
Ten je určite splnený v~prípade, keď platí $f(x)=x$. Ak ukážeme, že
to tak nie je vo zvyšných prípadoch, keď $f(x)<x$ alebo $f(x)>x$,
zostane nám jediná úloha~-- vyriešiť rovnicu $f(x)=x$.

Na vylúčenie prípadov $f(x)<x$ a $f(x)>x$ využijeme to, že funkcia
$f$ je zrejme rastúca. Vďaka tomu v~prípade $f(x)<x$ platí
štvorica nerovností
$$
f\biggl(f\Bigl(f\bigl(f(x)\bigr)\Bigr)\biggr)<f\Bigl(f\bigl(f(x)\bigr)\Bigr)<f\bigl(f (x)\bigr)<f(x)<x,
$$
lebo posledná nerovnosť vtedy platí a každá predchádzajúca nerovnosť je
dôsledkom bezprostredne nasledujúcej nerovnosti. Podobne sa
v~prípade $f(x)>x$ zdôvodní štvorica nerovností\fnote{K~dvom
uvedeným štvoriciam nerovností dodajme, že prípad $f(x)<x$
nastane pre každé $x>4$ a~prípad $f(x)>x$ pre každé $x\in(2,4)$.
Vyplýva to zrejme z rozkladu
$f(x)-x=(2-\sqrt{x})(\sqrt{x}+1)$, ktorý ďalej využijeme aj pri
riešení rovnice $f(x)=x$.}
$$
f\biggl(f\Bigl(f\bigl(f(x)\bigr)\Bigr)\biggr)>f\Bigl(f\bigl(f(x)\bigr)\Bigr)>f\bigl(f (x)\bigr)>f(x)>x.
$$
Tým je dôkaz tvrdenia o~ekvivalencii rovnosti $f\biggl(f\Bigl(f\bigl(f(x)\bigr)\Bigr)\biggr)=x$
s~rovnosťou $f(x)=x$ hotový.

Zostáva vyriešiť rovnicu $f(x)=x$ s~neznámou $x>2$,
čo je ľahké:
$$\def\ekv{\ \Leftrightarrow\ }
f(x)=x\ekv \sqrt{x}+2=x\ekv 0=(\sqrt{x}-2)(\sqrt{x}+1)\ekv
\sqrt{x}=2\ekv x=4.
$$
Dochádzame tak k rovnakému záveru ako v prvom riešení.

\poznamka
Úvahy o~funkcii $f$ z druhého riešenia je možné využiť aj na vyriešenie
rovnice~(1) z~prvého riešenia bez prechodu k rovnici štvrtého stupňa.
Rovnicu (1) možno totiž zapísať ako rovnicu $f\bigl(f(x)\bigr)=x$,
ktorá je však ekvivalentná s~jednoduchšou rovnicou~$f(x)=x$,
a to vďaka implikáciám
$$
f(x)<x\ \Rightarrow\ f\bigl(f(x)\bigr)<f(x)<x\qquad\hbox{a}\qquad
f(x)>x\ \Rightarrow\ f\bigl(f(x)\bigr)>f(x)>x,
$$
ktoré sa zdôvodnia rovnako ako v druhom riešení. Tam sme tiež
zjednodušenú rovnicu $f(x)=x$ vyriešili.

\schemaABC
Za úplné riešenie dajte 6 bodov. V~neúplných riešeniach oceňte čiastočné kroky nasledovne:
\item{A0.} Uvedenie podmienok $x\geqq2$ a $y\geqq2$: 0~bodov.
\item{A1.} Uhádnutie riešenia $(x,y)=(4,4)$ so skúškou: 1 bod.
\item{B1.} Dôkaz rovnosti $x=y$: 3 body.
\item{B2.} Vyriešenie úlohy za predpokladu $x=y$ vrátane vylúčenia iných hodnôt ako $x=4$: 3 body.
\item{C1.} Vyjadrenie sústavy v~tvare $f\bigl(f(x)\bigr)=y\land f\bigl(f(y)\bigr)=x$: 2 body.
\item{C2.} Odvodenie rovnosti $f(x)=x$ za~predpokladu $f\biggl(f\Bigl(f\bigl(f(x)\bigr)\Bigr)\biggr)=x$: 3 body.
\item{C3.} Vyriešenie rovnice $f(x)=x$: 1 bod.
\item{D1.} Vylúčenie rovnosti $f\biggl(f\Bigl(f\bigl(f(x)\bigr)\Bigr)\biggr)=x$ pre každé $x<4$: 2 body.
\item{D2.} Vylúčenie rovnosti $f\biggl(f\Bigl(f\bigl(f(x)\bigr)\Bigr)\biggr)=x$ pre každé $x>4$: 2 body.

\noindent
Celkovo potom dajte $\rm\max(A1,B1+B2,C1+C2+C3,C1+D1+D2)$ bodov.
Absenciu skúšky v~inak úplnom riešení nepenalizujte.
\endschema

}

{%%%%%   A-II-3
Najskôr si všimneme, že zo zadaných súmerností vyplývajú
rovnosti $|BR| = |BA|$ a $|SK| = |CD|$. Spolu s rovnosťami
zo zadania dostávame
$$
|AB|=|BC|=|CD|=|BR|=|SK|.
\tag1
$$

Podľa konštrukcie sú body $R$, $S$ zrejme rôzne, pritom stred $X$
úsečky $AR$ leží na jej osi~$BD$ a stred $Y$ úsečky $DS$ na jej
osi~$AC$. V~nasledujúcom odseku dokážeme, že bod $R$ leží vnútri
uhla $ABC$ a bod $S$ vnútri uhla $DCB$, ako je to na \obr{}.
Dokopy to bude znamenať, že body $A$, $D$, $R$, $S$ ležia
vnútri tej istej polroviny s~hraničnou priamkou~$BC$.
\inspsc{a72ii.31}{.8333}%

Z~predpokladu $|\uhol APD|<90^{\circ}$ vyplýva $|\uhol APB|>90^{\circ}$,
čo pre vnútorný bod $P$ základne~$AC$ rovnoramenného trojuholníka $ABC$
znamená, že $|\uhol ABP|<\frac12\cdot|\uhol ABC|$; odtiaľ však vyplýva
$|\uhol ABR|=2\cdot|\uhol ABP|<|\uhol ABC|$, teda bod $R$
naozaj leží vnútri uhla $ABC$. Analogicky z~nerovnosti
$|\uhol DPC|>90^{\circ}$ pre vnútorný bod $P$ základne $BD$
rovnoramenného trojuholníka $DCB$ usúdime, že bod $S$ naozaj leží
vnútri uhla $DCB$.

Ďalším dôsledkom nerovnosti $|\uhol APD|<90^{\circ}$ je, že pre
vyznačené vnútorné uhly pravouhlých trojuholníkov $APX$ a $DPY$ platí
$|\uhol XAP|=90^{\circ}-|\uhol APD|=|\uhol YDP|$, čiže
$|\uhol RAC|=|\uhol SDB|$.

Vráťme sa k rovnostiam~(1). Podľa nich je bod~$B$ stredom
kružnice opísanej trojuholníku $ARC$, ktorý ako vieme leží v~uhle $ABC$.
Podľa vety o~obvodovom a~stredovom uhle preto platí
$|\uhol RBC|=2\cdot|\uhol RAC|$. Podobnou úvahou o~strede~$C$
kružnice opísanej trojuholníku $BSD$ ležiacemu v~uhle $BCD$ dostaneme
$|\uhol SCB|=2\cdot|\uhol SDB|$.

Z~posledných dvoch odsekov vyplýva rovnosť $|\uhol RBC|=|\uhol SCB|$.
Tá spolu s~rovnosťami~(1) vedie k~záveru, že (rovnoramenné) trojuholníky
$RBC$ a $SCB$ sú podľa vety~\emph{sus} zhodné. Odtiaľ vyplýva zhodnosť
ich výšok z~vrcholov $R$ a $S$ na stranu~$BC$.
To vzhľadom na vyššie odvodenú polohu bodov $R$ a $S$ už znamená,
že $BC\parallel RS$.\fnote{Namiesto
úvahy o~zhodných trojuholníkoch $CBR$ a $BCS$ stačí konštatovať,
že zhodné úsečky $BR$ a $CS$ sú súmerne združené podľa osi
úsečky $BC$.}

\ineriesenie
Ukážeme, že
body $R$ a $S$ ležia na kružnici opísanej trojuholníku $BCP$.
Podrobný dôkaz zapíšeme len pre bod $R$, pre bod $S$
je totiž dôkaz analogický.

Rovnako ako v~prvom riešení odvodíme rovnosť (1) a poznatok, že
bod $R$ leží vnútri uhla $ABC$. Z~podmienky $|\uhol APD|<90^{\circ}$
zároveň vyplýva, že bod $R$ leží tiež v~polrovine~$ACD$.

Podľa (1) je bod $B$ stredom kružnice opísanej trojuholníku $ARC$,
ktorej stredový uhol $RBA$ s osou $BD$ je teda dvojnásobkom
obvodového uhla $RCA$. Preto sú zhodné tri uhly $PBA$, $RBP$
a $RCP$ vyznačené na \obr{}.
Zhodnosť posledných dvoch uhlov vzhľadom na predchádzajúci
odsek už znamená, že bod $R$ naozaj leží
na kružnici opísanej trojuholníku $BCP$. Pre bod $S$ to isté
platí vďaka analogickej zhodnosti uhlov $PCD$, $SCP$ a~$SBP$.
\inspsc{a72ii.32}{.8333}%

Z dokázaného vyplýva, že body $B$, $C$, $R$, $S$ ležia na jednej kružnici,
pritom body $R$ a~$S$ ležia v rovnakej polrovine s~hraničnou
priamkou $BC$. Odtiaľ vyplýva zhodnosť uhlov $BRC$ a~$BSC$,
ktorá spolu s rovnosťami $|BC|=|BR|=|CS|$ znamená, že
rovnoramenné trojuholníky~$RBC$ a $SCB$ sú zhodné. Zhodnosť ich
výšok z vrcholov $R$ a $S$ tak rovnako ako v~prvom riešení
už vedie k dokazovanému vzťahu $BC\parallel RS$.


\schemaABC
Za úplné riešenie dajte 6 bodov. V~neúplných riešeniach oceňte čiastočné kroky nasledovne:
\item{A1.} Zdôvodnenie rovnosti $|BR|=|BA|$ a $|SK|=|CD|$: 1 bod.
\item{A2.} Dôkaz, že $B$ je stred kružnice opísanej trojuholníku $ARC$: 1 bod.
\item{A3.} Dôkaz, že $C$ je stred kružnice opísanej trojuholníku $BSD$: 1 bod.
\item{B1.} Dôkaz, že body $A$, $D$, $R$, $S$ ležia vnútri tej istej polroviny s~hraničnou priamkou $BC$: 2 body.
\item{B2.} Odvodenie $|\angle CBR| = |\angle BCS|$: 2 body.
\item{B3.} Dôkaz, že trojuholníky $RBC$ a $SCB$ sú podobné: 3 body.
\item{C1.} Dôkaz, že body $R$ a $S$ ležia na kružnici opísanej trojuholníku $BCP$: 3 body za oba body~$R$ a~$S$, 2 body za jeden, ak nie je analógia pre druhý bod spomenutá.

\noindent
Body z B2, B3 a C1 dajte aj v prípade, keď riešiteľ neuvedie dôkaz z B1 a ani
potrebnú polohu bodov $R$ a $S$ explicitne nezmieni.

Celkom potom dajte $\rm \max(A1,A2+A3,B1+B2,B1+B3,B1+C1)$ bodov.
\endschema
}

{%%%%%   A-II-4
Najskôr si všimnime, že aspoň jedno z~čísel $a+b$, $a+c$ a $b+c$
musí byť párne. Naozaj, z troch čísel $a$, $b$, $c$
niektoré dve majú rovnakú paritu, teda ich súčet je párny.

Ak je súčin zo zadania mocninou prvočísla $p$,
tak každý zo štyroch činiteľov musí byť mocninou $p$. Ako už vieme,
niektorý z prvých troch činiteľov je párny, musí preto
platiť $p=2$.
Každý zo štyroch činiteľov je teda mocninou dvoch, ktorá je pritom
väčšia ako $1=2^0$, pretože čísla $a$, $b$, $c$ sú podľa zadania
prirodzené. Z~toho vyplýva, že každý činiteľ je párne číslo.

Ďalej pozorujme, že čísla $a+b$, $a+c$, $b+c$ sú všetky párne práve vtedy, keď čísla $a$, $b$,~$c$ majú všetky rovnakú paritu.
Keďže číslo $a+b+c+2036$ je párne, musia byť $a$, $b$ a $c$
párne čísla. Môžeme preto písať $a=2a_1$, $b=2b_1$ a $c=2c_1$,
kde $a_1$, $b_1$, $c_1$ sú prirodzené čísla. Potom však platí
$$
(a+b)(a+c)(b+c)(a+b+c+2036) =
2^4(a_1+b_1)(a_1+c_1)(b_1+c_1)(a_1+b_1+c_1+1018).
$$
Súčin posledných štyroch zátvoriek musí byť mocninou dvoch. Trojica
čísel $a_1$, $b_1$, $c_1$ je teda riešením pôvodnej úlohy
s~konštantou~1018 namiesto 2036. Z rovnakého dôvodu ako vyššie aj čísla $a_1$, $b_1$ a $c_1$ musia byť párne. Označíme
preto $a_2$, $b_2$ a $c_2$ postupne ich polovice (sú to opäť
prirodzené čísla) a dostaneme
$$
(a+b)(a+c)(b+c)(a+b+c+2036) = 2^8(a_2+b_2)(a_2+c_2)(b_2+c_2)(a_2+b_2+c_2+ 509).
$$
A~opäť tu máme rovnakú úlohu s~konštantou 509, takže aj čísla
$a_2$, $b_2$ a $c_2$ majú nutne rovnakú paritu. Keďže však číslo 509
je nepárne, musia byť $a_2$, $b_2$ a $c_2$ nepárne čísla.
Vidíme, že trojica $a_2=b_2=c_2=1$ úlohe vyhovuje (lebo $3+509=512$ je
mocnina dvoch), teda zodpovedajúca trojica $a=b=c=4$ je riešením pôvodnej úlohy.
Ukážeme, že je to riešenie jediné.

Pripusťme teda, že je niektoré z čísel $a_2$, $b_2$, $c_2$
väčšie ako jedna, nech je to $c_2$ bez ujmy na všeobecnosti.
Potom však mocnina dvoch rovná $a_2+c_2$ je väčšia ako 2, takže je
deliteľná štyrmi. To znamená, že po delení štyrmi
jedno z~nepárnych čísel $a_2$,~$c_2$ dáva zvyšok 1 a druhé zvyšok 3.
Tretie nepárne číslo $b_2$ tak má po delení štyrmi rovnaký zvyšok ako
jedno z~čísel $a_2$, $c_2$. Súčet $b_2$ s týmto číslom potom má
zvyšok 2 a~keďže tento súčet je zároveň mocninou dvoch, musí
ísť o~mocninu $2^{1}$. Z~toho vyplýva, že $b_2=1$ a~$a_2=1$
(rovnosť $c_2=1$ je vylúčená našim predpokladom $c_2>1$).
Zvyšok 3 po delení štyrmi tak nutne dáva číslo $c_2$.
Potom ale číslo $a_2+b_2+c_2+509$ rovné $c_2+511$ dáva po
delení štyrmi zvyšok 2, teda to nie je mocnina dvoch, a to je spor.

\zaver
Zadaniu úlohy vyhovuje jediná trojica $(a,b,c)=(4,4,4)$.

\ineriesenie
Nech bez ujmy na všeobecnosti platí $a \geq b \geq c$, takže potom
$a+b \geq {c+a} \geq {b+c}\geq 2$. Súčin $(a+b)(b+c)(c+a)(a+b+c+2036)$
je mocninou niektorého prvočísla práve vtedy, keď sú
mocninami tohto prvočísla všetky štyri
činitele $a+b$, $b+c$, $c+a$ a~$a+b+c+2036$. Nech teda
$a+b = p^k$, $c+a= p^l$ a~$b+c = p^m$ pre nejaké prvočíslo~$p$
a celé nezáporné čísla $k$, $l$ a $m$. Pre ne podľa úvodnej vety
platí $k \geq l \geq m \geq 1$.

Keby platilo $k>l$, a teda $k-1\geq l$ a $k-1\geq m$, vzhľadom
na $p\geq 2$ by sme mali
$$
a+b=p^k\geq p^{k-1} + p^{k-1}\geq p^l+p^m =(c+a)+(b+c)>a+b,
$$
a to je (podľa krajných výrazov) spor. Nutne teda $k=l$,
takže $a+b = p^k = p^l = c+a$, odkiaľ $b=c$. Potom však $p^m =b+c=2b$,
a teda $b=c=p^m/2$, teda $2\mid p$, čiže $p=2$,
a preto $b=c=2^{m-1}$ a $a+b=2^k$. Keďže $p=2$, je tiež $a+b+c+2036=2^n$
pre nejaké celé číslo $n$, ktoré zrejme spĺňa podmienku $2^n>2036$.

Podľa záveru predchádzajúceho odseku čísla $k$, $m$, $n$
spĺňajú rovnosť
$$
2^n = (a+b)+c+2036 = 2^k + 2^{m-1} + 2036.
$$
Všimnime si, že pri delení číslom 16 číslo 2036 dáva zvyšok 4, zatiaľ čo
väčšia mocnina~$2^n$ dáva určite zvyšok 0.\fnote{Motiváciu
k~úvahám o~deliteľnosti číslom 16 objasníme v~poznámke za
týmto riešením.} Z~vypísanej rovnosti tak
vyplýva $2^k + 2^{m-1} \equiv 12 \pmod{16}$. Sčítance $2^k$ a $2^{m-1}$
-- ako mocniny dvoch -- môžu pri delení 16 dávať iba zvyšky
1, 2, 4, 8 a 0. Ľahko nahliadneme, že odvodená kongruencia platí
jedine v~prípade, keď jedna z~mocnín $2^k$, $2^{m-1}$
dáva zvyšok 4 a druhá 8. Podľa týchto zvyškov ide nutne o~mocniny
$2^2$ a $2^3$, takže $\{k,m-1\}=\{2,3\}$. Keďže však
$2^{m-1}=b<a+b=2^k$, je nutne ${m-1}<k$, a preto $m-1=2$ a $k=3$,
teda $b=c=2^{m-1}=4$ a $a+b=2^k=8$, odkiaľ tiež $a=4$.
Dostávame tak jedinú možnú trojicu $(a,b,c)=(4,4,4)$.
Tá je skutočne riešením úlohy -- súčin zo zadania má vtedy hodnotu
$8\cdot8\cdot8\cdot2048$, pritom $8=2^3$ a $2048=2^{11}$.

\poznamka
Prečo sme rovnicu $2^n =2^k + 2^{m-1} + 2036$ riešili úvahami
o~deliteľnosti číslom 16? Bola to
vhodná forma riešenia tejto rovnice prepísanej do dvojkovej sústavy
(čomu sme sa chceli vyhnúť), v~ktorej má každá mocnina dvoch (ďalej
len \uv{mocnina}) zápis tvaru $100\dots0$, zatiaľ čo číslo 2036 má
11-ciferný zápis $111\,1111\,0100$. Aby sme z~neho pripočítaním
dvoch mocnín dostali opäť mocninu, je zrejmé, že
zápis menšej pripočítanej mocniny musí byť $100$ a tej väčšej potom $1000$,
\tj. jedná sa o~mocniny určené ich zvyškami po delení číslom $2^4=16$.
Podobne využijeme deliteľnosť číslom 16
aj pri všeobecnejšej rovnici $2^n=2036+2^{k-1}+2^{l-1}+2^{m-1}$
z~nasledujúceho tretieho riešenia.

\ineriesenie
Zachovajme označenie $p$, $k$, $l$, $m$, $n$ z~druhého riešenia.
Rovnako ako tam budeme predpokladať, že $a\geq b\geq c$, takže aj teraz
bude platiť $k\geq l\geq m\geq 1$. Ukážeme, ako je možné vynechať
odvodenie rovnosti $k=l$.

Najprv dokážeme, že $p=2$.\fnote{Ďalší spôsob odvodenia
rovnosti $p=2$ uvedieme v~poznámke za týmto riešením.}
Keby prvočíslo $p$ bolo nepárne, bol by nepárny aj súčet
čísel
$p^k + p^l + p^m$, ktorý je však vďaka úprave
$$
p^k + p^l + p^m = (a+b)+(c+a)+(b+c)=2(a+b+c)
$$
nutne párnym číslom, a to je spor. Prvočíslo $p$ je teda párne
a rovnosť $p=2$ je tak dokázaná. Podľa použitej úpravy, v ktorej
položíme $p=2$, navyše vidíme, že mocnina~$2^n$
(rovná štvrtému činiteľu $2036+a+b+c$) má vyjadrenie
$$
2^n=2036+(a+b+c)=2036+\dfrac12(2^k+2^l+2^m)=
2036+2^{k-1}+2^{l-1}+2^{m-1}.
$$
Odtiaľ podobne ako v~prvom riešení získame kongruenciu
$$
2^{k-1} + 2^{l-1} + 2^{m-1}\equiv 12 \pmod {16},
$$
tentoraz so súčtom troch mocnín dvoch na ľavej strane,
ktorých možné zvyšky po delení číslom~16 sú 1, 2, 4, 8, 0.
Rozborom možností pre súčet troch zvyškov zistíme,
že -- vzhľadom na $k\geq l\geq m$ --
zvyšky mocnín z~trojice $\bigl(2^{k-1},2^{l-1},2^{m-1}\bigr)$ tvoria
jednu z~trojíc $(4,4,4)$, $(8,2,2)$ alebo $(0,8,4)$.

(i) {\it Prípad trojice $(4,4,4)$}. Vtedy nutne platí
$2^{k-1}=2^{l-1}=2^{m-1}=4$, čiže $a+b=c+a=b+c=8$,
odkiaľ $(a,b,c) = (4,4,4)$. Dosadením overíme, že
táto trojica vyhovuje zadaniu úlohy.

(ii) {\it Prípad trojice $(8,2,2)$}. Vtedy máme $2^{k-1} = 8$ a
$2^{l-1} = 2^{m-1} = 2$, odkiaľ $a+b=16$ a $c+a=b+c=4$. To však
odporuje tomu, že $a+b<(c+a)+(b+c)$.

(iii) {\it Prípad trojice $(0,8,4)$}. Vtedy máme
$2^{l-1} = 8$ a $2^{m-1} = 4$, takže rovnosť
$2^n= 2^{k-1} + 2^{l-1} + 2^{m-1} + 2036$ získava tvar
$2^n=2^{k-1} + 2048$, kde $2048=2^{11}$, takže nutne $2^n>2^{11}$,
čiže $n\geqq12$. Z~rovnosti $2^n=2^{k-1}+2^{11}$ preto vyplýva
kongruencia $2^{k-1} \equiv 2^{11}\pmod{2^{12}}$, ktorá je splnená
iba pre $k=12$. Pre čísla $a$, $b$, $c$ tak platí
$a+b = 2^k = 2^{12}$, $c+a = 2^l=16$ a $b+c=2^m=8$, čo
opäť odporuje tomu, že $a+b<(c+a)+(b+c)$.

V~súhrne tak dochádzame k rovnakému záveru ako v~prvom riešení:
zadaniu úlohy vyhovuje jediná trojica $(a,b,c)=(4,4,4)$.

\poznamka
Predpokladajme, že čísla $a+b$, $b+c$, $c+a$ a $a+b+c+2036$
(väčšie ako~1) sú mocninami niektorého prvočísla $p$
a uveďme ďalší dôkaz rovnosti $p=2$.

Keďže ľavá strana rovnosti
$$
2(a+b+c+2036)-(a+b)-(b+c)-(c+a)=4072
$$
je deliteľná prvočíslom $p$, z~rozkladu $4072=2^3\cdot509$ vyplýva,
že $p$ je rovné jednému z~prvočísel 2 alebo 509. Pripusťme, že
$p=509$. Po vydelení uvedenej rovnosti číslom 509 potom dostaneme
$$
2\cdot\frac{a+b+c+2036}{509}-\frac{a+b}{509}-\frac{b+c}{509}
-\frac{c+a}{509}=8.
$$
Všetky štyri zlomky v tejto rovnosti sú celé čísla a súčasne
mocniny 509, dávajú teda po delení 509 zvyšky 0 alebo 1. Je preto
zrejmé, že ľavá strana rovnosti nemôže dávať po delení 509
zvyšok 8, ktorý dáva pravá strana. Tento spor už dokazuje, že
naozaj platí $p=2$.

\schemaABC
Za úplné riešenie dajte 6 bodov. V~neúplných riešeniach oceňte
čiastočné kroky nasledovne:
\item{A1.} Správna odpoveď (aj bez odvodenia a skúšky): 1 bod.
\item{B1.} Odvodenie rovnosti $p=2$ úvahou o~parite čísel $a$, $b$, $c$: 2 body.
\item{B2.} Odvodenie, že čísla $a$, $b$, $c$ sú párne, a prechod k~úlohe s~konštantou 1018: 1 bod.
\item{B3.} Prechod k~úlohe s~konštantou 509 a pozorovanie, že každé jej riešenie je trojica nepárnych čísel: 1 bod.
\item{B4.} Doriešenie úlohy s~konštantou 509: 0--2 body podľa miery úplnosti.
\item{C1.} Zdôvodnenie, že aspoň dve z~čísel $a$, $b$, $c$ sa rovnajú: 2 body.
\item{C2.} Odvodenie rovnosti $p=2$ za predpokladu z~C1: 1 bod.
\item{C3.} Odvodenie rovnice typu $2^n = 2^k + 2^{m-1}+2036$: 1 bod.
\item{C4.} Vyriešenie rovnice z~C3: 0--2 body podľa miery úplnosti.
\item{D1.} Odvodenie $p=2$ bez predpokladu, že aspoň dve z~čísel $a$, $b$, $c$ sa rovnajú: 2 body, z~toho 1~bod za zdôvodnenie $p\in \{2,509\}$.
\item{D2.} Odvodenie rovnice typu $2^n = 2^{k-1}+2^{l-1}+ 2^{m-1}+2036$: 1 bod.
\item{D3.} Vyriešenie rovnice z~D2: 0--3 body podľa miery úplnosti.

\noindent
Celkom potom dajte $\rm\max(A1,B1+B2+B3+B4,C1+C2+C3+C4,D1+D2+D3)$ bodov.
Absenciu skúšky v~inak úplnom riešení nepenalizujte.
\endschema

}

{%%%%%   A-III-1
Ukážeme, že hľadaný najmenší možný počet žetónov je 36.

V~prvej časti popíšeme stratégiu Bohuša, pri ktorej
s~36~žetónmi dokáže zabezpečiť, aby hra po žiadnom počte kôl
neskončila. Na začiatku Bohuš rozmiestni 36~žetónov
na každé druhé políčko hracieho plánu a napevno rozdelí
všetkých 72 políčok na 36~dvojíc susediacich políčok. Môže žetóny
posúvať tak, aby v~priebehu celej hry bol v~každej vytvorenej dvojici
políčok práve jeden žetón: V~každom kole totiž Alica
musí zvoliť prázdne políčko v~niektorej dvojici, Bohuš
potom naň presunie žetón z druhého políčka tejto dvojice.
Hra teda nikdy neskončí.

V~druhej časti riešenia budeme predpokladať, že
Bohuš na začiatku rozmiestni na hrací plán menej ako 36 žetónov.
Popíšeme stratégiu Alice, pri ktorej dokáže zabezpečiť, aby hra
skončila najneskôr 36.~kolom.

Na úvod si Alica predstaví, že políčka sú nastálo zafarbené striedavo
bielou a~čiernou farbou. V~každom kole potom Alica zvolí ktorékoľvek
prázdne biele políčko~-- také vždy nájde, lebo bielych políčok je
36, zatiaľ čo všetkých žetónov je menej. Bohuš tak bude nútený v~každom kole
presunúť žetón z~niektorého čierneho políčka na biele. S~každým
žetónom tak bude v priebehu celej hry môcť ťahať najviac raz a len s tými, ktoré na začiatku stáli na čiernom políčku.
Hra teda skutočne skončí najneskôr 36. kolom.

\ineriesenie
Uvedieme odlišný prístup iba k druhej časti pôvodného riešenia.
Budeme teda opäť predpokladať, že Bohuš na začiatku rozmiestni na hrací
plán menej ako 36~žetónov, teraz navyše tak, že žiadne tri susedné políčka
nebudú prázdne -- inak Alica hru ukončí prvým kolom tým, že
zvolí prostredné z týchto troch políčok. Ukážeme, že po nanajvýš
34 kolách si Alica vhodnou stratégiou vynúti situáciu,
keď takéto tri políčka už budú existovať.\fnote{V poznámke za týmto riešením
načrtneme, ako Alica môže túto stratégiu ďalej vylepšiť, aby
ukončila hru prípadne ešte skôr.}

Prázdne políčka sú teda rozdelené do
niekoľkých súvislých úsekov, tvorených vždy jedným alebo dvoma políčkami.
Také úseky s dvoma políčkami existujú aspoň dva -- aspoň jeden nájdeme pri
každom z~oboch rozdelení všetkých 72 políčok na~36 dvojíc susedných políčok,
lebo žetónov je najviac 35.

Alica umiestni medzi každé dve prázdne susedné políčka zarážku (a~po
každom kole vykoná korekciu polohy jednej z~nich). Na začiatku
tieto zarážky v~počte $z\geqq2$ rozdelia všetkých 72 políčok na $z$
úsekov. Každý z nich pritom obsahuje aspoň 3 políčka,
{\it začína aj končí prázdnym políčkom a neobsahuje dve susediace
prázdne políčka}. Alica určite môže z týchto úsekov vybrať
jeden, označme ho ďalej $U$, v ktorom je žetónov menej ako prázdnych
políčok (táto nerovnosť totiž platí pre ich celkové počty).

Nech $k\geqq1$ je to celé číslo, pri ktorom vybraný úsek
$U$ obsahuje $k+1$ prázdnych políčok a najviac $k$ žetónov.
Týchto žetónov však musí byť práve $k$ -- po jednom
v~každej z~$k$ \uv{medzier} medzi prázdnymi $k+1$ políčkami.
Úsek $U$ je tak tvorený nepárnym počtom $2k+1$ políčok, pre ktorý
navyše zrejme platí $2k+1\leqq 72-3=39$, čiže $k\leqq34$.
Pri zrejmom označení potom obsadenosť políčok v~okolí týchto
dvoch zarážok okolo úseku $U$ vyzerá takto:
$$
\dots0\mid \underbrace{0\,1\,0\,1\,\dots\,0\,1\,0}_{U}
\mid 0\dots
$$
Alica v~prvom kole zvolí v~úseku $U$ prvé políčko zľava.
Bohuš je potom vynútený k~presunu žetónu sprava -- tým sa ľavá
zarážka posunie o~dve pozície doprava, takže vznikne nový úsek $U'$
dĺžky $2k-1$:
$$
\dots 0\mid \underbrace{0\,1\,0\,1\,\dots\,0\,1\,0}_{U}\mid 0\dots\
\to\ \dots0\,1\,0\mid \underbrace{0\,1\,0\,1\,\dots\,0\,1\,0}_{U'}
\mid 0\dots
$$
V druhom kole Alica v~úseku $U'$ zvolí opäť prvé políčko zľava.
Procedúru neustále opakuje, až po $k$-tom kole, kde ako vieme
$k\leqq34$, dostane úsek medzi
dvoma zarážkami tvorený jedným políčkom, ktoré tak je prostredným
v~trojici susediacich prázdnych políčok. Tým je tvrdenie z~úvodného
odseku dokázané.

\poznamka
Možno dokázať, že Alica môže úsek $U$
z~predchádzajúceho riešenia vybrať tak, aby bol tvorený najviac 35 políčkami.
Okrem toho môže Alica svoju stratégiu pozmeniť tak, že v~úseku $U$
ukáže nie na krajné, ale buď na prostredné
prázdne políčko, alebo na jedno z~políčok vedľa prostredného obsadeného.
Potom sa po Bohušovom ťahu objaví v~úseku $U$ nová
zarážka, ktorá ho rozdelí na dva úseky~-- za $U'$ potom Alica vyberie
kratší z~nich. Opakovaním tejto procedúry dostane Alica postupnosť úsekov
s~počtami políčok, ktoré neprevyšujú postupne čísla 35, 17, 7, 3 a 1,
takže Alica hru ukončí najneskôr piatym kolom.
}

{%%%%%   A-III-2
Keďže $a_1,\ldots,a_n$ sú dĺžky strán $n$-uholníka, platia
zrejmé nerovnosti
$$\eqalign{
a_2 + a_3 + \ldots+ a_n &> a_1, \cr
a_1 + a_3 + \ldots+ a_n &> a_2 ,\cr
\omit\span\omit\hss\vdots\hss\cr
a_1 + a_2 + \ldots+ a_{n-1}&> a_n.}
$$
V prvej nerovnosti pripočítame k obom stranám $a_1$ a potom obe
strany vynásobíme kladným číslom $a_1$. Podobne v druhej
nerovnosti pripočítame k obom stranám $a_2$ a~potom ich obe vynásobíme
$a_2$ a tak ďalej. Dostaneme tak nerovnosti
$$\eqalign{
a_1(a_1 + a_2 + \ldots + a_n)&> 2a_1^2 , \cr
  a_2(a_1 + a_2 + \ldots + a_n)&> 2a_2^2 , \cr
\omit\span\omit\hss\vdots\hss\cr
a_n(a_1 + a_2 + \ldots + a_n)&> 2a_n^2.}
   $$
Ak spočítame všetkých $n$ týchto nerovností, dostaneme
$$
(a_1 + a_2 + \ldots + a_n)(a_1 + a_2 + \ldots + a_n)
>2(a_1^2 + a_2^2 + \ldots + a_n^2).
$$
Po odmocnení oboch (kladných) strán poslednej nerovnosti už
získame nerovnosť, ktorú sme mali dokázať.

\ineriesenie
Keďže v danej nerovnosti na označení dĺžok
strán nezáleží, môžeme predpokladať, že sú očíslované tak, že
platí
$a_n\geqq\max\{a_1, a_2, \ldots, a_{n-1}\}$. Zo všeobecne platnej
nerovnosti $a_1+a_2+\ldots+a_{n-1}>a_n$ potom dostaneme
$$\eqalign{
a_1 + a_2 + \ldots + a_n&=\sqrt{\left((a_1 + a_2 + \ldots +
a_{n-1}\right) + a_n)(a_1 + a_2 + \ldots + a_n)}>\cr
&>\sqrt{(a_n + a_n)(a_1 + a_2 + \ldots + a_n)}=\cr
&=\sqrt{2a_na_1 + 2a_na_2 + \ldots +2a_na_n}\geqq\cr
&\geq\sqrt{2a_1^2 + 2a_2^2 + \ldots +2a_n^2}.}
    $$
V poslednom kroku sme využili to, že pre každé $i\in\{1, 2, \ldots,n\}$
vďaka nášmu predpokladu $a_n\geq a_i>0$ platí $2a_na_i\geq2a_i^2$.
Tým je ostrá nerovnosť zo zadania úlohy dokázaná.
}

{%%%%%   A-III-3
Doplňme trojuholník $ABC$ na rovnobežník $ABPC$ (poz. \obr{}). Keďže
platí $HB\perp AC\parallel BP$, je uhol $HBP$ pravý. Podobne
z~$HC\perp AB\parallel CP$ vyplýva, že uhol~$HCP$ je tiež pravý.
Oba body $B$ a $C$ preto ležia na Tálesovej kružnici
s~priemerom~$HP$.
\inspsc{a72iii.31}{0.8333}%
Zrejme sa ďalej stačí zaoberať prípadom, keď platí $H\ne F$.
Vysvetlime, prečo potom stačí ukázať,
že {\it body $D$, $E$, $P$ ležia na jednej priamke}.
Vtedy totiž na tejto priamke leží aj bod $F$, takže uhol $HFP$
je pravý, a teda jeho vrchol $F$ leží
(spolu s~bodmi $B$, $C$ a~$H$) na kružnici s~priemerom $HP$.

V stredovej súmernosti so stredom v~bode $M$ označme
$L$ obraz bodu $D$ a~$J$~obraz bodu $I$. Z~tejto súmernosti vyplýva,
že $J$ je stredom kružnice vpísanej trojuholníku~$BCP$
a $L$ je bodom jej dotyku so stranou $BC$. Nech $KL$ je priemer
tejto kružnice. Bod $J$ je tak stredom úsečky $KL$.

Je dobre známe, že $D$ je bodom dotyku kružnice zvonka pripísanej strane
$BC$ trojuholníka $BCP$.\fnote{Súmerná združenosť bodu dotyku
pripísanej kružnice s bodom dotyku vpísanej kružnice podľa stredu
dotyčnej strany je dokázaná napríklad v~riešení úlohy
\pdfklink{63-B-S-3}{https://skmo.sk/dokument.php?id=1011}.}
Táto pripísaná kružnica je obrazom jeho kružnice
vpísanej v~rovnoľahlosti so stredom vo vrchole $P$ (a
koeficientom väčším ako 1). V~tej rovnoľahlosti
je dotyčnica~$BC$ kružnice pripísanej obrazom tej dotyčnice kružnice vpísanej,
ktorá je s ich spoločnou dotyčnicou~$BC$ rovnobežná, má však od vrcholu
$P$ menšiu vzdialenosť. Táto dotyčnica však prechádza bodom $K$,
keďže $KL$ je priemer kružnice vpísanej kolmý na obe dotyčnice.
Preto sa v~tejto rovnoľahlosti bod $K$ zobrazí na bod
$D$, a teda body $D$, $K$, $P$ ležia na jednej priamke. Zostáva tak
dokázať, že na tejto priamke leží aj bod~$E$. Na to stačí
ukázať, že úsečky $EP$ a~$DK$ sú rovnobežné. To sú však
strany postupne trojuholníkov $AEP$ a $LDK$ so strednými priečkami postupne
$IM$ a $MJ$, pre ktoré platí $IM\parallel EP$ a $MJ\parallel DK$;
odtiaľ už vyplýva vytúžená relácia $EP\parallel DK$, pretože $M$ je stredom
úsečky $IJ$.
}

{%%%%%   A-III-4
Keďže všetky členy $a_i$ sú prirodzené čísla,
platí
$$
a_5 = a_1a_2 + a_2a_3 + a_3a_4 - 1 \geq 1 + 1 + 1 - 1
=2,\ \hbox{a preto }\ a_5\ne1.
$$
Číslo $a_5$ je tak deliteľné aspoň jedným prvočíslom.


Teraz si všimnime, že pre každé $n\geq4$ platí
$$\eqalign{
a_n &=(a_1a_2 + a_2a_3 + \ldots + a_{n-3}a_{n-2})+ a_{n-2}a_{n-1}
- 1=\cr
&= (a_{n-1} + 1) + a_{n-2}a_{n-1} - 1=\cr
&= a_{n-1}(a_{n-2} + 1),}
     $$
a teda $a_{n-1} \mid a_n$. Nech $p$ je ľubovoľný prvočíselný deliteľ
$a_5$, potom z relácie $a_{n-1}\mid a_n$ vyplýva $p\mid a_6$, odkiaľ
zase vyplýva $p\mid a_7$ atď. Použitím princípu matematickej indukcie
vzhľadom na $n$ tak dostávame, že $p\mid a_n$ pre každé $n\geq5$.
Prvočíslo $p$ teda delí nekonečne veľa členov danej postupnosti.
Tým je dôkaz tvrdenia z~časti a) hotový.

Označme $\Cal P$ množinu všetkých prvočísel, ktoré sú deliteľmi nekonečne
mnohých členov postupnosti. Pripusťme, že (ako už vieme neprázdna)
množina $\Cal P$ je konečná, \tj. $\Cal P=\{p_1,\dots,p_k\}$
pre vhodné $k$. Zrejme pre každé $i\in\{1,2,\dots,k\}$ nájdeme
taký člen~$a_{n_i}$, ktorý je deliteľný
$p_i$ a má index $n_i\geqq5$. Vzhľadom na reláciu $a_{n-1}|a_n$
(dokázané skôr pre každé $n\geqq4$) potom platí $p_i\mid a_n$
pre všetky $n\geqq n_i$.
Ak teraz označíme $N=\max(n_1,\dots,n_k)$, bude číslo $a_N$
deliteľné všetkými prvočíslami $p_1,\dots,p_k$.
Prirodzené číslo $a_{N}+1>1$ potom nie je deliteľné žiadnym
prvočíslom z~$\Cal P$, musí teda
existovať prvočíslo $q\notin \Cal P$ s vlastnosťou $q\mid a_{N}+1$.
Toto prvočíslo $q$ je potom tiež deliteľom čísla
$a_{N+2}=a_{N+1}(a_{N}+1)$, teda platí $q\mid a_n$ pre
každé $n\geqq N+2$, a preto $q\in \Cal P$. Dostali sme tak spor,
ktorý dokazuje tvrdenie z časti b) úlohy.

\poznamka
Z riešenia časti a) vieme, že každé prvočíslo, ktoré delí niektorý
člen postupnosti počnúc tretím, delí aj všetky nasledujúce členy.
Stačí teda dokázať, že existuje nekonečne veľa prvočísel, ktoré
delia aspoň jeden člen danej postupnosti. Tento poznatok je
dôsledkom silnejšieho tvrdenia, že {\it pre každé $n\geqq1$ je
číslo~$a_{2n+3}$ deliteľné aspoň $n$ rôznymi prvočíslami}. Dôkaz
tohto tvrdenia tu uvádzať nebudeme.
}

{%%%%%   A-III-5
Stredná priečka $MP$ pretína ťažnicu $BN$ medzi bodmi $B$ a $G$,
takže bod~$K$ leží na polpriamke $PM$ a $BKGP$ je tetivový
štvoruholník. Podobne bod~$L$ leží na polpriamke $NM$ a $CLGN$ je
tetivový štvoruholník. Vzhľadom na $MP\parallel CA$ a $MN\parallel BA$
tak máme
$$
|\uhol BPK| = |\uhol BPM|=|\uhol BAC| =|\uhol MNC|=|\uhol LNC|,
$$
zatiaľ čo z oboch tetivových štvoruholníkov vyplýva
$$
|\uhol BKP| = |\uhol BGP| = |\uhol NGC| = |\uhol NLC|.
$$
\inspsc{a72iii.51}{.8333}%
Vidíme, že trojuholníky $BPK$ a $CNL$ sú podobné podľa vety \emph{uu} (\obr). Vďaka
tomu sú podobné aj trojuholníky $ABK$ a $ACL$, a to podľa vety \emph{sus},
lebo
$$
\postdisplaypenalty=1000
\eqalign{
&\hbox{\phantom{i}(i)}\quad|\uhol ABK| = |\uhol PBK| = |\uhol NCL| = |\uhol
ACL|,\cr
&\hbox{(ii)}\quad\frac{|AB|}{|BK|} = 2\cdot\frac{|PB|}{|BK|} =
2\cdot\frac{|NC|}{|CL|} =\frac{|AC|}{|CL|}.}
$$
Tým je rovnosť $|\uhol BAK| = |\uhol CAL|$ dokázaná.
}

{%%%%%   A-III-6
Jednotlivé štvorčeky papiera $n\times n$ nazývajme
ďalej {\it políčkami}.
Dokážeme, že prefarbenie\fnote{Budeme tým ďalej mať na mysli ako zmenu farby políčka z~bielej na čiernu, tak aj naopak.}
všetkých $n^2$ bielych políčok je po určitom počte krokov možné práve vtedy, keď platí $n>3$ a zároveň je číslo
$n$ deliteľné dvoma alebo troma.

Ukážme najprv, že pre $n=3$ požadované prefarbenie neexistuje.
Uvažujme za tým účelom na papieri $3\times 3$
tri políčka A, B, C podľa \obr{}.
\inspsc{a72iii.61}{.8333}%

Uvedomme si, že v~každom kroku sa prefarbí políčko A~a práve
jedno z~políčok~B alebo~C. Ak by sme po určitom počte krokov
všetkých 9~políčok prefarbili, počty prefarbení políčok
B a C by boli nepárne, teda počet prefarbení políčka A by bol párny,
a preto by sa vo výsledku políčko A neprefarbilo, čo je spor.
Ďalej už preto budeme predpokladať, že platí $n\geqq4$.
Tvrdenie z~úvodného odseku riešenia dokážeme v troch etapách.

(i) Ak je $n$ párne, využijeme opakovane postup podľa \obr{},
pri ktorom štyrmi krokmi v zeleno vyznačenom štvorci $4\times4$
prefarbíme práve 4 políčka jedného štvorca $2\times2$.
Papier $n \times n$ rozdelíme na $\left(\frac12n\right)^2$~štvorcov
$2\times2$ a postupne v~každom z~nich políčka
uvedeným postupom prefarbíme.
Pri tých štvorcoch, ktoré sú na hranici papiera, budeme konštrukciu
z~\obrr1{} vhodne otáčať, aby potrebný štvorec $4\times 4$
ležal celý vnútri papiera.
{
\def\O#1{$\vcenter{\epsfboxsc{a72iii.6#1}{.8333}}$}
\def\P{$\longrightarrow$}
\vadjust{\bigskip\centerline{\O2 \P \O3 \P \O4 \P \O5 \P \O6}\medskip\centerline{\Obr}\bigskip}
}

(ii) Ak je $n$ deliteľné tromi, využijeme opakovane postup podľa
\obr{}, ktorý najskôr využíva dvakrát konštrukciu z~\obrr2{}.
Takto vo fialovo vyznačenom obdĺžniku~$4\times5$ prefarbíme práve
9~políčok jedného štvorca $3\times3$. Postup podobne ako v~časti (i)
uplatníme na jednotlivé štvorce $3 \times 3$, na ktoré celý papier rozdelíme,
pričom pre hraničné štvorce konštrukciu z~\obrr1{} opäť vhodne otáčame,
aby potrebný obdĺžnik $4\times5$ ležal celý vnútri papiera.
{
\def\O#1#2{$\vcenter{\epsfboxsc{a72iii.#1#2}{.8333}}$}
\def\P{$\longrightarrow$}
\vadjust{\bigskip\centerline{\O67 \P \O68 \P \O69 \P \O70}\medskip\centerline{\Obr}\bigskip}
}

(iii) Teraz budeme predpokladať, že $n$ nie je deliteľné ani dvoma,
ani tromi. Políčka papiera $n\times n$ v~každom riadku
označíme postupne číslami 0, 1, 2, 0, \dots{} ako na \obr{}.
\inspsc{a72iii.71}{.8333}%

Nech $a_i$ označuje počet čierno zafarbených políčok s číslom $i$,
$i\in\{0, 1,2\}$. Všimnime si, že parita každého z troch čísel
$a_i$ sa po každom kroku zmení, lebo v~ňom meníme farbu
niektorých políčok iba v troch susedných stĺpcoch,
a~to v~každom z~nich pri nepárnom počte políčok.
Keďže na začiatku máme $a_0=a_1=a_2=0$, po ľubovoľnom počte
krokov bude platiť $a_0 \equiv a_1 \equiv a_2 \pmod 2$.

Vďaka predpokladu $3\nmid n$ je políčok s~číslom 0 o~$n$
(celý jeden stĺpec) viac ako tých s~číslom $2$.
Keby po niektorom počte krokov
boli všetky políčka zafarbené čierno, mali by sme potom $a_0-a_2=n$,
čo by vzhľadom na predpoklad $2\nmid n$ znamenalo, že čísla
$a_0$ a~$a_2$ majú rôznu paritu. To ale odporuje záveru
z predchádzajúceho odseku. Po žiadnom počte krokov sa nám tak
nepodarí zadanú úlohu splniť.
}

{%%%%%   B-S-1
\def\s{\hbox{p}}\def\l{\hbox{n}}%
Úlohu začneme riešiť výpočtom rozdielu $M-D$, teda
počtu \emph{zlých} vyplnení, pri ktorých súčin čísel v~žiadnom riadku ani stĺpci tabuľky nie je násobkom desiatich. To nastane práve vtedy, keď
s~číslom~5 nie je v rovnakom riadku ani stĺpci žiadne párne číslo.

Pred sľúbeným výpočtom najskôr ilustrujme situáciu príkladom
jedného typu zlého vyplnenia. Pozície pre štyri
párne čísla 2, 4, 6, 8 a štyri nepárne čísla 1, 3, 7 a 9 označíme
písmenami $\s$, resp.~$\l$:
$$
\matrix
\s&\l&\s\cr
\s&\l&\s\cr
\l&5&\l
\endmatrix
$$

Umiestnenie čísla 5 pre zlé vyplnenie môžeme zvoliť deviatimi spôsobmi.
Pri každom z nich potom všetky štyri párne čísla 2, 4, 6, 8
musia ležať v políčkach mimo riadok aj stĺpec umiestneného
čísla~5. Také políčka sú práve štyri a na ne môžu byť
párne čísla rozmiestnené akokoľvek, teda $4!$ spôsobmi.
Ak máme také rozmiestnenie vybrané, tak čísla 1, 3, 7, 9 môžu byť na
zvyšné štyri doposiaľ neobsadené políčka tiež rozmiestnené akokoľvek,
teda opäť $4!$ spôsobmi.
Celkový počet $M-D$ zlých vyplnení tak má hodnotu
$$
M-D=9\cdot 4!\cdot 4!.
$$
Keďže navyše zrejme platí $M=9!$, dokopy dostávame
$$
1-\frac DM = \frac{M-D}{M} = \frac{9\cdot 4!\cdot 4!}{9!} =
\frac{4\cdot 3\cdot 2}{8\cdot 7\cdot 6\cdot 5} = \frac1{70},
\quad\hbox{odkiaľ}\quad\frac DM=1-\frac1{70}=\frac{69}{70}.
$$

\zaver
Hľadaný pomer $D:M$ je rovný $69:70$.

\poznamka
Po úvahe z prvého odseku je možné prejsť k
ekvivalentnej úlohe o vypĺňaní tabuľky $3\times3$
jedným číslom 5, štyrmi písmenami $\l$ a štyrmi písmenami $\s$.
Pre zodpovedajúce hodnoty $M'$ a $D'$ pritom platí
$M'=9\cdot{8\choose4}$ a $M'-D'=9$. Táto obmena však neprináša
oproti pôvodnému postupu žiadnu výhodu, navyše je potrebné spomínanú
ekvivalenciu doložiť rovnosťami $M=(4!)^2\cdot M'$ a $D=(4!)^2\cdot D'$.

\ineriesenie
Vyplnenie tabuľky $3\times3$ číslami $1,\dots,~9$ nazveme \emph{nepárnym},
ak je nepárny \emph{súčet} troch čísel v každom riadku aj v každom stĺpci
tabuľky. V riešení druhej úlohy domáceho kola sme ukázali, že
nepárne sú práve tie vyplnenia, pri ktorých nepárne čísla 1, 3, \dots, 9
zaberajú jeden riadok a jeden stĺpec tabuľky, a že pre počet $L$ všetkých
nepárnych vyplnení platí vzťah $L=M/14$.

Skúmajme opäť vyplnenia, ktoré sme v prvom riešení aktuálnej
úlohy nazvali \emph{zlé}. Tam sme úvahou o riadku a
stĺpci s~číslom~5 vlastne ukázali, že
každé zlé vyplnenie je nepárne. Nie každé nepárne vyplnenie je však
zlé, ako ukazuje nasledujúci príklad:
$$
\matrix
\s&\l&\s\cr
\s&5&\s\cr
\l&\l&\l
\endmatrix
$$
Ukážme, že nepárnych vyplnení je päťkrát viac ako zlých
vyplnení. Naozaj, pre konštrukciu nepárnych vyplnení zvoľme najprv
ten riadok a ten stĺpec, ktoré budeme vypĺňať nepárnymi číslami.
Potom máme pre číslo 5 na výber päť políčok, pritom
na zlé vyplnenie povedie jediné z nich. (Po umiestnení čísla 5
potom pre zvyšné 4 nepárne čísla a~všetky 4 párne čísla máme vždy
rovnaký počet $4!\cdot4!$ spôsobov, ako ich
umiestniť.)\fnote{Absenciu tohto dodatku v zátvorke je možné v riešeniach
tolerovať.}
Platí preto
${L=5(M-D)}$, čiže ${M-D=L/5}$, odkiaľ vzhľadom na ${L=M/14}$
dostávame ${M-D=M/70}$, teda ${D=69M/70}$, čiže
${D:M=69:70}$.

\ineriesenie
V oboch predchádzajúcich riešeniach sme volili výhodnejší postup, keď sa
vlastne počítajú vyplnenia, ktoré zadaniu úlohy {\it nevyhovujú}.
Ukážeme teraz, že je schodná aj náročnejšia
cesta priameho určenia počtu vyhovujúcich vyplnení. Sú to zrejme
práve tie vyplnenia, pri ktorých sa niektoré párne číslo
nachádza v rovnakom riadku alebo stĺpci tabuľky ako číslo 5.

Všetky vyhovujúce vyplnenia rozdelíme do skupín podľa toho, koľko je
dokopy párnych čísel v tom riadku a v tom stĺpci tabuľky, v~ktorých sa nachádza číslo~5.
Tento počet označíme $p$ a určíme, koľko vyplnení
pre jednotlivé možné $p$ od~1 do~4 existuje (ako už vieme,
počet ${p=0}$ majú práve tie vyplnenia, ktoré zadaniu úlohy nevyhovujú).
Tieto počty zapíšeme vždy v tvare súčinu, ktorého prvý činiteľ
bude rovný~9 (počet spôsobov umiestnenia čísla~5) a druhý činiteľ bude
rovný počtu výberov $p$ políčok pre párne čísla zdieľajúce s~číslom~5
riadok alebo stĺpec; tretí činiteľ bude rovný počtu spôsobov
vyplnenia už vybraných $p$~políčok niektorými párnymi číslami a~štvrtý činiteľ
počtu spôsobov rozmiestnenia zvyšných $4-p$~párnych čísel
do 4~políčok mimo riadok a~stĺpec čísla~5; napokon piaty činiteľ~$4!$
bude rovný počtu spôsobov vyplnenia zvyšných 4~políčok
nepárnymi číslami 1, 3, 7 a~9.\hfil\break\indent
\vbox{\offinterlineskip
   \halign{\strut\hss$#$:\qquad&$\hss#\hss\cdot{}$&$\hss#\hss\cdot{}$&$\hss#\hss\cdot{}$&$ \hss#\hss\cdot{}$&$\hss#\hss$&${}=#$\hss\cr
p=1&9&4&4&(4\cdot3\cdot2)&4!&9\cdot(4!)^2\cdot16\cr
p=2&9&6&(4\cdot3)&(4\cdot3)&4!&9\cdot(4!)^2\cdot36\cr
p=3&9&4&(4\cdot3\cdot2)&4&4!&9\cdot(4!)^2\cdot16\cr
p=4&9&1&4!&1&4!&9\cdot(4!)^2\cr}}
   \hfil\break
Celkový počet $D$ vyhovujúcich vyplnení preto má hodnotu
$$
D=9\cdot(4!)^2(16+36+16+1)=9\cdot(4!)^2\cdot69.
$$
Vzhľadom na zrejmú hodnotu $M=9!$ tak dochádzame k výsledku
$$
\frac{D}{M}=\frac{9\cdot(4!)^2\cdot69}{9\cdot8!}=
\frac{4!\cdot69}{5\cdot6\cdot7\cdot8}=\frac{69}{70}.
$$

\schemaABC
Za úplné riešenie dajte 6 bodov. V~neúplných riešeniach oceňte
čiastočné kroky nasledovne:
\item{A0.} Uvedenie hodnoty $9!$ pre počet $M$: 0 bodov.
\item{A1.} Uvedenie príkladu nevyhovujúceho vyplnenia tabuľky (pričom je zjavné, že riešiteľ si túto jeho vlastnosť uvedomuje): 1 bod.
\item{B1.} Charakterizácia \emph{nevyhovujúcich} vyplnení tabuľky (\tj. vlastnosť rozmiestnenia ďalších čísel v~závislosti na pozícii čísla~5): 3 body.
\item{B2.} Určenie jednej z hodnôt $(M-D)/M$ alebo $M-D$ vrátane zdôvodnenia: 2 body, 1 bod za správnu metódu s numerickou chybou.
\item{B3.} Odpoveď zapísaná pomerom $69:70$ alebo zlomkom $69/70$: 1 bod.
\item{C1.} Uvedenie pomeru $L:M=1:14$ z riešenia úlohy domáceho kola spolu s opisom nepárnych vyplnení (oboje možno vyhlásiť za známe): 1 bod.
\item{C2.} Každé nevyhovujúce vyplnenie je nepárne: 2 body so zdôvodnením, 1 bod bez zdôvodnenia.
\item{C3.} Uvedenie vzťahu $L=5(M-D)$: 2 body so zdôvodnením, 1 bod bez zdôvodnenia.
\item{D1.} Charakterizácia vyhovujúcich vyplnení tabuľky (\tj. vlastnosť rozmiestnenia ďalších čísel v~závislosti na pozícii čísla~5): 1 bod.
\item{D2.} Rozdelenie všetkých vyhovujúcich vyplnení do štyroch skupín podľa celkového počtu párnych čísel, ktoré zdieľajú s číslom 5 rovnaký riadok alebo stĺpec: 1 bod.
\item{D3.} Určenie hodnoty $D$ vrátane zdôvodnenia: 3 body, z toho 2 body za úplnosť metódy kombinatorického počítania a 1 bod za numerickú bezchybnosť.

\noindent
Celkom potom dajte $\rm\max\bigl(A1,\max(B1+B2,C1+C2+C3,D1+D2+D3)+B3\bigr)$ bodov.
\endschema
}

{%%%%%   B-S-2
Podľa zvyčajnej konštrukcie stredu vpísanej kružnice budeme
s~dôkazom hotoví, keď ukážeme, že
bod $C$ leží na osiach dvoch vnútorných uhlov trojuholníka~$AEX$~-- tých
pri vrcholoch $A$ a $E$.

Najskôr si všimneme dve vlastnosti úsečky $AE$,
ktorej priesečník so stranou $CD$ označíme $G$: Striedavé uhly $DAE$ a $BEA$
sú zhodné (ako je vyznačené na \obr{}) a~bod $G$ je zrejme stredom
strany $CD$.\fnote{Dokázať to možno niekoľkými spôsobmi: úvahou
o~rovnobežníku $ACED$ alebo o~strednej priečke trojuholníka $ABE$, prípadne
použitím dvojice zhodných trojuholníkov $DAG$ a $CEG$. Riešitelia však môžu
tvrdenia považovať, rovnako ako my, za zrejmé a nedokazovať ich.}
\insp{b72s.1}%
Keďže $F$ a $G$ sú stredy strán $BC$, resp. $DC$,
zo súmernosti štvorca $ABCD$ podľa uhlopriečky $AC$ vyplýva
zhodnosť (podfarbených) trojuholníkov $DAG$ a $BAF$. Preto je uhol $DAG$ zhodný s~(tretím červeno vyznačeným) uhlom $BAF$
a vďaka súmernej združenosti bodov $F$, $G$ podľa priamky $AC$
leží bod $C$ naozaj na osi uhla $EAX$ (ako je na obrázku
vyznačené zeleno).

Teraz si všimnime trojuholníky $BAF$ a $XEF$ s~pravými uhlami pri vrcholoch $B$
a~$X$. Keďže sa zhodujú aj v uhle
pri spoločnom vrchole $F$, sú zhodné aj ich tretie uhly
$BAF$ a $XEF$ (preto je aj uhol $XEF$ na obr. vyznačený
červeno).\fnote{Túto zhodnosť môžeme tiež odvodiť pomocou
obvodových uhlov $BAX$ a $BEX$ v~kružnici nad priemerom $AE$ --
tá je totiž opísaná štvoruholníku $ABXE$, pretože
oba uhly $ABE$ a $AXE$ sú pravé.}
Dokopy dostávame zhodnosť uhlov $CEA$ a $CEX$, podľa ktorej
bod $C$ naozaj leží na osi uhla~$XEA$.
Tým je celý dôkaz ukončený.

\ineriesenie
K~zadanému štvorcu $ABCD$ a určenému bodu $E$ ešte prikreslíme
dva ďalšie štvorce $DCEH$ a $BIJC$ podľa \obr{}.
Stred $F$ strany $BC$ určite leží na úsečke $AJ$.
\insp{b72s.2}%
Uvažujme otočenie so stredom $C$ a (orientovaným) pravým uhlom
$DCB$. V~ňom sa úsečka~$AJ$ zobrazí na na ňu kolmú úsečku $IE$,
takže priesečníkom týchto dvoch úsečiek je bod $X$ zo zadania úlohy
(kolmý priemet bodu $E$ na priamku $AF$, čiže $AJ$).

Vlastné tvrdenie úlohy dokážeme ako v~prvom riešení -- overíme,
že bod $C$ leží na osiach dvoch vnútorných uhlov trojuholníka $AEX$.

Keďže body $J$ a $E$ sú súmerne združené podľa priamky $AC$,
leží bod $C$ na osi uhla $JAE$, čiže $XAE$.
Podobne zo súmernosti dvojice bodov $A$ a $I$
podľa priamky~$BE$ vyplýva, že jej bod~$C$ leží na osi uhla
$AEI$, čiže $AEX$. Tým je sľúbený dôkaz hotový.

\poznamky
Ukážme, že otočenie so stredom $C$, ktoré sme
využili v~prvej časti riešenia, je možné použiť aj na iný dôkaz
pre druhú časť. V~tomto otočení totiž okrem $AJ\to IE$ platí rovnako
$AE\to ID$. Preto existujú dve kružnice so stredom $C$: tej prvej
sa dotýkajú úsečky $AJ$ a $IE$, tej druhej zase úsečky $EA$ a $ID$.
Úsečky $ID$ a $IE$ sú však súmerne združené podľa
priamky $IC$, teda obe spomínané kružnice splývajú v~jednu, ktorá
je preto vpísaná trojuholníku $AEX$.

Kvôli pokynom na bodovanie dodajme, že obe časti
podaného riešenia sme mohli uviesť v~opačnom poradí: najprv
označiť priesečník úsečiek $EI$, $JA$ ako~$X'$ a~ukázať, že bod $C$
je stredom kružnice vpísanej trojuholníku $AEX'$, až potom odvodiť
rovnosť $X'=X$.

\ineriesenie
Ukážme, ako možno myšlienky z oboch predchádzajúcich riešení
stručne podať použitím základných poznatkov o smerniciach priamok
z~analytickej geometrie.

Uvažujme karteziánsku sústavu súradníc s~počiatkom
$A$ a kladnými polosami postupne $AB$ a $AD$. V~nej má priamka $AF$
smernicu $|BF|/|AB|= 1/2$, zatiaľ čo
priamka $AE$ má smernicu $|BE|/|AB| = 2$. Keďže tieto
dve smernice sú navzájom prevrátené čísla, priamky $AF$ a
$AE$ sú súmerne združené podľa osi prvého kvadrantu, ktorou
však je priamka $AC$. Inak povedané, bod $C$ leží na osi uhla $EAX$.

Keďže súčin smerníc každých dvoch navzájom kolmých priamok je ${-1}$,
priamka $EX$ kolmá na $AF$ má smernicu ${-2}$. Keďže priamka $EA$
má opačnú smernicu $2$, je os uhla $AEX$ rovnobežná s~druhou
súradnicovou osou, takže to je nutne priamka $EC$. Bod $C$ tak
leží aj na osi uhla $AEX$ -- celé riešenie je teda hotové.

\schemaABC
Za úplné riešenie dajte 6 bodov. V~neúplných riešeniach oceňte
čiastočné poznatky nasledovne:
\item{A0.} Priamka $AE$ rozpoľuje stranu $CD$, striedavé uhly $DAE$ a $CEA$ sú zhodné, štvoruholník $ABXE$ je tetivový: 0 bodov.
\item{A1.} Priamka $AC$ je osou uhla $EAX$: 2 body.
\item{A2.} Priamka $EC$ je osou uhla $AEX$: 3 body.
\item{B1.} Zhodnosť uhlov $CEA$ a $BAF$: 1 bod.
\item{B2.} Zhodnosť uhlov $BAF$ a $XEF$: 1 bod.
\item{C1.} Bod $X$ je priesečníkom úsečiek $AJ$ a~$EI$ z~druhého riešenia: 3 body.
\item{C2.} Bod $C$ je stredom kružnice vpísanej trojuholníku s~vrcholmi v~bodoch $A$,
$E$ a priesečníku úsečiek $AJ$ a~$EI$: 3 body.

\noindent
Namiesto zhodností uhlov v~B1 a B2 môžu byť uvedené zhodnosti či
podobnosti pravouhlých trojuholníkov s týmito uhlami. Prípadné neúplné
analytické riešenie hodnoťte podľa pokynov A1 a A2,
ak je vedené týmto smerom.

Celkovo potom dajte $\rm\max\bigl(A1+\max(A2,B1+B2),C1+C2\bigr)$ bodov.
\endschema
}

{%%%%%   B-S-3
Predpokladajme, že kladné celé čísla $a$, $b$ spĺňajú rovnosť
$a^2 - b^2 = 2^m$ pre nejaké nezáporné celé číslo $m$.
Potom zrejme $a>b$ a $(a+b)(a-b) = 2^m$, teda aj kladné celé čísla
$a+b$ a $a-b$ musia byť mocninami dvojky s~celočíselnými
nezápornými exponentmi.

Pokiaľ by čísla $a$, $b$ mali rôznu paritu, obe čísla $a+b$ a
$a-b$ by boli nepárne, a~museli by preto obe byť rovné mocnine $2^0=1$.
To však nie je možné, pretože $a+b\geq2$.
Čísla $a$, $b$ teda musia mať rovnakú paritu,
takže obe čísla $a+b$ a $a-b$ sú párne, a teda mocniny dvojky
s~kladnými exponentmi.

Všimnime si, že platí
$$
a^2 + b^2 = \frac{1}{2}\bigl((a+b)^2 + (a-b)^2)\bigr) =
\frac{1}{2}(a+b)^2 + \frac{1}{2}(a-b)^2,
$$
pričom oba posledné sčítance sú podľa predchádzajúceho odseku
mocninami dvojky s~nezápornými celočíselnými exponentmi.
Tým je dôkaz hotový.

\poznamka
Je zrejmé, že nájdené mocniny dvojky rovné
$\frac{1}{2}(a+b)^2$ a $\frac{1}{2}(a-b)^2$ majú nepárne,
a teda aj kladné exponenty.

\ineriesenie
Využijeme opäť rozklad $a^2-b^2=(a-b)(a+b)$, podľa ktorého zo
zadania úlohy vyplýva, že platí
$a+b = 2^k$ a $a-b =2^m$ pre niektoré nezáporné celé čísla
$k$ a~$m$. Ak sa pozrieme na tieto dve rovnosti ako na
sústavu dvoch rovníc s~neznámymi $a$ a~$b$,
jej jednoduchým vyriešením dostaneme
$$
a =\frac12\bigl(2^{k}+2^{m}\bigr)=2^{k-1}+2^{m-1}\ \quad\hbox{a}\
\quad b=\frac12\bigl(2^{k}-2^{m}\bigr)=2^{k-1}-2^{m-1}.
$$
Dosadením týchto vyjadrení do súčtu $a^2+b^2$ a následnou
úpravou obdržíme
$$
a^2+b^2=\bigl(2^{k-1} + 2^{m-1}\bigr)^2+
\bigl(2^{k-1} - 2^{m-1}\bigr)^2
= 2^{2k-1} + 2^{2m-1}.
$$
To už bude hľadané vyjadrenie, pokiaľ ukážeme, že celočíselné
exponenty $2k-1$ a~${2m-1}$ sú nezáporné, \tj. že
obe čísla $k$ a $m$ sú rôzne od
nuly. Možno to urobiť rovnako ako v~prvom riešení, ponúkneme však
iný postup: Keby platilo $k=0$
alebo $m=0$, bola by príslušná z mocnín $2^{2k-1}$, $2^{2m-1}$
rovná zlomku $1/2$, teda by sa mu museli rovnať obe mocniny,
aby ich súčet $a^2+b^2$ bol celým číslom. Rovnosť
$a^2+b^2=\frac12+\frac12=1$ je však vylúčená, pretože
$a^2+b^2\geqq1+1=2$.

\poznamka
Z druhého riešenia bezprostredne vyplýva, že dvojice $(a,b)$ spĺňajúce
zadanie úlohy existujú, že ich je nekonečne veľa a
že všetky sú tvaru
$$
(a,b)=\bigl(2^u+2^v, 2^u-2^v\bigr),
$$
kde $u$, $v$ sú ľubovoľné celé čísla s vlastnosťou
$u>v\geq0$.


\schemaABC
Za úplné riešenie dajte 6 bodov. V~neúplných riešeniach oceňte
čiastočné kroky nasledovne:
\item{A0.} Overenie tvrdenia úlohy iba pre konkrétne vyhovujúce dvojice $a$, $b$ alebo uvedenie rozkladu ${a^2 - b^2} = (a+b)(a-b)$ bez ďalších záverov: 0 bodov.
\item{A1.} Konštatovanie, že čísla $a+b$, $a-b$ sú mocniny dvojky s~celočíselnými nezápornými exponentmi: 2~body.
\item{A2.} Vylúčenie prípadu, že je niektoré z~čísel $a+b$, $a-b$ rovné $2^0$: 1 bod.
\item{A3.} Uvedenie rovnosti $a^2+b^2=\frac{1}{2}(a+b)^2 + \frac{1}{2}(a-b)^2$: 3 body.
\item{B1.} Konštatovanie, že $a+b = 2^k$ a $a-b =2^m$ pre celé nezáporné $k,m$: 2 body.
\item{B2.} Vyjadrenie čísel $a$, $b$ pomocou čísel $k$ a $m$: 1 bod.
\item{B3.} Odvodenie rovnosti $a^2+b^2=2^{2k-1} + 2^{2m-1}$: 3 body.
\item{B4.} Vylúčenie prípadu, že niektoré z~čísel $k$, $m$ z B1 je rovné nule: 1 bod.

\noindent
Celkom potom dajte
$\rm\max\bigl(A1+A2+A3, B1+\max(B2,B3)+B4\bigr)$ bodov.
\endschema
}

{%%%%%   B-II-1
Označme $d$ najväčšieho spoločného deliteľa čísel $a$, $b$. Potom
platí $a =da'$ a~$b = db'$, kde prirodzené čísla $a'$ a $b'$ sú
nesúdeliteľné a pritom podľa zadania platí $a'>b'$.

Zadanú podmienku na súčet $a+b$ vyjadríme rovnosťou $da'+db'=8d$.
Z~nej po vydelení číslom $d$ dostaneme
$a'+b'=8$. Dvojica $(a',b')$ prirodzených čísel so súčtom~8
a vlastnosťou $a'>b'$ je preto jedna z~dvojíc
$(7,1)$, $(6,2)$ alebo $(5,3)$, prostrednú
z nich však rovno vylúčime kvôli súdeliteľnosti čísel 6 a 2.


Pre dvojicu $(7, 1)$ vychádza
$$
a^2-b^2=d^2\bigl(7^2-1^2\bigr)=48d^2=3\cdot(4d)^2.
$$
Trojnásobok tohto čísla je teda druhá mocnina celého čísla~$12d$.

Pre dvojicu $(5,3)$ dostávame
$$
a^2-b^2=d^2\bigl(5^2-3^2\bigr) =16d^2=(4d)^2,
$$
čo je priamo druhá mocnina celého čísla $4d$.

Tvrdenie úlohy tak platí v~oboch prípadoch, ktoré prichádzali do úvahy.

\poznamka
Z~rovnosti $(ka)^2-(kb)^2=k^2(a^2 - b^2)$ pre každé celé $k>1$
si možno všimnúť, že číslo $(ka)^2-(kb)^2$ alebo jeho
trojnásobok je druhou mocninou celého čísla práve vtedy,
keď je také číslo $a^2 - b^2$. Preto stačí tvrdenie úlohy
dokázať pre prípad, keď čísla $a$ a~$b$ sú nesúdeliteľné, \tj.
platí $d=1$.


\schemaABC
Za úplné riešenie dajte 6 bodov. V~neúplných riešeniach oceňte
čiastočné kroky nasledovne.

\smallskip
\item{A1.} Vyjadrenie $a = a'd$, $b = b'd$: 1 bod.
\item{A2.} Odvodenie rovnosti $a'+b'=8$: 3 body.
\item{A3.} Dôkazy tvrdenia úlohy pre každý z~jednotlivých prípadov, keď platí $a'+b'=8$: 0 -- 3 body podľa miery úplnosti.
\item{B1.} Zdôvodnenie, že je možné predpokladať $d=1$ (a nezavádzať tak čísla $a'$, $b'$, \tj. namiesto nich písať $a$, $b$): 2 body.

\smallskip\noindent
Celkovo potom dajte $\rm\max(A1,A2,B1)+A3$ bodov.
\endschema
}

{%%%%%   B-II-2
Nerovnosť, ktorú čísla $x$, $y$, $z$ spĺňajú, najprv algebraicky
upravíme:
$$\eqalign{
(x+y+z)^2&>2(x^2+y^2+z^2),\cr
2xz +2yz&>x^2+y^2+z^2-2xy,\cr
2x(y+z)&>(x-y)^2+z^2.
    }$$
Keďže posledná pravá strana ako súčet dvoch druhých mocnín je
nezáporná, musí byť kladný súčin $x(y+z)$ z~ľavej strany tejto rovnosti.
Keďže pôvodná nerovnosť je zrejme v~premenných $x$, $y$, $z$
symetrická, jej dôsledkom sú tri nerovnosti
$$
x(y+z)>0,\qquad y(z+x)>0,\qquad z(x+y)>0.
\tag1
$$
Podľa \thetag1 sú všetky tri čísla $x$, $y$, $z$ rôzne od nuly.
Keby teda dokazovaný záver neplatil, nastal by jeden z dvoch
nasledujúcich prípadov.

\smallskip
\item{$\bullet$} Dve z~čísel $x$, $y$, $z$ sú kladné a zvyšné je záporné.
\item{$\bullet$} Dve z~čísel $x$, $y$, $z$ sú záporné a zvyšné je kladné.

\smallskip\noindent
V~oboch týchto prípadoch však jedna z~nerovností \thetag1 neplatí:
Ak totiž vynásobíme súčet dvoch kladných (resp. záporných) čísel
s~číslom záporným (resp. kladným), dostaneme vo výsledku záporné
číslo. Tento spor už dokazuje tvrdenie úlohy.

\ineriesenie
Ukážeme iný spôsob prevedenia dôkazu sporom. Všimnime si
na úvod, že zadaná nerovnosť sa nezmení, ak trojicu čísel $(x,y,z)$
do nej dosadíme v~akomkoľvek inom poradí. To isté platí
pri zámene trojice $(x,y,z)$ trojicou $({-x},{-y},{-z})$.

Predpokladajme teda, že tvrdenie úlohy neplatí. Majme tak čísla
$x$, $y$, $z$, ktoré spĺňajú zadanú nerovnosť, avšak všetky tri
nie sú ani kladné, ani záporné. To pri usporiadaní $x\geq y\geq z$,
ktoré vieme zariadiť, znamená, že $x\geqq0$ a $z\leqq0$. Platí teda
$x\geqq y\geqq0\geqq z$ alebo $x\geqq 0\geqq y\geqq z$. Stačí
uvažovať iba prvú možnosť, lebo od tej druhej prejdeme k prvej,
keď trojicu $(x,y,z)$ zameníme trojicou $({-z},{-y},{-x})$.

Nech teda $x\geqq y\geqq0\geqq z$. Vzhľadom na to zadanú
nerovnosť, ktorú čísla $x$, $y$, $z$ spĺňajú, upravíme
nasledovne:
$$\eqalign{
0&>x^2+y^2+z^2-2xy-2xz-2yz,\cr
0&>(x-y)^2+z^2-2xz-2yz.
    }$$
Všetky členy na pravej strane sú však nezáporné~-- prvé
dva ako druhé mocniny, zvyšné dve nerovnosti ${-2xz}\geq0$ a
${-2yz}\geq0$ platia vďaka nášmu predpokladu $x\geqq y\geq0\geq z$.
Tým je dôkaz sporom ukončený.

\ineriesenie
Zadanú nerovnosť tentoraz upravíme na tvar
$$
(x-y-z)^2-4yz<0,
$$
z~ktorého vďaka nerovnosti $(x-y-z)^2\geqq0$ vyplýva $yz>0$,
takže čísla $y$, $z$ sú buď obe kladné, alebo obe záporné.
Vzhľadom na symetriu východiskovej nerovnosti
to isté platí aj pre zvyšné dvojice čísel $x$, $z$ resp. $x$, $y$,
takže všetky tri čísla $x$, $y$, $z$ sú buď kladné, alebo záporné,
ako sme mali ukázať.

\ineriesenie
Zadanú nerovnosť upravíme štandardným spôsobom na kvadratickú
nerovnosť v~premennej $x$:
$$
x^2-2(y+z)x+(y-z)^2<0.
$$
Takto zapísaná nerovnica s~neznámou $x$
má podľa predpokladu riešenie. Preto grafom kvadratickej
funkcia s~predpisom
$$
f(x)=x^2-2(y+z)x+(y-z)^2
\tag2
$$
je taká parabola, ktorej časť leží pod osou $x$. Príslušná
kvadratická rovnica $f(x)=0$ preto má dva rôzne korene, teda
jej diskriminant~$D$ je kladný:
$$
0<D=4(y+z)^2-4(y-z)^2=16yz.
$$
Odtiaľto vyplýva $yz>0$. Podobne ako v predchádzajúcom riešení
zo symetrie vyplýva taktiež $xy>0$ a $xz>0$, teda všetky tri čísla
$x$, $y$, $z$ majú rovnaké znamienka.

\poznamka
Ukážme, ako niektoré poznatky z~predchádzajúcich riešení, a
vlastne aj celý nový dôkaz, vyplývajú z ďalších úvah
o~vyššie spomínanej parabole, ktorá je grafom kvadratickej funkcie~\thetag2.
Tá vďaka zadanému predpokladu nadobúda aspoň jednu zápornú hodnotu.

Doplnením na štvorec trojčlena z~pravej strany~\thetag2
získame vyjadrenie, ktoré sme predtým využili v~treťom riešení,
teda
$$
f(x)=(x-yz)^2-4yz.
$$
Z~neho vidíme, že funkcia $f$ má
najmenšiu hodnotu v~bode $x_0=y+z$, ktorá je pritom rovná
$f(x_0)={-4yz}$. Z~nerovnosti $f(x_0)<0$ preto
vyplýva, rovnako ako z~nerovnosti $D>0$ vo štvrtom riešení,
nerovnosť $yz>0$. Preto $y$ a $z$ sú nenulové čísla
s rovnakým znamienkom, ktoré tak má aj číslo $x_0=y+z$.
Keďže navyše platí $f(0)=(y-z)^2\geq0$,
dané číslo $x$ s~vlastnosťou $f(x)<0$ má (vďaka polohe
paraboly) rovnaké znamienko ako číslo~$x_0$,
a teda (tu bez úvah o~symetrii)
ako aj obe čísla $y$ a $z$, ako sme mali dokázať.


\schemaABC
Za úplné riešenie dajte 6 bodov. V~neúplných riešeniach oceňte
čiastočné kroky nasledovne.

\smallskip
\item{A1.} Odvodenie nerovnosti typu $x(y+z)>0$: 3 body.
\item{A2.} Zdôvodnené usporiadanie typu $x\geqq y\geq0\geq z$ pri dôkaze sporom: 2 body.
\item{A3.} Úprava nerovnosti na tvar typu $2xz+2yz>(x-y)^2+z^2$: 2 body.
\item{B1.} Úprava nerovnosti na tvar typu $(x-y-z)^2-4yz<0$: 3 body.
\item{B2.} Odvodenie nerovnosti typu $xy>0$: 4 body.

\smallskip\noindent
Celkovo potom dajte
$\rm\max(A1,A2+A3,B1,B2)$ bodov.
\endschema
}

{%%%%%   B-II-3
Označme $\beta=|\uhol ABC|$. Z~rovnoramenných
trojuholníkov $BCD$ a~$CDE$ tak postupne máme
$$
\beta=|\uhol ABC|=|\angle DBC|=|\angle BCD|=|\angle ECD|=|\angle CDE|.
$$
Dokazovaná rovnosť $|AE|=|BE|$ bude platiť práve vtedy, keď v trojuholníku
$ABE$, v~ktorom $|\uhol ABE|=\beta$, bude aj $|\uhol EAB|=\beta$, čiže
$|\uhol EAD|=\beta$. To vďaka
rovnosti $|\angle ECD|=\beta$ nastane práve vtedy, keď
štvoruholník $ADEC$ bude tetivový. Túto jeho vlastnosť dokážeme
dopočítaním niekoľkých ďalších uhlov.

Pre vonkajší uhol $ADC$ trojuholníka $BCD$ platí $|\uhol ADC|=2\beta$.
Preto z~doposiaľ nevyužitého rovnoramenného trojuholníka $ADC$ vyplýva
rovnosť
$$
|\angle CAD|= 2\beta.
$$
Výpočtom tretieho uhla v~trojuholníku $CDE$ dostávame
$$
|\angle DEC|= 180^\circ - 2\beta.
$$
Vidíme, že $|\angle CAD|+|\angle DEC|=180\st$, čo znamená,
že $ADEC$ je naozaj tetivový štvoruholník, ako sme sľúbili dokázať.
\inspsc{b72ii.1}{.8333}%

\ineriesenie
Ukážme postup bez použitia poznatku, že štvoruholníku $ADEC$ je možné
opísať kružnicu.

Rovnosť $|AE|=|BE|$ dostaneme zo zhodnosti trojuholníkov $AEC$ a $BED$.
V~nich podľa zadania platí $|AC|=|BD|$ a $|EC|=|ED|$. Preto vďaka
vete \emph{sus} stačí overiť rovnosť $|\uhol ECA|=|\uhol EDB|$.

Obe veľkosti uhlov môžeme ako v predchádzajúcom riešení
vyjadriť pomocou~$\beta$. Tak z trojuholníka $ABC$
podľa rovností $|\uhol ABC|=\beta$ a $|\uhol CAB|=|\angle
CAD| = 2\beta$ platí
$$
|\angle ECA|=|\angle BCA|=180^\circ-|\uhol ABC|-\uhol CAB|
=180^\circ -3\beta.
$$
Podobne z~priameho uhla pri vrchole $D$ vzhľadom na
$|\angle CDE| = \beta$ a $|\angle ADC| = 2\beta$ vyplýva
$$
|\angle EDB|= 180^\circ - |\angle CDE|-|\angle ADC|=
180^\circ -3\beta.
$$
Tým je potrebná rovnosť $|\uhol ECA|=|\uhol EDB|$ dokázaná.

\ineriesenie
Bez počítania veľkostí uhlov ukážeme, že
$E$ je hlavný vrchol rovnoramenného trojuholníka $BEA$.

Rovnoramenné trojuholníky $BCD$ a $DCE$ so
spoločným uhlom pri krajnom bode~$C$ ich základní
sú podobné, platí teda $|DC|/|BC|=|EC|/|DC|$.
Použitím tejto rovnosti a rovností zo zadania dostávame
$$
\frac{|AC|}{|BC|}=\frac{|DC|}{|BC|}=\frac{|EC|}{|DC|}
=\frac{|EC|}{|AC|}.
$$
Trojuholníky $ABC$ a $EAC$ sú tak tiež podobné, a to
podľa vety \emph{sus}, pretože pri vrchole~$C$ majú spoločný vnútorný uhol.
Táto podobnosť spolu so zhodnosťou uhlov pri základni $AD$
rovnoramenného trojuholníka $ADC$ vedie k~rovnostiam
$$
|\angle AEC|=|\angle BAC|=|\angle DAC|=|\angle ADC|.
$$
Rovnoramenný trojuholník $BCD$ má tak pri svojom hlavnom vrchole $D$
vonkajší uhol $ADC$ zhodný s~vonkajším uhlom $AEC$ trojuholníka $BEA$
pri vrchole~$E$. Keďže tieto dva trojuholníky majú navyše pri vrchole $B$
spoločný vnútorný uhol, je taktiež $E$ hlavný vrchol
rovnoramenného trojuholníka $BEA$, ako sme sľúbili ukázať.

\poznamka
Kľúčový poznatok posledného riešenia, teda zhodnosť uhlov $ADC$ a $AEC$,
je zrejme ekvivalentná s~tým, že štvoruholník $ADEC$ je tetivový.
Overenie tejto vlastnosti štvoruholníka $ADEC$
sme v~prvom riešení podali odlišným postupom.

\schemaABC
Za úplné riešenie dajte 6 bodov. V~neúplných riešeniach oceňte
čiastočné kroky nasledovne.

\smallskip
\emph{Cestou prvého riešenia}
\item{A0.} Pozorovanie, že dokazovaná rovnosť vyplýva zo zhodnosti uhlov $ABE$ a $EAB$: 0 bodov.
\item{A1.} Odvodenie, že dokazovaná rovnosť vyplýva zo zhodnosti uhlov $EAD$ a $ECD$ alebo z~tetivovosti štvoruholníka $ADEC$: 2 body.
\item{A2.} Dôkaz niektorej z~rovností $|\angle CAD|+|\angle DEC|=180^\circ$ alebo $|\angle CAD|=|\angle BED|$ alebo $|\angle ECA| +|\angle ADE|= 180^\circ$ alebo $|\angle EAC|=|\angle EDC|$: 0--3 body podľa úplnosti.
\item{A3.} Úplný dôkaz tetivovosti štvoruholníka $ADEC$: 4 body.

\smallskip
\emph{Cestou druhého riešenia}
\item{B1.} Pozorovanie, že dokazovaná rovnosť vyplýva zo zhodnosti trojuholníkov $AEC$ a $BED$: 2~body.
\item{B2.} Dôkaz zhodnosti uhlov $ECA$ a $EDB$: 0--3 body podľa úplnosti.
\item{B3.} Úplný dôkaz zhodnosti trojuholníkov $AEC$ a $BED$: 4 body.

\smallskip
\emph{Cestou tretieho riešenia}
\item{C1.} Dôkaz podobnosti trojuholníkov $ABC$ a $EAC$: 3~body.
\item{C2.} Dôkaz zhodnosti uhlov $ADC$ a $AEC$: 4 body.
\item{C3.} Dokončenie dôkazu: 0--2 body podľa úplnosti.

\smallskip\noindent
Celkom potom dajte
$\rm\max\bigl(A1+\max(A2,A3),B1+\max(B2,B3), \max(C1,C2)+C3,A1+C2\bigr)$ bodov.
\endschema
}

{%%%%%   B-II-4
Cifry pre vyhovujúce čísla budeme postupne vyberať zľava doprava.
Prvá z~nich nesmie byť rovná $0$ ani $3$, teda to môže byť
ktorákoľvek cifra z~osemprvkovej množiny $\{1, 2, 4, 5, 6, 7, 8, 9\}$.
Každú ďalšiu cifru potom vyberáme z deväťprvkovej
množiny $\{0, 1, 2, 4, 5, 6, 7, 8, 9\}$.

Predpokladajme teraz, že už sme vybrali všetky cifry
okrem tej poslednej na mieste jednotiek. Pripomeňme, že
prirodzené číslo je deliteľné tromi práve vtedy, keď je tromi deliteľný
jeho ciferný súčet. Aby teda bolo výsledné číslo deliteľné tromi,
posledná (zatiaľ nevybraná) cifra už má jednoznačne
určený zvyšok po delení tromi.

Ak preto rozdelíme 9 možných cifier pre miesto jednotiek
do troch skupín podľa ich zvyšku po delení tromi, z ich
výpisu
$$
\{0, 6, 9\},\qquad\{1, 4, 7\}\qquad\hbox{a}\qquad\{2, 5, 8\},
\tag1
$$
vidíme, že pre výber poslednej cifry vždy
máme práve 3 možnosti.

Celkovo dochádzame (použitím pravidla súčinu) k~záveru,
že hľadaný počet 33-ciferných čísel deliteľných 3
neobsahujúcich cifru 3 je rovný
$$
8\cdot9^{31}\cdot3=2^3\cdot\Bigl(3^2\Bigr)^{31}\cdot3
=2^3\cdot3^{63}.
$$

\poznamky
\item{1.} Všimnime si, že pri našom postupe nemusíme ako poslednú
vyberať cifru na mieste jednotiek -- akákoľvek pozícia okrem tej
prvej funguje rovnako dobre. Ak by sme však ako poslednú
vyberali prvú cifru zľava, nedalo by sa pravidlo súčinu priamo
použiť (niekedy by sme mali 2, niekedy 3 možnosti).

\item{2.} K správnemu výsledku je možné dôjsť aj na základe poznatku,
že práve tretina z~33-ciferných čísel neobsahujúcich cifru 3
(iné čísla až do konca poznámky neuvažujeme)
je deliteľná tromi. Tento poznatok {\it nemožno vyhlásiť za zrejmý}
a musí byť v~riešení dokázaný, napríklad v~tejto forme:
{\sl Ak rozdelíme všetky 33-ciferné čísla do troch skupín podľa
ich zvyšku po delení tromi, budú počty
čísel vo všetkých troch skupinách rovnaké}. Naozaj,
každé 33-ciferné číslo dostaneme z ~32-ciferného čísla pripísaním
jednej cifry sprava. Z~rozdelenia všetkých možných takto
pripisovaných cifier do trojíc z~\thetag1 potom vyplýva, že
ľubovoľné 32-ciferné číslo
uvedeným spôsobom prispeje tromi číslami
do každej z troch skupín 33-ciferných čísel,
ktoré po delení dávajú rovnaký zvyšok. Preto tieto tri
skupiny naozaj majú rovnaký počet prvkov (rovný teda
trojnásobku počtu všetkých 32-ciferných čísel, ktorých je ${8\cdot9^{31}}$).

\schemaABC
Za úplné riešenie dajte 6 bodov. V~neúplných riešeniach oceňte
čiastočné kroky nasledovne.

\smallskip
\emph{Cestou vzorového riešenia}
\item{A1.} Stratégia počítania možností (nezávislých) výberov jednotlivých cifier 33-ciferného čísla s~výsledkami 1-krát 8 a 32-krát 9: 1~bod.
\item{A2.} Zdôvodnenie, že kvôli požadovanej deliteľnosti tromi môžeme prvých 32 cifier vybrať akokoľvek a pre poslednú cifru na mieste jednotiek potom máme vždy 3~možnosti: 3~body.
\item{A3.} Použitie pravidla súčinu (nie je ho nutné spomínať) pre záver z~A2: 1~bod.

\smallskip
\emph{Postupom z~poznámky $2$}
\item{B1.} Určenie počtu $8\cdot9^{32}$ všetkých 33-ciferných čísel bez cifry 3: 2~body.
\item{B2.} Dôkaz tvrdenia, že práve tretina z~33-ciferných čísel bez cifry 3 je deliteľná tromi: 0--3~body podľa úplnosti.

\smallskip
\emph{Iné}
\item{C1.} Akékoľvek určenie hľadaného počtu, ktorý však
nie je upravený na požadovaný tvar: 1--5~bodov podľa úplnosti
zdôvodnenia, ktoré musí byť v súlade s predchádzajúcimi pokynmi,
pritom 1~bod dajte za správny počet bez akéhokoľvek zdôvodnenia,
napríklad pri konštatovaní, že čísel so zvyškom 0 je zrejme
práve toľko, ako čísel so zvyškom 1 aj čísel so zvyškom 2.

\smallskip
\emph{Záverom}
\item{D1.} Úprava úplne odvodeného počtu v~nehotovej forme na požadovaný súčin mocnín prvočísel: 1~bod.

\smallskip\noindent
Celkovo potom dajte $\rm\max(A1+A2+A3,B1+B2,C1)+D1$ bodov. Ak
riešiteľ predpokladá, že 33-ciferný zápis 33-ciferného čísla môže začínať
nulou, dajte mu najviac 2 body.
\endschema
}

{%%%%%   C-S-1
Ukážeme, že celočíselnú hodnotu nadobúda 18 z~uvažovaných zlomkov.

Ak označíme $d$ menovateľ jedného zo zlomkov,
tak jeho čitateľ je rovný $(72-d)^2$.
Zlomok potom pre každé uvažované $d\in\{1,2,\dots,72\}$
môžeme upraviť nasledovne:
$$
\frac{(72-d)^2}{d} = \frac{72^2 - 2\cdot 72\cdot d + d^2}{d} =
\frac{72^2}d - 2\cdot 72 + d.
$$
Taká hodnota je celé číslo práve vtedy, keď číslo $d$ je deliteľom
čísla~$72^2$. Zostáva tak ukázať, že číslo $72^2$
má celkom $18$ kladných deliteľov rovných najviac $72$.
Vykonáme to dvoma rôznymi spôsobmi.

\smallskip\noindent
\emph{Prvý spôsob}.
Číslo ${72^2=(8\cdot 9)^2= 2^6\cdot 3^4}$ má celkom ${(6+1)\cdot
(4+1)=35}$~kladných deliteľov, pretože to sú práve čísla tvaru
${2^{a}\cdot3^{b}}$
pre ľubovoľné ${a\in\{0,1,\dots,6\}}$ a ľubovoľné
${b\in\{0,1,\dots,4\}}$. Všetky tieto čísla, ktoré pritom sú
najviac rovné 72, vypíšeme. Urobíme to systematicky, a to
tak,\fnote{Zvolená metóda vychádza z~toho, že
všetky delitele čísla $72^2$ možno zapísať do tabuľky
s~5~riadkami a~7~stĺpcami.}
že do prvého riadka zapíšeme čísla s rozkladom $2^a$ (až do $2^6=64<72$),
do druhého čísla s rozkladom $2^a\cdot3$, do tretieho čísla s rozkladom
$2^a\cdot3^2$ a do štvrtého čísla s~rozkladom $2^a\cdot3^3$
(vypisovať posledný piaty riadok s~číslami $2^a\cdot3^4$ je zbytočné,
lebo kvôli nerovnosti $3^4=81>72$ by bol prázdny):
$$\vbox{\offinterlineskip
\halign{\vrule depth 3pt height 12pt width0pt\quad$#$\quad\hss\vrule&\hbox to 12mm{\hss$#$}&\hbox to 12mm{\hss$#$}&\hbox to 12mm{ \hss$#$}&\hbox to 12mm{\hss$#$}&\hbox to 12mm{\hss$#$}&\hbox to 12mm{\hss$#$}&\hbox to 12mm{\hss $#$}&\hss$#$\cr
\hfill a\hfill &0&1&2&3&4&5&6&\cr
\noalign{\hrule}
2^a&1&2&4&8&16&32&64&\cr
2^a\cdot3&3&6&12&24&48&&&(96>72)\cr
2^a\cdot3^2&9&18&36&72&&&&(144>72)\cr
2^a\cdot3^3&27&54&&&&&&(108>72)\cr
}}$$
Vidíme, že hľadaný počet deliteľov je $7+5+4+2$, \tj. naozaj 18.

\smallskip\noindent
\emph{Druhý spôsob}. Tentoraz delitele čísla $72^2$, ktoré
neprevyšujú 72, nebudeme vypisovať. Využijeme pritom iba
vyššie uvedený poznatok, že všetkých deliteľov čísla $72^2$ je 35.
Jedným z týchto deliteľov je číslo 72; ak je $d$ ľubovoľný
z~deliteľov menších ako~72, iný deliteľ $d'$
určený rovnosťou ${d\cdot d'=72^2}$ spĺňa zrejme nerovnosť
${d'>72}$. Naopak, každému deliteľu ${d'>72}$ bude z~rovnosti
${d\cdot d'=72^2}$ zodpovedať deliteľ ${d<72}$. Vidíme, že
deliteľov čísla $72^2$ menších ako 72 je práve toľko, koľko ich je
väčších ako~72. Dokopy ich je ${35-1=34}$, takže deliteľov
menších ako 72 je ${34:2=17}$, a preto deliteľov rovných najviac 72
je naozaj ${17+1=18}$, ako sme sľúbili ukázať.

\poznamka
K~hľadanému počtu $18$ deliteľov je možné dôjsť aj postupným testovaním,
ktoré z~čísel od 1 do 72 sú deliteľmi čísla $72^2$. K~tomu sa
samozrejme oplatí mať informáciu o tom, že $72^2$ má rozklad
$2^6\cdot3^4$. Testovanie je možné výrazne urýchliť
tak, že najprv z~vypísaného radu čísel od 1 do 72 postupne vyškrtneme ako
nevyhovujúce všetky násobky prvočísel 5, 7 a 11,
prípadne aj niekoľkých ďalších, alebo dokonca násobky
všetkých prvočísel väčších ako 3 a menších ako 72. V~poslednom prípade
už nám zostanú nevyškrtnuté iba čísla tvaru $2^a3^b$,
pritom vďaka nerovnostiam $2^7>72$ a~$3^4>72$
to všetko budú naozaj delitele čísla $72^2$.

Dodajme, že namiesto uvedeného testovania menovateľov zlomkov zo
zadania je možné testovať priamo celočíselnosť samotných zlomkov. Tento
náročný postup je skôr úlohou pre počítač, preto ho do našej
bodovacej schémy nezahrnieme; o~jeho hodnotení sa zmienime
v~záverečnom odseku pokynov.

\schemaABC
Za úplné riešenie dajte 6 bodov. V~neúplných
riešeniach oceňte čiastočné kroky alebo zistenia nasledovne:
\item{A1.} Zápis zlomkov v tvare $(72-d)^2/d$: 1 bod.
\item{A2.} Zlomok s~menovateľom $d$ vyhovuje zadaniu práve vtedy, keď je $d$ deliteľom čísla~$72^2$: 2 body.
\item{B1.} Každý deliteľ čísla $72^2$ je číslo tvaru $2^a3^b$, pričom $0\leqq a\leqq 6$ a $0\leqq b\leqq 4$: 1 bod.
\item{B2.} Určenie počtu 18 tých čísel z~B1, ktoré neprevyšujú 72, ich výpisom: 3 body.
\item{C1.} Určenie počtu 35 všetkých deliteľov čísla $72^2$: 1 bod.
\item{C2.} Rozdelenie všetkých deliteľov čísla $72^2$ rôznych od $72$ do dvojíc $(d,d')$ s~vlastnosťou $d\cdot d'=72^2$: 2 body.
\item{C3.} Určenie hľadanej odpovede na základe C1 a C2: 1 bod.
\item{D1.} Určenie hľadanej odpovede testovaním, ktoré čísla od 1 do 72 sú deliteľmi čísla $72^2$: 4~body, pritom 1~bod strhnite za každé chybne otestované číslo či za chybné spočítanie prvkov správne určeného súboru všetkých vyhovujúcich čísel. (Priebeh celého testovania musí byť zaznamenaný, aby bola zrejmá jeho úplnosť, inak za D1 dajte najviac 1 bod.)

\noindent
Celkovo potom dajte
$\rm\max(A1,A2)+\max(B1+B2,C1+C2+C3,D1)$ bodov. Správne
postupy s~drobnými numerickými chybami alebo vynechaním (napr.
deliteľov 1 alebo 72 čísla $72^2$) oceňte 4 alebo 5 bodmi.

Ak riešiteľ testuje priamo zlomky zo zadania, zo 6~bodov
strhnite 1~bod za každý chybne otestovaný alebo zabudnutý zlomok
či za chybné spočítanie prvkov správne určeného súboru
všetkých vyhovujúcich zlomkov; ak zo zápisu postupu nie je
zrejmé, že boli otestované všetky zlomky,
dajte najviac 1--2 body.
\endschema
}

{%%%%%   C-S-2
Keďže $KL$ je stredná priečka trojuholníka $ABC$, platí
$KL\parallel BC$. Vďaka predpokladu $BC\perp AC$ to znamená, že
rovnako $KL\perp AC$. Priamka $KL$ tak je osou úsečky~$AC$, lebo je na ňu
kolmá a prechádza jej stredom $L$. Preto zadaný priesečník~$P$
spĺňa rovnosť $|PA|=|PC|$. Z~rovnoramenného trojuholníka $ACP$
so základňou~$AC$ preto s~prihliadnutím na pravý uhol $BCA$ vyplýva
$$
|\uhol PAC|=|\uhol PCA|=|\uhol BCA|-|\uhol BCP| =90^{\circ}-|\uhol BCP|.
$$

Ďalej si všimneme (\obr), že podľa Tálesovej vety má trojuholník $BCP$
pravý uhol pri vrchole~$P$, teda pre jeho uhly
pri vrcholoch $B$ a $C$ platí $|\uhol PBC|=90^{\circ}-|\uhol BCP|$.

Dokopy dostávame, že uhly $PAC$ a $PBC$ majú rovnakú veľkosť,
a tým je ich zhodnosť dokázaná.
\insp{c72s.1}%

\poznamka
Obe časti postupu je možné vyložiť aj v~opačnom poradí:
najprv pre veľkosť druhého uhla $PBC$ získať
z~pravouhlého trojuholníka $BCP$ vyjadrenie
$$
|\uhol PBC|=90^{\circ}-|\uhol BCP|=|\uhol BCA|-|\uhol BCP|=|\uhol PCA|
$$
a potom dokázať zhodnosť prvého uhla $PAC$ s~uhlom $PCA$ úvahou
o~osi úsečky~$AC$.


\schemaABC
Za úplné riešenie dajte 6 bodov. V~neúplných
riešeniach oceňte čiastočné kroky alebo výsledky nasledovne:
\item{A1.} Priamka $KL$ je osou úsečky $AC$: 2 body so zdôvodnením, 1 bod bez zdôvodnenia.
\item{A2.} Odvodenie rovnosti $|\uhol PAC|=|\uhol PCA|$: 3 body. Ak pritom chýba zdôvodnenie použitého poznatku A1, dajte len 2 body.
\item{A3.} Rovnosť $|\uhol BPC|=90^{\circ}$ z~Tálesovej vety: 1 bod.
\item{A4.} Odvodenie rovnosti $|\uhol PBC|=|\uhol PCA|$: 3 body.

\noindent
Celkovo potom dajte $\rm\max(A1,A2)+\max(A3,A4)$ bodov.
\endschema
}

{%%%%%   C-S-3
Ak zotrieme (v jednom kroku) dve čísla párne, budú nahradené
jedným párnym číslom. Ak zotrieme dve čísla nepárne, budú tiež
nahradené párnym číslom.
Ak však zotrieme jedno číslo párne a jedno číslo nepárne, budú
nahradené nepárnym číslom. Preto sa celkový počet čísel po každom kroku
zmenší o 1 jedným z dvoch spôsobov:

\smallskip
\item{(i)} Počet párnych čísel sa zmenší o~1 a počet nepárnych čísel sa nezmení.
\item{(ii)} Počet párnych čísel sa zväčší o~1 a počet nepárnych čísel sa zmenší o~2.

\smallskip
Najprv odpovieme záporne na časť b) zadanej otázky.
Keďže na začiatku je na tabuli päť nepárnych čísel, zostane ich
počet podľa (i) a (ii) vždy nepárny, takže sa nikdy nebude rovnať
nule.\fnote{Jednoduchšie možno konštatovať, že počet nepárnych
čísel bude vždy 5, 3 alebo 1.} Prípad b) je tak vylúčený.

Na to, aby sme ukázali, že odpoveď na časť a) je kladná, stačí
uviesť jeden príklad postupnosti najviac 7 krokov, po ktorých zostane na
tabuli len niekoľko rovnakých nepárnych čísel. Predtým však pre
zaujímavosť vysvetlíme, že sa musí jednať o~postupnosť 4 alebo 6
krokov.\fnote{Zdôraznime, že toto vysvetlenie nie je nutnou
súčasťou riešenia.}

Všimneme si, že na začiatku sú na tabuli 4 párne čísla. Na zotretie
všetkých párnych čísel tak potrebujeme aspoň 4 kroky. Navyše
však počet párnych čísel po každom kroku zmení
podľa (i) a (ii) svoju paritu, takže k ich počtu rovnému 0
môže dôjsť len po párnom počte krokov.
Z počtov od 4 do 7 tak prichádzajú do úvahy iba počty 4 a~6.

Teraz už uvedieme sľúbené príklady. Po 4 krokoch zostane na
tabuli päť jednotiek, ak využijeme pri nahradzovaniach rovnosti
$$
(9-8)^2=(7-6)^2=(5-4)^2=(3-2)^2=1.
$$
Predĺžením tohto postupu o 2 kroky, keď najprv dve z piatich
jednotiek nahradíme nulou a potom dvojicu jednotky a nuly opäť
jednotkou, zostanú po 6~krokoch na tabuli práve 3~jednotky.

Spôsobov, ako získať tri rovnaké nepárne čísla po 6 krokoch, je viac.
Napríklad pre získanie troch deviatok je možné postupovať takto:
$$\def\pmb#1{\hbox{\bf#1}}\eqalign{
(1,\pmb{2},3,\pmb{4},5,6,7,8,9)&\mapsto (1,3,4,5,6,7,8,9)\cr
(\pmb{1},\pmb{3},4,5,6,7,8,9)&\mapsto (4,4,5,6,7,8,9)\cr
(4,4,\pmb{5},\pmb{6},7,8,9)&\mapsto (1,4,4,7,8,9)\cr
(1,4,4,\pmb{7},\pmb{8},9)&\mapsto (1,1,4,4,9)\cr
(1,\pmb{1},\pmb{4},4,9)&\mapsto (1,4,9,9)\cr
(\pmb{1},\pmb{4},9,9)&\mapsto (9,9,9)
}$$

\zaver
Mohli to byť čísla nepárne, nie však čísla párne.

\schemaABC
Za úplné riešenie dajte 6 bodov. V neúplných
riešeniach oceňte čiastočné kroky nasledovne:
\item{A1.} Možné zmeny počtu nepárnych čísel na tabuli po každom kroku: 1 bod.
\item{A2.} Pozorovanie, že počet nepárnych čísel na tabuli má stálu paritu: 2 body.
\item{A3.} Zdôvodnenie, že nikdy na tabuli nezostanú len párne čísla (stačí uviesť, že počet nepárnych čísel bude vždy 5, 3 alebo 1): 3 body.
\item{B1.} Príklad najviac 7 krokov, po ktorých zostanú na tabuli rovnaké čísla: 3 body.

\noindent
Celkovo potom dajte $\rm\max(A1,A2,A3)+B1$ bodov.
\endschema
}

{%%%%%   C-II-1
Ukážeme, že Mach celkom napísal 6 alebo 16 písomiek.

Posledná známka mohla byť 1 alebo 2, inak by sa celkový priemer
naopak zhoršil. Tieto dva prípady ďalej rozoberieme, pričom
v~oboch z~nich označíme $p$ počet známok pred poslednou písomkou
a $s$ ich súčet.

(i) Ak bola posledná známka 1, je zadanie úlohy vyjadrené
sústavou rovníc
$$
\frac sp=2{,}6=\frac{13}5\qquad\hbox{a}\qquad \frac{s+1}{p+1}=2{,}5=\frac 52,
$$
po úprave $5s=13p$ a $2s=5p+3$. Aby sme vylúčili neznámu $s$,
od prvej rovnice vynásobenej dvoma odpočítame druhú rovnicu
vynásobenú piatimi. Pre neznámu $p$ tak dostaneme rovnicu
$0=2\cdot13p-5(5p+3)$ s~jediným riešením $p=15$,
ktorému zodpovedá $s=13p/5=39$. Hodnoty $p=15$ a $s=39$
sú pritom možné, napríklad pri 6 dvojkách a 9 trojkách.
Mach teda naozaj mohol celkom napísať $p+1=16$ písomiek.

(ii) V prípade známky 2 získame rovnice $s/p=13/5$ a
$(s+2)/(p+1)=5/2$, po úprave $5s=13p$ a $2s=5p+1$. Podobným
odčítaním ako v (i) dostaneme rovnicu $0=2\cdot13p-5(5p+1)$
s jediným riešením $p=5$, ktorému zodpovedá $s=13p/5=13$.
Hodnoty $p=5$ a $s=13$ vyjdú napríklad pri 2~dvojkách a
3~trojkách.
Mach teda naozaj mohol celkom napísať $p+1=6$ písomiek.

\poznamka
Výpisy príkladov známok možno nahradiť konštatovaním, že súčet
známok s~daným počtom $p$ môže zrejme nadobudnúť
všetky celočíselné hodnoty $s$ od $p$ po~$5p$.

\ineriesenie
Rovnako ako v prvom riešení označíme $p$ počet známok
pred poslednou písomkou, $s$ ich súčet a ukážeme, že
$(p,s)$ je jedna z dvojíc $(5,13)$ alebo $(15,39)$.
Príklady možných známok pre tieto dvojice opakovať v druhom riešení
už nebudeme.

Z vyjadrenia priemeru $s/p$ zlomkom v základnom tvare $13/5$ vyplýva,
že číslo $p$ je násobkom piatich, teda $p=5k$ pre vhodné prirodzené
$k$. Z rovnosti $s/p=13/5$ potom máme $s=13k$,
teda podmienku na poslednú známku $z$ môžeme zapísať rovnicou
$$
2{,}5=\frac52=\frac{s+z}{p+1}=\frac{13k+z}{5k+1}.
$$
Úpravou $5(5k+1)=2(13k+z)$, odkiaľ získame ekvivalentnú podmienku
$2z=5-k$. Vidíme, že do úvahy prichádzajú iba nepárne $k=1$ a
$k=3$. Hodnote $k=1$ podľa $2z=5-k$ zodpovedá $z=2$ a
z rovností $p=5k$ a $s=13k$ vychádza $p=5$ a $s=13$.
Podobne pre $k=3$ sú zodpovedajúce hodnoty $z=1$, $p=15$ a
$s=39$.

\poznamka
Úvodnú úvahu o deliteľnosti možno pri predchádzajúcom postupe použiť aj
neskôr. Predtým totiž môžeme zostaviť pre neznáme $p$, $s$, $z$
sústavu rovníc
$$
\frac sp=\frac{13}5\qquad\hbox{a}\qquad\frac{s+z}{p+1}=\frac52,
$$
tú potom upraviť napríklad na tvar
$$
5s = 13p, \quad 2s + 2z = 5p +5
$$
a až teraz využiť deliteľnosť, alebo ešte
z posledných dvoch rovníc podobne ako v~prvom riešení vylúčiť
neznámu $s$. Vtedy dostaneme rovnicu
$$
5\cdot(2s+2z)-2\cdot 5s=5\cdot(5p+5)-2\cdot13p, \quad\hbox{čiže}\quad
10z=25-p.
$$
Odtiaľ už vyplýva, že číslo $p$ je nepárne, deliteľné piatimi a menšie ako 25.


\schemaABC
Za úplné riešenie dajte 6 bodov. V neúplných riešeniach oceňte čiastočné
kroky nasledovne:

\medskip\noindent
{\it Cestou prvého riešenia}

\medskip\noindent
\item{A1.} Konštatovanie, že poslednou známkou $z$ mohla byť jednotka alebo dvojka: 1 bod.
\item{A2.} Vyriešenie prípadov $z=1$ a $z=2$: 5 bodov za oba prípady, 3 body za jeden prípad. Ak chýbajú príklady možných známok, strhnite 1 bod (za jeden prípad aj celkom za oba).

\medskip\noindent
{\it Cestou druhého riešenia}

\medskip\noindent
\item{B1.} Zdôvodnenie, že $p=5k$ a $s=13k$: 1 bod.
\item{B2.} Vyjadrenie $z$ (alebo $2z$) pomocou $k$: 3 body
\item{B3.} Určenie oboch trojíc $(p,s,z)$ a vylúčenie iných: 1 bod.
\item{B4.} Príklady možných známok pre určené dvojice $(p,s)$: 1 bod.

\medskip\noindent
Celkovo potom dajte $\rm \max\left(A1+A2,B1+B2+B3+B4\right)$ bodov.
\endschema
}

{%%%%%   C-II-2
Z danej podmienky vyplýva $a=1-b-c$. Preto prvý činiteľ skúmaného
súčinu má vyjadrenie
$$
a+bc=(1-b-c)+bc=(1-b)(1-c).
$$
Analogicky prepíšeme aj zvyšné dva činitele. Celý súčin je tak
rovný
$$
(1-b)(1-c)\cdot(1-c)(1-a)\cdot(1-a)(1-b)=\bigl[(1-a)(1-b)(1 -c)\bigr]^2.
$$
Keďže druhá mocnina každého reálneho čísla je nezáporná,
posledný výraz má naozaj nezápornú hodnotu. Tá je pritom rovná
nule práve vtedy, keď je základ druhej mocniny rovný nule, \tj.
$(1-a)(1-b)(1-c)=0$.
To zrejme nastane len vtedy, keď je aspoň jedno z~čísel $a$, $b$,
$c$ rovné~1. Ak platí napríklad $a=1$, tak zadaná podmienka $a+b+c=1$
prejde na tvar $b+c=0$, takže ju spĺňajú práve
tie dvojice $(b,c)$, ktoré sú tvaru $(b,c)=(t,-t)$
pre vhodné reálne číslo $t$. Všetky hľadané trojice $(a,b,c)$
tak sú práve tvaru
$$
(1,t,-t),\qquad (t,1,-t),\qquad (t,-t,1),
$$
kde~$t$ je ľubovoľné reálne číslo.

\poznamka
Opíšme dva iné vhodné spôsoby úprav činiteľov
zadaného súčinu.
\smallskip
Pri prvom z nich využijeme rovnosť $a+b+c=1$ nasledovne:
$$
a+bc=a\cdot1+bc=a(a+b+c)+bc=a^2+ab+ac+bc=(a+b)(a+c).
$$
Podobne $b+ca=(b+c)(b+a)$ a $c+ab=(c+a)(c+b)$, takže dokopy
$$
(a+bc)(b+ca)(c+ab)=(a+b)(a+c)\cdot(b+c)(b+a)\cdot(c+a)(c+b) =
\bigl[(a+b)(b+c)(c+a)\bigr]^2.
$$
Odtiaľ vyplýva rovnaký záver ako v pôvodnom riešení,
lebo napríklad $a+b=1-c$.
\smallskip
Pri druhom spôsobe dosadíme vyjadrenie $a=1-b-c$ do všetkých troch
činiteľov:
$$\setbox7=\hbox{$c+b-b^2-bc$}\def\r#1{\hbox to \wd7{$#1$}}
\eqalign{
a+bc&=\r{1-b-c+bc}=(1-b)(1-c),\cr
b+ca&=\r{b+c-bc-c^2}=(b+c)(1-c),\cr
c+ab&=\r{c+b-b^2-bc}=(b+c)(1-b).
}$$
Vynásobením týchto troch rovností obdržíme
$$
(a+bc)(b+ca)(c+ab)=(1-b)(1-c)\cdot(b+c)(1-c)\cdot(b+c)(1-b) =
\bigl[(b+c)(1-b)(1-c)\bigr]^2.
$$
Odtiaľ opäť vyplýva rovnaký záver ako v pôvodnom riešení, pretože
$b+c=1-a$.

\schemaABC
Za úplné riešenie dajte 6 bodov. V neúplných riešeniach oceňte čiastočné kroky nasledovne:

\medskip
\item{A1.} Úprava niektorého z činiteľov daného súčinu na súčin dvoch výrazov typu $1-a$ alebo $b+c$: 1~bod, 2~body dajte za takú úpravu všetkých troch činiteľov.
\item{A2.} Dôkaz nezápornosti daného súčinu: 2 body.
\item{A3.} Popis všetkých trojíc $(a,b,c)$ pre ktoré je daný súčin rovný nule: 2~body, z toho 1 bod, ak je odvodená iba skupina podmienok typu $a=1$ alebo $b+c=0$, avšak chýba popis zodpovedajúcich trojíc $(a,b,c)$, ako je zadaním úlohy vyžadované. Vyhovujúci je aj slovný popis: Sú to práve tie trojice reálnych čísel, v ktorých je niektoré číslo rovné 1 a súčet zvyšných dvoch čísel je rovný 0. Za podmienky typu $a+bc=0$ však žiadny bod neudeľujte.

\medskip\noindent
Celkom potom dajte $\rm A1+A2+A3$ bodov.
\endschema
}

{%%%%%   C-II-3
Ukážeme, že porovnávané obsahy štvoruholníkov~$CDEF$ a~$AGBF$ sú
si rovné.
\inspinsp{c72ii.1}{c72ii.11}%

Zo súmerností pravidelného päťuholníka (podrobne posúdených
v~riešení príkladu~5 z~domáceho kola) vyplýva, že
body $D$, $F$, $G$ ležia v~tomto poradí na jednej priamke a~že platí
$BC\parallel AD$. Dokopy dostávame, že $F$ je priesečník
uhlopriečok lichobežníka $GCDA$ so základňami $GC$ a $DA$ (\obr).
Podľa jedného pravidla dva trojuholníky
ohraničené ramenami lichobežníka a jeho uhlopriečkami majú rovnaký
obsah, v~našom prípade tak platí $S_{DFC}=S_{AFG}$.
Pre úplnosť túto rovnosť dokážeme v~nasledujúcom odseku.

Vďaka $DA\parallel GC$ majú trojuholníky $DAC$ a $DAG$ zhodné
výšky z~vrcholov $C$ a $G$ na spoločnú stranu $DA$ (\obr).
Preto pre ich obsahy platí $S_{DAC}=S_{DAG}$, pritom
$$
S_{DAC}=S_{DAF}+S_{DFC}\qquad\hbox{a}\qquad
S_{DAG}=S_{DAF}+S_{AFG}.
$$
Porovnaním už získavame rovnosť $S_{DFC}=S_{AFG}$, ktorú sme
chceli dokázať.

Teraz si všimneme, že oba zadané štvoruholníky $CDEF$ a~$AGBF$
sú súmerné podľa priamky $DG$, ktorá tak rozpoľuje obsah každého
z~nich. Preto z~rovnosti $S_{DFC}=S_{AFG}$ po násobení
dvoma dostaneme $S_{CDEF}=S_{AGBF}$.
Tým je tvrdenie z~úvodnej vety riešenia dokázané.

\poznamka
Z dôvodu symetrie sme namiesto dvojice
trojuholníkov $DFC$ a $AFG$ mohli uvažovať dvojicu trojuholníkov $DFE$ a $BFG$.
Takéto obmeny ďalších riešení už spomínať nebudeme
(ani v~záverečných pokynoch na bodovanie).
Dohodneme sa tiež, že
zjavné dôsledky súmerností päťuholníka $ABCDE$
budeme ďalej uvádzať bez odkazov.

\ineriesenie
Zadané štvoruholníky $CDEF$, $AGBF$ doplníme tým istým trojuholníkom $BCF$
postupne na štvoruholník $BCDE$ a trojuholník $AGC$ (\obr).
Namiesto rovnosti $S_{CDEF}=S_{AGBF}$ ďalej dokážeme
ekvivalentnú rovnosť $S_{BCDE}=S_{AGC}$. Za tým účelom označíme
$a$ dĺžku strán daného päťuholníka, $u$ dĺžku jeho
uhlopriečok a~$v$~vzdialenosť medzi ľubovoľnou stranou
a s~ňou rovnobežnou uhlopriečkou.
\insp{c72ii.2}%

Pre obsah lichobežníka $BCDE$ platí vzorec
$S_{BCDE}=\frac12(a+u)v$. Vďaka rovnobežníku $AGBD$ má trojuholník $AGC$
stranu $GC$ dĺžky $|GB|+|BC|=u+a$ a výška na túto stranu má veľkosť
$v$, teda jeho obsah je tiež $\frac12(a+u)v$.
Tým je rovnosť $S_{BCDE}=S_{AGC}$ dokázaná.

\ineriesenie
Keďže oba štvoruholníky $CDEF$ a $AGBF$ majú navzájom
kolmé uhlopriečky, pre ich obsahy platia vzorce
$$
S_{CDEF}=\frac12\cdot|EC|\cdot |FD|\qquad\hbox{a}\qquad
S_{AGBF}=\frac12\cdot|AB|\cdot |GF|.
$$
Tieto obsahy tak budú rovnaké, ak bude platiť
$|EC|\cdot |FD|=|AB|\cdot |GF|$, čiže
$$
|EC|:|AB|=|GF|:|FD|.
\tag1
$$
Na dôkaz \thetag1 využijeme priečku $BF$ trojuholníka $GCD$, ktorá je rovnobežná
s~jeho stranou~$CD$ (\obr). Preto bod $F$ delí stranu $GD$ v rovnakom
pomere, ako bod~$B$ delí stranu~$GC$, \tj. platí
$|GF|:|FD|=|GB|:|BC|$. Vzťah \thetag1 teda môžeme prepísať
v~tvare $|EC|:|AB|=|GB|:|BC|$. Posledné naozaj platí, pretože
$|EC|=|AD|=|GB|$ a~$|AB|=|BC|$.
\insp{c72ii.3}%


\ineriesenie
Pri poslednom postupe dokážeme rovnosť $S_{EFC}=S_{GBF}$, z ktorej
takisto zrejme vyplýva želaný záver o rovnosti obsahov štvoruholníkov
$DCEF$ a $AGBF$ (\obrr1).

Podľa riešenia úlohy D3 k príkladu 5 domáceho kola vieme, že
v~pravidelnom päťuholníku pre dĺžku strán $a$ a dĺžku
uhlopriečok $u$ platí $u:a=a:(u-a)$ (tzv. zlatý rez), čo ďalej
využijeme ako rovnosť $a^2=(u-a)u$. Tú môžeme prepísať v tvare
$|FC|\cdot|FE|=|BF|\cdot|BG|$, pretože z~kosoštvorcov
$CDEF$ a $AGBD$ vyplýva $|FC|=|FE|=a$ a $|BG|=u$, odkiaľ
$|BF|=|BE|-|FE|= u-a$. Podľa známeho vzorca\fnote{Obsah
každého trojuholníka je rovný jednej polovici súčinu dĺžok dvoch jeho strán
so sínusom uhla, ktorý tieto dve strany zvierajú.} tak bude sľúbená
rovnosť $S_{EFC}=S_{GBF}$ dokázaná, ak overíme zhodnosť uhlov
$EFC$ a $GBF$. To sú ale vonkajšie uhly pri vrcholoch $F$ a $B$
trojuholníka $FBC$ so stranami $FC$ a $BC$ rovnakej dĺžky $a$. Preto sú
uhly $EFC$ a $GBF$ naozaj zhodné.

\schemaABC
Za úplné riešenie dajte 6 bodov. Zjavné dôsledky osových
súmerností pravidelného päťuholníka nie je nutné dokazovať (ani
spomínať, že sú známe z~domáceho kola); to isté sa týka
aj pravidla o~lichobežníku z~prvého riešenia.

Pri postupe z~prvého riešenia dajte 1 bod za prechod od porovnania
obsahov $CDEF$ a $AGBF$ k~porovnaniu ich polovíc, pri tomto
postupe obsahov trojuholníkov $CDF$ a $AGF$. Podľa úplnosti dôkazu ich rovnosti
dajte 0 -- 5 bodov, z~toho 1 bod za východiskové pozorovanie,
že $DA\parallel GC$.

Pri postupe z~druhého riešenia dajte 2 body za prechod
od porovnania obsahov $CDEF$ a $AGBF$ k~ekvivalentnému porovnaniu
obsahov štvoruholníka $BCDE$ a trojuholníka $AGC$.
Podľa úplnosti dôkazu ich rovnosti dajte 0 -- 4 body,
z~toho 1 bod za konštatovanie, že $BCDE$ je lichobežník a~2~body za rovnosť $|GB|=u$.

Pri postupe z~tretieho riešenia dajte 1 bod za vyjadrenie obsahov oboch
štvoruholníkov použitím súčinu dĺžok ich uhlopriečok a 0 -- 5 bodov
za úplnosť dôkazu ich rovnosti, z toho 2 body za záver
spojený s~priečkou~$BF$ trojuholníka $GCD$ a 2 body za zhodnosť úsečky
$GB$ s uhlopriečkami päťuholníka.

Pri postupe zo štvrtého riešenia dajte 1 bod za prechod od porovnania
obsahov $CDEF$ a $AGBF$ k~porovnaniu ich polovíc, pri tomto
postupe obsahov trojuholníkov $EFC$ a $GBF$. Za úplnosť dôkazu
ich rovnosti dajte
0 -- 5 bodov, z~toho 3 body za prepis rovnosti $u:a=a:(u-a)$ z textu
domáceho kola na tvar $|FC|\cdot|FE|=|BF|\cdot|BG|$,
1 bod za zhodnosť uhlov $EFC$, $GBF$ a 1 bod za dvojaké použitie
vzorca pre obsah všeobecného trojuholníka z poznámky pod čiarou.

Čiastkové bodové zisky z rôznych postupov nemožno sčítať,
celkovo dajte maximálny z nich. Len za uhádnutie
odpovede žiadny bod neudeľujte.
\endschema
}

{%%%%%   C-II-4
Ak je jedno z~prvočísel $q$, $r$, $s$ rovné dvom,
je pravá strana danej rovnosti párna, a teda podľa ľavej strany je
aj prvočíslo $p$ párne, teda $p=2$,
čo je vylúčené kvôli podmienke rôznosti jednotlivých prvočísel.
Podobne žiadne z~prvočísel $q$, $r$, $s$ nie je rovné trom,
lebo súčet na ľavej strane by potom bol násobkom troch,
a teda aj prvočíslo $p$ by muselo byť rovné trom.

Vieme teda, že všetky tri prvočísla $q$, $r$, $s$ sú väčšie ako
3. Tri najmenšie takéto prvočísla sú 5, 7 a 11, takže nutne platí
$$
q\cdot r\cdot s\geq 5\cdot 7\cdot 11=385,
$$
odkiaľ vyplýva
$$
p=q\cdot r\cdot s-72\ge 385-72=313.
$$
Ľahko ukážeme, že číslo $313$ je prvočíslo,
a že je to teda hľadaná najmenšia možná hodnota~$p$ (na pravej
strane zadanej rovnosti sú potom menšie prvočísla 5, 7 a 11).

Keby číslo 313 nebolo prvočíslo, bolo by vďaka odhadu $18^2=324>313$
deliteľné niektorým z najmenších prvočísel od 2 do 17.
Deliteľnosť prvočíslami 2, 3 a 5 je však vylúčená podľa známych kritérií
a pre zvyšné prvočísla 7, 11, 13 a 17 ľahko overíme, že
príslušné delenia\fnote{Pri overovaní si
možno ušetriť prácu úvahou:
Keďže $72=2^3\cdot3^2$, z rovnosti $72+313=5\cdot7\cdot11$ vyplýva, že
číslo 313 nie je deliteľné dvoma, tromi, piatimi, siedmimi ani jedenástimi.
Stačí preto vydeliť číslo~313 prvočíslami 13 a 17.}
vyjdú so zvyškami:
$$
313=7\cdot44+5=
11\cdot28+5=13\cdot24+1=17\cdot18+7.
$$

\schemaABC
Za úplné riešenie dajte 6 bodov, z toho po 1 bode za každé z
oboch vysvetlení, že žiadne z prvočísel $q$, $r$, $s$ nie je 2 ani 3,
ďalšie 3 body za nájdenie odhadu $p\geq 313$ a konečne 1 bod za
overenie, že 313~je prvočíslo (riešiteľ pritom môže konštatovať, že vylúčil deliteľnosť
všetkými prvočíslami od 2 do 17, čo nie je nutné zaznamenávať).

Pokiaľ riešiteľ hodnotu $p=313$ určí dosadením prvočísel
5, 7 a 11 do pravej strany rovnosti a~možnosť $p<313$ nevylúči,
dajte iba 1 bod.
\endschema
}

{%%%%%   vyberko, den 1, priklad 1
...}

{%%%%%   vyberko, den 1, priklad 2
...}

{%%%%%   vyberko, den 1, priklad 3
...}

{%%%%%   vyberko, den 1, priklad 4
...}

{%%%%%   vyberko, den 2, priklad 1
...}

{%%%%%   vyberko, den 2, priklad 2
...}

{%%%%%   vyberko, den 2, priklad 3
...}

{%%%%%   vyberko, den 2, priklad 4
...}

{%%%%%   vyberko, den 3, priklad 1
...}

{%%%%%   vyberko, den 3, priklad 2
...}

{%%%%%   vyberko, den 3, priklad 3
...}

{%%%%%   vyberko, den 3, priklad 4
...}

{%%%%%   vyberko, den 4, priklad 1
...}

{%%%%%   vyberko, den 4, priklad 2
...}

{%%%%%   vyberko, den 4, priklad 3
...}

{%%%%%   vyberko, den 4, priklad 4
...}

{%%%%%   vyberko, den 5, priklad 1
...}

{%%%%%   vyberko, den 5, priklad 2
...}

{%%%%%   vyberko, den 5, priklad 3
...}

{%%%%%   vyberko, den 5, priklad 4
...}

{%%%%%   trojstretnutie, priklad 1
...}

{%%%%%   trojstretnutie, priklad 2
...}

{%%%%%   trojstretnutie, priklad 3
...}

{%%%%%   trojstretnutie, priklad 4
...}

{%%%%%   trojstretnutie, priklad 5
...}

{%%%%%   trojstretnutie, priklad 6
...}

{%%%%%   IMO, priklad 1
...}

{%%%%%   IMO, priklad 2
...}

{%%%%%   IMO, priklad 3
...}

{%%%%%   IMO, priklad 4
...}

{%%%%%   IMO, priklad 5
...}

{%%%%%   IMO, priklad 6
...}

{%%%%%   MEMO, priklad 1
...}

{%%%%%   MEMO, priklad 2
...}

{%%%%%   MEMO, priklad 3
...}

{%%%%%   MEMO, priklad 4
...}

{%%%%%   MEMO, priklad t1
...}

{%%%%%   MEMO, priklad t2
...}

{%%%%%   MEMO, priklad t3
...}

{%%%%%   MEMO, priklad t4
...}

{%%%%%   MEMO, priklad t5
...}

{%%%%%   MEMO, priklad t6
...}

{%%%%%   MEMO, priklad t7
...}

{%%%%%   MEMO, priklad t8
...} 