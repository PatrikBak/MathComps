{%%%%% A-I-1
V~obore reálnych čísel riešte sústavu rovníc
$$\eqalign{ 2x+\lfloor{y}\rfloor &= 2022,\cr
3y+\lfloor{2x}\rfloor &= 2023.}
$$
(Symbol $\lfloor{a}\rfloor$ označuje \emph{dolnú celú časť} reálneho čísla $a$, \tj.
najväčšie celé číslo, ktoré nie je väčšie ako~$a$.
Napr. $\lfloor{1{,}9}\rfloor=1$ a~$\lfloor{-1{,}1}\rfloor=-2$.)
}
\podpis{Jaroslav Švrček}

{%%%%% A-I-2
Daný je ostrouhlý trojuholník $ABC$.
Na polpriamkach opačných k~$CA$ a~$BA$ ležia postupne body $B'$ a~$C'$ tak, že
$|B'C|=|AB|$ a~$|C'B|=|AC|$.
Dokážte, že stred kružnice opísanej trojuholníku $AB'C'$ leží na kružnici opísanej trojuholníku $ABC$.
}
\podpis{Patrik Bak}

{%%%%% A-I-3
Pre dané kladné celé číslo $n$ uvažujme obdĺžnikový hrací plán $2n\times 2$
a~na ňom $2n$~žetónov očíslovaných $1,2,\cdots,2n$ a rozmiestnených ako na obrázku vľavo.
V~jednom ťahu je možné posunúť jeden žetón z jeho políčka na políčko susediace stranou, pokiaľ je prázdne.
Koľkými najmenej ťahmi možno z~pôvodného rozostavenia získať rozostavenie na obrázku vpravo?
\insp{72-a-i-3}
}
\podpis{Josef Tkadlec}

{%%%%% A-I-4
Sú dané dve nepárne prirodzené čísla $k$ a~$n$.
Martin pre každé dve prirodzené čísla $i$, $j$ spĺňajúce $1\le i\leq k$ a~$1\le j \leq n$ napísal na tabuľu zlomok~$i/j$.
Určte medián všetkých týchto zlomkov, teda také reálne číslo~$q$, že
ak všetky zlomky na tabuli zoradíme podľa hodnoty od najmenšej po najväčšiu
(zlomky s rovnakou hodnotou v~ľubovoľnom poradí),
uprostred tohto zoznamu bude zlomok s~hodnotou~$q$.
}
\podpis{Martin Melicher}

{%%%%% A-I-5
Daný je ostrouhlý rôznostranný trojuholník $ABC$.
Os vnútorného uhla pri vrchole~$A$ a~osi strán $AB$, $AC$ vymedzujú trojuholník.
Dokážte, že priesečník jeho výšok leží na ťažnici z~vrcholu~$A$.}
\podpis{Josef Tkadlec}

{%%%%% A-I-6
Uvažujme postupnosť $(a_n)_{n=1}^\infty$ definovanú nasledovne:
$$
a_1=3 \qquad\hbox{a}\qquad a_n=a_1a_2a_3\cdots a_{n-1}-1\quad\text{pre všetky $n\ge 2$}.
$$
Dokážte, že existuje
\itemitem{a)} nekonečne veľa prvočísel deliacich aspoň jeden člen tejto postupnosti;
\itemitem{b)} nekonečne veľa prvočísel nedeliacich žiadny člen tejto postupnosti.
}
\podpis{Martin Melicher}

{%%%%% B-I-1
Na tabuľu napíšeme desať navzájom rôznych kladných celých čísel.
V každom kroku najskôr podčiarkneme každé číslo, ktoré nie je súčtom žiadnych dvoch rôznych čísel napísaných na tabuli, potom všetky podčiarknuté čísla zotrieme. Príklad je na \obr{}.
\inspdf{b72i_0.pdf}%
\itemitem{a)} Dokážte, že pre ľubovoľných desať napísaných čísel zostane po konečnom počte krokov tabuľa prázdna.
\itemitem{b)} Určte najväčší počet krokov, po ktorých prevedení ešte nemusí zostať tabuľa prázdna. Uveďte príklad desiatich čísel, pre ktoré tento počet dosiahneme.\endgraf\noindent}
\podpis{Patrik Bak}

{%%%%% B-I-2
Označme $M$ počet všetkých možných vyplnení tabuľky $3\times 3$ navzájom rôznymi prirodzenými číslami od $1$ do $9$.
Ďalej označme $N$ počet tých vyplnení, kde sú navyše súčty všetkých čísel v~každom riadku aj stĺpci nepárne čísla. Určte pomer $N:M$.
}
\podpis{Jaromír Šimša}

{%%%%% B-I-3
Určte všetky dvojice $(a,b)$ reálnych čísel, pre ktoré majú
kvadratické trojčleny $P(x)=x^2+ax+b$ a~$Q(x)=x^2+bx+a$ nasledujúcu vlastnosť:
každá z rovníc
$$aP(x)+bQ(x)=0 \qquad \hbox{a} \qquad aQ(x)+bP(x)=0$$
je kvadratickou rovnicou s~dvojnásobným koreňom.
}
\podpis{Jaroslav Švrček}

{%%%%% B-I-4
V~konvexnom päťuholníku~$ABCDE$ platí $BC \parallel DE$, $CD \parallel AE$, $|\angle BAD|=|\angle DAE|$ a~$|\angle CBD|=|\angle DBA|$. Dokážte, že $|CD|=|DE|$.}
\podpis{Patrik Bak}

{%%%%% B-I-5
Skúmajme trojice $(a,b,c)$ kladných celých čísel spĺňajúcich podmienku $ab=c^2$.
\itemitem{a)} Pre každé prvočíslo $p$ uveďte príklad trojice $(a,b,c)$, pre ktorú platí $a+b-2c=p$.
\itemitem{b)} Dokážte, že pre každú trojicu $(a,b,c)$ je $a+b+2c$ zložené číslo.
}
\podpis{Josef Tkadlec}

{%%%%% B-I-6
Daný je trojuholník $ABC$ s~pravým uhlom pri vrchole $B$. Označme $I$ stred kružnice jemu vpísanej, $M$ stred prepony~$AC$ a~$X$ priesečník priamky $IM$ s~priamkou $BC$. Dokážte, že ak ležia body $B$, $I$, $M$, $C$ na jednej kružnici, je trojuholník $ABX$ rovnoramenný.
}
\podpis{David Hruška}

{%%%%% C-I-1
Uvažujme 2022 zlomkov
$$\frac{0}{2022},\frac{1}{2021},\frac{2}{2020},\cdots,\frac{2021}{1}$$
v tvare podielu dvoch celých nezáporných čísel, ktorých súčet je pre každý zlomok rovný 2022. Koľko z~nich nadobúda celočíselné hodnoty?
}
\podpis{Jaroslav Zhouf}

{%%%%% C-I-2
Šebestová má z~päťminútoviek priemer známok presne $1{,}12$. Dokážte, že z~nich má aspoň 22 jednotiek.
(Možné známky sú $1$, $2$, $3$, $4$, $5$.)}
\podpis{Josef Tkadlec}

{%%%%% C-I-3
V trojuholníku $ABC$ označme $M$ stred strany $AB$, $N$ stred strany $AC$ a $P$ stred úsečky~$MN$. Dokážte, že ak $|MN|=|AP|$, tak $BP \perp CP$.}
\podpis{Patrik Bak, Eliška Macáková}

{%%%%% C-I-4
Mach hrá nasledujúcu hru. Na začiatku je na stole $k$ kôpok, na ktorých je postupne $1, 2, 3, \cdots, k$ žetónov. V~každom ťahu vyberie ľubovoľné dve kôpky a~odstráni z~oboch rovnaký počet žetónov. Jeho cieľom je, aby na stole zostal jediný žetón. Môže sa mu to podariť a)~pre $k=10$, b)~pre $k=11$?}
\podpis{Radek Horenský}

{%%%%% C-I-5
Nech $ABCDE$ je pravidelný päťuholník. Priesečník uhlopriečky $AC$ s~osou strany $AB$ označme $F$ (\obr). Dokážte, že trojuholníky $ABC$ a~$CDF$ majú rovnaký obsah.
\inspdf{c72i_0.pdf}%
}
\podpis{David Hruška}

{%%%%% C-I-6
Určte najväčšie prirodzené číslo $n\ge 10$ také, že pre ľubovoľných $10$ rôznych čísel z~množiny $\{1,2,\cdots, n\}$ platí nasledujúce tvrdenie: Ak nie je ani jedno z~týchto $10$ čísel prvočíslom, tak je súčet niektorých dvoch z~nich prvočíslom.}
\podpis{Ján Mazák}

{%%%%% A-S-1
V obore nezáporných reálnych čísel riešte sústavu rovníc
$$\eqalign{
\lfloor 3x+5y+7z\rfloor&=7z,\cr
\lfloor 3y+5z+7x\rfloor&=7x,\cr
\lfloor 3z+5x+7y\rfloor&=7y.}
$$}
\podpis{Tomáš Bárta}

{%%%%% A-S-2
V~konvexnom päťuholníku $ABCDE$ platí $|\angle CBA| = |\angle BAE| = |\angle AED|$. Na~stranách~$AB$ a~$AE$ existujú postupne body~$P$ a~$Q$ tak, že $|AP| = |BC| = |QE|$ a~$|AQ|=|BP|=|DE|$. Dokážte, že $CD \parallel PQ$.}
\podpis{Patrik Bak}

{%%%%% A-S-3
Dokážte tvrdenie: Ak vyberieme ľubovoľné štyri delitele čísla $720$, tak jeden z nich je deliteľom súčinu zvyšných troch.}
\podpis{Jaromír Šimša}

{%%%%% A-II-1
Na hracom pláne $8\times 5$ je rozmiestnených 8~bielych a~8~čiernych žetónov
ako na \obr{} vľavo. V~jednom ťahu je
možné posunúť žetón na prázdne políčko susediace stranou. Určte najmenší počet ťahov, ktorými možno z~pôvodného rozostavenia získať rozostavenie na
obrázku vpravo.
\insp{72-a-ii-1}
}
\podpis{Josef Tkadlec}

{%%%%% A-II-2
V~obore reálnych čísel riešte sústavu rovníc
$$\eqalign{\sqrt{\sqrt{x}+2}&=y-2,\cr
\sqrt{\sqrt{y}+2}&=x-2.}$$}
\podpis{Radek Horenský}

{%%%%% A-II-3
V~konvexnom štvoruholníku $ABCD$ platí $|AB|=|BC|=|CD|$.
Nech navyše pre priesečník $P$ jeho uhlopriečok platí $|\uhol APD|<90^{\circ}$.
Označme $R$ a~$S$ postupne obrazy bodov~$A$ a~$D$ v~osových súmernostiach podľa priamok $BD$ a~$AC$.
Dokážte, že úsečky $BC$ a~$RS$ sú rovnobežné.
}
\podpis{Patrik Bak}

{%%%%% A-II-4
Nájdite všetky trojice prirodzených čísel $a$, $b$, $c$, pre ktoré je súčin
$$(a+b)(b+c)(c+a)(a+b+c+2036)$$ rovný mocnine niektorého prvočísla s~celočíselným exponentom.}
\podpis{Ján Mazák}

{%%%%% A-III-1
Alica a~Bohuš hrajú hru na pláne so 72 políčkami rozmiestnenými po obvode kruhu.
Na začiatku Bohuš položí na niektoré políčka po jednom žetóne.
V~každom kole najskôr Alica zvolí jedno prázdne políčko
a~Bohuš potom naň musí posunúť žetón z~jedného susedného políčka.
Ak to nedokáže, hra končí; inak nasleduje ďalšie kolo. Určte najmenší počet žetónov, pre ktorý Bohuš vie zabezpečiť, že v hre prebehne aspoň 2023~kôl.}
\podpis{Václav Blažej}

{%%%%% A-III-2
Nech $n\ge3$ je celé číslo a $a_1, a_2, \cdots, a_n$ sú dĺžky strán ľubovoľného $n$-uholníka. Dokážte nerovnosť
$$a_1 + a_2 + \cdots + a_n > \sqrt{2\,(a_1^2+a_2^2+\cdots+a_n^2)}.$$}
\podpis{Jaroslav Švrček}

{%%%%% A-III-3
V~ostrouhlom trojuholníku $ABC$ označme~$H$ priesečník jeho výšok a~$I$ stred kružnice jemu vpísanej. Nech~$D$ je kolmým priemetom bodu~$I$ na priamku~$BC$ a~$E$ je obrazom bodu~$A$ v~súmernosti so stredom~$I$. Ďalej je $F$ kolmým priemetom bodu~$H$ na priamku~$ED$. Dokážte, že body $B$, $H$, $F$ a~$C$ ležia na jednej kružnici.}
\podpis{Patrik Bak}

{%%%%% A-III-4
Uvažujme postupnosť $(a_n)_{n=1}^\infty$ kladných celých čísel spĺňajúcu pre každý index $n\ge 3$ podmienku
$$
a_n=a_1a_2+a_2a_3+\cdots + a_{n-2}a_{n-1} - 1.
$$
\ite a) Dokážte, že niektoré prvočíslo je deliteľom nekonečne veľa členov tejto postupnosti.
\ite b) Dokážte, že takých prvočísel je nekonečne veľa.
}
\podpis{Tomáš Bárta}

{%%%%% A-III-5
V~trojuholníku $ABC$ označme $M$, $N$, $P$ postupne stredy strán $BC$, $CA$, $AB$ a~$G$ jeho ťažisko. Nech kružnica opísaná trojuholníku $BGP$ pretína priamku $MP$ v~bode~$K$ rôznom od $P$ a~kružnica opísaná trojuholníku $CGN$ pretína priamku $MN$ v~bode~$L$ rôznom od $N$. Dokážte, že $|\angle BAK| = |\angle CAL|$.}
\podpis{Josef Tkadlec}

{%%%%% A-III-6
Nech $n \geq 3$ je celé číslo. Uvažujme štvorčekový papier s~rozmermi $n \times n$, ktorého jednotlivé štvorčeky môžu mať buď bielu, alebo čiernu farbu. V každom kroku zmeníme farby piatich štvorčekov, ktoré tvoria útvar
\insp{72-a-iii-6}
v~ľubovoľnom natočení. Na začiatku sú všetky štvorčeky biele. Rozhodnite, pre ktoré $n$ možno po konečnom počte krokov dosiahnuť to, že všetky štvorčeky budú čierne.}
\podpis{Jaroslav Zhouf}

{%%%%% B-S-1
Označme $M$ počet všetkých možných vyplnení tabuľky $3 \times 3$
navzájom rôznymi prirodzenými číslami od~$1$ do~$9$.
Ďalej označme $D$ počet tých vyplnení, keď je navyše \emph{súčin} čísel v~niektorom riadku alebo stĺpci násobkom desiatich.
Určte pomer $D : M$.}
\podpis{Josef Tkadlec}

{%%%%% B-S-2
Daný je štvorec $ABCD$. Na polpriamke opačnej k~$CB$ leží bod $E$ tak, že $|BC|=|CE|$. Označme~$F$ stred strany $BC$ a~$X$ kolmý priemet bodu $E$ na priamku $AF$. Dokážte, že bod~$C$ je stredom kružnice vpísanej trojuholníku $AXE$.}
\podpis{David Hruška}

{%%%%% B-S-3
Nech $a$, $b$ sú kladné celé čísla také, že $a^2-b^2$ je mocninou dvojky. Dokážte, že $a^2+b^2$ je súčtom dvoch mocnín dvojky. (Mocninami dvojky rozumieme čísla $2^0$, $2^1$, $2^2\cdots$)}
\podpis{Zdeno Pezlar, Michal Pecho}

{%%%%% B-II-1
Nech $a$ a $b$ ($a>b$) sú prirodzené čísla, ktorých súčet je osemnásobkom ich najväčšieho spoločného deliteľa. Dokážte, že číslo $a^2-b^2$ alebo jeho trojnásobok je druhou mocninou celého čísla.}
\podpis{Zdeněk Pezlar, Michal Pecho}

{%%%%% B-II-2
Predpokladajme, že reálne čísla $x$, $y$, $z$ spĺňajú nerovnosť
$$
(x+y+z)^2>2\,(x^2+y^2+z^2).
$$
Dokážte, že čísla $x$, $y$, $z$ sú buď všetky kladné, alebo všetky
záporné.}
\podpis{Jaromír Šimša}

{%%%%% B-II-3
Na stranách $AB$ a~$BC$ daného trojuholníka $ABC$ ležia postupne také body~$D$ a~$E$, že $|BD|=|DC|=|CA|$ a~$|EC|=|ED|$. Dokážte, že $|AE|=|BE|$.}
\podpis{Patrik Bak}

{%%%%% B-II-4
Koľko $33$-ciferných čísel deliteľných~$3$ neobsahuje vo svojom zápise cifru~3? Výsledok zapíšte v tvare súčinu mocnín prvočísel.}
\podpis{Eliška Macáková, Patrik Bak}

{%%%%% C-S-1
Uvažujme $72$ zlomkov
$$ \frac{0\cdot 0}{72},\ \frac{1\cdot 1}{71},\ \frac{2\cdot 2}{70},\ \frac{3\cdot 3}{69}, \cdots,\ \frac{71\cdot 71}{1}.
$$
Koľko z~nich nadobúda celočíselnú hodnotu?}
\podpis{Josef Tkadlec}

{%%%%% C-S-2
V danom pravouhlom trojuholníku $ABC$ označme $K$ stred prepony $AB$ a $L$ stred kratšej odvesny $AC$. Kružnica s priemerom $BC$ pretína úsečku $KL$ v bode $P$. Dokážte, že uhly $PAC$ a $PBC$ sú zhodné.}
\podpis{Jaromír Šimša}

{%%%%% C-S-3
Na tabuli boli napísané čísla $1$, $2$, $3$, $4$, $5$, $6$, $7$, $8$, $9$. V~každom kroku sme dve čísla zotreli a~nahradili druhou mocninou ich rozdielu. Ak po nanajvýš 7 krokoch zostali na tabuli všetky čísla rovnaké, mohli to byť čísla
a)~nepárne, b)~párne?}
\podpis{Radek Horenský}

{%%%%% C-II-1
Mach si známkou z~poslednej päťminútovky zlepšil celkový priemer z~$2{,}6$ na $2{,}5$. Koľko päťminútoviek celkom písal? Určte všetky možnosti. (Možné známky sú $1$, $2$, $3$, $4$,~$5$.)}
\podpis{Josef Tkadlec}

{%%%%% C-II-2
Pre reálne čísla $a$, $b$, $c$ platí $a+b+c=1$. Dokážte, že súčin
$$
(a+bc)(b+ca)(c+ab)
$$
je nezáporný. Určte tiež všetky trojice $(a,b,c)$, pre ktoré je daný súčin rovný nule.}
\podpis{Patrik Bak}

{%%%%% C-II-3
V~pravidelnom päťuholníku $ABCDE$ označme~$F$ priesečník uhlopriečok~$AC$, $BE$ a $G$~priesečník predĺžených strán~$EA$, $CB$. Ktorý zo štvoruholníkov~$CDEF$ a~$AFBG$ má väčší obsah?}
\podpis{Mária Dományová}

{%%%%% C-II-4
Štyri navzájom rôzne prvočísla $p$, $q$, $r$, $s$ spĺňajú rovnosť
$$72+p=q\cdot r\cdot s. $$
Určte najmenšiu možnú hodnotu $p$.
}
\podpis{Josef Tkadlec}

{%%%%%   vyberko, den 1, priklad 1
Dané je celé číslo $k\ge 2$. Nájdite najmenšie celé číslo $n\ge k+1$, pre ktoré existuje $n$-prvková množina $S$ rôznych reálnych čísel s nasledujúcou vlastnosťou: Každý prvok $a$ množiny $S$ sa dá vyjadriť ako súčet $k$ rôznych prvkov množiny $S \setminus \{a\}$.
}
\podpis{IMO shortlist 2022,A2}

{%%%%%   vyberko, den 1, priklad 2
V rade vedľa seba je umiestnených $2022$ vedier s vodou. Každé z nich je ofarbené buď červenou, alebo modrou farbou. Losos Sally sa hrá hru nasledovným spôsobom:
\smallskip
Najprv si pozrie, ako sú ofarbené vedrá, a vyberie si vedro, v ktorom chce začať. Potom môže, koľkokrát chce, preskočiť buď do ďalšieho vedra v rade, alebo do vedra za týmto vedrom (t.j. stále skáče tým istým smerom a nemôže preskočiť ponad viac ako jedno vedro). V ľubovoľnom momente sa môže rozhodnúť ukončiť hru.
\smallskip
Keď ukončí hru, tak spočíta svoje skóre. To je dané ako absolútna hodnota rozdielu medzi počtom červených a počtom modrých vedier,  ktoré Sally navštívila počas hry.
\smallskip
Nájdite najväčšie $C$ také, že bez ohľadu na počiatočné ofarbenie vedier môže Sally vždy získať skóre aspoň $C$.}
\podpis{IMO shortlist 2022,C1}

{%%%%%   vyberko, den 1, priklad 3
Uhlopriečky rovnobežníka $ABCD$ sa pretínajú v bode $O$. Dotyčnice ku kružnici opísanej trojuholníku $AOD$ vedené bodmi $A$ a $D$ pretnú polpriamky opačné k polpriamkam $BC$ a $CB$ postupne v bodoch $P$ a $Q$. Dokážte, že tieto dotyčnice sa dotýkajú aj kružnice opísanej trojuholníku $POQ$.}
\podpis{Švédsko 2022}

{%%%%%   vyberko, den 1, priklad 4
Nech $d(k)$ označuje počet kladných deliteľov celého čísla $k$. Napríklad $d(6)=4$, lebo 6 má štyroch kladných deliteľov, konkrétne 1, 2, 3 a 6. Dokážte, že pre všetky kladné celé čísla $n$ platí
$$d(1)+d(3)+d(5)+\cdots+d(2n-1)\le d(2)+d(4)+d(6)+\cdots+d(2n).$$}
\podpis{}

{%%%%%   vyberko, den 2, priklad 1
Kružnica prechádzajúca vrcholmi $A$, $B$ ostrouhlého rôznostranného trojuholníka $ABC$ sa dotýka osi strany $BC$ v bode $K$ a pretína stranu $BC$ druhýkrát v bode $L\ne B$. Označme~$M$ stred úsečky $AC$. Dokážte, že $KM\perp AL$.}
\podpis{Mongolsko 2022}

{%%%%%   vyberko, den 2, priklad 2
Nech $a>1$ a $d>1$ sú nesúdeliteľné celé čísla. Nech $x_1=1$ a pre $k\ge 1$ definujeme
$$
x_{k+1}=
\begin{cases}
        x_k+d & \text{ak }a\not\mid x_k\cr
        x_k/a & \text{ak }a\mid x_k.\cr
\end{cases}
$$
V závislosti od celých čísel $a$ a $d$ určite najväčšie kladné celé číslo $n$, pre ktoré existuje index $k$ taký, že $a^n\mid x_k$. }
\podpis{IMO shortlist 2022,N3}

{%%%%%   vyberko, den 2, priklad 3
Označme ${\Cal F}$ množinu všetkých funkcií $f\colon\Bbb R\rightarrow\Bbb R$, ktoré spĺňajú $$f(x+f(y))=f(x)+f(y)$$ pre všetky $x,y\in\Bbb R$. Nájdite všetky racionálne čísla $q$ také, že pre každú funkciu $f\in{\Cal F}$ existuje nejaké $z\in\Bbb R$ také, že $f(z)=qz$.}
\podpis{IMO shortlist 2022, 6}

{%%%%%   vyberko, den 3, priklad 1
Nech $\Bbb Z^+$ označuje množinu všetkých kladných celých čísel. Nájdite všetky funkcie $f: \Bbb Z^+\rightarrow \Bbb Z^+$, ktoré spĺňajú $$f(a)+f(b)\mid (a+b)^2$$
pre všetky kladné celé čísla $a$, $b$.
}
\podpis{}

{%%%%%   vyberko, den 3, priklad 2
Dané je kladné celé číslo $n$. Máme $n$ kôpok kamienkov, na každej z nich je na začiatku jediný kamienok. Môžeme vykonávať ťahy nasledovného typu: vyberieme dve kôpky, z~každej vezmeme rovnaký počet kamienkov a z týchto kamienkov vytvoríme novú kôpku.
\smallskip
Pre každé kladné celé číslo $n$ nájdite najmenší možný počet neprázdnych kôpok, ktoré môžeme dosiahnuť konečnou postupnosťou krokov tohto typu.
}
\podpis{IMO shortlist 2022, C6}

{%%%%%   vyberko, den 3, priklad 3
V ostrouhlom trojuholníku $ABC$ splňajúcom $|AB|<|AC|$ označme $O$ stred kružnice opísanej a $D$ ľubovoľný bod na strane $BC$. Kolmica na $BC$ vedená bodom $D$ pretne úsečky $AO$, $AC$ a priamku $AB$ postupne v bodoch $W$, $X$, $Y$.
Kružnice opísané trojuholníkom $AXY$ a $ABC$ sa druhýkrát pretnú v bode $Z\ne A$.
Dokážte, že ak $|OW|=|OD|$, tak $DZ$ je dotyčnica kružnice opísanej trojuholníku $AXY$.}
\podpis{IMO shortlist 2022, G4}

{%%%%%   vyberko, den 4, priklad 1
Majme záhradu tvaru tabuľky $42 \times 42$. Na začiatku je v každom políčku strom výšky~$0$. V tejto záhrade sa {\it záhradník} a {\it drevorubač} hrajú nasledovnú hru, pri ktorej záhradník začína a potom sa striedajú v ťahoch:
\item{$\bullet$} Záhradník si vo svojom ťahu vyberie políčko záhrady. Strom na tomto políčku a všetky stromy na políčkach susediacich s týmto políčkom stranou alebo vrcholom (ktorých je najviac $8$) zväčšia svoju výšku o $1$.
\item{$\bullet$} Drevorubač si vo svojom ťahu vyberie ľubovoľné $4$ rôzne políčka. Všetky stromy na týchto $4$ políčkach, ktorých výška je väčšia ako $0$, zmenšia svoju výšku o $1$.
\endgraf\noindent
O strome povieme, že je {\it majestátny}, ak má výšku aspoň $10^6$. Nájdite najväčšie celé číslo $K$ také, že záhradník vie zabezpečiť, že v záhrade bude aspoň $K$ majestátnych stromov, bez ohľadu na to, ako hrá drevorubač.}
\podpis{IMO shortlist 2022, NC3}

{%%%%%   vyberko, den 4, priklad 2
Nech $\{a_n\}_{n = 1}^\infty$ je postupnosť kladných reálnych čísel. Predpokladajme, že existuje konštanta $M>0$ taká, že $a_1^2+a_2^2+\cdots+a_n^2<M a_{n+1}^2$ pre všetky $n\in \Bbb N$. Dokážte, že existuje konštanta $M'>0$ taká, že $a_1+a_2+\cdots+ a_n<M'a_{n+1}$ pre všetky $n\in \Bbb N$.}
\podpis{}

{%%%%%   vyberko, den 4, priklad 3
Pre každé $1\le i\le 9$ a každé kladné celé číslo $T$ definujeme $d_i(T)$ ako celkový počet výskytov cifry $i$ v cifernom zápise čísel $$2023,\quad 2\cdot 2023,\quad 3\cdot 2023,\quad\cdots,\quad T\cdot 2023$$ v desiatkovej sústave.

Dokážte, že existuje nekonečne veľa kladných celých čísel $T$ takých, že medzi číslami $d_1(T),d_2(T),\cdots,d_9(T)$ sú presne dve rôzne hodnoty.}
\podpis{IMO shortlist 2022, N5}

{%%%%%   vyberko, den 5, priklad 1
Nech $P$ je polynóm s celočíselnými koeficientami spĺňajúci $P(16)=36$, $P(14)=16$, $P(5)=25$. Určte všetky možné hodnoty $P(10)$}
\podpis{}

{%%%%%   vyberko, den 5, priklad 2
V ostrouhlom trojuholníku $ABC$ označme $F$ pätu výšky z vrchola $A$ a $P$ ľubovoľný bod na úsečke $AF$.
Rovnobežky vedené bodom $P$ so stranami $AC$, $AB$ pretnú stranu $BC$ postupne v bodoch $D$, $E$.
Body $X\ne A$ a $Y\ne A$ ležia postupne na kružniciach opísaných trojuholníkom $ABD$ a $ACE$ tak, že $|DA|=|DX|$ a $|EA|=|EY|$.
Dokážte, že body $B$, $C$, $X$, $Y$ ležia na jednej kružnici.}
\podpis{IMO shortlist 2022, G2}

{%%%%%   vyberko, den 5, priklad 3
Majo si na začiatku napíše na tabuľu $s$ celočíselných $2023$-tíc. Potom môže zobrať dve (nie nutne rôzne) $2023$-tice $\bm{v} = (v_{1}, \cdots, v_{2023})$ a $\bm{w} = (w_{1}, \cdots, w_{2023})$, ktoré sú už napísané na tabuli, a použiť jednu z nasledujúcich operácií, aby dostal novú $2023$-ticu
$$
\eqalign{
\bm{v} + \bm{w} &= (v_{1} + w_{1},\ \cdots,\ v_{2023} + w_{2023}),\cr
\bm{v} \vee \bm{w} &= (\max(v_{1}, w_{1}),\ \cdots,\ \max(v_{2023}, w_{2023}))\cr
}
$$
a napísal ju na tabuľu.

Ukázalo sa, že týmto spôsobom vie Majo napísať ľubovoľnú celočíselnú $2023$-ticu na tabuľu po konečne veľa krokoch. Nájdite najmenšiu možnú hodnotu $s$ počtu $2023$-tíc, ktoré Majo na začiatku napísal na tabuľu.}
\podpis{IMO shortlist 2022, C7}

{%%%%%   vyberko, den 5, priklad 1
...}
\podpis{...}

{%%%%%   vyberko, den 5, priklad 2
...}
\podpis{...}

{%%%%%   vyberko, den 5, priklad 3
...}
\podpis{...}

{%%%%%   vyberko, den 5, priklad 4
...}
\podpis{...}

{%%%%%   trojstretnutie, priklad 1
Dané je celé číslo $n \geq 3$. Nájdite najmenšie kladné celé číslo $k$ také, že ľubovoľné dva body v ľubovoľnom $n$-uholníku (alebo na jeho hranici) v rovine možno spojiť lomenou čiarou skladajúcou sa z práve $k$ úsečiek, ktoré sú celé obsiahnuté v danom $n$-uholníku (alebo na jeho hranici).}
\podpis{David Hruška, Česká rep.}

{%%%%%   trojstretnutie, priklad 2
Nech $a_1, a_2, \cdots, a_n$ sú také reálne čísla, že pre každé $k \in \{1, 2, \cdots, n\}$ platí nerovnosť
$$
n\cdot a_k\geq \sum_{i=1}^k a_i^2.
$$
Dokážte, že existuje aspoň $\frac{n}{10}$ indexov $k$ takých, že $a_k\leq 1000$.}
\podpis{Sándor Kisfaludi-Bak, Karol W\ę{}grzycki, Poľsko}

{%%%%%   trojstretnutie, priklad 3
Je daný konvexný štvoruholník $ABCD$, v ktorom platí $|\uhol BAD| = |\uhol BCD|$ a~$|\uhol ABC| < |\uhol ADC|$. Bod $M$ je stredom úsečky $AC$. Dokážte, že na úsečkách $AB$ a $BC$ existujú postupne body $X$ a $Y$ také, že $XY \perp BD$, $|MX| = |MY|$ a~$|\uhol XMY| = |\uhol ADC| - |\uhol ABC|$.}
\podpis{Mykhailo Shtandenko, Ukrajina}

{%%%%%   trojstretnutie, priklad 4
Nech $p$, $q$ a $r$ sú kladné reálne čísla také, že rovnica
$$
\lfloor pn\rfloor + \lfloor qn\rfloor + \lfloor rn\rfloor =n
$$
je splnená pre nekonečne veľa kladných celých čísel $n$.
\ite a) Dokážte, že $p$, $q$ a $r$ sú racionálne čísla.
\ite b) Nájdite počet kladných celých čísel $c$ takých, že existujú kladné celé čísla $a$,~$b$, pre ktoré je rovnica
     $$
     \left\lfloor \frac{n}{a}\right\rfloor+\left\lfloor \frac{n}{b}\right\rfloor +\left\lfloor\frac{cn}{202}\right\rfloor =n
     $$
     splnená pre nekonečne veľa kladných celých čísel $n$.\endgraf
}
\podpis{Walther Janous, Rakúsko}

{%%%%%   trojstretnutie, priklad 5
Daný je ostrouhlý trojuholník $ABC$ s ortocentrom $H$. Označme $D$ pätu výšky z bodu $A$ na stranu $BC$. Nech $T$ je taký bod na kružnici s priemerom $AH$, že táto kružnica má vnútorný dotyk s kružnicou opísanou trojuholníku $BDT$. Napokon označme $N$ stred úsečky $AH$. Dokážte, že $BT \perp CN$. }
\podpis{Michal Pecho, Slovensko}

{%%%%%   trojstretnutie, priklad 6
Dané je kladné celé číslo $n$. Majme tabuľku $n \times n$, ktorej všetky políčka sú na začiatku biele. Maliar Piet sa prechádza po tabuľke a prefarbuje navštívené políčka podľa nasledovných pravidiel. Piet začína každú \emph{prechádzku} v ľavom dolnom rohu tabuľky a pokračuje nasledovne:
\item{$\bullet$} ak stojí na bielom políčku, prefarbí ho načierno a pohne sa o jedno políčko nahor (alebo vyjde z~tabuľky, ak stojí na políčku v hornom riadku);
\item{$\bullet$} ak stojí na čiernom políčku, prefarbí ho nabielo a pohne sa o jedno políčko doprava (alebo vyjde z~tabuľky, ak stojí na políčku v stĺpci úplne napravo).

Pietova prechádzka sa skončí, keď vyjde z tabuľky. Nájdite najmenšie kladné celé číslo $s$ s nasledujúcou vlastnosťou: po \emph{presne} $s$ prechádzkach sú všetky políčka tabuľky opäť biele.

\smallskip\noindent
Napríklad pre $n = 3$ sú príslušné stavy tabuľky po prvých piatich prechádzkach ako na \obr.
\inspdf{cps-obr6.pdf}%
}
\podpis{\L{}ukasz Bożyk}

{%%%%%   IMO, priklad 1
Nájdite všetky zložené kladné celé čísla $n$ také,
že ak sú $d_1, d_2, \cdots, d_k$ všetky jeho kladné delitele
a~$1=d_1<d_2<\cdots<d_k=n$,
tak pre každé $i$ také,
že $1\le i\le k-2$,
platí,
že číslo $d_i$ je deliteľom čísla $d_{i+1}+d_{i+2}$.
}
\podpis{...}

{%%%%%   IMO, priklad 2
Nech $ABC$ je ostrouhlý trojuholník taký,
že $|AB|<|AC|$.
Označme $\varOmega$ jemu opísanú kružnicu
a $S$ stred jej oblúka s krajnými bodmi $C$ a $B$ obsahujúceho bod $A$.
Nech kolmica z~bodu $A$ na priamku $BC$ pretína úsečku $BS$ v~bode~$D$
a kružnicu $\varOmega$ v~bode~$E$ rôznom od $A$.
Nech rovnobežka s priamkou $BC$ cez bod $D$ pretína priamku $BE$ v~bode~$L$.
Označme $\omega$ kružnicu opísanú trojuholníku $BDL$.
Nech $P$ je priesečník kružníc $\omega$ a $\varOmega$ rôzny od $B$.
Dokážte,
že dotyčnica ku kružnici $\omega$ v bode $P$ pretína priamku $BS$ na osi uhla $BAC$.
}
\podpis{...}

{%%%%%   IMO, priklad 3
Nech $k$ je celé číslo také, že $k\ge2$. Určte všetky nekonečné postupnosti kladných celých čísel $(a_1,a_2,\cdots)$ také, že existuje polynóm $P$ taký, že
$$
P(x)=x^k+c_{k-1}x^{k-1}+\cdots+c_1x+c_0,
$$
kde $c_0, c_1, \cdots, c_{k-1}$ sú nezáporné celé čísla, a pre každé kladné celé číslo $n$ platí
$$
P(a_n)=a_{n+1}a_{n+2}\cdots a_{n+k}.
$$
}
\podpis{...}

{%%%%%   IMO, priklad 4
Nech $x_1, x_2, \cdots, x_{2023}$ sú navzájom rôzne kladné reálne čísla také, že pre každé $n$ z~množiny $\{1,2,\cdots,2023\}$ platí
$$
a_n=\sqrt{(x_1+x_2+\cdots+x_n)\left(\frac1{x_1}+\frac1{x_2}+\cdots+\frac1{x_n}\right)}
$$
a číslo $a_n$ je celé. Dokážte, že $a_{2023}\ge3034$.
}
\podpis{...}

{%%%%%   IMO, priklad 5
Nech $n$ je kladné celé číslo. Pod \emph{japonským trojuholníkom} rádu $n$ budeme rozumieť rozmiestnenie $1+2+\cdots+n$ dotykmi prepojených zhodných krúžkov do tvaru rovnostranného trojuholníka s $n$ riadkami tak, že pre každé $i$ z množiny $\{1,2,\cdots,n\}$ obsahuje $i$. riadok práve $i$ krúžkov,
z ktorých práve jeden je červený. V japonskom trojuholníku budeme pod \emph{nindža cestou} rozumieť postupnosť $n$ krúžkov, ktorá sa začína v najvrchnejšom krúžku, v každom kroku pokračuje jedným z dvoch krúžkov, ktoré sú hneď pod ním, a končí sa v najspodnejšom riadku. Na \obr{} je príklad japonského trojuholníka rádu $6$ a v ňom príklad nindža cesty obsahujúcej $2$ červené krúžky.
\inspdf{triangle-imo.pdf}%

Nájdite najväčšie $k$ také, že v každom japonskom trojuholníku rádu $n$ existuje nindža cesta obsahujúca aspoň $k$ červených krúžkov.
}
\podpis{...}

{%%%%%   IMO, priklad 6
Nech $ABC$ je rovnostranný trojuholník. Nech $A_1$, $B_1$, $C_1$ sú jeho vnútorné body také, že $|BA_1|=|A_1C|$, $|CB_1|=|B_1A|$, $|AC_1|=|C_1B|$
a
$$
|\uhol BA_1C| + |\uhol CB_1A| +|\uhol AC_1B| = 480\st.
$$
Nech sa priamky $BC_1$ a $CB_1$ pretínajú v bode $A_2$, priamky $CA_1$ a $AC_1$ v bode $B_2$ a~priamky $AB_1$ a $BA_1$ v bode $C_2$. Dokážte, že ak je trojuholník $A_1B_1C_1$ rôznostranný, tak existujú dva rôzne body ležiace na všetkých troch kružniciach opísaných trojuholníkom $AA_1A_2$, $BB_1B_2$, $CC_1C_2$.
}
\podpis{...}

{%%%%%   MEMO, priklad 1
Nech $\Bbb R$ označuje množinu všetkých reálnych čísel. Pre všetky dvojice $(\alpha, \beta)$  nezáporných reálnych čísel, pre ktoré $\alpha+\beta\ge2$, nájdite všetky funkcie $f\colon \Bbb R \to \Bbb R$ spĺňajúce $$f(x)f(y)\le f(xy)+\alpha x+\beta y$$ pre všetky reálne čísla $x$, $y$.}
\podpis{Walther Janous, Rakúsko}

{%%%%%   MEMO, priklad 2
Nájdite všetky celé čísla $n\ge 3$, pre ktoré možno nakresliť $n$ tetív jednej kružnice tak, že všetkých ich $2n$ koncových bodov je navzájom rôznych a každá tetiva pretína práve $k$ iných tetív, pre
\ite a) $k=n-2$,
\ite b) $k=n-3$.

\poznamka
Tetiva kružnice je úsečka, ktorej oba koncové body ležia na danej kružnici.}
\podpis{Josef Tkadlec, ČR}

{%%%%%   MEMO, priklad 3
Nech $ABC$ je trojuholník s vpísanou kružnicou $\omega$ so stredom v bode $I$. Kružnica $\omega$ sa dotýka strany $BC$ v bode $D$. Označme $E$, $F$ body spĺňajúce $AI \parallel BE \parallel CF$ a~$|\angle BEI| = |\angle CFI| = 90^\circ.$ Priamky $DE$ a $DF$ pretínajú druhýkrát kružnicu $\omega$ postupne v bodoch $E'$ a~$F'$. Dokážte, že $E'F' \perp AI$.}
\podpis{Patrik Bak, Slovensko}

{%%%%%   MEMO, priklad 4
Nech $n$, $m$ sú kladné celé čísla. Množinu kladných celých čísel $S$ nazývame \emph{$(n, m)$-dobrá}, ak spĺňa nasledovné tri podmienky:
\ite (i) Platí $m \in S$.
\ite (ii) Pre každé číslo $a \in S$ patria do $S$ aj všetky jeho kladné delitele.
\ite (iii) Pre všetky navzájom rôzne čísla $a, b \in S$ platí $a^n + b^n \in S$.

\noindent
Nájdite všetky dvojice $(n,m)$, pre ktoré je množina všetkých kladných celých čísel jedinou $(n,m)$-dobrou množinou.}
\podpis{Michael Reitmeir, Rakúsko}

{%%%%%   MEMO, priklad t1
Nech $\Bbb Z$ označuje množinu všetkých celých čísel a $\Bbb Z_{>0}$ množinu všetkých kladných celých čísel.
\ite a) Funkcia $f\colon\Bbb Z\to\Bbb Z$ sa nazýva \emph{$\Bbb Z$-ilinská}, pokiaľ spĺňa $f(a^2+b)=f(b^2+a)$ pre všetky $a,b\in\Bbb Z$.
Určte najväčší možný počet navzájom rôznych čísel, ktoré sa môžu nachádzať medzi $f(1),f(2), \cdots, f(2023)$, kde $f$ je $\Bbb Z$-ilinská funkcia.
\ite b) Funkcia $f\colon\Bbb Z_{>0}\to\Bbb Z_{>0}$ sa nazýva \emph{$\Bbb Z_{>0}$-ilinská}, pokiaľ spĺňa $f(a^2+b)=f(b^2+a)$ pre všetky $a,b\in\Bbb Z_{>0}$. Určte najväčší možný počet navzájom rôznych čísel, ktoré sa môžu nachádzať medzi $f(1),f(2), \cdots, f(2023)$, kde $f$ je $\Bbb Z_{>0}$-ilinská funkcia.\endgraf
}
\podpis{Josef Tkadlec, ČR}

{%%%%%   MEMO, priklad t2
Nech $a$, $b$, $c$, $d$ sú kladné reálne čísla spĺňajúce $abcd=1$.
Dokážte, že
$$
\postdisplaypenalty=10000
\frac{ab+1}{a+1}+\frac{bc+1}{b+1}+\frac{cd+1}{c+1}+\frac{da+1}{d+1}\ge4
$$
a nájdite všetky štvorice $(a,b,c,d)$, pre ktoré nastáva rovnosť.}
\podpis{Walther Janous, Rakúsko}

{%%%%%   MEMO, priklad t3
Nájdite najmenšie celé číslo $b$ s nasledujúcou vlastnosťou: Pre každé ofarbenie práve $b$~políčok na šachovnici rozmerov $8 \times 8$ na zeleno je možné umiestniť 7 strelcov na 7 zelených políčok tak, aby sa žiadni dvaja strelci neohrozovali.

\poznamka
Dvaja strelci sa ohrozujú, ak sa nachádzajú na rovnakej diagonále.
}
\podpis{Jozef Rajník, Slovensko}

{%%%%%   MEMO, priklad t4
Nech $c\ge4$ je párne celé číslo. V Slovenskej futbalovej lige má každý klub domáci a~hosťovský dres. Každý domáci dres je ofarbený dvomi rôznymi farbami, kým každý hosťovský dres je ofarbený jednou farbou. Farba hosťovského dresu klubu sa musí líšiť od oboch farieb jeho domáceho dresu. Na všetkých dresoch dokopy je nanajvýš $c$ rôznych farieb. Ak majú dva tímy rovnaké obe farby na ich domácich dresoch, potom majú rôzne farby na ich hosťovských dresoch.

Hovoríme, že pár dresov \emph{sa bije}, ak sa nejaká farba vyskytuje na oboch dresoch. Predpokladajme, že pre každý tím $X$ v lige platí, že neexistuje v lige tím $Y$ taký, že domáci dres tímu $X$ sa bije s oboma dresmi tímu $Y$. Nájdite najväčší možný počet tímov v lige.}
\podpis{Ivan Novak, Chorvátsko}

{%%%%%   MEMO, priklad t5
Majme konvexný štvoruholník $ABCD$, ktorého uhly nie sú pravé. Predpokladajme, že body $P$, $Q$, $R$, $S$ ležia postupne na jeho stranách $AB$, $BC$, $CD$, $DA$ tak, že platí $PS \parallel BD$, $SQ \perp BC$, $PR \perp CD$. Navyše predpokladajme, že sa priamky $PR$, $SQ$, $AC$ pretínajú v jednom bode. Dokážte, že body $P$, $Q$, $R$, $S$ ležia na jednej kružnici.}
\podpis{Patrik Bak, Slovensko}

{%%%%%   MEMO, priklad t6
Nech $ABC$ je ostrouhlý trojuholník, kde $|AB| < |AC|$. Nech $J$ je stred kružnice pripísanej k~strane $BC$ a $D$ kolmý priemet bodu $J$ na stranu $BC$. Osi uhlov $BDJ$ a~$JDC$ pretínajú priamky $BJ$ a $JC$ postupne v bodoch $X$ a $Y$. Úsečky $XY$ a $JD$ sa pretínajú v bode $P$. Nech $Q$ je kolmý priemet bodu $A$ na priamku $BC$. Dokážte, že os uhla $QAP$ je kolmá na priamku $XY$.

\poznamka
Kružnica pripísaná k strane $BC$ trojuholníka $ABC$ je kružnica mimo trojuholníka, ktorá sa dotýka priamok $AB$, $AC$ a úsečky $BC$.
}
\podpis{Dominik Burek, Poľsko}

{%%%%%   MEMO, priklad t7
Nájdite všetky kladné celé čísla $n$, pre ktoré existujú kladné celé čísla $a > b$ spĺňajúce
$$
n = \frac{4ab}{a-b}.
$$
}
\podpis{Josef Tkadlec, ČR}

{%%%%%   MEMO, priklad t8
Nech $A$ a $B$ sú kladné celé čísla. Uvažujme postupnosť kladných celých čísel $(x_n)_{n \geq 1}$ takú, že
$$
\postdisplaypenalty=10000
x_{n+1}=A \cdot \gcd(x_n, x_{n-1})+B\quad \text{pre všetky $n \geq 2$.} 
$$
Dokážte, že táto postupnosť dosahuje len konečne veľa rôznych hodnôt.

\poznamka
Zápis $\gcd(a, b)$ značí najväčšieho spoločného deliteľa kladných celých čísel $a$, $b$.
}
\podpis{Ivan Novak, Chorvátsko}

{%%%%%   CPSJ, priklad 1
Nech $ABC$ je trojuholník, ktorý spĺňa $|BC|=2|AC|$. Nech $M$ je stred strany $BC$ a $D$ je taký bod úsečky $AB$, pre ktorý platí $|AD|=2|BD|$. Dokážte $AM \perp MD$.}
\podpis{Patrik Bak}

{%%%%%   CPSJ, priklad 2
Pre kladné celé číslo $n$ označíme $d(n)$ počet kladných deliteľov čísla $n$. Nájdite všetky kladné celé čísla $n$ také, že $d(n)$ je druhý najväčší deliteľ $n$.}
\podpis{Zdeněk Pezlar}

{%%%%%   CPSJ, priklad 3
Daný je ostrouhlý trojuholník $ABC$. Nech $P$ je bod vnútri $ABC$ ležiaci na osi uhla $BAC$. Predpokladajme, že ortocentrum $H$ trojuholníka $ABP$ leží vnútri $ABC$. Označme $Q$ priesečník kolmice z~$H$ na~$AC$ s~$AP$. Dokážte, že $P$ a~$Q$ sú osovo súmerné podľa osi $BH$.}
\podpis{Patrik Bak}

{%%%%%   CPSJ, priklad 4
Každé políčko tabuľky $n\times n$ je modré alebo červené. Predpokladajme, že sú splnené nasledujúce podmienky:
\item{$\bullet$} Ak riadok a stĺpec obsahujú rovnaký počet červených políčok, tak políčko na ich prieniku je červené.
\item{$\bullet$} Ak riadok a stĺpec obsahujú rôzny počet červených políčok, tak políčko na ich prieniku je modré.\endgraf\noindent
Dokážte, že v celej tabuľke je párny počet modrých políčok.}
\podpis{Poľsko}

{%%%%%   CPSJ, priklad 5
Martin vykonáva aritmetickú operáciu na zlomkoch. V každom ťahu zvýši posledný výsledok o jeho prevrátenú hodnotu, čím dostane nový výsledok. Martin začína číslom $1$, po prvom ťahu preto dostane $2$, po druhom ťahu $\frac52$, potom $\frac{29}{10}$ atď. Po $300$ ťahoch dostane číslo $x$. Nájdite najväčšie celé číslo, ktoré nie je väčšie ako $x$.}
\podpis{Poľsko}

{%%%%%   CPSJ, priklad t1
Označme $S(n)$ součet všech číslic přirozeného čísla $n$. Určete všechna přirozená čísla~$n$, pro která jsou obě čísla $n+S(n)$ a $n-S(n)$ druhými mocninami nenulových celých čísel.}
\podpis{Poľsko}

{%%%%%   CPSJ, priklad t2
Na tabuli jsou napsána čísla $1,2,\ldots,2023$ v tomto pořadí. Můžeme s nimi opakovaně provádět následující operaci: Vybereme libovolný lichý počet po sobě napsaných čísel a tato čísla zapíšeme v~obráceném pořadí. Kolik různých pořadí těchto $2023$ čísel tak můžeme dostat?

\emph{Příklad:} Pokud začneme pouze s čísly $1,2,3,4,5,6$, můžeme provádět následující kroky
 $$\bold{1},\bold{2},\bold{3},4,5,6 \rightarrow 3,\bold2,\bold1,\bold4,\bold5,\bold6 \rightarrow 3,6,\bold5,\bold4,\bold1,2 \rightarrow 3,6,1,4,5,2 \rightarrow \cdots$$}
\podpis{Eliška Macáková}

{%%%%%   CPSJ, priklad t3
Na przyj\ę{}ciu spotka\l{}o si\ę{} $n$ osób, przy czym $n\geq 2$. Każda osoba nie lubi dok\l{}adnie jednej innej osoby obecnej na przyj\ę{}ciu (ale niekoniecznie ze wzajemności\ą{}, tzn. może si\ę{} zdarzyć, że $A$ lub $B$ pomimo, że $B$ nie lubi $A$) i~lubi wszystkie pozosta\l{}e. Wykaż, że gości można usadzić przy trzech sto\l{}ach w~taki sposób, aby każdy gość lubi\l{} wszystkie osoby przy swoim stole.}
\podpis{Poľsko}

{%%%%%   CPSJ, priklad t4
W trójk\ą{}cie $ABC$ punkty $M$ i~$N$ s\ą{} środkami odpowiednio boków $AB$ i~$AC$. Dwusieczne k\ą{}tów wewn\ę{}trznych $ABC$ oraz $BCA$ przecinaj\ą{} prost\ą{} $MN$ odpowienio w~punktach $P$ i~$Q$. Niech $p$ b\ę{}dzie styczn\ą{} do okr\ę{}gu opisanego na trójk\ą{}cie $AMP$ przechodz\ą{}c\ą{} przez punkt $P$, a~$q$ b\ę{}dzie styczn\ą{} do okr\ę{}gu opisanego na trójk\ą{}cie $ANQ$ przechodz\ą{}c\ą{} przez punkt $Q$. Wykaż, że proste $p$ i~$q$ przecinaj\ą{} si\ę{} na prostej $BC$.}
\podpis{Michal Pecho}

{%%%%%   CPSJ, priklad t5
Mazo vykonáva nasledujúcu operáciu na trojiciach nezáporných celých čísel: Ak aspoň jedno z nich je kladné, tak si vyberie jedno kladné číslo, zmenší ho o jedna a~ostatným dvom číslam vymení cifry na mieste jednotiek. Začína s trojicou $x$, $y$, $z$. Nájdite trojicu kladných celých čísel $x$, $y$, $z$ takých, že
$xy+yz+zx=1000$ ($\ast$) a počet operácii, ktoré môže Mazo následne s~trojicou $x$, $y$, $z$ vykonať je
\item{(a)} maximálny (t.\,j. neexistuje trojica kladných celých čísel spĺňajúca ($\ast$), ktorá by mu dovolila urobiť viac operácii);
\item{(b)} minimálny (t.\,j. každá trojica kladných celých čísel spĺňajúca ($\ast$) mu dovolí vykonať aspoň toľko operácií).\endgraf
}
\podpis{Ján Mazák}

{%%%%%   CPSJ, priklad t6
Daný je obdĺžnik $ABCD$. Body $E$ a $F$ ležia postupne na stranách $BC$ a $CD$ tak, že obsah trojuholníkov $ABE$, $ECF$, $FDA$ je rovný $1$. Určte obsah trojuholníka $AEF$.
}
\podpis{Poľsko}

{%%%%%   EGMO, priklad 1
...}
\podpis{...}

{%%%%%   EGMO, priklad 2
...}
\podpis{...}

{%%%%%   EGMO, priklad 3
...}
\podpis{...}

{%%%%%   EGMO, priklad 4
...}
\podpis{...}

{%%%%%   EGMO, priklad 5
...}
\podpis{...}

{%%%%%   EGMO, priklad 6
...}
\podpis{...}

{%%%%%   vyberko C, den 1, priklad 1
V trojuholníku $ABC$ je dĺžka ťažnice $BM$ rovná polovici dĺžky strany $BC$. Dokážte, že $|\uhol ABM| = |\uhol BCA| + |\uhol BAC|$.}
\podpis{...}

{%%%%%   vyberko C, den 1, priklad 2
Určte všetky dvojice celých čísel $a$ a $b$, $b > 0$, takých, že riešenia kvadratických rovníc
$$x^2 -ax + a = b \qquad \text{a} \qquad x^2 - ax + a = -b$$ sú štyri po sebe idúce celé čísla.}
\podpis{...}

{%%%%%   vyberko C, den 1, priklad 3
Pre ktoré $n \geq 3$ existuje $n$ priamok v rovine takých, že existuje aspoň $n-2$ bodov, ktorými prechádzajú aspoň tri z týchto priamok?}
\podpis{...}

{%%%%%   vyberko C, den 1, priklad 4
Nech $S\subset\{1,2,\cdots,2023\}$ je množina celých čísel taká, že neexistuje dvojica rôznych čísel z $S$, ktorých rozdiel by delil súčet nejakých dvoch rôznych čísel z $S$. Koľko najviac prvkov môže mať množina $S$?}
\podpis{...}

{%%%%%   vyberko C, den 1, priklad 5
Nech $ a,b,c$ sú kladné reálne čísla také, že absolútna hodnota rozdielu ľubovoľných dvoch je menšia ako $ 2$. Dokážte, že $$ a + b + c < \sqrt {ab + 1} + \sqrt {ac + 1} + \sqrt {bc + 1}.$$}
\podpis{...} 