{%%%%%   Z4-I-1
...}

{%%%%%   Z4-I-2
...}

{%%%%%   Z4-I-3
...}

{%%%%%   Z4-I-4
...}

{%%%%%   Z4-I-5
...}

{%%%%%   Z4-I-6
...}

{%%%%%   Z5-I-1
\napad
Koľko oviec bolo na lúke po odchode časti pastierov a oviec?

\riesenie
Po odchode tretiny oviec ich na lúke zostali dve tretiny z pôvodného počtu, teda 30 oviec ($\frac23\cdot45=30$).
Tie majú celkom 120 nôh ($30\cdot4=120$).

Pastieri, ktorí na lúke zostali, mali celkom 6 nôh ($126-120=6$), teda zostali 3 pastieri ($6:2=3$).
Pastierov odišla polovica, teda pôvodne ich bolo 6 ($3\cdot2=6$).
}

{%%%%%   Z5-I-2
\napad
Ktoré číslice sa môžu v~čísle vyskytovať?

\riesenie
V~hádanom čísle sa nevyskytujú číslice zo šedých políčok, teda hľadané číslo neobsahuje číslice 0, 3, 5, 6, 9.
Zostáva práve päť číslic --- 1, 2, 4, 7, 8 --- ktorých poradie sa postupne pokúsime určiť podľa zvyšných políčok.
V~každom kroku zohľadňujeme všetky predchádzajúce závery:
\begin{itemize}
\item Číslica 1 môže byť jedine na druhom mieste.
\item Číslice 2 nemôže byť ani na druhom, ani na prvom, štvrtom a~piatom mieste; môže byť jedine na treťom mieste.
\item Číslice 8 nemôže byť ani na druhom a~treťom, ani na prvom a~piatom mieste; môže byť jedine na štvrtom mieste.
\item Číslice 4 nemôže byť ani na druhom, treťom a~štvrtom, ani na prvom mieste; môže byť jedine na piatom mieste.
\item Číslice 7 nemôže byť na druhom, treťom, štvrtom a~piatom mieste; môže byť jedine na prvom mieste.
\end{itemize}

Ako jediná možnosť vychádza číslo 71284, ktoré vyhovuje všetkým uvedeným požiadavkám.
Marta môže číslo s~istotou uhádnuť, a to napr. vyššie popísaným spôsobom.

\poznamka
Súčasťou úlohy je aj popis postupu vedúceho k správnej odpovedi.
Uhádnuté číslo bez primeraného zdôvodnenia nemožno ohodnotiť ako plnohodnotné.
}

{%%%%%   Z5-I-3
\napad
Aké typy úsečiek vidíte na obvode daného trojuholníka?

\riesenie
Obvod trojuholníka je tvorený tromi hranami siete, jednou uhlopriečkou štvorca $1\times1$ a~raz uhlopriečkou obdĺžnika $1\times2$.
V~nových mnohouholníkoch stačí všetky tieto časti použiť v~dvojnásobnom počte.

To sa dá urobiť rozličnými spôsobmi; v~nasledujúcom výbere vyhovujúcich mnohouholníkov sú zastúpené rôzne nápady
(málo/veľa vrcholov, súmerné/nesúmerné, konvexné/nekonvexné):
\insp{z5-I-3a.eps}%
}

{%%%%%   Z5-I-4
\napad
Ktoré trojciferné čísla prichádzajú do~úvahy?

\riesenie
Najväčšie trojciferné číslo s navzájom rôznymi číslicami je 987.
Ak by toto bolo jedno z Nikoliných čísel, to druhé by muselo byť 11 (aby rozdiel bol 976).
To je dvojciferné číslo, avšak nie je tvorené rôznymi číslicami, teda nemohlo byť Nikoliným číslom.

Najbližšie menšie trojciferné číslo s navzájom rôznymi číslicami je 986.
Ak by toto bolo jedno z Nikoliných čísel, to druhé by muselo byť 10.
To je dvojmiestne číslo s rôznymi číslicami, teda možné Nikolino číslo.

Žiadna iná dvojica nepripadá do~úvahy, lebo 10 je najmenšie dvojciferné číslo.
Nikoline čísla boli 986 a~10, teda ich súčet bol 996.

\poznamky
Možno uvažovať aj obrátene, teda od najmenšieho dvojciferného čísla:
Ak by jedno z Nikoliných čísel bolo 10, to druhé by muselo byť 986 (aby rozdiel bol 976).
To je trojciferné číslo s navzájom rôznymi číslicami, teda možné Nikolino číslo.
Pre ďalšie dvojciferné čísla s rôznymi číslicami (12, 13, 14, 15, \dots) by druhé číslo buď obsahovalo dve rovnaké číslice (988, 989, 990, 991, \dots), alebo by nebolo trojciferné (pre dvojmiestne čísla väčšie ako 23).
Teda 10 a~986 boli Nikoline čísla.

Nie je ťažké odhaliť vyhovujúcu dvojicu čísel.
Súčasťou úlohy je aj rozbor ďalších možností, teda zdôvodnenie, že viac takýchto dvojíc nie je.
Riešenie bez primeraného rozboru nie je možné hodnotiť najlepším stupňom.
}

{%%%%%   Z5-I-5
\napad
Pokiaľ sa vám nedarí dostať na danú priečku, skúste inú.

\riesenie
\begin{itemize}
\item Malá žaba môže postupne skočiť hore o~2, o~2 a~o~3 priečky, čím sa dostane na siedmu priečku.
\item Veľká žaba môže postupne skočiť o~9 priečok hore a~o~6 priečok dole, čím sa dostane na tretiu priečku.
\item Stredná žaba vie skákať len o párne počty priečok, teda jej dosiahnuteľné priečky sú vždy párne.
Na prvú priečku sa táto žaba nedostane.
\end{itemize}

\poznamka
Uvedené príklady u~kladných odpovedí sú najjednoduchšie, ale nie jediné možné.
Malá žaba môže napr. skákať päťkrát o ~ 2 priečky nahor, raz o ~ 3 priečky dole a ~ dostane sa na tú istú priečku.
Pokiaľ sú úlohy tohto typu riešiteľné, potom mávajú neobmedzené množstvo riešení.
V~našom prípade možno navyše považovať iné (prípustné) poradie skokov za iné riešenia.}

{%%%%%   Z5-I-6
\napad
Koľko kociek mal Jakub v každej úplnej vrstve?

\riesenie
Nových 18 kociek od babičky predstavovalo polovicu všetkých kociek v nedokončenej (rovnako ako v každej inej) vrstve.
Teda v~ jednej vrstve bolo 36 kociek ($2\cdot18=36$).
Vrstvy sú vskutku štvorcové, a to so 6 radmi po 6 kockách ($6\cdot6=36$).

Včera mal Jakub vyskladaných šesť plných vrstiev a polovicu siedmej.
Celkom tak mal 234 kociek ($6\cdot36+18=234$).

\poznamky
V~uvedenom riešení pre zistený počet kociek vo vrstve overujeme, že mohli tvoriť štvorec.
Naopak je možné začať skúšaním možných rozmerov štvorcovej vrstvy tak, aby polovica predstavovala 18 kociek.
Rýchlo dospejeme k tej istej možnosti 6 radov po 6 kockách.

Polovicu vrstvy je možné preskladať do obdĺžnika, ktorého jedna strana je polovičná vzhľadom na stranu druhej.
Číslo 18 je možné v~tomto duchu rozložiť jedine ako $18=3\cdot6$.
Teda jedna vrstva pozostávala z~$6\cdot6=36$ kociek.}

{%%%%%   Z6-I-1
\napad
Mohol ich chovať napr. deväťdesiat?

\riesenie
Počet vtákov, ktoré choval pán Škovránok, musel byť deliteľný deviatimi (kvôli andulkám) a súčasne štyrmi (kvôli kanárikom).
Také čísla sú 36, 72, 108 atď., teda násobky 36.
Z~týchto čísel je jedine 72 v~uvedenom rozmedzí.

Pán Škovránok choval 72 vtákov.

\poznamka
Alternatívne môžeme písať, že anduliek bola $\frac19$ a~kanárikov $\frac14$ všetkých vtákov.
Anduliek a~kanárikov dohromady teda bolo $\frac19+\frac14=\frac{13}{36}$ všetkých vtákov.
Tento pomer je možné vyjadriť ako
$$
\frac{13}{36}=\frac{26}{72}=\frac{39}{108}=\cdots
$$
Jediná vyhovujúca možnosť je tá druhá.
Pre úplnosť dodajme, že anduliek teda bolo 8 a~kanárikov 18.}

{%%%%%   Z6-I-2
\napad
Pátranie je možné začať rozborom posledného medzivýsledku.

\riesenie
Pre jednoduchší popis riešenia úlohy označíme v~každom kroku číslice, na ktoré sústredíme pozornosť, písmenami, a~tie budeme postupne dopĺňať.

Číslica 6 vo výslednom čísle určite nezahŕňa prechod cez desiatku (predchádzajúci súčet $2+0+2$ je dostatočne malý).
Teda $b=5$ (jedine súčet $2+9+5=16$ končí číslicou 6)
a~$a=4$ (z~predchádzajúceho vidíme, že dochádza k~prechodu cez desiatku):
$$
\vbox{\let\\=\cr
\halign{&\hbox to1.0em{\hss$#$\hss}\\
   &&*&*&* \\
\x &&1&*&* \\
\noalign{\vskip4pt\hrule\vskip4pt}
  &2&2&*&* \\
  &9&0&*&& \\
  a&b&2&&& \\
\noalign{\vskip4pt\hrule\vskip4pt}
 5&6&*&*&* \\
}} \qquad
\vbox{\let\\=\cr
\halign{&\hbox to1.0em{\hss$#$\hss}\\
   &&*&*&* \\
\x &&1&*&* \\
\noalign{\vskip4pt\hrule\vskip4pt}
  &2&2&*&* \\
  &9&0&*&& \\
  4&5&2&&& \\
\noalign{\vskip4pt\hrule\vskip4pt}
 5&6&*&*&* \\
}}
$$

Posledný medzivýsledok (452) vzniká súčinom prvého čísla ($\overline{cde}$) s~prvou číslicou druhého čísla (čo je 1).
Teda prvé číslo bolo 452:
$$
\vbox{\let\\=\cr
\halign{&\hbox to1.0em{\hss$#$\hss}\\
   &&c&d&e \\
\x &&1&*&* \\
\noalign{\vskip4pt\hrule\vskip4pt}
  &2&2&*&* \\
  &9&0&*&& \\
  4&5&2&&& \\
\noalign{\vskip4pt\hrule\vskip4pt}
 5&6&*&*&* \\
}} \qquad
\vbox{\let\\=\cr
\halign{&\hbox to1.0em{\hss$#$\hss}\\
   &&4&5&2 \\
\x &&1&*&* \\
\noalign{\vskip4pt\hrule\vskip4pt}
  &2&2&*&* \\
  &9&0&*&& \\
  4&5&2&&& \\
\noalign{\vskip4pt\hrule\vskip4pt}
 5&6&*&*&* \\
}}
$$

Druhý medzivýsledok ($\overline{90g}$) vzniká súčinom prvého čísla (452) s~druhou číslicou druhého čísla (ozn. $f$).
Teda $f=2$ a~druhý medzivýsledok bol 904:
$$
\vbox{\let\\=\cr
\halign{&\hbox to1.0em{\hss$#$\hss}\\
   &&4&5&2 \\
\x &&1&f&* \\
\noalign{\vskip4pt\hrule\vskip4pt}
  &2&2&*&* \\
  &9&0&g&& \\
  4&5&2&&& \\
\noalign{\vskip4pt\hrule\vskip4pt}
 5&6&*&*&* \\
}} \qquad
\vbox{\let\\=\cr
\halign{&\hbox to1.0em{\hss$#$\hss}\\
   &&4&5&2 \\
\x &&1&2&* \\
\noalign{\vskip4pt\hrule\vskip4pt}
  &2&2&*&* \\
  &9&0&4&& \\
  4&5&2&&& \\
\noalign{\vskip4pt\hrule\vskip4pt}
 5&6&*&*&* \\
}}
$$

Prvý medzivýsledok ($\overline{22ij}$) vzniká súčinom prvého čísla (452) s~treťou číslicou druhého čísla (ozn. $h$).
Teda $h=5$ a~prvý medzivýsledok bol 2260:
$$
\vbox{\let\\=\cr
\halign{&\hbox to1.0em{\hss$#$\hss}\\
   &&4&5&2 \\
\x &&1&2&h \\
\noalign{\vskip4pt\hrule\vskip4pt}
  &2&2&i&j \\
  &9&0&4&& \\
  4&5&2&&& \\
\noalign{\vskip4pt\hrule\vskip4pt}
 5&6&*&*&* \\
}} \qquad
\vbox{\let\\=\cr
\halign{&\hbox to1.0em{\hss$#$\hss}\\
   &&4&5&2 \\
\x &&1&2&5 \\
\noalign{\vskip4pt\hrule\vskip4pt}
  &2&2&6&0 \\
  &9&0&4&& \\
  4&5&2&&& \\
\noalign{\vskip4pt\hrule\vskip4pt}
 5&6&*&*&* \\
}}
$$

Odtiaľ už je možné určiť výsledok --- Václavom hľadaný súčin bol 56\,500:
$$
\vbox{\let\\=\cr
\halign{&\hbox to1.0em{\hss$#$\hss}\\
   &&4&5&2 \\
\x &&1&2&5 \\
\noalign{\vskip4pt\hrule\vskip4pt}
  &2&2&6&0 \\
  &9&0&4&& \\
  4&5&2&&& \\
\noalign{\vskip4pt\hrule\vskip4pt}
 5&6&5&0&0 \\
}}
$$

\poznamka
Predposledné dva kroky v~uvedenom riešení sú zameniteľné.
V~takom prípade by (skrátená) rekonštrukcia výsledku vyzerala takto:
$$
\vbox{\let\\=\cr
\halign{&\hbox to1.0em{\hss$#$\hss}\\
   &&*&*&* \\
\x &&1&*&* \\
\noalign{\vskip4pt\hrule\vskip4pt}
  &2&2&*&* \\
  &9&0&*&& \\
  4&5&2&&& \\
\noalign{\vskip4pt\hrule\vskip4pt}
 5&6&*&*&* \\
}} \quad
\vbox{\let\\=\cr
\halign{&\hbox to1.0em{\hss$#$\hss}\\
   &&4&5&2 \\
\x &&1&*&* \\
\noalign{\vskip4pt\hrule\vskip4pt}
  &2&2&*&* \\
  &9&0&*&& \\
  4&5&2&&& \\
\noalign{\vskip4pt\hrule\vskip4pt}
 5&6&*&*&* \\
}} \quad
\vbox{\let\\=\cr
\halign{&\hbox to1.0em{\hss$#$\hss}\\
   &&4&5&2 \\
\x &&1&*&5 \\
\noalign{\vskip4pt\hrule\vskip4pt}
  &2&2&6&0 \\
  &9&0&*&& \\
  4&5&2&&& \\
\noalign{\vskip4pt\hrule\vskip4pt}
 5&6&*&*&* \\
}} \quad
\vbox{\let\\=\cr
\halign{&\hbox to1.0em{\hss$#$\hss}\\
   &&4&5&2 \\
\x &&1&2&5 \\
\noalign{\vskip4pt\hrule\vskip4pt}
  &2&2&6&0 \\
  &9&0&4&& \\
  4&5&2&&& \\
\noalign{\vskip4pt\hrule\vskip4pt}
 5&6&*&*&* \\
}} \quad
\vbox{\let\\=\cr
\halign{&\hbox to1.0em{\hss$#$\hss}\\
   &&4&5&2 \\
\x &&1&2&5 \\
\noalign{\vskip4pt\hrule\vskip4pt}
  &2&2&6&0 \\
  &9&0&4&& \\
  4&5&2&&& \\
\noalign{\vskip4pt\hrule\vskip4pt}
 5&6&5&0&0 \\
}}
$$

Vzhľadom na to, že na každom mieste môže byť najviac desať možných číslic, dá sa pri riešení úlohy postupovať systematickým preverovaním možností.

Súčasťou úlohy je aj popis postupu vedúceho k správnej odpovedi.
Vyplnený algebrogram bez primeraného komentára nemožno hodnotiť najlepším stupňom.}

{%%%%%   Z6-I-3
\napad
Bola dlhšia základňa, alebo rameno trojuholníka? A~o~koľko?

\riesenie
Obvod prvého štvoruholníka pozostáva z dvoch ramien a dvoch základní, obvod druhého zo štyroch ramien.
Pretože prvý štvoruholník mal o~4\,cm kratší obvod ako druhý, musela byť základňa o~2\,cm kratšia ako rameno.

Obvod každého trojuholníka pozostáva z jednej základne a dvoch ramien, čo zodpovedá trom ramenám bez 2\,cm.
Súčasne každý trojuholník mal obvod 100\,cm.
Teda súčet dĺžok troch ramien bol 102\,cm.

Ramená trojuholníkov merali 34\,cm ($102:3=34$) a ~ základňa 32\,cm ($34-2=32$).

\poznamka
Pokiaľ dĺžku základne označíme $z$ a~dĺžku ramena $r$, potom predchádzajúce úvahy možno zapísať nasledovne
(jednotky cm v~ďalšom neuvádzame):

Rozdiel obvodov štvoruholníkov bol
$$4r-2r-2z=2r-2z=4,$$
teda $r-z=2$ alebo $z=r-2$.
Obvod každého trojuholníka bol
$$2r+z=3r-2=100,$$
teda $3r=102$.
Odtiaľ dostávame $r=102:3=34$ a ~$z=34-2=32$.

Trojuholníky s takýmito stranami existujú, keďže sú splnené trojuholníkové nerovnosti ($34+32>34$ a $34+34>32$).
}

{%%%%%   Z6-I-4
\napad
Koľko rokov mali jednotliví trpaslíci?

\res
Súčet vekov prvých troch trpaslíkov bol 42 rokov, teda priemerne mali 14 rokov ($42:3=14$).
Veky týchto, resp. zvyšných štyroch trpaslíkov boli 13, 14, 15, resp. 16, 17, 18 a ~19 rokov.

Súčet vekov všetkých siedmich trpaslíkov bol 112 rokov, teda priemerne mali 16 rokov ($112:7=16$).
Ak by odišiel najmladší trpaslík, priemerný vek zvyšných šiestich by bol 16,5 rokov:
$$
(112-13):6=16{,}5.
$$
Podobne pre ostatných trpaslíkov určíme, čo by sa po ich odchode stalo s priemerným vekom zvyšných:
$$
\begintable
vek odchádzajúceho\|13|14|15|16|17|18|19\cr
priemer zvyšných\|16,5|16,$\overline{3}$|16,1$\overline{6}$|16|15,8$\overline{3}$|15,$\overline{6} $|15,5\endtable
$$

Priemerný vek sa nezmenil po odchode prostredného trpaslíka.
Trpaslík, ktorý odišiel so Snehulienkou, mal 16 rokov.

\poznamky
Priemerný vek trpaslíkov sa po odchode jedného z nich nezmenil.
Onen trpaslík sa teda narodil ako prostredný a jeho vek bol priemerným vekom všetkých siedmich.
Týmto spôsobom je možné nahradiť skúšanie možností v druhej časti riešenia.

Úvahy v~uvedenom riešení je možné uchopiť aj takto:
Pokiaľ napr. vek najmladšieho trpaslíka označíme $n$, potom súčet vekov prvých troch trpaslíkov bol
$$
n+(n+1)+(n+2)=3n+3=42.
$$
Teda $3n=39$ alebo $n=13$.
Veky trpaslíkov boli 13, 14, 15, 16, 17, 18, 19 rokov.
Súčet všetkých siedmich vekov bol 112 a ich priemer $112:7=16$.
Pokiaľ vek odchádzajúceho trpaslíka označíme $o$, potom sa priemer zvyšných nezmenil:
$$
(112-o):6=16.
$$
Teda $112-o=96$ alebo $o=16$.}

{%%%%%   Z6-I-5
\napad
Ktoré číslo mal Mat v predposlednom kroku?

\riesenie
Úlohu je možné znázorniť ako tzv. číselného hada:
\insp{z6-I-5a.eps}%

Postupne odzadu doplníme všetky medzivýsledky:
\insp{z6-I-5b.eps}%

Pat zadal Matovi číslo 21.

\poznamky
Pri spätnom postupe operácie v~jednotlivých krokoch obraciame
(napr. v~prvom kroku namiesto $x+7=57$ uvažujeme $x=57-7$).
S týmto postrehom je možné predchádzajúce riešenie zapísať nasledovne:
$$
\aligned
((((((((57-7)+4)\cdot2):3):3)\cdot2)+4)-7 &= \\
= (((((((50+4)\cdot2):3):3)\cdot2)+4)-7 &= \\
= ((((((54\cdot2):3):3)\cdot2)+4)-7 &= \\
& \hskip5pt \vdots \\
= 28-7 &=21.
\endaligned
$$

Číselný had v štvorcovej sieti, teda Matova skutočná cesta vyzerala takto:
\insp{z6-I-5c.eps}%

V zvýraznenom poli uprostred sa začínalo (číslom 21), prechádzalo sa ním po štvrtom kroku (medzivýsledok 36) a tiež sa v~ňom končilo (výsledok 57).
}

{%%%%%   Z6-I-6
\napad
Je možné vopred vylúčiť nejaké číslice z bežného vyjadrenia času?

\res
Ciferný súčet číslic ukazujúcich skutočný čas je 6, teda žiadna z~číslíc nie je väčšia ako 6.
Súčet čísel ukazujúcich skutočné hodiny a~minúty je 15.
Pomocou číslic od 0 do 6 je možné 15 vyjadriť ako súčet dvoch čísel nasledujúcimi spôsobmi:
$$
0+15, 4+14, 2+13, 3+12, 4+11, 5+10.
$$

Keď Borisove hodiny ukazujú 6:15, môže byť 0:15, 1:14, 2:13, 3:12, 4:11, 5:10, 10:05, 11:04, 12:03, 13:02, 14:01, alebo 15:00.

\poznamka
Súčasťou úlohy je aj popis postupu vedúceho k správnej odpovedi.
Uvedené časy bez primeraného komentára nemožno hodnotiť najlepším stupňom.}

{%%%%%   Z7-I-1
\napad
Koľko rokov majú všetky vnúčatá dohromady?

\res
Vnúčatá dohromady majú 35 rokov ($5\cdot7=35$).
Dedo, babička a vnúčatá spolu majú 182 rokov ($7\cdot26=182$).
Dedo a ~ babička dohromady teda majú 147 rokov ($182-35=147$).

Priemerný vek babičky a dedka je 73 a pol roka ($147:2=73{,}5$) a~babička je o rok mladšia ako dedo.
Teda babička má 73 rokov (a ~ dedo 74 rokov; $73{,}5\pm0{,}5$).

\poznamka
Pokiaľ značí vek babičky označíme $b$ a~ vek deda $d$, potom informácie zo zadania je možné zapísať ako
$$
(d+b+5\cdot7): 7=26, \qquad d=b+1.
$$
Z~prvého vzťahu dostávame $d+b=26\cdot7-5\cdot7=21\cdot7=147$.
Dosadením druhého vzťahu dostávame $2b+1=147$, teda $b=(147-1):2=73$.}

{%%%%%   Z7-I-2
\napad
Oplatí sa zamerať na skupiny navzájom zhodných uhlov.

\riesenie
Vnútorný uhol pri~vrchole $A$, resp. $D$ v~trojuholníku $AFD$ je tiež vnútorným uhlom trojuholníka $ABE$, resp. $DBC$.
Trojuholníky $ABE$ a~$DBC$ sú podľa predpokladov rovnoramenné a navzájom zhodné.
Teda aj vnútorné uhly pri~vrcholoch $A$ a~$D$ sú zhodné.

Preto je trojuholník $AFD$ rovnoramenný a~uhol $AFD$ je zhodný s~uhlom $ABE$, resp. $DBC$.
Tento uhol je vonkajším uhlom rovnostranného trojuholníka, teda má veľkosť 120\st.

Veľkosť uhla $AFD$ je 120\st.
\insp{z7-I-2a.eps}%

\ineriesenie
Vnútorný uhol v~rovnostrannom trojuholníku má veľkosť 60\st.
Pri otočení okolo bodu $B$ o~tento uhol v smere hodinových ručičiek sa bod $A$ zobrazí na $C$ a~bod $E$ sa zobrazí na $D$.
Teda priamka $AE$ sa zobrazí na priamku $CD$.

Uhol $AFC$, resp. $EFD$ má preto veľkosť 60\st.
Hľadaný uhol $AFD$ dopĺňa tento uhol do priameho uhla, teda má veľkosť 120\st.

\poznamky
Z predpokladov vyplýva rad ďalších vzťahov, ktoré môžu viesť k alternatívnym zdôvodneniam predchádzajúceho výsledku.
Stručne uvádzame niekoľko postrehov (pre dodatočné značenia viď obrázok nižšie):

Štvoruholník $ABEC$, resp. $BDEC$ je kosoštvorcom.
Uhlopriečky v kosoštvorci delia príslušné vnútorné uhly na polovice a sú navzájom kolmé.
\begin{itemize}
\item
Z~prvého faktu vyplýva, že uhol $BAE$, resp. $BDC$ má veľkosť 30\st.
Uhol $AFD$, čo je zvyšný vnútorný uhol v~trojuholníku $AFD$, má preto veľkosť 120\st\
(súčet vnútorných uhlov trojuholníka je 180\st).
\item
Z~druhého faktu poznáme vnútorné uhly pri~vrcholoch $G$ a~$H$ v štvoruholníku $BHFG$.
Vnútorný uhol pri~vrchole $B$ má veľkosť 60\st.
Uhol $AFD$, čo je zvyšný vnútorný uhol v štvoruholníku $BHFG$, má preto veľkosť 120\st\
(súčet vnútorných uhlov štvoruholníka je 360\st).
\end{itemize}

Zadaný útvar je časťou rovnostranného trojuholníka $ADI$, kde body $B$, $C$, $E$ sú stredy jeho strán.
Úsečky $AE$, $CD$ a~$IB$ sú výškami a súčasne osami súmernosti tohto trojuholníka.
Trojuholníky $AFD$, $DFI$ a~$IFA$ sú navzájom zhodné a~spolu tvoria trojuholník $ADI$.
Ich vnútorné uhly pri~vrchole $F$ preto majú veľkosť 120\st\
(tretina plného uhla).
\insp{z7-I-2c.eps}%
}

{%%%%%   Z7-I-3
\napad
Koľkokrát sú použité nuly v~každom z~hľadaných čísel?

\riesenie
Hľadané čísla sú sedemmiestne a ~obsahujú práve dve jednotky a~ práve dve dvojky, teda obsahujú práve tri nuly.
Tieto nuly oddeľujú v~zápise čísla štyri časti, v ktorých sa môžu nachádzať jednotky a~dvojky:
$$
\dots 0 \dots 0 \dots 0 \dots
$$

Niektoré časti môžu zostať prázdne, niektoré nuly môžu byť bezprostredne vedľa seba.
Jednotky môžu byť jedine v susedných častiach (oddelené jednou nulou),
dvojky ob jednu časť (oddelené dvoma nulami).
Všetky čísla s týmito vlastnosťami, ktoré nezačínajú nulou, sú v~nasledujúcej tabuľke:
\bgroup
\def\ctr#1{\hfil\hskip.7em#1\hskip.7em\hfil}
$$
\begintable
\| $\buildrel1\over\dots 0 \buildrel1\over\dots 0 \dots 0 \dots$ | $\dots 0 \buildrel1\over\dots 0 \buildrel1\over\dots 0 \dots$ | $\dots 0 \dots 0 \buildrel1\over\dots 0 \buildrel1\over\dots$ \crthick
$\buildrel2\over\dots 0 \dots 0 \buildrel2\over\dots 0$ \dots \| 1201020, 2101020 | 2010120, 2010210 | 2001201, 2002101 \cr
$\dots 0 \buildrel2\over\dots 0 \dots 0 \buildrel2\over\dots$ \| 1012002, 1021002 | --- | ---
\endtable
$$
\egroup

Obkročných čísel s~uvedenými vlastnosťami je osem.

\poznamka
Súčasťou úlohy je aj popis postupu vedúceho k správnej odpovedi.
V tomto prípade ide o nejaký systém, aby bolo zrejmé, že sa na nič nezabudlo.
Výpočet možností bez primeraného zdôvodnenia nemožno hodnotiť najlepším stupňom.}

{%%%%%   Z7-I-4
\napad
Jarko sa najskôr sústredil na opakujúce sa písmená.

\res
Vo dvojici za sebou stojacich slabík ZU ZA by musela byť číslica nahrádzajúca písmeno U~menšia ako číslica nahrádzajúca písmeno A.

Vo dvojici za sebou stojacich slabík LA LU by musela byť číslica nahrádzajúca písmeno A~menšia ako číslica nahrádzajúca písmeno U.

Tieto dve požiadavky sú nezlučiteľné, a preto Jarkova úloha nemá riešenie.}

{%%%%%   Z7-I-5
\napad
Obsah trojuholníka je polovicou obsahu nejakého obdĺžnika.

\res
Obsah trojuholníka $IJS$ je rovný polovici obsahu obdĺžnika so stranami $IJ$ a~$IA$.
Strana $IJ$ je stranou štvorca $IJHE$, strana $IA$ je súčtom strany štvorca $IJHE$ a~strany štvorca $EFCA$.

Štvorec $IJHE$ má dvojnásobnú stranu vzhľadom k štvorcu $EFCA$, ktorá meria 1\,cm.
Teda strana $IJ$ meria 2\,cm a~strana $IA$ meria 3\,cm.
Obsah trojuholníka $IJS$ je rovný
$$
\frac12\cdot2\cdot3 =3\,(\Cm^2).
$$
\insp{z7-I-5a.eps}%

\poznamka
Pri jemnejšom delení naznačenej štvorcovej siete je možné k tomu istému výsledku dospieť názorným počítaním štvorčekov, resp. trojuholníčkov.}

{%%%%%   Z7-I-6
\napad
Každé prirodzené číslo má konečný počet deliteľov.

\riesenie
Číslo 645 je možné zapísať ako súčin dvoch prirodzených čísel nasledujúcimi spôsobmi:
$$
645 =1\cdot645 =5\cdot129 =15\cdot43 = 3\cdot215.
$$
Súčinitelia sú dvojciferní iba v~predposlednom prípade, teda súčet Eviných čísel bol 43 a~rozdiel 15.

Čísla 43 a 15 vznikli tak, že k~väčšiemu z~Eviných čísel sa raz pripočítalo a~raz sa od neho odpočítalo to menšie.
Teda väčšie Evino číslo je presne medzi (tzn. je priemerom) 43 a~15 a~menšie Evino číslo je polovicou rozdielu medzi 43 a~15.
Eva si myslela čísla
$$
\frac{43+15}{2}=29, \qquad \frac{43-15}{2}=14.
$$

\poznamka
Ak väčšie z~Eviných čísel označíme $v$ a~to menšie $m$, potom predchádzajúce vzťahy môžeme zapísať ako
$$
v+m=43, \qquad v-m=15.
$$
Odtiaľ je možné formálne odvodiť, že
$$
(v+m)+(v-m) =43+15, \qquad
(v+m)-(v-m) =43-15,
$$
a~teda
$$
v =\frac{43+15}{2} =29, \qquad
m =\frac{43-15}{2} =14.
$$}

{%%%%%   Z8-I-1
\napad
Vyjadrite jedno z~čísel pomocou zvyšných dvoch.

\res
Označme tri čísla zo zadania ako $a$, $b$, $c$, pričom $a<b<c$.
Prvá podmienka znamená
$$
\frac{\frac{a+b}{2}+\frac{b+c}{2}}{2}=\frac{a+b+c}{3}. \tag{$*$}
$$
Túto rovnosť môžeme upraviť nasledovne:
$$\aligned
\frac{a+2b+c}{4} &=\frac{a+b+c}{3}, \\
3a+6b+3c &= 4a+4b+4c, \\
2b &= a+c, \\
b &= \frac{a+c}{2}.
\endaligned
$$
Teda $b$ je priemerom $a$ a~$c$, ktorého hodnotu však poznáme:
$b =\frac{a+c}{2} = 2022$.

Z~uvedeného plynie, že súčet daných čísel je rovný
$$
a+b+c =b+2b =6066.
$$

\poznamky
Zo zadaných údajov nie je možné určiť čísla ako také, iba ich súčet.
To, že úloha je zmysluplná, teda že existujú čísla s~uvedenými vlastnosťami, je možné ukázať napr. na trojici $a=2020$, $b=2022$, $c=2024$.

Hneď skraja si možno všimnúť, že pokiaľ $b$ je priemerom $a$ a~$c$, potom každá takáto trojica čísel vyhovuje prvej podmienke zo zadania: ako priemer priemerov $\frac{a+b}{2}$ a~ $\frac{b+c}{2}$, tak priemer všetkých $\frac{a+b+c}{3}$ je rovný $b=\frac{a+c}{2}$.
Pre podmienky typu ($*$) --- tj lineárne vzhľadom k~$a$, $b$, $c$ --- platí, že ktorékoľvek z~čísel $a$, $b$, $c$ je jednoznačne určené zvyšnými dvoma.
S týmto dodatkom možno predchádzajúci postreh považovať za vyhovujúcu náhradu vyššie uvedených úprav rovnosti ($*$).}

{%%%%%   Z8-I-2
\napad
Pomôžte si obrázkom s~doplnenou úsečkou $EG$.

\riesenie
Úsečka $EG$ rozdeľuje kosoštvorec $ABCD$ na dva zhodné kosodĺžniky $ABGE$ a~$EGCD$
(body $E$ a~$G$ sú stredy strán, teda úsečky $AE$, $BG$, $ED$, $GC$ sú navzájom zhodné a~priamky $AB$, $EG$, $CD$ navzájom rovnobežné ).
Zvyšné úsečky zo zadania sú uhlopriečkami týchto kosodĺžnikov a~body $F$, $H$ sú ich priesečníky:
\insp{z8-I-2a.eps}%

Každý kosodĺžnik je svojimi uhlopriečkami rozdelený na štyri trojuholníky s rovnakými obsahmi
(každé dva susedné trojuholníky majú zhodné strany ležiace na jednej priamke a~spoločnú výšku vzhľadom k~tejto priamke).

Kosoštvorec $ABCD$ je tak rozdelený na osem trojuholníkov s rovnakými obsahmi a ~ štvoruholník $EFGH$ je tvorený dvoma z ~ týchto ôsmich trojuholníkov.
Tento štvoruholník teda zaberá dve osminy, tj jednu štvrtinu obsahu daného kosoštvorca.

Obsah kosoštvorca $ABCD$ je rovný $6\cdot4 =24\,(\Cm^2)$, teda
$$
S_{EFGH} = \frac14 S_{ABCD} = 6\,(\Cm^2).
$$

\poznamky
V~uvedenom riešení nie je podstatné, ktorá z~uhlopriečok kosoštvorca je kratšia a~ktorá dlhšia.
Pretože sa zaujímame výhradne o ~obsahy, môžeme dokonca pracovať s~akýmkoľvek kosodĺžnikom, ktorý má rovnakú stranu a ~výšku ako pôvodný kosoštvorec.
Preto môžeme úlohu riešiť v~obdĺžniku so stranami 6\,cm a~4\,cm:
\insp{z8-I-2b.eps}%

Dodatočné delenie pomocou rovnobežiek so stranami rozdeľuje obdĺžnik $ABCD$ na 16 zhodných trojuholníkov, z ktorých 4 tvoria štvoruholník $EFGH$.
Tento štvoruholník teda zaberá $\frac4{16}$, tj jednu štvrtinu obsahu obdĺžnika.}

{%%%%%   Z8-I-3
\napad
Pomôžte si postupnosťou tvorenou poslednými číslicami.

\riesenie
Posledná číslica každého čísla zodpovedá zvyšku po delení onoho čísla desiatimi.
Stačí sa teda zaoberať postupnosťou zodpovedajúcich zvyškov:
$$
1,\ 3,\ 4,\ 7,\ 1,\ 8,\ \dots ,
$$
tzn. postupnosťou, v ktorej každé číslo počnúc tretím je zvyškom súčtu predchádzajúcich dvoch po delení desiatich.
Táto postupnosť sa po 12 členoch opakuje:
$$
1, 3, 4, 7, 1, 8, 9, 7, 6, 3, 9, 2; 1, 3, \dots
$$
Teda napr. 1., 13., 25., 145. či 2017. člen postupnosti tvoria rovnaké čísla.

Číslo 2023 po delení 12 dáva 168 a zvyšok 7.
Teda 2023. člen tejto postupnosti je rovnaký ako ten siedmy, a to je číslo 9.

2023. číslo v~danej postupnosti končí číslicou 9.}

{%%%%%   Z8-I-4
\napad
Aké rozmery mal vyznačený pozemok na pôvodnej mape?

\riesenie
Na pôvodnej mape mala strana pozemku dĺžku 2\,cm ($2\cm \cdot50\,000 = 100\,000\cm = 1\,\text{km}$).
Teda obsah príslušného štvorca bol 4\,cm$^2$.

Po zmenšení mapy bol obsah nového štvorca 2,56\cm$^2$ ($2{,}56+1{,}44=4$).
Teda strana tohto štvorca mala dĺžku 1,6\,cm ($1{,}6^2=2{,}56$).

Oných 1,6\,cm na mape zodpovedá stále rovnakému 1\,km v skutočnosti.
Mierka takto zmenšenej mapy teda bolo
$$
1{,}6 : 100\,000 = 1 : 62\,500.
$$}

{%%%%%   Z8-I-5
\napad
Oplatí sa zamerať na prvočísla.

\riesenie
Na porovnanie súčinov vzniknutých skupín čísel budú užitoční prvočíselní činitelia.
Aby súčiny čísel v obidvoch skupinách boli rovnaké, musia byť celkové počty jednotlivých prvočiniteľov párne.
Pokiaľ je počet niektorých prvočiniteľov nepárny, potom rozdelenie do skupín s rovnakým súčinom nie je možné.

Prvočísla medzi danými číslami sú iba 2, 3, 5, 7, pričom 5 a ~ 7 sa vyskytujú práve raz.
Vzhľadom k~výskytu 5 a~7 stačí uvažovať tri prípady, ktoré môžu viesť k~riešeniu, a~tie postupne rozoberieme:

\medskip
\noindent
a) Petra sčítala niečo s~5, aby dostala násobok 7.
To mohla urobiť dvojakým spôsobom:
\begin{itemize}
\item Zrátala $5+2$ a~dostala osemci čísel 1, 3, 4, 6, 7, 7, 8, 9.
Túto osmicu možno rozdeliť do skupín s rovnakým súčinom:
$$
1\cdot7\cdot8\cdot9 = 3\cdot4\cdot6\cdot7 = 2^3\cdot3^2\cdot7 = 504.
$$
\item Zrátala $5+9$ a~dostala osemci čísel 1, 2, 3, 4, 6, 7, 8, 14.
Túto osmicu možno rozdeliť do skupín s rovnakým súčinom:
$$
1\cdot4\cdot6\cdot14 = 2\cdot3\cdot7\cdot8 = 2^4\cdot3\cdot7 = 336.
$$
\end{itemize}

%\medskip
\noindent
b) Petra sčítala niečo so 7, aby dostala násobok 5.
To mohla urobiť dvojakým spôsobom:
\begin{itemize}
\item Zrátala $7+3$ a~dostala osmicu čísel 1, 2, 4, 5, 6, 8, 9, 10.
Túto osmicu nemožno rozdeliť do skupín s rovnakým súčinom,
pretože 3 sa v~prvočíselných rozkladoch týchto čísel vyskytuje celkom trikrát.
\item Zrátala $7+8$ a~dostala osmicu čísel 1, 2, 3, 4, 5, 6, 9, 15.
Túto osmicu nemožno rozdeliť do skupín s rovnakým súčinom,
pretože 3 sa v~prvočíselných rozkladoch týchto čísel vyskytuje celkom päťkrát.
\end{itemize}

\medskip
\noindent
c) Petra sčítala $5+7$ a~dostala osmicu čísel 1, 2, 3, 4, 6, 8, 9, 12.
Túto osmicu nemožno rozdeliť do skupín s rovnakým súčinom,
pretože 3 sa v~prvočíselných rozkladoch týchto čísel vyskytuje celkom päťkrát (či 2 sa vyskytuje deväťkrát).

\medskip
Petra mohla dostať buď súčin 504, alebo 336.
Najväčší možný súčin je teda 504.

\poznamky
Rozdelenia do skupín s rovnakým súčinom v~prípade a) nie sú jediné možné (napr. 1 môže byť kdekoľvek).

Počet všetkých dvojíc, ktoré je možné utvoriť z~daných deviatich čísel, je 36.
S~úvodným pozorovaním sme počet prípadov do diskusie podstatne znížili.

Súčasťou úlohy je aj popis postupu vedúceho k správnej odpovedi.
Nájdené súčiny bez primeraného zdôvodnenia, že iné možné nie sú, nemožno hodnotiť najlepším stupňom.}

{%%%%%   Z8-I-6
\napad
Oplatí sa zamerať na skupiny zhodných objektov.

\riesenie
Vzájomná poloha obdĺžnika a~dvoch rovnostranných trojuholníkov je znázornená na obrázku, v ktorom sú tiež vyznačené niektoré navzájom zhodné uhly:
\insp{z8-I-6a.eps}%

Rovnostrannosť trojuholníka $AEF$ vyplýva zo vzájomnej zhodnosti trojuholníkov $ABE$, $FCE$ a~$FDA$, a~tú zdôvodníme nasledovne:
\begin{itemize}
\item strany $AB$, $FC$ a~$FD$ sú navzájom zhodné, pretože
protiľahlé strany $AB$ a~$CD$ obdĺžnika sú zhodné a~trojuholník $CFD$ je rovnostranný,
\item strany $BE$, $CE$ a~$DA$ sú navzájom zhodné, pretože
protiľahlé strany $DA$ a~$BC$ obdĺžnika sú zhodné a~trojuholník $BEC$ je rovnostranný,
\item uhly $ABE$, $FCE$ a~$FDA$ sú navzájom zhodné, pretože všetky majú rovnakú veľkosť:
$$
|\uhel ABE| =|\uhel FDA| =90\st+60\st =150\st,
\qquad
|\uhel FCE| =360\st-2\cdot60\st-90\st =150\st.
$$
\end{itemize}

Podľa vety {\it sus\/} sú trojuholníky $ABE$, $FCE$ a~$FDA$ navzájom zhodné.
Zhodujú sa teda aj vo zvyšných stranách, čo sú zároveň strany trojuholníka $AEF$.
Preto je trojuholník $AEF$ rovnostranný.

\poznamky
V~úvodnom obrázku je znázornený obdĺžnik, ktorého strana $AB$ je dlhšia ako $BC$.
Tento predpoklad nehrá v~ďalšom žiadnu rolu --- rovnaké argumenty, a~teda aj~záver, platí pre akýkoľvek pravouholník (vrátane štvorca).

Pokiaľ by sme strany obdĺžnika označili $a$ a~$b$, možno sa k zhodnosti strán $AE$, $EF$ a~$FA$ dopočítať pomocou Pytagorovej vety vo vhodných pravouhlých trojuholníkoch.
Pre prehľadnosť opíšme danému útvaru obdĺžnik $AGHI$ ako na obrázku:
\insp{z8-I-6b.eps}%

\noindent
Doplnené rozmery sú odvodené z~rovnostrannosti trojuholníkov $BEC$ a~$CFD$.
Podľa Pytagorovej vety v~pravouhlom trojuholníku $AGE$ platí
$$\align
\vert{}AE\vert{}^2 =\vert{}AG\vert{}^2+\vert{}GE\vert{}^2
& =\left(a +\frac{\sqrt3}2b \right)^2 + \left( \frac12b \right)^2 = \\
&=a^2+\sqrt3ab+\frac34b^2+\frac14b^2
=a^2+b^2+\sqrt3ab.
\endalign
$$
Podobným spôsobom je možné vyjadriť aj stranu $EF$, resp. $FA$ ako preponu v~pravouhlom trojuholníku $EHF$, resp. $FIA$.
Vychádza zakaždým rovnaký výraz, teda trojuholník $AEF$ je rovnostranný.}

{%%%%%   Z9-I-1
\napad
Bolkovu a~Lolkovu diferenciu je možné vyjadriť pomocou jednej premennej.

\riesenie
Pomer Bolkovej a~Lolkovej diferencie bol $5:2$.
Teda Bolkova diferencia bola $5k$ a~Lolkova $2k$, kde $k$ je zatiaľ neznáme číslo, ktoré skoro určíme z~ostatných informácií.
Rozdiel Bolkovej a~Lolkovej diferencie potom vyjadríme ako $5k-2k=3k$.

Bolkova, resp. Lolkova postupnosť bola
$$ \aligned
& 2023,\hskip96pt 2023+5k,\hskip96pt 2023+10k,\quad \dots , \\
& 2023,\quad 2023+2k,\quad 2023+4k,\quad 2023+6k,\quad 2023+8k,\quad 2023+10k,\quad \dots ,
\endaligned
$$
spoločné čísla oboch postupností boli
$$
2023, \quad 2023 + 10k, \quad 2023 +20k, \quad 2023 +30k, \quad \dots . \tag{$*$}
$$
Spoločných čísel bolo 26 a~posledné bolo 3023.
Pritom 26. číslo v~postupnosti $(*)$ je tvaru $2023+250k$, teda $k=4$ ($2023+250\cdot4=3023$).

Rozdiel Bolkovej a~Lolkovej diferencie bol 12.}

{%%%%%   Z9-I-2
\napad
Súčty vnútorných uhlov v~trojuholníkoch či mnohouholníkoch poznáte.

\riesenie
Vnútorný uhol pri~vrchole $A$, resp. $D$ v štvoruholníku $AFDB$ je tiež vnútorným uhlom trojuholníka $ABE$, resp. $DBC$.
Trojuholníky $ABE$ a~$DBC$ sú podľa predpokladov rovnoramenné a navzájom zhodné.
Teda vnútorné uhly pri~vrcholoch $A$, $E$, $D$, $C$ v~týchto trojuholníkoch sú navzájom zhodné.
Veľkosť týchto uhlov závisí od uhla $ABD$, ktorý nie je presne vymedzený.
Označme diskutované uhly podľa obrázku:
\insp{z9-I-2a.eps}%

Súčet vnútorných uhlov v~trojuholníku $ABE$, resp. $DBC$ dáva vzťah medzi $\alpha$ a~$\omega$:
$$
2\alpha+\omega-60\st=180\st .
$$
Súčet vnútorných uhlov v~štvoruholníku $AFDB$ dáva vzťah s~hľadaným uhlom $\phi$:
$$
2\alpha+\omega+\phi=360\st.
$$
Z~prvého vzťahu plynie $2\alpha+\omega=240\st$, čo spoločne s druhým vzťahom dáva $\phi=120\st$.

Veľkosť uhla $AFD$ je 120\st.

\ineriesenie
Vnútorný uhol v~rovnostrannom trojuholníku má veľkosť 60\st.
Pri otočení okolo bodu $B$ o~60\st v smere hodinových ručičiek sa bod $A$ zobrazí na $C$ a~bod $E$ sa zobrazí na $D$.
Teda priamka $AE$ sa zobrazí na priamku $CD$.

Uhol $AFC$, resp. $EFD$ má preto veľkosť 60\st.
Hľadaný uhol $AFD$ dopĺňa tento uhol do priameho uhla, teda má veľkosť 120\st.

\poznamka
Výsledok $\phi=120\st$ naozaj nezávisí od veľkosti uhla $\omega$.
Obmedzenia v~zadaní nie sú podstatné a~slúžia len na ľahšie uchopenie úlohy.
V~medzných prípadoch dostávame špeciálne a spravidla jednoduchšie prípady:
pre $\omega=120\st$ body $C$, $E$ a~$F$ splývajú a~$\phi$ je súčtom dvoch vnútorných uhlov rovnostranných trojuholníkov,
pre $\omega=180\st$ dostávame zadanie ako v~úlohe {\bf Z7--I--2}.}

{%%%%%   Z9-I-3
\napad
Pokiaľ vám nejde kúzlenie priamo, skúste to nepriamo.

\riesenie
Päticu trojok možno vykúzliť, kúzelníci mohli (v~ľubovoľnom poradí) postupovať podľa nasledujúcej schémy:
$$\aligned
& 3 \\
& 8 \buildrel :2 \over \longmapsto 4 \buildrel -1 \over \longmapsto 3 \\
& 9 \buildrel -1 \over \longmapsto 8 \buildrel :2 \over \longmapsto 4 \buildrel -1 \over \longmapsto 3 \\
& 2 \buildrel :2 \over \longmapsto 1 \buildrel \cdot3 \over \longmapsto 3 \\
& 4 \buildrel -1 \over \longmapsto 3
\endaligned
$$
Žiadne z kúziel nebolo použité viac ako štyrikrát a žiadny medzivýsledok nebol väčší ako 8.

\medskip
Päticu päťok vykúzliť síce možno, ale nie s dodatočnými obmedzeniami na počet jednotlivých kúziel a veľkosti medzivýsledkov:

Najprv si uvedomme, že všetky medzivýsledky musia byť celočíselné, pretože pôvodné i~koncové číslo je celé:
Z~celého čísla možno neceločíselný medzivýsledok dostať použitím druhého kúzla (v menovateľovi bude mocnina dvojky); s~danými typmi kúziel by však tiež všetky nasledujúce čísla neboli celé.
K~celému číslu sa dá z~neceločíselného medzivýsledku dostať použitím tretieho kúzla (v menovateľovi bola mocnina trojky); s~danými typmi kúziel by však tiež všetky predchádzajúce čísla nemohli byť celé.

Ku koncovému číslu 5 je možné dospieť dvojakým spôsobom, buď $10 \buildrel :2 \over \longmapsto 5$, alebo $6 \buildrel -1 \over \longmapsto 5$.
Medzivýsledok 10 však nemôže vzniknúť z ~ čísla, ktoré by nebolo väčšie ako 10.
Teda pätica pätiek môže vzniknúť jedine jedným použitím prvého kúzla na pätici šestiek:
$$\aligned
\cdots\ 6 \buildrel -1 \over \longmapsto 5 \\
\cdots\ 6 \buildrel -1 \over \longmapsto 5 \\
\cdots\ 6 \buildrel -1 \over \longmapsto 5 \\
\cdots\ 6 \buildrel -1 \over \longmapsto 5 \\
\cdots\ 6 \buildrel -1 \over \longmapsto 5 \\
\endaligned
$$
Tým je maximálny počet použití prvého kúzla vyčerpaný a~otázka znie, či je možné vykúzliť päticu šestiek len pomocou druhého a~tretieho kúzla, z ktorých žiadne nie je použité viac ako päťkrát a~žiadny medzivýsledok nie je väčší ako 10.

K~číslu 6 je možné dospieť jediným spôsobom, a~to $2 \buildrel \cdot3 \over \longmapsto 6$.
Teda pätica šestiek môže vzniknúť jedine jedným použitím tretieho kúzla na pätici dvojiek:
$$\aligned
\cdots\ 2 \buildrel \cdot3 \over \longmapsto 6 \buildrel -1 \over \longmapsto 5 \\
\cdots\ 2 \buildrel \cdot3 \over \longmapsto 6 \buildrel -1 \over \longmapsto 5 \\
\cdots\ 2 \buildrel \cdot3 \over \longmapsto 6 \buildrel -1 \over \longmapsto 5 \\
\cdots\ 2 \buildrel \cdot3 \over \longmapsto 6 \buildrel -1 \over \longmapsto 5 \\
\cdots\ 2 \buildrel \cdot3 \over \longmapsto 6 \buildrel -1 \over \longmapsto 5 \\
\endaligned
$$
Tým je maximálny počet použití tretieho kúzla vyčerpaný a~otázka znie, či je možné vykúzliť päticu dvojiek len pomocou druhého kúzla, ktoré nie je užité viac ako päťkrát a~žiadny medzivýsledok nie je väčší ako 10.
To však pre danú päticu čísel 3, 8, 9, 2, 4 nie je možné (problém s~3 a~9).

S~danými obmedzeniami päticu pätiek vykúzliť nemožno, kúzelníci by sa do tohto vystúpenia radšej púšťať nemali.}

{%%%%%   Z9-I-4
\napad
Oplatí sa zamerať na prvočísla.

\riesenie
Z~uvedenej rovnosti vyplýva, že medzi prvočíselnými činiteľmi $a$ a~$b$ musí byť 7 a~11.
Pretože hľadáme najmenšie vyhovujúce čísla $a$ a~$b$, môžeme predpokladať, že iných prvočiniteľov tieto čísla nemajú.
Teda hľadáme čísla tvaru
$$
a=7^*\cdot11^*,\qquad b=7^*\cdot11^* , \tag{$*$}
$$
kde hviezdičky predstavujú zatiaľ neznáme (kladné celé) mocnitele.
Najmenšie vyhovujúce čísla zodpovedajú najmenším vyhovujúcim mocniteľom.

Zadaná rovnosť platí, práve keď sa prvočíselní činitelia vyskytujú na oboch stranách v rovnakých počtoch (mocninách).
Preberieme možné činitele, budeme skúšať postupne od najmenších počtov, diskusiu začíname na pravej strane:
\begin{itemize}
\item Ak $b=7\cdot11^*$, potom na pravej strane je $7^5$ (a~nejaká mocnina 11).
Rovnakú mocninu 7 na ľavej strane však dostať nemožno: $7\cdot7^3=7^4 < 7^5$ a~$7\cdot(7^2)^3=7^7 > 7^5$.
\item Ak $b=7^2\cdot11^*$, potom na pravej strane je $(7^2)^5 =7^{10}$ (a~nejaká mocnina 11).
Rovnakú mocninu 7 na ľavej strane dostať možno: $7.(7^3)^3=7^{10}$.
\end{itemize}
Teda hľadané čísla sú tvaru
$$
a=7^3\cdot11^*,\qquad b=7^2\cdot11^*.
$$
\begin{itemize}
\item Ak $b=7^*\cdot11$, potom na pravej strane je $11\cdot11^5 =11^6$ (a~nejaká mocnina 7).
Rovnakú mocninu 11 na ľavej strane dostať je možné: $(11^2)^3 =11^6$.
\end{itemize}
Teda hľadané čísla sú
$$
a=7^3\cdot11^2=41\,503,\qquad b=7^2\cdot11=539. \tag{$**$}
$$

\poznamky
Pokiaľ neznáme mocnitele v~$(*)$ označíme ako
$$
a=7^k\cdot11^m,\qquad b=7^l\cdot11^n ,
$$
potom po dosiahnutí a~úprave rovnice zo zadania dostávame
$$\align
7\cdot(7^k\cdot11^m)^3 &= 11\cdot(7^l\cdot11^n)^5 , \\
7^{3k+1}\cdot11^{3m} &= 7^{5l}\cdot11^{5n+1} .
\endalign
$$
Porovnaním zodpovedajúcich činiteľov dostávame lineárne rovnice
$$
3k+1 = 5l, \qquad 3m = 5n+1.
$$
Najmenšie kladné celočíselné riešenia týchto rovníc sú $k=3$, $m=2$, $l=2$, $n=1$, čo zodpovedá $(**)$.

Pre zaujímavosť dodávame, že všetky celočíselné riešenia zadanej rovnice sú tvaru
$$
a=41\,503\cdot c^5,\qquad b=539\cdot c^3 ,
$$
kde $c$ je ľubovoľné celé číslo.
}

{%%%%%   Z9-I-5
\napad
Vyjadrite vzťahy zo zadania pomocou neznámych, zasnite sa a~hľadajte vzťahy ďalšie.

\res
Označme cenu sna, ilúzie, šlofíka a ~nočné mory postupne ako $s$, $i$, $\check s$
a~$m$.
Potom podľa zadania platí
$$
4s=7i+2\check s+m, \qquad
7s = 4i + 4 \check s + 2m. \tag{$*$}
$$

Šlofíkov a nočných môr je v druhej rovnici dvakrát toľko čo v prvej rovnici.
Vyjadrením týchto komodít z~prvej rovnice a~dosadením do druhej dostaneme vzťah medzi snami a ilúziami:
$$\aligned
4s-7i = 2\check s + m, \qquad 7s = 4i+2(2\check s+m), \\
7s &=4i +8s-14i, \\
s &=10i.
\endaligned
$$

Jeden sen teda stál desať ilúzií.

\poznamky
Zo zadania nie je možné určiť zvyšné neznáme, iba odvodzovať ich ďalšie vzťahy, presnejšie povedané inak vyjadrovať zadané vzťahy.
Pre dané rovnice, je možné akúkoľvek ich manipuláciou (použitú súčasne na obe ich strany) odvodiť opäť platnú rovnicu.
Pri riešení lineárnych rovníc môžeme eliminovať neznáme vhodnými lineárnymi kombináciami daných rovníc.
Náš prípad $(*)$ je špeciálny v~tom, že je možné eliminovať dve zo štyroch neznámych naraz:
rozdiel druhej rovnice od dvojnásobku prvej dáva
$$\aligned
8s-7s &=14i+4\check s +2m-4i-4\check s-2m, \\
s &=10i.
\endaligned
$$
To je len inak uchopený ten istý postreh ako v predchádzajúcom riešení.

Sústava rovníc $(*)$ má nekonečne mnoho riešení, ktoré je možné vyjadriť napr.
$$
s=10i,\qquad
m=33i-2\check s,
$$
kde $i$ a~$\check s$ môžu byť ľubovoľné čísla.
Najmä štvorica $i=0$, $s=0$, $\check s=0$, $m=0$ je riešením a~všetky riešenia sú lineárnymi kombináciami štvoríc $i=0$, $s=0$, $ \check s=1$, $m=-2$ a ~$i=1$, $s=10$, $\check s=0$, $m=33$.

Aj pre hádaním a pokusmi odhalené riešenie sústavy $(*)$ bude určite platiť $s=10i$.
Zovšeobecnenie typu
$$
\text{,,$i=1$, $s=10$, $\check s=0$, $m=33$ je riešením, teda $s=10i$``}
$$
však nemožno bez dodatočného vysvetlenia považovať za vyhovujúce riešenie úlohy.}

{%%%%%   Z9-I-6
\napad
Dala by sa z~polámanej uhlopriečky určiť tá skutočná?

\riesenie
Zo zadania je možné určiť uhlopriečku štvorca, odkiaľ už ľahko vyjadríme jeho obsah.

Uhlopriečku $DB$ interpretujeme ako preponu v~pravouhlom trojuholníku $DKB$, ktorého odvesny sú rovnobežné s~časťami danej lomenej čiary (doplnené pomocné štvoruholníky sú obdĺžniky):
\insp{z9-I-6a.eps}%

Odvesny majú dĺžky
$$\aligned
\vert{}DK\vert{} &=\vert{}DE\vert{}+\vert{}FG\vert{}+\vert{}HB\vert{} =6+4+2 =12\, (\Cm), \\
\vert{}KB\vert{} &=\vert{}EF\vert{}+\vert{}GH\vert{} =4+1 =5\,(\Cm).
\endaligned
$$
Podľa Pytagorovej vety platí
$$
|DB|^2 =|DK|^2+|KB|^2 =12^2+5^2 =169\,(\Cm^2).
$$

Obsah štvorca $ABCD$ je rovný polovici štvorca, ktorého strana je uhlopriečkou štvorca $ABCD$, tzn.
$$
S_{ABCD} =\frac12|DB|^2 =84{,}5\cm^2.
$$

\poznamka
Posledný vzťah je možné považovať za dobre známy a nie je nutné ho odvodzovať.
Názorne je možné tento vzťah znázorniť takto:
\insp{z9-I-6b.eps}%

Inak (z~Pytagorovej vety v~trojuholníku $BAD$) vieme, že $|DB|^2=2|AB|^2$, teda
$$
|AB|^2=\frac12|DB|^2 ,
$$
čo je práve obsah štvorca $ABCD$.}

{%%%%%   Z4-II-1
...}

{%%%%%   Z4-II-2
...}

{%%%%%   Z4-II-3
...}

{%%%%%   Z5-II-1
Rozdiel hmotností $35-18=17$ (kg) zodpovedá polovici mlieka bez kanvy.
Všetko mlieko bez kanvy váži $2\cdot17=34$ (kg).

Teda rozdiel $35-34=1$ (kg) zodpovedá prázdnej kanvi.
Prázdna kanva váži 1 kg.

\ineriesenie
Kanva s polovičným množstvom mlieka váži 18 (kg). Teda hmotnosť $2\cdot15=56$ zodpovedá hmotnosti všetkého mlieka a dvoch kanví.
Rozdiel $36-35=1$ zodpovedá prázdnej kanvi. Prázdna kanva váži 1 kg.

\hodnotenie
4~body za pomocné výpočty a úvahy (napr. pri prvom riešení 2 body za hmotnosť polovičného množstva mlieka a 2~body za hmotnosť všetkého mlieka bez kanvy);
2~body za doriešenie úlohy a~výsledok.
\endhodnotenie
}

{%%%%%   Z5-II-2
Celú miestnosť je možné rozdeliť na štvorce so stranou 10 metrov ako na obrázku.
Časti miestnosti, ktoré vidia jednotliví strážcovia, sú vyznačené šrafovaním dvojakého druhu.
Tá časť, na ktorú vidia obaja súčasne, je teda šrafovaná dvojito:
\insp{z5-II-2a.eps}%

Táto časť pozostáva zo 4 štvorcov, celá miestnosť zo 16 štvorcov. Celá miestnosť je teda štyrikrát väčšia ako časť, kam môžu dohliadnuť obaja strážci.

\hodnotenie
2~body za znázornenie časti videnej oboma strážcami;
2~body za pomocné delenie miestnosti;
2~body za doriešenie úlohy a~výsledok.

\poznamka
Celú miestnosť je možné rozdeliť na hrubšie časti podľa toho, či sú pod kontrolou niektorého alebo oboch strážcov.
To sú obdĺžniky s~rozmermi $30\x20$, $20\x30$ a~štvorec $20\x20$ (všetko v~metroch).
Celá miestnosť zodpovedá obdĺžniku s rozmermi $80\x20$, dvojito šrafovaná časť má rozmery $20\x20$, čo je práve štvrtina celku.
\endhodnotenie
}

{%%%%%   Z5-II-3
Párne čísla majú na mieste jednotiek párnu číslicu, tj 0, 2, 4, 6 alebo 8. Pokiaľ číslica na mieste desiatok má byť aspoň o 4 väčšia ako číslica na mieste jednotiek, nemôže byť na mieste jednotiek ani 6 ani 8 (po pripočítaní 4 by sme dostali číslo väčšie ako 9). Na mieste jednotiek môže byť len 0,2 alebo 4.

Ak je na mieste jednotiek číslica 0, musí byť podľa štvrtej podmienky na mieste desiatok číslica väčšia alebo rovná 4. Aby platila podmienka o súčte číslic, musí byť na mieste desiatok aspoň číslica 7. Vyhovujúce možnosti sú
$$
90,\quad 80,\quad 70;\quad
$$

Ak je na mieste jednotiek číslica 2, musí byť podľa štvrtej podmienky na mieste desiatok číslica väčšia alebo rovná 6. Tým je splnená aj tretia podmienka o súčte číslic. Vyhovujúce možnosti sú
$$
92,\quad 82,\quad 72,\quad 62;\quad
$$

Ak je na mieste jednotiek číslica 4, musí byť podľa štvrtej podmienky na mieste desiatok číslica väčšia alebo rovná 8. Tým je splnená aj tretia podmienka o súčte číslic. Vyhovujúce možnosti sú
$$
94,\quad 84.
$$
Hore uvedených 9 čísel predstavuje všetky riešenia úlohy, keďže sme vyčerpali už všetky možnosti na skúšanie.

\hodnotenie
2~body za obmedzenie číslic na mieste jednotiek;
2~body za ďalšie čiastkové pozorovania;
2~body za úplnosť odpovede.
\endhodnotenie
}

{%%%%%   Z6-II-1
Postupne preberieme možnosti vzhľadom na počet detí:
\begin{itemize}
\item Ak by mala 1 dieťa, potom by cesto malo vychádzať na $1\cdot3+2=5$ a súčasne na $1\cdot4-1=3$ rožky.
Tieto hodnoty sú rôzne, teda 1 dieťa nemala.
\item Ak by mala 2 deti, potom by cesto malo vychádzať na $2\cdot3+2=8$ a súčasne na $2\cdot4-1=7$ rožkov.
Tieto hodnoty sú rôzne, teda 2 deti nemala.
\item Ak by mala 3 deti, potom by cesto malo vychádzať na $3\cdot3+2=11$ a súčasne na $3\cdot4-1=11$ rožky.
Tieto hodnoty sú rovnaké, teda mala 3 deti.
\end{itemize}

Pre väčšie počty detí rovnosť nenastane, lebo rozdiel porovnávaných hodnôt sa bude postupne zväčšovať.
Mamička piekla rožky pre svoje 3 deti.

\poznamka
Predchádzajúci rozbor možností je možné prehľadne zapísať takto ($d$ značí počet detí):
$$
\begintable
$d$\|1|2|\bf 3|4|5|\dots\crthick
$3d+2$\|5|8|\bf 11|14|17|\dots\cr
$4d-1$\|3|7|\bf 11|15|19|\dots\endtable
$$
K~vyhovujúcej možnosti je možné dospieť riešením rovnice
$$
3d+2 = 4d-1 .
$$

\hodnotenie
3~body za rozbor možností pre rôzne počty detí či zostavenie rovnice;
3~body za správny výsledok.
\endhodnotenie
}

{%%%%%   Z6-II-2
Stanov, príp. Janin výpočet môžeme prehľadne zapísať takto:
$$
\vbox{
\alggg{&a,&b&c \\ d&e,&f&}{5&0,&1&3}
}
\qquad
\vbox{
\alggg{a&b,&c \\ &d,&e&f&}{3&4,&0&2}
}
$$

Porovnaním hodnôt na mieste stotín v~prvom, príp. druhom výpočte dostávame $c=3$, príp. $f=2$.
Predchádzajúce výpočty môžeme vyjadriť takto:
$$
\vbox{
\alggg{&a,&b&3 \\ d&e,&2&}{5&0,&1&3}
}
\qquad
\vbox{
\alggg{a&b,&3 \\ &d,&e&2&}{3&4,&0&2}
}
$$
Porovnaním hodnôt na mieste desatín v~prvom, príp. druhom výpočte dostávame $b=9$, príp. $e=7$.
Predchádzajúce výpočty môžeme vyjadriť nasledovne, pričom máme na pamäti, že v obidvoch prípadoch dochádza k prechodu cez desiatku:
$$
\vbox{
\alggg{&a,&9&3 \\ d&7,&2&}{5&0,&1&3}
}
\qquad
\vbox{
\alggg{a&9,&3 \\ &d,&7&2&}{3&4,&0&2}
}
$$
Porovnaním hodnôt na mieste jednotiek v~prvom, príp. druhom výpočte dostávame $a=2$, príp. $d=4$.
Po doplnení zisťujeme, že rovnosť vládne aj na mieste desiatok:
$$
\vbox{
\alggg{&2,&9&3 \\ 4&7,&2&}{5&0,&1&3}
}
\qquad
\vbox{
\alggg{2&9,&3 \\ &4,&7&2&}{3&4,&0&2}
}
$$

Pôvodné trojciferné čísla boli 293 a~472.
Hľadaný súčet je
$$
\alggg{2&9&3 \\ 4&7&2}{7&6&5}
$$

\hodnotenie
1~bod za vhodný zápis problému;
3~body za čiastkové kroky v~určovaní neznámych číslic;
2~body za dopočítanie a~výsledok.
\endhodnotenie
}

{%%%%%   Z6-II-3
Každá kocka má šesť rovnakých stien.
Celkový povrch všetkých (nezlepených) kociek je
$$
6\cdot(1+4+9+16+25) =6\cdot55 = 330\ \text{štvorčekov}.
$$

Po lepení a~natieraní sú na každej kocke neofarbené celá dolná stena a~na hornej stene plocha zodpovedajúca stene susednej menšej kocky.
Neofarbené plochy teda zodpovedajú
$$
2\cdot(1+4+9+16)+25 =2\cdot30+25 = 85\ \text{štvorčekom}.
$$

Zafarbený povrch kociek zodpovedá $330-85=245$ štvorčekom.
Na ich zafarbenie Zuzke stačí $245:5=49$ vedierok farby.

\poznamky
Zafarbené plochy na jednotlivých kockách zodpovedajú postupne
5, 19, 41, 71 a ~ 109 štvorčekom
(tieto hodnoty sú vypočítané postupne ako $6-1$, $5\cdot4-1$, $5\cdot9-4$, $5\cdot16-9$ a ~$5\cdot25-16$).
Naozaj platí, že
$$
5+19+41+71+109=245 .
$$

Z~každej kocky sú zafarbené štyri bočné steny a~časť hornej steny.
Zafarbené časti horných stien dohromady zodpovedajú stene najväčšej kocky (bez ohľadu na umiestnenie jednotlivých kociek), čo je dobre viditeľné pri pohľade na vežu zhora:
\insp{z6-II-3a.eps}%

\noindent
K~uvedenému výsledku sa teda dá dopočítať aj takto:
$$
4\cdot(1+4+9+16+25)+25 =4\cdot55+25 = 245 .
$$

\hodnotenie
3~body za čiastočné pozorovania a~medzivýsledky;
2~bod za celkový zafarbený povrch;
1~bod za počet vedierok farby.
\endhodnotenie
}

{%%%%%   Z7-II-1
Celú miestnosť je možné rozdeliť na štvorce so stranou 10 metrov ako na obrázku.
Časti miestnosti, ktoré nevidia jednotliví strážcovia, sú vyznačené šrafovaním dvojakého druhu.
Tá časť, na ktorú nedohliadne ani jeden, je teda šrafovaná dvojito:
\insp{z7-II-1a.eps}%

Hraničné čiary šrafovania sú uhlopriečkami štvorca z~pomocného delenia.
Dvojito šrafovaná časť pozostáva z dvoch a~štvrť alebo $\frac94$ štvorca.
Plocha, na ktorú nevidí ani jeden zo strážcov, je
$$
\frac94\cdot100 =9\cdot25 =225\,(\text{m}^2).
$$

\poznamka
Celá miestnosť pozostáva z 13 štvorcov pomocného delenia.
Iná správna odpoveď na otázku zo zadania je, že
časť miestnosti, na ktorú nevidí ani jeden zo strážcov, tvorí
$
\frac{9}{4\cdot13} = \frac{9}{52}
$
celej miestnosti.

\hodnotenie
3~body za znázornenie príslušnej časti a~pomocné delenie;
3~body za doriešenie úlohy a~výsledok (v m$^2$ alebo ako zlomok).
\endhodnotenie
}

{%%%%%   Z7-II-2
Súčet vekov všetkých potomkov je
$$
2\cdot63 +4\cdot35 +5\cdot4 =286\ \text{rokov}.
$$
Ich priemerný vek je $286:11=26$ rokov.
Teda Anna má $3{,}5\cdot26=91$ rokov.

Za päť rokov bude mať Anna 96 rokov.
Aby priemerný vek Anny a Jozefa vtedy bol 100 rokov, Jozef musí mať za päť rokov presne 104 rokov.
Jozef má teraz teda 99 rokov.

\poznamka
Informáciu o ~priemernom veku Anny a ~Jozefa možno zapísať takto:
$$
\aligned
(91+5+j+5):2 &=100 , \\
%101+j &=200, \\
%j &= 99.
\endaligned
$$
kde $j$ značí aktuálny vek Jozefa.
Odtiaľ po úprave vyplýva, že $j=99$.

\hodnotenie
3~body za priemerný vek všetkých potomkov;
1~bod za Annin vek;
2~body za Jozefov vek.
\endhodnotenie
}

{%%%%%   Z7-II-3
Názvy tímov skracujeme na A, B, C a ~ D.
Každý tím hral tri zápasy so zvyšnými tímami, za každý zápas mohol získať 0, 1 alebo 3 body.
Rozdelenie bodov pre tímy A, B a ~ D je teda dané jednoznačne, pre tím C sú dve možnosti:
$$
7=3+3+1,\quad
4=3+1+0,\quad
3=3+0+0=1+1+1,\quad
2=1+1+0.
$$

Jeden bod vždy dostávajú dva tímy, teda celkový počet jednotiek v predchádzajúcich rozkladoch musí byť párny.
V~rozkladoch 7, 3 a~2 (teda v~ziskách tímov A, B a~D) sú celkom štyri jednotky.
Preto tím C svoje 3 body musel získať takto:
$$
3=3+0+0.
$$
Odtiaľ taktiež vyplýva, že remízou skončili dva zápasy.

\smallskip
Tím D remizoval dvakrát a~tímy A~a~B raz.
Teda tím D remizoval s tímami A~a~B.

Tím A~ raz remizoval a~dvakrát vyhral.
Pritom remizoval s~tímom D, teda vyhral nad tímami B a~C.

Tím B raz prehral, raz remizoval a raz vyhral.
Pritom prehral s tímom A~a~remizoval s tímom D, teda vyhral nad tímom C.

\poznamka
Z~uvedeného je možné zostaviť tabuľku výsledkov všetkých zápasov:
$$
\begintable
\|A|B|C|D\|dokopy\crthick
A\|$-$|3|3|1\|7\cr
B\|0|$-$|3|1\|4\cr
C\|0|0|$-$|3\|3\cr
D\|1|1|0|$-$\|2\endtable
$$

Odpoveď na otázku a) možno odvodiť aj takto:
Celkom bolo v~turnaji rozdelených $7+4+3+2=16$ bodov.
Pokiaľ by žiadny zápas neskončil remízou, bolo by celkom rozdelených $6\cdot3=18$ bodov.
Každá remíza do celkového súčtu prispieva dvoma bodmi (tj o jeden bod menej ako výhra), teda remízou skončili dva zápasy.

\hodnotenie
Po 2~bodoch za odpoveď na každú z~otázok a) a~b);
2~body za kvalitu komentára.
\endhodnotenie
}

{%%%%%   Z8-II-1
Cenu člna v zlatkách označíme $z$.
Pankrác zaplatil $\frac6{10}z$, zostávalo doplatiť
$$
\left (1-\frac6{10} \right)z =\frac4{10}z .
$$
Servác zaplatil $\frac4{10}\cdot\frac4{10}z = \frac{16}{100}z$, zostávalo doplatiť
$$
\left( \frac4{10}-\frac{16}{100} \right) z = \frac{24}{100}z =  \frac{6}{25}z .
$$
Chýbajúcu čiastku zaplatil Bonifác, čo predstavovalo 30 zlatiek.
Teda $\frac{6}{25}z =30$, odkiaľ vyplýva, že $z=125$.
Čln stál 125 zlatiek.

\poznamka
Predchádzajúce úvahy možno znázorniť nasledovne (celok predstavuje cenu člna, obdĺžnikmi sú vyznačené podiely jednotlivých prispievateľov):
\insp{z8-II-1.eps}%

\hodnotenie
2~body za vyjadrenie Pankrácovho príspevku a zvyšku;
2~body za vyjadrenie Servácovho príspevku a zvyšku;
2~body za dopočítanie a~odpoveď.
\endhodnotenie
}

{%%%%%   Z8-II-2
Úsečky $KC$ a~$CL$ sú časťami uhlopriečok vo štvorcoch, preto uhly $KCA$ a~$LCB$ majú veľkosť 45\st.
Uhol $ACB$ je pravý, teda uhol $KCL$ má veľkosť
$$
45\st+90\st+45\st=180\st .
$$
Odtiaľ vyplýva, že bod $C$ je vnútorným bodom úsečky $KL$.
\insp{z8-II-2a.eps}%

V~ďalšom predpokladáme také značenie vrcholov, že $|AC|:|CB|=1:3$.
Veľkosť odvesny $AC$ označíme $b$.
S~týmto značením je obsah pravouhlého trojuholníka $ABC$ vyjadrený ako
$$
\frac12(b\cdot3b)=\frac32 b^2 .
$$

Zo zadania vyplýva, že úsečky $KM$ a~$ML$ sú (predĺžené) stredné priečky trojuholníka $ABC$.
Teda trojuholník $KLM$ je tiež pravouhlý a navyše rovnoramenný, s veľkosťami ramien $\frac12b+\frac32b=2b$.
Obsah trojuholníka $KLM$ je tak rovný
$$
\frac12(2b\cdot2b) =2b^2.
$$
Pomer obsahov trojuholníkov $ABC$ a~$KLM$ je
$$
\frac32 b^2 : 2b^2 = 3 : 4 .
$$
\insp{z8-II-2b.eps}%

\hodnotenie
Po 2~bodoch za odpoveď na každú z~otázok a) a~b);
2~body za kvalitu komentára.

\poznamky
Namiesto predchádzajúceho značenia $|AC|=b$, $|AB|=3b$ atď. si možno vypomôcť s~dodatočným delením, resp. znázornením v štvorčekovej sieti:
\insp{z8-II-2c.eps}%

Ako kolineárnosť bodov $K$, $C$, $L$, tak pravouhlosť a~rovnoramennosť trojuholníka $KLM$ je platná pre všeobecný pravouhlý trojuholník $ABC$.
Daný pomer dĺžok odvesien ovplyvňuje iba pomer obsahov trojuholníkov $ABC$ a~$KLM$.

\endhodnotenie
}

{%%%%%   Z8-II-3
Karolína napísala čísla
$$
123, \quad 321.
$$
Nikola napísala čísla
$$
456, \quad 465, \quad 546, \quad 564, \quad 645, \quad 654.
$$
Obe čísla od Karolíny sú nepárne.
Pre párny súčet musel teda Jakub vybrať nepárne číslo od Nikoly.
Párny súčet dávajú práve tieto prípady:
$$
123+465, \quad 123+645, \quad 321+465, \quad 321+645 .
\tag{$*$}
$$
Vo všetkých prípadoch je číslica v súčine na mieste jednotiek určená nepárnym násobkom 5, teda to môže byť jedine 5.

\hodnotenie
Po 1~bode za vypísanie Karolínin a~Nikolinin čísel;
2~body za Kubov výber párneho súčtu;
2~body za určenie poslednej číslice súčinu.

\poznamka
K správnemu záveru nie je potrebné súčiny vyčíslovať.
Avšak pre štvoricu možností v~$(*)$ sú tieto súčiny postupne
$$
57\,195, \quad 79\,335, \quad 149\,265, \quad 207\,045 .
$$
\endhodnotenie
}

{%%%%%   Z9-II-1
Hľadané čísla majú mať deliteľa 1, 2, 3, 4, 5, 6, 7, 8, 9, teda musia byť deliteľné ich najmenším spoločným násobkom:
$$
5\cdot7\cdot8\cdot9=2520.
$$

Ak budeme odteraz skúšať násobky čísla 2520, budú mať tiež tých istých jednociferných deliteľov. Odlišný však bude počet ich štvorciferných deliteľov.

Ako prvé vyskúšame priamo číslo 2520. Číslo 2520 je štvorciferné, ale má len dvoch štvorciferných deliteľov, a to
$$
2520 =2520\cdot1 =1260\cdot2;
$$
a najbližším menším deliteľom je 840.

Ďalším násobkom čísla 2520 je 5040.
Toto číslo je štvorciferné a ~ má práve päť štvorciferných deliteľov, a to
$$
5040 =5040\cdot1 =2520\cdot2 =1680\cdot3 =1260\cdot4 =1008\cdot5.
$$

Ďalším násobkom čísla 2520 je 7560.
Toto číslo je štvorciferné, ale má viac ako päť štvorciferných deliteľov:
$$
7560 =7560\cdot1 = 3780\cdot2 =2520\cdot3 =1890\cdot4 =1512\cdot5 =1260\cdot6 =1080\cdot7 .
$$

Ďalšie násobky položky 2520 už nie sú štvorciferné.
Jediným riešením úlohy je číslo 5040.

\hodnotenie
2~body za najmenší spoločný násobok jednociferných deliteľov;
2~body za postupný rozbor možností;
2~body za záver a~kvalitu komentára.
\endhodnotenie}

{%%%%%   Z9-II-2
Obsah celého trojuholníka $ABC$ je
$$
\frac12\cdot24\cdot 25=300\,(\Cm^2).
$$

Rovnobežky so stranou $AB$ prechádzajúce bodmi $O$, $P$, $Q$ $R$ a~rovnobežky so stranou $BC$ prechádzajúce bodmi $K$, $L$, $M$, $N$, rozdeľujú trojuholník $ABC$ na menšie navzájom zhodné trojuholníky.
Vzájomná zhodnosť trojuholníkov plynie z konštrukcie a základných viet o zhodnostiach trojuholníkov ({\it sus\/}, {\it sss\/}, {\it usu\/}).
Najmenšími dielikmi v~tomto delení sú trojuholníky zhodné s~trojuholníkom $NBO$, a~tých je dokopy 25.
\insp{z9-II-2a.eps}%

Obsah trojuholníka $NBO$ je $\frac1{25}\cdot300=12\,(\Cm^2)$.
Lichobežník $KLQR$, resp. $MNOP$ pozostáva zo 7, resp. z 3 takých trojuholníkov.
Súčet obsahov týchto lichobežníkov je teda
$$
(7+3)\cdot12 =120\,(\Cm^2) .  \tag{1}
$$

\ineriesenie
Trojuholník $ABC$ doplníme do rovnobežníka $ABDC$.
Štvorica rovnobežiek po predĺžení delí rovnobežník na päť zhodných rovnobežníkov:
\insp{z9-II-2b.eps}%

Obsah celého rovnobežníka $ABDC$ je
$$
24\cdot25=600\,(\Cm^2).
$$
Súčet obsahov dvoch rovnobežníkov vyfarbených sivou je
$$
\frac25\cdot600 =240\,(\Cm^2).
$$

Trojuholníky $ABC$ a~$DCB$ sú zhodné, a to vrátane vyfarbených častí.
Teda súčet obsahov lichobežníkov $KLQR$ a~$MNOP$ je rovný polovici súčtu obsahov sivých rovnobežníkov:
$$
\frac12\cdot240 =120\,(\Cm^2) .  \tag{2}
$$

\ineriesenie
Trojuholníky $ABC$, $KBR$, $LBQ$, $MBP$ a~$NBO$ sú navzájom podobné, pretože majú spoločný uhol vo vrchole $B$ a~k~nemu protiľahlé strany sú navzájom rovnobežné.
Koeficienty podobnosti medzi prvými a zvyšnými štyrmi trojuholníkmi sú postupne $\frac45$, $\frac35$, $\frac25$ a~$\frac15$.
\insp{z9-II-2c.eps}%

Obsah celého trojuholníka $ABC$ je
$$
\frac12\cdot24\cdot 25=300\,(\Cm^2).
$$
Obsahy trojuholníkov $NBO$, $MBP$, $LBQ$, $KBR$ sú postupne
$$\aligned
&\frac12\left(\frac15\cdot24\right)\left(\frac15\cdot25\right) =12\,(\Cm^2) , \\
&\frac12\left(\frac25\cdot24\right)\left(\frac25\cdot25\right) =48\,(\Cm^2) , \\
&\frac12\left(\frac35\cdot24\right)\left(\frac35\cdot25\right) =108\,(\Cm^2) , \\
&\frac12\left(\frac45\cdot24\right)\left(\frac45\cdot25\right) =192\,(\Cm^2) .
\endaligned
$$

Obsah lichobežníka $KLQR$ je rozdielom obsahov trojuholníkov $KBR$ a~$LBQ$,
obsah lichobežníka $MNOP$ je rozdielom obsahov trojuholníkov $MBP$ a~$NBO$.
Súčet obsahov týchto dvoch lichobežníkov je teda
$$
(192-108) +(48-12) =120\,(\Cm^2) .   \tag{3}
$$

\hodnotenie
2~body za počiatočné úvahy (delenie, doplnenie, resp. zhodnosti/podobnosti);
2~body za pomocné výpočty a~výsledok;
2~body za kvalitu komentára.

\poznamka
Vo všetkých uvedených riešeniach možno namiesto postupného vyčíslovania obsahov pracovať s~výrazmi vyjadrujúcimi závislosť na obsahu trojuholníka $ABC$ a~hodnoty zo zadania dosadzovať až nakoniec.
Pokiaľ obsah trojuholníka $ABC$ označíme $S$, potom výpočty (1) a~(2) zodpovedajú postupne
$$
\frac{7+3}{25} S =\frac25 S , \quad
\frac12\cdot\frac25\cdot 2S =\frac25 S
$$
a~vzťah (3) je možné vyjadriť ako
$$
\left(\left(\frac45\right)^2-\left(\frac35\right)^2 + \left(\frac25\right)^2-\left(\frac15\right)^2\right)S
=\frac{16-9+4-1}{25} S =\frac25 S .
$$
\endhodnotenie
}

{%%%%%   Z9-II-3
Pri riešení úlohy si vystačíme so známymi poznatkami o~súčtoch a~súčinoch párnych ($P$) a~nepárnych ($N$) čísel, ktoré v skratke pripomíname v~nasledujúcich tabuľkách:
$$
\hskip-0.167\hsize
\begintable
$+$\|$P$|$N$\crthick
$P$\|$P$|$N$\cr
$N$\|$N$|$P$\endtable
\hskip-.667\hsize
\begintable
$\cdot$\|$P$|$N$\crthick
$P$\|$P$|$P$\cr
$N$\|$P$|$N$\endtable
$$

Prvý výraz $2023n$ je súčinom nepárneho čísla s~$n$, teda výsledok závisí od parity $n$.
Hodnota tohto výrazu teda nie je vždy nepárna (napr. pre $n=2$ dostávame 4046).

V druhom výraze je $n^2+n$ vždy párne číslo: pre $n$ párne, resp. nepárne sa jedná o súčet dvoch párnych, resp. dvoch nepárnych čísel.
Zostávajúci sčítanec je nepárne číslo, teda pre akékoľvek $n$ je hodnota výrazu $n^2+n+23$ vždy nepárna.

Tretí výraz $3n^3$ je súčinom nepárneho čísla s~$n^3$, teda výsledok závisí od parity $n$.
Hodnota tohto výrazu teda nie je vždy nepárna (napr. pre $n=2$ dostávame 24).

Vo štvrtom výraze je $10n^2$ vždy párne číslo, pretože 10 je párne číslo.
Zostávajúci sčítanec je nepárne číslo, teda pre akékoľvek $n$ je hodnota výrazu $10n^2+2023$ vždy nepárna.

Nepárnotvorné sú teda výrazy $n^2+n+23$ a $10n^2+2023$.

\hodnotenie
Po 1~bode za každú správnu a~odôvodnenú odpoveď;
2~body za celkovú kvalitu komentára.
\endhodnotenie

\poznamka
Vyššie diskutovaná nepárnosť časti druhého výrazu vyplýva aj z nasledujúcej úpravy: $n^2+n=n(n+1)$. Na pravej strane je súčin dvoch po sebe idúcich čísel, teda súčin párneho a nepárneho čísla (alebo naopak).
}

{%%%%%   Z9-II-4
Súčet veľkostí vnútorných uhlov v~každom trojuholníku je $180\st$, štvoruholníku $360\st$ atď.
Všeobecne platí, že každý $n$-uholník je možné zložiť z~$n-2$ trojuholníkov, teda súčet jeho vnútorných uhlov je $(n-2)\cdot180\st$.
\insp{z9-II-4a.eps}%

Súčet veľkostí doplnkových uhlov všeobecného $n$-uholníka je
$$
n\cdot360\st-(n-2)\cdot180\st=(n+2)\cdot180\st.
$$
Pomer týchto dvoch hodnôt je $(n-2):(n+2)$, čo má podľa zadania byť $3:5$.
Úpravami dostávame:
$$\aligned
\frac{n-2}{n+2} &=\frac35 , \\
5n-10 &=3n+6 , \\
2n &=16, \\
n &=8.
\endaligned
$$
Neznámy mnohouholník je osemuholník.

\ineriesenie
Súčty veľkostí vnútorných, resp. doplnkových uhlov sú vo všetkých mnohouholníkoch s rovnakým počtom vrcholov rovnaké.
Preto sa stačí zaoberať (napr.) len pravidelnými mnohouholníkmi.

Pravidelný $n$-uholník je možné zložiť z~$n$ navzájom zhodných rovnoramenných trojuholníkov.
Vnútorný uhol pri~hlavnom vrchole trojuholníka má veľkosť $\frac1n 360\st$, súčet vnútorných uhlov pri~základni sa rovná vnútornému uhlu pravidelného $n$-uholníka a~má veľkosť $180\st-\frac1n\cdot360\st$.
Doplnkový uhol pravidelného $n$-uholníka má teda veľkosť
$
360\st - (180\st-\frac1n\cdot360\st) =180\st+\frac1n\cdot360\st.
$
\insp{z9-II-4b.eps}%

Súčty veľkostí vnútorných a vonkajších uhlov sú postupne
$$\aligned
&n\Big(180\st-\frac1n\cdot360\st\Big) =(n-2)\cdot180\st, \\
&n\Big(180\st+\frac1n\cdot360\st\Big) =(n+2)\cdot180\st.
\endaligned
$$
Pomer týchto dvoch hodnôt je $(n-2):(n+2)$, čo má byť $3:5$.
Odtiaľ, rovnakými úpravami ako vyššie, dostávame $n=8$.
Neznámy mnohouholník je osemuholník.

\hodnotenie
2~body za prípravné vyjadrenia v~závislosti na $n$;
2~body za zostavenie a~doriešenie rovnice;
2~body za kvalitu komentára.
\endhodnotenie

\ineriesenie
Rovnako ako v predchádzajúcom riešení sa zameriame len na pravidelné mnohouholníky, resp. ich delenie na navzájom zhodné rovnoramenné trojuholníky, viď ilustrácie vyššie.

V~pravidelnom mnohouholníku je pomer súčtov veľkostí vnútorných a~doplnkových uhlov rovnaký ako pomer veľkostí týchto uhlov pri~každom jeho vrchole.
Tento pomer je $3:5$ práve vtedy, keď vnútorný uhol neznámeho mnohouholníka má veľkosť
$$
\frac38\cdot360\st =135\st
$$
($3+5=8$ dielov zodpovedá plnému uhlu).
Uhol pri~hlavnom vrchole pomocného rovnoramenného trojuholníka, t.j. stredový uhol mnohouholníka, je $45\st$
(aby súčet vnútorných uhlov trojuholníka bol $180\st$).
Tento uhol je osmina plného uhla.
Neznámy mnohouholník je osemuholník.

\poznamka
Predchádzajúci nápad je možné spracovať postupným vyjadrovaním stredového, vnútorného, resp. doplnkového uhla pravidelného $n$-uholníka v~závislosti na $n$ a~kontrolou požadovaného pomeru:
$$
\begintable
$n$\|3|4|5|6|7|8|\dots\crthick
stredový\|$120\st$|$90\st$|$72\st$|$60\st$|$\frac17 360\st$|$45\st$|\dots\cr
vnútorný\|$60\st$|$90\st$|$108\st$|$120\st$|$\frac17 900\st$|$135\st$|\dots\cr
doplnkový\|$300\st$|$270\st$|$252\st$|$240\st$|$\frac17 1620\st$|$225\st$|\dots\crthick
pomer\|$1:5$|$1:3$|$3:7$|$1:2$|$5:9$|$\boldmath{3:5}$|\dots\endtable
$$
Z~geometrickej predstavy vyplýva, že pre zväčšujúce sa $n$ hodnota pomeru vnútorného a~doplnkového uhla postupne rastie k~$1:1$.
Teda pokiaľ má úloha riešenie, je toto riešenie jediné.
\hodnotenie
Po 2~bodoch za veľkosti vnútorného a~stredového uhla pravidelného mnohouholníka;
2~body za doriešenie a~kvalitu komentára.

Pri postupnom skúšaní zohľadne úplnosť komentára.
Náhodne odhalené nezdôvodnené riešenie hodnoťte 2~body.
\endhodnotenie
}

{%%%%%   Z9-III-1
Matematickú olympiádu (MO) riešia všetky deti, ktoré majú rady matematiku, ale iba desatina tých, ktoré matematiku rady nemajú.

Dajme tomu, že $x$\,\% detí má rado matematiku, a teda rieši MO.
Potom $(46-x)$\,\% detí matematiku rado nemá a~tiež rieši MO.
Všetkých detí, ktoré matematiku rady nemajú, je desaťkrát viac, tj. $10(46-x)$\,\%.
Platí teda:
$$
x+10(46-x) =100 ,
$$
odkiaľ jednoduchými úpravami dostávame $x=40$.

Detí, ktoré majú rady matematiku, je 40\,\%.

\poznamky
Iné značenie vedie k iným úpravám.
Dajme tomu, že $y$\,\% detí nemá rado matematiku.
Potom $\frac9{10}y\,\%$ detí nerieši MO.
To zodpovedá 54\,\% detí $(100-46=54)$.
Teda platí:
$$
\frac9{10}y =54 ,
$$
odkiaľ jednoduchými úpravami dostávame $y=60$.

Uvedené vzťahy je možné znázorniť pomocou Vennovho diagramu takto:
\insp{z9-III-1.eps}%

\hodnotenie
2~body za pomocné značenia a~čiastkové pozorovania;
2~body za výsledok;
2~body za kvalitu komentára.
\endhodnotenie
}

{%%%%%   Z9-III-2
Vrcholy trojuholníka označíme $C$, $D$, $E$ a~vzdialenosti zo zadania priradíme (bez ujmy na všeobecnosti) veľkostiam úsečiek $BC$ a~$BD$.
Keďže $1+2=3$, leží bod $B$ na priamke $CD$.
Vzdialenosť $B$ od tretieho vrchola $E$ je preponou vyznačeného pravouhlého trojuholníka $BFE$:
\insp{z9-III-2.eps}%

Úsečka $EF$ je výškou rovnostranného trojuholníka $CDE$, teda
$$
|EF|=\frac{\sqrt3}2\cm ;
$$
to je známy vzťah, ktorý je možné nájsť v ~tabuľkách, príp. spočítať pomocou Pytagorovej vety v~trojuholníku $CFE$.
Bod $F$ je stredom úsečky $CD$, teda
$$
|BF|=|BC|+|CF|=\frac52\cm.
$$
Veľkosť prepony trojuholníka $BFE$ je podľa Pytagorovej vety rovná
$$
|BE|
%=\sqrt{|EF|^2+|FE|^2}
=\sqrt{\left (\frac52\right )^2+\left (\frac{\sqrt3}2\right )^2}
= \sqrt{\frac{28}4}
= \sqrt 7\,(\Cm) .
$$
Vzdialenosť bodu $B$ od tretieho vrcholu trojuholníka je $\sqrt 7$\,cm.

\hodnotenie
1~bod za zistenie, že $B$ leží na priamke $CD$;
3~body za dopočítanie $|BE|$;
2~body za kvalitu komentára.
\endhodnotenie
}

{%%%%%   Z9-III-3
Celá tabuľka je určená číslami na prvom riadku.
Prvých niekoľko riadkov tabuľky vyzerá takto:
$$
\begintable
1\|$a$|$b$|$c$|$d$\crthick
2\|$a+b$|$a-b$|$c+d$|$c-d$\cr
3\|$2a$|$2b$|$2c$|$2d$\cr
4\|$2(a+b)$|$2(a-b)$|$2(c+d)$|$2(c-d)$\cr
5\|$4a$|$4b$|$4c$|$4d$\cr
6\|$4(a+b)$|$4(a-b)$|$4(c+d)$|$4(c-d)$\cr
7\|$8a$|$8b$|$8c$|$8d$\cr
8\|$8(a+b)$|$8(a-b)$|$8(c+d)$|$8(c-d)$\cr
9\|$16a$|$16b$|$16c$|$16d$\cr
$\vdots$\||||\endtable
$$

a) Súčet čísel na treťom riadku je dvojnásobkom súčtu čísel na prvom riadku.
Ak je súčet prvého riadka rovný 0, je súčet tretieho riadka tiež rovný 0.

\smallskip

b)Súčty čísel na ďalších nepárnych riadkoch sa postupne zdvojnásobujú:
súčet 5. riadku je 4-násobkom súčtu čísel prvého riadka ($2\cdot2=4=2^2$),
súčet 7. riadku je 8-násobkom súčtu čísel prvého riadka ($2\cdot4=8=2^3$),
súčet 9. riadku je 16-násobkom súčtu čísel prvého riadku ($2\cdot8=16=2^4$)
atď.

Kladné nepárne čísla sú v tvare $2i+1$, kde $i=0,1,2,3,4, \dots$
Predchádzajúce pozorovania sú pri tomto značení zovšeobecnené nasledovne:
súčet $(2i+1)$-tého riadka je $2^i$-násobkom súčtu prvého riadka.

Keďže $25=2\cdot12+1$, je súčet 25. riadku rovný $2^{12}$-násobku prvého riadku.
Pokiaľ je súčet prvého riadku rovný 1, je súčet 25. riadku rovný
$$
2^{12} = 8^4 = 64^2 =4\,096 . \tag{$*$}
$$

\hodnotenie
1~bod za prípravné pozorovania (niekoľko riadkov tabuľky);
1~bod za odpoveď na otázku a);
2~body za odpoveď na otázku b);
2~body za kvalitu komentára.

\poznamky
Akýkoľvek z~výrazov v~($*$), či iné ekvivalentné vyjadrenie, považujte za správnu odpoveď na otázku b).

Taktiež pre párne riadky platí, že sa ich súčty postupne zdvojnásobujú.
Tieto súčty sú násobkom $a+c$, teda súčtu dvoch čísel z~prvého riadka.
\endhodnotenie
}

{%%%%%   Z9-III-4
Lichobežníkov s uvedenými vlastnosťami je nekonečne veľa.
Všetky majú rovnaké veľkosti základní, menia sa ich výšky.
Táto výška však nemôže byť väčšia ako vzdialenosť stredov kružníc:
\insp{z9-III-4a.eps}%

\noindent
Označme stredy kružníc $O$, $P$ a~pätu výšky z~bodu $O$ označme $Q$.
Ak body $P$ a~$Q$ nesplývajú, potom výška $OQ$ je odvesnou pravouhlého trojuholníka $OQP$, a~preto je menšia ako jeho prepona $OP$.

Teda lichobežník s najväčším obsahom je taký, ktorého základne sú kolmé na spojnicu stredov kružníc:
\insp{z9-III-4b.eps}%

\noindent
Základne lichobežníka majú dĺžky $|AB|=6$\,cm a ~$|CD|=14$\,cm, veľkosť výšky je $|OP|=\frac12(6+14)=10$\,cm.
Obsah tohto lichobežníka, a teda najväčší možný obsah, je
$$
\frac12\left(|AB|+|CD|\right)\cdot|OP|
= \frac12(6+14)\cdot10
= 100\,(\Cm^2).
$$

\hodnotenie
2~body za uvedomenie, že základne sa nemenia a~mení sa len výška;
2~body za obsah najväčšieho lichobežníka;
2~body za zdôvodnenie maxima výšky.

\poznamka
V zdôvodnení, že úsečka $OP$ je väčšia ako $OQ$, stačí argumentovať tým, že oproti väčšiemu vnútornému uhlu trojuholníka je väčšia strana.
\endhodnotenie
}

