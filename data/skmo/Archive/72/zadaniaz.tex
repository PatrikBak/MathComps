{%%%%%   Z4-I-1
...}
\podpis{...}

{%%%%%   Z4-I-2
...}
\podpis{...}

{%%%%%   Z4-I-3
...}
\podpis{...}

{%%%%%   Z4-I-4
...}
\podpis{...}

{%%%%%   Z4-I-5
...}
\podpis{...}

{%%%%%   Z4-I-6
...}
\podpis{...}

{%%%%% Z5-I-1
Na lúke bolo 45 oviec a niekoľko pastierov.
Potom ako z lúky odišla polovica pastierov a tretina oviec, mali zvyšní pastieri a ovce spolu 126 nôh.
Pritom všetky ovce a~všetci pastieri mali obvyklé počty nôh.
Koľko pastierov bolo pôvodne na lúke?}
\podpis{Libuše Hozová}

{%%%%% Z5-I-2
Marta hrá hru, v ktorej háda päťciferné číslo tvorené navzájom rôznymi číslicami.
Priebeh prvých troch kôl vyzerá takto:
\insp{z5-I-2.eps}%

Farba políčka prezrádza niečo o~číslici v~ňom napísanej:
\begin{itemize}
\item zelené políčko znamená, že číslica sa v~hádanom čísle vyskytuje, a~to presne na tom mieste,
\item žlté políčko znamená, že číslica sa v~hádanom čísle vyskytuje, ale na inom mieste,
\item šedé políčko znamená, že číslica sa v~hádanom čísle nevyskytuje.
\end{itemize}
Vysvetlite, či Marta môže alebo nemôže päťciferné číslo s~istotou uhádnuť v~nasledujúcom kole.}
\podpis{Josef Tkadlec}

{%%%%%   Z5-I-3
Vo štvorcovej sieti je daný trojuholník, ktorého vrcholy sú uzlovými bodmi siete:
\insp{z5-I-3.eps}%

V~dostatočne rozšírenej sieti nakreslite štyri rozdielne (navzájom nezhodné) mnohouholníky s~dvojnásobným obvodom ako má daný trojuholník.}
\podpis{Erika Novotná}

{%%%%%   Z5-I-4
Nikola mala v zošite napísané jedno trojciferné a~ jedno dvojciferné číslo.
Každé z~ týchto čísel bolo tvorené navzájom rôznymi číslicami.
Rozdiel Nikoliných čísel bol 976.
Aký bol ich súčet?}
\podpis{Libuše Hozová}

{%%%%%   Z5-I-5
Tri žaby sa naučili skákať po rebríku.
Každá dokáže skákať smerom hore, aj smerom dole, ale len o~určité počty priečok.
Žaby začínajú na zemi a~každá by sa rada dostala na svoju obľúbenú priečku:
\begin{itemize}
\item malá žaba vie skákať o~2 alebo o~3 priečky a~chce sa dostať na siedmu priečku,
\item stredná žaba vie skákať o~2 alebo o~4 priečky a~chce sa dostať na prvú priečku,
\item veľká žaba vie skákať o~6 alebo o~9 priečok a~chce sa dostať na tretiu priečku.
\end{itemize}
Pre jednotlivé žaby rozhodnite, či vedia doskočiť na svoju obľúbenú priečku.
Ak áno, popíšte ako.
Ak nie, vysvetlite prečo.}
\podpis{Veronika Bachratá}

{%%%%%   Z5-I-6
Jakub zbiera hracie kocky, všetky rovnakej veľkosti.
Včera našiel krabičku, do ktorej začal kocky ukladať.
Prvá vrstva kociek pokryla presne štvorcové dno krabičky.
Podobne vyskladal päť ďalších vrstiev, avšak v polovici nasledujúcej vrstvy mu došli kocky.
Dnes dostal Jakub od babičky 18 kociek a s prekvapením zistil, že mu presne chýbali na dokončenie neúplnej vrstvy v krabičke.
Koľko kociek mal Jakub včera?}
\podpis{Michaela Petrová}

{%%%%%   Z6-I-1
Pán Škovránok bol známym chovateľom vtákov.
Celkovo ich mal viac ako 50 a menej ako 100.
Andulky tvorili devätinu a~kanáriky štvrtinu celkového množstva vtákov.

Koľko vtákov choval pán Škovránok?}
\podpis{Libuše Hozová}

{%%%%%   Z6-I-2
Václav násobil dve trojciferné čísla obvyklým písomným spôsobom.
Overil si, že výsledok je naozaj správny a svoj výpočet niekam založil.
Po čase potreboval výsledok ukázať maminke.
Našiel síce svoj predchádzajúci výpočet, ale mnoho číslic bolo rozmazaných, takže sa nedali vôbec prečítať (hviezdičky nahrádzajú nečitateľné číslice):
$$
\begin{array}{ccccc}
 & & * & * & * \\
 & & 1 & * & * \\
\hline
 & 2 & 2 & * & * \\
 & 9 & 0 & * & \\
* & * & 2 & & \\
\hline
5 & 6 & * & * & * \\
\end{array}
$$

Václav si už nepamätal, ktoré čísla násobil, napriek tomu bol schopný určiť ich súčin.
Aký bol tento súčin?}
\podpis{L. Hozová}

{%%%%%   Z6-I-3
Magda si z papiera vystrihla dva rovnaké rovnoramenné trojuholníky, z ktorých každý mal obvod 100\,cm.
Najprv z týchto trojuholníkov zložila štvoruholník tak, že ich k sebe priložila ramenami.
Potom z~nich zložila štvoruholník tak, že ich k~sebe priložila základňami.
V prvom prípade jej vyšiel štvoruholník s obvodom o 4 cm kratším ako v druhom prípade.
Určite dĺžky strán Magdiných trojuholníkov.
\insp{z6-I-3.eps}%
}
\podpis{Eva Semerádová}

{%%%%%   Z6-I-4
Sedem trpaslíkov sa narodilo v rovnaký deň v siedmich po sebe idúcich rokoch.
Súčet vekov troch najmladších trpaslíkov je 42 rokov.
Keď jeden trpaslík odišiel so Snehulienkou po vodu, zistili zvyšní trpaslíci, že ich priemerný vek je rovnaký ako priemerný vek všetkých siedmich.
Koľko rokov mal trpaslík, ktorý šiel so Snehulienkou po vodu?}
\podpis{L. Hozová}

{%%%%%   Z6-I-5
Pat a~Mat si precvičovali počítanie.
Vo štvorcovej sieti orientovanej podľa svetových strán priradili posunu o~jedno políčko nasledujúce matematické operácie:
\begin{itemize}
\item pri posune na sever (S) pripočítali sedem,
\item pri posune na východ (V) odpočítali štyri,
\item pri posune na juh (J) vydelili dvoma,
\item pri posune na západ (Z) vynásobili tromi.
\end{itemize}
Napr. keď Mat zadal Patovi číslo 5 a cestu S-V-J, vyšlo im pri správnom počítaní číslo 4.

Ktoré číslo zadal Pat Matovi, ak pri ceste S - V - J - Z - Z - J - V - S pri správnom počítaní vyšlo na konci číslo 57?}
\podpis{Michaela Petrová}

{%%%%%   Z6-I-6
Boris má zvláštne digitálne hodiny.
Idú síce presne, ale namiesto hodín a minút ukazujú iné dve čísla:
prvé je ciferným súčtom čísla, ktoré by bolo na displeji bežných hodín,
druhé je súčtom hodín a minút (napr. v~7:30 ukazujú Borisove hodiny 10:37).
Aký môže byť skutočný čas, keď Borisove hodiny ukazujú 6:15? Určte všetky možnosti.}
\podpis{Monika Dillingerová}

{%%%%%   Z7-I-1
Priemerný vek dedka, babičky a ich piatich vnúčat je 26 rokov.
Priemerný vek samotných vnúčat je 7 rokov.
Babička je o rok mladšia ako dedo.
Koľko rokov má babička?}
\podpis{Libuše Hozová}

{%%%%%   Z7-I-2
Sú dané dva zhodné rovnostranné trojuholníky $ABC$ a~$BDE$ tak, že body $A$, $B$, $D$ ležia na jednej priamke a~body $C$, $E$ ležia v rovnakej polrovine vymedzenej priamkou AD.
Priesečník $CD$ a~$AE$ je označený $F$.
Určte veľkosť uhla $AFD$.
\insp{z7-I-2.eps}%
}
\podpis{Iveta Jančigová}

{%%%%%   Z7-I-3
{\it Obkročné číslo\/} je také prirodzené číslo, v ktorého zápise
\begin{itemize}
\item je každá nenulová číslica použitá práve dvakrát,
\item medzi dvoma rovnakými nenulovými číslicami sa nachádza práve toľko núl, aká je hodnota týchto číslic.
\end{itemize}
Príklady obkročných čísel sú napr. 40001041 alebo 300103100.

Zistite, koľko existuje sedemciferných obkročných čísel, v ktorých zápise sa vyskytujú práve jednotky, dvojky a nuly.}
\podpis{Matúš Papšo}

{%%%%%   Z7-I-4
Jarko mal napísanú postupnosť slabík:
$$
\text{ZU ZA NA NE LA LU CI SA MU EL}
$$
Písmená chcel nahradiť číslicami od 0 do 9 tak, aby rôznym písmenám zodpovedali rôzne číslice a aby (v~danom poradí) vznikla rastúca postupnosť dvojciferných čísel.

Zistite, či sa to dá a ako, alebo vysvetlite, prečo to možné nie je.}
\podpis{Jaroslav Zhouf}

{%%%%%   Z7-I-5
Na obrázku sú znázornené štvorce $ABCD$, $EFCA$, $GHCE$ a~$IJHE$.
Body $S$, $B$, $F$ a~$G$ sú po rade stredy týchto štvorcov.
Úsečka $AC$ je dlhá 1\,cm.
Určte obsah trojuholníka $IJS$.
\insp{z7-I-5.eps}%
}
\podpis{Eva Semerádová}

{%%%%%   Z7-I-6
Eva si myslela dve prirodzené čísla.
Tieto čísla najprv správne sčítala, potom od seba správne odčítala.
V obidvoch prípadoch dostala dvojciferný výsledok.
Súčin takto vzniknutých dvojciferných čísel bol 645.
Ktoré čísla si Eva myslela?}
\podpis{Erika Novotná}

{%%%%%   Z8-I-1
Sú dané tri navzájom rôzne čísla.
Priemer priemeru dvoch menších čísel a~priemeru dvoch väčších čísel je rovný priemeru všetkých troch čísel.
Priemer najmenšieho a~najväčšieho čísla je 2022.
Určite súčet troch daných čísel.}
\podpis{Karel Pazourek}

{%%%%%   Z8-I-2
Štvoruholník $ABCD$ je kosoštvorec so stranou dĺžky 6\,cm a~výškou 4\,cm.
Bod $E$ je stred strany $AD$,
bod $G$ je stred strany $BC$,
bod $F$ je priesečník úsečiek $AG$ a~$BE$,
bod $H$ je priesečník úsečiek $CE$ a~$DG$.
Určte obsah štvoruholníka $EFGH$.}
\podpis{Karel Pazourek}

{%%%%%   Z8-I-3
Pre postupnosť čísel začínajúcich
$$
1,\ 3,\ 4,\ 7,\ 11,\ 18,\ \dots
$$
platí, že každé číslo počnúc tretím je súčtom predchádzajúcich dvoch.

Akou číslicou končí číslo na 2023. mieste tejto postupnosti?}
\podpis{Ján Mazák}

{%%%%%   Z8-I-4
Cyril na mape s mierkou $1:50\,000$ vyznačil štvorcový pozemok a vypočítal si, že jeho strana v skutočnosti zodpovedá 1\,km.
Mapu zmenšil na kopírke tak, že vyznačený štvorec mal obsah o~1,44\,cm$^2$ menší ako na pôvodnej mape.
Aka bola mierka mapy po zmenšení?}
\podpis{Michaela Petrová}

{%%%%%   Z8-I-5
Petra mala na tabuli napísané všetky prirodzené čísla od 1 do 9, každé práve raz.
Dve z týchto čísel sčítala, zmazala a výsledný súčet napísala namiesto zmazaných sčítancov.
Mala tak teraz napísaných osem čísel, ktoré sa jej podarilo rozdeliť do dvoch skupín s rovnakým súčinom.
Určte aký najväčší mohol byť tento súčin.}
\podpis{Erika Novotná}

{%%%%%   Z8-I-6
Je daný obdĺžnik $ABCD$ a~body $E$, $F$ tak, že trojuholníky $BEC$ a~$CFD$ sú rovnostranné a~každý z~ich má s~pravouholníkom $ABCD$ spoločnú iba stranu.
Zdôvodnite, že aj trojuholník $AEF$ je rovnostranný.}
\podpis{Jaroslav Švrček}

{%%%%%   Z9-I-1
Aritmetická postupnosť je taká postupnosť čísel, v ktorej je rozdiel každého čísla od čísla jemu predchádzajúceho stále rovnaký; tomuto rozdielu sa hovorí {\it diferencia\/}.
(Napr. 2, 8, 14, 20, 26, 32 je aritmetická postupnosť s~diferenciou 6.)
Bolek a Lolek mali každý svoju aritmetickú postupnosť.
Aj Bolkova, aj Lolkova postupnosť začínala číslom 2023 a~končila číslom 3023.
Tieto dve postupnosti mali 26 spoločných čísel.
Pomer Bolkovej a~Lolkovej diferencie bol $5:2$.
Určte rozdiel Bolkovej a~Lolkovej diferencie.}
\podpis{Erika Novotná}

{%%%%%   Z9-I-2
Sú dané dva zhodné rovnostranné trojuholníky $ABC$ a~$BDE$ tak, že veľkosť uhla $ABD$ je väčšia ako $120\st$ a~menšia ako $180\st$ a~body $C$, $E$ ležia v rovnakej polrovine vymedzenej priamkou $AD$.
Priesečník $CD$ a~$AE$ je označený $F$.
Určte veľkosť uhla $AFD$.
\insp{z9-I-2.eps}%
}
\podpis{Iveta Jančigová}

{%%%%%   Z9-I-3
Traja kúzelníci kúzlia s číslami, každý však vie len jedno kúzlo:
\begin{itemize}
\item prvý kúzelník vie od ľubovoľného čísla odčítať číslo jedna,
\item druhý kúzelník vie ľubovoľné číslo vydeliť číslom dva,
\item tretí kúzelník vie ľubovoľné číslo vynásobiť číslom tri.
\end{itemize}
Kúzelníci sa pri čarovaní môžu ľubovoľne striedať, každý však môže svoje kúzlo počas jedného vystúpenia použiť najviac päťkrát
a~žiadny medzivýsledok nesmie byť väčší ako 10.
Pri jednom vystúpení mali z~pätice čísel 3, 8, 9, 2, 4 vykúzliť päticu trojok,
pri inom vystúpení mali z tej istej pätice čísel vykúzliť päticu pätiek.
Ako si mohli s~problémom poradiť?
Nájdite možné riešenia alebo vysvetlite, prečo to možné nie je.}
\podpis{Erika Novotná}

{%%%%%   Z9-I-4
Nájdite najmenšie kladné celé čísla $a$ a~$b$, pre ktoré platia
$$
7a^3=11b^5.
$$}
\podpis{Alžbeta Bohiniková}

{%%%%%   Z9-I-5
Na snovom trhovisku ponúkla Sfinga cestovateľovi za štyri sny sedem ilúzií, dva šlofíky a jednu nočnú moru.
Inému cestovateľovi tá istá Sfinga ponúkla za sedem snov štyri ilúzie, štyri šlofíky a dve nočné mory.
Sfinga je pri svojich ponukách spravodlivá a vymeriava vždy rovnako.
Koľko ilúzií stojí jeden sen?}
\podpis{Karel Pazourek}

{%%%%%   Z9-I-6
Vrcholy štvorca $ABCD$ spája lomená čiara $DEFGHB$.
Menšie uhly pri~vrcholoch $E$, $F$, $G$, $H$ sú pravé a~úsečky $DE$, $EF$, $FG$, $GH$, $HB$ po rade merajú 6\, cm, 4\,cm, 4\,cm, 1\,cm, 2\,cm.
Určte obsah štvorca $ABCD$.
\insp{z9-I-6.eps}%
}
\podpis{Monika Dillingerová}

{%%%%%   Z4-II-1
...}
\podpis{...}

{%%%%%   Z4-II-2
...}
\podpis{...}

{%%%%%   Z4-II-3
...}
\podpis{...}

{%%%%%   Z5-II-1
Kanva plná mlieka má hmotnosť 35 kg.
Tá istá kanva s polovičným množstvom mlieka má hmotnosť 18 kg.

Koľko kilogramov váži prázdna kanva?}
\podpis{Libuše Hozová}

{%%%%%   Z5-II-2
Dvaja strážcovia dohliadajú na miestnosť, ktorej tvar a~rozmery sú znázornené na obrázku.
Každé dve susedné steny sú navzájom kolmé, rozmery sú uvedené v~metroch.
Strážcovia stoja tesne pri stene v miestach označených štvorčekmi. Spoločne takto majú pod dohľadom celú miestnosť, pritom len na časť miestnosti môžu dohliadnuť obaja súčasne.
\ite a) Vyznačte v~obrázku časť miestnosti, kam môžu dohliadnuť obaja strážci.
\ite b) Koľkokrát je celá miestnosť väčšia ako časť, kam môžu dohliadnuť obaja strážci?
\insp{z5-II-2.eps}%
}
\podpis{Karel Pazourek}

{%%%%%   Z5-II-3
Nájdite všetky čísla s~nasledujúcimi vlastnosťami:
\ite $\bullet$ číslo je párne,
\ite $\bullet$ číslo je dvojciferné,
\ite $\bullet$ súčet jeho číslic je väčší ako 6,
\ite $\bullet$ číslica na mieste desiatok je aspoň o~4 väčšia ako číslica na mieste jednotiek.
}
\podpis{Miroslava Farkas Smitková}

{%%%%%   Z6-II-1
Mamička sa chystala piecť svojim deťom rožky, všetky z rovnako veľkých dielov cesta.
Ak by každému dieťaťu upiekla tri rožky, zostalo by jej cesto na ďalšie dva rožky.
Ak by každému dieťaťu chcela upiecť štyri rožky, chýbalo by jej cesto na jeden rožok.

Pre koľko detí piekla mamička rožky?}
\podpis{Libuše Hozová}

{%%%%%   Z6-II-2
Stano a ~Jana dostali dve trociferné čísla.
Stano si v~prvom čísle doplnil desatinnú čiarku za prvú číslicu, v druhom čísle za druhú číslicu, takto vzniknuté desatinné čísla sčítal a~dostal výsledok
50{,}13.
Jana si v~prvom čísle doplnila desatinnú čiarku za druhú číslicu, v druhom čísle za prvú číslicu, takto vzniknuté desatinné čísla sčítala a~dostala výsledok
34{,}02.

Určite súčet pôvodných trojciferných čísel.}
\podpis{Karel Pazourek}

{%%%%%   Z6-II-3
Zuzka mala päť štvorčekových kociek s hranami dĺžok od 1 do 5 štvorčekov:
\insp{z6-II-3.eps}%

Zo všetkých týchto kociek zlepila vežu, v ktorej menšie kocky stavala na väčšie, a to vždy celou jednou stenou.
Potom Zuzka celú vežu okrem podstavnej steny zafarbila.
Farbu mala vo vedierkach, z ktorých každé stačilo na zafarbenie plochy zodpovedajúcej presne 5 štvorčekov.

Koľko vedierok farby stačilo Zuzke na zafarbenie veže?}
\podpis{Erika Novotná}

{%%%%%   Z7-II-1
Dvaja strážcovia dohliadajú na poriadok v~miestnosti, ktorej pôdorys a~rozmery sú znázornené na obrázku.
Každé dve susedné steny sú navzájom kolmé, rozmery sú uvedené v~metroch.
Strážcovia stoja v rohoch označených štvorčekmi.

Aká veľká je časť miestnosti, na ktorú zo svojho miesta nedohliadne ani jeden zo strážcov?
\insp{z7-II-1.eps}%
}
\podpis{Eva Semerádová}

{%%%%%   Z7-II-2
Anna a Jozef sú manželia s jedenástimi potomkami.
Priemerný vek ich dvoch detí je 63 rokov,
priemerný vek ich štyroch vnúčat je 35 rokov
a~priemerný vek ich piatich pravnúčat je 4~roky.
Annin vek je triapolkrát väčší ako priemerný vek všetkých jedenástich potomkov.
Ak sa dožijú, bude za päť rokov priemerný vek Anny a Jozefa rovných 100 rokov.

Koľko rokov má Anna a~koľko Jozef?}
\podpis{Eva Semerádová}

{%%%%%   Z7-II-3
Na turnaji sa zišli tímy Akúska, Bovenska, Coľska a Demecka.
Každý tím sa stretol s každým presne raz.
Víťazný tím dostal tri body, porazený tím nedostal žiadny bod, pri nerozhodnom výsledku dostal každý z remizujúcich tímov po jednom bode.
Po odohraní všetkých šiestich zápasov malo Akúsko 7 bodov, Bovensko 4 body, Coľsko 3 body a Demecko 2~body.
\begin{enumerate}\alphatrue
\item Koľko zápasov skončilo remízou?
\item Ako dopadol zápas Bovenska s Coľskom?
\end{enumerate}
}
\podpis{Josef Tkadlec}

{%%%%%   Z8-II-1
Pankrác, Servác a Bonifác si kúpili čln.
Pankrác zaplatil 60\,\% ceny člna, Servác zaplatil 40\,\% zvyšku ceny a Bonifác doplatil chýbajúcu čiastku, čo bolo 30 zlatiek.

Koľko zlatiek stál čln, ktorý si chlapci kúpili?}
\podpis{Libuše Hozová}

{%%%%%   Z8-II-2
Je daný pravouhlý trojuholník $ABC$ s~pravým uhlom pri~vrchole $C$ a~s~dĺžkami odvesien v~pomere $1:3$.
Body $K$, resp. $L$ sú stredy štvorcov, ktoré majú jednu stranu spoločnú s~odvesnou $AC$, resp. $BC$ a~ktoré sa s~trojuholníkom $ABC$ neprekrývajú.
Bod $M$ je stredom prepony $AB$.

\begin{enumerate}\alphatrue
\item Zdôvodnite, že bod $C$ leží na úsečke $KL$.
\item Vypočítajte pomer obsahov trojuholníkov $ABC$ a~$KLM$.
\end{enumerate}
}
\podpis{Jaroslav Švrček}

{%%%%%   Z8-II-3
Karolína napísala všetky trojciferné čísla tvorené číslicami 1, 2 a~3, v ktorých sa žiadna číslica neopakovala a~v~ ktorých bola číslica 2 na mieste desiatok.
Nikola napísala všetky trojciferné čísla tvorené číslicami 4, 5 a ~6, v ktorých sa tiež žiadna číslica neopakovala.
Kubo si vybral jedno číslo od Karolíny a jedno číslo od Nikoly tak, aby súčet týchto dvoch čísel bol párny.

Aká bola číslica na mieste jednotiek v súčine čísel, ktoré si Kubo vybral?
Nájdite všetky možnosti.}
\podpis{Libuše Hozová}

{%%%%%   Z9-II-1
Nájdite všetky štvorciferné čísla, ktoré majú presne päť štvorciferných a deväť jednociferných deliteľov.}
\podpis{Svetlana Bednářová}

{%%%%%   Z9-II-2
Trojuholník $ABC$ má stranu $AC$ dlhú 24\,cm a ~výšku z ~vrcholu $B$ dlhú 25\,cm.
Strana $AB$ je rozdelená na päť zhodných častí, deliace body sú postupne od $A$ k~$B$ označené $K$, $L$, $M$, $N$.
Každým z týchto bodov prechádza rovnobežka so stranou $AC$.
Priesečníky rovnobežiek so stranou $BC$ sú postupne od $B$ k~$C$ označené $O$, $P$, $Q$, $R$.

Vypočítajte súčet obsahov lichobežníkov $KLQR$ a~$MNOP$.}
\podpis{Iveta Jančigová}

{%%%%%   Z9-II-3
Pomocou premennej $n$ boli zapísané nasledujúce štyri výrazy:
$$
2023n, \quad n^2+n+23, \quad 3n^3, \quad 10n^2+2023.
$$
Výraz nazveme \emph{nepárnotvorný}, ak pre každé prirodzené číslo $n$ platí, že hodnota výrazu je nepárna.
Rozhodnite, ktoré z~uvedených štyroch výrazov sú nepárnotvorné a zdôvodnite prečo.}
\podpis{Libuše Hozová}

{%%%%%   Z9-II-4
V~istom mnohouholníku platí, že pomer súčtu veľkostí jeho vnútorných uhlov a súčtu veľkostí k~nim doplnkových uhlov je $3:5$.
(Na vysvetlenie: doplnkový uhol dopĺňa daný uhol do uhla plného.)

Koľko vrcholov má tento mnohouholník?}
\podpis{Iveta Jančigová}

{%%%%%   Z9-III-1
Z prieskumu v našej škole vyplynulo, že
\begin{itemize}
\item všetky deti, ktoré majú rady matematiku, riešia Matematickú olympiádu,
\item 90\,\% detí, ktoré nemajú rady matematiku, Matematickú olympiádu nerieši,
\item Matematickú olympiádu rieši 46\,\% detí.
\end{itemize}
Koľko percent detí z našej školy má rado matematiku?}
\podpis{Libuše Hozová}

{%%%%%   Z9-III-2
Je daný bod $B$ a~rovnostranný trojuholník so stranami dĺžky 1\,cm.
Vzdialenosti bodu~$B$ od dvoch vrcholov tohto trojuholníka sú 2\,cm a~3\,cm.

Vypočítajte vzdialenosť bodu $B$ od tretieho vrcholu tohto trojuholníka.}
\podpis{Josef Tkadlec}

{%%%%%   Z9-III-3
Tabuľka čísel má 4 stĺpce a~99 riadkov
a~je vytvorená nasledujúcim spôsobom: počnúc druhým riadkom je štvorica čísel v každom riadku určená
číslami z riadku predchádzajúceho, a to postupne ako
súčet prvého a~druhého čísla, rozdiel prvého a~druhého čísla, súčet tretieho a~štvrtého čísla a~rozdiel tretieho a~štvrtého čísla.
\begin{enumerate} \alphatrue
\item Aký je súčet čísel na 3. riadku, ak je súčet čísel v prvom riadku rovný 0?
\item Aký je súčet čísel na 25. riadku, ak je súčet čísel v prvom riadku rovný 1?
\end{enumerate}
}
\podpis{Karel Pazourek}

{%%%%%   Z9-III-4
Sú dané dve kružnice s vonkajším dotykom.
Úsečka $AB$ je priemerom jednej kružnice a~má veľkosť 6\,cm, úsečka $CD$ je priemerom druhej kružnice a~má veľkosť 14\,cm.
Štvoruholník $ABCD$ je lichobežník so základňami $AB$ a~$CD$.

Zistite, aký najväčší obsah môže mať lichobežník $ABCD$, a~zdôvodnite, prečo nemôže byť väčší.
\insp{z9-III-4.eps}%
}
\podpis{Karel Pazourek}

