{%%%%%   Z4-I-1
Z päťciferných čísel $53\,827$ a $19\,763$ vyškrtni spolu dve číslice tak, aby súčet vzniknutých
čísel bol čo najväčší.}
\podpis{M. Dillingerová}

{%%%%%   Z4-I-2
Na pomarančovú limonádu potrebujeme šťavu z ôsmich pomarančov, dvoch citrónov, 2 čajové
lyžičky cukru a 6 decilitrov vody. Do džbánu sme si naliali 9 decilitrov vody. Koľko
musíme odšťaviť pomarančov, citrónov, koľko pridať lyžičiek cukru, aby sme dostali rovnako
kvalitnú limonádu?}
\podpis{S. Bodláková}

{%%%%%   Z4-I-3
Na drevenom plote je 70 latiek. Paľko s Aničkou ich mali všetky ponatierať farbou. Začali aj
skončili obaja naraz. Kým Anička natrela dve latky, prešli 4 minúty a za 8 minút stihol Paľko
ponatierať 3 latky. Ako dlho im trvalo natretie všetkých latiek?}
\podpis{M. Dillingerová}

{%%%%%   Z4-I-4
...}
\podpis{...}

{%%%%%   Z4-I-5
Martin a Jana si porovnávali svoje Mikulášske balíčky. Mali tam nasypané aj svoje obľúbené
čokoládky. Martin ich však mal iný počet ako Jana, tak venoval celú štvrtinu svojich
čokoládiek Jane. Jana si všetky svoje prepočítala a polovicu z nich venovala naspäť
Martinovi. Martin opäť venoval štvrtinu Jane. Po poslednom prepočítaní zistili, že majú obaja
po 9 čokoládok. Koľko čokoládok mal Martin pôvodne v balíčku? Koľko ich tam mala Jana?
(Počas počítania a presúvania ani jednu čokoládku nezjedli.)}
\podpis{M. Dillingerová}

{%%%%%   Z4-I-6
...}
\podpis{...}

{%%%%%   Z5-I-1
Náš kuchynský stôl má obdĺžnikovú vrchnú dosku s rozmermi $90\cm\times140\cm$. Chceme naň
ušiť obrus tak, aby na každej strane stola presahoval rovnako.
\begin{itemize}
\itemvar{a)} Koľko látky šírky $140\cm$ treba kúpiť, aby sme ju už nemuseli ďalej strihať?
\itemvar{b)} Koľko cm bude tento obrus na každej strane presahovať?
\end{itemize}
}
\podpis{S. Bednářová}

{%%%%%   Z5-I-2
...}
\podpis{...}

{%%%%%   Z5-I-3
V škôlke majú stavebnicu pozostávajúcu z rovnako veľkých molitanových kvádrov. Keď ich
deti všetky položia na seba, poskladajú ich vždy tak, aby na sebe ležali kvádre rovnakými
stenami a v žiadnom "poschodí" neboli kvádre dva. Takto sa im postupne podarilo postaviť tri
rôzne vysoké veže. Prvá mala $120\cm$, druhá $150\cm$ a tretia $130\cm$. Koľko kvádrov mohla
mať stavebnica, z ktorej stavali?}
\podpis{M. Dillingerová}

{%%%%%   Z5-I-4
Trojčatá práve oslávili svoje tretie narodeniny. O päť rokov bude súčet ich vekov rovný
dnešnému veku ich mamy. Koľko rokov bude mať ich mama o 5 rokov?}
\podpis{M. Krejčová}

{%%%%%   Z5-I-5
Číslo sa nazýva {\it prefíkané}, ak počnúc jeho treťou číslicou (počítané zľava) platí, že každá
jeho číslica je súčtom všetkých číslic ležiacich naľavo od nej.
\begin{itemize}
\itemvar{a)} Napíšte dve najväčšie prefíkané čísla.
\itemvar{b)} Koľko je všetkých štvorciferných prefíkaných čísel?
\end{itemize}
}
\podpis{S. Bednářová}

{%%%%%   Z5-I-6
...}
\podpis{...}

{%%%%%   Z6-I-1
Jurko kúpil dve čokolády v obchode oproti škole. Miško si kúpil také isté dve čokolády
v obchode za školou a Ivan si kúpil tiež takú čokoládu, ale v školskom bufete. Spolu potom
zistili, že priemerne ich spolu vyšla jedna čokoláda na 19{,}70 Sk. Takýmto spôsobom boli
všetky tri nákupy spolu o 6 Sk drahšie, ako keby chlapci nakupovali všetkých 5 čokolád
v obchode oproti škole a o 6{,}50 Sk lacnejšie, ako keby nakúpili iba v obchode za školou.
Koľko stáli čokolády v jednotlivých obchodoch?}
\podpis{M. Dillingerová}

{%%%%%   Z6-I-2
Miško mal farebné nálepky v tvare rovnoramenných pravouhlých trojuholníkov dvoch
veľkostí. Prvý druh mal ramená dĺžky $5\cm$, tých bolo 9. Druhý druh mal najdlhšiu stranu
dĺžky $10\cm$ a týchto nálepiek bolo 17. Najmenej koľko nálepiek prvého druhu si má Miško
ešte dokúpiť, aby svojimi nálepkami mohol úplne oblepiť (zakryť) steny kocky s hranou
dĺžky $10\cm$?}
\podpis{M. Dillingerová}

{%%%%%   Z6-I-3
V~rovine majú ležať body $A$, $B$, $C$, $D$ tak, aby platilo: $|AB| = 7\cm$, $|BC| = 8\cm$, $|CD| = 5\cm$
a $|DA| = 9\cm$.
\begin{itemize}
\itemvar{a)} Urči najväčšiu možnú vzdialenosť bodov $A$ a $C$.
\itemvar{b)} Urči najmenšiu možnú vzdialenosť bodov $A$ a $C$.
\end{itemize}
}
\podpis{L. Šimůnek}

{%%%%%   Z6-I-4
Pri chudokrvnosti sa odporúča piť mrkvovo-cviklovú šťavu, pričom cviklová šťava má
predstavovať len 1/5 z objemu nápoja. Z dvoch kg mrkvy získame v odšťavovači 7{,}5 dl šťavy,
z jedného kg cvikly 6 dl šťavy.
\begin{itemize}
\itemvar{a)} Aké množstvo mrkvy potrebujeme na 25 dag cvikly, aby sme získali správne
namiešanú mrkvovo-cviklovú šťavu?
\itemvar{b)} Aké množstvo mrkvovo-cviklovej šťavy takto získame?
\end{itemize}
}
\podpis{S. Bednářová}

{%%%%%   Z6-I-5
Ak povie mimozemšťan v rozhovore o Vianociach "haf quin lina", znamená to "veľké zlaté
hviezdy"; ak "kari lina mejk", znamená to "blikajúce zlaté kolieska"; ak "esca haf kari",
znamená to "veľké červené kolieska".
Ako sa povie "blikajúce hviezdy"? (Zapíš svoju úvahu.)}
\podpis{M. Volfová}

{%%%%%   Z6-I-6
Z trojciferných čísel $532$ a $179$ vyškrtni spolu dve číslice tak, aby súčin vzniknutých čísel bol
čo najväčší.}
\podpis{M. Dillingerová}

{%%%%%   Z7-I-1
Číslo nazveme trochu nešťastné, ak je násobkom čísla 13. Číslo, ktoré je násobkom čísla 17,
nazveme trochu usmievavé. Vezmime všetky prirodzené čísla od 1 do 1\,000\,000, ktoré
nekončia ani 0 ani 5. Koľko z nich je trochu nešťastných a zároveň trochu usmievavých?}
\podpis{M. Volfová}

{%%%%%   Z7-I-2
...}
\podpis{...}

{%%%%%   Z7-I-3
O dvanástej stáli na parkovisku české, nemecké a francúzske autá v pomere: české
k nemeckým $9:4$, nemecké k francúzskym $2:3$. V priebehu hodiny odišlo jedenásť a prišlo päť
českých áut, odišlo jedno a prišlo jedenásť nemeckých áut a odišli tri a prišlo šesť
francúzskych áut. Aký je pomer českých, nemeckých a francúzskych áut stojacich o 13:00 na
parkovisku, ak o 12:00 tam bolo dvanásť francúzskych áut?}
\podpis{Š. Ptáčková}

{%%%%%   Z7-I-4
...}
\podpis{...}

{%%%%%   Z7-I-5
Všetky políčka na šachovnici $4\times4$ vyfarbite štyrmi rôznymi farbami a vpíšte do nich písmená
L, E, T, O tak, aby v každom riadku aj v každom stĺpci boli zastúpené všetky farby aj všetky
písmená. Každé políčko bude celé jednofarebné a bude obsahovať práve jedno písmeno.
Každé písmeno musí byť napísané na políčku každej farby a na každej farbe musia byť
postupne umiestnené všetky písmená. Nájdi jedno riešenie.}
\podpis{M. Volfová}

{%%%%%   Z7-I-6
Na papieri je napísaných niekoľko po sebe idúcich prirodzených čísel. Je medzi nimi 12
takých, ktoré sú násobkom piatich a 10 takých, ktoré sú násobkom siedmich.
\begin{itemize}
\itemvar{a)} Koľko prirodzených čísel je napísaných na papieri?
\itemvar{b)} Nájdite jednu postupnosť prirodzených čísel, ktorá odpovedá vyššie opísaným
podmienkam.
\end{itemize}
}
\podpis{L. Šimůnek}

{%%%%%   Z8-I-1
Nájdite všetky štvorciferné čísla deliteľné tromi, ktoré po vynásobení číslom 17 dávajú číslo
končiace trojčíslím 519.}
\podpis{L. Hozová}

{%%%%%   Z8-I-2
Nájdite všetky trojice prirodzených čísel menších ako 10, pre ktoré platí, že ich súčin je
sedemnásobok ich súčtu.}
\podpis{L. Hozová}

{%%%%%   Z8-I-3
Jano si kúpil sedemmíľové čižmy. Jeho kamarát Honza z Čiech si kúpil lietajúci koberec.
Potom sa obaja zúčastnili na rozprávkových 12-hodinových pretekoch. Počas pretekov boli
hladní, a tak sa obaja zastavili najesť. Jedenie každému trvalo hodinu. Keby sa Honza
nezastavil na "vepřo-knedlo-zelo", predbehol by Jana o 51 rozprávkových míľ. Keby sa Jano
nezastavil na bryndzové halušky, predbehol by Honzu o 28 rozprávkových míľ. Ako ďaleko
od seba by skončili, keby nejedol ani jeden z nich? Kto z nich by bol prvý?}
\podpis{M. Dillingerová}

{%%%%%   Z8-I-4
V Tramtárii majú 5 lekárskych fakúlt, z ktorých každá môže do prvého ročníka prijať presne
200 študentov. Prijímacie skúšky na jednotlivé fakulty sa konajú v rôzne dni, preto si študenti
môžu podať prihlášku na viacero fakúlt. Pýtali sme sa na fakultách, koľko im prišlo prihlášok
na školský rok 2007/2008. Získali sme tieto odpovede:
\begin{itemize}
\itemvar{} 1. fakulta: "Dostali sme päťkrát viac prihlášok, ako sme mali voľných miest."
\itemvar{} 2. fakulta: "U nás počet uchádzačov prevýšil kapacitu o 320\%."
\itemvar{} 3. fakulta: "Na našu fakultu sa hlásilo o 520 uchádzačov viac, ako sme mali miest."
\itemvar{} 4. fakulta: "U nás na každé voľné miesto pripadli v priemere 3 prihlášky."
\itemvar{} 5. fakulta: "K nám sa hlásilo o tri štvrtiny záujemcov viac, ako sme mali miest."
\end{itemize}
\noindent
V akademickom roku 2007/2008 nakoniec štúdium začalo 1000 medikov. Zo štatistiky
vyplýva, že záujemca o štúdium medicíny podal na lekárske fakulty v priemere 2{,}5 prihlášky.
Koľko záujemcov sa nedostalo na žiadnu z lekárskych fakúlt Tramtárie?}
\podpis{L. Šimůnek}

{%%%%%   Z8-I-5
...}
\podpis{...}

{%%%%%   Z8-I-6
Uprostred námestia v Kocúrkove je štvorcový trávnatý záhon. Keď Kocúrkovčania zistili, že
zabudli urobiť chodník, tak z každého kraja záhonu naň ubrali 2 metre. Pred položením
zámkovej dlažby (a štrku pod ňu) bolo treba pod celú plochu chodníka urobiť $0{,}5\,\text{m}$ hlboký
výkop. Odkopaním trávy a hliny sa záhon zmenšil o $1\,200\,\text{m}^2$.
\begin{itemize}
\itemvar{a)} Aký obsah má teraz trávnatý záhon?
\itemvar{b)} Koľko $\text{m}^3$ štrku je pod dlažbou, ak je povrch dlažby zarovno s trávnatým záhonom a výška
dlaždice je $8\cm$?
\end{itemize}
}
\podpis{M. Smitková, M. Dillingerová}

{%%%%%   Z9-I-1
Nájdite všetky štvorciferné čísla končiace číslicou 9, ktoré sú deliteľné každou svojou
číslicou.}
\podpis{P. Tlustý}

{%%%%%   Z9-I-2
Peter sa pýtal babičky, koľko rokov má dedko. Babička mu takto odpovedala: "To vieš, už
dávno nemáme päťdesiat, ale zato ešte nemáme osemdesiat rokov. Ak vynásobíš súčet môjho
a dedkovho veku ich rozdielom a k výsledku pripočítaš oba naše veky, dostaneš 492."
"Aha", povedal po chvíli Peter, "tak to má dedko..."
Koľko rokov má Petrov dedko, ak viete, že je starší ako Petrova babička?}
\podpis{M. Raabová}

{%%%%%   Z9-I-3
...}
\podpis{...}

{%%%%%   Z9-I-4
Minulú divadelnú sezónu sa predávali vstupenky za jednotnú cenu 160 Sk. Pre tohtoročnú
sezónu sa sedadlá v hľadisku rozdelili do dvoch kategórií. Miesta I. kategórie stoja 180 Sk
a miesta II. kategórie 155 Sk. Ak sa všetky sedadlá v hľadisku vypredajú, bude celková tržba
rovnaká ako minulú sezónu pri vypredanom hľadisku. Riaditeľ divadla stále nie je spokojný
a pre budúcu sezónu plánuje zmenu: z najhorších miest súčasnej II. kategórie urobí
III. kategóriu. Aby sa však tržba za vypredané hľadisko nezmenila, tak rozhodol, že
vstupenky budú stáť 180 Sk (I. kategória), 160 Sk (II. kategória) a 130 Sk (III. kategória).
V akom pomere budú v budúcej sezóne počty sedadiel jednotlivých kategórií?}
\podpis{L. Šimůnek}

{%%%%%   Z9-I-5
Jurko kúpil dve čokolády v obchode oproti škole. Miško si kúpil také isté dve čokolády
v obchode za školou a Ivan si kúpil tiež takú čokoládu, ale v školskom bufete. Takýmto
spôsobom boli všetky tri nákupy spolu o 6 Sk drahšie, ako keby chlapci nakupovali všetkých
5 čokolád v obchode oproti škole a o 6,50 Sk lacnejšie, ako keby nakúpili iba v obchode za
školou. V školskom bufete predávajú jednu čokoládu za 19,50 Sk. Koľko stáli všetky
čokolády spolu? Koľko stojí čokoláda v obchode za školou?}
\podpis{M. Dillingerová}

{%%%%%   Z9-I-6
V rovine je daný štvoruholník $ABCD$. Zostrojte bod $K$, ktorý je vrcholom rovnobežníka
$BCDK$, a bod $L$, ktorý je vrcholom rovnobežníka $CDAL$. Ukážte, že priamka $KL$ prechádza
stredom strany $AB$ daného štvoruholníka $ABCD$.}
\podpis{J. Švrček}

{%%%%%   Z4-II-1
...}
\podpis{...}

{%%%%%   Z4-II-2
Z päťciferných čísel $53\,827$ a $19\,763$ vyškrtni spolu dve číslice tak, aby rozdiel
vzniknutých čísel bol čo najmenší.}
\podpis{M. Dillingerová}

{%%%%%   Z4-II-3
Martin narysoval obdĺžnik, ktorý sa dá bezo zvyšku rozstrihať na 3 štvorce. Aké rozmery
mohol mať obdĺžnik ak jeden zo štvorcov mal stranu dĺžky $7\cm$? Vypočítaj jeho obvod.}
\podpis{M. Dillingerová}

{%%%%%   Z5-II-1
Polovica detí 5.A chodí na tanečný krúžok. Dievčatá chodia všetky a z osemnástich chlapcov chodí iba jedna tretina.
\begin{itemize}
\itemvar{a)}  Koľko detí chodí do 5.A?
\itemvar{b)}  Koľko dievčat chodí do 5.A?
\end{itemize}
}
\podpis{M. Volfová}

{%%%%%   Z5-II-2
Majo sčítal štyri po sebe idúce dvojciferné čísla a súčet zaokrúhlil na desiatky. Juro vzal tie isté čísla, najprv ich zaokrúhlil na desiatky a potom ich sčítal. Jeho výsledok bol o 10 väčší ako Majov. Ktoré čísla sčítavali chlapci, ak ich výsledky neboli väčšie ako 100? Nájdi všetky možné riešenia.}
\podpis{M. Dillingerová}

{%%%%%   Z5-II-3
Mamička šije utierky z látky šírky $120\cm$. Hotová utierka má rozmery $60\cm\times38\cm$. Pri strihaní látky treba počítať $2\cm$ navyše na každom okraji na začistenie. Najmenej koľko centimetrov tejto látky treba kúpiť, aby z nej mamičke vyšlo 10 utierok?}
\podpis{M. Dillingerová}

{%%%%%   Z6-II-1
Na záhrade pána Kozla kvitlo niekoľko čerešní. Na každej čerešni sedeli tri škorce
a ešte jeden sedel na plote. Pes pána Kozla ich vyplašil a škorce uleteli. Za chvíľu sa
všetci vrátili a usadili sa opäť na čerešniach. Čerešňa, pod ktorou spal pes, zostala
prázdna a na každej z ostatných sa usadili štyri škorce. Koľko čerešní má pán Kozel
a koľko bolo na záhrade škorcov?}
\podpis{L. Hozová}

{%%%%%   Z6-II-2
Daný je trojuholník $ABC$, v ktorom päta $P$ kolmice z bodu $C$ na priamku $AB$ leží vo
vnútri úsečky $AB$. Z bodu $P$ sú vedené kolmice $p$, $q$ na priamky $AC$ a $BC$
(v uvedenom poradí). Označme $S$ priesečník priamky $BC$ a priamky $q$ a nech $T$ je
priesečník priamky $AC$ a priamky $p$. Vypočítaj veľkosť uhla $ACB$, ak vieš, že
$|\uhol APT|+|\uhol BPS|=20\st$.}
\podpis{M. Dillingerová}

{%%%%%   Z6-II-3
...}
\podpis{...}

{%%%%%   Z7-II-1
Daný je rovnobežník $ABCD$ ($|AB|\ne|BC|$) s vnútorným uhlom veľkosti $72\st$ pri vrchole
$A$. Jedným vrcholom tohto rovnobežníka vedieme dve priamky, ktoré rozdeľujú
rovnobežník na tri rovnoramenné trojuholníky. Určte veľkosti vnútorných uhlov
týchto trojuholníkov.}
\podpis{L. Hozová}

{%%%%%   Z7-II-2
Kráľ Lenivého kráľovstva vydal v nedeľu 1. apríla 2007 dekrét, ktorým vyradil zo
všetkých nasledujúcich týždňov piatky. Od tej doby v jeho kráľovstve nasleduje vždy
po štvrtku sobota a týždeň má iba šesť dní. Ktorý deň v týždni pripadne v Lenivom
kráľovstve na 9. apríla 2008?
(Nezabudnite, že rok 2008 je priestupný!)}
\podpis{L. Šimůnek}

{%%%%%   Z7-II-3
U Novákovcov napiekli svadobné koláče. Štvrtinu odviezli príbuzným na Moravu,
šestinu rozdali kolegom v práci a devätinu dali susedom. Keby im zostali o tri koláče
viacej, bola by to polovica pôvodného počtu. Koľko koláčov napiekli?}
\podpis{M. Volfová}

{%%%%%   Z8-II-1
Družstvo chce v sezóne vyhrať $\frac34$ všetkých svojich zápasov. V prvej tretine z nich
však vyhralo iba 55\% zápasov.
\begin{itemize}
\itemvar{a)} Koľko percent zvyšných zápasov by muselo družstvo vyhrať, aby dosiahlo
zamýšľaný cieľ?
\itemvar{b)} Keby družstvo vyhralo všetky zvyšné zápasy, koľko percent svojich zápasov by
v celej sezóne vyhralo?
\end{itemize}
}
\podpis{M. Volfová}

{%%%%%   Z8-II-2
Akú časť obsahu nerovnoramenného lichobežníka $KLMN$ ($KL\parallel MN$) tvorí obsah
trojuholníka $ABC$, kde $A$ je stred základne $KL$, $B$ je stred základne $MN$ a $C$ je stred
ramena $KN$?}
\podpis{L. Hozová}

{%%%%%   Z8-II-3
Aby prirodzené číslo prinášalo Liborovi šťastie, musí byť jeho druhá mocnina
deliteľná číslami sedem, osem, deväť a desať. Nájdite všetky prirodzené čísla menšie
ako $1\,000$, ktoré Liborovi nosia šťastie.}
\podpis{L. Šimůnek}

{%%%%%   Z9-II-1
Barbora si napísala dve rôzne celé čísla. Potom ich sčítala, odčítala, vynásobila a vydelila. Dostala štyri výsledky, ktorých súčet bol $\m100$. Keď vynechala výsledok sčítania a spočítala ostatné tri výsledky, dostala tiež súčet $\m100$. Ktoré celé čísla mohla Barbora pôvodne napísať?}
\podpis{Š. Černíčková}

{%%%%%   Z9-II-2
Z kociek o hrane $1\cm$ sme postavili kváder. Keby sme z kvádra odobrali jeden stĺpec (\tj. zvislý komínik), zvyšok stavby by sa skladal zo 602 kociek. Keby sme miesto toho odobrali jeden riadok hornej vrstvy, zostala by nám stavba zo 605 kociek. Aké rozmery má kváder?}
\podpis{L. Šimůnek}

{%%%%%   Z9-II-3
Daný je štvorec $ABCD$ so stranou dĺžky $a$ a úsečka $KL$ dĺžky $5a$ tak, že $A\equiv K$ a strana $AB$ leží na úsečke $KL$. Štvorec $ABCD$ otočíme okolo pravého dolného bodu o pravý uhol a takéto otáčania robíme dovtedy, kým celá strana $AB$ po prvý raz opäť nesplynie s časťou úsečky $KL$ a bod $B$ s bodom $L$.
\begin{itemize}
\itemvar{a)} Narysujte po akej dráhe sa bude pohybovať stred $S$ strany $AB$.
\itemvar{b)} Vypočítajte dĺžku krivky, po ktorej sa bod $S$ pohyboval.
\end{itemize}
}
\podpis{M. Volfová}

{%%%%%   Z9-II-4
Severských pretekov psích záprahov sa zúčastnilo spolu 315 dvojzáprahov a trojzáprahov. Do cieľa dorazilo v stanovenom limite 60\% všetkých dvojzáprahov a jedna tretina všetkých trojzáprahov, čo predstavovalo polovicu všetkých psov na štarte. Koľko dvojzáprahov a koľko trojzáprahov štartovalo?}
\podpis{L. Černíček}

{%%%%%   Z9-III-1
V Dlhej Lehote volili starostu. Kandidovali dvaja občania: Ing. Schopný a jeho manželka Dr. Schopná. V obci boli tri volebné miestnosti. V prvej i druhej miestnosti dostala viac hlasov Dr. Schopná. V prvej bol pomer hlasov $7:5$, v druhej $5:3$. V tretej volebnej miestnosti bol pomer hlasov $3:7$ v prospech Ing. Schopného. Voľby nakoniec skončili nerozhodne, obaja kandidáti totiž získali rovnaký počet hlasov. V akom pomere boli počty odovzdaných platných hlasovacích lístkov v jednotlivých volebných miestnostiach, ak vieme, že v prvej a druhej miestnosti odovzdal platný hlas rovnaký počet občanov?}
\podpis{L. Šimůnek}

{%%%%%   Z9-III-2
Je daný rovnoramenný lichobežník $ABCD$ ($AB\parallel CD$), kde $|AB|>|CD|$. Bodom $A$ sa dajú viesť dve priamky tak, aby rozdelili lichobežník na tri rovnoramenné trojuholníky. Určite veľkosti uhlov lichobežníka $ABCD$.}
\podpis{L. Hozová}

{%%%%%   Z9-III-3
Nájdite všetky prirodzené čísla $x$, $y$, pre ktoré platí
$$
1 + x + y + xy = 2\,008.
$$}
\podpis{L. Hozová}

{%%%%%   Z9-III-4
...}
\podpis{...}

