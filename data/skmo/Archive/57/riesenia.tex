{%%%%%   A-I-1
Označme zadané rovnice postupne \thetag1, \thetag2, \thetag3.
Predpokladajme, že čísla $a$, $b$, $c$ majú požadovanú
vlastnosť. Všimneme si najskôr, že každé
dve z~daných rovníc musia mať spoločný koreň, inak by
mali spolu šesť rôznych koreňov.

Spoločné korene dvoch z~daných troch kubických rovníc sú korene
kvadratických rovníc, ktoré dostaneme ich odčítaním.
Vypíšme všetky tri "rozdielové" rovnice, ktoré sú nezávislé od
parametrov $a$, $b$, $c$ (to je pre vyriešenie pozitívne zistenie),
a~rozložme hneď ich ľavé strany na koreňové činitele:
$$
\align
x^2-2x+1=(x-1)^2&=0,\tag{$2\hbox{--}1$}\\
2x^2-x-1=(2x+1)(x-1)&=0,\tag{$3\hbox{--}1$}\\
x^2+x-2=(x-1)(x+2)&=0.\tag{$3\hbox{--}2$}
\endalign
$$
Vidíme, že rovnice \thetag1 a~\thetag2 majú jediný spoločný koreň $x=1$,
takže majú spolu práve päť rôznych koreňov. Preto musí byť
každý z~koreňov rovnice~\thetag3 koreňom aspoň jednej z~rovníc \thetag1 alebo
\thetag2. Z~uvedených rozkladov vyplýva, že číslo $x=1$ je aj koreňom
rovnice~\thetag3.

Vysvetlime, prečo ostatné dva korene rovnice~\thetag3
nemôžu byť zároveň aj korene jednej z~rovníc \thetag1 alebo \thetag2. V~opačnom prípade
by jedna z~rovníc \thetag1, \thetag2 mala
s~rovnicou~\thetag3 rovnakú trojicu koreňov, a~preto by museli mať rovnaké
koeficienty nielen pri kubickom člene. To však neplatí, lebo pre
ľubovoľnú hodnotu parametra~$c$
sú čísla $c+1$, $c+2$, $c+3$ (\tj. absolútne členy rovníc) všetky navzájom rôzne.

Rovnica~\thetag3 má teda okrem koreňa $x=1$ ešte jeden spoločný
koreň s~rovnicou~\thetag1 a~jeden spoločný koreň s~rovnicou~\thetag2;
podľa rozkladov \thetag{$3\hbox{--}1$} a~\thetag{$3\hbox{--}2$} vidíme, že sa jedná
o~čísla $x=\m\frac12$ a~$x=\m2$. Ľavá strana rovnice~\thetag3 má preto rozklad
$$
(x-1)(x+2)\left(x+\frac12\right)=x^3+\frac32x^2-\frac32x-1.
$$
Odtiaľ už porovnaním s~koeficientmi zapísanými v~\thetag3 dostaneme
$a=\m\frac32$, $b=\m\frac72$, $c=\m2$.

Z~nášho postupu vyplýva, že pre nájdené
hodnoty $a$, $b$, $c$ má rovnica~\thetag3 trojicu
koreňov $1$, $\m\frac12$ a~$\m2$, že čísla $1$, $\m\frac12$ sú
korene rovnice~\thetag1 a~že čísla $1$, $\m2$ sú korene rovnice~\thetag2.
Musíme sa ešte presvedčiť, že tretie korene rovníc \thetag1
a~\thetag2 sú ďalšie dve (rôzne) čísla. Tieto tretie korene môžeme
výhodne nájsť pomocou Vi\`etových vzťahov. Keďže súčin troch koreňov
rovnice~\thetag1 je číslo opačné k~absolútnemu členu $c+2$ rovnému
nule, je číslo nula tretí koreň rovnice~\thetag1.
Podobne súčin troch koreňov rovnice~\thetag2 je rovný $\m1$,
takže tretí koreň je číslo $x=\frac12$.

\zaver
Jediným riešením úlohy sú čísla $a=\m\frac32$, $b=\m\frac72$, $c=\m2$.

\návody
Nájdite (ak existujú) všetky spoločné korene
kubických rovníc
$$
\align
x^3+\bigl(2\sqrt2-1\bigr)x^2-\bigl(9\sqrt2+10\bigr)x+10\sqrt2+16&=0,\\
x^3+\bigl(2\sqrt2-3\bigr)x^2+\bigl(\sqrt2-6\bigr)x-10\sqrt2+16&=0.
\endalign
$$
[Spoločné korene
musia vyhovovať kvadratickej rovnici $x^2-\bigl(5\sqrt2+2\bigr)x+10\sqrt2=0$,
ktorú dostaneme, keď dané rovnice od seba odčítame a~výsledok
vydelíme dvoma. Táto rovnica má diskriminant $54-20\sqrt2$ rovný
$\bigl(5\sqrt2-2\bigr)^2$, takže jej korene sú čísla $2$
a~$5\sqrt2$. Dosadením sa presvedčíme, že spoločný koreň je
iba číslo~2. Na
precvičenie je možné dopočítať aj zvyšné korene daných kubických
rovníc po znížení ich stupňa zvyčajnou metódou
vydelením koreňovým činiteľom (v~našom prípade rovným $x-2$).
Pri prvej z~nich sú to čísla $\sqrt2+1$ a~$\m2-3\sqrt2$, pri druhej
čísla $\sqrt2-1$ a~$2-3\sqrt2$. Náročnejším miestom výpočtu je
odmocňovanie diskriminantov tvaru $a+b\sqrt2$ cestou riešenia rovnice
$a+b\sqrt2=\bigl(u+v\sqrt2\bigr)^2$ s~neznámymi celými číslami $u$, $v$.]

Zistite, pre ktoré reálne číslo~$p$ majú rovnice
$$
\align
x^3+x^2-36x-p&=0,\\
x^3-2x^2-px+2p&=0\\
\endalign
$$
spoločný koreň v~obore reálnych čísel. [52--A--S--3]

\D% D1.
Zistite, pre ktoré reálne čísla~$p$ má sústava
$$
\align
  x^2y - 2x &= p, \\
  y^2x - 2y &= 2p - p^2
\endalign
$$
práve tri riešenia v~obore reálnych čísel. [B--51--I--5]
\endnávod
}

{%%%%%   A-I-2
\fontplace
\trpoint A; \tlpoint B; \bpoint C;
\lBpoint A_0; \rtpoint C_0;
\rBpoint V; \tpoint V'; \bpoint O;
\rBpoint k;
[1] \hfil\Obr

\fontplace
\trpoint A; \tlpoint B; \lBpoint C;
\tpoint A_0; \lBpoint C_0;
\lpoint V; \lpoint V'; \bpoint O;
\rBpoint k;
[3] \hfil\Obr

\fontplace
\trpoint A; \tlpoint B; \bpoint C;
\rBpoint\xy.5,.5 V; \lBpoint O;
[2] \hfil\Obr

Najskôr dokážme jedno všeobecne užitočné tvrdenie
o~priesečníku~$V$ výšok %$AA_0$, $CC_0$
ľubovoľného ostrouhlého trojuholníka $ABC$.
Označme $V'$ priesečník priamky obsahujúcej výšku~$CC_0$ s~kružnicou
opísanou trojuholníku $ABC$ (\obr).
\inspicture{}
Pravouhlé trojuholníky $C_0V\!A$ a~$A_0VC$ sú podobné (zhodujú se ešte
v~uhle pri vrchole~$V$), preto $|\uhol BAA_0|=|\uhol BCC_0|$.
Uhly $BCC_0$ a~$V'\!AB$ sú zhodné obvodové uhly nad oblúkom~$V'\!B$,
takže {\it body $V$ a~$V'$ sú súmerne združené podľa priamky~$AB$}.

Ak označíme uhly v~trojuholníku $ABC$ zvyčajným spôsobom, tak $|\uhol ACV'|=
|\uhol ACC_0|=90^{\circ}-\al$, takže pre dĺžku úsečky~$AV$
%% z~pravoúhlého \tr-u $ACC_0$
vďaka uvedenej súmernosti
%% podle sinové věty pro \tr- $ACV'$
dostaneme
$$
|AV|=|AV'|=2r\sin(90^{\circ}-\al)=2r\cos\alpha,     \tag1
$$
pričom $r$ je veľkosť polomeru kružnice~$k$ opísanej trojuholníku $ABC$ (a~zároveň
aj trojuholníku $AV'C$).
Rovnaký vzťah~\thetag1 platí pre trojuholník $ABC$ s~ostrým vnútorným uhlom~$\al$
pri vrchole~$A$ aj v~prípade,
keď jeden z~ostatných dvoch vnútorných uhlov (napr\. pri vrchole~$B$)
je pravý alebo tupý (\obr). Celú úvahu môžeme doslova zopakovať.
\inspicture{}

Teraz sa už pustíme do riešenia úlohy
so zadanými bodmi $A$, $V$ a~danou veľkosťou ostrého uhla~$\al$.
Vzťah~\thetag1 nás privádza k~záveru, že
kružnice opísané všetkým uvažovaným trojuholníkom $ABC$
budú mať rovnaký polomer
$$
r=\frac{|AV|}{2\cos\al},    \tag2
$$
takže ich stredy~$O$ budú mať od daného bodu~$A$ pevnú, práve
určenú vzdialenosť~$r$. Je však potrebné určiť,
akú časť kružnice
$l(A,r)$ stredy~$O$ vyplnia; určite to bude množina súmerná podľa
priamky~$AV$, pretože súmernosť podľa osi~$AV$ zobrazuje vyhovujúci trojuholník
na vyhovujúci trojuholník.
S~týmto cieľom vyjadríme veľkosť uhla $V\!AO$ pomocou
vnútorných uhlov $\be=|\uhol ABC|$ a~$\ga=|\uhol ACB|$.
Budeme pritom predpokladať,
že $\be\ge\ga$
(v~opačnom prípade môžeme od úplného začiatku označenie
vrcholov $B$, $C$ navzájom vymeniť).

Predpokladajme najskôr, že $\be<90^{\circ}$, takže trojuholník $ABC$ je
ostrouhlý a~môžeme opäť pracovať s~\obrr2. Z~rovnoramenného trojuholníka
$ABO$ s~vnútorným uhlom~$2\ga$ pri hlavnom vrchole~$O$ vidíme, že
$|\uhol BAO|=90^{\circ}-\ga$, z~pravouhlého trojuholníka $BAA_0$ zasa
vyplýva $|\uhol BAV|=90^{\circ}-\be$. Vzhľadom na to, že oba
body $O$, $V$ ležia v~polrovine $ABC$,
dostávame pre uhol $V\!AO$ vyjadrenie
$$
\postdisplaypenalty 1000
|\uhol V\!AO|=|\uhol BAO|-|\uhol BAV|=
(90^{\circ}-\ga)-(90^{\circ}-\be)=\be-\ga
$$
(pripomeňme, že $\be\ge\ga$).

V~prípade $\be\ge90^{\circ}$ podľa \obrr1{} podobne zistíme, že
$|\uhol BAO|=90^{\circ}-\ga$ a~$|\uhol BAV|=\be-90^{\circ}$,
teda
$$
|\uhol V\!AO|=|\uhol BAO|+|\uhol BAV|=
(90^{\circ}-\ga)+(\be-90^{\circ})=\be-\ga.
$$
Vidíme, že $|\uhol V\!AO|=\be-\ga$ bez ohľadu na to,
či je trojuholník $ABC$ ostrouhlý, pravouhlý alebo tupouhlý.

Teraz už ľahko dokončíme riešenie úlohy. Z~odvodenej veľkosti
uhla $V\!AO$ vyplýva odhad
$$
|\uhol V\!AO|=\be-\ga<\be+\ga=180^{\circ}-\al,
$$
takže bod~$O$ leží vnútri oblúka kružnice $l(A,r)$ určeného
nerovnosťou
$$
|\uhol V\!AO|<180^{\circ}-\al.
$$
Ak naopak zvolíme uhol~$\ep$,
$0^{\circ}\le \ep<180^{\circ}-\al$, jednoducho
vypočítame, akú veľkosť musia mať vnútorné uhly $\be$ a~$\ga$,
aby platilo $|\uhol V\!AO|=\ep$:
$$
\be=\frac{180^{\circ}-\al+\ep}{2},\quad
\ga=\frac{180^{\circ}-\al-\ep}{2}.
$$
Ak teda vpíšeme do akejkoľvek kružnice s~polomerom~$r$ zo vzťahu \thetag2
pomocný trojuholník $A'B'C'$ s~daným uhlom~$\al$ pri vrchole~$A'$
a~vypočítanými uhlami $\be$, $\ga$ pri vrcholoch $B'$, resp\. $C'$,
pre jeho ortocentrum~$V'$ a~stred~$O'$ opísanej kružnice
budú splnené rovnosti $|A'V'|=|AV|$ a~$|\uhol V'A'O'|=\ep$.
V~zhodnom zobrazení, ktoré zobrazí úsečku~$A'V'$ na úsečku~$AV$,
sa potom trojuholník $A'B'C'$ zobrazí na vyhovujúci trojuholník $ABC$, ktorého
stred~$O$ opísanej kružnice bude ležať na kružnici~$l$ a~vyhovovať
rovnosti $|\uhol VAO|=\ep$.

\zaver
Hľadanou množinou stredov~$O$ opísaných
kružníc je oblúk kružnice so stredom~$A$
a~polomerom $r=\frac12|AV|/\cos\al$
určený nerovnosťou $|\uhol V\!AO|<180^{\circ}-\al$
(krajné body tohto oblúka teda do výslednej množiny nepatria, \obr).
\inspicture{}


\návody
Ukážte, že dve priamky, na ktorých ležia osi vnútorných
uhlov (resp\. výšky, resp\. spojnice vrcholov so stredom kružnice opísanej)
trojuholníka, zvierajú uhol, ktorého veľkosť závisí iba od
veľkosti vnútorného uhla pri zostávajúcom vrchole. Platí také tvrdenie
aj pre priamky, na ktorých ležia ťažnice trojuholníka?

Dokážte, že body súmerne združené s~priesečníkom výšok podľa
priamok, na ktorých ležia strany ostrouhlého trojuholníka,
ležia na kružnici trojuholníku opísanej. [Viď riešenie súťažnej úlohy.]
Odvoďte odtiaľ zaujímavé tvrdenie o~troch kružniciach, ktoré sú
obrazmi opísanej kružnice v~spomenutých troch osových súmernostiach.
[Tieto tri kružnice prechádzajú jedným bodom, totiž priesečníkom výšok.]

Vyjadrite vzdialenosti priesečníka výšok ostrouhlého trojuholníka od
jeho vrcholov v~závislosti od kosínusov jeho vnútorných uhlov a~polomeru kružnice opísanej.
[Viď riešenie súťažnej úlohy.]

\D% D1.
Ukážte, že z~úlohy~N2 vyplýva rovnaké tvrdenie aj pre tupouhlý trojuholník
pomocou tejto úvahy: Ak je $V$ priesečník výšok trojuholníka
$ABC$ s~tupým uhlom pri vrchole~$C$, tak je bod~$C$
priesečníkom výšok ostrouhlého trojuholníka $ABV$.
\endnávod
}

{%%%%%   A-I-3
Venujme sa najskôr najjednoduchšej situácii, keď $n=2$. Danú
množinu~$\mm M$ tak tvoria štyri kladné čísla,
ktoré označíme podľa veľkosti
$$
a_1<a_2<a_3<a_4.
$$
Máme iba tri možnosti, ako požadovaným spôsobom zostaviť
dvojicu obdĺžnikov. Vypíšme na troch riadkoch ich rozmery:
$$
\align
&a_1\times a_2\quad\text{a}\quad a_3\times a_4,\\
&a_1\times a_3\quad\text{a}\quad a_2\times a_4,\\
&a_1\times a_4\quad\text{a}\quad a_2\times a_3,\\
\endalign
$$
a~ukážme, že súčty obsahov týchto obdĺžnikov
sú v~uvedenom poradí klesajúce, \tj. že platí
$$
a_1a_2+a_3a_4>a_1a_3+a_2a_4>a_1a_4+a_2a_3.      \tag1
$$
Namiesto dvoch jednoduchých dôkazov\niedorocenky{ (urobte sami)} poznamenajme,
že obe nerovnosti sú rovnakého typu a~možno ich zdôvodniť
všeobecným pravidlom
$$
a<b,\ c<d\quad\Longrightarrow\quad ac+bd>ad+bc,        \tag2
$$
ktoré platí pre ľubovoľnú štvoricu reálnych čísel $a$, $b$, $c$, $d$ vďaka
rovnosti
$$
(ac+bd)-(ad+bc)=(b-a)(d-c).
$$

Naozaj, ľavú nerovnosť z~\thetag1 dostaneme z~pravidla~\thetag2 voľbou
$$
a=a_1,\quad b=a_4,\quad c=a_2,\quad d=a_3\quad
(\text{platí } a_1<a_4 \text{ a~}  a_2<a_3),
$$
pravú nerovnosť zasa voľbou
$$
a=a_1,\quad b=a_2,\quad c=a_3,\quad d=a_4\quad
(\text{platí } a_1<a_2 \text{ a~}  a_3<a_4).
$$
Tým je úloha v~prípade $n=2$ vyriešená. Táto skúsenosť
nás iste privedie k~odhadu výsledku pre všeobecné $n\ge2$:

{\sl
Ak $a_1<a_2<\dots<a_{2n}$ sú prvky danej množiny~$\mm M$, tak v~súčte
najväčší obsah má
jediná $n$-tica obdĺžnikov s~rozmermi
$
a_1\times a_2, a_3\times a_4,\dots,a_{2n-1}\times a_{2n}
$;
v~súčte najmenší obsah má
jediná $n$-tica obdĺžnikov s~rozmermi
$
a_1\times a_{2n}, a_2\times a_{2n-1},\dots,a_{n}\times a_{n+1}
$.}

Pre dôkaz prvého záveru predpokladajme, že vyhovujúca $n$-tica
obdĺžnikov je zostavená tak, že čísla
$a_1$, $a_2$ nie sú rozmery toho istého obdĺžnika. Potom v~takej
$n$-tici sú obdĺžniky $a_1\times a_i$
a~$a_2\times a_j$, kde $i,j>2$. Nahraďme ich obdĺžnikmi $a_1\times a_2$ a~$a_i\times
a_j$. Dostaneme (inú) vyhovujúcu $n$-ticu obdĺžnikov, ktorá bude
mať oproti pôvodnej $n$-tici v~súčte väčší obsah, lebo platí
$$
a_1a_2+a_ia_j>a_1a_i+a_2a_j,
$$
a~to opäť vďaka pravidlu~\thetag2 pre čísla $a_1<a_j$ a~$a_2<a_i$.
Z~tejto úvahy vyplýva, že v~súčte najväčší obsah môže mať len taká
$n$-tica uvažovaných obdĺžnikov, medzi ktorými je obdĺžnik $a_1\times
a_2$. Tento obdĺžnik môžeme teda dať bokom a~uvažovať úlohu
o~najmenšom obsahu pre
redukovanú množinu~$\mm M'$ s~$2n-2$ prvkami
$a_3<a_4<\dots<a_{2n}$. Opakovaním predchádzajúceho postupu vytvoríme
obdĺžnik $a_3\times a_4$ a~urobíme ďalšiu redukciu množiny atď.
(formálne
môžeme využiť matematickú indukciu). Hypotéza o~zostave obdĺžnikov
v~súčte s~najväčším obsahom je tak dokázaná.

Veľmi podobne dokážeme záver o~zostave v~súčte s~najmenším
obsahom. Ak $a_1$, $a_{2n}$ nie sú rozmery toho istého obdĺžnika,
sú medzi uvažovanými obdĺžnikmi aj obdĺžniky $a_1\times a_i$
a~$a_j\times a_{2n}$ (pričom $1<i,j<2n$), ktoré nahradíme obdĺžnikmi
$a_1\times a_{2n}$ a~$a_i\times a_j$. Tým sa v~súčte obsah obdĺžnikov
zmenší, lebo podľa pravidla~\thetag2 pre čísla $a_1<a_j$ a~$a_i<a_{2n}$ platí
$$
a_1a_i+a_ja_{2n}>a_1a_{2n}+a_ia_j.
$$
V~súčte najmenší obsah preto
môže mať len taká vyhovujúca $n$-tica obdĺžnikov,
medzi ktorými je obdĺžnik $a_1\times a_{2n}$. Tento obdĺžnik dáme bokom
a~uvažujeme úlohu o~najmenšom obsahu
pre redukovanú množinu $\mm M'$ s~$2n-2$ prvkami
$a_2<a_3<\dots<a_{2n-1}$. Všetko ostatné je už zbytočné opakovať.

\návody
Dokážte pravidlo~\thetag2 z~riešenia úlohy. [Dôkaz je v~uvedenom riešení.]

\D% D1.
Pravidlo~\thetag2 spomenuté v~úlohe~N1 využite na dôkaz tzv. {\it
nerovností usporiadania}: Ak sú $a_1\le a_2\le\dots\le a_n$
a~$b_1\le b_2\le\dots\le b_n$ dve $n$-tice reálnych čísel
a~$(x_1,x_2,\dots,x_n)$, resp. $(y_1,y_2,\dots,y_n)$ ich
ľubovoľné permutácie, tak pre súčet
$S=x_1y_1+x_2y_2+\cdots+x_ny_n$ platí
$S_{\min}\le S\le S_{\max}$, kde
$S_{\min}=a_1b_n+a_2b_{n-1}+\cdots+a_nb_n$
a~$S_{\max}=a_1b_1+a_2b_2+\cdots+a_nb_n$.
[Návod: Ako v~riešení súťažnej úlohy ukážte, že súčet~$S$
možno zväčšiť, ak sú medzi jeho sčítancami
$x_ky_k$ členy $a_1b_i$ a~$a_jb_1$, pričom $a_j>a_1$
a~$b_i>b_1$. Podobne možno súčet~$S$ zmenšiť v~prípade sčítancov
$a_1b_i$ a~$a_jb_n$, pokiaľ $a_j>a_1$ a~$b_n>b_i$. Takých
zväčšení (zmenšení) možno opakovane urobiť len konečne veľa.]

% D2.
Nerovnosť usporiadania z~úlohy~D1 využite na dôkaz nerovností
$$
a^3b+b^3c+c^3a\le a^4+b^4+c^4\quad\text{a}\quad
\frac{a^3}{b}+\frac{b^3}{c}+\frac{c^3}{a}\ge a^2+b^2+c^2
$$
pre ľubovoľné kladné čísla $a$, $b$, $c$. [Ak $p\le q\le r$ je neklesajúce
poradie čísel $a$, $b$, $c$, tak $p^3\le q^3\le r^3$
a~$p^{\m1}\ge q^{\m1}\ge r^{\m1}$.]

% D3.
Zachovajme predpoklady a~označenie z~úlohy~D2.
Ukážte, že sčítaním $n$~vhodných
nerovností usporiadania možno odvodiť tzv. {\it Čebyševove
nerovnosti}
$$
n\cdot S_{\min}\le(a_1+a_2+\cdots+a_n)(b_1+b_2+\cdots+b_n)\le
n\cdot S_{\max}.
$$
S~ich pomocou potom dokážte, že pre ľubovoľné kladné $a$, $b$, $c$
platí
$$
(a^2+b^2+c^2)(a^3+b^3+c^3)\le 3(a^5+b^5+c^5)\le
(a^7+b^7+c^7)(a^{-2}+b^{-2}+c^{-2}).
$$

% D4.
Čísla od 1 do 2\,000 boli rozdelené na 1\,000 (disjunktných)
dvojíc $(a_i,b_i)$ tak, že pre každé $i=1,2,\dots,1\,000$ je
rozdiel $|a_i-b_i|$ rovný jednému z~čísel 1 alebo 6. Určte, akou
číslicou končí desiatkový zápis čísla
$$
S=|a_1-b_1|+|a_2-b_2|+\dots+|a_{1000}-b_{1000}|.
$$
[Nulou. Platí $S=1\,000+5p$, kde $p$ je počet dvojíc $(a_i,b_i)$
s~vlastnosťou $|a_i-b_i|=6$. Počet tých dvojíc, v~ktorých sú
obe čísla nepárne, sa musí rovnať počtu tých dvojíc, kde sú obe
párne. Preto je číslo~$p$ párne.]

% D5.
Pre dané prirodzené $n\ge2$ rozdeľme množinu
$\{1,2,3,\dots,2n\}$ ľubovoľným spôsobom na dve (disjunktné)
$n$-prvkové množiny $\mm A$ a~$\mm B$. Prvky z~$\mm A$ označme v~rastúcom
poradí $a_1<a_2<\dots<a_n$, prvky z~$\mm B$ v~klesajúcom
poradí $b_1>b_2>\dots>b_n$. Nájdite všetky možné hodnoty
súčtu
$$
S=|a_1-b_1|+|a_2-b_2|+\dots+|a_n-b_n|.
$$
[Všetky súčty majú rovnakú hodnotu
$(n+1)+(n+2)+\dots+2n-(1+2+\dots+n)=n^2$. Návod: Pre každé~$i$
je menšie z~čísel $a_i$, $b_i$ menšie než $n-i$ čísel z~jednej množiny
a~$i$~čísel z~druhej množiny, čo dokopy znamená, že je menšie ako niektorých
$n$~čísel z~celej množiny $\{1,2,3,\dots,2n\}$, musí preto ležať
v~množine $\{1,2,\dots,n\}$. Podobne väčšie z~čísel $a_i$, $b_i$
musí ležať v~množine $\{n+1,n+2,\dots,2n\}$.]
\endnávod
}

{%%%%%   A-I-4
Zo zadania úlohy vyplýva, že všetky členy uvažovaných postupností
budú deliteľmi ich posledného člena, rovného číslu
$969\,969$. Nájdime preto najskôr rozklad tohto čísla na
súčin prvočísel:
$$
969\,969=3\cdot7\cdot11\cdot13\cdot17\cdot19.      \tag1
$$
Teraz už ľahko môžeme vytvárať príklady vyhovujúcich postupností
rôznych dĺžok. Vypíšme napríklad tú najkratšiu, jednu
z~najdlhších a~ešte jednu ďalšiu:
$$
\align
(a_1,a_2)&=(1,969\,969),\\
(a_1,a_2,a_3,a_4,a_5,a_6,a_7)&=(1,13,91,1\,729,5\,187,57\,057,969\,969),\\
(a_1,a_2,a_3,a_4)&=(1,21,4\,641,88\,179,969\,969).
\endalign
$$
(Skontrolujte uvedené príklady výpočtom podielu $a_{i+1}/a_{i}$
pre všetky prípustné~$i$).

Experimentovaním s~konkrétnymi postupnosťami dôjdeme k~poznaniu
ich spoločných vlastností, ktoré ich úplne charakterizujú:

{\sl
Ľubovoľný člen~$a_i$ každej vyhovujúcej postupnosti $a_1,a_2,\dots,a_k$
je súčinom niekoľkých (v~prípade $i=1$ žiadneho, v~prípade $i=k$
všetkých) z~šiestich rôznych prvočísel z~rozkladu~\thetag1,
pritom (v~prípade $i<k$) člen~$a_{i+1}$ má okrem všetkých činiteľov
člena~$a_{i}$ ešte aspoň jedného nového činiteľa navyše
(postupnosť má byť rastúca!).
Naopak, každá takáto konečná postupnosť je vyhovujúca.
}

Z~uvedeného vyplýva spôsob, ako "úsporne" zadať každú
vyhovujúcu postupnosť; stačí len uviesť, ako sa nové činitele
postupne objavujú, \tj. zadať postupnosť podielov
$$
\frac{a_2}{a_1},\ \frac{a_3}{a_2},\ \frac{a_4}{a_3},\
\dots,\ \frac{a_{k-1}}{a_{k-2}},\ \frac{a_k}{a_{k-1}},
\tag2
$$
do ktorých rozkladov na súčin prvočísel je šesť prvočísel
z~\thetag1 rozdelených (v~každom aspoň jedno). Preto je hľadaný počet
vyhovujúcich postupností rovný počtu rozdelení šiestich daných
prvočísel do jednej alebo niekoľkých {\it očíslovaných\/}
neprázdnych skupín (zodpovedajúcich prvočiniteľom podielov~\thetag2,
takže na poradí prvočísel v~skupine nezáleží).
Slovo "očíslovaných" znamená,
že na poradí skupín záleží. Napríklad pre rozdelenie do dvoch
skupín $\{3,11,19\}$, $\{7,13,17\}$ dostaneme podľa toho,
v~akom poradí obe skupiny vezmeme, dve vyhovujúce postupnosti
$(1,u,uv)$ a~$(1,v,uv)$, pričom $u=3\cdot11\cdot19$
a~$v=7\cdot13\cdot17$.

Dospeli sme tak ku kombinatorickej úlohe určenia hodnoty $P(6)$,
pričom $P(n)$ označuje
počet rozdelení $n$-prvkovej množiny~$X$
na ľubovoľný počet očíslovaných neprázdnych podmnožín
$X_1$, $X_2$, $X_3$,~\dots{} Nie je ľahké hodnotu $P(6)$
vypočítať {\it priamo}, avšak bude možné hodnoty $P(n)$ počítať
{\it postupne\/} pre $n=1$, $n=2$, atď\. až po potrebné $n=6$.
Takému spôsobu výpočtu hovoríme {\it rekurentný}. V~našej úlohe
bude výpočet založený na rekurentnom vzťahu
$$
P(n)=\binom{n}{1}P(n-1)+\binom{n}{2}P(n-2)+\dots+
\binom{n}{n-1}P(1)+1
\tag3
$$
platnom pre každé $n\ge2$, ako teraz ukážeme.

Všetky uvažované rozdelenia $n$-prvkovej množiny~$X$
rozdelíme na $n$~skupín podľa
počtu~$j$ prvkov prvej podmnožiny~$X_1$ ($1\le j\le n$).
Prvú podmnožinu~$X_1$ s~$j$~prvkami možno vybrať
práve $\binom{n}{j}$ spôsobmi, práve $P(n-j)$ spôsobmi potom možno
zvyšnú množinu $X'=X\setminus X_1$ rozdeliť na neprázdne
očíslované podmnožiny $X_2$, $X_3$, $X_4$, \dots{}
(Platí to aj v~prípade $j=n$, keď položíme $P(0)=1$, keďže už
nie je čo rozdeľovať.) Podľa pravidla súčinu je preto
počet všetkých rozdelení pôvodnej množiny~$X$ s~prvou množinou~$X_1$
majúcou $j$~prvkov rovný $\binom{n}{j}P(n-j)$. Tým je vzťah~\thetag3,
na ktorého pravej strane posledný člen~1 zodpovedá hodnote
$j=n$, dokázaný.

Zo zrejmej hodnoty $P(1)=1$ vypočítame opakovaným použitím vzťahu~\thetag3
ďalšie hodnoty
$P(2)=3$, $P(3)=13$, $P(4)=75$, $P(5)=541$ a~$P(6)=4\,683$.

\zaver
Existuje práve $4\,683$ vyhovujúcich postupností.


\návody
Koľkými spôsobmi možno číslo $49\,000$ rozložiť na súčin dvoch
celých čísel väčších ako~1, keď na poradí činiteľov nezáleží?
[23 spôsobov. Návod: číslo $49\,000=2^3\cdot5^3\cdot7^2$
má $(3+1)\cdot(3+1)\cdot(2+1)=48$ deliteľov, z~ktorých môžeme
utvoriť 24~neusporiadaných dvojíc $\{a,b\}$ s~vlastnosťou
$ab=49\,000$. Jedna z~nich je nevyhovujúca dvojica $\{1,49\,000\}$.]

Určte počet $P(n)$ spôsobov, ako si rozdeliť vianočnú zásobu
$n$~rovnakých cukríkov na zjedenie v~priebehu niekoľkých prvých
dní nového roku.
(V~každom z~daných dní musíme mať aspoň jeden cukrík.
Je možné aj také "rozdelenie", keď
všetky cukríky zjeme na Nový rok.) Ako sa zmení odpoveď,
keď každé dva z~daných
$n$~cukríkov budú rôzne? [Pre prípad rovnakých cukríkov platí
$P(n)=2^{n-1}$ podľa pravidla súčinu použitého na postupné
rozdeľovanie cukríkov: prvý predelíme na Nový rok
a~každý nasledujúci cukrík buď na rovnaký deň
ako predchádzajúci cukrík, alebo na deň nasledujúci. Pre prípad
rôznych cukríkov je situácia zložitejšia a~vedie na rekurentný
vzťah~\thetag3 z~riešenia súťažnej úlohy.]

\D% D1.
Určte, koľko čísel môžeme vybrať z~množiny
$\{1,2,3,\dots,75\,599,75\,600\}$ tak, aby medzi nimi bolo číslo 75\,600
a~aby pre ľubovoľné dve vybrané čísla $a$, $b$ platilo, že $a$ je
deliteľom $b$ alebo $b$ deliteľom $a$. (Uveďte všetky možnosti.)
[51--B--I--2]

% D2.
Uvažujme množinu $\{1,2,4,5,8,10,16,20,32,40,80,160\}$
a~všetky jej trojprvkové podmnožiny. Rozhodnite, či je viac tých,
ktoré majú súčin svojich prvkov väčší ako 2006,
alebo tých, ktoré majú súčin svojich prvkov menší ako 2006.
[56--A--S--2]
\endnávod
}

{%%%%%   A-I-5
\fontplace
\rtpoint M; \ltpoint N; \bpoint\xy-.8,0 S;
\tpoint\xy-1.6,0 O; \rtpoint P; \tpoint R;
\bpoint\up\unit A; \lbpoint B;
\lpoint p'; \lpoint p; \tpoint q;
\rBpoint k; \bpoint l; \tpoint m;
[4] \hfil\Obr

Jedna z~vyhovujúcich kružníc~$l$ je znázornená na \obr. Bod~$A$
vonkajšieho dotyku kružníc $k$, $l$ je ich (vnútorným) stredom
rovnoľahlosti, v~ktorej dotyčnici~$p$ kružnice~$l$ zodpovedá s~ňou
rovnobežná dotyčnica~$p'$ kružnice~$k$.
%% Snadno vysvětlíme, že přímka
%% $p'$ je pro všechny vyhovující kružnice $l$ společná:
Jej bod dotyku~$M$ s~kružnicou~$k$ leží na osi~$q$ kružnice~$k$,
ktorá je kolmá na~priamku~$p$. Pritom z~dvoch priesečníkov $M$, $N$
priamky~$q$ s~kružnicou~$k$ je bod~$M$ ten vzdialenejší od
priamky~$p$, lebo úsečka spájajúca rovnoľahlé
body dotyku $M$ a~$B$ pretína kružnicu~$k$ v~bode~$A$
(v~strede príslušnej rovnoľahlosti).
\inspicture{}

Bod~$M$ teda od voľby kružnice~$l$
nezávisí. Body $A\in k$ a~$B\in p$ pochopiteľne áno,
ukážme však, že ich vzájomná poloha na polpriamke s~počiatkom~$M$
je viazaná podmienkou
$$
|MA|\cdot|MB|=|MN|\cdot|MP|,       \tag1
$$
kde $P$ je priesečník kolmíc $p$ a~$q$. To jednoducho vyplýva
z~podobnosti
$$
|MA|:|MN|=|MP|:|MB|
$$
pravouhlých trojuholníkov $AMN$, $PMB$. Vzťah~\thetag1
možno tiež zdôvodniť pomocou mocnosti bodu~$M$ ku kružnici
zostrojenej nad priemerom~$NB$ (ktorá prechádza bodmi $P$, $A$
podľa Tálesovej vety).

Až teraz vstúpi do našich úvah daný bod~$O$. Na \obrr1{} je
kružnica~$l$ vybraná tak, že zodpovedajúca priamka~$AB$ bodom~$O$
neprechádza, takže existuje kružnica~$m$ opísaná trojuholníku $OAB$.
Podľa zadania $O\notin k$, a~teda $O\ne M$, takže je určená
polpriamka~$MO$, ktorá okrem bodu~$O$ bude mať s~kružnicou~$m$
spoločný ešte jeden bod, ktorý označíme~$R$
(v~prípade, keď $MO$ je dotyčnica kružnice~$m$, položíme
$R=O$).\footnote{Zdôraznime, že vzhľadom na vzájomnú polohu
bodov $M$, $A$, $B$
leží bod~$M$ vo vonkajšej oblasti {\it každej\/}
kružnice prechádzajúcej bodmi $A$, $B$, teda aj kružnice~$m$.
Polpriamka~$MO$ má teda s~kružnicou~$m$, ak nie je jej dotyčnicou,
spoločné skutočne dva rôzne body.}
Dvojakým vyjadrením mocnosti bodu~$M$ ku kružnici~$m$ potom dostaneme
$$
|MA|\cdot|MB|=|MO|\cdot|MR|,
$$
odkiaľ porovnaním s~\thetag1 zistíme, že úsečka~$MR$ má dĺžku
$$
|MR|=\frac{|MN|\cdot|MP|}{|MO|},
$$
ktorá zrejme nezávisí od voľby kružnice~$l$. Keďže
bod~$R$ navyše leží na pevnej polpriamke~$MO$, je v~prípade
$|MR|\ne|MO|$ bod~$R$ spoločným bodom všetkých kružníc~$m$ ($R\ne O$),
v~prípade $|MR|=|MO|$ je priamka~$MO$ ich spoločná dotyčnica. Tým je
riešenie úlohy na konci.


\návody
{\everypar{}%
Zopakujte si najskôr učebnicové poznatky o~rovnoľahlosti dvoch
kružníc (obzvlášť prípad, keď sa kružnice dotýkajú)
a~ich rovnobežných (špeciálne spoločných) dotyčníc. Pripomeňte si
tiež vlastnosť všetkých sečníc danej kružnice prechádzajúcich daným bodom,
vyjadrenú mocnosťou bodu ku kružnici.
\par}
V~rovine je daná kružnica~$k$, priamka~$p$ a~bod $B\in p$.
Zostrojte kružnicu~$l$, ktorá sa dotýka ako kružnice~$k$, tak
priamky~$p$, a~to v~bode~$B$. [Jedna zo známych
tzv. {\it Pappových úloh}.]

\D% D1.
V~rovine sú dané kružnice $k_1(S_1,r_1)$ a~$k_2(S_2,r_2)$ tak,
že $S_2\in k_1$ a~$r_1>r_2$. Spoločné dotyčnice oboch kružníc sa
dotýkajú kružnice~$k_1$ v~bodoch $P$ a~$Q$. Dokážte, že priamka~$PQ$
sa dotýka kružnice~$k_2$. [52--A--S--2]

% D2.
Sú dané kružnice $k$ a~$l$ s~rôznymi polomermi, ktoré sa zvonka
dotýkajú v~bode~$T$. Priesečníkom~$M$ ich spoločných vonkajších
dotyčníc veďme sečnicu~$s$ oboch kružníc. Označme $X$ ten
z~oboch priesečníkov kružnice~$k$ so sečnicou~$s$, ktorý je vzdialenejší
od bodu~$M$. Podobne označme $Y$ ten z~oboch priesečníkov kružnice~$l$
so sečnicou~$s$, ktorý je vzdialenejší od bodu~$M$. Nech $P$ je
taký bod, že $XTY\!P$ je rovnobežník. Určte množinu bodov~$P$
zodpovedajúcich všetkým takým sečniciam~$s$. [49--B--I--2]

% D3.
Je daný rovnoramenný trojuholník $ABC$ so základňou~$AB$. Na jeho výške~$CD$
je zvolený bod~$P$ tak, že kružnice vpísané trojuholníku $ABP$
a~štvoruholníku $PECF$ sú zhodné; pritom bod~$E$ je
priesečník priamky~$AP$ so stranou~$BC$ a~$F$ priesečník priamky~$BP$
so stranou~$AC$. Dokážte, že aj kružnice vpísané trojuholníkom $ADP$
a~$BCP$ sú zhodné. [49--A--III--2]
\endnávod
}

{%%%%%   A-I-6
Začneme trochu obšírnejšie prípadom $n=1$.
Nájdeme {\it všetky\/} celé čísla~$a$ s~vlastnosťou $5\mid a^3-a+1$.
Najskôr zostavíme tabuľku
hodnôt $r^3-r+1$ pre všetky možné zvyšky~$r$ po delení piatimi,
teda pre $r\in\{0,1,2,3,4\}$:
$$
\vbox{\everycr{\noalign{\hrule}}\offinterlineskip\let\\=\cr
\halign{\strut\vrule#&&\enspace\hss$#$\enspace \hss\vrule\cr
&r&0&1&2&3&4\\
&r^3-r+1&1&1&7&25&61\cr
}}
$$
Pre ostatné celé čísla~$a$ už hodnoty $a^3-a+1$ počítať nemusíme.
Ak je totiž $r$ zvyšok čísla~$a$ po delení piatimi, teda
$a=5q+r$ pre vhodné celé~$q$, tak čísla $a^3-a+1$ a~$r^3-r+1$
dávajú po delení piatimi rovnaký zvyšok, lebo ich rozdiel
$$
(a^3-a+1)-(r^3-r+1)=(a^3-r^3)-(a-r)=(a-r)(a^2+ar+r^2-1)
$$
je deliteľný číslom $a-r=5q$, je teda násobkom
piatich.\footnote{Rovnako ľahko sa dokáže všeobecnejší užitočný poznatok:
pre ľubovoľný mnohočlen~$F$
s~celočíselnými koeficientmi a~ľubovoľné celé $a$, $b$ je
rozdiel $F(a)-F(b)$ celočíselným násobkom rozdielu $a-b$.} Z~uvedenej
tabuľky vidíme, že pre celé~$a$ platí $5\mid a^3-a+1$ práve vtedy, keď
$a=5q+3$.

Zadanú úlohu vyriešime tak, že indukciou
vzhľadom na číslo~$n$ dokážeme existenciu
celého čísla~$a_n$ z~intervalu $(1,5^n)$,
ktoré vyhovuje podmienke $5^n\mid a_n^3-a_n+1$.
Pre $n=1$ podľa prvého odstavca
dokazované tvrdenie spĺňa (v~intervale $(1,5)$!)
jediné číslo $a_1=3$.

V~druhom indukčnom kroku predpokladajme, že
pre niektoré prirodzené~$k$ poznáme číslo~$a_k$ z~intervalu
$(1,5^k)$ s~vlastnosťou $5^k\mid a_k^3-a_k+1$, a~na základe znalosti
$a_k$ zostrojme vyhovujúce číslo $a_{k+1}$.
Zvyškom čísla $a_k^3-a_k+1$ po delení
číslom $5^{k+1}$ musí byť číslo deliteľné~$5^k$, teda jedno
z~čísel
$$
0,\ 5^k,\ 2\cdot5^k,\ 3\cdot5^k,\ 4\cdot5^k.
$$
Zapíšme preto tento zvyšok v~tvare $r\cdot5^k$, pričom
$r\in\{0,1,2,3,4\}$, a~hľadajme číslo $a_{k+1}$ v~tvare
$a_{k+1}=a_k+s\cdot5^k$ pre vhodné $s\in\{0,1,2,3,4\}$. (Je hneď
jasné, že v~prípade $r=0$ môžeme zobrať $a_{k+1}=a_k$, teda $s=0$).
Aj keď hodnotu~$s$ vyberieme až za chvíľu, z~podmienky $1<a_k<5^{k}$
a~nerovností $a_k\le a_{k+1}\le a_k+4\cdot5^k$ už teraz vyplýva,
že podmienka $1<a_{k+1}<5^{k+1}$ bude splnená (nech dopadne výber $s$
akokoľvek). Pre číslo $a_{k+1}$
zvoleného tvaru dostávame
$$
\align
\frac{a_{k+1}^3-a_{k+1}+1}{5^{k+1}}&=
\frac{\left(a_k+s\cdot5^k\right)^3-\left(a_k+s\cdot5^k\right)
+1}{5^{k+1}}=\\
&=\frac{a_k^3+3a_k^2s\cdot5^k+3a_ks^2{\cdot}5^{2k}+s^3{\cdot}5^{3k}-
a_k-s\cdot5^k+1}{5^{k+1}}=\\
&=3a_ks^2{\cdot}5^{k-1}+s^3{\cdot}5^{2k-1}+
\frac{\left(a_k^3-a_k+1\right)-r\cdot5^k}{5^{k+1}}+
\frac{\left(3a_k^2-1\right)s+r}{5}.
\endalign
$$
Ak budú oba záverečné zlomky celočíselné, bude taká aj hodnota celého posledného súčtu.
Prvý zlomok túto vlastnosť má vďaka tomu,
ako sme zaviedli číslo $r\in\{0,1,2,3,4\}$. Preto je
len potrebné nájsť také $s\in\{0,1,2,3,4\}$, aby aj druhý zlomok
bol celočíselný, teda aby číslo $\left(3a_k^2-1\right)s+r$ bolo
deliteľné piatimi. Stačí ukázať, že päť čísel
$$
c(s)=\left(3a_k^2-1\right)s+r,\quad\text{pričom}\quad s\in\{0,1,2,3,4\},
$$
dáva po delení piatimi navzájom rôzne zvyšky (jeden z~nich potom bude
nula). Keby to tak nebolo, platilo by $5\mid c(s)-c(s')$ pre
niektoré dve rôzne $s,s'\in\{0,1,2,3,4\}$;
z~vyjadrenia
$$
c(s)-c(s')=\left(3a_k^2-1\right)(s-s')
$$
by sme potom usúdili, že číslo $3a_k^2-1$ je deliteľné piatimi.
Vzťah $5\mid 3a^2-1$ však neplatí pre žiadne celé~$a$; podľa úvah
z~prvého odstavca sa stačí o~tom presvedčiť pre päť hodnôt
$a\in\{0,1,2,3,4\}$:
$$
\vbox{\everycr{\noalign{\hrule}}\offinterlineskip\let\\=\cr
\halign{\strut\vrule#&&\enspace\hss$#$\enspace \hss\vrule\cr
&a&0&1&2&3&4\\
&3a^2-1&\m1&2&11&26&47\cr
}}
$$

Tým je celý dôkaz matematickou indukciou ukončený.
Pre zaujímavosť dodajme, že sme schopní ľahko vysvetliť, že
naše číslo $3a_k^2-1$ dáva po delení piatimi vždy zvyšok~1
(takže v~prípade $r\ne0$ vyhovuje $s=5-r$). Naozaj, vzhľadom
na to, že $k\ge1$, z~podmienky $5^k\mid a_k^3-a_k+1$ vyplýva
$5\mid a_k^3-a_k+1$, čo je podľa prvého odstavca splnené
práve vtedy, keď $a_k=5k+3$; číslo $3a_k^2-1$ teda po delení piatimi
dáva rovnaký zvyšok ako číslo $3\cdot3^2-1=26$.

\návody
Dokážte, že pre každé prirodzené číslo~$n$ existuje celé číslo~$a$
také, že $2^n\mid a^2+2\,007$. [Indukciou nájdeme celé~$a_n$
s~vlastnosťou $2^n\mid a_n^2+2\,007$.
Zrejme vyhovuje $a_1=a_2=a_3=1$,
a~keď máme pre niektoré $k\ge3$ vyhovujúce $a_k$,
tak položíme buď $a_{k+1}=a_{k}$, alebo $a_{k+1}=a_{k}+2^{k-1}$,
podľa toho, či číslo
$a_k^2+2\,007$ dáva po delení číslom $2^{k+1}$ zvyšok~$0$, alebo
zvyšok~$2^k$ (iný zvyšok podmienka $2^k\mid a_k^2+2\,007$
vylučuje). V~druhom prípade
$$
\frac{a_{k+1}^2+2\,007}{2^{k+1}}=
\frac{a_{k}^2+2\,007-2^k}{2^{k+1}}+
\frac{a_k+1}{2}+2^{k-3},
$$
čo je celé číslo, lebo $a_k$ je vzhľadom na
$2^k\mid a_k^2+2\,007$ nutne nepárne.]

\D% D1.
Číslo $1\,997^{2^{n}}-1$ je deliteľné číslom $2^{n+2}$ pre
každé prirodzené číslo~$n$. Dokážte. [47--A--I--1]

% D2.
Číslo $1\,997^{3^n}+1$ je deliteľné číslom~$3^{n+3}$ pre každé
prirodzené číslo~$n$. Dokážte. [47--A--II--1]
\endnávod
}

{%%%%%   B-I-1
Opakovaným násobením číslom~$6$ zistíme, že posledné dvojčíslia
mocnín~$6^k$ pre $k=1,2,3,\dots$ sú postupne
$$
06,36,16,96,76,56,36,16,96,76,56,\dots,
\tag1
$$
opakujú sa teda od druhého člena s~periódou dĺžky~$5$. Podobne
opakovaným násobením číslom~$7$ zistíme, že posledné dvojčíslia
mocnín~$7^m$ pre $m=1,2,3,\dots$ sú postupne
$$
07,49,43,01,07,49,43,01,\dots,
\tag2
$$
opakujú sa teda už od prvého člena s~periódou dĺžky~$4$.

\smallskip
a) Pretože každá mocnina šiestich je zakončená číslicou~$6$, bude
číslo $6^k\cdot 7^{2007-k}$ zakončené dvojkou jedine vtedy,
keď bude číslo $7^{2007-k}$ zakončené dvojčíslím~$07$
(iné dvojčíslie z~\thetag2 nevyhovuje). Násobením
číslami $6$, $36$, $16$, $96$, $76$ a~$56$ však zistíme, že číslo $6^k\cdot
7^{2007-k}$ môže mať v~takom prípade na predposlednom mieste iba
niektorú z~číslic $1$, $3$, $4$, $5$, $7$, $9$. Zakončenie dvojčíslím~$02$
preto nie je možné.

\smallskip
b) Keďže každá mocnina šiestich je zakončená číslicou $6$, bude
číslo $6^k\cdot 7^{2007-k}$ zakončené štvorkou práve vtedy, keď
$7^{2007-k}$ bude zakončené dvojčíslím~$49$ (iné dvojčíslie
z~\thetag2 nevyhovuje). Násobením všetkými rôznymi číslami z~\thetag1 zistíme,
že $6^k\cdot 7^{2007-k}$ je
zakončené dvojčíslím~$04$ jedine vtedy, keď $6^k$ končí dvojčíslím~$96$.
Číslo~$6^k$ končí na~$96$ práve vtedy, keď je exponent~$k$
tvaru $k=4+5a$; číslo $7^{2007-k}$ končí na~$49$ práve vtedy,
keď príslušný exponent má tvar $2\,007-k=2+4b$. Dosadením $k=4+5a$
dostaneme rovnicu $2\,007-4-5a=2+4b$, pričom $a$ a~$b$ sú celé
nezáporné čísla. Z~nej vychádza
$$
b=\frac{2\,001-5a}4=500-a-\frac{a-1}4.
$$
Aby bolo $b$
celé, musí byť $a-1$ deliteľné štyrmi, teda $a=4c+1$; potom
$b=499-5c$, $k=9+20c$. Exponent $2\,007-k$ rovný $1\,998-20c$
nemôže byť záporný, preto $c\le99$.

 Číslo $6^k\cdot 7^{2007-k}$ je zakončené dvojčíslím~$04$ práve vtedy,
keď je číslo~$k$ tvaru $k=9+20c$, pričom $c\in\{0, 1, 2, \dots, 99\}$.

\niedorocenky{
\poznamka
Rovnica tvaru $ax+by=c$, kde $a$, $b$, $c$ sú dané
celé čísla a~$x$, $y$ celočíselné neznáme, sa nazýva {\it lineárna
diofantická rovnica\/} o~dvoch neznámych. Žiaci by sa mali oboznámiť
s~riešením takých rovníc.}

\návody
Zistite, ktorým dvojčíslím končí dekadický zápis čísla
$7^{2007}$. [43]

Určte všetky prirodzené čísla~$k$, pre ktoré dekadický zápis
čísla $7^k+9^k$ končí dvojčíslím~$22$. [$k=20n+8$]

Zistite, pre koľko prirodzených čísel~$n$ menších ako $1\,000$
je súčet $n^{2007}+2\,007^n+1$ deliteľný siedmimi. [95.
Po delení siedmimi sú zvyšky čísel $n^{2\,007}$ rovnaké ako
zvyšky čísel~$n^3$ a~pre jednotlivé~$n$ sa opakujú s~periódou
dĺžky~$7$, pri číslach $2\,007^{n}$ sú zvyšky rovnaké ako pri číslach~$5^{n}$
a~opakujú sa s~periódou dĺžky~$6$. Vyhovujúce~$n$ majú
dvojaké vyjadrenie $n=7k=6l+3$, $n=7k+1=6l+1$, $n=7k+2=6l+1$ alebo
$n=7k+4=6l+1$.]
\endnávod
}

{%%%%%   B-I-2
\epsplace b57.1 \hfil\Obr\par
\epsplace b57.2 \hfil\Obr

V~kosoštvorci (resp. vo štvorci, v~nasledujúcich úvahách to budeme často vynechávať) sú
vzdialenosti oboch dvojíc protiľahlých strán rovnaké. Našou úlohou je
teda viesť bodmi $M$ a~$N$ rovnobežky, ktorých vzdialenosť sa rovná vzdialenosti~$d$
rovnobežiek $p$ a~$q$. Päta~$P$ kolmice z~bodu~$M$ na stranu hľadaného
kosoštvorca prechádzajúcu bodom~$N$ leží na Tálesovej kružnici nad
priemerom~$MN$ a~má od bodu~$M$ vzdialenosť~$d$ (\obr).
\inspicture{}
Odtiaľ vyplýva {\it konštrukcia}:

Zostrojíme Tálesovu kružnicu~$k$ nad priemerom~$MN$
a~kružnicu~$l$ so stredom~$M$, ktorej polomer sa rovná vzdialenosti~$d$
priamok $p$ a~$q$. Označíme $P$ priesečník kružníc $k$ a~$l$. Na
priamke~$PN$ leží jedna zo strán hľadaného kosoštvorca. Protiľahlá
strana prechádza bodom~$M$ a~je s~priamkou~$PN$ rovnobežná.

Vzniknutý rovnobežník je skutočne kosoštvorec alebo štvorec, lebo zo
zhodnosti výšok vyplýva zhodnosť strán.
\inspicture{}

\diskusia
Existencia riešenia je podmienená existenciou bodu~$P$. Zrejme potom nemôže
byť $NP\parallel q$, pretože by to znamenalo, že je $|MP|<d$. Takže
rovnobežky prechádzajúce bodmi $M$, $N$ vždy vytnú požadovaný rovnobežník.
Ak $|MN|>d$, majú kružnice $k$ a~$l$ dva rôzne priesečníky
$P_1\ne P_2$ (\obr),
takže úloha má dve riešenia. Ak $|MN|=d$, potom
$P=N$; stranu kosoštvorca prechádzajúcu bodom~$N$ zostrojíme ako
kolmicu na $MN$ a~úloha má len jedno riešenie. V~prípade $|MN|<d$
nemá úloha riešenie.


\návody
V~rovine sú dané body $A$, $B$, priamka $p$ a~úsečka dĺžky~$v$.
Zostrojte trojuholník $ABC$, ktorého vrchol~$C$ leží na priamke~$p$
a~ktorého výška na stranu~$BC$ má dĺžku~$v$.

\D%
V~rovine je daná priamka~$p$ a~body $M$, $N$, $S$. Zostrojte
pravouholník $ABCD$ tak, aby vrchol~$A$ ležal na priamke~$p$, bod~$M$
na priamke~$AB$, bod~$N$ na priamke~$BC$ a~aby $S$ bol priesečník
jeho uhlopriečok. [Uvažujeme obrazy priamky~$p$ a~bodu~$M$
v~otočení o~$90\st$ so stredom~$S$, ktoré zobrazí stranu~$AB$
na stranu~$BC$.]

V~rovine sú dané tri rovnobežné priamky $a$, $b$, $c$ a~priamka~$d$
s~nimi rôznobežná. Zostrojte štvorec $ABCD$ tak, aby $A\in a$,
$B\in b$, $C\in c$, $D\in d$.
[Zostrojíme najskôr ľubovoľný štvorec $ABCD$, ktorý spĺňa prvé
tri podmienky: Zvoľme bod $B\in b$, vrchol~$A$ takého štvorca
potom leží na priamke~$a$ a~zároveň na priamke~$c$ otočenej okolo bodu~$B$
o~$90\st$. Hľadaný štvorec dostaneme posunutím v~smere rovnobežiek
$a$, $b$, $c$, v~ktorom sa priamka $d'\parallel d$ obsahujúca vrchol~$D$
zobrazí na priamku~$d$.]
\endnávod
}

{%%%%%   B-I-3
Tvrdenie dokážeme sporom.
Pripusťme, že platí $x+y>2$. Potom $y>2-x$, takže
$y^3>(2-x)^3$, lebo funkcia $s=t^3$ je v~premennej~$t$ rastúca
na celom obore reálnych čísel. Preto platí
$$
x^3+y^3>x^3+(2-x)^3=8-12x+6x^2=6(x-1)^2+2\ge2.
$$
To je v~spore s~predpokladom. Tým je tvrdenie dokázané.

\ineriesenie
Dvojčlen $x^3+y^3$ rozložíme na súčin
$(x+y)(x^2-xy+y^2)$. Keby platilo $x+y>2$, pre druhý činiteľ $x^2-xy+y^2$
by sme mali odhad
$$
x^2-xy+y^2=\frac14(x+y)^2+\frac34(x-y)^2>1.
$$
Pre výraz $x^3+y^3$ by potom platilo
$$
x^3+y^3=(x+y)(x^2-xy+y^2)>2\cdot1=2.
$$
To je opäť v~spore s~predpokladom. Tým je tvrdenie dokázané.

\návody
Ak $x$, $y$ sú reálne čísla, pre ktoré platí $x+y>2$, tak
$x^2+y^2>2$; dokážte.
[Tvrdenie vyplýva z~nerovností $(x+y)^2\le2(x^2+y^2)$.]

\D%
Nech $a$, $b$, $c$ sú reálne čísla, ktorých súčet je väčší ako~$1$.
Dokážte, že súčet ich druhých mocnín je väčší ako~$\frac13$.
[Tvrdenie vyplýva z~tzv\. {\it Cauchyho nerovnosti\/}
$(x+y+z)^2\le3(x^2+y^2+z^2)$, ktorú možno overiť úpravou na
tvar $(x-y)^2+(x-z)^2+(y-z)^2\ge0$.]

Nech $x$, $y$ sú nezáporné čísla, pre ktoré platí $x^2+y^2>2$.
Dokážte, že potom $x^3+y^3>2$.
[Potrebnú nerovnosť $(x^2+y^2)^3\le2(x^3+y^3)^2$ dokážeme tak,
že rozdiel medzi pravou a~ľavou stranou upravíme na tvar
$(x-y)^2(x^4+2xy^3+2xy^3+y^4)$.]
\endnávod
}

{%%%%%   B-I-4
\epsplace b57.3 \hfil\Obr

a) Nech $a$ aj $b$ sú odvesny (\obr).
Potom podľa Pytagorovej vety
$$
t_a=\sqrt{b^2+\left(\frac a2\right)^2},\quad
t_b=\sqrt{a^2+\Bigl(\frac b2\Bigr)^2},
$$
takže podmienka $a+t_a=b+t_b$ má tvar
$$
a+\sqrt{b^2+\left(\frac a2\right)^2}=
b+\sqrt{a^2+\Bigl(\frac b2\Bigr)^2}.
$$
\inspicture{}

Keďže z~nerovnosti $a>b$ vyplýva\niedorocenky{ (viď návodná úloha~1)}
$t_b>t_a$, sú nasledujúce úpravy ekvivalentné:
$$
% \gather
\align
2a-2b&=\sqrt{4a^2+b^2}-\sqrt{4b^2+a^2},\\
4a^2-8ab+4b^2&=5a^2+5b^2-2\sqrt{(4a^2+b^2)(4b^2+a^2)},\\
2\sqrt{4a^4+17a^2b^2+4b^4}&=a^2+8ab+b^2,\\
16a^4+68a^2b^2+16b^4&=a^4+16a^3b+66a^2b^2+16ab^3+b^4,\\
15a^4-16a^3b+2a^2b^2-16ab^3+15b^4&=0.
% \endgather
\endalign
$$

Mnohočlen na ľavej strane poslednej rovnice je zrejme deliteľný
dvojčlenom $a-b$ (pre $a=b$ je totiž rovný nule). Delením
zistíme, že výsledný mnohočlen tretieho stupňa má opäť rovnakú
vlastnosť, takže po opakovanom delení prevedieme skúmanú rovnicu
na súčinový tvar
$$
(a-b)^2(15a^2+14ab+15b^2)=0.
$$
Ostatná rovnosť platí práve vtedy, keď $a=b$, pretože
$15a^2+14ab+15b^2>0$ pre každú dvojicu reálnych čísel $a$, $b$.

V~prípade~a) môžeme postupovať aj nasledovne:
Odčítaním rovností
$$
t_a^2=b^2+\left(\frac a2\right)^2,\quad
t_b^2=a^2+\Bigl(\frac b2\Bigr)^2
$$
dostaneme
$$
t_a^2-t_b^2=\tfrac34(b^2-a^2).
$$
Na oboch stranách ostatnej rovnice sú rozdiely druhých mocnín.
Prevedieme ich na súčiny a~potom využijeme danú rovnosť
$a+t_a=b+t_b$ upravenú na tvar $t_a-t_b=b-a$:
$$
\align
(t_a-t_b)(t_a+t_b)&=\tfrac34(b-a)(b+a),\\
(b-a)(t_a+t_b)&=\tfrac34(b-a)(a+b).
\endalign
$$

Keby bolo $a\ne b$, vyjde $t_a+t_b=\frac34(a+b)$;
to spolu s~rovnosťou $t_a-t_b=b-a$ dáva $t_a=\frac78b-\frac18a$,
teda $t_a<b$, čo odporuje tomu, že $t_a$ je prepona a~$b$
odvesna toho istého pravouhlého trojuholníka (\obrr1). Preto musí platiť
rovnosť $a=b$.

\smallskip
b) Nech napr\. $a$ je prepona (ak je preponou~$b$, stačí
strany $a$, $b$ v~nasledujúcom texte navzájom vymeniť).
Potom z~Tálesovej a~Pytagorovej vety vyplýva
$$
t_a=\frac a2,\quad
t_b=\sqrt{c^2+\Bigl(\frac b2\Bigr)^2}
   =\sqrt{a^2-b^2+\Bigl(\frac{b}{2}\Bigr)^2},
$$
čiže rovnosť zo zadania má tvar
$$
\frac{3a}2=b+\sqrt{a^2-b^2+\Bigl(\frac b2\Bigr)^2}.
$$
Keďže prepona~$a$ je dlhšia ako odvesna~$b$, \tj. $a>b$, sú
nasledujúce úpravy ekvivalentné:
$$
% \catcode`\&=10
% \gather
\align
3a-2b&=\sqrt{4a^2-3b^2},\\
9a^2-12ab+4b^2&=4a^2-3b^2,\\
5a^2-12ab+7b^2&=0,\\
(a-b)(5a-7b)&=0,\\
5a-7b&=0.
\endalign
% \endgather
$$

\zaver
Rovnosť $a+t_a=b+t_b$ platí pre pravouhlé
rovnoramenné trojuholníky s~odvesnami $a=b$ a~pre pravouhlé
trojuholníky, ktoré majú strany v~pomere $5:\sqrt{24}:7$,
a~pritom najkratšia z~nich je (tretia) strana~$c$.

\návody
Dokážte, že ťažnice všeobecného trojuholníka
(či už je pravouhlý alebo nie) majú rovnakú vlastnosť ako jeho
výšky: ku kratšej strane smeruje dlhšia ťažnica. Odvoďte odtiaľ, že
rovnosť $a+t_b=b+t_a$ platí práve vtedy, keď $a=b$.
[Nerovnosti medzi
stranami $a$, $b$ a~medzi časťami ťažníc $\frac23t_a$, $\frac23t_b$
porovnajte na
základe toho, že vrchol~$C$ aj ťažisko~$T$ trojuholníka $ABC$ ležia
v~jednej polrovine určenej osou strany~$AB$.]

Vypočítajte dĺžku ťažnice~$t_b$ trojuholníka $ABC$, ak
$a=96$, $b=144$, $t_a=107$.
[Dokážte, že v~každom trojuholníku platí
$t_a^2-t_b^2=\frac34(b^2-a^2)$.]

\D%
V~ľubovoľnom trojuholníku $ABC$ označme $T$ ťažisko, $D$
stred strany~$AC$ a~$E$ stred strany~$BC$. Nájdite všetky
pravouhlé trojuholníky $ABC$ s~preponou~$AB$, pre ktoré je
štvoruholník $CDTE$ dotyčnicový. [56--B--I--4]
\endnávod
}

{%%%%%   B-I-5
Zo zadania vyplýva, že $a\ne0$, $b\ne0$ (inak by rovnice
neboli kvadratické) a~$a\ne b$ (inak by rovnice boli totožné,
a~ak by mali dva reálne korene, boli by oba spoločné).

Označme $x_0$ spoločný koreň oboch rovníc, takže
$$
ax_0^2+2bx_0+1=0,\quad bx_0^2+2ax_0+1=0.
$$
Odčítaním oboch rovníc dostaneme
$(a-b)(x_0^2-2x_0)=x_0(a-b)(x_0-2)=0$.
Keďže $a\ne b$ a~$0$ zrejme koreňom
daných rovníc nie je, musí byť spoločným koreňom číslo $x_0=2$.
Dosadením do daných rovníc tak dostaneme jedinú podmienku
$4a+4b+1=0$, čiže
$$
b=\m a-\tfrac14.
$$

Diskriminant druhej z~daných rovníc je potom
$4a^2-4b=4a^2+4a+1=(2a+1)^2$, takže rovnica má dva rôzne
reálne korene pre ľubovoľné $a\ne\m\frac12$.
Podobne
diskriminant prvej z~daných rovníc je
$4b^2-4a=4b^2+4b+1=(2b+1)^2$. Rovnica má teda dva rôzne
reálne korene pre ľubovoľné $b\ne\m\frac12$, čiže $a\ne\frac14$.

Z~uvedených predpokladov však zároveň vyplýva $a\ne-\frac14$ ($b\ne0$) a~$a\ne-\frac18$ ($a\ne b$).

\zaver
Vyhovujú všetky dvojice $(a, \m a-\frac14)$, kde
$a\in\Bbb R\setminus\{\m\frac12, \m\frac14, \m\frac18, 0, \frac14\}$.

\návody
Nájdite spoločné korene rovníc $2x^3+3x^2-6x+2=0$
a~$2x^3+7x^2+2x-6=0$. [$\m1\pm\sqrt3$, spoločný koreň je koreňom
kvadratickej rovnice, ktorú dostaneme odčítaním kubických rovníc.]

Zistite, pre ktoré hodnoty parametra~$a$ majú rovnice
$x^2+ax-3=0$ a~$x^2+3x-a=0$ aspoň jeden spoločný koreň.
[$a=3$, $a=\m2$]

Nájdite všetky dvojice $(a,b)$ reálnych čísel, pre ktoré má
každá z~rovníc $x^2+(a-2)x+b-3=0$, $x^2+(a+2)x+3b-5=0$
dvojnásobný koreň. [$(6,7)$, $(2,3)$]
\endnávod
}

{%%%%%   B-I-6
\epsplace b57.4 \hfil\Obr\par
\ifrocenka\else\epsplace b57.5 \hfil\Obr\par\fi
Budeme hľadať obdĺžnik s~čo najmenším obsahom,
v~ktorom musí byť obsiahnutý pravouholník majúci všetky rohové
políčka rovnakej farby. Šírka~$2$ nestačí (pri ľubovoľnej dĺžke by
napríklad mohol byť jeden celý riadok čierny a~druhý biely). Uvažujme
teda obdĺžnik šírky~$3$. Jeho stĺpce môžu byť ofarbené ôsmimi
spôsobmi ako na \obr.

\inspicture{}

Ak je obdĺžnik zložený iba zo šiestich stĺpcov 2 až 7, nemá žiadny pravouholník
s~rozmermi väčšími ako $1$ v~ňom obsiahnutý všetky rohové políčka
jednej farby. Uvedených šesť stĺpcov totiž predstavuje všetky
možnosti, ako ofarbiť stĺpec zložený z~troch políčok dvoma
farbami tak, aby nebol jednofarebný (jednofarebné sú zvyšné
dva stĺpce 1 a~8). Keby v~takom obdĺžniku existoval pravouholník
s~rohovými políčkami jednej farby, boli by príslušné
stĺpce rovnaké.

Avšak ak má obdĺžnik šírky~$3$ dĺžku aspoň~$7$, sú v~ňom buď dva
rovnaké stĺpce, alebo v~ňom je niektorý z~jednofarebných stĺpcov (1
a~8). V~prípade dvoch rovnakých stĺpcov je existencia pravouholníka
s~rohovými políčkami jednej farby zrejmá. Ak nie sú žiadne dva
stĺpce rovnaké, ale je tam jednofarebný stĺpec farby~A, musí
v~obdĺžniku byť aj stĺpec, ktorého dve políčka majú farbu~A. Tento
stĺpec a~jednofarebný stĺpec farby~A~určujú pravouholník,
ktorého všetky rohové políčka majú farbu~A.

Daný obdĺžnik $2\,005\times2\,007$ teraz rozdelíme na dve časti
$2\,002\times2\,007$ a~$3\times2\,007$.
Keďže $2\,002=7\cdot286$, $2\,007=3\cdot669$, skladá sa prvá časť
z~$286\cdot669$ neprekrývajúcich sa obdĺžnikov
$7\times3$. V~druhej časti je ešte ďalších
286 obdĺžnikov $7\times3$. Obdĺžnikov $7\times3$ je teda celkom
$286\cdot669+286=286\cdot670=191\,620$. V~každom z~nich je obsiahnutý
aspoň jeden pravouholník, ktorý má všetky rohové políčka jednej
farby. Pre najmenej polovicu takto nájdených obdĺžnikov, teda
pre aspoň 95\,810, je potom farba rohových políčok rovnaká.

\návody
Obdĺžnik $6\times4$ je rozdelený na 24 jednotkových štvorčekov.
Každý z~nich ofarbite čiernou alebo bielou farbou tak, aby žiadne
štyri rovnako ofarbené štvorčeky neboli rohovými štvorčekmi jedného
pravouholníka (nájdite aspoň jedno riešenie).
[Viď \obr: vzhľadom na jeho "farebné súmernosti" stačí
overiť neexistenciu pravouholníka len pre jednu farbu rohových
políčok.]
\inspicture{}

Obdĺžnik $3\times7$ je rozdelený na 21 jednotkových štvorčekov,
z~ktorých každý je ofarbený bielou alebo čiernou farbou. Dokážte, že
niektoré štyri rovnako ofarbené štvorce sú rohovými štvorcami jedného
pravouholníka. [Viď riešenie súťažnej úlohy.]

Obdĺžnik $6\times4$ je rozdelený na 24 jednotkových štvorčekov.
Trinásť z~nich je ofarbených čiernou farbou. Dokážte, že niektoré
štyri čierne štvorce sú rohovými štvorcami jedného pravouholníka.
[Aspoň jeden zo šiestich stĺpcov musí obsahovať aspoň tri čierne
štvorce. Ak sú dokonca štyri, máme pre umiestnenie zvyšných
deviatich štvorcov päť stĺpcov, takže v~jednom z~nich musia byť
aspoň dva. Ak sú práve tri, buď existuje aspoň jeden ďalší
stĺpec s~tromi čiernymi štvorcami (v~týchto dvoch stĺpcoch už
požadovaný pravouholník nájdeme), alebo v~každom zo zvyšných
piatich stĺpcov sú práve dva čierne štvorce; je iba $\binom42=6$
možností, ako štyri štvorce v~stĺpci takto ofarbiť, pričom tri
z~nich už dávajú požadovaný pravouholník s~pôvodne nájdenými tromi
čiernymi štvorcami v~stĺpci.]
\endnávod
}

{%%%%%   C-I-1
Vysvetlíme, prečo prvočíselný rozklad hľadaného čísla musí obsahovať
len vhodné mocniny prvočísel 2, 3 a~5. Každé prípadné ďalšie prvočíslo by sa v~rozklade
čísla~$n$ muselo vyskytovať v~mocnine, ktorej exponent je deliteľný dvoma, tromi
aj piatimi zároveň\niedorocenky{ (viď návodná úloha~1)}. Po vyškrtnutí takého prvočísla by sa číslo~$n$ zmenšilo
a~skúmané odmocniny by pritom ostali celočíselné.

Položme preto $n=2^a{\cdot}3^b{\cdot}5^c$, pričom $a$, $b$, $c$
sú prirodzené čísla. Čísla $\root3\of{3n}$ a~$\root5\of{5n}$ sú celé,
preto je exponent~$a$ násobkom troch a~piatich. Aj $\sqrt{2n}$ je celé číslo,
preto musí byť číslo~$a$ nepárne. Je teda nepárnym násobkom pätnástich: $a\in\{15,45,75,\dots\}$.
Analogicky je exponent~$b$ taký násobok desiatich, ktorý po delení tromi dáva
zvyšok~2: $b\in\{20,50,80,\dots\}$. Napokon $c$ je násobok šiestich,
ktorý po delení piatimi dáva zvyšok~4: $c\in\{24,54,84,\dots\}$.
Z~podmienky, že $n$ je najmenšie, dostávame $n=2^{15}{\cdot}3^{20}{\cdot}5^{24}$.

Presvedčíme sa ešte, že dané odmocniny sú prirodzené čísla:
$$
\sqrt{2n}=2^{8}{\cdot}3^{10}{\cdot}5^{12},\quad
\root3\of{3n}=2^{5}{\cdot}3^{7}{\cdot}5^{8},\quad
\root5\of{5n}=2^{3}{\cdot}3^{2}{\cdot}5^{5}.
$$

\zaver
Hľadaným číslom je $n=2^{15}{\cdot}3^{20}{\cdot}5^{24}$.

\návody
Ak $m$, $k$ a~$\root k\of m$ sú celé čísla väčšie ako~1, tak v~rozklade
čísla~$m$ na súčin prvočísel sa každé prvočíslo vyskytuje v~mocnine, ktorej
exponent je násobkom čísla~$k$. Dokážte. [Rozklad čísla~$m$ dostaneme, keď
rozklad čísla $\root k\of m$ umocníme na $k$-tu.]

\D%
Nájdite všetky trojice prirodzených čísel $a$, $b$, $c$, pre ktoré súčasne
platí
$$
n(ab,c)=2^8,\quad
n(bc,a)=2^9,\quad
n(ca,b)=2^{11},
$$
pričom $n(x,y)$ označuje najmenší spoločný násobok prirodzených čísel $x$, $y$. [50--C--S--1]

Pre koľko usporiadaných trojíc prirodzených čísel $x$, $y$, $z$ platí  $xyz = 1\,000\,000$?
      [Návod: $1\,000\,000=2^6{\cdot}5^6$. Položme $x=2^a{\cdot}5^p$, $y=2^b{\cdot}5^q$, $z=2^c{\cdot}5^r$
      a~preskúmajme všetky možnosti pre $a+b+c=6$  a~pre $p+q+r=6$.
      Nakoniec zistíme hľadaný počet: $28\cdot28=784$.]
\endnávod
}

{%%%%%   C-I-2
\fontplace
\rpoint A; \tpoint B; \lpoint C; \lBpoint D;
\rtpoint K; \ltpoint L; \lpoint M; \bpoint N;
\lbpoint\xy1.3,-.5 S;
\cpoint\a; \cpoint\a;
\cpoint\b; \cpoint\b;
\cpoint\g; \cpoint\g;
\cpoint\d; \cpoint\d;
[3] \hfil\Obr

Päty kolmíc spustených zo stredu~$S$ vpísanej kružnice na strany $AB$, $BC$, $CD$ a~$DA$
označme postupne $K$, $L$, $M$ a~$N$ (\obr). Pravouhlé trojuholníky
$ASK$ a~$ASN$ sú zhodné podľa vety $Ssu$. Majú totiž spoločnú preponu~$AS$
a~zhodné odvesny $SK$ a~$SL$, ktorých dĺžka je rovná polomeru vpísanej kružnice.
Zo zhodnosti týchto trojuholníkov vyplýva jednak známe tvrdenie o~dĺžkach dotyčníc
($|AK|=|AN|$), jednak zhodnosť uhlov $ASK$ a~$ASN$, ktorých spoločnú
veľkosť označíme~$\alpha$:
$$
|\uhol ASK|=|\uhol ASN|=\a.
$$
\inspicture{}

Analogicky zistíme zhodnosť trojuholníkov $SBK$ a~$SBL$, ďalej $SCL$
a~$SCM$, a~nakoniec $SDM$ a~$SDN$. Na základe uvedených zhodností môžeme položiť
$$
|\uhol BSK|=|\uhol BSL|=\b,\quad
|\uhol CSL|=|\uhol CSM|=\g,\quad
|\uhol DSM|=|\uhol DSN|=\d.
$$
Odtiaľ a~z~\obrr1{} potom dostávame
$$
\align
|\uhol ASD|-|\uhol CSD|=&(\a+\d)-(\g+\d)=\a-\g=\\
                   =&(\a+\b)-(\g+\b)=|\uhol ASB|-|\uhol BSC|=40\st.
\endalign
$$

\zaver
$|\uhol ASD|-|\uhol CSD|=40\st$.


\návody
Dotyčnice vedené ku kružnici $k(O, r)$ z~bodu~$A$ sa dotýkajú kružnice~$k$
v~bodoch $T$ a~$U$. Dokážte, že
        a) $|AT|=|AU|$,
        b) $|\uhol AOT|=|\uhol AOU|$.

Lichobežníku $ABCD$ ($AB\parallel CD$) je vpísaná kružnica so stredom~$O$.
Dokážte, že
        a)~$|\uhol AOD|=90\st$,
        b)~$|\uhol DOC|=|\uhol DAO|+|\uhol ABO|$.
%         c)~$|\uhol BCD|=|\uhol AOB|$.

Dotyčnice vedené ku kružnici $k(O, r)$ z~bodu~$A$ sa dotýkajú kružnice~$k$ v~bodoch
$T$ a~$U$. Tretia dotyčnica pretína úsečky $AT$ a~$AU$ postupne v~bodoch $B$ a~$C$.
Určte obvod trojuholníka $ABC$, ak $|AT|=12\cm$. [$24\cm$; pre bod~$V$ dotyku
dotyčnice~$BC$ platí $|CV|=|CT|$ a~$|BV|=|BU|$, takže $|BC|=|CT|+|BU|$.]
\endnávod
}

{%%%%%   C-I-3
Keď označíme $x$ počet krabičiek a~$y$ počet guľôčok, dostaneme zo zadania sústavu
rovníc
$$
x+n=y \qquad\hbox{a}\qquad (x-n)\cdot n=y  \tag1
$$
s~neznámymi $x$, $y$ a~$n$ z~oboru prirodzených čísel. Vylúčením neznámej~$y$
dostaneme rovnicu $x+n=(x-n)\cdot n$, ktorá pre $n=1$ nemá riešenie.
Pre $n\ge2$ dostaneme
$$
x={n^2+n\over n-1}=n+2+{2\over n-1}, \tag2
$$
odkiaľ vidíme, že (prirodzené) číslo $n-1$ musí byť deliteľom čísla~$2$. Teda $n\in\{2,3\}$.
Prípustné hodnoty~$n$ dosadíme do \thetag1 a~sústavu vyriešime (možno tiež využiť vzťah \thetag2).
Pre $n=2$ dostaneme $x=6$, $y=8$ a~pre $n=3$ určíme $x=6$ a~$y=9$.

\skuska
Majme šesť krabičiek a~osem guľôčok. Keď do každej krabičky
dáme práve jednu guľôčku, ostane $n=2$ guľôčok. Keď však odoberieme dve
krabičky, môžeme do zostávajúcich štyroch rozdeliť guľôčky práve po dvoch.
Podmienky úlohy sú teda splnené.
Pre šesť krabičiek a~deväť guľôčok urobíme skúšku rovnako ľahko.

\zaver
Buď máme šesť krabičiek a~osem guľôčok, alebo šesť krabičiek
a~deväť guľôčok.

\návody
     Určte všetky celé čísla~$n$, pre ktoré nadobúda zlomok $\frc{(4n+27)}{(n+3)}$
     celočíselné hodnoty. [$n\in\{-18,-8,-6,-4,-2,0,2,12\}$, číslo~$n+3$ je
     deliteľom čísla~15.]

   Nováková, Vašková a~Sudková vyhrali štafetu a~okrem diplomov dostali
   aj bonboniéru, ktorú hneď po pretekoch zjedli. Keby zjedla Petra
   o~3~bonbóny viac, zjedla by ich práve toľko, koľko Miška s~Janou dokopy.
   A~keby si Jana pochutnala ešte na siedmich bonbónoch, tiež by ich mala
   toľko, ako druhé dve spolu. Ešte vieme, že počet bonbónov, ktoré zjedla
   Vašková, je deliteľný tromi a~že Sudková si pochutila na siedmich bonbónoch.
   Ako sa volali dievčatá? Koľko bonbónov zjedla každá z~nich?
   [56--Z9--II--3]
\endnávod
}

{%%%%%   C-I-4
\fontplace
\tpoint4; \lpoint2;
[2] \hfil\Obr

a) Daný obdĺžnik sa zložiť dá (\obr).
\inspicture{}

b) Celková dĺžka "iracionálnych" strán všetkých dielov tangramu je $10\sqrt2\cm$.
Je teda rovná obvodu obdĺžnika, ktorý máme zložiť. Odtiaľ\niedorocenky{ a~z~textu
návodnej úlohy~1} vyplýva, že všetky "iracionálne" strany dielov tangramu musia byť
umiestnené na hranici skladaného obdĺžnika.\dorocenky{\footnote{Používame známe tvrdenie, že pre
celé čísla $a$, $b$, $c$, $d$ platí $a+b\sqrt2=c+d\sqrt2$ práve vtedy, keď $a=c$ a~$b=d$.}}
To však nie je možné, lebo protiľahlé
"iracionálne" strany kosodĺžnikového dielu majú vzdialenosť menšiu ako $1\cm$,
ale najmenšia vzdialenosť protiľahlých strán obdĺžnika je $\sqrt2\cm$.

\zaver
Obdĺžnik $2\cm\times4\cm$ sa z~tangramu zložiť dá, obdĺžnik $\sqrt2\cm\times4\sqrt2\cm$ sa zložiť nedá.


\návody
Dokážte, že pre celé nezáporné čísla $a$, $b$, $c$, $d$ platí:
Dĺžku úsečky možno vyjadriť v~tvare $a+b\sqrt2$ a~súčasne v~tvare $c+d\sqrt2$
práve vtedy, keď $a=c$ a~$b=d$.
[Rovnosť $a+b\sqrt2=c+d\sqrt2$ je ekvivalentná so vzťahom $a-c=(d-b)\sqrt2$,
ktorého ľavá strana je celé číslo, ale pravá strana je pre $b\ne d$ iracionálna.
Rovnosť nastáva, len keď $b=d$ a~$a=c$.]

\D%
Dokážte, že z~tangramu nemožno zložiť kosodĺžnik so základňou dĺžky $2\cm$
a~výškou $4\cm$.
[Z~dielov tangramu sa dajú zostaviť iba tie uhly, ktorých veľkosť je násobkom
$45\st$. Preto musí mať skladaný kosodĺžnik veľkosti vnútorných uhlov $45\st$ a~$135\st$.
Keďže má výšku $4\cm$, má jeho dlhšia strana dĺžku $4\sqrt2\cm$.
Tangram má sedem dielov, z~ktorých jedine štvorec má všetky strany celočíselnej dĺžky.
Pozdĺž oboch dlhších strán kosodĺžnika je preto nutné umiestniť
po jednej "iracionálnej" strane každého zo šiestich zvyšných "iracionálnych" dielov.
Ostanú tak práve dve strany dĺžky $\sqrt2\cm$, ktoré musia byť vnútri
skladaného kosodĺžnika. Jedna z~nich
zrejme patrí dielu tvaru kosodĺžnika (lebo ten nemôže mať kvôli svojej malej výške
obe protiľahlé "iracionálne" strany na hranici skladaného obrazca), druhá
dielu tvaru trojuholníka s~"iracionálnymi" odvesnami. V~dôsledku vety z~predošlej
úlohy musia byť tieto strany umiestnené pozdĺž jednej priamky. To však nie je možné,
pretože môžu byť umiestnené jedine v~smeroch navzájom kolmých.]

Určte všetky dvojice $(a, b)$ prirodzených čísel, pre ktoré platí
$a+b\sqrt5=b+a\sqrt5$.
[56--C--I--1]

Určte všetky dvojice $(a, b)$ prirodzených čísel, ktorých rozdiel $a - b$
je piatou mocninou niektorého prvočísla a~pre ktoré platí $a-4\sqrt b=b+4\sqrt a$.
[56--C--S--3]

Nájdite všetky dvojice $(a, b)$ nezáporných reálnych čísel, pre ktoré platí
$$
\sqrt{a^2+b}+\sqrt{b^2+a}=
\sqrt{a^2+b^2}+\sqrt{a+b}.
$$
[48--C--S--1]
\endnávod
}

{%%%%%   C-I-5
\fontplace
\tpoint A; \tpoint B; \lpoint C; \bpoint D;
\bpoint E; \rbpoint F;
[4] \hfil\Obr

a) Označme $A$, $B$ dve osoby, ktoré sa nepoznajú, a~pridajme k~nim ľubovoľné
ďalšie dve osoby $X$ a~$Y$. Keby ani osoba~$X$, ani osoba~$Y$ nebola
spoločným známym osôb $A$ a~$B$, mali by sme zo všetkých šiestich dvojíc vo štvorici
$ABXY$ aspoň tri dvojice neznámych: dvojicu~$AB$, dvojicu $AX$ alebo $BX$
a~dvojicu $AY$ alebo $BY$. Dvojice známych vo štvorici $ABXY$ by tak boli
najviac tri, čo odporuje predpokladu zo zadania časti~a). Tým je časť~a) dokázaná.

\smallskip
b) Skupina požadovaných vlastností existuje pre všetky $n\ge4$. Ako príklad
stačí zvoliť skupinu, v~ktorej sa osoba~$A$ nepozná s~nikým a~ostatní sa poznajú
navzájom. Potom existuje dokonca $n-1$ dvojíc osôb, ktoré sa ani nepoznajú, ani nemajú
spoločného známeho, a~medzi každými štyrmi osobami sú aspoň tri dvojice známych.

\smallskip
c) Budeme predpokladať, že šestica osôb s~opísanou vlastnosťou existuje.
Využijeme grafické znázornenie, v~ktorom osoby zakreslíme ako body. Plnou
(resp\. prerušovanou) úsečkou, ktorou niektoré dva z~týchto bodov spojíme,
vyznačíme dvojicu známych (resp\. dvojicu neznámych).

Z~každého bodu grafického znázornenia skupiny šiestich osôb vychádza práve
päť úsečiek. Podľa Dirichletovho
princípu preto aspoň tri úsečky, ktoré vychádzajú z~jedného bodu, majú rovnaký typ
(sú buď prerušované, alebo plné). Označme body $A$, $B$, $C$, $D$, $E$ a~$F$
tak, aby mali rovnaký typ úsečky $AB$, $AC$ a~$AD$, a~predpokladajme
najskôr, že označujú dvojice známych. Vo štvorici $ABCD$
sú však podľa predpokladu práve tri dvojice neznámych, a~preto je trojuholník
$BCD$ v~grafickom znázornení zakreslený prerušovane. Vo štvorici $BCDE$ potom
úsečky $EB$, $EC$, $ED$ nutne predstavujú dvojice známych (\obr).
Odtiaľ vyplýva, že vo štvorici $ABDE$ sú aspoň štyri dvojice
známych, ktoré na \obrr1{} znázorňujú úsečky $AB$, $AD$, $EB$ a~$ED$, čo je v~rozpore
s~naším predpokladom.
Prípad, keď úsečky $AB$, $AC$ a~$AD$ predstavujú dvojice neznámych, vedie
ku sporu analogicky (v~predchádzajúcich úvahách stačí zameniť
vzťahy {\it poznať sa\/} a~{\it nepoznať sa\/} a~samozrejme aj prerušované
a~plné úsečky).

\inspicture{}

\zaver
Neexistuje skupina šiestich osôb, ktorá má v~každej svojej
štvorici práve tri dvojice známych a~práve tri dvojice neznámych.

\návody
V~skupine piatich osôb sa v~každej štvorici vyskytujú práve tri dvojice známych.
\item{a)} Dokážte, že v~skupine nemôže byť trojica osôb, ktoré sa poznajú
          navzájom (tzv\. trojuholník známych), ani osoba, ktorá má aspoň troch
          známych.
\item{b)} Dokážte, že tu nemôže byť trojuholník neznámych ani osoba,
          ktorá sa nepozná aspoň s~tromi osobami.
\item{c)} %Najděte příklad takové skupiny osob.
          Nakreslite graf známostí v~takej skupine osôb.
\endnávod
}

{%%%%%   C-I-6
Hľadajme pôvodné číslo $x=100a+10b+c$, ktorého cifry sú $a$, $b$, $c$.
Cifru, ktorá sa vyskytuje na prostredných dvoch miestach výsledného súčinu,
označme~$d$. Zo zadania vyplýva
$$
9(100a+10b+c)=1\,000a+100d+10d+(a+b+c),    \tag1
$$
pričom výraz v~poslednej zátvorke predstavuje cifru zhodnú s~poslednou
cifrou súčinu~$9c$. To však znamená, že nemôže byť $c\ge5$: pre také~$c$
totiž končí číslo~$9c$ cifrou neprevyšujúcou~$5$, a~pretože $a\ne0$, platí
naopak $a+b+c>c\ge5$.

Zrejme tiež $c\ne0$ (v~opačnom prípade by platilo $a=b=c=x=0$). Ostatné
možnosti vyšetríme zostavením nasledujúcej tabuľky.
$$
\vbox{\offinterlineskip\everycr{\noalign{\hrule}}
 \catcode`@\active\def@{\phantom0}
 \halign{\strut\vrule#&&\hbox to 4.5em{\hss$#$\hss}\vrule\cr
&c   &9c &a+b+c& a+b\cr \noalign{\vskip3pt\hrule}
&1   &@9 & 9   &8\cr
&2   &18 & 8   &6\cr
&3   &27 & 7   &4\cr
&4   &36 & 6   &2\cr
}}
$$

Rovnosť~\thetag1 možno prepísať na tvar
$$
100(b-a-d)=10d+a+11b-8c.          \tag2
$$
Hodnota pravej strany je aspoň $\m72$ a~menšia ako $200$, lebo každé z~čísel $a$,
$b$, $c$, $d$ je najviac rovné deviatim. Takže buď $b-a-d=0$, alebo $b-a-d=1$.

V~prvom prípade po substitúcii $d=b-a$ upravíme vzťah~\thetag2 na tvar $8c=3(7b-3a)$,
z~ktorého vidíme, že $c$ je násobkom troch. Z~prvej tabuľky potom vyplýva $c=3$, $a=4-b$,
čo po dosadení do rovnice $8c=3(7b-3a)$ vedie k~riešeniu $a=b=2$, $c=3$.
Pôvodné číslo je teda $x=223$ a~jeho deväťnásobok $9x=2\,007$.

V~druhom prípade dosadíme $d=b-a-1$ do \thetag2 a~zistíme, že $8c+110=3(7b-3a)$.
Výraz $8c+110$ je teda deliteľný tromi, preto číslo~$c$ dáva po delení
tromi zvyšok~$2$. Dosadením jediných možných hodnôt $c=2$ a~$b=6-a$ do poslednej
rovnice zistíme, že $a=0$, čo je v~rozpore s~tým, že číslo $x=100a+10b+c$ je
trojciferné.

\zaver
Klárka dostala štvorciferné číslo $2\,007$.

\poznamka
Prvá tabuľka ponúka jednoduchší, ale numericky prácnejší postup priameho dosadzovania
všetkých prípustných hodnôt čísel $a$, $b$, $c$ do rovnice~\thetag1. Počet všetkých
možností možno obmedziť na desať odhadom $b\ge a$, ktorý zistíme pomocou vhodnej
úpravy vzťahu~\thetag1 napríklad na tvar~\thetag2. Riešenie uvádzame v~druhej tabuľke.
$$
\vbox{%\offinterlineskip\everycr{\noalign{\hrule}}
 \catcode`@\active\def@#1{\text{\bf #1}}
 \everycr{\noalign{\vrule}}\dimen1=1.8em
 \centerline{\valign{\hrule#&&\hbox to\dimen1{ \strut\hss$#$ }\hrule\cr
 &a  &b  &c  &9x\cr   \noalign{\global\dimen1=3em \hskip3pt\vrule}
 &1  &7  &1  &1\,539\cr
 &2  &6  &1  &2\,349\cr
 &3  &5  &1  &3\,159\cr
 &4  &4  &1  &3\,969\cr
 &1  &5  &2  &1\,368\cr
 &2  &4  &2  &2\,178\cr
 &3  &3  &2  &2\,988\cr
 &1  &3  &3  &1\,197\cr
&@2 &@2 &@3 &@{2\,007}\cr
 &1  &1  &4  &1\,026\cr}}}
$$


\návody
\D%
K~prirodzenému číslu~$m$ zapísanému rovnakými ciframi sme pripočítali štvorciferné
prirodzené číslo~$n$. Získali sme štvorciferné číslo s~opačným poradím cifier,
ako má číslo~$n$. Určte všetky také dvojice čísel $m$ a~$n$. [52--C--I--5]

Žiaci mali vypočítať príklad $x+y\cdot z$ pre trojciferné číslo~$x$
a~dvojciferné čísla $y$, $z$. Martin vie násobiť a~sčítať čísla zapísané
v~desiatkovej sústave, ale zabudol na pravidlo prednosti násobenia pred sčítaním.
Preto mu vyšlo síce zaujímavé číslo, ktoré sa píše rovnako zľava doprava ako sprava doľava,
správny výsledok bol ale o~$2\,004$ menší. Určte čísla $x$, $y$, $z$. [53--C--II--4]
\endnávod
}

{%%%%%   A-S-1
Sčítaním všetkých troch rovníc po zrušení kvadratických členov
dostaneme
$$
x+y+z=0.               \tag1
$$
Odtiaľ vyjadríme $z=\m x-y$ a~dosadíme do prvej rovnice sústavy.
Obdržíme
$x^2-y=(\m x-y)^2$, čo po úprave dá rovnicu $y(2x+y+1)=0$.
Rozoberieme preto dva prípady podľa toho, ktorý
z~oboch činiteľov na jej ľavej strane je rovný nule.

V~prípade $y=0$ z~rovnice~\thetag1 obdržíme $z={\m x}$ a~po dosadení
$y$, $z$ do pôvodnej sústavy dostaneme pre neznámu~$x$
jedinú podmienku $x(x-1)=0$, ktorú spĺňa iba $x=0$
a~$x=1$. Tomu zodpovedajú riešenia $(x,y,z)$ tvaru $(0,0,0)$
a~$(1,0,\m1)$.

V~prípade, keď $2x+y+1=0$, čiže $y=\m2x-1$, z~\thetag1 máme
$z=\m x-y={x+1}$. Po dosadení $y$, $z$ do pôvodnej sústavy
dostaneme pre neznámu~$x$ jedinú podmienku $x({x+1})=0$,
ktorú spĺňajú iba $x=0$ a~$x=\m1$. Tomu zodpovedajú riešenia
$(x,y,z)$ tvaru $(0,\m1,1)$ a~$(\m1,1,0)$.

\zaver
Daná sústava má práve štyri riešenia $(x,y,z)$: trojice
$(0,0,0)$, $(1,0,\m1)$, $(0,\m1,1)$ a~$(\m1,1,0)$.

\ineriesenie
Sčítaním dvoch prvých rovníc danej sústavy eliminujeme neznámu~$x$
a~dostaneme rovnicu $y^2-z^2=y+z$, ktorú možno zapísať
v~tvare súčinu
$$
(y+z)(y-z-1)=0.            \tag2
$$
Rozoberieme opäť dva prípady podľa toho, ktorý z~dvoch činiteľov v~poslednej rovnici sa
rovná nule.

V~prípade $y+z=0$ z~tretej rovnice danej sústavy vyjde $x=0$
a~z~prvých dvoch rovníc po dosadení $x=0$ a~$z=\m y$ dostaneme pre
neznámu~$y$ jedinú podmienku $y(y+1)=0$, teda $y=0$ alebo
$y=\m1$. Zodpovedajúce riešenia $(x,y,z)$ sú $(0,0,0)$
a~$(0,\m1,1)$.

V~prípade, keď $y-z-1=0$, čiže $z=y-1$, získame
z~tretej rovnice sústavy
$x=z^2-y^2=(y-1)^2-y^2=1-2y$. Dosadením $x$, $z$ dostaneme pre
neznámu~$y$ jedinú podmienku $y(y-1)=0$, teda $y=0$ alebo
$y=1$. Zodpovedajúce riešenia $(x,y,z)$ sú $(1,0,\m1)$
a~$(\m1,1,0)$.


\nobreak\medskip\petit\noindent
Za úplné riešenie dajte 6~bodov. Odvodenie rovnice
\thetag1 alebo aspoň jednej z~troch analogických rovníc
v~tvare súčinu~\thetag2 oceňte 2~bodmi.
\endpetit
\bigbreak}

{%%%%%   A-S-2
Každý $n$-boký hranol má práve $n$~vrcholov v~každej zo svojich
podstáv, takže $v=2n$. Z~každého vrcholu vychádza $n-3$
uhlopriečok ležiacich v~podstave a~dve uhlopriečky ležiace v~bočných
stenách, celkom je to $n-1$ stenových uhlopriečok. Z~$2n$ vrcholov
teda vychádza $2n(n-1)$ stenových uhlopriečok, každá z~nich je však
započítaná dvakrát, preto $s=n({n-1})$. Podobne určíme počet $t$
telesových uhlopriečok: z~každého vrcholu ich vychádza $n-3$ (do
všetkých vrcholov druhej podstavy s~výnimkou tých troch vrcholov,
s~ktorými je daný vrchol spojený hranou
alebo uhlopriečkou v~bočnej stene), preto
$t=2n(n-3)/2=n(n-3)$.

Hľadáme tie celé $n\ge3$, pre ktoré čísla
$$
v=2n,\quad s=n(n-1)\quad\text{a}\quad t=n(n-3)
$$
tvoria vo vhodnom poradí trojicu $x$, $y$, $z$ s~vlastnosťou $y-x=z-y$,
čiže $y=\frac12(x+z)$. Jednoduchým dosadením zistíme, že pre $n=3$
máme nevyhovujúcu trojicu čísel $6$, $6$, $0$, zatiaľ čo pre $n=4$
vychádza vyhovujúca trojica $8$, $12$, $4$ (platí $8=\frac12(4+12)$).
Pre ľubovoľné $n\ge5$ máme $n-1>n-3\ge2$, odkiaľ po násobení
číslom~$n$ dostaneme
$s>t\ge v$, takže požadovaná rovnosť s~aritmetickým priemerom
musí byť tvaru $t=\frac12(v+s)$. Po dosadení dostávame rovnicu
$$
n(n-3)=\frac{2n+n(n-1)}{2}
$$
s~jediným prípustným koreňom $n=7$ (koreň $n=0$ nemá reálny
zmysel).

\zaver
Vyhovujú jedine $n=4$ a~$n=7$.


\nobreak\medskip\petit\noindent
Za úplné riešenie dajte 6~bodov, z~toho 1~bod za
vyjadrenie počtu~$s$ a~2~body za vyjadrenie počtu~$t$ (v~závislosti
od premennej~$n$), ďalšie body podľa úplnosti diskusie, v~akom
poradí môžu čísla $v$, $s$, $t$ tvoriť aritmetickú postupnosť.
Pokiaľ riešiteľ zabudne na riešenie $n=4$ (napr. prehlási za zrejmé
nerovnosti $s>t>v$), dajte najviac 5~bodov.
\endpetit
\bigbreak}

{%%%%%   A-S-3
\fontplace
\rBpoint A; \lBpoint B; \bpoint C;
\lBpoint S; \rBpoint T; \rpoint k;
\lBpoint r; \cpoint\omega; \cpoint\frac\gamma2;
\lpoint k_1; \tpoint X; \tpoint\xy-1,0 Y;
[5] \hfil\Obr

\fontplace
\rBpoint A; \lBpoint B; \bpoint C;
\bpoint\xy.6,.9 S; \rpoint k; \cpoint\omega;
\lpoint k_1; \tpoint X; \tpoint Y;
\bpoint K; \bpoint\ L;
[6] \hfil\Obr

Označme $r$ polomer danej kružnice~$k$
a~$\om$ veľkosť daného (konvexného) uhla $XSY$. V~ľubovoľnom
vyhovujúcom trojuholníku $ABC$ označme zvyčajným spôsobom vnútorné uhly.
V~trojuholníku $ABS$ platí (\obr)
$$
\om=|\uhol ASB|=180^{\circ}-|\uhol SAB|-|\uhol SBA|=
180^{\circ}-\frac{\al+\be}{2}=90^{\circ}+\frac{\ga}{2}.
$$
Odtiaľ vyplýva, že hľadaná množina je prázdna, ak
$\om\le90^{\circ}$ alebo $\om=180^{\circ}$, a~že všetky
vyhovujúce trojuholníky $ABC$ majú vnútorný uhol~$\ga$, pre ktorého veľkosť platí
$$
\ga=2\om-180^{\circ}.
$$
\inspicture{}

Z~pravouhlého trojuholníka $CST$, pričom $T$ je bod dotyku kružnice~$k$
so stranou~$AC$ (\obrr1), vyjadríme dĺžku prepony~$SC$ vzťahom
$$
\postdisplaypenalty10000
|SC|=\frac{|ST|}{\sin\frac12\ga}=\frac{r}{\sin(\om-90^{\circ})}.
$$
Bod~$C$ preto leží na kružnici~$k_1$ so stredom~$S$
a~polomerom $r_1=r/\sin(\om-90\st)$.

Rovnako ako uhol $ASB$ sú aj uhly $ASC$ a~$BSC$
(čiže uhly $XSC$ a~$YSC$) tupé, lebo
$$
|\uhol ASC|=90^{\circ}+\frac{\be}{2}\quad\text{a}\quad
|\uhol BSC|=90^{\circ}+\frac{\al}{2}.
\tag1
$$
Spolu tak dostávame, že bod~$C$ je vnútorným bodom
oblúka~$KL$ kružnice~$k_1$, ktorý leží zvonka daného uhla $XSY$ a~ktorého krajné body $K$,
$L$ sú určené pravými uhlami $XSK$ a~$YSL$ (\obr).
\inspicture{}

Ak naopak vyberieme ľubovoľný vnútorný bod~$C$ oblúka~$KL$,
polpriamky $SX$, $SY$ a~$SC$ rozdelia rovinu na tri tupé
uhly, pričom polpriamka~$CS$ oddelí body $X$ a~$Y$.
Z~rovnosti $|SC|=r_1$ vyplýva, že dotyčnica
z~bodu~$C$ ku kružnici~$k$ zostrojená v~polrovine $CSX$ zviera
s~polpriamkou~$CS$ ostrý uhol~$\omega-90\st$,
takže pretne polpriamku~$SX$ v~bode, ktorý označíme~$A$.
Analogicky dotyčnica
z~bodu~$C$ ku kružnici~$k$ zostrojená v~polrovine $CSY$
pretne polpriamku~$SY$ v~bode, ktorý označíme~$B$.

Zvoľme teraz hodnoty $\a$, $\b$, $\gamma$ tak, aby
$\omega-90\st=\frac12\ga$,
$|\uh CSK|=\frac12\b$, $|\uh CSL|=\frac12\a$, potom z~plného uhla pri vrchole~$S$ vyplýva
$$
{\a+\b\over2}=180\st-\omega=90\st-\frac\gamma2
\quad\text{čiže $\a+\b+\gamma=180\st$}.
$$
Ako ľahko spočítame,
dotyčnica z~nájdeného bodu~$A$ ku kružnici~$k$ súmerne združená s~dotyčnicou~$AC$
podľa priamky~$SX$ pretína polpriamku~$CS$ pod uhlom $\frac12\ga+\a$,
a~podobne vyjde, že analogická dotyčnica z~nájdeného bodu~$B$ pretne tú istú
polpriamku pod uhlom $\frac12\ga+\b$. Súčet oboch uvedených uhlov je však $180\st$,
preto sú obe dotyčnice ku kružnici~$k$ rovnobežné, a~teda totožné (oba príslušné
body dotyku musia totiž ležať vnútri konvexného uhla $XSY$).
Nájdený trojuholník $ABC$ má preto požadované vlastnosti.

%% podle jejichž velikostí určíme kladné úhly
%% $$
%% \al=2|\uhol YSC|-180^{\circ},\
%% \be=2|\uhol XSC|-180^{\circ},\
%% \ga=2\om-180^{\circ}
%% \tag2
%% $$
%% o~součtu $180^{\circ}$. Z~rovnosti $|SC|=r_1$ pak plyne, že tečny
%% ke kružnici $k$ vedené bodem $C$ svírají úhel $\ga$, takže na
%% nich leží strany \tr-u $ABC$, který má vepsanou kružnici
%% $k$ a~úhly $\al$, $\be$ u~vrcholů $A$, resp. $B$. Jak víme,
%% v~takovém \tr-u mají úhly $ASC$, $BSC$ velikosti dané vztahy (1),
%% kam po dosazení hodnot $\al$, $\be$ samozřejmě
%% dostaneme hodnoty $|\uhol XSC|$, resp. $|\uhol YSC|$.
%% To znamená, že vrcholy $A$, $B$ leží na
%% polopřímkách $SX$, resp. $SY$, jak vyžadovalo zadání.

\nobreak\medskip\petit\noindent
Za úplné riešenie dajte 6~bodov. Za určenie uhla~$\ga$
dajte 1~bod, ďalšie 2~body za určenie polomeru $r_1=|SC|$
kružnice $k_1$ a~2~body za vymedzenie jej oblúka~$KL$.
Ak chýba záverečné zdôvodnenie, že každý vnútorný bod~$C$ oblúka~$KL$
je vrcholom vyhovujúceho trojuholníka $ABC$,
môže riešiteľ získať najviac 5~bodov.
\endpetit}

{%%%%%   A-II-1
Ak $p=0$, druhá rovnica má tvar $12=0$, nemá teda žiadne riešenie. Pre všetky hľadané štvorice je teda $p\ne0$ a~obe rovnice sú kvadratické.

{\it Rôzne\/} čísla $p$, $q$ sú koreňmi kvadratickej rovnice $x^2+rx+s-1=0$ práve vtedy, keď spĺňajú Vi\`etove vzťahy
$$
p+q=\m r,\qquad pq=s-1.
\tag1
$$
Podobne sú {\it rôzne\/} čísla $r$, $s$ koreňmi kvadratickej rovnice $px^2+p(q-1)x+12=0$ práve vtedy, keď spĺňajú Vi\`etove vzťahy
$$
r+s=\m\frac{p(q-1)}p,\qquad rs=\frac{12}p.
\tag2
$$
Z~\thetag1 vyjadríme $r=\m p-q$, $s=pq+1$ a~dosadíme do \thetag2. Postupnými úpravami dostaneme rovnosti
$$
\aligned
  \m p-q+pq+1&=\m\frac{p(q-1)}p,\\
  \m p+pq+1-q&=1-q,\\
           pq&=p,
\endaligned
\qquad\qquad
\aligned
  (\m p-q)(pq+1)&=\frac{12}p,\\
  \m p(p+q)(pq+1)&=12.
\endaligned
$$
Keďže $p\ne0$, z~rovnosti naľavo máme $q=1$ a~po dosadení do rovnosti napravo získame rovnicu $\m p(p+1)^2=12$, ktorá po úprave prejde na rovnicu tretieho stupňa
$$
  p^3+2p^2+p+12=0.
\tag3
$$
Tá má koreň $p=\m3$ a~po vyňatí výrazu $(p+3)$ pred zátvorku ju upravíme na súčinový tvar $(p+3)(p^2-p+4)=0$. Keďže kvadratická rovnica $p^2-p+4=0$ nemá žiadne reálne riešenie (jej diskriminant je $\m15$), je $p=\m3$ jediným riešením rovnice~\thetag3. Podľa \thetag1 potom $r=\m(\m3)-1=2$, $s=(\m3)\cdot1+1=\m2$. Skúškou ľahko overíme, že štvorica $(p,q,r,s)=(\m3,1,2,\m2)$ spĺňa Vi\`etove vzťahy \thetag1, \thetag2 a~je teda jedinou štvoricou vyhovujúcou zadaniu.

\nobreak\medskip\petit\noindent
Za úplné riešenie dajte 6~bodov. Z~toho 2~body za zostavenie Vi\`etových vzťahov \thetag1, \thetag2 alebo podobných rovností umožňujúcich priame vyjadrenie dvoch premenných v~závislosti od iných dvoch. Také rovnosti možno získať napr. vhodnou úpravou sústavy
$$
\aligned
  p^2+rp+s-1&=0,\\
  q^2+rq+s-1&=0,\\
  pr^2+p(q-1)r+12&=0,\\
  ps^2+p(q-1)s+12&=0
\endaligned
$$
vyplývajúcej zo zadania (len za zostavenie tejto sústavy body nedajte), prípadne zo známeho vzorca na výpočet koreňov kvadratických rovníc. Ďalší 1~bod dajte za vyjadrenie $q=1$, resp. inú redukciu sústavy na rovnicu s~jedinou neznámou (taká rovnica bude spravidla kubická). Úplné vyriešenie tejto rovnice ohodnoťte ďalšími 2~bodmi (z~týcho 2~bodov nedajte žiadny, ak žiak iba uhádne riešenie uvedenej kubickej rovnice a~nevysvetlí, prečo iné reálne riešenie neexistuje). Posledný 1~bod dajte za dokončenie riešenia.

Posúďte, či v~riešeniach ašpirujúcich na úplnosť nechýba skúška (pokiaľ sú úpravy ekvivalentné, nie je nutná),
jej absenciu penalizujte stratou 1~bodu.

Pokiaľ žiak nezíska za úlohu iné body, za nájdenie (uhádnutie) štvorice $(\m3,1,2,\m2)$ bez zdôvodnenia, prečo žiadne ďalšie riešenie neexistuje, dajte 1~bod.

\endpetit
\bigbreak}

{%%%%%   A-II-2
Po vypísaní uvedenej tabuľky $3\times3$, prípadne $5\times5$, ihneď zistíme, že pre tieto hodnoty $n$ nemožno urobiť žiadny krok, ktorý by čísla zmenil. Totiž v~každej dvojici čísel na susedných políčkach je jedno párne a~jedno nepárne číslo, takže ich súčet je nepárny a~aritmetický priemer nie je celé číslo.

Takáto situácia nastáva pri všetkých nepárnych hodnotách~$n$: Ak ofarbíme tabuľku ako šachovnicu (pričom číslo~$1$ bude na bielom políčku), bude každý riadok začínať opačnou farbou ako má prvé políčko predošlého riadku. To má pri nepárnom~$n$ rovnakú farbu ako posledné políčko v~riadku. Takže čísla $1,2,3,\dots,n^2$ budú v~tomto poradí napísané striedavo na políčkach s~farbou biela, čierna, biela, \dots, biela, teda nepárne čísla budú na bielych políčkach a~párne na čiernych. Keďže v~každej dvojici susedných políčok je jedno biele a~jedno čierne, nie je aritmetický priemer žiadnych dvoch susedných čísel celým číslom.

Fakt, že pre nepárne~$n$ sú na ľubovoľných dvoch susedných políčkach čísla s~opačnou paritou, možno zdôvodniť aj inak: Každé číslo $k$ susedí v~tabuľke s~číslami $k-1$, $k+1$, $k-n$ a~$k+n$ (prípadne len s~niektorými z~nich, ak leží na okraji tabuľky). Všetky tieto čísla majú opačnú paritu ako $k$.

\smallskip
Ukážeme, že ani žiadne párne $n$ nespĺňa podmienky zadania. Súčet~$S$ všetkých čísel v~tabuľke sa zrejme po žiadnom kroku nezmení. Stále má rovnakú hodnotu ako na začiatku, \tj.
$$
S=1+2+\cdots+n^2=\frac{(n^2+1)n^2}2.
$$
Ak by bolo po niekoľkých krokoch všetkých $n^2$ čísel rovnakých, museli by sa rovnať číslu
$$
\frac S{n^2}=\frac{n^2+1}2.
$$
Avšak pre párne $n$ je čitateľ $n^2+1$ nepárny a~uvedený zlomok nie je celým číslom.

\zaver
Tabuľku, v~ktorej sú všetky čísla rovnaké, nemožno dostať pre žiadne~$n$.

\nobreak\medskip\petit\noindent
Za úplné riešenie dajte 6~bodov. Vylúčenie nepárnych hodnôt $n$ so správnym zdôvodnením oceňte 1~bodom.
Za pozorovanie, že súčet čísel v~tabuľke sa nemení, dajte 3~body, ďalší 1~bod za určenie hodnoty $(n^2+1)/2$, ktorá by musela byť na všetkých políčkach a~posledný bod za zdôvodnenie, že pre párne $n$ to nie je celé číslo.
Za úvahy, v~ktorých žiak iba konštatuje, že párne hodnoty~$n$ nevyhovujú, lebo na políčkach by na konci muselo byť necelé číslo (bez argumentov o~nemeniacom sa súčte) dajte 1~bod.
Za vylúčenie konečného počtu hodnôt~$n$ nedajte žiadne body.
\endpetit
\bigbreak}

{%%%%%   A-II-3
Označme $S$ stred strany~$AB$ a~$G$ obraz bodu~$F$ v~stredovej
súmernosti podľa stredu~$S$. Veľkosť uhla $BCA$ označme~$\gamma$. Z~ostrouhlosti trojuholníka $ABC$ je zrejmé, že bod~$G$ leží v~polrovine opačnej k~polrovine $ABC$ (\obr).
\insp{a57.1}

Štvoruholník $ADEC$ je tetivový (body $D$, $E$ ležia na Tálesovej kružnici nad priemerom~$AC$), preto
$$
|\uhol ADE|=180\st-\gamma\quad\text{a}\quad|\uhol ADG|=180\st-|\uhol ADE|=\gamma.
$$
Aj štvoruholník $BDFC$ je tetivový (body $D$, $F$ ležia na Tálesovej kružnici nad priemerom~$BC$), preto
$$
|\uhol BDF|=180\st-\gamma\quad\text{a}\quad|\uhol ADF|=180\st-|\uhol BDF|=\gamma.
$$
V~trojuholníku $FDG$ teda splývajú ťažnica a~os uhla z~vrcholu~$D$ a~ten je preto rovnoramenný. Takže úsečka~$FS$ je kolmá na úsečku~$AS$. Keďže bod~$F$ leží na Tálesovej kružnici nad priemerom $AB$, ktorá má
stred v~bode~$S$, je trojuholník $SFA$ rovnoramenný a~pravouhlý, a teda veľkosť uhla $BAC$ je $45^\circ$.

\ineriesenie
Podobne ako v~prvom riešení označme  body $S$ a~$G$. Veľkosť uhla $BAC$ označme~$\alpha$.
\insp{a57.2}

Bod $S$ je stredom Tálesovej kružnice~$k$ nad priemerom~$AB$. Na tejto kružnici ležia zrejme okrem bodov $A$, $B$ aj body $E$, $F$, a~$G$ -- pri pätách výšok sú pravé uhly a~bod~$G$ je od stredu súmernosti~$S$ rovnako vzdialený
ako bod~$F$ (\obr). Štvoruholník $ADEC$ je tetivový (body $D$, $E$ ležia na Tálesovej kružnici nad priemerom~$AC$), preto
$$
|\uhol DEC|=180\st-\alpha\quad\text{a}\quad|\uhol GEB|=180\st-|\uhol DEC|=\alpha.
$$
Keďže uhly $GEB$ a~$GAB$ sú obvodové uhly nad tetivou~$GB$ kružnice~$k$, aj uhol $GAB$ má veľkosť~$\alpha$. Preto uhol $FAG$ má veľkosť~$2\alpha$. Tento uhol je
však pravý, lebo úsečka~$FG$ je priemerom kružnice~$k$. Z~toho dostávame, že hľadaná veľkosť uhla $\alpha$ je $45^\circ$.

\nobreak\medskip\petit\noindent
Za úplné riešenie dajte 6~bodov.

Za prvé uvedené riešenie prideľte body takto: výpočet veľkostí uhlov
$ADF$ a~$ADG$ (so zdôvodnením) po 1~bode, kolmosť $FS$ na $AS$ 1~bod,
zdôvodnenie rovnoramennosti trojuholníka $ASF$ 1~bod, výpočet veľkosti
uhla $BAC$ 1~bod a~nakoniec za spojenie čiastkových úvah dokopy 1~bod.

Za druhé uvedené riešenie prideľte body takto: zdôvodnenie, že body
$A$, $G$, $B$, $E$, $F$ ležia na kružnici 2~body, výpočet veľkosti
uhla $GEB$ (so zdôvodnením) 1~bod, určenie veľkosti uhla $GAB$ 1~bod,
výpočet veľkosti uhla~$\alpha$ 2~body.

Neúplné riešenie, v~ktorom nie je určená veľkosť uhla $BAC$, môže byť
ohodnotené nanajvýš 4~bodmi.

\endpetit
\bigbreak}

{%%%%%   A-II-4
Z~nerovnosti medzi aritmetickým a~geometrickým priemerom trojice (nezáporných) čísel $x^2$, $1$, $1$ vyplýva odhad
$$
\frac{x^2+1+1}3\ge\root3\of{x^2\cdot1\cdot1},
$$
čiže $x^2+2\ge3\root3\of{x^2}$. Stačí teda dokázať nerovnosť $3\root3\of{x^2}\ge 3xy$. Po
vykrátení a~umocnení na šiestu dostaneme ekvivalentnú nerovnosť $x^4\ge x^6y^6$ a~po dosadení zadanej rovnosti $y^6=2-x^2$ a~prevedením na jednu stranu postupne získame
$$
\align
  x^4&\ge x^6(2-x^2),\\
  x^4-2x^6+x^8&\ge0,\\
  x^4(x^2-1)^2&\ge0.
\endalign
$$
Ostatná nerovnosť zjavne platí, platí teda aj zadaná nerovnosť.

\ineriesenie
Z~nerovnosti medzi aritmetickým a~geometrickým priemerom šestice (nezáporných) čísel $y^6$, $x^2$, $x^2$, $x^2$, $1$, $1$
vyplýva
$$
\frac{y^6+x^2+x^2+x^2+1+1}6\ge\root{6}\of{y^6\cdot x^2\cdot x^2\cdot x^2\cdot1\cdot1}=xy.
$$
Odtiaľ po dosadení zadanej rovnosti máme $(2x^2+4)/6\ge xy$. Vynásobením tromi získame dokazovanú nerovnosť.

\ineriesenie
Keď (ekvivalentne) umocníme dokazovanú nerovnosť na šiestu, budeme sa
môcť dosadením zadanej rovnosti zbaviť premennej $y$:
$$
(x^2+2)^6\ge3^6{\cdot}x^6y^6 = 3^6{\cdot}x^6(2-x^2).
$$
Po substitúcii $t=x^2$ a úprave dostaneme polynomickú nerovnosť s~jednou premennou
$$
t^6+12t^5+789t^4-1298t^3+240t^2+192t+64\ge 0.
$$
Mnohočlen na ľavej strane má dvojnásobný koreň~$1$, preto ho vieme
rozložiť na súčin dvoch činiteľov (napríklad použijeme známy postup pre delenie mnohočlena koreňovým činiteľom). Nerovnosť má potom tvar
$$
(t-1)^2\cdot (t^4+14t^3+816t^2+320t+64)\ge 0.
$$
Druhý činiteľ je pre $t=x^2\ge 0$ zjavne nezáporný, z čoho vidíme, že
ostatná nerovnosť platí, a preto platí aj s ňou ekvivalentná
dokazovaná nerovnosť.

\nobreak\medskip\petit\noindent
Za úplný dôkaz dajte 6~bodov. (Žiaci môžu bez dôkazu používať známe
nerovnosti, ako napr. váženú AG-nerovnosť alebo Cauchyho-Schwarzovu
nerovnosť.)\newline
V~prípade neúplného riešenia prideľte body takto (uvedené body sa nesčítajú!):
Za zredukovanie nerovnosti na polynomickú nerovnosť v~jednej premennej
dajte 3~body, ak je možné túto nerovnosť vyriešiť uhádnutím koreňa a~rozkladom na súčin podobne ako v~treťom riešení.
Ak žiak použije AG-nerovnosť spôsobom, ktorý vedie k~riešeniu (v~takej AG-nerovnosti musí nastávať rovnosť pre $x=y=1$), dajte nanajvýš 3~body (podľa náročnosti nedokončených úvah).
Za objavenie prípadu $x=y=1$, kedy nastáva rovnosť, dajte 1~bod.
\endpetit}

{%%%%%   A-III-1
Predpokladajme, že koeficienty $p$, $q$, $r$ spĺňajú zadané podmienky. Úpravou vzťahov $f(p)=p^3$, $f(q)=q^3$ získame
$$
\aligned
p^3+p^3+pq+r&=p^3,\\
p^3+pq+r&=0,
\endaligned
\qquad
\aligned
q^3+pq^2+q^2+r&=q^3,\\
pq^2+q^2+r&=0.
\endaligned
$$
Odčítaním oboch výsledných rovností a~ďalšou úpravou máme postupne
$$
\aligned
p^3-pq^2+pq-q^2&=0,\\
p(p-q)(p+q)+q(p-q)&=0,\\
(p-q)(p^2+pq+q)&=0.
\endaligned
$$
Podľa zadania $p\ne q$, po vydelení nenulovým výrazom $(p-q)$ preto ďalej dostaneme
$$
\aligned
p^2+pq+q&=0,\\
q(p+1)&= -p^2,
\endaligned
$$
odkiaľ vyplýva, že $p\ne\m1$. Za tohto predpokladu môžeme vyjadriť
$$
q = -\frac{p^2}{p+1} = -\frac{p^2-1+1}{p+1} = 1-p-\frac{1}{p+1}.
$$
Keďže $p$ aj $q$ sú celé čísla, musí byť aj zlomok $1/(p+1)$ celé číslo, teda $p+1\in\{1,\m1\}$. Vzhľadom na podmienku $p\ne0$ nutne $p=\m2$. Potom $q=1-(\m2)-1/(\m2+1)=4$ a~ľahko dopočítame, že $r=16$. Skúškou overíme, že trojica $(p,q,r)=(\m2,4,16)$ spĺňa zadané podmienky.}

{%%%%%   A-III-2
Označme vnútorné uhly trojuholníka $ABC$ zvyčajným spôsobom. Ďalej označme $l$ kružnicu opísanú trojuholníku $BDS$ a~$m$ kružnicu opísanú trojuholníku $AES$. Body $D$ a~$E$ ležia na Tálesovej kružnici~$k$ so stredom~$S$ a~priemerom~$AB$, preto trojuholníky $BSD$, $ASE$ sú rovnoramenné so základňami $BD$, $AE$. Aby sme nemuseli rozoberať rôzne prípady, dokážeme najskôr, že bod~$K$ leží vždy vnútri trojuholníka $ABC$.

Zrejme oblúk~$BD$ kružnice~$l$ {\it neobsahujúci\/} bod~$S$ sa nepretína s~oblúkom~$AE$ kružnice~$m$ {\it neobsahujúcim\/} bod~$S$. Totiž prvý z~nich leží celý v~polrovine opačnej k~polrovine $BCA$, druhý leží celý v~polrovine opačnej k~polrovine $ACB$, takže pretínať by sa mohli len v~uhle~$\gamma'$, ktorý je vrcholovým uhlom k~vnútornému uhlu~$\gamma$ trojuholníka $ABC$ (\obr).
\insp{a57.3}%
Avšak aspoň jeden z~uhlov $BSD$, $ASE$ je ostrý (keďže ich súčet je menej ako $180^\circ$), nech je to napríklad uhol $ASE$. Potom je sledovaný oblúk~$AE$ "kratším" oblúkom svojej kružnice (lebo k~nemu prislúcha tupý obvodový uhol) a~teda leží celý v~polrovine $BEA$, ktorá nemá s~uhlom $\gamma'$ žiadny spoločný bod.

Oblúk~$BD$ kružnice~$l$ {\it neobsahujúci\/} bod~$S$ sa nepretína ani s~oblúkom~$AE$ kružnice~$m$ {\it obsahujúcim\/} bod~$S$, lebo prvý z~nich leží zvonka kružnice~$k$, zatiaľ čo druhý leží vnútri kružnice~$k$ (\obr).
\insp{a57.4}%
Z~podobných dôvodov sa nepretínajú ani oblúk~$BD$ kružnice~$l$ {\it obsahujúci\/} bod~$S$ s~oblúkom~$AE$ kružnice~$m$ {\it neobsahujúcim\/} bod~$S$.

Kružnice $l$, $m$ sa teda musia pretínať na tých oblúkoch $BD$ a~$AE$, ktoré obsahujú bod~$S$. Prvý z~nich leží v~polrovine $BCA$, druhý v~polrovine $ABC$, takže ich priesečník musí ležať v~uhle~$\gamma$. Z~rovnoramennosti trojuholníkov $BSD$, $ASE$ vyplýva
$$
|\uhol BDS|=\beta,\qquad |\uhol AES|=\alpha,\tag1
$$
takže uhly $BDS$, $AES$ sú ostré. Preto oblúky $BS$, $AS$ kružníc $l$, $m$ neobsahujúce postupne body $D$, $E$ sú "kratšími" oblúkmi svojich kružníc (prislúcha im tupý obvodový uhol) a~nemôžu sa pretínať (oddeľuje ich os strany~$AB$, \obr).
\insp{a57.5}%
Kružnice $l$, $m$ sa teda nemôžu pretínať v~polrovine opačnej k~polrovine $ABC$. Spolu s~predošlým zistením dostávame, že ich priesečník (rôzny od $S$) musí ležať vnútri trojuholníka $ABC$.

\smallskip
a)
Aby sme dokázali, že body $D$, $E$, $V$, $K$ ležia na jednej kružnici, stačí dokázať, že bod~$K$ leží na kružnici prechádzajúcej bodmi $D$, $E$, $V$, ktorou je zrejme Tálesova kružnica~$t$ s~priemerom~$CV$. Z~tetivových štvoruholníkov $BSKD$, $ASKE$ vyplýva $|\uhol SKD|=180^\circ-\beta$, $|\uhol SKE|=180^\circ-\alpha$. Preto
$$
\aligned
|\uhol EKD|&=360^\circ-|\uhol SKD|-|\uhol SKE|=360^\circ-(180^\circ-\beta)-(180^\circ-\alpha)=\alpha+\beta=\\
&=180^\circ-\gamma=180^\circ-|\uhol DCE|.
\endaligned
$$
Takže štvoruholník $CEKD$ je tetivový a~bod~$K$ naozaj leží na kružnici~$t$ (\obr).
\insp{a57.6}%

\smallskip
b)
Ak $V=K$, Tak body $F$, $V$, $K$ celkom určite ležia na jednej priamke. Zaoberajme sa ďalej len prípadom $V\ne K$\footnote{Dá sa ukázať, že predpoklady zadania vylučujú prípad $V=K$, ale pri riešení to nepotrebujeme.}. Ukážeme najprv, že body $A$, $V$, $K$, $B$ ležia na jednej kružnici. Z~pravouhlých trojuholníkov $ABD$, $ABE$ dostávame $|\uhol BAD|=90^\circ-\beta$, $|\uhol ABE|=90^\circ-\alpha$, preto z~trojuholníka $ABV$ máme
$$
|\uhol AVB|=180^\circ-(90^\circ-\beta)-(90^\circ-\alpha)=\alpha+\beta
$$
(uvedený vzťah možno odvodiť aj z~tetivového štvoruholníka $CEVD$ na \obrr1{}). Podľa \thetag1 a~z~vlastností obvodových uhlov v~tetivových štvoruholníkoch $BSKD$, $ASKE$ máme
$$
|\uhol AKS|=|\uhol AES|=\alpha,\qquad |\uhol BKS|=|\uhol BDS|=\beta.
$$
Takže $|\uhol AKB|=\alpha+\beta=|\uhol AVB|$ a~body $A$, $V$, $K$, $B$ naozaj ležia na jednej kružnici, označme ju $u$ (\obr).
\insp{a57.7}%

Body $V$, $K$ sú teda priesečníkmi kružnice $u$ s~kružnicou~$t$ z~časti~a). Priamka~$VK$ je preto chordálou kružníc $t$, $u$. Aby sme ukázali, že na nej leží bod $F$, stačí ukázať, že jeho mocnosť k~obom kružniciam je rovnaká. To je však pravda, lebo z~mocnosti bodu $F$ ku kružnici $k$ vyplýva
$$
|FE|\cdot|FD|=|FA|\cdot|FB|.
$$
Pravá strana tejto rovnosti je zároveň mocnosťou bodu~$F$ ku kružnici~$t$, ľavá strana je jeho mocnosťou ku kružnici~$u$. Tým je časť b) dokázaná\footnote{Možno argumentovať aj priamejšie: priamky $ED$, $VK$, $AB$ sú chordálami kružníc $k$, $t$, $u$ (každá priamka prislúcha jednej dvojici kružníc) preto sa pretínajú v~jednom bode.}.

\poznamka
Pomocou mocnosti bodu~$C$ ku kružniciam $k$, $l$, $m$ možno podobne odvodiť, že $K$ leží na priamke~$CS$. Tento fakt môže jednak pomôcť dokázať, že $K$ leží vnútri trojuholníka $ABC$ (iným postupom, ako sme to urobili tu), jednak poskytuje alternatívne možnosti na dôkaz časti a) (dá sa ukázať, že priamky $VK$ a~$CS$ sú na seba kolmé).
}

{%%%%%   A-III-3
Najskôr dokážeme, že pre ľubovoľné párne~$n$ je možné dostať požadovanú tabuľku. Ukážeme jeden z~mnohých možných postupov. Zrejme ak sú na dvoch susedných políčkach celé čísla líšiace sa práve o~$1$, po konečnom počte krokov vieme dostať (bez zmeny ostatných políčok) tabuľku, v~ktorej väčšie z~oboch čísel bude nahradené číslom $365$ a~menšie číslom $364$: Ak sú na daných dvoch susedných políčkach čísla $k$ a~$k+1$, stačí $|364-k|$-krát spraviť krok, v~ktorom na oboch týchto políčkach čísla o~$1$ zväčšíme (keď $k\le364$), resp. zmenšíme (keď $k>364$). Na začiatku sú na susedných políčkach v~každom riadku čísla líšiace sa práve o~$1$ (pričom číslo "napravo" je vždy väčšie), navyše v~každom riadku je párne veľa políčok. Môžeme teda políčka každého riadku rozdeliť do dvojíc a~každú dvojicu čísel na nich po konečnom počte krokov zmeniť na dvojicu $(364,365)$. Dostaneme tak tabuľku, v~ktorej sú na políčkach v~nepárnych stĺpcoch len čísla $364$ a~v~párnych stĺpcoch len čísla $365$. Teraz už stačí rozdeliť do dvojíc políčka v~nepárnych stĺpcoch a~každú dvojicu čísel $(364,364)$ nahradiť po jednom kroku dvojicou $(365,365)$. Pre hodnotu $n=6$ je postup načrtnutý na \obr.
\insp{a57.8}

\smallskip
Zaoberajme sa ďalej prípadom, keď $n$ je nepárne. Ofarbime celú tabuľku striedavo čiernou a~bielou farbou ako šachovnicu, pričom prvé políčko (\tj. to, na ktorom je na začiatku číslo~$1$) bude biele. Čísla, ktoré sú na bielych políčkach, nazývajme {\it biele}, čísla na čiernych políčkach nazývajme {\it čierne}. Keďže susedné políčka majú opačnú farbu, v~každom kroku zmeníme jedno biele a~jedno čierne číslo. Ak teda označíme $B$ súčet všetkých bielych čísel a~$C$ súčet všetkých čiernych čísel, rozdiel $R=B-C$ sa po žiadnom kroku nezmení (v~každom kroku sa buď $B$ aj $C$ zväčšia o~$1$, alebo sa obe zmenšia o~$1$). Na začiatku sú biele čísla všetky nepárne a~čierne všetky párne, takže
$$
\aligned
 R&=(1+3+\cdots+n^2)-\bigl(2+4+\cdots+(n^2-1)\bigr)=\\
  &=1+(3-2)+(5-4)+\cdots+\bigl(n^2-(n^2-1)\bigr)=\underbrace{1+1+\cdots+1}_{\text{$(n^2+1)/2$-krát}}=\frac{n^2+1}2.
\endaligned
$$
Bielych políčok je o~jedno viac ako čiernych. Ak by teda bolo po nejakom počte krokov na všetkých políčkach číslo $365$, mal by rozdiel~$R$ hodnotu $365$. To znamená, že nutnou podmienkou, aby sa tabuľka dala zmeniť na požadovaný tvar, je rovnosť
$$
  \frac{n^2+1}2=365,
$$
ktorá nastáva jedine pre $n=27$.

\smallskip
Dokázali sme, že pre nepárne čísla rôzne od $27$ nie je možné dostať tabuľku so všetkými číslami rovnými $365$. Zostáva ukázať, že pre $n=27$ to možné je. Zrejme nech sú na dvoch susedných políčkach ľubovoľné celé čísla, vieme po konečnom počte krokov (bez zmeny ostatných políčok) dosiahnuť, že jedno z~nich bude rovné $365$ (stačí čísla na oboch políčkach príslušný počet krát zväčšiť alebo zmenšiť).

Ukážeme, že týmto postupom dokážeme tabuľku zmeniť tak, že na všetkých políčkach okrem posledného (v~pravom dolnom rohu) bude číslo $365$. Môžeme to urobiť napríklad nasledovne: Najprv číslo na prvom políčku zmeníme pomocou druhého políčka na $365$, potom číslo na druhom políčku pomocou tretieho, atď. Tak dostaneme v~celom prvom riadku až na jeho posledné políčko číslo $365$. Rovnaký postup aplikujeme v~každom riadku (\obr).
\insp{a57.9}%
Tým dostaneme číslo $365$ na všetkých políčkach okrem posledného stĺpca. Keď teraz rovnaký "riadkový" postup aplikujeme na posledný stĺpec, získame číslo $365$ všade okrem posledného políčka.

Nech na poslednom políčku vzniklo uvedeným postupom číslo~$k$. Ako sme dokázali už skôr, rozdiel~$R=B-C$ nikdy nemení svoju hodnotu. Keďže na začiatku bola jeho hodnota $(27^2+1)/2=365$, musí byť rovnaká aj teraz, čiže
$$
365=B-C=(\underbrace{365+365+\cdots+365}_{\text{364-krát}}+k)-(\underbrace{365+365+\cdots+365}_{\text{364-krát}})=k.
$$
Teda na poslednom políčku muselo pri uvedenom postupe vzniknúť číslo $365$ a~dostali sme priamo požadovanú tabuľku.

\zaver
Tabuľku, v~ktorej sa všetky čísla rovnajú $365$, možno dostať pre všetky párne $n$ a~pre $n=27$.
}

{%%%%%   A-III-4
Podľa známeho vzťahu pre rozklad rozdielu dvoch tretích mocnín na súčin máme
$$
27^n-n^{27} = (3^n)^3-(n^9)^3=(3^n-n^9)(9^n+3^n\cdot n^9+n^{18}).
$$
Druhá zátvorka je zrejme väčšia ako $1$. Stačí teda dokázať, že prvá zátvorka nie je nikdy rovná $1$.

Najprv matematickou indukciou dokážeme, že pre $n\ge 30$ je $3^n>n^9+1$.

\krok1
Pre $n=30$ máme
$$
3^{30}=3^{10}\cdot (3^5)^4>3^{10}\cdot (2\cdot 10^2)^4=3^9\cdot48\cdot 10^8>3^9(10^9+1)>30^9+1.
$$

\krok2
Predpokladajme, že tvrdenie platí pre hodnotu $n=k$, pričom $k\ge30$. Máme teda $3^k>k^9+1$. Potom
$$
3^{k+1}=3\cdot3^k>3\cdot(k^9+1)=3k^9+3=(\root9\of3k)^9+3>(k+1)^9+3>(k+1)^9+1,
$$
čiže tvrdenie platí aj pre hodnotu $n=k+1$. Pri úprave sme okrem indukčného predpokladu použili nerovnosť $\root9\of3k>k+1$. Tá naozaj platí, lebo
$$
1{,}1^9=(1{,}1^3)^3=1{,}331^3<1{,}4^3=2{,}744<3,
$$
a~teda $\root9\of3k>1{,}1k>k+1$ (keďže $k\ge30$).

\smallskip
Ostáva dokázať, že $V=3^n-n^9\ne 1$ aj pre $n<30$. Nebolo by ťažké pre každé zostávajúce $n$ hodnotu~$V$ priamo vypočítať\footnote{Dá sa ukázať, že pre $n\ge28$ je $V>1$, pre $n=27$ je $V=0$, pre $2\le n\le26$ je $V<0$ a~pre $n=1$ je $V=2$.}. Uvedieme však rýchlejší postup používajúc zvyšky po delení rôznymi číslami.

Ak $n$ je nepárne, tak $V$ je rozdielom dvoch nepárnych čísel, teda párne, a~preto rôzne od~$1$.

Ak $n$ je deliteľné tromi, tak $3\mid 3^n$ a~súčasne $3\mid n^9$, čiže aj $V$ je deliteľné tromi, a~preto rôzne od~$1$.

Ak $n$ dáva po delení tromi zvyšok~$1$, tak $3\mid 3^n$ a~$n^9$ dáva po delení tromi zvyšok~$1$. Teda $V=3^n-n^9$ dáva zvyšok~$2$ a~je rôzne od~$1$.

Ostáva preveriť čísla, ktoré sú párne a~dávajú zvyšok~$2$ po delení tromi, \tj. čísla z~množiny $\{2,8,14,20,26\}$. Pre $n=2$ a~pre $n=8$ je zrejme $3^n-n^9<0$. Podobne
$$
\aligned
  3^{14}-14^9&=9^7-14^9<0,\\
  3^{20}-20^9&=9^{10}-2^9\cdot10^9<10^{10}-512\cdot 10^9<0.
\endaligned
$$
Napokon aj $3^{26}-26^9\ne1$, lebo zápis čísla $3^{26}=81^6\cdot9$ končí cifrou~$9$, zápis čísla $26^9$ končí cifrou~$6$, teda číslo $3^{26}-26^9$ dáva po delení desiatimi zvyšok~$3$ a~je rôzne od~$1$.}

{%%%%%   A-III-5
Pokúsme sa dokazovanú nerovnosť zapísať ako súčet niekoľkých nerovností, o~ktorých vieme, že platia.
Podľa AG-nerovnosti\dorocenky{\global\edef\strankaAG{\the\pageno}}\footnote{AG-nerovnosť je skrátené označenie známej nerovnosti medzi aritmetickým a~geometrickým priemerom. Podľa nej pre ľubovoľné prirodzené číslo~$n$ a~nezáporné čísla $a_1,\dots,a_n$ platí $$\frac{a_1+\cdots+a_n}n\ge\root n\of{a_1\dots a_n}.$$ } pre ľubovoľné prirodzené čísla $r$, $s$ platí
$$
\frac{rx^k+sy^k+sz^k}{r+2s} \ge \root{r+2s}\of{(x^k)^r(y^k)^s(z^k)^s}=\root{r+2s}\of{x^{k(r-s)}(xyz)^{ks}},
\tag1
$$
a~keďže podľa zadania $xyz=1$, máme
$$
rx^k+sy^k+sz^k \ge (r+2s)x^{\frac{(r-s)k}{r+2s}}.
\tag2
$$
Cyklickou zámenou dostaneme podobné nerovnosti
$$
sx^k+ry^k+sz^k \ge (r+2s)y^{\frac{(r-s)k}{r+2s}}\qquad\text{a}\qquad
sx^k+sy^k+rz^k \ge (r+2s)z^{\frac{(r-s)k}{r+2s}}.
\tag3
$$
Hľadajme také prirodzené čísla $r$, $s$, aby súčtom uvedených troch nerovností bola dokazovaná nerovnosť, resp. nejaký jej násobok. Keďže na pravej strane potrebujeme dostať $m$-té mocniny, nutnou podmienkou je rovnosť
$$
m=\frac{(r-s)k}{r + 2s}.\tag4
$$
Po sčítaní nerovností budú koeficienty na ľavej aj pravej strane rovné $r+2s$, čo nám vyhovuje (po vydelení výrazom $r+2s$ dostaneme priamo dokazovanú nerovnosť). Stačí teda splniť rovnosť \thetag4. Tá platí napríklad pre hodnoty $m=r-s$, $k=r+2s$, odkiaľ ľahko vyjadríme $r=\frac13(2m+k)$, $s=\frac13(k-m)$. Keďže však chceme, aby $r$, $s$ boli prirodzené\footnote{V skutočnosti to nepotrebujeme, nerovnosť \thetag1 totiž podľa všeobecnejšie sformulovanej AG-nerovnosti platí pre ľubovoľné kladné reálne čísla $r$, $s$.}, zoberme hodnoty $r=2m+k$, $s=k-m$ (keďže $k$, $m$ sú prirodzené čísla a~$k>m$, sú takéto $r$, $s$ naozaj prirodzené). Pre ne platí
$$
\frac{(r-s)k}{r+2s}=\frac{3m\cdot k}{3k}=m,
$$
teda tiež spĺňajú \thetag4. Sčítaním troch nerovností \thetag2, \thetag3 pre uvedené hodnoty $r$, $s$ tak dostaneme
$$
(r+2s)(x^k + y^k + z^k)\ge(r+2s)(x^m + y^m + z^m),
$$
Z~čoho už priamo vyplýva zadaná nerovnosť.

\ineriesenie
Podľa známej nerovnosti medzi mocninovými priemermi platí
$$
\root k\of{\frac{x^k+y^k+z^k}3}\ge\root m\of{\frac{x^m+y^m+z^m}3},
$$
Z~čoho po jednoduchých ekvivalentných úpravách dostávame
$$
x^k+y^k+z^k\ge(x^m+y^m+z^m)^{\frac km}\cdot 3^{1-\frac km}.\tag5
$$
Podľa AG-nerovnosti (s~využitím zadaného predpokladu $xyz=1$) platí
$$
x^m+y^m+z^m\ge3\root 3\of{x^my^mz^m}=3,
$$
a~keďže $k>m$, čiže $\frac km-1>0$, tak aj
$$
(x^m+y^m+z^m)^{\frac km-1}\ge 3^{\frac km-1}.
$$
Odtiaľ
$$
(x^m+y^m+z^m)^{\frac km}\cdot 3^{1-\frac km}\ge x^m+y^m+z^m,
$$
čo spolu s~\thetag5 dáva dokazovanú nerovnosť.
}

{%%%%%   A-III-6
Ak $a=b$, ani jeden z~výrazov nenadobúda žiadnu hodnotu (menovatele sú nulové). Zaoberajme sa preto len trojuholníkmi, ktorých strany $a$, $b$ majú rôzne dĺžky.

\smallskip
a) Podľa známych vyjadrení dĺžok ťažníc pomocou dĺžok strán (ktoré možno ľahko odvodiť pomocou kosínusových viet) platí
$$
t_a^2 = \frac{2b^2+2c^2-a^2}4,\qquad t_b^2 = \frac{2a^2+2c^2-b^2}4.
\tag1
$$
Takže priamym dosadením dostávame
$$
\frac{t_a^2-t_b^2}{b^2-a^2}=\frac{(2b^2+2c^2-a^2)-(2a^2+2c^2-b^2)}{4(b^2-a^2)}=\frac{3b^2-3a^2}{4(b^2-a^2)}=\frac34.
$$
Teda jediná možná hodnota prvého výrazu je $\frac34$.

\smallskip
b) Skúmaním rôznych "degenerovaných" prípadov najskôr uhádneme výsledok. Napríklad ak $b=1$ a~bod~$B$ sa nachádza "blízko" stredu strany~$AC$, tak $t_a\approx\frac34$, $t_b\approx0$, $a\approx\frac12$, čiže
$$
\frac{t_a-t_b}{b-a}\approx\frac{\frac34-0}{1-\frac12}=\frac32.
$$
Ak $b=1$ a~bod~$B$ sa nachádza "blízko" bodu~$C$, tak $t_a\approx1$, $t_b\approx\frac12$, $a\approx0$, čiže
$$
\frac{t_a-t_b}{b-a}\approx\frac{1-\frac12}{1-0}=\frac12.
$$
Skúsme teda dokázať, že
$$
\frac12<\frac{t_a-t_b}{b-a}<\frac32.
\tag2
$$
S~použitím výsledku časti~$a)$ máme
$$
\frac{t_a-t_b}{b-a}=\frac{t_a^2-t_b^2}{t_a+t_b}\cdot\frac{b+a}{b^2-a^2}=\frac34\cdot\frac{a+b}{t_a+t_b}.
$$
Nerovnosti \thetag2 sú preto ekvivalentné s~nerovnosťami
$$
\frac23<\frac{a+b}{t_a+t_b}<2,
\tag3
$$
ktoré ľahko dokážeme pomocou trojuholníkových nerovností.
\insp{a57.10}

Naozaj, ak označíme $S_A$, $S_B$ postupne stredy strán $BC$, $AC$, tak z~trojuholníkových nerovností v~trojuholníkoch $ACS_A$, $BCS_B$ máme (\obr)
$$
b+\frac a2>t_a,\qquad a+\frac b2>t_b.
$$
Odtiaľ sčítaním dostaneme $\frac32(a+b)>t_a+t_b$, čo je ekvivalentné s~prvou nerovnosťou v~\thetag3.

A~ak označíme $T$ ťažisko trojuholníka $ABC$, ktoré rozdeľuje každú ťažnicu v~známom pomere (\obrr1), tak z~trojuholníkových nerovností v~trojuholníkoch $ATS_B$, $BTS_A$ máme
$$
\frac23t_a+\frac{t_b}3>\frac b2,\qquad \frac23t_b+\frac{t_a}3>\frac a2.
$$
Odtiaľ sčítaním dostaneme $t_a+t_b>\frac12(a+b)$, čo je ekvivalentné s~druhou nerovnosťou v~\thetag3.

Dokázali sme teda, že platí \thetag2. Aby sme dokončili riešenie úlohy, musíme ešte ukázať, že zadaný výraz môže nadobúdať všetky možné hodnoty z~intervalu $(\frac12,\frac32)$. Na to stačí uvažovať trojuholníky so stranami dĺžok $a=1$, $b=2$, $c\in(1,3)$. Potom podľa \thetag1 máme
$$
\align
\frac{t_a-t_b}{b-a}&=\frac{\frac12\sqrt{8+2c^2-1}-\frac12\sqrt{2+2c^2-4}}{2-1}=\frac{\sqrt{2c^2+7}-\sqrt{2c^2-2}}2=\\
&=\frac{9/2}{\sqrt{2c^2+7}+\sqrt{2c^2-2}}
\tag4
\endalign
$$
Tento výraz nadobúda pre $c=1$ hodnotu $\frac32$ a~pre $c=3$ hodnotu $\frac12$. Ak ho teda chápeme ako funkciu premennej~$c$, musí na intervale $(1,3)$ nadobúdať všetky hodnoty z~intervalu $(\frac12,\frac32)$. To vyplýva zo spojitosti uvedenej funkcie. Z~jej vyjadrenia dokonca vidíme, že je na skúmanom intervale klesajúca.

\zaver
Druhý výraz môže nadobúdať ľubovoľnú hodnotu z~intervalu $(\frac12,\frac32)$.

\poznamka
To, že výraz \thetag4 nadobúda pre $c\in(1,3)$ všetky hodnoty z~intervalu $(\frac12,\frac32)$, možno dokázať aj bez použitia vedomostí o~spojitých funkciách. Stačí ukázať, že pre každé $h\in(\frac12,\frac32)$ má rovnica
$$
\frac{\sqrt{2c^2+7}-\sqrt{2c^2-2}}2=h
$$
s~neznámou $c$ riešenie v~intervale $(1,3)$. Ľahko možno vyjadriť riešenie
$$
c=\sqrt{1+\frac{(9-4h^2)^2}{32h^2}}.
$$
Tento výraz s~rastúcim $h$ pre $h\in\langle\frac12,\frac32\rangle$ klesá a~keďže pre hodnoty $h=\frac12$, $h=\frac32$ nadobúda postupne hodnoty $c=3$, $c=1$, bude pre $h\in(\frac12,\frac32)$ riešenie $c$ vždy v~intervale $(1,3)$.
}

{%%%%%   B-S-1
Ľubovoľné prvočíslo $p$ sa dá napísať v~tvare $p=30a+z$,
kde $a$ je celé nezáporné a~$z\in\{1,2,\dots29\}$ je zvyšok po delení čísla $p$ tridsiatimi (keď $p$ je prvočíslo, môžeme nulový zvyšok $z$ vylúčiť).

Ak $p$ je prvočíslo menšie ako $30$, zrejme $z=p$ je tiež prvočíslo.

Predpokladajme teda, že $p$ je prvočíslo väčšie ako~$30$, \tj. $a\ge 1$.
Pripusťme, že zvyšok~$z$ nie je ani číslo~$1$ ani prvočíslo a~označme $q$ jeho najmenší prvočíselný deliteľ.
Zrejme $q^2\le z<30<7^2$, odkiaľ $q<7$, čiže $q\in \{2,3,5\}$.
Keďže číslo $30$ je deliteľné dvoma, tromi aj piatimi, je deliteľné prvočíslom~$q$.
Takže aj číslo $p=30a+z$ je prvočíslom~$q$ deliteľné. Nemôže to teda
byť prvočíslo.

\ineriesenie
Vyjadríme číslo $p$ v~tvare $p=30a+z$. Keby bolo zvyškom~$z$ niektoré z~čísel $0$, $4$, $6$, $8$, $10$, $12$, $14$, $16$, $18$, $20$, $22$, $24$, $26$, $28$, bolo by $p$ párne a~pritom väčšie ako $2$, takže by nebolo prvočíslom. Keby bolo zvyškom niektoré z~čísel $9$, $15$, $21$, $27$, bolo by $p$ deliteľné tromi a~pritom väčšie ako $3$ a~nemohlo by byť prvočíslom. Nakoniec pri zvyšku $25$ by bolo $p$ deliteľné piatimi a~pritom väčšie ako $5$, opäť by to teda nebolo prvočíslo.

\nobreak\medskip\petit\noindent
Za úplné riešenie dajte 6~bodov.
Za vylúčenie všetkých devätnástich zvyškov (zložených čísel a nuly) dajte 6 bodov.
Za vyjadrenie čísla $p$ v tvare $p=30a+z$, kde $a$ je celé nezáporné a $z\in \{0,1,\dots,29\}$, dajte 1~bod; 2~body za postreh, že každé zložené číslo menšie ako 30 je deliteľné dvoma, tromi alebo piatimi a ďalšie 3 body za správne dokončenie dôkazu.
\endpetit
\bigbreak}

{%%%%%   B-S-2
Nech $x_0$ je spoločný koreň oboch rovníc. Potom platí
$$
x_0^2+(3a+b)x_0+4a=0, \quad x_0^2+(3b+a)x_0+4b=0.
$$
Odčítaním týchto rovníc dostaneme $(2a-2b)x_0+4(a-b)=0$, odkiaľ po úprave získame $(a-b)(x_0+2)=0$.

Rozoberieme dve možnosti:

Ak $a=b$, majú obidve dané rovnice rovnaký tvar $x^2+4ax+4a=0$. Aspoň jeden koreň (samozrejme spoločný) existuje práve vtedy, keď je diskriminant $16a^2-16a$ nezáporný, teda $a\in(\m\infty,0\rangle\cup\langle1,\infty)$.

Ak $x_0=\m2$, dostaneme z~prvej aj z~druhej rovnice $4-2a-2b=0$, teda $b=2-a$. Dosadením do zadania dostaneme rovnice
$$
x^2+(2a+2)x+4a=0,\quad x^2+(6-2a)x+8-4a=0,
$$ ktoré majú pri ľubovoľnej hodnote parametra~$a$ spoločný koreň~$\m2$.

\zaver
Dané rovnice majú aspoň jeden spoločný koreň pre všetky dvojice $(a,a)$, kde $a\in (-\infty, 0\rangle \cup \langle 1,\infty)$, a~pre všetky dvojice tvaru ${(a,2-a)}$, kde $a$ je ľubovoľné.

\nobreak\medskip\petit\noindent
Za úplné riešenie dajte 6~bodov. Za odvodenie podmienky $(a-b)(x_0+2)=0$ dajte 2~body, 2~body za správny rozbor možnosti $a=b$, 2~body za nájdenie riešenia $b=2-a$.
\endpetit
\bigbreak}

{%%%%%   B-S-3
Zo zhodnosti trojuholníkov $ABM$ a~$CBM$ $(sus)$ vyplýva $|CM|=|AM|$; preto musí bod~$C$ ležať na kružnici so stredom~$M$ a~polomerom~$|AM|$ (\obr). Uhlopriečky kosoštvorca (štvorca) sú na seba kolmé, preto body $B$ a~$D$ ležia na kolmici vedenej bodom~$M$ na priamku~$AC$.
\insp{b57.6}

\konstrukcia
Zostrojíme kružnicu~$k$ so stredom~$M$ a~polomerom~$|AM|$. Priesečník tejto kružnice s~priamkou~$q$ je bod~$C$. Bodom~$M$ vedieme kolmicu na priamku~$AC$. Jej priesečníky s~priamkami $p$ a~$q$ sú body $B$ a~$D$ (\obr).
Zostrojený štvoruholník má zrejme všetky požadované vlastnosti.
\insp{b57.7}

\diskusia
Ak je vzdialenosť bodu~$M$ od priamky~$q$ väčšia ako jeho vzdialenosť od bodu~$A$, nemá kružnica~$k$ s~priamkou~$q$ spoločný bod a~úloha nemá riešenie.

Ak má bod~$M$ rovnakú vzdialenosť od priamky~$q$ ako od bodu~$A$, má kružnica~$k$ s~priamkou~$q$ jediný spoločný bod~$C$. Pokiaľ bod~$M$ neleží na osi pásu medzi rovnobežkami $p$ a~$q$, nie je priamka~$AC$ kolmá na $p$, preto kolmica vedená bodom~$M$ na priamku~$AC$ nie je s~priamkou~$p$ rovnobežná a~úloha má jedno riešenie. Pokiaľ ale bod~$M$ leží na osi pásu (je to teda priesečník osi pásu s~kolmicou vedenou bodom~$A$ na priamku~$p$), nemá úloha riešenie.

Ak je vzdialenosť bodu~$M$ od priamky~$q$ menšia ako jeho vzdialenosť od bodu~$A$, pretína kružnica~$k$ priamku~$q$ v~dvoch bodoch. Pokiaľ bod~$M$ leží na osi pásu medzi rovnobežkami $p$ a~$q$, leží jeden z~priesečníkov na kolmici vedenej bodom~$A$ na priamku~$p$ a~úloha má jedno riešenie; ak bod~$M$ na osi pásu neleží, má úloha dve riešenia.

\ineriesenie
Priesečník~$S$ uhlopriečok kosoštvorca (štvorca) $ABCD$ musí ležať na osi pásu medzi rovnobežkami $p$ a~$q$.

Ak leží bod~$M$ na osi pásu, musí platiť $S=M$; bod~$C$ je potom priesečník priamok $AS$ a~$q$, $B$ a~$D$ sú priesečníky kolmice na priamku~$AC$ vedenú bodom~$M$ s~priamkami $p$ a~$q$. Ak $AM \perp p$, nemá úloha riešenie, inak má jedno riešenie.

Ak bod~$M$ neleží na osi pásu, je uhol $ASM$ pravý. Preto je bod~$S$ priesečníkom osi pásu s~Tálesovou kružnicou nad priemerom~$AM$. Body $C$, $B$, $D$ potom nájdeme podobne ako je uvedené vyššie. Podľa počtu spoločných bodov osi pásu a~Tálesovej kružnice má potom úloha dve riešenia, jedno riešenie alebo nemá žiadne riešenie.

\ineriesenie
Bod~$M$ leží na osi uhla $ADC$, preto má od priamok $AD$ a~$q$ rovnakú vzdialenosť. Priamka~$AD$ je teda dotyčnicou kružnice, ktorá má stred~$M$ a~dotýka sa priamky~$q$.

\konstrukcia
Zostrojíme kružnicu~$h$ so stredom~$M$, ktorá sa dotýka priamky~$q$. Vrchol~$D$ hľadaného kosoštvorca (štvorca) je priesečník priamky~$q$ s~dotyčnicou kružnice~$h$ prechádzajúcou bodom~$A$. Body $B$ a~$C$ potom už nájdeme ľahko.

\diskusia
Ak má bod~$M$ od bodu~$A$ menšiu vzdialenosť ako od priamky~$q$, neprechádza bodom~$A$ žiadna dotyčnica kružnice~$h$ a~úloha nemá riešenie.

Ak má bod~$M$ od bodu~$A$ rovnakú vzdialenosť ako od priamky~$q$, leží bod~$A$ na kružnici~$h$ a~prechádza ním jedna dotyčnica tejto kružnice. Pokiaľ pritom bod~$M$ leží na osi pásu medzi rovnobežkami $p$ a~$q$, je touto dotyčnicou priamka~$p$, ktorá priamku~$q$ nepretína, a~úloha nemá riešenie. Pokiaľ ale bod~$M$ na osi pásu neleží, dotyčnica je s~priamkou~$q$ rôznobežná a~úloha má jedno riešenie.

Ak má bod~$M$ od bodu~$A$ väčšiu vzdialenosť ako od priamky~$q$, existujú dve dotyčnice kružnice~$h$ prechádzajúce bodom~$A$. Pokiaľ pritom bod~$M$ leží na osi pásu, je jednou z~dotyčníc priamka~$p$ a~úloha má jedno riešenie; pokiaľ bod~$M$ na osi pásu neleží, sú obidve dotyčnice s~$q$ rôznobežné a~úloha má dve riešenia.

\bigskip\petit\noindent
Za úplné riešenie dajte 6~bodov.

Pri postupe ako v~prvom riešení dajte 2~body za nájdenie bodu~$C$, 2~body za zostrojenie bodov $B$ a~$D$ a~dva body za úplnú diskusiu.

Pri postupe ako v~druhom riešení za poznatok, že bod~$S$ leží na osi pásu, dajte 1~bod; za vyriešenie úlohy pre prípad, keď $M$ leží na osi pásu, dajte 2~body (z~toho 1~bod za diskusiu); za vyriešenie úlohy pre prípad, keď $M$ na osi pásu neleží, dajte 3~body (z~toho 1~bod za diskusiu).

Pri postupe ako v~treťom riešení dajte 3~body za zostrojenie bodu~$D$, 1~bod za dokončenie konštrukcie a~2~body za diskusiu.

Dôkaz správnosti konštrukcie je pri všetkých troch uvedených riešeniach natoľko zrejmý, že môže chýbať v~inak úplných riešeniach ohodnotených 6~bodmi.
\endpetit}

{%%%%%   B-II-1
Odčítaním oboch daných rovníc dostaneme rovnosť $(b-a)x+a-b=0$, čiže $(b-a)(x-1)=0$. Odtiaľ vyplýva, že $b=a$ alebo $x=1$.

Ak $b=a$, majú obidve rovnice tvar $x^2-ax-a=0$. Práve jedno riešenie existuje práve vtedy, keď diskriminant $a^2+4a$ je nulový. To platí pre $a=0$ a~pre $a=\m4$. Pretože $b=a$, má súčet $a+b$ v~prvom prípade hodnotu~$0$ a~v~druhom prípade hodnotu $\m8$.

Ak $x=1$, dostaneme z~daných rovníc $a+b=1$, teda $b=1-a$. Rovnice potom majú tvar
$$
x^2-ax+a-1=0\quad\text{a}\quad x^2+(a-1)x-a=0.
$$
Prvá má korene $1$ a~$a-1$, druhá má korene $1$ a~$\m a$. Práve jedno spoločné riešenie tak dostaneme vždy s~výnimkou prípadu, keď $a-1=\m a$, čiže $a=\frac12$ -- vtedy sú spoločné riešenia dve.

\zaver
Najmenšia hodnota súčtu $a+b$ je $\m8$ a je dosiahnutá pre $a=b=\m4$. Najväčšia hodnota súčtu $a+b$ je $1$; túto hodnotu má súčet $a+b$ pre všetky dvojice $(a,1-a)$, kde $a\ne\frac12$ je ľubovoľné reálne číslo.

\nobreak\medskip\petit\noindent
Za úplné riešenie dajte 6~bodov.
Jeden bod dajte za odvodenie podmienky $(b=a) \vee (x=1)$, dva body za vyriešenie prípadu $b=a$, dva body za vyriešenie prípadu $x=1$, jeden bod za správny záver.
\endpetit
\bigbreak}

{%%%%%   B-II-2
Z~dvojakého vyjadrenia obsahu trojuholníka $ABC$ dostaneme rovnosť $av_a=bv_b$. Dosadením $b=a+2v_a-2v_b$ do tejto rovnosti dostaneme $av_a=av_b+2v_av_b-2v_b^2$ a~po úprave $(a-2v_b)(v_a-v_b)=0$.

Sú dve možnosti: Ak $a=2v_b$, tak $\sin\gamma =\frac{v_b}a=\frac12$, a~teda $\gamma=30^\circ$ alebo $\gamma=150^\circ$; potom $\beta=180^\circ-\alpha-\gamma$. Ak $v_a=v_b$, tak $a=b$, a~teda $\beta=\alpha=20^\circ$.

Úloha má tri riešenia: $\beta=130^\circ$ a~$\gamma=30^\circ$, $\beta=10^\circ$ a~$\gamma=150^\circ$ alebo $\beta=20^\circ$ a~$\gamma=140^\circ$.

\ineriesenie
Z~vyjadrenia výšok pomocou uhla~$\gamma$, \tj.~$v_a=b\sin\gamma$ a~$v_b=a\sin\gamma$,
dostaneme dosadením do zadaného vzťahu rovnosť $a+2b\sin\gamma=b+2a\sin\gamma$, ktorá platí práve vtedy, keď
$(a-b)(1-2\sin\gamma)=0$.

Ak $a=b$, vychádza $\beta=\alpha=20^\circ$, takže $\gamma=140\st$. Inak musí byť $\sin \gamma =\frac12$, takže $\gamma=30^\circ$ alebo $\gamma=150^\circ$; uhol $\beta$ v~oboch prípadoch dopočítame ako $\beta=180^\circ-\alpha-\gamma$.

Dostaneme tak rovnakú trojicu riešení ako pri prvom postupe.

\nobreak\medskip\petit\noindent
Za úplné riešenie dajte 6~bodov. Pri prvom uvedenom postupe dajte jeden bod za použitie rovnosti $av_a=bv_b$, ďalší bod za podmienku $(a-2v_b)(v_a-v_b)=0$. Za úplné vyriešenie prípadu $a=2v_b$ tri body a~za vyriešenie prípadu $v_a=v_b$ jeden bod. Pri druhom postupe dajte jeden bod za vyjadrenie výšok pomocou uhla~$\gamma$, ďalšie dva body za podmienku $(a-b)(1-2\sin\gamma)=0$, dva body za vyriešenie prípadu $1-2\sin\gamma=0$ a~jeden bod za vyriešenie prípadu $a=b$.
\endpetit
\bigbreak}

{%%%%%   B-II-3
Podľa Tálesovej vety je uhol $AMD$ pravý. Preto je aj uhol $DMC$ pravý (\obr). Strany $BC$ a~$AD$ sú rovnobežné, preto je uhlopriečka~$BD$ kolmá aj na stranu~$BC$. Body $M$ a~$B$ teda ležia na Tálesovej kružnici s~priemerom~$CD$. Od stredu úsečky~$CD$ majú potom rovnakú vzdialenosť, a~preto tento stred leží na osi úsečky~$MB$.
\insp{b57.8}

\nobreak\medskip\petit\noindent
Za úplné riešenie dajte 6~bodov. Dva body dajte za zistenie, že uhly $DMC$ a~$DBC$ sú pravé, dva body za poznatok, že body $M$ a~$B$ ležia na Tálesovej kružnici a~dva body za z~toho vyplývajúci záver.
\endpetit
\bigbreak}

{%%%%%   B-II-4
a) Turnaj sa skladá z~deviatich kôl. Ak má každé družstvo iný počet bodov, musí mať prvý v~tabuľke aspoň o~9~bodov viac ako posledný. Na zisk deviatich bodov je potrebné odohrať aspoň päť stretnutí; to znamená, že už muselo prebehnúť aspoň 5 kôl, takže do konca turnaja ostávajú nanajvýš štyri kolá. V~nich môže posledný v~tabuľke získať maximálne 8~bodov a~prvého už nemôže dostihnúť.

\smallskip
b) V~turnaji prebehne 11~kôl (každé družstvo desaťkrát hrá a~raz má voľno). Ak má každé družstvo iný počet bodov, muselo už byť udelených aspoň $0+1+2+\dots+10=55$ bodov. V~jednom kole sa odohrá 5~stretnutí, takže sa rozdelí $5\cdot2=10$~bodov. Preto už muselo byť odohraných aspoň 6~kôl a~do konca ich ostáva nanajvýš~5.

Keby bol medzi niektorými susedmi v~tabuľke väčší rozdiel ako jednobodový, mal by prvý aspoň o~11~bodov viac ako posledný a~v~zostávajúcich nanajvýš piatich kolách by ním nemohol byť dostihnutý. Pripusťme teda, že rozdiely medzi susedmi v~tabuľke sú iba jednobodové. Ak má posledný $b$~bodov (zrejme $0\le b<11$), je celkový počet udelených bodov $b+(b+1)+(b+2)+\dots+(b+10)=11b+55$. Na to je potrebné odohrať
$$
k=\frac{11b+55}{10}=b+5+\frac{b+5}{10}
$$
kôl. Počet odohraných kôl je celé číslo, preto $10\mid b+5$. Odtiaľ vyplýva $b=5$, a~teda $k=11$. To znamená, že sú odohrané všetky kolá a~posledné miesto v~tabuľke je definitívne.

\medskip
{\bf Iné riešenie časti b).}
Rovnako ako v~prvom riešení dokážeme, že už muselo prebehnúť aspoň 6~kôl. Medzi prvým a~posledným v~tabuľke je aspoň desaťbodový rozdiel. Keby prebehlo aspoň 7~kôl, ostávali by do konca najviac 4~kolá a~v~nich by nemohol posledný najmenej desaťbodový náskok prvého vyrovnať. Predpokladajme teda, že prebehlo presne 6~kôl, v~ktorých bolo rozdelených $6\cdot10=60$~bodov. Keby mal posledný v~tabuľke aspoň jeden bod, bol by celkový počet udelených bodov aspoň $1+2+3+\dots+11=66>60$. Posledný teda musel byť bez bodu. Potom ale prvý musel mať viac ako 10~bodov, pretože v~opačnom prípade by bol bodový zisk všetkých družstiev $0+1+2+\dots+10=55<60$~bodov. Prvý teda mal pred posledným aspoň jedenásťbodový náskok a~ten nemôže byť v~zostávajúcich piatich kolách vyrovnaný.

\nobreak\medskip\petit\noindent
Za úplné riešenie dajte 6~bodov.
Za vyriešenie časti~a) dajte 2~body, z~toho jeden bod za zistenie, že už muselo prebehnúť aspoň 5~kôl. V~časti~b) jeden bod za dôkaz, že už prebehlo aspoň 6~kôl, jeden bod za vyriešenie úlohy za predpokladu, že medzi niektorými susedmi v~tabuľke je väčší rozdiel ako jeden bod, dva body za vyriešenie úlohy v~prípade jednobodových rozdielov medzi susedmi.
Podobne dajte (pri druhom postupe) jeden bod za dôkaz, že už prebehlo aspoň 6~kôl; jeden bod za dôkaz, že posledný nemôže dostihnúť prvého, ak sa už odohralo viac ako 6~kôl; 2~body za dôkaz, že sa to nemôže stať ani vtedy, ak sa už odohralo presne 6~kôl.
\endpetit
\bigbreak}

{%%%%%   C-S-1
Z~podmienky pre súčin vyplýva, že $a$ aj $b$ sú mocninami toho istého
prvočísla~$p$: $a = p^r$, $b=p^s$, pričom $r$, $s$ sú celé kladné čísla.
Keby bolo $p$ nepárne, bol by súčet $a+b$ deliteľný okrem čísla~$p$ aj číslom~$2$,
takže by nebol mocninou prvočísla. Ak $p=2$ a~$r<s$, je súčet $a+b=2^r(1+2^{s-r})$
opäť číslo párne deliteľné nepárnym číslom väčším ako~$1$,
nie je teda mocninou prvočísla. K~rovnakému záveru dôjdeme aj v~prípade,
keď $r>s$. Ostáva preto jediná možnosť: $a=b=2^r$, pričom $r$ je celé kladné číslo.
Skúška $a + b = 2^r + 2^r = 2^{r +1}$
a~$ab = 2^{2r}$ potvrdzuje, že riešením sú všetky dvojice $(a,b)=(2^r,2^r)$,
kde $r$ je celé kladné číslo.

\nobreak\medskip\petit\noindent
Za úplné riešenie dajte 6~bodov.
Za zistenie, že $a$, $b$ sú mocniny jedného prvočísla, dajte 1~bod, ďalší bod za
zdôvodnenie, že $p = 2$. Ďalej potom 2~body za dôkaz $r = s$
(vzhľadom na symetriu možno priamo predpokladať $r\le s$) a~1~bod za skúšku.

\endpetit
\bigbreak}

{%%%%%   C-S-2
\fontplace
\rtpoint A; \ltpoint B; \bpoint C; \bpoint D;
\lpoint E; \bpoint\xy1.7,0.4 F;
\bpoint a\mathbin{\smash-}x; \bpoint x; \lpoint v;
\cpoint15; \cpoint14;
[5] \hfil\Obr

\fontplace
\tpoint A; \tpoint B; \bpoint C; \bpoint D;
\lpoint E; \tpoint F;
\cpoint S_1; \cpoint S_2;
\cpoint S_2; \cpoint S_1+2S_2;
[6] \hfil\Obr

\fontplace
\tpoint A; \tpoint B; \bpoint C; \bpoint D;
\lpoint E; \tpoint F; \bpoint G; \rpoint H;
\cpoint S_1; \cpoint S_2;
\cpoint S_2; \cpoint S_2; \cpoint S_2; \cpoint\up\unit S_1;
[7] \hfil\Obr

%\def\S#1 {S_{#1}}
Označme $v$ vzdialenosť bodu~$C$ od priamky~$AB$, $a = |AB|$ a~$x =|AF|$.
Pre obsahy trojuholníkov $AFD$ a~$FBE$ (\obr) platí
$\frac12x \cdot v~=15$, $\frac12(a - x)\cdot \frac12v =14$.
Odtiaľ $xv = 30$, $av - xv = 56$.
Sčítaním oboch rovností nájdeme obsah rovnobežníka $ABCD$: $S_{ABCD}
= av =86\cm^2$. Obsah štvoruholníka $FECD$ je teda
$S_{FECD}=S_{ABCD}-(S_{AFD} +S_{FBE})= 57\cm^2$.

\twocpictures

\ineriesenie
Trojuholníky $BEF$ a~$ECF$ majú spoločnú výšku z~vrcholu~$F$ a~zhodné základne
$BE$ a~$EC$. Preto sú obsahy oboch trojuholníkov rovnaké.
Z~\obr{} vidíme, že obsah trojuholníka $CDF$ je polovicou obsahu
rovnobežníka $ABCD$ (oba útvary majú spoločnú základňu~$CD$
a~rovnakú výšku). Druhú polovicu tvorí súčet obsahov trojuholníkov
$AFD$ a~$BCF$. Odtiaľ $S_{FECD} =S_{ECF} +S_{CDF} =S_{ECF} +(S_{AFD} +S_{BCF})=
S_{AFD} +3S_{FBE} =57\cm^2$.

\ineriesenie
Do rovnobežníka dokreslíme úsečky $FG$ a~$EH$ rovnobežné
so stranami $BC$ a~$AB$ tak, ako znázorňuje \obr.
\inspicture{}
Rovnobežníky $AFGD$ a~$FBEH$ sú svojimi
uhlopriečkami $DF$ a~$EF$ rozdelené na dvojice zhodných trojuholníkov.
Takže $S_{GDF} = S_{AFD} = 15\cm^2$ a~$S_{HFE} =S_{BEF} =14\cm^2$. Zo
zhodnosti rovnobežníkov $HECG$ a~$FBEH$ navyše ľahko nahliadneme, že
všetky štyri trojuholníky $FBE$, $EHF$, $HEC$ a~$CGH$ sú zhodné,
takže obsah štvoruholníka $FECD$ je $S_{AFD} +3S_{FBE} =57\cm^2$.

\nobreak\medskip\petit\noindent
Za úplné riešenie dajte 6~bodov.
Za každú číselnú chybu alebo nepodstatnú chybu pri zdôvodňovaní
výpočtu niektorého obsahu strhnite bod, za hrubú chybu pri
zdôvodnení strhnite dva body. Ak je úloha nedoriešená, dajte dva
body za každú správne vypočítanú časť obsahu štvoruholníka $FECD$.

\endpetit
\bigbreak}

{%%%%%   C-S-3
\fontplace
\bpoint D; \rpoint E; \tpoint A; \tpoint B; \lpoint C;
[8] \hfil\Obr

\fontplace
\rpoint A; \blpoint X; \lpoint F;
[9] \hfil\Obr

\fontplace [10] \hfil\Obr\par
\fontplace [11] \hfill\rlap{\quad\Obr}\par
\fontplace [12]

Jednotlivé osoby označíme písmenami $A$, $B$, $C$, $D$, $E $ a~$F$.
Aspoň jedna z~nich (označme ju~$A$) má aspoň štyroch známych (ak by mala každá
osoba najviac troch známych, bolo by dvojíc známych menej ako desať). Keby
mala dokonca päť známych, dozvie sa správu od každého v~skupine
a môže ju komukoľvek v~skupine povedať.
\twocpictures

Ak má osoba~$A$ práve štyroch známych,
napríklad osoby $B$, $C$, $D$ a~$E$, existuje v~skupine osôb
$A$, $B$, $C$, $D$, $E$ najviac 10~známostí (\obr, dvojice známych znázorňujú
úsečky), a~tak sa osoba~$F$ musí poznať s~niektorou osobou $X \in\{ B,C,D,E\}$.
Možnosť šírenia správy od ľubovoľnej osoby ku ktorejkoľvek inej ľahko
overíme podľa \obr.

\ineriesenie
Znázornenie ktorejkoľvek množiny práve jedenástich dvojíc známych
v~skupine šiestich osôb dostaneme odstránením štyroch z~pätnástich
hrán úplného grafu (\obr, v~ňom z~každého uzla vychádza práve päť hrán).
Po odstránení iba štyroch hrán z~grafu na \obrr1{} musí teda z~každého
vrcholu vychádzať aspoň jedna hrana. V~skupine teda neexistuje
človek, ktorý by nikoho nepoznal. Aby sa preto správa nemohla od
niektorej z~osôb rozšíriť ku všetkým ostatným, musela by v~príslušnom grafe
existovať buď aspoň jedna oddelená dvojica, alebo dve oddelené trojice,
v~ktorých sa osoby môžu poznať navzájom.
V~žiadnej z~týchto situácií však počet dvojíc známych
neprevyšuje sedem, ako vidíme z~\obr. Tým je tvrdenie úlohy dokázané.
\midinsert
\centerline{\inspicture-!\hss\inspicture-!\hss\inspicture-!}
\endinsert

\nobreak\medskip\petit\noindent
Za úplné riešenie dajte 6~bodov,
z~toho 2~body za dôkaz tvrdenia, že niektorá osoba má
aspoň štyroch známych, a~1~bod za vysvetlenie, že neexistuje osoba,
ktorá by nemala známeho.
\endpetit
\bigbreak}

{%%%%%   C-II-1
\fontplace
\tpoint A; \tpoint B; \bpoint C;
\tpoint x; \rBpoint x; \tpoint y; \lBpoint y; \lBpoint z; \rBpoint z;
\tpoint K; \lBpoint L; \rBpoint M;
[13] \hfil\Obr\strut

Označme $x=|AK|=|AM|$, $y=|BL|=|BK|$, $z=|CM|=|CL|$ (\obr)
\inspicture{}
zhodné úseky dotyčníc
z~jednotlivých vrcholov trojuholníka ku vpísanej kružnici. Zrejme
$$
  a = y+z,\quad b = z+x,\quad c = x+y. \tag1
$$
Z~uvedených rovností vidíme, že daná podmienka
$$
  b+c < 3a \tag2
$$
je ekvivalentná nerovnosti
$$
  x < y+z, \tag3
$$
čo je nutná podmienka existencie trojuholníka
so stranami dĺžok $x$, $y$ a~$z$.

Dosadením z~\thetag1 do podmienok $b\le c$ a~$a\le b$ zistíme, že $z\le y$
a~$y\le x$. To znamená, že ďalšie dve trojuholníkové nerovnosti $y<z+x$ a~$z<x+y$
sú automaticky splnené, takže nerovnosť~\thetag3, a~tým aj \thetag2 je
podmienkou postačujúcou. Tým je tvrdenie úlohy dokázané.

\nobreak\medskip\petit\noindent
Za úplné riešenie dajte 6~bodov, z~toho 1~bod za nájdenie rovností \thetag1, 2~body za
zistenie ekvivalencie vzťahov \thetag2 a~\thetag3 a~3~body za dôkaz,
že podmienka \thetag3 je nielen nutná, ale
aj postačujúca.
\endpetit
\bigbreak}

{%%%%%   C-II-2
Označme $x$ menšie a~$y$ väčšie z~násobených čísel. Podľa zadania platí
$xy-400=67x+56$,
čiže
$$
x(y-67) = 456.  \tag1
$$
Číslo~$x$ je teda dvojciferný deliteľ čísla $456=2^3\cdot3\cdot19$.
Zo zadania navyše vyplýva, že číslo~$x$ je väčšie ako príslušný zvyšok~$56$.
Najmenšie také $x$ je $x=3\cdot19=57$. Pre každý ďalší taký deliteľ
platí $x\ge4\cdot19=76$ a~$y-67\le2\cdot3=6$, takže $y\le73<x$,
čo odporuje zvolenému označeniu $x<y$. Teda $x=57$ a~$y=75$.
Ľahko overíme, že tieto čísla vyhovujú zadaniu úlohy.

\zaver
Klárka násobila čísla $57$ a~$75$.


% \medskip
% {\bf Jiné řešení.}
% Vyjdeme ze vztahu (4) přepsaného na tvar
% $y= 456/x+ 67$ a~dosazujeme za $x$ ty dělitele čísla~456, kteří jsou větší než~56.


\nobreak\medskip\petit\noindent
Za úplné riešenie dajte 6~bodov. Také je samozrejme aj riešenie, v~ktorom žiak preverí
všetkých 16~deliteľov čísla~$456$ v~\thetag1.
Za nájdenie rovnice \thetag1 či iného ekvivalentného vzťahu dajte 3~body,
2~body za analýzu ich riešení a~1~bod za skúšku, \tj. overenie,
že dvojica $(x,y)=(57,75)$ spĺňa všetky
podmienky úlohy. Iba za uhádnutie hľadaných čísel (bez zostavovania rovnice)
dajte celkom 2~body.
\endpetit
\bigbreak}

{%%%%%   C-II-3
\fontplace
\tpoint A; \tpoint B; \bpoint C; \bpoint D;
\tpoint\xy-1,0 E; \tpoint F;
[14] \hfil\Obr\strut

Nazvime $A$ osobu (prípadne jednu z~osôb), ktorá má v~danej skupine
najviac známych, a~tento počet známych označme~$n$. Zrejme $n\le5$.

Ak $n=5$, existuje medzi zostávajúcimi osobami aspoň päť ďalších dvojíc známych.
Ktorákoľvek z~týchto dvojíc potom tvorí s~osobou~$A$ trojicu známych.

Ak $n=4$, existuje osoba~$B$, ktorá sa s~$A$ nepozná, a~tá má tiež
najviac štyroch známych. Preto
sa medzi známymi osoby~$A$ vyskytujú aspoň dve dvojice známych.
Osoba $A$ s~jednou z~týchto dvojíc tvorí opäť trojicu známych.

Situácia $n\le3$ nemôže nastať, pretože celkový počet dvojíc známych v~skupine
by vtedy bol nanajvýš $\frac12 \cdot6n\le9$.
% (Počet dvojic spočteme jako
% . V~takovém případě má totiž každá z šesti osob
% nejvýše tři známé, což je celkem nejvýše~dvojic (každou dvojici
% jsme totiž započetli dvakrát).

\inspicture{}
Príklad skupiny šiestich osôb s~deviatimi dvojicami, ale
s~žiadnou trojicou známych, je znázornený grafom na \obr. V~ňom
body $A$, $B$, $C$, $D$, $E$ a~$F$ predstavujú jednotlivé osoby
a~dvojice známych sú vyznačené úsečkami. Pritom žiadne tri z~úsečiek
netvoria trojuholník. Pokiaľ je v~skupine menej ako deväť dvojíc známych,
%  ($x \in \{1,2,\dots,9\}$),
zostrojíme vhodný príklad odstránením príslušného počtu
% libovolných $x$~
úsečiek z~\obrr1{} (pritom určite žiadny trojuholník nevznikne).
% \inspicture{}

\ineriesenie
Ak je v~šestici osôb aspoň 10 dvojíc známych, je v~nej najviac
5~dvojíc neznámych, lebo všetkých dvojíc je práve~15. Budeme
preto naopak predpokladať, že v~každej trojici sa nájde dvojica
neznámych, a~dokážeme, že v~celej šestici je takých dvojíc
aspoň~6. Pri uvedenom predpoklade môžeme označenie osôb zvoliť tak,
aby v~trojiciach $ABC$ a~$DEF$ boli dvojice neznámych $AB$ a~$DE$. Potom
ďalšie štyri rôzne dvojice neznámych nájdeme
(po jednej) v~trojiciach $ACD$, $AEF$, $BCE$, $BDF$ (presvedčte sa,
že každá dvojice sa vyskytuje najviac v~jednej z~uvedených štyroch trojíc
a~žiadna z~týchto trojíc neobsahuje ani dvojicu~$AB$, ani dvojicu~$DE$).

Príklad pre menší počet dvojíc známych zostrojíme rovnako ako v~predchádzajúcom
riešení.

\nobreak\medskip\petit\noindent
Za úplné riešenie dajte 6~bodov.
Za dôkaz existencie trojice známych pre aspoň $10$~dvojíc známych dajte 3~body.
Ďalšie 3~body dajte, ak riešiteľ nájde pre 9~dvojíc známych príklad,
v~ktorom nie je trojica známych,
a~ukáže, ako postupovať pre menší počet dvojíc známych.
Ak uvedie iba príklad pre 9~dvojíc známych, strhnite 1~bod.
Ak uvedie iba príklady pre niektoré počty dvojíc menšie ako~9,
strhnite 2~body.
\endpetit
\bigbreak
}

{%%%%%   C-II-4
Rovnicu prepíšeme na tvar
$$
  x-y = (z-y)\sqrt3 + (x-z)\sqrt7
$$
a~umocníme. Po jednoduchej úprave dostaneme
$$
  (x-y)^2 - 3(z-y)^2 - 7(x-z)^2 = 2(x-z)(z-y)\sqrt{21}. \tag1
$$
Pre $x\ne z$ a~$y\ne z$ nemôže rovnosť~\thetag1 platiť, pretože jej pravá strana
je v~takom prípade
číslo iracionálne, zatiaľ čo ľavá strana je číslo celé. Rovnosť teda môže nastať,
len keď $x=z$ alebo $y=z$.

V~prvom prípade po dosadení $x=z$ do pôvodnej rovnice dostaneme
$z-y=\sqrt3(z-y)$. Odtiaľ $z=y=x$.

V~druhom prípade, keď $y=z$, dôjdeme analogicky k~rovnakému výsledku.

\zaver
Riešením danej rovnice sú všetky trojice $(x,y,z)=(k,k,k)$, kde $k$ je ľubovoľné
celé číslo.

\nobreak\medskip\petit\noindent
Za úplné riešenie dajte 6~bodov,
z~toho 1~bod za uhádnutie
koreňov $x=y=z\in\Bbb Z$
a~5~bodov za správny dôkaz, že iné riešenia rovnica nemá.
\endpetit
\bigbreak}

{%%%%%   vyberko, den 1, priklad 1
...}

{%%%%%   vyberko, den 1, priklad 2
...}

{%%%%%   vyberko, den 1, priklad 3
...}

{%%%%%   vyberko, den 1, priklad 4
...}

{%%%%%   vyberko, den 2, priklad 1
...}

{%%%%%   vyberko, den 2, priklad 2
...}

{%%%%%   vyberko, den 2, priklad 3
...}

{%%%%%   vyberko, den 2, priklad 4
...}

{%%%%%   vyberko, den 3, priklad 1
...}

{%%%%%   vyberko, den 3, priklad 2
...}

{%%%%%   vyberko, den 3, priklad 3
...}

{%%%%%   vyberko, den 3, priklad 4
...}

{%%%%%   vyberko, den 4, priklad 1
...}

{%%%%%   vyberko, den 4, priklad 2
...}

{%%%%%   vyberko, den 4, priklad 3
...}

{%%%%%   vyberko, den 4, priklad 4
...}

{%%%%%   vyberko, den 5, priklad 1
...}

{%%%%%   vyberko, den 5, priklad 2
...}

{%%%%%   vyberko, den 5, priklad 3
...}

{%%%%%   vyberko, den 5, priklad 4
...}

{%%%%%   trojstretnutie, priklad 1
Predpokladajme, že trojica $(x,y,z)$ kladných reálnych čísel je riešením zadanej sústavy. Rozoberieme tri prípady podľa toho, ktoré z~čísel $x$, $y$, $z$ je najmenšie. Ukáže sa, že v~každom z~týchto prípadov stačí uvažovať len jednu rovnicu sústavy. Viackrát použijeme známu nerovnosť medzi aritmetickým a~geometrickým priemerom $n$-tice kladných reálnych čísel\dorocenky{\footnote{Poz. poznámku pod čiarou na str.~\strankaAG.}} (v~našom prípade bude $n\in\{2,3,4\}$), v~ktorej rovnosť platí práve vtedy, keď je všetkých $n$~čísel rovnakých.

\prip1
Ak $x\ge y$, $z\ge y$, tak zrejme platia nerovnosti
$$
2x^3+(z^2+1) \ge 2yx^2+(z^2+1) \ge 2yx^2+2z \ge 2yx^2+2y=2y(x^2+1),
$$
teda $2x^3\ge 2y(x^2+1)-(z^2+1)$. Pritom rovnosť nastáva len v~prípade, keď $2x^3=2yx^2$, $z^2+1=2z$ a~$2z=2y$, čiže $x=y$, $z=1$ a~$z=y$. Týmto podmienkam, a~teda aj prvej rovnici sústavy, vyhovuje jedine trojica $x=y=z=1$. Ľahko overíme, že táto trojica spĺňa aj zvyšné dve rovnice.

\prip2
Ak $x\ge z$, $y\ge z$, dostávame
$$
\aligned
2y^4+2(x^2+1) &\ge 2y^4+2\cdot2x = (y^4+y^4+x)+3x \ge\\
              &\ge (y^4+y^2z^2+z)+3z \ge 3\root3\of{y^6z^3}+3z = 3z(y^2+1),
\endaligned
$$
teda $2y^4\ge 3z(y^2+1)-2(x^2+1)$. Rovnosť nastáva jedine v~prípade, keď sú splnené podmienky $x=1$, $y^4=y^2z^2$, $x=z$ a~$y^4=y^2z^2=z$. Tomu, čiže aj druhej rovnici sústavy, vyhovuje jedine trojica $x=y=z=1$.

\prip3
Ak $y\ge x$, $z\ge x$, podobne ako v~predošlom prípade máme
$$
\aligned
2z^5+3(y^2+1) &\ge 2z^5+3\cdot2y = (z^5+z^5+y+y)+4y \ge\\
              &\ge (z^5+z^3x^2+x+x)+4x \ge 4\root4\of{z^8x^4}+4x = 4x(z^2+1),
\endaligned
$$
teda $2z^5\ge 4x(z^2+1)-3(y^2+1)$. Rovnosť dostaneme iba pri dodržaní podmienok $y=1$, $z^5=z^3x^2$, $y=x$ a~$z^5=z^3x^2=x$. Tretia rovnica sústavy je teda splnená jedine pre trojicu $x=y=z=1$.

\odpoved
Jediným riešením sústavy je trojica $(1,1,1)$.
}

{%%%%%   trojstretnutie, priklad 2
Označme v~trojuholníku $ACE$ veľkosti vnútorných uhlov pri vrcholoch $A$, $C$, $E$ postupne $\alpha$, $\gamma$, $\varepsilon$. Trojuholníky $ACB$, $CED$, $EAF$ sú podľa zadania rovnoramenné. Označme veľkosti ich vnútorných uhlov pri základniach postupne $\beta$, $\delta$, $\varphi$ (\obr). Tvrdenie dokážeme použitím C\`evovej vety. Kvôli tomu označme ešte $P$, $Q$, $R$ priesečníky priamok $AD$, $CF$, $EB$ postupne so stranami $CE$, $EA$, $AC$ trojuholníka $ACE$.
\insp{cps.1}

Zo sínusovej vety v~trojuholníku $ABR$ máme
$$
\frac{|AR|}{\sin|\angle ABE|}=\frac{|BR|}{\sin\beta},\qquad\text{teda}\qquad
|AR|=\frac{|BR|\cdot\sin|\angle ABE|}{\sin\beta}.
\tag1
$$
Zo sínusovej vety v~trojuholníku $ABE$ máme
$$
\frac{|AE|}{\sin|\angle ABE|}=\frac{|BE|}{\sin(\alpha+\beta)},\qquad\text{teda}\qquad
\sin|\angle ABE|=\frac{|AE|\cdot\sin(\alpha+\beta)}{|BE|}.
$$
Dosadením do \thetag1 dostávame
$$
|AR|=\frac{|BR|\cdot|AE|\cdot\sin(\alpha+\beta)}{|BE|\cdot\sin\beta}.
$$
Zrejme analogicky (zo sínusových viet v~trojuholníkoch $CBR$ a~$CBE$) možno odvodiť
$$
|CR|=\frac{|BR|\cdot|CE|\cdot\sin(\gamma+\beta)}{|BE|\cdot\sin\beta}.
$$
Preto
$$
\frac{|AR|}{|CR|}=\frac{|AE|\cdot\sin(\alpha+\beta)}{|CE|\cdot\sin(\gamma+\beta)}.
$$
Opäť analogicky možno vyjadriť pomery
$$
\frac{|CP|}{|EP|}=\frac{|CA|\cdot\sin(\gamma+\delta)}{|EA|\cdot\sin(\varepsilon+\delta)}
\qquad\text{a}\qquad
\frac{|EQ|}{|AQ|}=\frac{|EC|\cdot\sin(\varepsilon+\varphi)}{|AC|\cdot\sin(\alpha+\varphi)}.
$$
Odtiaľ
$$
\frac{|AR|}{|CR|}\cdot\frac{|CP|}{|EP|}\cdot\frac{|EQ|}{|AQ|}=
\frac{\sin(\alpha+\beta)}{\sin(\gamma+\beta)}\cdot
\frac{\sin(\gamma+\delta)}{\sin(\varepsilon+\delta)}\cdot
\frac{\sin(\varepsilon+\varphi)}{\sin(\alpha+\varphi)}.
\tag2
$$
Avšak podľa zadania platí $\varphi+\alpha+\beta=\beta+\gamma+\delta=\delta+\varepsilon+\varphi$. Preto
$$
\alpha+\beta=\varepsilon+\delta,\quad \gamma+\delta=\alpha+\varphi,\quad \varepsilon+\varphi=\gamma+\beta
$$
a~súčin \thetag2 je rovný~$1$. Podľa C\`evovej vety sa teda priamky $AD$, $BE$ a~$CF$ pretínajú v~jednom bode.

\ineriesenie
Označme $P$ priesečník osí vnútorných uhlov daného šesťuholníka pri vrcholoch $B$ a~$D$ (\obr). Dokážeme, že šesťuholníku $ABCDEF$ sa dá vpísať kružnica, ktorej stredom je $P$. Zadané tvrdenie bude potom vyplývať z~Brianchonovej vety\footnote{Uvedená veta hovorí, že ak sa strany šesťuholníka $ABCDEF$ dotýkajú jednej kužeľosečky, tak priamky $AD$, $BE$ a~$CF$ sa pretínajú v~jednom bode.}.
\insp{cps.2}

Z~rovnosti $|AB|=|BC|$ vyplýva, že trojuholníky $ABP$ a~$CBP$ sú zhodné podľa vety {\it sus}. Preto $|\angle BAP|=|\angle BCP|=\alpha$. Rovnako sú zhodné trojuholníky $CDP$ a~$EDP$, \tj. $|\angle DCP|=|\angle DEP|=\beta$.

Z~uvedených zhodností navyše máme $|AP|=|CP|=|EP|$, odkiaľ spolu so zadanou rovnosťou $|AF|=|EF|$ dostávame podľa vety {\it sss\/} zhodnosť trojuholníkov $AFP$ a~$EFP$. Preto os vnútorného uhla pri vrchole~$F$ prechádza cez bod~$P$ a~$|\angle FAP|=|\angle FEP|=\gamma$.

Rovnosti $|\uhol FAB|=|\uhol BCD|=|\uhol DEF|$ sú ekvivalentné s~rovnosťami $\gamma+\alpha=\alpha+\beta=\beta+\gamma$, z~ktorých triviálne vyplýva $\alpha=\beta=\gamma$. Preto aj osi vnútorných uhlov pri vrcholoch $A$, $C$ a~$E$ prechádzajú cez bod~$P$ a~šesťuholníku $ABCDEF$ sa dá vpísať kružnica so stredom $P$.
}

{%%%%%   trojstretnutie, priklad 3
Nech $M=\{1,2,\dots,p-1\}$ je množina všetkých nenulových zvyškov po delení~$p$. Pre každé $k\in M$ je kombinačné číslo
$$
\binom pk=\frac{p!}{k!(p-k)!}
$$
deliteľné prvočíslom~$p$, lebo všetky činitele súčinu $k!(p-k)!$ v~menovateli sú menšie ako~$p$ (a~teda nesúdeliteľné s~$p$), zatiaľ čo čitateľ $p!$ zrejme prvočíslom~$p$ deliteľný je. Každý zo sčítancov súčtu v~zadaní je teda deliteľný číslom~$p^2$ a~našou úlohou je zistiť, pre ktoré prvočísla~$p$ je súčet
$$
S=\frac1{p^2}\binom p1^2+\frac1{p^2}\binom p2^2+\cdots+\frac1{p^2}\binom p{p-1}^2
\tag1
$$
deliteľný~$p$.

Pre každé $k\in M$ skúmajme, aký dáva prirodzené číslo
$$
a_k=\frac1{p^2}\binom pk^2=\left(\frac{(p-1)!}{k!(p-k)!}\right)^2
\tag2
$$
zvyšok po delení~$p$. Keďže pre každé $i=k,k+1,p-1$ máme $p-i\equiv\m i\pmod p$, tak
$$
\aligned
(p-k)!&=(p-k)\left(p-(k+1)\right)\dots\left(p-(p-1)\right)\equiv\\
      &\equiv(-1)^{p-k}k(k+1)\dots(p-1)\pmod p
\endaligned
$$
Z~toho úpravou vzťahu~\thetag2 dostávame
$$
\aligned
((p-1)!)^2&=a_k(k!(p-k)!)^2\equiv a_k(k!(-1)^{p-k}k(k+1)\dots(p-1))^2=\\
          &=a_k\cdot k^2((p-1)!)^2\pmod p.
\endaligned
$$
Túto kongruenciu môžeme vydeliť výrazom $((p-1)!)^2$, ktorý je nesúdeliteľný s~$p$. Teda
$$
1\equiv a_k\cdot k^2\pmod p.
\tag3
$$
Ako vieme, ku každému zvyšku $k\in M$ existuje práve jeden zvyšok $z_k\in M$ taký, že $z_k\cdot k\equiv1\pmod p$; ak navyše $k,l\in M$ sú rôzne, tak aj $z_k$, $z_l$ sú rôzne\footnote{Existencia zvyšku $z_k$ vyplýva z~existencie celých čísel $a$, $b$ takých, že $ak+bp=1$. Jednoznačnosť je zrejmá: ak $1\equiv z_k\cdot k\equiv z_k'\cdot k\pmod p$, tak vydelením $k$ máme $z_k\equiv z_k'\pmod p$. Rôznosť triviálne vyplýva z~jednoznačnosti a~z~vlastnosti $z_{z_k}=k$: ak $z_k=z_l$, tak $k=z_{z_k}=z_{z_l}=l$.}. Teda množina $M'=\{z_1,z_2,\dots,z_{p-1}\}$ má rovnako veľa prvkov ako množina $M$, a~keďže $M'\subset M$, nutne $M'=M$.

Z~definície prvku~$z_k$ dostávame $1=1^2\equiv(z_k\cdot k)^2=z_k^2\cdot k^2\pmod p$. Spolu s~\thetag3 potom $a_k\cdot k^2\equiv z_k^2\cdot k^2\pmod p$ a~po vydelení $k^2$ máme $a_k\equiv z_k^2\pmod p$. Pre zvyšok súčtu~\thetag1 teda platí
$$
\aligned
S=a_1+a_2+\cdots+a_{p-1}&\equiv z_1^2+z_2^2+\cdots+z_{p-1}^2=1^2+2^2+\cdots+(p-1)^2=\\
                        &=\tfrac16p(p-1)(2p-1)\pmod p
\endaligned
$$
(využili sme dokázanú množinovú rovnosť $\{z_1,z_2,\dots,z_{p-1}\}=\{1,2,\dots,p-1\}$ a~známy vzorec pre súčet druhých mocnín). Ľahko možno priamym dosadením overiť, že výraz $\frac16p(p-1)(2p-1)$ pre $p=2,3$ nie je násobkom~$p$. Naopak, každé prvočíslo $p\ge5$ je nesúdeliteľné s~číslom~$6$, čiže $p$ je deliteľom čísla $p\cdot\frac16(p-1)(2p-1)$.

\odpoved
Zadaný súčet je deliteľný číslom~$p^3$ pre všetky prvočísla väčšie ako~$4$.
}

{%%%%%   trojstretnutie, priklad 4
Nech $p$ je dané prvočíslo. Skúmajme, aký zvyšok po delení $p$ môže dávať číslo $k^2+k$. Na to stačí za $k$ dosadiť čísla $0$, $1$, \dots, $p-1$, ďalej sa už budú zvyšky periodicky opakovať. Pre $p=2,3,5,7$ dostaneme zvyšky uvedené v~tabuľke.
\input graphicx
\def\clap#1{\hbox to 0pt{\hss#1\hss}}
\def\sikma{\smash{\clap{\kern-2.3mm\rotatebox{-27}{\rule[24mm]{0.95cm}{.4pt}}}}}
$$
\sikma
\vbox{\offinterlineskip \let\\=\cr
\def\ab{\raise\ht\strutbox\hbox{}%\special{{em: point 1}}%
        \,\lower1pt\hbox{$p$}\quad
        \raise.2ex\hbox{$k$}\,\lower\dp\strutbox\hbox{}}%\special{{{em: point 2}}}
\halign{\vrule\strut\hss#\hfil\vrule
        &\enspace\hss$#$\enspace&\hss$#$\enspace&\hss$#$\enspace&\hss$#$\enspace&\hss$#$\enspace&\hss$#$\enspace&\hss$#$\enspace\vrule\\
\noalign{\hrule}
\ab &0 &1 &2 &3 &4 &5 &6\\
\noalign{\hrule}
 2  &0 &0 & & & & &\\
 3  &0 &2 &0 & & & &\\
 5  &0 &2 &1 &2 &0 & &\\
 7  &0 &2 &6 &5 &6 &2 &0\cr
\noalign{\hrule}
}}%\special{em: line 1,2}
$$
Vidíme, že v~postupnosti zvyškov sa niektorý zvyšok neobjaví. Napríklad pre $p=2$ nedáva $k^2+k$ nikdy zvyšok~$1$, pre $p=3$ nedostaneme zvyšok $1$, pre $p=5$ zvyšok $3$ ani $4$, atď. Aby sme to dokázali pre všeobecné~$p$, stačí overiť, že niektorý zvyšok sa v~postupnosti objaví aspoň dvakrát. Počet rôznych zvyškov je totiž $p$ a~dĺžka postupnosti je tiež $p$, teda akonáhle sa v~postupnosti nejaký zvyšok zopakuje, nebude už v~nej dosť miesta pre všetky rôzne zvyšky.

Opakujúcim sa zvyškom je napríklad $0$, platí totiž
$$
0^2+0\equiv0\pmod p \qquad\text{aj}\qquad (p-1)^2+(p-1)=p^2-p\equiv0\pmod p,
$$
čiže zvyšok~$0$ dostaneme pre $k=0$ aj pre $k=p-1$.

Nech $\{p_1,p_2,\dots,p_m\}$ je množina všetkých prvočísel menších ako $2008$. Pre každé $j=1,2,\dots,m$ označme $r_{p_j}$ ľubovoľný zo zvyškov po delení prvočíslom~$p_j$, pre ktorý $k^2+k \not\equiv r_{p_j}\pmod{p_j}$ pre všetky celé čísla~$k$ (už sme dokázali, že taký zvyšok existuje). Aby sme vyhoveli zadaniu, stačí zvoliť $n$, ktoré spĺňa
$$
\aligned
n&\equiv -r_{p_1}\pmod{p_1},\\
n&\equiv -r_{p_2}\pmod{p_2},\\
 &\ \,\vdots\\
n&\equiv -r_{p_m}\pmod{p_m},\\
\endaligned
$$
potom totiž $k^2+k+n\equiv k^2+k-r_{p_j}\not\equiv0\pmod{p_j}$ pre všetky $j=1,2,\dots,m$. Existencia požadovaného $n$ už priamo vyplýva z~čínskej zvyškovej vety\footnote{Podľa nej, ak $q_1,\dots,q_m$ sú navzájom nesúdeliteľné čísla a~$a_1,\dots,a_m$ sú ľubovoľné celé čísla, tak existuje celé číslo~$x$ spĺňajúce
$$
x\equiv a_1\pmod{q_1},\quad x\equiv a_2\pmod{q_2},\quad\dots,\quad x\equiv a_m\pmod {q_m}.
$$
Túto vetu možno jednoducho dokázať priamou konštrukciou $x$: Prvú kongruenciu spĺňa $x=kq_1+a_1$ pre ľubovoľné celé~$k$. Ak za $k$ dosadíme postupne $0,1,\dots,q_2-1$, pre $x$ dostaneme $q_2$ rôznych zvyškov po delení $q_2$, jeden z~nich $k'q_1+a_1$ teda bude rovný $a_2$. Čiže aby sme splnili aj druhú kongruenciu, stačí zvoliť $x=(k'+lq_2)q_1+a_1$ pre ľubovoľné celé~$l$. Za $l$ dosadíme postupne $0,1,\dots,q_3-1$, dostaneme $q_3$ rôznych zvyškov po delení $q_3$, jeden z~nich bude rovný $a_3$, atď.}, keďže prvočísla $p_1,p_2,\dots,p_m$ sú navzájom nesúdeliteľné.
}

{%%%%%   trojstretnutie, priklad 5
Bez ujmy na všeobecnosti predpokladajme, že pravidelný päťuholník $ABCDE$ má dĺžku strany~$1$. Potom každá z~jeho uhlopriečok má dĺžku\footnote{Dĺžku uhlopriečky $u$ pravidelného päťuholníka $ABCDE$ so stranou dĺžky~$1$ možno jednoducho vypočítať napríklad z~podobnosti rovnoramenných trojuholníkov $CAB$ a~$DEX$, kde $X$ je priesečník uhlopriečok $AD$ a~$EC$. Totiž $ABCX$ je kosoštvorec a~teda $|EX|=u-1$, čiže $(u-1):1=1:u$.}
$$
u=\frac{1+\sqrt5}2.
$$

Použitím Ptolemaiovej nerovnosti\footnote{Ak $X$, $Y$, $Z$, $W$ sú ľubovoľné štyri body v~rovine, tak podľa Ptolemaiovej nerovnosti platí $|XY|\cdot|ZW|+|YZ|\cdot|WX|\ge|XZ|\cdot|YW|$, pričom rovnosť podľa Ptolemaiovej vety platí práve vtedy, keď body $X$, $Y$, $Z$, $W$ ležia (v~tomto poradí) na jednej kružnici. Ak $XYZW$ je štvoruholník, tak Ptolemaiova nerovnosť (resp. veta) hovorí, že súčet súčinov dĺžok protiľahlých strán nie je menší ako súčin dĺžok uhlopriečok, pričom rovnosť nastáva práve vtedy, keď štvoruholník $XYZW$ je tetivový.} pre štvoruholníky $APBE$, $APBD$, $APBC$ (\obr), resp. príslušné štvorice bodov, pokiaľ body v~uvedenom poradí netvoria štvoruholníky, dostávame
$$
\aligned
|PA|\cdot u+|PB|\cdot 1 &\ge 1\cdot|PE|,\\
|PA|\cdot u+|PB|\cdot u &\ge 1\cdot|PD|,\\
|PA|\cdot 1+|PB|\cdot u &\ge 1\cdot|PC|.
\endaligned
\tag1
$$
Sčítaním týchto nerovností už získame priamo dolné ohraničenie pre výraz zo zadania:
$$
(|PA|+|PB|)\cdot(2u+1)\ge|PC|+|PD|+|PE|,
$$
odkiaľ
$$
\frac{|PA|+|PB|}{|PC|+|PD|+|PE|}\ge\frac1{2u+1}.
\tag2
$$
\inspinsp{cps.3}{cps.4}%

Pritom rovnosť vo všetkých nerovnostiach v~\thetag1, čiže aj v~\thetag2, platí práve vtedy, keď sú štvoruholníky $APBE$, $APBD$, $APBC$ tetivové (pripúšťa sa aj možnosť $P=A$, resp. $P=B$), \tj. keď bod~$P$ leží na kratšom oblúku~$AB$ kružnice opísanej päťuholníku $ABCDE$ (\obr). Najmenšia možná hodnota zadaného výrazu je preto
$$
\frac1{2u+1}=\frac1{2\cdot\frac{1+\sqrt5}2+1}=\frac1{\sqrt5+2}=\sqrt{5}-2.
$$}

{%%%%%   trojstretnutie, priklad 6
Zrejme každý pravouholník, ktorého dĺžka aspoň jednej strany je násobkom~$k$, sa dá rozdeliť na pravouholníky rozmerov $1\times k$. Pokúsme sa teda rozdeliť štvorec $m\times m$ na jeden štvorec $n\times n$ a~niekoľko pravouholníkov s~uvedenou vlastnosťou. Samozrejme, zmysel má zaoberať sa iba prípadom $m\ge n$.
\inspinspab{cps.5}{cps.6}%

Ak $k\mid m-n$, môžeme štvorec $m\times m$ rozdeliť tak, ako je znázornené na \obr{}a, oba pravouholníky $\Cal A$, $\Cal B$ totiž majú jednu stranu dĺžky $m-n$, ktorá je násobkom~$k$.

Ak $k\mid m+n$ a~$n+r\le m$, pričom $r$ je zvyšok, ktorý dáva číslo~$m$ po delení~$k$, dá sa štvorec $m\times m$ rozdeliť tak, ako na \obrr1b. Pravouholníky $\Cal D$, $\Cal E$ majú jednu stranu dĺžky $m-r$, ktorá je násobkom~$k$. Podmienka $k\mid m+n$ zabezpečuje, že násobkom $k$ je aj číslo $n+r$, \tj. dĺžka jednej zo strán v~pravouholníkoch $\Cal C$, $\Cal F$. Vďaka nerovnosti $n+r\le m$ majú pravouholníky $\Cal E$, $\Cal F$ stranu nezápornej dĺžky $m-n-r$, uvedené rozdelenie teda naozaj je možné (strany dĺžky~$0$ sú povolené, v~takom prípade jednoducho na pokrytie degenerovaného pravouholníka rozmerov $0\times l$ nepotrebujeme žiadny pravouholník $1\times k$).

Takže aby trojica $(k,m,n)$, pričom $m\ge n$, vyhovovala zadaniu, stačí, aby bola splnená aspoň jedna z~podmienok
\ite (a) $k\mid m-n$;
\ite (b) $k\mid m+n$ a~súčasne $n+r\le m$, kde $r$ je zvyšok, ktorý dáva číslo~$m$ po delení~$k$.

\smallskip
Ukážeme, že tieto podmienky sú zároveň nutné. Najskôr dokážeme, že ak pre trojicu $(k,m,n)$, pričom $m>n$, existuje vyhovujúce rozdelenie, tak $n+r\le m$ (to pri $m>n$ triviálne platí, aj keď je splnená podmienka~(a), nemusíme teda rozlišovať dva prípady). Predpokladajme sporom, že máme vyhovujúce rozdelenie a~pritom $n+r>m$. Bez ujmy na všeobecnosti nech štvorec $n\times n$ sa nedotýka spodnej strany štvorca $m\times m$ (\obr). Keďže $m-n<r<k$, všetky jednotkové štvorčeky dotýkajúce sa priemetu štvorca $n\times n$ na spodnú stranu štvorca $m\times m$ (na \obrr1{} znázornené sivou farbou) musia byť pokryté "ležiacimi" pravouholníkmi $1\times k$ (\tj. takými, ktoré majú dlhšiu stranu rovnobežnú so spodnou stranou štvorca $m\times m$); "stojace" pravouholníky $1\times k$ ich nemôžu pokrývať, lebo by mali spoločný prienik so štvorcom $n\times n$.
\insp{cps.7}%

Nech $p$ je počet "ležiacich" pravouholníkov $1\times k$ pokrývajúcich spomenutých~$n$~sivých jednotkových štvorčekov. Keďže tieto pravouholníky pokrývajú len štvorčeky pri spodnej strane štvorca $m\times m$ a~zároveň pokrývajú minimálne $n$ sivých jednotkových štvorčekov, platia nerovnosti $n\le pk\le m$. Spojením s~nerovnosťou $n+r>m$ dostávame
$$
m-r<pk\le m,
$$
čo je v~spore s~tým, že $r$ je zvyšok, ktorý dáva $m$ po delení $k$ (medzi číslami $m-r$ a~$m$ nemôže ležať žiadny násobok čísla~$k$).

\smallskip
Ostáva dokázať, že $k\mid m-n$ alebo $k\mid m+n$. Hlavná myšlienka bude nasledovná. Do každého jednotkového štvorčeka napíšeme jedno číslo. Pritom celé očíslovanie urobíme tak, aby v~každom pravouholníku $1\times k$ bol súčet čísel rovný~$0$. To znamená, že v~každom vyhovujúcom rozdelení bude musieť byť súčet všetkých čísel vo štvorci $m\times m$ rovnaký ako súčet čísel v~menšom štvorci $n\times n$. Porovnaním oboch súčtov stanovíme nutné podmienky pre $k$, $m$ a~$n$.
\insp{cps.8}

Výhodné bude očíslovanie pomocou komplexných čísel. Nech $z=\cos\frac{2\pi}k+i\sin\frac{2\pi}k$. Teda $z$ je $k$-ta komplexná odmocnina z~čísla~$1$ s~najmenším uhlom (\obr). Pritom
$$
z^0+z^1+\cdots+z^{k-1}=\frac{z^k-1}{z-1}=\frac0{z-1}=0.
\tag1
$$
Očíslujme štvorčeky tak, ako je naznačené na \obr, \tj. ak štvorec $m\times m$ je umiestnený do prvého kvadrantu súradnicovej sústavy s~vrcholom v~počiatku, tak do štvorčeka, ktorého ľavý dolný vrchol má súradnice $(x,y)$, napíšeme číslo $z^{x+y}$.
\insp{cps.9}%

Uvažujme ľubovoľný pravouholník $1\times k$. Nech v~jeho štvorčeku s~najmenšou \hbox{$x$-ovou} (ak sa jedná o~"ležiaci" pravouholník), resp. najmenšou $y$-ovou (ak je to "stojaci" pravouholník) je napísané číslo $z^t$. Potom súčet všetkých čísel v~ňom napísaných je (s~využitím \thetag1)
$$
z^t+z^{t+1}+\cdots+z^{t+k-1}=z^t(z^0+z^1+\cdots+z^{k-1})=0,
$$
teda očíslovanie spĺňa požadovanú podmienku.

Súčet čísel v~ľubovoľnom štvorci $n\times n$, ktorého ľavý dolný štvorček má číslo~$z^t$, je rovný (sčitujúc po jednotlivých riadkoch)
$$
\aligned
&(z^t+\cdots+z^{t+n-1})+(z^{t+1}+\cdots+z^{t+n})+\cdots+(z^{t+n-1}+\cdots+z^{t+2n-2})=\\
&=(z^t+z^{t+1}+\cdots+z^{t+n-1})(z^0+z^1+\cdots+z^{n-1})=\\
&=z^t(z^0+z^1+\cdots+z^{n-1})^2=z^t\left(\frac{z^n-1}{z-1}\right)^2.
\endaligned
$$
Tento vzťah môžeme použiť aj na výpočet súčtu v~celom štvorci $m\times m$. Ten má v~ľavom dolnom štvorčeku číslo $z^0=1$, takže súčet čísel v~ňom je $(z^m-1)^2/(z-1)^2$.

Ak teda máme vyhovujúce rozdelenie, pričom v~ľavom dolnom štvorčeku štvorca $n\times n$ je napísané číslo~$z^t$, musí platiť
$$
z^t\left(\frac{z^n-1}{z-1}\right)^2=\left(\frac{z^m-1}{z-1}\right)^2.
$$
Aby sa dve komplexné čísla rovnali, musia sa rovnať aj ich absolútne hodnoty. Dôsledkovými úpravami predošlej rovnosti (využijúc zrejmý vzťah $|z|=1$) tak postupne dostávame
$$
\aligned
\left|z^t\left(\frac{z^n-1}{z-1}\right)^2\right|&=\left|\left(\frac{z^m-1}{z-1}\right)^2\right|,\\
|z|^t\frac{|z^n-1|^2}{|z-1|^2}&=\frac{|z^m-1|^2}{|z-1|^2},\\
|z^n-1|^2&=|z^m-1|^2,\\
|z^n-1|&=|z^m-1|.\\
\endaligned
$$
Nech $r$, $s$ sú zvyšky, ktoré dávajú $m$, $n$ po delení $k$. Keďže $z^k=1$, zrejme $z^m=z^r$ a~$z^n=z^s$. Pre ktoré čísla $r,s\in\{0,1,\dots,k-1\}$ majú komplexné čísla $z^r-1$, $z^s-1$ rovnakú absolútnu hodnotu? Prvou možnosťou samozrejme je, že $r=s$. V~takom prípade dávajú $m$ a~$n$ rovnaký zvyšok po delení~$k$, teda $k\mid m-n$. Zaoberajme sa ďalej len prípadom $r\ne s$. Čísla $z^r$, $z^s$ ležia v~komplexnej rovine na jednotkovej kružnici so stredom v~$0$ (\obrr2), takže $z^r-1$, $z^s-1$ ležia na jednotkovej kružnici so stredom v~$\m1$. Aby mali dve rôzne čísla na tejto kružnici rovnakú absolútnu hodnotu, musia byť rovnako vzdialené od~$0$, čo zrejme nastáva jedine v~prípade, keď $z^r-1$, $z^s-1$ sú navzájom komplexne združené, \tj. keď $r+s=k$ (\obr). V~tomto prípade teda $k\mid m+n$.
\insp{cps.10}%

}

{%%%%%   IMO, priklad 1
Osi úsečiek $A_1A_2$, $B_1B_2$, $C_1C_2$ sú zároveň osami strán $BC$, $CA$, $AB$, takže sa pretínajú v~bode $O$, ktorý je stredom kružnice opísanej trojuholníku $ABC$. Preto je $O$ jediný bod, ktorý môže byť stredom požadovanej kružnice. Označme strany a~uhly v~trojuholníku štandardným spôsobom. Ďalej nech $r=|OA|$ je veľkosť polomeru opísanej kružnice, $C_0$ je stred strany~$AB$ a~$P$ päta výšky z~vrcholu~$C$ (\obr). Dokážeme, že všetkých šesť bodov zo zadania má od bodu~$O$ rovnakú vzdialenosť. Použitím Pytagorovej vety vo viacerých trojuholníkoch najprv vyjadríme dĺžku úsečky~$OC_1$ pomocou iných dĺžok v~trojuholníku.
\insp{mmo.1}%

Z~pravouhlých trojuholníkov $OC_1C_0$, $C_0HP$,\footnote{Ak $P=C_0$, tak $C_0HP$ nie je trojuholník, ale rovnosť~\thetag2 aj tak triviálne platí.} $OAC_0$, $HAP$ máme
$$
\align
|OC_1|^2&=|C_1C_0|^2+|OC_0|^2,\tag1\\
|HC_0|^2&=|HP|^2+|PC_0|^2,\tag2\\
|OC_0|^2&=r^2-(\tfrac12c)^2,\tag3\\
|HP|^2&=|AH|^2-|AP|^2.\tag4
\endalign
$$
Keďže podľa zadania $|C_1C_0|=|HC_0|$, dosadením \thetag2, \thetag3 do \thetag1 a~následným dosadením \thetag4 dostávame
$$
|OC_1|^2=|HP|^2+|PC_0|^2+r^2-(\tfrac12c)^2=|AH|^2-|AP|^2+|PC_0|^2+r^2-\tfrac14c^2.
$$
Bez ohľadu na to, či bod $P$ leží na úsečke $AC_0$, alebo na úsečke $C_0B$, platí
$$
|PC_0|^2=\left|\tfrac12c-|AP|\right|^2=\left(\tfrac12c-|AP|\right)^2=\tfrac14c^2-c|AP|+|AP|^2.
$$
Dosadením do predošlého vyjadrenia dostaneme
$$
|OC_1|^2=|AH|^2-|AP|^2+\left(\tfrac14c^2-c|AP|+|AP|^2\right)+r^2-\tfrac14c^2=|AH|^2+r^2-c|AP|.
$$
Napokon, z~pravouhlého trojuholníka $CAP$ máme $|AP|=b\cos\alpha$, takže
$$
|OC_1|^2=|AH|^2+r^2-cb\cos\alpha.
$$
Zrejme zopakovaním totožného postupu (len vymeníme úlohu vrcholov $B$ a~$C$) možno vyjadriť
$$
|OB_2|^2=|AH|^2+r^2-bc\cos\alpha,
$$
takže $|OC_1|=|OB_2|$. Analogicky odvodíme aj rovnosti $|OA_1|=|OC_2|$ a~$|OB_1|=|OA_2|$. Spolu s~triviálnymi rovnosťami $|OA_1|=|OA_2|$, $|OB_1|=|OB_2|$, $|OC_1|=|OC_2|$ dostávame
$$
|OA_1|=|OA_2|=|OB_1|=|OB_2|=|OC_1|=|OC_2|,
$$
teda body $A_1$, $A_2$, $B_1$, $B_2$, $C_1$, $C_2$ ležia na jednej kružnici so stredom~$O$.

\ineriesenie
Označme stredy strán $BC$, $CA$, $AB$ postupne $A_0$, $B_0$, $C_0$ a~kružnice spomínané v~zadaní so stredmi v~týchto bodoch postupne $k_a$, $k_b$, $k_c$. Úsečka $A_0B_0$ je spojnicou stredov kružníc $k_a$, $k_b$. Zároveň je ako stredná priečka trojuholníka $ABC$ rovnobežná so stranou $AB$ a~kolmá na priamku~$CH$ obsahujúcu výšku na stranu~$AB$. Priamka~$CH$ je preto chordálou\footnote{Chordála dvoch kružníc je množina bodov, ktoré majú k~obom kružniciam rovnakú mocnosť. Je kolmá na spojnicu stredov kružníc a~prechádza spoločnými bodmi oboch kružníc, pokiaľ sa tieto pretínajú alebo dotýkajú.} kružníc $k_a$, $k_b$ (\obr).
\insp{mmo.2}%

Keďže bod~$C$ leží na chordále kružníc $k_a$, $k_b$, má ku obom kružniciam rovnakú mocnosť, čiže $|CA_1|\cdot|CA_2|=|CB_1|\cdot|CB_2|$. Z~tejto rovnosti a~zo známeho "obráteného" tvrdenia o~mocnosti bodu ku kružnici už priamo vyplýva, že body $A_1$, $A_2$, $B_1$, $B_2$ ležia na jednej kružnici~$k$, bez ohľadu na to, či bod~$C$ leží vnútri oboch priemerov $A_1A_2$, $B_1B_2$, alebo mimo nich. (Nie je možné, aby ležal vnútri jedného priemeru a~mimo druhého, keďže $C$ leží na chordále oboch kružníc. Pri ostrouhlom trojuholníku $ABC$ navyše možno ľahko ukázať, že body $A_1$, $A_2$, resp. $B_1$, $B_2$ ležia vnútri strán $BC$, $CA$, takže $C$ leží určite mimo oboch priemerov $A_1A_2$, $B_1B_2$).

Stred kružnice~$k$ pritom musí byť priesečníkom osí úsečiek $A_1A_2$, $B_1B_2$, \tj. stred~$O$ opísanej kružnice. Odtiaľ máme $|OA_1|=|OA_2|=|OB_1|=|OB_2|$. Zrejme analogicky (argumentáciou o~chordále~$BH$ kružníc $k_a$, $k_c$) dostaneme $|OA_1|=|OA_2|=|OC_1|=|OC_2|$, odkiaľ dostaneme rovnaký záver ako pri prvom riešení.
}

{%%%%%   IMO, priklad 2
a)
Zaveďme substitúciu
$$
\frac x{x-1}=a,\quad\frac y{y-1}=b,\quad\frac z{z-1}=c,\quad\text{\tj.}\quad
x=\frac a{a-1},\quad y=\frac b{b-1},\quad z=\frac c{c-1}.
$$
Chceme dokázať nerovnosť $a^2+b^2+c^2\ge1$ pre ľubovoľné reálne čísla $a,b,c\ne1$ spĺňajúce rovnosť
$$
\frac a{a-1}\cdot\frac b{b-1}\cdot\frac c{c-1}=1
\tag1
$$
pochádzajúcu z~podmienky $xyz=1$. Ekvivalentnými úpravami z~\thetag1 dostávame
$$
\align
  abc&=(a-1)(b-1)(c-1),\\
  ab+bc+ca&=a+b+c-1,\\
  (a+b+c)^2-(a^2+b^2+c^2)&=2(a+b+c-1),\\
  (a+b+c)^2-2(a+b+c)&=a^2+b^2+c^2-2,\\
  (a+b+c-1)^2&=a^2+b^2+c^2-1.\tag2
\endalign
$$
Ľavá strana poslednej rovnosti je vždy nezáporná, teda naozaj platí $a^2+b^2+c^2\ge1$.

\smallskip
b)
Aby sme našli trojice racionálnych čísel $x,y,z\ne1$ spĺňajúce $xyz=1$, pre ktoré platí v~zadanej nerovnosti rovnosť, stačí nájsť trojice racionálnych čísel $a,b,c\ne1$ spĺňajúce rovnosti \thetag1 a~$a^2+b^2+c^2=1$ a~použiť uvedenú substitúciu (zachovávajúcu racionálnosť) na výpočet $x$, $y$, $z$. Pritom prvú z~rovností sme ekvivalentne upravili na tvar~\thetag2. Hľadáme teda v~obore racionálnych čísel nekonečne veľa riešení sústavy
$$
\aligned
(a+b+c-1)^2&=a^2+b^2+c^2-1,\\
a^2+b^2+c^2&=1,
\endaligned
$$
ktorá je zrejme ekvivalentná so sústavou
$$
a^2+b^2+c^2=a+b+c=1.
$$
Vyjadrením $c=1-a-b$ a~dosadením do rovnice $a^2+b^2+c^2=1$ dostávame jedinú rovnicu $a^2+b^2+ab-a-b=0$, ktorú možno v~premennej~$b$ prepísať ako kvadratickú rovnicu
$$
b^2+(a-1)b+a(a-1)=0
\tag3
$$
s~diskriminantom
$$
D=(a-1)^2-4a(a-1)=(1-a)(1+3a).
$$
Aby sme dostali racionálnu trojicu $(a,b,c)$, stačí zobrať racionálne číslo~$a$ také, že súčin $(1-a)(1+3a)$ bude druhou mocninou racionálneho čísla. Potom totiž budú racionálnymi aj čísla
$$
b=\frac{1-a\pm\sqrt{(1-a)(1+3a)}}2\qquad\text{a}\qquad c=1-a-b.
\tag4
$$
Hľadajme $a$ v~tvare podielu celých čísel $k/m$. Potom
$$
1-a=\frac{m-k}m,\qquad 1+3a=\frac{m+3k}m.
$$
Vhodnou voľbou teda bude napríklad $m=k^2-k+1$, kde $k$ je ľubovoľné celé číslo. Potom zrejme $m\ne0$, $m-k=(k-1)^2$, $m+3k=(k+1)^2$, čiže $D=(k^2-1)^2/m^2$. Dosadením do \thetag4 (zvolíme napríklad väčší z~dvoch koreňov kvadratickej rovnice~\thetag3) dostaneme
$$
b=\frac{m-k+k^2-1}{2m}=\frac{m+(m-2)}{2m}=\frac{m-1}m,\qquad c=1-\frac km-\frac{m-1}m=\frac{1-k}m.
$$
Pre rôzne hodnoty $k$ takto zrejme dostaneme nekonečne veľa rôznych racionálnych trojíc $(a,b,c)$, pričom podmienka $a,b,c\ne1$ vylučuje iba hodnoty $k=0$ a~$k=1$. Ak by sme sa vrátili k~pôvodným premenným $x$, $y$, $z$, po jednoduchej úprave by sme dostali trojice
$$
x=-\frac k{(k-1)^2},\qquad y=k-k^2,\qquad z=\frac{k-1}{k^2},
$$
avšak dôkaz je úplný aj bez tohto vyjadrenia.
}

{%%%%%   IMO, priklad 3
Nech $N$ je ľubovoľné prirodzené číslo a~$p$ je prvočíselný deliteľ čísla $N^2+1$. Označme~$z$ zvyšok, ktorý dáva $N$ po delení prvočíslom~$p$ (zrejme $0<z<p$). Potom máme
$$
z^2\equiv N^2\equiv-1\pmod p\qquad\text{a~tiež}\qquad (p-z)^2\equiv z^2\equiv-1\pmod p.
$$
Zoberme za $n$ menšie z~dvojice čísel $z$, $p-z$. Platí $0<n\le p/2$ a~zároveň $p\mid n^2+1$. Navyše
$$
(p-2n)^2\equiv 4n^2\equiv-4\pmod p,
$$
a~keďže $(p-2n)^2>0$, dostávame $(p-2n)^2\ge p-4$. Podľa doterajšieho $p-2n\ge0$, ak je teda $p\ge5$, po odmocnení a~úprave máme
$$
\align
p-2n&\ge\sqrt{p-4},\\
p&\ge2n+\sqrt{p-4}.\tag1
\endalign
$$
Ak $p>20$, tak $\sqrt{p-4}>4$ a~z~\thetag1 vyplýva $p>2n+4$. Potom $p-4>2n$, čiže $\sqrt{p-4}>\sqrt{2n}$ a~dosadením do \thetag1 dostávame požadovanú nerovnosť $p>2n+\sqrt{2n}$.

\smallskip
Ukázali sme, že pre každé prvočíslo $p>20$, ku ktorému existuje také číslo~$N$, že $p\mid N^2+1$ (teda pre každé prvočíslo, ku ktorému je $\m1$ kvadratickým zvyškom) existuje číslo~$n$ s~požadovanou vlastnosťou. Ak by bolo takých $n$ len konečne veľa, muselo by existovať iba konečne veľa opísaných prvočísel, platí totiž $p\mid n^2+1$, \tj. $p\le n^2+1$. Avšak prvočísel s~kvadratickým zvyškom~$\m1$ je nekonečne veľa. Dôkazom tohto známeho tvrdenia ukončíme riešenie úlohy.

Nech $M$ je ľubovoľné prirodzené číslo. Potom ľubovoľné prvočíslo $p$, ktoré je deliteľom čísla $(M!)^2+1$, má medzi kvadratickými zvyškami zvyšok $\m1$. Zároveň $p>M$, lebo zrejme žiadne z~čísel $2$, $3$, \dots, $M$ nie je deliteľom čísla $(M!)^2+1$. Ku každému~$M$ teda existuje prvočíslo $p>M$ s~požadovanou vlastnosťou. Takých prvočísel je preto nekonečne veľa.
}

{%%%%%   IMO, priklad 4
Predpokladajme, že funkcia~$f$ vyhovuje zadaniu. Budeme za $w$, $x$, $y$, $z$ dosadzovať rôzne štvorice kladných čísel spĺňajúce $wx=yz$ a~stanovovať tak podmienky, ktoré musí $f$ spĺňať, a~ktoré budeme ďalej používať.

Po dosadení $w=x=y=z=1$ máme $f^2(1)/f(1)=1$, teda $f(1)=1$. Zoberme ľubovoľné $t>0$ a~dosaďme $w=t$, $x=1$, $y=z=\sqrt t$. S~využitím $f(1)=1$ postupnými úpravami dostávame
$$
\align
\frac{f^2(t)+1}{2f(t)} &= \frac{t^2+1}{2t},\\
tf^2(t)+t &= t^2f(t)+f(t),\\
tf(t)\bigl(f(t)-t\bigr) &= f(t)-t,\\
\bigl(f(t)-t\bigr)\bigl(tf(t)-1\bigr) &=0.
\endalign
$$
Takže pre každé $t>0$ platí buď $f(t)=t$, alebo $f(t)=1/t$. Priamym dosadením do zadania možno ľahko overiť, že obe funkcie
$$
f(t)=t\quad\text{pre všetky $t>0$}\qquad\text{a}\qquad f(t)=\frac1t\quad\text{pre všetky $t>0$}
\tag1
$$
vyhovujú (prvá funkcia vyhovuje očividne, pri druhej treba previesť triviálnu úpravu a~použiť podmienku $wx=yz$). Ukážeme, že žiadna iná funkcia podmienky zadania nespĺňa, \tj. že $f$ nemôže pre niektoré $t\ne1$ nadobúdať hodnotu~$t$ a~pre nejaké iné hodnotu $1/t$.

Predpokladajme, že $f$ nie je ani jedna z~funkcií zapísaných v~\thetag1. Teda pre nejaké $a,b>0$ platí $f(a)\ne a$ a~$f(b)\ne 1/b$. Podľa odvodených podmienok potom nutne $f(a)=1/a$, $f(b)=b$. Dosadením $w=a$, $x=b$, $y=z=\sqrt{ab}$ do zadanej rovnosti a~úpravou dostávame
$$
\align
\frac{\frac1{a^2}+b^2}{2f(ab)}&=\frac{a^2+b^2}{2ab},\\
f(ab)&=\frac{ab(a^{-2}+b^2)}{a^2+b^2}.\tag2
\endalign
$$
Vieme, že $f(ab)=ab$ alebo $f(ab)=1/ab$. Ak $f(ab)=ab$, podľa \thetag2 máme $a^{\m2}+b^2=a^2+b^2$, odkiaľ $a=1$. Avšak $f(1)=1$, čo je v~spore s~predpokladom $f(a)\ne a$. Podobne ak $f(ab)=1/ab$, z~\thetag2 máme
$$
a^2b^2(a^{-2}+b^2)=a^2+b^2,\qquad\text{čiže}\qquad b^2+a^2b^4=a^2+b^2,
$$
odkiaľ $b^4=1$, \tj. $b=1$, čo je v~spore s~predpokladom $f(b)\ne 1/b$.

Dve funkcie zapísané v~\thetag1 sú teda jediné vyhovujúce.
}

{%%%%%   IMO, priklad 5
Postupnosti, ktoré vedú do stavu opísaného v~zadaní (lampy od $1$ po $n$ zapnuté, lampy od $n+1$ po $2n$ vypnuté), nazvime {\it vyhovujúce}. Vyhovujúce postupnosti, v~ktorých navyše ani raz nezapneme žiadnu z~lámp od $n+1$ po $2n$, nazvime {\it špeciálne}. Máme teda $N$ vyhovujúcich postupností, z~ktorých je $M$ špeciálnych.

V~každej vyhovujúcej postupnosti je každá z~lámp $1$, \dots, $n$ na konci zapnutá, takže bola prepnutá nepárny počet krát. Naopak, každá z~lámp $n+1$, \dots, $2n$ je na konci vypnutá, takže bola prepnutá párny počet krát.

Zrejme $M>0$, \tj. existuje aspoň jedna špeciálna postupnosť (stačí raz zapnúť každú z~lámp od $1$ po $n$ a~potom zvoliť jednu z~nich a~prepnúť ju $(k-n)$-krát, čo je podľa zadania párne číslo).

Nech $\Cal P$ je ľubovoľná špeciálna postupnosť. Zvoľme ktorúkoľvek lampu~$l$, kde $1\le l\le n$. Označme $k_l$ celkový počet prepnutí lampy~$l$ (ako sme spomenuli skôr, $k_l$ je nepárne). Vyberme spomedzi nich ľubovoľnú podmnožinu obsahujúcu párne veľa prepnutí a~nahraďme ich prepnutím lampy $n+l$. To môžeme urobiť $2^{k_l-1}$ spôsobmi, keďže každá $k_l$-prvková množina má $2^{k_l-1}$ podmnožín s~párnou mohutnosťou\footnote{Tento známy fakt možno odvodiť jednoduchou kombinatorickou úvahou: Keď vytvárame podmnožinu s~párnou mohutnosťou, pri každom spomedzi $k_l$~prvkov sa môžeme rozhodnúť, či do podmnožiny bude alebo nebude patriť, len pri poslednom prvku na výber nemáme -- musíme alebo nesmieme ho do podmnožiny pridať, aby sme dodržali paritu. Celkový počet podmnožín je teda $$\underbrace{2\cdot2\cdot\dots\cdot2}_{\text{$(k_l-1)$-krát}}\cdot1=2^{k_l-1}.$$}.

Uvedené zmeny prepnutí môžeme urobiť nezávisle s~každou lampou pre $l=1,\dots,n$. Keďže $k_1+\cdots+k_n=k$, celkový počet rôznych postupností, ktoré dostaneme, je
$$
2^{k_1-1}\cdot2^{k_1-1}\cdot\dots\cdot2^{k_n-1}=2^{k-n}.
$$
V~každej vytvorenej postupnosti je každá z~lámp od $n+1$ do $2n$ prepnutá párny počet krát a~každá z~lámp od $1$ do $n$ nepárny počet krát, jedná sa preto o~vyhovujúcu postupnosť. Z~každej špeciálnej postupnosti~$\Cal P$ vieme takto vytvoriť $2^{k-n}$ vyhovujúcich postupností.

Zrejme každú vyhovujúcu postupnosť~$\Cal Q$ možno vytvoriť opísaným spôsobom. Stačí každé prepnutie lampy $l>n$ nachádzajúce sa v~$\Cal Q$ nahradiť prepnutím lampy $l-n$. Vo výslednej postupnosti~$\Cal P$ nebudú lampy od $n+1$ do $2n$ prepnuté ani raz. Keďže v~$\Cal Q$ bola každá lampa $l>n$ prepnutá párny počet krát, každá lampa $l\le n$ bude v~$\Cal P$ prepnutá nepárny počet krát, \tj. postupnosť~$\Cal P$ bude špeciálna. Ak teraz obrátime postup a~vrátime príslušné prepnutia naspäť na lampy od $n+1$ do $2n$, dostaneme postupnosť $\Cal Q$. Pritom obrátený postup zmeny $\Cal P$ na $\Cal Q$ prebieha presne tak, ako sme opísali v~predošlých odsekoch.

Našli sme zobrazenie z~množiny vyhovujúcich postupností do množiny špeciálnych postupností, pričom vzor každej špeciálnej postupnosti pri tomto zobrazení obsahuje $2^{k-n}$ vyhovujúcich postupností. Preto $N/M=2^{k-n}$.
}

{%%%%%   IMO, priklad 6
Pri riešení použijeme dve pomocné lemy, ktoré najprv sformulujeme a~dokážeme.

\smallskip\noindent
{\it Lema 1}. Nech $ABCD$ je konvexný štvoruholník. Ak existuje kružnica, ktorá sa dotýka polpriamky~$BA$ za bodom~$A$, polpriamky~$BC$ za bodom~$C$ a~priamok $AD$ a~$CD$, tak $|AB|+|AD|=|CB|+|CD|$.
\insp{mmo.3}%

\smallskip\noindent
{\it Dôkaz}.
Označme dotykové body kružnice a~priamok $AB$, $BC$, $CD$, $DA$ postupne $K$, $L$, $M$, $N$ (\obr). Máme
$$
\align
|AB|+|AD|&=(|BK|-|AK|)+(|AN|-|DN|),\\
|CB|+|CD|&=(|BL|-|CL|)+(|CM|-|DM|).
\endalign
$$
Zrejme $|BK|=|BL|$, $|DN|=|DM|$, $|AK|=|AN|$ a~$|CL|=|CM|$ (vzdialenosti bodu od príslušných dotykových bodov na kružnici sú rovnaké). Odtiaľ už priamo dostávame $|AB|+|AD|=|CB|+|CD|$.

\smallskip\noindent
{\it Lema 2}. V~danom trojuholníku $ABC$ označme $P$ bod, v~ktorom sa vpísaná kružnica~$\omega_1$ dotýka strany~$AC$. Nech $PP'$ je priemer vpísanej kružnice a~$Q$ je priesečník priamky~$BP'$ so stranou~$AC$. Potom $Q$ je bodom dotyku strany~$AC$ a~kružnice pripísanej k~strane~$AC$.
\insp{mmo.4}%

\smallskip\noindent
{\it Dôkaz}.
Priesečníky dotyčnice k~$\omega_1$ vedenej bodom~$P'$ so stranami $BA$, $BC$ označme postupne $A'$, $C'$ (\obr). Kružnica $\omega_1$ je pripísanou kružnicou ku strane~$A'C'$ trojuholníka $A'BC'$ a~dotýka sa strany~$A'C'$ v~bode~$P'$. Keďže $A'C'\parallel AC$, v~rovnoľahlosti so stredom~$B$ a~koeficientom $|BQ|/|BP'|$ sa trojuholník $A'BC'$ zobrazí na trojuholník $ABC$, kružnica~$\omega_1$ na kružnicu pripísanú k~strane~$AC$ trojuholníka $ABC$ a~bod~$P'$ (ktorý je bodom dotyku $\omega_1$ a~strany~$A'C'$) na bod~$Q$ (ktorý preto musí byť bodom dotyku pripísanej kružnice a~strany~$AC$).

\smallskip
Pripomeňme ešte známy fakt, že ak sa v~trojuholníku $ABC$ dotýka vpísaná kružnica strany~$AC$ v~bode~$P$ a~pripísaná kružnica k~tejto strane sa jej dotýka v~bode $Q$, tak $|AP|=|CQ|$.

Vráťme sa k~pôvodnému zadaniu. Nech $\omega_1$ sa dotýka uhlopriečky~$AC$ v~bode~$P$ a~$\omega_2$ v~bode~$Q$. Podľa známych vzorcov pre dĺžky úsekov medzi vrcholmi trojuholníka a~dotykovými bodmi strán a~vpísanej kružnice dostávame
$$
|AP|=\tfrac12(|AC|+|AB|-|BC|),\qquad |CQ|=\tfrac12(|AC|+|CD|-|AD|).
$$
Keďže z~prvej lemy vyplýva $|AB|-|BC|=|CD|-|AD|$, dostávame $|AP|=|CQ|$. Preto $Q$ je zároveň bodom dotyku kružnice pripísanej k~strane~$AC$ trojuholníka $ABC$. Analogicky $P$ je bodom dotyku kružnice pripísanej k~strane~$AC$ trojuholníka $ADC$. Navyše $P\ne Q$, lebo $|AB|\ne|BC|$.

Nech $PP'$, $QQ'$ sú priemery kružníc $\omega_1$, $\omega_2$ kolmé na uhlopriečku~$AC$ (\obr). Podľa druhej lemy ležia body $B$, $P'$, $Q$ na jednej priamke. Takisto ležia na jednej priamke body $D$, $Q'$, $P$.
\insp{mmo.5}%

Uvažujme priemer kružnice $\omega$, ktorý je kolmý na $AC$. Nech $T$ je ten jeho krajný bod, ktorý je bližšie k~$AC$. V~rovnoľahlosti so stredom~$B$ a~koeficientom $|BT|/|BP'|$ sa $\omega_1$ zobrazí na $\omega$, preto body $B$, $P'$, $T$ ležia na jednej priamke. Podobne sa v~rovnoľahlosti so stredom $D$ a~koeficientom $\m|DT|/|DQ'|$ zobrazí $\omega_2$ na $\omega$ a~na jednej priamke ležia body $D$, $Q'$, $T$.

Takže bod~$T$ je priesečníkom priamok $P'Q$ a~$PQ'$. Keďže $PP'\parallel QQ'$, kružnice $\omega_1$, $\omega_2$ s~priemermi $PP'$, $QQ'$ sú rovnoľahlé so stredom~$T$. Pritom koeficient tejto rovnoľahlosti je očividne kladný (lebo $T$ neleží na spoločnej vnútornej dotyčnici~$AC$ oboch kružníc), teda $T$ je zároveň priesečníkom vonkajších dotyčníc kružníc $\omega_1$, $\omega_2$, čo sme chceli dokázať.
}

{%%%%%   MEMO, priklad 1
\podla{Jaromíra Šimšu{\rm,} Česká republika}
Pre každé $n\ge1$ a~štvoricu indexov $(n,n+1,n+1,n+2)$ podľa zadania platí
$$
a_{n+2}-a_{n+1}\ge(a_{n+1}-a_{n})+1.
$$
Keďže $a_2-a_1\ge1$, jednoduchým dôkazom matematickou indukciou dostávame $a_{n+1}-a_n\ge n$ pre $n\ge1$. Takže $a_{n+1}\ge n+a_n$. Z~toho s~využitím $a_1\ge1$ opäť triviálnou matematickou indukciou odvodíme nerovnosť
$$
a_n\ge\frac12(n^2-n+2).
$$

Pritom postupnosť $a_n=\frac12(n^2-n+2)$ spĺňa podmienky zadania. Nerovnosť $a_i+a_l>a_j+a_k$ je totiž pre ňu (pri rovnosti $i+l=j+k$) ekvivalentná s~nerovnosťou $i^2+l^2>j^2+k^2$, ktorá po substitúcii $i=d-y$, $l=d+y$,
$j=d-x$, $k=d+x$ (kde $0\le x<y$) prejde na zrejmú nerovnosť $2d^2+2y^2>2d^2+2x^2$.

\zaver
Najmenšia možná hodnota čísla $a_{2008}$ je $\frac12(2008^2-2008+2)=2\,015\,029$.
}

{%%%%%   MEMO, priklad 2
Množinu $k$~políčok uložených od jedného okraja šachovnice po druhý v~smere niektorej uhlopriečky (pričom $1\le k\le n$) nazývajme $k$-diagonála. Počet disjunktných diagonál v~jednom smere je $2n-1$ (\obr), medzi $2n-2$ zvolenými políčkami však nemôžu byť naraz obe políčka na $1$-diagonálach (keďže tie sú obe súčasťou $n$-diagonály majúcej druhý smer). Preto na každej $k$-diagonále pre $k>1$ musí byť zvolené práve jedno políčko a~práve dve políčka musia byť zvolené v~rohoch (nie však protiľahlých).
\insp{memo.6}%

Uvažujme množinu~$P$ všetkých takých dvojíc $(z,v)$, že $z$ je zvolené políčko a~$v$ je {\it voľné\/} (čiže nezvolené) políčko na rovnakej diagonále ako $z$. Na šachovnici je práve $n^2-2n+2$ voľných políčok, pričom dve z~nich sú rohové. Každé zo zvyšných $n^2-2n$ voľných políčok~$v$ leží na dvoch $k$-diagonálach pre $k>1$, existujú k~nemu preto práve dve políčka~$z$ také, že $(z,v)\in P$. Celkový počet~$p$ dvojíc v~množine~$P$ je teda
$$
p=2(n^2-2n)+2=2n^2-4n+2,\tag1
$$
pričom $\p2$ je príspevok dvoch voľných rohových políčok (každé z~nich má jediné príslušné zvolené políčko v~protiľahlom rohu).

Ak zvolené políčko~$z$ leží na prieniku $k_1$-diagonály a~$k_2$-diagonály pre $k_1,k_2>1$, tak počet počet voľných políčok~$v$ takých, že $(z,v)\in P$, je rovný $k_1+k_2-2$. To isté platí aj pre rohové políčka, pre ktoré $\{k_1,k_2\}=\{1,n\}$. Zrejme pre každé zvolené políčko~$z$ platí $k_1+k_2\ge n+1$, pričom rovnosť platí práve vtedy, keď sa políčko nachádza na okraji šachovnice. Takže počet takých voľných políčok~$v$, že $(z,v)\in P$, je aspoň $n-1$. Odtiaľ
$$
p\ge(2n-2)(n-1)=2n^2-4n+2.
$$
Podľa \thetag1 vieme, že v~predošlej nerovnosti platí rovnosť, preto všetky zvolené políčka musia ležať na okraji šachovnice.

Ak zvolíme ľubovoľné políčka (napríklad aj žiadne) z~prvého riadka šachovnice, zvyšné okrajové políčka (ležiace mimo prvého riadka), ktoré musíme zvoliť, sú jednoznačne určené. Pre rohové políčka je to zrejmé, pre ostatné políčka stačí pre každé $k=2,3,\dots,n-1$ uvažovať obdĺžnik tvorený dvoma $k$-diagonálami v~jednom smere a~$(n+1-k)$-diagonálami v~druhom smere (v~každom takom obdĺžniku musia byť medzi zvolenými políčkami dva protiľahlé rohy, \obr). Celkový počet rôznych výberov políčok je teda rovnaký, ako počet rôznych podmnožín $n$-prvkovej množiny (tvorenej políčkami prvého riadka), čiže $2^n$.
\inspinsp{memo.7}{memo.8}%

\ineriesenie
\podla{Bernda Mulanskeho{\rm,} Nemecko}
Dva rôzne smery diagonál označme $A$ a~$B$. Z~úvodu prvého riešenia vieme, že na každej $k$-diagonále pre $k>1$ je zvolené práve jedno políčko a~práve dve políčka sú zvolené v~neprotiľahlých rohoch. Každý vyhovujúci výber políčok môžeme vytvoriť nasledujúcim postupom pozostávajúcim z~$n$~krokov:
\item{$\triangleright$}
Krok 1: Zvolíme políčko na jednej z~dvoch $1$-diagonál smeru~$A$.
\item{$\triangleright$}
Krok $k$ ($2\le k\le n-1$): Zvolíme dve políčka, každé na jednej z~dvoch $k$-diagonál smeru~$A$.
\item{$\triangleright$}
Krok $n$: Zvolíme políčko na $n$-diagonále smeru~$A$.

Zrejme pre každé $m=1,2,\dots,n-1$ po urobení $m$~krokov (takých, že žiadne dve zvolené políčka nie sú na rovnakej diagonále smeru~$B$) sa na každej spomedzi $2m-1$ najdlhších $k$-diagonál smeru~$B$ (\tj. $k\ge n+1-m$) nachádza zvolené políčko (\obr). Ak $m<n-1$, v~nasledujúcom kroku $m+1$ musia byť obe políčka zvolené na kraji oboch $(m+1)$-diagonál smeru~$A$ (ostatné políčka týchto dvoch diagonál ležia na už "obsadených" diagonálach smeru~$B$), čo možno urobiť práve dvoma spôsobmi. Podobne je to v~prípade $m+1=n$. Máme teda dve možnosti v~každom z~$n$~krokov a~celkový počet rôznych vyhovujúcich výberov je $2^n$.


\ineriesenie
\podla{Pavla Novotného{\rm,} Slovensko}
Ofarbime políčka šachovnice ako zvyčajne, pričom ľavý horný roh bude čierny. Z~podobnej úvahy ako v~úvode prvého riešenia vyplýva, že musíme zvoliť $n-1$ bielych a~$n-1$ čiernych políčok. Počet $p_n$ všetkých vyhovujúcich výberov $2n-2$ políčok na šachovnici $n\times n$ sa rovná súčinu $b_n\cdot c_n$, pričom $b_n$ a~$c_n$ sú počty vyhovujúcich výberov $n-1$ bielych, resp. čiernych políčok. Zrejme $b_2=b_3=2$, $c_2=2$ a~$c_3=4$. Ľahko možno ukázať, že pre každé $n\ge4$ platí $b_n=2b_{n-2}$,\footnote{Odstránime dve biele $2$-diagonály v~jednom smere a~dve biele $(n-1)$-diagonály v~druhom smere; zvyšné biele políčka vytvárajú rovnaké diagonály ako biele políčka šachovnice $(n-2)\times(n-2)$.} $c_n=2b_{n-1}$,\footnote{Odstránime jednu čiernu $n$-diagonálu; zvyšné čierne políčka vytvárajú rovnaké diagonály ako biele políčka šachovnice $(n-1)\times(n-1)$.} takže $p_n=b_nc_n=4b_{n-2}b_{n-1}=2c_{n-1}b_{n-1}=2p_{n-1}$, odkiaľ už triviálne vyplýva $p_n=2^n$.
}

{%%%%%   MEMO, priklad 3
Bez ujmy na všeobecnosti predpokladajme, že $|AF|<|AG|$. Rozoberme najprv situáciu, keď $G$ je na kratšom oblúku~$DE$.

Označme $J$ dotykový bod vpísanej kružnice so stranou~$AC$. Z~vlastností súhlasných, úsekového a~obvodových uhlov
máme $|\angle CAB|=|\angle CJE|=|\angle JDE|=|\angle JFE|$ (\obr), takže $AJFK$ je tetivový štvoruholník. Preto z~obvodových, vrcholových a~úsekového uhla dostávame $|\angle AJK|=|\angle AFK|=|\angle EFG|=|\angle LEB|$, teda trojuholníky $AJK$ a~$BEL$ sú zhodné. Keďže $K$ a~$L$ sú vnútorné body úsečky~$AB$, z~rovnosti $|AK|=|BL|$ vyplýva $|DK|=|DL|$.

Ak $G$ leží na dlhšom oblúku~$DE$ (medzi bodmi $E$ a~$J$), tak $K$, $A$, $B$, $L$ ležia v~tomto poradí na priamke a~tetivovým štvoruholníkom je $AKJF$. Ostatné argumenty sú rovnaké ako v~predošlom prípade.
\inspinsp{memo.1}{memo.2}%

\ineriesenie
\podla{Tomáša Pavlíka{\rm,} Česká republika}
Označme $X$ priesečník priamky~$AF$ so stranou~$BC$ (\obr). Z~mocnosti bodu~$X$ ku vpísanej kružnici platí $|XE|^2=|XF|\cdot|XG|$, čiže
$$
{|XG|\over|XE|}={|XE|\over|XF|}.  \tag1
$$
Podľa Menelaovej vety pre trojuholník $ABX$ a~priamky $EG$ a~$EF$ máme
$$
{|AL|\over|LB|}\cdot {|BE|\over|EX|}\cdot {|XG|\over|GA|}=1
\quad\text{a}\quad
{|AK|\over|KB|}\cdot {|BE|\over|EX|}\cdot {|XF|\over|FA|}=1.    %\tag2
$$
Použitím \thetag1 môžeme tieto dve rovnosti prepísať na
$$
{|XE|\over |XF|}\cdot {|AL|\cdot|BE|\over|LB|\cdot|GA|}=1
\quad\text{a}\quad
{|XE|\over|XF|}\cdot {|KB|\cdot|FA|\over|AK|\cdot|BE|}=1.     %\tag2
$$
Odtiaľ postupne
$$
\gather
{|AL|\cdot|BE|\over|LB|\cdot|GA|}={|KB|\cdot|FA|\over|AK|\cdot|BE|},\\
{|AK|\cdot|AL|\cdot|BE|^2\over|KB|\cdot|LB|\cdot|FA|\cdot|GA|}=1. \tag2
\endgather
$$
Z~mocnosti bodu~$A$ ku vpísanej kružnici platí $|AF|\cdot|AG|=|AD|^2$, odkiaľ spolu so zrejmými rovnosťami $|AD|=|BD|=|BE|$ máme $|AF|\cdot|AG|=|BE|^2$. Spojením s~\thetag2 dostávame
$$
|AK|\cdot|AL|=|KB|\cdot|LB|.
$$

V~závislosti od polohy bodu~$G$ ležia body $K$ a~$L$ buď oba vnútri, alebo mimo úsečky~$AB$. Podľa toho pre niektoré znamienko plus alebo mínus platí
$$
|AK|\cdot (|AB|\pm |BL|)=|AK|\cdot |AL|=|KB|\cdot |LB|=(|AB|\pm |AK|)\cdot |BL|.
$$
V~oboch prípadoch po úprave $|AK|=|BL|$, čo je ekvivalentné s~rovnosťou $|DK|=|DL|$.
}

{%%%%%   MEMO, priklad 4
Keďže číslo $4n+1$ je nepárne, z~rovnosti $k-4=k(4n+1)-4(kn+1)$ vidíme, že $4n+1$ a~$kn+1$ sú nesúdeliteľné, ak $k-4$ nemá žiadneho nepárneho deliteľa $p>1$, \tj. keď $k-4=\pm2^m$ pre nejaké nezáporné celé číslo~$m$.

Na druhej strane, ak $k-4$ má nepárneho deliteľa $p>1$, ľahko nájdeme násobok~$p$ tvaru $4n+1$ (je ním napríklad číslo~$p^2$ alebo jednoducho jedno z~dvojice čísel $p$, $3p$). Pre každé číslo $4n+1$, ktoré je násobkom~$p$, z~rovnosti uvedenej na začiatku riešenia vyplýva $p\mid kn+1$, teda $4n+1$ a~$kn+1$ nie sú nesúdeliteľné.

\odpoved
Hľadanými číslami sú $k=4\pm2^m$, pričom $m=0,1,2,\dots$
}

{%%%%%   MEMO, priklad t1
Dosadením $x=y=0$ do zadanej rovnosti
$$
xf(x+xy)=xf(x)+f(x^2)f(y)
$$
dostaneme $f(0)=0$. Po dosadení $y=\m1$ do zadanej rovnosti tak máme
$$
xf(x)+f(x^2)f(-1)=0.
\tag1
$$
Rozoberme postupne prípady $f(-1)=0$ a $f(-1)\ne0$.

\pripad{$f(-1)=0$}
Z~\thetag1 potom vyplýva $f(x)=0$ pre všetky $x\ne0$. Keďže už vieme, že aj $f(0)=0$, dostávame konštantnú nulovú funkciu $f(x)=0$, ktorá očividne vyhovuje.

\pripad{$f(-1)\ne0$}
Dosadením $x=\m1$ do \thetag1 dostávame $f(1)=1$. S~využitím toho po dosadení $x=1$ do \thetag1 máme
$f(\m1)=\m1$ a~teda \thetag1 môžeme prepísať na
$$
xf(x)=f(x^2).
\tag2
$$
Dosaďme teraz do zadanej rovnosti $y=x-1$. Odtiaľ
$$
xf(x^2)=xf(x)+f(x^2)f(x-1).
\tag3
$$
Sčítaním \thetag2 a~\thetag3 získame po úprave rovnosť
$$
f(x^2)(f(x-1)-(x-1))=0.
\tag4
$$
Predpokladajme, že $f(a)=0$ pre nejaké $a\ne0$.
Potom podľa \thetag2 máme $f(a^2)=0$ a~teda po dosadení $x=a$ do zadanej rovnosti dostaneme
$af(a+ay)=0$, čiže $f(a+ay)=0$. Keďže $y$ môže byť ľubovoľné, nutne aj $f(\m1)=0$, čo nesúhlasí s~prípadom, ktorý rozoberáme. Preto pre každé $x\ne0$ platí $f(x)\ne0$ a~takisto aj $f(x^2)\ne0$.
Z~\thetag4 potom $f(x-1)=x-1$ pre každé $x\ne0$, takže $f(x)=x$ pre každé $x\ne\m1$. Keďže z~predošlého vieme, že aj $f(\m1)=\m1$, dostávame funkciu $f(x)=x$, ktorá tiež očividne vyhovuje.

\zaver
Hľadanými funkciami sú $f(x)=0$ a~$f(x)=x$.
}

{%%%%%   MEMO, priklad t2
Ak začneme s~$n$-ticou $(2,2,1,1,\dots,1)$, pričom $n\ge3$, v~každej $n$-tici, ktorú z~nej po ľubovoľnom počte krokov dostaneme, bude počet členov nadobúdajúcich maximálnu hodnotu {\it párny}. Preto nevyhovuje žiadna nepárna hodnota $n\ge3$.

Matematickou indukciou dokážeme, že každé párne $n\ge2$ vyhovuje. Pre $n=2$ je to zrejmé. Ak $n\ge4$ je párne, podľa indukčného predpokladu vieme ľubovoľnú $n$-ticu po konečnom počte krokov zmeniť na $(a,a,\dots,a,b,b)$. Ak $a\ne b$, opakovane urobíme niektorú z~nasledujúcich sérií krokov, ktoré vždy vedú na $n$-ticu tvaru
$$
(\underbrace{a,\dots,a}_{k},\underbrace{b,\dots,b}_{n-k})
$$
(v~nej $k$ môže mať inú hodnotu ako počiatočné $k=n-2$, stále však bude {\it párne\/}):
$$
\aligned
\text{séria $\alpha$:}&\quad
(\underbrace{a,\dots,a}_{k},\underbrace{b,\dots,b}_{n-k})\to
(\underbrace{2a,\dots,2a}_{k},\underbrace{b,\dots,b}_{n-k}),\\
\text{séria $\beta$:}&\quad
(\underbrace{a,\dots,a}_{k},\underbrace{b,\dots,b}_{n-k})\to
(\underbrace{a,\dots,a}_{k},\underbrace{2b,\dots,2b}_{n-k}),\\
\text{séria $\gamma_1$ (ak $k\le n-k$):}&\quad
(\underbrace{a,\dots,a}_{k},\underbrace{b,\dots,b}_{n-k})\to
(\underbrace{a+b,\dots,a+b}_{2k},\underbrace{b,\dots,b}_{n-2k}),\\
\text{séria $\gamma_2$ (ak $k\ge n-k$):}&\quad
(\underbrace{a,\dots,a}_{k},\underbrace{b,\dots,b}_{n-k})\to
(\underbrace{a,\dots,a}_{2k-n},\underbrace{a+b,\dots,a+b}_{2(n-k)}).
\endaligned
$$
Kvôli ďalším úvahám zaveďme označenie $c=2^{P(c)}N(c)$ pre ľubovoľné prirodzené číslo $c$, pričom $P(c)\ge0$
a~$N(c)$ je nepárne. Na $n$-ticu
$$
(\underbrace{a,\dots,a}_{k},\underbrace{b,\dots,b}_{n-k})
$$
pričom $a\ne b$, použijeme
\item{$\triangleright$} sériu $\alpha$, ak $P(a)<P(b)$,
\item{$\triangleright$} sériu $\beta$, ak $P(a)>P(b)$,
\item{$\triangleright$}  sériu $\gamma_1$ alebo $\gamma_2$, ak $P(a)=P(b)$ (a~teda $N(a)\ne N(b)$).


Pri použití sérií $\alpha$ a~$\beta$ sa čísla $N(a)$, $N(b)$ nemenia, zatiaľ čo pri použití $\gamma_1$ a~$\gamma_2$ sa zmení práve jedno z~nich, konkrétne
$$
N(b)\to\frac{N(a)+N(b)}{2^m},\quad \text{alebo}\quad
N(b)\to\frac{N(a)+N(b)}{2^m},
$$
pričom $m=P(N(a)+N(b))\ge1$ a~teda
$$
\frac{N(a)+N(b)}{2^m}\le\frac{N(a)+N(b)}{2}<\max(N(a),N(b))
$$
(pripomíname, že $N(a)\ne N(b)$). Z~uvedeného vyplýva, že hodnota $\max(N(a),N(b))$ nikdy nerastie, teda po konečnom počte krokov musí byť konštantná. Od toho momentu musíme mať stále buď $N(a)\ge N(b)$, alebo $N(a)\le N(b)$. To vylučuje z~ďalšieho použitia buď sériu $\ga_1$, alebo sériu $\ga_2$. Všetky ďalšie zmeny parametra $k$ sú potom buď  $k\to2k$, alebo $(n-k)\to2(n-k)$. Keďže toto možno zopakovať iba $r$-krát, kde $2^r\le n$, na konci musíme dostať $n$-ticu $(a,\dots,a,b,\dots,b)$, ktorú (ak $a\ne b$) už môžeme meniť len sériami $\alpha$ a~$\beta$. Použitím série $\al$ alebo $\be$ práve $|P(a)-P(b)|$-krát dostaneme $n$-ticu $(a',\dots,a',b',\dots,b')$, v~ktorej $P(a')=P(b')$. Keďže použitie $\gamma_1$, $\gamma_2$ sme už vylúčili, nutne $a'=b'$, čím je indukčný krok ukončený.

\ineriesenie
\podla{nemeckého družstva{\rm, upravené}}
Dokážeme {\it bez\/} matematickej indukcie vzhľadom na $n$, že vyhovuje každé párne $n=2k$.
Najskôr v~začiatočnej $2k$-tici $(a_1,\dots,a_{2k})$ nahradíme
každú dvojicu $(a_{2i-1},a_{2i})$ (pre $i=1,\dots,k$) dvojicou
$(a_{2i-1}+a_{2i},a_{2i-1}+a_{2i})$. Odteraz budeme mať na $(2i-1)$-tej a~$2i$-tej pozícii vždy
rovnaké čísla. Preto kvôli prehľadnosti budeme pracovať s~$k$-ticami $(x,y,z,\dots)$ namiesto $2k$-tic
$(x,x,y,y,z,z,\dots)$. S~$k$-ticami môžeme robiť nasledujúce zmeny:
\item{$\triangleright$}
zvolíme dve čísla $x$, $y$ a~nahradíme každé z~nich ich súčtom
(to zodpovedá dvom krokom $(\dots,x,x,\dots,y,y,\dots)\to
(\dots,x+y,x,\dots,x+y,y,\dots)\to(\dots,x+y,x+y,\dots,x+y,x+y,\dots)$
vykonaných na $2k$-tici);
\item{$\triangleright$} zvolíme jedno číslo $x$ a~vynásobíme ho dvoma
(to zodpovedá jednému kroku $(\dots,x,x,\dots)\to(\dots,x+x,x+x,\dots$);
\item{$\triangleright$} predelíme všetky čísla dvoma (to samozrejme nič neovplyvní; formálne si môžeme pamätať, koľkokrát sme delenie dvoma vykonali a~na konci môžeme všetky čísla vynásobiť príslušnou mocninou dvoch).

Naším cieľom je dostať $k$ rovnakých čísel. Získame ich opakovaním nasledovného algoritmu:
\item{1.} Kým existujú aspoň dve nepárne čísla, nájdeme najmenšie a najväčšie nepárne číslo a nahradíme každé z~nich ich (párnym) súčtom.
\item{2.} Ak po skončení prvého kroku je v~$k$-tici jedno nepárne číslo, vynásobíme ho dvoma.
\item{3.} Vydelíme všetky čísla dvoma.

Zrejme po každom vykonaní celého algoritmu sa najväčšie číslo spomedzi všetkých $k$ čísel buď zmenší, alebo nezmení. Keďže toto maximum je stále prirodzeným číslom, po konečnom počte opakovaní algoritmu musí nadobudnúť hodnotu~$M$, ktorá sa už nebude meniť. Od tohto momentu sledujme počet čísel majúcich hodnotu $M$
v~našej $k$-tici. Tento počet označme $N$.

Zrejme $M$ je nepárne (inak by sa v~treťom kroku algoritmu zmenšilo). Ak $N<k$, tak v~$k$-tici existuje aspoň jedno číslo $m$ menšie ako $M$. Ak $m$ je nepárne, po vykonaní algoritmu sa $N$ zmenší. Keďže $N$ sa nemôže nikdy zväčšiť, po konečnom počte krokov už musí ostať konštantné a všetky čísla v $k$-tici menšie ako $M$ musia byť párne. Ale každé párne $m$ sa po vykonaní algoritmu vydelí dvoma a~po niekoľkých vykonaniach algoritmu sa nutne objaví nepárne číslo menšie ako $M$. Preto v~$k$-tici neexistujú čísla menšie ako $M$, čo sme chceli dokázať.
}

{%%%%%   MEMO, priklad t3
Podmienka $|\angle ADB|=|\angle CDE|$ nabáda zobraziť bod~$B$ v~osovej súmernosti podľa priamky~$AE$ do bodu $B'$ (\obr). Potom ležia body $C$, $D$ a~$B'$ na jednej priamke a~$|\angle EAB'|=|\angle EAB|=|\angle ECD|=|\angle ECB'|$, takže $B'ACE$ je tetivový štvoruholník. Odtiaľ $|\angle ECA|=180^{\circ}-|\angle EB'A|=180^{\circ}-|\angle EBA|=180^{\circ}-|\angle ACB|$, čiže $|\angle ECA|+|\angle ACB|=180^{\circ}$ a~teda body $B$, $C$, $E$ ležia na jednej priamke.
\insp{memo.3}%

\poznamky
Rovnako dobre môžeme zobraziť $C$ v~osovej súmernosti podľa~$AE$ do $C'$ ležiaceho na jednej priamke s~$B$, $D$. Tetivový je potom štvoruholník $ABEC'$ a~ďalej $|\angle ECA|=|\angle EC'A|=180^{\circ}-|\angle EBA|=180^{\circ}-|\angle
ACB|$, \tj. opäť $|\angle ECA|+|\angle ACB|=180^{\circ}$.

Kvôli dokázanej kolineárnosti bodov $B$, $C$, $E$ z~podmienky $|\angle ACB|=|\angle EBA|$ vyplýva $|AB|=|AC|$, zatiaľ čo z~podmienky $|\angle BAD|=|\angle ECD|$ vyplýva, že štvoruholník $ABCD$ je tetivový. To naznačuje, ako možno úlohu riešiť iným spôsobom. Predtým ešte poznamenajme, že rozloženie bodov opísané v~zadaní {\it môže\/} nastať a~všetky také rozloženia sú tohto typu: $ABC$ je rovnoramenný trojuholník, pričom $|AB|=|AC|$, body $B$, $C$, $E$ ležia na jednej priamke ($C$ medzi $B$ a~$E$) a~$AE$ pretína kružnicu opísanú trojuholníku $ABC$ v~bode~$D$.

\ineriesenie
Predpokladajme, že $B$, $C$, $E$ nie sú kolineárne.
Priamka prechádzajúca cez $B$ rovnobežná s~$CE$ pretína priamky $CD$ a~$AD$ postupne v~bodoch $C'$ a~$E'$. Keďže $|\angle E'C'D|=|\angle ECD|=|\angle BAD|$, štvoruholník $ABC'D$ je tetivový (\obr).
Označme $\Cal K$ jemu opísanú kružnicu.
Máme $|\angle AC'B|=|\angle ADB|=|\angle CDE|=|\angle C'DE|=|\angle ABC'|$,
\tj. $|\angle AC'B|=|\angle ABC'|$ (teda $ABC'$ je rovnoramenný trojuholník).

Predpokladajme, že $C$ leží vnútri úsečky~$C'D$. Potom $C$ leží vnútri $\Cal K$ (v~rovnakej polrovine určenej priamkou~$AB$ ako bod~$C'$), preto $|\angle ACB|>|\angle AC'B|=|\angle ABC'|=|\angle ABE'|>|\angle ABE|$
(lebo $E$ leží medzi $A$ a~$E'$), čo je v~spore s~$|\angle ACB|=|\angle EBA|$.

Podobne ak $C$ neleží na úsečke~$C'D$, tak $C$ leží zvonka $\Cal K$ (v~rovnakej polrovine určenej priamkou~$AB$ ako bod~$C'$), preto $|\angle ACB|<|\angle AC'B|=|\angle ABC'|=|\angle ABE'|<|\angle ABE|$ (lebo $E'$ leží medzi $A$ a~$E$), čo je opäť v~spore s~$|\angle ACB|=|\angle EBA|$.
\inspinsp{memo.5}{memo.4}

\ineriesenie
\podla{Karla Horáka{\rm,} Česká republika}
Z~daných rovností veľkostí uhlov vyplýva, že trojuholníky $ABD$ a~$CED$ sú podobné (\obr).
Z~toho okamžite dostávame, že aj trojuholníky $ACD$ a~$BED$ sú podobné (podľa $sus$; rovnako veľký uhol pri spoločnom vrchole~$D$ a~úmerné strany). Z~rovnosti uhlov $BED$ a~$ACD$ potom vyplýva, že súčet veľkostí troch uhlov $BCA$, $ACD$ a~$DCE$ je rovný súčtu veľkostí uhlov v~trojuholníku $ABE$, teda $E$, $C$ a~$B$ sú kolineárne.
}

{%%%%%   MEMO, priklad t4
Nech prvočíselný rozklad čísla $n$ je $n=p_1^{s_1}p_2^{s_2}\dots p_k^{s_k}$, pričom
$p_1$, \dots, $p_k$ sú rôzne prvočísla a~$s_i\ge1$ pre každé $i$. Predpokladajme,
že súčet všetkých kladných deliteľov čísla~$n$, ktorý možno vypočítať ako
$$
(1+p_1+p_1^2+\cdots+p_1^{s_1})
(1+p_2+p_2^2+\cdots+p_2^{s_2})\dots(1+p_k+p_k^2+\cdots+p_k^{s_k}),
$$
je mocninou dvoch. Potom každý z~činiteľov
$$
f_i=1+p_i+p_i^2+\cdots+p_i^{s_i}
$$
musí byť tiež mocninou dvoch väčšou ako $1$
a~teda $p_i$ aj $s_i$ sú nepárne. Ak $s_i>1$, tak
$$
f_i=(1+p_i)(1+p_i^2+p_i^4+\cdots+p_i^{s_i-1}).
$$
Keďže $f_i$ nemá žiadneho nepárneho deliteľa väčšieho ako $1$, nepárne celé číslo $s_i-1$ (o~ktorom predpokladáme, že je kladné) musí byť tvaru $4k+2$ a~preto vieme urobiť ďalší rozklad
$$
f_i=(1+p_i)(1+p_i^2)(1+p_i^4+p_i^8+\cdots+p_i^{s_i-3}).
$$
Obe čísla $1+p_i$ a~$1+p_i^2$ sú mocniny dvoch, teda $1+p_i\mid 1+p_i^2$, čo je v~spore s~rovnosťou $1+p_i^2=(1+p_i)(p_i-1)+2$ (keďže zrejme $1+p_i\nmid 2$).
Preto pre každé $i$ platí $s_i=1$ a~počet deliteľov čísla~$n$ je rovný $2^k$.

\poznamka
Uvedené riešenie možno ukončiť aj bez pozorovania, že $1+p_i$ a~$1+p_i^2$ nemôžu byť súčasne mocniny dvoch. Opakovaním postupných rozkladov na súčin dostaneme
$$
f_i=(1+p_i)\bigl(1+p_i^2\bigr)\bigl(1+p_i^4\bigr)\dots
\bigl(1+p_i^{2^{t_i}}\bigr),
$$
takže $s_i=2^{t_i+1}-1$ pre nejaké $t_i\ge0$ a~pre každé $i$ a~teda počet deliteľov čísla~$n$ je rovný
$2^{k+t_1+t_2+\cdots+t_k}$. (Z~uvedeného riešenia akurát navyše vieme, že $t_i=0$ pre každé~$i$.)
}
