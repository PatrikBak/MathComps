{%%%%%   A-I-1
Nájdite všetky také trojice reálnych čísel $a$, $b$, $c$, že
každá z~rovníc
$$
\align
x^3+(a+1)x^2+(b+3)x+(c+2)&=0, \\
x^3+(a+2)x^2+(b+1)x+(c+3)&=0, \\
x^3+(a+3)x^2+(b+2)x+(c+1)&=0
\endalign
$$
má v~obore reálnych čísel tri rôzne korene, spolu je to však
iba päť rôznych čísel.}
\podpis{Jaromír Šimša}

{%%%%%   A-I-2
V~rovine je daná úsečka~$AV$ a~ostrý uhol veľkosti~$\alpha$.
Určte množinu stredov kružníc opísaných všetkým tým trojuholníkom $ABC$
s~vnútorným uhlom~$\alpha$ pri vrchole~$A$, ktorých výšky sa
pretínajú v~bode~$V$.}
\podpis{Pavel Leischner}

{%%%%%   A-I-3
Množinu~$\mm M$ tvorí $2n$ rôznych kladných reálnych čísel,
pričom $n\ge2$. Uvažujme $n$~obdĺžnikov, ktorých rozmery
sú čísla z~množiny~$\mm M$, pričom každý prvok z~$\mm M$ je použitý
práve raz. Určte, aké rozmery majú tieto obdĺžniky, ak je
súčet ich obsahov
\ite a) najväčší možný;
\ite b) najmenší možný.}
\podpis{Jaroslav Švrček}

{%%%%%   A-I-4
Určte počet konečných rastúcich
postupností prirodzených čísel $a_1,a_2,\dots,a_k$ všetkých možných
dĺžok~$k$, pre ktoré platí $a_1=1$,
$a_i\mid a_{i+1}$ pre $i=1,2,\dots,k-1$ a~$a_{k}=969\,969$.}
\podpis{Martin Panák}

{%%%%%   A-I-5
Je daná kružnica~$k$, bod~$O$, ktorý na nej neleží,
a~priamka $p$, ktorá ju nepretína.
Uvažujme ľubovoľnú kružnicu~$l$, ktorá má vonkajší
dotyk s~kružnicou~$k$ a~dotýka sa aj priamky~$p$. Príslušné body
dotyku označme $A$ a~$B$. Pokiaľ body $O$, $A$, $B$
neležia na jednej priamke, zostrojíme kružnicu~$m$
opísanú trojuholníku $OAB$.
Dokážte, že všetky také kružnice~$m$ prechádzajú spoločným
bodom rôznym od bodu~$O$, alebo sa dotýkajú jednej priamky.}
\podpis{Ján Mazák}

{%%%%%   A-I-6
Dokážte, že pre každé prirodzené číslo~$n$ existuje celé číslo~$a$
($1<a<5^n$)
také, že platí $5^n\mid a^3-a+1$.}
\podpis{Ján Mazák}

{%%%%%   B-I-1
Nájdite všetky prirodzené čísla~$k$, pre ktoré je dekadický zápis
čísla $6^k\cdot7^{2007-k}$ ukončený dvojčíslím  a)~  $02$;   b)~  $04$.}
\podpis{Eva Řídká}

{%%%%%   B-I-2
V~páse medzi rovnobežkami $p$, $q$ sú dané dva rôzne body $M$ a~$N$.
Zostrojte kosoštvorec alebo štvorec, ktorého dve protiľahlé
strany ležia na priamkach $p$ a~$q$ a~zostávajúce dve strany
prechádzajú bodmi $M$ a~$N$ (každá jedným).}
\podpis{Jaromír Šimša}

{%%%%%   B-I-3
Dokážte, že ak $x$ a~$y$ sú reálne čísla, pre ktoré platí $x^3+y^3\le2$,
tak $x+y\le2$.}
\podpis{Ján Mazák}

{%%%%%   B-I-4
Nájdite všetky pravouhlé trojuholníky s~dĺžkami strán~$a$, $b$,
$c$ a~dĺžkami ťažníc $t_a$, $t_b$, $t_c$, pre ktoré platí
$a+t_a=b+t_b$. Uvažujte oba prípady, keď $AB$ je a)~prepona, b)~odvesna.}
\podpis{Pavel Novotný}

{%%%%%   B-I-5
Určte všetky dvojice $a$, $b$ reálnych čísel, pre ktoré má každá
z~kvadratických rovníc
$$
ax^2+2bx+1=0,\quad bx^2+2ax+1=0
$$
dva rôzne reálne korene, pričom práve jeden z~nich je spoločný obom rovniciam.}
\podpis{Jaroslav Švrček}

{%%%%%   B-I-6
Obdĺžnik $2\,005\times2\,007$ je rozdelený na čierne a~biele
jednotkové štvorčeky. Dokážte, že pre niektorú z~farieb (čiernu alebo bielu)
existuje viac ako
95\,800 pravouholníkov so šírkou aspoň~2, ktoré sa skladajú
z~jednotkových štvorčekov, navzájom sa neprekrývajú a~ktorých rohové
políčka majú túto farbu.}
\podpis{Pavel Leischner}

{%%%%%   C-I-1
Určte najmenšie prirodzené číslo~$n$, pre ktoré aj čísla
$\sqrt{2n}$, $\root3\of{3n}$, $\root5\of{5n}$
sú prirodzené.}
\podpis{Jaroslav Švrček}

{%%%%%   C-I-2
Štvoruholníku $ABCD$ je vpísaná kružnica so stredom~$S$.
Určte rozdiel $|\uhol ASD|-|\uhol CSD|$, ak $|\uhol ASB|-|\uhol BSC|=40\st$.}
\podpis{Jaromír Šimša}

{%%%%%   C-I-3
Máme určitý počet krabičiek a~určitý počet guľôčok. Ak dáme do
každej krabičky práve jednu guľôčku, ostane nám $n$~guľôčok. Keď však
necháme práve $n$~krabičiek bokom, môžeme všetky guľôčky
rozmiestniť tak, aby ich v~každej zostávajúcej krabičke bolo práve~$n$.
Koľko máme krabičiek a~koľko guľôčok?}
\podpis{Vojtech Bálint}

{%%%%%   C-I-4
Tangram je skladačka, ktorú možno vyrobiť z~papiera rozrezaním
vystrihnutého štvorca na sedem dielov podľa čiar vyznačených na
\insp{c.1}%
\obr. Predpokladajme, že dĺžka strany štvorca je
$2\sqrt2\cm$. Rozhodnite, či možno z~dielov tangramu zložiť:
\ite a) obdĺžnik $2\cm\times4\cm$,
\ite b) obdĺžnik $\sqrt2\cm\times4\sqrt2\cm$.}
\podpis{Pavel Leischner}

{%%%%%   C-I-5
V~skupine $n$~ľudí ($n\ge4$) sa niektorí poznajú.
Vzťah "poznať sa" je vzájomný: ak osoba~$A$ pozná
osobu~$B$, tak aj $B$ pozná $A$ a~nazývame ich dvojicou známych.
\ite a) Dokážte, že ak medzi každými štyrmi osobami sú aspoň štyri dvojice známych,
        tak každé dve osoby, ktoré sa nepoznajú, majú spoločného známeho.
\ite b) Zistite, pre ktoré $n\geq4$ existuje skupina osôb, v~ktorej sú
        medzi každými štyrmi osobami aspoň tri dvojice známych a~súčasne
        sa niektoré dve osoby ani nepoznajú, ani nemajú spoločného známeho.
\ite c) Rozhodnite, či v~skupine šiestich osôb môžu byť v~každej štvorici práve
        tri dvojice známych a~práve tri dvojice neznámych.

   }
\podpis{Ján Mazák}

{%%%%%   C-I-6
Klárka mala na papieri napísané trojciferné číslo. Keď ho správne
vynásobila deviatimi, dostala štvorciferné číslo, ktoré začínalo rovnakou
číslicou ako pôvodné číslo, prostredné dve číslice sa rovnali
a~posledná číslica bola súčtom číslic pôvodného čísla.
Ktoré štvorciferné číslo mohla Klárka dostať?}
\podpis{Peter Novotný}

{%%%%%   A-S-1
V~obore reálnych čísel riešte sústavu rovníc
$$
\align
x^2-y&=z^2,\\
y^2-z&=x^2,\\
z^2-x&=y^2.
\endalign
$$
}
\podpis{Ján Mazák}

{%%%%%   A-S-2
Podstavy hranola sú tvorené dvoma zhodnými konvexnými $n$-uholníkmi.
Počet $v$ vrcholov tohto telesa, počet $s$ jeho stenových
uhlopriečok a~počet $t$ jeho telesových uhlopriečok tvoria v~istom
poradí prvé tri členy aritmetickej postupnosti. Pre ktoré $n$ to
platí?

(Poznámka: Steny hranola sú bočné steny aj podstavy.
Telesová uhlopriečka je úsečka spájajúca
dva vrcholy hranola, ktoré neležia v~rovnakej stene.)}
\podpis{Vojtech Bálint}

{%%%%%   A-S-3
V~rovine je daný uhol $XSY$ a~kružnica~$k$ so stredom~$S$. Uvažujme
ľubovoľný trojuholník $ABC$ s~vpísanou kružnicou~$k$,
ktorého vrcholy $A$ a~$B$ ležia postupne na polpriamkach
$SX$ a~$SY$. Určte množinu vrcholov~$C$
všetkých takých trojuholníkov $ABC$.}
\podpis{Jaromír Šimša}

{%%%%%   A-II-1
Nájdite všetky štvorice $p$, $q$, $r$, $s$ navzájom rôznych reálnych čísel, pre ktoré sú $p$,~$q$ koreňmi rovnice
$$
x^2+rx+s-1=0
$$
a $r$, $s$ koreňmi rovnice
$$
px^2+p(q-1)x+12=0.
$$}
\podpis{Tomáš Jurík}

{%%%%%   A-II-2
V~tabuľke $n\times n$, pričom $n\ge2$, sú po riadkoch napísané všetky čísla $1,2,\dots,n^2$ v~tomto poradí (v~prvom riadku sú za sebou napísané čísla $1,2,\dots,n$, v~druhom riadku $n+1, n+2,\dots, 2n$, atď.). V~jednom kroku môžeme zvoliť ľubovoľné dve čísla na susedných políčkach (\tj. na takých, ktoré majú spoločnú stranu), a~ak je ich aritmetický priemer celé číslo, obe nahradíme týmto priemerom. Pre ktoré $n$ možno po konečnom počte krokov dostať tabuľku, v~ktorej sú všetky čísla rovnaké?}
\podpis{Peter Novotný}

{%%%%%   A-II-3
Daný je ostrouhlý trojuholník $ABC$ s~pätami výšok $D$, $E$, $F$ ležiacimi postupne na stranách $AB$, $BC$, $CA$.
Obraz bodu $F$ v~stredovej súmernosti podľa stredu strany~$AB$ leží na priamke~$DE$. Určte veľkosť uhla $BAC$.}
\podpis{Ján Mazák}

{%%%%%   A-II-4
Dokážte, že pre nezáporné reálne čísla $x$, $y$ spĺňajúce vzťah
$x^2+y^6=2$ platí
$$
x^2+2\ge 3xy.
$$}
\podpis{Ján Mazák}

{%%%%%   A-III-1
Určte koeficienty $p$, $q$, $r$ polynómu $f(x)=x^3+px^2+qx+r$, ak
viete, že sú to nenulové navzájom rôzne celé čísla a~že $f(p)=p^3$, $f(q)=q^3$.}
\podpis{Vojtech Bálint}

{%%%%%   A-III-2
V~ostrouhlom trojuholníku $ABC$, v~ktorom $|AC|\ne|BC|$, označme $D$ a~$E$ päty výšok z~vrcholov $A$ a~$B$. Nech $V$ je priesečník výšok trojuholníka $ABC$, bod~$F$ je priesečník priamok $AB$ a~$DE$ a~bod~$S$ je stred strany~$AB$. Ďalej nech $K$ je priesečník kružníc opísaných trojuholníkom $BDS$ a~$AES$ rôzny od bodu~$S$.
\ite a) Dokážte, že body $D$, $E$, $V$, $K$ ležia na jednej kružnici.
\ite b) Dokážte, že body $F$, $V$, $K$ ležia na jednej priamke.}
\podpis{Ján Mazák}

{%%%%%   A-III-3
V~tabuľke $n\times n$, pričom $n\ge2$, sú po riadkoch napísané všetky čísla $1,2,\dots,n^2$ v~tomto poradí (v~prvom riadku sú za sebou napísané čísla $1,2,\dots,n$, v~druhom riadku $n+1, n+2,\dots, 2n$, atď.). V~jednom kroku môžeme zvoliť ľubovoľné dve čísla na susedných políčkach (\tj. na takých, ktoré majú spoločnú stranu) a~buď obidve súčasne zväčšiť o~$1$ alebo obidve súčasne zmenšiť o~$1$. Pre ktoré $n$ možno po konečnom počte krokov dostať tabuľku, v~ktorej sú všetky čísla rovné $365$?}
\podpis{Peter Novotný}

{%%%%%   A-III-4
Dokážte, že pre žiadne prirodzené číslo $n$ nie je číslo $27^n-n^{27}$ prvočíslom.}
\podpis{Ján Mazák}

{%%%%%   A-III-5
Nech $x$, $y$, $z$ sú kladné reálne čísla, ktorých súčin je $1$. Dokážte, že ak $k$, $m$ sú kladné celé
čísla, pričom $k>m$, tak
$$
x^k+y^k+z^k \ge x^m + y^m + z^m.
$$}
\podpis{Pavel Novotný}

{%%%%%   A-III-6
Označme zvyčajným spôsobom dĺžky strán a~ťažníc daného trojuholníka. Nájdite všetky možné hodnoty výrazu
$$
\text{a)}\quad\frac{t_a^2-t_b^2}{b^2-a^2};\qquad\qquad
\text{b)}\quad\frac{t_a-t_b}{b-a}.
$$}
\podpis{Pavel Novotný}

{%%%%%   B-S-1
Keď ľubovoľné prvočíslo vydelíme tridsiatimi,
bude zvyškom číslo~$1$ alebo prvočíslo. Dokážte.}
\podpis{Vojtech Bálint}

{%%%%%   B-S-2
Určte všetky dvojice $(a,b)$ reálnych čísel, pre ktoré majú rovnice
$$
x^2+(3a+b)x+4a=0,\quad x^2+(3b+a)x+4b=0
$$
spoločný reálny koreň.}
\podpis{Jaroslav Švrček}

{%%%%%   B-S-3
V~rovine sú dané dve rovnobežky $p$ a~$q$, bod~$A$
na priamke~$p$ a~bod~$M$ ležiaci vnútri
pásu medzi priamkami $p$ a~$q$. Zostrojte
kosoštvorec alebo štvorec $ABCD$ tak, aby strana~$AB$ ležala na priamke~$p$, strana~$CD$ na
priamke~$q$ a~aby uhlopriečka~$BD$ prechádzala
bodom~$M$.}
\podpis{Jaromír Šimša}

{%%%%%   B-II-1
Uvažujme dve kvadratické rovnice
$$
x^2-ax-b=0,\quad x^2-bx-a=0
$$
s~reálnymi parametrami $a$, $b$. Zistite, akú najmenšiu a~akú najväčšiu hodnotu môže nadobudnúť súčet $a+b$, ak existuje práve jedno reálne číslo~$x$, ktoré súčasne vyhovuje obom rovniciam. Určte ďalej všetky dvojice $(a,b)$ reálnych parametrov, pre ktoré tento súčet tieto hodnoty nadobúda.}
\podpis{Jaroslav Švrček}

{%%%%%   B-II-2
V~trojuholníku $ABC$ má uhol $\alpha$ veľkosť $20^\circ$. Vypočítajte veľkosti uhlov $\beta$ a~$\gamma$, ak platí rovnosť $a+2v_a=b+2v_b$.}
\podpis{Pavel Novotný}

{%%%%%   B-II-3
V~rovine je daný rovnobežník $ABCD$, ktorého uhlopriečka~$BD$ je kolmá na stranu~$AD$. Označme $M$ $(M\ne A)$ priesečník priamky~$AC$ s~kružnicou majúcou priemer~$AD$. Dokážte, že os úsečky~$BM$ prechádza stredom strany~$CD$.}
\podpis{Jaroslav Švrček}

{%%%%%   B-II-4
Hokejový turnaj sa hrá systémom "každý s~každým". V~priebehu turnaja sa každá dvojica družstiev stretne práve raz. Turnaj sa odohráva po jednotlivých kolách. Pri párnom počte družstiev odohrá každé v~jednom kole jedno stretnutie, pri nepárnom počte má v~každom kole jedno družstvo voľno. Za remízu dostane každý zo súperov po jednom bode. Ak sa stretnutie neskončí remízou, dostane víťaz dva body, porazený nezíska žiadny bod. O~poradí v~tabuľke rozhoduje predovšetkým počet bodov, pri rovnosti bodov potom skóre. Po odohratí niekoľkých kôl nemala žiadna dvojica družstiev ten istý počet bodov. Dokážte, že v~tom prípade už posledný v~tabuľke stratil nádej na celkové prvenstvo. Úlohu riešte pre turnaj
\ite a) desiatich družstiev,
\ite b) jedenástich družstiev.}
\podpis{Martin Panák}

{%%%%%   C-S-1
Nájdite všetky dvojice prirodzených čísel
$a$, $b$ väčších ako~$1$ tak, aby ich
súčet aj súčin boli mocniny prvočísel.}
\podpis{Ján Mazák}

{%%%%%   C-S-2
V~danom rovnobežníku $ABCD$ je bod~$E$ stred
strany~$BC$ a~bod~$F$ leží vnútri strany~$AB$.
Obsah trojuholníka $AFD$ je $15\cm^2$
a~obsah trojuholníka $FBE$ je $14\cm^2$.
Určte obsah štvoruholníka $FECD$.}
\podpis{Peter Novotný}

{%%%%%   C-S-3
V~skupine šiestich ľudí existuje
práve 11~dvojíc známych. Vzťah "poznať sa" je
vzájomný, \tj.~ak osoba~$A$ pozná osobu~$B$, tak aj $B$ pozná $A$.
Keď sa ktokoľvek zo skupiny dozvie
nejakú správu, povie ju všetkým svojim známym.
Dokážte, že sa týmto spôsobom nakoniec správu dozvedia
všetci.}
\podpis{Vojtech Bálint}

{%%%%%   C-II-1
Trojuholník $ABC$ spĺňa pri zvyčajnom
označení dĺžok strán podmienku $a\le b\le c$.
Vpísaná kružnica sa dotýka strán $AB$,
$BC$ a~$AC$ postupne v~bodoch $K$,
$L$ a~$M$. Dokážte, že z~úsečiek
$AK$, $BL$ a~$CM$ možno zostrojiť trojuholník
práve vtedy, keď platí $b+c<3a$.}
\podpis{Jaroslav Švrček}

{%%%%%   C-II-2
Klárka urobila chybu pri písomnom násobení dvoch
dvojciferných čísel, a~tak jej vyšlo číslo o~$400$
menšie, ako bol správny výsledok. Pre kontrolu
vydelila číslo, ktoré dostala, menším
z~násobených čísel. Tentoraz počítala správne
a~vyšiel jej neúplný podiel $67$ a~zvyšok~$56$. Ktoré
čísla Klárka násobila?}
\podpis{Jaromír Šimša}

{%%%%%   C-II-3
Dokážte, že pokiaľ v~skupine
šiestich osôb existuje aspoň desať dvojíc známych, tak v~nej
možno nájsť tri osoby, ktoré sa poznajú navzájom.
Vzťah "poznať sa" je
vzájomný, \tj. ak osoba~$A$ pozná osobu~$B$, tak aj $B$ pozná $A$. Ukážte, že taká
trojica existovať nemusí, ak v~skupine šiestich osôb
je menej ako desať dvojíc známych.}
\podpis{Vojtech Bálint}

{%%%%%   C-II-4
Nájdite všetky trojice celých čísel $x$, $y$, $z$, pre
ktoré platí
$$
x+y\sqrt3+z\sqrt7=y+z\sqrt3+x\sqrt7.
$$}
\podpis{Ján Mazák}

{%%%%%   vyberko, den 1, priklad 1
Body v~rovine s~celočíselnými súradnicami sú ofarbené troma farbami, pričom každá farba je použitá aspoň raz. Dokážte, že vieme nájsť pravouhlý trojuholník, ktorého vrcholy majú celočíselné súradnice a~sú rôznych farieb.}
\podpis{Ondrej Budáč:niekde z~mathlinksu}

{%%%%%   vyberko, den 1, priklad 2
Nájdite všetky štvorice $k$, $l$, $m$, $n$ prirodzených čísel, ktoré spĺňajú
$$
(1+n^k)^l=1+n^m.
$$}
\podpis{Ondrej Budáč:Moskovská olympiáda, 1999}

{%%%%%   vyberko, den 1, priklad 3
V~trojuholníku $ABC$ označme $P$ priesečník osi uhla $BAC$ so stranou~$BC$ a~$Q$ priesečník osi uhla $ABC$ so stranou~$AC$. Označme $M$ priesečník osi uhla $BAC$ a~kružnice opísanej trojuholníku $ABC$ (rôzny od $A$) a~$N$ priesečník osi uhla $ABC$ a~kružnice opísanej $ABC$ (rôzny od $B$). Ďalej na priamke $AB$ zvoľme body $D$, $E$ tak, aby $D$ ležal na polpriamke opačnej k~$AB$ a~$E$ na polpriamke opačnej k~$BA$ a~zároveň $|AD|=|AC|$ a~$|BE|=|BC|$. Ďalej nech $U$ je stred kružnice opísanej trojuholníku $BEM$ a~$V$ je stred kružnice opísanej trojuholníku $ADN$. Označme $X$ priesečník $AU$ a~$BV$. Ukážte, že $CX$ je kolmá na $PQ$.}
\podpis{Ondrej Budáč:srbské CK, 2008}

{%%%%%   vyberko, den 2, priklad 1
Máme daných päť reálnych čísel. Urobíme rozdiel súčtu ľubovoľných troch z~nich a~súčtu zvyšných dvoch. Tento rozdiel je vždy kladné číslo. Dokážte, že súčin všetkých takýchto desiatich rozdielov je nanajvýš súčin druhých mocnín týchto piatich čísel.}
\podpis{Michal Prusák, Michal Takács:Estónsko 2000}

{%%%%%   vyberko, den 2, priklad 2
Nech $b$, $n$ sú prirodzené čísla väčšie ako~$1$. Pre každé $k>1$ existuje celé číslo~$a_k$ také, že $b-a_k^n$ je deliteľné číslom~$k$. Dokážte, že $b$ je $n$-tou mocninou celého čísla.
%{\it Zadanie bude zverejnené po IMO 2008.}
}
\podpis{Michal Prusák, Michal Takács:Shortlist 2007, N2}

{%%%%%   vyberko, den 2, priklad 3
Daný je rovnoramenný trojuholník $ABC$ so základňou~$BC$. Bod~$M$ je stredom strany~$BC$. Nech $X$ je bod na kratšom oblúku~$MA$ kružnice opísanej trojuholníku $ABM$. Nech $T$ je bod v~časti roviny určenej uhlom $BMA$ taký, že $|\uhol TMX|=90^\circ$ a~$|TX|=|BX|$. Dokážte, že $|\uhol MTB|-|\uhol CTM|$ nezáleží na voľbe bodu~$X$.
%{\it Zadanie bude zverejnené po IMO 2008.}
}
\podpis{Michal Prusák, Michal Takács:Shortlist 2007, G2}

{%%%%%   vyberko, den 2, priklad 4
Majme $3n$ čísel, označme ich $a_1,a_2,\dots,a_{3n}$. Po {\it pozoruhodnom\/} premiešaní budú čísla v~poradí
$$
a_3,a_6,\dots,a_{3n},a_2,a_5,\dots,a_{3n-1},a_1,a_4,\dots,a_{3n-2}.
$$
Začnime s~číslami $1,2,3,\dots,192$. Môžeme po konečnom počte pozoruhodných premiešaní dostať čísla v~poradí $192,191,190,\dots,1$?}
\podpis{Michal Prusák, Michal Takács:Japonsko 2000}

{%%%%%   vyberko, den 3, priklad 1
Uhlopriečky daného lichobežníka $ABCD$ sa pretínajú v~bode~$P$. Bod~$Q$ leží medzi rovnobežnými priamkami $BC$ a~$AD$ tak, že $|\uhol AQD|=|\uhol CQB|$ a~body $P$ a~$Q$ ležia v~opačných polrovinách určených priamkou~$CD$. Dokážte, že $|\uhol BQP|=|\uhol DAQ|$.
%{\it Zadanie bude zverejnené po IMO 2008.}
}
\podpis{Martin Potočný:Shortlist 2007, G3}

{%%%%%   vyberko, den 3, priklad 2
Nech $S$ je konečná množina bodov v~rovine takých, že žiadne tri z~nich neležia na jednej priamke. Pre každý konvexný mnohouholník~$P$, ktorého vrcholy patria do $S$, nech $a(P)$ je počet jeho vrcholov a~nech $b(P)$ je počet bodov z~$S$ ležiacich zvonku~$P$. Dokážte, že pre každé reálne číslo~$x$ platí
$$
\sum_P x^{a(P)}(1-x)^{b(P)}=1,
$$
kde suma sa berie cez všetky konvexné mnohouholníky s~vrcholmi v~$S$.

{\it Poznámka.}
Úsečka, bod a prázdna množina sa považujú za konvexný dvoj-, jedno- a~nulauholník.}
\podpis{Martin Potočný:Shortlist 2006, C3}

{%%%%%   vyberko, den 3, priklad 3
V~trojuholníku $ABC$ platí $|\uhol ACB|<|\uhol BAC|<\pi/2$ a~bod~$D$ leží na strane~$AC$ tak, že $|BD|=|BA|$. Kružnica vpísaná trojuholníku $ABC$ sa dotýka strany~$AB$ v~bode~$K$ a~strany~$AC$ v~bode~$L$. Bod~$J$ je stred kružnice vpísanej trojuholníku $BCD$. Dokážte, že priamka~$KL$ rozpoľuje úsečku~$AJ$.
}
\podpis{Martin Potočný:Shortlist 2006, G4}

{%%%%%   vyberko, den 3, priklad 4
Uvažujme funkcie $f\colon\Bbb N\to\Bbb N$ vyhovujúce podmienke
$$
f(m+n)\geq f(m)+f(f(n))-1
$$
pre všetky $m,n\in\Bbb N$. Nájdite všetky možné hodnoty $f(2007)$.
%{\it Zadanie bude zverejnené po IMO 2008.}
}
\podpis{Martin Potočný:Shortlist 2007, A2}

{%%%%%   vyberko, den 4, priklad 1
Nech $ABCD$ je rovnobežník, pričom žiadny z~jeho vnútorných uhlov nemá veľkosť $60^\circ$. Nájdite všetky dvojice bodov $E$ a~$F$ také, že trojuholníky $AEB$ a~$BFC$ sú rovnoramenné so základňami $AB$ a~$BC$ a~trojuholník $DEF$ je rovnostranný.}
\podpis{Hana Budáčová:rumunské výberové sústredenie 2007, úloha 20}

{%%%%%   vyberko, den 4, priklad 2
Vo vrcholoch konvexného mnohouholníka s~párnym počtom strán sedia poľovníci. Vnútri mnohouholníka mimo jeho uhlopriečok sa nachádza líška. Poľovníci naraz vystrelia smerom na líšku, ale líška sa uhne a~guľky z~pušiek letia ďalej a~preletia cez strany mnohouholníka. Dokážte, že aspoň jednu stranu netrafí žiadna guľka.}
\podpis{Hana Budáčová:rumunské výberové sústredenie 2007, úloha 1}

{%%%%%   vyberko, den 4, priklad 3
Nájdite všetky funkcie $f\colon\Bbb Q\to\Bbb R$, ktoré pre všetky racionálne čísla $x$ a~$y$ spĺňajú nerovnosť
$$
|f(x)-f(y)|\le (x-y)^2.
$$}
\podpis{Hana Budáčová:rumunské výberové sústredenie 2007, úloha 3}

{%%%%%   vyberko, den 4, priklad 4
Nech $P$ je množina všetkých prvočísel. Nech $M$ je podmnožina množiny~$P$, ktorá má aspoň tri prvky a~navyše pre každú vlastnú konečnú podmnožinu~$A$ množiny~$M$ patria všetky prvočísla z~prvočíselného rozkladu čísla
$$
-1 + \prod_{p\in A} p
$$
do množiny~$M$. Dokážte, že $M=P$.}
\podpis{Hana Budáčová:rumunské výberové sústredenie 2003, úloha 10}

{%%%%%   vyberko, den 5, priklad 1
Polynóm~$P$ nepárneho stupňa spĺňa pre každé reálne číslo~$x$ rovnosť
$$
P(x^2-1)=P(x)^2-1.
$$
Dokážte, že $P(x)=x$ pre každé reálne číslo $x$.}
\podpis{Peter Novotný:Poľsko 2000 (Problems and Solutions From Around the World, Problem 6)}

{%%%%%   vyberko, den 5, priklad 2
Nech $X$ je množina $10\,000$ rôznych celých čísel, z~ktorých žiadne nie je deliteľné číslom~$47$. Dokážte, že existuje jej $2008$-prvková podmnožina $Y$ taká, že číslo $a-b+c-d+e$ nie je deliteľné číslom~$47$ pre žiadnu päticu (nie nutne rôznych) čísel $a,b,c,d,e\in Y$.
%{\it Zadanie bude zverejnené po IMO 2008.}
}
\podpis{Peter Novotný:Shortlist 2007, N3}

{%%%%%   vyberko, den 5, priklad 3
Daný je trojuholník $ABC$. Kružnica pripísaná ku strane~$BC$ má stred~$J$ a~dotýka sa strany~$BC$ v~bode~$A_1$ a~priamok $AC$, $AB$ postupne v~bodoch $B_1$, $C_1$. Predpokladajme, že priamky $A_1B_1$ a~$AB$ sú navzájom kolmé a~pretínajú sa v~bode~$D$. Nech $E$ je päta kolmice spustenej z~bodu~$C_1$ na priamku~$DJ$. Určte veľkosti uhlov $BEA_1$ a~$AEB_1$.}
\podpis{Peter Novotný:Shortlist 2006, G5}

{%%%%%   vyberko, den 1, priklad 4
...}
\podpis{...}

{%%%%%   vyberko, den 5, priklad 4
...}
\podpis{...}

{%%%%%   trojstretnutie, priklad 1
Určte všetky trojice $(x,y,z)$ kladných reálnych čísel, ktoré sú riešením sústavy rovníc
$$
\eqalign{2x^3&=2y(x^2+1)-(z^2+1),\cr
2y^4&=3z(y^2+1)-2(x^2+1),\cr
2z^5&=4x(z^2+1)-3(y^2+1).\cr}
$$}
\podpis{Adam Osekowski}

{%%%%%   trojstretnutie, priklad 2
Daný je konvexný šesťuholník $ABCDEF$, pričom $|\uhol FAB|=|\uhol BCD|=|\uhol DEF|$ a~$|AB|=|BC|$, $|CD|=|DE|$, $|EF|=|FA|$. Dokážte, že priamky $AD$, $BE$ a~$CF$ sa pretínajú v~jednom bode.}
\podpis{Waldemar Pompe}

{%%%%%   trojstretnutie, priklad 3
Nájdite všetky prvočísla~$p$, pre ktoré je číslo
$$
\binom{p}{1}^2+\binom{p}{2}^2+\cdots+\binom{p}{p-1}^2
$$
deliteľné číslom~$p^3$.}
\podpis{Jaros\l aw Wr\'oblewski}

{%%%%%   trojstretnutie, priklad 4
Dokážte, že existuje také prirodzené číslo~$n$, že číslo $k^2+k+n$ nemá žiadneho prvočíselného deliteľa menšieho ako $2008$ pre žiadne celé číslo~$k$.}
\podpis{Jaros\l aw Wr\'oblewski}

{%%%%%   trojstretnutie, priklad 5
Daný je pravidelný päťuholník $ABCDE$. Určte najmenšiu hodnotu výrazu
$$
\frac{|PA|+|PB|}{|PC|+|PD|+|PE|},
$$
pričom $P$ je ľubovoľný bod ležiaci v~rovine päťuholníka $ABCDE$.}
\podpis{Waldemar Pompe}

{%%%%%   trojstretnutie, priklad 6
Nájdite všetky trojice $(k,m,n)$ prirodzených čísel majúce nasledujúcu vlastnosť:
Štvorec s~dĺžkou strany~$m$ sa dá rozdeliť na niekoľko pravouholníkov s~rozmermi $1\times k$ a~práve jeden štvorec s~dĺžkou strany~$n$.}
\podpis{Jaros\l aw Wr\'oblewski}

{%%%%%   IMO, priklad 1
V~ostrouhlom trojuholníku $ABC$ označme $H$ priesečník jeho výšok. Kružnica so stredom v~strede strany~$BC$ prechádzajúca bodom~$H$ pretína priamku~$BC$ v~bodoch $A_1$ a~$A_2$. Podobne kružnica so stredom v~strede strany~$CA$ predchádzajúca bodom~$H$ pretína priamku $CA$ v~bodoch $B_1$ a~$B_2$ a~kružnica so stredom v~strede strany~$AB$ predchádzajúca bodom~$H$ pretína priamku $AB$ v~bodoch $C_1$ a~$C_2$. Dokážte, že body $A_1$, $A_2$, $B_1$, $B_2$, $C_1$, $C_2$ ležia na jednej kružnici.}
\podpis{Rusko}

{%%%%%   IMO, priklad 2
\ite a) Dokážte, že
$$
\frac{x^2}{(x-1)^2} + \frac{y^2}{(y-1)^2} + \frac{z^2}{(z-1)^2} \ge 1
$$
pre všetky reálne čísla $x$, $y$, $z$ rôzne od $1$ spĺňajúce $xyz = 1$.
\ite b) Dokážte, že v~uvedenej nerovnosti platí rovnosť pre nekonečne veľa trojíc racionálnych čísel $x$, $y$, $z$ rôznych od $1$ spĺňajúcich $xyz = 1$.}
\podpis{Rakúsko}

{%%%%%   IMO, priklad 3
Dokážte, že existuje nekonečne veľa kladných celých čísel $n$ takých, že $n^2+1$ má prvočíselného deliteľa väčšieho ako $2n+\sqrt{2n}$.}
\podpis{Litva}

{%%%%%   IMO, priklad 4
Nájdite všetky funkcie $f\colon(0,\infty) \to (0,\infty)$ (\tj. funkcie z~kladných reálnych čísel do kladných reálnych čísel) také, že
$$
\frac{f^2(w)+f^2(x)}{f(y^2)+f(z^2)} = \frac{w^2 + x^2}{y^2 + z^2}
$$
pre všetky kladné reálne čísla $w$, $x$, $y$, $z$ spĺňajúce $wx=yz$.}
\podpis{Južná Kórea}

{%%%%%   IMO, priklad 5
Nech $n$ a~$k$ sú kladné celé čísla, kde $k\ge n$ a~$k-n$ je párne číslo. Daných je $2n$ lámp označených $1,2,\dots,2n$, pričom každá z~nich môže byť buď {\it zapnutá\/} alebo {\it vypnutá}. Na začiatku sú všetky lampy vypnuté. Uvažujeme postupnosti {\it krokov\/}: v~každom kroku jednu z~lámp prepneme (zo zapnutej na vypnutú alebo z~vypnutej na zapnutú).

Nech $N$ je počet takých postupností pozostávajúcich z~$k$~krokov, ktoré vedú do stavu, že všetky lampy od $1$ po $n$ sú zapnuté a~všetky lampy od $n+1$ po $2n$ sú vypnuté.

Nech $M$ je počet takých postupností pozostávajúcich z~$k$~krokov, ktoré vedú do stavu, že všetky lampy od $1$ po $n$ sú zapnuté a~všetky lampy od $n+1$ po $2n$ sú vypnuté, pričom žiadna z~lámp od $n+1$ po $2n$ nebola nikdy zapnutá.

Určte podiel $N/M$.}
\podpis{Francúzsko}

{%%%%%   IMO, priklad 6
Nech $ABCD$ je konvexný štvoruholník, pričom $|BA|\ne|BC|$. Označme postupne $\omega_1$ a~$\omega_2$ kružnice vpísané do trojuholníkov $ABC$ a~$ADC$. Predpokladajme, že existuje kružnica~$\omega$ dotýkajúca sa polpriamky~$BA$ za bodom~$A$ a~polpriamky~$BC$ za bodom~$C$, ktorá sa dotýka aj priamok $AD$ a~$CD$. Dokážte, že spoločné vonkajšie dotyčnice kružníc $\omega_1$ a~$\omega_2$ sa pretínajú na kružnici~$\omega$.}
\podpis{Rusko}

{%%%%%   MEMO, priklad 1
Nech $(a_n)_{n=1}^\infty$ je rastúca postupnosť celých kladných čísel s~nasledovnou vlastnosťou: pre každú štvoricu indexov $(i,j,k,l)$, kde $1\le i<j\le k<l$ a~$i+l=j+k$, platí nerovnosť $a_i+a_l>a_j+a_k$. Určte najmenšiu možnú hodnotu čísla $a_{2008}$.}
\podpis{Rakúsko}

{%%%%%   MEMO, priklad 2
Uvažujme šachovnicu $n\times n$, kde $n>1$. Koľkými spôsobmi môžeme vybrať $2n-2$ políčok tejto šachovnice tak, aby spojnica stredov žiadnych dvoch vybraných políčok nebola rovnobežná so žiadnou diagonálou šachovnice?}
\podpis{Švajčiarsko}

{%%%%%   MEMO, priklad 3
Nech $ABC$ je rovnoramenný trojuholník s~ramenami $AC$ a~$BC$. Kružnica vpísaná tomuto trojuholníku sa dotýka strany $AB$ v~bode~$D$ a~strany~$BC$ v~bode~$E$. Priamka rôzna od $AE$ prechádza bodom~$A$ a~pretína vpísanú kružnicu v~bodoch $F$ a~$G$. Priamky $EF$ a~$EG$ pretínajú priamku~$AB$ v~bodoch $K$ a~$L$. Dokážte, že platí rovnosť $|DK|=|DL|$.}
\podpis{Maďarsko}

{%%%%%   MEMO, priklad 4
Nájdite všetky také celé čísla $k$, že čísla $4n+1$ a~$kn+1$ sú nesúdeliteľné pre každé celé číslo~$n$.}
\podpis{Maďarsko}

{%%%%%   MEMO, priklad t1
Nájdite všetky funkcie $f\colon \Bbb R \to \Bbb R$, pre ktoré platí
$$
xf(x+xy)=xf(x)+f(x^2)f(y)
$$
pre všetky reálne čísla $x$, $y$.}
\podpis{Švajčiarsko}

{%%%%%   MEMO, priklad t2
Na tabuli je napísaných $n$ celých kladných čísel, pričom $n\ge2$. V~jednom kroku vyberieme dve z~napísaných čísel a~každé z~nich nahradíme ich súčtom.
Určte všetky hodnoty~$n$, pre ktoré môžeme z~akejkoľvek začiatočnej $n$-tice prirodzených čísel po konečnom počte krokov dostať $n$-ticu rovnakých čísel.}
\podpis{Slovensko, Peter Novotný}

{%%%%%   MEMO, priklad t3
Nech $ABC$ je ostrouhlý trojuholník. Bod~$E$ leží v~opačnej polrovine s~hraničnou priamkou~$AC$ ako bod $B$ a~$D$ je vnútorný bod úsečky~$AE$. Nech $|\uhol ADB|=|\uhol CDE|$, $|\uhol BAD|=|\uhol ECD|$ a~$|\uhol ACB|=|\uhol EBA|$. Dokážte, že body $B$, $C$ a~$E$ sú kolineárne.}
\podpis{Slovinsko}

{%%%%%   MEMO, priklad t4
Nech súčet všetkých kladných deliteľov celého kladného čísla~$n$ je mocninou čísla~$2$. Dokážte, že aj počet týchto deliteľov je mocninou čísla~$2$.}
\podpis{Česká rep.} 