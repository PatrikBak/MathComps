{%%%%%   Z4-I-1
...}
\podpis{...}

{%%%%%   Z4-I-2
...}
\podpis{...}

{%%%%%   Z4-I-3
...}
\podpis{...}

{%%%%%   Z4-I-4
...}
\podpis{...}

{%%%%%   Z4-I-5
...}
\podpis{...}

{%%%%%   Z4-I-6
...}
\podpis{...}

{%%%%%   Z5-I-1
...}
\podpis{...}

{%%%%%   Z5-I-2
...}
\podpis{...}

{%%%%%   Z5-I-3
...}
\podpis{...}

{%%%%%   Z5-I-4
Jarda vystřihl z~rohu šachovnice následující útvar:
\insp{z5-I-4.eps}%

Následně odstřihl několik dalších polí, a~to tak, že výsledný útvar
neobsahoval díry a~nerozpadal se,
měl stejný počet černých a~bílých polí
a~měl největší možný obsah.
Navíc zjistil, že ze všech možných útvarů s~těmito vlastnostmi měl ten jeho největší možný obvod.

Která pole Jarda dodatečně odstřihl?
Určete všechny možnosti.
}
\podpis{M. Petrová}

{%%%%%   Z5-I-5
...}
\podpis{...}

{%%%%%   Z5-I-6
...}
\podpis{...}

{%%%%%   Z6-I-1
...}
\podpis{...}

{%%%%%   Z6-I-2
...}
\podpis{...}

{%%%%%   Z6-I-3
...}
\podpis{...}

{%%%%%   Z6-I-4
...}
\podpis{...}

{%%%%%   Z6-I-5
...}
\podpis{...}

{%%%%%   Z6-I-6
...}
\podpis{...}

{%%%%%   Z7-I-1
...}
\podpis{...}

{%%%%%   Z7-I-2
...}
\podpis{...}

{%%%%%   Z7-I-3
Na stranách trojúhelníku $ABC$ jsou dány body $D$, $E$, $F$, $G$, viz obrázek.
Přitom platí, že čtyřúhelník $DEFG$ je kosočtverec a~úsečky $AD$, $DE$ a~$EB$ jsou navzájem shodné.

Určete velikost úhlu $ACB$.
\insp{z7-I-3.eps}%
}
\podpis{I. Jančigová}

{%%%%%   Z7-I-4
...}
\podpis{...}

{%%%%%   Z7-I-5
...}
\podpis{...}

{%%%%%   Z7-I-6
...}
\podpis{...}

{%%%%%   Z8-I-1
...}
\podpis{...}

{%%%%%   Z8-I-2
...}
\podpis{...}

{%%%%%   Z8-I-3
...}
\podpis{...}

{%%%%%   Z8-I-4
...}
\podpis{...}

{%%%%%   Z8-I-5
...}
\podpis{...}

{%%%%%   Z8-I-6
...}
\podpis{...}

{%%%%%   Z9-I-1
...}
\podpis{...}

{%%%%%   Z9-I-2
...}
\podpis{...}

{%%%%%   Z9-I-3
Je dán pravidelný trojboký hranol s~podstavnou hranou délky 3,2\,cm a~výškou 5\,cm.
Jeho plášť omotáváme šachovnicovou fólií, která sestává z~neprůhledných a~průhledných čtvercových polí se stranami délky 1\,cm.
Začátek fólie lícuje s~hranou hranolu (viz obrázek) a~vystačí právě na dvojí omotání celého pláště.

Kolik procent pláště hranolu bude přes fólii po omotání vidět?
Tloušťku fólie zanedbejte.
\insp{z9-I-3.eps}%
}
\podpis{K. Pazourek}

{%%%%%   Z9-I-4
...}
\podpis{...}

{%%%%%   Z9-I-5
...}
\podpis{...}

{%%%%%   Z9-I-6
...}
\podpis{...}

{%%%%%   Z4-II-1
...}
\podpis{...}

{%%%%%   Z4-II-2
...}
\podpis{...}

{%%%%%   Z4-II-3
...}
\podpis{...}

{%%%%% Z5-II-1
V~obchode majú jeden druh lízaniek a~jeden druh nanukov.
Cena ako lízaniek, tak nanukov je uvedená v~celých grošoch.

Barborka kúpila tri lízanky.
Eliška kúpila štyri lízanky a~niekoľko nanukov -- vieme len, že to bolo viac ako jeden a~menej ako desať nanukov.
Janko kúpil jednu lízanku a~jeden nanuk.
Barborka platila 24 grošov a~Eliška 109 grošov.

Koľko grošov platil Janko?
}
\podpis{Libuše Hozová}

{%%%%% Z5-II-2
Xénia mala obdĺžnik s~rozmermi 24\,cm a~30\,cm.
Rozdelila ho tromi úsečkami na niekoľko rovnakých obdĺžnikových dielov.

Aké mohli byť rozmery týchto dielov? Určte štyri možnosti.
}
\podpis{Karel Pazourek}

{%%%%% Z5-II-3
Vojto sa snaží uložiť svojich 20 hračiek do škatúľ tak, aby v~každej škatuli bola aspoň jedna hračka a~v~žiadnych dvoch škatuliach nebol rovnaký počet hračiek.
\begin{enumerate}\alphatrue
\item Opíšte, ako môže hračky uložiť do piatich škatúľ.
\item Môže hračky uložiť do šiestich škatúľ?
\end{enumerate}
}
\podpis{Josef Tkadlec}

{%%%%%   Z6-II-1
...}
\podpis{...}

{%%%%%   Z6-II-2
...}
\podpis{...}

{%%%%%   Z6-II-3
...}
\podpis{...}

{%%%%%   Z7-II-1
...}
\podpis{...}

{%%%%%   Z7-II-2
...}
\podpis{...}

{%%%%%   Z7-II-3
...}
\podpis{...}

{%%%%%   Z8-II-1
...}
\podpis{...}

{%%%%%   Z8-II-2
...}
\podpis{...}

{%%%%%   Z8-II-3
...}
\podpis{...}

{%%%%% Z9-II-1
Trojciferné číslo má ciferný súčet 16.
Ak v~tomto čísle zameníme cifry na miestach stoviek a~desiatok, číslo sa o~360 zmenší.
Ak v~pôvodnom čísle zameníme cifry na miestach desiatok a~jednotiek, číslo sa o~54 zväčší.

Nájdite ono trojciferné číslo.}
\podpis{Libuše Hozová}

{%%%%% Z9-II-2
Deltoid $ABCD$ je súmerný podľa uhlopriečky $AC$.
Dĺžka $AC$ je 12\,cm, dĺžka $BC$ je 6\,cm a~vnútorný uhol pri vrchole~$B$ je pravý.
Na stranách $AB$, $AD$ sú dané body $E$, $F$ tak, že trojuholník $ECF$ je rovnostranný.

Určte dĺžku úsečky $EF$.
}
\podpis{Karel Pazourek}

{%%%%% Z9-II-3
Ľudovít si pri istom príklade na delenie všimol, že keď delenec zdvojnásobí a~deliteľ zväčší o~12, dostane svoje obľúbené číslo. To isté číslo by dostal, aj keby pôvodný delenec zmenšil o~42 a~pôvodný deliteľ zmenšil na polovicu.

Určte Ľudovítovo obľúbené číslo.
}
\podpis{Michaela Petrová}

{%%%%% Z9-II-4
V~konvexnom štvoruholníku $ABCD$ platí, že $E$ je priesečníkom uhlopriečok, trojuholníky $ADE$, $BCE$, $CDE$ majú postupne obsahy 12\,cm$^2$, 45\,cm$^2$, 18\,cm$^2$ a~dĺžka strany $AB$ je 7\,cm.

Určte vzdialenosť bodu $D$ od priamky $AB$.}
\podpis{Michaela Petrová}

{%%%%%   Z9-III-1
...}
\podpis{...}

{%%%%%   Z9-III-2
...}
\podpis{...}

{%%%%%   Z9-III-3
...}
\podpis{...}

{%%%%%   Z9-III-4
...}
\podpis{...}

