{%%%%%   Z4-I-1
...}

{%%%%%   Z4-I-2
...}

{%%%%%   Z4-I-3
...}

{%%%%%   Z4-I-4
...}

{%%%%%   Z4-I-5
...}

{%%%%%   Z4-I-6
...}

{%%%%%   Z5-I-1
...}

{%%%%%   Z5-I-2
...}

{%%%%%   Z5-I-3
...}

{%%%%%   Z5-I-4
...}

{%%%%%   Z5-I-5
...}

{%%%%%   Z5-I-6
...}

{%%%%%   Z6-I-1
...}

{%%%%%   Z6-I-2
...}

{%%%%%   Z6-I-3
...}

{%%%%%   Z6-I-4
...}

{%%%%%   Z6-I-5
...}

{%%%%%   Z6-I-6
...}

{%%%%%   Z7-I-1
...}

{%%%%%   Z7-I-2
...}

{%%%%%   Z7-I-3
...}

{%%%%%   Z7-I-4
...}

{%%%%%   Z7-I-5
...}

{%%%%%   Z7-I-6
...}

{%%%%%   Z8-I-1
...}

{%%%%%   Z8-I-2
...}

{%%%%%   Z8-I-3
...}

{%%%%%   Z8-I-4
...}

{%%%%%   Z8-I-5
...}

{%%%%%   Z8-I-6
...}

{%%%%%   Z9-I-1
...}

{%%%%%   Z9-I-2
...}

{%%%%%   Z9-I-3
...}

{%%%%%   Z9-I-4
...}

{%%%%%   Z9-I-5
...}

{%%%%%   Z9-I-6
...}

{%%%%%   Z4-II-1
...}

{%%%%%   Z4-II-2
...}

{%%%%%   Z4-II-3
...}

{%%%%%   Z5-II-1
Barborka platila za tri lízanky 24 grošov, teda jedna lízanka stála 8 grošov ($24:3=8$).

Eliška za štyri lízanky a~niekoľko nanukov platila 109 grošov.
Štyri lízanky stáli 32~grošov ($4\cdot8=32$), teda niekoľko Eliškiných nanukov stálo 77 grošov ($109-32=77$).

Táto cena je súčinom počtu nanukov a~ceny jedného nanuka.
Nanukov bolo viac ako jeden a~menej ako desať, teda ich bolo 7 (žiadne iné číslo v~danom rozsahu nedelí 77 bezo zvyšku).
Jeden nanuk stál 11 grošov ($77:7=11$).

Janko za jednu lízanku a~jeden nanuk platil 19 grošov ($8+11=19$).

\hodnotenie
1~bod za cenu jednej lízanky;
2~body za celkovú cenu Eliškiných nanukov;
2~body za cenu jedného nanuka;
1~bod za Jankovu útratu.
\endhodnotenie
}

{%%%%%   Z5-II-2
Aby diely po rozdelení boli obdĺžnikové, musia byť deliace úsečky rovnobežné so stranami pôvodného obdĺžnika.
Podľa počtu úsečiek rovnobežných s~dlhšou, príp. kratšou stranou pôvodného obdĺžnika môžeme rozlišovať nasledujúce prípady:
\insp{z5-II-2.eps}%

Novovzniknuté obdĺžniky majú byť zhodné, teda každá zo strán pôvodného obdĺžnika musí byť násobkom strany menšieho obdĺžnika.
Delením rozmerov 24\,cm a~30\,cm podľa vyššie uvedených možností zisťujeme rozmery novovzniknutých obdĺžnikov:
$$
30\cm\times6\cm, \quad 15\cm\times8\cm, \quad 10\cm\times12\cm, \quad 7{,}5\cm\times24\cm.
$$

\hodnotenie
Po 1~bode za každú možnosť;
2~body za rozbor, náčrtky a pod.
Za spracovanie obsahujúce iba obrázky dajte nanajvýš 3~body.

\poznamka
Je možné uvažovať komplikovanejšie delenia, keď niektoré deliace úsečky sú kratšie ako strany daného obdĺžnika.
V~našom prípade však žiadne ďalšie riešenie nenájdeme.
\endhodnotenie
}

{%%%%%   Z5-II-3
Aby v~každej škatuli bola aspoň jedna hračka a~v~žiadnych dvoch škatuliach nebol rovnaký počet hračiek, môže ich Vojto začať ukladať tak, že do jednej škatule dá jednu hračku, do druhej dve, do tretej tri atď.
Ak nejaké hračky zvýšia, môže ich umiestniť do poslednej škatule.

\begin{enumerate}\alphatrue
\item Do piatich škatúľ možno 20 hračiek uložiť napr. takto: 1, 2, 3, 4, $5+5$.
\item Do šiestich škatúľ je potrebných aspoň $1+2+3+4+5+6=21$ hračiek.
Teda 20~hračiek sa do šiestich škatúľ uvedeným spôsobom uložiť nedá.
\end{enumerate}

\hodnotenie
Po 3~bodoch za každú časť úlohy.
Zápornú odpoveď na otázku b) bez uvedenia dôvodu hodnoťte 1~bodom.
\endhodnotenie
}

{%%%%%   Z6-II-1
...}

{%%%%%   Z6-II-2
...}

{%%%%%   Z6-II-3
...}

{%%%%%   Z7-II-1
...}

{%%%%%   Z7-II-2
...}

{%%%%%   Z7-II-3
...}

{%%%%%   Z8-II-1
...}

{%%%%%   Z8-II-2
...}

{%%%%%   Z8-II-3
...}

{%%%%%   Z9-II-1
Označme pôvodné trojciferné číslo $\overline{abc}=100a+10b+c$.
Podľa prvej informácie zo zadania platí
$$
a+b+c=16. \tag{1}
$$
Podľa druhej informácie platí $\overline{bac}=\overline{abc}-360$, teda
$$\eqalignno{
100b+10a+c &=100a+10b+c-360, & \cr
360 &=90a-90b, & \cr
4 &=a-b. & (2) }
$$
Podľa tretej informácie platí $\overline{acb}=\overline{abc}+54$, teda
$$\eqalignno{
100a+10c+b &=100a+10b+c +54, & \cr
9c-9b &=54, & \cr
c-b &=6. & (3) }
$$

Ak z druhej, resp. tretej informácie vyjadríme $a=b+4$, resp. $c=b+6$ a~dosadíme do prvej, dostaneme $3b+10=16$, teda $b=2$.
Dosadením tohto výsledku do predchádzajúcich vyjadrení získame $a=6$ a~$c=8$.
Pôvodné trojciferné číslo bolo 628.

\hodnotenie
Po 2~bodoch za každú z~rovníc (2) a~(3);
2~body za doriešenie sústavy a~určenie neznámeho čísla.
Správne riešenie bez ďalšieho komentára hodnoťte 1~bodom.

\poznamky
Rozdiely čísel vzniknutých zámenou dvoch cifier sú vždy násobkom deviatich, pričom príslušný násobok zodpovedá miestam zamenených cifier.
Napr. v~predchádzajúcom riešení vidíme $\overline{abc}-\overline{bac}=90(a-b)$ a~$\overline{abc}-\overline{acb}=9(b-c)$, podobne $\overline{abc}-\overline{cba}=99(a-b)$.
Taká či podobná úvaha pred samotným riešením úlohy dovoľuje rýchlejšie odvodenie vzťahov (2) a~(3).

Druhú, resp. tretiu informáciu zo zadania možno názorne zapísať takto:
$$
\alggg{&a&b&c \\-&b&a&c}{&3&6&0} \qquad
\alggg{&a&c&b \\-&a&b&c}{&&5&4}
$$
Porovnaním miest pri najvyšších rádoch zisťujeme, že rozdiel $a-b$ je 3 alebo 4, resp. rozdiel $c-b$ je 5 alebo 6 (dve možnosti pri každom rozdiele zodpovedajú tomu, či uvažujeme prechod cez desiatku alebo nie).
Tieto obmedzenia spolu s~(1) dávajú jediné riešenie, ktoré možno odhaliť systematickým skúšaním možností.
Napr. podmienkam $a-b=4$ a~$a+b+c=16$ vyhovujú čísla 952, 844, 736 a~628, z~ktorých iba posledné uvedené vyhovuje aj obmedzeniu na rozdiel $c-b$.
\endhodnotenie
}

{%%%%%   Z9-II-2
Trojuholníky $ACB$ a~$ACD$ sú súmerné podľa spoločnej strany $AC$, body $E$, $F$ ležia na stranách $AB$, $AD$ a~trojuholník $ECF$ je rovnostranný, teda aj tento trojuholník je súmerný podľa $AC$.
Preto úsečky $EF$ a~$AC$ sú kolmé; ich priesečník označíme $G$.
\insp{z9-II-2a.eps}%

Vnútorné uhly rovnostranného trojuholníka majú veľkosť 60\st.
Os súmernosti, resp. výška rovnostranného trojuholníka tento uhol rozpoľuje a~rozdeľuje trojuholník na dva zhodné pravouhlé trojuholníky so zvyšnými vnútornými uhlami 30\st a~60\st.
Pomer prepony a~kratšej odvesny v~týchto menších trojuholníkoch je práve $2:1$.
Tieto všeobecné poznatky využijeme v~našej úlohe niekoľkorakým spôsobom:

Jednak priamka $AC$ je osou súmernosti trojuholníka $ECF$, teda uhol $ACE$ má veľkosť 30\st.
Jednak pomer prepony a~kratšej odvesny v~pravouhlom trojuholníku $ABC$ je podľa zadania $2:1$, teda sa jedná o~polovicu rovnostranného trojuholníka, ktorého zvyšné vnútorné uhly $BAC$ a~$ACB$ majú veľkosti 30\st a~60\st.

Z toho odvodzujeme, že uhol $BCE$ má veľkosť 30\st (rozdiel uhlov $ACB$ a~$ACE$).
Teda trojuholník $ABC$ pozostáva z troch navzájom zhodných pravouhlých trojuholníkov $AGE$, $CGE$ a~$CBE$ (zhodujú sa vo vnútorných uhloch a~každé dva majú spoločnú stranu).
\insp{z9-II-2b.eps}%

Dlhšia odvesna v~každom z~týchto troch trojuholníkov sa zhoduje so stranou $BC$, ktorá má dĺžku 6\,cm.
Navyše pomer prepony a~kratšej odvesny je $2:1$.
Ak veľkosť prepony označíme $a$, tak podľa Pytagorovej vety platí
$$
a^2 = 36+ \left( \frac{a}2 \right)^2.
$$
Po úprave dostávame $a^2=48$, teda $a=4\sqrt{3}\doteq 6{,}93$\,(cm).
A~to je veľkosť úsečky $EF$.

\hodnotenie
2~body za určenie potrebných uhlov, resp. rozpoznanie polovíc rovnostranných trojuholníkov;
2~body za dopočítanie veľkosti $EF$;
2~body za zrozumiteľnosť a~kvalitu komentára.

\poznamky
Záverečný výpočet môže byť nahradený odkazom na známy vzťah medzi výškou a~stranou rovnostranného trojuholníka.
S~uvedeným označením platí $6=\frac{\sqrt3}2a$, teda $a=4\sqrt3$.

So znalosťou Tálesovej vety, resp. vety opačnej možno k~veľkosti uhla $ACB$ dospieť nasledovne:
Keďže pri vrcholoch $B$ a~$D$ je pravý uhol, ležia oba na kružnici s~priemerom $AC$.
Stred tejto kružnice je stredom úsečky $AC$ a~označíme ho $G$ (ten istý bod ako v~riešení uvedenom vyššie).
Podľa zadania je polomer tejto kružnice rovný dĺžke strany $BC$ ($\frac12|AC|=|BC|$).
Teda trojuholník $GCB$ je rovnostranný, preto vnútorný uhol pri vrchole $C$ má veľkosť 60\st.
\insp{z9-II-2c.eps}%

Z~uvedeného o. i. vyplýva, že body $A$, $B$, $C$, $D$ sú štyri z~vrcholov pravidelného šesťuholníka.
\endhodnotenie
}

{%%%%%   Z9-II-3
Ak účinkujúcich v~pôvodnom príklade označíme ako $a:b$, tak Ľudovítovo pozorovanie môžeme zapísať takto:
$$
2a:(b+12)=(a-42):\frac{b}2.
$$
To je dvojaké vyjadrenie Ľudovítovho obľúbeného čísla, ktoré kvôli následným úpravám označíme $\ell$.

Z~vyjadrenia vľavo dostávame
$$
2a=b \ell+12\ell,
$$
z~vyjadrenia vpravo dostávame
$$\aligned
a-42 &=\frac{b}2\cdot\ell, \\
2a &=b\ell +84.
\endaligned
$$
Porovnaním týchto dvoch vyjadrení zisťujeme, že $12\ell=84$, teda Ľudovítovo obľúbené číslo je $\ell=7$.

\hodnotenie
3~body za zápis vzťahov zo zadania a~pomocné úpravy;
3~body za dopočítanie, výsledok a~kvalitu komentára.

\poznamky
Dvojíc čísel $a$ a~$b$ vyhovujúcich Ľudovítovej rovnosti je neobmedzené množstvo.
Náhodne odhalené možnosti vedúce k správnemu výsledku (napr. $a=49$ a~$b=2$) nemožno považovať za úplné riešenie úlohy -- také spracovanie hodnoťte nanajvýš 3~bodmi.

Bez dodatočného označenia podielu môžeme upravovať úvodnú rovnosť:
$$\aligned
2a\cdot\frac{b}2 &=(a-42)(b+12), \\
ab &=ab-42b+12a-12\cdot42, \\
42(b+12) &=12a.
\endaligned
$$
Odtiaľ po krátení dostávame $2a=7(b+12)$, teda hľadaný podiel je $2a:(b+12)=7$.
\endhodnotenie}

{%%%%%   Z9-II-4
Zo znalosti obsahov jednotlivých trojuholníkov určíme obsah trojuholníka $ABD$.
Zo znalosti dĺžky strany $AB$ určíme vzdialenosť bodu $D$ od priamky $AB$.
\insp{z9-II-4.eps}%

Trojuholníky $ADE$ a~$CDE$ majú spoločnú stranu $DE$ a~strany $AE$ a~$EC$ ležiace na jednej priamke.
Pomer ich obsahov je teda rovnaký ako pomer dĺžok strán $AE$ a~$EC$,
$$
|AE|:|EC| =12:18 =2:3. \tag{$*$}
$$
Podobne trojuholníky $ABE$ a~$BCE$ majú spoločnú stranu $BE$ a~strany $AE$ a~$EC$ ležiace na jednej priamke.
Pomer ich obsahov je teda rovnaký ako pomer dĺžok strán $AE$ a~$EC$,
$$
S_{ABE}:45 =2:3.
$$
Z toho dostávame $S_{ABE}=30\cm^2$.
Obsah trojuholníka $ABD$ je súčtom obsahov trojuholníkov $ABE$ a~$ADE$,
$$
S_{ABD}=30+12 =42\,(\Cm^2).
$$

Tento obsah je rovný polovici súčinu veľkosti strany $AB$ a~vzdialenosti bodu $D$ od tejto strany,
$S_{ABD}=\frac12|AB|\cdot v$.
Obsah $ABD$ aj veľkosť $AB$ poznáme, teda hľadaná vzdialenosť je rovná
$$
v=\frac{2\cdot42}{7}=12\,(\Cm).
$$

\hodnotenie
4~body za obsah trojuholníka $ABD$;
2~body za hľadanú vzdialenosť a~kvalitu komentára.

\poznamky
Úvahy a~vzťahy z~prvej časti uvedeného riešenia možno stručne zapísať ako
$S_{ADE}:S_{CDE} =S_{ABE}:S_{BCE}$,
ekvivalentne ako
$S_{BCE}:S_{CDE} =S_{ABE}:S_{ADE}$,
pričom jedinou neznámou je obsah $S_{ABE}$.

Tiež platí, že pomer obsahov $S_{ABD}:S_{BCD}$ je rovnaký ako pomer výšok trojuholníkov vzhľadom na spoločnú stranu $BD$, a~ten je rovnaký ako pomer úsečiek $|AE|:|EC|$.
Pritom $S_{BCD}=S_{BCE}+S_{CDE}$ poznáme zo zadania a~pomer $|AE|:|EC|$ je odvodený v~($*$).
Z toho je možné vyjadriť $S_{ABD}$.
\endhodnotenie}

{%%%%%   Z9-III-1
...}

{%%%%%   Z9-III-2
...}

{%%%%%   Z9-III-3
...}

{%%%%%   Z9-III-4
...}

