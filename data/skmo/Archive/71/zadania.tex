{%%%%%   A-I-1
Je možné vyplniť tabuľku $n \times n$ jednotkami a dvojkami tak, aby bol súčet čísel v každom riadku deliteľný piatimi a súčet čísel v každom stĺpci deliteľný siedmimi? Riešte
a) pre $n=9$,
b) pre $n=12$.}
\podpis{Tomáš Bárta}

{%%%%%   A-I-2
Je daný lichobežník $A B C D$ so základňami $A B$ a $C D$. Označme $k_{1}$ a $k_{2}$ kružnice s priemermi $B C$ a $A D$. Ďalej označme $P$ priesečník priamok $B C$ a $A D$. Dokážte, že dotyčnice z bodu $P$ ku kružnici $k_{1}$ zvierajú rovnaký uhol ako dotyčnice z bodu $P$ ku kružnici $k_{2}$.}
\podpis{Patrik Bak}

{%%%%%   A-I-3
Nájdite všetky celé čísla $n>2$ také, že číslo $n^{n-2}$ je $n$-tá mocnina celého čísla.}
\podpis{Patrik Bak}

{%%%%%   A-I-4
V obore reálnych čísel riešte sústavu rovníc
$$
\begin{aligned}
& x y+1=z^{2} \\
& y z+2=x^{2} \\
& z x+3=y^{2}
\end{aligned}
$$}
\podpis{Tomáš Jurík}

{%%%%%   A-I-5
V rôznostrannom trojuholníku $A B C$ označme $I$ stred vpísanej kružnice a $k$ kružnicu opísanú. Polpriamky $B I$ a $C I$ pretnú kružnicu $k$ po rade v bodoch $S_{b} \neq B$ a $S_{c} \neq C$. Dokážte, že dotyčnica ku kružnici $k$ v bode $A$, priamka vedená bodom $I$ rovnobežne so stranou $B C$ a priamka $S_{b} S_{c}$ sa pretínajú v jednom bode.}
\podpis{Patrik Bak}

{%%%%%   A-I-6
Uvažujme nekonečnú postupnosť $a_{0}, a_{1}, a_{2}, \ldots$ celých čísel, ktorá spĺňa podmienky $a_{0} \geqq 2$ a $a_{n+1} \in\left\{2 a_{n}-1,2 a_{n}+1\right\}$ pre všetky indexy $n \geqq 0$. Dokážte, že každá taká postupnosť obsahuje nekonečne veľa zložených čísel.}
\podpis{Martin Melicher, Josef Tkadlec}

{%%%%%   B-I-1
Pravouhlý trojuholník má celočíselné dĺžky strán a obvod 11990. Navyše vieme, že jedna jeho odvesna má prvočíselnú dĺžku. Určite ju.}
\podpis{Patrik Bak}

{%%%%%   B-I-2
Nech $A B C$ je ostrouhlý trojuholník s najdlhšou stranou $B C$. Vnútri strán $A B$ a $A C$ ležia po rade body $D$ a $E$ tak, že $|C D|=|C A|$ a $|B E|=|B A|$. Označme $F$ taký bod, že $A B F C$ je rovnobežník. Dokážte, že $|F D|=|F E|$.}
\podpis{Patrik Bak, Josef Tkadlec}

{%%%%%   B-I-3
Určite počet deväťmiestnych čísel, v ktorých sa číslice $0-9$ vyskytujú najviac raz a v ktorých sa súčty číslic na 1. až 3. mieste, na 3. až 5. mieste, na 5. až 7. mieste a na 7. až 9. mieste všetky rovnajú tomu istému číslu 10. Nájdite takisto najmenšie a najväčšie z týchto čísel.}
\podpis{Jaroslav Zhouf}

{%%%%%   B-I-4
Určite počet reálnych koreňov rovnice $x|x+6 A|=36 \mathrm{v}$ závislosti od reálneho parametra $A$.}
\podpis{Vojtech Bálint}

{%%%%%   B-I-5
Pravidelný $n$-uholník označme $A_{1} A_{2} \ldots A_{n}$. Bod $A_{3}$ zobrazíme v osovej súmernosti s osou $A_{2} A_{4}$, získame bod $A_{3}^{\prime}$. Potom bod $A_{3}^{\prime}$ zobrazíme v osovej súmernosti s osou $A_{1} A_{3}$, získame bod $A_{3}^{\prime \prime}$. Pre ktoré $n \geqq 4$ je bod $A_{3}^{\prime \prime}$ totožný s priesečníkom priamok $A_{1} A_{2}$ a $A_{3} A_{4}$?}
\podpis{Jaroslav Zhouf}

{%%%%%   B-I-6
Je daná šachovnica $m \times n$, ktorej políčka sú ofarbené čierno a bielo klasickým spôsobom, pričom ľavé horné políčko je čierne. Ťahom rozumieme vzájomnú výmenu dvoch riadkov alebo vzájomnú výmenu dvoch stĺpcov šachovnice. Škvrnou rozumieme takú neprázdnu množinu čiernych políčok, ktorá je tvorená všetkými políčkami, do ktorých
je možné z jedného jej políčka prejsť po ceste pozostávajúcej zo stranou susediacich čiernych políčok. Napríklad na obrázku je šachovnica $4 \times 4 \mathrm{~s}$ práve štyrmi škvrnami. V závislosti od prirodzených čísel $m$ a $n$ určite, koľko najmenej škvŕn môže byť na šachovnici $m \times n$ po vykonaní konečného počtu ťahov.
\Image{71-b-i-6.pdf}
}
\podpis{David Hruška}

{%%%%%   C-I-1
Na školskej záhrade hrá skupina žiakov hru zvanú molekuly. Učiteľ im najprv uložil, aby sa rozdelili do trojíc. Jeden žiak zvýšil, a tak z ďalšej hry vypadol. Zvyšní žiaci sa potom mali rozdeliť do štvoríc. Opäť jeden žiak zvýšil a vypadol. Potom sa zvyšní žiaci mali rozdeliť do pätíc, zase jeden žiak zvýšil a vypadol. Učiteľ teraz ukladá, aby sa zvyšní žiaci rozdelili do šestíc. Dokážte, že opäť jeden žiak zvýši.}
\podpis{Josef Tkadlec}

{%%%%%   C-I-2
Určite všetky štvorice rôznych dvojmiestnych prirodzených čísel, pre ktoré zároveň platí:
(i) Súčet tých čísel z danej štvorice, ktoré obsahujú číslicu 2, je 80.
(ii) Súčet tých čísel z danej štvorice, ktoré obsahujú číslicu 3, je 90.
(iii) Súčet tých čísel z danej štvorice, ktoré obsahujú číslicu 5 , je 60.}
\podpis{Jaroslav Zhouf}

{%%%%%   C-I-3
Vnútri strany $B C$ ľubovoľného trojuholníka $A B C$ sú dané body $D, E$ tak, že $|B D|=|D E|=|E C|$, vnútri strany $A C$ body $F, G$ tak, že $|A G|=|G F|=|F C|$. Uvažujme trojuholník vymedzený úsečkami $A E, G D, B F$. Dokážte, že pomer obsahu tohoto trojuholníka a obsahu trojuholníka $A B C$ má jedinú možnú hodnotu, a určite ju.}
\podpis{Jaroslav Zhouf}

{%%%%%   C-I-4
Tabuľka $10 \times 10$ je vyplnená číslami $1 \mathrm{a}-1$ tak, že súčet čísel v každom riadku až na jeden je rovný nule a zároveň súčet čísel v každom stĺpci až na jeden je rovný nule. Určite najväčší možný súčet všetkých čísel v tabuľke.}
\podpis{Patrik Bak}

{%%%%%   C-I-5
Je daný rovnostranný trojuholník $A B C$ a vnútri jeho strany $A B$ bod $D$. Na polpriamke opačnej k $B C$ uvažujme bod $E$ taký, že $|C D|=|D E|$. Dokážte, že platí $|A D|=|B E|$.}
\podpis{Jaroslav Švrček}

{%%%%%   C-I-6
Určite všetky možné hodnoty súčtu $a+b+c+d$, kde $a, b, c, d$ sú prirodzené čísla spĺňajúce rovnosť
$$
\left(a^{2}-b^{2}\right)\left(c^{2}-d^{2}\right)+\left(b^{2}-d^{2}\right)\left(c^{2}-a^{2}\right)=2021.
$$}
\podpis{Mária Dományová, Patrik Bak}

{%%%%% A-S-1
Nájdite najväčšie celé číslo $d$, pre ktoré možno tabuľku
$43\times 47$ vyplniť jednotkami a~dvojkami tak, aby súčet
čísel v~každom riadku aj v~každom stĺpci bol deliteľný číslom~$d$.
(Dokážte tiež, že žiadne väčšie číslo $d$ zadaniu úlohy nevyhovuje.)
}
\podpis{Tomáš Bárta}

{%%%%% A-S-2
V~trojuholníku $ABC$ označme $I$ stred kružnice vpísanej.
Priamky $BI$, $CI$ pretnú kružnicu opísanú trojuholníku $ABC$
postupne v~bodoch $S\ne B$, $T\ne C$. Úsečka~$ST$ pretína strany
$AB$, $AC$ v~bodoch $K$, $L$. Dokážte, že štvoruholník $AKIL$ je
kosoštvorec (prípadne štvorec).
}
\podpis{Josef Tkadlec}

{%%%%% A-S-3
Určte všetky dvojice kladných celých čísel $a$ a $b$, pre
ktoré platí $a^{a-b}=b^a$.}
\podpis{Jaromír Šimša}

{%%%%% A-II-1
Je možné vyplniť tabuľku $8\times 8$ šestkami a sedmičkami
tak, aby súčet čísel v~každom riadku bol deliteľný piatimi a súčet
čísel v~každom stĺpci bol deliteľný siedmimi?
}
\podpis{Josef Tkadlec}

{%%%%% A-II-2
V~obore kladných reálnych čísel riešte sústavu rovníc
$$\align
x^2 + 2y^2 &= \hphantom{1}x + 2y + 3z, \\
y^2 + 2z^2 &= 2x + 3y + 4z, \\
z^2 + 2x^2 &= 3x + 4y + 5z.
\endalign$$
}
\podpis{Patrik Bak}

{%%%%% A-II-3
Daný je rovnoramenný trojuholník $ABC$ so základňou $AB$ a bod~$P$
vnútri jeho výšky z~vrcholu $C$.
Priamka $AP$ pretína kružnicu opísanú trojuholníku $ABC$ v~bode~$Q\ne A$.
Rovnobežka so základňou $AB$ vedená bodom $P$ pretína rameno~$BC$
v~bode~$R$. Dokážte, že polpriamka $QR$ je osou uhla $AQB$.}
\podpis{Jaroslav Švrček}

{%%%%% A-II-4
Uvažujme nekonečnú postupnosť $a_0,a_1,a_2,\ldots$
celých čísel, ktorá spĺňa podmienky $a_0\geqq1$ a
$$
a_{n+1}\in\{2022a_n-1, 2022a_n+1\}
$$
pre všetky indexy $n\geqq0$. Dokážte, že každá taká postupnosť
obsahuje nekonečne veľa zložených čísel.}
\podpis{Martin Melicher}

{%%%%% A-III-1
Na papieri je v~rade vedľa seba napísaných 71 nenulových
reálnych čísel. Platí, že každé číslo okrem prvého a posledného
je o~jedna menšie ako súčin jeho dvoch susedov. Dokážte, že prvé
a posledné číslo sa rovnajú.
}
\podpis{Josef Tkadlec}

{%%%%% A-III-2
Hovoríme, že kladné celé číslo $d$ je {\it spravodlivé},
ak počet $2021$-ciferných palindrómov, ktoré sú násobkami $d$,
je rovnaký ako počet $2022$-ciferných palindrómov, ktoré sú
násobkami $d$. Obsahuje množina $M=\{1,2,\dots,35\}$ viac tých
čísel, ktoré sú spravodlivé, alebo tých, ktoré spravodlivé
nie sú? (Palindrómom nazývame prirodzené číslo, ktorého dekadický
zápis sa číta zľava doprava rovnako ako sprava doľava.)
}
\podpis{David Hruška, Josef Tkadlec}

{%%%%% A-III-3
V~ostrouhlom rôznostrannom trojuholníku $ABC$ označme $M$
stred strany $BC$ a~$N$~stred oblúka $BAC$ jeho kružnice
opísanej.
Ďalej označme $\omega$ kružnicu s~priemerom $BC$ a $D$, $E$
priesečníky $\omega$ s~osou uhla $BAC$. Body $D'$, $E'$ ležia na
kružnici~$\omega$ tak, že štvoruholník $DED'E'$ je pravouholník.
Dokážte, že body $D'$, $E'$, $M$, $N$ ležia na jednej kružnici.
}
\podpis{Patrik Bak}

{%%%%% A-III-4
V~konvexnom štvoruholníku $ABCD$ platí $|AB|=|BC|=|CD|$.
Označme~$P$ priesečník jeho uhlopriečok a~$O_1$,~$O_2$
stredy kružníc opísaných postupne trojuholníkom $APB$ a~$DPC$.
Dokážte, že štvoruholník $O_1BCO_2$ je rovnobežník.
}
\podpis{Patrik Bak}

{%%%%% A-III-5
Nájdite všetky celé čísla $n$, pre ktoré je číslo
$$
2^n+n^2
$$
druhou mocninou nejakého celého čísla.
}
\podpis{Tomáš Jurík}

{%%%%% A-III-6
Pri pokuse o~kolonizáciu Marsu zaplavilo ľudstvo slnečnú
sústavu 50 satelitmi, ktoré medzi sebou vytvorili 225
komunikačných línií (každá línia existuje medzi jednou dvojicou
satelitov a žiadne dva satelity medzi sebou nemajú viac ako jednu
líniu). Hovoríme, že trojica satelitov je {\it prepojená}, ak aspoň
jeden z~nich má vytvorené komunikačné línie s~oboma ostatnými
satelitmi. Určte najmenší a najväčší možný počet prepojených trojíc
satelitov.
}
\podpis{Ján Mazák, Josef Tkadlec}

{%%%%% B-S-1
Hovoríme, že prirodzené číslo je {\it strakaté},
ak je v~jeho dekadickom zápise
každá cifra iná a všetky súčty troch susedných cifier daného čísla
nadobúdajú práve dve rôzne hodnoty. (Napríklad číslo 162\,735 nie je strakaté,
pretože posudzované súčty $1+6+2=9$, $6+2+7=15$, $2+7+3=12$ a $7+3+5=15$
nadobúdajú tri rôzne hodnoty.)
\ite a) Uveďte príklad šesťciferného strakatého čísla.
\ite b) Existuje sedemciferné strakaté číslo?
}
\podpis{Josef Tkadlec, Martin Melicher}

{%%%%% B-S-2
Daný je ostrouhlý trojuholník $ABC$ s~najdlhšou stranou~$BC$.
Vnútri jeho strán $AB$ a~$AC$ ležia postupne body $D$ a~$E$ tak,
že $|CD|=|CA|$ a~$|BE|=|BA|$. Uvažujme ďalej body $F$ a~$G$ tak,
že $ABCF$ a $ACBG$ sú rovnobežníky. Dokážte, že~$|FD|=|GE|$.}
\podpis{Patrik Bak}

{%%%%% B-S-3
Pravouhlý trojuholník má celočíselné dĺžky strán. Jeho obvod je druhá mocnina
prirodzeného čísla. Tiež vieme, že jedna jeho odvesna má dĺžku
rovnú druhej mocnine prvočísla. Určte všetky možné hodnoty tejto
dĺžky.}
\podpis{Patrik Bak}

{%%%%% B-II-1
a) Rozhodnite, či existuje také prirodzené číslo $n$, že
$2n$ je druhou mocninou prirodzeného čísla a $3n$ je treťou mocninou
prirodzeného čísla.

b) Rozhodnite, či existuje také prirodzené číslo $n$,
ktoré spĺňa podmienku časti~a) a navyše ešte je $4n$
štvrtou mocninou prirodzeného čísla.}
\podpis{Josef Tkadlec}

{%%%%% B-II-2
Určte počet reálnych koreňov rovnice $x^2+4=a|x|$
v~závislosti od reálneho parametra~$a$.}
\podpis{Mária Dományová}

{%%%%% B-II-3
Pravidelný $n$-uholník označme $A_1A_2\dots A_n$.
Pre ktoré $n\geqq 5$ platí, že obraz bodu~$A_3$ v~osovej súmernosti
podľa priamky~$A_1A_2$ leží na priamke~$A_4A_5$?}
\podpis{Josef Tkadlec}

{%%%%% B-II-4
Tabuľka $10\times 10$ je vyplnená číslami ${-4}$, $3$ a $10$ tak, že súčet čísel
v~každom riadku až na jeden je nanajvýš~0 a súčet čísel v~každom stĺpci až
na jeden je nanajvýš~0. Určte najväčší možný súčet čísel v~tabuľke.}
\podpis{Radovan Švarc}

{%%%%% C-S-1
Tabuľka $10\times10$ je vyplnená číslami $1$ a ${-1}$ tak, že súčet
čísel v~každom riadku aj stĺpci je deliteľný tromi. Určte
najväčší možný súčet čísel v~tabuľke a dokážte, že väčší byť
nemôže. Uveďte tiež príklad tabuľky s~určeným najväčším
súčtom.}
\podpis{Michal Rolínek}

{%%%%% C-S-2
V~rovnobežníku $ABCD$ platí, že os uhla $ABC$ prechádza stredom~$L$ strany~$CD$. Dokážte, že $AL\perp BL$.}
\podpis{Jaroslav Švrček}

{%%%%% C-S-3
Nájdite všetky štvorice $a>b>c>d$ celých čísel so súčtom 71,
ktoré spĺňajú rovnicu
$$
(a-b)(c-d)+(a-d)(b-c)=26.
$$
}
\podpis{Josef Tkadlec}

{%%%%% C-II-1
Dokážte, že pre ľubovoľné celé čísla $a$, $b$ platí nerovnosť
$$
a(a+1)+b(b-1)\geqq2ab.
$$
Zistite tiež, kedy nastáva rovnosť.}
\podpis{Jaromír Šimša}

{%%%%% C-II-2
Daný je rovnobežník $ABCD$ a na jeho obvode body $X$ a~$Y$ rôzne
od bodu~$A$ tak, že priamky $AX$ a $AY$ delia tento rovnobežník na
tri časti s~rovnakým obsahom. Určte pomer obsahu trojuholníka $AXY$ a~obsahu rovnobežníka $ABCD$.}
\podpis{David Hruška}

{%%%%% C-II-3
Na tabuli je napísaných niekoľko rôznych dvojciferných prirodzených čísel.
Cifru~$c$ nazveme {\it dobrou}, ak je súčet tých čísel
z~tabule, ktoré obsahujú cifru~$c$, rovný číslu~71.
\ite a) Ktoré z~cifier 0 až 9 môžu byť dobré?
\ite b) Koľko najviac cifier môže byť súčasne dobrých?}
\podpis{Josef Tkadlec}

{%%%%% C-II-4
Tabuľka $10\times 10$ je vyplnená číslami $1$ a~${-1}$ tak, že súčet čísel
v~každom riadku až na jeden je rovný~0 a súčet čísel v~každom stĺpci až
na jeden je rovný rovnakému číslu~$s$. Určte najväčšiu možnú
hodnotu $s$ a dokážte, že väčšia byť nemôže. Uveďte tiež príklad tabuľky
s~určenou najväčšou hodnotou~$s$.}
\podpis{Josef Tkadlec}

{%%%%%   vyberko, den 1, priklad 1
[A0]
Nájdite všetky polynómy $P(x)$ s reálnymi koeficientami, ktoré spĺňajú nasledujúce dve podmienky:
\item{a)} $P(2022)=2021$,
\item{b)} $(P(x)+1)^2=P(x^2+1)$ pre všetky reálne čísla $x$.
}
\podpis{...}

{%%%%%   vyberko, den 1, priklad 2
[C0]
K dispozícii máme nekonečne veľa guľôčok z každej zo 100 farieb $C_1,\dots,C_{100}$. Hovoríme, že celé číslo $m\ge 101$ je {\it farebné}, ak je možné umiestniť $m$ guľôčok do kruhu tak, že skupina ľubovoľných 101 po sebe idúcich guľôčok obsahuje aspoň jednu guľôčku farby $C_i$ pre každé $i=1,\dots,100$. Dokážte, že existuje iba konečne veľa kladných celých čísel, ktoré nie sú farebné, a~nájdite najväčšie z nich.}
\podpis{IMO shortlist 2021,C2}

{%%%%%   vyberko, den 1, priklad 3
[G0]
Nech $ABC$ je trojuholník spĺňajúci $|AB|=|AC|$. Bod~$K$ leží vnútri trojuholníka~$ABC$ na jeho výške z~vrcholu~$A$. Bod~$L$ je zvolený na priamke~$BK$ tak, že $AL \parallel BC$. Dokážte, že ak $KC \perp CL$, tak bod~$L$ leží na osi vonkajšieho uhla~$ACB$.}
\podpis{...}

{%%%%%   vyberko, den 1, priklad 4
[N1]
Nech $n\ge 3$ je celé číslo a $a_1,a_2,\dots,a_n$ sú nenulové celé čísla také, že
$$
a_1a_2\cdots a_n \left( \frac{1}{a_1^2}+\frac{1}{a_2^2}+\cdots+ \frac{1}{a_n^2}\right)
$$
je celé číslo. Vyplýva z toho nutne, že súčin $a_1a_2\cdots a_n$ je deliteľný $a_i^2$ pre všetky $1\le i\le n$?}
\podpis{...}

{%%%%%   vyberko, den 2, priklad 1
[N0]
Nájdite všetky trojice kladných celých čísel $(a,b,c)$ také, že $2^a+2^b+2^c-21$ je druhou mocninou celého čísla.}
\podpis{...}

{%%%%%   vyberko, den 2, priklad 2
[A2]
Nájdite všetky neklesajúce funkcie $f\colon\Bbb R\rightarrow\Bbb R$ také, že pre všetky reálne čísla $x$, $y$ platí
$$
f(f(x^2)+y+f(y))=x^2+2f(y).
\belowdisplayskip0pt
$$}
\podpis{...}

{%%%%%   vyberko, den 2, priklad 3
[G3]
A teraz nájdite všetky celé čísla $n \ge 3$ s~nasledovnou vlastnosťou: každý konvexný $n$-uholník, ktorého všetky strany majú dĺžku~1, obsahuje rovnostranný trojuholník so stranou dĺžky~1. (Každý mnohouholník obsahuje aj svoju hranicu.)}
\podpis{IMO shortlist 2021,C2}

{%%%%%   vyberko, den 3, priklad 1
[G1]
Nech~$ABC$ je rôznostranný ostrouhlý trojuholník a~nech~$M$ je stred jeho strany~$BC$. Body~$D$ a~$E$ sú päty výšok postupne z~vrcholov~$C$ a~$B$. Nech~$L$ a~$K$ sú postupne stredy úsečiek~$MD$ a~$ME$. Priamka~$KL$ pretína priamky~$AB$ a~$AC$ postupne v~bodoch~$X$ a~$Y$. Dokážte, že štvoruholník~$AXMY$ je tetivový.}
\podpis{...}

{%%%%%   vyberko, den 3, priklad 2
[C2]
Je daná postupnosť 51 kladných celých čísel, ktorých súčet je 101. Dokážte, že pre každé celé číslo $k$ spĺňajúce $1\le k\le 100$ vieme nájsť niekoľko po sebe idúcich čísel tejto postupnosti, ktorých súčet je $k$, alebo niekoľko po sebe idúcich čísel tejto postupnosti, ktorých súčet je $101-k$.}
\podpis{...}

{%%%%%   vyberko, den 3, priklad 3
[N3]
Patrik má dané racionálne číslo $r>1$ a priamku s dvoma bodmi $M\ne C$. V~bode~$M$ sa nachádza modrý žetón a v bode $C$ je červený žetón. Patrik hrá hru prevádzaním postupnosti ťahov. V každom ťahu si vyberie (nie nutne kladné) celé číslo $k$ a žetón, ktorým chce pohnúť. Ak je tento žetón položený v bode $X$ a druhý žetón je v bode~$Y$, tak Patrik posunie vybraný žetón do bodu $X'$ na polpriamke $\overrightarrow{YX}$ pre ktorý platí $|YX'|=r^k \cdot |YX|$.
Patrikov cieľ je posunúť červený žetón do bodu $M$. Nájdite všetky racionálne čísla $r>1$, pre ktoré vie Patrik dosiahnuť svoj cieľ na najviac 2022 ťahov.}
\podpis{IMO shortlist 2021,C2}

{%%%%%   vyberko, den 4, priklad 1
[C1]
Jožko a Majo hrajú hru na šachovnici $m\times n$, kde $m$ a $n$ sú kladné celé čísla. Striedajú sa v ťahoch, pričom Jožko začína. Jožko musí počas svojho ťahu umiestniť kameň na políčko, na ktorom zatiaľ žiaden kameň nie je. Majo tiež musí počas svojho ťahu umiestniť kameň na prázdne políčko, ale navyše toto  políčko musí susediť stranou s políčkom, na ktoré umiestnil kameň Jožko v predošlom ťahu.
\smallskip
Majo vyhrá, ak je celá šachovnica zaplnená kameňmi. Jožko vyhrá, ak Majo nemôže umiestniť kameň počas jeho ťahu a na šachovnici je aspoň jedno prázdne políčko.
Pre každú dvojicu kladných celých čísel $(m,n)$ určte, ktorý z hráčov má víťaznú stratégiu.}
\podpis{...}

{%%%%%   vyberko, den 4, priklad 2
[N2]
Nájdite všetky kladné celé čísla $n$ s~nasledujúcou vlastnosťou: množina všetkých kladných deliteľov čísla $n$ má permutáciu $(d_1,\dots,d_k)$ takú, že pre každé $i=1,2,\dots,k$ je číslo $d_1+\cdots+d_i$ druhou mocninou celého čísla.}
\podpis{IMO shortlist 2021, N3}

{%%%%%   vyberko, den 4, priklad 3
[A3]
Nech $n\ge 2$ je celé číslo a $a_1,\dots,a_n$ sú kladné reálne čísla také, že $a_1+a_2+\dots+a_n=1$. Dokážte, že
$$
\sum_{k=1}^n\frac{a_k}{1-a_k}(a_1+a_2+\dots+a_{k-1})^2<\frac 13.
\belowdisplayskip0pt
$$}
\podpis{IMO shortlist 2021, A5}

{%%%%%   vyberko, den 5, priklad 1
[A1]
Nech $n$ je kladné celé číslo a nech $A$ je podmnožina množiny $\{0,1,2,\dots,5^n\}$ so $4n+2$ prvkami. Dokážte, že existujú $a,b,c\in A$ také, že $a<b<c$ a $c+2a>3b$.}
\podpis{IMO shortlist 2021, A5}

{%%%%%   vyberko, den 5, priklad 2
[G2]
Nech $ABCD$ je rovnobežník spĺňajúci $|AC|=|BC|$. Bod~$P$ leží na predĺžení úsečky~$AB$ za bodom~$B$. Kružnica opísaná trojuholníku~$ACD$ pretína úsečku~$PD$ znova v~bode~$Q$. Kružnica opísaná trojuholníku~$APQ$ pretína úsečku~$PC$ znova v~bode~$R$. Dokážte, že priamky $CD$, $AQ$ a~$BR$ sa pretínajú v~jednom bode.}
\podpis{IMO shortlist 2021, G1}

{%%%%%   vyberko, den 5, priklad 3
[C3]
A teraz nech $S$ je nekonečná množina kladných celých čísel taká, že existujú štyri po dvoch rôzne čísla $a,b,c,d\in S$ spĺňajúce $\nsd(a,b)\ne \nsd(c,d)$. Dokážte, že existujú tri po dvoch rôzne čísla $x,y,z\in S$ také, že $\nsd(x,y)=\nsd(y,z)\ne \nsd(z,x)$.

\poznamka
Výraz $\nsd(a,b)$ označuje najväčšieho spoločného deliteľa čísel $a,b$.
}
\podpis{IMO shortlist 2021, C1}

{%%%%%   vyberko, den 2, priklad 4
...}
\podpis{...}

{%%%%%   vyberko, den 3, priklad 4
...}
\podpis{...}

{%%%%%   vyberko, den 4, priklad 4
...}
\podpis{...}

{%%%%%   vyberko, den 5, priklad 4
...}
\podpis{...}

{%%%%%   trojstretnutie, priklad 1
...}
\podpis{...}

{%%%%%   trojstretnutie, priklad 2
...}
\podpis{...}

{%%%%%   trojstretnutie, priklad 3
...}
\podpis{...}

{%%%%%   trojstretnutie, priklad 4
...}
\podpis{...}

{%%%%%   trojstretnutie, priklad 5
...}
\podpis{...}

{%%%%%   trojstretnutie, priklad 6
...}
\podpis{...}

{%%%%%   IMO, priklad 1
...}
\podpis{...}

{%%%%%   IMO, priklad 2
...}
\podpis{...}

{%%%%%   IMO, priklad 3
...}
\podpis{...}

{%%%%%   IMO, priklad 4
...}
\podpis{...}

{%%%%%   IMO, priklad 5
...}
\podpis{...}

{%%%%%   IMO, priklad 6
...}
\podpis{...}

{%%%%%   MEMO, priklad 1
}
\podpis{}

{%%%%%   MEMO, priklad 2
}
\podpis{}

{%%%%%   MEMO, priklad 3
}
\podpis{}

{%%%%%   MEMO, priklad 4
}
\podpis{}

{%%%%%   MEMO, priklad t1
}
\podpis{}

{%%%%%   MEMO, priklad t2
}
\podpis{}

{%%%%%   MEMO, priklad t3
}
\podpis{}

{%%%%%   MEMO, priklad t4
}
\podpis{}

{%%%%%   MEMO, priklad t5
}
\podpis{}

{%%%%%   MEMO, priklad t6
}
\podpis{}

{%%%%%   MEMO, priklad t7
}
\podpis{}

{%%%%%   MEMO, priklad t8
}
\podpis{}

{%%%%%   CPSJ, priklad 1
...}
\podpis{...}

{%%%%%   CPSJ, priklad 2
...}
\podpis{...}

{%%%%%   CPSJ, priklad 3
...}
\podpis{...}

{%%%%%   CPSJ, priklad 4
...}
\podpis{...}

{%%%%%   CPSJ, priklad 5
...}
\podpis{...}

{%%%%%   CPSJ, priklad t1
...}
\podpis{...}

{%%%%%   CPSJ, priklad t2
...}
\podpis{...}

{%%%%%   CPSJ, priklad t3
...}
\podpis{...}

{%%%%%   CPSJ, priklad t4
...}
\podpis{...}

{%%%%%   CPSJ, priklad t5
...}
\podpis{...}

{%%%%%   CPSJ, priklad t6
...}
\podpis{...}

{%%%%%   EGMO, priklad 1
...}
\podpis{...}

{%%%%%   EGMO, priklad 2
...}
\podpis{...}

{%%%%%   EGMO, priklad 3
...}
\podpis{...}

{%%%%%   EGMO, priklad 4
...}
\podpis{...}

{%%%%%   EGMO, priklad 5
...}
\podpis{...}

{%%%%%   EGMO, priklad 6
...}
\podpis{...}

{%%%%%   vyberko C, den 1, priklad 1
Tabuľka $8 \times 8$ má vyfarbené niektoré políčka, pričom v~žiadnom riadku, stĺpci ani na diagonále neležia viac ako štyri vyfarbené políčka. Koľko najviac políčok môže byť vyfarbených?

\poznamka
Dve políčka ležia na diagonále, ak priamka prechádzajúca ich stredmi zviera so stranami tabuľky uhol $45^\circ$. Príklady diagonál sú na \obr.
\Image{grid.pdf}{0.5}%
}
\podpis{...}

{%%%%%   vyberko C, den 1, priklad 2
Nech $ABCDE$ je konvexný päťuholník s~piatimi rovnako dlhými stranami a~pravými uhlami pri vrcholoch $C$ a~$D$. Jeho uhlopriečky $AC$ a~$BD$ sa pretínajú v~bode~$P$. Dokážte, že priamky $PE$ a~$AD$ sú kolmé.
}
\podpis{...}

{%%%%%   vyberko C, den 1, priklad 3
Určte všetky trojice $(x,y,z)$ reálnych čísel, ktoré spĺňajú
$$
\align
x^2+y^2+25z^2&=6xz+8yz,\\
3x^2+2y^2+z^2&=240.
\endalign
\belowdisplayskip0pt
$$
}
\podpis{...}

{%%%%%   vyberko C, den 1, priklad 4
Nájdite všetky kladné celé čísla~$n$ s~nasledujúcou vlastnosťou: Všetkých kladných deliteľov čísla~$n$ vieme rozdeliť do dvojíc takých, že v~každej dvojici jedno číslo delí to druhé.}
\podpis{...}

{%%%%%   vyberko C, den 1, priklad 5
V trojuholníku $ABC$ spĺňajúcom $|AB|=|AC|$ označíme~$M$ stred~$BC$, $N$ stred~$AM$ a~$P$ priesečník priamok $CN$ a~$AB$. Vypočítajte pomer $|AC|:|AP|$.}
\podpis{...} 