{%%%%%   A-I-1
...}

{%%%%%   A-I-2
...}

{%%%%%   A-I-3
...}

{%%%%%   A-I-4
...}

{%%%%%   A-I-5
...}

{%%%%%   A-I-6
...}

{%%%%%   B-I-1
...}

{%%%%%   B-I-2
...}

{%%%%%   B-I-3
...}

{%%%%%   B-I-4
...}

{%%%%%   B-I-5
...}

{%%%%%   B-I-6
...}

{%%%%%   C-I-1
...}

{%%%%%   C-I-2
...}

{%%%%%   C-I-3
...}

{%%%%%   C-I-4
...}

{%%%%%   C-I-5
...}

{%%%%%   C-I-6
...}

{%%%%%   A-S-1
Odpoveď je $d=47$.

V~prvej časti riešenia dokážeme, že pre $d=47$ požadované
vyplnenie tabuľky naozaj existuje:
Zapíšme do každého riadku všetkých 47 čísel rovnakých, a~to do 39~riadkov
jednotky a~do zvyšných 4 riadkov dvojky (pozri \obr).
Potom súčet čísel v~každom riadku je rovný 47 alebo $2\cdot 47$
a~súčet čísel v~každom stĺpci je rovný $39\cdot 1 + 4\cdot
2=47$, čo sú všetko násobky čísla 47, ako sme chceli.
\inspdf{a71s1.pdf}%

V druhej časti riešenia dokážeme sporom, že pre žiadne $d>47$ požadované
vyplnenie neexistuje. Pripusťme naopak, že tabuľku $43\times 47$
máme vyplnenú jednotkami a~dvojkami tak, že pre nejaké $d>47$ je
súčet čísel v~každom riadku aj každom stĺpci násobkom čísla~$d$.
Keďže každý z~týchto súčtov je kladný a~nanajvýš rovný~$2\cdot47$,
a~teda menší ako $2d$, musí sa jednať o~násobok čísla $d$ rovný priamo číslu $d$.
Sčítaním všetkých čísel tabuľky po riadkoch tak dôjdeme k~hodnote $43d$,
zatiaľ čo pri sčítaní po stĺpcoch dostaneme hodnotu $47d$. Musí
teda platiť $43d=47d$, čo je hľadaný spor.

\Pozn
Podané riešenie možno zrejmým spôsobom zovšeobecniť
na dôkaz výsledku, že pre každú tabuľku $m\times n$, pričom
$m<n\le 2m$, je najväčšie vyhovujúce $d$ rovné číslu~$n$.

\schemaABC
Za úplné riešenie dajte 6 bodov, z~toho
2~body za dôkaz prvej časti (konštrukcia pre $d=47$)
a~4~body za dôkaz druhej časti (dôkaz nemožnosti pre každé~$d>47$).
Z~prvej časti možno získať 1~čiastočný bod za vyhovujúcu konštrukciu
pre niektoré celé $d$, $1<d<47$. Z~druhej časti možno získať
1~čiastočný bod za dôkaz nemožnosti pre každé~$d\geqq86$.
Za uhádnutie odpovede $d=47$ dajte 1~bod, ak nie je
dosiahnutý žiadny z~vyššie uvedených ziskov.
\endschema}

{%%%%%   A-S-2
Z~návodnej úlohy N3 k~5.~úlohe domáceho kola vieme, že priamka $ST$ je
osou úsečky $AI$.\fnote{Tento známy poznatok nemusia riešitelia dokazovať.
Pripomeňme len, že vyplýva zo zhodnosti trojuholníkov $ATS$
a~$ITS$ podľa vety $usu$: Majú totiž spoločnú stranu $ST$ a~uhly
$ATS$, $ITS$ sú zhodné rovnako ako uhly $AST$, $IST$ -- jedná sa
totiž o~obvodové uhly v~kružnici $k$ prislúchajúce zhodným oblúkom
$AS$, $CS$, resp. zhodným oblúkom $AT$, $BT$.}
Platí preto $|KA|=|KI|$ a~$|LA|=|LI|$, ako je vyznačené na
\obr{} vľavo. Štvoruholník $AKIL$ je teda súmerný
podľa svojej uhlopriečky $KL$, jedná sa preto buď
o~deltoid\fnote{Zdôraznime, že pod \emph{deltoidom} sa zvyčajne rozumie
konvexný štvoruholník, ktorý má práve dve dvojice zhodných
susedných strán. Kosoštvorec potom nie je zvláštnym prípadom
deltoidu. Niekedy sa aj štvorec nepovažuje za kosoštvorec, preto
v~našom texte dopisujeme dodatky o~štvorci v~zátvorkách.},
alebo o~kosoštvorec (či štvorec).
Keďže však jeho druhá uhlopriečka $AI$ rozpoľuje uhol $KAL$,
musí to byť kosoštvorec (či štvorec).\fnote{Tento
zrejmý záver nemusia riešitelia dokazovať,
napriek tomu jeho dôkaz uvedieme: Rovnoramenné trojuholníky $AIK$
a~$AIL$ sú zhodné, lebo majú pri spoločnej základni $AI$ zhodné
vnútorné uhly $KAI$ a~$LAI$. Z toho vyplýva, že štvoruholník $AKIL$
má zhodné všetky štyri strany.}
\inspdf{a71s2.pdf}%

\Jres
Dokážeme, že konvexný štvoruholník $AKIL$ má protiľahlé strany
rovnobežné. Že je to nielen rovnobežník, ale
dokonca kosoštvorec (či štvorec), potom ako v~prvom riešení
zrejme vyplynie z~toho, že jeho uhlopriečka $AI$ leží na osi
jeho vnútorného uhla pri vrchole $A$.\fnote{Zo zhodnosti uhlov
$KAI$ a~$LAI$ potom totiž dostaneme, že sú s~nimi zhodné aj striedavé uhly
$LIA$, resp. $KIA$, takže oba trojuholníky $AIK$ a~$AIL$ sú naozaj
rovnoramenné so spoločnou základňou $AI$. Ani toto zrejmé
vysvetlenie nemusia riešitelia zapisovať.}

Z dvoch rovnobežností $AK\parallel LI$ a~$AL\parallel KI$
dokážeme iba tú prvú v~podobe $AB\parallel LI$; druhá rovnobežnosť
$AL\parallel KI$ sa dokáže úplne analogicky.
Postup založíme na vlastnostiach obvodových uhlov. V~kružnici $k$
(pri zvyčajnom označení $\gamma=|\uhel ACB|$) tak
rovnakú veľkosť $\frac12\gamma$ majú nielen uhly $ACT$ a~$TCB$,
ale aj uhol $TSB$ (pozri \obrr1{} vpravo).
Prvý a~tretí z~nich sú ale uhly $LCI$ a~$LSI$,
takže vďaka ich zhodnosti je štvoruholník~$LICS$ tetivový --
jemu opísaná kružnica je na obrázku vykreslená. V~nej je
obvodový uhol $LIS$ zhodný s~uhlom $LCS$, čo je
vlastne obvodový uhol $ACS$ v~kružnici~$k$, ktorý je zhodný
s~uhlom $ABS$. Zistili sme tak, že úsečky $LI$ a~$AB$
zvierajú s~priamkou $BS$ zhodné súhlasné uhly\fnote{Namiesto
priamky $BS$ sme mohli uvážiť aj priamku $AC$: $|\uhel ILC|=|\uhel
ISC|=|\uhel BSC|=|\uhel BAC|$.}, a~sú preto
rovnobežné, ako sme potrebovali dokázať. Riešenie je ukončené.

\Pozn
Záverečným krokom oboch riešení boli úvahy
o~druhej uhlopriečke $AI$ štvoruholníka $AKIL$. Tie možno nahradiť
dôkazom rovnosti $|KA|=|LA|$, ktorý teraz uvedieme: Stačí ukázať,
že sú zhodné uhly $AKL$ a~$ALK$, na ktoré sa pozrieme ako na
vonkajšie uhly trojuholníkov $AKT$, resp. $ALS$. Pre ich vnútorné uhly
ale platí $|\uhel ATK|=|\uhel ATS|=|\uhel ABS|=\frac12\beta$,
$|\uhel TAK|= |\uhel TAB|=|\uhel TCB|=\frac12\gamma$ a~analogicky
$|\uhel ASL|=\frac12\gamma$, $|\uhel SAL|=\frac12\beta$. Uhly $AKL$, $ALK$
tak majú tú istú veľkosť $\frac12\beta+\frac12\gamma$.

\schemaABC
Za úplné riešenie dajte 6~bodov.
V~neúplných riešeniach dajte čiastkové body za nasledujúce
poznatky (\emph{dokázané}, ak sa na ne nevzťahuje posledný odsek
pokynov):
\item{a)} Uhlopriečka $AI$ leží na osi uhla $KAL$ -- 0 bodov.
\item{b)} Rovnosť $|KA|=|LA|$ (zo záverečnej poznámky) -- 2 body.
\item{c)} Rovnosti $|KA|=|KI|$ a~$|LA|=|LI|$ -- za jedinú 3 body, za obe 4
body (aj pri konštatovaní, že štvoruholník $AKIL$ je súmerný
podľa svojej uhlopriečky $KL$).
\item{d)} Rovnobežnosti $AK\parallel LI$ a~$AL\parallel KI$ -- za jedinú 3
body, za obe 4 body.
\item{e)} Štvoruholník $LICS$ (prípadne $LIBT$, prípadne oba) je tetivový -- 2 body.
\endgraf\noindent
Čiastočné zisky za položky a)\,--\,e) \emph{sa nedajú} sčítať, \tj. počíta sa
najväčší z~nich.

V~poznámkach pod čiarou sme uviedli, ktoré poznatky možno prehlásiť
za známe alebo zrejmé. Platí to aj pre ďalšie poznatky z~riešenia
návodných a~dopĺňajúcich úloh k~domácemu kolu,
ak sa riešiteľ na text k~tomuto kolu odvolá.
\endschema}

{%%%%%   A-S-3
Po vynásobení zadanej rovnosti číslom $a^b/b^a$ dostaneme
$$
\biggl(\frac{a}{b}\biggr)^a=a^b.
$$
Mocnina na ľavej strane je celočíselná práve vtedy, keď je jej základ
$k=a/b$ celé (v~našom prípade kladné) číslo.\fnote{Toto zrejmé tvrdenie nemusia
riešitelia dokazovať. Vyplýva z~toho, že ak má zlomok $a/b$ po
skrátení základný tvar $u/v$, je každá jeho mocnina $(a/b)^n$
zlomkom so základným tvarom $u^n/v^n$.}
Po dosadení $a=kb$ sa zmení prepísaná
rovnosť na tvar $k^{kb}=(kb)^{b}$, odkiaľ po vydelení exponentov
číslom $b$ dostaneme $k^k=kb$, čiže $b=k^{k-1}$, teda
$a=kb=k^k$. Všetky hľadané dvojice sú preto tvaru
$(a,b)=\bigl(k^k,k^{k-1}\bigr)$, pričom $k\geqq1$ je ľubovoľné celé číslo.
Skúška vzhľadom na ekvivalentné úpravy nie je nutná.

\Pozn
Ukážeme, ako kľúčový poznatok o~tom, že číslo $a$ je násobkom
čísla~$b$, možno získať aj bez prepisu zadanej rovnosti na tvar so
zlomkom $a/b$. Využijeme na to predpoklad, že čísla $a$
a~$b$ sú základy sebe rovných mocnín $a^{a-b}$ a~$b^a$, pre
ktorých exponenty platí $a-b<a$.\fnote{Zatiaľ čo nerovnosť
$b\leqq a$ je za takéhoto predpokladu zrejmá (\uv{väčší exponent
-- menší základ}), menej zrejmú deliteľnosť $b\mid a$
je nutné v~riešení dokázať.} Pre postup uvážime
ľubovoľné prvočíslo $p$ a~počty jeho výskytov v~rozkladoch čísel
$a$, $b$ na súčin prvočísel. Máme vlastne ukázať, že pre tieto počty,
ktoré označíme $a_p$
a~$b_p$, platí nerovnosť $b_p\leqq a_p$. Porovnaním výskytov
prvočísla $p$ na oboch stranách rovnosti $a^{a-b}\!=\!b^a$ získame
vzťah $(a-b)\cdot a_p\!=\!a\cdot b_p$. Z toho v~prípade
$a_p=0$ máme aj $b_p=0$; v~prípade $a_p>0$ potom dostávame
$b_p/a_p=1-b/a<1$. Tým je želaná nerovnosť $b_p\leqq a_p$
dokázaná.


\Jres %Martin Melicher
Označme $d$ najväčší spoločný deliteľ čísel $a$ a~$b$.
Potom $a=pd$ a~$b=qd$, pričom $p,q\geqq1$ sú nesúdeliteľné celé čísla.
Dosadením do zadanej rovnosti a~následnými ekvivalentnými úpravami
postupne dostaneme
$$
\eqalign{
(pd)^{(p-q)d} &= (qd)^{pd},\cr
(pd)^{p-q} &= (qd)^{p},\cr
p^{p-q} &= q^p\cdot d^q.
}
$$
Keďže $p\geqq1$, je činiteľ z~pravej strany poslednej rovnosti
násobkom~$q$, takže násobkom~$q$ je aj~celočíselná mocnina $p^{p-q}$
z~ľavej strany.
Tá je však súčasne s~číslom $q$ nesúdeliteľná, lebo taký
je jej základ $p$. Preto musí byť nutne $q=1$.
Dosadením tohto $q$ do upravenej rovnosti
dostaneme $p^{p-1}=d$, takže všetky riešenia sú tvaru $a=pd=p^p$
a~$b=qd=p^{p-1}$, pričom $p$ je ľubovoľné celé kladné číslo.
Prichádzame tak k rovnakému záveru
ako v~prvom riešení. Skúška opäť nie je nutná.


\schemaABC
Za úplné riešenie dajte 6 bodov, z~toho:
\item{a)} 1 bod za správny popis \emph{všetkých} riešení (aj bez dôkazu,
uhádnutím či experimentovaním),
\item{b)} 1 bod za poznatok (aj bez dôkazu),
že $a$ je násobkom $b$ (resp. že $q=1$ pri označení z~druhého riešenia),
ďalšie 2 body za jeho dôkaz,
\item{c)} 2 body za postup vedúci od poznatku z~bodu b)
k~opisu z~bodu a)
\endgraf\noindent
Body za a)\,--\,c) možno sčítať. Za chýbajúcu skúšku body nestrhávajte.
\endschema
}

{%%%%%   A-II-1
Nie je to možné, dôkaz urobíme sporom.

Predpokladajme teda, že tabuľku $8\times 8$ máme požadovaným spôsobom
vyplnenú. Postupne zistíme, koľko riadkov a stĺpcov obsahuje koľko
šestiek a sedmičiek. Spor nakoniec vyplynie z~toho, že také
kombinácie počtov sa nedajú dosiahnuť.

Zamerajme sa najskôr na ľubovoľný riadok vyplnenej tabuľky.
Súčet všetkých čísel v~ňom je aspoň $8\cdot 6=48$ a nanajvýš
$8\cdot 7=56$. Keďže je to podľa predpokladu násobok piatich,
musí byť rovný jednému z~čísel 50 alebo 55. Zrejme je
v~prvom prípade v~riadku 6~šestiek a 2~sedmičky, v druhom
prípade v~ňom je 1~šestka a 7~sedmičiek.

Podobne súčet všetkých čísel v~ľubovoľnom stĺpci je podľa zadania
násobok siedmich opäť z~intervalu $\langle48;56\rangle$,
musí byť preto rovný jednému z~čísel 49 alebo 56. Zrejme je
v~prvom prípade v stĺpci 7~šestiek a 1 sedmička, v druhom
prípade v~ňom je 8~sedmičiek (a nie je žiadna
šestka).\fnote{Okamžite to vyplýva aj z~úvahy, že počet
šestiek v~každom stĺpci musí byť deliteľný siedmimi.}

Skúmajme ďalej iba počty šestiek (rovnako úspešne možno pracovať
iba s~počtami sedmičiek). Vzhľadom na ich vyššie určené možné
počty označíme $k\in\langle0;8\rangle$ počet riadkov,
ktoré obsahujú 6 šestiek, a $l\in\langle 0;8\rangle$ počet stĺpcov,
ktoré obsahujú 7 šestiek. Ako vieme, zvyšných $8-k$ riadkov obsahuje 1~šestku
a zvyšných $8-l$ stĺpcov neobsahuje žiadnu šestku. Prehľadne to zachytíme
na \obr{} (poradia riadkov ani stĺpcov nie sú podstatné).
\inspdf{a71ii1-1.pdf}%

Najskôr ukážeme, že nutne platí $k=l=4$ (ako máme na obrázku).
Z~počítania po stĺpcoch vyplýva, že celkový počet šestiek v~tabuľke
je $7l$, teda násobok siedmich, ktorý je z~počítania po riadkoch rovný číslu
$k\cdot6+(8-k)\cdot1=5k+8$. Pre možné $k\in\langle0;8\rangle$
je však číslo $5k+8$ deliteľné siedmimi zrejme iba pre $k=4$,
keď je rovné~28. To však už znamená, že aj $l=4$.

Situácia $k=l=4$ ale nastať nemôže (ako napovedá pohľad na našu
tabuľku): Z~rovnosti $l=4$ vyplýva, že v 4~stĺpcoch nie je žiadna šestka,
takže v~žiadnom riadku nemôžu byť viac ako štyri šestky.
To je v spore s~rovnosťou $k=4$, ktorá znamená,
že v štyroch riadkoch je šestiek dokonca 6.

\poznamka
Poznatok, že nutne platí $k=l=4$, možno odvodiť aj nasledujúcou
úvahou o~súčte $S$ všetkých čísel v~tabuľke.

Keďže súčet čísel v~každom riadku je deliteľný piatimi, je aj
číslo $S$ deliteľné piatimi. Rovnakou úvahou pre sčítanie po stĺpcoch
zistíme, že číslo $S$ je tiež deliteľné siedmimi.
Keďže čísla 5 a 7 sú nesúdeliteľné, je súčet $S$
deliteľný aj číslom $5\cdot 7=35$. Navyše zo sčítania po riadkoch
(osem sčítancov, každý ako vieme rovný 50 alebo 55) zisťujeme,
že číslo $S$ leží v~intervale
$\langle8\cdot 50;8\cdot 55\rangle=\langle400;440\rangle$.
V~ňom sa však nachádza jediný násobok čísla 35, konkrétne číslo
$12\cdot 35=420$. Preto platí nutne $S=420$, takže počet
šestiek~$N$ v~celej tabuľke spĺňa rovnicu $N\cdot 6 + (64-N)\cdot 7 =
420$, z ktorej vychádza $N= 28$. V~celej tabuľke je tak 28 šestiek
(a 36 sedmičiek), čo už vedie k rovnostiam $k=l=4$.

\schemaABC
Za úplné riešenie dajte 6 bodov. V~neúplných riešeniach (vrátane
pokusov o dôkaz sporom) oceňte čiastkové výsledky nasledovne.
\item{A1.} Správna (\tj. negatívna) odpoveď (aj bez zdôvodnenia, avšak slovne zapísaná): 1 bod.
\item{B1.} Určenie oboch možností $(6,2)$ a $(1,7)$ pre počty šestiek a sedmičiek v~každom riadku: 1~bod.
\item{B2.} Určenie oboch možností $(7,1)$ a $(0,8)$ pre počty šestiek a sedmičiek v~každom stĺpci: 1~bod.
\item{B3.} Dôkaz rovností $k=l=4$ (\tj. práve štyri riadky sú typu $(6,2)$ a práve štyri stĺpce sú typu $(7,1)$, pozri B1, resp. B2): 2 body.
\item{B4.} Spor v prípade $k=l=4$: 1 bod.
\item{C1.} Spor v každom z~prípadov, keď neplatí $k=l=4$: 5 bodov.
\endgraf\noindent
Celkovo potom dajte $\rm\max(A1,\,B1+B2+B3+B4,\,C1)$ bodov.

Určovanie počtov šestiek a sedmičiek v jednom riadku alebo stĺpci (či
v~celej tabuľke) podľa ich zadaného súčtu je natoľko zrejmé, že
ho nie je nutné opisovať (ako sme to urobili my až pre súčet
čísel v celej tabuľke v poznámke za riešením).
\endschema
}

{%%%%%   A-II-2
Dokážeme, že úloha má jediné riešenie $(x;y;z)=(10/3;5/3;10/3)$.

Všimnime si, že pravá strana druhej rovnice je aritmetickým
priemerom pravých strán zvyšných dvoch rovníc.
Ak preto odčítame podobne ako vo štvrtej úlohe domáceho kola od
súčtu prvej a tretej rovnice dvojnásobok druhej rovnice, vyjde
rovnica s~nulovou pravou stranou
$$
(x^2+2y^2) + (z^2+2x^2) - 2(y^2+2z^2)=0,\quad\hbox{čiže}\quad
3x^2-3z^2=0.
$$
Keďže $x$ a $z$ sú kladné čísla, vyplýva z~toho $x=z$.
Po dosadení $x$ za $z$ sa zmení pôvodná sústava na tvar
$$
\eqalign{
x^2 + 2y^2 &= 4x + 2y = 2(2x+y), \cr
2x^2 + y^2 &= 6x + 3y = 3(2x+y), \cr
3x^2 &= 8x + 4y = 4(2x+y).
}
$$
Porovnaním prvej a tretej rovnice dostávame
$2(x^2+2y^2)=4(2x+y)=3x^2$, teda $x^2=4y^2$, z~čoho vzhľadom
na $x,y>0$ vyplýva $x=2y$.
Po dosadení $2y$ za $x$ sa zmení upravená sústava na tvar
$$
\eqalign{
6y^2 &= 10y, \cr
9y^2 &= 15y, \cr
12y^2 &= 20y.}
$$
Keďže $y\ne 0$, je každá z troch rovníc ekvivalentná s~$y=5/3$.
Vzhľadom na odvodené vzťahy $x=2y$ a $z=x$ tak nutne platí
$(x;y;z)=(10/3;5/3;10/3)$. Skúška pri tomto postupe nie je nutná.

\poznamka
Ak si nevšimneme závislosť medzi pravými stranami zadaných rovníc,
môžeme na danú sústavu pozerať ako na sústavu troch
lineárnych rovníc s~neznámymi $x^2$, $y^2$, $z^2$ a parametrami $x$, $y$, $z$.
Jej riešením dostaneme vyjadrenia
$$
x^2=\frac{3x+4y+5z}{3},\quad y^2=\frac{y+2z}{3},\quad
z^2=\frac{3x+4y+5z}{3}.
$$
Teraz už je očividné, že platí $x^2=z^2$, čiže $x=z$.
Po dosadení $x$ za $z$ tak z~odvodených vyjadrení dostaneme
$$
x^2=\frac{8x+4y}{3}=\frac{4(2x+y)}{3}\quad\hbox{a}\quad
y^2=\frac{2x+y}{3}.
$$
Teraz vidíme, že $x^2=4y^2$, čiže $x=2y$. Nutne teda platia
rovnosti $x=2y=z$. Ďalej už postupujeme rovnako ako v~pôvodnom
riešení.

Doplňme túto poznámku ešte zmienkou o~iných dôsledkoch zadanej
sústavy rovníc, ktorými sú kvadratické rovnice pre dve z troch
neznámych $x$, $y$, $z$. Ak napríklad od dvojnásobku
tretej rovnice odčítame druhú rovnicu, dostaneme
$$
4x^2-y^2=2(3x+4y+5y)-(2x+3y+4z)=4x+5y+6z.
$$
Ak sem dosadíme za $z$ (či rovno za $3z$) z~prvej rovnice,
získame pre neznáme $x$, $y$ po jednoduchej úprave kvadratickú rovnicu
$4x^2-5y^2=2x+y$. Podobne po odčítaní tretej rovnice od dvojnásobku
prvej rovnice možno použitím druhej rovnice získať pre neznáme $y$, $z$
rovnicu $3y^2=y+2z$. Najlepší výsledok ale dáva odčítanie prvej
rovnice od dvojnásobku druhej rovnice, keď s~prihliadnutím na tretiu
rovnicu získame pre neznáme $x$, $z$ rovnicu $x^2-z^2=0$.

\schemaABC
Za úplné riešenie dajte 6 bodov. V žiadnom riešení absenciu skúšky nájdeného riešenia nepenalizujte.

V~neúplných riešeniach ohodnoťte čiastkové výsledky nasledovne.
\item{A1.} Uvedenie riešenia $(x;y;z)=(10/3;5/3;10/3)$ (aj bez zdôvodnenia): 1~bod.
\item{A2.} Odvodenie aspoň jednej z~kvadratických rovníc pre dve neznáme, ktorá nie je dôsledkom vzťahov $x=2y=z$, ako napr. $4x^2-5y^2=2x+y$ alebo
$3y^2=y+2z$: 1~bod.
\item{B1.} Odvodenie vzťahu $x=z$: 3 body.
\item{B2.} Odvodenie vzťahu $x=2y$ alebo $z=2y$: 2 body.
\endgraf\noindent
Celkovo potom dajte $\rm\max(A1+A2,\, B1+B2)$ bodov.

Ak riešiteľ eliminuje jednu z~neznámych, napríklad $z$,
a to tak, že vyjadrenie $z$ z~prvej rovnice dosadí do zvyšných
dvoch rovníc, žiadny bod neudeľujte, ak nie je pri pokuse o~riešenie
získanej sústavy dvoch rovníc štvrtého stupňa o~dvoch neznámych dosiahnutý
významný pokrok, akým sú vzťahy z~A2, B1 či~B2.
\endschema
}

{%%%%%   A-II-3
Z~rovnobežnosti $PR\parallel AB$ a zo zhodnosti obvodových uhlov
$BAQ$ a~$BCQ$ v~kružnici $k$ opísanej trojuholníku $ABC$ vyplýva zhodnosť
uhlov $RPQ$ a $RCQ$. Naozaj,
$$
|\uhel RPQ|=|\uhel BAQ|=|\uhel BCQ|=|\uhel RCQ|.
$$
Body $P$, $R$, $Q$ a $C$ teda ležia v~tomto poradí na jednej
kružnici (\tj. štvoruholník $PRQC$ je tetivový).
Uvedieme dva spôsoby, ako riešenie dokončiť.

\emph{Prvý spôsob}.
V~kružnici opísanej štvoruholníku $PRQC$ sú zhodné
obvodové uhly $PQR$ a $PCR$ (pozri \obr{} vľavo).
Uhol $PCR$ je však tiež zhodný s~uhlom $PCA$,
lebo v~rovnoramennom trojuholníku leží výška
z~hlavného vrcholu na osi vnútorného uhla. Napokon s~prihliadnutím
na zhodné obvodové uhly $ACB$ a $AQB$ v~kružnici $k$
dokopy dostávame
$$
|\uhel PQR|=|\uhel PCR|=\frac12|\uhel ACB|=\frac12|\uhel AQB|,
$$
a preto polpriamka $QR$ je osou uhla $AQB$, ako sme mali dokázať.

\emph{Druhý spôsob}.
Označme $S$ taký bod, že úsečka $CS$ je priemerom kružnice $k$
(pozri \obrr1{} vpravo).
Keďže štvoruholník $PRQC$ je tetivový, je pravý nielen uhol $CPR$,
ale aj uhol $CQR$, navyše je pravý aj uhol $CQS$ v~kružnici $k$.
Zhodnosť uhlov $CQR$ a~$CQS$ znamená, že polpriamky $QR$ a $QS$
splývajú. Vďaka rovnosti $|AC|=|BC|$ platí aj rovnosť
$|AS|=|BS|$, teda bod $S$ je stredom toho oblúka $AB$
kružnice $k$, ktorý neobsahuje bod~$Q$. Polpriamka $QS$,
čiže $QR$, je tak naozaj osou uhla $AQB$, lebo zhodným oblúkom
$AS$ a $BS$ kružnice $k$ prislúchajú zhodné obvodové uhly $AQS$ a $BQS$.
\inspdf{a71ii3-1.pdf}%

\ineriesenie
Úlohu vyriešime, keď dokážeme, že bod $R$ je stredom kružnice vpísanej
trojuholníku $PBQ$. Na to stačí overiť, že polpriamky $PR$ a $BC$ sú
osami vnútorných uhlov $BPQ$, resp. $PBQ$ (pozri \obr).
Urobíme to osobitne v dvoch odsekoch.

\item{$\triangleright$} \emph{pre os uhla $BPQ$}: Potrebnú zhodnosť uhlov $BPR$ a
$QPR$ odvodíme z~toho, že $PR\parallel AB$ a že $AB$ je základňa
rovnoramenného trojuholníka $ABP$. Z toho už máme
$$
|\uhel BPR|=|\uhel PBA| =|\uhel PAB|=|\uhel QPR|.
$$

\item{$\triangleright$} \emph{Pre os uhla $PBQ$}: Potrebnú zhodnosť uhlov $PBC$
a $QBC$ odvodíme z~toho, že uhol $PBC$ je podľa osi $CP$
súmerne združený s~uhlom $PAC$ a že $QAC$ a $QBC$ sú
zhodné obvodové uhly v~kružnici $k$. Z toho už máme
$$
|\uhel PBC|=|\uhel PAC|=|\uhel QAC|=|\uhel QBC|.
$$
\inspdf{a71ii3-2.pdf}%


\schemaABC
Za úplné riešenie dajte 6 bodov.
V~neúplných riešeniach ohodnoťte čiastkové výsledky nasledovne.
\item{A1.} Štvoruholník $PRQC$ je tetivový: 3 body s~dôkazom, 0 bodov bez dôkazu.
\item{A2.} Dokončenie dôkazu za predpokladu A1 (aj nedokázaného): 2 body.
\item{B1.} Polpriamka $PR$ je osou uhla $BPQ$: 2 body s~dôkazom, 0 bodov bez dôkazu.
\item{B2.} Polpriamka $BC$ je osou uhla $PBQ$: 2 body s~dôkazom, 0 bodov bez dôkazu.
\item{B3.} Dokončenie dôkazu za predpokladov B1 a B2 (aj nedokázaných): 1~bod.
\endgraf\noindent
Celkovo potom dajte $\rm\max(A1+A2,\,B1+B2+B3)$ bodov.
\endschema
}

{%%%%%   A-II-4
Tvrdenie dokážeme sporom. Pripusťme preto, že postupnosť
obsahuje iba konečne veľa zložených čísel, teda že od určitého člena
$a_m$ vrátane obsahuje postupnosť už iba prvočísla.
Keďže postupnosť je rastúca (lebo pre každé $t\ge 1$ platí
$2022t+1>2022t-1>t$), môžeme spomenutý index $m$ vybrať tak, aby
navyše platilo $a_m>5$, a teda aj $a_i>5$ pre všetky indexy $i\ge m$.

Skúmajme zvyšky členov $a_m, a_{m+1},a_{m+2},\dots$ po delení
piatimi. Schéma na \obr{} vpravo znázorňuje, ktoré zvyšky po delení
piatimi dávajú čísla $2022t+1$ (modré šípky) a~$2022t-1$ (červené šípky)
v~závislosti od toho, ktorý zvyšok dáva číslo $t$. Keďže nami
skúmané členy $a_i$ ($i\ge m$) nie sú násobkami piatich
(vďaka výberu indexu $m$ to sú totiž prvočísla väčšie ako 5),
nezakreslili sme do nášho grafu šípky vychádzajúce z~uzla $t=0$.
\inspdf{a71ii4-1.pdf}%

Keďže nekonečná postupnosť prvočísel $a_m, a_{m+1},a_{m+2},\dots$
neobsahuje žiadne číslo deliteľné piatimi, je zo schémy zrejmé, že
nastane práve jeden z troch prípadov:

\item{a)} Všetky jej členy dávajú zvyšok 4.
\item{b)} Všetky jej členy dávajú zvyšok 1.
\item{c)} Od istého jej člena dávajú členy striedavo zvyšky 2 a 3.

\noindent
Každý z~týchto troch prípadov dovedieme ku sporu podobným spôsobom.
Za tým účelom vždy využijeme pomocné tvrdenie, ktoré vo všeobecnej podobe
najskôr sformulujeme a~dokážeme.

\smallskip\noindent{\it Tvrdenie}.
{\sl
Uvažujme nekonečnú rastúcu postupnosť čísel
$x_0,x_1,x_2,\dots$, ktorá pre každý index $i\ge 0$
spĺňa rovnosť $x_{i+1}=q\cdot x_i+d$, pričom $x_0\ge2$, $q\ge 1$ a $d$
sú celé čísla. Potom niektorý jej člen $x_i$ je zložené číslo.
}

\smallskip
Dôkaz tvrdenia urobíme sporom podobne ako
v druhom riešení šiestej úlohy domáceho kola.
Predpokladajme teda ďalej, že všetky členy $x_0$, $x_1$
$x_2,\dots$ z~nášho tvrdenia sú prvočísla.

Keďže je daná postupnosť rastúca, vyberieme index $i\ge 0$ taký,
že člen $x_i$ je rovný takému prvočíslu $p$, ktoré je s~číslom~$q$
nesúdeliteľné. Dokážeme, že sa potom niektorý násobok
tohto prvočísla $p=x_i$ rovná jednému z~väčších prvočísel
$x_{i+1},x_{i+2},\dots$ Dôkaz sporom tak bude ukončený.

Uvažujme zobrazenie $f\colon z\mapsto qz+d$ na množine
$M_p=\{0,1,2,\dots,p-1\}$ všetkých zvyškov po delení $p$. Ukážeme,
že zobrazenie $f$ je prosté. Naozaj, zo vzťahu
$$
p\mid f(z)-f(z')=(qz+d)-(qz'+d)=q(z-z')
$$
vďaka nesúdeliteľnosti $p$ a $q$ vyplýva $p\mid z-z'$,
takže pre rôzne zvyšky $z,z'\in M_p$ platí $f(z)\nequiv f(z')\pmod p$.
Podľa návodnej úlohy N3 k~šiestej úlohe
domáceho kola sa preto zvyšky čísel $x_i,x_{i+1},x_{i+2},\dots$
opakujú periodicky od prvého miesta (\tj. bez predperiódy).
Keďže zvyšok prvého člena $x_i=p$ je rovný $0$, má rovnaký
zvyšok $0$ po delení prvočíslom~$p$
dokonca nekonečne členov $x_j$ s~indexmi $j>i$. Tým je dôkaz tvrdenia ukončený.

\smallskip
Teraz použitím dokázaného tvrdenia rozoberieme vyššie určené prípady
a)~--~c).

\item{a)} Pre každé $i\ge m$ platí $a_{i+1}=2022a_i+1$, takže sporný záver
vyplýva okamžite z~nášho tvrdenia pre postupnosť $a_m,a_{m+1},\dots$ a
hodnoty $q=2022$, $d=1$.
\item{b)} Pre každé $i\ge m$ platí $a_{i+1}=2022a_i-1$, takže sporný záver
vyplýva okamžite z~nášho tvrdenia pre postupnosť $a_m,a_{m+1},\dots$ a
hodnoty $q=2022$, $d={-1}$.
\item{c)} Označme $n\ge m$ niektorý index, pre ktorý platí
$a_n\equiv 2\pmod 5$. Potom postupnosť $a_n,a_{n+2},a_{n+4},\dots$
pre každé $k\ge 0$ spĺňa rovnosti
$$\eqalign{
a_{n+2k+2}&=2022a_{n+2k+1}+1 = 2022\cdot(2022a_{n+2k}-1)+1 =\cr
&=2022^2\cdot a_{n+2k}-2021,}
$$
takže sporný záver vyplýva z~nášho tvrdenia pre postupnosť
$a_n,a_{n+2},a_{n+4},\dots$ a~hodnoty $q=2022^2$, $d={-2021}$.

\noindent
Tým je celé riešenie úlohy ukončené.

%%%%%%%%%%%%%%%%%%%%%%%%%%%%%%%%%%%%%%%%%%%%%%%%

\poznamka
Ukážeme stručne iný spôsob, ako môžeme vylúčiť všetky tri prípady a)~--~c) z~predchádzajúceho riešenia bez toho, aby sme použili dokázané tvrdenie.

\item{a)} Pre každé $i\ge m$ je $a_{i+1}=2022a_i+1$, takže $a_{i+1}\equiv a_i+1\pmod{2021}$. Z toho indukciou pre každé celé $k\ge0$ dostávame $a_{m+k}\equiv a_m+k\pmod{2021}$. Vidíme, že $2021\mid a_{m+k}$ zakaždým, keď $2021\mid a_m+k$, čo spĺňa nekonečne veľa celých $k\ge0$.


\item{b)} Pre každé $i\ge m$ je $a_{i+1}=2022a_i-1$, odkiaľ indukciou pre každé celé $k\ge0$ vyplýva $a_{m+k}\equiv a_m-k \pmod{2021}$ s~týmto sporným záverom: $2021\mid a_{m+k}$ zakaždým, keď $2021\mid a_m-k$.


\item{c)} Vyberieme $n\ge m$ také, že pre každé celé $i\ge0$ platí $a_{n+2i+1}=2022a_{n+2i}+1$ a~$a_{n+2i+2}=2022a_{n+2i+1}-1$, odkiaľ $a_{n+2i+1}\equiv\m a_{n+2i}+1\pmod{2023}$ a $a_{n+2i+2}\equiv\m a_{n+2i+1}-1\pmod{2023}$. Vylúčením $a_{n+2i+1}$ z~posledných dvoch kongruencií dostaneme $a_{n+2i+2}\equiv a_{n+2i}-2\pmod{2023}$, takže indukciou pre každé celé $k\ge0$ vychádza $a_{n+2k}\equiv a_{n}-2k\pmod{2023}$ s~týmto sporným záverom: $2023\mid a_{n+2k}$ zakaždým, keď $2023\mid a_{n}-2k$.

\schemaABC
Za úplné riešenie dajte 6 bodov.
Poznatky z~riešenia šiestej úlohy domáceho kola (vrátane návodných a~dopĺňajúcich úloh) možno prehlásiť za známe.
V~neúplných riešeniach podľa vzorového postupu ohodnoťte
čiastkové výsledky nasledovne.
\item{A1.} Rozhodnutie uvažovať zvyšky členov $a_i$ po delení číslom 5: 0 bodov.
\item{A2.} Dôkaz, že nastane jeden z troch prípadov a), b), c): 3 body.
\item{A3.} Vyriešenie aspoň jedného z~prípadov a), b): 1 bod.
\item{A4.} Vyriešenie prípadu c): 1 bod.
\endgraf\noindent
Za neúplné riešenia potom dajte $\rm A2+A3+A4$ bodov.
\endschema
}

{%%%%%   A-III-1
Dokážeme, že postupnosť 71 čísel na tabuli musí byť periodická
s~periódou~5. Keďže rozdiel $71-1=70$ je násobok piatich, bude tým úloha
vyriešená.

Označme ľubovoľných šesť po sebe napísaných čísel postupne
$a,b,c,d,e,f$. Sľúbili sme dokázať, že $f=a$. Za tým účelom pomocou čísel $a$, $b$
postupne vyjadrujme čísla $c,d,e,f$. Zo zadania máme $b=ac-1$,
čo možno vďaka podmienke $a\ne 0$ prepísať na $c=(b+1)/a$. Podobne
ďalej dostaneme
$$\eqalign{
d&=\frac{c+1}b = \frac{\frac{b+1}a+1}b = \frac{a+b+1}{ab},\cr
e&=\frac{d+1}{c} = \frac{\frac{a+b+1}{ab} +1}{\frac{b+1}a}=
\frac{ab+a+b+1}{ab}\cdot \frac{a}{b+1} =
\frac{(a+1)(b+1)a}{ab(b+1)}= \frac{a+1}b,\cr
f&=\frac{e+1}{d} =\frac{\frac{a+1}b+1}{ \frac{a+b+1}{ab}} =
\frac{a+b+1}{b}\cdot \frac{ab}{a+b+1} = a,\cr
}$$
pričom krátenie číslami $b+1$ a $a+b+1$ bolo korektné, pretože
sa jedná o~čitateľa zlomkov, ktorými sme predtým vyjadrili nenulové čísla
$c$ a $d$. Sme hotoví.

\ineriesenie
Iným spôsobom dokážeme, že pre každých šesť
po sebe napísaných čísel $a$, $b$, $c$, $d$, $e$, $f$ platí $a=f$.
Odvodíme totiž rovnosť $abcde=bcdef$, z~ktorej požadovaný záver
vyplynie po vydelení oboch strán nenulovým súčinom $bcde$.

Vďaka zadanej podmienke (aplikovanej nižšie na podčiarknuté súčiny)
môžeme písať
$$\eqalign{
abcde&=
(\underline{\vphantom{f}ac})(\underline{\vphantom{f}bd})e=(b+1)(c+1)e
=(b+1)(\underline{\vphantom{f}ce}+e)=(b+1)(d+e+1)\cr
&=\underline{\vphantom{f}bd}+be+b+d+e+1=be+b+c+d+e+2.
}$$
Analogicky pre druhý súčin $bcdef$ vychádza
$$\eqalign{
bcdef=fedcb &= (\underline{fd})(\underline{\vphantom{f}ec})b
=(e+1)(d+1)b=(e+1)(\underline{\vphantom{f}bd}+b)=(e+1)(c+b+1)\cr
&=\underline{\vphantom{f}ec}+eb+e+c+b+1=be+b+c+d+e+2.
}$$
Vidíme, že rovnosť $abcde=bcdef$ naozaj
platí.\fnote{Výpočet druhého súčinu $fedcb$ nebol nevyhnutný.
Stačilo konštatovať, že podmienka zo zadania úlohy nezávisí od toho,
v~akom z~oboch možných smerov napísaný rad čísel prečítame, a že
výsledok pre prvý súčin $abcde$ závisí iba od štvorice $(b,c,d,e)$
a je rovnaký ako pre štvoricu $(e,d,c,b)$.}

\poznamka
Tvrdenie úlohy všeobecne neplatí, ak pripustíme, že niektoré
z~napísaných čísel sa môžu rovnať nule. V~tom prípade
možno totiž ľubovoľne za seba zoraďovať bloky troch dĺžok:
bloky $B_5=(0,{-1},a,{-a-1},{-1})$ dĺžky 5 (napríklad aj pre rôzne reálne
čísla $a$), bloky $B_3=(0,{-1},{-1})$ dĺžky 3 a~bloky $B_2=(0,{-1})$ dĺžky 2. Tak napríklad vyhovujúci rad 71~čísel
$$
\underbrace{B_5B_5\dots B_5}_{13\hbox{\sevenrm-krát}} B_3B_3
$$
začína číslom 0 a končí číslom ${-1}$.
}

{%%%%%   A-III-2
Ukážeme, že zadaná 35-prvková množina $M$ obsahuje aspoň 18
spravodlivých čísel, teda viac ako je v~nej čísel,
ktoré spravodlivé nie sú.

Stačí nám dokázať, že každý kladný deliteľ čísla $2^{1010}\cdot 3^2\cdot 5^{1010}$
je číslo spravodlivé. Z~toho totiž vyplynie, že všetkých 18 čísel
v~podmnožine
$$
S=\{1,2,3,4,5,6,8,9,10,12,15,16,18,20,24,25,30,32\}
$$
množiny $M$ je spravodlivých. Ideu dôkazu spravodlivosti každého
$k$ s~vlastnosťou $k\mid 2^{1010}\cdot 3^2\cdot 5^{1010}$ uvedieme
hneď potom, čo v~nasledujúcom odseku rozdelíme všetky palindrómy
dĺžok 2021 a 2022 do vhodných skupín, zavedieme pre ne účelné označenia
a doplníme poznatky o~deliteľnosti, ktoré sprehľadnia ďalší dôkaz.

Uvažujme ľubovoľné 1010-ciferné číslo $n$ a označme $A_n$ (resp.
$B_n$) množinu tých 2021-ciferných (resp. 2022-ciferných) palindrómov,
ktorých prvé 1010-číslie je dané $n$. Týmto spôsobom máme všetky skúmané
palindrómy rozdelené do skupín $A_n$, resp.~$B_n$,
pričom čísla v~jednej skupine sa zhodujú nielen v~prvom, ale aj
v~poslednom 1010-číslí. Teda k~zadaniu čísla z~$A_n$, resp. $B_n$ stačí
určiť jeho prostrednú cifru, resp. jeho prostredné dvojčíslie, a
to ľubovoľným výberom z~množiny
$$
J=\{0,1,2,\dots,9\},\quad\hbox{resp.}\quad
D=\{00,11,22,33,44,55,66,77,88,99\}.
$$
Preto je v~$A_n$ aj $B_n$ práve 10 čísel a my ich ďalej budeme
označovať zápismi $[n,j]$, resp. $[n,d]$,
pričom $j\in J$ a $d\in D$. Fakt, že všetky palindrómy $[n,j]$ a
$[n,d]$ s~daným $n$ končia tým 1010-číslím, ktoré dostaneme prečítaním čísla $n$
sprava doľava, bude pre nás mať dva dôsledky: Čísla $[n,j]$ a $[n,d]$
jednak dávajú po delení číslami 3 a 9
rovnaké zvyšky ako čísla $2n+j$, resp. $2n+d$ (lebo
$10^i\equiv1\pmod{9}$ pre každé celé $i\geqq0$),
jednak je ich deliteľnosť číslami $2^a$ a $5^a$, pričom
$a\in\langle1,1010\rangle$ je celé, určená
prvými $a$~ciframi daného~$n$.

Teraz už sme pripravení dokázať spravodlivosť každého čísla
$k=2^a\cdot 3^b\cdot 5^c$, pričom $a,c\in\langle0,1010\rangle$ a
$b\in\langle0,2\rangle$ sú (aj všade ďalej) celé čísla.
Za tým účelom ukážeme, že počet násobkov takého $k$ je
v~oboch množinách $A_n$ a $B_n$ rovnaký, nech už je prípustné~$n$ akékoľvek.
Sčítaním týchto počtov cez všetky možné 1010-ciferné čísla $n$
potom dostaneme požadovaný záver.

Tvrdenie o~zhodnom počte násobkov daného $k$ v~množinách
$A_n$ a $B_n$ s daným (všade ďalej pevným) číslom $n$ je zrejmé
v~prípade~$k=1$ (oba počty sú 10).
Pre ostatné vymedzené $k$ rozlíšime 4 prípady,
v~ktorých budeme poznatky o~deliteľnosti (z~tretieho odseku riešenia)
používať bez odkazov.

\item{a)}
Prípad $k=2^a5^c$, pričom $\max(a,c)\in\langle1,1010\rangle$.
Deliteľnosť čísel $[n,j]$ a $[n,d]$ takým číslom $k$ je
určená prvými $\max(a,c)$ ciframi čísla $n$, takže tvrdenie platí --
buď sú násobky $k$ všetky čísla v~$A_n$ aj $B_n$ (po desiatich),
alebo ním nie je žiadne číslo v~$A_n$ ani~$B_n$.

\item{b)}
Prípad $k=3^2=9$.
Čísla $[n,j]$ a $[n,d]$ dávajú po delení deviatimi rovnaký zvyšok
ako čísla $2n+j$, resp. $2n+d$. Zvyšky čísel $d\in D$ po delení deviatimi
sú podľa zápisu $D$ v~a) postupne
$0,\,2,\,4,\,6,\,8,\,1,\,3,\,5,\,7,\,0$, teda rovnako ako pre čísla
$j\in J$ sa každý zvyšok až na nulu vyskytuje raz a nula sa
vyskytuje dvakrát. Ak je teda $n$ násobkom deviatich, obsahujú
množiny $A_n$, $B_n$ po dvoch násobkoch deviatich, v~opačnom prípade
po jednom násobku deviatich.

\item{c)}
Prípad $k=3^1$.
Čísla $[n,j]$ a $[n,d]$ môžeme modulo 3 opäť zameniť číslami $2n+j$,
resp. $2n+d$. Po delení tromi dávajú čísla $j\in J$ postupne zvyšky
$0,\,1,\,2,\,0,\,1,\,2,\,0,\,1,\,2,\,0$ a
dvojčíslia $d\in D$ zvyšky $0,\,2,\,1,\,0,\,2,\,1,\,0,\,2,\,1,\,0$.
Ak je teda $n$ násobkom troch, obsahujú obe množiny $A_n$, $B_n$
po štyroch násobkoch troch, inak obsahujú po troch násobkoch troch.

\item{d)}
Prípad $k=2^a\cdot3^b\cdot 5^c$, pričom $\max(a,c)\in\langle1,1010\rangle$
a $b\in\langle1,2\rangle$. Uvažujme najskôr číslo~$k'=2^a5^c$.
Z~dôkazu prípadu a) vyplýva, že buď sú všetky čísla
v~$A_n$ aj $B_n$ násobky~$k'$, alebo žiadne z~nich také nie je.
Ak nastáva prvá možnosť, z~výsledku prípadu $k=3^b$ vyplýva,
že obe množiny $A_n$ aj $B_n$ obsahujú rovnaký počet násobkov $3^b$,
a~teda aj násobkov čísla $3^b\cdot k'=k$. Ak nastáva druhá možnosť,
žiadna z~množín $A_n$, $B_n$ neobsahuje násobok~$k'$,
a~teda ani žiadny násobok~$k$.

\smallskip
Tým je celé riešenie ukončené.

\poznamka
Použitím počítača možno overiť, že zadaná množina $M$
obsahuje \emph{práve} 18~spravodlivých čísel, ktoré sme teda
v~našom riešení určili v~plnom počte.
}

{%%%%%   A-III-3
Bez ujmy na všeobecnosti predpokladajme, že platí $|AB|<|AC|$ a že
bod~$D$ leží na úsečke $AE$ (ako máme na oboch obrázkoch).
V iných prípadoch stačí vymeniť označenie bodov $B\leftrightarrow C$,
resp. $D\leftrightarrow E$.

Označme ešte $k$ kružnicu opísanú trojuholníku $ABC$ a
$S$ jej priesečník s~osou uhla $BAC$ ($S\ne A$).
Tento (tzv. Švrčkov) bod $S$ je v našom prípade
stredom \emph{kratšieho} oblúka~$BC$ kružnice $k$,
lebo podľa zadania je uhol $BAC$ ostrý. Z~definície bodu~$N$
potom vyplýva, že úsečka $SN$ je priemerom kružnice $k$ ležiacim na osi
jej tetivy $BC$. Stred~$M$ tejto tetivy preto leží na úsečke~$SN$
tak, že platí $|MS|<|MN|$. Súčasne uhol $BSC$ je tupý,
čo vzhľadom na pravé uhly $BDC$ a $BEC$ znamená,
že bod~$S$ leží vnútri úsečky~$DE$.
Keďže $DED'E'$ je pravouholník, sú úsečky $DD'$, $EE'$ (rovnako
ako $BC$) priemery zadanej kružnice~$\omega$, takže jej stred $M$
je aj stredom úsečiek $DD'$ a $EE'$.

Po týchto úvodných pozorovaniach uvedieme niekoľko spôsobov, ako
riešenie dokončiť. Budeme v nich bez odkazov využívať
známe vlastnosti mocnosti bodu ku kružnici a~obvodových uhlov.

\medskip
\emph{Prvý spôsob}. Vyjdeme z toho, že bod $M$ je spoločným
vnútorným bodom tetív $SN$, $BC$ kružnice $k$, ako aj tetív
$BC$, $DD'$, $EE'$ kružnice $\omega$. Platí tak reťazec rovností
$$
|MS|\cdot|MN|=|MB|\cdot|MC|=|MD|\cdot |MD'|=|ME|\cdot |ME'|.
$$
Z toho vyplývajúca rovnosť prvého súčinu posledným dvom súčinom
znamená práve to, že oba štvoruholníky $SDND'$ a $SENE'$
(s~priesečníkmi uhlopriečok v~bode~$M$) sú tetivové.
Vďaka tomu platí (ako je farebne vyznačené na \obr)
$$
|\uhol ND'M|=|\uhol ND'D|=|\uhol NSD|\quad\hbox{a}\quad
|\uhol NE'M|=|\uhol NE'E|=|\uhol NSE|.
$$
\inspdf{a71iii_p31.pdf}%

Keďže však $|\uhol NSD|+|\uhol NSE|=180^\circ$, máme aj
$|\uhol ND'M|+|\uhol NE'M|=180^\circ$. Štyri body z poslednej
rovnosti tak naozaj ležia na jednej kružnici (ako máme dokázať),
ak ležia vrcholy $D'$, $E'$ uhlov $ND'M$, resp. $NE'M$
v~opačných polrovinách s~hraničnou priamkou~$MN$.
Táto priamka však pretína úsečku $DE$ (v~bode $S$), a~teda
aj úsečku~$D'E'$ (súmerne združenú podľa stredu $M$),
takže sme hotoví.

\medskip
\emph{Druhý spôsob}. Uvážime obraz $S'$ bodu $S$ v~súmernosti
podľa stredu $M$, v ktorej $D\mapsto D'$ a $E\mapsto E'$.
Keďže podľa úvodnej časti bod $S$ leží vnútri úsečky $DE$
a platí $|MS|<|MN|$, leží bod $S'$ vnútri úsečiek
$D'E'$ a $MN$ (\obr). Stačí nám dokázať rovnosť
$|S'D'|\cdot|S'E'|=|S'M|\cdot|S'N|$.
\inspdf{a71iii_p32.pdf}%

Najskôr zo súmernosti podľa stredu $M$ a z~definície
mocnosti bodu~$S$ vzhľadom ku kružnici~$\omega(M,|BM|)$ máme
$$
|S'D'|\cdot|S'E'| =|SD|\cdot|SE| =|BM|^2-|SM|^2.
$$
Ďalej použitím rovnosti $|S'M|=|SM|$ a
Euklidovej vety pre výšku $BM$ pravouhlého trojuholníka $SNB$
dostávame
$$
|S'M|\cdot|S'N|=|S'M|\cdot (|MN|-|S'M|) =|SM|\cdot|MN|
-|SM|^2 = |BM|^2-|SM|^2.
$$
Tým je avizovaná rovnosť dokázaná.

\medskip
\emph{Tretí spôsob}. Opäť uvážime bod $S'$ z druhého postupu
(pozri \obrr1) a tentoraz využijeme kružnicovú inverziu
podľa zadanej kružnice $\omega$. Keďže pri tomto zobrazení
sú body $D',E'\in\omega$ samodružné a stred~$M$ kružnice~$\omega$
neleží na priamke~$D'E'$, bude, ako je známe, obrazom tejto priamky
kružnica prechádzajúca bodmi $D'$, $E'$, $M$. Na nej ale bude
tiež ležať obraz bodu $S'$, lebo $S'\in D'E'$. Ak teda ukážeme, že zmieneným obrazom bodu $S'$ je práve bod $N$,
budeme s riešením úlohy hotoví.

Euklidova veta pre výšku $BM$ pravouhlého trojuholníka
$SNB$ dáva rovnosť $|BM|^2=|MS|\cdot |MN|=|MS'|\cdot |MN|$.
Keďže $M$ je stred a $|BM|$ polomer kružnice~$\omega$,
podľa ktorej invertujeme, a keďže bod $N$ leží na
polpriamke~$MS'$,
je podľa rovnosti $|BM|^2=|MS'|\cdot|MN|$ bod $N$ naozaj
obrazom bodu $S'$, ako sme sľúbili ukázať.
}

{%%%%%   A-III-4
V prvej časti riešenia dokážeme, že úsečky $BO_1$ a $CO_2$ ležia v
polrovine $BCP$ a sú rovnobežné, v druhej časti ukážeme, že
tieto úsečky majú rovnakú dĺžku. Dokopy to už bude znamenať, že
$O_1BCO_2$ je rovnobežník.

\smallskip
\emph{Prvá časť}.
Rovnoramenné trojuholníky $ABC$, $BCD$ majú pri svojich základniach $AC$,
resp. $BD$ ostré vnútorné uhly, ktorých veľkosť označíme $\alpha$,
resp. $\beta$. Tieto uhly sú na \obr{} vyznačené jedným oblúčikom.
Ako hneď zdôvodníme, dvoma oblúčikmi zodpovedajúcej farby sú
vyznačené uhly veľkosti $90^{\circ}-\alpha$, resp. $90^{\circ}-\beta$.
\inspdf{a71iii_p4.pdf}%

Keďže trojuholník $ABP$ má pri vrchole $A$ ostrý uhol $\alpha$, leží stred
$O_1$ jemu opísanej kružnice v polrovine $BPA$ a konvexný stredový
uhol $PO_1B$ má veľkosť $2\alpha$. Preto v~rovnoramennom trojuholníku $BPO_1$
majú uhly pri vrcholoch $B$, $P$ avizovanú veľkosť $90^{\circ}-\alpha$.
Analogicky sa dokáže, že stred $O_2$ leží v polrovine $CPD$ a
v~rovnoramennom trojuholníku $CPO_2$ majú uhly pri vrcholoch $C$, $P$ veľkosť
$90^{\circ}-\beta$.

Keďže ostré uhly $O_1BP$ a $CBP$ ležia na odlišných stranách
od spoločného ramena~$BP$, je uhol $O_1BC$ s vnútorným bodom $P$
konvexný, leží tak v polrovine $BCP$ a~má navyše veľkosť
$$
|\uhel O_1BC|=|\uhel O_1BP|+|\uhel CBP|=(90^\circ-\alpha)+\beta.
$$
Analogicky v polrovine $BCP$ leží tiež uhol $O_2CB$ a
má veľkosť $(90^\circ-\beta)+\alpha$. Dokopy nám vychádza
$|\uhel O_1BC|+|\uhel O_2CB|=180^{\circ}$ a s~prvou časťou
riešenia sme tak hotoví.

\smallskip
\emph{Druhá časť} riešenia bude kratšia. Rovnosť $|O_1B|=|O_2C|$
ľahko dokážeme použitím rozšírenej sínusovej vety pre trojuholníky $ABP$ a
$CDP$. Tie sa totiž zhodujú vo veľkosti uhlov pri spoločnom
vrchole~$P$
aj v~dĺžke protiľahlých strán $AB$ a $CD$, takže platí
$$
2\cdot|O_1B| = \frac{|AB|}{\sin|\uhel APB|} =
\frac{|CD|}{\sin|\uhel CPD|} = 2\cdot|O_2C|.
$$}

{%%%%%   A-III-5
Ukážeme, že zadaniu vyhovujú iba celé čísla $n=0$ a $n=6$.

V prípade celého $n<0$ zrejme nie je číslo $2^n+n^2$ celé,
tobôž nie druhá mocnina celého čísla.
Pre $n\in\{0,1,2,3,4,5,6\}$ nadobúda výraz $2^n+n^2$
postupne hodnoty $1$, $3$, $8$, $17$, $32$, $57$, $100$, z ktorých
sú druhými mocninami iba krajné hodnoty $1^2$ a~$10^2$,
ktoré zodpovedajú číslam $n=0$ a $n=6$. Ak ďalej dokážeme sporom,
že žiadne $n\ge 7$ nevyhovuje, budeme s~úvodnou vytýčenou úlohou hotoví.

Pripusťme teda, že pre nejaké celé číslo $n\ge 7$ platí
$$
2^n+n^2=m^2,
\tag1
$$
pričom $m$ je celé číslo, o ktorom môžeme predpokladať, že je
kladné, a teda podľa (1) väčšie ako $n$. Podľa parity $n$
rozlíšime dva prípady.

\smallskip
a) \emph{Prípad nepárneho $n$}. Vtedy aj číslo $m$ z rovnosti $2^n+n^2=m^2$
je nepárne. Upravme túto rovnosť na tvar
$$
2^n=m^2-n^2=(m+n)(m-n).
$$
Vidíme, že obe kladné čísla $m+n>m-n$ sú mocninami dvoch,
takže $m-n\mid m+n$. Teda $m-n$ delí aj
číslo $(m+n)-(m-n)$ rovné $2n$, pričom ale $n$ je nepárny činiteľ,
takže $4\nmid m-n$. Mocnina dvoch rovná $m-n$ je ale párne číslo,
lebo čísla $m$, $n$ sú nepárne. Preto z $4\nmid m-n$ vyplýva
$m-n=2$. Po dosadení $m=n+2$ prejde (1) po
úprave na rovnosť $2^n=4n+4$. My však matematickou indukciou
ukážeme, že pre každé celé $n\geqq5$ platí $2^n>4n+4$, a tým
prípad a) dovedieme ku sporu. Naozaj, pre $n=5$ sa jedná o platnú nerovnosť
$32>24$. Ak teraz predpokladáme, že
pre nejaké $n\geqq5$ platí $2^n>4n+4$, tak po vynásobení dvoma
dostaneme $2^{n+1}>8n+8$, odkiaľ s prihliadnutím na zrejmú
nerovnosť $8n+8>4(n+1)+4$ dostávame dokazovanú
nerovnosť pre hodnotu $n+1$. Tým je celý rozbor prípadu~a) ukončený.

\smallskip
b) \emph{Prípad párneho $n$}. Vtedy aj číslo $m$ z rovnosti $2^n+n^2=m^2$
je párne. Položme $n=2k$, pričom $k$ je celé číslo, ktoré vďaka predpokladu
$n\geqq7$ spĺňa nerovnosť $k\geqq4$. Dokážeme ďalej, že platia nerovnosti
$\bigl(2^k\bigr)^2 < m^2 <\bigl(2^k+2\bigr)^2$, čo bude želaný spor,
keďže medzi kvadrátmi dvoch po sebe nasledujúcich párnych čísel nemôže ležať
ďalší kvadrát párneho čísla.

Nerovnosti, ktoré sme sľúbili dokázať, majú po nahradení $m^2$
súčtom $2^{2k}+(2k)^2$ a rozpísaní mocniny $(2^k+2)^2$ tvar
$$
2^{2k}<2^{2k}+(2k)^2<2^{2k}+4\cdot2^{k}+4.
$$
Ľavá nerovnosť je vďaka $(2k)^2>0$ triviálna, pravá nerovnosť po
zrušení rovnakých členov~$2^{2k}$ a vydelení oboch strán štyrmi
sa zmení na ekvivalentnú nerovnosť $k^2<2^k+1$. Tú opäť dokážeme pre
každé celé $k\geqq4$ matematickou indukciou. Pre $k=4$ sa jedná o
platnú nerovnosť $16<17$. Ak teraz predpokladáme, že
pre nejaké $k\geqq4$ platí $k^2<2^k+1$, tak po vynásobení dvoma
a odčítaní jednotky dostaneme $2k^2-1<2^{k+1}+1$. Želaná
nerovnosť pre hodnotu $k+1$ tak bude dokázaná, ak ukážeme, že
platí $(k+1)^2<2k^2-1$, čiže $2<k(k-2)$, čo je však zrejmé
dokonca pre každé $k\geqq3$. Tým je aj rozbor prípadu b) ukončený.

\ineriesenie
Úpravu rovnice $2^n+n^2=m^2$ na súčinový tvar
$$
2^n=(m-n)(m+n),
\tag2
$$
ktorú sme v prvom riešení využili iba na rozbor prípadu nepárneho
$n$, teraz uplatníme univerzálne, totiž na vyriešenie pôvodnej rovnice
$2^n+n^2=m^2$ v obore všetkých celých čísel $m>n\geqq1$. Ukážeme, že
jediné riešenie je $(m,n)=(10,6)$.

Vyjdeme opäť z toho, že na pravej strane rovnice (2) musia stáť dve
mocniny dvoch. Platí teda $m-n=2^a$ a $m+n=2^b$, pričom celé čísla
$0\leqq a<b$ spĺňajú vďaka vzťahu $2^n=2^a\cdot2^b$
rovnosť $a+b=n$. Pôvodné neznáme $m$ a~$n$
potom majú zrejme vyjadrenie
$$
m=\frac{2^b+2^a}{2}\quad\hbox{a}\quad n=\frac{2^b-2^a}{2},
$$
z ktorých vyplýva, že $a\ne0$, inak by oba zlomky mali nepárne
čitatele a čísla $m$, $n$ by tak neboli celé. Ak dosadíme
do $a+b=n$ vyjadrenie čísla $n$,
dostaneme po vynásobení dvoma ekvivalentnú rovnicu
$$
2a+2b=2^b-2^a
\tag3
$$
pre nové celočíselné neznáme $a$ a $b$ z oboru $1\leqq a<b$.

Tvrdíme, že v prípade $b\geqq6$ rovnosť (3) nemôže nastať.
Keďže z nerovnosti $a<b$ zrejme vyplýva $2a+2b<4b$ a na druhej
strane zároveň $2^b-2^a\geqq 2^b-2^{b-1}=2^{b-1}$, stačí nám ukázať, že
$4b<2^{b-1}$ pre každé celé $b\geqq6$. Použijeme na to matematickú indukciu.
Pre $b=6$ dostávame platnú nerovnosť $24<32$. Ak teraz predpokladáme, že
pre nejaké $b\geqq6$ platí $4b<2^{b-1}$, tak po vynásobení dvoma
dostaneme $8b<2^{b}$, odkiaľ s~prihliadnutím na zrejmú
nerovnosť $4(b+1)<8b$ dostávame dokazovanú nerovnosť pre hodnotu
$b+1$ a dôkaz indukciou je ukončený.

Pre zvyšných 10 prípadov, keď platí $1\le a<b\le 5$, priamym dosadením do
rovnice~(3) overíme, že rovnosť v~nej nastane iba pre dvojicu $(a,b)=(2,4)$,
ktorej zodpovedá dvojica $(m,n)=(10,6)$:
$$
\def\v{\phantom{1}}
\def\fakebold#1{\pdfliteral{2 Tr .3 w}#1\pdfliteral{0 Tr 0 w}}
\def\bm#1{\hbox{\bf #1}}
\def\bmi#1{\hbox{\typoscale[700/700]\bf #1}}
\eqalign{
2\cdot 1+2\cdot 2 &=\v 6 \ne\v2= 2^2-2^1,\cr
2\cdot 1+2\cdot 3 &= \v 8 \ne \v6= 2^3-2^1,\cr
2\cdot 1+2\cdot 4 &= 10 \ne 14= 2^4-2^1,\cr
2\cdot 1+2\cdot 5 &= 12 \ne 30= 2^5-2^1,\cr
2\cdot 2+2\cdot 3 &= 10 \ne\v 4= 2^3-2^2,\cr
}
\qquad
\eqalign{
\bold{2}\fakebold{\cdot} \bold{2}\fakebold+\bold{2}\fakebold\cdot \bold{4} &= 12 = 12= \bold{2}^{\bold{4}}\fakebold-\bold{2}^{\bold{2}},\cr
2\cdot 2+2\cdot 5 &= 14 \ne 28= 2^5-2^2,\cr
2\cdot 3+2\cdot 4 &= 14 \ne \v 8= 2^4-2^3,\cr
2\cdot 3+2\cdot 5 &= 16 \ne 24= 2^5-2^3,\cr
2\cdot 4+2\cdot 5 &= 18 \ne 16= 2^5-2^4.
}
$$
}

{%%%%%   A-III-6
Dokážeme v dvoch oddelených častiach, že satelity môžu vytvoriť
najmenej~600 a najviac 5\,400 prepojených trojíc. V celom texte
bude $S$ označovať množinu dotyčných satelitov, $n=50$ ich počet a
$m=225$ počet komunikačných línií.

\medskip
a) \emph{Najväčší možný počet prepojených trojíc je $5\,400$.}

Trojica satelitov je prepojená práve vtedy, keď obsahuje dve
alebo tri komunikačné línie. Vyplatí sa nám preto pre každú možnú hodnotu $i\in\{0,1,2,3\}$
označiť ako $t_i$ počet tých trojíc satelitov, medzi ktorými je práve
$i$ komunikačných línií. Tvrdíme, že tieto počty spĺňajú rovnosť
$$
\sum_{i=0}^3 i\cdot t_i=m\cdot(n-2).
\tag1
$$
Naozaj, obe strany vyjadrujú celkový počet usporiadaných dvojíc
$(T,l)$, pričom $T$ je trojica satelitov a $l$ je komunikačná
línia medzi dvoma satelitmi tejto trojice, lebo na ľavej strane
započítavame pre jednotlivé $T$, v koľkých dvojiciach $(T,l)$ dané
$T$ vystupuje, zatiaľ čo pravou stranou (1) vyjadrujeme fakt, že každé~$l$
sa vyskytuje v $n-2$ dvojiciach $(T,l)$.

Z dokázanej rovnosti (1) prepísanej na tvar $t_1+2t_2+3t_3=m(n-2)$
vyplýva pre počet $t_2+t_3$ všetkých prepojených trojíc satelitov
odhad
$$
t_2+t_3 =\frac{m(n-2)-t_1-t_3}{2}\leqq \frac{m(n-2)}{2}=5400,
$$
pritom želaná rovnosť $t_2+t_3=5\,400$ nastane práve vtedy, keď bude
platiť $t_1=t_3=0$. To možno dosiahnuť, ak budú satelity
prepojené ako na \obr{} vľavo (pričom naozaj máme $5+45=50$ satelitov
a $5\cdot 45=225$ komunikačných línií).
\inspdf{mo71-a-iii-6-sk.pdf}%

\medskip
b) \emph{Najmenší možný počet prepojených trojíc je $600$.}

O prepojenej trojici satelitov $T$ a jej satelite $s$ hovoríme,
že $s$ je \emph{centrálny satelit} trojice $T$, ak
je prepojený s~oboma ostatnými satelitmi tejto trojice. Všimnime si,
že každá prepojená trojica má buď jeden, alebo tri centrálne satelity.

Definujme ešte \emph{stupeň} $d_s$ satelitu $s$ ako
počet komunikačných línií, ktoré má. Potom $s$ je centrálnym
satelitom v práve $\binom{d_s}2=\frac12d_s(d_s-1)$
prepojených trojiciach. Keďže každá prepojená trojica má
nanajvýš 3 centrálne satelity, pre celkový počet $P$ prepojených
trojíc platí
$$
\qquad P\ge \frac13 \sum_{s\in S}\binom{d_s}2
= \frac16 \left(\sum_{s\in S} d_s^2 - \sum_{s\in S} d_s\right)
\ge \frac16\left( \frac1n\bigg(\sum_{s\in S} d_s\bigg)^2 -
\sum_{s\in S} d_s\right),
$$
pričom posledná nerovnosť vyplýva z~nerovnosti medzi kvadratickým a
aritmetickým priemerom pre $n$-ticu čísel $\bigl(d_s\mid s\in
S\bigr)$, ktorá je sama dôsledkom Cauchyho-Schwarzovej
nerovnosti
$$
\left(\sum_{s\in S} 1\cdot d_s\right)^2\leqq
\left(\sum_{s\in S} 1^2\right)\cdot
\left(\sum_{s\in S} d_s^2\right).
$$

Súčet $\sum_{s\in S} d_s$, ktorý sa v získanom odhade objavil,
je však rovný dvojnásobku počtu~$m$ všetkých komunikačných línií,
lebo je v ňom každá línia započítaná práve dvakrát
(raz za každý jej koniec). Dvojakým
dosadením hodnoty $2m=450$ za $\sum_{s\in S} d_s$ spolu s~hodnotou $n=50$ už z odvodenej nerovnosti dostávame sľúbený odhad
$$
P\geqq \frac16\left(\frac1{50}\cdot450^2-450\right)=600.
$$
Z nášho postupu navyše vyplýva, že rovnosť $P=600$ nastane
práve vtedy, keď všetkých $P$ prepojených trojíc má po troch
centrálnych satelitoch a zároveň všetkých $n=50$
satelitov má ten istý stupeň, rovný teda ako vieme
$2m/n=9$.\fnote{Našťastie vyšlo celé číslo.}
To možno dosiahnuť, ak budú satelity prepojené
ako na \obrr1{} vpravo (všetkých 50~satelitov je rozdelených do
piatich komunikačne izolovaných skupín
po desiatich navzájom prepojených satelitoch, majúcich teda naozaj
ten istý stupeň~9).

\poznamky
\nopagebreak
\item{1.}
Nie je ťažké dokázať, že konštrukcie potrebných príkladov v oboch
častiach riešenia sú jediné možné.

\item{2.}
Odhad počtu $P$ všetkých prepojených trojíc v~druhej časti
riešenia možno získať tiež použitím {\it Jensenovej nerovnosti\/}
pre konvexnú funkciu
$f(x)=\frac12x(x-1)$:
$$
P\ge\frac13\sum_{s\in S}\binom{d_s}2
= \frac n3\cdot \frac1n\sum_{s\in S}\binom{d_s}2\ge
\frac n3\cdot\binom{\frac1n\sum_{ s\in S}d_s}{2}
= \frac n3\cdot\binom{2m/n}2.
$$
}

{%%%%%   B-S-1
a) Šesťciferné strakaté číslo je napríklad 573\,482, lebo
súčty $5+7+3$, ${7+3+4}$, $3+4+8$, $4+8+2$ majú práve dve rôzne
hodnoty 14 a 15.
\smallskip
b) Dokážeme, že žiadne sedemciferné strakaté číslo neexistuje.

Dôkaz urobíme sporom. Predpokladajme, že také číslo existuje,
vyberme jedno z~nich a jeho sedem navzájom rôznych cifier
označme zľava doprava $a$, $b$, $c$, $d$, $e$, $f$, $g$. Navyše
jeho trojiciam susedných cifier, ktorých je celkom päť, priradíme
poradové čísla: prvá bude trojica $(a,b,c)$, druhá trojica
$(b,c,d)$ atď., až piata bude trojica $(e,f,g)$.

Všimnime si, že súčet prvej trojice je $a+b+c$ a súčet druhej
trojice je ${b+c+d}$. Keby sa tieto dva súčty rovnali, muselo by
platiť $a=d$, čo odporuje tomu, že každá zastúpená cifra je iná.
Súčet prvej trojice je teda rôzny od súčtu druhej trojice.
Z~rovnakého dôvodu musia byť rôzne súčty čísel aj
v~každých dvoch ďalších trojiciach, ktoré spolu v~našej pätici susedia
(lebo majú vždy dve čísla spoločné).

Keďže vybrané číslo je strakaté a už súčty prvých
dvoch trojíc, označme ich $S={a+b+c}$ a $R=b+c+d$,
sú navzájom rôzne, musí podľa záveru predchádzajúceho odseku platiť:
súčet tretej trojice je $S$ (susedná druhá trojica má totiž súčet~$R$),
a~tak súčet štvrtej trojice je zasa $R$, a~teda súčet piatej
trojice je znova~$S$. Zapíšme to prehľadne do jedného riadka:
$$
a+b+c=S,\ b+c+d=R,\ c+d+e=S,\ d+e+f=R\quad\hbox{a}\quad e+f+g=S.
$$
Vyjadrime z toho rozdiel $S-R$ jednak z~prvých dvoch, jednak
z~posledných dvoch rovností:
$$\eqalign{
S-R&=\hphantom{(e+f+g)}\llap{$(a+b+c)$}-\rlap{$(b+c+d)$}\hphantom{(d+e+f)}=a-d,\cr
S-R&=(e+f+g)-(d+e+f)=g-d.
}$$
Porovnaním dostávame $a-d=g-d$, čiže $a=g$, napriek tomu, že $a$ a $g$
sú navzájom rôzne cifry. To je definitívny
spor, ktorým je existencia sedemciferného strakatého čísla
vyvrátená.

\poznamka
Obe časti a) a b) uvedeného riešenia teraz osobitne okomentujeme.

\smallskip
a) Aj keď sa dá na šesťciferné strakaté číslo prísť skusmo, opíšme
spôsob, ako do jeho hľadania vniesť určitý systém.
Ak si uvedomíme, že súčty trojíc susedných cifier musia
nadobúdať striedavo dve rôzne hodnoty, dôjdeme k~záveru, že cifry
šesťciferného strakatého čísla musia byť zľava doprava tvaru
$$
a,\,b,\,c,\,a+x,\,b-x,\,c+x,\quad\hbox{kde}\quad x\ne0.
\tag1
$$
Naozaj, podmienka $x\ne0$ pre tvar $a+x$ štvrtej cifry musí
byť splnená, aby sa táto štvrtá cifra nerovnala prvej
cifre $a$. Tvar piatej cifry $b-x$
je potom riešením~$y$ rovnice $c+(a+x)+y=a+b+c$,
tvar šiestej cifry $c+x$ potom riešením $z$
rovnice $(a+x)+(b-x)+z=b+c+(a+x)$.

Zdôraznime, že sme našli nutné vyjadrenie $(a,b,c,a+x,b-x,c+x)$
cifier každého šesťciferného strakatého čísla. Potrebujeme však
ešte, aby taká šestica neobsahovala dve rovnaké cifry.
Ak zvolíme napríklad $a=1$ a $x=1$, dostaneme
šesticu $(1,b,c,2,b-1,{c+1})$ a vidíme, že $b$ musí byť aspoň 4.
Voľbou $b=4$ dostaneme šesticu $(1,4,c,2,3,c+1)$, ktorá bude
zrejme vyhovovať pre ľubovoľné $c\in\{5,6,7,8\}$. Tak pre
$c=5$ získame strakaté číslo 145\,236.

Dodajme ešte, že šesťciferné číslo $\overline{abcdef}$ s~navzájom
rôznymi ciframi je strakaté práve vtedy, keď platí
$a-d=e-b=c-f$. Sú to totiž zjednodušene zapísané rovnosti
$$
(a+b+c)-(b+c+d)=(c+d+e)-(b+c+d)=(c+d+e)-(d+e+f),
$$
ktoré vyjadrujú práve to, že súčty trojíc susedných cifier
nadobúdajú striedavo dve hodnoty (ktoré sú rôzne vďaka $a\ne
d$).

\smallskip
b) Po odvodení sústavy rovností v~podanom riešení
$$
a+b+c=S,\ b+c+d=R,\ c+d+e=S,\ d+e+f=R,\ e+f+g=S
\tag2
$$
môžeme tiež dôjsť ku sporu inými algebraickými manipuláciami,
akými sú napríklad porovnávanie rôznych vyjadrení hodnôt $S$ alebo
$R$. Nebudeme ich tu uvádzať, namiesto toho objasníme, prečo krátke
odvodenie spornej rovnosti $a=g$ v~podanom riešení nie je tak trikové,
ako by sa na prvý pohľad mohlo zdať.

Ak budeme hľadať čo najjednoduchšie dôsledky sústavy rovností~(2),
nemôžeme opomenúť možnosť odčítať od seba súčty
susedných trojíc (dôjde totiž ku zrušeniu dvoch sčítancov v~každej
trojici). Dostaneme tak štyri rôzne vyjadrenia toho istého
rozdielu~$S-R$, a to
$$
S-R=a-d=e-b=c-f=g-d.
$$
Tak objavíme spornú rovnosť $a-d=g-d$, čiže $a=g$.
Alebo inak:
Ak vieme z~časti a) tejto poznámky, že prvých šesť cifier
sedemciferného strakatého čísla je nutne tvaru $a$, $b$, $c$, $a+x$, $b-x$
a $c+x$ (pozri (1)), musí byť siedma cifra opäť rovná~$a$, aby sa
súčet piatej, šiestej a siedmej cifry rovnal $a+b+c$.


\schemaABC
Za úplné riešenie dajte 6 bodov. V~neúplných riešeniach oceňte čiastkové kroky nasledovne.
\item{A1.} Uvedenie príkladu šesťciferného strakatého čísla (aj bez zdôvodnenia): 2 body, z~toho 1~bod, ak sa príklad nepodarí nájsť, avšak je dosiahnutá nejaká parametrizácia šestice cifier podobne ako v~časti~a) poznámky, alebo sú odvodené tamojšie rovnosti $a-d=e-b=c-f$.
\item{B1.} Negatívna odpoveď na otázku b): 0 bodov.
\item{B2.} Voľba metódy dôkazu sporom: 0 bodov.
\item{B3.} Zdôvodnenie, že susedné trojice majú rôzne súčty: 1 bod.
\item{B4.} Zdôvodnenie, že rovnaké súčty majú prvá, tretia a piata trojica a tiež druhá a štvrtá trojica: 1~bod.
\endgraf\noindent
Celkom potom dajte $\rm A1+B3+B4$ bodov.
\endschema
}

{%%%%%   B-S-2
Uvažujme ešte bod $H$ taký, že $CABH$ je rovnobežník.
Tvrdenie úlohy získame ako zrejmý dôsledok dvoch analogických rovností
$|HA|=|GE|$ a $|HA|=|FD|$. Na dôkaz prvej z~nich
využijeme \obr{} bez vyznačeného bodu $D$.
\inspdf{b71s2-1.pdf}%

Keďže úsečky $GB$ a $BH$ sú protiľahlými stranami ku strane~$AC$
v~rovnobežníkoch $GBCA$, resp. $BHCA$, sú úsečky $GB$, $AC$ a $BH$
zhodné a rovnobežné. Bod~$B$ je teda stredom úsečky $GH$ a navyše
platí $GH\parallel AC$, a teda aj $GH\parallel AE$. Keďže $ABE$ je
rovnoramenný trojuholník so základňou $AE$, leží jeho hlavný vrchol $B$
na osi základne $AE$. Táto os je však aj osou úsečky $GH$ (je
totiž na ňu kolmá a prechádza jej stredom). Podľa tejto spoločnej osi
je teda súmerný lichobežník $GHEA$, ktorý je preto rovnoramenný.
Keďže uhlopriečky rovnoramenného lichobežníka sú zhodné, je
avizovaná rovnosť $|HA|=|GE|$ dokázaná.\fnote{Riešenie možno
zakončiť aj bez zmienky o~rovnoramennom lichobežníku: Podľa
spoločnej osi úsečiek $AE$ a $GH$ sú totiž úsečky $HA$ a $GE$
súmerne združené, a teda zhodné.}

Rovnakým spôsobom sa dokáže, že aj úsečky $AD$ a $FH$ majú
spoločnú os (čiže $FADH$ je rovnoramenný lichobežník, pozri
\obr{}, odkiaľ vyplynie druhá potrebná rovnosť $|HA|=|FD|$.
Tým je podľa úvodného odseku celé riešenie ukončené.
\inspdf{b71s2-2.pdf}%

\ineriesenie
Ukážeme, že trojuholníky $GBE$ a $DCF$ sú zhodné podľa
vety~\emph{sus}, uplatnenej na dvojice ich strán
so spoločným vrcholom $B$, resp. $C$ -- pozri \obr{}.
Tak budeme s~riešením hotoví, lebo zhodnosť tretích strán $GE$ a $DF$
je práve tým tvrdením, ktoré máme dokázať.
\inspdf{b71s2-3.pdf}%

Začneme s~dôkazom zhodnosti uhlov $GBE$ a $DCF$. Na to budeme
zhodné uhly na obrázku vyznačovať oblúčikmi jednej farby. Najskôr
porovnáme všetky vnútorné uhly rovnoramenných trojuholníkov $ADC$ a $EAB$.
Tie pri ich základniach $AD$, resp.~$EA$ sú na
obrázku všetky vyznačené namodro, lebo pre ich veľkosti platí:
$$
|\uhel CDA|=|\uhel CAD|=|\uhel CAB|=|\uhel EAB|=|\uhel BEA|.
$$
Z toho vyplýva aj zhodnosť zvyšných vnútorných uhlov $DCA$ a $ABE$,
vyznačených na obrázku červenou. Okrem nich sú tam ešte
vyznačené namodro uhly $GBA$ a~$ACF$, ktoré sú totiž striedavé
(a teda zhodné) s~uhlom $CAB$ vďaka tomu, že $GB\parallel AC$ a~$AB\parallel FC$. Teraz už vidíme, že platí
$$
|\uhel GBE|=|\uhel GBA|+|\uhel ABE|=|\uhel ACF|+|\uhel ACD|=
|\uhel DCF|,
$$
ako sme mali dokázať.

Ostáva dokázať, že sú zhodné ako strany $GB$ a $DC$, tak
aj strany $BE$ a~$CF$. To je ale jednoduché,
lebo $|GB|=|AC|=|DC|$ (prvá rovnosť vyplýva z~rovnobežníka $GBCA$,
druhá z~definície bodu $D$) a $|BE|=|BA|=|CF|$ (tu prvá rovnosť vyplýva z~definície bodu~$E$ a~druhá z~rovnobežníka $ABCF$). Tým je dôkaz zhodnosti trojuholníkov $GBE$ a~$DCF$, ktorý sme sľúbili v~úvodnom odseku, ukončený.


\schemaABC
Za úplné riešenie dajte 6 bodov. V~neúplných riešeniach oceňte
čiastkové kroky nasledovne.
\item{A1.} Prikreslenie tretieho rovnobežníka $CABH$: 1 bod.
\item{A2.} Konštatovanie, že $GHEA$ a $HFAD$ sú rovnoramenné lichobežníky (prípadne štvoruholníky so súmerne združenými uhlopriečkami): bez dôkazu 2 body, s~dôkazom 4 body.
\item{B1.} Hypotéza o~zhodnosti $\triangle GBE\cong\triangle DCF$ (ďalej iba \uv{hypotéza}): 0 bodov.
\item{B2.} Zdôvodnené rovnosti $|GB|=|DC|$ a $|GB|=|CF|$ (bez hypotézy): 0 bodov.
\item{B3.} Zdôvodnená zhodnosť uhlov $GBE$ a $DCF$ (bez hypotézy): 0 bodov.
\item{B4.} Hypotéza a zdôvodnené rovnosti $|GB|=|DC|$ a $|GB|=|CF|$: 2 body, z~toho 1 bod za jednu rovnosť.
\item{B5.} Hypotéza a zdôvodnená zhodnosť uhlov $GBE$ a $DCF$: 3 body.
\endgraf\noindent
Celkom potom dajte $\rm\max\,(A1,\,A2,\,B4+B5)$ bodov.
\endschema
}

{%%%%%   B-S-3
Hľadáme všetky prvočísla $p$, pre ktoré existuje opísaný trojuholník
s~dĺžkou jednej odvesny $p^2$. Dĺžku jeho
druhej odvesny označme $b$ a dĺžku prepony~$c$. Potom podľa
Pytagorovej vety platí rovnosť $c^2=p^4+b^2$, ktorú upravíme na
súčinový tvar
$$
p^4=c^2-b^2=(c+b)(c-b).
\tag1
$$
Keďže $c>b>0$ (prepona je dlhšia ako
odvesna), stojí na pravej strane (1) súčin dvoch
prirodzených čísel $c+b$ a $c-b$, pričom $c+b>c-b$.
Číslo $p^4$ z~ľavej strany~(1) možno ale rozložiť na súčin
dvoch rôznych prirodzených čísel práve dvoma spôsobmi (keď väčší činiteľ
zapíšeme ako prvý): $p^4\cdot 1$ a $p^3\cdot p$. Teda súčet
$c+b$ je rovný buď $p^4$, alebo $p^3$. V~prvom prípade má obvod
$o=p^2+b+c$ nášho trojuholníka hodnotu $o=p^2+p^4$, v~druhom prípade
je $o=p^2+p^3$.

Podľa zadania je obvod $o$ druhou mocninou prirodzeného čísla.
Keby taká bola hodnota $p^2+p^4=p^2(p^2+1)$, potom vzhľadom
na činiteľ $p^2$ by musel byť druhou mocninou aj o~jednotku väčší
činiteľ $p^2+1$, čo je zjavne nemožné -- je to číslo o~$2p$
menšie ako kvadrát $(p+1)^2$, ktorý po kvadráte $p^2$
nasleduje.\fnote{Zrejmé tvrdenie, že dve po sebe nasledujúce
prirodzené čísla nemôžu byť obe druhými mocninami, nemusia
riešitelia v~riešeniach dokazovať.}

Musí teda nastať druhý prípad, keď je druhou mocninou hodnota
$o=p^2+p^3=p^2(p+1)$. To nastane práve vtedy, keď
bude druhou mocninou činiteľ~$p+1$, čo spĺňa,
ako vzápätí ukážeme, jediné prvočíslo $p=3$.
Naozaj, z~rovnosti $p+1=k^2$ pre celé
$k>1$ máme $p=(k-1)(k+1)$, odkiaľ vzhľadom na $0<k-1<k+1$ vyplýva
$k-1=1$ a $k+1=p$, \tj. $k=2$ a $p=3$.

Zistili sme, že do úvahy prichádza jediné prvočíslo $p=3$, pritom
musí platiť ${c+b}=p^3=27$ a $c-b=p=3$, odkiaľ $c=15$ a $b=12$.
Trojuholník so stranami dĺžok 15, 12 a $9\,(=\!p^2)$ je naozaj pravouhlý
a jeho obvod 36 je druhou mocninou čísla 6.\fnote{Previerka
týchto dvoch vlastností \emph{v~uvedenom postupe} nie je nevyhnutná:
máme zaručenú rovnosť z~Pytagorovej vety a vieme,
že obvod $p^2(p+1)$ je druhou mocninou, pretože je také číslo $p+1=4$.}

\zaver
Dĺžka odvesny rovnej druhej mocnine prvočísla má
jedinú možnú hodnotu, ktorou je číslo 9.

\poznamka
Zachovajme označenie $p$, $b$, $c$, $o$ a stručne opíšme iné,
trochu komplikovanejšie riešenie. Tentoraz rovnosť $p^4+b^2=c^2$
upravíme na súčinový tvar $b^2=(c+p^2)(c-p^2)$.
Zdôraznime, že tu (nie nutne nesúdeliteľné) čísla $c+p^2$ a $c-p^2$
nemusia byť kvadráty (aj keď ich súčin taký je).
Ak ale uvážime ich najväčší spoločný deliteľ $d$, budeme mať
$c+p^2=du^2$, $c-p^2=dv^2$, $b=duv$ a $o=du(u+v)$,
pričom $u>v$ sú nesúdeliteľné prirodzené čísla. Teraz z~rovností
$$
2p^2=\bigl(c+p^2\bigr)-\bigl(c-p^2\bigr)=du^2-dv^2=d\bigl(u^2-v^2\bigr)
$$
vidíme, že $d$ je deliteľ čísla $2p^2$, ktorý je pritom menší ako
$p^2$, lebo určite $u^2-v^2>2$. Máme tak nutne $d\in\{1,2,p,2p\}$.
Tieto hodnoty možno jednotlivo posúdiť dosadením
do rovnosti $2p^2={d(u+v)(u-v)}$. Z nej potom vždy určíme
činitele $u+v$ a~${u-v}$.\fnote{Pre hodnoty $d=1$ a $d=2$ si
pritom vypomôžeme poznatkom, že vďaka nesúdeliteľnosti čísel $u$, $v$
môžu mať čísla $u+v$, $u-v$ jediného spoločného deliteľa väčšieho ako
1, a to číslo 2.} Zapíšme to tu iba
pre hodnotu $d=2p$ (ktorá jediná nevedie ku sporu):
Pre ňu dostaneme ${(u+v)(u-v)}=p$,
takže $u-v=1$ a $u+v=p$. Keď z toho vyplývajúce hodnoty $u$, $v$
dosadíme spolu s~$d=2p$ do vzorca $o=du(u+v)$,
dostaneme po úprave $o=p^2(p+1)$ a~postup ľahko dokončíme
rovnako ako v~pôvodnom riešení.

\schemaABC
Za úplné riešenie dajte 6 bodov. V~neúplných riešeniach oceňte čiastkové kroky nasledovne.
\item{A1.} Odvodenie rovnice (1) (v~súčinovom tvare!): 1 bod.
\item{A2.} Odvodenie záveru, že $b+c\in\{p^4,p^3\}$: 1 bod.
\item{A3.} Vylúčenie prípadu $b+c=p^4$: 1 bod.
\item{A4.} Odvodenie, že v~prípade $b+c=p^3$ platí $p=3$: 2 body.
\item{A5.} Overenie, že hodnota $p=3$ (hoci aj uhádnutá) vyhovuje: 1 bod.
\item{B1.} Odvodenie rovnice $b^2=(c+p^2)(c-p^2)$: 0 bodov.
\item{B2.} Pozorovanie, že najväčší spoločný deliteľ činiteľov $c+p^2$, $c-p^2$ (z~rovnice v~B1) je deliteľ čísla $2p^2$, \tj. jedna z~hodnôt 1, 2, $p$, $2p$, $p^2$, $2p^2$: 1 bod.
\item{B3.} Vylúčenie hodnôt 1, 2, $p$, $p^2$, $2p^2$ z~bodu B2: 3 body
\item{B4.} Vyriešenie prípadu s~hodnotou $2p$ z~bodu B2 (vrátane skúšky): 2 body
\endgraf\noindent
Celkom potom dajte $\rm\max(A1+A2+A3+A4+A5,\,B2+B3+B4)$ bodov.

Riešenie, ktoré neobsahuje požadovaný záver, že teda jediná možná dĺžka dotyčnej odvesny je 9, môže byť ohodnotené nanajvýš 5 bodmi.
\endschema
}

{%%%%%   B-II-1
a) Také číslo existuje. Obe podmienky spĺňa napríklad číslo
$n=72$, lebo preň je $2n=144=12^2$ a $3n=216=6^3$.

Predchádzajúci odsek je úplným riešením časti a), ukážeme si
však ešte, ako sa na (najmenšie vyhovujúce) číslo 72 príde.
Uvažujme prvočíselný rozklad hľadaného čísla~$n$ v tvare
$$
n=2^{a_2}\cdot 3^{a_3}\cdot\dots\cdot p^{a_p},
\tag1$$
pričom $p$ je najväčšie prvočíslo, ktoré delí $n$, a čísla
$a_2, \dots,$ $a_p$ sú celé a nezáporné. Potom číslo~$2n$ má
prvočíselný rozklad
$$
2n = 2^{a_2+1}\cdot 3^{a_3}\cdot\dots\cdot p^{a_p}.
$$
Uvedomme si, že toto číslo je druhou mocninou prirodzeného čísla
práve vtedy, keď sú všetky exponenty $a_2+1$, $a_3,
\dots,$ $a_p$ čísla párne. Prvočíselný rozklad čísla $3n$ je
$$
3n=2^{a_2}\cdot 3^{a_3+1}\cdot\dots\cdot p^{a_p}.
$$
Toto číslo je treťou mocninou práve vtedy, keď sú všetky
exponenty $a_2$, $a_3+1,\dots,$ $a_p$ čísla deliteľné
tromi. Číslo $n$ preto spĺňa obe podmienky zo zadania
práve vtedy, keď platí súčasne: číslo $a_2$ je nepárne a deliteľné tromi,
číslo $a_3$ je párne a po delení tromi dáva zvyšok 2, a napokon
pre každé prvočíslo $q>3$ je $a_q$ deliteľné súčasne dvoma aj tromi
(teda je deliteľné šiestimi). Vidíme, že exponenty $a_2=3$ a $a_3=2$
sú najmenšie, ktoré spĺňajú zodpovedajúce podmienky, a pritom v~(1)
môžeme voliť $p=3$ (\tj. položiť $a_q=0$ pre každé $q>3$).
Dostaneme tak
najmenšie možné $n=2^3\cdot 3^2=72$, ktoré spĺňa zadanie časti a).
Zároveň z~našich úvah vyplýva, že všetky vyhovujúce čísla $n$
sú tvaru
$$
n=2^{6a-3}\cdot 3^{6b-4}\cdot M^6,\quad\hbox{alebo jednoduchšie}\quad
n=2^{3}\cdot 3^{2}\cdot N^6,
$$
pričom $a$, $b$, $M$, $N$ sú kladné celé čísla (o~čísle $M$ môžeme
navyše predpokladať, že je nesúdeliteľné s~číslom 6).

\medskip
b) Dokážeme sporom, že také číslo $n$ neexistuje. Jeho
prvočíselný rozklad (1) by totiž musel spĺňať podmienky
určené v~časti a) nášho riešenia a ešte navyše by číslo
$$
4n=2^{a_2+2}\cdot 3^{a_3}\cdot\dots\cdot p^{a_p}
$$
muselo byť štvrtou mocninou, teda všetky exponenty
$a_2+2$, $a_3,\dots,$ $a_p$ by museli byť deliteľné štyrmi.
Špeciálne číslo $a_2+2$ by muselo byť deliteľné štyrmi, aj keď
samo číslo~$a_2$ je -- ako vieme z~riešenia časti a) -- nutne nepárne.
Týmto sporom je dôkaz neexistencie čísla $n$ ukončený.

\inerieseniecc{časti b) bez úvah o~prvočíselných rozkladoch}
Hľadané číslo $n$ musí spĺňať rovnosť $2n=k^2$
pre nejaké prirodzené $k$. Keďže $\sqrt{4n}=\sqrt{2k^2}=k\cdot\sqrt{2}$
a~číslo $\sqrt{2}$ je (ako je známe) iracionálne, nie je odmocnenec $4n$
druhou, a teda ani štvrtou mocninou prirodzeného čísla.
%\medskip

\schemaABC
Za úplné riešenie dajte 6 bodov, z toho 3 body za časť a) a
3 body za časť b). Za drobný nedostatok v~argumentácii k časti b)
strhnite 1~bod.
\endschema
}

{%%%%%   B-II-2
Ako prvé uveďme grafické riešenie úlohy. Grafom funkcie na ľavej
strane rovnice je parabola roztvorená nahor,
ktorá má vrchol v~bode $[0,4]$.
Grafom funkcie na pravej strane je dvojica polpriamok
vychádzajúcich z~počiatku, ktoré sú súmerne združené podľa
osi~$y$. Polpriamka smerujúca doprava (graf funkcie pre
$x>0$) má smernicu $a$, polpriamka smerujúca doľava má
smernicu ${-a}$.

Uvedomme si, že celá situácia je súmerná
podľa osi $y$ (obe zastúpené funkcie ${x^2+4}$ a~$a|x|$ sú párne). Môžu
teda nastať tri prípady. Pre dostatočne malé hodnoty~$a$ polpriamky
parabolu nepretnú (tak to bude pre každé $a\leqq0$ a~tiež pre
niektoré ${a>0}$ ako na \obr{} vľavo) -- rovnica potom nebude
mať žiadny reálny koreň. Pre určitú hraničnú hodnotu $a=a_0>0$
budú obe polpriamky dotyčnicami paraboly (pozri \obrr1{} uprostred) --
v~tom prípade bude rovnica mať 2 reálne korene. Pri hodnotách
$a>a_0$ každá z~oboch polpriamok pretína parabolu v dvoch bodoch
(\obrr1{} vpravo) -- rovnica potom bude mať celkom 4 reálne korene.
Stačí teda dopočítať onú hraničnú hodnotu $a_0>0$.
Pre ňu musí mať kvadratická rovnica $x^2+4=a_0x$ práve jeden
dvojnásobný koreň, čo nastane práve vtedy, keď jej
diskriminant $a_0^2-16$ bude rovný nule. Tomu vyhovuje jediné
kladné $a_0=4$ (dvojnásobným koreňom je potom naozaj \emph{kladné}
číslo $x=2$).
\inspdf{b71ii_p2.pdf}%

\zaver
Pri každom $a>4$ má rovnica 4 reálne korene, pre $a=4$ má dva
reálne korene a~pre ľubovoľné $a<4$ nemá žiadny reálny koreň.


\ineriesenie
Podajme teraz čisto algebrické riešenie. Všimnime si, že číslo $x=0$
zrejme nie je koreňom zadanej rovnice (pri žiadnom $a$),
a~hľadajme najskôr jej kladné korene.

V~obore všetkých kladných čísel $x$ riešime rovnicu $x^2+4=ax$, čiže
$x^2-ax+4=0$. Jej prípadné reálne korene sú tvaru
$$
\frac{a\pm\sqrt{a^2-16}}2.
$$
Vidíme, že pri $|a|<4$ je diskriminant záporný a~rovnica
nemá žiadny reálny koreň. Pri $|a|=4$ má rovnica dvojnásobný
koreň $\frac12a$. Ten je však kladný práve vtedy, keď je
$a>0$, teda v~našom prípade $a=4$. Pri $|a|>4$ má rovnice dva
rôzne reálne korene. Vtedy ale z~$a^2-16<a^2$ vyplýva $\sqrt{a^2-16}<|a|$,
a~preto oba korene majú rovnaké znamienko ako parameter $a$. Teda pri $a>4$
má rovnica dva kladné korene, zatiaľ čo pri $a<-4$ nemá žiadny kladný koreň
(oba korene sú záporné).

Teraz môžeme podobne riešiť rovnicu v~obore všetkých záporných čísel $x$, v
ktorom má tvar $x^2+4={-ax}$. Zistíme, že pri $a>4$ má táto rovnica
dva záporné korene, pri $a=4$ iba jeden a~pri $a<4$ nemá žiadny
záporný koreň. Namiesto týchto výpočtov ale stačí konštatovať, že
z~tvaru pôvodnej rovnice $x^2+4=a|x|$ vyplýva, že číslo $x$ je jej
koreňom práve vtedy, keď je koreňom číslo ${-x}$. Preto sa výsledky
o~počte riešení v~obore $x>0$ bezo zmeny prenesú do oboru $x<0$.

Získané poznatky dokopy vedú k rovnakému záveru
ako v~prvom riešení.

\schemaABC
Za úplné riešenie dajte 6 bodov.

\smallskip
V~prípade neúplného grafického riešenia dajte:
\item{A1.} 1 bod za dostatočný popis oboch grafov alebo ich nakreslenie pre niektoré $a$.
\item{A2.} 3 body za vizuálne úvahy o~vplyve smerníc $\pm a$ oboch polpriamok na počet ich priesečníkov s~parabolou a~sformulovanie záveru o~existencii hodnoty $a_0$ s~vlastnosťou: Rovnica bude mať celkom 4, 2 resp. žiadne riešenie podľa toho, či $a>a_0$, $a=a_0$, resp. $a<a_0$ (bez výpočtu~$a_0$). Ak pritom využíva riešiteľ zrejmú súmernosť oboch grafov podľa osi $y$, zabudne ju však uviesť, tolerujte to.
\item{A3.} 2 body za určenie hodnoty $a_0$, keď sa obe polpriamky dotýkajú paraboly.
\endgraf\noindent
Pri hodnotení neúplného algebrického riešenia postupujte
nasledovne. Ak je konštatované, že vďaka symetrii
stačí rovnicu riešiť v~obore $x\geqq0$ (prípadne
$x>0$, ak je hodnota $x=0$ apriori vylúčená), dajte čiastkové
body podľa nasledujúcej schémy. Polovicu v~nej uvedených bodov
udeľujte vždy za každý z~oborov $x>0$ a~$x<0$, ak nie je symetria
spomenutá.
\item{B1.} 2 body za dôkaz, že pri $|a|<4$ nebude mať rovnica žiadne riešenie.
\item{B2.} 2 body za úplné vyriešenie oboch prípadov $a=\pm4$ (pri tvrdení, že rovnica bude mať riešenie aj pre $a={-4}$, dajte 0 bodov).
\item{B3.} 2 body za úplné vyriešenie prípadu $|a|>4$ (pri tvrdení, že rovnica bude mať riešenie aj pre niektoré $a<{-4}$, dajte 0 bodov).
\endgraf\noindent
Celkom potom dajte
$\rm\max\bigl({A1}+{A2}+{A3},{B1}+{B2}+{B3}\bigr)$
bodov. Len za pozorovanie symetrie bez ďalších poznatkov
(okrem prípadného vylúčenia hodnoty $x=0$) dajte 1 bod.
\endschema
}

{%%%%%   B-II-3
Označme $S$ stred pravidelného $n$-uholníka $A_1A_2\dots A_n$
a~$P$ priesečník polpriamok $A_1A_2$ a~$A_5A_4$ (\obr). Bod $P$
bude existovať pre každé $n>6$, zvyšné prípady $n=5$ a~$n=6$
rozoberieme na konci riešenia. Teraz teda predpokladajme, že $n>6$.
\inspdf{b71ii_p31.pdf}%

Zo súmernosti podľa priamky $SA_3$ je zrejmé, že
bod $P$ leží na polpriamke $SA_3$. V~zadaní úlohy vystupujúci
obraz bodu $A_3$ v~osovej súmernosti podľa priamky $A_1A_2$
označme~$A_3'$. Ďalej označme ešte
$\alpha=|\uhel A_1SA_2|=|\uhel A_2SA_3|=360^{\circ}/n$.
V~rovnoramennom trojuholníku $SA_1A_2$ máme $|\uhel SA_2A_1|=90^{\circ}-\frac12\alpha$, takže
pre jeho vonkajší uhol platí
$|\uhel SA_2P|={180^{\circ}-|\uhel SA_2A_1|}=90^{\circ}+\frac12\alpha$,
a~teda z~trojuholníka $SA_2P$ vychádza
$|\uhel A_3PA_2|=|\uhel SPA_2|=180^{\circ}-\alpha-(90^{\circ}+\frac12\alpha)=
90^{\circ}-\frac32\alpha$.
Vďaka súmernej združenosti bodov $A_2$, $A_4$ podľa priamky $SA_3=SP$
platí tiež $|\uhel SPA_4|=90^{\circ}-\frac32\alpha$. Podobne vďaka
súmernej združenosti bodov $A_3$, $A_3'$ podľa priamky $A_1A_2$
platí aj $|\uhel A_2PA'_3|=90^{\circ}-\frac32\alpha$.
Dokopy to znamená, že (nie nutne konvexný) uhol~$A_3'PA_4$
s~vnútorným bodom $S$ má veľkosť
$3\cdot(90^{\circ}-\frac32\alpha)=270^{\circ}-\frac92\alpha$. Našou úlohou je
nájsť všetky $n>6$, keď naposledy určená veľkosť je
$180^{\circ}$ (práve vtedy totiž bod $A_3'$ leží na priamke $A_4A_5$).
Rovnosť $270^{\circ}-\frac92\alpha=180^{\circ}$ ale nastane práve vtedy,
keď bude $\alpha=20^{\circ}$. Keďže ako vieme $\alpha=360^{\circ}/n$,
je $n=18$ jediným riešením úlohy v~obore všetkých čísel $n>6$.

Ostáva rozobrať prípady $n=6$ a~$n=5$. Pri $n=6$ sú priamky
$A_1A_2$ a~$A_4A_5$ rovnobežné a~priamka $A_4A_5$ leží celá
v~polrovine opačnej k~polrovine $SA_3A_3'$, a~preto bod~$A_3'$
neleží na priamke $A_4A_5$. Pri $n=5$ ležia bod $A_3'$
a~celá priamka $A_4A_5$ v~opačných polrovinách s~hraničnou priamkou
$A_1A_3$ (ktorá je totiž rovnobežná s~priamkou~$A_4A_5$), takže
ani vtedy bod $A_3'$ na priamke $A_4A_5$ neleží.

\zaver
Jediné riešenie úlohy je $n=18$.

\ineriesenie
Uveďme teraz odlišné riešenie, ktoré je založené na
porovnávaní dĺžok použitím goniometrických funkcií.
Zostrojme kolmicu z~bodu $A_3$ na priamku $A_1A_2$ a~označme jej
priesečníky s~priamkami $A_1A_2$ a~$A_4A_5$ postupne $O$ a~$X$ ako na
\obr{}. Rola bodu $X$ je jasná: našou úlohou je zistiť,
kedy bod $X$ je bodom $A_3'$ súmerne združeným s~bodom $A_3$
podľa priamky $A_1A_2$. Nastane to práve vtedy, keď bude platiť rovnosť
$|A_3X|=|A_3A_3'|$, čiže $|A_3X|=2|A_3O|$
a~súčasne bod $X$ bude ležať na polpriamke $A_3O$.

Naše riešenie povedieme nasledovne. Najskôr dokážeme, že ak
bod~$X$ na polpriamke~$A_3O$ leží (ako na \obrr1), je
rovnosť $|A_3X|=2|A_3O|$ splnená iba pre hodnotu $n=18$.
Na záver potom ukážeme, že pre $n=18$ bod~$X$ na
polpriamke~$A_3O$ naozaj leží.

Pre ďalšie výpočty označíme $\beta$ veľkosť uhla $A_3A_2O$
(na \obrr1{} vyznačeného červenou). Ukážeme, že $\beta=360^{\circ}/n$.
Naozaj, ak je $S$ stred kružnice opísanej nášmu $n$-uholníku,
platí $|\uhel A_1SA_3|=2\cdot|\uhel A_1SA_2|=2\cdot360^{\circ}/n$
a~podľa vety o~stredovom a~obvodovom uhle je
$|\uhel A_1A_2A_3|=180^{\circ}-|\uhel A_1SA_3|/2$,
dokopy už vychádza $\beta=|\uhel A_3A_2O|={180^{\circ}-
|\uhel A_1A_2A_3|}=|\uhel A_1SA_3|/2=360^{\circ}/n$. Ďalej ešte
využijeme, že veľkosť $180^{\circ}-\beta$ má nielen uhol
$A_1A_2A_3$, ale aj uhol $A_2A_3A_4$.
\inspdf{b71ii_p32.pdf}%

Vyjadrime teraz pomocou $\beta$ veľkosti
vnútorných uhlov trojuholníka $A_3A_4X$. Vďaka osovej súmernosti
podľa priamky $AP$ platí $|\uhel A_3A_4X|=\beta$. Druhý uhol $XA_3A_4$
má (na základe plného uhla s~vrcholom $A_3$)
veľkosť $|\uhel XA_3A_4|= 360^{\circ}-
|\uhel XA_3A_2|-|\uhel A_2A_3A_4|=360^{\circ}-(90^{\circ}-\beta)-
(180^{\circ}-\beta)=90^{\circ}+2\beta$. Pre tretí uhol $A_3XA_4$
tak dostávame $|\uhel A_3XA_4|=180^{\circ}-\beta-(90^{\circ}+2\beta)=
90^{\circ}-3\beta$.

Označme teraz $a$ dĺžku strany nášho $n$-uholníka. Keďže
z~trojuholníka $A_3A_2O$ vyplýva $|A_3O|=a \sin \beta$, želaná rovnosť
$|A_3X|=2|A_3O|$ nastane práve vtedy, keď bude platiť
$|A_3X|=2a\sin\beta$. Podľa sínusovej vety pre trojuholník $A_3A_4X$
máme
$$
|A_3X|:|A_3A_4|=\sin|\uhel A_3A_4X|:\sin |\uhel A_3XA_4|.
$$
Ak sem dosadíme určené veľkosti uhlov spolu s~dĺžkou
$|A_3A_4|=a$, získame
$$
|A_3X|=\frac{|A_3A_4|\cdot\sin|\uhel A_3A_4X|}{\sin |\uhel
A_3XA_4|}=\frac{a\cdot\sin\beta}{\sin(90^{\circ}-3\beta)}
=\frac{a\cdot\sin\beta}{\cos3\beta}.
$$
Posledný výraz má požadovanú hodnotu $2a\sin\beta$ práve vtedy, keď
$\cos 3\beta=\frac12$. To nastane práve v~prípade
$3\beta=60^{\circ}$, čiže $\beta=360^{\circ}/n=20^{\circ}$.
Táto rovnosť je zrejme splnená pre jediné $n=18$.

Ostáva ukázať, že v~prípade $n=18$ náš priesečník $X$ leží na polpriamke
$A_3O$. To sme pri predchádzajúcich výpočtoch využili, keď sme
veľkosti uhlov $A_3A_4X$ a~$XA_3A_4$ počítali vlastne ako veľkosti uhlov
$A_3A_4P$ a~$OA_3A_4$. Platí teda $|\uhel A_3A_4P|=\beta$
a~$|\uhel OA_3A_4|=90^{\circ}+2\beta$. Z toho v~prípade $n=18$, keď
$\beta=20^{\circ}$, dostávame
$$
|\uhel A_3A_4P|+|\uhel OA_3A_4|=90^{\circ}+3\beta=150^{\circ}
<180^{\circ}.
$$
Táto nerovnosť už znamená, že polpriamky $A_3O$ a~$A_4P$ sa
pretínajú (a~ich priesečníkom je teda bod $X$).
Tým je celé riešenie ukončené.

\schemaABC
Za úplné riešenie dajte 6 bodov. \emph{Vypočítaním} v~nasledujúcej
schéme pre neúplné riešenie rozumieme vyjadrenie závislosti
od $n$, $\alpha$, $\beta$, $a$ alebo iného zvoleného prvku
pravidelného $n$-uholníka.
Za čiastkové kroky dajte:
\item{X0.} 0 bodov za zavedenie bodov $O$, $P$, $X$.
\item{X1.} 0 bodov za vypočítanie veľkostí ľubovoľných uhlov pri vrcholoch $n$-uholníka.
\item{A1.} 1 bod za pozorovanie, že požadovaná situácia nastane práve vtedy, keď $|\uhel A_3'PA_4|=180^{\circ}$.
\item{A2.} 1 bod za pozorovanie, že tri uhly pri vrchole $P$ (červenou vyznačené na \obrr2{}) sú vďaka dvom osovým súmernostiam zhodné.
\item{A3.} 3 body za vypočítanie veľkostí všetkých uhlov $A_3'PA_2$, $A_2PA_3$ a~$A_3PA_4$ (za každý uhol po 1~bode).
\item{B1.} 1 bod za konštatovanie, že požadovaná situácia nastane práve vtedy, keď $|A_3X|=|A_3A_3'|$.
\item{B2.} 1 bod za vypočítanie $|A_3A_3'|$ (čiže $2|A_3O|$ v~našom riešení).
\item{B3.} 3 body za vypočítanie $|A_3X|$.
\endgraf\noindent
Celkom potom dajte
$\rm\max\bigl({A1}+\max({A2},{A3}),
{B1}+{B2}+{B3}\bigr)$ bodov.
Absenciu rozboru prípadov $n=5$, $n=6$ (pri prvom postupe)
či záverečnej diskusie o~polohe bodu $X$ (pri druhom postupe)
v~inak úplnom riešení nepenalizujte.
\endschema

}

{%%%%%   B-II-4
Skúsme si najskôr rozmyslieť, aký najväčší súčet môže byť
v~riadku tabuľky, ak je tvorený číslami ${-4}$, 3 a~10 a~má byť
nanajvýš 0. Všimnime si, že všetky tri čísla ${-4}$, 3 a~10
dávajú zvyšok 3 po delení siedmimi. Preto súčet desiatich čísel jedného
riadka bude dávať po delení siedmimi rovnaký zvyšok ako číslo
$10\cdot 3=30$, teda zvyšok 2. Najväčšie nekladné číslo s~touto vlastnosťou
je ${-5}$. V~každom z~deviatich riadkov, kde je
súčet nekladný, je teda súčet nutne nanajvýš ${-5}$.
V~poslednom riadku, na ktorý nemáme žiadne obmedzenia, je súčet
nanajvýš $10\cdot 10=100$. Celkový súčet čísel v~tabuľke
teda určite nepresiahne $9\cdot (-5)+100=55$.

Nájdime teraz tabuľku $10\times 10$ so súčtom čísel 55, ktorá
vyhovuje podmienkam zo zadania. Predchádzajúci odsek nám
napovedá, ako ju hľadať, keď si ešte uvedomíme, že súčet
${-5}$ v~riadku alebo stĺpci môžeme dosiahnuť použitím dvoch desiatok,
jednej trojky a~siedmich mínus štvoriek, lebo
$2\cdot10+1\cdot3+7\cdot({-4})={-5}$.
Preto posledný riadok aj posledný stĺpec vyplníme desiatkami
a~zvyšnú tabuľku $9\times 9$ sa budeme snažiť vyplniť tak,
aby v~každom jej riadku a~aj každom jej stĺpci bola
jedna desiatka, jedna trojka a~sedem mínus štvoriek. To možno
dosiahnuť spôsobom založeným na cyklickom posúvaní poradí čísel
v~riadkoch (a~teda aj stĺpcoch):
$$%PRVNÍ TABULKA
%\begin{tabular}{ | c | c| c | c | c | c | c | c | c | c | }
\TAB{ %nove
\hline
10 & 3 & -4 & -4 & -4 & -4 & -4 & -4 & -4 & 10 \\
\hline
-4 & 10 & 3 & -4 & -4 & -4 & -4 & -4 & -4 & 10 \\
\hline
-4 & -4 & 10 & 3 & -4 & -4 & -4 & -4 & -4 & 10 \\
\hline
-4 & -4 & -4 & 10 & 3 & -4 & -4 & -4 & -4 & 10 \\
\hline
-4 & -4 & -4 & -4 & 10 & 3 & -4 & -4 & -4 & 10 \\
\hline
-4 & -4 & -4 & -4 & -4 & 10 & 3 & -4 & -4 & 10 \\
\hline
-4 & -4 & -4 & -4 & -4 & -4 & 10 & 3 & -4 & 10 \\
\hline
-4 & -4 & -4 & -4 & -4 & -4 & -4 & 10 & 3 & 10 \\
\hline
3 & -4 & -4 & -4 & -4 & -4 & -4 & -4 & 10 & 10 \\
\hline
10 & 10 & 10 & 10 & 10 & 10 & 10 & 10 & 10 & 10 \\ \hline
%\end{tabular}
}
$$


\Zav
Najväčší možný súčet čísel v~tabuľke je rovný 55.

\Jres
Ak si nevšimneme, že všetky tri čísla 10, 3 a~${-4}$ dávajú
zvyšok~3 po delení siedmimi, môžeme dokázať inak, že súčty v~deviatich
riadkoch neprevyšujú ${-5}$, konkrétne rozborom prípadov možných
počtov zastúpených desiatok:
\item{$\bullet$} Keby v~riadku boli aspoň tri desiatky, dali by v~súčte najmenej~30 a~súčet čísel v~zvyšných nanajvýš siedmich políčkach by bol najmenej $7\cdot(-4)=-28$, \tj. celkový súčet by bol kladný, čo odporuje zadaniu.
\item{$\bullet$} Ak sú v~riadku dve desiatky, potrebujeme ich \uv{vyvážiť} piatimi mínus štvorkami ($2\cdot10+5\cdot({-4})=0$). Ostávajú nám tri políčka, stoja v~nich trojky a~mínus štvorky tak, že ich súčet je nekladný. Určite nemôžu byť dve z~týchto troch čísel kladné ($3+3-4>0$). Najväčší možný nekladný súčet pri dvoch desiatkach je teda ${3-4-4={-5}}$.
\item{$\bullet$} Ak je v~riadku práve jedna desiatka, musia tam byť aspoň tri mínus štvorky. Keďže $10+3\cdot({-4})={-2}$, v zvyšných šiestich políčkach stoja trojky a~mínus štvorky so súčtom nanajvýš 2. Keďže $3+3+3+3-4-4=4>2$, najväčší súčet menší ako 3 je $3+3+3-4-4-4=-3$. Celkový súčet v~riadku potom je $1\cdot 10+3\cdot 3+{6\cdot({-4})}={-5}$.
\item{$\bullet$} Ak v~riadku nie je žiadna desiatka, potom vzhľadom na $6\cdot3+4\cdot({-4})>0$ je najvyšší možný nekladný súčet rovný $5\cdot3+5\cdot({-4})={-5}$.
\endgraf\noindent
Ako sme sľúbili, rozborom možností sme dokázali, že ak je
súčet čísel v~riadku nekladný, potom je nanajvýš rovný ${-5}$.

Dodajme, že prevedenie rozboru možno skrátiť nasledujúcim spôsobom.
Ak je v~riadku tabuľky $x$ desiatok, $y$ trojok a~$10-x-y$ mínus
štvoriek, má ich súčet hodnotu
$$
s=10x+3y+(10-x-y)\cdot({-4})=14x+7y-40.
$$
Vidíme, že v~prípade $s\leqq0$ sú možné iba hodnoty $x=0,1,2$. Im
zodpovedajú hodnoty~$s$ postupne rovné $7y-40$, $7y-26$ a~$7y-12$,
ktoré majú najväčšie nekladné hodnoty postupne rovné $35-40={-5}$,
$21-26={-5}$ a~$7-12={-5}$. V~prípade $x=0$ ale nebude v~tabuľke
stĺpec so samými desiatkami, ktorý pre získanie celkového
súčtu~55 potrebujeme.

\poznamka
Rozbor prípadov v druhom riešení ukazuje, že tabuľka $10\times10$
spĺňajúca zadanie úlohy má celkový súčet čísel 55 práve vtedy, keď má
jeden riadok a~jeden stĺpec vyplnený samými desiatkami a~v~každom
z~ostatných deviatich riadkov a~deviatich stĺpcov má súčet čísel (až na
ich poradie) tvaru $2\cdot10+1\cdot3+7\cdot({-4})$ alebo tvaru
$1\cdot10+3\cdot3+6\cdot({-4})$.
To nám dáva ďalšie príklady takých tabuliek:
$$\line{%
%\begin{tabular}{ | c | c| c | c | c | c | c | c | c | c | }
$\TAB{ %nove
\hline
3 & 3 & 3 & -4 & -4 & -4 & -4 & -4 & -4 & 10 \\
\hline
-4 & 3 & 3 & 3 & -4 & -4 & -4 & -4 & -4 & 10 \\
\hline
-4 & -4 & 3 & 3 & 3 & -4 & -4 & -4 & -4 & 10 \\
\hline
-4 & -4 & -4 & 3 & 3 & 3 & -4 & -4 & -4 & 10 \\
\hline
-4 & -4 & -4 & -4 & 3 & 3 & 3 & -4 & -4 & 10 \\
\hline
-4 & -4 & -4 & -4 & -4 & 3 & 3 & 3 & -4 & 10 \\
\hline
-4 & -4 & -4 & -4 & -4 & -4 & 3 & 3 & 3 & 10 \\
\hline
3 & -4 & -4 & -4 & -4 & -4 & -4 & 3 & 3 & 10 \\
\hline
3 & 3 & -4 & -4 & -4 & -4 & -4 & -4 & 3 & 10 \\
\hline
10 & 10 & 10 & 10 & 10 & 10 & 10 & 10 & 10 & 10 \\ \hline
%\end{tabular}
} %nove
\hfill
%TŘETÍ TABULKA
%\begin{tabular}{ | c | c| c | c | c | c | c | c | c | c | }
\TAB{ %nove
\hline
10 & 3 & -4 & -4 & -4 & -4 & -4 & -4 & -4 & 10 \\
\hline
-4 & 10 & 3 & -4 & -4 & -4 & -4 & -4 & -4 & 10 \\
\hline
-4 & -4 & 10 & 3 & -4 & -4 & -4 & -4 & -4 & 10 \\
\hline
-4 & -4 & -4 & 3 & 3 & 3 & -4 & -4 & -4 & 10 \\
\hline
-4 & -4 & -4 & -4 & 3 & 3 & 3 & -4 & -4 & 10 \\
\hline
-4 & -4 & -4 & -4 & -4 & 3 & 3 & 3 & -4 & 10 \\
\hline
-4 & -4 & -4 & -4 & -4 & -4 & 3 & 3 & 3 & 10 \\
\hline
-4 & -4 & -4 & 3 & -4 & -4 & -4 & 3 & 3 & 10 \\
\hline
3 & -4 & -4 & -4 & 3 & -4 & -4 & -4 & 3 & 10 \\
\hline
10 & 10 & 10 & 10 & 10 & 10 & 10 & 10 & 10 & 10 \\ \hline
%\end{tabular}
}$% %nove
}$$
%\medskip


\schemaABC
Za úplné riešenie dajte 6 bodov. V~prípade neúplného riešenia dajte:
\item{A1.} 2 body za príklad (alebo úplný popis) tabuľky so súčtom 55.
\item{B0.} 0 bodov za hypotézu, že súčet čísel v~tabuľke nepresiahne 55.
\item{B1.} 1 bod za hypotézu, že súčty v~deviatich riadkoch (resp. stĺpcoch) nemôžu byť väčšie ako~${-5}$, a~z~toho vyplývajúci záver, že súčet čísel v~tabuľke nepresiahne~55.
\item{B2.} 4 body za dôkaz, že súčet čísel v~tabuľke nemôže byť väčší ako~55, z~toho 3 body za dôkaz, že súčty v~deviatich riadkoch (resp. stĺpcoch) nemôžu byť väčšie ako~${-5}$, a~1 bod za získanie horného odhadu celkového súčtu číslom 55. Ak je rozbor prípadov neúplný alebo s~chybami, dajte nanajvýš 2~body z~týchto~4.
\endgraf\noindent
Celkom potom dajte
$\rm{A1}+\max({B1},{B2})$ bodov.
\endschema
}

{%%%%%   C-S-1
Uvažujme akúkoľvek tabuľku $10\times10$ vyplnenú podľa zadania a
odhadnime súčty jej čísel v~jednotlivých riadkoch, keď vieme,
že to sú násobky čísla~3. To isté bude zrejme platiť aj pre súčty čísel v stĺpcoch.

V~jednom riadku je desať čísel~$\pm1$, takže pre ich súčet
prichádzajú do úvahy hodnoty, ktoré určíme zostupne podľa podľa
počtu zastúpených plus jednotiek: 10~(desať jednotiek),
8~(deväť jednotiek), 6 (osem jednotiek), atď.
Vidíme, že deliteľná tromi je až tretia najväčšia hodnota 6.
Súčet čísel v~ľubovoľnom riadku je preto nanajvýš 6.
Z toho vyplýva, že súčet všetkých čísel v~tabuľke (ktorá má 10 riadkov)
neprevyšuje hodnotu $10\cdot6=60$.
Ak uvedieme príklad správne vyplnenej tabuľky
so súčtom čísel rovným 60, bude to naozaj jeho
najväčšia možná hodnota a úloha bude vyriešená.

Nájsť požadovaný príklad nám pomôže poznatok z~predchádzajúceho
odseku, podľa ktorého tabuľku máme vlastne vyplniť tak,
aby v~každom riadku aj stĺpci bolo práve osem jednotiek. Môžeme to
spraviť mnohými spôsobmi. Dva z~nich (vykazujúce určitú pravidelnosť)
sú uvedené na \obr{}, v ktorom sme všetky políčka s~číslami 1
vyfarbili (takže biele zostali políčka s~číslami ${-1}$).
\inspsc{c71s1-1.eps}{0.8333}%

\schemaABC
Za úplné riešenie dajte 6~bodov. V neúplných riešeniach oceňte čiastkové kroky nasledovne.
\item{A1.} Pozorovanie, že v~jednom riadku (alebo stĺpci) je súčet čísel nanajvýš~6 (alebo ekvivalentne, je v~ňom nanajvýš 8 jednotiek): 2 body.
\item{A2.} Dôkaz toho, že súčet čísel v~tabuľke je nanajvýš~60: 3 body.
\item{B1.} Sformulovanie úlohy nájsť tabuľku, ktorá má v každom riadku aj stĺpci 8 jednotiek: 1~bod.
\item{B2.} Uvedenie tabuľky so súčtom čísel 60 (alebo jej kompletný opis): 3 body.
\item{C1.} Uhádnutie odpovede: 0 bodov.
\endgraf\noindent
Celkovo potom dajte $\rm\max(A1,\,A2)+\max(B1,\,B2)$ bodov.
\endschema
}

{%%%%%   C-S-2
Podľa zadania leží bod $L$ na osi uhla $ABC$, takže
$|\uhel ABL|=|\uhel CBL|$. Podľa vety o~striedavých uhloch
platí tiež $|\uhel ABL|=|\uhel CLB|$. Dokopy dostávame
$|\uhel CBL|=|\uhel CLB|$, takže trojuholník $CLB$ je rovnoramenný so
základňou~$LB$, \tj. $|BC|=|LC|$ (\obr).
\inspsc{c71s2-1.eps}{.8333}%

Keďže v~rovnobežníku $ABCD$ platí $|AD|=|BC|$ a bod $L$ je stred
strany $CD$, možno dokázanú rovnosť $|BC|=|LC|$ prepísať
na $|AD|=|LD|$. Trojuholník $ALD$ je teda rovnoramenný
so základňou $AL$, a preto $|\uhel DAL|=|\uhel DLA|$.
Uhol $DLA$ je však zhodný so striedavým uhlom $BAL$,
teda $|\uhel DAL|=|\uhel BAL|$.
To znamená, že bod $L$ leží nielen na osi uhla $ABC$,
ale aj na osi uhla $BAD$. Tým pádom platí
$$
|\uhel BAL|+|\uhel LBA|=
\tfrac12\,|\uhel BAD|+\tfrac12\,|\uhel ABC|=
\tfrac12\,\bigl(|\uhel BAD|+|\uhel ABC|\bigr)=90^{\circ},
$$
pričom sme využili to, že vďaka $BC\parallel AD$ platí
$|\uhel BAD|+|\uhel ABC|=180^{\circ}$.

Došli sme k~záveru, že v~trojuholníku $ABL$ je súčet vnútorných uhlov
pri vrcholoch $A$ a~$B$ rovný $90^{\circ}$, takže pri treťom vrchole~$L$
je uhol pravý, ako sme mali dokázať.

\ineriesenie
Označme $K$ stred strany~$AB$. Keďže štvoruholník $KBCL$
je rovnobežník\fnote{Toto pozorovanie nie je nutné dokazovať~--~vyplýva z~toho, že protiľahlé strany $KB$ a $LC$ sú zhodné a~rovnobežné.}, platí v ňom $|KL|=|BC|$. Ak preto rovnako
ako v~prvom riešení dokážeme rovnosti
$|BC|=|CL|=\frac12|CD|=|KA|=|KB|$, budeme dokopy mať
$|KL|=|KA|=|KB|$. Z~toho vďaka Tálesovej vete už
dostávame, že $ABL$ je pravouhlý trojuholník s~preponou $AB$ (\obr).
\inspsc{c71s2-2.eps}{.8333}%

\poznamka
Z rovností uvedených v druhom riešení vyplýva,
že rovnobežník $KBCL$ je kosoštvorec (prípadne štvorec),
a preto platí $KC\perp BL$. Po tomto zistení možno
riešenie dokončiť takto (pozri \obr{}):
Keďže aj $AKCL$ je zrejme rovnobežník (z rovnakého dôvodu ako
$KBCL$, pozri poznámku pod čiarou), platí
v~ňom $AL\parallel KC$, odkiaľ už vďaka $KC\perp BL$ máme $AL\perp BL$.
\inspsc{c71s2-3.eps}{.8333}%

\schemaABC
Za úplné riešenie dajte 6~bodov. V~neúplných riešeniach oceňte
čiastkové kroky nasledovne.
\item{A1.} Dôkaz rovnosti $|BC|=|LC|$: 2 body.
\item{A2.} Dôkaz rovnosti $|\uhel DAL|=|\uhel BAL|$: 2 body.
\item{B1.} Zavedenie stredu $K$ strany $AB$: 0 bodov.
\item{B2.} Pozorovanie, že $KBCL$ je rovnobežník: 1 bod.
\item{B3.} Dôkaz zhodnosti úsečky $KL$ s úsečkami $KA$, $KB$: 5 bodov.
\item{B4.} Dôkaz kolmosti $KC\perp BL$: 4 body
\item{B5.} Pozorovanie, že $AKCL$ je rovnobežník: 1 bod
\endgraf\noindent
Celkovo potom dajte $\rm\max(A1+A2,\,B2,\,B3,\,B4+B5)$ bodov.
\endschema
}

{%%%%%   C-S-3
Ľavú stranu zadanej rovnice roznásobíme a~upravíme:
$$
\eqalign{
(a-b)(c-d)+(a-d)(b-c)&=(ac-bc-ad+bd)\ + \ (ab-bd-ac+cd)=\cr
&= -bc-ad+ab+cd= (a-c)(b-d).
}
$$
Máme teda rovnicu $(a-c)(b-d)=26=2\cdot13$. Z~podmienky $a>b>c$ vyplýva,
že celé čísla~$a>c$ sa líšia o~aspoň~2, \tj. $a-c\ge2$.
Podobne z~$b>c>d$ vyplýva $b-d\ge2$. Tým pádom v~rovnici
$(a-c)(b-d)=2\cdot13$ sa činitele $a-c$, $b-d$ rovnajú číslam~2 a~13
v~niektorom poradí (lebo 13 je prvočíslo). Tieto dve možnosti teraz
rozoberieme.

\item{$\triangleright$}
$a-c=2$ a $b-d=13$.
Potom platí $a=c+2$ a $d=b-13$. Podmienka $a>b>c$
tak prejde na tvar $c+2>b>c$. Keďže však medzi celými
číslami $c+2$ a~$c$ leží jediné ďalšie celé číslo $c+1$, tak nutne
platí $b=c+1$, odkiaľ $d=b-13=c-12$. Naša štvorica
$(a,b,c,d)$ má teda tvar $(c+2,c+1,c,c-12)$. Z~podmienky
$a+b+c+d=71$ potom vyplýva $4c-9=71$, čiže $c=20$, a~preto
$(a,b,c,d)=(22,21,20,8)$. Skúška nie je nutná.

\item{$\triangleright$}
$b-d=2$ a~$a-c=13$. Podobne ako v~prvom prípade vďaka $b=d+2$
prejde podmienka $b>c>d$ na tvar $d+2>c>d$, podľa
ktorého $c=d+1$, a teda $a=c+13=d+14$. Naša štvorica
$(a,b,c,d)=(d+14,d+2,d+1,d)$ po dosadení do rovnosti $a+b+c+d=71$
dáva $4d+17=71$, odkiaľ $d=27/2$, čo ale nie je celé číslo.
Tento prípad je tak vylúčený.

\zaver
Zadaniu úlohy vyhovuje jediná štvorica $(a,b,c,d)$,
a to $(22,21,20,8)$.

\schemaABC
Za úplné riešenie dajte 6~bodov. V žiadnom riešení nepenalizujte
absenciu skúšky. V~neúplných riešeniach ohodnoťte čiastkové kroky
nasledovne.
\item{A1.} Rozklad ľavej strany rovnice na súčin $(a-c)(b-d)$, prípadne $(c-a)(d-b)$: 2 body.
\item{B1.} Vylúčenie prípadov, keď $\{a-c,b-d\}=\{1,26\}$: 1 bod
\item{B2.} Vyriešenie jedného či oboch prípadov, keď $\{a-c,b-d\}=\{2,13\}$: 1 prípad 2 body, 2 prípady 3~body.
\item{B3.} Pozorovanie, že pre celé čísla $x$, $y$, $z$ vyplýva zo vzťahov $x>y>z$ a~$x-z=2$ rovnosť $y=z+1$: 1 bod.
\item{B4.} Uhádnutie odpovede: 1 bod.
\endgraf\noindent
Celkovo potom dajte $\rm A1+\max(B1,\,B2,\,B3,\,B4)$ bodov.
\endschema
}

{%%%%%   C-II-1
Zadanú nerovnosť postupne ekvivalentne upravujeme
$$
\eqalign{
a(a+1)+b(b-1)&\ge 2ab,\cr
(a^2+a)+(b^2-b)-2ab&\ge0,\cr
(a^2-2ab+b^2)+(a-b)&\ge0,\cr
(a-b)^2+(a-b)&\ge0,\cr
(a-b)(a-b+1)&\ge0.
}
$$
Poslednú nerovnosť posúdime pre celé čísla $a$ a~$b$ v dvoch
prípadoch.
\item{a)} Ak je $a \ge b$, tak $a-b \ge 0$, a~teda aj $a-b+1>0$, takže
dokopy máme
${(a-b)(a-b+1)} \ge 0$ s~rovnosťou práve pre $a=b$.
\item{b)} Ak je naopak $a<b$, tak $a-b<0$, čo pre celé číslo $a-b$ znamená,
že $a-b \leq {-1}$, čiže $a-b+1 \leq 0$. Spolu s~$a-b<0$ tak
máme $(a-b)(a-b+1) \ge 0$ s~rovnosťou práve pre $a=b-1$.
\endgraf\noindent
Dôkaz nerovnosti je teda ukončený a~rovnosť nastane práve vtedy, keď
pre celé čísla $a$, $b$ platí $a=b$ alebo $a=b-1$.

\poznamky
\item{1.} Potrebnú úpravu nerovnosti aj následnú diskusiu možno
spraviť aj tak, že uvážime celé číslo $x=a-b$. Ak potom dosadíme
$a=b+x$ do zadanej nerovnosti, po roznásobení a~zrušení rovnakých
členov na oboch stranách nám zostane nerovnosť $x^2+x\geqq0$, čiže
$x(x+1)\geqq0$ a~dokončenie je jednoduché.
\item{2.} Uveďme ešte malú obmenu druhej časti uvedeného riešenia.
Z~upravenej nerovnosti $(a-b)(a-b+1)\geqq0$ hneď vidíme, že
aj v~pôvodnej nerovnosti nastane rovnosť práve v dvoch prípadoch: $a=b$
a~$a=b-1$. Ak nenastane žiadny z~nich, nebude celé číslo $a$
rovné žiadnemu z dvoch susedných celých čísel $b-1$ a~$b$, takže
bude platiť buď $a>b$, alebo $a<b-1$. V~každom z~týchto
prípadov budú mať oba činitele $a-b$, $a-b+1$ rovnaké znamienko
a~ich súčin tak bude kladný.


\schemaABC
Za úplné riešenie dajte 6~bodov. V~prípade čiastočných riešení dajte:
\item{$\bullet$} 2~body za úpravu nerovnosti na súčinový tvar $(a-b)(a-b+1)\geq0$, prípadne na tvar $x^2+x\geqq0$ po zavedení substitúcie $x=a-b$.
\item{$\bullet$} 2~body za vyriešenie prípadu $a\geqq b$, resp. $x\geqq0$.
\item{$\bullet$} 2~body za vyriešenie prípadu $a<b$, resp. $x<0$.
\endgraf\noindent
Dva body z~prvej položky možno udeliť aj za úpravu na
tvar ako je $(a-b)^2+(a-b)\geqq0$ či $(b-a)^2\geqq b-a$,
ale iba ak žiak ďalej dokáže aj s~takou nerovnosťou vyriešiť
\emph{oba} prípady $a\geqq b$ a~$a<b$ (potom však sa jedná o~úplné
riešenie). Ak s~ňou vyrieši iba jeden prípad $a\geqq b$, resp. $a<b$,
dajte celkom 2, resp. 3 body.

V~každom z~oboch uvedených prípadov možno strhnúť po 1~bode za drobné nedostatky
v~nerovnostnej argumentácii. Ak riešiteľ vyčlení triviálny prípad $a=b$ samostatne, za ten žiadny bod neudeľujte,
hodnoťte ho spolu s~prípadom $a>b$. V~prípade chybnej analýzy prípadov
rovnosti strhnite dokopy nanajvýš 1~bod. Ak chýba zmienka
o~ekvivalentnosti úprav, body nestrhávajte, ak sú úpravy zrejmé ako
v~našom riešení. Len za uhádnutie \emph{oboch} prípadov rovnosti
($a=b$ a~$a=b-1$) dajte 1 bod, ktorý sa ale nedá pripočítať ku 2
bodom za úpravu nerovnosti z~prvej položky pokynov.
\endschema
}

{%%%%%   C-II-2
Aby priamky~$AX$ a~$AY$ rozdeľovali rovnobežník $ABCD$
na tri časti, musia byť zrejme body $X$ a~$Y$ navzájom rôzne
a~žiadny z~nich nemôže ležať ani na strane $AB$, ani na strane $AD$.
Každý z~nich teda leží na strane~$BC$ alebo~$CD$. Zdôvodnime
v~ďalšom odseku, že jeden z~bodov $X$, $Y$ musí
ležať vnútri strany~$BC$ a~druhý vnútri strany $CD$,
keď podľa zadania všetky tri časti rozdeleného rovnobežníka majú
v porovnaní s~ním tretinový obsah.

Keďže uhlopriečka $AC$ rozpoľuje obsah rovnobežníka $ABCD$,
žiadna z troch častí s~tretinovým obsahom nemôže obsahovať ani celý trojuholník $ABC$,
ani celý trojuholník $ACD$. Keby však oba body ležali na strane $BC$
ako na \obr{}, jedna z troch častí by obsahovala celý trojuholník
$ACD$. Rovnako tak sa vylúči prípad, keď oba body $X$, $Y$ ležia na strane
$CD$.
\inspsc{c71ii_p21.eps}{.8333}%


Keďže označenie $X$ a~$Y$ môžeme navzájom vymeniť, budeme
ďalej predpokladať, že bod $X$ leží vnútri strany $BC$ a~bod $Y$ vnútri
strany $CD$ (\obr). Rovnobežník $ABCD$ je potom rozdelený na dva trojuholníky
$ABX$, $AY\!D$ a~štvoruholník $AXCY$.
\inspsc{c71ii_p22.eps}{.8333}%

Označme $S$ obsah celého rovnobežníka $ABCD$.
Podľa zadania pre obsahy trojuholníkov $ABX$ a~$AY\!D$
platí $S_{ABX}=S_{AY\!D}=S/3$. Keďže
trojuholníky $ABX$ a~$ABC$ majú spoločnú výšku z~vrcholu~$A$, dĺžky
protiľahlých strán sú v~pomere
$$
|BX|:|BC|=S_{ABX}:S_{ABC}=(S/3):(S/2)=2:3,
$$
odkiaľ $|BX|=\frac23|BC|$, takže $|CX|=\frac13|BC|$. Podobne porovnaním trojuholníkov
$AY\!D$ a~$ACD$ dostaneme $|CY|=\frac13|CD|$. Dokopy dostávame, že
podľa vety \emph{sus} sú trojuholníky $CXY$ a~$CBD$ podobné v~pomere $1:3$.
Pre obsah prvého z~nich teda platí
$$
S_{CXY} =
\biggl(\frac 13\biggr)^2 \cdot S_{CBD} =
\left(\frac 13\right)^2 \cdot \frac{S}{2} =
\frac{S}{18}.
$$
Štvoruholník $AXCY$ má tiež obsah $S/3$
a~je zložený z~trojuholníkov $AXY$ a~$CXY$. Obsah druhého z~nich už
poznáme, takže prvý z~nich má obsah
$$
S_{AXY}=\frac{S}{3}-S_{CXY}=\frac{S}{3}-\frac{S}{18}
=\frac{5S}{18}.
$$

\zaver
Hľadaný pomer obsahov je rovný $5:18$.


\schemaABC
Za úplné riešenie dajte 6~bodov. V~prípade čiastočných
riešení dajte:
\item{$\bullet$} 1~bod za vylúčenie prípadu, keď oba body $X$ a~$Y$ ležia na jednej zo strán $BC$ alebo $CD$.
\item{$\bullet$} 2~body za určenie aspoň jedného z~pomerov, v~akom body $X$, $Y$ delia príslušnú zo strán $BC$, resp.~$CD$.
\item{$\bullet$} 2~body za výpočet pomeru obsahu trojuholníka $CXY$ k~obsahu rovnobežníka $ABCD$.
\endgraf\noindent
Absenciu vylúčenia polôh bodov $X$ a~$Y$ na stranách $AB$ a~$AD$
nepenalizujte. Nepenalizujte ani zabudnutie prípadov, keď $X$ alebo
$Y$ je totožný s~jedným z~vrcholov $B$, $C$, $D$. Jeden bod však
strhnite, ak chýba vysvetlenie, prečo oba body~$X$ a~$Y$ nemôžu
ležať na jednej zo strán $BC$, $AD$.
\endschema
}

{%%%%%   C-II-3
a) Nasledujúce príklady vždy jedného alebo dvoch
čísel napísaných na tabuli ukazujú, že cifry 1, 2, 3, 4, 5,
7 môžu byť dobré.\fnote{Dá sa dokázať, že k cifrám~1, 2 a~3 existuje postupne~43,
11 a~5 príkladov vyhovujúcich množín dvojciferných čísel s dotyčnou cifrou.
Pri ostatných vypísaných cifrách 4, 5 a~7 sú uvedené príklady jediné možné.
Pre cifry 4 a~7 to zdôvodníme v~časti~b) riešenia, pre cifru 5
to možno urobiť podobne ľahko.}
$$
\eqalign{
1{:}\quad &\{71\},\cr
2{:}\quad &\{29,42\},\cr
3{:}\quad &\{32,39\},\cr
4{:}\quad &\{24,47\},\cr
5{:}\quad &\{15,56\},\cr
7{:}\quad &\{71\}.
}$$

%\def\van{\hbox{\ttset\rm \char"20}}
\def\van{{\llcorner\mskip-6mu\lrcorner}}
Ukážme, že žiadna zo zvyšných cifier 0, 6, 8 a~9 nemôže byť nikdy
dobrá. Dokážeme to pre ne jednotlivo, budeme pritom vždy hovoriť
o~výskytoch danej cifry v~číslach na tabuli.
\item{$\bullet$} Cifra 0 môže byť iba na miestach jednotiek. Ale súčet takých čísel vždy končí cifrou~0, a~nie požadovanou cifrou 1.
\item{$\bullet$} Keby boli cifry 6 iba na miestach jednotiek, bol by súčet čísel s~cifrou~6 párny, a~teda rôzny od 71. Majme teda aspoň jedno číslo tvaru $6\van$. Preň však platí $6\van<71<60+16$, takže súčet~71 sa nedá získať.
\item{$\bullet$} Cifra~8 je príliš veľká na to, aby bola niekde na mieste desiatok. Musí teda byť všade na miestach jednotiek, súčet takých čísel je však párny.
\item{$\bullet$} Cifra~9 môže byť (z rovnakého dôvodu ako cifra 8) iba na mieste jednotiek. Aby súčet takých čísel končil na 1, museli by sme ich sčítať aspoň~9, ale $9\cdot19>71$.

\zaver
Dobré môžu byť práve cifry 1, 2, 3, 4, 5 alebo 7.

\bigskip
b) Dokážeme najskôr, že všetky do úvahy prichádzajúce cifry
1, 2, 3, 4, 5 a~7, ktorých je celkom 6, nemôžu byť dobré súčasne.
Na to stačí ukázať, že dobré nemôžu byť súčasne cifry 4 a~7.

Skúmajme teda, kedy je cifra~4 dobrá. Keďže číslo 71 je
nepárne, všetky sčítané čísla nemôžu mať cifru 4 na mieste
jednotiek. Aspoň jedno číslo tak má cifru 4 na mieste
desiatok, navyše dve také čísla zrejme existovať nemôžu. Máme
teda práve jedno číslo začínajúce na 4 a~k~tomu aspoň
jedno číslo končiace na 4. Aj toto číslo je však jediné, lebo
$40+14+24>71$. Nutne tak máme (na tabuli) práve dve čísla s~cifrou~4,
konkrétne $4\van$ a~$\van4$, a~ich súčet je 71. Jedná sa určite o~čísla $47$
a~$24$.

Predpokladajme teraz, že je dobrá cifra 7. Keby sa vyskytovala
iba na miestach jednotiek, muselo by to tak byť v~aspoň~3
číslach, aby ich súčet končil cifrou 1, avšak $17+27+37>71$.
Dobrá cifra 7 sa tak niekde vyskytuje na mieste desiatok -- vtedy je
na tabuli zrejme jediné číslo s~cifrou 7, konkrétne číslo~71.

Z~posledných dvoch odsekov už vyplýva to, čo sme sľúbili ukázať:
cifry 4 a~7 nemôžu byť súčasne dobré -- na tabuli by museli
súčasne byť čísla 47, 24 a~71, takže cifra 7 by nebola dobrá.
Podľa úvodu z~časti~b) to znamená, že počet dobrých cifier
na tabuli je vždy nanajvýš~5 a~že kandidátmi na päticu súčasne dobrých cifier
sú iba pätice $(1,2,3,4,5)$ a~$(1,2,3,5,7)$. Zadanie úlohy
splníme, ak uvedieme {\it jeden\/} príklad čísel na tabuli s~piatimi
dobrými ciframi. Pre zaujímavosť uvedieme také príklady
\emph{štyri}, všetky pre prvú päticu $(1,2,3,4,5)$:
\def\black{\pdfliteral{0 g}}
$$
\def\d#1{{\pdfliteral{0 .651 .318 rg}\aftergroup\black#1}}
%\def\d#1{{\localcolor\Green#1}}
%\def\d#1{{\gr#1}}
\eqalign{
\{\d10, \d1\d1, \d15, \d16, \d19\}
&\cup \{\d20, \d24, \d27\}
\cup \{\d3\d3, \d38\}
\cup \{2\d4, \d47\}
\cup \{1\d5, \d56\}, \cr
\{\d10, \d1\d1, \d15, \d17, \d18\}
&\cup \{\d20, \d24, \d27\}
\cup \{\d3\d3, \d38\}
\cup \{2\d4, \d47\}
\cup \{1\d5, \d56\}, \cr
\{\d15, \d16, \d19, 2\d1\}
&\cup \{\d21, \d24, \d26\}
\cup \{\d3\d3, \d38\}
\cup \{2\d4, \d47\}
\cup \{1\d5, \d56\}, \cr
\{\d15, \d17, \d18, 2\d1\}
&\cup \{\d21, \d24, \d26\}
\cup \{\d3\d3, \d38\}
\cup \{2\d4, \d47\}
\cup \{1\d5, \d56\}.
}$$


\zaver
Najväčší možný počet súčasne dobrých cifier je rovný 5.

\poznamka
Vysvetlíme najskôr, prečo žiadny príklad čísel s~päticou dobrých
cifier $(1,2,3,5,7)$ neexistuje. V~druhej časti poznámky
potom aspoň naznačíme, ako sa dostať k~niektorému príkladu
s~päticou dobrých cifier $(1,2,3,4,5)$.
(Bez dôkazu ponecháme fakt, že v~závere
nášho riešenia sme vypísali všetky štyri možné príklady.)

Nie je ťažké dokázať, že cifra 5 je dobrá, iba keď vystupuje v~dvoch číslach, konkrétne 15 a~56. Ak je preto zároveň dobrá
aj cifra 7, sú na tabuli čísla 15, 56 a~71, teda cifra 1
nie je dobrá a~prvý vytýčený cieľ je splnený.

\def\gr#1{{\pdfliteral{0.647 0.1647 0.1647 rg}\aftergroup\black#1}}
%\def\gr{\localcolor\Brown}
Ku konštrukcii príkladu pre päticu $(1,2,3,4,5)$: Vieme už, že na
tabuli musia byť s~cifrou~4 alebo 5 práve čísla \gr{24}, \gr{47},
\gr{15} a~\gr{65}. Analýzou čísel s~cifrou~3 možno nájsť päť
do úvahy prichádzajúcich
možností $\{32, 39\},\{33,38\},\{34,37\},\penalty0
\{35,36\}, \{13, 23, 35\}$. Posledné tri z~nich hneď vylúčime
kvôli prítomnosti cifier~4 alebo~5. Prvá možnosť $\{32, 39\}$ vedie
k~problému s~cifrou~2 -- vyjadriť jej \uv{deficit} $71-24-32=15$ nie je možné.
Cifra~3 tak bude nutne zastúpená
práve v~číslach \gr{33} a~\gr{38}. Následne skúmaním zastúpenia
cifry~2 dôjdeme k dvom možnostiam: k~číslu 24 pridať $\{20,27\}$
alebo $\{21,26\}$. V~oboch prípadoch ostáva analyzovať cifru~1,
pričom pri pridaní $\{\gr{21},\gr{26}\}$ je to jednoduchšie --
hľadáme tam totiž vyjadrenie \uv{deficitu} $71-21-15=35$, pre ktorý
ľahko nájdeme obe vyhovujúce možnosti $\gr{16}+\gr{19}$
a~$\gr{17}+\gr{18}$.


\schemaABC
Za úplné riešenie dajte 6~bodov, z~toho 3 body za časť
a) a~3 body za časť b). V~neúplných riešeniach oceňte čiastkové
kroky nasledovne.
\item{A1.} Určenie všetkých šiestich cifier 1, 2, 3, 4, 5, 7, ktoré môžu byť
dobré, spolu s~uvedením vyhovujúcich príkladov -- 1~bod. Tento bod neudeľujte,
ak chýba čo i len jedna cifra alebo príklad pre ňu, alebo ak je uvedená
naopak niektorá cifra, ktorá nemôže byť dobrá.
\item{A2.} Určenie všetkých štyroch cifier 0, 6, 8, 9, ktoré nemôžu byť dobré,
podložené patričnými zdôvodneniami -- 2~body. Čiastočný 1 bod je možné udeliť,
ak je iba jedna zo štyroch cifier vynechaná.
\item{B1.} Dôkaz tvrdenia, že cifry 4 a~7 nemôžu byť súčasne dobré -- 1 bod.
(Iná dvojica cifier
z~množiny $\{1, 2, 3, 4, 5, 7\}$ túto negatívnu vlastnosť nemá.) Tento bod
možno získať aj za dôkaz tvrdenia, že cifry 1, 5, 7 nemôžu byť
súčasne dobré, alebo aj iného tvrdenia vedúceho k rovnakému
potrebnému záveru, že dobrých cifier zároveň nemôže byť viac ako päť.
\item{B2.} Nájdenie niektorého (zo štyroch možných) príkladov čísel s~piatimi dobrými
ciframi -- 2 body.
\endgraf\noindent
Celkovo potom dajte $\rm A1+A2+B1+B2$ bodov.
Ak žiak nepostupuje podľa uvedenej schémy, dosiahne však
relevantné zistenia, je možné udeliť až 2~body (ak je
napríklad zdôvodnené, ktoré z~cifier 4, 5, 6, 7, 8 a~9 môžu byť dobré
a~navyše aké čísla na tabuli s~každou z~týchto dobrých cifier musia byť).
Len za hypotézu, že najväčší možný počet dobrých čísel na
tabuli je 5, ale žiadny bod neudeľujte.
\endschema
}

{%%%%%   C-II-4
V~prvej časti riešenia ukážeme, že hodnota $s$ nikdy
neprevyšuje číslo $2$. Pre hodnotu $s=2$ potom v~druhej časti uvedieme
príklad vyhovujúcej tabuľky.

\smallskip
Majme teda ľubovoľnú tabuľku $10\times10$ vyplnenú podľa zadania úlohy
a~okrem čísla~$s$ uvažujme aj súčet $S$ všetkých čísel v~tejto tabuľke.
Keďže súčet čísel v~každom riadku až na jeden je rovný~0, tak hodnota~$S$
je rovná súčtu 10~čísel v~tomto jednom riadku, rovnému vždy nanajvýš $10$.
Platí teda $S\leq10$.

Na druhej strane, pri počítaní súčtu~$S$ po stĺpcoch našej tabuľky
dostaneme podľa zadania za deväť stĺpcov dokopy hodnotu $9s$.
K~nej ešte musíme pripočítať súčet 10 čísel v zvyšnom desiatom stĺpci,
čo je vždy najmenej ${-10}$. Tým pádom platí $S\geq 9s-10$.

Spojením nerovností $S\le10$ a~$S\ge9s-10$ dostávame
$10\geq S\geq9s-10$, odkiaľ $10 \ge 9s-10$, čiže $9s\leq20$.
Číslo $s$ je však celé, a~preto posledná nerovnosť už vedie
k~avizovanému odhadu $s\leqq2$.

\smallskip
Sľúbený príklad tabuľky, ktorá bude spĺňať zadanie úlohy pre
$s=2$, založíme na triku s~\uv{posúvaním diagonály}, známom zo
vzorových riešení úloh oboch predchádzajúcich súťažných kôl. Tmavo
sú vyfarbené políčka s~číslami ${-1}$, políčka s~číslami $1$ sú
biele (\obr). Iba v~desiatom riadku je súčet čísel rôzny od 0 (je rovný~10),
iba v~prvom stĺpci je súčet čísel rôzny od $2$ (je rovný ${-8}$).
\inspsc{c71ii_p41.eps}{.4167}%

\zaver
Najväčšia možná hodnota $s$ je rovná číslu 2.

\poznamka
Je zrejmé, že každý riadkový aj každý stĺpcový súčet vyplnenej
tabuľky ${10\times10}$ je číslo párne
(rovné totiž číslu $2j-10$, pričom $j$ je počet zastúpených jednotiek). Preto je
párnym číslom aj hľadaná maximálna hodnota~$s$.
Na tom možno založiť odvodenie odhadu
$s\leqq2$ cestou, keď postupne prechádzame najväčšie párne hodnoty
$s=10$, 8, 6,~4 a~úvahami o~počte jednotiek v~celej tabuľke
vždy ľahko prichádzame ku sporu. Ukážeme, ako možno celú diskusiu
o~možných počtoch jednotiek spraviť úspornejšie.

Informácia o~riadkových súčtoch nám hovorí, že v~niektorých deviatich
riadkoch je po práve 5~jednotkách. Keďže vo zvyšnom desiatom riadku
ich je nanajvýš~10, v~celej tabuľke je tak
nanajvýš $9\cdot5+10=55$ jednotiek. Keby však platilo $s\geqq3$, potom by v~každom z~deviatich stĺpcov so súčtom~$s$ muselo byť aspoň 7 jednotiek -- dokopy by ich bolo aspoň $7\cdot9=63$,
a~to je spor. Nutne tak platí $s\leqq2$.


\schemaABC
Za úplné riešenie dajte 6~bodov, z~toho 3~body za dôkaz
odhadu $s\leq2$ a~3~body za konštrukciu tabuľky pre hodnotu $s=2$.
V~prípade čiastočných riešení dajte:
\item{$\bullet$} 1 bod za vylúčenie {\it všetkých\/} nepárnych $s\in\{3,5,7,9\}$, napríklad všeobecnejším konštatovaním, že každé možné $s$ je nutne párne (to možno považovať za zrejmé, \tj. uviesť bez vysvetlenia).
\item{$\bullet$} 2 body za vylúčenie {\it všetkých\/} párnych $s\in\{4,6,8,10\}$, z~toho 1 bod za vylúčenie {\it oboch\/} hodnôt $s\in\{8,10\}$ a~1 bod za vylúčenie {\it oboch\/} hodnôt $s\in\{4,6\}$.
\item{$\bullet$}
1 bod v~prípade iba uhádnutia odpovede $s=2$ (bez príkladu
tabuľky), ak ale celkový zisk týmto neprekročí 3 body.
\endgraf\noindent
Za triviálnu nerovnosť $s\leqq10$ žiadny bod neudeľujte. Preto
aj hodnoty $s>10$ v~položkách pre bodovanie vôbec nezmieňujeme.
\endschema
}

{%%%%%   vyberko, den 1, priklad 1
...}

{%%%%%   vyberko, den 1, priklad 2
...}

{%%%%%   vyberko, den 1, priklad 3
...}

{%%%%%   vyberko, den 1, priklad 4
...}

{%%%%%   vyberko, den 2, priklad 1
...}

{%%%%%   vyberko, den 2, priklad 2
...}

{%%%%%   vyberko, den 2, priklad 3
...}

{%%%%%   vyberko, den 2, priklad 4
...}

{%%%%%   vyberko, den 3, priklad 1
...}

{%%%%%   vyberko, den 3, priklad 2
...}

{%%%%%   vyberko, den 3, priklad 3
...}

{%%%%%   vyberko, den 3, priklad 4
...}

{%%%%%   vyberko, den 4, priklad 1
...}

{%%%%%   vyberko, den 4, priklad 2
...}

{%%%%%   vyberko, den 4, priklad 3
...}

{%%%%%   vyberko, den 4, priklad 4
...}

{%%%%%   vyberko, den 5, priklad 1
...}

{%%%%%   vyberko, den 5, priklad 2
...}

{%%%%%   vyberko, den 5, priklad 3
...}

{%%%%%   vyberko, den 5, priklad 4
...}

{%%%%%   trojstretnutie, priklad 1
...}

{%%%%%   trojstretnutie, priklad 2
...}

{%%%%%   trojstretnutie, priklad 3
...}

{%%%%%   trojstretnutie, priklad 4
...}

{%%%%%   trojstretnutie, priklad 5
...}

{%%%%%   trojstretnutie, priklad 6
...}

{%%%%%   IMO, priklad 1
...}

{%%%%%   IMO, priklad 2
...}

{%%%%%   IMO, priklad 3
...}

{%%%%%   IMO, priklad 4
...}

{%%%%%   IMO, priklad 5
...}

{%%%%%   IMO, priklad 6
...}

{%%%%%   MEMO, priklad 1
...}

{%%%%%   MEMO, priklad 2
...}

{%%%%%   MEMO, priklad 3
...}

{%%%%%   MEMO, priklad 4
...}

{%%%%%   MEMO, priklad t1
...}

{%%%%%   MEMO, priklad t2
...}

{%%%%%   MEMO, priklad t3
...}

{%%%%%   MEMO, priklad t4
...}

{%%%%%   MEMO, priklad t5
...}

{%%%%%   MEMO, priklad t6
...}

{%%%%%   MEMO, priklad t7
...}

{%%%%%   MEMO, priklad t8
...} 