{%%%%%   A-I-1
Vzhľadom na to, že rastúcu aritmetickú postupnosť tvoria štyri navzájom
rôzne reálne čísla, musí mať prvá z~daných rovníc štyri rôzne reálne korene. Preto $a\ne 0$.

Označme $x_0$ spoločný koreň oboch rovníc. Potom je $x_0$ tiež koreňom
rovnice, ktorá vznikne odčítaním druhej z~daných rovníc od prvej, teda rovnice
$ax^4-ax=0.$ Tú ďalej upravíme na tvar $ax(x^3-1)=0.$
Pre spoločný reálny koreň~$x_0$ oboch daných rovníc odtiaľ vyplýva $x_0=0$ alebo
$x_0=1$.

Dosadením $x_0=0$ do prvej z~daných rovníc dostaneme $a=1$, takže táto rovnica má
tvar $x^4+bx^2=0.$
Táto rovnica ale pre žiadne reálne číslo~$b$ nemá štyri rôzne reálne korene
(číslo~0 je jej aspoň dvojnásobným koreňom), preto $x_0\ne0$.

Jediným spoločným koreňom oboch rovníc je teda $x_0=1$. Dosadením tejto
hodnoty do ktorejkoľvek z~oboch daných rovníc dostaneme $b=1-2a$. Prvú rovnicu
potom môžeme zapísať v tvare $ax^4+(1-2a)x^2+a-1=0$, z~ktorého vidíme, že má
i~koreň~$\m1$, a~po vyňatí súčinu koreňových činiteľov $(x-1)(x+1)$
dostaneme rovnicu
$$
(x-1)(x+1)(ax^2-a+1)=0.     \tag1
$$

Kvadratický dvojčlen $ax^2-(a-1)$  má mať dva rôzne korene, ktorými musia
byť dve navzájom opačné (nenulové) čísla $\xi$ a~$-\xi$. To je splnené práve vtedy, keď
$(a-1)/a>0$, čiže práve vtedy, keď $a>1$ alebo $a<0$.
Ak zvolíme označenie tak, že $\xi>0$, dostávame pre aritmetickú postupnosť
všetkých štyroch koreňov dve možnosti podľa toho, či je $0<\xi<1$ alebo $\xi>1$.


V~prvom prípade tvoria štyri korene rovnice~(1) aritmetickú postupnosť $-1$, ${-\xi}$, $\xi$,~$1$,
ktorá má zrejme diferenciu~$\frac23$, preto $\xi=1-\frac23=\frac13$.
Toto číslo~$\xi$ je koreňom rovnice~(1) práve vtedy, keď $a=1/(1-\xi^2)=\frac98$.
Potom $b=1-2a=-\frac54$.

V~druhom prípade tvoria štyri korene rovnice~(1) aritmetickú postupnosť ${-\xi}$, $-1$, $1$, $\xi$
s~diferenciou~2, preto $\xi=1+2=3$.
Číslo~3 je koreňom rovnice~(1) práve vtedy, keď $a=1/(1-3^2)=-\frac18$.
Potom $b=1-2a=\frac54$.

\zaver
Úlohe vyhovujú práve dve dvojice reálnych čísel $(a,b)$, a~to
$$
(a,b)\in\left\{\left(-\frac18,\frac54\right),\left(\frac98,-\frac54\right)\right\}.
$$

\návody
Určte všetky hodnoty reálnych parametrov $p$ a~$q$, pre ktoré má každá
z~rovníc
$$
x(x-p)=3+q, \qquad x(x+p)=3-q
$$
v~obore reálnych čísel dva rôzne korene, ktorých aritmetický priemer je
jedným z~koreňov zvyšnej rovnice.
\vpravo{[59--B--S--1]}

Určte všetky dvojice $(a,b)$ reálnych čísel, pre ktoré má každá
z~kvadratických rovníc
$$
ax^2+2bx+1=0, \qquad bx^2+2ax+1=0
$$
dva rôzne reálne korene, pričom práve jeden z~nich je pre obe rovnice
spoločný.
\vpravo{[57--B--I--5]}

Určte všetky dvojice $(a,b)$ reálnych čísel, pre ktoré majú rovnice
$$
x^2+(3a+b)x+4a=0, \qquad x^2+(3b+a)x+4b=0
$$
spoločný reálny koreň.
\vpravo{[57--B--S--2]}

\D
Určte všetky trojčlenné aritmetické postupnosti prvočísel
s~diferenciou~1970.
\vpravo{[20--B--P--2]}

Uvažujme dve kvadratické rovnice
$$
x^2-ax-b=0, \qquad x^2-bx-a=0
$$
s~reálnymi parametrami $a$, $b$.
Zistite, akú najmenšiu a~akú najväčšiu hodnotu môže nadobudnúť súčet $a+b$,
ak existuje práve jedno reálne číslo~$x$, ktoré súčasne vyhovuje obom
rovniciam. Určte ďalej všetky dvojice $(a,b)$ reálnych parametrov, pre ktoré
uvažovaný súčet tieto hodnoty nadobúda.
\vpravo{[57--B--II--1]}

Reálne čísla $a$, $b$ majú nasledovnú vlastnosť: kvadratická rovnica
$x^2-ax+b-1=0$ má v~množine reálnych čísel dva rôzne korene, ktorých
  rozdiel je kladným koreňom rovnice $x^2-ax+b+1=0$.
\item{a)} Dokážte nerovnosť $b>3$.
\item{b)} Vyjadrite korene oboch rovníc pomocou $b$.\endgraf
\vpravo{[59--B--I--6]}
\endnávod
}

{%%%%%   A-I-2
Ukážeme, že pre nesúdeliteľné prirodzené čísla $r$ a~$s$, kde $r>2$ a~$s>2$,
existuje prirodzené číslo~$n$ s~vlastnosťou
$$
r\mid n-1\qquad\hbox{a}\qquad s\mid n+1.
$$
Pre také číslo $n$ a~číslo $k=rs$ nie je Adamova úvaha
správna, pretože z~predpokladu, že číslo~$k$ delí číslo $(n-1)(n+1)$, nevyplýva, že
$k$ delí $n-1$ ani že $k$ delí $n+1$. Keby totiž $k=rs$ delilo napríklad $n-1$,
delilo by číslo~$s$ obidve čísla $n+1$ i~$n-1$, čo vzhľadom na rovnosť $(n+1)-(n-1)=2$
nie je možné, lebo $s>2$.

Existenciu čísla $n$ z~prvej vety riešenia dokážeme tak, že zoberieme $s$~čísel
$$
2,\ r+2,\ 2r+2,\ \dots,\ (s-1)r+2.
$$
Tie dávajú pri delení číslom $s$ navzájom rôzne zvyšky. Keby totiž
niektoré dve z~nich, povedzme $ir+2$ a~$jr+2$ ($0\le i<j\le s-1$),
dávali pri delení číslom~$s$ rovnaký zvyšok, potom by číslo $s$ delilo
aj ich rozdiel $(i-j)r$, a~vzhľadom na nesúdeliteľnosť čísel $r$ a~$s$
aj rozdiel $i-j$, čo nie je možné, pretože $|i-j|<s$. Uvedených $s$~čísel teda
dáva úplnú sústavu zvyškov modulo~$s$, preto medzi nimi existuje
číslo, ktoré pri delení číslom~$s$ dáva zvyšok~0; nech je to číslo $lr+2$.
Potom ale pre číslo $n=lr+1$ platí, že $r$ delí $n-1$ a~$s$ delí $n+1$.


Uvedomme si, že každé
číslo~$k$ deliteľné dvoma nepárnymi prvočíslami sa dá zapísať ako súčin
dvoch nesúdeliteľných čísel väčších ako~2. Adamova úvaha môže byť teda správna iba
pre tie čísla~$k$, ktoré sú deliteľné nanajvýš jedným nepárnym prvočíslom.
To znamená, že číslo~$k$ má jeden z~nasledovných troch tvarov:
$$
k=2^s,\qquad k=p^t,\qquad k=2p^t,
$$
kde $p$ je nepárne prvočíslo, $s$ celé nezáporné a~$t$ prirodzené číslo.


Nech $k=2^s$, kde $s$~je celé nezáporné číslo. Pre $s=0$ nie je Adamova úvaha
správna, pretože číslo $k=2^0=1$ delí každé prirodzené číslo, teda delí obidve
čísla $n-1$ i~$n+1$. Ani pre $s=1$ nie je Adamova úvaha správna, pretože
pokiaľ $k=2^1=2$ delí číslo $(n-1)(n+1)$, je jeden z~činiteľov párny, ale
potom je párny i~druhý činiteľ. Pre číslo $s=2$, teda pre
$k=2^2=4$, je Adamova úvaha správna. Ak totiž $4$ delí číslo $(n-1)(n+1)$,
je aspoň jeden z~oboch činiteľov párny, takže ide o~dve po sebe idúce {\it párne\/}
čísla, z~ktorých práve jedno je deliteľné štyrmi.
Nakoniec, pre žiadne $s\ge3$ Adamova úvaha správna nie je, stačí vziať
číslo $n=2^{s-1}-1$.

Nech $k=p^t$, kde $p$ je nepárne prvočíslo a~$t$ prirodzené číslo. Potom je
Adamova úvaha správna, lebo obe čísla $n-1$ a~$n+1$ nemôžu byť súčasne
deliteľné tým istým nepárnym prvočíslom~$p$, a~preto je práve jedno z~nich deliteľné
číslom $p^t=k$.

Nech $k=2p^t$, kde $p$ je nepárne prvočíslo a~$t$ prirodzené číslo. Potom je
Adamova úvaha tiež správna: obe čísla $n-1$ a~$n+1$ sú nutne párne a~pritom nemôžu byť
súčasne deliteľné tým istým nepárnym prvočíslom~$p$, preto je práve jedno z~nich deliteľné
číslom $2p^t=k$.

\zaver
Adamova úvaha je správna pre každé prirodzené číslo $n$ len
pre prirodzené čísla $k$ jedného z~tvarov
$$
k=4,\qquad k=p^t,\qquad k=2p^t,
$$
kde $p$ je nepárne prvočíslo a~$t$ prirodzené číslo.

\návody
Ak je číslo $(n -1)(n + 1)$ deliteľné štyrmi, je práve jeden z~činiteľov $n-1$, $n+1$
deliteľný štyrmi. Dokážte.

Zistite, pre ktoré dvojciferné čísla $n$ je $n^3-n$ deliteľné číslom 100.
\vpravo{[50--C--S--3]}

Nájdite všetky trojciferné čísla $n$ také, že posledné trojčíslie čísla
$n^2$ je zhodné s~číslom~$n$.
\vpravo{[50--C--I--1]}

Koľko existuje prirodzených čísel  $x\leq 1\,992\,000$
takých, že číslo $1\,992\,000$ delí číslo $x^3-x$?
\vpravo{[41--B--I--6]}
\endnávod
}

{%%%%%   A-I-3
\epsplace a60.1 \hfil\Obr\par
\epsplace a60.2 \hfil\Obr\par
Z~rovnosti obvodového a~úsekového uhla prislúchajúceho k~tetive~$AK$ kružnice~$k$ vyplýva (\obr)
\hbox {$|\angle KNA|=|\angle LKA|$} a~podobne z~rovnosti obvodového a~úsekového uhla prislúchajúceho
k~tetive~$AL$ kružnice~$l$ vyplýva $|\angle VLM|=|\angle LAM|$, pričom sme označili $V$
nejaký bod polpriamky opačnej k~polpriamke~$LK$.

\inspicture

Štvoruholník $KLMN$ je tetivový práve vtedy, keď platí $|\angle KNA|=|\angle VLM|$ čiže
$|\angle LKA|=|\angle LAM|$. Posledná rovnosť ale platí
práve vtedy, keď je $LAM$ úsekovým uhlom prislúchajúcim k~obvodovému uhlu~$LKA$ tetivy~$LA$
kružnice opísanej trojuholníku $AKL$, teda práve vtedy, keď je priamka $MN$ dotyčnicou tejto kružnice.

Tým je tvrdenie úlohy dokázané.

\ineriesenie
Vyriešme úlohu najskôr za predpokladu, že priamky $KL$ a~$MN$ sú rovnobežné.
V~takom prípade sú zrejme oba trojuholníky $ANK$ a~$MAL$ rovnoramenné, pretože osi strán
$AN$, resp.~$MA$ prechádzajú príslušným vrcholom~$K$, resp.~$L$ (čiže bodom dotyku
dotyčnice rovnobežnej s~tetivou $AN$, resp.~$MA$ kružnice~$k$, resp.~$l$).
Teda $|LA|=|LM|$ a~$|KN|=|KA|$.
Pritom štvoruholník $KLMN$ je tetivový práve vtedy, keď je to rovnoramenný
lichobežník, \tj. $|LM|=|KN|$.
To podľa predchádzajúcej dvojice rovností nastane práve vtedy, keď je trojuholník $KLA$
rovnoramenný, čiže
práve vtedy, keď $MN$ je dotyčnicou
kružnice opísanej tomuto trojuholníku vedenou vrcholom~$A$.
(Vzhľadom na to, že potom sú trojuholníky $ANK$ a~$MAL$ zhodné, uvedená situácia nastane
práve vtedy, keď sú kružnice $k$ a~$l$ zhodné.)

Predpokladajme ďalej, že
priamky $MN$ a~$KL$ sú rôznobežné, a~označme $V$ ich priesečník (\obr).
Použitím mocnosti bodu~$V$ ku kružniciam $k$ a~$l$ dostaneme
$$
|VK|^2=|VA|\cdot |VN| \qquad \text{a} \qquad  |VL|^2=|VM|\cdot |VA|.
$$

\inspicture

Vynásobením oboch vzťahov dostaneme
$$
|VK|^2\cdot |VL|^2=|VN|\cdot |VA|^2 \cdot |VM|. \tag1
$$
Štvoruholník $KLMN$ je ale tetivový práve vtedy, keď platí
\niedorocenky{(pozri návodnú úlohu~N1)}
$$
|VK|\cdot |VL|=|VN|\cdot |VM|
$$
čiže~-- s~prihliadnutím na (1)~-- práve vtedy, keď platí
$$
|VK|\cdot |VL|=|VA|^2.
$$
Posledná rovnosť ale platí práve vtedy, keď priamka $MN$ (prechádzajúca bodom~$A$)
je dotyčnicou kružnice opísanej trojuholníku~$AKL$.
Tým je tvrdenie úlohy dokázané.

\návody
{\everypar{}%
Zopakujte si najskôr učebnicové poznatky o~obvodovom, stredovom
a~úsekovom uhle i~s~ich dôkazmi. Pripomeňte si
taktiež vlastnosť všetkých sečníc danej kružnice idúcich daných bodom,
ktorá je vyjadrená mocnosťou bodu ku kružnici.
\par}
Priamky $KL$ a~$MN$ sa pretínajú v~bode $V$, ktorý leží buď vnútri, alebo
mimo oboch úsečiek $KL$ a~$MN$. Dokážte, že body $K$, $L$, $M$, $N$ ležia na jednej
kružnici práve vtedy, keď platí ${|VK|\cdot|VL|}=|VM|\cdot|VN|$. [Uvážte mocnosť
bodu~$V$ ku kružnici opísanej trojuholníku $KLM$ a~posúďte, kedy druhý priesečník~$N'$
tejto kružnice s~priamkou $VM$ splýva s~bodom~$N$.]

V~rovine je daná priamka $p$ a~body $A,B$ ($A\ne B$), ktoré ležia v~tej istej
polrovine vyťatej priamkou $p$. Zostrojte kružnicu,
ktorá prechádza bodmi $A$, $B$ a~dotýka sa priamky $p$. [Uvažujte priamku~$q$, na
ktorej ležia body $A$, $B$, a~využite mocnosť jej priesečníka s~$p$ k~hľadanej kružnici.]

V~rovine je daný pravouhlý lichobežník $ABCD$ s~dlhšou základňou~$AB$
a~pravým uhlom pri vrchole~$A$. Kružnica~$k_1$ zostrojená nad
stranou~$AD$ ako priemerom a~kružnica~$k_2$, ktorá prechádza vrcholmi $B$
a~$C$ a~dotýka sa priamky~$AB$, majú vonkajší dotyk v~bode $P$. Dokážte,
že uhly $CPD$ a~$ABC$ sú zhodné.
\vpravo{[52--B--I--5]}

V~rovine je daný pravouhlý lichobežník $ABCD$ s~dlhšou základňou~$AB$
a~pravým uhlom pri vrchole~$A$. Označme $k_1$ kružnicu zostrojenú
nad stranou~$AD$ ako nad priemerom a~$k_2$ kružnicu prechádzajúcu vrcholmi
$B$, $C$ a~dotýkajúcu sa priamky~$AB$. Ak majú kružnice $k_1$, $k_2$ vonkajší
dotyk v~bode $P$, je priamka~$BC$ dotyčnicou kružnice opísanej trojuholníku $CDP$.
Dokážte.
\vpravo{[52--B--II--4]}

\D
Nech $L$ je ľubovoľný vnútorný bod kratšieho oblúka~$CD$ kružnice opísanej
štvorcu $ABCD$. Označme $K$ priesečník priamok $AL$ a~$CD$,
$M$ priesečník priamok $AD$ a~$CL$ a~$N$ priesečník priamok $MK$ a~$BC$. Dokážte,
že body $B$, $L$, $M$, $N$ ležia na jednej kružnici.
\vpravo{[53--A--III--5]}
\endnávod
}

{%%%%%   A-I-4
Žiadne tri žetóny tej istej farby neležia na jednej kôpke, teda na každej
kôpke leží aspoň jeden žetón zvolenej farby.
Každé vyhovujúce rozdelenie žetónov na kôpky je potom charakterizované tým,
na ktorej z~nich leží práve jeden z~troch žetónov
jednotlivých farieb.
%% libovolně zvolené barvy.

Predpokladajme, že v~jednej z~kôpok je práve $l$~farieb zastúpených jedným
žetónom a~zvyšných $2n-l$ farieb dvoma. Jednoduchým výpočtom $l+2(2n-l)=3n$
zistíme, že to je možné len pri $l=n$.
Preto je skúmaný počet~$p_n$ rovný počtu rozdelení $2n$ žetónov navzájom rôznych farieb
na dve (neusporiadané) skupiny po $n$~žetónoch, teda
$$
p_n=\frac12\binom{2n}{n}=\frac{(2n)!}{2(n!)^2}=
\frac{2n\cdot(2n-1)!}{2n\cdot(n-1)!\,n!}=\binom{2n-1}{n}.  \tag1
$$

Našou úlohou je dokázať, že posledné kombinačné
číslo je nepárne práve vtedy, keď je číslo~$n$ mocnina dvoch.
Tento poznatok (a~vlastne i~metódu jeho dôkazu)
je možné vypozorovať z~dobre známej schémy všetkých
kombinačných čísel v~podobe Pascalovho trojuholníka:
$$
\hbox{\epsfbox{pascal.1}}
$$
V~našej schéme nie sú samotné kombinačné čísla,
ale ich zvyšky $0$ alebo $1$ po delení dvoma. Pre ich určenie
nie je nutné kombinačné čísla vôbec počítať, pretože
z~rekurentných vzorcov
$$
\binom{n}{0}=\binom{n}{n}=1\quad\text{a}\quad
\binom{n}{i}=\binom{n-1}{i-1}+\binom{n-1}{i}\quad (1\le i\le n-1)
\tag2
$$
môžeme postupne po jednotlivých riadkoch namiesto kombinačných
čísel priamo písať ich zvyšky pri delení akýmkoľvek
pevným číslom, v~našom prípade číslom~2.

Všimnime si, čo naša schéma napovedá. Niektoré riadky
(vyznačené obdĺžničkami) sú zostavené zo samých jednotiek.
Vďaka rekurentným vzorcom~(2) pod každým takým riadkom zrejme
vznikne trojuholník zostavený zo samých núl (tri také trojuholníky sú
vyznačené sivou farbou) olemovaný zľava aj sprava samými
jednotkami; bezprostredne pod ním
opäť leží riadok zo samých jednotiek. Pretože zvyšky všetkých skúmaných
čísel $\binom{2n-1}{n}$ (v~našej schéme vyznačených krúžkami)
ležia v~opísaných obdĺžničkoch alebo
trojuholníkoch, bude také kombinačné číslo nepárne práve vtedy, keď bude
mať pozíciu v~niektorom obdĺžničku.

Naše pozorovanie teraz opíšeme presnejšie a~priamo ho overíme matematickou
indukciou.

{\it Riadky zo samých jednotiek sú práve riadky s~kombinačnými
číslami $\binom {n-1}i$ $(0\le i\le {n-1})$, kde $n$ je tvaru $n=2^k$.}\par
Tvrdenie triviálne platí pre $k=1$.
Predpokladajme teda, že platí pre nejaké $k\ge1$, a~označme $P_n$
prvých $n=2^k$~riadkov
schémy.
Ďalších $n$~riadkov si môžeme
predstaviť ako tri rovnostranné trojuholníky čísel: prvý a~tretí s~$n$~riadkami
sú rovnakej veľkosti ako~$P_n$,
medzi nimi je potom $(n-1)$-riadkový trojuholník (vrcholom nadol), ktorý je vďaka
jednotkám v~základni trojuholníka~$P_n$ a~rekurentným vzorcom~(2) zostavený zo samých núl.
Preto majú prvý a tretí trojuholník jednotky nielen v~horných vrcholoch
a~na stranách ležiacich na hranici celej schémy, ale i~na stranách,
ktorými priliehajú k~druhému trojuholníku, teda na začiatku i na konci každého zo
svojich $n$~riadkov. Vyplýva to opäť zo vzorcov~(2), ktoré potom vedú
k~ďalšiemu, pre nás hlavnému záveru: Prvý a tretí trojuholník sú totožné s trojuholníkom
$P_n$. Môžeme teda zhrnúť, že každý
z~$n-1$ pridaných riadkov obsahuje aspoň jednu nulu, ale $n$-tý riadok
(zložený z dvoch $n$-tých riadkov trojuholníka $P_n$) obsahuje samé jednotky. Tvrdenie
teda platí i~pre $2n=2\cdot 2^k=2^{k+1}$ riadkov Pascalovho trojuholníka modulo~2,
\tj. i~pre číslo $k+1$.

Vzhľadom na to, že skúmané číslo $p_n=\binom{2n-1}n=\binom{2n-1}{n-1}$ leží vždy v strede
párneho riadku Pascalovho trojuholníka, je zrejmé, že leží buď v~niektorom obdĺžničku,
alebo v~niektorom sivom trojuholníku, ktoré sa postupne striedajú.
Číslo $p_n$ je teda naozaj nepárne práve vtedy, keď $n$ je mocnina dvoch.

\ineriesenie
Počet $p_n$ opísaných rozdelení žetónov určíme rovnako ako
v~prvom riešení vzorcom
$$
p_n=\frac{(2n)!}{2(n!)^2},
$$
ktorý ďalej upravíme na tvar
$$
\align
p_n=&1\cdot3\cdot\dots\cdot(2n-1)\cdot{2\cdot4\cdot\dots\cdot(2n-2)(2n)\over2(n!)^2}
=1\cdot3\cdot\dots\cdot(2n-1)\cdot{2^nn!\over2(n!)^2}=\\
=&1\cdot3\cdot\dots\cdot(2n-1)\cdot{2^{n-1}\over n!}.
\tag3
\endalign
$$

Pre najvyššiu mocninu $2^a$, ktorá delí $n!$, platí \niedorocenky{(pozri návodnú úlohu N1)}
$$
a=\Bigl\lfloor \frac{n}{2}\Bigr\rfloor
 +\Bigl\lfloor \frac{n}{2^2}\Bigr\rfloor+\dots
 +\Bigl\lfloor \frac{n}{2^m}\Bigr\rfloor,
$$
kde $2^m\le n<2^{m+1}$
a~$\lfloor x\rfloor$ označuje {\it dolnú celú časť čísla $x$}, teda najväčšie
celé číslo, ktoré nie je väčšie ako $x$.
Odtiaľ pre exponent $a$ vyplýva odhad
$$
a\le\frac{n}{2}
 + \frac{n}{2^2}+\dots
 + \frac{n}{2^m}=n\Bigl(1-{1\over2^m}\Bigr)=n-{n\over2^m}\le n-1.
$$

Z~vyjadrenia (3) teda vidíme, že číslo $p_n$ je nepárne práve vtedy, keď $a=n-1$ čiže
$n$ je tvaru $2^m$.

\návody
Odvoďte tzv. Legendreovu formulu: Najvyššia mocnina prvočísla $p$, ktorá delí
$n!$, má stupeň rovný súčtu
$$
\Bigl\lfloor\frac{n}{p}\Bigr\rfloor+
\Bigl\lfloor\frac{n}{p^2}\Bigr\rfloor+
\Bigl\lfloor\frac{n}{p^3}\Bigr\rfloor+
\dots %\eqno\hbox{[34--A--I--2]}
$$
[Uvážte, že prvý sčítanec je rovný počtu tých čísel od 1 do $n$, ktoré sú
deliteľné číslom~$p$, druhý sčítanec je rovný počtu takých čísel, ktoré
sú deliteľné nielen~$p$, ale i~$p^2$ atď.
Do daného súčtu preto každé z~čísel od~1 do~$n$ prispeje práve toľkokrát,
koľkokrát je prvočíslo~$p$ zastúpené v~jeho rozklade na prvočinitele.]

Zistite, koľkými nulami končí zápis čísla $2\,010!$. [501]

\D
O~tom, či zadané kombinačné číslo $\binom{n}{k}$, kde $0\le
k\le n$, je párne alebo nepárne, sa dá jednoducho rozhodnúť
pomocou číslic $c_i,d_i$ z~binárnych zápisov parametrov $n$ a~$k$:
$$
n=c_0+c_1 2^1+c_2 2^2+c_3 2^3+\dots,\quad
k=d_0+d_1 2^1+d_2 2^2+d_3 2^3+\dots
$$
 Pomocou Legendreovej formuly dokážte nasledovné kritérium:
číslo $\binom{n}{k}$ je nepárne práve vtedy, keď pre každý index
$i\ge0$ platí $c_i\ge d_i$. Potom odvoďte jeho dva dôsledky:
\item{a)} Všetky kombinačné čísla $\binom{n}{k}$ s~daným $n$
a~$k\in\{0,\dots,n\}$ sú nepárne práve vtedy, keď je číslo $n+1$
mocninou čísla~2.
\item{b)} Počet tých kombinačných čísel $\binom{n}{k}$ s~daným $n$
a~$k\in\{0,\dots,n\}$, ktoré sú nepárne, je mocninou
čísla 2 pre každé $n$.
\endgraf
[Podľa Legendreovej formuly je číslo $\binom{n}{k}$ nepárne práve vtedy, keď
vo všetkých všeobecne platných nerovnostiach
$$
\left\lfloor{\frac{n}{2^i}}\right\rfloor-\left\lfloor{\frac{k}{2^i}}\right\rfloor-\left\lfloor{\frac{n-k}{2^i}}\right\rfloor\ge0
\quad(i\in\ssize\Bbb N)
$$
nastane rovnosť. Po dosadení binárnych zápisov do týchto rovností
dostaneme po úprave ekvivalentnú sústavu
$$
\left\lfloor{\frac{(c_0-d_0)+(c_1-d_1)2+\dots+(c_{i-1}-d_{i-1})2^{i-1}}
{2^i}}\right\rfloor=0\quad(i\in\ssize\Bbb N).
$$
Tá je zrejme splnená práve vtedy, keď je súčet
$$
(c_0-d_0)+(c_1-d_1)2+\dots+(c_{i}-d_{i})2^{i}
$$
nezáporný pre každé $i\ge0$, teda práve vtedy, keď sa celý výpočet
rozdielu $n-k$ v~dvojkovej sústave odohrá v~jednotlivých rádoch
samostatne. Ďalej uvážte, že takto prebehne
odčítanie $n-k$ pre každé $k$ práve vtedy, keď je číslo $n$ zapísané
skupinou jednotiek. Pri všeobecnom $n$ je počet rozdielov $n-k$
s~takým priebehom výpočtu zrejme rovný $2^j$, kde $j$ je počet
jednotiek v~zápise čísla~$n$.]
\endnávod
}

{%%%%%   A-I-5
V~každom kroku sa súčet všetkých čísel na stenách kocky zväčší o~$2$, jeho
parita sa teda nezmení. Ak sú na všetkých stenách kocky rovnaké
čísla, je ich súčet násobkom šiestich, a~je teda deliteľný dvoma. Nutnou
podmienkou pre to, aby sme tento stav dosiahli, teda je, aby i~na začiatku
bol súčet všetkých čísel na stenách kocky deliteľný dvoma.

Táto podmienka je súčasne aj postačujúca. Predpokladajme, že súčet
všetkých šiestich celých čísel na stenách kocky je na začiatku deliteľný dvoma.
Ukážeme, ako po určitom počte krokov dosiahneme, aby na všetkých stenách
kocky boli rovnaké čísla.

Označme steny kocky $S_1,S_2,\dots,S_6$,
pričom stena $S_1$ je oproti stene $S_6$, stena~$S_2$ oproti $S_5$ a~$S_3$
oproti~$S_4$. (Podobne sú očíslované i~steny bežnej hracej kocky:
súčet bodov na protiľahlých stenách dáva~7.)
Krok, v~ktorom zväčšíme čísla na stenách $S_i$, $S_j$, označíme~$k_{ij}$.
A~pretože nás zaujíma len relatívna hodnota očíslovania stien, \tj. či a~o~koľko
sa líšia od najmenšej hodnoty zo všetkých šiestich čísel, budeme ďalej pracovať len s~týmito
relatívnymi hodnotami (čo budú nezáporné celé čísla s~najmenšou hodnotou~0).

Postupnosťou krokov $k_{12}$, $k_{23}$, $k_{35}$, $k_{54}$,
$k_{41}$ zabezpečíme, že sa číslo na každej stene okrem steny~$S_6$
zväčší o~$2$, čo vzhľadom na náš dohovor vlastne znamená, že sme (relatívnu)
hodnotu čísla na stene~$S_6$ o~2 zmenšili.
Podobným spôsobom môžeme o~$2$ "zmenšiť" číslo na ľubovoľnej stene kocky.
Je teda zrejmé, že opísaným spôsobom dosiahneme, že (relatívne) hodnoty
čísel na stenách budú len 0 alebo~1; nula medzi nimi musí byť aspoň jedna (podľa
významu relatívnych hodnôt).
Teraz už stačí vyšetriť nasledovné možnosti (pripomeňme, že súčet všetkých
šiestich čísel je párny):
\item{a)} Na stenách kocky sú samé nuly; tvrdenie potom platí triviálne.
\item{b)} Na stenách kocky sú práve dve 1 (na zvyšných stenách 0). Bez ohľadu na to,
či sú obe jednotky na susedných alebo protiľahlých stenách, vždy môžeme
rozdeliť štyri steny s~nulami na dve dvojice susedných stien
a~v~dvoch krokoch zväčšiť ich čísla o~1.
\item{c)} Na stenách kocky sú práve štyri 1 (na zvyšných dvoch stenách sú~0).
Tento prípad vyriešime tak, že najprv znížime (spôsobom opísaným vyššie) hodnotu každej
steny s~jednotkou o~dve, čím (v~relatívnych hodnotách) dostaneme presne
situáciu opísanú v~b).

\zaver
Dosiahnuť, aby po konečnom počte krokov boli
na všetkých stenách kocky napísané rovnaké čísla, je možné
práve vtedy, keď je súčet čísel na všetkých šiestich stenách kocky
deliteľný dvoma.

\poznamka
Časť c) predchádzajúceho riešenia je možné vyriešiť i~takto:
Ak sú obe 0 na susedných stenách, môžeme ich jediným krokom zväčšiť na~1.
Ak sú obe~0 na protiľahlých stenách (bez ujmy
na všeobecnosti nech sú to napríklad $S_1$ a~$S_6$), pomocou krokov $k_{12}$, $k_{36}$, $k_{15}$, $k_{46}$
dosiahneme, že na každej stene kocky bude napísané číslo~2.

\návody
Na každej stene kocky je napísané práve jedno číslo, pričom všetky čísla
nie sú rovnaké. V~jednom kroku čísla na každej stene kocky nahradíme
aritmetickým priemerom čísel na všetkých štyroch susedných stenách.
Rozhodnite, či po niekoľkých krokoch môžu byť na~stenách kocky opäť pôvodné
čísla.
[Nie. Označme $M$ najväčšie z~čísel. Ak $M$ po prvom kroku zo stien
zmizne, už sa na nich nikdy neobjaví. Ak nezmizne, bude po prvom kroku
na práve dvoch stenách a~zmizne po druhom kroku.]

Na tabuli sú napísané celé nezáporné čísla od 0 do $1\,234$. Uvažujme
nasledovnú operáciu: Zmažeme ľubovoľné dve čísla a~namiesto nich napíšeme na
tabuľu ich rozdiel (od väčšieho čísla odčítame menšie). Túto operáciu
opakujeme, kým na tabuli neostane posledné číslo. Môže na tabuli ostať
číslo~2?
[Nie. Uvedenou operáciou sa nemení parita súčtu všetkých čísel
napísaných na tabuli a tento súčet je v~našom prípade nepárny.]

Na tabuli sú napísané všetky prirodzené čísla od 1 do 100. Uvažujme
nasledovnú operáciu: Zmažeme ľubovoľné dve čísla a~namiesto nich napíšeme na
tabuľu ich súčet. Túto operáciu opakujeme, kým na tabuli neostanú
posledné tri čísla. Môžeme týmto spôsobom nakoniec získať tri po
sebe idúce čísla?
[Súčet troch po sebe idúcich čísel je deliteľný tromi,
ale nemeniaci sa súčet všetkých čísel na tabuli deliteľný tromi nie je.]

Na stole je $n$ pohárov, všetky sú postavené dnom nahor.
V~jednom kroku môžeme otočiť ľubovoľných $k$ pohárov naopak ($k$ je
dané, nemenné). Je možné, aby po konečnom počte krokov bolo všetkých
$n$ pohárov postavených dnom nadol? Riešte najprv pre $n=9$ a~$k=5$,
potom pre $n=9$ a~$k=4$.
[Pre $n=9$ a~$k=5$ to zrejme možné je. Pre $n=9$
a~$k=4$ to možné nie je, pretože všeobecnejšie platí: pri párnom $k$ a~ľubovoľnom
$n$ sa nemení parita počtu pohárov postavených dnom nahor (\tj.~tento počet
je buď stále párny alebo stále nepárny).]

Je daných $n$ ($n\ge 2$) prirodzených čísel, s~ktorými môžeme
vykonať nasledovnú operáciu: vyberieme niekoľko z~nich, ale nie
všetky, a~nahradíme ich ich aritmetickým priemerom. Zistite,
či je možné pre ľubovoľnú začiatočnú $n$-ticu dostať po konečnom
počte krokov všetky čísla rovnaké, ak $n$ je
a) 2\,000, b) 35, c) 3, d) 17.
\vpravo{[51--B--I--4]}
\endnávod
}

{%%%%%   A-I-6
\epsplace a60.3 \hfil\Obr\par
\epsplace a60.4 \hfil\Obr\par
\epsplace a60.5  \par
\epsplace a60.6  \par
\epsplace a60.7 \hfil\Obr\par

\inspicture r(1.5)
Ak $a=b$, je $\alpha=\beta$, takže $\cos(\alpha-\beta)=1$ a~dokazovaná nerovnosť
platí ako rovnosť $a^2+a^2=2a^2$ (dodajme, že bez
ohľadu na to, či je uhol $\gamma$ ostrý alebo nie).
Keďže dokazovaná nerovnosť je symetrická v $a$, $b$ (kosínus je párna funkcia),
môžeme bez ujmy na všeobecnosti predpokladať, že $a>b$ čiže $\alpha>\beta$.

Keďže $\alpha>\beta$, môžeme uhol $BAC$ veľkosti $\alpha$
rozdeliť pomocou bodu $D\in BC$ na dva uhly $CAD$ a~$DAB$ veľkostí $\beta$,
a $\alpha-\beta$ (\obr). Trojuholník $DAC$ je potom zmenšením trojuholníka $ABC$
s~koeficientom podobnosti $k=b:a$, takže $|AD|=bc/a$
a~$|DC|=b^2/a$; odtiaľ $|BD|=|BC|-|DC|=(a^2-b^2)/a$.


Vyjadrenia $|AD|$, $|BD|$ dosadíme do rovnosti z~kosínusovej vety pre
trojuholník $ABD$ a~upravíme:
$$
\align
|BD|^2&=|AB|^2+|AD|^2-2|AB|\cdot|AD|\cos(\alpha-\beta),\\
\frac{(a^2-b^2)^2}{a^2}&=c^2+\frac{b^2c^2}{a^2}
-\frac{2bc^2\cos(\alpha-\beta)}{a},\\
(a^2-b^2)^2&=\delta \cdot c^2,\quad\text{kde}\quad
\delta =a^2+b^2-2ab\cos(\alpha-\beta)>0.
\tag1
\endalign
$$
(Posledná nerovnosť vyplýva z~toho, že pre $\alpha\ne\beta$ je $\cos(\alpha-\beta)<1$.)
Vzťah~(1) spolu s~rovnosťou $c^2=a^2+b^2-2ab\cos\gamma$
teraz využijeme na úpravu rozdielu $\Delta$
pravej a~ľavej strany dokazovanej nerovnosti, ktorý
navyše ešte vynásobíme výrazom~$2ab$:
$$
\align
2ab\Delta&=2ab\bigl(2ab-(a^2+b^2)\cos(\alpha-\beta)\bigr)=
4a^2b^2-(a^2+b^2)\cdot2ab\cos(\alpha-\beta)=\\
&=4a^2b^2-(a^2+b^2)(a^2+b^2-\delta )=\delta (a^2+b^2)-(a^2-b^2)^2=\\
&=\delta (a^2+b^2)-\delta \cdot c^2=\delta (a^2+b^2-c^2)=\delta \cdot2ab\cos\gamma.
\endalign
$$
Po vydelení výrazom $2ab$ dostávame vzťah $\Delta=\delta \cos\gamma$, takže
vzhľadom na $\delta >0$ má výraz $\Delta$ rovnaké znamienko ako
$\cos\gamma$ (zopakujme, že za predpokladu $\alpha\ne \beta$). Odtiaľ vyplýva,
že v~prípade, keď $\gamma<90^\circ$ a~$\alpha\ne \beta$, platí nerovnosť zo
zadania úlohy ako {\it ostrá}. Tým je úloha vyriešená a~odpoveď na
jej záverečnú otázku znie:
v~dokázanej nerovnosti (v~zadanej situácii, \tj. pri ostrom uhle $\gamma$)
nastane rovnosť práve vtedy, keď $a=b$.

\smallskip
{\it Poznámka 1.} Odvodený vzťah $\Delta=\delta \cos\gamma$ sa bez
pomocných označení prepíše ako identita
$$
2ab-(a^2+b^2)\cos(\alpha-\beta)=
\bigl(a^2+b^2-2ab\cos(\alpha-\beta)\bigr)\cos\gamma,
\tag2
$$
ktorá platí pre {\it ľubovoľný\/} trojuholník $ABC$ (k~nášmu odvodeniu
stačí pridať triviálne overenie rovnosti~(2) v~prípade $a=b$).
Výsledok (2) umožňuje ľahko urobiť diskusiu o~jednotlivých
prípadoch relácie
$$
(a^2+b^2)\cos(\alpha-\beta)\lesseqqgtr 2ab,
$$
lebo prvý činiteľ na pravej strane (2) je vždy nezáporný:
$$
a^2+b^2-2ab\cos(\alpha-\beta)\ge a^2+b^2-2ab=(a-b)^2\ge0.
$$
Relácia dopadá takto: rovnosť nastane práve vtedy, keď $a=b$ alebo
$\gamma=90^\circ$; v~prípade $a\ne b$ potom platí ostrá nerovnosť
$<$ alebo $>$ podľa toho, či je $\gamma<90^\circ$ alebo $\gamma>90^\circ$.

\ineriesenie
Pôvodné riešenie je celé založené na vzťahu (1),
preto jeho odlišné odvodenie teraz uvedieme ako "iné riešenie".
Tvar kladného výrazu $\delta $ v~(1) je motiváciou k~úvahe o~pomocnom trojuholníku,
ktorého dve strany majú dĺžky $a$, $b$
a~zvierajú uhol veľkosti $\alpha-\beta$ (opäť predpokladáme, že $a>b$).
Nás zaujíma dĺžka jeho tretej strany, ktorú označíme~$d$, takže pre výraz $\delta $ vo
vzťahu~(1), ktorý sa chystáme dokázať, budeme mať $\delta =d^2$.
Ukážme, že taký trojuholník so stranami $a$, $b$, $d$ je~-- okrem pôvodného
trojuholníka so stranami $a$, $b$, $c$~--
druhým riešením úlohy {\it zostrojiť trojuholník $ABC$, ak sú dané
strany $a$, $b$ a~uhol~$\beta$}. Konštrukciu oboch riešení $A_1BC$
a~$A_2BC$ vidíme na \obr. Súčet uhlov pri vrcholoch $A_1$ a~$A_2$
(vyznačených oblúčikmi) je zrejme $180^\circ$.
V~jednom z~trojuholníkov je to uhol $\alpha$,
v~druhom teda uhol $180^\circ-\alpha$, takže uhol pri vrchole~$C$
druhého trojuholníka je práve $\alpha-\beta$, ako sme si želali. (V~prípade
$\alpha=90^\circ$ síce platí $A_1=A_2$, ale na celej našej úvahe
netreba nič meniť: v takom prípade totiž $\alpha-\beta=\gamma$ a~$c=d$.)
Úsečky $A_1B$, $A_2B$ teda majú (v~niektorom poradí) dĺžky
$c$ a~$d$. Z~mocnosti bodu~$B$ k~zostrojenej kružnici so stredom~$C$
a~polomerom~$b$ vyplýva rovnosť
$$
cd=a^2-b^2,     \tag3
$$
z~ktorej po umocnení na druhú dostávame $c^2d^2=(a^2-b^2)^2$.
A to je kľúčový vzťah~(1) z~pôvodného riešenia, lebo,
ako už sme naznačili, podľa kosínusovej vety platí
$$
d^2=a^2+b^2-2ab\cos(\alpha-\beta).   \tag4
$$
\inspicture{}


\medskip
{\it Poznámka 2}. V~pôvodnom riešení sme zo vzťahu (1)
odvodili identitu zapísanú v~Poznámke~1 ako (2).
Práve uvedený alternatívny dôkaz (1) s~využitím
konštrukčnej úlohy $(a,b,\beta)$ má zaujímavý dôsledok: vďaka
"rovnoprávnosti" oboch riešení z~\obrr1{} musí platiť aj identita
$$
2ab-(a^2+b^2)\cos\gamma=c^2\cos(\alpha-\beta),    \tag5
$$
získaná z~(2) výmenou úloh trojuholníkov s~trojicami strán $(a,b,c)$
a~$(a,b,d)$\niedorocenky{ a~uvedená v~doplňujúcej úlohe~D1; v~jej návode
naznačujeme odlišné trigonometrické odvodenie}.

\ineriesenie
Pomocný trojuholník so stranami $a$, $b$ ($a>b$)
zvierajúcimi uhol ${\alpha-\beta}$ a~treťou stranou~$d$ danou vzťahom~(4)
môžeme využiť na riešenie úlohy i~bez objavu "mocnosťovej" rovnosti~(3)
nasledovným postupom\niedorocenky{, ktorý môže byť blízky riešiteľom,
a~preto ho opisujeme i v~návodnej úlohe~N1}.

Uvedený trojuholník je možné k~trojuholníku $ABC$ vhodne prikresliť dvoma spôsobmi,
ktoré vidíme na \obr. Vľavo je to trojuholník $BCD$ (ten poznáme už
\vadjust{\medskip\centerline{\inspicture-!\hss \inspicture-!}\centerline\Obr\bigskip}%
z~predchádzajúceho riešenia), vpravo to je trojuholník $BCE$; ľahko potom overíme, že
oba vyznačené uhly $BCD$ a~$CBE$ majú požadovanú veľkosť
$\alpha-\beta$. (Oba obrázky zodpovedajú prípadu
$\alpha<90^\circ$, v~úplnom riešení by nemal chýbať obrázok pre prípad
$\alpha\ge90^\circ$, ktorý tu posudzovať nebudeme, pretože ďalší postup
vyžaduje len malú obmenu.)
Pomocou dĺžky $d$ zo vzťahu (4) teraz upravíme dokazovanú
(ostrú) nerovnosť:
$$
\align
(a^2+b^2)\cos(\alpha-\beta)&<2ab,\\
(a^2+b^2)\cdot2ab\cos(\alpha-\beta)&<4a^2b^2,\\
(a^2+b^2)(a^2+b^2-d^2)&<4a^2b^2,\\
(a^2-b^2)^2&<(a^2+b^2)d^2.          \tag6
\endalign
$$
Nakoniec využijeme Pytagorovu vetu pre dvojice
pravouhlých trojuholníkov z~\obrr1; v~oboch variantoch
(ako s~trojuholníkom $BCD$, tak s trojuholníkom~$BCE$) potom platí
$$
a^2=(d+x)^2+v^2\quad\text{a}\quad b^2=x^2+v^2,
$$
takže $a^2-b^2=d^2+2dx=d(d+2x)$. Po dosadení do ľavej strany
nerovnosti~(6) a~skrátení výrazom~$d^2$ dostaneme
ekvivalentnú nerovnosť
$$
(d+2x)^2<a^2+b^2\quad\text{čiže}\quad c^2<a^2+b^2,
$$
ktorá (vďaka kosínusovej vete) presne vyjadruje podmienku
$\gamma<90^\circ$ zo zadania úlohy. Tým je celé jej riešenie hotové, pretože
v~prípade $a=b$ zrejme v~dokazovanej nerovnosti nastane rovnosť.


\ineriesenie
Ešte jedným spôsobom za predpokladov $\gamma<90^\circ$
a~$a>b$ (čiže $\alpha>\beta$) dokážeme ostrú nerovnosť
$$
(a^2+b^2)\cos(\alpha-\beta)<2ab.
$$
Najprv ju ekvivalentne upravíme, keď položíme
$\phi=\frac12(\alpha-\beta)>0$ a~využijeme vzorec
$\cos2\phi=1-2\sin^2\phi$:
$$
\align
(a^2+b^2)(1-2\sin^2\phi)&<2ab,\\
(a-b)^2&<2(a^2+b^2)\sin^2\phi,\\
2\cdot\Bigl(\frac{a-b}{2\sin\phi}\Bigr)^{\!2}&<a^2+b^2.
\endalign
$$
To je (podľa sínusovej vety) nerovnosť $2r^2<a^2+b^2$ pre polomer $r$
kružnice opísanej ľubovoľnému trojuholníku so stranou $a-b$ a~protiľahlým
vnútorným uhlom $\phi$. Taký trojuholník dostaneme, keď ako na
\obr{} stranu $CA$
\inspicture{}
trojuholníka $ABC$ predĺžime za bod~$A$ do bodu $F$ tak, aby platilo
$|CF|=a$  ($a>b$). Potom má trojuholník $ABF$ stranu~$AF$ dĺžky $a-b$
s~protiľahlým uhlom $ABF$, ktorého veľkosť určíme takto: rovnoramenný trojuholník~$BCF$
má pri základni $BF$ zhodné uhly
$90^\circ-\frac12\gamma=\frac12(\alpha+\beta)$, takže
$$
|\angle ABF|=|\angle CBF|-|\angle CBA|=\frac{\alpha+\beta}{2}-\beta=\phi.
$$
Preto je polomer kružnice opísanej trojuholníku $ABF$ naozaj rovný
skúmanej hodnote~$r$. Pre ňu tak dostaneme z~predpokladu
$\gamma<90^\circ$ odhad
$$
r=\frac{|AB|}{2\sin|\angle AFB|}=\frac{c}{2\sin(90^\circ-\frac12\gamma)}<
\frac{c}{2\sin45^\circ}=\frac{c}{\sqrt2}
$$
čiže $2r^2<c^2$; z toho istého predpokladu $\gamma<90^\circ$
vyplýva (podľa kosínusovej vety pre~trojuholník $ABC$)
ďalšia nerovnosť $c^2<a^2+b^2$.
Dokopy dostávame $2r^2<c^2<a^2+b^2$ a~požadovaná nerovnosť
$2r^2<a^2+b^2$ je tak dokázaná.

Dodajme ešte, že v~prípade
$\gamma>90^\circ$ z rovnakých dôvodov platí $2r^2>c^2>a^2+b^2$, čo
(za predpokladu $a\ne b$) dokazuje~opačnú nerovnosť
$$
(a^2+b^2)\cos(\alpha-\beta)>2ab.
$$

\ineriesenie
Uvedieme ešte jedno trigonometrické riešenie.
Pre ľubovoľný trojuholník  $ABC$ platí totiž tzv. {\it Mollweidov vzorec}
$$
{a-b\over c}={\sin\frac1{2}(\alpha-\beta)\over\cos\frac1{2}\gamma},
$$
\niedorocenky{o~ktorom pojednáva návodná úloha N2 a~}z ktorého vyplýva
nasledovné vyjadrenie hodnoty $\cos(\alpha-\beta)$:
$$
\cos(\alpha-\beta)=1-2\sin^2\frac{\alpha-\beta}{2}=
1-\frac{2(a-b)^2\cos^2\frac1{2}{\gamma}}{c^2}.
$$
Dosadením do ľavej strany dokazovanej nerovnosti dostaneme
$$
\align
(a^2+b^2)\cos(\alpha-\beta)&\le2ab,\\
(a^2+b^2)\biggl(1-\frac{2(a-b)^2\cos^2\frac1{2}{\gamma}}{c^2}\biggr)
&\le2ab,\\
(a-b)^2&\le\frac{2(a^2+b^2)(a-b)^2\cos^2\frac1{2}{\gamma}}{c^2}.
\endalign
$$
Vidíme, že v~prípade $a=b$ nastane rovnosť. V~prípade $a\ne b$
po vydelení kladným výrazom~$(a-b)^2$ a~ďalšej zrejmej ekvivalentnej
úprave dostaneme
$$
c^2\le2(a^2+b^2)\cos^2\frac{\gamma}{2}.
$$
Ak sem dosadíme z~rovností
$$
c^2=a^2+b^2-2ab\cos\gamma\quad\text{a}\quad
2\cos^2\frac{\gamma}{2}=1+\cos\gamma,
$$
dostaneme po odčítaní súčtu $a^2+b^2$ od oboch strán nerovnosť
$$
-2ab\cos\gamma\le(a^2+b^2)\cos\gamma\quad\text{čiže}\quad
0\le(a+b)^2\cos\gamma,
$$
čo vďaka zadanému predpokladu $\gamma<90^\circ$ naozaj platí ako
ostrá nerovnosť. Tým je nerovnosť zo zadania úlohy dokázaná;
rovnosť v~nej nastane práve vtedy, keď $a=b$. (Aj pri
tomto postupe je možné odvodiť všeobecnejšie závery uvedené v~Poznámke 1
za prvým riešením.)

\návody
Najprv uvážte, ako k~danému trojuholníku $ABC$, v ktorom platí $a>b$
a~$\gamma<90^\circ$, vhodne prikresliť trojuholník s~dvoma stranami $a$, $b$,
ktoré by zvierali uhol $\alpha-\beta$. Označte $d$ dĺžku tretej strany
takého trojuholníka a~ukážte, že nerovnosť zo zadania súťažnej úlohy je
ekvivalentná s~nerovnosťou $(a^2-b^2)^2\le(a^2+b^2)^2d^2$.
Tú potom dokážte tak, že do ľavej strany dosadíte vyjadrenie prepôn
$a$, $b$ vo vhodných pravouhlých trojuholníkoch %% "\tr-" se tu nehodí
pomocou Pytagorovej vety.
[Celý postup je podrobne opísaný v~treťom riešení súťažnej úlohy.]

Pre všeobecný trojuholník $ABC$ dokážte tzv. Mollweidov vzorec
$$
{a-b\over c}={\sin\frac1{2}(\alpha-\beta)\over\cos\frac1{2}\gamma},
$$
s~pomocou ktorého je možné vyriešiť zadanú súťažnú úlohu.
[Mollweidov vzorec je triviálny v~prípade $a=b$; v~prípade $a>b$
použite sínusovú vetu pre trojuholník $ABF$, kde $F$ je ten bod ležiaci
na predĺžení strany $CA$ za bod $A$, pre ktorý platí $|AF|=a-b$;
prípad $a<b$ sa dá previesť na predchádzajúci zámenou strán $a$ a~$b$.
Riešenie súťažnej úlohy pomocou Mollweidovho vzorca je
v~našom texte uvedené ako posledné.]

\D
Pre všeobecný trojuholník $ABC$ dokážte rovnosť
$$
2ab-(a^2+b^2)\cos\gamma=c^2\cos(\alpha-\beta).
$$
[S~využitím rovnosti $a^2+b^2=c^2+2ab\cos\gamma$ sa dá
dokazovaný vzorec upraviť na tvar
$2ab\sin^2\gamma=c^2\bigl(\cos(\alpha-\beta)+\cos\gamma\bigr)$.
Ukážte ďalej, že platí $\cos(\alpha-\beta)+\cos\gamma=2\sin\alpha\sin\beta$, a~potom
využite, že rovnosť $ab\sin^2\gamma=c^2\sin\alpha\sin\beta$ je dôsledkom
sínusovej vety.]

Pre všeobecný trojuholník $ABC$ dokážte druhý Mollweidov vzorec
$$
{a+b\over c}={\cos\frac12(\alpha-\beta)\over\sin\frac1{2}{\gamma}},
$$
ktorý spolu s~prvým Mollweidovým vzorcom z~úlohy N2 vedie
k~rovnosti
$$
\frac{a-b}{a+b}=\frac{\tg\frac12(\alpha-\beta)}{\tg\frac12(\alpha+\beta)}
$$
označovanej spolu s~ďalšími dvoma analogickými rovnosťami
pre dvojice strán $a$, $c$ a~$b$, $c$ ako tangensová veta pre trojuholník $ABC$.
[Použite sínusovú vetu pre trojuholník $ABG$, kde bod $G$ leží na predĺžení
strany $BC$ za bod $C$ tak, že $|BG|=a+b$. Pre odvodenie tangensovej
vety porovnajte podiel ľavých a~podiel pravých strán oboch
Mollweidových vzorcov a~k~tomu uvážte, že
$\tg\frac12(\alpha+\beta)=\cotg\frac12{\gamma}$.]
\endnávod
}

{%%%%%   B-I-1
Umocnením a~odčítaním prvých dvoch rovností dostaneme
$x^2-z^2=({z+1})^2-(x+1)^2$, čo upravíme na
$2(x^2-z^2)+2(x-z)=0$ čiže
$$
(x-z)(x+z+1)=0.
\tag1
$$
Analogicky by sme dostali ďalšie dve rovnice, ktoré vzniknú z~(1)
cyklickou zámenou neznámych $x\to y\to z$. Vzhľadom na túto symetriu (daná sústava
se nezmení dokonca pri ľubovoľnej permutácii neznámych) stačí rozobrať len nasledovné
dve možnosti:

Ak $x=y=z$, prejde pôvodná sústava na jedinú rovnicu
$\sqrt{2x^2}=x+1$, ktorá má dve riešenia $x_{1,2}=1\pm\sqrt2$.
Každá z~trojíc $\bigl(1\pm\sqrt2,1\pm\sqrt2,1\pm\sqrt2\bigr)$ je zrejme riešením
pôvodnej sústavy.

Ak sú niektoré dve z~čísel $x$, $y$, $z$ rôzne,
napríklad $x\ne z$, vyplýva z~(1) rovnosť $x+z=\m1$. Dosadením $x+1=\m z$ do druhej rovnice
sústavy dostávame $y=0$ a~potom z~tretej rovnice máme $x^2+(x+1)^2=1$ čiže $x(x+1)=0$.
Posledná rovnica má dve riešenia $x=0$ a~$x=\m1$, ktorým zodpovedajú $z=\m1$ a~$z=0$.
Ľahko overíme, že obe nájdené trojice $(0,0,\m1)$ a~$(\m1,0,0)$ sú riešeniami danej
sústavy, rovnako aj trojica $(0,\m1,0)$, ktorú dostaneme ich permutáciou.

Daná sústava má päť riešení:
$$
\hbox{$(0,0,-1)$, $(0,-1,0)$, $(-1,0,0)$,
$\bigl(1+\sqrt2,1+\sqrt2,1+\sqrt2\bigr)$ a~$\bigl(1-\sqrt2,1-\sqrt2,1-\sqrt2\bigr)$.}
$$

\návody
V~obore reálnych čísel vyriešte sústavu
$$
\align
x^2=&y+z+1,\\
y^2=&z+x+1,\\
z^2=&x+y+1.
\endalign
$$
[$(0,0,-1)$, $(0,-1,0)$, $(-1,0,0)$]

V~obore reálnych čísel vyriešte sústavu
$$\align
\sqrt{x-y^2}=&z-1,\\
\sqrt{y-z^2}=&x-1,\\
\sqrt{z-x^2\vphantom y}=&y-1.
\endalign
$$
\vpravo{[59--A--S--1]}

Určte všetky trojice $(x,y,z)$ reálnych čísel, pre ktoré platí
$$
\postdisplaypenalty 10000
\align
x^2+xy=&y^2+z^2,\\
z^2+zy=&y^2+x^2.
\endalign
$$
\vpravo{[58--B--I--2]}

V~obore reálnych čísel riešte sústavu rovníc
$$
\align
x+y=&1,\\
x-y=&a,\\
-4ax+4y=&z^2+4
\endalign
$$
s~neznámymi $x$, $y$, $z$ a~reálnym parametrom $a$.
\vpravo{[58--B--II--1]}

V~obore reálnych čísel riešte sústavu rovníc
$$
\align
x^2+2yz=&6(y+z-2),\\
y^2+2zx=&6(z+x-2),\\
z^2+2xy=&6(x+y-2).
\endalign
$$
\vpravo{[A--53--S--3]}
\endnávod
}

{%%%%%   B-I-2
\epsplace b60.1 \hfil\Obr\par
Uhlopriečka~$AC$ daného obdĺžnika $ABCD$ je podľa zadania strednou priečkou
v~trojuholníku
$PQR$, a~teda $AC\parallel QR$, takže aj $AC\parallel MN$. Úsečka~$MN$
je tak jednoznačne
\inspicture%
určená tým, že je rovnobežná s~$AC$, leží v~opačnej polrovine určenej priamkou~
$AC$ ako bod $P$ a~pre jej dĺžku platí $|MN|=|AB|$.
Konštrukciu bodov $M$ a~$N$ je možné urobiť
niekoľkými spôsobmi. Dá sa napríklad využiť rovnobežník $AMNE$
(\obr), v~ktorom platí $|AE|=|MN|=|AB|$.

Keďže úsečka~$MN$ súčasne určuje priamku, na ktorej leží strana~$QR$ trojuholníka $PQR$,
je zrejmé, že vrchol~$P$ musí ležať na priamke~$p$, ktorá je obrazom priamky~$MN$ v~osovej
súmernosti podľa priamky~$AC$ (obsahujúcej strednú priečku trojuholníka $PQR$).
Priamka~$p$ má s~vnútrom daného obdĺžnika
spoločné vnútro úsečky~$M'N'$ (ktorá je navyše obrazom nájdenej úsečky~$MN$
v~stredovej súmernosti podľa stredu daného obdĺžnika).

Ľahko vidíme, že aj naopak ku každému vnútornému bodu~$P$ úsečky~$M'N'$ ležia
zodpovedajúce body $Q$, $R$ na priamke~$MN$ a~body $M$, $N$ sú tak priesečníky priamky~$QR$
so stranami $AB$, $BC$, takže vyhovujú podmienkam úlohy.

\zaver
Hľadanou množinou všetkých bodov $P$ danej vlastnosti je vnútro vyššie
opísanej úsečky~$M'N'$.

\návody
Je daná kružnica $k$ s~priemerom $AB$. K~ľubovoľnému bodu $Y$ kružnice
$k$, $Y\ne A$ zostrojme na polpriamke
$AY$ bod $X$, pre ktorý platí $|AX|=|YB|$.
Určte množinu všetkých takých bodov~$X$.
\vpravo{[56--B--I--2]}

V~rovine daného štvorca $KLMN$ určte množinu všetkých bodov $P$, pre ktoré
sú uhly $NPK$, $KPL$ a~$LPM$ zhodné.
\vpravo{[53--A--I--2]}

Je daný rovnostranný trojuholník $MPQ$. Nájdite množinu vrcholov $C$ všetkých
trojuholníkov $ABC$ takých, že
body $P$, $Q$ sú päty výšok z~vrcholov $A$, $B$ a~bod $M$ je stred strany~$AB$.
\vpravo{[51--B--I--6]}

Sú dané kružnice $k$ a~$l$ s~rôznymi polomermi, ktoré majú vonkajší dotyk
v~bode $T$. Priesečníkom $M$ ich
spoločných vonkajších dotyčníc veďme sečnicu $s$ oboch kružníc. Označme $X$ ten
z~oboch priesečníkov kružnice $k$ so sečnicou
$s$, ktorý je vzdialenejší od bodu $M$. Podobne označme $Y$ ten z~oboch
priesečníkov kružnice $l$ so sečnicou $s$, ktorý
je vzdialenejší od bodu $M$. Nech $P$ je taký bod, že $XTYP$ je
rovnobežník. Určte množinu bodov~$P$ zodpovedajúcich
všetkým takým sečniciam~$s$.
\vpravo{[49--B--I--4]}
\endnávod
}

{%%%%%   B-I-3
Stačí ukázať, že súčet troch skúmaných čísel neprevyšuje $24$:
$$
(ab+bc)+(bc+ca)+(ca+ab)=2(ab+bc+ca)\le \frac23(a+b+c)^2=24,
$$
kde nerovnosť je dôsledkom nerovnosti
$$
3(ab+bc+ca) \le (a+b+c)^2=36,
$$
ktorá je ekvivalentná s nerovnosťou $0\le (a-b)^2+(b-c)^2+(c-a)^2$;
tá je splnená pre ľubovoľné tri reálne čísla $a$, $b$, $c$.

\ineriesenie
Vzhľadom na symetriu predpokladajme, že
platí $a=\min\{a,b,c\}$. Z~rovnosti $a+b+c=6$ potom vyplýva
$a\le2$ a~$b+c\ge4$. Preto tretie skúmané číslo,
rovné $a(b+c)$, má rovnaké znamienko ako číslo~$a$, takže je
určite menšie ako~$8$, ak platí $a\le0$. Ak je naopak
$0<a\le2$, všimnime si,
že zo zrejmej nerovnosti $0\le(u-v)^2$, platnej pre
ľubovoľné reálne čísla $u,v$, vyplýva úpravou odhad $4uv\le(u+v)^2$;
ak sem dosadíme $u=2a$ a~$v=b+c$, dostaneme
$$
8a(b+c)\le(2a+b+c)^2=(a+6)^2\le8^2=64;
$$
odtiaľ po vydelení ôsmimi dostaneme nerovnosť $a(b+c)\le8$.

\návody
Dokážte, že pre ľubovoľné reálne čísla $a$, $b$, $c$ z~intervalu $\langle
0,1\rangle$ platí
$$
1\le a+b+c+2(ab+bc+ca)+3(1-a)(1-b)(1-c)\le 9.
$$
\vpravo{[55--B--II--4]}

Ak reálne čísla $a$, $b$, $c$, $d$ vyhovujú rovnostiam
$$
a^2+b^2=b^2+c^2=c^2+d^2=1,
$$
tak platí nerovnosť
$$
ab+ac+ad+bc+bd+cd\le 3.
$$
Dokážte a~zistite, kedy pritom nastane rovnosť.
\vpravo{[55--C--II--2]}
\endnávod
}

{%%%%%   B-I-4
Zlomok
$$
\frac{n^3+2\,010}{n^2+2\,010}=n-\frac{2\,010(n-1)}{n^2+2\,010}
$$
je celé číslo práve vtedy, keď $n^2+2\,010$ je deliteľ čísla $2\,010(n-1)=2\cdot3\cdot5\cdot67(n-1)$.

Ak nie je $n$ násobok prvočísla~$67$, sú čísla $n^2+2\,010$ a~$67$ nesúdeliteľné,
preto ${n^2+2\,010}$ musí byť deliteľom čísla
$30(n-1)$. Keďže $|30(n-1)|<n^2+2\,010$, vyhovuje len $n=1$.

Nech $n=67m$, kde $m$ je celé. Potom
$$
\frac{2\,010(n-1)}{n^2+2\,010}=\frac {30(67m-1)}{67m^2+30}.
$$
Ak nie je $m$ násobkom piatich, musí byť číslo $67m^2+30$ deliteľom čísla $6(67m-1)$. Pre $|m|\le 4$
to tak ale nie je a~pre $|m|\ge 6$ je
$|6(67m-1)|<67m^2+30$. Teda $m=5k$, kde $k$ je celé. Potom
$$
\frac{30(67m-1)}{67m^2+30}=\frac {6(335k-1)}{335k^2+6}.
$$
Pre $|k|\ge 7$ je absolútna hodnota tohto zlomku nenulová a~menšia ako~$1$.
Zo zvyšných čísel vyhovujú $k=0$ a~$k=\m6$.

Číslo $(n^3+2\,010)/(n^2+2\,010)$ je teda celé práve vtedy, keď celé~$n$
má niektorú z~hodnôt $0$, $1$ alebo $\m2\,010$.

\návody
Nájdite najmenšie prirodzené číslo $n$, pre ktoré je podiel
$\frac{n^2+15n}{33\,000}$ prirodzené číslo.
\vpravo{[56--B--S--3]}

Nájdite všetky dvojice $(p,q)$ reálnych čísel také, že mnohočlen
$x^2+pxq$ je deliteľom mnohočlena $x^4+px^2+q$.
\vpravo{[56--B--I--5]}
\endnávod
}

{%%%%%   B-I-5
\epsplace b60.2 \hfil\Obr\par
Hoci odporúčame riešiť obe časti úlohy oddelene (\tj.~najprv analyzovať
situáciu v~pravouhlom trojuholníku), opíšeme priamo ich spoločné riešenie. Celú
úlohu môžeme totiž formulovať ako dôkaz tvrdenia,
že šesť zostrojených bodov leží na kružnici práve vtedy, keď je uhol $ACB$ pravý.

Uvažujme teda ľubovoľný trojuholník $ABC$ s~ostrými uhlami $\alpha$, $\beta$ a~označme $M$
stred výšky~$CP$ a~$D$, $E$, $F$, $G$, $H$, $I$ uvažované
priesečníky tak, aby s~vrcholmi $A$, $B$, $C$ a~pätou výšky~$P$
ležali na hranici trojuholníka v~poradí
$$
A,\ D,\ P,\ E,\ B,\ F,\ G,\ C,\ H,\ I.
$$
Z~konštrukcie vyplýva, že body $M$, $D$, $I$ sú stredy strán
pravouhlého trojuholníka $ACP$ a~body $M$, $E$, $F$ sú stredy strán
pravouhlého trojuholníka $BCP$. Oba štvoruholníky $PMID$ a~$PMFE$ sú
teda pravouholníky, takže i~$DEFI$ je pravouholník (\obr). Jeho vrcholy
$D$, $E$, $F$, $I$ preto {\it vždy\/} ležia na jednej kružnici
a~úsečky $DF$ a~$EI$ sú jej priemery. Našou úlohou je preto
zistiť, kedy na tejto kružnici ležia i~body $G$ a~$H$. To sa dá podľa
Tálesovej vety vyjadriť podmienkou, že uhly $DGF$ a~$EHI$ sú
pravé. Keďže ${DG\parallel AC}$ a~$EH\parallel BC$, sú oba uhly
$DGF$ a~$EHI$ zhodné s~uhlom $ACB$ a~ekvivalencia s~podmienkou
pravého uhla $ACB$ je tak dokázaná.
\inspicture

\návody
Označme $S$ stred kružnice vpísanej danému trojuholníku $ABC$ a~$P$, $Q$
päty kolmíc z~vrcholu $C$ na priamky,
na ktorých ležia osi vnútorných uhlov $BAC$ a~$ABC$. Dokážte, že priamky $AB$
a~$PQ$ sú rovnobežné.
\vpravo{[51--A--S--2]}

Vnútri strán $BC$, $CA$, $AB$ daného ostrouhlého trojuholníka $ABC$ sú
postupne vybrané
body $X$, $Y$ a~$Z$. Dokážte, že každému zo štvoruholníkov $ABXY$, $BCYZ$
a~$CAZX$ sa dá opísať kružnica práve vtedy, keď
body $X$, $Y$, $Z$ sú päty výšok trojuholníka $ABC$.
\vpravo{[51--B--S--2]}
\endnávod
}

{%%%%%   B-I-6
Nech $n$ je číslo vyhovujúce podmienkam zadania. Škrtnutím dvoch posledných
cifier zmenšíme $n$ aspoň stokrát,
preto sa môžeme obmedziť na škrtanie cifier, ktoré nie sú posledné.
Po škrtnutí dvoch susedných cifier ostanú z~čísla $n$ dve časti,
pritom prvá časť môže byť prázdna, ak sme škrtli jeho prvé dve cifry.

Nech $a$ je číslo určené prvou časťou čísla~$n$ (nula v~prípade, že
prvá časť je prázdna), $b$ je číslo určené vyškrtnutými dvoma ciframi
a~$c$ je určené poslednou časťou čísla~$n$ (počet cifier tejto časti označme~$k$).
Podľa zadania platí
$$
99(a\cdot 10^k + c) = a\cdot 10^{k+2}+b\cdot 10^k+c,
$$
po úprave $98c = 10^k(a+b)$. Keďže $c < 10^k$, musí byť $98 > a+b$.
Navyše číslo~$49$ delí $a+b$, lebo je samo nesúdeliteľné s~$10^k$.
Kladný celočíselný podiel $(a+b)/49$ je menší ako~$2$, musí teda byť rovný
$1$, takže $a+b=49$. Odtiaľ vyplýva rovnosť
$$
c = {10^k\over 2} = 5\cdot 10^{k-1},
$$
kde číslo $k$ je súčasne určené počtom cifier čísla~$a$
(ak označíme $l$ počet číslic čísla~$a$, je $k=10-l-2$, pričom v~prípade $a=0$
položíme prirodzene $l=0$).

Z~uvedeného postupu vyplýva, že pre každé $a\in\{0,1,2,\dots,49\}$
a~$b=49-a$ existuje  práve jedno číslo~$c$, pre ktoré opísané
číslo~$n$ vyhovuje podmienkam zadania, a~že iné vyhovujúce~$n$ neexistujú.
Ukážeme, že všetkých 50~takých~$n$
(končiacich siedmimi, šiestimi alebo piatimi nulami) je navzájom rôznych.

Zostrojené $n$ končiace siedmimi nulami je jediné ($a=0$). Šiestimi
nulami končí 9~zostrojených čísel ($a\in\{1,2,\dots,9\}$)
a~sú navzájom rôzne, lebo začínajú rôznymi ciframi. Piatimi
nulami končí 40~zostrojených čísel ($a\in\{10,11,\dots,49\}$)
a~sú navzájom rôzne, lebo začínajú rôznymi dvojčísliami.

Pre názornosť  vypíšme ešte niekoľko čísel vyhovujúcich zadaniu tak, ako ich
dostaneme pomocou našich úvah:
pre $a=0$ máme $b=49$, $c=50\,000\,000$ a~$n=4\,950\,000\,000$, pre $a=1$ je $b=48$,
$c=5\,000\,000$ a~$n=1\,485\,000\,000$,
pre $a=2$ je $n=2\,475\,000\,000$,~$\dots$,
pre $a=9$ je $n=9\,405\,000\,000$,
pre $a=10$ je $b=39$, $c=500\,000$ a~$n=1\,039\,500\,000$,~$\dots$,
pre $a=49$ je $b=0$, $c=500\,000$ a~$n=4\,900\,500\,000$.

\zaver
Existuje $50$~čísel, ktoré vyhovujú zadaniu.

\návody
Ukážte, že škrtnutím posledných dvoch číslic aspoň trojciferného čísla
dostaneme číslo minimálne
stokrát menšie ako pôvodné číslo.

Prirodzené číslo nazveme {\it vlnitým}, ak pre každé tri po sebe
idúce číslice $a$, $b$, $c$ jeho
dekadického zápisu platí ${(a-b)(b-c)<0}$. Dokážte, že z~číslic 0, 1, \dots, 9 je
možné zostaviť viac ako 25\,000
desaťciferných vlnitých čísel, ktoré obsahujú všetky číslice od nuly po
deviatku (číslica~0 nemôže byť na prvom mieste).
\vpravo{[56--B--II--3]}

Určte najväčšie dvojciferné číslo~$k$ s~nasledovnou vlastnosťou: existuje
prirodzené číslo~$N$, z~ktorého po
škrtnutí prvej číslice zľava dostaneme číslo $k$-krát menšie. (Po vyškrtnutí
číslice môže zápis čísla začínať
jednou alebo niekoľkými nulami.) K~určenému číslu~$k$ potom nájdite najmenšie
vyhovujúce číslo~$N$.
\vpravo{[56--C--II--4]}

Určte počet všetkých trojíc dvojciferných prirodzených čísel $a$, $b$, $c$,
ktorých súčin $abc$ má zápis,
v ktorom sú všetky číslice rovnaké. Trojice líšiace sa iba poradím čísel
považujeme za rovnaké,
\tj. započítavame ich len raz.
\vpravo{[54--C--I--5]}
\endnávod
}

{%%%%%   C-I-1
Označme hľadané čísla $a$, $b$. Keďže $b\ne0$, nutne $a+b\ne a-b$. Každé
z čísel $a\cdot b$, $a:b$ je rovné buď $a+b$, alebo $a-b$. Stačí teda
rozobrať štyri prípady a v každom z nich vyriešiť sústavu rovníc. Ukážeme si však rýchlejší postup.

Ak by platilo
$$
a+b=a\cdot b\quad\text{a}\quad a-b=a:b\qquad\text{alebo}\qquad a+b=a:b\quad\text{a}\quad a-b=a\cdot b,
$$
vynásobením rovností by sme v~oboch prípadoch dostali $a^2-b^2=a^2$, čo je v~spore s~$b\ne0$.
Preto sú čísla $a\cdot b$ a $a:b$ buď obe rovné $a+b$ alebo obe rovné $a-b$. Tak či tak musí platiť $a\cdot b=a:b$, odkiaľ po úprave $a(b^2-1)=0$.
Keďže $a\ne0$, nutne $b\in\{1,\m1\}$. Ale ak $b=1$, tak štyri výsledky sú postupne $a+1$, $a-1$, $a$, $a$,
čo sú pre každé~$a$ až tri rôzne hodnoty. Pre $b=\m1$ máme výsledky $a-1$, $a+1$, $\m a$, $\m a$.
Dva rôzne výsledky to budú práve vtedy, keď $a-1=\m a$ alebo $a+1=\m a$. V~prvom prípade dostávame $a=\frac12$, v~druhom $a=\m\frac12$.

Lucia mohla na začiatku na tabuľu napísať buď čísla $\frac12$ a~$\m1$, alebo čísla $\m\frac12$ a~$\m1$.

\návody
Máme tri čísla so súčtom $2010$, pričom každé z nich je aritmetickým priemerom zvyšných dvoch.
Aké sú to čísla?
[Zostavíme a vyriešime sústavu rovníc, čísla musia byť rovnaké a teda sú rovné $670$.]

Máme tri čísla, o ktorých vieme, že každé z nich je aritmetickým priemerom niektorých dvoch
z našich troch čísel. Dokážte, že naše tri čísla sú rovnaké.
[Predpokladajme, že niektoré z našich čísel je priemerom seba a iného z našich čísel. Potom ich vieme
označiť $a$, $b$, $c$ tak, že $a = (a+b)/2$. Z tejto rovnosti vyplýva $a=b$. Číslo $c$ je buď priemerom
čísel $a$ a $b$, z čoho hneď máme, že je týmto číslam rovné, alebo je priemerom seba a niektorého
z čísel $a$, $b$, čiže $c = (c+a)/2$, z toho opäť dostaneme $c=a=b$. Ak každé z našich čísel je
aritmetickým priemerom zvyšných dvoch, riešime predošlú úlohu.]

\D
Nech $n$ je prirodzené číslo väčšie ako $2$. Máme $n$ čísel so súčtom $n$, pričom každé z nich je aritmetickým priemerom
ostatných čísel. Aké sú to čísla?
[Usporiadajme si naše čísla podľa veľkosti, nech $x_1\le x_2\le \cdots\le x_n$. Aritmetický priemer
skupiny čísel je aspoň taký, ako najmenšie z nich. Aritmetický priemer
čísel $x_2, x_3, \dots, x_n$ je preto aspoň $x_2$, a je rovný $x_1$ len v prípade,
že žiadne z čísel $x_3,\dots,x_n$ nie je väčšie ako $x_2$. Z toho hneď dostávame,
že všetky naše čísla musia byť rovnaké a teda rovné $1$.]
\endnávod
}

{%%%%%   C-I-2
Predpokladajme, že pre dvojicu prirodzených čísel $x$, $y$ platí $50\mid 23x+y$.
Potom pre nejaké prirodzené číslo $k$ platí $23x+y=50k$.
Z tejto rovnosti dostaneme $y=50k-23x$, čiže $19x+3y=19x+3(50k-23x)=150k-50x=50(3k-x)$,
takže číslo $19x+3y$ je násobkom čísla $50$.

Podobne to funguje aj z druhej strany.
Ak pre nejakú dvojicu prirodzených čísel $x$, $y$ platí $50\mid 19x+3y$,
tak $19x+3y=50l$ pre nejaké prirodzené číslo $l$. Z tejto rovnosti vyjadríme číslo $y$;
dostaneme $y=(50l-19x)/3$ (ďalší postup by bol podobný, aj keby sme vyjadrili $x$ miesto $y$).
Po dosadení dostaneme
$$
23x+y = 23x + {50l-19x\over 3} = {69x + 50l - 19x\over 3} = {50\cdot (x+l)\over 3}.
$$
O výslednom zlomku vieme, že je to prirodzené číslo. Čitateľ tohto zlomku je deliteľný číslom $50$.
V menovateli je len číslo $3$, ktoré je nesúdeliteľné s $50$, preto sa číslo $50$ nemá s čím z menovateľa vykrátiť
a teda číslo $23x+y$ je deliteľné $50$.

\ineriesenie
Zrejme $3\cdot (23x+y)-(19x+3y)=50x$, čiže ak $50$ delí jedno z čísel $23x+y$ a $19x+3y$, tak delí aj druhé z nich.

\návody
Ukážte, že každé prvočíslo väčšie ako $3$ sa dá napísať v tvare $6k+1$ alebo $6k-1$ pre vhodné prirodzené číslo $k$.
[Každé prvočíslo sa dá napísať v tvare $6k+z$, kde $z$ je jeho zvyšok po delení šiestimi.
Čísla $6k$, $6k+2$ a $6k+4$ sú evidentne deliteľné dvoma, $6k+3$ je deliteľné tromi, preto ostávajú len
čísla v tvare $6k+1$ a $6k+5$.]

Nech $x + 5y$ dáva zvyšok $1$ po delení $7$. Aký zvyšok po delení $7$ dáva číslo $3x+15y$? A~číslo $4x+13y$?
[Keďže $x+5y=7k+1$ pre vhodné $k$, máme $3x+15y = 3(7k+1) = 7\cdot 3k + 3$, čiže zvyšok je $3$.
Podobne $4x+20y=4(7k+1)=7\cdot 4k+4$, pritom číslo $4x+13y$ sa od $4x+20y$ líši len o násobok $7$, preto dáva rovnaký zvyšok.]

\D
Dokážte, že ak pre celé čísla $a$, $b$, $c$ platí $7\mid a-3b+5c$, tak platí aj $7\mid 4a+2b-c$.
Zistite, či platí opačná implikácia.
[Platí aj opačná implikácia. Návod: $(4a+2b-c)-4(a-3b+5c)=14b-21c=7(2b-3c)$.]

Dokážte, že ku každému celému číslu $x$ existuje celé číslo $y$ také, že $19x+3y$ je deliteľné $50$.
[Číslo $19x$ dáva po delení $50$ zvyšok, ktorý označíme $z$. Chceme ukázať, že pre ľubovoľné $z$
vieme nájsť $y$ tak, aby číslo $3y$ dávalo zvyšok $50-z$. Vezmime si čísla $3\cdot 1, 3\cdot 2, 3\cdot 3,\dots,
3\cdot 50$. Keby dve z týchto čísel, povedzme $3i$ a $3j$, dávali rovnaký zvyšok, musí byť ich rozdiel $3(i-j)$
deliteľný $50$. Pritom $3$ a $50$ sú nesúdeliteľné, preto $50\mid i-j$. To však nie je možné, lebo $1\le i-j\le 49$.
Preto vymenované čísla dávajú všetky možné rôzne zvyšky po delení $50$, a teda jedno z nich dáva zvyšok $50-z$.]
\endnávod
}

{%%%%%   C-I-3
\epsplace c60.2 \hfil\Obr\par
\epsplace c60.3 \hfil\Obr\par
a) Štvoruholníky $ABQK$ a~$DAPL$ sú zhodné (jeden z~nich je obrazom druhého
v~otočení o~$90^\circ$ so stredom v~strede štvorca $ABCD$).
Preto majú aj rovnaký obsah, čiže $S_A+S_B=S_A+S_D$. Z~toho hneď dostaneme $S_B=S_D$.

\smallskip
b) Ľahko sa nám podarí vypočítať obsah pravouhlého lichobežníka $ABQK$, lebo poznáme dĺžky základní aj výšku.
Dostaneme
$$
S_A+S_B=\left(\frac12+\frac13\right)\cdot\frac12=\frac5{12}\cm^2.
$$
Podobne výpočtom obsahu lichobežníka $PBCL$ dostaneme
$$
S_C+S_B=\left(\frac12+\frac23\right)\cdot\frac12=\frac7{12}\cm^2.
$$
Odčítaním prvej získanej rovnosti od druhej dostávame $S_C-S_A=\frac7{12}-\frac5{12}=\frac16\cm^2$.

\smallskip
c) Nerovnosť medzi obsahmi $S_A+S_C$ a~$S_B+S_D$ (ktorých priame výpočty nie sú v~silách
žiakov 1.~ročníka) môžeme zdôvodniť nasledovným spôsobom: Súčet
týchto dvoch obsahov je $1\cm^2$, takže sa nerovnajú práve vtedy, keď je jeden
z~nich menší ako $\frac12\cm^2$. Bude to obsah $S_B+S_D$ (rovný $2S_B$, ako už vieme),
keď ukážeme, že obsah $S_B$ je menší ako $\frac14\cm^2$. Urobíme to tak, že do
celého štvorca $ABCD$ umiestnime bez prekrytia štyri kópie štvoruholníka $PBQX$.
Ako ich umiestnime, vidíme na \obr, pričom $M$, $N$ sú
stredy strán $BC$, $AB$ a~$R$, $S$ body, ktoré delia strany $CD$,
$DA$ v~pomere $1:2$.

\twocpictures

\medskip\noindent
{\bf Iné riešenie} časti~c).
Tentoraz namiesto nerovnosti $S_B+S_D<\frac12\cm^2$ dokážeme ekvivalentnú
nerovnosť $S_A+S_C>\frac12\cm^2$. Preto sa pokúsime "premiestniť"
štvoruholník $APXK$ tak, aby ležal pri štvoruholníku
$XQCL$ a~aby sa ich obsahy dali geometricky sčítať. Uhly $AKQ$ a~$DLP$
sú zhodné a~$|AK|=|DL|$, preto môžeme štvoruholník $APXK$ premiestniť vo štvorci $ABCD$ do jeho
"rohu"~$D$ tak, že k~štvoruholníku~$XQCL$ priľahne pozdĺž strany~$LX$ svojou
stranou~$LY$, pričom $Y$ je priesečník úsečiek $SM$ a~$PL$ z~pôvodného riešenia (\obr). Obsah
$S_A+S_C$ je potom obsahom šesťuholníka $DSYXQC$. Prečo je väčší ako $\frac12\cm^2$,
môžeme zdôvodniť napríklad takto:

Úsečka spájajúca bod~$L$ so stredom~$U$ úsečky~$KQ$
pretne úsečku~$SM$ v~jej strede~$V$.
Štvoruholník $UQMV$ má obsah rovný polovici obsahu rovnobežníka $KQMS$, teda
rovný obsahu trojuholníka $KMS$. Preto má šesťuholník $DSVUQC$ obsah
rovný obsahu štvoruholníka $KMCD$, \tj. polovici obsahu štvorca $ABCD$.
Obsah $S_A+S_C$ je ešte väčší, a~to o~obsah štvoruholníka $XUVY$.
Teda naozaj $S_A+S_C>\frac12\cm^2$.

\návody
Daný je lichobežník $ABCD$ s~dlhšou základňou~$AB$ a~priesečníkom uhlopriečok~$P$.
Vieme, že obsah trojuholníka $ABP$ je $16$ a~obsah trojuholníka $BCP$ je $10$.
\item{a)} Vypočítajte obsah trojuholníka $ADP$.
\item{b)} Vypočítajte obsah lichobežníka $ABCD$.
\endgraf
[Trojuholníky $ABC$ a~$ABD$ majú spoločnú stranu~$AB$ a~rovnaké výšky na túto stranu, teda majú
rovnaký obsah. Preto majú rovnaký obsah trojuholníky $ADP$ a~$BCP$. Obsah trojuholníka $CDP$ vyrátame
napríklad z~jeho podobnosti s~trojuholníkom $ABP$, pomer podobnosti je $|AP|/|CP|=S_{ABP}/S_{CBP}$.
Dostaneme $S_{ABCD}=169/4$.]

Vo štvorci $ABCD$ s~obsahom~$1$ označme $K$, $L$ po rade stredy strán $AB$, $AD$.
Priamky $CK$ a~$BL$ sa pretínajú v~bode~$M$, priamky $CL$ a~$KD$ sa pretínajú v~bode~$N$.
Ukážte, že súčet obsahov trojuholníkov $KBM$, $KLN$ a~$CDN$ nie je väčší ako $3/8$.
[Priamo vypočítať obsahy jednotlivých trojuholníkov ide len ťažko.
Pomohlo by premiestniť tieto trojuholníky "viac k~sebe", aby sa ich obsahy dali geometricky sčítať.
Napríklad vďaka osovej súmernosti podľa priamky~$AC$ je trojuholník $KLN$ zhodný s~trojuholníkom
$KLM$. A~obsah trojuholníka $KBL$ už vypočítame ľahko, je to $1/8$. Ostáva ukázať,
že obsah trojuholníka $DCN$ je menší ako $1/4$. To hneď vidno z~toho,
že trojuholník $DCN$ je súčasťou trojuholníka $DCL$ s~obsahom $1/4$.]

\D
V~ostrouhlom trojuholníku $ABC$ označme $D$ pätu výšky z~vrcholu $C$ a~$P$, $Q$
zodpovedajúce päty kolmíc vedených bodom~$D$ na strany $AC$ a~$BC$. Obsahy trojuholníkov
$ADP$, $DCP$, $DBQ$, $CDQ$ označme postupne $S_1$, $S_2$, $S_3$, $S_4$.
Vypočítajte $S_1 : S_3$, ak
$S_1 : S_2 = 2 : 3$ a $S_3 : S_4 = 3 : 8$.
\vpravo{[C-55-I-5]}

V~ľubovoľnom konvexnom štvoruholníku $ABCD$ označme $E$ stred strany $BC$ a~$F$
stred strany~$AD$. Dokážte, že trojuholníky $AED$ a~$BFC$ majú rovnaký obsah práve
vtedy, keď sú strany $AB$ a~$CD$ rovnobežné.
\vpravo{[C-54-I-3]}

Spojnica stredov strán $AB$ a~$CD$ konvexného štvoruholníka $ABCD$ rozdelí tento štvoruholník na dve časti
s~rovnakým obsahom. Ukážte, že priamky $AB$ a~$CD$ sú rovnobežné.
[Označme $S$ a~$T$ po rade stredy strán $AB$ a~$CD$. Trojuholníky $DST$ a~$CST$ majú rovnaký obsah (rovnako dlhé strany
$DT$ a~$CT$, spoločná výška). Preto trojuholníky $ADS$ a~$BCS$ majú rovnaký obsah, a~keďže majú rovnako dlhé
strany $AS$ a~$BS$, musia mať aj rovnaké výšky, čiže body $D$ a~$C$ sú rovnako vzdialené od priamky~$AB$.]

Nájdite všetky konvexné štvoruholníky $ABCD$ s~nasledujúcou vlastnosťou:
v~rovine štvoruholníka $ABCD$ existuje bod~$P$ taký, že každá priamka vedená bodom~$P$ rozdelí štvoruholník
$ABCD$ na dve časti s~rovnakým obsahom.
\vpravo{[49-A-II-4]}
\endnávod
}

{%%%%%   C-I-4
a) Jediný spôsob, ako rozdeliť 7~žiakov na dve nanajvýš štvorčlenné skupiny, je mať jednu trojčlennú a~jednu štvorčlennú skupinu.
Každý žiak zo štvorčlennej skupiny pritom bude mať vo svojej skupine kamaráta pri hocijakom rozdelení,
pretože sa nemôže stať, že by všetci jeho kamaráti boli v~trojčlennej skupine (sú aspoň štyria).

Takže stačí rozdeliť žiakov tak, že každý v~trojčlennej skupine má v nej kamaráta. Preto do nej dáme hociktorého zo žiakov
a~k~nemu niektorých jeho dvoch kamarátov.

\smallskip
b) Vezmime hocijaké rozdelenie 8~žiakov na dve štvorčlenné skupiny.
Ak toto rozdelenie nevyhovuje učiteľkinmu zámeru, máme nejakého žiaka~$X$,
ktorý je {\it zle zaradený\/} -- má všetkých svojich štyroch kamarátov $A$, $B$, $C$, $D$ v~druhej skupine.
Ukážeme, že vieme vymeniť $X$ a~niektorého zo žiakov $A$, $B$, $C$, $D$ tak,
že počet zle zaradených žiakov sa zmenší.

Po každej zo štyroch výmen prichádzajúcich do úvahy $X$ prestane byť zle zaradený
a~všetci traja žiaci, ktorí budú s~$X$ v~skupine, budú dobre zaradení, lebo sú to kamaráti žiaka~$X$.
Žiaci $K$, $L$, $M$, ktorí boli pred výmenou v~skupine s~$X$, môžu byť po výmene zle zaradení len vtedy,
ak boli zle zaradení aj predtým (lebo $X$ nemal ani jedného z~nich za kamaráta).
Keďže žiak~$K$ má štyroch kamarátov a~nekamaráti sa s~$X$, musí mať aspoň jedného kamaráta~$Y$
aj v~skupine obsahujúcej žiakov $A$, $B$, $C$, $D$, a~keď žiaka~$Y$ vymeníme s~$X$,
bude mať vo svojej novej skupine za kamaráta~$K$.

Ukázali sme teda, že výmenou žiakov $X$ a~$Y$ počet zle zaradených žiakov klesol.
Dostali sme nejaké nové rozdelenie; ak v~ňom je aspoň jeden žiak zle zaradený, môžeme zopakovať predošlý postup a~opäť
znížiť počet zle zaradených žiakov. Po nanajvýš ôsmich krokoch dostaneme rozdelenie, v~ktorom už nie sú žiadni zle zaradení žiaci.

\medskip\noindent
{\bf Iné riešenie} časti b).
Uvažujme všetky možné rozdelenia žiakov na dve štvorčlenné skupiny.
Rozdelenia, kde niekto nemá vo svojej skupine žiadneho kamaráta, budeme nazývať {\it zlé}, ostatné budú {\it dobré}.

Koľko je zlých rozdelení? Ak má žiak~$X$ aspoň päť kamarátov, aspoň jeden z~nich musí byť v~jeho skupine.
Ak má žiak~$X$ iba štyroch kamarátov, a~všetci sú v~druhej skupine, máme len jedno jediné rozdelenie s~touto vlastnosťou.
Celkovo teda k~danému žiakovi~$X$ existuje nanajvýš jedno rozdelenie, ktoré je zlé.
Za $X$ môžeme zobrať jedného z~8~rôznych žiakov, preto zlých rozdelení je nanajvýš~8 (niektoré sme možno zarátali viackrát).
Pritom všetkých rozdelení je ${7\choose 3}=35$, čiže aspoň 27 z~nich je dobrých.

\návody
V~istej triede má každý žiak aspoň jedného kamaráta. Ukážte, že vieme žiakov rozdeliť
na dve skupiny tak, že každý má v~druhej skupine aspoň jedného kamaráta.
[K~úlohe sa dá pristupovať viacerými poučnými spôsobmi, pozri
vzorové riešenie úlohy č.~5 v~3.~sérii zimnej časti korešpondenčného matematického seminára KMS, ročník 2005/6, {\tt http://kms.sk/archiv}.]

Každý zo šiestich žiakov istej triedy má medzi ostatnými piatimi aspoň troch kamarátov. Kamarátstvo je vzájomné.
Ukážte, že vieme týchto žiakov rozdeliť do dvoch (neprázdnych) skupín tak, že každý žiak má vo svojej skupine aspoň jedného kamaráta.
Vedeli by sme to spraviť aj vtedy, keby každý žiak mal presne dvoch kamarátov?
[Ak rozdelíme žiakov hocijakým spôsobom na dvojicu a~štvoricu, tak každý žiak zo štvorice má v~nej aspoň jedného kamaráta, lebo z~jeho aspoň troch kamarátov sú nanajvýš dvaja v~druhej skupine. Čiže stačí zobrať dvojicu kamarátov a~ostatných dať do druhej skupiny.
Ak má každý presne dvoch kamarátov, tiež vieme žiakov rozdeliť: vezmeme žiaka~$A$ a~jeho dvoch kamarátov $B$ a~$C$ a~všetkých ich dáme do prvej skupiny.
Zvyšní traja žiaci $D$, $E$, $F$ budú tvoriť druhú skupinu. Ak by niektorý žiak z~druhej skupiny, povedzme~$D$, mal za kamarátov $B$ aj $C$, tak žiaci $E$ a~$F$ budú mať nanajvýš po jednom kamarátovi. Preto $D$ má za kamaráta nanajvýš jedného z~$B$ a~$C$, nemôže sa kamarátiť s~$A$, čiže musí mať za kamaráta aspoň jedného zo žiakov $E$ a~$F$. Podobne to funguje pre žiakov $E$ a~$F$. O~situácii so šiestimi žiakmi, kde každý má presne dvoch kamarátov, vieme povedať dokonca viac. Ak si zakreslíme žiakov ako body a~kamarátsky vzťah reprezentujeme spojením bodov zodpovedajúcich dvom kamarátom, môžeme dostať len dva rôzne obrázky: dva trojuholníky, alebo šesťuholník (pri vhodnom rozmiestnení bodov v~rovine).]

\D
V~skupine $n$~ľudí ($n\ge 4$) sa niektorí poznajú. Vzťah "poznať sa" je vzájomný:
ak osoba~$A$ pozná osobu~$B$, tak aj $B$ pozná $A$ a~nazývame ich dvojicou známych.
\item{a)} Dokážte, že ak medzi každými štyrmi osobami sú aspoň štyri dvojice známych,
        tak každé dve osoby, ktoré sa nepoznajú, majú spoločného známeho.
\item{b)} Zistite, pre ktoré $n\ge 4$ existuje skupina osôb, v~ktorej sú medzi každými
   štyrmi osobami aspoň tri dvojice známych a~súčasne sa niektoré dve osoby
   ani nepoznajú, ani nemajú spoločného známeho.
\item{c)} Rozhodnite, či v~skupine šiestich osôb môžu byť v~každej štvorici práve tri
   dvojice známych a~práve tri dvojice neznámych.
\vpravo{[C-57-I-5]}

Istý panovník pozval na oslavu svojich narodenín $28$~rytierov. Každý z~rytierov mal
medzi ostatnými práve troch nepriateľov.
\item{a)} Ukážte, že panovník môže rytierov rozsadiť k~dvom stolom tak, aby každý rytier
         sedel pri rovnakom stole najviac s~jedným nepriateľom.
\item{b)} Ukážte, že v prípade ľubovoľného takéhoto rozsadenia sedí pri každom stole
         najviac $16$~rytierov.
\endgraf
(Nepriateľstvo je vzájomný vzťah: Ak $A$ je nepriateľom~$B$, tak aj $B$ je nepriateľom~$A$.)
\vpravo{[51-C-I-6]}
\endnávod
}

{%%%%%   C-I-5
Nerovnosť by bolo ľahké dokázať, ak by niektorý z~dvoch sčítancov na ľavej strane bol sám osebe aspoň taký,
ako pravá strana. Číslo $[a,b]$ je zjavne násobkom čísla~$a$. Ak $[a,b]\ge 2a$, tak $b[a,b]\ge 2ab$ a~v~zadanej nerovnosti platí dokonca ostrá nerovnosť,
lebo číslo $a(a,b)$ je kladné.
Ak $[a,b]<2a$, tak neostáva iná možnosť, ako $[a,b]=a$. To však nastane iba v~prípade, keď $b\mid a$. V~tomto prípade $(a,b)=b$ a~v~zadanej nerovnosti nastane rovnosť.

\ineriesenie
Označme $d = (a, b)$, takže $a = ud$ a~$b = vd$ pre nesúdeliteľné prirodzené čísla $u$, $v$.
Z~toho hneď vieme, že $[a, b] = uvd$. Keďže
$$
\align
a~\cdot (a, b) + b \cdot [a, b] &= ud^2 + uv^2d^2 = u(1 + v^2)d^2 ,\cr
                                    2ab &= 2uvd^2,
\endalign
$$
je vzhľadom na $ud^2 > 0$ nerovnosť zo zadania ekvivalentná s~nerovnosťou $1 + v^2 \ge 2v$,
čiže $(v-1)^2 \ge 0$, čo platí pre každé~$v$. Rovnosť nastane práve vtedy, keď $v = 1$, čiže $b\mid a$.

\ineriesenie
Označme $d=(a,b)$. Je známe, že $[a,b]\cdot (a,b)=ab$. Po vyjadrení $[a,b]$ z~tohto vzťahu,
dosadení do zadanej nerovnosti a~ekvivalentnej úprave dostaneme ekvivalentnú nerovnosť
$d^2+b^2\ge 2bd$, ktorá platí, lebo $(d-b)^2\ge 0$.
Rovnosť nastáva pre $d=b$, čiže v~prípade $b\mid a$.


\návody
Nech $d$ je najväčší spoločný deliteľ prirodzených čísel $a$ a~$b$.
Ukážte, že čísla $a/d$ a~$b/d$ sú celé a~nesúdeliteľné.

Dokážte, že pre ľubovoľné prirodzené čísla $a$, $b$ platí vzťah $[a,b]\cdot (a,b)=ab$.
[Úvaha o~exponentoch jednotlivých prvočísel, alebo štandardným spôsobom:
nech $d=(a,b)$, potom $a=xd$, $b=yd$ pre nesúdeliteľné $x$ a~$y$, čiže $[a,b]=xyd$.]

Ukážte, že výraz $[a,15]/a$, kde $a$ je prirodzené číslo, môže nadobúdať len štyri rôzne hodnoty,
ktoré sú všetky celočíselné. Koľko rôznych celočíselných hodnôt môže nadobudnúť výraz $[120, b]/2b$?
[Výraz $[60, b]/2b$ môže nadobudnúť celočíselné hodnoty $1$, $2$, $3$, $5$, $6$, $10$, $15$, $30$,
okrem toho nadobúda hodnoty $1/2$, $3/2$, $5/2$, $15/2$.]

Dokážte, že pre kladné reálne čísla $a$, $b$ platí
$$
4ab\le (a+b)^2\le 2(a^2+b^2).
$$
[Obe nerovnosti sa dajú priamočiaro ukázať z~toho, že štvorec reálneho čísla je nezáporný.]

\D
Nájdite všetky trojice prirodzených čísel $a$, $b$, $c$, pre ktoré súčasne platí $[ab,c]=2^8$, $[bc,a]=2^9$, $[ca,b]=2^{11}$.
\vpravo{[50-C-S-1]}

Nájdite všetky dvojice prirodzených čísel $a$, $b$, pre ktoré platí $[a, b] + (a, b) = 63$.
\vpravo{[50-C-I-3]}

Nájdite všetky dvojice kladných celých čísel $a$, $b$, pre ktoré má výraz
$$
\displaystyle {a\over b}+{14b\over 9a}
$$
celočíselnú hodnotu.
[Nech $d=(a,b)$, potom $a=xd$, $b=yd$ pre nesúdeliteľné $x$ a~$y$. Skúmaný výraz bude po dosadení $(9x^2+14y^2)/(9xy)$, takže $9x\mid 14y^2$
a~z~nesúdeliteľnosti $x$ a~$y$ máme $x\mid 14$, navyše $3\mid y$. Podobne $y\mid 9$; vyskúšame konečne veľa možností.]

Dokážte, že pre ľubovoľné rôzne kladné čísla $a$, $b$ platí
$$
{a+b\over 2} < {2(a^2+ab+b^2)\over 3(a+b)} < \sqrt{a^2+b^2\over 2}.
$$
\vpravo{[58-C-I-6]}
\endnávod
}

{%%%%%   C-I-6
\epsplace c60.4 \hfil\Obr\par
\epsplace c60.5 \hfil\Obr\par
a)
Priamky $AB$, $CD$ a $KL$ sú rovnobežné, preto v~našej situácii vieme
nájsť viacero dvojíc podobných trojuholníkov (sú podobné podľa vety {\it uu}).
Tieto podobnosti vieme výhodne zapísať pomocou pomerov vzdialeností,
čo využijeme v dôkaze toho, že úsečky $KL$ a~$MN$ majú rovnakú dĺžku.

Označme $x$ vzdialenosť priamok $AB$ a~$KL$ a~$y$ vzdialenosť priamok $KL$ a~$CD$.
Tieto vzdialenosti nám umožnia vyjadriť koeficient podobnosti trojuholníkov --
tento koeficient je rovný nielen pomeru zodpovedajúcich si strán, ale aj zodpovedajúcich si výšok.

Trojuholníky $APD$ a~$KLD$ sú podobné, preto
$$
{|KL|\over |AP|}={y\over x+y}.
$$
Aj trojuholníky $BPC$ a~$NMC$ sú podobné, preto
$$
{|MN|\over |PB|}={y\over x+y}.
$$
Celkovo dostávame
$$
{|KL|\over |AP|}={y\over x+y}={|MN|\over |PB|},
$$
a~keďže $|AP|=|PB|$, máme $|KL|=|MN|$.

\smallskip
b) Chceme zostrojiť bod~$L$ taký, že $|KL|=|LM|$.
Rozoberieme dva prípady podľa toho, či je priamka $PC$ rovnobežná s priamkou $AD$, alebo nie.

Ak je priamka~$PC$ rovnobežná s~$AD$,
tak štvoruholník $APCD$ je rovnobežník a~jediný vyhovujúci bod~$L$ je stred úsečky~$PD$, čiže priesečník uhlopriečok rovnobežníka $APCD$ (podmienka $|KL|=|LM|$ tu vyjadruje zhodnosť trojuholníkov $KLD$ a~$MLP$, ktorá nastane práve vtedy, keď $|LD|=|LP|$, \obr).

\twocpictures

Ak sa priamky $PC$ a~$AD$ pretínajú v~nejakom bode~$R$ (\obr),
tak bod~$L$ bude priesečníkom úsečky~$DP$ s~priamkou, na ktorej leží ťažnica trojuholníka $APR$.
Požadovaná vlastnosť $|KL|=|LM|$ vyplýva z~toho, že rovnoľahlosť so stredom v~bode~$R$
zobrazujúca úsečku~$AP$ na úsečku~$KM$ zobrazí stred úsečky~$AP$ na stred úsečky~$KM$.

Z~uvedených konštrukcií vyplýva, že vyhovujúci bod~$L$ je vždy jediný,
čiže vieme skonštruovať práve jednu rovnobežku s~priamkou~$AB$ s~vyhovujúcimi vlastnosťami.

\poznamka
Ako sme uviedli, v~prípade, že priamky $PC$ a~$AD$ sú rovnobežné, bude vyhovujúcim bodom~$L$ priesečník uhlopriečok rovnobežníka $APCD$.
Ak priamky $PC$ a~$AD$ rovnobežné nie sú, štvoruholník $APCD$ už nebude rovnobežník, ale jeho priesečník uhlopriečok je výborným kandidátom
na bod~$L$. Výpočtom s~využitím podobnosti sa dá ukázať, že je to naozaj tak a~jediným vyhovujúcim bodom~$L$ je priesečník uhlopriečok lichobežníka $APCD$.

\návody
V~lichobežníku $ABCD$ s~priesečníkom uhlopriečok~$P$ zostrojíme rovnobežku so základňou~$AB$
prechádzajúcu bodom~$P$. Táto priamka pretne ramená $AD$ a~$BC$ v~bodoch $K$ a~$L$. Ukážte, že bod~$P$
je stredom úsečky~$KL$. Vypočítajte dĺžku úsečky~$KL$, ak viete, že $|AB|=a$, $|CD|=c$.
[Využijeme podobnosť dvojíc trojuholníkov
$DKP$ a~$DAB$, $CPL$ a~$CAB$, $PAB$ a~$PCD$. Ak označíme $v_1$ výšku trojuholníka $PAB$
a~$v_2$ výšku trojuholníka $PCD$, tak $|KP| = |LP| = a\cdot v_2/(v_1+v_2)$.
Z~toho $|KL| = 2ac/(a+c)$.]

Daný je lichobežník $ABCD$ s~dlhšou základňou~$AB$.
Nech $X$, $Y$ sú po rade priesečníky dvojíc priamok $AD$ a~$BC$, $AC$ a~$BD$.
Dokážte, že body $X$, $Y$ a~stredy základní lichobežníka $ABCD$ ležia na jednej priamke.
[Rovnoľahlosť so stredom v~bode~$X$ zobrazujúca úsečku~$AB$ na úsečku~$CD$
zobrazí stred jednej základne do stredu druhej základne, preto stredy
základní a~bod~$X$ ležia na priamke. Analogicky stredy základní
a~bod~$Y$ ležia na priamke. Je vhodné spraviť aj riešenie využívajúce
len podobnosť trojuholníkov bez spoliehania sa na vlastnosti rovnoľahlosti.]

\D
Daný je lichobežník $ABCD$ so základňami $AB$ a~$CD$. Označme $E$ stred strany~$AB$,
$F$~stred úsečky~$DE$ a~$G$ priesečník úsečiek $BD$ a~$CE$. Vyjadrite obsah lichobežníka
$ABCD$ pomocou jeho výšky~$v$ a~dĺžky~$d$ úsečky~$FG$ za predpokladu, že body $A$, $F$, $C$
ležia na jednej priamke.
\vpravo{[56-C-I-4]}

Zostrojte lichobežník $ABCD$ s~výškou $3\cm$ a~zhodnými stranami $BC$, $CD$ a~$DA$,
pre ktorý platí: Na základni~$AB$ existuje bod~$E$ taký, že úsečka~$DE$ má dĺžku $5\cm$ a~delí
lichobežník na dve časti s~rovnakými obsahmi.
\vpravo{[52-C-I-4]}
\endnávod
}

{%%%%%   A-S-1
Predpokladajme, že číslo~$c$ má požadovanú vlastnosť. Diferenciu príslušnej aritmetickej
postupnosti označme~$d$. Rozoberieme dva prípady podľa toho,
či číslo~$c$ leží medzi koreňmi $x_1$ a~$x_2$ danej kvadratickej rovnice, alebo nie.

\smallskip
a) Ak je $c$ prostredným členom predpokladanej aritmetickej postupnosti,
platí $x_1=c-d$ a~$x_2=c+d$. Pre súčet koreňov tak podľa Vi\`etových vzťahov
dostávame $\m\frac52=x_1+x_2=2c$, odkiaľ $c=\m\frac54$. Navyše pre záporné~$c$
je diskriminant danej rovnice kladný, takže má dva reálne korene.
(Pre $c=\m\frac54$ má daná rovnica
korene $x_{1,2}=\m\frac54\pm\frac34\sqrt{5}$.)

\smallskip
b) Ak je koeficient~$c$ krajným členom predpokladanej aritmetickej postupnosti,
označme korene danej rovnice tak, aby platilo
$x_1=c+d$, $x_2=c+2d$. Pre ich súčet %kořenů dané kvadratické rovnice
teraz vychádza
$\m\frac52=x_1+x_2=2c+3d$. Ak z~toho vyjadríme $d={\m\frac56}-\frac23c$ a~dosadíme do
vzťahov $x_1=c+d$ a~$x_2=c+2d$, dostaneme
$x_1=\frac16(2c-5)$, $x_2=\m\frac13(c+5)$. Po dosadení oboch výrazov do
Vi\`etovho vzťahu $x_1x_2=c$ %dané kvadratické rovnice,
obdržíme po
úprave kvadratickú rovnicu $2c^2+23c-25=0$, ktorá má korene $1$ a~$\m\frac{25}2$.
(Podmienku na diskriminant tentoraz overovať
nemusíme, lebo uvedeným postupom máme zaručené, že reálne čísla $x_{1,2}$
zodpovedajúce obom nájdeným hodnotám~$c$ spĺňajú oba Vi\`etove vzťahy,
takže sú naozaj koreňmi príslušnej rovnice. Pre $c=1$ má daná kvadratická rovnica korene
$x_1=\m\frac12$, $x_2=\m2$; pre $c=\m\frac{25}2$ má rovnica korene
$x_1=\m5$, $x_2=\frac52$.)

\zaver
Úlohe vyhovujú reálne čísla~$c$ z~množiny $\{\m\frac{25}2;\m\frac54;1\}$.

\nobreak\medskip\petit\noindent
Za úplné vyriešenie úlohy dajte 6~bodov. Za nájdenie každého z~troch
riešení úlohy dajte po 2~body. Body pritom
nestrhávajte, ak súťažiaci neurčí pre nájdené hodnoty koeficientu~$c$
korene príslušnej kvadratickej rovnice, lebo to úloha nevyžaduje.
\endpetit
\bigbreak
}

{%%%%%   A-S-2
\epsplace a60.9 \hfil\Obr\par
\epsplace a60.8 \hfil\Obr\par
Označme $M'$ stred úsečky~$CQ$ (\obr).
Keďže $PM'$ a~$RM'$ sú stredné priečky trojuholníkov $AQC$ a~$BQC$, ktoré sú podľa Tálesovej vety
rovnoramenné so základňami $AC$ a~$BC$, sú štvoruholníky
$CAPM'$ a~$CBRM'$ rovnoramenné lichobežníky a~im
opísané kružnice, ktoré sa pretínajú v~bodoch $C$ a~$M'$, sú zároveň aj opísanými
kružnicami uvažovaných trojuholníkov $APC$ a~$BRC$. Takže $M=M'$ a~tvrdenie úlohy je tým dokázané.
\inspicture

\ineriesenie
Označme $c$ dĺžku prepony~$AB$ daného pravouhlého trojuholníka $ABC$.
Kružnice opísané trojuholníkom $APC$ a~$BRC$ označme postupne $k$,
$l$ (\obr). Vzhľadom na to, že ${|QP|\cdot|QA|}={|QR|\cdot|QB|}=\frac14c\cdot
\frac12c$, má stred~$Q$ prepony~$AB$ rovnakú mocnosť $m=\frac14c\cdot
\frac12c$ k~obom kružniciam $k$ aj $l$,
a~leží preto na ich chordále~$CM$. Navyše podľa Tálesovej vety
%% ($ABC$ je pravoúhlý \tr-)
platí $|QC|=|QA|=\frac12c$. Z~rovnosti $|QM|\cdot|QC|=m$ tak vyplýva
$|QM|=\frac14c=\frac12|QC|$, takže $M$ je stredom úsečky~$CQ$.
\inspicture


\nobreak\medskip\petit\noindent
Za úplné vyriešenie úlohy dajte 6~bodov. Za čiastočné pozorovania, ktoré
vedú k~riešeniu úlohy, dajte najviac 3~body.
\endpetit
\bigbreak
}

{%%%%%   A-S-3
Keďže $p$, $q$ sú rôzne {\it nepárne\/} prvočísla, platí $|p-q|\ge2$.
% Postupnou úpravou levé strany dané nerovnosti
% (s~využitím nerovnosti mezi aritmetickým a~geometrickým průměrem pro dvojici
% $p$, $q$ různých prvočísel) máme
Pre ľavú stranu danej nerovnosti teda máme
$$
L=\bigg|\frac{p}{q}-\frac{q}{p}\bigg|=\bigg|\frac{p^2-q^2}{pq}\bigg|
     =\frac{|p-q|\cdot(p+q)}{pq}\ge
    \frac{2\,(p+q)}{pq}.
%     >\frac{4\,\sqrt{pq}}{pq}=\frac{4}{\sqrt{pq}}.
$$
Aby sme dokázali požadovanú nerovnosť
$$
L>\dfrac{4}{\sqrt{pq}},
$$
stačí dokázať nerovnosť $p+q>2\sqrt{pq}$. To je však nerovnosť, ktorá
je triviálnym dôsledkom nerovnosti $(\sqrt p-\sqrt q)^2>0$, ktorá platí pre
ľubovoľné dve rôzne kladné čísla $p$, $q$.
Tým je daná nerovnosť dokázaná.

\nobreak\medskip\petit\noindent
Za úplné vyriešenie úlohy dajte 6~bodov, z~toho 1~bod za správnu úpravu výrazu na ľavej strane nerovnosti, ďalej 2~body za využitie
podmienky $|p-q|\ge 2$ a~ďalšie 3~body za~dokončenie dôkazu využitím uvedenej "ostrej"
nerovnosti či prípadnej nerovnosti medzi aritmetickým a~geometrickým priemerom dvojice
rôznych kladných čísel $p$, $q$.
\endpetit
}

{%%%%%   A-II-1
Najskôr určíme počet $u$ všetkých osemciferných čísel deliteľných štyrmi.
Každé také číslo má vo svojom zápise na prvom mieste zľava nenulovú
cifru. Máme tak 9~možností. Na nasledujúcich piatich miestach má ľubovoľnú cifru
desiatkovej sústavy, \tj. pre každú pozíciu máme 10~možností, a~končí
dvojčíslím, ktoré je deliteľné štyrmi, \tj. $00$, $04$, $08$, $12$, $16$, $20$,
$24$, \dots, $96$, celkom teda 25~možností. Preto
$$
u=9\cdot10^5\cdot25=22\,500\,000.
$$

Podobnú úvahu možno urobiť aj pri hľadaní počtu $v$ všetkých osemciferných čísel
deliteľných štyrmi, ktoré vo svojom desiatkovom zápise {\it neobsahujú\/} cifru~$1$.
Pre prvú pozíciu zľava máme teraz 8~možností a~pre každú ďalšiu z~piatich
nasledujúcich pozícií máme 9~možností. Na posledných dvoch miestach sprava musí
byť dvojčíslie deliteľné štyrmi, ktoré však neobsahuje cifru~$1$. Sú to
všetky dvojčíslia z~predchádzajúceho odseku okrem $12$ a~$16$, teda 23~možností.
Preto
$$
v=8\cdot9^5\cdot23=10\,865\,016.
$$

\zaver
Keďže $u>2v$, je medzi osemcifernými násobkami čísla~$4$ viac
tých, ktoré vo svojom (desiatkovom) zápise cifru~$1$ obsahujú,
ako tých, ktoré ju neobsahujú.

\poznamka
Počet $u$ všetkých osemciferných násobkov čísla $4$
možno určiť aj jednoduchou úvahou: najmenší násobok je $A=10\,000\,000$,
najväčší je $B=99\,999\,996$, takže hľadaný počet je
$\frac14(B-A)+1=\frac14(B+4-A)=22\,500\,000$.
% $\frac14(B-A)+1=22\,499\,999+1$.

Na dôkaz nerovnosti $u>2v$ nie je nutné $v$ vyčísliť, pretože podiel
$$
\frac uv=\frac{9\cdot 10^5 \cdot 25}{8\cdot 9^5\cdot
23}=\frac98\cdot\Bigl(\frac{10}9\Bigr)^{\!5}\cdot\frac{25}{23}
$$
sa dá dobre odhadnúť pomocou binomickej vety
$$
\Bigl(\frac{10}9\Bigr)^{\!5}=\Bigl(1+\frac{1}9\Bigr)^{\!5}>1+5\cdot\frac19+10\cdot
\frac1{9^2}=
    \frac{136}{81}=\frac{8\cdot17}{9^2},
$$
takže
$$
\frac uv=\frac98\cdot\Bigl(\frac{10}9\Bigr)^{\!5}\cdot\frac{25}{23}>
   \frac98\cdot \frac{8\cdot17}{9^2}\cdot
\frac{25}{23}=\frac{17\cdot25}{9\cdot23}=
   \frac{425}{207}>2.
$$

\nobreak\medskip\petit\noindent
Za úplné riešenie dajte 6~bodov, z~toho 2~body za správne
vyjadrenie~$u$, 3~body za správne vyjadrenie~$v$ a~1~bod za dôkaz nerovnosti
$u>2v$ alebo nerovnosti s~ňou ekvivalentnou. Za
numerické chyby pri výpočte strhnite nanajvýš 1~bod.

\endpetit
\bigbreak
}

{%%%%%   A-II-2
\epsplace a60.12 \hfil\Obr\par
\epsplace a60.34 \hfil\Obr\par
\epsplace a60.10 \hfil\Obr\par
\epsplace a60.11 \hfil\Obr\par
Označme $\Cal U$ trojuholník s~vrcholmi v~stredoch strán $BC$, $CA$, $AB$
daného trojuholníka $ABC$.
Obsah trojuholníka $XYZ$ budeme označovať $S_{XYZ}$.

%% Uvažujme označení dle~\obr.
Keďže body $A'$, $B'$, $C'$ sú zároveň obrazmi bodu~$O$ v~rovnoľahlostiach
so stredmi v~zodpovedajúcich vrcholoch trojuholníka $ABC$ a~koeficientom~$2$, vyplýva z~predpokladu úlohy,
že body $A'$, $B'$, $C'$ ležia postupne vnútri trojuholníkov $A_0CB$, $CB_0A$ a~$BAC_0$
(to sú obrazy trojuholníka $\Cal U$
%% s~vrcholy ve středech stran trojúhelníku $ABC$
v~uvedených rovnoľahlostiach, na \obr{} je $\Cal U$ vyznačený sivou farbou).
Hranica trojuholníka $A'B'C'$ teda pretne strany $AB$, $BC$, $CA$ postupne
v~ich vnútorných bodoch $D$, $E$, $F$, $G$, $H$, $I$.
\inspicture{}
% \midinsert
% \inspicture{}
% \endinsert

Keďže trojuholník $A'B'C'$ je obrazom trojuholníka $ABC$
v~stredovej súmernosti podľa stredu~$O$, sú
navzájom si zodpovedajúce strany rovnobežné a~v~tej istej súmernosti si zodpovedajú
dvojice bodov $D$ a~$G$, $E$ a~$H$ aj $F$ a~$I$. Preto podľa vety {\it uu\/} je každý
z~trojuholníkov $ADI$, $EBF$, $HGC$ podobný
trojuholníku $ABC$. Označme $k_1$, $k_2$, $k_3$ koeficienty podobností,
ktoré zobrazia trojuholník $ABC$ postupne na trojuholníky $ADI$, $EBF$,
$HGC$. Obrazy trojuholníkov $ADI$, $EBF$, $HGC$ v~stredovej súmernosti podľa
stredu~$O$ sú postupne trojuholníky $A'GF$, $HB'I$, $EDC'$.
Tie sú tiež podobné s~trojuholníkom $ABC$, pričom zodpovedajúce koeficienty
podobností, ktoré na ne zobrazia trojuholník $ABC$,
%% na trojúhelníky $A'NM$, $OB'P$, $LKC'$,
sú opäť $k_1$, $k_2$, $k_3$. Ak označíme $c$ dĺžku strany~$AB$,
%% je z~obr. 3 patrné, že
platí pre dĺžky úsekov na strane~$AB$
$$
c_1=|AD|=k_1c,\quad
c_2=|EB|=k_2c,\quad
c_3=|DE|=k_3c,
$$
takže
$$
c=c_1+c_2+c_3=k_1c+k_2c+k_3c=(k_1+k_2+k_3)c,\quad \hbox{odkiaľ} \quad k_1+k_2+k_3=1.
$$
Z~podobnosti trojuholníkov $ABC$ a~$EDC'$ ďalej vyplýva, že veľkosť výšky
z~vrcholu~$C'$ na stranu~$AB$ v~trojuholníku $ABC'$ je rovná $k_3v_c$, pričom
$v_c$ je veľkosť výšky z~vrcholu~$C$ v~trojuholníku $ABC$.
Preto
$$
S_{ABC'}=\frac12 c\cdot k_3v_c=k_3\Bigl(\frac12 cv_c\Bigr)=k_3S.
$$
Analogicky $S_{BCA'}=k_1S$ a~$S_{CAB'}=k_2S$. Pre obsah $S'$ šesťuholníka
$AC'BA'CB'$ tak platí
$$
S'=S_{ABC}+S_{BCA'}+S_{CAB'}+S_{ABC'}=(1+k_1+k_2+k_3)S=2S.
$$

\ineriesenie
Označme $K$, $L$, $M$ stredy strán $AB$, $BC$, $CA$. Rovnoľahlosť so stredom~$A$ a~koeficientom~$2$ zobrazí trojuholník $MKO$ na trojuholník $CBA'$ (\obr), preto $S_{CBA'}=4\cdot S_{MKO}$. Podobne $S_{ACB'}=4\cdot S_{KLO}$ a $S_{BAC'}=4\cdot S_{LMO}$. Odtiaľ
$$
S_{CBA'}+S_{ACB'}+S_{BAC'}=4\cdot S_{KLM}=S,
$$
a teda šesťuholník $AC'BA'CB'$ má obsah $2S$.
\inspicture{}


\ineriesenie
Ak je bod~$O$ totožný s~ťažiskom~$T$ trojuholníka $ABC$ ($O=T$), je
tvrdenie úlohy splnené, lebo $S_{A'BC}=S_{TBC}$,
$S_{B'CA}=S_{TCA}$ a~$S_{C'AB}=S_{TAB}$ (\obr).


Predpokladajme teraz, že bod~$O$ sa pohybuje vnútri trojuholníka $\Cal U$ po priamke~$p$
rovnobežnej so stranou~$BC$. Ukážeme, že sa pritom
obsah šesťuholníka $AC'BA'CB'$ nemení.
Body $A'$, $B'$ a~$C'$ ležia na rovnobežkách s~priamkou~$p$, preto sa nemení obsah trojuholníka $A'BC$ ani obsahy rovnobežníka
$BCB'C'$ a~trojuholníka $B'C'A$ (\obr). Obsah šesťuholníka $AC'BA'CB'$
teda od polohy bodu~$O$ na priamke~$p$ nezávisí. Podobne možno ukázať, že sa obsah šesťuholníka
$AC'BA'CB'$ nemení, ani keď sa bod~$O$ pohybuje po rovnobežke so stranou~$AC$.
\twopictures

Ľubovoľný vnútorný bod~$O$ trojuholníka~$\Cal U$ pritom získame ako obraz
ťažiska~$T$ trojuholníka $ABC$ v~zobrazení zloženom
z~dvoch posunutí, a~to z~posunutia v~smere rovnobežnom so stranou~$BC$
a~z~posunutia v~smere rovnobežnom so stranou~$AC$.
Preto pre každý bod~$O$ vnútri trojuholníka~$\Cal U$ má šesťuholník
$AC'BA'CB'$ rovnaký obsah ako šesťuholník zodpovedajúci bodu $O=T$,
teda obsah~$2S$, čo sme chceli dokázať.

\ineriesenie
Označme $O'$ ľubovoľný vnútorný bod trojuholníka $ABC$. Keďže obsah
skúmaného šesťuholníka je rovný súčtu obsahov troch štvoruholníkov
$AC'BO'$, $BA'CO'$ a~$CB'AO'$, bude tvrdenie úlohy zrejme platiť, ak
dokážeme bod~$O'$ vybrať tak, aby všetky tri spomenuté
štvoruholníky boli rovnobežníky. Keďže
$$
O=\frac{A+A'}{2}=\frac{B+B'}{2}=\frac{C+C'}{2},
$$
majú body $A'$, $B'$, $C'$ vyjadrenia
$$
A'=2O-A,\quad B'=2O-B,\quad C'=2O-C,
$$
takže potrebné rovnosti
$$
\frac{A+B}{2}=\frac{C'+O'}{2},\quad
\frac{B+C}{2}=\frac{A'+O'}{2},\quad
\frac{C+A}{2}=\frac{B'+O'}{2}
$$
budú splnené práve vtedy, keď bod~$O'$ bude mať vyjadrenie
$O'=A+B+C-2O$, čiže $O'=3T-2O$, pričom $T=\frac13(A+B+C)$ je
ťažisko trojuholníka $ABC$. Odvodená rovnosť zapísaná v~tvare $O'-T=2(T-O)$
znamená, že želaný bod~$O'$ je určený ako obraz bodu~$O$ v~rovnoľahlosti
so stredom~$T$ a~koeficientom $\m2$. V~nej je však
obrazom trojuholníka~$\Cal U$ %%s~vrcholy ve středech stran \tr-u $ABC$
pôvodný trojuholník $ABC$, takže vnútorný bod~$O$ trojuholníka~$\Cal U$ sa
naozaj zobrazí na vnútorný bod~$O'$ trojuholníka~$ABC$, ako sme
potrebovali dokázať.


\nobreak\medskip\petit\noindent
Za úplné riešenie dajte 6~bodov.
Za uvedenie (a~zdôvodnenie) akýchkoľvek
čiastočných výsledkov, ktoré vedú na úplné riešenie úlohy,
dajte v~súčte najviac 4~body.
\endpetit
\bigbreak
}

{%%%%%   A-II-3
Najskôr si uvedomme, že s~každou dvojicou $(m,n)$ prirodzených čísel, ktorá úlohe
vyhovuje, jej vyhovuje aj dvojica $(n,m)$.
Preto môžeme bez ujmy na všeobecnosti predpokladať, že $m\ge n$.

Ak prirodzené číslo $A=(m+n)^2$ delí prirodzené číslo $B=4(mn+1)$,
nutne platí
$$
(m+n)^2\le 4(mn+1),\qquad \hbox{čiže}\qquad (m-n)^2\le4.
$$
Preto $0\le m-n\le 2$. Nastane teda práve jedna z~troch nasledujúcich
možností:
\item{$\triangleright$}
$m=n$, potom $A=4n^2$, $B=4n^2+4$ a~$A$ delí $B$
práve vtedy, keď $4n^2$ delí~$4$, teda $n=1$. Dostávame jedno riešenie
$(m,n)=(1,1)$.
\item{$\triangleright$}
$m=n+1$, potom $A=4n^2+4n+1$, $B=4n^2+4n+4=A+3$.
Číslo $A$ delí $B$ práve vtedy, keď $4n^2+4n+1$ delí~$3$. Avšak pre kladné
celé čísla~$n$ platí ${4n^2+4n+1}\ge 4+4+1=9$, preto v~tomto
prípade nemá úloha riešenie.
\item{$\triangleright$}
$m=n+2$, potom $A=4n^2+8n+4$, $B=4n^2+8n+4$.
Vidíme, že $A=B$, teda každá dvojica $(m,n)=(n+2,n)$ kladných celých čísel
je riešením zadanej úlohy.

\zaver
Úlohe vyhovuje dvojica $(1,1)$ a~ďalej (vzhľadom na symetriu
neznámych $m$, $n$) každá z~dvojíc $(k+2,k)$ a~$(k,k+2)$, pričom $k$ je ľubovoľné prirodzené číslo.


\nobreak\medskip\petit\noindent
Za úplné riešenie dajte 6~bodov, z~toho 3~body za odvodenie nerovnosti
$(m-n)^2\le4$. Ďalej dajte 1~bod za nájdenie riešenia $(1,1)$ (1~bod strhnite
za jeho zabudnutie v~inak úplnom riešení) a~2~body za
nájdenie všetkých zostávajúcich riešení. Za uvedenie všetkých riešení bez akéhokoľvek
zdôvodnenia dajte nanajvýš 2~body.
\endpetit
\bigbreak
}

{%%%%%   A-II-4
Označme steny kocky $S_1$, $S_2$, \dots, $S_6$ tak, že stena $S_1$ je
oproti stene $S_6$, stena $S_2$ oproti $S_5$ a~$S_3$ oproti $S_4$.
Číslo na stene $S_i$ označme $c_i$.
Zrejme ľubovoľný vrchol kocky
patrí vždy práve jednej stene z~dvojice protiľahlých stien. %$S_1$ a~$S_6$, $S_2$ a~$S_5$, $S_3$ a~$S_4$.
To znamená, že sa v~každom kroku zväčší o~$1$ aj hodnota súčtov
$c_1+c_6$, $c_2+c_5$ a~$c_3+c_4$ čísel na protiľahlých stenách.
Ak má teda na konci platiť $c_1=c_2=c_3=c_4=c_5=c_6$, čiže aj
$$
c_1+c_6=c_2+c_5=c_3+c_4,         \tag1
$$
musia byť súčty čísel na protiľahlých stenách kocky rovnaké už na
začiatku (a~zostanú rovnaké aj po každom kroku).

Ukážeme, že podmienka \thetag{1} je zároveň postačujúca. Nech teda čísla na
stenách kocky spĺňajú \thetag{1}.
Opíšeme postupnosť krokov, po ktorých budú na všetkých stenách kocky rovnaké
čísla. Krok, v~ktorom zväčšíme čísla na stenách $S_i$, $S_j$, $S_m$,
označme $k_{ijm}$. Bez ujmy na všeobecnosti nech $c_1=p$ je najväčšie zo šiestich
čísel na kocke. Urobíme $(p-c_2)$-krát krok $k_{246}$
a~$(p-c_3)$-krát krok $k_{356}$. Dosiahneme tak, že na stenách $S_1$,
$S_2$, $S_3$ budú rovnaké čísla~$p$. Vďaka
podmienke~\thetag1 je aj na stenách $S_4$, $S_5$, $S_6$
rovnaké číslo, ktoré označme~$q$. Ak ešte neplatí $p=q$, stačí už len
$(p-q)$-krát urobiť krok $k_{456}$ (ak $p>q$), resp. $(q-p)$-krát krok
$k_{123}$ (ak $q>p$).

Našou úlohou je teda určiť počet takých množín
$\mm M=\{c_1,c_2,c_3,c_4,c_5,c_6\}$ navzájom rôznych prirodzených čísel, pre ktoré
platí
$$
c_1+c_2+c_3+c_4+c_5+c_6=60\quad \hbox{a} \quad c_1+c_6=c_2+c_5=c_3+c_4.
$$
Odtiaľ vyplýva $3(c_1+c_6)=60$, teda
$$
c_1+c_6=c_2+c_5=c_3+c_4=20. \tag{2}
$$
Bez ujmy na všeobecnosti môžeme zrejme predpokladať, že $c_1<c_2<c_3<c_4<c_5<c_6$,
%% Protože $20/2=10$, platí
čiže (vzhľadom na rovnosti~\thetag2)
$$
c_1<c_2<c_3<10<c_4<c_5<c_6.
$$
Pritom ku každej trojici $(c_1,c_2,c_3)$ spĺňajúcej $c_1<c_2<c_3<10$ zvyšné
čísla $c_4$, $c_5$, $c_6$ jednoznačne dopočítame z~\thetag{2}.
Počet všetkých vyhovujúcich množín $\mm M$ je teda rovný počtu
rôznych trojíc prirodzených čísel $(c_1,c_2,c_3)$, ktoré
spĺňajú $c_1<c_2<c_3<10$, čo je
$$
{9 \choose 3}=\frac{9\cdot8\cdot7}{1\cdot2\cdot3}=84.
$$

\nobreak\medskip\petit\noindent
Za úplné riešenie dajte 6 bodov, z~toho 2 body za nájdenie a~správne
zdôvodnenie nutnej podmienky \thetag{1}; za dôkaz, že \thetag1 je aj postačujúcou podmienkou,
dajte 2~body. Ďalšie 2~body dajte za určenie počtu množín~$\mm M$.
\endpetit
\bigbreak
}

{%%%%%   A-III-1
\epsplace a60.13 \hfil\Obr\par
Štvoruholník $KBCM$ je tetivový práve vtedy, keď
$|\uhol CMB|=|\uhol CKB|$, čiže $|\uhol AKL|=|\uhol AML|$ (\obr).
Pritom štvoruholník $AKLM$ je tetivový práve vtedy, keď $|\uhol AKL|+|\uhol AML|=180\st$.
V~skúmanom prípade preto musia byť všetky
štyri uvedené uhly pravé, $K$ a~$M$ sú tak päty výšok v~trojuholníku $ABC$, ktorý
je teda ostrouhlý, a~bod~$L$ je priesečníkom jeho výšok. Kružnica opísaná
štvoruholníku $KBCM$ je Tálesovou kružnicou nad priemerom~$BC$
a~kružnica opísaná štvoruholníku $AKLM$ je Tálesovou kružnicou nad
priemerom~$AL$.
\inspicture

Kružnice opísané uvedeným štvoruholníkom sú zhodné práve vtedy, keď
sú zhodné ich priemery $BC$ a~$AL$.
Označme veľkosti vnútorných uhlov v~trojuholníku $ABC$
zvyčajným spôsobom $\alpha$, $\beta$, $\gamma$.
Pravouhlé trojuholníky $CKB$ a~$AKL$ sú podobné, lebo
pre ich uhly pri zodpovedajúcich vrcholoch $C$ a~$A$ platí
$|\uhol BAL|=|\uhol BCK|=90\st-\beta$. Zrejme preto $|BC|=|AL|$ platí
práve vtedy, keď $|AK|=|CK|$, teda keď $AKC$ je pravouhlý rovnoramenný trojuholník.

Vidíme, že trojuholník $ABC$ vyhovuje podmienkam úlohy práve vtedy, keď je ostrouhlý
s~uhlom $\alpha=45\st$. Pre {\it ostré\/} uhly $\beta$ a~$\gamma$ potom platí $\beta+\gamma=135\st$.

\zaver
Riešením sú trojice uhlov $(\alpha,\beta,\gamma)=(45\st,45\st+\phi,90\st-\phi)$, pričom
$\phi\in(0\st,45\st)$.

\poznamka
Druhú časť riešenia možno založiť aj na úvahe, že v~zadaní uvedené kružnice sú zhodné práve vtedy, keď obvodové uhly $MAK$ a~$MCK$ nad ich spoločnou tetivou~$MK$ (s~vrcholmi v~opačných polrovinách určených priamkou~$MK$) majú rovnakú veľkosť. Keďže tieto uhly sú vnútornými uhlami pravouhlého trojuholníka $AKC$, rovnakú veľkosť majú práve vtedy, keď $\alpha=45\st$.

\ifrozsirenevzoraky
\epsplace a60.130 \hfil\Obr\par
\ineriesenie
Nech $K$, $M$ sú vnútorné body strán $AB$, $AC$ trojuholníka $ABC$, ktoré
vyhovujú podmienkam úlohy. Vzhľadom na to, že
kružnice opísané štvoruholníkom $AKLM$, $KBCM$ sú zhodné, zhodujú sa
aj príslušné obvodové uhly nad spoločnou tetivou~$KM$ oboch kružníc\footnote{Dve zhodné kružnice
so spoločnou tetivou môžu byť buď totožné, alebo súmerne združené podľa spoločnej tetivy;
prvá možnosť tu neprichádza do úvahy.} (\obr):
$$
|\uhol KCM|=|\uhol KBM|=|\uhol KAM|=\alpha.
$$
Odtiaľ $|\uhol MBC|=\beta-\alpha$ a~$|\uhol KCB|=\gamma-\alpha$, a~preto $\alpha$
je najmenším vnútorným uhlom uvažovaného trojuholníka.
\inspicture

Keďže $AKLM$ je tetivový štvoruholník, je vnútorný uhol pri vrchole~$A$ zhodný
s~vedľajším uhlom pri protiľahlom vrchole~$L$, \tj. platí
$$
\alpha=(\beta-\alpha)+(\gamma-\alpha),
\quad\text{čiže}\quad
3\alpha=\beta+\gamma=180\st-\alpha.
$$
Odtiaľ vychádza $\alpha=45^{\circ}$. Trojuholník $ABM$ je teda rovnako ako trojuholník $ACK$
rovnoramenný pravouhlý, takže $CK$ a~$BM$ sú výšky trojuholníka $ABC$ a~bod~$L$ je jeho
priesečníkom výšok. A~keďže bod~$L$ leží vnútri trojuholníka $ABC$, je trojuholník $ABC$ ostrouhlý.

Vzhľadom na to, že $\alpha=45^{\circ}$ je najmenším z~vnútorných uhlov
trojuholníka $ABC$, ľahko nahliadneme, že hľadané trojice
$(\alpha,\beta,\gamma)$ majú tvar
$(45^{\circ},45^{\circ}+\varphi,90^{\circ}-\varphi)$,
pričom pre parameter~$\varphi$ platí $0^{\circ}<\varphi<45^{\circ}$.

Naopak, v~každom ostrouhlom trojuholníku $ABC$ s~uhlom~$45\st{}$ pri vrchole~$A$, pre ktorého
zvyšné uhly platí $\beta+\gamma=135\st$, majú zrejme päty výšok $K$, $M$ z~vrcholov $C$ a~$B$
požadované vlastnosti, pretože oba štvoruholníky $AKLM$, $KBCM$ sú tetivové
podľa Tálesovej vety a~z~rovnosti uhlov $|\uhol KCM|=|\uhol KAM|=45\st$ nad spoločnou
tetivou~$KM$ vyplýva, že im opísané kružnice sú zhodné.
\fi
}

{%%%%%   A-III-2
Ukážeme, že danej rovnici vyhovujú práve tri trojice prvočísel $(p,q,r)$,
a~to $(2,3,5)$, $(5,3,3)$ a~$(7,5,2)$.

\smallskip
Danú rovnicu najskôr upravíme na tvar
$$
\Bigl(1+\frac{1}{p}\Bigr)\Bigl(1+\frac{2}{q}\Bigr)\Bigl(1+\frac{3}{r}\Bigr)=4.
$$
Keďže $3^{3}<4\cdot 2^{3}$, musí byť aspoň jeden
z~troch činiteľov na ľavej strane upravenej rovnice väčší
ako~$\frac32$. Pre prvočísla $p$, $q$, $r$ tak nutne platí $p<2$ alebo $q<4$
alebo $r<6$. Vzhľadom na to, že neexistuje žiadne prvočíslo
menšie ako~$2$, zostáva prešetriť nasledujúcich päť možností: $q \in \{2,3\}$ a~$r\in \{2,3,5\}$.
Ty teraz rozoberieme jednotlivo, pritom uvažovanú hodnotu $q$ či $r$ vždy
dosadíme do danej rovnice, ktorú potom (v~obore prvočísel) vyriešime pre zostávajúce
dve neznáme.

\item{$\triangleright$} Pre $q=2$ dostaneme $(p+1)(r+3)=2pr$, odkiaľ vyplýva
      $r=3+\frc{6}{(p-1)}$, čo je celé číslo len pre prvočísla $p\in\{2,3,7\}$. Im však
        prislúchajú $r\in\{9,6,4\}$, ktoré nie sú prvočíslami.
\item{$\triangleright$} Pre $q=3$ dostaneme $5(p+1)(r+3)=12pr$, odkiaľ vyplýva, že $p=5$
        alebo $r=5$. Pre $p=5$ dostaneme riešenie $(5,3,3)$ a~pre $r=5$ riešenie $(2,3,5)$.
\item{$\triangleright$} Pre $r=2$ dostaneme $5(p+1)(q+2)=8pq$, odkiaľ vyplýva, že
      $p=5$ alebo $q=5$. Pre $p=5$
      nedostaneme žiadne riešenie v~obore prvočísel, zatiaľ čo pre $q=5$ dostávame tretie riešenie danej
      rovnice, ktorým je trojica $(7,5,2)$.
\item{$\triangleright$} Pre $r=3$ dostaneme $(p+1)(q+2)=2pq$, odkiaľ
      $q=2+\frc{4}{(p-1)}$, čo je celé číslo iba pre prvočísla
      $p\in\{2,3,5\}$. Medzi prislúchajúcimi hodnotami $q\in\{6,4,3\}$ je jediné prvočíslo,
      pre ktoré dostávame riešenie $(p,q,r)=(5,3,3)$, ktoré už poznáme.
\item{$\triangleright$} Pre $r=5$ dostaneme $2(p+1)(q+2)=5pq$, odkiaľ vyplýva, že
      $p=2$ alebo $q=2$. Pre $p=2$
% $q=\frcp{4p+4}{3p-2}$. Pro $p=2$ vychází $q=3$ a pro $p=3$ $q=\frac{16}{7}$. Pro $p\geq 5$ evidentně platí
% $1<\frac{4p+4}{3p-2}<2$, takže uvažovaný zlomek nemůže nabývat celočíselné
% hodnoty.
       dostávame už známe riešenie $(2,3,5)$, zatiaľ čo pre $q=2$ vychádza $p=4$.

% \smallskip\noindent
% Tím je úvodní tvrzení uvedeného řešení prokázáno.

\ineriesenie
Pre každé prvočíslo~$q$ platí nerovnosť $q+2\le 2q$. Pre prvočísla $p$ a~$r$
tak dostaneme nerovnicu $2(p+1)(r+3)\ge 4pr$, ktorú
upravíme na tvar ${(p-1)(r-3)}\le 6$. Keďže $p-1\ge1$,
musí byť $r-3\le 6$, čiže $r\le 9$. Odtiaľ vyplýva, že nutne $r\in\{2,3,5,7\}$.
Postupným rozborom každej z~týchto štyroch možností dospejeme
(analogicky ako v~predchádzajúcom riešení) k~trom trojiciam prvočísel $(p,q,r)$:
$(2,3,5)$, $(5,3,3)$ a~$(7,5,2)$, ktoré sú jedinými riešeniami úlohy.

\ifrozsirenevzoraky
\ineriesenie
Rovnicu upravíme na tvar $(1+1/p)(1+2/q)(1+3/r)=4$. Keby bolo $p\ge5$, $q\ge5$, $r\ge5$, tak
$$
\Bigl(1+\frac1p\Bigr)\Bigl(1+\frac2q\Bigr)\Bigl(1+\frac3r\Bigr)\le\frac65\cdot\frac75\cdot\frac85<4.
$$
Preto aspoň jedno z čísel $p$ ,$q$, $r$ je z~množiny $\{2,3\}$. Stačí teda preskúmať 6~možností:

\item{$\triangleright$} $p=2$: Rovnicu $3(q+2)(r+3)=8qr$ upravíme na $(5q-6)(5r-9)=144$, v~obore prvočísel je riešenie $q=3$, $r=5$.
\item{$\triangleright$} $p=3$: Rovnicu $4(q+2)(r+3)=12qr$ upravíme na tvar $(q-1)(2r-3)=9$, v~obore prvočísel nie je riešenie.
\item{$\triangleright$} $q=2$: Rovnicu $4(p+1)(r+3)=8pr$ upravíme na tvar $(p-1)(r-3)=6$, v~obore prvočísel nie je riešenie.
\item{$\triangleright$} $q=3$: Rovnicu $5(p+1)(r+3)=12pr$ upravíme na tvar $(7p-5)(7r-15)=180$, v~obore prvočísel sú riešenia $p=5$, $r=3$ a~$p=2$, $r=5$.
\item{$\triangleright$} $r=2$: Rovnicu $5(p+1)(q+2)=8pq$ upravíme na tvar $(3p-5)(3q-10)=80$, v~obore prvočísel je riešenie $p=7$, $q=5$.
\item{$\triangleright$} $r=3$: Rovnicu $6(p+1)(q+2)=12pq$ upravíme na tvar $(p-1)(q-2)=4$, v~obore prvočísel je riešenie $p=5$, $q=3$.

\zaver
V~obore prvočísel sú riešením zadanej rovnice nasledovné trojice $(p,q,r)$: $(2,3,5)$, $(5,3,3)$ a~$(7,5,2)$.
\fi

\poznamka
V~obore kladných celých čísel má rovnica až 28 riešení, z~toho 13 v~obore celých čísel väčších ako $1$\ifrozsirenevzoraky ($(2,2,9)$, $(2, 3, 5)$, $(2, 6, 3)$, $(2, 30, 2)$, $(3, 2, 6)$, $(3, 4, 3)$, $(3, 10, 2)$, $(4, 2, 5)$, $(5, 3, 3)$, $(5, 6, 2)$, $(7, 2, 4)$, $(7, 5, 2)$, $(15, 4, 2)$)\fi.
}

{%%%%%   A-III-3
\epsplace a60.15 \hfil\Obr\par
\epsplace a60.16 \hfil\Obr\par
a) Podľa zadania platí $(x+y)^2=(12-z)^2$ a~$x^2+y^2=54-z^2$, teda
$$
2xy=(x+y)^2-(x^2+y^2)=(12-z)^2-(54-z^2)=2\bigl((z-6)^2+9\bigr)\tag{1}
$$
a
$$
0\le(x-y)^2=x^2+y^2-2xy=54-z^2-2\bigl((z-6)^2+9\bigr)=-3\bigl((z-4)^2-4\bigr).\tag{2}
$$
Z~\thetag{1} vyplýva $xy=(z-6)^2+9\ge9$, z~\thetag{2} nerovnosť $(z-4)^2\le4$, čiže
% $z\in\langle2,6\rangle$.
$2\le z\le6$.
Preto $(z-6)^2\le (2-6)^2=16$,
čo spolu s~\thetag{1} dáva $xy=(z-6)^2+9\le 25$.  Vzhľadom na symetriu
platia odvodené nerovnosti $9\le xy\le 25$ aj pre súčiny $yz$, $zx$ namiesto $xy$.

\smallskip
b) Z~danej sústavy rovníc dostávame
$$
xy+yz+zx=\frac{(x+y+z)^2-(x^2+y^2+z^2)}2=\frac{12^2-54}2=45.
$$
Ďalej platí
$$\displaylines{
  \qquad (x-3)(y-3)+(y-3)(z-3)+(z-3)(x-3)=\hfill\cr
  \hfill =xy+yz+zx-6(x+y+z)+27=45-6\cdot12+27=0.\qquad
 }
$$
Odtiaľ vyplýva, že čísla $x-3$, $y-3$, $z-3$ nemôžu byť súčasne všetky
kladné, aspoň jedno z~čísel $x$, $y$, $z$ je teda nanajvýš~$3$.  Podobne zo
vzťahu
$$\displaylines{
  \qquad(x-5)(y-5)+(y-5)(z-5)+(z-5)(x-5)=\hfill\cr
  \hfill=xy+yz+zx-10(x+y+z)+75=45-10\cdot12+75=0\qquad
  }
$$
vidíme, že čísla $x-5$, $y-5$, $z-5$ nemôžu byť súčasne všetky
záporné, preto minimálne jedno z~čísel $x$, $y$, $z$ je aspoň~$5$.

\ineriesenie
Náročnejší trik v~časti~b) predošlého riešenia môžeme nahradiť dôkazom
implikácií
$$
\left(x>3\right)\wedge \left(y>3\right)\Rightarrow z<3\qquad\hbox{a}\qquad
  \left(x<5\right)\wedge \left(y<5\right)\Rightarrow z>5.
$$

Uvažujme kvadratický trojčlen $F(t)=(t-x)(t-y)$. Ak sú oba jeho korene $x$
a~$y$ väčšie ako~$3$, platí $F(3)>0$. Avšak podľa zadania a~\thetag{1} platí
$$
0<F(3)=3^2-3(x+y)+xy=9-3(12-z)+(z-6)^2+9=(z-3)(z-6).
$$
Z~tejto nerovnosti a~z~odhadu $z\le 6$ dokázaného v~časti~a) predošlého
riešenia tak dostávame požadovaný odhad $z<3$. Podobne, ak sú obe čísla $x$
a~$y$ menšie ako~$5$, tak $F(5)>0$. Avšak podľa zadania a~\thetag{1} platí
$$
0<F(5)=5^2-5(x+y)+xy=25-5(12-z)+(z-6)^2+9=(z-2)(z-5).
$$
Z~tejto nerovnosti a~odhadu $z\ge 2$ dokázaného v~časti~a) predošlého riešenia
tak dostávame požadovaný odhad $z>5$.

\ineriesenie
Vyriešime časť~b) geometricky.
V~karteziánskej sústave súradníc s~počiatkom $O$ a~osami
 $x$, $y$, $z$ určuje prvá rovnica rovinu $\sigma$,
ktorá prechádza bodom $S=[4,4,4]$ a~je kolmá na úsečku~$OS$, zatiaľ čo
druhá rovnica je rovnicou guľovej plochy ${G}(O,r=\sqrt{54})$.
% (vektor $(4,4,4)$ je normálový vektor roviny~$\sigma$).
Prienikom oboch útvarov je kružnica $k(S,\rho)$. Určíme najskôr jej polomer
a~priesečníky kružnice s~rovinou, podľa ktorej sú osi $x$ a~$y$ súmerne združené.

Označme $S_{x}$, $S_{y}$ a~$S_{z}$ kolmé priemety bodu~$S$ do
súradnicových osí $x$, $y$ a~$z$. Na \obr{} je rez rovinou
$OSS_{z}$. Platí $|OS_{1}|=4\sqrt{2}$, $|OS|=4\sqrt{3}$ (stenová
a~telesová uhlopriečka kocky s~hranou dĺžky~$4$) a~$|OA|=\sqrt{54}$. Z~pravouhlého
trojuholníka $OAS$ pomocou Pytagorovej vety určíme
$\rho=|SA|=\sqrt{6}$ a~z~podobnosti trojuholníkov $SAU\sim OSS_{1}$
dostaneme $|US|=2$ a~$|AU|=\sqrt{2}$. Odtiaľ $A=[5,5,2]$
a~(vďaka symetrii podľa~$S$) $D=[3,3,6]$.
\twocpictures

Analogickým rozborom pre roviny $OSS_{y}$ a~$OSS_{x}$ (alebo len
cyklickou zámenou, ktorú možno vzhľadom k~symetriu použiť)
nájdeme ich priesečníky s~kružnicou~$k$:
$$
B=[3,6,3],\quad E=[5,2,5]\quad\text{a}\quad C=[2,5,5],\quad F=[6,3,3].
$$
Nájdené body $A$, $B$, $C$, $D$, $E$, $F$ rozdeľujú kružnicu~$k$ na šesť oblúkov
(\obr{} znázorňuje pohľad na kružnicu~$k$ v~smere osi~$z$), pre ktorých
body zrejme platí:
\def\oblouk#1#2{\wideparen{#1#2}}
$$
\let\ =\enspace
\align
[x,y,z]\in\oblouk AB\ \Rightarrow\ &2\leq z\leq 3,\ 5\leq y\leq 6,\ 3\leq x\leq 5,\\
[x,y,z]\in\oblouk BC\ \Rightarrow\ &2\leq x\leq 3,\ 5\leq y\leq 6,\ 3\leq z\leq 5,\\
[x,y,z]\in\oblouk CD\ \Rightarrow\ &2\leq x\leq 3,\ 5\leq z\leq 6,\ 3\leq y\leq 5,\\
[x,y,z]\in\oblouk DE\ \Rightarrow\ &2\leq y\leq 3,\ 5\leq z\leq 6,\ 3\leq x\leq 5,\\
[x,y,z]\in\oblouk EF\ \Rightarrow\ &2\leq y\leq 3,\ 5\leq x\leq 6,\ 3\leq z\leq 5,\\
[x,y,z]\in\oblouk FA\ \Rightarrow\ &2\leq z\leq 3,\ 5\leq x\leq 6,\ 3\leq y\leq 5.
\endalign
$$
Tým je však tvrdenie~b) dokázané.

\ineriesenie
a) Dosadením z~prvej rovnice do druhej dostaneme
$$
x^2+y^2+xy-12x-12y+45=0\qquad \text{a odtiaľ}\qquad x=\frac{12-y\pm\sqrt{-3y^2+24y-36}}2.
$$
Preto $-3y^2+24y-36\ge0$, takže $2\le y\le6$. Ďalej máme
$$
2xy=12y-y^2\pm y\sqrt{-3y^2+24y-36}.
$$

Pripusťme, že $2xy<18$. Potom $12y-y^2-y\sqrt{-3y^2+24y-36}<18$ čiže
$$
0<12y-y^2-18<y\sqrt{-3y^2+24y-36},
$$
odkiaľ po umocnení a úprave dostaneme $(y-3)^4<0$, čo však nie je možné.

Podobne z~nerovnosti $2xy>50$ by vyplývalo
$$
\align
12y-y^2+y\sqrt{-3y^2+24y-36}&>50,\\
y\sqrt{-3y^2+24y-36}&>y^2-12y+50>0,
\endalign
$$
a po umocnení a~úprave $(y^2-2y+25)(y-5)^2<0$, čo tiež neplatí.

Preto $9\le xy\le 25$ a~vzhľadom na symetriu aj $9\le yz\le 25$ a~$9\le zx\le 25$.

\smallskip
b) Položme $x=4+a$, $y=4+b$, $z=4+c$. Potom $a+b+c=0$, $a^2+b^2+c^2=6$. Bez ujmy na všeobecnosti môžeme predpokladať, že
$|a|\ge|b|\ge|c|$. Potom čísla $a$ a~$b$ majú opačné znamienka a~$a^2\ge2$, preto $|a|\ge\sqrt2$\ifrozsirenevzoraky{} (dá sa dokázať dokonca $|a|\ge\sqrt3$)\fi, a~teda
$x\le 4-\sqrt2<3$ alebo $x\ge 4+\sqrt2>5$. Z~nerovnosti $|b|<1$ by vyplývalo $|c|<1$, ale potom $|a|\le|b|+|c|<2$ a~$a^2+b^2+c^2<6$; preto $|b|\ge1$.
Môžu teda nastať dva prípady:

\item{$\triangleright$} Ak $a>0$, tak $b<0$, teda $b\le-1$; preto $x>5$ a $y\le3$.
\item{$\triangleright$} Ak $a<0$, tak $b>0$, teda $b\ge1$; preto $x<3$ a $y\ge5$.
}

{%%%%%   A-III-4
Ak Adam nahradí koeficient pri lineárnom člene, získa trojčlen
$ax^2+({a+c})x+c$, ktorý má dva rôzne reálne korene práve vtedy, keď je jeho
diskriminant $(a+c)^2-4ac=(a-c)^2$ kladný. To nastane práve vtedy, keď $a\ne
c$. V~tomto prípade vyššie opísaným ťahom Adam zvíťazí.
Ak Adam nahradí koeficient pri absolútnom člene, získa trojčlen
$ax^2+bx+(a+b)$, ktorý má dva rôzne reálne korene práve vtedy, keď je jeho diskriminant
$b^2-4a(a+b)=\left(b(1+\sqrt2)+2a\right)\left(b(\sqrt2-1)-2a\right)$
kladný. Vzhľadom na podmienky úlohy to nastane práve vtedy, keď
$b(\sqrt2-1)>2a$. Keďže diskriminant kvadratického trojčlena je symetrická
funkcia koeficientov pri kvadratickom a~absolútnom člene, nastane rovnaká
situácia aj v~prípade, keď Adam nahradí koeficient pri kvadratickom člene.

Zhrňme úvahy z~predošlého odseku. Ak $a\ne c$ alebo
$b>\frac2{\sqrt2-1}a=2(\sqrt2+1)a$, môže Adam svojím prvým ťahom vyhrať.

\smallskip
Predpokladajme, že $a=c$ a~súčasne $b\le 2(\sqrt2+1)a$. Po Adamovi je na
ťahu Boris, ktorý bude nahrádzať koeficienty jedného z~trojčlenov

\smallskip
{a)} $ax^2+bx+(a+b)$ alebo $(a+b)x^2+bx+a$,
% \item
\hfil
{b)} $ax^2+2ax+a$.

\smallskip
a) Ak v~tomto prípade nahradí Boris koeficient pri lineárnom člene,
dostane jeden z~trojčlenov $a x^2+a(a+b) x+(a+b)$ alebo
$(a+b)x^2+a(a+b) x +a$, ktoré majú oba diskriminant
$a^2(a+b)^2-4a(a+b)=a(a+b)\left(a(a+b)-4\right)$, ktorý je vzhľadom
na podmienky $a\ge2$, $b\ge2$ kladný. Preto Boris týmto ťahom zvíťazí.

b) Ak Boris nahradí koeficient pri lineárnom člene, dostane kvadratický
trojčlen $ax^2+a^2x+a$, ktorý má dva reálne korene práve vtedy,
keď je jeho diskriminant $a^4-4a^2=a^2(a+2)(a-2)$ kladný. Vzhľadom
na podmienky úlohy to nastane práve vtedy, keď $a>2$. Keby Boris v~prípade $a=2$
nahradil koeficient pri kvadratickom alebo absolútnom člene, zanechal by
Adamovi jeden z~trojčlenov $8x^2+4x+2$ alebo $2x^2+4x+8$. Z~úvah v~prvom
odseku vyplýva, že v~takom prípade by zvíťazil Adam. Preto v~prípade $a=2$
musí Boris, aby neprehral, nahradiť koeficient pri lineárnom člene, a~zanechá tak Adamovi
trojčlen $2x^2+4x+2$.

Z~odsekov a) a~b) vyplýva: Ak Adam nemôže zvíťaziť prvým ťahom, môže svojím
ťahom zvíťaziť Boris práve vtedy, keď $a\ne 2$. V~prípade $a=2$ svojím prvým ťahom Boris
neprehrá, len ak po ňom zanechá trojčlen $2x^2+4x+2$.

Zatiaľ teda nepoznáme víťaznú stratégiu niektorého z~hráčov, ak po prvom
Borisovom ťahu zostane trojčlen $2x^2+4x+2$. Z~predošlých úvah
vyplýva, že Adam neprehrá, len ak nahradí koeficient
pri lineárnom člene, takže zanechá súperovi rovnaký trojčlen. Na tento trojčlen
musí Boris, aby neprehral, reagovať náhradou koeficientu pri lineárnom člene,
teda aj on zanechá rovnaký trojčlen a~hra v~tomto prípade nemá pri správnej
hre oboch hráčov víťaza.

\zaver
Pre trojčlen $ax^2+bx+c$ platí:
\item{$\triangleright$} Ak $a\ne c$ alebo $b>2(\sqrt2+1)a$, má víťaznú stratégiu
Adam a~môže prvým ťahom vyhrať.
\item{$\triangleright$} Ak $a=c>2$ a~$b\le 2(\sqrt2+1)a$, má víťaznú
stratégiu Boris a~môže svojím prvým ťahom vyhrať.
\item{$\triangleright$} Ak $a=c=2$ a~$b\le 2(\sqrt2+1)a$, musia obaja hráči, aby
neprehrali, v~každom ťahu zanechať trojčlen $2x^2+4x+2$. V~tomto
prípade žiadny z~hráčov nemá víťaznú stratégiu.
}

{%%%%%   A-III-5
\epsplace a60.14 \hfil\Obr\par
\epsplace a60.17 \hfil\Obr\par
\inspicture
Ak je trojuholník $ABC$ rovnoramenný so základňou~$AB$, leží celá úsečka~$OV$ na priamke~$EP$ a~tvrdenie platí triviálne. V~ďalších úvahách teda môžeme predpokladať, že $|AC|\ne|BC|$, čiže priamky $CV$, $CO$ nie sú totožné.

Je známe, že bod~$V'$ súmerne združený s~priesečníkom výšok~$V$ podľa strany~$AB$ uvažovaného trojuholníka $ABC$ leží na kružnici tomuto trojuholníku
opísanej, preto je bod~$P$ stredom úsečky~$VV'$ (\obr).
Trojuholník $CV'O$ je rovnoramenný s~hlavným
vrcholom~$O$, a~keďže stred~$E$ úsečky~$CD$ je súčasne stredom kružnice
opísanej pravouhlému trojuholníku $CPD$ s~preponou~$CD$, je aj trojuholník $CPE$
rovnoramenný. Oba rovnoramenné trojuholníky $CV'O$ a~$CPE$ sú pritom
rovnoľahlé (so stredom rovnoľahlosti v~bode~$C$)~-- zhodujú sa totiž
v~spoločnom uhle pri základni a~body $C$, $P$, $V'$ ležia na jednej priamke rovnako ako body
$C$, $E$, $O$. Preto $PE\parallel V'O$.

% \goodbreak
Keďže $P$ je stredom strany~$VV'$ trojuholníka $V'OV$, leží na priamke~$PE$
stredná priečka tohto trojuholníka, ktorá je rovnobežná s~jeho stranou~$V'O$. Priamka~$PE$ teda pretína úsečku~$OV$ v~jej strede, čo sme chceli
dokázať.

\ineriesenie
Označme $S$ stred úsečky~$OV$ a~$A'$, $B'$, $C'$ postupne stredy strán $BC$, $CA$, $AB$. Bod~$O$ je zrejme ortocentrom ostrouhlého trojuholníka $A'B'C'$, ktorý je podobný s~trojuholníkom $ABC$. Preto $O$ leží vnútri trojuholníka $A'B'C'$ a~bod~$E$ leží vnútri úsečky~$OC$ (na jej priesečníku s~priečkou~$A'B'$).
\inspicture
Keďže $S$ leží vnútri úsečky~$OV$ a~$P$ leží mimo úsečky~$CV$ (\obr), na dôkaz toho, že body $P$, $S$, $E$ ležia na jednej priamke, stačí podľa Menelaovej vety aplikovanej na trojuholník $VOC$ ukázať, že súčin
$$
s=\frac{|VS|}{|SO|}\cdot\frac{|OE|}{|EC|}\cdot\frac{|CP|}{|VP|}
\tag1
$$
je rovný $1$. Avšak $|VS|=|SO|$ a~ak označíme $P'$ pätu kolmice spustenej z~bodu~$O$ na priečku~$A'B'$, z~podobnosti trojuholníkov $ABC$ a~$A'B'C'$ máme $|CP|:|VP|=|C'P'|:|OP'|$ (keďže $O$ je ortocentrom trojuholníka $A'B'C'$). Navyše $|EC|=|DE|$, takže po dosadení do \thetag1 dostávame
$$
s=1\cdot\frac{|OE|}{|DE|}\cdot\frac{|C'P'|}{|OP'|}=1.
$$
Posledná rovnosť platí vďaka tomu, že bod~$O$ delí každú úsečku majúcu jeden krajný bod na strane~$AB$ a~druhý na priečke $A'B'$ (rovnobežnej s~$AB$) v~rovnakom pomere, \tj. $|OE|:|DE|=|OP'|:|C'P'|$.

\ifrozsirenevzoraky
\ineriesenie
Uvažujme rovnaké označenie bodov ako v~predošlom riešení. Odlišným spôsobom ukážeme, že súčin \thetag1 je rovný $1$. Ak označíme veľkosti vnútorných uhlov trojuholníka $ABC$ zvyčajným spôsobom, tak $|\uhol COA'|=\alpha$, $|\uhol OA'E|=90\st-\beta$, $|\uhol EA'C|=\beta$, $|\uhol OCA'|=90\st-\alpha$, takže zo sínusových viet v~trojuholníkoch $EOA'$ a~$ECA'$ máme
$$
\frac{|OE|}{|EC|}=\frac{\displaystyle\frac{|EA'|}{\sin\alpha}\cdot\sin(90\st-\beta)}{\displaystyle\frac{|EA'|}{\sin(90\st-\alpha)}\cdot\sin\beta}=
\cotg\alpha\cdot\cotg\beta.
$$
Avšak $|\uhol AVP|=\beta$, odkiaľ $|VP|=|AP|/\tg\beta$; a~zároveň $|CP|=|AP|\cdot\tg\alpha$, teda $|CP|:|VP|=\tg\alpha\cdot\tg\beta$. Spolu dostávame
$$
s=1\cdot\cotg\alpha\cdot\cotg\beta\cdot\tg\alpha\cdot\tg\beta=1.
$$

\fi

\ineriesenie
Zvoľme v~rovine karteziánsku súradnicovú sústavu s~počiatkom v~bode~$P$, a~s~$x$-ovou osou totožnou s~priamkou~$AB$. Teda $P=[0,0]$ a~pre vhodné $a<0$, $b>0$ a~$c>0$ platí $A=[a,0]$, $B=[b,0]$, $C=[0,c]$. Postupne ľahko vypočítame súradnice bodov $V$, $O$, $D$, $E$ a~stredu~$S$ úsečky~$VO$ (zrejme všetky menovatele sú nenulové):
$$
\alignedat3
V&=\left[0,-\frac{ab}c\right],\qquad &
O&=\left[\frac{a+b}2,\frac{ab+c^2}{2c}\right],\qquad &
D&=\left[\frac{c^2(a+b)}{c^2-ab},0\right],\\
E&=\left[\frac{c^2(a+b)}{2(c^2-ab)},\frac c2\right],\qquad &
S&=\left[\frac{a+b}4,\frac{c^2-ab}{4c}\right]. &&
\endalignedat
$$
Overenie, že $S$ leží na priamke~$PE$, sa tak redukuje na overenie triviálnej identity
$$
\frac{c^2(a+b)}{2(c^2-ab)}:\frac c2 = \frac{a+b}4 : \frac{c^2-ab}{4c}.
$$

\ifrozsirenevzoraky
\ineriesenie
Zvoľme v~rovine karteziánsku súradnicovú sústavu tak, že $A=[0,0]$, $B=[1,0]$, $C=[c_1,c_2]$, pričom $c_2>0$, $0<c_1<1$. Potom
$$
\gathered
O=\left[\frac12, \frac {c_1^2-c_1+c_2^2}{2c_2}\right],\
P=[c_1,0], \
V=\left[c_1,\frac{c_1-c_1^2}{c_2}\right], \
S=\left[\frac{c_1}2+\frac14, \frac{c_2^2+c_1-c_1^2}{4c_2}\right],\\
D=\left[\frac{c_1^2+c_2^2-c_1^3-c_1c_2^2}{c_2^2+c_1-c_1^2},0\right], \quad
E=\left[\frac{2c_1^2-2c_1^3+c_2^2}{2\left(c_1-c_1^2+c_2^2\right)}, \frac{c_2}2\right].
\endgathered
$$
Stačí už len overiť lineárnu závislosť vektorov $S-P$ a~$E-P$, teda rovnosť
$$
\frac{\frac14-\frac{c_1}2}{\frac{c_2^2+c_1-c_1^2}{4c_2}}=
\frac{\frac{2c_1^2-2c_1^3+c_2^2}{2\left(c_1-c_1^2+c_2^2\right)}-c_1}{\frac{c_2}2}.
$$

\ineriesenie
Ak $b=|AC|=|BC|=a$, ležia všetky štyri body $P$, $E$, $O$, $V$ na jednej priamke a~na nej leží aj stred úsečky~$OV$.

Nech teda napríklad $b>a$. Platí $|\angle ACO|=|\angle PCB|=90^\circ-\beta$, a~teda $|\angle DCP|=\beta-\alpha$. Bod $E$ je stred prepony $CD$ pravouhlého trojuholníka $CDP$, preto $|\angle DPE|=|\angle PDE|=90^\circ+\alpha-\beta$.

Označme $r$ polomer kružnice opísanej trojuholníku $ABC$, $S$ stred úsečky $OV$ a $F$, $G$ päty kolmíc z~bodov $O$, $S$ na priamku~$AB$. Potom $|OF|=r\cos\gamma=\frac12c\cotg\gamma$, $|PB|=a\cos\beta$,
$|PV|=a\cos\beta\cotg\alpha$,
$|SG|=\frac12(|OF|+|PV|)=\frac14c\cotg\gamma+\frac12a\cos\beta\cotg\alpha$,
$|GP|=\frac12(|FB|-|PB|)=\frac14 c-\frac12a\cos\beta$,
$$
\gathered
\tg|\angle GPS|=\frac{|SG|}{|GP|}=
\frac{\frac14c\cotg\gamma+\frac12a\cos\beta\cotg\alpha}{\frac 14c-\frac12a\cos\beta}=
\frac{\frac14c\cotg\gamma+\frac12\frac c{\sin\gamma}\cos\beta\cos\alpha}{\frac 14c-\frac12\frac{c\sin\alpha}{\sin\gamma}\cos\beta}=\\
=\frac{\cos\gamma+2\cos\alpha\cos\beta}{\sin\gamma-2\sin\alpha\cos\beta}=
\frac{-\cos(\alpha+\beta)+2\cos\alpha\cos\beta}{\sin(\alpha+\beta)-2\sin\alpha\cos\beta}=
\cotg(\beta-\alpha)=\tg|\angle DPE|;
\endgathered
$$
odtiaľ $|\angle GPS|=|\angle DPE|$, a~teda body $S$, $P$ a $E$ ležia na jednej priamke.
\fi
}

{%%%%%   A-III-6
Ukážeme, že jediná funkcia~$f$, ktorá spĺňa podmienky úlohy, je
$$
f(x)=1+\frac1x.
$$

Zo zadania vyplýva, že $f(y)\ne0$ pre každé $y>0$, teda
$$
f\bigl(x\,f(y)\bigr)=f(x)-\frac{1}{xy\,f(y)}. \tag{1}
$$

Označme $f(1)=a>0$.
Voľbou $x=1$, resp. $y=1$ v~rovnici \thetag{1} postupne dostaneme
$$
\gather
f\bigl(f(y)\bigr)=f(1)-\frac{1}{y\,f(y)}=a-\frac{1}{y\,f(y)}\qquad(y\in\Bbb R^+),   \tag{2}\\
f(ax)=f(x)-\frac{1}{ax} \qquad (x\in\Bbb R^+).         \tag3
\endgather
$$

Voľbou $x=1$ v~rovnici \thetag{3} obdržíme
$$
f(a)=f(1)-\frac{1}{a}=a-\frac1a. \tag{4}
$$

Voľbou $x=a$ v~rovnici \thetag{1} a~použitím \thetag{4} dostaneme
$$
f\bigl(a\,f(y)\bigr)=f(a)-\frac{1}{ay\,f(y)}=a-\frac{1}{a}-\frac{1}{ay\,f(y)}
\qquad (y\in\Bbb R^+),
$$
zatiaľ čo pomocou vzťahov \thetag{3} a~\thetag{2} môžeme ľavú stranu predošlej
rovnice upraviť na tvar
$$
f\bigl(a\,f(y)\bigr)=f\bigl(f(y)\bigr)-\frac{1}{a\,f(y)}=a-\frac{1}{y\,f(y)}-\frac{1}{a\,f(y)}.
$$
Porovnaním pravých strán predošlých dvoch rovníc vypočítame
$$
f(y)=1+\frac{a-1}y \qquad (y\in\Bbb R^+). \tag{4}
$$

Ak teda existuje riešenie danej rovnice, musí mať tvar \thetag{4}. Dosadením
do rovnice v~zadaní a~následnou úpravou zistíme, že pre všetky kladné
reálne $x$ a~$y$ má platiť ${(a-1)^2}=1$.
Vzhľadom na predpoklad $a>0$ platí táto rovnosť práve vtedy, keď $a=2$.
Týmto krokom sme zároveň urobili skúšku správnosti nájdeného riešenia.

\ifrozsirenevzoraky
\ineriesenie
Vzťahy \thetag{1} a~\thetag{2} z~predchádzajúceho riešenia môžeme využiť
aj nasledujúcim spôsobom.
Pre ľubovoľné reálne číslo~$t$ je $f(t)>0$.  Voľbou $x=f(t)$
v~rovnici \thetag{1} obdržíme pomocou \thetag{2}
$$
f\bigl(f(t)\,f(y)\bigr)=f\bigl(f(t)\bigr)-\frac1{f(t)\,y\,f(y)}=
  a-\frac1{t\,f(t)}-\frac1{f(t)\,y\,f(y)} \qquad (t,y\in\Bbb R^+).
$$
Zámenou premenných $t$ a~$y$ odtiaľ získame
$$
f\bigl(f(y)\,f(t)\bigr)=a-\frac1{y\,f(y)}-\frac1{f(y)\,t\,f(t)} \qquad (t,y\in\Bbb R^+).
$$
Keďže výrazy na ľavých stranách predošlých dvoch rovníc sú zhodné, musia
byť zhodné aj výrazy na pravých stranách, takže platí
$$
a-\frac1{t\,f(t)}-\frac1{f(t)\,y\,f(y)}=a-\frac1{y\,f(y)}-\frac1{f(y)\,t\,f(t)}
\qquad (t,y\in\Bbb R^+).
$$
Úpravou odtiaľ dostaneme
$$
t\bigl(f(t)-1\bigr)=y\bigl(f(y)-1\bigr) \qquad (t,y\in\Bbb R^+).
$$
Voľbou $t=1$ v~tejto rovnici získame rovnosť
$$
f(y)=1+\frac{a-1}y   \qquad (y\in\Bbb R^+),
$$
ktorú využijeme rovnako ako v~prvom riešení.
\fi
}

{%%%%%   B-S-1
Aby bola ľavá strana rovnice definovaná, musia byť oba výrazy pod odmocninami nezáporné,
čo je splnené práve pre všetky $x\ge 0$.
Pre nezáporné $x$ potom $p=\sqrt{x+3}+\sqrt{x\vphantom3}\ge\sqrt3$,
rovnica môže teda mať riešenie iba pre $p\ge\sqrt3$.

Upravujme danú rovnicu:
$$
\align
   \sqrt x+\sqrt{x+3}=&p,\\
2x +3 +2\sqrt{x(x+3)}=&p^2,\\
       2\sqrt{x(x+3)}=&p^2-2x-3,\\
              4x(x+3)=&(p^2-2x-3)^2,\\
             4x^2+12x=&p^4+4x^2+9-4p^2x-6p^2+12x,\\
                    x=&\frac{(p^2-3)^2}{4p^2}   .\tag1
\endalign
$$
Keďže sme danú rovnicu umocňovali na druhú, je nutné sa presvedčiť skúškou, že
vypočítané $x$ je pre hodnotu parametra $p\ge\sqrt3$ riešením pôvodnej rovnice:
$$
\align
\sqrt{\frac{(p^2-3)^2}{4p^2}+3} +\sqrt{\frac{(p^2-3)^2}{4p^2}}
     =&\sqrt{\frac{p^4-6p^2+9+12p^2}{4p^2}}+\sqrt{\frac{(p^2-3)^2}{4p^2}}=\\
     =&\sqrt{\frac{(p^2+3)^2}{4p^2}}+\sqrt{\frac{(p^2-3)^2}{4p^2}}=\frac{p^2+3}{2p}+\frac{p^2-3}{2p}=p.
\endalign
$$
Pri predposlednej úprave sme využili podmienku $p\ge\sqrt3$ (a~teda aj $p^2-3\ge 0$ a~$p>0$),
takže $\sqrt{(p^2-3)^2}=p^2-3$ a~$\sqrt{4p^2}=2p$.

\poznamka
Namiesto skúšky stačí overiť, že pre nájdené $x$ sú všetky umocňované výrazy nezáporné,
teda vlastne stačí overiť, že
$$
p^2-2x-3=\frac{(p^2-3)(p^2+3)}{2p^2}\ge0.
$$
Pre $p\ge\sqrt3$ to tak naozaj je.

Vynechať skúšku možno aj takouto úvahou: Funkcia $\sqrt{x+3}+\sqrt{x}$ je zrejme rastúca, v~bode $0$ (ktorý je krajným bodom jej definičného oboru) nadobúda hodnotu $\sqrt3$ a~zhora je neohraničená. Preto každú hodnotu $p\ge\sqrt3$ nadobúda pre práve jedno $x\ge0$. Z~toho vyplýva, že pre $p\ge\sqrt3$ má zadaná rovnica práve jedno riešenie, a~teda (jediné) nájdené riešenie \thetag1 musí vyhovovať.

\nobreak\medskip\petit\noindent
Za úplné riešenie dajte 6 bodov. Za
zabudnutie skúšky (resp. vynechanie zdôvodnenia, prečo ju robiť netreba) strhnite jeden
bod. Pri neuvedení podmienky $p\ge\sqrt3$ dajte 3~body (či už skúška je,
alebo nie~-- ak je, tak je chybne).
\endpetit
\bigbreak
}

{%%%%%   B-S-2
a) Medzi 16~číslami napísanými pozdĺž kružnice sa
nachádza práve 16~úsekov piatich susedných čísel
(ak vyberieme ľubovoľne jedno z~napísaných čísel a~od neho označíme čísla pozdĺž
kružnice postupne ako prvé, druhé,~\dots, šestnáste, bude prvý úsek tvorený
prvým až piatym číslom, druhý úsek druhým až šiestym číslom,~
\dots{} a~posledný šestnásty úsek bude tvorený šestnástym, prvým, druhým, tretím a~štvrtým číslom).

Tvrdenie dokážeme sporom. Predpokladajme, že uvažované tvrdenie neplatí,
teda že čísla v~každom z~16~úsekov majú súčet menší ako~$2$.
%% a z tohoto předpokladu odvodíme  nepravdivé tvrzení.
%% Pokud by tedy součet libovolných pěti sousedních čísel byl menší než $2$,
%% pak by
Celkový súčet~$S_5$
%% součtů jednotlivých úseků byl
všetkých 16 súčtov čísel v~jednotlivých päticiach
je tak menší ako $16\cdot 2=32$.
Avšak každé číslo na kružnici je súčasťou práve piatich úsekov piatich susedných
čísel, teda každé z~16~čísel je v~uvedenom súčte započítané
práve päťkrát.
Preto je súčet~$S_5$ zároveň rovný päťnásobku
súčtu všetkých čísel na kružnici, čo je~$35$. To je v~spore s~odvodenou nerovnosťou
$S_5<32$.
%% a proto $S_5=5\cdot 7=35$. Nemůže však být současně $S_5\le 32$ i $S_5=35$,
%% dostáváme kýžený spor
%% a náš předpoklad tudíž neplatí.
Na kružnici teda musí existovať
päť po sebe idúcich čísel, ktorých súčet je aspoň~$2$ (dokonca viac ako~$2$).

\smallskip
b)
%% Ukážeme, že nejmenší hledané $k$ je rovno~7.
Najskôr ukážeme, že neplatí $k\le6$. Na to stačí
pozdĺž kružnice rozmiestniť 16~zhodných čísel so súčtom~$7$.
Súčet čísel v~ľubovoľnom úseku $k$ čísel tak bude
$$
k\cdot \frac7{16}\le \frac{42}{16}<3.
$$

Nech teraz $k=7$. Zopakovaním úvahy z~časti~a) dokážeme, že vhodný úsek už
existuje: Predpokladajme naopak,
že súčet ľubovoľných siedmich po sebe idúcich čísel
(z~daných šestnástich) je menší ako tri.
Takých úsekov je pozdĺž kružnice šestnásť (ich počet od čísla~$k$ nezávisí!),
takže súčet~$S_7$ všetkých 16~súčtov čísel v~jednotlivých sedmiciach
je menší ako $16\cdot3=48$. Každé z~daných 16~čísel je v~súčte~$S_7$
započítané sedemkrát, teda $S_7=7\cdot 7=49$, čo odporuje predošlému odhadu $S_7<48$.

Hľadaným číslom~$k$ je číslo~$7$.

\nobreak\medskip\petit\noindent
Za úplné riešenie dajte 6~bodov.
Za dôkaz každej z~častí a) a~b) dajte po 3~bodoch. Ak v~časti~b) chýba
príklad, že $k\le6$ nevyhovuje, strhnite 2~body.
\endpetit
\bigbreak
}

{%%%%%   B-S-3
\epsplace b60.3 \hfil\Obr\par
Keďže oba uhly $BCG$ a~$DAC$ sú pravé,
%% uvažujme otočení o~$90\st$ kolem vrcholu~$C$ daného \tr-u,
uvažujme otočenie okolo vrcholu~$C$ daného trojuholníka,
v~ktorom sa bod~$B$ zobrazí na bod~$G$. V~ňom je zrejme
obrazom bodu~$D$ bod~$A$ a~obrazom úsečky~$BD$ úsečka~$GA$ (\obr).
Odtiaľ vyplýva, že $|AG|=|BD|$, a~tiež, že úsečky $AG$ a~$BD$ sú navzájom kolmé.
\inspicture{}

Označme postupne $K$, $L$, $M$, $N$ stredy %%úseček $AB$, $BG$, $GD$ a~$DA$.
strán štvoruholníka $ABGD$.
(Body $N$ a~$L$ sú teda stredmi uvažovaných štvorcov.)
Vzhľadom na to, že úsečka~$KL$ je strednou priečkou
trojuholníka $AGB$ a~úsečka~$MN$ strednou priečkou
trojuholníka $AGD$, máme $|KL|=\frac12|AG|=|KL|$ a~zároveň $MN\parallel AG\parallel KL$.
Podobne $|KN|=\frac12|BD|=|LM|$ a~zároveň $KN\parallel BD\parallel LM$.
To znamená, že $KLMN$ je rovnobežník. Keďže však vieme, že
$|AG|=|BD|$ a~navyše $AG\perp BD$, je $KLMN$ štvorec.
Tým sú všetky tvrdenia úlohy dokázané.

\ineriesenie
Úlohu vyriešime bez úvahy o~otočení. Pre dôkaz rovnosti $|AG|=|BD|$ ukážeme, že
trojuholníky $ACG$ a~$DCB$ sú zhodné podľa vety $sus$. Naozaj,
$|AC|=|DC|$, $|CG|=|CB|$
a~$|\angle ACG|=|\angle ACB|+|\angle BCG|=|\angle ACB|+90^\circ=|\angle
ACB|+|\angle ACD|=|\angle DCB|$.

Úsečky $AG$ a~$BD$ ako strany zhodných trojuholníkov teda majú rovnakú dĺžku.
Aby sme overili, že sú navyše navzájom kolmé, označíme $P$ ich priesečník
a~porovnáme vnútorné uhly v~trojuholníkoch $APQ$ a~$DCQ$, pričom $Q$ je priesečník úsečiek $AC$
a~$BD$. Pri~vrcholoch $A$~a~$D$ sú uhly zhodné vďaka overenej zhodnosti
trojuholníkov $ACG$ a~$DCB$, uhly pri vrchole~$Q$ sa tiež zhodujú (sú vrcholové), takže sa
zhodujú aj ich uhly pri vrcholoch $P$ a~$C$, sú teda oba pravé.

Z~dokázanej zhodnosti aj kolmosti úsečiek $AG$ a~$BD$ odvodíme, že $KLMN$ je štvorec,
rovnako ako v~prvom riešení.

% \medskip
% {\it Poznámka}.
%% Poznamenejme ještě, že jsme vlastně ukázali, že středy stran libovolného
%% konvexního čtyřúhelníku
%% tvoří vrcholy rovnoběžníku, jehož strany jsou rovnoběžné s úhlopříčkami
%% daného čtyřúhelníku.
%% Tomuto rovnoběžníku se říká Varignonův rovnoběžník.
%% Je tedy čtyřúhelník $KLMN$  Varignonův rovnoběžník čtyřúhelníku $ABGD$.

\nobreak\medskip\petit\noindent
Za úplné riešenie dajte 6~bodov.
Za dôkaz zhodnosti úsečiek $AG$ a~$BD$ dajte 2~body, za dôkaz ich kolmosti ďalšie
2~body. Za dokončenie dôkazu tiež 2~body.
\endpetit
}

{%%%%%   B-II-1
Pomocou rovností $abc=60$, $a+b+c=15$ daný výraz $(a+b)(a+c)$
upravíme a~potom odhadneme na základe AG-nerovnosti pre dvojicu
hodnôt $a$ a~$\frc{4}{a}$:
$$
\align
(a+b)(a+c)&=a^2+(b+c)a+bc=a^2+(15-a)\cdot a+\frac{60}{a}=\\
&=15a+\frac{60}{a}=15\Bigl(a+\frac{4}{a}\Bigr)\ge
15\cdot2\cdot\sqrt{a\cdot\frac{4}{a}}=60.
\endalign
$$

Nerovnosť je dokázaná. Rovnosť nastane práve vtedy, keď $a=\frc{4}{a}$,
čiže $a=2$. Zo vzťahov $b+c=15-a=13$ a~$bc=\frc{60}{a}=30$
máme $\{b,c\}=\{3,10\}$. Rovnosť preto spĺňajú práve dve vyhovujúce
trojice $(a,b,c)$, a~to $(2,3,10)$ a~$(2,10,3)$.

\ineriesenie
Okrem rovností $abc=60$, $a+b+c=15$
využijeme AG-nerovnosť pre dvojicu hodnôt $bc$ a~$a(a+b+c)$:
$$
(a+b)(a+c)=bc+a(a+b+c)\ge2\cdot\sqrt{bc\cdot a(a+b+c)}=2\sqrt{60\cdot15}=60.
$$
Rovnosť nastane práve vtedy, keď $bc=a(a+b+c)$,
čiže $\frc{60}{a}=15a$, odkiaľ $a=2$, takže záver je rovnaký
ako v~prvom riešení.

\nobreak\medskip\petit\noindent
Za úplné riešenie dajte 6~bodov. Za dôkaz nerovnosti dajte 4b. Za zistenie,
kedy platí rovnosť, udeľte zvyšné 2b.
\endpetit
\bigbreak
}

{%%%%%   B-II-2
\def\iite #1 {\itemitem{#1}}
Keďže číslo~$a$ delí číslo~$b$, môžeme písať $b=ka$, pričom $k$ je kladné celé číslo.
Stačí teda nájsť kladné celé čísla~$a$, pre ktoré existuje kladné celé číslo
$k$ také, že číslo $3a+4$
je (kladným) násobkom čísla $ka+1$ ($=b+1$). Z~tejto podmienky dostávame
nerovnosť $ka+1\le 3a+4$, z~ktorej vyplýva $k-3\le(k-3)a\le3$, a~teda $k\le 6$. Navyše
pre $k\ge 3$ je už $2(ka+1)>3a+4$ pre ľubovoľné $a\ge1$, takže môže byť jedine
$ka+1=3a+4$.
Preberieme všetkých šesť možností pre číslo~$k$:

\iite $k=1$: $a+1\mid 3a+4$, a~keďže $a+1\mid 3a+3$, muselo by platiť
$a+1\mid 1$, čo nie je možné, lebo $a+1>1$.

\iite $k=2$: $2a+1\mid 3a+4=(2a+1)+(a+3)$, teda $2a+1\mid a+3$. Keďže však
pre ľubovoľné prirodzené~$a$ platí $2\cdot(2a+1)> a+3$, musí byť $2a+1=a+3$,
čiže $a=2$ a~odtiaľ $b=ka=4$.

\iite $k=3$: $3a+1=3a+4$, čo nie je možné.

\iite $k=4$: $4a+1=3a+4$, teda $a=3$, $b=12$.

\iite $k=5$: $5a+1=3a+4$, čo nespĺňa žiadne celé~$a$.

\iite $k=6$: $6a+1=3a+4$, teda $a=1$, $b=6$.

Riešením sú dvojice $(1,6)$, $(2,4)$ a~$(3,12)$.


\nobreak\medskip\petit\noindent
Za úplné riešenie dajte 6~bodov.
Len za uvedenie všetkých dvojíc vyhovujúcich zadaniu dajte~1b.
\endpetit
\bigbreak
}

{%%%%%   B-II-3
\epsplace b60.4 \hfil\Obr\par
Zo zadania vyplýva, že $|BM|=|CN|$, $|AC|=|BC|$ a~$|\uhol ACN|=|\uhol CBM|=60\st$,
takže trojuholníky $ACN$ a~$CBM$ sú zhodné podľa vety {\it sus}.
Preto platí aj $|\angle ANC|=|\angle CMB|$, takže
štvoruholník $BNPM$ je tetivový (uhol $ANC$ je doplnok do priameho uhla
k~uhlu $ANB$, ktorý je protiľahlým uhlom k~uhlu $CMB$ v~spomenutom štvoruholníku, \obr).
\inspicture

Označme $S$ stred strany~$AB$ daného rovnostranného trojuholníka $ABC$.
Keďže $|SB|=\frac12|AB|$, je $|SB|:|MB|=3:2$,
a~keďže aj $|CB|:|NB|=3:2$, sú trojuholníky $SBC$
a~$MBN$ podobné podľa vety {\it sus}.
Uhol $CSB$ je pravý, preto musí byť pravý aj uhol $NMB$. Kružnica
opísaná štvoruholníku $BNPM$ je tak Tálesovou kružnicou nad priemerom~$BN$, a~teda
je pravý aj uhol~$BPN$, čo sme chceli dokázať.


\nobreak\medskip\petit\noindent
Za úplné riešenie dajte 6~bodov.
Za odvodenie faktu, že uhol $BMN$ je pravý
(podobnosť trojuholníkov $SBC$ a~$MBN$)
alebo za dôkaz toho, že štvoruholník $BNPM$ je tetivový, dajte 2b.
Za neúplné riešenie (napr. odvodenie oboch predchádzajúcich tvrdení bez dokončenia
dôkazu) však nedávajte viac ako 3b.
\endpetit
\bigbreak
}

{%%%%%   B-II-4
Výsledný ciferný súčet je určený jednoznačne a~je ním číslo~$33$.

Pre vyriešenie úlohy bude výhodné najskôr zistiť súčet~$S$ všetkých
päťciferných čísel obsahujúcich každú z~cifier $4$, $5$, $6$, $7$, $8$.
Týchto čísel je zrejme práve toľko,
koľko je rôznych poradí uvedených piatich cifier, teda $5!=120$.
Navyše každá z~daných cifier sa medzi týmito 120~číslami objavuje rovnomerne
v~každom ráde, teda 24-krát.
Súčet~$S$ tak môžeme rozpísať po jednotlivých rádoch ako
$$
\align
S=&10^4\cdot(24\cdot 4+24\cdot 5+24\cdot 6+24\cdot 7+24\cdot 8)+\\
  &+10^3\cdot(24\cdot 4+24\cdot 5+24\cdot 6+24\cdot 7+24\cdot 8)+\cdots=\\
 =&24\cdot(4+5+6+7+8)\cdot(10^4+10^3+10^2+10+1)=24\cdot 30\cdot 11\,111.
\endalign
$$

Obráťme teraz pozornosť na možné hodnoty ciferného súčtu čísla $S-a$, pričom
$a$ je päťciferné číslo spomínaného tvaru, teda $a=33\,333+b$, pričom $b$ je
päťciferné číslo obsahujúce každú z~cifier $1$, $2$, $3$, $4$, $5$. Teda
$$
S-a= 11\,111 \cdot 24 \cdot 30 -a = 7\,999\,920 - 33\,333-b=7\,966\,587-b.
$$
Pri odčítaní čísla~$b$ však nenastáva v~jednotlivých rádoch prechod cez
desiatku, preto je ciferný súčet čísla $S-a$ rovný
$(7 + 9 + 6 + 6 + 5 + 8 + 7) - (1 + 2 + 3 +4+5) = 48 - 15 = 33$
pre ľubovoľné päťciferné číslo~$a$ obsahujúce každú z~cifier $4$, $5$, $6$, $7$, $8$.

\nobreak\medskip\petit\noindent
Za úplné riešenie dajte 6~bodov.
\endpetit
\bigbreak
}

{%%%%%   C-S-1
Označme
%% délku okruhu v metroch $d$ a
rýchlosti bežcov $v_1$ a~$v_2$ tak, že
$v_1>v_2$ (rýchlosti udávame v~okruhoch za minútu).
Predstavme si, že atléti vyštartujú z~rovnakého miesta, ale opačným smerom.
V~okamihu ich ďalšieho stretnutia po 10~minútach bude súčet dĺžok
oboch prebehnutých úsekov
zodpovedať presne dĺžke jedného okruhu, teda $10v_1 + 10v_2 = 1$.

Ak bežia atléti z~rovnakého miesta rovnakým smerom, dôjde k~ďalšiemu
stretnutiu, akonáhle rýchlejší atlét zabehne o~jeden okruh viac ako
pomalší. Preto $40v_1-40v_2 = 1$.

Dostali sme sústavu dvoch lineárnych rovníc %s parametrom $d$ a
s~neznámymi $v_1$, $v_2$:
$$
\align
10v_1+10v_2 &=1,\\
40v_1-40v_2 &=1,
\endalign
$$
ktorú vyriešime napríklad tak, že k~štvornásobku prvej rovnice
pripočítame druhú, čím dostaneme $80v_1=5$, čiže $v_1=\frac1{16}$.
Zaujíma nás, ako dlho trvá rýchlejšiemu bežcovi prebehnúť jeden okruh, teda hodnota
podielu $1/v_1$. Po dosadení vypočítanej hodnoty~$v_1$ dostaneme {\it odpoveď\/}: 16~minút.

\poznamka
Úlohu možno riešiť aj úvahou: za 40~minút ubehnú atléti spolu 4~okruhy
(to vyplýva z~prvej podmienky),
pritom rýchlejší o~1~okruh viac ako pomalší (to vyplýva z~druhej podmienky).
To teda znamená, že prvý za uvedenú dobu ubehne 2{,}5~okruhu a~druhý 1{,}5~okruhu,
takže rýchlejší ubehne jeden okruh za $40/2{,}5=16$~minút.

\nobreak\medskip\petit\noindent
Za úplné riešenie dajte 6~bodov, z~toho po 2~bodoch za zostavenie
jednotlivých rovníc a~2~body za výpočet požadovanej hodnoty.
\endpetit
\bigbreak
}

{%%%%%   C-S-2
\epsplace c60.11 \hfil\Obr\par
\epsplace c60.12 \hfil\Obr\par
Ak priečka delí štvorec na dva štvoruholníky, musia ich koncové body
ležať na protiľahlých stranách štvorca. V~takom prípade sú oba
štvoruholníky lichobežníkmi alebo pravouholníkmi
(pre potreby tohto riešenia budeme pravouholník považovať
za špeciálny lichobežník). Označme daný štvorec $ABCD$, koncové body
priečky označme $K$ a~$L$. Predpokladajme, že bod~$K$ leží na strane~$AD$,
potom bod~$L$ leží na strane~$BC$. Jeden zo štvoruholníkov $KABL$ a~$KDCL$ má podľa
zadania obsah $12\cm^2$; nech je to napr. lichobežník $KABL$.

Obsah lichobežníka vypočítame ako súčin jeho výšky s~dĺžkou strednej
priečky. Výška je v~našom prípade rovná dĺžke strany štvorca, čiže
$6\cm$. Jeho stredná priečka má teda dĺžku $2\cm$. Z~toho vyplýva, že stred úsečky~$KL$
musí ležať na osi strany~$AB$ vo vzdialenosti $2\cm$ od stredu strany~$AB$ (\obr).
Platí to aj naopak: Ak stred úsečky~$KL$ leží v~opísanej polohe,
bude štvoruholník $KABL$ lichobežník s~obsahom $12\cm^2$.
\twocpictures

Ak budeme namiesto lichobežníka $KABL$ uvažovať lichobežník $KDCL$,
vyjde stred priečky~$KL$ na osi úsečky~$CD$ vo vzdialenosti $2\cm$
od stredu strany~$CD$.

Ak priečka~$KL$ spája body na stranách $AB$ a~$CD$, dostaneme
ďalšie dva možné body ležiace na spojnici stredov úsečiek $AD$ a~$BC$.
Hľadanú množinu teda tvoria štyri body, ktoré ležia na priečkach spájajúcich
stredy protiľahlých strán štvorca vo vzdialenosti $1\cm$ od jeho stredu (\obr).

\nobreak\medskip\petit\noindent
Za úplné riešenie dajte 6~bodov. Ak riešiteľ nezdôvodní, že nájdené body
majú požadovanú vlastnosť (má iba dôkaz nutnej podmienky), dajte
nanajvýš 5~bodov. Za objavenie jedného z~bodov skúmanej množiny bez dôkazu
správnosti dajte len 1~bod a~za opis celej správnej množiny bez dôkazu správnosti 2~body.
\endpetit
\bigbreak
}

{%%%%%   C-S-3
Na základe predpokladu zo zadania vieme, že existujú kladné celé čísla $m$
a~$n$, pre ktoré platí
$$
\aligned
3x + 5y = & 60m,\\
5x + 2y = & 60n.
\endaligned
$$
Na tieto vzťahy sa môžeme pozerať ako na sústavu lineárnych rovníc
s~neznámymi $x$ a~$y$ a~parametrami $m$ a~$n$. Vyriešiť ju vieme ľubovoľnou
štandardnou metódou, napríklad od dvojnásobku prvej rovnice odčítame
päťnásobok druhej a~vyjadríme~$x$, potom dopočítame~$y$. Dostaneme
$$
x =  {60(5n-2m)\over 19}, \qquad y ={60(5m-3n)\over 19}.
$$
Keďže čísla $19$ a~$60$ sú nesúdeliteľné, sú obe čísla $x$ a~$y$
deliteľné~$60$. Preto aj súčet $2x+3y$ je deliteľný~$60$.

\ineriesenie
Vieme, že $60 = 3\cdot 4\cdot 5$. Pritom čísla $3$, $4$, $5$ sú po dvoch
nesúdeliteľné, preto na dôkaz deliteľnosti~$60$ stačí dokázať deliteľnosť
jednotlivými číslami $3$, $4$, $5$.

Keďže číslo $3x+5y$ je deliteľné $5$, je aj $x$ deliteľné $5$.
Podobne z~relácie $5\mid 5x+2y$ vyplýva $5\mid y$. Preto $5$ delí aj $2x+3y$.

Keďže číslo $3x+5y$ je deliteľné $3$, je $y$ deliteľné $3$.
Vzhľadom na $3\mid 5x+2y$ máme tiež $3\mid 5x$, a~teda $3\mid x$.
Preto $3$ delí aj $2x+3y$.

Keďže $4\mid 3x+5y$ a~$4\mid 5x+2y$, máme aj $4\mid (3x+5y)+(5x+2y) = 8x+7y$,
takže $4\mid y$. Ďalej napríklad $4\mid 3x+5y$, takže $4\mid3x$, čiže $4\mid x$.
Preto $4$ delí aj $2x+3y$.

\ineriesenie
Vyjadríme výraz $2x+3y$ pomocou $3x+5y$ a~$5x+2y$. Budeme hľadať čísla $p$
a~$q$ také, že $2x+3y = p(3x+5y)+q(5x+2y)$ pre každú dvojicu celých čísel
$x$, $y$. Jednoduchou úpravou dostaneme rovnicu
$$
(2-3p-5q)x + (3-5p-2q)y = 0.             \tag1
$$
Ak budú hľadané čísla $p$ a~$q$ spĺňať sústavu
$$
\aligned
3p + 5q = & 2,\\
5p + 2q = & 3,
\endaligned
$$
bude zrejme rovnosť \thetag1 splnená pre každú dvojicu $x$, $y$. Vyriešením sústavy
dostaneme $p = 11/19$, $q = 1/19$. Dosadením do \thetag1 dostávame vyjadrenie
$$
            19(2x + 3y) = 11(3x + 5y) + (5x + 2y),
$$
z~ktorého vyplýva, že spolu s~číslami $3x+5y$ a~$5x+2y$ je súčasne
deliteľné~$60$ aj číslo $2x+3y$, pretože čísla $19$ a~$60$ sú nesúdeliteľné.

\nobreak\medskip\petit\noindent
Za úplné riešenie dajte 6~bodov. Ak chýba zmienka o~nesúdeliteľnosti
čísel $19$ a~$60$ a~táto nesúdeliteľnosť je v~riešení potrebná, dajte
nanajvýš $5$~bodov. Za čiastočný pokrok v~prvom riešení dajte 2~body za
zostavenie sústavy rovníc a~4~body za
zostavenie a~vyriešenie sústavy rovníc alebo vyjadrenie $x$ a~$y$ pomocou
parametrov.
\endpetit
}

{%%%%%   C-II-1
Označme čísla napísané na tabuli $a$, $b$, $c$. Súčet $a+b$ sa tiež nachádza na
tabuli, je teda rovný jednému z~čísel $a$, $b$, $c$. Keby $a+b$ bolo
rovné $a$ alebo $b$, bola by na tabuli aspoň jedna nula. Rozoberieme preto tri
prípady podľa počtu núl napísaných na tabuli.

Ak sú na tabuli aspoň dve nuly, ľahko sa presvedčíme, že
súčet každých dvoch čísel z~tabule je tam tiež.
Dostávame, že trojica $t,0,0$ je pre ľubovoľné reálne číslo~$t$ riešením úlohy.

Ak je na tabuli práve jedna nula, je tam trojica $a,b,0$, pričom $a$
aj $b$ sú nenulové čísla.
Súčet $a+b$ teda nie je rovný ani~$a$, ani~$b$, musí preto byť rovný~$0$.
Dostávame tak ďalšiu trojicu
$t,-t,0$, ktorá je riešením úlohy pre ľubovoľné reálne číslo~$t$.

Ak na tabuli nie je ani jedna nula, súčet $a+b$ nie je rovný ani~$a$,
ani~$b$, preto $a+b = c$. Z~rovnakých dôvodov je $b+c = a$ a~$c+a = b$. Dostali sme
sústavu troch lineárnych rovníc s~neznámymi $a$, $b$, $c$, ktorú môžeme vyriešiť.
Avšak hneď z~prvých dvoch rovníc po dosadení vyjde $b+(a+b)=a$, čiže $b=0$.
To je v~spore s~tým, že na tabuli žiadna nula nie je.

\zaver
Úlohe vyhovujú trojice $t,0,0$ a~$t,-t,0$ pre ľubovoľné reálne
číslo~$t$ a~žiadne iné.

\nobreak\medskip\petit\noindent
Za úplné riešenie dajte 6~bodov.
Ak riešiteľ neoverí správnosť
nájdených trojíc a~táto správnosť nevyplýva priamo z~jeho postupu, dajte najviac
5~bodov.
Len za objavenie všetkých trojíc dajte 2~body. Za objavenie neúplnej,
ale nekonečnej množiny vyhovujúcich trojíc dajte 1~bod.
\endpetit
\bigbreak
}

{%%%%%   C-II-2
Zrejme $n^2+6n > n^2$ a~zároveň $n^2+6n < n^2+6n+9 = (n+3)^2$. V~uvedenom intervale ležia iba dve druhé mocniny celých čísel:
$(n+1)^2$ a~$(n+2)^2$.

V~prvom prípade máme $n^2+6n = n^2+2n+1$, teda $4n = 1$, tomu však žiadne
celé číslo $n$ nevyhovuje.

V~druhom prípade máme $n^2+6n = n^2+4n+4$, teda $2n = 4$. Dostávame tak jediné
riešenie $n=2$.

\ineriesenie
Budeme skúmať rozklad $n^2+6n = n(n+6)$.
Spoločný deliteľ oboch čísel $n$ a~$n+6$ musí deliť aj ich rozdiel, preto
ich najväčším spoločným deliteľom môžu byť len čísla $1$, $2$, $3$ alebo~$6$.
Tieto štyri možnosti rozoberieme.

Keby boli čísla $n$ a~$n+6$ nesúdeliteľné, muselo by byť každé z~nich
druhou mocninou. Rozdiel dvoch druhých mocnín prirodzených čísel však nikdy nie je~$6$.
Pre malé čísla sa o~tom ľahko presvedčíme, a~pre $k\ge4$ už je rozdiel
susedných štvorcov $k^2$ a~$(k-1)^2$ aspoň~$7$. Vlastnosť, že $1$, $3$, $4$, $5$ a~$7$ je
päť najmenších rozdielov dvoch druhých mocnín, využijeme aj ďalej.

Ak je najväčším spoločným deliteľom čísel $n$ a~$n+6$ číslo~$2$, je
$n = 2m$ pre vhodné~$m$, ktoré navyše nie je
deliteľné tromi. Ak $n(n+6) = 4m(m+3)$ je štvorec, musí byť aj $m(m+3)$
štvorec. Čísla $m$ a~$m+3$ sú však nesúdeliteľné, preto musí byť každé z~nich
druhou mocninou prirodzeného čísla.
To nastane len pre $m =1$, čiže $n = 2$. Ľahko overíme, že $n(n+6)$
je potom naozaj druhou mocninou celého čísla.

Ak je najväčším spoločným deliteľom čísel $n$ a~$n+6$ číslo~$3$, je
$n = 3m$ pre vhodné nepárne~$m$. Ak $n(n+6)
= 9m(m+2)$ je štvorec, musia byť nesúdeliteľné čísla $m$ a~$m+2$ tiež
štvorce. Také dva štvorce však neexistujú.

Ak je najväčším spoločným deliteľom čísel $n$ a~$n+6$ číslo~$6$, je
$n = 6m$ pre vhodné $m$. Ak $n(n+6) =
36m(m+1)$ je štvorec, musia byť štvorce aj obe nesúdeliteľné čísla $m$ a~$m+1$,
čo nastane len pre $m = 0$, my však hľadáme len kladné čísla~$n$.

Úlohe vyhovuje jedine $n = 2$.


\nobreak\medskip\petit\noindent
Za úplné riešenie dajte 6~bodov.
Ak riešiteľ úlohu zredukuje na zvládnuteľný konečný počet možností pre~
$n$ (napríklad ako v~prvom riešení), udeľte 4~body.
Za úvahu typu "$n$ a~$n+6$ sú pre $n$ nedeliteľné $2$ a~$3$ nesúdeliteľné,
preto to musia byť štvorce" bez redukcie na konečný počet možností dajte najviac
3~body.
Za objav riešenia $n=2$ bez dôkazu neexistencie iného riešenia nedávajte žiadny bod.
\endpetit
\bigbreak
}

{%%%%%   C-II-3
\epsplace c60.6 \hfil\Obr\par
Štvoruholník $AECD$ je rovnobežník, pretože jeho strany $AE$ a~$CD$
sú rovnobežné a~rovnako dlhé (obe merajú $6\cm$).
Na jeho uhlopriečke~$AC$ tak leží ťažnica trojuholníka $ADE$ z~vrcholu~$A$
aj ťažnica trojuholníka $CDE$ z~vrcholu~$C$,
a~preto na tejto priamke ležia aj body $K$ a~$L$ (\obr).
Navyše vieme, že ťažisko trojuholníka delí jeho ťažnice v~pomere $2:1$, preto sú úsečky $AK$,
$KL$ a~$LC$ rovnako dlhé.
\inspicture

Bod~$L$ je stredom úsečky~$KC$, preto na osi súmernosti úsečky~$KM$
leží nielen výška rovnostranného trojuholníka $KLM$, ale
aj stredná priečka trojuholníka~$KMC$. Preto je priamka~$CM$ kolmá na~$KM$.

% Bod $M$ leží na těžnici $CP$ v trojúhelníku $EBC$.
% Trojúhelník $KMC$ je podobný trojúhelníku $APC$ podle věty {\it sus},
% neboť body $K$ a $M$ leží ve třetinách úseček $AC$ a $PC$.
% Proto $|KM| = \frac23 \cdot |AP| = \frac23\cdot 12\unit{cm} = 8\unit{cm}$.

Ostáva vypočítať dĺžky ramien lichobežníka $ABCD$. Označme $P$ stred úsečky~$EB$.
Keďže $CM$ je kolmá na~$KM$, je ťažnica~$CP$ kolmá na $EB$, takže trojuholník $EBC$
je rovnoramenný, a~teda aj daný lichobežník $ABCD$ je rovnoramenný.
Dĺžku ramena~$BC$ teraz vypočítame z~pravouhlého trojuholníka $PBC$, v~ktorom poznáme
dĺžku odvesny~$PB$. Pre druhú odvesnu~$CP$ zrejme platí
% Tuto délku víme vyjádřit než $1\hbox{,}5$-násobek délky $CM$. Přitom $|CM| =
% 2|LS|$ a $LS$ je výška v rovnostranném trojúhelníku $KLM$, v n +mž umíme určit
% délku strany~$KM$
% $8$ cm.
% Jestliže naznačený plán zrealizujeme, dostaneme
$$
|CP| = \frac32|CM|=3 \cdot\frac{\sqrt{3}}2|KM|,
$$
čo jednoducho vyplýva z~vlastností trojuholníka $KMC$.
A~keďže z~podobnosti trojuholníkov $KMC$ a~$APC$ máme $|KM|=\frac23|AP|$, dostávame (počítané
v~centimetroch)
$$
|CP| = 3 \cdot\frac{\sqrt{3}}2|KM|=3 \cdot\frac{\sqrt{3}}2\cdot\frac23|AP|=
 \sqrt{3} \cdot\frac23|AB|= 12\sqrt 3.
$$
Potom
$$
|BC| = \sqrt{|PB|^2+|PC|^2}=\sqrt{36+12^2\cdot 3}=6\sqrt{1+12} = 6\sqrt{13}.
$$

Ramená daného lichobežníka majú dĺžku $6\sqrt{13}\cm$.

\medskip\noindent
{\it Alternatívny dôkaz kolmosti priamok $KM$ a~$CM$}.
Keďže bod~$L$ je stredom úsečky~$KC$ a~zároveň $|LK| = |LM|$,
lebo trojuholník $KLM$ je rovnostranný,
leží bod~$M$ na Tálesovej kružnici nad priemerom~$KC$, takže trojuholník
$KMC$ je pravouhlý.

% {\it Náznak dôkazu kolmosti přímek KM a CM, nějž nevyužíva to, že bod $L$ je
% středem úsečky $KC$\/}:
%
% Označme $R$ a $S$ průsečíky přímky $KM$ s úsečkami $EL$ a $EC$ a $P$ střed
% úsečky $EB$.
% Priamky $AB$, $KM$ a $CD$ jsou rovnobežné, to nám umožní využiť viacero
% dvojic podobných trojúhelníků.
% Víme například vypočítat délky $|RS|=1\unit{cm}$ a $|SM|=4\unit{cm}$.
% Označme $X$ pätu kolmice z $L$ na $AB$.
% Bod $R$ je středem úsečky $LE$, neboť priamka $KM$ delí výšku lichoběžníku
% $ABCD$ v tretine a bod $L$ leží v dvou tretinách.
% Proto $|EX| = 2|RS| = 2\unit{cm}$. Potom $|XP|=4\unit{cm} = |SM|$.
% Štvoruholník $XPMS$ je tedy pravouhlý rovnobežník, proto $SM\perp CP$.


\nobreak\medskip\petit\noindent
Za úplné riešenie dajte 6~bodov, z~toho 3~body za dôkaz kolmosti priamok $KM$
a~$CM$ a~3~body za výpočet dĺžok oboch ramien lichobežníka $ABCD$.
Neúplné riešenie hodnoťte podľa pokroku, ktorý žiak dosiahol. V~uvedenom riešení
by rozdelenie bodov bolo nasledujúce:
dôkaz kolmosti $KM$ a~$CM$ -- 3~body, z~toho 1~bod za zdôvodnenie toho, že
bod~$L$ je stredom úsečky~$KC$;
výpočet dĺžky úsečky~$KM$ -- 1~bod;
výpočet dĺžky jednotlivých ramien -- po 1~bode.
\endpetit
\bigbreak
}

{%%%%%   C-II-4
Ukážeme, že ak je číslo $xy+yz+zx-3$ záporné, je číslo $x+y+z-xyz$ kladné.

Ak $xy+yz+zx<3$, je aspoň jedno z~čísel $xy$, $yz$, $zx$ menšie ako~$1$,
napr. $xy$. Potom $x+y+z-xyz=x+y+z(1-xy)$ je zjavne súčet troch kladných čísel.

\ineriesenie
Ukážeme, že ak je číslo $x+y+z-xyz$ záporné, tak číslo $xy+yz+zx-3$ je
kladné.

Predpokladajme, že $x+y+z < xyz$. Tým skôr $x < xyz$. Po skrátení kladného
čísla~$x$ dostaneme $yz > 1$.
Podobne odvodíme odhady $xy > 1$ a~$zx > 1$. Teraz ich stačí sčítať a~máme
$xy+yz+zx > 3$.

\ineriesenie
Tvrdenie úlohy dokážeme sporom.
Predpokladajme, že $x+y+z < xyz$ a~zároveň $xy+yz+zx < 3$. Obe tieto
nerovnosti sú symetrické, preto môžeme predpokladať, že čísla $x$, $y$, $z$
sú označené tak, že $z$ je najmenšie. Z~druhej nerovnosti dostaneme, že
$xy< 3$. Potom však $x+y+z < xyz < 3z$, teda $x + y < 2z$. To je však spor s~tým, že
číslo $z$ je najmenšie.

\nobreak\medskip\petit\noindent
Za úplné riešenie dajte 6~bodov.
\endpetit
\bigbreak
}

{%%%%%   vyberko, den 1, priklad 1
...}

{%%%%%   vyberko, den 1, priklad 2
...}

{%%%%%   vyberko, den 1, priklad 3
...}

{%%%%%   vyberko, den 1, priklad 4
...}

{%%%%%   vyberko, den 2, priklad 1
...}

{%%%%%   vyberko, den 2, priklad 2
...}

{%%%%%   vyberko, den 2, priklad 3
...}

{%%%%%   vyberko, den 2, priklad 4
...}

{%%%%%   vyberko, den 3, priklad 1
...}

{%%%%%   vyberko, den 3, priklad 2
...}

{%%%%%   vyberko, den 3, priklad 3
...}

{%%%%%   vyberko, den 3, priklad 4
...}

{%%%%%   vyberko, den 4, priklad 1
...}

{%%%%%   vyberko, den 4, priklad 2
Let $P(x,y)$ be the assertion $f(x^2+xy+f(y))=f^2(x)+xf(y)+y$

\item{$\bullet$} $f(0)=0$

Let $f(0)=a$

\itemitem{(1)} $P(4a,0)$ implies $f(16a^2+a)=f^2(4a)+4a^2$
\itemitem{(2)} $P(-4a,0)$ implies $f(16a^2+a)=f^2(-4a)-4a^2$
\itemitem{(3)} $P(-4a,4a)$ implies $f(f(4a))=f^2(-4a)-4af(4a)+4a$
\itemitem{(4)} $P(0,4a)$ implies $f(f(4a))=4a+a^2$

---------------------------------------------------

$(1)-(2)+(3)-(4)$ : $0=(f(4a)-2a)^2+3a^2$ and so $a=0$
Q.E.D.

\item{$\bullet$} $f(x)=x$

\itemitem{(1)} $P(0,x)$ implies $f(f(x))=x$ and so $f(x)$ is a bijection
\itemitem{(2)} $P(-x,x)$ implies $f(f(x))=f^2(-x)-xf(x)+x$
\itemitem{(3)} $P(x,0)$ implies $f(x^2)=f^2(x)$
\itemitem{(4)} $P(-x,0)$ implies $f(x^2)=f^2(-x)$

-------------------------------------------------

$-(1)+(2)+(3)-(4)$ : $0=f(x)(f(x)-x)$

\noindent
Since $f(0)=0$ and $f(x)$ is a bijection, we have $f(x)\ne 0$ $\forall x\ne 0$ and the above equality becomes $f(x)=x$ $\forall x\ne 0$.
And so $f(x)=x$ $\forall x$ which indeed is a solution.}

{%%%%%   vyberko, den 4, priklad 3
...}

{%%%%%   vyberko, den 4, priklad 4
...}

{%%%%%   vyberko, den 5, priklad 1
...}

{%%%%%   vyberko, den 5, priklad 2
...}

{%%%%%   vyberko, den 5, priklad 3
...}

{%%%%%   vyberko, den 5, priklad 4
...}

{%%%%%   trojstretnutie, priklad 1
Sčítaním troch AG-nerovností
$$
\aligned
4a^3b+b^3c+2c^3a&\ge7a^2bc,\\
4b^3c+c^3a+2a^3b&\ge7b^2ca,\\
4c^3a+a^3b+2b^3c&\ge7c^2ab
\endaligned
$$
dostaneme
$$
a^3b+b^3c+c^3a\ge a^2bc+b^2ca+c^2ab. \tag1
$$
Z predpokladu $a^2<bc$ vynásobením $ab$ vyplýva $b^2ca>a^3b$, čo spolu s~\thetag1
dáva
$$
b^3c+c^3a>a^2bc+c^2ab
\quad\text{čiže}\quad
b^3+ac^2>ab(a+c).
$$

\ineriesenie
Podľa AG-nerovnosti a danej nerovnosti platí
$$
\align
\tfrac25 b^3+ \tfrac35 a c^2 &\ge \root5\of{b^6 a^3 c^6} = b c \root5\of{a^3
b c }> a b c,\\
\tfrac35 b^3+ \tfrac25 a c^2 &\ge \root5\of{b^9 a^2 c^4} = b \root5\of{a^2
b^4 c^4}> a^2b.
\endalign
$$
Ich sčítaním získame hľadaný odhad.}

{%%%%%   trojstretnutie, priklad 2
{\it Odpoveď}. Súčet daných čísel musí byť mocnina čísla~$2$.

\smallskip
Predpokladajme, že dané čísla nie sú všetky nulové.
Označme $S$ ich celkový súčet a~$D$ ich najväčší spoločný deliteľ.
Na začiatku je $D=1$, zatiaľ čo na konci by malo byť $D=S$, pritom súčet~$S$
všetkých čísel na tabuli sa nemení.

Po každom opísanom kroku sa aktuálna hodnota~$D$ buď nezmení, alebo narastie na
dvojnásobok. To vyplýva z~toho, že buď $\nsd(x-y,2y)=\nsd(x-y,y)=\nsd(x,y)$,
alebo $\nsd(x-y,2y)=2\nsd(x-y,y)=2\nsd(x,y)$. Vzhľadom na to, že
$\nsd(a,b,c)=\nsd\bigl(\nsd(a,b),c\bigr)$, ľahko uvedený postreh rozšírime na
najväčšieho spoločného deliteľa všetkých čísel na tabuli.
Ak teda zostane nakoniec na tabuli jediné nenulové číslo, musí tým číslom
byť mocnina dvoch.

Ak je naopak $S$ mocninou čísla~$2$, ukážeme, ako postupovať, aby sme
dostali ${n-1}$~núl. Zapíšme všetky čísla na tabuli v~dvojkovej sústave.
Ak sú na tabuli ešte aspoň dve nenulové čísla, vezmeme dve z~nich,
ktorých zápis končí najmenším počtom núl (také čísla sú aspoň dve, keďže
celkový súčet je mocnina dvoch). Po opísanej operácii namiesto nich zrejme
dostaneme dve čísla, ktoré majú na konci aspoň o~jednu nulu viac. Je teda jasné, že
po konečnom počte krokov musíme skončiť tým, že na tabuli bude jediné
nenulové číslo.}

{%%%%%   trojstretnutie, priklad 3
Najskôr sformulujeme a~dokážeme pomocné tvrdenie.

\Lema
Dané sú kružnice $\Gamma_1$, $\Gamma_2$ pretínajúce sa v~bodoch $K$, $L$, pričom stred $S_2$ kružnice $\Gamma_2$ leží na $\Gamma_1$.
Ak $M\in\Gamma_1$ ($M\ne K,L$) a~priamka $KM$ pretína $\Gamma_2$ v~bode~$N$ (rôznom od $K$), tak $|MN|=|ML|$.

\dokaz
Ak $M=S_2$, tvrdenie lemy je triviálne. Zaoberajme sa len prípadom $M\ne S_2$. Najskôr dokážeme, že priamka $MS_2$ je osou uhla $NML$.

Ak $M$ leží na oblúku~$KL$ neobsahujúcom $S_2$ (\obr{}a), tak uhly $KMS_2$, $S_2ML$ sú obvodovými uhlami nad zhodnými tetivami $S_2K$, $S_2L$, teda majú rovnakú veľkosť. Ak $M$ leží na oblúku~$KL$ obsahujúcom $S_2$, môžeme bez ujmy na všeobecnosti predpokladať, že leží na oblúku $S_2L$ neobsahujúcom bod $K$ (\obrr1b). Potom s~využitím tetivovosti štvoruholníka $KS_2ML$ máme $|\uhol S_2MN|=180\st-|\uhol S_2MK|=180\st-|\uhol S_2LK|=180\st-|\uhol S_2KL|=|\uhol S_2ML|$.
\inspinspab{cps.2}{cps.3}%

Uvažujme osovú súmernosť podľa osi $MS_2$. V~nej sa kružnica $\Gamma_2$ zobrazí sama na seba (lebo jej stred leží na osi súmernosti). Priamka $ML$ sa zobrazí na priamku $MN$ (lebo $MS_2$ je osou uhla $NML$). Takže množina $ML\cap\Gamma_2$ sa zobrazí na množinu ${MN\cap\Gamma_2}=\{K,N\}$. Keďže $L\in ML\cap\Gamma_2$, musí sa $L$ zobraziť na $K$ alebo $N$. Avšak okrem prípadu, keď $MS_2$ je priemerom $\Gamma_1$, priamky $KL$ a~$MS_2$ nie sú navzájom kolmé, teda $L$ sa nemôže zobraziť na $K$ a~musí sa zobraziť na $N$, odkiaľ $|ML|=|MN|$. Prípad, keď $MS_2$ je priemerom $\Gamma_1$, nemusíme brať do úvahy, keďže vtedy $MK$ je dotyčnicou kružnice $\Gamma_2$, čiže ju nepretína v~ďalšom bode~$N\ne K$.
\qed

\smallskip
Označme $S$ priesečník kolmice na $AB$ vedenej cez $E$ a~oblúka~$CD$. Nech $k_1$ je kružnica so stredom $A$ prechádzajúca bodom $C$ a $k_2$ je kružnica so stredom $B$ prechádzajúca bodom $D$. Kružnicu opísanú tetivovému štvoruholníku $ABCD$ označme~$k$. Priamka $SC$ pretína kružnicu $k_1$ v~$C'$ a~priamka $SD$ kružnicu $k_2$ v~$D'$. Nech ${k\cap k_1}=\{C,C''\}$, ${k\cap k_2}=\{D,D''\}$. Priesečník kružníc $k_1$, $k_2$ rôzny od $E$ označme $E'$. Sporom dokážeme, že $C'' = D''= E'$.

Bod $S$ leží na chordále kružníc $k_1$, $k_2$, preto z~jeho mocnosti k~týmto kružniciam máme $|SC|\cdot|SC'|=|SD|\cdot|SD'|$. Podľa zadania $|SC|=|SD|$, takže aj $|SC'|=|SD'|$. S~využitím pomocnej lemy potom máme $|SC''|=|SC'|=|SD'|=|SD''|$.

Ak by $C''$, $D''$ boli rôzne body, trojuholník $SD''C''$ by bol rovnoramenný a~jeho výška z~vrcholu~$S$ by prechádzala stredom kružnice~$k$. Štvoruholník $CDD''C''$ (resp. $CDC''D''$) by bol rovnoramenný lichobežník ($|SC|=|SD|$). Body $A$, $B$ sa však dajú určiť ako priesečníky osí úsečiek $CC''$, $DD''$ s~kružnicou $k$, takže aj $ABCD$ by bol vzhľadom na symetriu rovnoramenný lichobežník, čo je v~spore s~predpokladom $AB\nparallel CD$.
\insp{cps.1}%

Označme $|\uhol DE'S|=|\uhol SE'C|=\alpha$ a~$|\uhol AE'D|=\beta$.
Potom $|\uhol CE'B|=2\alpha-\beta$, pretože oblúk~$AB$ je dvakrát dlhší ako oblúk~$CD$ (\obr).
Z~rovností $|BD|=|BE'|$, $|AC|=|AE'|$ vyjadríme veľkosti uhlov v~trojuholníku $ABE'$:
$$
\aligned
2\alpha+\beta&=|\uhol AE'C|=|\uhol ACE'|=|\uhol ABE'|,\\
4\alpha-\beta&=|\uhol BE'D|=|\uhol BDE'|=|\uhol BAE'|.
\endaligned
$$
Odtiaľ
$$
180^\circ= 2\alpha+\beta + 4\alpha-\beta + 4\alpha=10\alpha, \quad\text{čiže $\alpha=18\st$}
$$
a~$|\uhol ACB|=180^\circ-|\uhol AE'B|=180^\circ-4\alpha=108^\circ$.
}

{%%%%%   trojstretnutie, priklad 4
Dokážeme tvrdenie úlohy sporom.
Predpokladajme, že $P$ nie je konštantný, a uvažujme najskôr prípad, keď je
mnohočlen $P$ lineárny, teda $P(x)=ax+b$ pre nejaké
$a, b \in \Bbb Z$, $a \ne 0$. Vezmime $Q(x)=ax^2+(b+1)x$, potom
$$
P(Q(x))=a(ax^2+(b+1)x)+b=a^2x^2+a(b+1)x+b=(ax+b)(ax+1)
$$
je rozklad mnohočlena $P(Q(x))$ na súčin dvoch nekonštantných mnohočlenov
$ax+b$ a~$ax+1$, čo odporuje predpokladu úlohy.

Ak je stupeň mnohočlena $P$ aspoň~$2$, teda $P(x) = a_nx^{n} +
a_{n-1}x^{n-1}+\cdots+a_1x+a_0$, pričom $n>1$ a $a_n\ne 0$, vezmime mnohočlen
$Q(x)=P(x)+x$. Pre mnohočlen $P(Q(x))$, ktorý má stupeň $n^2>n$, platí
$$
P(Q(x))-P(x)=P(P(x)+x)-P(x)=\sum_{i=0}^{n} a_i\bigl((P(x)+x)^{i}-x^{i}\bigr).
$$
Z~rovností $a^i-b^i=(a-b)(a^{i-1}+a^{i-2}b+\ldots+b^{i-1})$ však vyplýva,
že každý z~mnohočlenov $(P(x)+x)^{i}-x^{i}$ je deliteľný mnohočlenom
$P(x)$. Preto aj mnohočlen $P(Q(x))$ je deliteľný mnohočlenom $P(x)$. To vedie opäť
k~sporu, pretože stupeň mnohočlena $P(Q(x))$ je väčší
ako stupeň mnohočlena $P(x)$, a~ten je teda netriviálnym deliteľom
mnohočlena $P(Q(x))$. Tým je tvrdenie dokázané.}

{%%%%%   trojstretnutie, priklad 5
Označme $P$ stred uhlopriečky~$BD$ a~$Q$ priesečník priamok $BD$,
$KM$ a~$LN$. Bez ujmy na všeobecnosti predpokladajme, že bod~$B$ leží medzi
bodmi $Q$ a~$D$ (\obr). Keďže $PM$ a~$PN$ sú stredné priečky trojuholníkov
$ABD$ a~$DCB$, je $PM\parallel AB$ a~${PN\parallel CD}$.
Takže $|\uhol PNL|=|\uhol NLC|=|\uhol MKA|=|\uhol KMP|$, čo znamená, že body
$P$, $M$, $Q$, $N$ ležia na jednej kružnici.
\insp{cps.4}%
Z~rovnosti obvodových uhlov nad tetivou~$NQ$
tak vyplýva $|\uhol KMN|=|\uhol QMN|=|\uhol QPN|=|\uhol BDC|$. Podobne vychádza
$|\uhol LNM|=180^{\circ}-|\uhol QNM|=180^{\circ}-|\uhol QPM|=|\uhol MPD|=|\uhol ABD|$.
}

{%%%%%   trojstretnutie, priklad 6
Tvrdenie najskôr dokážeme pre $a=0$. Vtedy má druhá podmienka tvar ${p\mid m^3}$,
takže je splnená pre každé prvočíslo~$p$ voľbou $m=p$. Stačí teda dokázať, že
medzi deliteľmi čísel $n^2+3$ ($n\in\Bbb Z$) je nekonečne veľa prvočísel.
Pripusťme, že všetkých takých prvočísel je naopak konečne veľa, a~označme
ich $p_1$, $p_2$, \dots, $p_r$. Číslo $(3p_1p_2\ldots p_r)^2+3=3({3p_1^2p_2^2\ldots p_r^2 + 1})$
však má netriviálneho deliteľa $3p_1^2p_2^2\ldots p_r^2 + 1$, ktorý nie je
deliteľný žiadnym z~prvočísel $p_1$, $p_2$, \dots, $p_r$, a~to je spor.

Teraz tvrdenie dokážeme pre $a\ne0$. Z~rovností
$$
(9a^2k^3)^2 + 3 = 3(27a^4k^6+1)
$$
a
$$
(9a^3k^4)^3 - a = a(3^6a^8k^{12}-1) = a(27a^4k^6-1)(27a^4k^6+1)
$$
vyplýva, že pre každé celé $k$ je číslo $27a^4k^6+1$ spoločným
deliteľom čísel $n^2+3$ a~$m^3 - a$, pričom $n=9a^2k^3$ a~$m=9a^3k^4$. Stačí teda dokázať, že
pre ľubovoľné dané $a$ medzi deliteľmi čísel $27a^4k^6 + 1$ ($k\in\Bbb Z$)
existuje nekonečne veľa rôznych prvočísel.

Predpokladajme naopak, že takých prvočísel je len konečne veľa,
a~označme ich $p_1$, $p_2$, \dots, $p_r$. Pre $k=p_1p_2\ldots p_r + 1$ je však
zrejmé, že číslo $27a^4k^6+1>1$ nie je deliteľné žiadnym z~prvočísel
$p_1$, $p_2$, \dots, $p_r$. Má teda ďalšieho prvočiniteľa
$p\notin \{p_i\colon1\le i\le r\}$. Dospeli sme tak k~sporu, ktorý dokazuje
tvrdenie úlohy.}

{%%%%%   IMO, priklad 1
\podla{Natálie Karáskovej}
Keďže dvojíc $(i,j)$ spĺňajúcich $1\le i<j\le 4$ je iba šesť, určite $n_A\le6$.

Bez ujmy na všeobecnosti predpokladajme, že $a_1<a_2<a_3<a_4$. Potom platí $0<a_1+a_2<a_3+a_4$, z~čoho po pripočítaní výrazu $a_3+a_4$ dostaneme
$$
a_3+a_4<s_A<2(a_3+a_4),\qquad\text{čiže}\quad \tfrac12s_A<a_3+a_4<s_A.
$$
Z~toho priamo vyplýva, že $a_3+a_4\nmid s_A$.

Podobne máme $0<a_1+a_3<a_2+a_4$ a~po pripočítaní $a_2+a_4$ dostaneme
$$
a_2+a_4<s_A<2(a_2+a_4),\qquad\text{čiže}\quad \tfrac12s_A<a_2+a_4<s_A,
$$
z~čoho vyplýva $a_2+a_4\nmid s_A$.

Ukázali sme, že minimálne dva zo súčtov $a_i+a_j$ nedelia $s_A$, teda $n_A\le4$. Predpokladajme, že pre množinu~$A$ platí $n_A=4$ a~zaoberajme sa jej vlastnosťami. Keďže $a_2+a_4$ ani $a_3+a_4$ nedelia $s_A$, určite musia byť deliteľmi $s_A$ všetky zvyšné štyri súčty $a_1+a_2$, $a_1+a_3$, $a_1+a_4$, $a_2+a_3$. Keďže zrejme ani jeden z~nich nie je rovný $s_A$, musí byť každý z~nich menší alebo rovný $\frac12s_A$.

Keby v~niektorej z~nerovností $a_1+a_4\le\frac12s_A$, $a_2+a_3\le\frac12s_A$ platila ostrá nerovnosť (\tj. neplatila by rovnosť), ich sčítaním by sme dostali $a_1+a_2+a_3+a_4<s_A$, čo je spor. Preto nutne $a_1+a_4=a_2+a_3=\frac12s_A$.

Ďalej vieme, že $a_1+a_2<a_1+a_3<a_2+a_4=\frac12s_A$, čiže $a_1+a_3\le\frac13s_A$. Predpokladajme, že $a_1+a_3<\frac13s_A$. Keďže $a_1+a_3\mid s_A$, tak $a_1+a_3\le\frac14s_A$, a~tiež $a_1+a_2<a_1+a_3\le\frac14s_A$. Sčítaním dostaneme
$$
a_1+a_3+a_1+a_2\le\tfrac14s_A+\tfrac14s_A=\tfrac12s_A=a_2+a_3,
$$
teda $2a_1<0$, čo je spor. Takže musí byť $a_1+a_3=\frac13s_A$.

Z~predošlého vieme, že $a_1+a_2<a_1+a_3=\frac13s_A$ a~zároveň $a_1+a_2\mid s_A$, \tj. ${a_1+a_2}=s_A/x$ pre nejaké celé číslo $x\ge4$. Keď prvé dve z~troch rovností
$$
a_1+a_2=\frac1x s_A,\quad a_1+a_3=\frac13s_A,\quad a_2+a_3=\frac12s_A
\tag1
$$
sčítame a~tretiu od nich odčítame, dostaneme
$$
2a_1=\Bigl(\frac1x+\frac13-\frac12\Bigr)s_A=\Bigl(\frac1x-\frac16\Bigr)s_A,
\qquad\text{\tj.}\quad
a_1=\frac12\Bigl(\frac1x-\frac16\Bigr)s_A.\tag2
$$
Keďže $a_1>0$, musí byť $1/x-\frc16>0$, čiže $x<6$. Vzhľadom na pôvodné ohraničenie $x\ge4$ ostávajú len možnosti $x=4$ a~$x=5$. Pre každú z~oboch hodnôt dosadením do~\thetag2 vyjadríme $a_1$ a~následne z~rovností \thetag1 vyjadríme aj $a_2$, $a_3$, $a_4$:

Ak $x=4$, tak
$$
\align
a_1&=\frac12\Bigl(\frac14-\frac16\Bigr)s_A=\frac1{24}s_A,\\
a_2&=\Bigl(\frac14-\frac1{24}\Bigr)s_A=\frac5{24}s_A,\\
a_3&=\Bigl(\frac13-\frac1{24}\Bigr)s_A=\frac7{24}s_A,\\
a_4&=\Bigl(\frac12-\frac1{24}\Bigr)s_A=\frac{11}{24}s_A,
\endalign
$$
teda $A=\{k,5k,7k,11k\}$ pre nejaké prirodzené číslo~$k$. Ľahko overíme, že pre každú takúto množinu naozaj $n_A=4$.

Ak $x=5$, tak
$$
\align
a_1&=\frac12\Bigl(\frac15-\frac16\Bigr)s_A=\frac1{60}s_A,\\
a_2&=\Bigl(\frac15-\frac1{60}\Bigr)s_A=\frac{11}{60}s_A,\\
a_3&=\Bigl(\frac13-\frac1{60}\Bigr)s_A=\frac{19}{60}s_A,\\
a_4&=\Bigl(\frac12-\frac1{60}\Bigr)s_A=\frac{29}{60}s_A,
\endalign
$$
teda $A=\{k,11k,19k,29k\}$ pre nejaké prirodzené číslo~$k$. Aj pre takúto množinu platí $n_A=4$.

\odpoved
Najväčšia možná hodnota $n_A$ je 4 a~nadobúda sa pre množiny $A$ tvaru $\{k,5k,7k,11k\}$ a $\{k,11k,19k,29k\}$, kde $k$ je ľubovoľné prirodzené číslo.
}

{%%%%%   IMO, priklad 2
Priamka~$l$ delí rovinu na dve časti. Jednu z~nich ofarbime sivou a druhú bielou farbou. Všimnime si, že keď sa vo veternom mlyne mení pivot z~bodu~$T$ na bod~$U$, po zmene sa $T$ nachádza na tej istej strane priamky~$l$, na ktorej sa pred zmenou nachádzal bod~$U$ (\obr). Takže počet bodov z~$\Cal S$ nachádzajúcich sa v~sivej časti ostáva stále konštantný (ak neuvažujeme okamihy, v~ktorých sa mení pivot); to isté platí samozrejme aj pre bielu časť.
\insp{mmo.1}%

Uvažujme najskôr prípad, keď $\Cal S$ obsahuje nepárne veľa bodov, \tj. $|\Cal S|=2n+1$ pre nejaké prirodzené~$n$. Tvrdíme, že každým bodom $T\in\Cal S$ možno viesť "rozpoľujúcu" priamku, \tj. priamku, ktorá má na každej strane práve $n$~bodov z~$\Cal S$. Existencia rozpoľujúcej priamky vyplýva z~jednoduchej úvahy: Veďme cez $T$ ľubovoľnú priamku neprechádzajúcu žiadnym ďalším bodom množiny~$\Cal S$. Nech na jednej jej strane je $n+r$ a~na druhej $n-r$ bodov z~$\Cal S$. Ak $r=0$, našli sme hľadanú priamku. Ak $r\ne0$, tak pri postupnom otáčaní priamky stále okolo bodu~$T$ sa počet bodov na jej stranách zmení pri každom pretnutí bodu z~$\Cal S$ o~$1$. Pritom po otočení o~$180\st$ sa počty zrejme vymenia, takže na prvej strane bude $n-r$ bodov a~na druhej $n+r$ bodov. Je jasné, že v~istom momente počas otáčania musel byť počet bodov na oboch stranách presne~$n$.

Zvoľme teda bod~$P\in\Cal S$ ľubovoľne a~za priamku~$l$ zoberme k~nemu prislúchajúcu rozpoľujúcu priamku. Veterný mlyn s~takýmto začiatkom musí navštíviť počas otáčania o~prvých $180\st$ každý bod z~$\Cal S$ ako pivota. Pre každý bod $T\in\Cal S$ totiž existuje rozpoľujúca priamka~$t$. Rozpoľujúca priamka žiadneho iného bodu z~$\Cal S$ nemôže byť rovnobežná s~$t$ (pretože každé rovnobežné posunutie priamky~$t$ do iného bodu zmení počty bodov na jednotlivých stranách). Takže v~momente, keď je priamka veterného mlyna rovnobežná s~$t$, musí to byť~práve $t$ (z~úvodnej úvahy vieme, že priamka veterného mlyna musí byť stále rozpoľujúca).

Zaoberajme sa ďalej prípadom, keď $\Cal S$ má párne veľa bodov, čiže $|\Cal S|=2n$. Teraz budeme za "rozpoľujúcu" považovať takú {\it orientovanú\/} priamku vedúcu niektorým bodom $T\in\Cal S$, ktorá má na sivej strane $n-1$ bodov a~na bielej strane $n$~bodov z~$\Cal S$. Takú priamku možno viesť každým bodom~$T$, pretože (podobne ako v~nepárnom prípade) pri otočení pevne zvolenej priamky o~$180\st$ okolo $T$ sa zmení počet bodov z~$n-1+r$ na $n-r$, pričom zmena je pri prechode každým bodom z~$\Cal S$ vždy o~$1$ (zrejme v~závislosti od znamienka $r$ platí buď $n-1+r\le n-1\le n-r$ alebo $n-1+r\ge n-1\ge n-r$, takže hodnota $n-1$ sa v~niektorom momente musí nadobudnúť).

Aj v~tomto prípade zvolíme $P\in\Cal S$ ľubovoľne a~za priamku~$l$ zoberieme rozpoľujúcu priamku, pričom v~sivej časti bude $n-1$ bodov z~$\Cal S$. Tvrdíme, že veterný mlyn s~takýmto začiatkom navštívi počas otáčania o~prvých $360\st$ každý bod z~$\Cal S$ ako pivota. Pre každý bod $T\in\Cal S$ totiž existuje rozpoľujúca priamka~$t$ s~$n-1$ bodmi v~sivej časti a rozpoľujúca priamka žiadneho iného bodu s~ňou nie je rovnobežná a~zároveň rovnako orientovaná. Takže v~momente, keď je priamka veterného mlyna rovnobežná s~$t$ a~má aj rovnakú orientáciu, musí to byť $t$.

Ukázali sme, že veterný mlyn sa dá zvoliť tak, aby prechádzal každým bodom ako pivotom (počas otočenia o~prvých $180\st$, resp. $360\st$). Z~uvedeného je tiež zrejmé, že pri otáčaní o~ďalšie násobky $180\st$, resp. $360\st$ bude mlyn prechádzať tými istými pivotmi, takže každý bod bude pivotom nekonečne veľa krát.
}

{%%%%%   IMO, priklad 3
V~zadanej funkcionálnej nerovnici sa ako argumenty funkcie objavujú výrazy $x+y$ a~$x$. Aby sme sa zbavili súčtu v~argumente, použijeme substitúciu $y=t-x$. Pre všetky reálne čísla $x$, $t$ potom platí
$$
f(t)\le tf(x)-xf(x)+f(f(x)).\tag1
$$
V~ďalšom kroku eliminujeme člen $f(f(x))$ tak, že do \thetag1 dosadíme najskôr $t=f(a)$, $x=b$ a~potom $t=f(b)$, $x=a$. Dostaneme
$$
\align
f(f(a))-f(f(b))&\le f(a)f(b)-bf(b),\\
f(f(b))-f(f(a))&\le f(a)f(b)-af(a).
\endalign
$$
Sčítaním dostávame, že pre všetky $a$, $b$ platí
$$
2f(a)f(b)\ge af(a)+bf(b).
$$
Voľbou $b=2f(a)$ dosiahneme, že ľavá strana poslednej nerovnosti bude rovnaká ako druhý sčítanec na pravej strane. Po ich odčítaní tak ostane nerovnosť $af(a)\le0$, ktorá musí byť splnená pre všetky $a\in\Bbb R$. Preto
$$
f(a)\ge0\quad\text{pre všetky $a<0$.}\tag2
$$

Ak by pre nejaké $x$ platilo $f(x)>0$, bola by pre takúto hodnotu pravá strana nerovnosti \thetag1 v~premennej $t$ rastúcou lineárnou funkciou, teda by nadobúdala na obore záporných čísel určite aj záporné hodnoty. Potom by však záporné hodnoty na obore záporných čísel musela nadobudnúť aj ľavá strana, čiže funkcia $f$, čo je v~spore s~\thetag2. Preto
$$
f(x)\le0\quad\text{pre všetky $x\in\Bbb R$.}\tag3
$$
Spojením \thetag2 a~\thetag3 ihneď máme $f(x)=0$ pre všetky $x<0$. Ostáva určiť hodnotu $f(0)$. Ak v~\thetag1 položíme $t=x<0$, dostaneme $0\le0-0+f(0)$, čiže $f(0)\ge0$. Vzhľadom na \thetag3 už potom nutne $f(0)=0$.
}

{%%%%%   IMO, priklad 4
\podla{Natálie Karáskovej}
Závažia, ktoré máme k~dispozícii, majú nasledujúcu vlastnosť: Najťažšie závažie je ťažšie ako všetky ostatné dokopy, ba dokonca všeobecnejšie -- {\it každé závažie je ťažšie ako súčet hmotností všetkých od neho ľahších závaží}.

Z~toho dôvodu sa nikdy nemôže stať, že by súčet nejakých ľahších závaží prevážil ťažšie závažie. Inými slovami, nutná a~postačujúca podmienka, ktorú musíme pri ukladaní závaží splniť, je, že v~každom okamihu najťažšie závažie spomedzi všetkých dovtedy uložených musí byť na ľavej strane.

V~skutočnosti teda nezáleží na tom, koľko závažia vážia. Ak označíme $P(k)$ počet spôsobov, koľkými vieme uložiť závažia s~hmotnosťami $2^0$, $2^1$, \dots, $2^{k-1}$, tak aj hocakých iných $k$ závaží spĺňajúcich vlastnosť uvedenú kurzívou v~prvom odseku vieme uložiť $P(k)$ spôsobmi.

Počet $P(n)$ teda môžeme určiť "induktívne". Pre $n=1$ je zrejme jediná možnosť (závažie musíme uložiť naľavo), \tj. $P(1)=1$.

Predpokladajme, že poznáme hodnotu $P(k)$. Skúmajme, koľkými spôsobmi vieme uložiť $k+1$ závaží. Najprv sa rozhodneme, ktorých $k$ závaží spomedzi všetkých ${k+1}$ umiestnime najskôr. Bez ohľadu na to, ktoré vezmeme, vieme ich vždy uložiť $P(k)$ spôsobmi (keďže majú avizovanú vlastnosť). Každý z~týchto spôsobov môžeme doplniť posledným závažím. Ak ako posledné ukladáme najťažšie závažie (s~hmotnosťou $2^{k}$), musíme ho dať nutne na ľavú misku. Ak ukladáme jedno z~$k$ ľahších závaží, najťažšie je už určite vľavo, a~teda bez ohľadu na to, kam dáme posledné závažie, ostane ľavá strana ťažšia. V~tomto prípade máme preto dve rôzne možnosti. Dostávame
$$
P(k+1)=P(k)+k\cdot P(k)\cdot2=P(k)\cdot(2k+1).
$$

Platí teda
$$
\align
P(2)&=P(1)\cdot3=1\cdot3,\\
P(3)&=P(2)\cdot5=1\cdot3\cdot5,\\
&\,\,\,\vdots\\
P(n)&=P(n-1)\cdot(2n-1)=1\cdot3\cdot5\cdot\dots\cdot(2n-1).
\endalign
$$

\odpoved
Hľadaný počet spôsobov je rovný súčinu prvých $n$ nepárnych čísel (skrátene sa tento výraz niekedy označuje $(2n-1)!!$).
}

{%%%%%   IMO, priklad 5
\podla{Ondreja Kováča}
Podľa zadania pre ľubovoľné celé čísla $a$, $b$ platí
$$
f(a-b)\mid f(a)-f(b).\tag1
$$
Voľbou $a=c$, $b=0$ dostávame $f(c)\mid f(c)-f(0)$, odkiaľ $f(c)\mid f(0)$ pre každé celé~$c$. Z~voľby $a=0$, $b=\m c$ vyplýva $f(c)\mid f(0)-f(\m c)$, a~keďže $f(c)\mid f(0)$, nutne $f(c)\mid f(\m c)$. To však platí pre ľubovoľné celé číslo~$c$, takže zároveň aj $f(\m c)\mid f(c)$, z~čoho vzhľadom na kladnosť funkcie $f$ vyplýva $f(c)=f(\m c)$ pre všetky celé~$c$.

Predpokladajme, že $f(m)\le f(n)$. Ak $f(m)=f(n)$, tak triviálne platí $f(m)\mid f(n)$ a~nemáme čo dokazovať. Ďalej sa preto obmedzíme na prípad $f(m)<f(n)$.

Dosadením $a=n$, $b=m$ do \thetag1 dostaneme $f(n-m)\mid f(n)-f(m)$, teda existuje kladné číslo~$k$ také, že
$$
f(n)-f(m)=f(n-m)\cdot k=f(m-n)\cdot k\tag2
$$
(rovnosť $f(n-m)=f(m-n)$ vyplýva z~úvodného odseku; kladnosť $k$ vyplýva z~predpokladu $f(m)<f(n)$ a~z~kladnosti funkcie~$f$).

Dosadením $a=m$, $b=m-n$ do \thetag1 dostaneme $f(n)\mid f(m)-f(m-n)$, teda existuje celé číslo $l$ také, že
$$
f(n)\cdot l = f(m)-f(m-n).
$$
Ostatnú rovnosť prenásobíme číslom~$k$ a~využijeme \thetag2:
$$
f(n)\cdot lk = \bigl(f(m)-f(m-n)\bigr)k=f(m)\cdot k - \bigl(f(n)-f(m)\bigr)=f(m)(k+1)-f(n).
$$
Z~toho po jednoduchej úprave máme
$$
(kl+1)\cdot f(n)=(k+1)\cdot f(m).
$$
Keďže výraz na pravej strane je kladný, musí byť kladný aj činiteľ $kl+1$, čiže $l\ge0$. Z~nerovnosti $f(n)>f(m)$ a~z~poslednej rovnosti potom vyplýva $kl+1<k+1$, čiže $l<1$. Jedinou prípustnou hodnotou je teda $l=0$, odkiaľ $f(n)=(k+1)f(m)$, z~čoho už priamo vyplýva $f(m)\mid f(n)$.

}

{%%%%%   IMO, priklad 6
Bod dotyku priamky~$l$ a~kružnice $\Gamma$ označme~$T$ a~vrcholy trojuholníka určeného priamkami $l_a$, $l_b$, $l_c$ označme $A'$, $B'$, $C'$ tak ako na \obr.
\insp{mmo.2}%
Nech $A''$ je taký bod na kružnici $\Gamma$, že $A$ je stredom oblúka $TA''$ (\tj. $|TA|=|AA''|$). Podobne sú definované body $B''$, $C''$. V~celom riešení budeme kvôli stručnosti predpokladať, že situácia vyzerá tak, ako na \obrr1. V~riešení pre každú inú možnú polohu by sme postupovali podobne -- s~využívaním analogických úvah.\footnote{Osobitne treba uvažovať najmä polohu, keď je $l$ rovnobežná s~niektorou stranou trojuholníka $ABC$.}

Ak označíme $M$ priesečník priamok $l$ a~$B''C''$ a~$D$ priesečník priamok $l$ a~$l_a$, s~využitím úsekových uhlov a~toho, že $B$, $C$ sú stredy oblúkov $TB''$, $TC''$, máme
$$
\align
&|\uhol TMC''|=180\st-|\uhol TC''M|-|\uhol MTC''|=|\uhol TC''B''|-2|\uhol MTC|=\\
             &\quad=2(|\uhol TC''B|-|\uhol MTC|)=2(|\uhol TCB|-|\uhol MTC|)=2|\uhol TDC|=|\uhol TDC'|.
\endalign
$$
Takže priamky $l_a$ a~$B''C''$ sú rovnobežné. Podobnými úvahami možno odvodiť ${l_b\parallel A''C''}$ a~$l_c\parallel A''B''$. Z~toho vyplýva, že trojuholníky $A'B'C'$, $A''B''C''$ sú buď rovnoľahlé\footnote{Hovoríme, že dva trojuholníky sú rovnoľahlé, ak existuje rovnoľahlosť, ktorá zobrazí jeden na druhý.}, alebo je jeden posunutím druhého. V~ďalšom dokážeme, že sú rovnoľahlé, a~že stred ich rovnoľahlosti~$K$ leží na kružnici~$\Gamma$. Z~toho už bude vyplývať, že ich opísané kružnice sú tiež rovnoľahlé so stredom rovnoľahlosti~$K$, a~preto sa v~ňom dotýkajú.

\smallskip
Dokážeme dve pomocné tvrdenia.

\smallskip\noindent
{\it Tvrdenie 1.}
Priesečník~$X$ priamok $B''C$, $BC''$ leží na priamke~$l_a$.

\dokaz
Bod~$B$ je stredom oblúka $TB''$, teda $|\uhol BCT|=|\uhol BCB''|$ a~priamka $B''C$ je obrazom priamky~$TC$ v~osovej súmernosti podľa~$BC$. Podobne je priamka~$BC''$ obrazom priamky~$BT$. Takže $X$ je obrazom bodu~$T$ v~tejto súmernosti, \tj. leží na $l_a$.\qed

\smallskip\noindent
{\it Tvrdenie 2.}
Priesečník~$I$ priamok $BB'$, $CC'$ leží na kružnici~$\Gamma$.
\insp{mmo.3}%

\dokaz
Označme $E=AC\cap l$, $F=AB\cap l$ (\obr) a~veľkosti vnútorných uhlov trojuholníka $ABC$ označme štandardne $\alpha$, $\beta$, $\gamma$. Zo zadaných osových súmerností vyplýva, že bod~$B$ je priesečníkom osí uhlov pri vrcholoch $D$, $F$ v~trojuholníku $FDB'$. Preto je $B$ stredom kružnice vpísanej tomuto trojuholníku a~leží aj na osi uhla pri vrchole~$B'$. Máme tak
$$
|\uhol BB'C'|=\tfrac12(180\st-|\uhol B'FD|-|\uhol FDB'|)=90\st-|\uhol BFD|-|\uhol BDF|=90\st-\beta.
$$
Podobne je bod~$C$ stredom kružnice pripísanej k~strane~$EC'$ trojuholníka $EDC'$, odkiaľ
$$
|\uhol CC'B'|=\tfrac12(|\uhol EDC'|+|\uhol DEC'|)=|\uhol EDC|+|\uhol CED|-90\st=90\st-\gamma.
$$
Dopočítaním tretieho uhla v~trojuholníku $B'C'I$ dostávame
$$
|\uhol B'IC'|=180\st-(90\st-\beta+90\st-\gamma)=\beta+\gamma=180\st-\alpha,
$$
teda bod~$I$ leží na kružnici~$\Gamma$.\qed

\smallskip
Označme $K$ druhý priesečník priamky~$B'B''$ s~kružnicou~$\Gamma$. Dôkaz dokončíme použitím Pascalovej vety pre šesticu bodov $K$, $B''$, $C$, $I$, $B$, $C''$ na kružnici~$\Gamma$. Podľa nej ležia body $B'=KB''\cap IB$, $X=B''C\cap BC''$ a~$S=CI\cap C''K$ na jednej priamke. Preto $S=C'$, čiže body $K$, $C''$, $C'$ ležia na jednej priamke. Bod~$K$ je potom priesečníkom priamok $B'B''$, $C'C''$, z~čoho vyplýva, že $K$ je stredom rovnoľahlosti zobrazujúcej $A'B'C'$ na $A''B''C''$ a~leží na kružnici~$\Gamma$.
}

{%%%%%   MEMO, priklad 1
Nahradením jedného z~čísel na tabuli štvoricou opísanou v~zadaní sa zväčší aritmetický priemer druhých mocnín čísel na tabuli.
Dokážeme, že toto zväčšenie je dostatočne veľké. Začneme pomocným tvrdením.

\medskip

\Lema
Ak $a_1$, $a_2$, $a_3$, $a_4$ sú štyri navzájom rôzne celé čísla také, že ich aritmetický priemer $a = \frac14 (a_1+a_2+a_3+a_4)$ je tiež celé číslo,
tak platí
$$
{a_1^2+a_2^2+a_3^2+a_4^2\over 4}-a^2\ge {5\over 2}.
$$

\medskip

\dokaz
Ľavá strana dokazovanej nerovnosti sa dá prepísať takto:
$$
\align
{a_1^2+a_2^2+a_3^2+a_4^2\over 4}-a^2 &= {a_1^2+a_2^2+a_3^2+a_4^2-2a(a_1+a_2+a_3+a_4)+4a^2\over 4}\\
&={(a_1-a)^2+(a_2-a)^2+(a_3-a)^2+(a_4-a)^2\over 4}.
\endalign
$$
Čísla $a_1-a$, $a_2-a$, $a_3-a$, $a_4-a$ sú navzájom rôzne celé čísla so súčtom $0$. Ak žiadne z nich nie je nulové, bude súčet ich štvorcov aspoň $1^2+(-1)^2+2^2+(-2)^2=10$.
Ak jedno z nich je nula, musí aspoň jedno z nich mať absolútnu hodnotu aspoň $3$; v tomto prípade súčet štvorcov bude aspoň $3^2+1^2+(-1)^2=11$.
V oboch prípadoch je platnosť tvrdenia lemy evidentná.

\medskip
Vráťme sa k~dokazovanému tvrdeniu. Označme $S_k$ aritmetický priemer druhých mocnín čísel na tabuli po vykonaní $k$ krokov.
Keď použijeme dokázanú lemu pre štvorice čísel vzniknuté nahradením každého zo $4^k$ čísel, ktoré sú na tabuli po $k$ krokoch,
dostaneme, že $S_{k+1}-S_k\ge 5/2$ pre každé $k\ge 0$.
Preto
$$
S_{30}\ge S_0+30\cdot \frac 52= 44^2+75 = 2011.
$$
}

{%%%%%   MEMO, priklad 2
Ukážeme, že $S = (n^2+3n-6)/2$ pre všetky $n\ge 3$. Pre $n = 3$ je to zjavne pravda, ďalej budeme uvažovať $n > 3$.

Pozrime sa na situáciu najprv z pohľadu Marienky. V každej triangulácii, ktorú vytvorí, bude presne $n-2$ trojuholníkov.
Každý trojuholník z triangulácie má najviac dve strany na obvode pôvodného mnohouholníka a trojuholníky obsahujúce dve strany pôvodného mnohouholníka musia byť aspoň dva.
Ukážeme, že pre Marienku je najlepšie zvoliť trianguláciu, v ktorej sú spomínané trojuholníky práve dva.

Nazvime trojuholník {\it zlý}, ak všetky jeho strany sú diagonálami pôvodného trojuholníka. Ukážeme, že Marienka chce zvoliť trianguláciu bez zlých trojuholníkov.
Predpokladajme, že to tak nie je, \tj. existuje triangulácia optimálna pre Marienku, ktorá obsahuje zlý trojuholník (budeme takéto triangulácie volať {\it zlé\/}).
Pre každú zlú trianguláciu $T$ označme $d(T)$ dĺžku najkratšej možnej strany zlého trojuholníka v $T$. Zo všetkých zlých triangulácií s~najmenším možným počtom zlých trojuholníkov vezmime trianguláciu $T_0$, pre ktorú je hodnota $d(T)$ minimálna.
\insp{memo.4}%

Nech $ABC$ je zlý trojuholník v $T_0$ taký, že $|AB| = d(T_0)$. V $T_0$ máme tiež trojuholník $ABD$ pre $D\ne C$. Strana $AB$ je v trojuholníku $ABC$ najkratšia, čiže uhol $ACB$ je najmenší a teda ostrý. Body $A$, $C$, $B$, $D$ ležia na kružnici v tomto poradí, preto uhol $ADB$ je tupý a teda $AD$ aj $BD$ sú kratšie ako $AB$. Zmeňme trianguláciu $T_0$ na $T_1$ tak, že trojuholníky $ABC$ a $ABD$ nahradíme trojuholníkmi $ACD$ a $BCD$ (\obr). Ohodnotenia úsečiek $AD$ a $BD$ nech sú $a$ a $b$. Zmenu hodnoty $S$ vieme vyjadriť ako
$$
S(T_1)-S(T_0) = a+b-ab-1 = -(a-1)(b-1)\le 0.
$$
Avšak triangulácia $T_0$ bola optimálna, preto aj $T_1$ musí byť optimálna. Pritom počet zlých trojuholníkov v $T_0$ bol najmenší možný, a teda aspoň jedna z úsečiek $AD$ a $BD$ je diagonálou. Keďže sú obe tieto úsečky kratšie ako $AB$, dostávame spor s voľbou $T_0$.

V triangulácii bez zlých trojuholníkov sú práve dva trojuholníky obsahujúce po dve susedné strany pôvodného mnohouholníka; všetky ostatné trojuholníky obsahujú presne jednu stranu pôvodného mnohouholníka. Vzhľadom na nerovnosť $ab > a+b$, platnú pre každú dvojicu prirodzených čísel $a$, $b$ väčších ako $1$, sa Marienka už ľahko rozhodne, ktoré dva trojuholníky budú obsahovať dve strany pôvodného mnohouholníka: jeden z nich bude obsahovať stranu ohodnotenú $1$ a susednú stranu rôznu od $2$, druhý zase naopak stranu ohodnotenú $2$ a susednú stranu rôznu od $1$. Týmto spôsobom Marienka vie zaručiť, že hodnota $S$ nikdy nebude väčšia ako $3+4+\dots+(n-2) + 1\cdot (n-1) + 2n = (n^2+3n-6)/2$.

Na druhej strane Janko vie donútiť Marienku k voľbe aspoň takejto hodnoty $S$ tým, že vo svojom ťahu označí strany mnohouholníka postupne číslami
$$
1,\ n-1,\ 4,\ n-3,\ 5,\ n-4,\ \dots,\ n-2,\ 3,\ n,\ 2.
$$
}

{%%%%%   MEMO, priklad 3
Priamka $AD$ je dotyčnicou ku $k_2$, preto úsekový uhol $DAB$ má takú istú veľkosť ako uhol $BCA$.
Podobne $|\uhol ADB|=|\uhol BAC|$. Preto
$$
|\uhol DBC| = 360^\circ - |\uhol DBA| - |\uhol CBA| = 2|\uhol DAB|+2|\uhol CAB| = 2 |\uhol DAC|.
$$
\insp{memo.2}%
Body $D$, $L$, $E$, $B$, $C$, $K$ ležia na kružnici $k_3$.
Aby sme sa vyhli diskusii viacerých prípadov možného poradia týchto bodov na kružnici, budeme používať orientované uhly.
Symbolom $\widehat{XY}$ označíme veľkosť uhla $XZY$ takého, že bod $Z$ leží na kružnici $k_3$ a body $X$, $Z$, $Y$ sú pozdĺž kružnice $k_3$ usporiadané proti smeru hodinových ručičiek.
Keďže bod $E$ je stredom oblúka $CD$, platí
$$
|\uhol AKE| = \frac12 \widehat{DC} = \frac 12(180^\circ - \widehat{CD}) = \frac12 (180^\circ -|\uhol DBC|)=90^\circ - |\uhol DAC|.
$$
Preto priamka~$KE$ je kolmá na $AD$ (\obr). Podobne priamka $LE$ je kolmá na $AC$, preto bod $E$ je priesečníkom výšok trojuholníka $KLA$
a teda priamka $AE$ je kolmá na priamku $KL$, čo bolo treba dokázať.

\poznamka
Čiastočne iné riešenie tejto úlohy je možné založiť na pozorovaní, že bod $E$ je stredom opísanej kružnice trojuholníka $ACD$.
K riešeniu taktiež vedie použitie kružnicovej
inverzie so stredom v~bode~$A$; ak obraz bodu~$X$ v~takejto inverzii označíme~$X'$, bude štvoruholník $AD'B'C'$ rovnobežník a štvoruholník $AC'E'D'$ deltoid.
}

{%%%%%   MEMO, priklad 4
Označme $d$ najväčšieho spoločného deliteľa čísel $k$ a $m$.
Pre vhodné celé čísla $a$ a $b$ platí $k = da$, $m = db$, pričom $a$ a $b$ sú nesúdeliteľné a $a > b$.
Číslo
$$
{km(k^2-m^2)\over k^3-m^3} = {d^4ab(a^2-b^2)\over d^3(a^3-b^3)}={dab(a+b)\over a^2+ab+b^2}
$$
je podľa predpokladu zo zadania celé, preto $a^2+ab+b^2\mid dab(a+b)$.
Z~nesúdeliteľnosti čísel $a$, $b$ vyplýva, že $a^2+ab+b^2$ je nesúdeliteľné s~$a$, $b$ aj $a+b$; prvé dve nesúdeliteľnosti sú evidentné, na dôkaz tretej môžeme použiť Euklidov algoritmus:
$$
\nsd(a+b, a^2+ab+b^2) = \nsd(a+b, a(a+b)+b^2) = \nsd(a+b, b^2) = 1.
$$
Z~toho dostávame $a^2+ab+b^2\mid d$. Preto $d\ge a^2+ab+b^2=(a-b)^2+3ab > 3ab$. Takže
$$
(k-m)^3 = d^3(a-b)^3\ge d^3> d^2\cdot 3ab = 3km.
$$
}

{%%%%%   MEMO, priklad t1
Po dosadení $y=0$ dostaneme $x^2f(0)=x^2$ pre každé reálne číslo $x$, preto $f(0)=1$.

Zaveďme novú funkciu $g:\Bbb R \to \Bbb R$ takú, že $g(x) = f(x)-1$. Zadaná rovnica po prepísaní s~$g$ namiesto $f$ prejde na tvar
$$
y^2g(x)+x^2g(y) = xy\, g(x+y),
\tag1
$$
pričom vieme, že $g(0) = 0$.

Každá funkcia v tvare $g(x)=cx$ pre ľubovoľnú reálnu konštantu $c$ je riešením našej rovnice.
Označme $h(x) = g(x)-g(1)x$. Ukážeme, že $h(x) = 0$ pre každé reálne číslo $x$.

Funkcia $h$ spĺňa pre každú dvojicu reálnych čísel $x$, $y$ rovnosť
$$
y^2h(x)+x^2h(y) = xy\, h(x+y);
\tag2
$$
navyše vieme, že $h(0) = h(1) = 0$. Po dosadení $x=y=1$ do \thetag2 dostaneme $h(2)=0$, z dosadenia $x=\m1$, $y=1$ máme $h(\m1)=0$.

Predpokladajme, že existuje reálne číslo $a$ také, že $h(a)\ne 0$ (zjavne $a\ne 0$). Dosadením $x=1$, $y=a+1$ do \thetag2 dostaneme $h(a+1) = (a+1) h(a+2)$.
Dosadením $x = 2$, $y = a$ do \thetag2 dostaneme $2h(a) = a h(a+2)$. Z týchto dvoch rôznych vyjadrení hodnoty $h(a+2)$ dostávame
$$
{h(a+1)\over a+1} = {2h(a)\over a}.
\tag3
$$
Pritom z dosadenia $x=1$, $y=a$ do \thetag2 vyplýva, že $h(a) = a h(a+1)$, čo spolu so vzťahom \thetag3 dáva $a = \m\frac12$.
Avšak dosadenie $x=y=\m\frac12$ do \thetag2 nám po využití $h(\m1)=0$ prezradí, že $h(\m\frac12)=0$, a to je spor s voľbou čísla~$a$.

Jedinými riešeniami sú teda funkcie $f(x) = cx+1$ pre ľubovoľné reálne číslo $c$; ich správnosť ľahko overíme skúškou.


\ineries
Tak ako v prvom riešení zavedieme funkciu $g$ a dokážeme, že $g(0)=0$ a $g(\m x)= \m g(x)$.
Po dosadení $y = 1$ a $y = -1$ do rovnice \thetag1 dostaneme
$$
\align
g(x)+x^2g(1) &= x\, g(x+1),\tag4\\
g(x)+x^2g(-1) &= -x\,g(x-1).\tag5
\endalign
$$
Keď vzťah \thetag5 prepíšeme s $x+1$ miesto $x$,
dostaneme spolu so vzťahom \thetag4 sústavu dvoch rovníc s neznámymi $g(x+1)$, $g(x)$. Zo sústavy vyjadríme $g(x)$:
$$
g(x)(x^2+x+1)=g(1)x(x^2+x+1).
$$
Keďže číslo $x^2+x+1$ je vždy kladné, jedinou možnosťou je $g(x)=g(1)x$ a teda $f(x)=cx+1$.
Skúškou overíme, že takáto funkcia $f$ vyhovuje pre každé reálne číslo $c$.
}

{%%%%%   MEMO, priklad t2
Zaveďme substitúciu $a = 2x$, $b = 2y$, $c = 2z$. Chceme dokázať nerovnosť
$$
\sqrt{x}+\sqrt{y}+\sqrt{z}\ge {1\over \sqrt x}+{1\over \sqrt y}+{1\over \sqrt z},
$$
pričom platí rovnosť
$$
{1\over 1+2x}+{1\over 1+2y}+{1\over 1+2z} = 1,
$$
ktorá vyplýva zo zadanej podmienky po trojnásobnom využití vzťahu
$$
{2t\over 1+2t} = 1 - {1\over 1+2t}.
$$

Dokazovaná nerovnosť je symetrická, preto môžeme predpokladať, že $x\ge y\ge z$.
Ľahko nahliadneme, že
$$
{x-1\over 2x+1}\ge {y-1\over 2y+1}\ge {z-1\over 2z+1}.
\tag1
$$
Taktiež platí
$$
{2x+1\over \sqrt x}\ge {2y+1\over \sqrt y}\ge {2z+1\over \sqrt z}.
\tag2
$$
Ľahko to dokážeme z~ekvivalentného tvaru týchto nerovností: $(x-y)(4xy-1)\ge 0$ a~$(y-z)(4yz-1)\ge 0$. Keby bolo $4xy < 1$, bude
$$
{1\over 1+2x}+{1\over 1+2y} > 1,
$$
a to by bol spor s~väzbou.

Vďaka vzťahom \thetag1 a \thetag2 môžeme použiť Čebyševovu nerovnosť\footnote{Pod skráteným zápisom $\displaystyle\sum_{\text{cykl.}}V(x)$ rozumieme "cyklický" súčet $V(x)+V(y)+V(z).$}:
$$
\sum_{\text{cykl.}} {x-1\over \sqrt x}=\sum_{\text{cykl.}} \left({x-1\over 2x+1}\cdot {2x+1\over \sqrt x}\right)\ge {1\over 3} \sum_{\text{cykl.}} {x-1\over 2x+1} \sum_{\text{cykl.}} {2x+1\over \sqrt x} = 0.
$$

\ineriesenie (Náznak.)
Po substitúcii $x = 1/(a+1)$, $y = 1/(b+1)$, $z = 1/(c+1)$ platí $x+y+z=1$ a pôvodné premenné vieme vyjadriť ako $a = (y+z)/x$, $b = (x+z)/y$, $c = (x+y)/z$.
Chceme dokázať nerovnosť
$$
\sqrt{x+y\over 2z}+\sqrt{y+z\over 2x}+\sqrt{z+x\over 2y}\ge \sqrt{2x\over y+z}+\sqrt{2y\over z+x}+\sqrt{2z\over y+x}.
$$
Táto nerovnosť platí pre všetky trojice kladných reálnych čísel $x$, $y$, $z$: Spravíme substitúciu $p = x+y$, $q = y+z$, $r=z+x$. Čísla $p$, $q$, $r$ sú potom dĺžkami strán trojuholníka a môžeme písať $p=2R\sin\alpha$, $q = 2R\sin\beta$, $r=2R\sin \gamma$, pričom $R$ je polomer opísanej kružnice a $\alpha+\beta+\gamma=\pi$. Použitím základných vzorcov pre prácu s~goniometrickými funkciami vieme ukázať, že
$$
\sqrt{x+y\over 2z}=\sqrt{p\over q+r-p} = \sqrt{\sin\frac\alpha2\over 2\sin\frac\beta2\sin\frac\gamma2}.
$$
Podobne sa dajú upraviť všetky členy na ľavej aj pravej strane dokazovanej nerovnosti; po ďalších ekvivalentných úpravách redukujeme úlohu na dôkaz nerovnosti
$$
\sin\frac\alpha2 + \sin\frac\beta2 + \sin\frac\gamma2 \ge \left(\sin\frac\alpha2 + \sin\frac\beta2+\sin\frac\gamma2\right)^2-\left(\sin^2\frac\alpha2 + \sin^2\frac\beta2+\sin^2\frac\gamma2\right).
$$
Keďže funkcia sínus je na intervale $(0,\pi)$ konkávna, podľa Jensenovej nerovnosti platí
$$
\sin\frac\alpha2 + \sin\frac\beta2 + \sin\frac\gamma2\le 3\sin{\alpha+\beta+\gamma\over 6} = {3\over 2}.
$$
Na dokončenie dôkazu si stačí uvedomiť, že podľa nerovnosti medzi aritmetickým a~kvadratickým priemerom platí
$$
\sin^2\frac\alpha2 + \sin^2\frac\beta2+\sin^2\frac\gamma2\ge {1\over 3}\left(\sin\frac\alpha2 + \sin\frac\beta2+\sin\frac\gamma2\right)^2.
$$
}

{%%%%%   MEMO, priklad t3
Množina
$$
S = \left(\{1\}\times\{2,\dots,n\}\right)\cup\left(\{2,\dots,n\}\times\{1\}\right)
$$
má $2n-2$ prvkov a neobsahuje žiadne tri body tvoriace vrcholy pravouhlého trojuholníka. Ukážeme, že každá vyhovujúca množina $S$ má najviac $2n-2$ prvkov.

Vezmime vyhovujúcu množinu $S$. Označme $S_x$ množinu tých bodov $(x,y)$ z $S$, ktoré majú unikátnu $x$-ovú súradnicu,
čiže v $S$ nie je žiaden bod $(x,y')$ pre $y'\ne y$.
Podobne označme $S_y$ množinu bodov z $S$ s unikátnou $y$-ovou súradnicou.

Najprv sporom dokážeme, že $S = S_x \cup S_y$. Keby nejaký bod $P$ patril do $S$,
ale nepatril by ani do $S_x$, ani do $S_y$, tak k nemu vieme nájsť bod $P_x$ s rovnakou $x$-ovou súradnicou
aj bod $P_y$ s rovnakou $y$-ovou súradnicou. Potom však body $P$, $P_x$, $P_y$ tvoria pravouhlý trojuholník s pravým uhlom pri vrchole $P$.

Ak $|S_x| = n$, tak $S = S_x$, potom však množina $S$ má $n$ prvkov, a to je pre každé $n\ge 3$ menej ako $2n-2$.
Podobne vybavíme množiny, pre ktoré $|S_y| = n$. V každom inom prípade však $|S_x|\le n-1$ a $|S_y|\le n-1$,
teda množina $S$ má nanajvýš $2n-2$ prvkov.
}

{%%%%%   MEMO, priklad t4
Jazyky budeme voliť náhodne a ukážeme, že pravdepodobnosť priaznivej voľby je kladná. Z toho potom hneď vyplýva, že existuje požadovaná voľba jazykov.

Nech $p\in [0,1]$. Každý jazyk bude zvolený s pravdepodobnosťou $p$;
voľby pre jednotlivé jazyky sú nezávislé. (Formálne, elementárnou udalosťou je $n$-tica
$\omega=(a_1, a_2, \dots, a_n)$, kde $a_i=1$, ak $i$-ty jazyk bol zvolený, inak $a_i=0$;
pravdepodobnosť elementárnej udalosti $\omega$ je rovná $p^k(1-p)^{n-k}$, kde $k$ je počet jednotiek v $\omega$.)

Budeme skúmať dve náhodné premenné $A$ a $B$, kde $A$ označuje počet zvolených jazykov a $B$ počet účastníkov, ktorých všetky tri jazyky boli zvolené.
Vypočítame stredné hodnoty týchto dvoch premenných.
$$
\align
E(A) &= \sum_{\hbox{jazyk $l$}} 1\cdot P(\hbox{zvolili sme jazyk $l$}) = np\\
E(B) &= \sum_{\hbox{študent $s$}} P(\hbox{všetky jazyky študenta $s$ sú zvolené})= 3n\cdot p^3.\\
\endalign
$$
Ďalej využijeme nerovnosť
$$
P(X\ge E(X)) > 0,
$$
ktorá evidentne platí pre každú náhodnú premennú $X$. V našom prípade pre $X = A-B$ dostávame
$$
P(A-B\ge np-3np^3) > 0.
$$
Z toho vyplýva, že existuje voľba jazykov (\tj. elementárna udalosť $\omega$) taká, že ${A(\omega)-B(\omega)}\ge np - 3np^3$. Pre túto voľbu jazykov vieme odstrániť spomedzi zvolených jazykov jeden jazyk za každého účastníka, pre ktorého boli zvolené všetky tri jazyky, ktorými hovorí, a~stále ostane aspoň $A-B$ jazykov. Pre $p = \frac13$ však $A-B\ge \frac29n$, a~teda sme dokázali, čo sme potrebovali.
}

{%%%%%   MEMO, priklad t5
Označme $\alpha$ veľkosť uhla $EAD$. Takú istú veľkosť má aj uhol $EDA$, a~zo zadanej veľkosti $|\uhol ASE|=60^\circ$ ľahko dopočítame $|\uhol AEC| = 120^\circ-\alpha$ a~$|\uhol DEC| = 60^\circ -\alpha$. Označme $F$ obraz bodu $A$ v~osovej súmernosti podľa priamky~$CE$. Uhol $FED$ má veľkosť $(120^\circ-\alpha)-(60^\circ -\alpha)=60^\circ$, a~keďže $|EF| = |EA| = |ED|$,
je trojuholník $DEF$ rovnostranný. Trojuholníky $ABC$ a~$FDC$ sú teda zhodné podľa vety~{\it sss}.
\inspinspab{memo.5}{memo.6}%

Ak sa bod $D$ nachádza mimo trojuholníka $ACF$ (\obr{}a), tak body $B$ a $D$ sú súmerné podľa priamky $CE$,
čiže $|EB| = |ED|$ a štvoruholník $BCDE$ je kosoštvorec. Potom $BC\parallel DE$.

Ak bod $D$ leží v trojuholníku $ACF$ (\obrr1b), sú body $C$ a $F$ súmerné podľa osi $AD$, pretože uhly $ADC$ aj $ADF$ majú veľkosť $60^\circ+\alpha$
(využili sme, že $|\uhol ECD| = |\uhol CED| = 60^\circ-\alpha$). Potom však $|AF| = |AC|$ a trojuholníky $ABC$ a $AEF$ sú zhodné.
Bod $E$ musí ležať mimo trojuholníka $ACF$, inak by päťuholník $ABCDE$ nebol konvexný, čo je v~rozpore so zadaním.
Zo zhodnosti trojuholníkov $ABC$ a $AEF$ potom vyplýva, že body $B$ a $E$ sú súmerné podľa priamky $AD$. Preto $|DB| = |DE|$,
čiže štvoruholník $ABDE$ je kosoštvorec a platí $AB\parallel DE$.
}

{%%%%%   MEMO, priklad t6
Dotyk priamky s~kružnicou vieme popísať pomocou mocnosti bodu ku kružnici.
V~našom prípade z~mocnosti bodu~$B$ ku kružnici opísanej trojuholníku $AXC_0$ dostaneme $|BX|^2 = |BA|\cdot|BC_0|$.
Podobne $|CX|^2=|CA|\cdot|CB_0|$.
\insp{memo.1}%
Označme $A_0$ pätu výšky z~vrcholu~$A$. Štvoruholník $ACA_0C_0$ je tetivový, preto pre mocnosť bodu~$B$ k~jemu opísanej kružnici platí
$|BA|\cdot|BC_0| = |BA_0|\cdot|BC|$; podobne dostaneme $|CA|\cdot|CB_0| = |CA_0|\cdot|CB|$.
Celkovo dostávame
$$
|BX|^2 + |CX|^2 = |BA|\cdot|BC_0| + |CA|\cdot|CB_0| = |BA_0|\cdot|BC| + |CA_0|\cdot|BC| = |BC|^2,
$$
čiže uhol $BXC$ je pravý (\obr). Okrem toho platí $|BX|^2 = |BA_0|\cdot|BC|$ a preto z~Euklidovej vety o~odvesne v~pravouhlom trojuholníku $BXC$ vyplýva, že bod $A_0$ je pätou výšky z~vrcholu~$X$ na preponu~$BC$. Inak povedané, priamky $AA_0$ a $XA_0$ sú totožné, preto je priamka $AX$ kolmá na priamku $BC$.

\ineriesenie
Z mocnosti bodu ku kružnici dostávame
$$
|BX|^2 = |BA|\cdot|BC_0|, \qquad |CX|^2 = |CA|\cdot|CB_0|.
$$
Pre daný trojuholník $ABC$ tieto rovnosti jednoznačne určujú vzdialenosti bodu $X$ od bodov $B$ a $C$.
Do úvahy teda prichádzajú len dva možné body $X$, avšak jeden z~nich vždy leží mimo trojuholníka $ABC$.
Preto je bod $X$ pre daný trojuholník $ABC$ jediný.
Ukážeme, že tento bod $X$ je totožný s~bodom~$Y$, ktorý je priesečníkom Tálesovej kružnice nad priemerom~$BC$
a~výšky trojuholníka $ABC$ z~vrcholu~$A$. To zjavne bude stačiť na dôkaz zadaného tvrdenia.
\insp{memo.7}%

Štvoruholník $BCB_0Y$ je tetivový, preto $|\uhol CBB_0| = |\uhol CYB_0|$. Potom $|\uhol CAY| = 90^\circ - |\uhol ACB| = |\uhol CBB_0| = |\uhol CYB_0|$, a~podľa vety o~úsekovom uhle sa priamka~$CY$ dotýka kružnice opísanej trojuholníku $AYB_0$ (\obr). Analogicky vieme ukázať, že sa priamka $BY$ dotýka kružnice opísanej trojuholníku $AYC_0$, a teda $X = Y$.
}

{%%%%%   MEMO, priklad t7
Stačí ukázať, že existujú $a\in A$ a $b\in B$ také, že $a\equiv 2b\pmod {11}$,
pretože pre takéto $a$, $b$ platí
$$
a^3+ab^2+b^3\equiv 8b^3+2b^3+b^3\equiv 0\pmod {11}.
$$
Predpokladajme, že k nejakému $b\in B$ množina $A$ neobsahuje žiadne vyhovujúce $a$.
Potom však číslo so zvyškom $2b$ po delení $11$ musí byť v množine $B$
(využívame, že ak $b$ nie je deliteľné $11$, ani jeho dvojnásobok nebude).
Ak ani k $2b$ nevieme nájsť vyhovujúce $a\in A$, tak aj číslo so zvyškom $4b$ je v $B$.
A tak ďalej, postupne dôjdeme k záveru, že čísla so zvyškami $b$, $2b$, $4b$, \dots, $2^9b$ sú všetky v $B$.
Tieto čísla však dávajú navzájom rôzne zvyšky po delení $11$ a preto množina $B$ obsahuje $10$ čísel.
Potom však množina $A$ je prázdna, a to je spor s predpokladom zo zadania.
}

{%%%%%   MEMO, priklad t8
Zvoľme najprv čísla $x_1$, $x_2$, \dots, $x_{2011}$ tak, aby čísla
$$
y_1=x_1^2(x_1+2),\quad y_2=x_2^2(x_2+2),\quad\dots,\quad y_{2011}=x_{2011}^2(x_{2011}+2)
$$
boli navzájom nesúdeliteľné. Môžeme zvoliť napríklad $x_1 =1$ a $x_i = {y_1y_2\dots y_{i-1}-1}$ postupne pre každé $i\in\{2,3,\dots,2011\}$.
Takto zvolené číslo $x_i$ zabezpečí, že $y_i$ je nesúdeliteľné s $y_1$, $y_2$, \dots, $y_{i-1}$.

Podstata našej voľby čísel $x_i$ spočíva v tom, že každé číslo deliteľné číslom v tvare $x^2(x+2)$ je úžasné.
Naozaj, ak $n = x^2(x+2)m$, tak $n=\nsd(b,c)\cdot\nsd(a,bc)+\nsd(c,a)\cdot\nsd(b, ca)+\nsd(a,b)\cdot\nsd(c,ab)$  pre $a = mx^2$, $b = mx$, $c = x$.

Keďže čísla $y_1$, $y_2$, \dots, $y_{2011}$ sú navzájom nesúdeliteľné, podľa čínskej zvyškovej vety existuje kladné celé číslo $k$ také, že
$$
k\equiv -i\pmod {y_i}\qquad \hbox{pre každé}\quad i\in\{1,2,\dots,2011\}.
$$
Preto číslo $k+i$ je deliteľné číslom $y_i$ pre každé $i\in\{1,2,\dots,2011\}$.
Takže čísla $k+1$, $k+2$, \dots, $k+2011$ sú všetky úžasné a tvoria hľadaných $2011$ po sebe idúcich úžasných čísel.
}

