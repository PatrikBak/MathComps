{%%%%%   Z4-I-1
Doplň do prázdnych políčok čísla od $1$ do $7$ každé raz tak, aby matematické operácie boli vypočítané správne.
\insp{z60.41}%
}
\podpis{M. Smitková}

{%%%%%   Z4-I-2
Miško a~Jarka sú súrodenci. Jarka má narodeniny niekedy v~januári. O~Miškovi vieme, že v~roku 2010 bola od Jarkiných narodenín po Miškove narodeniny presne jedna sobota trinásteho. Zisti, v~ktorom mesiaci sa narodil Miško. Nájdi všetky možnosti.}
\podpis{M. Dillingerová}

{%%%%%   Z4-I-3
Koľko trojciferných čísiel má prvú číslicu trikrát väčšiu ako druhú a~tretiu číslicu o~$4$ menšiu ako prvú? Vypíš všetky také čísla.}
\podpis{M. Smitková, M. Dillingerová}

{%%%%%   Z4-I-4
Jožkovi sa podarilo rozlámať čokoládu na takéto kúsky:
\insp{z60.42}%

\ite a) Dala by sa táto čokoláda bez ďalšieho lámania spravodlivo rozdeliť dvom kamarátom? Ako?
\ite b) Dala by sa táto čokoláda spravodlivo rozdeliť bez ďalšieho lámania trom kamarátom? Ako?

Ak sa to dá, nájdi vždy aspoň jeden spôsob.
}
\podpis{M. Dillingerová}

{%%%%%   Z4-I-5
Na stôl do kuchyne položila mamička vylúskaný hrach v~miske. Danka a~Janka pochúťku objavili a~začali hrášky z~misky vyjedať. Dohodli sa, že Danka si bude z~misky brať vždy 2~guľôčky hrachu. Janka si bude pravidelne brať 2, 4, 1 a~1 guľôčku hrachu a~potom začne opäť od začiatku. Najskôr si vzala z~misky Danka 2~hrášky, potom Janka~2, opäť Danka~2, Janka svoje~4, atď. Zrazu prišla do kuchyne ich mamička a~prekvapene zhíkla: "Veď v~miske už zostala iba polovica hrachu!" Dievčatá začali byť zvedavé a~spočítali, že tam ostalo 45~guľôčok hrachu. Ak sa mamička nemýlila a~zvyšných 45~guľôčok bola naozaj polovica z~toho, čo bolo v~miske na začiatku, zjedli potom dievčatá rovnako alebo niektorá zjedla viac? Koľko hráškov zjedla Danka? A~koľko ich zjedla Janka?}
\podpis{M. Dillingerová}

{%%%%%   Z4-I-6
V našej bytovke je 10 bytov. Niektoré majú 4, niektoré 3 a niektoré 2 okná. Na našej bytovke je celkom 27 okien. Bytov s dvomi oknami je v bytovke najviac. Koľko je ktorých bytov?}
\podpis{M. Dillingerová}

{%%%%%   Z5-I-1
Vlado má napísané dve čísla, $541$ a~$293$.
Zo šiestich použitých cifier má najskôr vyškrtnúť dve tak, aby súčet dvoch takto
získaných čísel bol najväčší možný.
Potom má z~pôvodných šiestich cifier vyškrtnúť dve tak, aby rozdiel dvoch takto
získaných čísel bol najmenší možný (odčíta menšie číslo od väčšieho).
Ktoré cifry má v~jednotlivých prípadoch vyškrtnúť?}
\podpis{M. Petrová}

{%%%%%   Z5-I-2
V~Trpasličom kráľovstve merajú vzdialenosti v~rozprávkových míľach (rm),
v~rozprávkových siahach (rs) a~v~rozprávkových lakťoch (rl).
Na vstupnej bráne do Trpasličieho kráľovstva je nasledujúca tabuľka na
prevody medzi ich jednotkami a~našimi:
  \itemitem{$\bullet$}  1\,rm = 385\,cm,
  \itemitem{$\bullet$}  1\,rs = 105\,cm,
  \itemitem{$\bullet$}  1\,rl = 250\,mm.

Kráľ Trpaslík~I. nechal premerať vzdialenosť od zámockej brány k~rozprávkovému
jazierku. Traja pozvaní zememerači dospeli k~týmto výsledkom:
prvý nameral 4\,rm~4\,rs 18\,rl,
druhý 3\,rm~2\,rs~43\,rl a~tretí 6\,rm~1\,rs~1\,rl.
Jeden z~nich sa však pomýlil.
Aká je vzdialenosť v~centimetroch od zámockej brány k~rozprávkovému jazierku?
O~koľko centimetrov sa pomýlil nepresný zememerač?}
\podpis{M. Petrová}

{%%%%%   Z5-I-3
Štyria kamaráti Adam, Mojmír a~dvojčatá Peter a~Pavol získali na hodinách
matematiky celkom 52 smajlíkov, každý aspoň~1.
Pritom dvojčatá dokopy majú 33, ale najúspešnejší bol Mojmír.
Koľko ich získal Adam?}
\podpis{M. Volfová}

{%%%%%   Z5-I-4
Pán Tik a~pán Tak predávali budíky v~predajniach "Pred Rohom" a~"Za
Rohom". Pán Tik tvrdil, že "Pred Rohom" predali o~30 budíkov viac ako
"Za Rohom", zatiaľ čo pán Tak tvrdil, že "Pred Rohom" predali trikrát
viac budíkov ako "Za Rohom".
Nakoniec sa ukázalo, že Tik aj Tak mali pravdu.
Koľko budíkov predali v~oboch predajniach celkom?}
\podpis{L. Hozová}

{%%%%%   Z5-I-5
Do krúžkov na obrázku doplňte čísla $1$, $2$, $3$, $4$, $5$, $6$ a~$7$ tak, aby
súčet čísel na každej vyznačenej línii bol rovnaký.
Žiadne číslo pritom nesmie byť použité viackrát.
\insp{z60.1}%
}
\podpis{M. Smitková}

{%%%%%   Z5-I-6
Pani Šikovná čakala večer hostí. Najskôr pre nich pripravila 25~chlebíčkov. Potom
spočítala, že by si každý hosť mohol zobrať dva, ale po troch by už na všetkých
nevyšlo. Povedala si, že keby vyrobila ešte 10~chlebíčkov, mohol by si každý
hosť vziať tri, ale štyri nie každý. To sa jej zdalo stále málo. Nakoniec
prichystala dokopy 52~chlebíčkov. Každý hosť by si teda mohol vziať štyri
chlebíčky, ale po päť by už na všetkých nevyšlo. Koľko hostí pani Šikovná
čakala? Ona sama drží diétu a~večer nikdy neje.}
\podpis{L. Šimůnek}

{%%%%%   Z6-I-1
Keď Bernard natieral dvere garáže, pretrel omylom aj stupnicu nástenného
vonkajšieho teplomera. Trubička s~ortuťou však zostala nepoškodená, a~tak
Bernard pôvodnú stupnicu prelepil pásikom vlastnej výroby. Na nej starostlivo
narysoval dieliky, všetky boli rovnako veľké a~označené číslami. Jeho dielik mal
však inú veľkosť ako pôvodný dielik, ktorý predstavoval jeden stupeň
Celzia, a~aj nulu Bernard
umiestnil inde, ako bolo $0\st$C. Takto začal Bernard merať teplotu vo
vlastných jednotkách: bernardoch. Keď by mal teplomer ukazovať teplotu
$18\st$C, ukazoval 23~bernardov. Keď by mal ukazovať $9\st$C, ukazoval 8~bernardov.
Aká je teplota v~$\st$C, ak vidí Bernard na svojom teplomere teplotu 13~bernardov?}
\podpis{L. Šimůnek}

{%%%%%   Z6-I-2
Firma vyrábajúca mikrovlnné rúry predávala na trhu vždy po krátkej prezentácii svoje modely. Vo
štvrtok predala osem rovnakých mikrovlniek. Deň nato už ponúkala aj svoj nový model a~ľudia si
tak mohli kúpiť ten istý ako vo štvrtok alebo nový. V~sobotu chceli všetci
záujemcovia nový model a~firma ich predala v~ten deň šesť. V~jednotlivých
dňoch utŕžila 590~€, 720~€ a~840~€, neprezradíme však, ktorá suma patrí
ku ktorému dňu.
  \itemitem{$\bullet$} Koľko stál starší model mikrovlnky?
  \itemitem{$\bullet$} Koľko nových modelov predala firma v~piatok?
  
{\it Poznámka.} Cena každej mikrovlnky bola v~celých eurách.
%%Začínající písničkář prodával vždy po vystoupení CD se svou hudbou. Ve
%%čtvrtek prodal osm stejných CD. Den nato už nabízel i~své nové CD a~lidé si
%%tak mohli koupit to samé jako ve čtvrtek nebo nové. V~sobotu chtěli všichni
%%posluchači nové CD a~písničkář jich prodal ten den šest. V~jednotlivých
%%dnech utržil 590~Kč, 720~Kč a~840~Kč, neprozradíme však, která částka patří
%%ke~kterému dni.
%%  \itemitem{$\bullet$} Kolik stálo starší CD?
%%  \itemitem{$\bullet$} Kolik nových CD prodal v~pátek?
}
\podpis{L. Šimůnek}

{%%%%%   Z6-I-3
Vojto napísal číslo $2010$ stokrát bez medzier za sebou.
Koľko štvorciferných a~koľko päťciferných súmerných čísel bolo skrytých v~tomto
zápise?
(Súmerné číslo je také číslo, ktoré je rovnaké, či ho čítame spredu alebo zozadu, napr. $39193$.)}
\podpis{L. Hozová}

{%%%%%   Z6-I-4
Súčin vekov deda Vendelína a~jeho vnúčat je $2010$.
Súčet vekov všetkých vnúčat je $12$ a~žiadne dve vnúčatá nemajú rovnako veľa rokov.
Koľko vnúčat má dedo Vendelín?}
\podpis{L. Hozová}

{%%%%%   Z6-I-5
Na tábore sa dvaja vedúci s~dvoma táborníkmi a~psom potrebovali dostať cez
rieku a~k~dispozícii mali iba jednu loďku s~nosnosťou 65~kg. Našťastie všetci
(okrem psa) dokázali loďku cez rieku priviezť. Každý vedúci vážil
približne 60~kg, každý táborník 30~kg a~pes 12~kg.
Ako si mali počínať? Koľkokrát najmenej musela loďka prekonať rieku?}
\podpis{M. Volfová}

{%%%%%   Z6-I-6
Karol obstaval krabicu s~obdĺžnikovým dnom obrubou z~kocôčok.
Použil práve 22~kocôčok s~hranou~1\,dm, ktoré staval tesne vedľa seba v~jednej vrstve.
Medzi obrubou a~stenami krabice nebola medzera a~celá táto stavba mala
obdĺžnikový pôdorys.
Aké rozmery mohlo mať dno krabice?}
\podpis{M. Krejčová}

{%%%%%   Z7-I-1
Súčin cifier ľubovoľného viacciferného čísla je vždy menší ako toto číslo.
Ak počítame súčin cifier daného čísla, potom súčin cifier tohto
súčinu, potom znova súčin cifier nového súčinu atď., nutne po nejakom
počte krokov dospejeme k~jednocifernému číslu.
Tento počet krokov nazývame {\it perzistencia\/} čísla.
Napr. číslo $723$ má perzistenciu~$2$, lebo ${7\cdot2\cdot3}=42$ (1.~krok)
a~$4\cdot2=8$ (2.~krok).
  \itemitem{$\bullet$} Nájdite najväčšie nepárne číslo, ktoré má navzájom rôzne cifry
    a~perzistenciu~$1$.
  \itemitem{$\bullet$} Nájdite najväčšie párne číslo, ktoré má navzájom rôzne nenulové
    cifry a~perzistenciu~$1$.
  \itemitem{$\bullet$} Nájdite najmenšie prirodzené číslo, ktoré má perzistenciu~$3$.
}
\podpis{S. Bednářová}

{%%%%%   Z7-I-2
Ondro na výlete utratil $\frac23$ peňazí a~zo zvyšku dal ešte $\frac23$ na školu pre
deti z~Tibetu. Za $\frac23$ nového zvyšku kúpil malý darček pre mamičku.
Z~deravého vrecka stratil $\frac45$ zvyšných peňazí, a~keď zo zvyšných dal polovicu malej
sestričke, ostalo mu práve jedno euro.
S~akou sumou išiel Ondro na výlet?}
\podpis{M. Volfová}

{%%%%%   Z7-I-3
Silvia prehlásila:

"Sme tri sestry, ja som najmladšia, Lívia je staršia o~tri roky
a~Edita o~osem.
Naša mamka rada počuje, že všetky (aj s~ňou) máme
v~priemere 21~rokov. Pritom keď som sa narodila, mala mamka už~29."

Pred koľkými rokmi sa Silvia narodila?}
\podpis{M. Volfová}

{%%%%%   Z7-I-4
Juro mal napísané štvorciferné číslo. Toto číslo zaokrúhlil na desiatky, na
stovky a~na tisícky a~všetky tri výsledky zapísal pod pôvodné číslo. Všetky
štyri čísla správne sčítal a~dostal $5\,443$.
Ktoré číslo mal Juro napísané?}
\podpis{M. Petrová}

{%%%%%   Z7-I-5
Laco narysoval kružnicu so stredom~$S$ a~body $A$, $B$, $C$, $D$, ako ukazuje
obrázok. Zistil, že úsečky $SC$ a~$BD$ sú rovnako dlhé. V~akom pomere sú
veľkosti uhlov $ASC$ a~$SCD$?
\insp{z60.3}%
}
\podpis{L. Hozová}

{%%%%%   Z7-I-6
Nájdite všetky trojciferné prirodzené čísla, ktoré sú bezo zvyšku deliteľné
číslom~$6$ a~v~ktorých môžeme vyškrtnúť ktorúkoľvek cifru a~vždy dostaneme
dvojciferné prirodzené číslo, ktoré je tiež bezo zvyšku deliteľné číslom~$6$.}
\podpis{L. Šimůnek}

{%%%%%   Z8-I-1
Martin má na papieri napísané päťciferné číslo s~piatimi rôznymi ciframi a~nasledujúcimi vlastnosťami:
  \itemitem{$\bullet$} škrtnutím druhej cifry zľava (\tj. cifry na mieste tisícok)
    dostane číslo, ktoré je deliteľné dvoma,
  \itemitem{$\bullet$} škrtnutím tretej cifry zľava dostane číslo, ktoré je deliteľné tromi,
  \itemitem{$\bullet$} škrtnutím štvrtej cifry zľava dostane číslo, ktoré je deliteľné štyrmi,
  \itemitem{$\bullet$} škrtnutím piatej cifry zľava dostane číslo, ktoré je deliteľné piatimi,
  \itemitem{$\bullet$} ak neškrtne žiadnu cifru, má číslo deliteľné šiestimi.

Ktoré najväčšie číslo môže mať Martin napísané na papieri?}
\podpis{M. Petrová}

{%%%%%   Z8-I-2
Karol sa snažil do prázdnych políčok na obrázku vpísať prirodzené čísla od $1$ do
$14$ tak, aby žiadne číslo nebolo použité viackrát a~súčet všetkých čísel na každej
priamej línii bol rovnaký.
Po chvíli si uvedomil, že to nie je možné.
Ako by ste Karolovo pozorovanie zdôvodnili vy?
(Pod priamou líniou rozumieme skupinu všetkých susediacich políčok, ktorých stredy
ležia na jednej priamke.)
\insp{z60.2}%
}
\podpis{S. Bednářová}

{%%%%%   Z8-I-3
Cena encyklopédie "Hádanky, rébusy a~hlavolamy" bola znížená o~62,5\,\%. Matej zistil, že obe
ceny (pred znížením aj po ňom) sú dvojciferné čísla a~dajú sa vyjadriť
rovnakými ciframi, len v~rôznom poradí.
O~koľko eur bola encyklopédia zlacnená?}
\podpis{M. Volfová}

{%%%%%   Z8-I-4
Rozdeľte kocku s~hranou 8\,cm na menšie zhodné kocôčky tak, aby súčet
ich povrchov bol päťkrát väčší ako povrch pôvodnej kocky.
Aký bude objem malej kocôčky a~koľko centimetrov bude merať jej hrana?}
\podpis{M. Volfová}

{%%%%%   Z8-I-5
Klára, Lenka a~Matej si precvičovali písomné delenie so zvyškom. Ako delenca
mal každý zadané iné prirodzené číslo, ako deliteľa však mali všetci
rovnaké prirodzené číslo. Lenkin delenec bol o~$30$ väčší ako Klárin. Matejov
delenec bol o~$50$ väčší ako Lenkin. Kláre vyšiel vo výsledku zvyšok~$8$, Lenke
zvyšok~$2$ a~Matejovi zvyšok~$4$. Všetci počítali bez chyby. Aký deliteľ mali
žiaci zadaný?}
\podpis{L. Šimůnek}

{%%%%%   Z8-I-6
V~rovnoramennom lichobežníku $ABCD$ sú uhlopriečky $AC$ a~$DB$ na seba kolmé, ich
dĺžka je 8\,cm a~dĺžka najdlhšej strany~$AB$ je tiež 8\,cm.
Vypočítajte obsah tohto lichobežníka.}
\podpis{M. Krejčová}

{%%%%%   Z9-I-1
Pán Vlk čakal na zastávke pred školou na autobus.
Z~okna počul slová učiteľa:

"Aký povrch môže mať %%kvádr se čtvercovou podstavou,
pravidelný štvorboký hranol, ak viete, že
dĺžky všetkých jeho hrán sú v~centimetroch vyjadrené celými číslami a~že jeho
objem je..."

Toto dôležité číslo pán Vlk nepočul, pretože práve prešlo
okolo auto. Za~chvíľu počul žiaka oznamujúceho výsledok $918\text{cm}^2$.
Učiteľ na to povedal:

"Áno, ale úloha má celkom štyri riešenia. Hľadajte ďalej."

Viac sa pán Vlk už nedozvedel, lebo nastúpil do svojho autobusu.
Keďže matematika bola vždy jeho hobby, vybral si v~autobuse ceruzku a~papier
a~po čase určil aj zvyšné tri riešenia učiteľovej úlohy. Spočítajte ich aj vy.}
\podpis{L. Šimůnek}

{%%%%%   Z9-I-2
Na obrázku sú bodkovanou čiarou znázornené hranice štyroch rovnako veľkých
obdĺžnikových parciel. Sivou farbou je vyznačená zastavaná plocha. Tá má tvar
obdĺžnika, ktorého jedna strana tvorí zároveň hranice parciel. Zapísané čísla
vyjadrujú obsah nezastavanej plochy na jednotlivých parcelách, a~to v~$\text{m}^2$.
Vypočítajte obsah celkovej zastavanej plochy.
\insp{z60.4}%
}
\podpis{L. Šimůnek}

{%%%%%   Z9-I-3
Vĺčkovci lisovali jablkový mušt. Mali ho v~dvoch rovnako objemných súdkoch,
v~oboch takmer rovnaké množstvo. Keby z~prvého preliali do druhého 1~liter,
mali by v~oboch rovnako, ale to by ani jeden súdok nebol plný. Tak radšej
preliali 9~litrov z~druhého do prvého. Potom bol prvý súdok úplne plný a~mušt
v~druhom zapĺňal práve tretinu objemu.
Koľko litrov muštu vylisovali, aký bol objem súdkov a~koľko muštu v~nich bolo
pôvodne?}
\podpis{M. Volfová}

{%%%%%   Z9-I-4
Pán Rýchly a~pán Ťarbák v~rovnakom čase vyštartovali na tú istú turistickú trasu, len
pán Rýchly ju išiel zhora z~horskej chaty a~pán Ťarbák naopak od autobusu
dolu v~mestečku na chatu smerom nahor. Keď bolo 10~hodín, stretli sa na trase. Pán Rýchly
sa ponáhľal a~už o 12:00 bol v~cieli. Naopak pán Ťarbák postupoval pomaly,
%%vše si prohlížel, fotil kamzíky a~sviště a~vodopády
a~tak dorazil na chatu až o~18:00.
O~koľkej páni vyrazili na cestu, ak vieme, že každý z~nich
išiel celý čas svojou stálou rýchlosťou?}
\podpis{M. Volfová}

{%%%%%   Z9-I-5
Kružnici so stredom~$S$ a~polomerom 12\,cm sme opísali pravidelný
šesťuholník $ABCDEF$ a~vpísali pravidelný šesťuholník $TUVXYZ$ tak, aby
bod~$T$ bol stredom strany~$BC$.
Vypočítajte obsah a~obvod štvoruholníka $TCUS$.}
\podpis{M. Krejčová}

{%%%%%   Z9-I-6
Peter a~Pavol oberali v~sade jablká a~hrušky. V~pondelok zjedol Peter o~2~hrušky
viac ako Pavol a~o~2~jablká menej ako Pavol. V~utorok Peter zjedol o~4~hrušky
menej ako v~pondelok. Pavol zjedol v~utorok o~3~hrušky viac ako Peter
a~o~3~jablká menej ako Peter. Pavol zjedol za oba dni 12~jabĺk a~v~utorok zjedol
rovnaký počet jabĺk ako hrušiek. V~utorok večer obaja chlapci zistili, že
počet jabĺk, ktoré spolu za oba dni zjedli, je rovnako veľký ako počet
spoločne zjedených hrušiek.
Koľko jabĺk zjedol Peter v~pondelok a~koľko hrušiek zjedol Pavol v~utorok?}
\podpis{L. Hozová}

{%%%%%   Z4-II-1
Doplň do prázdnych políčok 8 za sebou idúcich jednociferných čísel každé raz tak, aby matematické operácie boli vypočítané správne.
\insp{z60ii.41}%
}
\podpis{M. Dillingerová, M. Smitková}

{%%%%%   Z4-II-2
Pred školou v~Kocúrkove stáli bicykle a~autá. Keby sa tu zastavilo ešte jedno auto, bolo by ich toľko ako bicyklov. Keby sa tu zastavilo ešte 5~bicyklov, mali by rovnako veľa kolies ako autá. Koľko stálo pred školou áut? Koľko tam stálo bicyklov?}
\podpis{M. Dillingerová}

{%%%%%   Z4-II-3
Janko dostal na Vianoce knihu, ktorú hneď v~ten deň začal čítať. V~posledný januárový deň nového roka Janko zistil, že prečítal 60~strán, čo je polovica knihy a~povedal si, že ak chce knihu celú dočítať do svojich narodenín, musí každý deň prečítať 5~strán. Kedy (presne ktorý deň a~mesiac) má Janko narodeniny?
}
\podpis{M. Smitková}

{%%%%%   Z5-II-1
Miro napísal do radu všetky násobky čísla sedem počnúc $7$
a~končiac $70$. Medzi číslami nepísal čiarky ani medzery a~čísla napísal
od najmenšieho po najväčšie. V~tomto rade čísel potom škrtol jedenásť cifier. Zistite,
aké najväčšie a~aké najmenšie číslo mohol dostať.
}
\podpis{M. Petrová}

{%%%%%   Z5-II-2
Rytier Miloslav sa chystal na turnaj do Veselína. Turnaj sa konal v~stredu.
Keďže cesta z~Rytierova, kde býva, do Veselína trvá až dva dni,
vyrazil už v~pondelok. Cesta vedie cez ďalšie dve mestá, Kostín a~Zubín.
Prvý deň jazdy prešiel Miloslav 25~míľ a~prenocoval v~Zubíne. Druhý deň, v~utorok, šťastne došiel do Veselína. Turnaj s~prehľadom
vyhral, takže keď sa vo štvrtok vracal späť, išiel rýchlejšie. Prešiel o~6~míľ
viac ako v~pondelok a~prenocoval v~Kostíne. V~piatok prešiel zvyšných 11~míľ
a~bol doma. Zisti vzdialenosť medzi Zubínom a~Veselínom.
}
\podpis{M. Petrová}

{%%%%%   Z5-II-3
Správca kúpeľov pán Slniečko kúpil pre kúpeľných hostí 58~slnečníkov.
Niektoré boli červené a~niektoré žlté. Červené boli balené v~krabiciach
po deviatich kusoch. Žlté boli balené v~krabiciach po štyroch kusoch.
Oba druhy slnečníkov nakupoval po celých baleniach. Koľko mohlo byť žltých slnečníkov?
}
\podpis{L. Hozová}

{%%%%%   Z6-II-1
Pani Hundravá mala 1.~júla 2010 na svojom mobile kredit 3{,}14€.
Z~kreditu sa postupne odpočítavajú čiastky za hovory a~to tak, že za každú
začatú minútu sa odčíta 9~centov. Textové správy pani Hundravá nepíše a~nevyužíva ani žiadne ďalšie platené služby. Svoj kredit dobíja podľa potreby a~to vždy sumou 8€. Dňa 31.~decembra 2010 bol jej kredit 7{,}06€. Koľkokrát minimálne dobíjala pani Hundravá za uvedený polrok svoj kredit?}
\podpis{L. Šimůnek}

{%%%%%   Z6-II-2
V~obdĺžniku $KLMN$ je vzdialenosť priesečníka jeho uhlopriečok od priamky~$KL$ o~$2\text{cm}$  menšia ako
jeho vzdialenosť od priamky~$LM$. Obvod obdĺžnika je $56\text{cm}$. Aký je obsah obdĺžnika $KLMN$?}
\podpis{L. Hozová}

{%%%%%   Z6-II-3
V~lete sa u babičky stretlo všetkých jej šesť vnúčat. O~vnúčatách nám babička prezradila, že
\begin{itemize}
  \iitem Martinka sa niekedy musí starať o~bračeka Tomáška, ktorý je o~8~rokov mladší,
  \iitem Vierka, ktorá je o~7~rokov staršia ako Ivana, rada rozpráva strašidelné príbehy,
  \iitem s~Martinkou sa často hašterí o~rok mladší Jaromír,
  \iitem Tomáško je o~11~rokov mladší ako Katka,
  \iitem Ivana často hnevá svojho o~4~roky staršieho brata Jaromíra,
  \iitem chlapci majú dokopy 13~rokov.
\end{itemize}
Koľko rokov majú jednotlivé deti?}
\podpis{M. Volfová}

{%%%%%   Z7-II-1
Mám kartičku, na ktorej je napísané štvorciferné prirodzené číslo. V~tomto čísle môžeme
vyškrtnúť akékoľvek dve cifry a~vždy dostaneme dvojciferné prirodzené číslo,
ktoré je bezo zvyšku deliteľné číslom~$5$. Koľko takých štvorciferných prirodzených čísel existuje?
(Pozor, napr. $06$ nie je dvojciferné číslo!)}
\podpis{L. Šimůnek}

{%%%%%   Z7-II-2
Karol a~Vojto zistili, že kuchynské hodiny na chalupe každú hodinu nadbehnú o~1{,}5~minúty
a~hodiny v~spálni každú hodinu pol minúty meškajú. Druhého apríla
na pravé poludnie nastavili hodiny na rovnaký a~správny čas.
Urči, kedy opäť budú (bez ďalšieho opravovania)
\begin{enumerate}
  \iitem kuchynské hodiny ukazovať správny čas;
  \iitem hodiny v~spálni ukazovať správny čas;
  \iitem oboje hodiny ukazovať rovnaký (aj~keď nie nutne správny) čas.
\end{enumerate}
(Hodiny v~kuchyni aj v~spálni majú dvanásťhodinový ciferník.)}
\podpis{M. Volfová}

{%%%%%   Z7-II-3
V~trojuholníku $ABC$ označíme stredy strán $CB$ a~$CA$ písmenami $K$ a~$L$.
Vieme, že štvoruholník $ABKL$ má obvod $10\text{cm}$ a~trojuholník $KLC$ má obvod
$6\text{cm}$.
Vypočítaj dĺžku úsečky~$KL$.}
\podpis{J. Mazák}

{%%%%%   Z8-II-1
Myslím si dvojciferné prirodzené číslo. Súčet cifier tohto čísla je deliteľný tromi.
Keď odčítam od mysleného čísla číslo~$27$, dostanem iné dvojciferné prirodzené
číslo, zapísané pomocou tých istých cifier, ale v~opačnom poradí. Aké čísla si môžem
myslieť?}
\podpis{L. Hozová}

{%%%%%   Z8-II-2
Martina si vymyslela postup na výrobu číselnej postupnosti. Začala číslom $52$.
Z~neho odvodila ďalší člen postupnosti takto: $2^2+2\cdot 5=4+10=14$.
Potom pokračovala rovnakým spôsobom ďalej a~z~čísla $14$ dostala
$4^2+2\cdot 1=16+2=18$. Vždy teda vezme číslo, odtrhne z~neho cifru na mieste
jednotiek, túto odtrhnutú cifru umocní na druhú a~k~výslednej mocnine pripočíta
dvojnásobok čísla, ktoré ostalo po odtrhnutí poslednej cifry. Aké je $2011$.~číslo takto vytvorenej postupnosti?
%Martina si vymyslela postup na výrobu číselnej postupnosti.
%Začala číslom $118$. Z~neho odvodila ďalší člen postupnosti takto: $8^2-11=64-11=53$.
%Potom pokračovala rovnakým spôsobom ďalej a~z~čísla $53$ dostala $3^2-5=9-5=4$.
%Vždy teda vezme číslo, odtrhne z~neho cifru na mieste jednotiek,
%túto odtrhnutú cifru umocní na druhú a~od výslednej mocniny odpočíta zvyšok
%pôvodného čísla po odtrhnutí poslednej cifry. (Ak je číslo jednociferné, teda po odtrhnutí neostane žiadne číslo na odčítanie,
%odčíta Martina od druhej mocniny číslo $0$.)
%Aké je 2011.~číslo takto vytvorenej postupnosti?
}
\podpis{M. Dillingerová}

{%%%%%   Z8-II-3
V~kružnici~$k$ so stredom~$S$ a~polomerom $52\text{mm}$ sú dané dve na seba kolmé
tetivy $AB$ a~$CD$.
Ich priesečník $X$ je od stredu $S$ vzdialený $25\text{mm}$.
Aká dlhá je tetiva~$CD$, ak dĺžka tetivy~$AB$ je $96\text{mm}$?}
\podpis{L. Hozová}

{%%%%%   Z9-II-1
Koľko existuje dvojíc štvorciferných palindrómov, ktorých rozdiel je $3674$?
Palindróm je číslo, ktoré ostane rovnaké, keď ho napíšeme odzadu.
Štvorciferný palindróm je teda také štvorciferné prirodzené číslo, ktoré má na mieste jednotiek
rovnakú cifru ako na mieste tisícok a~na mieste desiatok rovnakú cifru ako na
mieste stoviek.
%Štvorciferný palindróm je každé štvorciferné prirodzené číslo, ktoré
%má na mieste jednotiek rovnakú cifru ako na mieste tisícok, a~ktoré
%má na mieste desiatok rovnakú cifru ako na mieste stoviek.
%Koľko existuje dvojíc štvorciferných palindrómov, ktorých rozdiel je
%$3674$?
}
\podpis{L. Šimůnek}

{%%%%%   Z9-II-2
Na obrázku sú rovnostranné trojuholníky $ABC$, $DBE$, $IEF$ a~$HIG$.
Obsahy trojuholníkov $DBE$, $IEF$ a~$HIG$ sú v~pomere $9:16:4$. V~akom pomere sú
  \itemitem{1.}dĺžky úsečiek $HI$ a~$IE$,
  \itemitem{2.}obsahy trojuholníkov $ABC$ a~$HEC$?
%\epsfbox{zdroje-II/z9-trojuhelnik.eps}
\insp{z60ii.91}
}
\podpis{K. Pazourek}

{%%%%%   Z9-II-3
Dané sú štvorce $ABCD$ a~$KLMN$. Dĺžky strán oboch štvorcov sú v~centimetroch
vyjadrené celým číslom. Bod $K$ je vnútorným bodom úsečky $AB$, bod $L$ leží v~bode
$B$ a~bod $M$ je vnútorným bodom úsečky $BC$. Obsah šesťuholníka $AKNMCD$ je
$225\text{cm}^2$.
Aký môže byť obvod tohto šesťuholníka? Nájdite všetky možnosti.
}
\podpis{L. Šimůnek}

{%%%%%   Z9-II-4
Martina si vymyslela postup na výrobu číselnej postupnosti.
Začala číslom $128$.
Z~neho odvodila ďalší člen postupnosti takto: $8^2+5=64+5=69$.
Potom pokračovala rovnakým spôsobom a~z~čísla $69$ dostala $9^2+5=81+5=86$.
Vždy teda z predchádzajúceho člena postupnosti vezme cifru na mieste jednotiek,
umocní ju na druhú a~k~tejto mocnine pripočíta konštantu~$5$.
  \itemitem{1.}  Aké je 2011.~číslo tejto postupnosti?
  \itemitem{2.} Martina opäť začala číslom $128$, ale namiesto čísla $5$ zvolila ako
    konštantu iné prirodzené číslo.
    Tentoraz jej na 2011.~mieste vyšlo číslo $16$.
    Akú konštantu zvolila v~tomto prípade?\endgraf
}
\podpis{M. Dillingerová}

{%%%%%   Z9-III-1
Usporiadateľom výstavy "Na Mesiac a~ešte ďalej" sa po prvom výstavnom dni zdalo,
že málo ľudí si kúpilo na pamiatku leták o~rakete Apollo~11. Preto znížili jeho
cenu o~12~centov. Tým sa síce druhý deň zvýšil počet kupcov letáku o~10\,\%, ale
celková denná tržba za letáky sa znížila o~5\,\%.
Koľko centov stál leták Apollo~11 po zľave?}
\podpis{M. Petrová}

{%%%%%   Z9-III-2
Lichobežník $ABCD$, v~ktorom strana~$AB$ je rovnobežná so stranou~$CD$, je
rozdelený uhlopriečkami, ktoré sa pretínajú v~bode $M$, na štyri časti.
Určte jeho obsah, keď viete, že trojuholník $AMD$ má obsah
$8\text{cm}^2$ a~trojuholník $DCM$  má obsah $4\text{cm}^2$.}
\podpis{M. Volfová}

{%%%%%   Z9-III-3
Cyril a~Mirka počítali zo zbierky tú istú úlohu. Zadané boli dĺžky hrán
kvádra v~milimetroch a~úlohou bolo vypočítať jeho objem a~povrch. Cyril
najskôr previedol zadané dĺžky na centimetre. Počítalo sa mu tak ľahšie,
pretože aj po prevode boli všetky dĺžky vyjadrené celými číslami. Obom vyšli
správne výsledky, Mirke v~$\text{mm}^3$ a~$\text{mm}^2$, Cyrilovi v~$\text{cm}^3$ a~$\text{cm}^2$. Mirkin výsledok
v~$\text{mm}^3$ bol o~$17\,982$ väčší ako Cyrilov výsledok v~$\text{cm}^3$. Mirkin výsledok v~$\text{mm}^2$
bol o~$5\,742$ väčší ako Cyrilov výsledok v~$\text{cm}^2$. Určte dĺžky hrán kvádra.}
\podpis{L. Šimůnek}

{%%%%%   Z9-III-4
Na tabuli sú napísané čísla $1$, $\frac12$, $\frac13$, $\frac14$,
$\frac15$ a~$\frac16$.
Na tabuľu môžeme pripísať súčet alebo súčin ľubovoľných dvoch čísel z~tabule.
Je možné takýmto pripisovaním dosiahnuť, aby sa na tabuli objavilo
číslo
$$
\text{a)}\ \frac1{60};\qquad
\text{b)}\ \frac{2011}{375};\qquad
\text{c)}\ \frac17?
$$}
\podpis{V. Bachratá, J. Mazák}

