{%%%%%   A-I-1
Korene rovnice
$$
ax^4+bx^2+a=1
$$
v~obore reálnych čísel sú štyri po sebe idúce členy rastúcej aritmetickej postupnosti.
Pritom jeden z~týchto členov je zároveň riešením rovnice
$$
bx^2+ax+a=1.
$$
Určte všetky možné hodnoty reálnych parametrov $a$, $b$.
}
\podpis{Peter Novotný}

{%%%%%   A-I-2
Nech $k$, $n$ sú prirodzené čísla. Z~platnosti tvrdenia "číslo $(n-1)(n+1)$
je deliteľné číslom~$k$" Adam usúdil, že buď číslo $n-1$, alebo číslo $n+1$ je deliteľné~$k$.
Určte všetky prirodzené čísla~$k$, pre ktoré je Adamova úvaha správna pre každé
prirodzené~$n$.
}
\podpis{Ján Mazák}

{%%%%%   A-I-3
Dané sú kružnice $k$, $l$, ktoré sa pretínajú v~bodoch $A$, $B$. Označme $K$, $L$
postupne dotykové body ich spoločnej dotyčnice zvolené tak, že bod~$B$ je vnútorným
bodom trojuholníka $AKL$. Na kružniciach $k$ a~$l$ zvoľme postupne body $N$ a~$M$ tak,
aby bod~$A$ bol vnútorným bodom úsečky~$MN$. Dokážte, že
štvoruholník $KLMN$ je tetivový práve vtedy, keď priamka~$MN$ je dotyčnicou kružnice opísanej
trojuholníku $AKL$.
}
\podpis{Jaroslav Švrček}

{%%%%%   A-I-4
Majme $6n$ žetónov až na farbu zhodných, po troch z~každej z~$2n$ farieb.
Pre každé prirodzené číslo $n > 1$ určte počet $p_n$ všetkých takých rozdelení $6n$ žetónov
na dve kôpky po $3n$~žetónoch, že
žiadne tri žetóny rovnakej farby nie sú v~rovnakej kôpke. Dokážte, že $p_n$ je nepárne číslo
práve vtedy, keď $n = 2^k$ pre vhodné prirodzené~$k$.
}
\podpis{Jaromír Šimša}

{%%%%%   A-I-5
Na každej stene kocky je napísané práve jedno celé číslo. V~jednom kroku
zvolíme ľubovoľné dve susedné steny kocky a~čísla na nich napísané
zväčšíme o~$1$. Určte nutnú a~postačujúcu podmienku pre očíslovanie stien kocky
na začiatku, aby po konečnom počte vhodných krokov mohli byť na
všetkých stenách kocky rovnaké čísla.
}
\podpis{Peter Novotný}

{%%%%%   A-I-6
Dokážte, že v~každom trojuholníku $ABC$ s~ostrým uhlom pri vrchole~$C$
(pri zvyčajnom označení dĺžok strán a~veľkostí vnútorných uhlov)
platí nerovnosť
$$
(a^2 + b^2) \cos(\alpha - \beta) \le 2ab.
$$
Zistite, kedy nastane rovnosť.
}
\podpis{Jaromír Šimša}

{%%%%%   B-I-1
V~obore reálnych čísel vyriešte sústavu
$$
\align
\sqrt{x^2+y^2}&=z+1,\\
\sqrt{y^2+z^2}&=x+1,\\
\sqrt{z^2+x^2}&=y+1.
\endalign
$$
}
\podpis{Tomáš Jurík}

{%%%%%   B-I-2
Uvažujme vnútorný bod~$P$ daného obdĺžnika $ABCD$ a~označme postupne $Q$, $R$ obrazy bodu~$P$
v~súmernostiach podľa stredov $A$, $C$. Predpokladajme, že priamka~$QR$ pretne strany
$AB$ a~$BC$ vo vnútorných bodoch $M$ a~$N$.
Zostrojte množinu všetkých bodov~$P$, pre ktoré platí $|MN|=|AB|$.
}
\podpis{Jaroslav Švrček}

{%%%%%   B-I-3
Nech $a$, $b$, $c$ sú reálne čísla, ktorých súčet je~$6$.
Dokážte, že aspoň jedno z~čísel
$$
ab+bc,\quad bc+ca,\quad ca+ab
$$
nie je väčšie ako $8$.
}
\podpis{Ján Mazák}

{%%%%%   B-I-4
Nájdite všetky celé čísla~$n$, pre ktoré je zlomok
$$
\frac {n^3+2\,010}{n^2+2\,010}
$$
rovný celému číslu.}
\podpis{Pavel Novotný}

{%%%%%   B-I-5
Zaoberajme sa otázkou, ktoré trojuholníky $ABC$ s~ostrými uhlami
pri vrcholoch $A$ a~$B$ majú nasledujúcu vlastnosť:
Ak vedieme stredom výšky z~vrcholu~$C$ tri priamky
rovnobežné so stranami trojuholníka $ABC$, pretnú ich tieto priamky
v~šiestich bodoch ležiacich na jednej kružnici.
\ite a) Ukážte, že vyhovuje každý trojuholník $ABC$ s~pravým uhlom pri vrchole~$C$.
\ite b) Vysvetlite, prečo žiadny iný trojuholník $ABC$ nevyhovuje.}
\podpis{Jaromír Šimša}

{%%%%%   B-I-6
Určte počet desaťciferných čísel, v~ktorých možno škrtnúť dve susedné cifry
a~dostať tak číslo $99$-krát menšie.}
\podpis{Ján Mazák}

{%%%%%   C-I-1
Lucia napísala na tabuľu dve nenulové čísla. Potom medzi ne postupne vkladala znamienka
plus, mínus, krát a~delené a~všetky štyri príklady správne vypočítala. Medzi výsledkami
boli iba dve rôzne hodnoty. Aké dve čísla mohla Lucia na tabuľu %%na začátku
napísať?}
\podpis{Peter Novotný}

{%%%%%   C-I-2
Dokážte, že výrazy $23x + y$, $19x + 3y$ sú deliteľné číslom~$50$ pre rovnaké dvojice
prirodzených čísel $x$,~$y$.}
\podpis{Jaroslav Zhouf}

{%%%%%   C-I-3
Máme štvorec $ABCD$ so stranou dĺžky $1\cm$. Body $K$ a~$L$ sú
stredy strán $DA$ a~$DC$. Bod~$P$ leží na strane~$AB$ tak, že $|BP| =
2|AP|$. Bod~$Q$ leží na strane~$BC$ tak, že $|CQ| = 2|BQ|$. Úsečky
$KQ$ a~$PL$ sa pretínajú v~bode~$X$. Obsahy štvoruholníkov $APXK$,
$BQXP$, $QCLX$ a~$LDKX$ označíme postupne $S_A$, $S_B$, $S_C$, $S_D$ (\ifobrazkyvedla{}ako na obrázku\else\obr{}\fi).
\ite a) Dokážte, že $S_B=S_D$.
\ite b) Vypočítajte rozdiel $S_C-S_A$.
\ite c) Vysvetlite, prečo neplatí $S_A + S_C = S_B + S_D$.
\ifobrazkyvedla\else\insp{c60.1}\fi%
}
\podpis{Peter Novotný}

{%%%%%   C-I-4
V~skupine $n$~žiakov sa spolu niektorí kamarátia. Vieme, že každý má medzi ostatnými %spolužáky
aspoň štyroch kamarátov. Učiteľka chce žiakov rozdeliť na dve nanajvýš štvorčlenné skupiny
tak, že každý bude mať vo svojej skupine aspoň jedného kamaráta.
\ite a) Ukážte, že v~prípade $n = 7$ sa dajú žiaci požadovaným spôsobom vždy rozdeliť.
\ite b) Zistite, či možno žiakov takto vždy rozdeliť aj v~prípade $n = 8$.}
\podpis{Tomáš Jurík}

{%%%%%   C-I-5
Dokážte, že najmenší spoločný násobok $[a, b]$ a~najväčší spoločný deliteľ $(a, b)$ ľubovoľných
dvoch kladných celých čísel $a$, $b$ spĺňajú nerovnosť
$$
a~\cdot (a, b) + b \cdot [a, b] \ge 2ab.
$$
Zistite, kedy v~tejto nerovnosti nastane rovnosť.}
\podpis{Jaromír Šimša}

{%%%%%   C-I-6
Je daný lichobežník $ABCD$. Stred základne~$AB$ označme~$P$.
Uvažujme rovnobežku so základňou~$AB$, ktorá pretína úsečky $AD$, $PD$, $PC$, $BC$ postupne
v~bodoch $K$, $L$, $M$,~$N$.
\ite a) Dokážte, že $|KL| =|MN|$.
\ite b) Určte polohu priamky~$KL$ tak, aby platilo aj $|KL|=|LM|$.}
\podpis{Jaroslav Zhouf}

{%%%%%   A-S-1
Určte všetky reálne čísla~$c$, ktoré možno s~oboma koreňmi kvadratickej rovnice
$$
x^2+\frac52x+c=0
$$
usporiadať do trojčlennej aritmetickej postupnosti.}
\podpis{Pavel Calábek, Jaroslav Švrček}

{%%%%%   A-S-2
Nech $P$, $Q$, $R$ sú body prepony~$AB$ pravouhlého trojuholníka $ABC$, pre ktoré platí $|AP|=|PQ|=|QR|=|RB|=\frac14|AB|$. Dokážte, že priesečník~$M$ kružníc opísaných trojuholníkom $APC$ a~$BRC$, ktorý je rôzny od bodu~$C$, je totožný so stredom~$S$ úsečky~$CQ$.}
\podpis{Peter Novotný}

{%%%%%   A-S-3
Dokážte, že pre ľubovoľné dve rôzne prvočísla $p$, $q$ väčšie ako $2$ platí nerovnosť
$$
\left|\frac{p}{q}-\frac{q}{p}\right|>\frac{4}{\sqrt{pq}}.
$$
}
\podpis{Jaromír Šimša}

{%%%%%   A-II-1
Rozhodnite, či medzi všetkými osemcifernými násobkami čísla $4$ je viac tých, ktoré vo svojom dekadickom zápise obsahujú cifru~$1$, alebo tých, ktoré cifru $1$ neobsahujú.}
\podpis{Ján Mazák}

{%%%%%   A-II-2
Daný je trojuholník $ABC$ s~obsahom~$S$. Vnútri trojuholníka, ktorého vrcholmi sú stredy strán trojuholníka $ABC$,
je ľubovoľne zvolený bod~$O$. Označme $A'$, $B'$, $C'$ postupne obrazy bodov $A$, $B$, $C$ v~stredovej súmernosti podľa bodu~$O$.
Dokážte, že šesťuholník $AC'BA'CB'$ má obsah $2S$.}
\podpis{Pavel Leischner}

{%%%%%   A-II-3
Určte všetky dvojice $(m,n)$ kladných celých čísel, pre ktoré je číslo $4(mn+1)$ deliteľné číslom $(m+n)^2$.}
\podpis{Tomáš Jurík}

{%%%%%   A-II-4
Nech $\mm M$ je množina šiestich navzájom rôznych kladných celých čísel, ktorých súčet je~$60$. Všetky ich napíšeme na steny kocky, na každú práve jedno z~nich. V~jednom kroku zvolíme ľubovoľné tri steny kocky, ktoré majú spoločný vrchol, a~každé z~čísel
na týchto troch stenách zväčšíme o~$1$. Určte počet všetkých takých množín~$\mm M$, ktorých čísla možno napísať na steny kocky uvedeným spôsobom tak, že po konečnom počte vhodných krokov budú na všetkých stenách rovnaké čísla.}
\podpis{Peter Novotný}

{%%%%%   A-III-1
Určte veľkosti vnútorných uhlov všetkých trojuholníkov $ABC$ s~vlastnosťou:
Vnútri strán $AB$, $AC$ existujú postupne body $K$, $M$, ktoré s~priesečníkom~$L$
priamok $MB$ a~$KC$ tvoria tetivové štvoruholníky $AKLM$ a~$KBCM$ so
zhodnými opísanými kružnicami.}
\podpis{Jaroslav Švrček}

{%%%%%   A-III-2
Určte všetky trojice prvočísel $(p,q,r)$, pre ktoré platí
$$
\postdisplaypenalty=10000
(p+1)(q+2)(r+3)=4pqr.
$$
}
\podpis{Jaromír Šimša}

{%%%%%   A-III-3
Predpokladajme, že reálne čísla $x$, $y$, $z$ vyhovujú sústave rovníc
$$
x+y+z=12, \qquad x^2+y^2+z^2=54.
$$
Dokážte, že potom platí nasledujúce tvrdenie:
\ite a) Každé z~čísel $xy$, $yz$, $zx$ je aspoň $9$, avšak nanajvýš $25$.
\ite b) Niektoré z~čísel $x$, $y$, $z$ je nanajvýš $3$ a~iné z~nich je aspoň~$5$.
}
\podpis{Jaromír Šimša}

{%%%%%   A-III-4
Uvažujme kvadratický trojčlen $ax^2+bx+c$ s~reálnymi koeficientmi $a\ge2$, ${b\ge 2}$, $c\ge 2$.
Adam a~Boris hrajú nasledujúcu hru: Ak je na ťahu Adam, vyberie jeden z~koeficientov trojčlena a~nahradí ho
{\it súčtom\/} zvyšných dvoch. Ak je na ťahu Boris, vyberie jeden
z~koeficientov a~nahradí ho {\it súčinom\/} zvyšných dvoch. Adam začína a~hráči
sa pravidelne striedajú. Hru vyhráva ten, po ktorého ťahu má vzniknutý trojčlen
dva rôzne reálne korene. V~závislosti od koeficientov $a$, $b$, $c$ počiatočného trojčlena
určte, ktorý z~hráčov má víťaznú stratégiu.
}
\podpis{Michal Rolínek}

{%%%%%   A-III-5
V~ostrouhlom trojuholníku $ABC$, ktorý nie je rovnostranný, označme $P$ pätu výšky z~vrcholu~$C$ na
stranu~$AB$, $V$ priesečník výšok, $O$ stred kružnice
opísanej, $D$ priesečník polpriamky~$CO$ so stranou~$AB$ a~$E$ stred úsečky~$CD$.
Dokážte, že priamka~$EP$ prechádza stredom úsečky~$OV$.}
\podpis{Karel Horák}

{%%%%%   A-III-6
Označme $\Bbb R^+$ množinu všetkých kladných reálnych čísel.
Určte všetky funkcie $f\colon\Bbb R^{+}\to \Bbb R^{+}$ také, že pre ľubovoľné
$x,y\in \Bbb R^{+}$ platí
$$
f(x)\cdot f(y)=f(y)\cdot f\bigl(x\cdot f(y)\bigr)+\frac{1}{xy}.
$$}
\podpis{Pavel Calábek}

{%%%%%   B-S-1
V~obore reálnych čísel vyriešte rovnicu
$$
\sqrt{x+3}+\sqrt{x}=p
$$
s~neznámou~$x$ a~reálnym parametrom~$p$.}
\podpis{Vojtech Bálint}

{%%%%%   B-S-2
Pozdĺž kružnice je rozmiestnených 16 reálnych čísel so súčtom $7$.
\ite a) Dokážte, že existuje úsek piatich susedných čísel so súčtom aspoň $2$.
\ite b) Určte najmenšie $k$ také, že v~opísanej situácii možno vždy nájsť úsek $k$~susedných čísel so súčtom aspoň $3$.\endgraf}
\podpis{Ján Mazák}

{%%%%%   B-S-3
Zvonka daného trojuholníka $ABC$ sú zostrojené štvorce $ACDE$, $BCGF$.
Dokážte, že $|AG|=|BD|$. Ďalej ukážte, že stredy oboch štvorcov spolu so stredmi úsečiek $AB$ a~$DG$ sú vrcholmi štvorca.}
\podpis{Pavel Leischner}

{%%%%%   B-II-1
Súčin kladných reálnych čísel $a$, $b$, $c$ je
$60$ a~ich súčet je $15$. Dokážte nerovnosť
$$
(a+b)(a+c)\ge60
$$
a~zistite, pre ktoré také čísla $a$, $b$, $c$
nastane rovnosť.}
\podpis{Jaromír Šimša}

{%%%%%   B-II-2
Nájdite všetky dvojice kladných celých čísel $a$, $b$, pre ktoré číslo~$b$ je deliteľné číslom~$a$ a~súčasne číslo $3a+4$ je deliteľné číslom $b+1$.}
\podpis{Pavel Novotný}

{%%%%%   B-II-3
Nech $M$, $N$ sú postupne vnútorné body strán $AB$, $BC$ rovnostranného trojuholníka $ABC$, pre ktoré platí $|AM|:|MB|=|BN|:|NC|=2:1$.
Označme $P$ priesečník priamok $AN$ a~$CM$. Dokážte, že priamky $BP$ a~$AN$ sú navzájom kolmé.}
\podpis{Jaroslav Švrček}

{%%%%%   B-II-4
Zapíšeme všetky päťciferné čísla, v~ktorých sa každá z~cifier $4$, $5$, $6$, $7$, $8$ vyskytuje práve raz.
Potom jedno (ľubovoľné z~nich) škrtneme a~všetky zvyšné sčítame.
Aké sú možné hodnoty ciferného súčtu takého výsledku?}
\podpis{Šárka Gergelitsová}

{%%%%%   C-S-1
Po okruhu behajú dvaja atléti, každý inou konštantnou rýchlosťou. Keď bežia opačnými smermi, stretávajú sa každých 10~minút, keď bežia rovnakým smerom, stretávajú sa každých 40~minút. Za aký čas zabehne okruh rýchlejší atlét?}
\podpis{Vojtech Bálint}

{%%%%%   C-S-2
Daný je štvorec so stranou dĺžky $6\cm$. Nájdite množinu stredov všetkých priečok štvorca, ktoré ho delia na dva štvoruholníky, z~ktorých jeden má obsah $12\cm^2$. (Priečka štvorca je úsečka, ktorej krajné body ležia na stranách štvorca.)}
\podpis{Pavel Leischner}

{%%%%%   C-S-3
Nech $x$, $y$ sú také kladné celé čísla, že obe čísla $3x+5y$ a~$5x+2y$ sú deliteľné číslom~$60$. Zdôvodnite, prečo číslo $60$ delí aj súčet $2x+3y$.}
\podpis{Jaromír Šimša}

{%%%%%   C-II-1
Na tabuli sú napísané práve tri (nie nutne rôzne) reálne čísla. Vieme, že súčet ľubovoľných dvoch z~nich je tam napísaný tiež. Určte všetky trojice takých čísel.}
\podpis{Ján Mazák}

{%%%%%   C-II-2
Nájdite všetky kladné celé čísla~$n$, pre ktoré je číslo $n^2+6n$ druhou mocninou celého čísla.}
\podpis{Vojtech Bálint}

{%%%%%   C-II-3
V~lichobežníku $ABCD$ má základňa~$AB$ dĺžku $18\cm$ a~základňa~$CD$ dĺžku~$6\cm$. Pre bod~$E$ strany~$AB$ platí $2|AE|=|EB|$. Body $K$, $L$, $M$, ktoré sú postupne ťažiskami trojuholníkov $ADE$, $CDE$, $BCE$, tvoria vrcholy rovnostranného trojuholníka.
\itemitem{a)} Dokážte, že priamky $KM$ a~$CM$ zvierajú pravý uhol.
\itemitem{b)} Vypočítajte dĺžky ramien lichobežníka $ABCD$.}
\podpis{Pavel Calábek}

{%%%%%   C-II-4
Nech $x$, $y$, $z$ sú kladné reálne čísla. Ukážte, že aspoň jedno z~čísel $x+y+z-xyz$ a ${xy+yz+zx-3}$ je nezáporné.}
\podpis{Stanislava Sojáková}

{%%%%%   vyberko, den 1, priklad 1
Nájdite všetky konečné množiny $S$ bodov v~rovine s~nasledujúcou vlastnosťou:
pre každé tri body $A$, $B$, $C$ z~množiny~$S$ existuje bod~$D$ z~množiny~$S$ taký, že body $A$, $B$, $C$, $D$ sú vrcholmi rovnobežníka.}
\podpis{Ján Mazák, Róbert Tóth:USA TST 2005}

{%%%%%   vyberko, den 1, priklad 2
a) Dokážte, že množinu celých čísel vieme rozložiť na dve disjunktné podmnožiny $A$ a~$B$ tak, že každý prvok z~$A$ sa dá vyjadriť ako súčet dvoch rôznych prvkov z~$B$ a~každý prvok z~$B$ sa dá vyjadriť ako súčet dvoch rôznych prvkov z~$A$.
b) Rozhodnite, či je možné rozložiť množinu celých čísel na 2011 po dvoch disjunktných podmnožín $A_1, A_2,\dots, A_{2011}$ s~nasledujúcou vlastnosťou:
ak $i, j\in\{1,2,\dots,2011\}$ a~$i\ne j$, tak každý prvok množiny $A_i$ vieme vyjadriť ako súčet dvoch rôznych prvkov z~množiny~$A_j$.
}
\podpis{Ján Mazák, Róbert Tóth:Ján Mazák}

{%%%%%   vyberko, den 1, priklad 3
Daný je ostrouhlý rovnoramenný trojuholník $ABC$ so základňou~$BC$.
Pre bod~$P$ ležiaci vnútri trojuholníka $ABC$ označíme postupne $M$ a~$N$
priesečníky kružnice so stredom~$A$ a~polomerom $|AP|$ so stranami $AB$ a~$AC$.
Nájdite bod~$P$, pre ktorý je súčet $|MN|+|BP|+|CP|$ minimálny.}
\podpis{Ján Mazák, Róbert Tóth:junior balkan MO selection test, Rumunsko}

{%%%%%   vyberko, den 1, priklad 4
Daný je tetivový štvoruholník $ABCD$.
Polpriamky $CB$ a~$DA$ sa pretínajú v~bode~$P$, polpriamky $AB$ a~$DC$ sa pretínajú v~bode~$Q$.
Stredy uhlopriečok $AC$ a~$BD$ označíme $L$ a~$M$ (v~tomto poradí).
Nakoniec $K$ nech je ortocentrum trojuholníka $MPQ$.
Dokážte, že body $P$, $Q$, $K$, $L$ ležia na kružnici.}
\podpis{Ján Mazák, Róbert Tóth:}

{%%%%%   vyberko, den 2, priklad 1
Nájdite najmenšie prirodzené číslo~$n$, pre ktoré existuje $n$-tica rôznych prirodzených čísel $\{s_1,s_2,\dots,s_n\}$ taká, že
$$
\left(1-\frac{1}{s_1}\right)
\left(1-\frac{1}{s_2}\right)
\cdots
\left(1-\frac{1}{s_n}\right)
=\frac{42}{2010}.
$$
%{\it Zadanie bude zverejnené po IMO 2011.}
}
\podpis{Jakub Konečný, Peter Novotný, Filip Sládek:Shortlist 2010, N1'}

{%%%%%   vyberko, den 2, priklad 2
Označme $\Bbb Q^+$ množinu všetkých kladných racionálnych čísel.
Určte všetky funkcie $f\colon\Bbb Q^{+}\to \Bbb Q^{+}$ také, že pre ľubovoľné
$x,y\in \Bbb Q^{+}$ platí
$$
f\left((f(x))^2\cdot y\right)=x^3\cdot f(xy).
$$
%{\it Zadanie bude zverejnené po IMO 2011.}
}
\podpis{Jakub Konečný, Peter Novotný, Filip Sládek:Shortlist 2010, A5}

{%%%%%   vyberko, den 2, priklad 3
Daný je konvexný päťuholník $ABCDE$, pričom $BC\parallel AE$, $|AB|=|BC|+|AE|$ a~$|\uhol ABC|=|\uhol CDE|$. Nech $M$ je stred úsečky~$CE$ a~$O$ je stred kružnice opísanej trojuholníku $BCD$. Dokážte, že ak $|\uhol DMO|=90^{\circ}$, tak $2|\uhol BDA|=|\uhol CDE|$.
%{\it Zadanie bude zverejnené po IMO 2011.}
}
\podpis{Jakub Konečný, Peter Novotný, Filip Sládek:Shortlist 2010, G5}

{%%%%%   vyberko, den 3, priklad 1
Nájdite všetky konečné rastúce aritmetické postupnosti prvočísel, v~ktorých je počet členov väčší ako diferencia.}
\podpis{Tomáš Jurík, Martin Kollár:Moskovská matematická olympiáda, ročník 2007}

{%%%%%   vyberko, den 3, priklad 2
Dané je prirodzené číslo $n\ge 2$. Štvorec je rozdelený na $n\times n$ štvorčekov. Dva protiľahlé rohové štvorčeky sú zafarbené na čierno. Operáciou nazveme prefarbenie všetkých štvorčekov jedného riadku alebo jedného stĺpca na "opačnú" farbu. Koľko najmenej bielych štvorčekov musíme najprv zafarbiť na čierno, aby bolo potom možné týmito operáciami začierniť celý štvorec $n\times n$?}
\podpis{Tomáš Jurík, Martin Kollár:Moskovská matematická olympiáda, ročník 2007}

{%%%%%   vyberko, den 3, priklad 3
Nech $H$ a~$O$ sú postupne ortocentrum a~stred opísanej kružnice~$k$ trojuholníka $ABC$. Priamky $AH$ a~$AO$ pretínajú kružnicu~$k$ postupne v~bodoch $M$ a~$N$ (rôznych od~$A$). Označme $P$, $Q$, $R$ postupne priesečníky priamok $BC$ a~$HN$, $BC$ a~$OM$, $HQ$ a~$OP$. Dokážte, že $AORH$ je rovnobežník.}
\podpis{Tomáš Jurík, Martin Kollár:}

{%%%%%   vyberko, den 3, priklad 4
Dokážte, že pre kladné reálne čísla $a$, $b$, $c$ spĺňajúce $\frac1a+\frac1b+\frac1c=1$ platí nerovnosť
$$
\sqrt{a+bc}+\sqrt{b+ac}+\sqrt{c+ab} \ge \sqrt{abc}+\sqrt{\vphantom{b}a}+\sqrt{b}+\sqrt{\vphantom{b}c}.
$$
}
\podpis{Tomáš Jurík, Martin Kollár:}

{%%%%%   vyberko, den 4, priklad 1
Nech $ABC$ je ostrouhlý trojuholník a~nech $D$, $E$ a~$F$ sú postupne päty výšok na strany $BC$, $AC$ a~$AB$. Nech $P$ je priesečník kružnice opísanej trojuholníku $ABC$ a~priamky~$EF$. Priamky $BP$ a~$DF$ sa pretínajú v~bode~$Q$. Dokážte, že $|AP| = |AQ|$.
}
\podpis{Miroslav Baláž, Peter Csiba, Tomáš Kocák:Shortlist 2010, G1}

{%%%%%   vyberko, den 4, priklad 2
V~senáte je 51~senátorov. Senátorov potrebujeme rozdeliť do $n$~výborov. Každý senátor neznáša práve troch iných senátorov. Ak senátor~$A$ neznáša senátora~$B$, neznamená to nevyhnutne, že aj senátor~$B$ neznáša senátora~$A$. Nájdite najmenšie~$n$, pre ktoré je vždy možné rozdeliť senátorov do výborov tak, že všetci senátori v~jednom výbore sa navzájom znášajú.}
\podpis{Miroslav Baláž, Peter Csiba, Tomáš Kocák:http://www.artofproblemsolving.com/Forum/viewtopic.php?f=44&t=6165}

{%%%%%   vyberko, den 4, priklad 3
Nájdite všetky reálne funkcie~$f$ také, že pre všetky reálne čísla $x$, $y$ platí rovnosť
$$
f(x^2+xy+f(y)) = f^2(x)+xf(y)+y.
$$
}
\podpis{Miroslav Baláž, Peter Csiba, Tomáš Kocák:http://www.artofproblemsolving.com/Forum/viewtopic.php?f=37&t=365859}

{%%%%%   vyberko, den 4, priklad 4
Nech $m$ a~$n$ sú prirodzené čísla a~nech $d$ je ich najväčší spoločný deliteľ. Ďalej nech $x = 2^m-1$ a $y = 2^n + 1$.
\itemitem{a)} Ak $m/d$ je nepárne, dokážte, že najväčší spoločný deliteľ $x$ a $y$ je $1$.
\itemitem{b)} Ak $m/d$ je párne, nájdite najväčší spoločný deliteľ $x$ a $y$.
}
\podpis{Miroslav Baláž, Peter Csiba, Tomáš Kocák:http://www.artofproblemsolving.com/Forum/viewtopic.php?f=59&t=52928}

{%%%%%   vyberko, den 5, priklad 1
Daný je konvexný päťuholník $ABCDE$, v~ktorom $|DC| = |DE|$ a~$|\uhol DCB| = |\uhol DEA| = 90^{\circ}$.
Nech $F$ je bod vo vnútri strany~$AB$, pre ktorý platí $|AF|:|BF| = |AE|:|BC|$. Dokážte, že
$|\uhol FCE| = |\uhol ADE|$ a~$|\uhol FEC| = |\uhol BDC|$.}
\podpis{Michal Hagara, Richard Kollár:Polsko 1997 (48.roc), III.kolo, uloha c.5}

{%%%%%   vyberko, den 5, priklad 2
Určte najmenšie prirodzené číslo~$n$ také, že existujú celé čísla $x_1, x_2, \dots, x_n$, pre ktoré platí
$$
x_1^3 + x_2^3 + \dots + x_n^3 = 2002^{2011}.
$$}
\podpis{Michal Hagara, Richard Kollár:IMO shortlist 2003, 104 NT problems p.31}

{%%%%%   vyberko, den 5, priklad 3
V~škatuli sú jednofarebné gule $n$ rôznych veľkostí a~$n$ rôznych farieb. Viete, že ak v~škatuli nie je guľa farby~$F$ a~veľkosti~$V$, potom celkový počet gúľ v~škatuli, ktoré majú veľkosť~$V$ alebo farbu~$F$, je aspoň~$n$. Dokážte, že v~škatuli je aspoň $n^2/2$ gúľ. Môže byť ich počet presne $n^2/2$?}
\podpis{Michal Hagara, Richard Kollár:Zarubeznije mat olymp 24.11, p.72, zdroj Rumunsko 1978}

{%%%%%   vyberko, den 2, priklad 4
...}
\podpis{...}

{%%%%%   vyberko, den 5, priklad 4
...}
\podpis{...}

{%%%%%   trojstretnutie, priklad 1
Nech $a$, $b$, $c$ sú kladné reálne čísla také, že $a^2<bc$. Dokážte, že $b^3+ac^2>ab(a+c)$.}
\podpis{Pavel Novotný}

{%%%%%   trojstretnutie, priklad 2
Na tabuli je napísaných $n$ nezáporných celých čísel, ktorých najväčší spoločný deliteľ je $1$. V~jednom kroku môžeme zotrieť dve také čísla $x$, $y$, že $x\ge y$ a~nahradiť ich dvojicou čísel $x-y$, $2y$. Určte, pre ktoré $n$-tice pôvodných čísel môžeme dosiahnuť stav, keď medzi číslami na tabuli bude $n-1$ núl.
}
\podpis{Poľsko}

{%%%%%   trojstretnutie, priklad 3
Body $A$, $B$, $C$, $D$ ležia v~tomto poradí na kružnici, pričom $AB\nparallel CD$ a~dĺžka oblúka~$AB$, ktorý obsahuje body $C$, $D$, je dvakrát väčšia ako dĺžka oblúka~$CD$, ktorý neobsahuje body $A$, $B$. Nech~$E$ je taký bod v~polrovine $ABC$, že $|AC|=|AE|$ a~$|BD|=|BE|$. Dokážte, že ak kolmica z~bodu~$E$ na priamku~$AB$ prechádza stredom oblúka~$CD$ neobsahujúceho body $A$, $B$, tak $|\angle ACB|=108^{\circ}$.}
\podpis{Tomáš Jurík}

{%%%%%   trojstretnutie, priklad 4
Mnohočlen $P(x)$ s~celočíselnými koeficientmi spĺňa nasledujúcu podmienku: Ak pre mnohočleny $F(x)$, $G(x)$, $Q(x)$ s~celočíselnými koeficientmi platí
$$P(Q(x)) = F(x) \cdot G(x),$$
tak aspoň jeden z~mnohočlenov $F(x)$, $G(x)$ je konštantný.
Dokážte, že $P(x)$ je konštantný mnohočlen.
}
\podpis{Poľsko}

{%%%%%   trojstretnutie, priklad 5
V~konvexnom štvoruholníku $ABCD$ označme $M$, $N$ postupne stredy strán $AD$, $BC$. Na stranách $AB$ a~$CD$ sú postupne zvolené také body $K$ a~$L$, že $|\angle MKA|=|\angle NLC|$. Dokážte, že ak priamky $BD$, $KM$, $LN$ prechádzajú jedným bodom, tak
$$
|\angle KMN|=|\angle BDC| \qquad \text{a}\qquad |\angle LNM|=|\angle ABD|.
$$
}
\podpis{Poľsko}

{%%%%%   trojstretnutie, priklad 6
Nech $a$ je ľubovoľné celé číslo. Dokážte, že existuje nekonečne veľa prvočísel~$p$ takých, že
$$
p\mid n^2+3 \qquad \text{a} \qquad p\mid m^3-a
$$
pre nejaké celé čísla $n$, $m$.
}
\podpis{Poľsko}

{%%%%%   IMO, priklad 1
Pre ľubovoľnú množinu $A = \{a_1, a_2, a_3, a_4\}$ obsahujúcu štyri rôzne kladné celé čísla položme $s_A=a_1+a_2+a_3+a_4$.
Označme $n_A$ počet takých dvojíc $(i, j)$ spĺňajúcich $1\le i<j\le 4$, pre ktoré je číslo $a_i + a_j$ deliteľom čísla~$s_A$.
Určte všetky množiny~$A$ obsahujúce štyri rôzne kladné celé čísla, pre ktoré je hodnota~$n_A$ najväčšia možná.}
\podpis{Mexiko}

{%%%%%   IMO, priklad 2
Daná je konečná množina~$\Cal S$ aspoň dvoch bodov v~rovine, pričom žiadne tri body množiny~$\Cal S$ neležia na jednej priamke. Pojmom 
{\it veterný mlyn} rozumieme proces, ktorý začína ľubovoľnou priamkou~$l$
prechádzajúcou práve jedným bodom~$P$ množiny~$\Cal S$. Táto priamka sa otáča v~smere hodinových ručičiek okolo {\it pivota}~$P$, až kým po prvýkrát neprechádza ďalším bodom množiny~$\Cal S$. Tento bod, označme ho $Q$, sa stáva novým pivotom, \tj. priamka sa ďalej otáča v~smere hodinových ručičiek okolo bodu~$Q$, až kým neprechádza ďalším bodom množiny~$\Cal S$. Uvedený proces pokračuje donekonečna.
Dokážte, že sa dá vybrať bod $P\in\Cal S$ a~priamka~$l$ prechádzajúca bodom~$P$ tak, že pre príslušný veterný mlyn je každý bod množiny $\Cal S$ pivotom nekonečne veľa krát.}
\podpis{Veľká Británia}

{%%%%%   IMO, priklad 3
Nech $f\colon\Bbb R\to\Bbb R$ je funkcia z~množiny reálnych čísel do množiny reálnych čísel spĺňajúca
$$
 f(x+y)\le yf(x)+f(f(x))
$$
pre všetky reálne čísla $x$ a~$y$.
Dokážte, že $f(x)=0$ pre všetky $x\le 0$.}
\podpis{Bielorusko}

{%%%%%   IMO, priklad 4
Nech $n>0$ je celé číslo. K~dispozícii máme rovnoramenné váhy a~$n$ závaží s~hmotnosťami $2^0,2^1, \dots, 2^{n-1}$.
Jednotlivé závažia máme v~nejakom poradí ukladať na misky váh tak, aby obsah pravej misky nebol v~žiadnom okamihu ťažší ako obsah ľavej misky. V~každom kroku vyberieme jedno zo závaží, ktoré ešte nie je na váhach, a~položíme ho buď na ľavú alebo na pravú misku váh. Tak postupujeme, kým neminieme všetky závažia. Určte, koľkými spôsobmi to celé môžeme urobiť.}
\podpis{Irán}

{%%%%%   IMO, priklad 5
Nech $f$ je funkcia z~množiny celých čísel do množiny kladných celých čísel.
Predpokladajme, že pre ľubovoľné dve celé čísla $m$ a~$n$ je rozdiel $f(m)-f(n)$ deliteľný číslom $f(m-n)$.
Dokážte, že pre každé dve celé čísla $m$ a~$n$ také, že $f(m)\le f(n)$, je číslo $f(n)$ deliteľné číslom $f(m)$.}
\podpis{Irán}

{%%%%%   IMO, priklad 6
Daný je ostrouhlý trojuholník $ABC$ a~kružnica~$\Gamma$ jemu opísaná. Nech $l$ je dotyčnica kružnice~$\Gamma$ a~nech
$l_a$, $l_b$, $l_c$ sú obrazy priamky~$l$ v~osových súmernostiach podľa priamok $BC$, $CA$, $AB$.
Dokážte, že kružnica opísaná trojuholníku určenému priamkami $l_a$, $l_b$, $l_c$ sa dotýka kružnice~$\Gamma$.}
\podpis{Japonsko}

{%%%%%   MEMO, priklad 1
Na začiatku je na tabuli napísané číslo $44$. Celé číslo $a$ na tabuli môžeme nahradiť štyrmi navzájom rôznymi celými číslami $a_1$, $a_2$, $a_3$, $a_4$ takými, že ich aritmetický priemer $\frac 14(a_1+a_2+a_3+a_4)$ je rovný číslu~$a$. V~každom kroku naraz nahradíme všetky čísla na tabuli vyššie opísaným spôsobom. Po 30~krokoch bude na tabuli $n=4^{30}$ celých čísel $b_1,b_2,\dots,b_n$. Dokážte, že
$$
\frac{b_1^2+b_2^2+\ldots+b_n^2}{n} \ge 2011.
$$
}
\podpis{Chorvátsko}

{%%%%%   MEMO, priklad 2
Dané je celé číslo $n\ge 3$. Janko a Marienka hrajú nasledujúcu hru: Najskôr Janko označí strany pravidelného $n$-uholníka číslami $1,2,\dots,n$ v~ľubovoľnom poradí, pričom každé číslo použije práve raz. Potom Marienka rozdelí uvedený $n$-uholník na trojuholníky pomocou $n-3$ uhlopriečok, ktoré sa vnútri $n$-uholníka nepretínajú. Všetky tieto uhlopriečky označíme číslom~$1$. Dovnútra každého trojuholníka napíšeme súčin čísel na jeho stranách. Nech $S$ je súčet týchto $n-2$ súčinov. Určte, aká bude hodnota~$S$, ak Marienka chce, aby bolo $S$ čo najmenšie, Janko chce, aby bolo $S$ čo najväčšie a obaja hrajú najlepšie ako sa dá.
}
\podpis{Chorvátsko}

{%%%%%   MEMO, priklad 3
V~rovine sa kružnice $k_1$, $k_2$ so stredmi $I_1$, $I_2$ pretínajú v~dvoch bodoch $A$ a~$B$. Predpokladajme, že uhol $I_1AI_2$ je tupý. Dotyčnica ku $k_1$ vedená bodom~$A$ pretína kružnicu~$k_2$ znova v~bode~$C$ a~dotyčnica ku $k_2$ vedená bodom~$A$ pretína kružnicu~$k_1$ znova v~bode~$D$.
Nech $k_3$ je kružnica opísaná trojuholníku $BCD$. Označme $E$ stred oblúka~$CD$ kružnice~$k_3$ obsahujúceho bod~$B$. Priamky $AC$ a~$AD$ pretínajú kružnicu~$k_3$ znova postupne v~bodoch $K$ a~$L$. Dokážte, že priamky $AE$ a~$KL$ sú na seba kolmé.}
\podpis{Slovinsko}

{%%%%%   MEMO, priklad 4
Nech $k$ a~$m$ sú kladné celé čísla, pričom $k>m$ a~číslo
$km(k^2-m^2)$ je deliteľné číslom $k^3-m^3$. Dokážte, že $(k-m)^3>3km$.}
\podpis{Poľsko}

{%%%%%   MEMO, priklad t1
Nájdite všetky funkcie $f\colon\Bbb R\to\Bbb R$ také, že rovnosť
$$
y^2 f(x) + x^2 f(y) + xy = xy f(x+y) + x^2 + y^2
$$
platí pre všetky dvojice $x,y\in\Bbb R$, pričom $\Bbb R$ je množina všetkých reálnych čísel.}
\podpis{Chorvátsko}

{%%%%%   MEMO, priklad t2
Nech $a$, $b$, $c$ sú kladné reálne čísla spĺňajúce rovnosť
$$
\frac{a}{1+a}+\frac{b}{1+b}+\frac{c}{1+c}=2.
$$
Dokážte, že
$$
\frac{\sqrt{a}+\sqrt{b}+\sqrt{c}}{2}\ge \frac{1}{\sqrt{a}}+\frac{1}{\sqrt{b}}+\frac{1}{\sqrt{c}}.
$$
}
\podpis{Chorvátsko}

{%%%%%   MEMO, priklad t3
Pre celé číslo $n\ge3$ označme $\Cal{M}$ množinu $\{(x,y);\ x,y\in\Bbb Z, 1\le x\le n, 1\le y\le n\}$ pozostávajúcu z~bodov roviny. (Symbol $\Bbb Z$ označuje množinu celých čísel.) Aký je najväčší možný počet prvkov podmnožiny $S\subseteq \Cal{M}$, ktorá neobsahuje žiadne tri body ležiace vo vrcholoch pravouhlého trojuholníka?}
\podpis{Maďarsko}

{%%%%%   MEMO, priklad t4
Nech $n\ge 3$ je prirodzené číslo.
Súťaže podobnej MEMO sa zúčastnilo $3n$ účastníkov, ktorí dokopy hovoria $n$ rôznymi jazykmi. Každý účastník ovláda práve tri z týchto jazykov.
Dokážte, že vieme vybrať aspoň $\lceil\frac29n\rceil$ zo spomínaných $n$ jazykov tak, aby žiadny účastník neovládal viac ako dva z~nich.
(Symbol $\lceil x\rceil$ označuje najmenšie celé číslo, ktoré nie je menšie ako $x$.)
}
\podpis{Chorvátsko}

{%%%%%   MEMO, priklad t5
Konvexný päťuholník $ABCDE$ má všetky strany rovnako dlhé. Uhlopriečky $AD$ a~$EC$ sa pretínajú v~bode~$S$ tak, že $|\uhol ASE| = 60^\circ$.
Dokážte, že päťuholník $ABCDE$ má niektoré dve strany rovnobežné.}
\podpis{Michal Szabados}

{%%%%%   MEMO, priklad t6
Daný je ostrouhlý trojuholník $ABC$. Označme postupne $B_0$ a~$C_0$ päty výšok z~vrcholov $B$ a~$C$.
Bod~$X$ leží vnútri trojuholníka $ABC$ tak, že priamka~$BX$ sa dotýka kružnice opísanej trojuholníku $AXC_0$
a~priamka~$CX$ sa dotýka kružnice opísanej trojuholníku $AXB_0$. Dokážte, že priamky $AX$ a~$BC$ sú na seba kolmé.}
\podpis{Česká rep.}

{%%%%%   MEMO, priklad t7
Nech $A$ a~$B$ sú disjunktné neprázdne množiny také, že $A\cup B = \{1,2,3,\dots,10\}$.
Dokážte, že existujú prvky $a\in A$ a~$b\in B$ také, že číslo $a^3+ab^2+b^3$ je deliteľné $11$.}
\podpis{Poľsko}

{%%%%%   MEMO, priklad t8
Kladné celé číslo $n$ nazveme {\it úžasným}, ak existujú kladné celé čísla $a$, $b$, $c$ spĺňajúce rovnosť
$$
n=\nsd(b,c)\cdot\nsd(a,bc)+\nsd(c,a)\cdot\nsd(b,ca)+\nsd(a,b)\cdot\nsd(c,ab).
$$
Dokážte, že existuje 2011 po sebe idúcich kladných celých čísel, ktoré sú úžasné.
%(Symbolom $(m,n)$ označujeme najväčšieho spoločného deliteľa kladných celých čísel $m$ a~$n$.)
}
\podpis{Litva}
