{%%%%%   A-I-1
V~každej zo štyroch miestností je niekoľko predmetov. Nech $n\ge2$ je prirodzené
číslo. Jednu $n$-tinu predmetov z~prvej miestnosti prenesieme do druhej miestnosti.
Následne jednu $n$-tinu (z~nového počtu) predmetov prenesieme z~druhej miestnosti do
tretej. Podobne potom z~tretej miestnosti do štvrtej a~zo štvrtej do prvej. (Vždy pritom
prenášame celé predmety.) Ak viete, že na konci bol v~každej miestnosti rovnaký počet predmetov, určte,
koľko najmenej predmetov mohlo byť na začiatku v~druhej miestnosti. Pre ktoré~$n$ sa tak môže stať?
}
\podpis{Vojtech Bálint, Michal Rolínek}

{%%%%%   A-I-2
Nájdite najmenšie reálne číslo~$m$, pre ktoré možno nájsť reálne čísla $a$, $b$ tak,
aby nerovnosť
$$
|x^2+ax+b| \le m
$$
platila pre každé $x \in \langle 0, 2 \rangle$.
}
\podpis{Leo Boček}

{%%%%%   A-I-3
Daný je pravouhlý trojuholník $ABC$ s~preponou~$AB$ a~dlhšou odvesnou~$BC$. Nech
$D$ je päta výšky z~vrcholu~$C$.
Kružnica~$k$ so stredom~$D$ a~polomerom~$CD$ pretína odvesnu~$BC$ v~bode~$Q$ a~ďalej
priamku~$AB$ v~bodoch $E$ a~$F$ ($E \ne F$),
pričom $F$ je bodom prepony~$AB$. Úsečka~$QE$ pretína odvesnu~$AC$ v~bode~$P$. Dokážte,
že $|PE|=|QF|$.}
\podpis{Jaroslav Švrček}

{%%%%%   A-I-4
Nela s~Janou zvolia prirodzené číslo~$k$ a~následne hrajú hru s~tabuľkou
majúcou rozmery $9\times9$. Začínajúca Nela vždy vo svojom ťahu vyberie jedno prázdne
políčko a~vpíše doňho nulu. Jana vo svojom ťahu do nejakého prázdneho políčka
napíše jednotku. Navyše po každom ťahu Nely nasleduje $k$~ťahov Jany. Ak sa kedykoľvek
počas hry stane, že súčet čísel v~každom riadku aj v~každom stĺpci je nepárny, vyhrá
Jana. Ak dievčatá vyplnia celú tabuľku bez toho, aby sa tak stalo, vyhrá Nela. Nájdite
najmenšiu hodnotu~$k$, pre ktorú má Jana vyhrávajúcu stratégiu.
}
\podpis{Michal Rolínek}

{%%%%%   A-I-5
Daný je trojuholník $ABC$ s~najkratšou stranou~$BC$. Na stranách $AB$,
$AC$ a~na polpriamkach opačných k~polpriamkam $BC$, $CB$ zvoľme postupne body $X$,
$Y$, $K$, $L$ tak, aby platilo $|BX| = |BK| = |BC| = |CY| = |CL|$. Priamky $KX$ a~$LY$ sa
pretínajú v~bode~$M$. Dokážte, že ťažisko trojuholníka $KLM$ je totožné so stredom
kružnice vpísanej do trojuholníka $ABC$.
}
\podpis{Tomáš Jurík}

{%%%%%   A-I-6
Na tabuli je napísaný súčin
$$
1 \cdot 2 \cdot 3 \cdot \dots \cdot n.
$$
Pre ktoré prirodzené čísla $n \ge 2$ je možné za niektoré z~činiteľov dopísať výkričník
a~nahradiť ich tak ich faktoriálmi, aby výsledný súčin bol rovný druhej
mocnine prirodzeného čísla?}
\podpis{Michal Rolínek}

{%%%%%   B-I-1
Pre prirodzené čísla $k$, $l$, $m$ platí
$$
\frac {k+m+klm} {lm+1} = \frac {2\,051} {404}.
$$
Určte všetky možné hodnoty súčinu $klm$.
}
\podpis{Aleš Kobza}

{%%%%%   B-I-2
Do štvorcovej tabuľky $11 \times 11$ sme vpísali prirodzené čísla
$1, 2, \dots, 121$ postupne po riadkoch zľava doprava a~zhora nadol. Štvorcovou
doštičkou $4 \times 4$ sme všetkými možnými spôsobmi zakryli práve 16~políčok.
Koľkokrát bol súčet zakrytých 16~čísel druhou mocninou celého čísla?
}
\podpis{Vojtech Bálint, Tomáš Jurík}

{%%%%%   B-I-3
V~pravouhlom trojuholníku $ABC$ s~preponou~$AB$ a~odvesnami dĺžok
$|AC| = 4\cm$ a~$|BC| = 3\cm$
ležia navzájom sa dotýkajúce kružnice $k_1 (S_1; r_1)$
a~$k_2 (S_2; r_2)$ tak, že $k_1$ sa dotýka
strán $AB$ a~$AC$, zatiaľ čo $k_2$ sa dotýka strán $AB$ a~$BC$.
Určte najmenšiu a~najväčšiu možnú hodnotu polomeru~$r_2$.
}
\podpis{Pavel Novotný}

{%%%%%   B-I-4
Počet všetkých párnych deliteľov niektorého prirodzeného čísla je o~3
väčší ako počet všetkých jeho nepárnych deliteľov. Aký je podiel súčtu
všetkých jeho párnych deliteľov a~súčtu všetkých jeho nepárnych deliteľov?
Nájdite všetky možné odpovede.
}
\podpis{Erika Novotná}

{%%%%%   B-I-5
Vrcholy konvexného šesťuholníka $ABCDEF$ ležia na kružnici, pričom
$|AB| = |CD|$. Úsečky $AE$ a~$CF$ sa pretínajú v~bode~$G$ a~úsečky $BE$ a~$DF$
sa pretínajú v~bode~$H$.
Dokážte, že úsečky $GH$, $AD$ a~$BC$ sú navzájom rovnobežné.
}
\podpis{Šárka Gergelitsová}

{%%%%%   B-I-6
Kladné reálne čísla $a$, $b$, $c$ sú také, že hodnoty
$$
x_1 = a, \qquad x_2 = b, \qquad x_3 = c, \
x_4 = \frac {2a^2} {b+c}, \qquad x_5 = \frac {2b^2} {c+a}, \qquad x_6 = \frac {2c^2} {a+b}
$$
sú navzájom rôzne. Zapíšme ich od najmenšej po najväčšiu:
$$
x_{i_1} <x_{i_2} <x_{i_3} <x_{i_4} <x_{i_5} <x_{i_6}.
$$
Zistite, koľko rôznych poradí $(i_1, i_2, \dots, i_6)$ indexov 1 až 6
môžeme dostať, keď budeme rôzne voliť čísla $a$, $b$, $c$.
}
\podpis{Jaromír Šimša}

{%%%%%   C-I-1
Nájdite všetky možné hodnoty súčinu prvočísel $p$, $q$, $r$,
pre ktoré platí
$$
p^2-(q+r)^2 = 637.
$$
}
\podpis{Vojtech Bálint, Jaromír Šimša}

{%%%%%   C-I-2
Určte, koľkými spôsobmi možno k~jednotlivým vrcholom kocky $ABCDEFGH$ pripísať
čísla 1, 3, 3, 3, 4, 4, 4, 4 tak, aby
súčin čísel pripísaných ľubovoľným trom vrcholom každej zo stien kocky
bol párny.
}
\podpis{Jaroslav Švrček}

{%%%%%   C-I-3
Uvažujme výraz
$$
2x^2+y^2-2xy+2x+4.
$$
\ite a) Nájdite všetky reálne čísla $x$ a~$y$, pre ktoré daný výraz nadobúda svoju
najmenšiu hodnotu.
\ite b) Určte všetky dvojice celých nezáporných čísel $x$ a~$y$, pre
ktoré je hodnota daného výrazu rovná číslu~$16$.\endgraf
}
\podpis{Aleš Kobza}

{%%%%%   C-I-4
Vnútri strán $AB$, $AC$ daného trojuholníka $ABC$ sú zvolené postupne
body $E$, $F$, pričom $EF \parallel BC$. Úsečka~$EF$
je potom rozdelená bodom~$D$ tak, že platí
$$
p = |ED|:|DF| = |BE|:|EA|.
$$
\ite a) Ukážte, že pomer obsahov trojuholníkov $ABC$ a~$ABD$ je pre $p=2:3$ rovnaký
ako pre $p=3:2$.
\ite b) Zdôvodnite, prečo pomer obsahov trojuholníkov $ABC$ a~$ABD$ má hodnotu
aspoň~$4$.\endgraf
}
\podpis{Vojtěch Žádník}

{%%%%%   C-I-5
Máme kartičky s~číslami $5,6,7, \dots, 55$ (na každej kartičke je
jedno číslo). Koľko najviac kartičiek môžeme vybrať tak, aby
súčet čísel na žiadnych dvoch vybraných kartičkách nebol
palindróm? (Palindróm je číslo, ktoré je rovnaké pri čítaní zľava
doprava i~sprava doľava.)
}
\podpis{Tomáš Jurík}

{%%%%%   C-I-6
Daná je kružnica $k_1 (A; 4\cm)$, jej bod~$B$
a~kružnica $k_2 (B; 2\cm)$. Bod~$C$ je stredom úsečky~$AB$
a~bod~$K$ je stredom úsečky~$AC$. Vypočítajte obsah pravouhlého
trojuholníka~$KLM$, ktorého vrchol~$L$ je jeden z~priesečníkov
kružníc $k_1$, $k_2$ a~ktorého prepona~$KM$ leží na priamke~$AB$.
}
\podpis{Šárka Gergelitsová}

{%%%%% A-S-1
Prvočíslo nazveme {\it pekné}, ak sa dá zapísať ako
rozdiel dvoch tretích mocnín prirodzených čísel. Určte
posledné cifry všetkých pekných prvočísel.}
\podpis{Patrik Bak, Michal Rolínek}

{%%%%% A-S-2
Kladné reálne čísla $a$, $b$, $c$, $d$ spĺňajú rovnosti
$$
a=c+\frac1d \qquad\text a\qquad b=d+\frac1c .
$$
Dokážte nerovnosť $ab\ge4$ a~nájdite najmenšiu možnú hodnotu výrazu
$ab + cd$.}
\podpis{Jaromír Šimša}

{%%%%% A-S-3
Daný je lichobežník $ABCD$ ($AB\parallel CD$), v~ktorom
platí $|BC|=|AB|+|CD|$. Dokážte, že
\ite a) na ramene~$AD$ leží nejaký bod kružnice majúcej priemer~$BC$,
\ite b) na ramene~$BC$ leží nejaký bod kružnice majúcej priemer~$AD$.
}
\podpis{Josef Tkadlec}

{%%%%% A-II-1
Na tabuli sú napísané rôzne kladné celé čísla.
Ich aritmetický priemer je desatinné číslo, ktorého desatinná
časť je presne $0{,}2016$. Akú najmenšiu hodnotu môže tento
priemer mať?}
\podpis{Patrik Bak}

{%%%%% A-II-2
Daný je štvorec $ABCD$ so~stranou dĺžky~1. Na jeho strane~$CD$ zvolíme bod~$E$ tak, aby platilo $|\uhol BAE| = 60^\circ$. Ďalej
zvoľme ľubovoľný vnútorný bod úsečky~$AE$ a~označme ho~$X$. Bodom~$X$ potom veďme kolmicu na priamku~$BX$ a~jej priesečník s~priamkou~$BC$ označme~$Y$. Aká je najmenšia možná dĺžka úsečky~$BY$?}
\podpis{Michal Rolínek}

{%%%%% A-II-3
Koľkými spôsobmi sa dá rozdeliť množina $\{1,2,\ldots,12\}$ na
šesť disjunktných dvojprvkových podmnožín takých, že každá
z~nich obsahuje navzájom nesúdeliteľné čísla (teda také, ktoré nemajú
spoločného deliteľa väčšieho ako~1)?}
\podpis{Martin Panák}

{%%%%% A-II-4
Určte najmenšie reálne číslo~$m$, pre ktoré možno nájsť reálne čísla
$a$ a~$b$ tak, aby nerovnosť
$$
|x^2+ax+b|\le m(x^2+1)
$$
platila pre každé $x\in\langle-1,1\rangle$.}
\podpis{Jaromír Šimša}

{%%%%% A-III-1
Nech $p>3$ je dané prvočíslo. Určte počet všetkých
usporiadaných šestíc $(a,b,c,d,e,f)$ kladných celých čísel, ktorých
súčet je rovný $3p$, a~pritom všetky zlomky
$$
\frac{a+b}{c+d}, \quad \frac{b+c}{d+e}, \quad \frac{c+d}{e+f},
\quad \frac{d+e}{f+a}, \quad \frac{e+f}{a+b}
$$
majú celočíselné hodnoty.
}
\podpis{Jaromír Šimša, Jaroslav Švrček}

{%%%%% A-III-2
Označme postupne $r$ a~$r_a$ polomery kružnice vpísanej
a~kružnice pripísanej k~strane~$BC$ trojuholníka $ABC$. Dokážte,
že ak platí
$$
r+r_a= |BC|,
$$
tak trojuholník $ABC$ je pravouhlý.}
\podpis{Michal Rolínek}

{%%%%% A-III-3
Medzi obyvateľmi istého mesta sú populárne matematické
kluby. Každé dva z~nich majú aspoň jedného spoločného
člena. Dokážte, že môžeme obyvateľom mesta rozdať kružidlá
a~pravítka tak, že iba jeden obyvateľ dostane oboje, a~pritom každý
klub bude mať pri plnej účasti svojich členov k~dispozícii ako
pravítko, tak kružidlo.}
\podpis{Josef Tkadlec}

{%%%%% A-III-4
Pre kladné čísla $a$, $b$, $c$ platí
$$
(a+c)(b^2+ac) = 4a.
$$
Určte maximálnu hodnotu výrazu $b+c$ a~nájdite všetky trojice
čísel $(a,b,c)$, pre ktoré výraz túto hodnotu nadobúda.}
\podpis{Michal Rolínek}

{%%%%% A-III-5
V~trojuholníku $ABC$ platí $|BC|=1$ a~zároveň na strane
$BC$ existuje práve jeden bod~$D$ taký, že $|DA|^2 = |DB| \cdot
|DC|$. Určte všetky možné hodnoty obvodu trojuholníka $ABC$.}
\podpis{Patrik Bak}

{%%%%% A-III-6
Na niektoré políčko šachovnice $6\times 6$ postavíme figúrku
kráľoviča. Tá môže v~jednom ťahu poskočiť buď v~zvislom, alebo
vo vodorovnom smere. Dĺžka tohto skoku je striedavo jedno a~dve políčka, pričom skokom dĺžky jedna (\tj. na susedné políčko) figúrka začína.
Rozhodnite, či sa dá zvoliť východisková pozícia figúrky tak, aby po
vhodnej postupnosti 35~skokov navštívila každé políčko šachovnice
práve raz.}
\podpis{Peter Novotný}

{%%%%% B-S-1
Koľkými spôsobmi je možné vyplniť štvorcovú tabuľku $3 \times 3$
číslami 2, 2, 3, 3, 3, 4, 4, 4, 4 tak, aby súčet čísel
v~každom štvorci $2 \times 2$ tejto tabuľky bol rovný~14?}
\podpis{Tomáš Jurík}

{%%%%% B-S-2
Daná je úsečka~$AB$, jej stred~$C$ a~vnútri úsečky~$AB$ bod~$D$. Kružnice
$k(C, |BC|)$ a~$m(B, |BD|)$ sa pretínajú v~bodoch $E$ a~$F$ a~polpriamka~$FD$ pretína kružnicu~$k$ v~bode~$K$, $K\ne F$. Rovnobežka
s~priamkou~$AB$ prechádzajúca bodom~$K$ pretína kružnicu~$k$ v~bode~$L$, $L \ne K$.
Dokážte, že $|KL| = |BD|$.}
\podpis{Šárka Gergelitsová}

{%%%%% B-S-3
Dané sú dve rôzne reálne čísla $a$, $b$ väčšie ako~$1$. Zapíšte
všetky možné poradia hodnôt výrazov
$$
1+a, \quad 1+b, \quad 1+ \frac {a+b} 2, \quad \frac {a^2+b^2-2} {a+b-2}
$$
od najmenšej po najväčšiu.}
\podpis{Jaromír Šimša}

{%%%%% B-II-1
Určte všetky trojice celých kladných čísel $k$, $l$ a~$m$, pre ktoré
platí
$$
\frac {3l+1} {3kl+k+3} = \frac {lm+1} {5lm+m+5}.
$$
}
\podpis{Jaromír Šimša}

{%%%%% B-II-2
Daná je úsečka $AB$, jej stred $C$ a~vnútri úsečky $AB$ bod~$D$. Kružnice
$k(C, |BC|)$ a~$m (B, |BD|)$ sa pretínajú v~bodoch $E$ a~$F$. Zdôvodnite,
prečo je polpriamka $FD$ osou uhla $AFE$.}
\podpis{Šárka Gergelitsová}

{%%%%% B-II-3
Nájdite všetky prirodzené čísla $n$, ktoré majú práve šesť deliteľov,
pričom súčet druhého najväčšieho a~druhého najmenšieho z~nich je~54.}
\podpis{Pavel Novotný}

{%%%%% B-II-4
Dané je prirodzené číslo $k$, $4\le k\le900$.
Adam a~Braňo hrajú hru: Adam napíše na tabuľu $k$ rôznych
trojciferných čísel, Braňo si z~nich vyberie štyri rôzne. Ak rozdiel
medzi dvoma najmenšími aj rozdiel medzi dvoma najväčšími vybranými
číslami je nanajvýš~22, vyhráva Braňo, inak vyhráva Adam. V~závislosti
od hodnoty~$k$ určte, kto má vyhrávajúcu stratégiu.}
\podpis{Tomáš Jurík}

{%%%%% C-S-1
Nájdite všetky štvorciferné čísla $\overline{abcd}$, pre ktoré platí
$\overline{abcd}=20\cdot\overline{ab}+16\cdot\overline{cd}$.}
\podpis{Tomáš Jurík}

{%%%%% C-S-2
Pri stole sedí niekoľko ľudí (aspoň dvaja) a~hrajú takúto hru:
V~každom kole tajným hlasovaním každý hráč udelí hlas jednému
hráčovi (môže aj sám sebe). Potom sa kolo vyhodnotí: každý hráč, ktorý
dostal práve jeden hlas, z~hry vypadáva.
\ite a) Koľko ľudí mohlo sedieť pri stole na začiatku,
ak v~prvom kole vypadol z~hry práve jeden hráč?
\ite b) Mohla mať hra jediného víťaza, teda človeka,
ktorý po určitom počte kôl zostal v~hre sám?\endgraf
}
\podpis{Róbert Tóth}

{%%%%% C-S-3
V~kružnici so stredom~$S$ zostrojíme priemer~$AB$ a~ľubovoľnú
naň kolmú tetivu~$CD$. Zdôvodnite, prečo je obvod trojuholníka~$ACD$
menší ako dvojnásobok obvodu trojuholníka~$SBC$.}
\podpis{Šárka Gergelitsová}

{%%%%% C-II-1
Nájdite najmenšiu možnú hodnotu výrazu
$$
3x^2-12xy+y^4,
$$
v ktorom $x$ a~$y$ sú ľubovoľné celé nezáporné čísla.}
\podpis{Jaromír Šimša}

{%%%%% C-II-2
Určte, koľkými spôsobmi možno všetky hrany kocky $ABCDEFGH$
ofarbiť štyrmi danými farbami (celú hranu bez krajných bodov
vždy jednou farbou), aby pritom každá stena kocky mala hrany všetkých
štyroch farieb.}
\podpis{Jaroslav Švrček}

{%%%%% C-II-3
V~pravouhlom lichobežníku $ABCD$ s~pravým uhlom pri vrchole~$A$ základne~$AB$ je
bod~$K$ priesečníkom výšky~$CP$ lichobežníka s~jeho uhlopriečkou~$BD$.
Obsah štvoruholníka $APCD$ je polovicou obsahu lichobežníka
$ABCD$. Určte, akú časť obsahu trojuholníka $ABC$ zaberá trojuholník $BCK$.}
\podpis{Lucie Růžičková}

{%%%%% C-II-4
Adam s~Barborou hrajú so zlomkom
$$
{10a+b\over 10c+d}
$$
takúto hru na štyri ťahy: Hráči striedavo nahrádzajú ľubovoľné
z~doposiaľ neurčených písmen $a$, $b$, $c$, $d$
nejakou cifrou od~$1$ do~$9$. Barbora vyhrá, keď
výsledný zlomok bude rovný buď celému číslu, alebo číslu
s~konečným počtom
desatinných miest; inak vyhrá Adam (napríklad keď vznikne
zlomok~$\frac{11}{29}$). Ak začína Adam, ako
má hrať Barbora, aby zaručene vyhrala? Ak začína Barbora, je možné
poradiť Adamovi tak, aby vždy vyhral?}
\podpis{Tomáš Jurík}

{%%%%%   vyberko, den 1, priklad 1
Určte všetky hodnoty~$n$, pre ktoré je možné rozdeliť trojuholník na $n$ menších trojuholníkov, pričom žiadne tri vrcholy neležia na priamke a~z~každého bodu vychádza rovnaký počet úsečiek.}
\podpis{Michal Kopf, Miroslav Psota:Brazil 2004, national round}

{%%%%%   vyberko, den 1, priklad 2
V~ostrouhlom trojuholníku $ABC$ leží bod~$D$ na strane~$BC$. Nech body $O_1$, $O_2$ označujú stredy opísaných kružníc trojuholníkom $ABD$ a~$ACD$. Dokážte, že priamka spájajúca stred opísanej kružnice $ABC$ a~ortocentrum trojuholníka $O_1O_2D$ je rovnobežná s~$BC$.}
\podpis{Michal Kopf, Miroslav Psota:India 2014, national round}

{%%%%%   vyberko, den 1, priklad 3
Pre dané $a\in\Bbb{R}$ a~$n\in\Bbb{N}$, dokážte:
\ite a) Existuje práve jedna postupnosť reálnych čísel $x_0,x_1,\dots,x_n,x_{n+1}$ spĺňajúca
$$
x_0=x_{n+1}=0\qquad\text{a}\qquad\frac{1}{2}(x_i+x_{i+1})=x_i+x_i^3-a^3,\quad\text{pre $i=1,2,\dots,n$.}
$$
\ite b) Postupnosť $x_0,x_1,\dots,x_n,x_{n+1}$ z~a) spĺňa $|x_i|\le |a|$ pre $i=0,1,\dots,n+1$.}
\podpis{Michal Kopf, Miroslav Psota:China 2004, national round}

{%%%%%   vyberko, den 1, priklad 4
Nech $n$ je kladné celé číslo a~$\mm S_n$ je množina všetkých kladných deliteľov čísla $n$ (vrátane $1$ a~$n$). Dokážte, že najviac polovica čísel v $\mm S_n$ má ako poslednú cifru 3.}
\podpis{Michal Kopf, Miroslav Psota:2003 China Girls Math Olympiad}

{%%%%%   vyberko, den 2, priklad 1
Nech $ABCD$ je tetivový štvoruholník vpísaný do kružnice so stredom~$O$. Nech $E$, $F$ sú postupne stredy oblúkov $AB$, $CD$, ktoré neobsahujú zvyšné vrcholy štvoruholníka. Bodmi $E$, $F$ veďme priamky rovnobežné s~uhlopriečkami štvoruholníka $ABCD$. Tieto štyri priamky sa pretínajú v~bodoch $E$, $F$, $K$, $L$. Dokážte, že body $K$, $O$, $L$ ležia na jednej priamke.}
\podpis{Jozef Rajník, Martin Vodička:Sharygin 2013}

{%%%%%   vyberko, den 2, priklad 2
Nech funkcia $f\colon\Bbb R^3\to\Bbb R$ spĺňa pre všetky pätice reálnych čísel $a$, $b$, $c$, $d$, $e$ rovnosť
$$
f(a,b,c)+f(b,c,d)+f(c,d,e)+f(d,e,a)+f(e,a,b)=a+b+c+d+e.
$$
Dokážte, že pre všetky reálne čísla $x_1,x_2,\dots,x_n$, kde $ n\ge 5$, platí
$$
f(x_1,x_2,x_3)+f(x_2,x_3,x_4)+\ldots +f(x_{n-1},x_n,x_1)+f(x_n,x_1,x_2)=x_1+x_2+\ldots+x_n.
$$}
\podpis{Jozef Rajník, Martin Vodička:Polsko 2008}

{%%%%%   vyberko, den 2, priklad 3
Nech $\mm M_0 \subset  \Bbb N$ je neprázdna a~konečná množina. Vytvárame postupne množiny $\mm M_1,\mm M_2,\dots$, nasledujúcim spôsobom: V~$n$-tom kroku zvolíme ľubovoľné prirodzené číslo~$b_n$ a~definujeme
$$
\mm M_n = \{ b_nm+1;\qquad m\in \mm M_{n-1} \}.
$$
Dokážte, že existuje prirodzené číslo~$k$ také, že v~množine~$\mm M_k$ sa nenachádzajú dve rôzne prirodzené čísla také, že jedno z~nich delí druhé.}
\podpis{Jozef Rajník, Martin Vodička:Iran 2015}

{%%%%%   vyberko, den 2, priklad 4
Nesmrteľná žaba skáče po reálnej číselnej osi. Začína na čísle~$0$ a~v~$n$-tom skoku skočí o~$2^n+1$, a~to ľubovoľným smerom. Zistite, či žaba vie skákať tak, aby na každé prirodzené číslo niekedy doskočila.}
\podpis{Jozef Rajník, Martin Vodička:ARO 2015}

{%%%%%   vyberko, den 3, priklad 1
Daný je ostrouhlý trojuholník $ABC$, ktorého priesečník výšok označme~$H$. Nech $D$ je taký bod, že štvoruholník $HABD$ je rovnobežníkom (t.\,j. $AB\parallel HD$ a~${AH\parallel BD}$). Označme $E$ ten bod ležiaci na priamke $DH$, pre ktorý priamka~$AC$ prechádza stredom úsečky~$HE$.
Bod~$F$ je ďalším priesečníkom priamky~$AC$ s~kružnicou opísanou trojuholníku $DCE$.
Dokážte, že~$|EF|=|AH|$.}
\podpis{Peter Novotný:Shortlist 2015, G1}

{%%%%%   vyberko, den 3, priklad 2
Postupnosť $a_1,a_2,\dots$ kladných reálnych čísel spĺňa nerovnosť
$$
{a_{k+1}\ge \frac{ka_k}{a_k^2+k-1}}
$$
pre každé kladné celé číslo~$k$.
Dokážte, že
$$
{a_1+a_2+\ldots +a_n\ge n}
$$
pre každé $n\ge 2$.}
\podpis{Peter Novotný:Shortlist 2015, A1}

{%%%%%   vyberko, den 3, priklad 3
Pre danú konečnú množinu~$\mm A$ kladných celých čísel nazveme jej rozdelenie na dve disjunktné neprázdne množiny $\mm A_1$, $\mm A_2$ {\it pekné}, ak najmenší spoločný násobok všetkých prvkov množiny $\mm A_1$ je rovný najväčšiemu spoločnému deliteľovi všetkých prvkov množiny $\mm A_2$. Určte najmenšiu hodnotu $n$ takú, že existuje $n$-prvková množina $\mm A$ majúca práve 2015 pekných rozdelení.}
\podpis{Peter Novotný:Shortlist 2015, C3}

{%%%%%   vyberko, den 4, priklad 1
Po štvorcovej tabuľke $(4n + 2)\times (4n + 2)$ sa pohybuje korytnačka len medzi štvorčekmi susediacimi hranou. Korytnačka spraví nasledovnú prechádzku: začne v~rohovom štvorčeku tabuľky, prejde každým štvorčekom práve raz a~skončí na mieste kde začala. V~závislosti od~$n$ určte najväčšie prirodzené číslo~$k$ také, že v~tabuľke musí existovať riadok alebo stĺpec, do ktorého korytnačka vstúpila aspoň $k$-krát (vstúpiť do riadku/stĺpca znamená presunúť sa z~iného riadku/stĺpca do tohto riadku/stĺpca).}
\podpis{Matej Králik, Filip Hanzely:Canada 2015, national round}

{%%%%%   vyberko, den 4, priklad 2
Nájdite všetky $n$ také, že pre ľubovoľnú $n$-ticu nezáporných reálnych čísel $x_1, x_2, \dots, x_n$ platí
$$
\max_{i=1,...,n}\{2x_ix_{i+1}\}\ge\min_{j=1,...,n}\{x_j^2+x_{j+1}^2\},
$$
pričom $x_1=x_{n+1}$.}
\podpis{Matej Králik, Filip Hanzely:EGMO, 2016, upravene}

{%%%%%   vyberko, den 4, priklad 3
Nájdite všetky funkcie $f\colon\Bbb{N}_0\to\Bbb{N}_0$ také, že
\item $f(n)=n$ pre $0\le n\le 2016$,
\item $f(n^2+m^2)=f(n)^2+f(m)^2$ pre $m,n\in\Bbb{N}_0.$
\endgraf}
\podpis{Matej Králik, Filip Hanzely:Komal zbierka 1994-1997, N40, upravene}

{%%%%%   vyberko, den 4, priklad 4
V~tetivovom štvoruholníku $ABCD$ sa dotyčnice k~jeho opísanej kružnici v~bodoch $A$, $C$ pretínajú na priamke~$BD$. Označme $M$ stred~$AC$. Rovnobežka s~$BC$ cez $D$ pretína priamku~$BM$ v~bode~$E$ a~kružnicu opísanú $ABCD$ v~bode~$F$, $F\ne D$. Dokážte, že $BCEF$ je rovnobežník.}
\podpis{Matej Králik, Filip Hanzely:European cup 2014, upravene}

{%%%%%   vyberko, den 5, priklad 1
Nech $I$ je stred vpísanej kružnice trojuholníka $ABC$, pričom $|AB|\ne |AC|$. Označme $M$ stred strany~$BC$ a~$D$ bod dotyku vpísanej kružnice so stranou~$BC$. Kružnica so stredom v~bode~$M$ a~polomerom~$MD$ pretína priamku~$AI$ v~bodoch $P$ a~$Q$. Dokážte, že $|\angle BAC|+|\angle PMQ|=180^\circ$.}
\podpis{Patrik Bak, Tomáš Jurík:Rumunsko 2015}

{%%%%%   vyberko, den 5, priklad 2
Nájdite všetky nepárne prirodzené čísla~$M$, pre ktoré postupnosť $a_0,a_1,a_2,\dots$ definovaná ako $a_0=\frac12(2M+1)$ a $a_{k+1}=a_k\lfloor a_k\rfloor$ pre $k=0,1,2,\dots$ obsahuje aspoň jedno celé číslo.}
\podpis{Patrik Bak, Tomáš Jurík:Shortlist 2015, N1}

{%%%%%   vyberko, den 5, priklad 3
Na párty s 2016 hosťami bolo medzi ľubovoľnými 7~hosťami najviac 12~podaní rúk. Určte najväčší možný celkový počet podaní rúk.}
\podpis{Patrik Bak, Tomáš Jurík:Turecko 2015}

{%%%%%   vyberko, ...
...}
\podpis{...}

{%%%%%   vyberko, ...
...}
\podpis{...}

{%%%%%   trojstretnutie, priklad 1
Daný je $n$-uholník $\Cal P$, pričom $n>4$. Dokážte, že existujú tri jeho rôzne vrcholy $A$,~$B$,~$C$ spĺňajúce nasledujúcu vlastnosť:
Ak $l_1$, $l_2$, $l_3$ sú dĺžky troch lomených čiar, na ktoré vrcholy $A$, $B$, $C$ delia obvod mnohouholníka~$\Cal P$, tak existuje trojuholník so stranami dĺžok $l_1$, $l_2$, $l_3$.}
\podpis{Poľsko}

{%%%%%   trojstretnutie, priklad 2
Nech $m,n > 2$ sú dané párne čísla. Uvažujme tabuľku s~rozmermi $m \times n$, ktorej každé políčko je zafarbené buď čiernou, alebo bielou farbou. Hádajúca Hanka ofarbenie nevidí, ale môže klásť veštcovi určité otázky. Presnejšie, môže sa spýtať na dve susedné políčka (majúce spoločnú stranu) a~veštec odhalí, či tieto dve políčka majú rovnakú, alebo rôznu farbu. Hanka má za úlohu určiť paritu počtu dvojíc susedných políčok, ktoré majú rôznu farbu. Koľko minimálne otázok Hanke určite stačí na to, aby dokázala určiť správnu odpoveď?}
\podpis{Česká rep.}

{%%%%%   trojstretnutie, priklad 3
Nech $n$ je dané kladné celé číslo. Pre konečnú množinu $\mm M$ kladných celých čísel a~pre každé $i \in \{0, 1, \dots, n-1\}$ označíme $s_i$ počet takých neprázdnych podmnožín množiny $\mm M$, ktorých súčet prvkov dáva po delení $n$ zvyšok $i$. Hovoríme, že množina~$\mm M$ je {\it $n$-vyvážená}, pokiaľ $s_0 = s_1 = \dots = s_{n-1}$. Napríklad pre $n=5$ a~$\mm M=\{1,3,4\}$ máme $s_0 = s_1 = s_2 = 1$ a~$s_3 = s_4 = 2$, takže $\mm M$ nie je $5$-vyvážená.
Dokážte, že pre každé nepárne kladné číslo~$n$ existuje $n$-vyvážená podmnožina množiny $\{1, 2, \dots, n\}$.}
\podpis{Česká rep.}

{%%%%%   trojstretnutie, priklad 4
Nájdite všetky štvorice $(a,b,c,d)$ reálnych čísel, ktoré sú riešením sústavy
$$\align
(a+b)(a^2+b^2) &= (c+d)(c^2+d^2), \\
(a+c)(a^2+c^2) &= (b+d)(b^2+d^2), \\
(a+d)(a^2+d^2) &= (b+c)(b^2+c^2).
\endalign
$$}
\podpis{Patrik Bak}

{%%%%%   trojstretnutie, priklad 5
Dokážte, že pre každé nezáporné celé číslo~$n$ existujú celé čísla $x$, $y$, $z$ také, že
$$
\nsd(x, y, z) = 1\qquad\text{a}\qquad
x^2 + y^2 + z^2 = 3^{2^n}.
$$}
\podpis{Poľsko}

{%%%%%   trojstretnutie, priklad 6
Daný je ostrouhlý trojuholník $ABC$, pričom $|AB| < |AC|$. Označme $\Omega$ jeho opísanú kružnicu. Dotyčnica ku kružnici~$\Omega$ vedená bodom~$A$ pretína priamku $BC$ v~bode~$D$. Nech $G$ je ťažisko trojuholníka $ABC$ a~priamka~$AG$ pretína kružnicu~$\Omega$ v~bode $H\ne A$. Predpokladajme, že priamka~$DG$ pretína priamky $AB$ a~$AC$ postupne v~bodoch $E$ a~$F$. Dokážte, že $|\uhol EHG|=|\uhol GHF|$.}
\podpis{Patrik Bak}

{%%%%%   IMO, priklad 1
Trojuholník $BCF$ má pravý uhol pri vrchole $B$. Nech $A$ je bod priamky $CF$ taký, že $|FA|=|FB|$ a~bod~$F$ leží medzi bodmi $A$ a~$C$. Nech bod~$D$ je taký, že $|DA|=|DC|$ a~priamka~$AC$ je osou uhla $DAB$. Nech bod~$E$ je taký, že $|EA|=|ED|$ a~priamka~$AD$ je osou uhla $EAC$. Nech bod~$M$ je stred úsečky~$CF$. Nech bod~$X$ je taký, že $AMXE$ je rovnobežník (pričom $AM\parallel EX$ a~$AE\parallel MX$). Dokážte, že priamky $BD$, $FX$ a~$ME$
prechádzajú tým istým bodom.}
\podpis{Belgicko}

{%%%%%   IMO, priklad 2
Nájdite všetky kladné celé čísla~$n$, pre ktoré sa dá do každého políčka tabuľky $n \times n$ napísať práve jedno z~písmen {\it I}, {\it M\/} a~{\it O\/} tak, že platia obe nasledujúce podmienky:
\ite $\bullet$ V~každom riadku aj v~každom stĺpci obsahuje jedna tretina políčok písmeno~{\it I}, jedna tretina políčok písmeno~{\it M\/} a~jedna tretina políčok písmeno~{\it O}.
\ite $\bullet$ V~každom šikmom rade, ktorého počet políčok je násobkom troch, obsahuje jedna tretina políčok písmeno~{\it I}, jedna tretina políčok písmeno~{\it M\/} a~jedna tretina políčok písmeno~{\it O}.

\poznamka
Riadky a~stĺpce tabuľky $n \times n$ sú označené kladnými celými číslami od $1$ do $n$ v~obvyklom poradí, takže každé jej políčko zodpovedá dvojici kladných celých čísel $(i,j)$, kde $1 \le i,j \le n$. Ak $n > 1$, tak tabuľka má $4n-2$ {\it šikmých radov\/} dvoch typov. Šikmý rad prvého typu obsahuje práve všetky políčka $(i,j)$ také, že $i+j$ je konštanta, a~šikmý rad druhého typu obsahuje práve všetky políčka $(i,j)$ také, že $i-j$ je konštanta.}
\podpis{Austrália}

{%%%%%   IMO, priklad 3
Nech $\Cal P$, kde $\Cal P=A_1A_2\dots A_k$, je konvexný mnohouholník v~rovine. Jeho vrcholy $A_1,A_2,\dots,A_k$ majú celočíselné súradnice a~ležia na kružnici. Nech $S$ je obsah mnohouholníka~$\Cal P$. Nech $n$ je nepárne kladné celé číslo také, že druhé mocniny dĺžok strán mnohouholníka~$\Cal P$ sú celé čísla deliteľné~$n$. Dokážte, že $2S$ je celé číslo deliteľné~$n$.}
\podpis{Rusko}

{%%%%%   IMO, priklad 4
Množinu kladných celých čísel nazveme {\it voňavá}, ak obsahuje aspoň dva prvky a~každý jej prvok má s~nejakým iným jej prvkom aspoň jedného spoločného prvočíselného deliteľa. Nech $P(n)=n^2+n+1$. Nájdite najmenšie kladné celé číslo $b$ také, že existuje nezáporné celé číslo~$a$ také, že množina
$$
\{P(a+1), P(a+2), \dots, P(a+b) \}
$$
je voňavá.}
\podpis{Luxembursko}

{%%%%%   IMO, priklad 5
Na tabuli je napísaná rovnica
$$
(x-1)(x-2)\cdots(x-2016) = (x-1)(x-2)\cdots(x-2016)
$$
s~2016 lineárnymi dvojčlenmi na každej strane. Nájdite najmenšiu hodnotu~$k$, pre ktorú je možné vymazať práve $k$ z~týchto 4032 dvojčlenov tak,
že na každej strane ostane aspoň jeden dvojčlen a~výsledná rovnica nebude mať žiaden reálny koreň.}
\podpis{Rusko}

{%%%%%   IMO, priklad 6
V~rovine leží $n$~úsečiek, kde $n\ge 2$, a~to tak, že každé dve majú spoločný vnútorný bod, ale žiadne tri rôzne nemajú spoločný bod. Bohuš pre každú z~nich vyberie jeden jej koncový bod, položí naň žabu a~otočí ju smerom k~druhému koncovému bodu tejto úsečky. Potom bude $(n-1)$-krát tlieskať. Vždy keď tleskne, každá žaba okamžite skočí na nasledujúci priesečník na svojej úsečke. Žaba nikdy nemení smer svojich skokov. Bohuš si želá umiestniť žaby tak, aby žiadne dve z~nich nikdy neboli v~tom istom okamihu na tom istom priesečníku.
\ite a) Dokážte, že ak $n$ je nepárne, Bohuš si toto svoje želanie splniť môže.
\ite b) Dokážte, že ak $n$ je párne, Bohuš si toto svoje želanie splniť nemôže.
}
\podpis{Česká rep., Josef Tkadlec}

{%%%%%   MEMO, priklad 1
Nech $n \ge 2$ je kladné celé číslo a~$x_1,x_2,\dots,x_n$ sú reálne čísla spĺňajúce obidve podmienky
\ite i) $x_j>\m1$ pre $j=1,2,\dots,n$;
\ite ii) $x_1+x_2+\dots+x_n=n$.\endgraf
\noindent
Dokážte nerovnosť
$$
\sum_{j=1}^n \frac{1}{1+x_j} \ge \sum_{j=1}^n \frac{x_j}{1+x_j^2}
$$
a~určte, kedy nastáva rovnosť.}
\podpis{Rakúsko}

{%%%%%   MEMO, priklad 2
Majme na tabuli napísaných $n\ge 3$ kladných celých čísel. Ťah pozostáva z~výberu troch čísel $a$, $b$, $c$ napísaných na tabuli, ktoré sú stranami nedegenerovaného nerovnostranného trojuholníka, a~nahradením ich číslami $a+b-c$, $b+c-a$ a~$c+a-b$.
Dokážte, že nemôže existovať nekonečná postupnosť takýchto ťahov.}
\podpis{Švajčiarsko}

{%%%%%   MEMO, priklad 3
Nech $ABC$ je ostrouhlý trojuholník so stredom opísanej kružnice $O$, pričom $|\uhol BAC| > 45^{\circ}$. Bod~$P$ leží v~jeho vnútri tak, že body $A$, $P$, $O$, $B$ ležia na kružnici a~priamka~$BP$ je kolmá na $CP$. Bod~$Q$ leží na úsečke~$BP$ tak, že $AQ$ a~$PO$ sú rovnobežky. Dokážte, že $|\uhol QCB|=|\uhol PCO|$.}
\podpis{Slovensko, Patrik Bak}

{%%%%%   MEMO, priklad 4
Nájdite všetky funkcie  $f\colon \Bbb{N} \to \Bbb{N}$ také, že $f(a) + f(b)$ delí $2(a+b-1)$ pre všetky $a,b\in\Bbb N$.
}
\podpis{Chorvátsko}

{%%%%%   MEMO, priklad t1
Určte všetky trojice reálnych čísel $(a, b, c)$, spĺňajúce sústavu rovníc
$$
\align
a^2 + ab + c &= 0, \\
b^2 + bc + a &= 0, \\
c^2 + ca + b &= 0.
\endalign
$$}
\podpis{Chorvátsko}

{%%%%%   MEMO, priklad t2
Nech $\Bbb{R}$ je množina reálnych čísel. Určte všetky funkcie $f\colon\Bbb{R}\to\Bbb{R}$, pre ktoré
$$
f(x)f(y)=xf(f(y-x))+xf(2x)+f(x^2)
$$
platí pre všetky reálne čísla $x$ a $y$.}
\podpis{Litva}

{%%%%%   MEMO, priklad t3
Územie tvaru štvorca $8 \times 8$, ktorého strany majú orientáciu sever-juh a~východ-západ, je tvorené 64 menšími štvorcovými pozemkami $1 \times 1$. Na každom jednotlivom pozemku môže stáť najviac jeden dom. Každý dom môže stáť len na jednom štvorcovom pozemku $1 \times 1$. Hovoríme, že dom je {\it blokovaný od slnka}, ak stoja tri domy na pozemkoch bezprostredne na východ, západ a~juh od neho. Aký je maximálny počet domov, ktoré môžu naraz stáť v~tomto území tak, že žiaden z~nich nie je blokovaný od slnka?

\poznamka
Podľa definície, domy na východnej, západnej a~južnej hranici územia nie sú nikdy blokované od slnka.}
\podpis{Chorvátsko}

{%%%%%   MEMO, priklad t4
Trieda stredoškolských študentov písala test. Každá otázka bola bodovaná buď 1~bodom za správnu odpoveď, alebo 0~bodmi inak. Každá otázka bola správne zodpovedaná aspoň jedným študentom a~nie všetci študenti dosiahli rovnaký celkový počet bodov. Dokáže, že v~teste bola otázka s~nasledujúcou vlastnosťou: Študenti, ktorí zodpovedali túto otázku správne, dosiahli vyšší priemerný celkový počet bodov ako tí, čo na túto otázku neodpovedali správne.}
\podpis{Rakúsko}

{%%%%%   MEMO, priklad t5
Nech $ABC$ je ostrouhlý trojuholník, pričom $|AB| \ne |AC|$. Označme $O$ stred jeho opísanej kružnice~$\omega$. Priamka~$AO$ pretína kružnicu~$\omega$ druhýkrát v~bode~$D$ a~priamku~$BC$ v~bode~$E$. Kružnica opísaná trojuholníku $CDE$ pretína priamku~$CA$ druhýkrát v~bode~$P$. Priamka~$PE$ pretína priamku~$AB$ v~bode~$Q$. Priamka cez bod~$O$ rovnobežná s~$PE$ pretína výšku trojuholníka $ABC$ z~bodu~$A$ v~bode~$F$. Dokážte, že $|FP|=|FQ|$.}
\podpis{Chorvátsko}

{%%%%%   MEMO, priklad t6
Daný je trojuholník $ABC$, pričom $|AB|\ne|AC|$. Body $K$, $L$, $M$ sú postupne stredmi strán $BC$, $CA$, $AB$. Kružnica so stredom~$I$ vpísaná trojuholníku $ABC$ sa dotýka strany~$BC$ v~bode~$D$. Priamka~$g$, ktorá prechádza stredom úsečky~$ID$ a~je kolmá na $IK$, pretína priamku~$LM$ v~bode~$P$. Dokážte, že $|\uhol PIA|=90^\circ$.}
\podpis{Poľsko}

{%%%%%   MEMO, priklad t7
Kladné celé číslo~$n$ nazveme {\it mozartovské číslo}, ak je v~postupnosti $1,2,\dots,n$ napísaná každá cifra párny počet krát (v~desiatkovej sústave). Dokážte nasledujúce tvrdenia:
\ite a) Všetky mozartovské čísla sú párne.
\ite b) Existuje nekonečne veľa mozartovských čísel.\endgraf
}
\podpis{Slovensko, Patrik Bak}

{%%%%%   MEMO, priklad t8
Uvažujme rovnicu $a^2+b^2+c^2+n=abc$, kde $a$, $b$, $c$ sú kladné celé čísla. Dokážte nasledujúce tvrdenia:
\ite a) Pre $n=2017$ rovnica nemá riešenie $(a, b, c)$.
\ite b) Pre $n=2016$ v~každom riešení $(a,b,c)$ musí byť $a$ deliteľné tromi.
\ite c) Pre $n=2016$ má rovnica nekonečne veľa riešení $(a, b, c)$.
\endgraf
}
\podpis{Rakúsko}

{%%%%%   CPSJ, priklad 1
V~rovine je daná úsečka~$AB$ so stredom~$M$. Uvažujme množinu všetkých pravouhlých trojuholníkov $ABC$ s~preponou~$AB$, v~ktorých $D$
je päta výšky z~vrcholu~$C$ a~$K$, $L$ sú päty kolmíc spustených z~bodu~$D$ postupne na odvesny $BC$, $AC$. Určte, aký je najväčší možný
obsah štvoruholníka $MKCL$.}
\podpis{Jaroslav Švrček}

{%%%%%   CPSJ, priklad 2
Pre reálne čísla $x$, $y$ platí $x^2+y^2-1<xy$. Dokážte, že potom $x+y-|x-y|<2$.
}
\podpis{Pavol Kossaczký}

{%%%%%   CPSJ, priklad 3
Určte všetky celé čísla $n\ge 3$ také, že vrcholy pravidelného $n$-bokého hranola sa dajú označiť navzájom rôznymi kladnými celými číslami tak, aby vrcholy s~číslami $a$ a~$b$ boli spojené hranou práve vtedy, keď $a\mid b$ alebo $b\mid a$.}
\podpis{Barbara Roszkowska-Lech}

{%%%%%   CPSJ, priklad 4
Daný je ostrouhlý trojuholník $ABC$, v~ktorom ${|AB|<|AC|<|BC|}$. Na stranách $AC$ a~$BC$ sú postupne zvolené také body $K$ a~$L$, že ${|AB|=|CK|=|CL|}$. Osi úsečiek $AK$ a~$BL$ pretínajú priamku $AB$ postupne v~bodoch $P$ a~$Q$. Úsečky $KP$ a~$LQ$ sa pretínajú v~bode~$M$. Dokážte, že ${|AK|+|KM|=|BL|+|LM|}$.
}
\podpis{\L{}ukasz Bożyk}

{%%%%%   CPSJ, priklad 5
Určte najmenšie celé číslo~$j$, pri ktorom možno do jednotlivých políčok štvorcovej tabuľky $10\times10$ vpísať celé čísla od 1 do 100 tak, aby každých 10 po sebe idúcich čísel ležalo v~niektorej štvorcovej časti $j\times j$ celej tabuľky.
}
\podpis{Jaromír Šimša}

{%%%%%   CPSJ, priklad t1
Je dán pravoúhlý trojúhelník $ABC$ s přeponou $AB$. Označme $D$ patu výšky z~vrcholu $C$. Nechť $Q$, $R$ a $P$ jsou po řadě středy úseček $AD$, $BD$ a $CD$. Dokažte, že platí
$$
|\uh APB|+|\uh QCR|=180^{\circ}.
$$
}
\podpis{Jaroslav Švrček}

{%%%%%   CPSJ, priklad t2
Najděte největší možné celé číslo $d$, které současně dělí tři trojciferná čísla $\overline{abc}$, $\overline{bca}$ a~$\overline{cab}$, kde $a$,~$b$~ a~$c$ jsou vhodné nenulové a navzájem různé číslice.
}
\podpis{Jaromír Šimša}

{%%%%%   CPSJ, priklad t3
Na p\l{}aszczyźnie poprowadzono pewn\ą{} liczb\ę{} prostych tak, że każda przecina dok\l{}adnie 15 innych. Ile prostych poprowadzono? Scharakteryzuj wszystkie możliwe konfiguracje i~uzasadnij, że nie ma innych.
}
\podpis{Barbara Roszkowska-Lech}

{%%%%%   CPSJ, priklad t4
Pewn\ą{} liczb\ę{} p\l{}ytek przystaj\ą{}cych do przedstawionej na rysunku należy umieścić wewn\ą{}trz kwadratu o~wymiarach $11\times 11$ podzielonego na pola b\ę{}d\ą{}ce kwadratami jednostkowymi w~taki sposób, aby każda p\l{}ytka pokrywa\l{}a dok\l{}adnie $6$ pól, żadna nie wystawa\l{}a poza kwadrat oraz żadne dwie nie pokrywa\l{}y tego samego pola.\insp{5cpsj-prop.30}
\ite{a)} Wyznacz najwi\ę{}ksz\ą{} możliw\ą{} liczb\ę{} p\l{}ytek, któr\ą{} można umieścić wewn\ą{}trz kwadratu w~opisany sposób.
\ite{b)} Znajdź wszystkie pola, które musz\ą{} zostać przykryte przy każdym pokryciu z~użyciem maksymalnej liczby p\l{}ytek.\endgraf
%
}
\podpis{Jerzy Bednarczuk}

{%%%%%   CPSJ, priklad t5
Daný je trojuholník $ABC$ taký, že $|AB|:|AC|:|BC| = 5:5:6$. Označme $M$ stred strany~$BC$ a~$N$ taký bod na strane~$BC$, že $|BN| = 5\cdot|CN|$. Dokážte, že stred kružnice opísanej trojuholníku $ABN$ je stredom úsečky spájajúcej stredy kružníc vpísaných trojuholníkom $ABC$ a~$ABM$.
}
\podpis{Patrik Bak}

{%%%%%   CPSJ, priklad t6
Dané je kladné celé číslo~$k$. Nájdite všetky trojice kladných celých čísel $a$, $b$, $c$, ktoré spĺňajú rovnosti
$$
\aligned
a+b+c&=3k+1,\\
ab+bc+ca&=3k^2+2k.
\endaligned
$$
}
\podpis{Patrik Bak}
