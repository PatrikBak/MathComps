{%%%%%   Z4-I-1
...}
\podpis{...}

{%%%%%   Z4-I-2
...}
\podpis{...}

{%%%%%   Z4-I-3
...}
\podpis{...}

{%%%%%   Z4-I-4
...}
\podpis{...}

{%%%%%   Z4-I-5
...}
\podpis{...}

{%%%%%   Z4-I-6
...}
\podpis{...}

{%%%%% Z5-I-1
Čísla 1, 2, 3, 4, 5, 6, 7, 8 a~9 cestovali vlakom.
Vlak mal tri vagóny a~v~každom sa viezli práve tri čísla.
Číslo 1 sa viezlo v~prvom vagóne a~v~poslednom vagóne boli všetky čísla nepárne.
Sprievodca cestou spočítal súčet čísel v~prvom, druhom aj~poslednom vagóne a~zakaždým mu vyšiel rovnaký súčet.
Určte, ako mohli byť čísla do vagónov rozdelené.
}
\podpis{Veronika Hucíková}

{%%%%% Z5-I-2
Marta niesla svojej chorej kamarátke Majke 7~jabĺk, 6~hrušiek a~3~pomaranče.
Cestou ale dva kusy ovocia zjedla.
Určte, ktoré z~nasledujúcich situácií mohli nastať a~aké dva kusy ovocia by
Marta v~takom prípade musela zjesť:
\begin{enumerate}\alphatrue
\item Majka nedostala žiadny pomaranč.
\item Majka dostala menej hrušiek ako pomarančov.
\item Majka dostala rovnaký počet jabĺk, hrušiek aj~pomarančov.
\item Majka dostala rovnaký počet kusov ovocia dvojakého druhu.
\item Majka dostala viac jabĺk ako ostatných kusov ovocia dokopy.
\end{enumerate}
}
\podpis{Libuše Hozová}

{%%%%% Z5-I-3
Mamička vyprala štvorcové utierky a~vešia ich vedľa seba na bielizňovú šnúru natiahnutú medzi dvoma stromami.
Použila šnúru s~dĺžkou 7{,}5~metra, pričom na uviazanie okolo kmeňov potrebovala na každej strane~8\,dm.
Všetky utierky majú šírku 45\,cm.
Medzi krajnou utierkou a~kmeňom mamička necháva medzeru aspoň 10\,cm, utierky sa jej neprekrývajú
a~nemá ich zložené ani skrčené.
Koľko najviac utierok môže takto zavesiť na natiahnutú šnúru?
}
\podpis{Lenka Dedková}

{%%%%% Z5-I-4
Keď pán Baran zakladal chov, mal bielych oviec o~8 viac ako čiernych.
V~súčasnosti má bielych oviec štyrikrát viac ako na začiatku a~čiernych trikrát viac ako na začiatku.
Bielych oviec je teraz o~42 viac ako čiernych.
Koľko teraz pán Baran chová bielych a~čiernych oviec dokopy?
}
\podpis{Libor Šimůnek}

{%%%%% Z5-I-5
Štvorcová sieť sa skladá zo štvorcov so stranou dĺžky $1\cm$.
Narysujte do nej aspoň tri rôzne útvary také, aby každý mal obsah
$6\cm^2$ a~obvod $12\cm$ a~aby ich strany splývali s~priamkami siete.
}
\podpis{Eva Semerádová}

{%%%%% Z5-I-6
V~nepriestupnom roku bolo 53 nedieľ. Na aký deň týždňa pripadol Štedrý deň?
}
\podpis{Marta Volfová}

{%%%%% Z6-I-1
Archeológovia zistili, že vlajka bájneho matematického kráľovstva bola
rozdelená na šesť políčok, tak ako na obrázku.
V skutočnosti bola vlajka trojfarebná a~každé políčko bolo vyfarbené jednou
farbou.
\insp{z6-I-1.eps}%
Vedci už vybádali, že na vlajke bola použitá červená, biela a~modrá farba, že
vnútorné obdĺžnikové políčko bolo biele a~že spolu nesusedili dve políčka rovnakej
farby.
Určte, koľko možností vzhľadu vlajky musia archeológovia v~tejto fáze výskumu zvažovať.
}
\podpis{Veronika Hucíková}

{%%%%% Z6-I-2
Juraj išiel do služby k~čarodejníkovi.
Ten mal v~prvej pivnici viac múch ako pavúkov, v~druhej naopak.
V~každej pivnici mali muchy a~pavúky dokopy 100 nôh.
Určte, koľko mohlo byť múch a~pavúkov v~prvej a~koľko v~druhej pivnici.
}
\podpis{Marie Krejčová}

{%%%%% Z6-I-3
Na obrázku je štvorec $ABCD$, štvorec $EFGD$ a~obdĺžnik $HIJD$.
Body $J$ a~$G$ ležia na strane~$CD$, pričom platí $|DJ|<|DG|$,
a~body $H$ a~$E$ ležia na strane~$DA$, pričom platí $|DH|<|DE|$.
Ďalej vieme, že $|DJ|=|GC|$.
Šesťuholník $ABCGFE$ má obvod $96\cm$, šesťuholník $EFGJIH$ má obvod $60\cm$
a~obdĺžnik $HIJD$ má obvod $28\cm$.
Určte obsah šesťuholníka $EFGJIH$.
\insp{z6-I-3.eps}%
}
\podpis{Libor Šimůnek}

{%%%%% Z6-I-4
Na obrázku je obdĺžnik rozdelený na 7~políčok.
Na každé políčko sa má napísať práve jedno z~čísel 1, 2 a~3.
\insp{z6-I-4.eps}%
Miro tvrdí, že sa to dá spraviť tak, aby súčet dvoch vedľa seba napísaných čísel bol zakaždým iný.
Zuzka naopak tvrdí, že to nie je možné.
Rozhodnite, kto z~nich má pravdu.
}
\podpis{Veronika Hucíková}

{%%%%% Z6-I-5
Pán Cuketa mal obdĺžnikovú záhradu, ktorej obvod bol 28 metrov.
Obsah celej záhrady vyplnili práve štyri štvorcové záhony, ktorých rozmery
v~metroch boli vyjadrené celými číslami.
Určte, aké rozmery mohla mať záhrada. Nájdite všetky možnosti.
}
\podpis{Libuše Hozová}

{%%%%% Z6-I-6
V~zámockej kuchyni pripravujú rezancovú polievku v~hrncoch a~kotloch.
V~pondelok uvarili 25~hrncov a~10~kotlov polievky.
V~utorok uvarili 15~hrncov a~13~kotlov.
V~stredu uvarili 20~hrncov a~vo štvrtok 30~kotlov.
Pritom v~pondelok a~v~utorok uvarili rovnaké množstvo polievky.
Koľkokrát viac polievky uvarili vo štvrtok ako v~stredu?
}
\podpis{Karel Pazourek}

{%%%%% Z7-I-1
Myška Hryzka našla 27 rovnakých kocôčok syra.
Najskôr si z~nich poskladala veľkú kocku a~chvíľu počkala, kým sa syrové kocôčky k~sebe prilepili.
Potom z~každej steny veľkej kocky vyhrýzla strednú kocôčku.
Napokon zjedla aj kocôčku, ktorá bola v~strede veľkej kocky.
Zvyšok syra chce Hryzka spravodlivo rozdeliť svojim štyrom mláďatám, a~preto ho chce rozrezať na štyri kusy rovnakého tvaru aj veľkosti.
Rezať bude iba pozdĺž stien kocôčok a~nič k~sebe už lepiť nebude.
Aký tvar môžu mať kusy syra pre mláďatá? Nájdite aspoň dve možnosti.
}
\podpis{Veronika Hucíková}

{%%%%% Z7-I-2
Vlčkovci majú 4~deti.
Ondrej je o~3~roky starší ako Matej a~Jakub o~5~rokov starší ako najmladšia Jana.
Vieme, že majú dokopy 30 rokov a~pred 3~rokmi mali dokopy 19~rokov.
Určte, koľko má ktoré dieťa rokov.
}
\podpis{Marta Volfová}

{%%%%% Z7-I-3
Vnútri pravidelného päťuholníka $ABCDE$ je bod $P$ taký, že trojuholník
$ABP$ je rovnostranný.
Aký veľký je uhol $BCP$?
}
\podpis{Libuše Hozová}

{%%%%% Z7-I-4
V~škole pre robotov do jednej triedy chodí dvadsať robotov Robertov, ktorí sú očíslovaní Robert~1 až Robert~20.
V~triede je práve napätá atmosféra, rozprávajú sa spolu iba niektorí roboti.
Roboti s~nepárnym číslom sa nerozprávajú s~robotmi s~párnym číslom.
Medzi Robertmi s~nepárnym číslom sa spolu rozprávajú iba roboti, ktorí majú číslo s~rovnakým počtom cifier.
Roberti s~párnym číslom sa rozprávajú iba s~tými, ktorých číslo začína rovnakou cifrou.
Koľko dvojíc robotov Robertov sa môže spolu navzájom rozprávať?
}
\podpis{Karel Pazourek}

{%%%%% Z7-I-5
V~kocúrkovskej škole používajú zvláštnu číselnú os.
Vzdialenosť medzi číslami 1 a~2 je 1\,cm, vzdialenosť medzi číslami 2 a~3 je 3\,cm,
medzi číslami 3 a~4 je 5\,cm a~tak ďalej:
vzdialenosť medzi každou nasledujúcou dvojicou prirodzených čísel sa vždy zväčší o~2\,cm.
Medzi ktorými dvoma prirodzenými číslami je na kocúrkovskej číselnej osi
vzdialenosť 39\,cm?
Nájdite všetky možnosti.
}
\podpis{Karel Pazourek}

{%%%%% Z7-I-6
Na výstave dlhosrstých mačiek sa zišlo spolu desať vystavujúcich.
Vystavovalo sa v~obdĺžnikovej miestnosti, v~ktorej boli dva rady stolov ako na obrázku.
\insp{z7-I-6.eps}%
Mačky boli označené navzájom rôznymi číslami v~rozsahu 1 až 10 a~na každom stole sedela jedna mačka.
Určte číslo mačky, ktorá bola na výstave hodnotená najlepšie, ak viete, že:
\begin{itemize}
\item súčet čísel mačiek sediacich oproti sebe bol vždy rovnaký,
\item súčet čísel každých dvoch mačiek sediacich vedľa seba bol párny,
\item súčin čísel každých dvoch mačiek sediacich vedľa seba v~dolnom rade bol násobok čísla~8,
\item mačka číslo 1 nebola na kraji a~bola viac vpravo ako mačka číslo~6,
\item vyhrala mačka sediaca v~pravom dolnom rohu.
\end{itemize}
}
\podpis{Martin Mach}

{%%%%% Z8-I-1
Mišo mal na poličke malú klaviatúru, ktorú vidíte na obrázku.
Na bielych klávesoch boli vyznačené ich tóny.
\insp{z8-I-1.eps}%
Klaviatúru našla malá Klára.
Keď ju brala z~poličky, vypadla jej z~ruky a~všetky biele klávesy sa z~nej vysypali.
Aby sa brat nehneval, začala ich Klára skladať späť.
Všimla si pritom, že sa dali vložiť iba na niektoré miesta, lebo im prekážali čierne klávesy umiestnené presne doprostred medzi dva biele.
Kláre sa podarilo klávesy nejako zložiť, avšak tóny na nich boli pomiešané,
keďže ešte nepoznala hudobnú stupnicu.
Zistite, koľkými spôsobmi mohla Klára klávesy poskladať.
}
\podpis{Erika Novotná}

{%%%%% Z8-I-2
Na lúke sa pasú kone, kravy a~ovce, spolu ich je menej ako 200.
Keby bolo kráv 45-krát viac, koní 60-krát viac a~oviec 35-krát viac ako ich je teraz, ich počty by sa rovnali.
Koľko sa spolu na lúke pasie koní, kráv a~oviec?
}
\podpis{Marie Krejčová}

{%%%%% Z8-I-3
Daný je rovnoramenný lichobežník $ABCD$, v~ktorom platí
$$
|AB|=2|BC|=2|CD|=2|DA|.
$$
Na jeho strane~$BC$ je bod~$K$ taký, že $|BK|=2|KC|$, na jeho strane~$CD$
je bod~$L$ taký, že $|CL|=2|LD|$, a~na jeho strane~$DA$ je bod~$M$ taký,
že $|DM|=2|MA|$.
Určte veľkosti vnútorných uhlov trojuholníka $KLM$.
}
\podpis{Jaroslav Zhouf}

{%%%%% Z8-I-4
V~komore, kde sa rozbilo svetlo a~všetko z~nej musíme brať naslepo, máme ponožky štyroch rôznych farieb.
Ak si chceme byť istí, že vytiahneme aspoň dve biele ponožky, musíme ich z~komory priniesť~28.
Aby sme mali takú istotu pre sivé ponožky, musíme ich priniesť tiež~28,
pre čierne ponožky stačí~26 a~pre modré ponožky~34.
Koľko je spolu v~komore ponožiek?
}
\podpis{Eva Semerádová}

{%%%%% Z8-I-5
Číslo dňa je poradové číslo daného dňa v~príslušnom mesiaci (teda napr. číslo dňa 5.~augusta 2016 je 5).
Ciferný súčet dňa je súčet hodnôt všetkých cifier v~dátume tohto dňa (teda
napr. ciferný súčet dňa 5.~augusta 2016 je $5+8+2+0+1+6=22$).
Šťastný deň je taký deň, ktorého číslo dňa je rovné cifernému súčtu dňa.
Určte, koľko šťastných dní je v~roku 2016 a~ktoré dni to sú.
}
\podpis{Lucie Růžičková}

{%%%%% Z8-I-6
Katka narysovala trojuholník $ABC$.
Stred strany~$AB$ označila $X$ a~stred strany~$AC$ označila $Y$.
Na strane~$BC$ chce nájsť taký bod~$Z$, aby obsah štvoruholníka $AXZY$ bol čo najväčší.
Akú časť trojuholníka $ABC$ môže maximálne zaberať štvoruholník $AXZY$?
}
\podpis{Alžbeta Bohiniková}

{%%%%% Z9-I-1
Objem vody v~mestskom bazéne s~obdĺžnikovým dnom je 6\,998{,}4 hektolitrov.
Propagačný leták uvádza, že keby sme chceli všetku vodu z~bazéna preliať do pravidelného štvorbokého hranola s~hranou podstavy rovnajúcou sa priemernej hĺbke bazéna, musel by byť hranol vysoký ako blízky televízny vysielač a~potom by bol naplnený až po okraj.
Dodávame, že keby sme chceli preplávať vzdialenosť rovnakú, ako je výška
vysielača, museli by sme preplávať buď osem dĺžok, alebo pätnásť šírok bazéna.
Aký vysoký je vysielač?
}
\podpis{Libor Šimůnek}

{%%%%% Z9-I-2
Úžasným číslom nazveme také párne číslo, ktorého rozklad na súčin prvočísel má práve tri nie nutne rôzne činitele a~súčet všetkých jeho deliteľov je rovný dvojnásobku tohto čísla.
Nájdite všetky úžasné čísla.
}
\podpis{Martin Mach}

{%%%%% Z9-I-3
Juro zostrojil štvorec $ABCD$ so~stranou $12\cm$.
Do tohto štvorca narysoval štvrťkružnicu~$k$, ktorá mala stred v~bode~$B$ a~prechádzala bodom~$A$,
a~polkružnicu~$l$, ktorá mala stred v~strede strany~$BC$ a~prechádzala bodom~$B$.
Rád by ešte zostrojil kružnicu, ktorá by ležala vnútri štvorca a~dotýkala
sa štvrťkružnice~$k$, polkružnice~$l$ aj strany~$AB$.
Určte polomer takej kružnice.
}
\podpis{Marta Volfová}

{%%%%% Z9-I-4
V~tabuľke je kurzový lístok zmenárne, avšak niektoré hodnoty sú v~ňom nahradené otáznikmi.
\insp{65-z9-i-4}
Zmenáreň vymieňa peniaze v~uvedených kurzoch a~neúčtuje si iné poplatky.

1. Koľko eur dostane zákazník, ak tu zmení 4\,200 českých korún?

Ak zmenárnik vykúpi od zákazníka 1\,000 libier a~potom ich všetky predá,
jeho celkový zisk je 2\,200~českých korún.
Keby namiesto toho zmenárnik predal 1\,000 libier a~potom by všetky utŕžené
české koruny zmenil s~iným zákazníkom za libry, zarobil by na tom 68{,}75 libier.

2. Za koľko českých korún zmenárnik nakupuje a~za koľko predáva 1~libru?
}
\podpis{Libor Šimůnek}

{%%%%% Z9-I-5
Betka si myslela prirodzené číslo s~navzájom rôznymi ciframi a~napísala ho na tabuľu.
%Betka napísala prirodzené číslo s~navzájom rôznymi ciframi.
Podeň zapísala cifry pôvodného čísla odzadu a~tak získala nové číslo.
Sčítaním týchto dvoch čísel dostala číslo, ktoré malo rovnaký počet cifier
ako myslené číslo a~skladalo sa iba z~cifier mysleného čísla
(avšak nemuselo obsahovať všetky jeho cifry).
Erike sa Betkino číslo zapáčilo a~chcela nájsť iné číslo s~rovnakými vlastnosťami.
Zistila, že neexistuje menšie také číslo ako Betkino a~väčšie sa jej hľadať nechcelo.
Určte, aké číslo si myslela Betka a~aké číslo by mohla nájsť Erika, keby
mala viac trpezlivosti.
}
\podpis{Katarína Jasenčáková}

{%%%%% Z9-I-6
Na stranách $AB$ a~$AC$ trojuholníka $ABC$ ležia postupne body $E$ a~$F$,
na úsečke~$EF$ leží bod~$D$.
Priamky $EF$ a~$BC$ sú rovnobežné a~súčasne platí
$$
|FD|:|DE|=|AE|:|EB|= 2:1.
$$
Trojuholník $ABC$ má obsah 27 hektárov a~úsečkami $EF$, $AD$ a~$DB$ je rozdelený na štyri časti.
Určte obsahy týchto štyroch častí.
}
\podpis{Vojtěch Žádník}

{%%%%%   Z4-II-1
...}
\podpis{...}

{%%%%%   Z4-II-2
...}
\podpis{...}

{%%%%%   Z4-II-3
...}
\podpis{...}

{%%%%% Z5-II-1
Mamička zavára slivky do fliaš tak, že slivky z~jednej fľaše jej vystačia buď na 16~šatôčok, alebo na 4~koláčiky, alebo na polovicu plechu ovocných rezov.
V~špajze má 4~také fľaše a~chce upiecť jeden plech ovocných rezov a~6~koláčikov.
Na koľko šatôčok jej vystačia zvyšné slivky?}
\podpis{Michaela Petrová}

{%%%%% Z5-II-2
Strýko Fero mal dve záhrady: mrkvová mala tvar štvorca, jahodová mala tvar obdĺžnika.
Pritom šírka jahodovej záhrady bola trikrát menšia ako šírka mrkvovej záhrady a~dĺžka jahodovej záhrady bola o~8~metrov väčšia ako dĺžka mrkvovej záhrady. Keď strýko záhrady oplotil, zistil, že obe mali rovnaký obvod. Určte rozmery mrkvovej a~jahodovej záhrady.}
\podpis{Erika Novotná}

{%%%%% Z5-II-3
Jeden mesiac mal štyri pondelky, päť nedieľ a~jeden piatok trinásteho. Na ktorý deň v~týždni pripadol toho roku Nový rok?
Ktorým dňom v~týždni bude Nový rok v~roku nasledujúcom?}
\podpis{Marta Volfová}

{%%%%% Z6-II-1
Pani učiteľka napísala na tabuľu dve čísla pod seba a~vyvolala Adama, aby ich sčítal.
Adam ich správne sčítal a~výsledok 39 napísal pod zadané čísla.
Pani učiteľka zotrela najvrchnejšie číslo, a~tak zvyšné dve čísla vytvorili nový príklad na sčítanie.
Tentoraz správny výsledok zapísala pod čísla Barbora.
Pani učiteľka opäť zotrela najvrchnejšie číslo, novo vzniknutý príklad na sčítanie správne vypočítal Cyril a~vyšlo mu 96.
Určte dve čísla, ktoré boli pôvodne napísané na tabuli.
}
\podpis{Libor Šimůnek}

{%%%%% Z6-II-2
Vnútri obdĺžnika $ABGH$ sú dva zhodné štvorce $CDEF$ a~$IJKL$.
Strana~$CF$ prvého štvorca leží na strane~$BG$ obdĺžnika a~strana~$IL$ druhého z~nich leží na strane~$HA$ obdĺžnika.
Obvod osemuholníka $ABCDEFGH$ je 48\,cm, obvod dvanásťuholníka $ABCDEFGHIJKL$ je 58\,cm.
Tento dvanásťuholník je súmerný podľa vodorovnej osi a~dĺžky všetkých jeho strán sú v~centimetroch vyjadrené celými číslami.
Určte dĺžky strán obdĺžnika $ABGH$.
\ifobrazkyvedla\else\insp{z6-II-2.eps}\fi
}
\podpis{Libor Šimůnek}

{%%%%% Z6-II-3
Na Černíčkovom statku mali tri sliepky.
Prvá znášala každý deň jedno vajce, druhá znášala každý druhý deň jedno vajce a~tretia každý tretí deň jedno vajce.
Pán Černíček prikúpil na trhu dve nové sliepky, ktoré tiež znášali vajíčka úplne pravidelne
-- vždy jedno po niekoľkých dňoch, pričom jedna z~nich znášala dvakrát viac ako druhá.
Pani Černíčková spočítala, že všetkých päť sliepok znieslo za 60~dní spolu 155 vajec.
Ako často znášali nové sliepky?
}
\podpis{Libuše Hozová}

{%%%%%   Z7-II-1
V~roku 1966 žilo v~obci Bezdíkov o~30 žien viac ako mužov.
Do súčasnosti sa počet žien žijúcich v~obci zmenšil štyrikrát a~počet mužov žijúcich v~obci klesol o~196.
Teraz je v~Bezdíkove o~10 žien viac ako mužov. Koľko žien a~mužov dokopy žije v~súčasnosti v~Bezdíkove?
}
\podpis{Libor Šimůnek}

{%%%%%   Z7-II-2
Pani učiteľka napísala na tabuľu dve čísla pod seba a~vyvolala Kláru, aby ich sčítala.
Klára správny výsledok zapísala pod zadané čísla.
Pani učiteľka zotrela najvrchnejšie číslo, a~tak zvyšné dve čísla vytvorili nový príklad na sčítanie.
Tentoraz správny výsledok zapísal pod čísla Lukáš.
Pani učiteľka opäť zotrela najvrchnejšie číslo, novo vzniknutý príklad na sčítanie správne vypočítala Magda a~vyšlo jej~94.
Jedno z~dvoch čísel, ktoré pani učiteľka pôvodne napísala na tabuľu, bolo 14, ale neprezradíme, ktoré.
Aké mohlo byť druhé z~pôvodne napísaných čísel?
Určte všetky možnosti.}
\podpis{Libor Šimůnek}

{%%%%%   Z7-II-3
V~obdĺžniku $ABCD$ so stranou~$AD$ majúcou dĺžku 5\,cm leží bod~$P$ tak, že trojuholník $APD$ je
rovnostranný.
Polpriamka~$AP$ pretína stranu~$CD$ v~bode~$E$, úsečka~$CE$ meria 5\,cm.
Aká dlhá je úsečka~$AE$ a~aká je veľkosť uhla $AEB$?
}
\podpis{Libuše Hozová}

{%%%%% Z8-II-1
Janko má v~komore na chalupe krabicu s~pastelkami, v~ktorej je 5~modrých, 9~červených, 6~zelených a~4~žlté pastelky.
Je tma, svetlo v~komore nesvieti a~Janko pri sebe nemá žiadny zdroj osvetlenia.
Farbu pasteliek teda nedokáže rozlíšiť.
\ite a) Koľko najmenej pasteliek musí vziať, aby mal istotu, že prinesie aspoň jednu pastelku z~každej farby?
\ite b) Koľko najviac pasteliek môže vziať, aby mal istotu, že v~krabici zostane z~každej farby aspoň jedna pastelka?
\ite c) Koľko najviac pasteliek môže vziať, aby mal istotu, že v~krabici zostane aspoň päť červených pasteliek?\endgraf
}
\podpis{Marta Volfová}

{%%%%% Z8-II-2
Dedo chová husi, prasatá, kozy a~sliepky -- celkom 40~kusov.
Na každú kozu pripadajú 3~husi.
Keby bolo sliepok o~8 menej, bolo by ich rovnako ako husí a~prasiat dokopy.
Keby dedo vymenil štvrtinu husí za sliepky v~pomere 3~sliepky za
1~hus, mal by celkom 46~kusov zvierat.
Koľko ktorých zvierat dedo chová?
}
\podpis{Marta Volfová}

{%%%%% Z8-II-3
Dobrákovci pestovali tulipány na štvorcovom záhone so stranou 6~metrov.
Neskôr pristavali k~svojmu domu štvorcovú terasu so stranou 7~metrov.
Jeden vrchol terasy ležal presne uprostred tulipánového záhona a~jedna strana terasy delila stranu tulipánového záhona v~pomere $1:5$.
V~akom pomere delila druhá strana terasy druhú stranu záhona?
O~koľko metrov štvorcových sa stavbou terasy zmenšil záhon tulipánov?
}
\podpis{Libuše Hozová}

{%%%%% Z9-II-1
V~Zverimexe vypredávali rybičky z~jedného akvária.
Ondrej chcel polovicu všetkých rybičiek, ale aby nemuseli žiadnu rybičku deliť, dostal o~polovicu rybičky viac, ako požadoval.
Matej si želal polovicu zvyšných rybičiek, ale rovnako ako Ondrej dostal o~polovicu rybičky viac, ako požadoval.
Nakoniec Petrík chcel polovicu zvyšných rybičiek, ale tiež dostal o~polovicu rybičky viac, ako požadoval.
Potom bolo akvárium bez rybičiek. Koľko rybičiek bolo pôvodne v~akváriu a~koľko ich dostal Ondrej, koľko Matej a~koľko Petrík?
}
\podpis{Marta Volfová}

{%%%%% Z9-II-2
Zuzka vpísala do deviatich políčok na nasledujúcom obrázku celé čísla od 1 do 9, každé práve raz.
Pomer súčtov čísel napísaných v~kruhoch, trojuholníkoch a~šesťuholníkoch bol $2:3:6$.
Zistite, aké číslo mohlo byť napísané v~hornom trojuholníku; určte všetky možnosti.
\ifobrazkyvedla\else\insp{z9-ii-2.eps}\fi
}
\podpis{Erika Novotná}

{%%%%% Z9-II-3
Dané sú kružnice $k_1$, $k_2$, $k_3$ a~$k_4$ so stredmi postupne $S_1$, $S_2$, $S_3$ a~$S_4$.
Kružnice $k_1$ a~$k_3$ sa zvonka dotýkajú všetkých ostatných kružníc, polomer kružnice~$k_1$ je 5\,cm, vzdialenosť stredov $S_2$ a~$S_4$ je 24\,cm a~štvoruholník $S_1S_2S_3S_4$ je kosoštvorec.
Určte polomery kružníc $k_2$, $k_3$ a~$k_4$.

{\it Poznámka.}
Obrázok je len ilustračný.
\ifobrazkyvedla\else\insp{z9-ii-3.eps}\fi
}
\podpis{Eva Semerádová}

{%%%%% Z9-II-4
Pred každé z~čísel v~nasledujúcich dvoch zoznamoch doplňte buď znamienko plus, alebo mínus tak, aby hodnota takto zapísaných výrazov bola rovná nule:
%\begin{enumerate}\alphatrue
\itemitem{a)} 1 2 3 4 5 6 7 8 9 10,
\itemitem{b)} 1 2 3 4 5 6 7 8 9 10 11.
%\end{enumerate}

Pri oboch úlohách uveďte aspoň jedno riešenie alebo zdôvodnite, že úloha riešenie nemá.
}
\podpis{Marta Volfová}

{%%%%% Z9-III-1
Obdĺžnik má dĺžky strán v~pomere $2:5$.
Keď predĺžime všetky jeho strany o~9\,cm, dostaneme obdĺžnik, ktorého dĺžky strán sú v~pomere $3:7$.
V~akom pomere budú dĺžky strán obdĺžnika, ktorý vznikne predĺžením všetkých strán o~ďalších 9\,cm?
}
\podpis{Michaela Petrová}

{%%%%% Z9-III-2
Traja kamaráti si mysleli tri navzájom rôzne nenulové cifry, z~ktorých jedna bola~3.
Z~týchto cifier vytvorili všetkých šesť možných trojciferných čísel, ktoré potom rozdelili do troch dvojíc.
Rozdielom prvej dvojice čísel bolo jednociferné číslo, rozdielom druhej dvojice čísel bolo dvojciferné číslo a~rozdielom tretej dvojice čísel bolo trojciferné číslo deliteľné piatimi. Zistite, aké tri cifry si mohli kamaráti myslieť. Určte všetky možnosti.}
\podpis{Erika Novotná}

{%%%%% Z9-III-3
Na papieri bolo napísaných niekoľko bezprostredne po sebe idúcich kladných násobkov určitého prirodzeného čísla väčšieho ako jedna.
Rado ukázal na jedno z~napísaných čísel: keď ho vynásobil číslom, ktoré s~ním susedilo naľavo, dostal súčin o~216 menší, ako keď ho vynásobil číslom, ktoré s~ním susedilo napravo. Na ktoré číslo mohol Rado ukázať? Nájdite všetky možnosti.
}
\podpis{Libor Šimůnek}

{%%%%% Z9-III-4
Eva vpísala do daného trojuholníka kružnicu.
Potom dokreslila tri úsečky, ktoré sa dotýkali vpísanej kružnice a~v~pôvodnom trojuholníku vytvárali tri menšie trojuholníky, pozri obrázok.
Obvody týchto troch trojuholníkov boli 12\,cm, 14\,cm a~16\,cm. Určte obvod pôvodného trojuholníka.
\ifobrazkyvedla\else\insp{z9-iii-4.eps}\fi
}
\podpis{Erika Novotná}

