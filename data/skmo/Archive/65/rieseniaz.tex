{%%%%%   Z4-I-1
...}

{%%%%%   Z4-I-2
...}

{%%%%%   Z4-I-3
...}

{%%%%%   Z4-I-4
...}

{%%%%%   Z4-I-5
...}

{%%%%%   Z4-I-6
...}

{%%%%%   Z5-I-1
\napad
Zistite, aký bol súčet čísel v~každom vagóne.

\riesenie
Súčet všetkých čísel vo všetkých vagónoch je
$$
1+2+3+4+5+6+7+8+9=45.
$$
Súčet čísel v~každom vagóne teda bol $45:3=15$.

V~treťom vagóne sa viezli tri nepárne čísla iné ako 1,
z~nich možno získať súčet 15 len ako $3+5+7$.
V~prvom vagóne sa vedľa 1 viezli ešte niektoré dve čísla z~2, 4, 6, 8, 9.
Z~týchto čísel možno získať súčet 15 len ako $1+6+8$.
Do druhého vagóna tak zostávajú čísla 2, 4, 9 (pre kontrolu $2+4+9=15$).

Úloha má jediné riešenie:
v~prvom vagóne sa viezli čísla 1, 6, 8, v~druhom vagóne 2, 4, 9,
v~treťom vagóne 3, 5, 7.

\ineriesenie
Aj bez určenia súčtu čísel v~každom vagóne sa dá uvedené riešenie objaviť skúšaním.
Najmenej možností je v~poslednom vagóne, kde sa viezli niektoré tri čísla
z~3, 5, 7, 9:
\begin{itemize}
\item Trojica 5, 7, 9 má súčet 21 a~rovnaký súčet by musel byť
aj v~prvom vagóne. Z~dvoch zvyšných čísel a~1 však možno získať najviac
$1+6+8=15$, čo nevyhovuje.
\item Trojica 3, 7, 9 má súčet 19; v~prvom vagóne by potom mohol byť
súčet najviac $1+6+8=15$, čo tiež nevyhovuje.
\item Trojica 3, 5, 9 má súčet 17; v~prvom vagóne by potom mohol byť
súčet najviac $1+7+8=16$, čo tiež nevyhovuje.
\item Trojica 3, 5, 7 má súčet 15; v~prvom vagóne by potom mohla byť
trojica 1, 6, 8 so súčtom 15, čo je vyhovujúca možnosť.
\end{itemize}
Do druhého vagóna tak zostávajú čísla 2, 4, 9, ktoré majú tiež súčet~15.
}

{%%%%%   Z5-I-2
\napad
Zvážte postupne všetky možnosti, ktoré kusy ovocia mohla Marta zjesť.

\riesenie
Existuje iba niekoľko málo možností, ktoré kusy ovocia mohla Marta cestou zjesť.
Preberieme všetky možnosti a~porovnáme s~ponúkanými situáciami a)~--~e).
Jednotlivé druhy ovocia označujeme ich začiatočnými písmenami:
$$
\begintable
Marta zjedla \| Majka dostala \crthick
 $2j$ \| $5j+6h+3p$ \cr
$j+h$ \| $6j+5h+3p$ \cr
$j+p$ \| $6j+6h+2p$ \cr
  $2h$\| $7j+4h+3p$ \cr
$h+p$ \| $7j+5h+2p$ \cr
 $2p$ \| $7j+6h+p$ \endtable
$$
Z~toho vidíme, že situácie a), b), c), e) nemohli nastať nikdy a~situácia~d)
iba v~treťom prípade:
Marta by vtedy cestou zjedla jablko a~pomaranč, Majke by doniesla 6~jabĺk,
6~hrušiek a~2~pomaranče.
}

{%%%%%   Z5-I-3
\napad
Najskôr zistite, koľko šnúry môže byť využitej na samotné vešanie utierok.

\riesenie
Všetky rozmery budeme vyjadrovať v~rovnakých jednotkách, a~to v~dm.
Dĺžka napnutej šnúry medzi stromami je rovná $75-2\cdot 8=59$\,(dm).
Z~každej strany má navyše zostať voľný 1\,dm.
Na samotné vešanie teda môže byť použitých $59-2\cdot 1=57$\,(dm).

Každá utierka je široká 4{,}5\,dm, jedna dvojica utierok teda zaberá
aspoň 9\,dm. Šesť dvojíc utierok zaberá aspoň 54\,dm, v~takom
prípade zvýšia nanajvýš 3\,dm šnúry ($57=6\cdot 9+3$).
Do tohto priestoru sa už žiadna ďalšia utierka nevôjde.
Na šnúru možno uvedeným spôsobom zavesiť najviac 12 utierok.
}

{%%%%%   Z5-I-4
\napad
Uvažujte situáciu, keď sa trikrát zväčší ako počet čiernych, tak počet bielych oviec.

\riesenie
Keby sa počiatočné počty čiernych aj bielych oviec zväčšili trikrát, bolo by
teraz bielych oviec o~24 viac ako čiernych (lebo $3\cdot8=24$). Počet
bielych oviec sa však zväčšil nie trikrát, ale štyrikrát a~bielych oviec
je teraz o~42 viac ako čiernych. Rozdiel $42-24=18$ zodpovedá rozdielu
štvornásobku a~trojnásobku pôvodného počtu bielych oviec, čo je práve onen
pôvodný počet. Na začiatku teda bolo 18~bielych a~$18-8=10$ čiernych oviec.

V~súčasnosti pán Baran chová $4\cdot18=72$ bielych oviec a~$3\cdot10=30$
čiernych oviec, čo je spolu $72+30=102$ oviec.

\poznamka
Predošlé úvahy je možné graficky znázorniť takto:
\figure{z5-I-4}%

}

{%%%%%   Z5-I-5
\napad
Načrtnite si nejaký útvar s~obsahom $6\cm^2$ a~upravujte ho tak,
aby boli splnené ostatné podmienky.

\riesenie
Jednoduchým útvarom s~obsahom $6\cm^2$ je napr. obdĺžnik so stranami
dĺžok 2\,cm a~3\,cm. Ten má však obvod iba 10\,cm; potrebujeme
presunúť časť jeho plochy tak, aby sa obvod o~2\,cm zväčšil. To si možno
v~rámci zadanej štvorcovej siete predstaviť tak, že skúšame presúvať
jednotlivé štvorce obsiahnuté v~obdĺžniku na iné miesta. Niekoľko možných
riešení je na obrázku:
\figure{z5-I-5}%


\poznamka
Dôsledný riešiteľ sa môže zamyslieť nad ďalšími, príp. všetkými možnými riešeniami.
Na to si stačí povšimnúť, že pri presúvaní štvorcov mysleného
obdĺžnika sa obvod zväčší buď o~2\,cm, alebo o~4\,cm, a~to podľa toho,
či je tento štvorec rohový, alebo nie.
}

{%%%%%   Z5-I-6
\napad
Zistite, koľko je v~roku plných týždňov a~koľko je dní navyše.

\riesenie
Nepriestupný rok má 365 dní, \tj. 52~plných týždňov a~jeden deň navyše
($365=52\cdot 7+1$).
V~52~plných týždňoch je 52~nedieľ, preto musí byť onen deň navyše nedeľa.
Taký rok preto začínal aj končil nedeľou.
Štedrý deň je presne týždeň pred posledným dňom v~roku, preto bol
aj Štedrý deň v~nedeľu.
}

{%%%%%   Z6-I-1
\napad
Začnite vyfarbovať a~zvažujte, kedy je následný postup jednoznačný
a~kedy existuje viac možností.

\riesenie
Trojuholníkové políčko susediace s~bielym obdĺžnikom môže byť buď červené,
alebo modré:

Ak je červené, potom pravouhlé lichobežníky musia
byť modré (susedia s~bielym obdĺžnikom a~červeným trojuholníkom)
a~posledné lichobežníkové políčko musí byť červené (susedí s~bielym
obdĺžnikom a~modrými lichobežníkmi). Zvyšné trojuholníkové políčko potom môže
byť buď biele, alebo modré (susedí s~červeným lichobežníkom).

Ak je trojuholníkové políčko susediace s~bielym obdĺžnikom modré,
tak príslušná diskusia je veľmi podobná predošlej, akurát sú
vymenené farby červená a~modrá.

Celkom teda dostávame $2+2=4$ možnosti, ktoré musia archeológovia zvažovať.
\figure{z6-I-1a}%

}

{%%%%%   Z6-I-2
\napad
Uvedomte si, koľko má ktorý z~tvorov nôh za predpokladu, že žiadnemu z~nich
žiadna noha nechýba.

\riesenie
Musíme zistiť, aké počty múch a~pavúkov dajú dokopy 100~nôh.
V~nasledujúcej tabuľke postupne uvažujeme rôzne počty pavúkov ($p$),
určíme, koľko majú celkom nôh ($P=8p$) a~koľko nôh ostáva na muchy
($M=100-P$); ak je tento počet deliteľný šiestimi, dostávame možné
riešenie ($m=M:6$). Keďže všetky čísla musia byť kladné, stačí
vyskúšať len niekoľko možností:
\bgroup
\def\ctr#1{\hfil\ \ #1\ \ \hfil}
$$
\begintable
$p$\|1|2|3|4|5|6|7|8|9|10|11|12\crthick
$P$\|8|16|24|32|40|48|56|64|72|80|88|96\cr
$M$\|92|84|76|68|60|52|44|36|28|20|12|4\crthick
$m$\|--|14|--|--|10|--|--|6|--|--|2|--\endtable
$$
\egroup

Odtiaľ vidíme štyri možnosti:
v~prvej pivnici mohlo byť 14~múch a~2~pavúky, alebo 10~múch a~5~pavúkov;
v~druhej pivnici mohlo byť 8~pavúkov a~6~múch, alebo 11~pavúkov a~2~muchy.

\poznamka
Vzhľadom na to, že všetky počty nôh sú párne, je možné si trochu ušetriť počítanie v~tabuľke tým, že tieto počty delíme dvoma.
To je to isté, akoby sme počítali iba ľavé (alebo iba pravé) nohy jednotlivých tvorov.
Inými slovami, namiesto hľadania $p$ a~$m$ takých, aby platilo $8p+6m=100$, riešime $4p+3m=50$.
}

{%%%%%   Z6-I-3
\napad
Dokážete určiť dĺžku niektorej úsečky bez toho, aby ste použili viac ako jeden zadaný rozmer?

\riesenie
Zistíme rozmery štvorca $EFGD$ a~obdĺžnika $HIJD$, aby sme stanovili ich obsahy.
Rozdiel týchto obsahov predstavuje želaný obsah šesťuholníka $EFGJIH$.

Zadaný obvod šesťuholníka $EFGJIH$ je rovný obvodu štvorca $EFGD$,
pretože $|JI|=|DH|$ a~$|HI|=|DJ|$.
Strana~$GD$ má teda veľkosť $60:4=15$\,(cm).
Podobne zadaný obvod šesťuholníka $ABCGFE$ je rovný obvodu štvorca $ABCD$,
veľkosť strany~$CD$ je teda $96:4=24$\,(cm).
Rozdiel dĺžok strán týchto dvoch štvorcov je rovný dĺžke úsečky~$GC$, ktorá je
podľa zadania rovná dĺžke úsečky~$DJ$:
$$
|DJ|=|GC| =24-15=9\,(\Cm).
$$
Pomocou známeho obvodu obdĺžnika $HIJD$ a~dĺžky strany~$DJ$ stanovíme
aj druhý rozmer tohto obdĺžnika:
$$
|JI|= (28-2\cdot 9):2=5\,(\Cm).
$$
Už máme všetky údaje potrebné na stanovenie obsahov štvorca $EFGD$ a~obdĺžnika $HIJD$:
$$
S_{EFGD}=15\cdot 15=225\,(\Cm^2),
\quad
S_{HIJD}=9\cdot 5=45\,(\Cm^2).
$$
Hľadaný obsah šesťuholníka teda je
$$
S_{EFGJIH}=225-45=180\,(\Cm^2).
$$
}

{%%%%%   Z6-I-4
\napad
Zistite, ktoré rôzne súčty možno získať.

\riesenie
Všetky možné dvojice, ktoré možno z~daných čísel zložiť, sú
$$
(1,1);\ (1,2), (2,1);\ (1,3), (2,2), (3,1);\ (2,3), (3,2);\ (3,3).
$$
Tieto možnosti dávajú 5~rôznych súčtov, a~to 2, 3, 4, 5, 6 (dvojice s~rôznymi
súčtami sú oddelené bodkočiarkami).
Na uvedenom obrázku však potrebujeme 6~dvojíc s~rôznymi súčtami, pravdu má
teda Zuzka.

\poznamky
Na určenie možných súčtov nie je nutné vypisovať všetky prípustné dvojice:
najmenší súčet je $1+1=2$, najväčší je $3+3=6$.
Z~toho vyplýva, že možných súčtov nie je viac ako 5, čo je menej ako
požadovaných~6.

\smallskip
Riešenie úlohy pomocou všetkých možných vyplnení tabuľky a~kontrolou takto
získaných súčtov je extrémne prácne.
Ak však také riešenie je úplné, musí sa považovať za správne.
}

{%%%%%   Z6-I-5
\napad
Uvedomte si, že štvorce nemusia mať rovnaké rozmery.


\riesenie
Obvod $28=2\cdot 14$ metrov možno pomocou kladných celých čísel vyjadriť iba
niekoľkými spôsobmi.
Postupne všetky preberieme a~zistíme, či možno zodpovedajúci záhon rozdeliť
na štyri štvorce s~celočíselnými rozmermi:
\begin{itemize}
  \item $28=2\cdot(13+1)$, v~takom prípade potrebujeme 13~štvorcov:
    \figure{z6-I-5a}%
  \item $28=2\cdot(12+2)$, v~takom prípade potrebujeme aspoň 6~štvorcov:
    \figure{z6-I-5b}%
  \item $28=2\cdot(11+3)$, v~takom prípade potrebujeme aspoň 6~štvorcov:
    \figure{z6-I-5c}%
  \item $28=2\cdot(10+4)$, v~takom prípade stačia 4~štvorce:
    \figure{z6-I-5d}%
  \item $28=2\cdot(9+5)$, v~takom prípade potrebujeme aspoň 6~štvorcov:
    \figure{z6-I-5e}%
  \item $28=2\cdot(8+6)$, v~takom prípade stačia 4~štvorce:
    \figure{z6-I-5f}%
  \item $28=2\cdot(7+7)$, v~takom prípade by bol záhon štvorcový a~nie obdĺžnikový.
\end{itemize}

Záhrada mohla mať rozmery $10\times 4$ alebo $8\times 6$ metrov.

\ineriesenie
Uvažujme, ako možno zložiť jeden obdĺžnik zo štyroch štvorcov (vo všeobecnosti s~rôznymi
celočíselnými rozmermi).
To možno urobiť len nasledujúcimi spôsobmi:
\figure{z6-I-5g}%


Ak veľkosť strany najmenšieho štvorca v~metroch označíme~$a$, tak obvod
obdĺžnika v~jednotlivých prípadoch je:
\begin{itemize}
  \item $2\cdot(4a+a)=10a$, čo nie je rovné $28$ pre žiadne celé $a$.
  \item $2\cdot(5a+2a)=14a$, čo je rovné $28$ práve vtedy, keď $a=2$; obdĺžnik má v~takom prípade rozmery $10\times 4$ metrov.
  \item $2\cdot(5a+3a)=16a$, čo nie je rovné $28$ pre žiadne celé $a$.
  \item $2\cdot(4a+3a)=14a$, čo je rovné $28$ práve vtedy, keď $a=2$; obdĺžnik má v~takom prípade rozmery $8\times 6$ metrov.
\end{itemize}
}

{%%%%%   Z6-I-6
\napad
Zistite, ako sa líšili počty hrncov, resp. kotlov polievky uvarených v~pondelok a~utorok.

\riesenie
V~pondelok uvarili o~10~hrncov polievky viac ako v~utorok, zatiaľ čo v~utorok uvarili
o~3~kotly viac ako v~pondelok.
Keďže v~tieto dni uvarili rovnaké množstvo polievky, má 10~hrncov rovnaký objem
ako 3~kotly.

V~stredu uvarili $20=2\cdot 10$ hrncov polievky, čo zodpovedá
$2\cdot 3=6$ kotlom.
Vo štvrtok potom uvarili $30=5\cdot 6$ kotlov polievky, čo je päťkrát viac
ako v~stredu.
}

{%%%%%   Z7-I-1
\napad
Určte, koľko syra dostane každé z~mláďat.

\riesenie
Pôvodná kocka bola zložená z~27 kocôčok.
Hryzka vyhrýzla po jednej kocôčke z~každej steny a~jednu prostrednú,
prehryzená kocka teda obsahovala $27-6-1=20$ kocôčok.
Každé zo štyroch mláďat teda dostane $20:4=5$ kocôčok syra.

Teraz je potrebné predstavovať si rôzne útvary zložené z~5~kocôčok tak, aby zo štyroch takých útvarov bolo možné zložiť prehryzenú kocku.
Tu sú všetky riešenia:
\figure{z7-I-1}%

}

{%%%%%   Z7-I-2
\napad
Zamerajte sa na súčet vekov súrodencov pred tromi rokmi.

\riesenie
Označme aktuálny vek detí v~rokoch začiatočnými písmenami ich mien (Jakubov vek označíme~$k$)
a~takto postupne vyjadríme všetky vzťahy zo zadania:
$$
o=m+3,\quad
k=j+5,\quad
o+m+k+j =30,
$$
odkiaľ po dosadení dostávame
$$
\align
(m+3)+m+(j+5)+j =2m+2j+8 &=30, \\
2m+2j &=22, \\
m+j &=11.
\endalign
$$

Zároveň má platiť, že pred tromi rokmi mali deti dokopy 19~rokov.
Avšak rozdiel $30-19=11$ nie je násobkom~3, takže tušíme nejaký problém.
Vzhľadom na to, že $11=3\cdot 3+2$, znamená to, že najmladšia Jana nebola
pred tromi rokmi ešte na svete a~teraz má 2~roky.
Z~predchádzajúcich vzťahov postupne odvodzujeme
$$
m=11-j=9,\quad
k=j+5=7,\quad
o=m+3=12.
$$
Vek detí Vlčkovcov je teda nasledujúci:
Ondrej má 12 rokov, Matej 9, Jakub 7 a~Jana 2~roky.

\poznamka
Ak rovno neodhalíme, že $j=2$, môžeme akúkoľvek inú možnosť vylúčiť
podobným dosadením ako vyššie a~porovnaním s~požadovaným súčtom pred tromi rokmi.
Vzhľadom na to, že $m+j=11$ a~že Jana je najmladšia, stačí vyskúšať
nasledujúce možnosti: $j=1,2,3,4$ a~5.
}

{%%%%%   Z7-I-3
\napad
Uvedomte si, že trojuholník $BCP$ nie je všeobecný.

\riesenie
Päťuholník $ABCDE$ je pravidelný, takže platí $|AB|=|BC|$.
Trojuholník $ABP$ je rovnostranný, preto $|AB|=|BP|$.
Odtiaľ vidíme, že $|BP|=|BC|$, teda že trojuholník $BCP$ je rovnoramenný.
Jeho vnútorné uhly pri vrcholoch $P$ a~$C$ sú preto zhodné; na ich určenie
stačí poznať uhol pri vrchole~$B$
(súčet veľkostí vnútorných uhlov v~ľubovoľnom trojuholníku je $180\st$).
Pritom uhol $PBC$ je rozdielom uhlov $ABC$ a~$ABP$, z~ktorých prvý je
vnútorným uhlom pravidelného päťuholníka (vyjadríme ho za chvíľu) a~druhý je
vnútorným uhlom rovnostranného trojuholníka (má veľkosť $\alpha=60\st$).
\figure{z7-I-3}%


Päťuholník $ABCDE$ môžeme rozdeliť na päť trojuholníkov so spoločným
vrcholom~$P$. Súčet vnútorných uhlov päťuholníka je rovný súčtu
vnútorných uhlov všetkých piatich trojuholníkov bez uhlov pri vrchole~$P$, \tj.
$
5\cdot 180\st -360\st =540\st.
$
V~pravidelnom päťuholníku sú všetky vnútorné uhly zhodné, každý má
teda veľkosť
$
540\st:5 =108\st.
$

Odtiaľ konečne vieme vyjadriť
$$
\beta=|\uhol PBC| =|\uhol ABC|-|\uhol ABP| =108\st-60\st =48\st
$$
a~následne
$$
\gamma=|\uhol BCP|=|\uhol BPC| =\frac{180\st-48\st}2 =\frac{132\st}2 =66\st.
$$
Veľkosť uhla $BCP$ je $66\st$.

\poznamka
Veľkosť vnútorného uhla pravidelného päťuholníka je možné odvodiť aj
pomocou rozdelenia na päť zhodných rovnoramenných trojuholníkov ako na
nasledujúcom obrázku ($S$ je stred päťuholníka, \tj. stred jeho opísanej
kružnice).
\figure{z7-I-3a}%


Uhol pri vrchole~$S$ v~každom z~týchto trojuholníkov má veľkosť
$360:5=72\st$; súčet uhlov pri základni je rovný $180\st-72\st=108\st$,
čo je tiež veľkosť vnútorného uhla pravidelného päťuholníka.
}

{%%%%%   Z7-I-4
\napad
Najskôr rozdeľte robotov do skupín, v~rámci ktorých sa môžu navzájom rozprávať.

\riesenie
Najskôr vyjadríme všetky skupiny robotov, ktorí sa môžu medzi sebou rozprávať
(v~nasledujúcich zoznamoch sú tieto skupiny vyznačené zátvorkami).
Roboti s~nepárnymi číslami sú rozdelení podľa počtu cifier, to sú dve skupiny:
$$
(1,\ 3,\ 5,\ 7,\ 9),\
(11,\ 13,\ 15,\ 17,\ 19).
$$
Roboti s~párnymi číslami sú rozdelení podľa začiatočnej cifry:
$$
(2,\ 20),\ (4),\ (6),\ (8),\ (10,\ 12,\ 14,\ 16,\ 18).
$$

Stačí teda spočítať počty dvojíc, ktoré možno v~rámci každej skupiny vytvoriť.
Máme tri skupiny s~jediným robotom -- v~nich nevytvoríme žiadnu dvojicu;
jednu skupinu s~dvoma robotmi -- v~tej máme jedinú dvojicu;
tri skupiny po piatich robotoch -- v~každej takej skupine možno vytvoriť 10 dvojíc.
Celkom dostávame $1+3\cdot 10=31$ dvojíc robotov, ktorí sa spolu môžu rozprávať.
}

{%%%%%   Z7-I-5
\napad
Vypíšte si vzdialenosti medzi rôznymi dvojicami čísel na kocúrkovskej osi.

\riesenie
Vzdialenosť 39\,cm môže byť realizovaná medzi rôznymi dvojicami čísel.
Budeme systematicky vypisovať vzdialenosti medzi niekoľkými prvými číslami
kocúrkovskej osi.
V~nasledujúcej schéme je nad čiarou vypísaných prvých 10~čísel a~pod čiarou
skutočné vzdialenosti (v~cm) medzi rôznymi dvojicami týchto čísel~--
v~prvom riadku pod čiarou sú postupne vzdialenosti medzi susednými číslami,
v~druhom riadku pod čiarou sú vzdialenosti medzi dvojicami čísel, ktoré majú medzi sebou práve jedno číslo, atď.
(Napr. 21 v~treťom riadku pod čiarou označuje skutočnú vzdialenosť medzi číslami 3 a~6
na kocúrkovskej osi a~je určené ako $5+7+9$).
Hviezdičkou sú označené príliš veľké čísla, ktoré nás nezaujímajú.
$$
\vbox{
\def\strut{\vrule width 0pt height1.1em depth.55em\relax}
\halign{#&&\strut\hfil\ #\ \hfil\cr
1&&2&&3&&4&&5&&6&&7&&8&&9&&10\cr
\noalign{\hrule}
&1&&3&&5&&7&&9&&11&&13&&15&&17&\cr
&&4&&8&&12&&16&&20&&24&&28&&32&&36\cr
&&&9&&15&&21&&27&&33&&\bf 39&&$*$&&$*$&\cr
&&&&16&&24&&32&&40&&$*$&&$*$&&$*$&&$*$\cr
&&&&&25&&35&&45&&$*$&&$*$&&$*$&&$*$&\cr
&&&&&&36&&48&&$*$&&$*$&&$*$&&$*$&&$*$\cr
&&&&&&&49&&$*$&&$*$&&$*$&&$*$&&$*$&\cr
}}
$$
Ihneď vidíme (z~tretieho riadku pod čiarou), že vzdialenosť 39\,cm je medzi číslami 6
a~9  a~že sa určite neobjavuje medzi číslami, ktoré majú medzi sebou viac ako dve čísla (\tj. od štvrtého riadku pod čiarou).
Vzdialenosť 39\,cm sa určite tiež nemôže objavovať medzi číslami, ktoré majú medzi sebou práve jedno číslo,
pretože všetky tieto vzdialenosti sú párne (druhý riadok pod čiarou).
Ostáva teda preskúmať vzdialenosti medzi susednými číslami (prvý riadok pod čiarou):

Postupnosť vzdialeností medzi susednými číslami môžeme vyjadriť ako
$$
1,\quad 3=1+2,\quad 5=1+2\cdot2,\quad 7=1+2\cdot3,\quad 9=1+2\cdot4,\quad \dots
$$
Všeobecne, vzdialenosť medzi číslami $i$ a~$i+1$ na kocúrkovskej osi je
rovná
$$
1+2(i-1)=2i-1\,(\Cm).
$$
Táto vzdialenosť teda bude rovná 39\,cm práve vtedy, keď $i=20$.

Vzdialenosť 39\,cm na kocúrkovskej číselnej osi je medzi dvojicami čísel 6, 9
a~20, 21.

\poznamky
Záverečnú úvahu možno nahradiť vypísaním a~spočítaním všetkých nepárnych čísel
až po~39.
Ak je výpis úplný, je také riešenie správne.

\smallskip
Naopak, úvodné vypisovanie sa dá celé nahradiť úvahou, príp. výpočtom:
Všetky vzdialenosti v~tabuľke sú súčtom rôznych počtov nepárnych čísel,
pričom tieto počty sú buď nepárne (pre susediace čísla a~dvojice čísel, medzi
ktorými je párny počet čísel), alebo párne (pre dvojice čísel, medzi ktorými
je nepárny počet čísel). V~jednotlivých riadkoch sa teda objavujú buď
iba nepárne, alebo iba párne čísla. Vzdialenosť 39\,cm sa teda môže
objavovať iba medzi susednými číslami a~dvojicami, medzi ktorými je na
kocúrkovskej osi párny počet čísel.

Predošlé vypisovanie postupnosti vzdialeností medzi susednými číslami má
nasledujúcu analógiu pre dvojice čísel, medzi ktorými sú dve čísla:
$$
9,\quad 15=9+6,\quad 21=9+6\cdot2,\quad 27=9+6\cdot3,\quad \dots
$$
Všeobecne, vzdialenosť medzi číslami $i$ a~$i+3$ na kocúrkovskej osi
je rovná
$$
9+6(i-1)=6i+3\,(\Cm).
$$
Táto vzdialenosť teda bude rovná 39\,cm práve vtedy, keď $i=6$.
Podobne možno vyjadriť akúkoľvek inú vyššie vypisovanú postupnosť.

\smallskip
Riešenie úlohy možno zjednodušiť pomocou nasledujúceho poznatku:
Súčet nepárneho počtu po sebe idúcich nepárnych čísel je rovný súčinu počtu
týchto čísel a~prostredného z~nich.
Usilovným riešiteľom odporúčame tento poznatok zdôvodniť a~riešenie domyslieť.

\smallskip
V~uvedenej schéme si môžeme všimnúť, že všetky čísla v~prvom šikmom stĺpci
sú druhými mocninami prirodzených čísel.
To nie je náhoda~-- všeobecne platí, že súčet prvých $k$ po sebe idúcich nepárnych
čísel je rovný~$k^2$.
Odporúčame porovnať toto tvrdenie s~poznatkom v~predchádzajúcej poznámke.
}

{%%%%%   Z7-I-6
\napad
Môžu oproti sebe, príp. vedľa seba sedieť mačka s~párnym a~mačka s~nepárnym číslom?

\riesenie
Postupne rozoberieme dôsledky jednotlivých poznatkov zo zadania:
\begin{enumerate}\alphatrue
\item
Čísla mačiek sediacich oproti sebe tvoria 5 párov s~rovnakým súčtom.
Súčet čísel všetkých mačiek je $1+2+\cdots+10=55$, takže každý pár musí mať
súčet $55:5=11$; jediné možnosti sú
$1+10$, $2+9$, $3+8$, $4+7$, $5+6$.
\item
Párne číslo sa nedá získať súčtom párneho a~nepárneho čísla.
V~jednom rade preto môžu sedieť iba mačky s~nepárnymi číslami, v~druhom iba
mačky s~párnymi číslami.
\item
Násobok čísla~8 nemožno získať súčinom nepárnych čísel.
Z~toho a~z~predchádzajúceho dôsledku vyplýva, že v~dolnom rade sedeli iba mačky
s~párnymi číslami, \tj. 2, 4, 6, 8, 10.
Súčinom dvoch takých čísel možno získať násobok 8 práve vtedy, keď jeden
z~činiteľov je 4 alebo 8.
Preto nemôžu byť mačky s~číslami 4 a~8 na krajoch, ani uprostred.
\item
Podľa dôsledku~a) vieme, že oproti mačke s~číslom~1 sedela mačka s~číslom~10.
Z~toho vyplýva, že ani mačka s~číslom~10 nemôže byť na kraji a~je viac vpravo
ako mačka s~číslom~6.
\item
Z~doterajších informácií vieme, že v~pravom dolnom rohu sedela mačka
s~párnym číslom rôznym od 4, 8, 10 a~6.
\end{enumerate}
Vyhrala teda mačka s~číslom~2.

\poznamka
Z~uvedeného takmer vieme určiť rozmiestnenie všetkých mačiek v~miestnosti:
poradie mačiek v~spodnom rade mohlo byť
$$
\text{buď}\quad 6,\ 4,\ 10,\ 8,\ 2,\quad
\text{alebo}\quad 6,\ 8,\ 10,\ 4,\ 2,
$$
poradie mačiek v~hornom rade je potom jednoznačne určené podľa dôsledku~a).
}

{%%%%%   Z8-I-1
\napad
Ktoré klávesy mohla Klára zameniť a~ktoré nie?

\riesenie
Rozsypané, tzn. biele klávesy sú trojakého typu:
\begin{enumerate}
  \item klávesy C a~F, ktoré majú čiernu klávesu sprava,
  \item klávesy E a~H, ktoré majú čiernu klávesu zľava,
  \item klávesy D, G a~A, ktoré majú čierne klávesy z~oboch strán.
\end{enumerate}
Je zrejmé, že Klára mohla popliesť vždy len klávesy rovnakého typu.
Klávesy prvého typu mohla poskladať dvojakým spôsobom:
$$
\text{C}\ *\ *\ \text{F}\ *\ *\ *, \quad
\text{F}\ *\ *\ \text{C}\ *\ *\ *.
$$
Klávesy druhého typu mohla poskladať tiež dvojakým spôsobom:
$$
*\ *\ \text{E}\ *\ *\ *\ \text{H}, \quad
*\ *\ \text{H}\ *\ *\ *\ \text{E}.
$$
Klávesy tretieho typu mohla poskladať šiestimi spôsobmi:
$$
\gathered
*\ \text{D}\ *\ *\ \text{G}\ \text{A}\ *, \quad
*\ \text{D}\ *\ *\ \text{A}\ \text{G}\ *, \\
*\ \text{G}\ *\ *\ \text{A}\ \text{D}\ *, \quad
*\ \text{G}\ *\ *\ \text{D}\ \text{A}\ *, \\
*\ \text{A}\ *\ *\ \text{D}\ \text{G}\ *, \quad
*\ \text{A}\ *\ *\ \text{G}\ \text{D}\ {*}.
\endgathered
$$
Uvedené tri skupiny možných skladaní sú od seba úplne nezávislé (možno ich ľubovoľne kombinovať).
Preto je celkový počet možností, ako mohla Klára klávesy poskladať, rovný
$2\cdot2\cdot6=24$.
}

{%%%%%   Z8-I-2
\napad
Aké sú pomery počtov jednotlivých druhov zvierat?

\riesenie
Pomer medzi počtom kráv a~koní je
$$
60:45=4:3
$$
a~pomer medzi počtom oviec a~koní je
$$
60:35=12:7.
$$
Počet koní teda musí byť nejakým násobkom čísla~3 a~súčasne čísla~7, teda
násobkom čísla~21.

Keby na lúke bolo 21~koní, potom by tam bolo $21\cdot4:3=28$ kráv
a~$21\cdot12:7=36$ oviec, celkom teda $21+28+36=85$ zvierat.
Keby na lúke bolo 42~koní, potom by všetky počty boli dvojnásobné, celkom by
teda bolo $2\cdot85=170$ zvierat.
Keby na lúke bolo 63~koní, potom by všetky počty boli trojnásobné, celkom
by teda bolo $3\cdot85=255$ zvierat, čo je však viac ako 200.

Na lúke sa teda páslo buď 85, alebo 170 zvierat.

\poznamka
K~rovnakému výsledku možno dôjsť tiež rozkladom daných násobkov na súčiny prvočísel:
$$
45=3\cdot3\cdot5,\quad
60=2\cdot2\cdot3\cdot5,\quad
35=5\cdot7.
$$
Aby sa zodpovedajúce násobky počtov jednotlivých zvierat rovnali, musia byť
v~ich prvočíselných rozkladoch zastúpené všetky predchádzajúce prvočísla
(vrátane ich násobností).
Najmenší možný počet kráv teda je $2\cdot2\cdot7=28$, koní $3\cdot7=21$
a~oviec $2\cdot2\cdot3\cdot3=36$, celkom $28+21+36=85$ zvierat.
}

{%%%%%   Z8-I-3
\napad
Zamerajte sa najskôr na vnútorné uhly lichobežníka $ABCD$.

\riesenie
Z~predpokladov vyplýva, že spojnica stredu úsečky~$AB$ s~vrcholmi $C$ a~$D$ rozdeľuje lichobežník $ABCD$ na tri zhodné rovnostranné trojuholníky.
Preto veľkosti vnútorných uhlov v~lichobežníku pri vrcholoch $A$ a~$B$ sú rovné $60\st$ a~pri vrcholoch $C$ a~$D$ sú $120\st$.

Zo zadania ďalej vyplýva, že trojuholníky $LCK$ a~$MDL$ sú zhodné
(podľa vety {\it sus\/}).
Preto aj úsečky $KL$ a~$LM$ a~vyznačené dvojice uhlov sú zhodné;
veľkosti týchto uhlov označíme $\alpha$ a~$\beta$.
Trojuholník $KLM$ je rovnoramenný a~uhly pri základni sú tiež zhodné;
ich veľkosť označíme~$\delta$ a~veľkosť uhla~$KLM$ označíme~$\gamma$.
\figure{z8-I-3}%


Zo súčtu vnútorných uhlov v~trojuholníku $KCL$ odvodíme
$$
\alpha+\beta=180\st-120\st=60\st.
$$
Súčet troch vyznačených uhlov s~vrcholom~$L$ je priamy uhol, takže
$$
\gamma=180\st-(\alpha+\beta)=120\st.
$$
Napokon zo súčtu vnútorných uhlov v~trojuholníku $KLM$ odvodíme
$$
\delta=\frac{180\st-120\st}2=30\st.
$$
Veľkosti vnútorných uhlov trojuholníka $KLM$ sú $30\st$ a~$120\st$.}

{%%%%%   Z8-I-4
\napad
Koľko je v~komore nebielych, nesivých, nečiernych, resp. nemodrých ponožiek?

\riesenie
Ak vytiahneme najskôr všetky nebiele ponožky a~až potom dve biele,
vytiahneme práve 28~ponožiek.
Nebielych ponožiek je teda~26.
Rovnakou úvahou dospejeme k~tomu, že nesivých ponožiek je tiež~26, nečiernych je~24
a~nemodrých je~32.

Nebiele ponožky zahŕňajú ponožky ostatných troch farieb; naopak biele ponožky sú zahrnuté medzi nesivými, nečiernymi a~nemodrými.
Podobne je to s~ostatnými prípadmi.
Súčet všetkých nebielych, nesivých, nečiernych a~nemodrých ponožiek je preto rovný trojnásobku počtu všetkých ponožiek v~komore.
Tento súčet je $26+26+24+32=108$, v~komore je teda $108:3=36$ ponožiek.

\poznamky
Z~výsledného súčtu a~z~predchádzajúcich pozorovaní možno ľahko odvodiť počty ponožiek
jednotlivých farieb
(napr. bielych ponožiek je $36-26=10$).

\smallskip
Ak počty ponožiek jednotlivých farieb označíme počiatočnými písmenami oných farieb,
tak predchádzajúce myšlienky môžeme zapísať takto:
$$
\aligned
s+\check{c}+m &=26, \\
b+\check{c}+m &=26, \\
b+s+m &=24, \\
b+s+\check{c} &=32,
\endaligned
$$
odkiaľ sčítaním dostávame
$$
3(b+s+\check{c}+m) =108,
$$
z~čoho po delení tromi vyplýva
$$
b+s+\check{c}+m=36.
$$}

{%%%%%   Z8-I-5
\napad
Vyjadrite definíciu šťastného dňa pomocou rovnice.

\riesenie
Číslo dňa je nanajvýš dvojciferné číslo v~rozsahu od 1 po 31, ktoré označíme
$10a+b$;
cifra~$a$ môže byť 0, 1, 2, alebo 3.
Číslo mesiaca je nanajvýš dvojciferné číslo v~rozsahu od 1 po 12, ktoré označíme
$10c+d$;
cifra~$c$ môže byť buď 0, alebo 1.
Pri tomto označení je šťastný deň taký, že platí
$$
\aligned
10a+b &=a+b+c+d+2+0+1+6, \\
9(a-1) &=c+d.
\endaligned
$$
Z~toho vyplýva, že $b$ môže byť ľubovoľné, $a$ musí byť buď 2, alebo 3 (a~súčet
$c+d$ je deliteľný 9).
\begin{itemize}
\item Ak $a=2$, tak $c+d=9$, čo znamená, že $c=0$ a~$d=9$.
\item Ak $a=3$, tak $c+d=18$, čo vzhľadom na vyššie formulované obmedzenia
nemá vyhovujúce riešenie.
\end{itemize}
Všetky šťastné dni sú teda v~mesiaci september, a~to od 20. do 29., celkom 10~dní.
}

{%%%%%   Z8-I-6
\napad
Určte, akú časť trojuholníka $ABC$ zaberá trojuholník $AXY$.

\riesenie
Zo zadania vyplýva, že úsečka~$XY$ je strednou priečkou trojuholníka $ABC$, ktorá
je rovnobežná so stranou~$BC$.
Jej dĺžka je teda polovičná vzhľadom na dĺžku strany~$BC$ a~veľkosť
výšky z~bodu $A$ na $XY$ je tiež polovičná vzhľadom na veľkosť výšky
z~toho istého bodu na $BC$. To znamená, že trojuholník $AXY$ má štvrtinový
obsah vzhľadom na obsah trojuholníka $ABC$.
\figure{z8-I-6}%


Teraz zvoľme bod~$Z$ na strane~$BC$.
Keďže úsečky $BC$ a~$XY$ sú rovnobežné, je obsah trojuholníka $XYZ$,
a~teda aj štvoruholníka $AXZY$, rovnaký pre akokoľvek zvolený bod~$Z$.
Keďže vzdialenosť rovnobežiek $BC$ a~$XY$ je rovnaká ako vzdialenosť $XY$
od vrcholu~$A$, majú trojuholníky $AXY$ a~$XYZ$ tú istú veľkosť výšky
na ich spoločnú stranu~$XY$, a~preto majú taký istý obsah.
Každý z~týchto dvoch trojuholníkov zaberá štvrtinu trojuholníka $ABC$,
štvoruholník $AXZY$ preto zaberá polovicu trojuholníka $ABC$.
}

{%%%%%   Z9-I-1
\napad
Odvoďte najskôr vzťah medzi výškou vysielača a~priemernou hĺbkou bazéna.

\riesenie
Všetky dĺžky budeme vyjadrovať v~metroch, a~to pri tomto označení:
$h=$~priemerná hĺbka bazéna, $d=$~dĺžka bazéna, $\check{s}=$~šírka bazéna,
$v=$~výška vysielača.
Objem bazéna preto vyjadríme v~metroch kubických:
$6998{,}4\,\text{hl}=699{,}84\,\text{m}^3$.
Informácie v~zadaní sa dajú zapísať nasledujúcimi rovnicami:
$$
\eqalignno{
699{,}84 =h\cdot\check{s}\cdot d &=v\cdot h^2, & (1) \cr
v=8d &=15\check{s}. & (2) }
$$
Z~rovností (2) môžeme vyjadriť $d=\frc{v}8$ a~$\check{s}=\frc{v}{15}$.
Po dosadení do jednej z~rovností v~(1), vydelení $h$, resp. $v$ (ktoré sú
určite nenulové) a~úprave dostávame:
$$
\aligned
h\cdot\frac{v}{15}\cdot\frac{v}8 &=v\cdot h^2, \\
\frac{v}{120} &=h, \\
v&=120h.
\endaligned
$$
Tento vzťah dosadíme do inej rovnosti v~(1) a~zistíme hodnotu neznámej~$h$:
$$
\aligned
699{,}84 &=120h\cdot h^2, \\
5{,}832 &=h^3, \\
h &=1,8.
\endaligned
$$
Vysielač je vysoký $v=120\cdot 1{,}8=216$ metrov.
}

{%%%%%   Z9-I-2
\napad
Koľko najviac deliteľov môže mať číslo, ktoré je súčinom troch nie nutne rôznych
prvočísel?

\riesenie
Keďže úžasné číslo je párne, aspoň jeden z~jeho prvočíselných deliteľov je~2;
zvyšné dva prvočíselné delitele označme $b$ a~$c$.
Úžasné číslo je teda rovné súčinu~$2bc$.
Všetky delitele takéhoto čísla sú $1$, $2$, $b$, $c$, $2b$, $2c$, $bc$, $2bc$,
pričom niektoré z~týchto čísel sa môžu rovnať.
Postupne rozoberieme všetky možnosti podľa počtu a~typu rôznych prvočíselných
deliteľov.

\smallskip a)
Predpokladajme, že všetky prvočíselné delitele sú rovnaké, teda $b=c=2$.
V~takom prípade by úžasné číslo bolo~8 a~všetky jeho delitele by boli
1, 2, 4, 8.
Súčet všetkých deliteľov by bol 15, čo nie je dvojnásobok čísla~8.
Prípad $b=c=2$ teda nie je možný.

\smallskip b)
Predpokladajme, že dva prvočíselné delitele sú rovné 2, teda $b=2$.
V~takom prípade by úžasné číslo bolo $4c$ a~všetky jeho delitele by boli
1, 2, $c$, 4, $2c$, $4c$.
Súčet všetkých deliteľov by bol $7+7c$ a~podľa zadania má platiť
$$
7+7c=8c.
$$
To platí práve vtedy, keď $c=7$; zodpovedajúce úžasné číslo je $4c=28$.

\smallskip c)
Predpokladajme, že dva prvočíselné delitele sú rovnaké, ale oba rôzne od~2,
teda $b=c\ne 2$.
V~takom prípade by úžasné číslo bolo $2b^2$ a~všetky jeho delitele by boli
$1$, $2$, $b$, $2b$, $b^2$, $2b^2$.
Súčet všetkých deliteľov by bol $3+3b+3b^2$ a~podľa zadania má platiť
$$
\aligned
3+3b+3b^2 &=4b^2, \\
3(1+b) &=b^2.
\endaligned
$$
Číslo naľavo je násobkom čísla~3, preto číslo napravo má tiež byť násobkom~3.
Vzhľadom na to, že $b$ je prvočíslo, muselo by byť $b=3$.
V~takom prípade by však naľavo bolo $3\cdot 4=12$, zatiaľ čo napravo $3^2=9$.
Prípad $b=c\ne 2$ teda nie je možný.

\smallskip d)
Predpokladajme, že prvočíselné delitele sú navzájom rôzne, teda
$2\ne b\ne c\ne 2$.
V~takom prípade by úžasné číslo bolo $2bc$ a~všetky jeho delitele by boli
$1$, $2$, $b$, $c$, $2b$, $2c$, $bc$, $2bc$.
Súčet všetkých deliteľov by bol $3+3b+3c+3bc$ a~podľa zadania má platiť
$$
\aligned
3+3b+3c+3bc &=4bc, \\
3(1+b+c) &= bc.
\endaligned
$$
Číslo naľavo je násobkom čísla~3, preto číslo napravo má tiež byť násobkom~3.
Vzhľadom na to, že $b$ a~$c$ sú prvočísla, muselo by byť buď $b=3$,
alebo $c=3$.
Pre $b=3$ by predchádzajúca rovnosť prešla na $3\cdot(4+c)=3c$, čo však
neplatí pre žiadne~$c$.
Diskusia pre $c=3$ je obdobná.
Prípad $b\ne c\ne 2$ teda nie je možný.

\smallskip
Jediné úžasné číslo je 28.

\poznamka
Nemožnosť prípadu~c) môže byť zdôvodnená aj takto:
Každé prvočíslo $b\ne2$ je nepárne, preto číslo $b^2$ je tiež nepárne, zatiaľ čo
číslo $1+b$ (rovnako ako akýkoľvek jeho násobok) je párne.
V~uvedenej rovnosti na pravej strane by teda malo byť nepárne číslo, zatiaľ čo
na ľavej strane párne.
Podobný argument však nič neodhalí v~prípade~d).
}

{%%%%%   Z9-I-3
\napad
Porozmýšľajte, ako by ste pomocou polomeru hľadanej kružnice vyjadrili vzdialenosť jej stredu od úsečky~$AB$, príp. $BC$.

\riesenie
Počas riešenia sa odkazujeme na nasledujúci obrázok, v~ktorom $O$ označuje stred
strany~$BC$, $S$ označuje stred Jurovej vytúženej kružnice $h$, $K$ označuje dotykový
bod kružníc $h$ a~$k$, $L$ označuje dotykový bod kružníc $h$ a~$l$ a~$M$ označuje
dotykový bod kružnice~$h$ a~úsečky~$AB$.
Ďalej budeme odkazovať na pomocný bod~$E$, ktorý je pätou kolmice z~bodu~$S$
na stranu~$BC$.
Hľadaný polomer kružnice~$h$ v~cm označíme~$r$.
\figure{z9-I-3}%


Vzdialenosť bodu~$S$ od úsečky~$AB$ je rovná $r=|SM|=|EB|$.
Vzdialenosť bodu~$S$ od úsečky~$BC$ je rovná veľkosti úsečky~$SE$, ktorá je odvesnou ako v~pravouhlom trojuholníku $SEO$, tak v~trojuholníku $SEB$.
Všetky zvyšné strany v~oboch trojuholníkoch ľahko vyjadríme pomocou $r$;
z~toho pomocou Pytagorovej vety budeme vedieť určiť neznámu~$r$.

Body $S$ a~$O$ sú stredmi kružníc $h$ a~$l$, ktoré sa dotýkajú v~bode~$L$.
Tieto tri body ležia na jednej priamke, vzdialenosť $SO$ je preto rovná
$$
|SO|=|SL|+|LO|=r+6.
$$
Podobne, vzdialenosť $SB$ je rovná
$$
|SB|=|BK|-|KS|=12-r,
$$
lebo $S$ a~$O$ sú stredy kružníc $h$ a~$k$ a~$K$ je ich dotykovým bodom.
Vzdialenosť $OE$ je rovná
$$
|OE|=|OB|-|BE|=6-r.
$$
Z~toho a~z~Pytagorovej vety v~trojuholníkoch $SEO$ a~$SEB$ dostávame
$$
\align
\def|{\vert}
|SE|^2=|SO|^2-|OE|^2 &=|SB|^2-|BE|^2, \\
(6+r)^2-(6-r)^2 &=(12-r)^2-r^2, \\
12r+12r &=144-24r, \\
48r &=144, \\
r &=3.
\endalign
$$
Polomer hľadané kružnice je 3\,cm.
}

{%%%%%   Z9-I-4
\napad
Pri 2. časti úlohy si po každej transakcii pomocou neznámych zapíšte, koľko zmenárnikovi
pribudlo či ubudlo korún a~koľko mu pribudlo či ubudlo libier.

\riesenie
1.
Keď má zmenáreň vydať eurá zákazníkovi, znamená to, že zmenáreň eurá predáva.
Pracujeme preto s~hodnotou v~stĺpci "predaj", \tj. 28~CZK.
Zákazník dostane $4\,200:28=150$ eur.

\smallskip
2.
Neznámu v~stĺpci "nákup" označíme~$n$, v~stĺpci "predaj"
použijeme~$p$.
Keď zmenáreň vykúpi 1\,000 libier a~potom ich všetky predá, množstvo libier
v~zmenárni sa nezmení;
počet korún sa najskôr zmenší o~$1\,000n$ a~potom sa zväčší o~$1\,000p$.
Zisk 2\,200~korún môžeme vyjadriť nasledujúcou rovnicou, ktorú hneď upravíme:
$$
\eqalignno{
-1\,000n + 1\,000p &= 2\,200, & \cr
1\,000p &= 2\,200 + 1\,000n, & (1) \cr p &= 2{,}2 + n. }
$$

Keď zmenárnik predá 1\,000 libier a~potom všetky utŕžené koruny zmení
s~iným zákazníkom za libry, počet korún v~zmenárni sa síce prechodne
zväčší o~$1\,000p$, ale nakoniec zostane rovný východiskovej hodnote.
Suma libier sa najskôr zmenší o~1\,000 a~potom sa zväčší o~počet libier, ktoré
zmenárnik nakúpi za $1\,000p$ korún, tzn. o~$1\,000\frc{p}n$ libier.
Zisk 68{,}75 libier môžeme vyjadriť nasledujúcou rovnicou, ktorú hneď upravíme:
$$
\eqalignno{
-1\,000 +1\,000\frac{p}n &= 68{,}75, & \cr
1\,000p &= 1\,068{,}75 n, & (2) \cr p &= 1{,}06875 n. & }
$$
Porovnaním (1) a~(2) dostávame
$$
\aligned
2\,200 + 1\,000n &=1\,068{,}75 n, \\
68{,}75 n &= 2\,200, \\
n &=32.
\endaligned
$$
Odtiaľ dosadením do (1), resp. (2) získame $p =34{,}2$.
Zmenáreň teda nakupuje jednu libru za 32~CZK a~predáva ju za 34{,}20~CZK.
}

{%%%%%   Z9-I-5
\napad
Zvažujte postupne možnosti, keď je myslené číslo jednociferné, dvojciferné atď.
V~jednotlivých prípadoch porozmýšľajte postupne nad možnými súčtami na mieste
jednotiek, desiatok atď.

\riesenie
Najskôr nájdeme Betkino číslo, \tj. najmenšie číslo s~uvedenými vlastnosťami.

\smallskip 1)
Predpokladajme, že Betkino číslo je jednociferné, a~označme ho~$a$.
Potom by podľa zadania muselo platiť $a+a=a$, čo platí iba vtedy, keď $a=0$.
Nula však nie je prirodzené číslo, takže Betkino myslené číslo nemôže byť
jednociferné.

\smallskip 2)
Predpokladajme, že Betkino číslo je dvojciferné, a~označme ho~$\overline{ab}$.
Nech už súčet $\overline{ab}+\overline{ba}$ dopadne akokoľvek, na mieste jednotiek
čítame buď $b+a=a$, alebo $b+a=b$.
Z~toho dostávame buď $b=0$, alebo $a=0$.
V~takom prípade by však buď číslo $\overline{ba}$, alebo číslo $\overline{ab}$
nebolo dvojciferné.
Betkino myslené číslo teda nemôže byť dvojciferné.

\smallskip 3)
Predpokladajme, že Betkino číslo je trojciferné, a~označme ho $\overline{abc}$.
Z~rovnakého dôvodu ako vyššie nemôžu byť čísla $a$ a~$c$ nuly, preto v~súčte
$\overline{abc}+\overline{cba}$ sa na mieste jednotiek môže objaviť jedine~$b$:
$$
\alggg{a&b&c\\c&b&a}{*&*&b}
$$
Súčasne $c+a$ nemôže byť väčšie ako 9, pretože potom by celkový súčet
$\overline{abc}+\overline{cba}$ nebol trojciferný.
Z~toho sa dozvedáme, že
$$
a+c=b, \eqno (1)
$$
čo okrem iného znamená, že ani cifra~$b$ nemôže byť~0.

Z~toho vyplýva, že súčet $b+b$ na mieste desiatok nemôže byť menší ako~10;
v~takom prípade by totiž súčet bol rovný jednému z~čísel $a$, $b$, $c$,
čo vždy vedie k~nejakému sporu s~predchádzajúcimi poznatkami:
\begin{itemize}
\item
Ak $b+b=a$ alebo $b+b=c$, tak podľa (1) dostávame $2a+2c=a$ alebo $2a+2c=c$, teda $a=\m2c$ alebo $c=\m2a$, čo nie je možné.
\item Ak $b+b=b$, tak $b=0$, čo nie je možné.
\end{itemize}
Súčet $b+b$ na mieste desiatok však nemôže byť ani väčší ako 9.
V~takom prípade by totiž súčet na mieste stoviek bol $a+c+1$ a~toto číslo má byť
rovné jednému z~čísel $a$, $b$, $c$; to vždy vedie k~nejakému sporu:
\begin{itemize}
\item
Ak $a+c+1=a$ alebo $a+c+1=c$, tak $c=\m1$ alebo $a=\m1$, čo nie je možné.
\item Ak $a+c+1=b$, tak podľa (1) dostávame $b+1=b$, teda
$1=0$,
čo nie je možné.
\end{itemize}
Betkino myslené číslo teda nemôže byť ani trojciferné.

\smallskip 4)
Predpokladajme, že Betkino číslo je štvorciferné, a~označme ho
$\overline{abcd}$.
Z~rovnakého dôvodu ako vyššie nemôžu byť čísla $a$ a~$d$ nuly, teda v~súčte
$\overline{abcd}+\overline{dcba}$ sa na mieste jednotiek môže objaviť buď $b$,
alebo~$c$:
$$
\alggg{a&b&c&d\\d&c&b&a}{*&*&*&b}
\hskip16mm
\alggg{a&b&c&d\\d&c&b&a}{*&*&*&c}
$$
Súčasne $d+a$ nemôže byť väčšie ako 9, pretože inak by celkový súčet $\overline{abcd}+\overline{dcba}$ nebol štvorciferný.
Z~toho sa dozvedáme, že
$$
\eqalignno{
\text{buď}\quad a+d&=b, & (2) \cr
\text{alebo}\quad a+d&=c. & (3) }
$$
To okrem iného znamená, že buď $b\ne0$, alebo $c\ne0$.

Teraz predpokladáme, že súčet $c+b$ na mieste desiatok je menší ako 10, tzn. tento
súčet je rovný jednému z~čísel $a$, $b$, $c$, $d$, a~preskúmame jednotlivé prípady.
Najskôr uvažujme platnosť (2), a~teda $b\ne0$:
\begin{itemize}
\item
Ak $b+c=a$ alebo $b+c=d$, tak podľa (2) dostávame $a+d+c=a$ alebo $a+d+c=d$, teda $c=\m d$ alebo $c=\m a$, čo nie je možné.
\item Ak $b+c=b$, tak $c=0$ (čo ničomu nevadí).
\item Ak $b+c=c$, tak $b=0$, čo nie je možné.
\end{itemize}
Podobne za predpokladu (3) zistíme, že jediná prípustná možnosť je
\begin{itemize}
\item $b+c=c$, teda $b=0$.
\end{itemize}
Celkom tak objavujeme dva možné prípady:
$$
\alggg{a&b&0&d\\d&0&b&a}{b&b&b&b}
\hskip16mm
\alggg{a&0&c&d\\d&c&0&a}{c&c&c&c}
\eqno (4)
$$

Keďže Betkino číslo je najmenšie číslo vyhovujúce všetkým uvedeným
podmienkam, vôbec sa nemusíme zaoberať prípadom, keď súčet $c+b$ je väčší
ako 9, a~sústredíme sa len na druhú z~vyššie menovaných možností,
\tj. $b=0$.
Dosadíme najmenšie možné číslo na miesto tisícok $a=1$ a~zisťujeme,
že $c=d+1$.
Najmenšia vyhovujúca možnosť je $d=2$ a~$c=3$.
Betka si teda myslela číslo 1\,032 a~jej výpočet vyzeral takto:
$$
\alggg{1&0&3&2\\2&3&0&1}{3&3&3&3}
$$

\smallskip
Z~vyššie uvedeného je teraz jednoduché doplniť nejaké iné číslo s~uvedenými
vlastnosťami, teda nejaké Erikino číslo.
Napr. stačí v~Betkinom čísle zameniť cifry na mieste jednotiek a~tisícok alebo
cifry na mieste desiatok a~stoviek, príp. uvažovať akékoľvek čísla tvaru (4).
Medzi možnými riešeniami sú aj čísla, keď súčet $c+b$ je väčší ako 9.
Tu je niekoľko riešení, na ktoré mohla Erika prísť, keby nebola taká
netrpezlivá:
$$
\alggg{1&0&4&3\\3&4&0&1}{4&4&4&4}
\hskip16mm
\alggg{1&3&0&2\\2&0&3&1}{3&3&3&3}
\hskip16mm
\alggg{1&8&9&7\\7&9&8&1}{9&8&7&8}
$$

\poznamky
Pokiaľ vieme zdôvodniť, že hľadané Betkino číslo musí byť aspoň
štvorciferné, tak ho môžeme ľahko nájsť skúšaním:

Najmenšie štvorciferné číslo s~navzájom rôznymi ciframi je 1\,023.
Toto číslo však nie je riešením, pretože $1\,023+3\,201=4\,224$.
Ak nás napadne vymeniť cifry 2 a~3, dostaneme vyhovujúce riešenie:
$1\,032+2\,301=3\,333$.
Aby sme sa presvedčili, že toto riešenie je najmenšie možné, stačí overiť, že žiadne
číslo medzi 1\,023 a~1\,032 nevyhovuje všetkým uvedeným podmienkam.

\smallskip
Nahradenie ostatných úvah skúšaním je tiež možné, avšak často veľmi prácne.
No ak je riešenie založené na skúšaní úplné, nech je považované za
správne.
Akékoľvek čiastočné všeobecné postrehy môžu počet možností na preskúšanie zaujímavo
znižovať
(napr. počet trojíc rôznych čísel od~1 do~9 vyhovujúcich rovnosti~(1) určite nie je väčší ako~32).
}

{%%%%%   Z9-I-6
\napad
Začnite s~obsahom trojuholníka $AEF$.

\riesenie
Najskôr si zadanie verne znázorníme:
\figure{z9-I-6}%


Priamky $EF$ a~$BC$ sú rovnobežné,
súhlasné uhly pri vrcholoch $E$ a~$B$, resp. pri vrcholoch $F$ a~$C$ sú zhodné,
trojuholníky $AEF$ a~$ABC$ sú teda podobné.
Zodpovedajúci koeficient podobnosti je rovný
$$
|AE|:|AB|=|AE|:(|AE|+|EB|) =2:3.
$$
Obsahy týchto trojuholníkov sú teda v~pomere
$$
S_{AEF}:S_{ABC}=4:9,
$$
takže
$S_{AEF}=S_{ABC}\cdot 4:9 =12$~hektárov.

Úsečka~$AD$ delí trojuholník $AEF$ na dva trojuholníky, ktorých obsahy sú
v~rovnakom pomere ako dĺžky úsečiek $FD$ a~$DE$, teda
$$
S_{ADF}:S_{ADE} =|FD|:|DE| =2:1.
$$
Z~toho vyplýva, že
$S_{ADE} =S_{AEF}:3 =4$~hektáre a~$S_{ADF} =2\cdot S_{ADE} =8$~hektárov.

Úsečka~$DE$ delí trojuholník $ABD$ na dva trojuholníky, ktorých obsahy sú
v~rovnakom pomere ako dĺžky úsečiek $AE$ a~$EB$, teda
$$
S_{ADE}:S_{BDE} =|AE|:|EB| =2:1.
$$
Z~toho vyplýva, že $S_{BDE} =S_{ADE}:2 =2$~hektáre.

Teraz poznáme obsahy troch zo štyroch častí trojuholníka $ABC$, obsah tej poslednej je rovný rozdielu
$S_{BCFD} =S_{ABC}-S_{AEF}-S_{BDE} =13$~hektárov.
Obsahy častí trojuholníka $ABC$ v~hektároch sú
$$
S_{BED}=2,\quad S_{AED}=4,\quad S_{ADF}=8,\quad S_{BCFD}=13.
$$
}

{%%%%%   Z4-II-1
...}

{%%%%%   Z4-II-2
...}

{%%%%%   Z4-II-3
...}

{%%%%%   Z5-II-1
Najskôr zistíme, koľko ktorých dobrôt sa dá pripraviť zo všetkých sliviek:
\begin{itemize}
\item Štyri fľaše sliviek vystačia na $4\cdot\frac12=2$ celé plechy
ovocných rezov.
\item Štyri fľaše sliviek vystačia na $4\cdot 4=16$ koláčikov.
\item Štyri fľaše sliviek vystačia na $4\cdot 16=64$ šatôčok.
\end{itemize}
\noindent
V~spotrebe sliviek teda platia rovnosti:
$$
\text{2 plechy = 16 koláčikov = 64 šatôčok.}
$$

Keď mamička spotrebuje slivky na jeden plech ovocných rezov, spotrebuje
polovicu všetkých sliviek. Zvýšia tak slivky na
$$
\text{1 plech = 8 koláčikov = 32 šatôčok.}
$$
Keď spotrebuje ešte slivky na 6~koláčikov, zvyšné slivky už vystačia len
na 2~koláčiky, teda na štvrtinu predchádzajúceho množstva. Zo zvyšných sliviek
teda možno upiecť
$$
\text{2 koláčiky = 8 šatôčok.}
$$

\hodnotenie
2~body za vyjadrenie množstva jednotlivých druhov, ktoré možno upiecť zo všetkých sliviek;
po 1~bode za vyjadrenie spotreby sliviek na 1~plech a~6~koláčikov;
2~body za vyjadrenie spotreby sliviek na šatôčky a~záver.
\endhodnotenie

\poznamka
Úvodné prevádzajúce vzťahy možno zapísať aj takto:
$$
\text{1 plech = 8 koláčikov,\quad 1 koláčik = 4 šatôčky.}
$$
Predchádzajúce úvahy potom môžeme zjednodušiť nasledovne:

Po upečení plechu ovocných rezov zvýšia slivky na 1~plech, \tj. na 8~koláčikov.
Po upečení ďalších 6~koláčikov zvýšia slivky na 2~koláčiky, \tj. na 8~šatôčok.

Pri takomto postupe hodnoťte po 2~bodoch každý z~troch uvedených krokov.


\ineriesenie
Spotrebu sliviek budeme počítať vo fľašiach:

Keďže jedna fľaša sliviek vystačí na polovicu plechu ovocných rezov,
na jeden plech treba dve fľaše. Pre ďalšie použitie tak ostávajú
2~fľaše.

Keďže jedna fľaša vystačí na 4~koláčiky, na 6~koláčikov treba
jeden a~pol fľaše. Pre ďalšie použitie tak ostáva pol fľaše.

Keďže jedna fľaša vystačí na 16~šatôčok, zo zvyšnej polovice sa dá
upiecť 8~šatôčok.

\hodnotenie
2~body za poznatok, že po upečení plechu zvýšia dve fľaše;
2~body za poznatok, že po upečení ďalších šiestich koláčikov ostane polovica fľaše;
2~body za dopočítanie a~záver.
\endhodnotenie}

{%%%%%   Z5-II-2
Jahodová záhrada bola štvorcová, preto jej dĺžka bola rovnaká ako jej
šírka, a~tento rozmer bol rovný trojnásobku šírky jahodovej záhrady. Zo
zadania navyše vieme, že dĺžka jahodovej záhrady bola o~8~metrov väčšia ako
trojnásobok jej šírky.
\insp{z5-II-2.eps}%

Obvody oboch záhrad vyjadrené pomocou šírky jahodovej záhrady teda boli
$$
\aligned
&\text{obvod mrkvovej záhrady} = 12 \times \text{šírka jahodovej záhrady}, \\
&\text{obvod jahodovej záhrady} = 8 \times \text{šírka jahodovej záhrady} + \text{16 metrov}. \\
\endaligned
$$
Keďže obe záhrady mali rovnaký obvod, platilo, že
$$
12 \times \text{šírka jahodovej záhrady}
= 8 \times \text{šírka jahodovej záhrady} + \text{16 metrov}.
$$
Z~toho vyplýva, že šírka jahodovej záhrady bola $16:4=4$ metre.
Teda rozmery mrkvovej záhrady boli $12\times 12$ metrov a~rozmery jahodovej
záhrady boli $20\times 4$ metre.

\hodnotenie
2~body za vyjadrenie dĺžky jahodovej záhrady pomocou jej šírky;
2~body za vyjadrenie obvodov oboch záhrad;
1~bod za vyjadrenie šírky jahodovej záhrady;
1~bod za rozmery oboch záhrad.
\endhodnotenie
}

{%%%%%   Z5-II-3
Keďže v~piatok bolo 13., muselo byť v~nedeľu 15. a~v~pondelok 16.
Na nedele tak pripadajú dátumy 1., 8., 15., 22., 29. a~na pondelky
pripadajú dátumy 2., 9., 16., 23., prípadne~30.
Keďže v~uvažovanom mesiaci bolo päť nedieľ a~štyri pondelky, mal tento
mesiac práve 29~dní, jednalo sa teda o~február v~priestupnom roku.

Z~predchádzajúceho vieme, že 1.~februára bola nedeľa, preto 31.~januára bola sobota.
Ostatné januárové soboty boli 24., 17., 10. a~3.
Nový rok v~uvažovanom roku teda bol vo štvrtok.

Priestupný rok má 366 dní, \tj. 52 plných týždňov a~dva dni navyše
($366={52\cdot 7+2}$).
Nový rok v~nasledujúcom roku teda bude v~sobotu.

\hodnotenie
3~body za zistenie, že sa jednalo o~február v~priestupnom roku;
2~body za zistenie, že Nový rok v~uvažovanom roku bol vo štvrtok;
1~bod za zistenie, že Nový rok v~nasledujúcom roku bude v~sobotu.
\endhodnotenie
}

{%%%%%   Z6-II-1
Znázorníme všetkých päť čísel v~poradí, ako boli písané na tabuľu
(hviezdičky označujú zatiaľ neznáme cifry):
$$
\vbox{\let\\=\cr
\halign{&\hbox to1.0em{\hss$#$\hss}\cr
*&*\\*&*\\3&9\\*&*\\9&6\\}}
$$

V~naposledy riešenom príklade na sčítanie $39+**=96$ chýba iba jeden člen; číslo v~4.~riadku je teda $96-39=57$.

Tým dostávame podobnú situáciu v~príklade ${**}+39=57$; číslo v~2.~riadku je teda $57-39=18$.
Podobne doplníme chýbajúci člen prvého príkladu: $39-18=21$.

Na tabuli boli pôvodne napísané čísla 21 a~18.

\hodnotenie
Po 1~bode za každé neznáme číslo;
2~body podľa kvality vysvetlenia;
1~bod za správne formulovaný záver.
\endhodnotenie
}

{%%%%%   Z6-II-2
Značné časti obvodov osemuholníka $ABCDEFGH$ a~dvanásťuholníka $ABCDEFGHIJKL$ sú obom útvarom spoločné.
Zato úsečky $IJ$, $JK$ a~$KL$ sú súčasťou iba obvodu dvanásťuholníka a~úsečka $IL$ je súčasťou iba obvodu osemuholníka.
Tieto štyri úsečky sú navzájom zhodné, ich dĺžku označíme $a$.
Rozdiel obvodov útvarov ${58-48}=10$\,(cm) teda zodpovedá dvom $a$; tzn. $a=5$\,cm.

Podobne rozdiel medzi obvodom osemuholníka $ABCDEFGH$ a~obvodom obdĺžnika $ABGH$ je rovný dvom $a$, čiže 10\,cm.
Preto je obvod obdĺžnika rovný $48-10=38$\,(cm) a~súčet dĺžok jeho strán je $38:2=19$\,(cm).

Vzhľadom na to, že útvar je súmerný podľa vodorovnej osi, musia byť úsečky $AL$, $BC$, $FG$ a~$IH$ zhodné.
Vzhľadom na to, že dĺžky všetkých strán dvanásťuholníka sú v~centimetroch vyjadrené celými číslami, môžu dĺžky strán obdĺžnika $ABGH$ nadobúdať iba nasledujúce hodnoty (v~cm):
$$
\aligned
\vert AB\vert&=11,12,13,\dots, \\
\vert BG\vert&=7,9,11,\dots
\endaligned
$$
Aby bol súčet týchto dĺžok rovný 19, musia byť dĺžky strán obdĺžnika $ABGH$ 12\,cm a~7\,cm.

\poznamka
Ak dĺžku zhodných úsečiek $AL$, $BC$, $FG$ a~$IH$ označíme $b$ a~dĺžku zhodných úsečiek $KD$ a~$EJ$ označíme $c$, tak možno obvod dvanásťuholníka vyjadriť takto:
$$
\aligned
10a+4b+2c&=58, \\
4b+2c&=8.
\endaligned
$$
Dĺžky všetkých strán dvanásťuholníka (v~cm) sú celočíselné, preto sú $b$ a~$c$ celočíselné a~zrejme aj kladné.
Úloha má teda jediné riešenie: $b=1$ a~$c=2$.
Dĺžky strán obdĺžnika $ABGH$ sú $2a+c=12$\,(cm) a~$2b+a=7$\,(cm).
\insp{z6-II-2a.eps}%


\hodnotenie
3~body za dĺžku strany štvorca $a$ (z~toho 1~bod za zdôvodnenie);
2~body za vzťah $|AB|+|BG|=19$\,cm alebo jeho obdobu (napr. $4b+2c=8$);
1~bod za vyčíslenie a~správny záver.
\endhodnotenie
}

{%%%%%   Z6-II-3
Prvá stará sliepka znášala každý deň jedno vajce, \tj. za 60 dní 60 vajec.
Druhá stará sliepka znášala každý druhý deň jedno vajce, \tj. za 60 dní 30 vajec.
Tretia stará sliepka znášala každý tretí deň jedno vajce, \tj. za 60 dní 20 vajec.
Staré sliepky zniesli za 60 dní spolu $60+30+20=110$ vajec.

Staré aj nové sliepky zniesli za 60 dní spolu 155 vajec, takže iba nové sliepky zniesli za 60 dní spolu $155-110=45$ vajec.
Pritom jedna z~nových sliepok znášala dvakrát viac ako tá druhá.
Rozdelením 45 na tri rovnaké diely zisťujeme, že jedna nová sliepka zniesla za 60 dní 15 vajec, tzn.
jedno vajce každý štvrtý deň ($60:15=4$).
Druhá nová sliepka tak zniesla za 60 dní 30 vajec, tzn. jedno vajce každý druhý deň.

\hodnotenie
2~body za celkový počet vajec, ktoré zniesli tri staré sliepky za 60 dní;
1~bod za celkový počet vajec, ktoré zniesli dve nové sliepky za 60 dní;
1~bod za rozdelenie vajec od nových sliepok na tri rovnaké diely;
2~body za určenie znášky nových sliepok.
\endhodnotenie
}

{%%%%%   Z7-II-1
Rozdiel pôvodného počtu žien a~súčasného počtu mužov je $30+196=226$,
čo je o~10 viac ako rozdiel pôvodného a~súčasného počtu žien.
To znamená, že tri štvrtiny pôvodného počtu žien sú rovné $226-10=216$.
Zvyšná jedna štvrtina zodpovedajúca aktuálnemu počtu žien je teda $216:3=72$.
Mužov je teraz $72-10=62$.

V~obci teraz žije spolu $62+72=134$ mužov a~žien.

Predchádzajúce úvahy môžeme graficky znázorniť takto:
\insp{z7-II-1.eps}%

\poznamka
Aktuálny počet žien možno určiť algebraicky riešením rovnice.
Ak tento počet označíme $z$, tak pôvodný počet žien bol $4z$, aktuálny počet mužov je $z-10$ a~pôvodný počet mužov bol $4z-30$.
Rozdiel pôvodného a~súčasného počtu mužov je 196, teda:
$$
\align
(4z-30)-(z-10)&=196,\\
3z-20&=196,\\
z&=72.
\endalign
$$

\hodnotenie
2~body za aktuálny počet žien; 1~bod za správny záver; 3~body podľa kvality vysvetlenia.
\endhodnotenie
}

{%%%%%   Z7-II-2
Prvé číslo napísané na tabuli označíme~$x$, druhé~$y$.
Klára napísala súčet
$$
x+y,
$$
Lukáš doplnil
$$
y+(x+y)=x+2y
$$
a~nakoniec Magda pridala
$$
(x+y)+(x+2y)=2x+3y.
$$
Hodnota naposledy uvedeného súčtu je~94.
Zadanie nestanovuje, ktorá z~neznámych $x$ a~$y$ je~14, rozoberieme obe možnosti:
\ite a) Ak $x=14$, tak
$$
\aligned
2\cdot 14+3y&=94,\\
3y&=66,\\
y&=22.
\endaligned
$$
\ite b) Ak $y=14$, tak %riešime rovnicu
$$
\aligned
2x+3\cdot 14&=94,\\
2x&=52,\\
x&=26.
\endaligned
$$

Na začiatku mohlo byť na tabuli okrem čísla~14 napísané číslo 22 alebo 26.

\poznamka
Predchádzajúce dve možnosti môžeme uvažovať od samého začiatku.
Potom súčty, ktoré boli postupne písané na tabuľu, sú vyjadrené takto:
\ite a) $14+y$, $14+2y$ a~$28+3y$, čo vedie na~rovnicu $28+3y=94$, ktorej riešením je $y=22$.
\ite b) $x+14$, $x+28$ a~$2x+42$, čo vedie na~rovnicu $2x+42=94$, ktorej riešením je $x=26$.

\hodnotenie
Po 1~bode za každé riešenie;
3~body podľa kvality a~úplnosti vysvetlenia;
1~bod za správne formulovaný záver.
\endhodnotenie
}

{%%%%%   Z7-II-3
Trojuholník $APD$ je rovnostranný, teda dĺžky všetkých jeho strán sú 5\,cm a~veľkosti všetkých vnútorných uhlov sú $60\st$.

Vnútorné uhly obdĺžnika $ABCD$ sú pravé, takže veľkosti uhlov $PDE$ a~$PAB$ sú~$30\st$.
Uhol $PED$ je tretím uhlom trojuholníka $AED$ (alebo tiež striedavým uhlom k~uhlu $EAB$), preto je jeho veľkosť $30\st$.
Uhly $PDE$ a~$PED$ sú zhodné, teda trojuholník $DEP$ je rovnoramenný a~$|PD|=|PE|=5$\,cm.
Úsečka $AE$ je dlhá
$$
|AE|=|AP|+|PE|=5+5=10\,(\Cm).
$$

Úsečka $CE$ je zhodná s~úsečkou $CB$, takže trojuholník $EBC$ je rovnoramenný
s~pravým uhlom pri vrchole~$C$.
Zvyšné dva vnútorné uhly sú preto zhodné s~veľkosťou~$45\st$.
Veľkosť uhla $AEB$ je rovná
$$
|\angle AEB|=180\st-|\angle AED|-|\angle BEC|=180\st-30\st-45\st=105\st.
$$
\insp{z7-II-3.eps}%


\hodnotenie
Po 1~bode za poznatky o~rovnoramennosti trojuholníkov $DPE$ a~$EBC$;
po 1~bode za vyčíslenie veľkosti úsečky $AE$ a~uhla $AEB$;
2~body podľa kvality komentára.
\endhodnotenie
}

{%%%%%   Z8-II-1
Uvažujme vždy tú najmenej vhodnú situáciu:
\ite a) Keby Janko vzal všetky pastelky zo všetkých farieb okrem jednej, táto farba by mu stále chýbala.
Najnepriaznivejšia situácia nastáva, keď mu chýba žltá, pretože týchto pasteliek je najmenej.
Počet všetkých modrých, červených a~zelených pasteliek je $9+6+5=20$, Janko preto musí vziať
aspoň 21~pasteliek.
\ite b) Keby Janko vzal všetky pastelky jednej farby, táto farba by potom v~krabici chýbala.
Najnepriaznivejšia situácia nastáva, keď vyberá samé žlté, pretože tých je najmenej.
Žlté pastelky sú~4, Janko preto môže vziať najviac 3~pastelky.
\ite c) Najnepriaznivejšia situácia nastáva, keď Janko berie len červené pastelky.
Tých je spolu~9, môže teda vziať najviac 4~pastelky.

\hodnotenie
Po 1 bode za každú správnu odpoveď;
3~body podľa kvality vysvetlenia.
\endhodnotenie
}

{%%%%%   Z8-II-2
Počty jednotlivých druhov zvierat označíme ich počiatočnými písmenami.
Informácie zo zadania môžeme postupne zapísať takto:
$$
\aligned
h+p+k+s&=40, \\
h&=3k, \\
s-8&=h+p, \\
40-\frac14h+\frac34h&=46.
\endaligned
$$

Z~poslednej rovnice vyplýva $\frac12h=6$, teda $h=12$.
Z~druhej rovnice zisťujeme, že $k=4$.
Dosadením týchto hodnôt do zvyšných dvoch rovníc dostávame
$$
\aligned
12+p+4+s&=40, \quad\text{teda}\quad p+s=24, \\
s-8&=12+p, \quad\text{teda}\quad s=p+20.
\endaligned
$$
Z toho ďalej vyplýva
$$
\aligned
p+(p+20)&=24, \\
2p&=4, \\
p&=2, \quad\text{a~teda}\quad s=22.
\endaligned
$$

Dedo chová 12 husí, 2 prasatá, 4 kozy a~22~sliepok.

\hodnotenie
2~body za určenie počtu husí;
1~bod za určenie počtu kôz;
3~body za určenie počtu prasiat a~sliepok.
\endhodnotenie
}

{%%%%%   Z8-II-3
Vrcholy štvorca tulipánového záhona označíme $A$, $B$, $C$, $D$, stred tohto štvorca $S$ a~priesečníky strán štvorca so stranami terasy $E$, $F$, pozri obrázok.
\insp{z8-II-3.eps}%

Pri otáčaní okolo stredu $S$ o~celočíselné násobky uhla $90\st$ sa štvorec $ABCD$ zobrazuje sám na seba.
Uvažujme napr. otočenie, pri ktorom sa vrchol $A$ zobrazuje na vrchol $B$, a~teda strana $DA$ na stranu $AB$.
Body $E$ a~$F$ ležia práve na týchto stranách, uhol $ESF$ je podľa zadania pravý, a~preto sa bod $E$ zobrazuje na bod $F$.
Obe strany záhona sú teda stranami terasy rozdelené v~rovnakom pomere, \tj.
$$
|DE|:|EA|=|AF|:|FB|=1:5.
$$

Ešte označme $G$ a~$H$ priesečníky priamok $SE$ a~$SF$ so zvyšnými stranami štvorca $ABCD$.
Pri uvažovanom otočení sa bod $B$ zobrazuje na bod $C$, bod $F$ sa zobrazuje na bod $G$ atď.
Preto všetky štvoruholníky $SEAF$, $SFBG$, $SGCH$ a~$SHDE$ sú navzájom zhodné.
Tieto štyri štvoruholníky tvoria celý štvorec $ABCD$, ktorého obsah je $6\cdot 6=36\,(\text{m}^2)$.
Obsah každého z~nich je tak rovný $36:4=9\,(\text{m}^2)$.
Stavbou terasy sa záhon tulipánov zmenšil o~$9\,\text{m}^2$.

\ineriesenie
Pri rovnakom označení ako vyššie rozdeľme štvorec $ABCD$ pomocnou štvorčekovou sieťou na štvorčeky so stranami 1\,m
a~predpokladajme, že bod $E$ delí stranu $DA$ v~pomere $1:5$.
Body $E$ a~$S$ sú mrežovými bodmi štvorčekovej siete, pozri obrázok.
\insp{z8-II-3b.eps}%


Uhol $ESF$ je pravý práve vtedy, keď vyznačené pravouhlé trojuholníky sú zhodné,
čo nastáva práve vtedy, keď $F$ je mrežovým bodom deliacim stranu $AB$ v~pomere $1:5$.
Obe strany záhona sú teda stranami terasy rozdelené v~rovnakom pomere.

Obsah štvoruholníka $SEAF$ je rovný súčtu obsahov troch vyznačených častí, z~ktorých každá má obsah $3\,\text{m}^2$.
Stavbou terasy sa záhon tulipánov zmenšil o~$9\,\text{m}^2$.

\hodnotenie
1~bod za určenie pomeru, v~akom druhá strana terasy delila druhú stranu záhona;
2~body za obsah časti záhona obsadeného stavbou terasy;
3~body podľa kvality komentára.
\endhodnotenie
}

{%%%%%   Z9-II-1
Budeme uvažovať odzadu:

Petrík dostal o~polovicu rybičky viac, ako bola polovica všetkých rybičiek, ktoré
zvýšili po Matejovi. Keďže potom bolo akvárium prázdne, bola oná polovica
rybičky navyše práve polovicou toho, čo zvýšilo po Matejovi. Po Matejovom
nákupe teda ostala v~akváriu jedna rybička.

Matej dostal o~polovicu rybičky viac, ako bola polovica všetkých rybičiek, ktoré
zvýšili po Ondrejovi. Keďže potom ostala v~akváriu jedna rybička, bola táto
rybička a~polovica rybičky navyše práve polovicou toho, čo zvýšilo po
Ondrejovi. Po Ondrejovom nákupe ostali v~akváriu tri rybičky.

Ondrej dostal o~polovicu rybičky viac, ako bola polovica všetkých rybičiek, ktoré
boli pôvodne v~akváriu. Keďže potom ostali v~akváriu tri rybičky, boli
tieto tri rybičky a~polovica rybičky navyše práve polovicou pôvodného množstva
rybičiek. Pôvodne bolo v~akváriu sedem rybičiek. Teda Ondrej dostal štyri
rybičky, Matej dve a~Petrík jednu rybičku.

\hodnotenie
1~bod za určenie zvyšku po Matejovi;
2~body za určenie zvyšku po Ondrejovi;
3~body za určenie pôvodného množstva a~počtov rybičiek, ktoré si odniesli
jednotliví chlapci.
\endhodnotenie

\ineriesenie
Ak pôvodný počet rybičiek v~akváriu označíme $x$, tak môžeme ďalšie
počty postupne vyjadriť takto:
$$
\def\tstrut{\vrule height3.6ex depth2ex width0pt}%
\begintable
\|dostal|zvýšilo\crthick
Ondrej\|$\dfrac{x+1}2$|$\dfrac{x-1}2$\cr
Matej\|$ \dfrac{x+1}4$|$\dfrac{x-3}4$\cr
Petrík\|$\dfrac{x+1}8$|$\dfrac{x-7}8$\endtable
$$
Z~toho je zrejmé, že po Petríkovom nákupe mohlo byť akvárium bez rybičiek vtedy a~len vtedy, keď $x=7$.
Dosadením ľahko určíme počty rybičiek, ktoré si odniesli jednotliví chlapci.

\hodnotenie
2~body za predposledný riadok tabuľky;
2~body za posledný riadok tabuľky;
1~bod za vyčíslenie neznámej;
1~bod za počty rybičiek pre jednotlivých chlapcov.
\endhodnotenie
}

{%%%%%   Z9-II-2
Súčet všetkých čísel vpísaných do políčok na obrázku je rovný
$$
1+2+\cdots+9=45.
$$
Čísla v~trojuholníkoch a~čísla v~šesťuholníkoch sú navzájom rôzne
a~pomer ich súčtov je $3:6=1:2$. Z~toho vyplýva, že súčet čísel
v~trojuholníkoch je~15 a~súčet čísel v~šesťuholníkoch je~30. Keďže
pomer súčtov čísel v~kruhoch a~v~trojuholníkoch je $2:3$, súčet čísel
v~kruhoch je~10.

Na obrázku sú štyri kruhy, teda v~kruhoch musia byť čísla 1, 2, 3 a~4
(akákoľvek iná štvorica by mala súčet väčší ako~10). Z~týchto štyroch
čísel sú dve obsiahnuté aj v~trojuholníkoch~-- vo zvyšnom hornom
trojuholníku má byť také číslo, aby súčet týchto troch čísel bol~15.
Najväčší možný súčet čísel v dvoch spodných trojuholníkoch je~7, preto
najmenšie možné číslo v~hornom trojuholníku je~8. Súčasne v~hornom
trojuholníku nemôže byť väčšie číslo ako~9:
\begin{itemize}
\item ak je v~hornom trojuholníku číslo~8, tak v~spodných dvoch trojuholníkoch musia byť čísla 3 a~4,
\item ak je v~hornom trojuholníku číslo~9, tak v spodných dvoch trojuholníkoch musia byť čísla 2 a~4.
\end{itemize}
V~oboch prípadoch možno ľahko doplniť ostatné čísla tak, že sú splnené
všetky požiadavky zo zadania, pozri obrázky.
V~hornom trojuholníku mohlo byť vpísané číslo~8 alebo~9.
\insp{z9-II-2a.eps}%


\hodnotenie
2~body za vyjadrenie súčtov čísel v~trojuholníkoch a~šesťuholníkoch;
1~bod za vyjadrenie štvorice čísel v~kruhoch;
2~body za určenie dvoch možných čísel v~hornom trojuholníku;
1~bod za doplnenie obrázka a~overenie.
\endhodnotenie
}

{%%%%%   Z9-II-3
Na obrázku je znázornené jediné možné usporiadanie kružníc, ktoré vyhovuje
všetkým podmienkam zo zadania.

Keďže sa kružnice dotýkajú, sú dĺžky strán štvoruholníka
$S_1S_2S_3S_4$ rovné súčtom polomerov prislúchajúcich kružníc. Keďže je
tento štvoruholník kosoštvorcom, sú všetky jeho strany zhodné. Teda
$$
s=r_1+r_2=r_2+r_3=r_3+r_4=r_4+r_1,
$$
pričom $r_1$, $r_2$, $r_3$, $r_4$ sú veľkosti polomerov prislúchajúcich
kružníc a~$s$ je veľkosť strany kosoštvorca. Z~toho vyplýva, že
protiľahlé kružnice sú navzájom zhodné, teda $r_1=r_3$ a~$r_2=r_4$.
Uhlopriečky v~kosoštvorci sú na seba kolmé a~navzájom sa rozpoľujú. Preto
sú trojuholníky určené stranami a~uhlopriečkami kosoštvorca pravouhlé
a~navzájom zhodné.
\insp{z9-II-3a.eps}%


Odvesny majú podľa zadania veľkosti $r_1=r_3=5\cm$
a~$\frac12|S_2S_4|=12\cm$. Veľkosť prepony, \tj. veľkosť strany
kosoštvorca, je podľa Pytagorovej vety rovná
$$
s=\sqrt{12^2+5^2}=\sqrt{169}=13\,(\Cm).
$$
Z~toho vyplýva, že
$$
r_2=r_4=13-5=8\,(\Cm).
$$
Polomery kružníc $k_1$ a~$k_3$ sú 5\,cm, polomery kružníc $k_2$
a~$k_4$ sú 8\,cm.

\hodnotenie
3~body za zistenie a~zdôvodnenie, že $r_1=r_3$ a~$r_2=r_4$;
2~body za určenie strany kosoštvorca;
1~bod za určenie polomerov $r_2$ a~$r_4$.
\endhodnotenie
}

{%%%%%   Z9-II-4
Aby bol výsledok rovný nule, musí byť súčet všetkých čísel so znamienkom
plus rovnaký ako súčet všetkých čísel so znamienkom mínus. Z~toho
vyplýva, že súčet všetkých uvedených čísel musí byť párny.

V~prípade~a) je celkový súčet
$$
1+2+3+\dots+10 = 55.
$$
Keďže je tento súčet nepárny, úloha nemá riešenie.

V~prípade~b) je celkový súčet
$$
1+2+3+\dots+11 = 66.
$$
Ako čísla so znamienkom plus, tak čísla so znamienkom mínus preto musia mať
súčet~33.
Jedno z~mnohých možných riešení úlohy je napr.
$$
+1+2-3+4+5+6+7+8-9-10-11.
$$

\hodnotenie
3~body za zistenie, že celkový súčet musí byť párny;
1~bod za zistenie, že úloha a) nemá riešenie;
2~body za nájdenie jedného riešenia úlohy b).
\endhodnotenie
}

{%%%%%   Z9-III-1
Dĺžky strán pôvodného obdĺžnika sú v~pomere $2:5$.
To znamená, že ich dĺžky (v~centimetroch) môžeme označiť $2x$ a~$5x$.
Po prvom predĺžení strán má obdĺžnik rozmery $2x+9$ a~$5x+9$.
Keďže dĺžky strán tohto obdĺžnika sú v~pomere $3:7$, musí platiť:
$$
\frac{2x+9}{5x+9}=\frac37.
$$
Po vyriešení rovnice dostávame $x=36$\,(cm).
Rozmery pôvodného obdĺžnika sú $2\cdot 36=72$\,(cm) a~$5\cdot 36=180$\,(cm).

Po druhom predĺžení strán má obdĺžnik rozmery $72+18=90$\,(cm) a~$180+18=198$\,(cm).
Pomer dĺžok strán tohto obdĺžnika teda je
$$
90:198=5:11.
$$

\hodnotenie
2~body za označenie rozmerov pôvodného obdĺžnika a~vyjadrenie ich zmeny;
2~body za zostavenie a~vyriešenie rovnice;
2~body za výpočet konečných rozmerov obdĺžnika a~ich pomeru.
\endhodnotenie}

{%%%%%   Z9-III-2
Myslené cifry označíme $a$, $b$ a~$c$, pričom bez ujmy na všeobecnosti predpokladáme, že $a<b<c$.
Z~uvažovaných čísel mohol jednociferný rozdiel vzniknúť jedine ako rozdiel dvoch čísel začínajúcich rovnakou cifrou.
Preto stačí uvažovať nasledujúce tri možnosti:

\smallskip 1)
Ak by jednociferný rozdiel vznikol ako
$$
\overline{acb}-\overline{abc}
=(100a+10c+b)-(100a+10b+c)
=9(c-b),
$$
tak by platilo $c-b=1$, tzn. $c=b+1$ (zodpovedajúci rozdiel by bol 9).
Ostatné rozdiely vzniknuté zo zvyšných čísel $\overline{cba}$, $\overline{cab}$, $\overline{bca}$ a~$\overline{bac}$ by potom mohli byť:
$$
\align
\overline{cba}-\overline{cab}&=9(b-a), \\
\overline{cba}-\overline{bca}&=90, \\
\overline{cba}-\overline{bac}&=100+10(b-a)+(a-c), \\
\overline{cab}-\overline{bca}<\overline{cab}-\overline{bac}&=99, \\
\overline{bca}-\overline{bac}&=9(c-a).
\endalign
$$
Trojciferný rozdiel možno dostať iba ako $\overline{cba}-\overline{bac}$.
Tento rozdiel má byť podľa zadania deliteľný piatimi.
Poslednou cifrou rozdielu nemôže byť nula, lebo $a\ne c$, rozdiel preto končí cifrou~5.
A~keďže sme stanovili, že $c>a$,
muselo by platiť $c=a+5$,
a~teda $b=a+4$ (zodpovedajúci rozdiel by bol $100+40-5=135$).
Rozdiel zvyšných čísel $\overline{cab}$ a~$\overline{bca}$ by potom bol dvojciferný ($\overline{cab}-\overline{bca}=100-50+4=54$).
Jediná vyhovujúca trojica cifier obsahujúca 3 je 3, 7, 8.

\smallskip 2)
Ak by jednociferný rozdiel vznikol ako $\overline{bca}-\overline{bac}$, tak by platilo $c-a=1$, tzn. $c=a+1$.
V~takom prípade neexistuje $b$, pre ktoré by platilo $a<b<c$.

\smallskip 3)
Ak by jednociferný rozdiel vznikol ako $\overline{cba}-\overline{cab}$, tak by platilo $b-a=1$, tzn. $b=a+1$ (zodpovedajúci rozdiel by bol 9).
Podobnými úvahami ako v~prvom prípade zistíme, že trojciferný rozdiel možno zo zvyšných čísel $\overline{bca}$, $\overline{bac}$, $\overline{acb}$ a~$\overline{abc}$ dostať iba ako $\overline{bca}-\overline{abc}$.
Aby bol tento rozdiel deliteľný piatimi, muselo by platiť $c=a+5$, a~teda $c=b+4$
(zodpovedajúci rozdiel by bol $100+40-5=135$).
Rozdiel zvyšných čísel $\overline{bac}$ a~$\overline{acb}$ by potom bol dvojciferný ($\overline{bac}-\overline{acb}=100-50+4=54$).
Jediné vyhovujúce trojice cifier obsahujúce 3 sú 2, 3, 7 a~3, 4, 8.

\smallskip
Kamaráti si mohli myslieť cifry 3, 7, 8 alebo 2, 3, 7 alebo 3, 4, 8.

\hodnotenie
1~bod za možnosti vzniku jednociferného rozdielu;
2~body za všetky vyhovujúce trojice čísel (1~bod za aspoň jednu takú trojicu);
3~body podľa kvality a~úplnosti zdôvodnenia, že viac trojíc neexistuje.
\endhodnotenie
}

{%%%%%   Z9-III-3
Prirodzené číslo, ktorého násobky boli napísané na papieri, označme $n$; podľa zadania je $n>1$.
Dotyčné čísla na papieri tak môžeme označiť
$$
(k-1)n,\quad kn,\quad (k+1)n,
$$
pričom neznáma $k$ je prirodzené číslo; aby boli všetky tri výrazy kladné, musí byť ${k>1}$.
Rado tak dostal súčiny $(k-1)kn^2$ a~$(k+1)kn^2$, ktorých rozdiel je $2kn^2$.
Platí ${2kn^2=216}$, po úprave
$$
kn^2=108=2\cdot2\cdot3\cdot3\cdot3.
$$
Pri rešpektovaní podmienok $n>1$ a~$k>1$ dostávame tri rôzne riešenia:
$$
\begintable
$n$\|2|3|6\crthick
$k$\|27|12|3\cr
$kn$\|54|36|18\endtable
$$
Existujú tri možné čísla, na ktoré mohol Rado ukázať: 54, 36 alebo 18.

\hodnotenie
2~body za rovnicu $kn^2=108=2^2\cdot3^3$ alebo jej obdobu;
po 1~bode za každé riešenie vyhovujúce zadaniu;
1~bod za správne formulovaný záver.

Vypísanie dotyčných troch členov postupnosti nie je nevyhnutnou súčasťou riešenia.
Pre názornosť ich uvádzame:
a) 52, 54, 56; b) 33, 36, 39; c) 12, 18, 24.
\endhodnotenie
}

{%%%%%   Z9-III-4
Označme ako na nasledujúcom obrázku
vrcholy pôvodného trojuholníka $A$, $B$, $C$,
body dotyku vpísanej kružnice $D$, $E$, $F$ a~jej stred $S$,
krajné body dokreslených úsečiek $G$, $H$, $I$, $J$, $K$, $L$
a~ich body dotyku s~kružnicou $M$, $N$, $O$.
\insp{z9-III-4a.eps}%


Úsečky $HD$ a~$HM$ sú dotyčnicami z~bodu $H$ ku kružnici, preto sú uhly $HDS$ a~$HMS$ pravé.
Zodpovedajúce trojuholníky $HDS$ a~$HMS$ majú spoločnú stranu $HS$ a~zhodné odvesny $SD$ a~$SM$ tvoriace polomery vpísanej kružnice.
Z~Pytagorovej vety vyplýva, že aj odvesny $HD$ a~$HM$ sú zhodné, tzn. $|HD|=|HM|$.

Z~rovnakého dôvodu platí aj $|GF|=|GM|$, teda obvod trojuholníka $AHG$ je rovný
$$
|AH|+|HM|+|MG|+|GA|=|AH|+|HD|+|FG|+|GA|=|AD|+|AF|. \eqno{(*)}
$$
To znamená, že obvod rohového trojuholníka $AHG$ je rovný súčtu vzdialeností bodu $A$ od dotykových bodov na stranách $AB$ a~$AC$.

Rovnaká vlastnosť platí aj pre obvody zvyšných dvoch rohových trojuholníkov.
Obvod trojuholníka $ABC$ je preto rovný
$$
12+14+16=42\,(\cm).
$$

\hodnotenie
2~body za poznatok $|HD|=|HM|$ a~jeho použitie;
1~bod za jeho zdôvodnenie (možno akceptovať aj odkaz na súmernosť podľa priamky $HS$);
2~body za vyjadrenie obvodu rohového trojuholníka ($*$);
1~bod za vyčíslenie obvodu trojuholníka $ABC$.
\endhodnotenie
}
