{%%%%%   A-I-1
Pri analýze počtu predmetov po jednotlivých krokoch
budeme postupovať "odzadu". Ukážeme najskôr, ako
možno z~počtov predmetov v~dvoch miestnostiach po odovzdávke určiť počty
predmetov pred ňou. Povedzme, že v~miestnostiach $A$ a~$B$ je pred odovzdávkou
z~$A$ do $B$ postupne $a$ a~$b$ predmetov. Tieto počty po odovzdávke
označme $a'$, $b'$. Podľa zadania platí
$$
a' = \frac{n-1}{n} a, \qquad b' = b + \frac{1}{n}a.
$$
Z~prvej rovnosti a~následne zo vzťahu $a+b = a'+b'$ nájdeme
$$
a~= \frac{n}{n-1} a', \qquad b = b' - \frac{1}{n-1}a'.
$$

Označme teraz $M$ počet predmetov nachádzajúcich sa na konci
v~každej zo štyroch miestností. Opakovaným použitím odvodeného vzťahu
$(a',b')\to(a,b)$ sa dopracujeme
až k~vyjadreniu počiatočných počtov pomocou hodnôt $M$ a~$n$:
$$
\alignat5
\text{Na konci:} &\quad & M&, &M&, &M&, &M;&\\
\text{pred } 4 \to 1{:}&\quad &\frac{n-2}{n-1}M&, &M&, &M&, &\frac{n}{n-1}M;\\
\text{pred } 3 \to 4{:}&\quad &\frac{n-2}{n-1}M&, &M&, &\frac{n}{n-1}M&, &M;\\
\text{pred } 2 \to 3{:}&\quad &\frac{n-2}{n-1}M&, &\frac{n}{n-1}M&, &M&, &M;\\
\text{pred } 1 \to 2{:}&\quad &\frac{n(n-2)}{(n-1)^2}M&,\quad &\frac{(n-1)^2+1}{(n-1)^2}M&,\quad &M&,\quad &M.
\endalignat
$$

Keďže bol počet predmetov v~prvej miestnosti na začiatku kladný, musí byť
$n\ge3$. Teraz už ľahko určíme najmenšiu možnú hodnotu výrazu
$$
V_2 =\frac{(n-1)^2+1}{(n-1)^2}M.
$$
Čitateľ a~menovateľ zlomku sa líšia
o~jedna, a~teda zlomok sa nedá krátiť. Ak má vyjsť celé číslo, musí nutne byť
$M=k~\cdot (n-1)^2$ pre vhodné $k$, a~preto $V_2 = k\bigl((n-1)^2+1\bigr)$.
Pre $n\ge3$ je však $(n-1)^2+1 \ge 5$, preto aj~$V_2 \ge 5$.
Voľbou $n = 3$, $k=1$ a~$M=4$ potom dosiahneme hodnotu $V_2 = 5$, pričom sa
ľahko presvedčíme, že zodpovedajúca štvorica $(3, 5, 4, 4)$ vyhovuje
podmienkam úlohy: po jednotlivých odovzdávkach z~nej dostaneme štvoricu
$(2, 6, 4, 4)$, potom $(2, 4, 6, 4)$, potom $(2, 4, 4, 6)$ a~napokon
$(4, 4, 4, 4)$. Hľadaný minimálny počet predmetov v~druhej miestnosti je teda naozaj 5
a~možno ho dosiahnuť iba pre $n=3$, pretože pre $n\ge4$ je $V_2 \ge 3^2+1=10$.

\ineriesenie
Označme počiatočné počty predmetov v~miestnostiach postupne
$a$, $b$, $c$,~$d$ a~rad za radom určujme počty predmetov
v~jednotlivých miestnostiach po jednotlivých odovzdávkach.
$$
\alignat5
\text{Na začiatku:}&\quad &a&,\quad &b&,\quad &c&,\quad &d;&\\
\text{po } 1\to 2{:}&\quad &\frac{n-1}{n}a&, &b+\frac{a}{n}&, &c&, &d;\\
\text{po } 2\to 3{:}&\quad &\frac{n-1}{n}a&, &\frac{n-1}{n}\left(b+\frac{a}{n}\right)&, &c+\frac{1}{n}\left(b+\frac{a}{n}\right)&, &d.
\endalignat
$$

Pre zjednodušenie označme $t = b+\frc{a}{n}$ (zdôraznime, že $t$ je celé).
Po ďalšom kroku $3\to4$ dostaneme štvoricu počtov
$$
\frac{n-1}{n}a,\quad \frac{n-1}{n}t,\quad \frac{n-1}{n}\Big(c+\frac{t}{n}\Big),\quad d + \frac{1}{n}\Big(c+\frac{t}{n}\Big).
$$
Keďže sa počty predmetov v~druhej a~tretej miestnosti už ďalej nebudú meniť,
môžeme ich porovnať už teraz a~zistiť tak, že
$$
t = c+\frac{t}{n}, \quad \text{čiže} \quad c = \frac{n-1}{n}t.
$$
Keďže $n$ a~$n-1$ sú nesúdeliteľné čísla, usúdime, že $n$ delí~$t$.

Po dosadení za $c$ môžeme štvoricu po tretej odovzdávke prepísať na
$$
\frac{n-1}{n}a,\quad \frac{n-1}{n}t,\quad \frac{n-1}{n}t,\quad d + \frac{t}{n},
$$
a~preto po poslednom kroku $4 \to 1$ dôjdeme ku štvorici
$$
\frac{n-1}{n}a+\frac{1}{n}\Big(d +\frac{t}{n} \Big),\quad \frac{n-1}{n}t,\quad \frac{n-1}{n}t,\quad
\frac{n-1}{n}\Big(d + \frac{t}{n}\Big).
$$
Keďže sa má jednať o~štyri rovnaké čísla,
porovnaním tretieho a~štvrtého z~nich zistíme, že
$$
t = d+\frac{t}{n}, \quad \text{čiže} \quad d = \frac{n-1}{n}t.
$$
Vďaka tomu môžeme záverečnú štvoricu ešte zjednodušiť na
$$
\frac{n-1}{n}a+\frac{1}{n}t,\quad \frac{n-1}{n}t,\quad \frac{n-1}{n}t,\quad \frac{n-1}{n}t.
$$
To sú štyri rovnaké čísla práve vtedy, keď platí
$$
\frac{n-1}{n}a+\frac{1}{n}t= \frac{n-1}{n}t,
\quad \text{čiže} \quad
a=\frac{n-2}{n-1}t.
$$
Keďže $a>0$, je nutne $n \ge 3$. Navyše z~nesúdeliteľnosti
čísel $n-1$ a~$n-2$ vyplýva, že $n-1$ delí~$t$.
Zo vzťahu $t = b + \frc{a}{n}$ možno teraz aj $b$ vyjadriť iba pomocou
$t$ a~$n$ ako
$$
b = \frac{(n-1)^2+1}{n(n-1)} t.
$$

Ako už vieme, obe nesúdeliteľné čísla $n$ a~$n-1$ delia číslo~$t$, takže
ho delí aj ich súčin, a~preto $t\ge n(n-1)$. Vzhľadom
na $n \ge3$ tak získavame odhad
$$
b = (\underbrace{(n-1)^2}_{\ge 4}+1) \cdot
\underbrace{\left(\frac{t}{n(n-1)}\right)}_{\ge 1} \ge 5,
$$
pričom rovnosť zrejme nastáva jedine pre $n = 3$ a~$t = 6$. Pre také
$n$, $t$ (zodpovedajúce minimálnemu $b=5$) zo skôr odvodených
vzťahov dopočítame, že $a~= 3$, $c = d = 4$. Skúška vďaka uvedenému
postupu nie je nutná.

\odpoved
V~druhej miestnosti muselo byť minimálne päť predmetov a~mohlo sa tak
stať iba v~prípade $n = 3$.



\návody
Prirodzené čísla $a$, $b$ nazveme nesúdeliteľné, ak je ich
najväčší spoločný deliteľ rovný~1, teda
$\nsd(a,b) = 1$. Pripomeňte si základné vlastnosti nesúdeliteľných čísel:
\item{a)} Po sebe idúce prirodzené čísla sú nesúdeliteľné.
\item{b)} Ak sú $a,b,c \in \Bbb{N}$ a~ak platí $\nsd(a,b) = 1$
a~$b\mid ac$, tak platí aj~$b \mid c$.
\item{c)} Ak sú $a,b,c \in \Bbb{N}$ a~ak platí $\nsd(a,b) = 1$,
$a~\mid c$ a~$b \mid c$, tak platí aj~$ab \mid c$.

Riešte podobnú úlohu pre tri miestnosti namiesto štyroch.
\endnávod
}

{%%%%%   A-I-2
Na úvod si uvedomme, že žiadne záporné číslo $m$ požiadavkám úlohy zjavne
nevyhovuje (absolútna hodnota je nezáporné číslo).

Úlohu interpretujme geometricky. Podmienka zo zadania hovorí, že graf
nejakej kvadratickej funkcie $y=x^2+ax+b$ na intervale $\langle 0, 2\rangle$
má ležať v~horizontálnom páse medzi priamkami $y=\p m$ a~$y=\m m$ (\obr).
Otázka tak znie: Do akého nejtenšieho pásu tohto druhu možno graf nejakej
takej funkcie na intervale $\langle 0, 2 \rangle$ "zovrieť"?
\insp{a65.1}%

Dobrým kandidátom na najtenší pás sa podľa \obrr1{} zdá byť funkcia
$$
f(x)=(x-1)^2 - \tfrac12 = x^2 -2x + \tfrac12,
$$
pre ktorú $a=\m2$ a~$b=\frac12$
a~ktorá, ako hneď ukážeme, na intervale $\langle 0, 2 \rangle$ spĺňa
nerovnosti $\m\frac12\le f(x)\le\frac12$.

Naozaj, vďaka vyjadreniu $f(x) =(x-1)^2 - \frac12$ sa jedná
o~nerovnosti $0\le(x-1)^2\le1$, ktoré platia súčasne práve pre $x\in\<0, 2\>$.
Kvadratická funkcia $f(x)=x^2 -2x + \frac12$ teda vyhovuje podmienke úlohy
pre $m = \frac12$.

V~druhej časti riešenia ukážeme, že pre žiadne $m < \frac12$ vyhovujúca
kvadratická funkcia neexistuje.

Kľúčovým faktom pre nás bude, že pre ľubovoľnú funkciu $f(x) = x^2 + ax
+ b$ je aspoň jeden z~rozdielov $f(0) - f(1)$ a~$f(2) - f(1)$ väčší či
rovný jednej. Z~toho vyplynie, že šírka~$2m$ zvierajúceho pásu\footnote{Pozri
geometrickú interpretáciu z~úvodu riešenia.} musí byť väčšia alebo rovná jednej,
a~hodnoty $m < \frac12$ tak naozaj môžeme vylúčiť.
Ak je totiž napríklad $f(0) - f(1) \ge 1$ (v~prípade $f(2) - f(1) \ge 1$ by sme
postupovali podobne), dostaneme želaný odhad $2m \ge 1$ ľahko
zo všeobecne platnej trojuholníkovej nerovnosti $|a-b| \le |a| +|b|$:
$$
1 \le |f(0) - f(1)| \le |f(0)| + |f(1)| \le 2m.
$$

Na zakončenie celého riešenia preto ostáva dokázať platnosť aspoň jednej
z~nerovností $f(0) - f(1) \ge 1$ a~$f(2) - f(1) \ge 1$ pre ľubovoľnú
$f(x) = x^2 + ax + b$.
Keďže
$$
f(0) = b, \quad f(1) = 1 + a~+ b, \quad f(2) = 4 + 2a + b,
$$
platí
$$
\aligned
f(0) - f(1)=-1-a~&\ge 1 \quad \Leftrightarrow \quad a~\le-2,\\
f(2) - f(1)=\phantom- 3+a~&\ge 1 \quad \Leftrightarrow \quad a~\ge-2.
\endaligned
$$
Aspoň jedna z~nerovností $f(0) - f(1) \ge 1$ či $f(2) - f(1)\ge 1$ teda
platí vždy (bez ohľadu na voľbu čísel $a$, $b$).

\odpoved
Hľadaná minimálna hodnota $m$ je $\frac12$.


\návody
Určte najmenšie reálne číslo~$m$, pre ktoré platí $|x^2-2| \le m$
pre každé $x \in \langle \m2, 2 \rangle$.

Ukážte, že pre každú funkciu $f(x) = x^2+ax+b$ existujú čísla $u,
v~\in \Bbb{R}$ také, že $f(x) = (x-u)^2 + v$. Graf
každej takej funkcie je teda posunutím paraboly $y = x^2$.

Sú dané tri reálne čísla $a$, $b$, $c$, pričom každé dve sa líšia
aspoň o~1. Ukážte, že ak nejaké $m \in \Bbb{R}$ spĺňa
$|a|\le m$, $ |b| \le m$, $ |c| \le m$, tak $m \ge 1$.

Dokážte, že pre ľubovoľnú funkciu tvaru $f(x) = x^2 + ax + b$ platí
aspoň jedna z~nerovností $f(\m1) - f(0) \ge 1$, $f(1) - f(0) \ge 1$.
Platí záver aj vtedy, keď nahradíme trojicu čísel $\m1$, $0$, $1$ trojicou
čísel $t-1$, $t$, $t+1$ pre ľubovoľné $t \in \Bbb{R}$?

\D
Rozhodnite, či existujú čísla $a, b, c \in \Bbb{R}$ také, že
rovnica $ax^2 + bx + c + t = 0$ má dva reálne korene, nech zvolíme parameter
$t \in\Bbb{R}$ akokoľvek.

Nech $a$, $b$, $c$ sú reálne čísla. Dokážte, že aspoň jedna z~rovníc
$$
\align
x^2 + (a-b)x + (b-c) &= 0, \\
x^2 + (b-c)x + (c-a) &= 0, \\
x^2 + (c-a)x + (a-b) &= 0,
\endalign
$$
má reálny koreň. [Ruská MO 2007. Uvážte, že stačí, aby niektorá z~troch
kvadratických funkcií z~ľavých strán mala nekladnú hodnotu pre $x = 0$. Môže sa
stať, že by hodnoty v~nule vyšli všetky tri záporné?]

Nech $P(x)$ a~$Q(x)$ sú kvadratické trojčleny, pre ktoré platí,
že rovnica $P(Q(x)) = 0$ má korene $\m22$, $7$, $13$. Určte štvrtý koreň
tejto rovnice, ak viete, že je celočíselný. [Ukážte, že z~hodnôt
$Q(\m22)$, $Q(7)$, $Q(13)$ musia byť dve zhodné. Čo to potom znamená pre os
súmernosti paraboly~-- grafu funkcie $Q(x)$?]
\endnávod
}

{%%%%%   A-I-3
Vidíme, že daná kružnica~$k$ je Tálesovou kružnicou s~priemerom~$EF$
a~stredom~$D$ (\obr). Pritom trojuholník $EFC$ je zrejme pravouhlý rovnoramenný,
preto $|EC|=|FC|$. Ukážeme, že trojuholníky $EPC$ a~$FQC$ sú zhodné, čím bude
tvrdenie úlohy dokázané.
\insp{a65.2}%

Uhly $CEQ$ a~$CFQ$ sú zhodné, pretože sa jedná o~obvodové uhly nad tetivou~$CQ$
kružnice~$k$. Napokon, oba uhly $ECF$ a~$ACB$ sú zhodné (pravé), preto sú
zhodné aj ich neprekrývajúce sa časti, čiže uhly $ECA$ a~$FCB$
(a~teda aj uhly $ECP$ a~$FCQ$). Trojuholníky $EPC$ a~$FQC$ sa teda naozaj
zhodujú podľa vety {\it usu}.

\návody
Zopakujte si vetu o~stredovom a~obvodovom uhle.

Daný je štvorec $ABCD$. Na kratšom oblúku~$AB$ jemu opísanej kružnice
zvolíme bod~$X$ tak, že $|\uhol ADX| = 30^\circ$. Priesečníky
úsečiek $XC$ a~$XD$ so stranou~$AB$ označme postupne $Y$ a~$Z$. Určte
veľkosti vnútorných uhlov v~trojuholníku $XYZ$.

\D
Daný je štvorec $ABCD$. Na kratšom oblúku~$AB$ jemu opísanej kružnice
zvolíme bod~$X$. Priesečník úsečky~$XC$ so stranou~$AB$ označme
$Y$ a~priesečník úsečky~$XD$ s~uhlopriečkou~$AC$ označme $Z$. Dokážte, že
$YZ \perp AC$. [Nájdite skrytú štvoricu bodov, ktoré ležia na jednej
kružnici.]

Na stranách $BC$ a~$CD$ štvorca $ABCD$ zvoľme postupne body $K$
a~$L$ tak, že $|\uhol LAK| = 45^\circ$. Dokážte, že $|BK| + |DL| = |KL|$.
[Otočte bod~$K$ o~$90^\circ$ okolo~$A$ a~použite zhodnosti vhodných
trojuholníkov.]
\endnávod
}

{%%%%%   A-I-4
Ukážeme najskôr, že v~prípade $k=3$ vyhrá Jana. Pracujme so štvorcami $A_1$,
$A_2$ a~$A_3$ o~rozmeroch $3 \times 3$ (\obr). Štvorec $3 \times 3$
považujeme za {\it pokrytý}, ak sa nachádza v~každom jeho riadku
aj stĺpci práve jedna jednotka. Ak Jana pokryje štvorce $A_1$, $A_2$
a~$A_3$ bez toho, aby zahrala do iných štvorcov, zabezpečí si výhru, pretože všetky riadkové
aj stĺpcové súčty budú rovné nepárnemu číslu~1.
\insp{a65.3}%

Je zrejmé, že ak je po ťahu Nely v~niektorom štvorci $3 \times 3$ zapísaná nanajvýš jedna
nula (a~žiadna jednotka), môže Jana vďaka hodnote $k=3$ tento štvorec trojicou svojich
ťahov pokryť. Stratégia Jany je teda nasledujúca: Ak Nela svojim ťahom zahrá do
niektorého nepokrytého štvorca $A_1$, $A_2$ alebo~$A_3$, pokryje vzápätí Jana
tento štvorec. V~opačnom prípade pokryje Jana ľubovoľný z~doposiaľ nepokrytých štvorcov
$A_1$, $A_2$ a~$A_3$. Po prvých troch trojiciach ťahov Jana takto vždy vyhrá.

Tvrdíme, že v~prípadoch $k\in\{1, 2\}$ má vyhrávajúcu stratégiu Nela.
Najskôr si uvedomme, že
ak nejaký ťah ponúka Jane výhru (nazvime ho {\it víťazný\/} ťah), znamená
to, že pred jeho zahraním je nepárny súčet presne v~ôsmich stĺpcoch
aj ôsmich riadkoch, pričom onen víťazný ťah je ťahom Jany na priesečník jediného
"párneho" riadku s~jediným "párnym" stĺpcom. Z~toho vyplýva, že
ak má niekedy Jana víťazný ťah, je potom taký ťah presne jeden.

Teraz je zrejmé, ako Nela dosiahne výhru v~prípade $k=1$. Ak má Jana po svojom ťahu
k~dispozícii víťazný ťah, Nela na ono políčko napíše nulu, a~Jana tak o~svoju
(jedinú) možnosť výhry v~ďalšom ťahu príde. Ak naopak Jana nemá po svojom ťahu
k~dispozícii víťazný ťah, pripíše Nela ďalším ťahom nulu kamkoľvek. Tým sa
súčty v~riadkoch ani stĺpcoch nemenia, takže Jana ďalším ťahom nevyhrá.
Takto Nela dosiahne vyplnenie celej tabuľky bez toho, aby Jane dovolila vyhrať.

V~prípade $k=2$ bude hrať Nela podľa rovnakej stratégie ako pri $k=1$,
a~zabráni tak tomu, aby Jana mohla niekedy vyhrať po prvom zo svojej dvojice ťahov.
V~tom druhom však
Jana nikdy vyhrať nemôže, lebo po jeho zahraní bude v~tabuľke spolu párny
počet jednotiek, a~bude tak vylúčené, že by nepárny počet jednotiek bol v~každom
z~(nepárneho počtu) deviatich riadkov.

\odpoved
Najmenšia hodnota $k$, pre ktorú má Jana vyhrávajúcu stratégiu, je $k~= 3$.


\návody
Riešte danú hru najskôr v~tabuľke $3 \times 3$.

Na čarovnom strome vyrástlo 25~citrónov a~30~pomarančov. Sadár odtrhne každý
deň dva plody, potom cez noc vždy na strome vyrastie jeden nový plod,
a~to pomaranč (resp. citrón), ak boli odtrhnuté plody rovnaké (resp. rôzne).
Aký plod vyrastie na strome posledný? [Citrón~-- ich počet je
totiž po každej noci nepárny.]

\D
Simona a~Lenka hrajú hru. Pre dané celé číslo~$k$ také, že $0 \le
k\le 64$, vyberie Simona $k$~políčok šachovnice $8\times 8$ a~každé
z~nich označí krížikom. Lenka potom šachovnicu nejakým spôsobom vyplní
tridsiatimi dvoma dominovými kockami. Ak je počet kociek pokrývajúcich dva
krížiky nepárny, vyhráva Lenka, inak vyhráva Simona. V~závislosti od~$k$
určte, ktoré z~dievčat má vyhrávajúcu stratégiu.
[\hbox{64--C--I--3}]

V~ľavom hornom rohu šachovnice $8 \times 8$ stojí figúrka kráľa.
Dvaja hráči sa striedajú v~ťahoch, pričom každý svojim ťahom
(legálnym šachovým ťahom) posunie figúrku na miesto, na ktorom ešte nestála. Kto
nemá kam urobiť ťah, prehral. Dokážte, že hráč hrajúci ako prvý má vyhrávajúcu
stratégiu. [Rozdeľte šachovnicu na obdĺžničky $2\times 1$ a~nájdite pre
prvého hráča stratégiu, v~ktorej do žiadneho obdĺžnička neťahá ako prvý.]

V~ľavom hornom rohu šachovnice $8 \times 8$ stojí figúrka jazdca.
Dvaja hráči sa striedajú v~ťahoch, pričom každý svojim ťahom
(legálnym šachovým ťahom) posunie figúrku na miesto, na ktorom ešte nestála. Kto
nemá kam urobiť ťah, prehral. Dokážte, že začínajúci hráč má vyhrávajúcu
stratégiu. [Rozdeľte šachovnicu na obdĺžničky $2\times 4$ a~v~nich
políčka rozdeľte do dvojíc s~rovnakým úmyslom ako v~úlohe~D2.]
\endnávod
}

{%%%%%   A-I-5
Keďže uhol $ABC$ je vonkajším uhlom rovnoramenného trojuholníka~$XKB$
s~hlavným vrcholom~$B$ (\obr), je zrejme
priamka~$KX$ rovnobežná s~osou uhla $ABC$.
\insp{a65.4}%

Z~hodnoty pomeru $|LB|:|LK| = 2:3$ potom ale vyplýva, že spomenutá os uhla
$ABC$ prechádza ťažiskom trojuholníka $KLM$. Ak totiž označíme $LL_1$
jeho ťažnicu a~$L_2$ jej priesečník s~osou uhla $ABC$,
tak z~podobnosti trojuholníkov $LBL_2$ a~$LKL_1$
(podľa vety {\it uu}) získame
$$
\frac{|LL_2|}{|LL_1|} = \frac{|LB|}{|LK|} = \frac23.
$$
Bod~$L_2$ teda leží v~dvoch tretinách ťažnice~$LL_1$ od vrcholu~$L$, takže je
naozaj ťažiskom trojuholníka $KLM$.

Zo symetrie zadania úlohy vyplýva, že aj os uhla $BCA$ prechádza ťažiskom
trojuholníka~$KLM$. Keďže priesečník osí vnútorných uhlov trojuholníka
je stredom jeho kružnice vpísanej, je tvrdenie úlohy dokázané.



\návody
Dokážte, že v~každom trojuholníku $ABC$ je os vnútorného uhla
kolmá na os vonkajšieho uhla pri tom istom vrchole.

Daný je trojuholník $ABC$ a~jeho ťažisko~$T$. Rovnobežka so stranou~$BC$
vedená bodom~$T$ oddelí menší trojuholník $ADE$. Určte, aký je
pomer obsahov trojuholníkov $ABC$ a~$ADE$.

\D
V~trojuholníku $ABC$ označme $I$ stred kružnice vpísanej a~$I_a$
stred kružnice pripísanej strane~$BC$. Dokážte, že
\item{a)} body $B$, $C$, $I$, $I_a$ ležia na kružnici s~priemerom~$II_a$
[použite výsledok úlohy~N1],
\item{b)} stred úsečky~$II_a$ leží na osi úsečky~$BC$,
\ite{c)} body dotyku kružnice vpísanej a~kružnice pripísanej strane~$BC$ so stranou~$BC$
sú súmerne združené podľa osi úsečky~$BC$.

Daný je trojuholník $ABC$ s~tupým uhlom pri vrchole~$C$. Os $o_1$
úsečky~$AC$ pretína stranu~$AB$ v~bode~$K$, os $o_2$ úsečky~$BC$ pretína
stranu~$AB$ v~bode~$L$. Priesečník osí $o_1$ a~$o_2$ označme~$O$. Dokážte,
že stred kružnice vpísanej do trojuholníka $KLC$ leží na kružnici opísanej
trojuholníku $OKL$.
[64--A--II--1]

V~tetivovom štvoruholníku $ABCD$ označme $L$, $M$ stredy kružníc
vpísaných postupne do trojuholníkov $BCA$, $BCD$. Ďalej označme $R$
priesečník kolmíc vedených z~bodov $L$ a~$M$ postupne na priamky $AC$
a~$BD$. Dokážte, že trojuholník $LMR$ je rovnoramenný.
[56--A--III--2]
\endnávod
}

{%%%%%   A-I-6
Exponent (prípadne aj nulový) prvočísla~$p$ v~prvočíselnom rozklade čísla~$n$ budeme označovať
$v_p(n)$. Uvedomme si niekoľko zrejmých poznatkov:

\bulet Pre všetky $m$, $n$ a~každé prvočíslo $p$ platí $v_p(mn) = v_p(m)+v_p(n)$.
\bulet Pre každé prvočíslo $p$ platí $v_p(p!) = v_p(p) = 1$.
\bulet Pre každé prvočíslo $p$ platí $v_p((p+1)!) = 1$, $v_p(p+1) =0$.
\bulet Pre každé prvočíslo $p$ a~každé $n < p$ platí $v_p(n!) = v_p(n) = 0$.
\bulet Číslo $n$ je druhou mocninou prirodzeného čísla práve vtedy, keď $v_p(n)$ je párne
pre každé prvočíslo~$p$.

Označme $S=n!$ počiatočnú hodnotu súčinu na tabuli a~$S'$ jeho konečnú
hodnotu po dopísaní faktoriálov. Vďaka uvedeným vlastnostiam funkcií~$v_p$
je jasné, že ak je $n$ rovné nejakému
prvočíslu~$p$, tak bude platiť $v_p(S) = v_p(p!) = 1$ rovnako ako
$v_p(S') = 1$, pretože pripisovanie faktoriálov zastúpenie prvočísla~$p=n$
v~súčine čísel na tabuli nezvýši. Číslo~$v_p(S')$ tak bude nepárne,
a~preto $S'$ nebude druhou mocninou prirodzeného čísla.

V~druhej časti riešenia budeme naopak predpokladať, že dané číslo
$n\ge2$ nie je prvočíslo (takže $n\ge4$), a~ukážeme,
že faktoriály môžeme pripísať
tak, aby výsledný súčin
$$
S'=f_1\cdot f_2\cdot f_3\cdot\dots\cdot f_n,
$$
pričom $f_k$ je jedno z~čísel $k$ alebo $k!$ pre každé $k$,
bol druhou mocninou prirodzeného čísla. To, ako vieme, nastane práve
vtedy, keď v~súčine $S'$ bude každé prvočíslo~$p$
zastúpené s~párnym exponentom~$v_p(S')$.
Keďže samo $n$ podľa predpokladu prvočíslo nie je, súčin~$S'$
určite obsahuje iba prvočísla menšie ako~$n$.
Keďže každé také prvočíslo~$p$ nie je zastúpené v~činiteľoch
$f_1,f_2,\dots,f_{p-1}$ vôbec a~v~činiteli $f_{p}$ práve raz,
príslušný mocniteľ $v_p(S')$ je rovnaký ako mocniteľ
daného prvočísla~$p$ v~"skrátenom" súčine
$$
p\cdot f_{p+1}\cdot f_{p+2}\cdot\dots\cdot f_{n}. \tag1
$$
Ako teda zabezpečiť, aby každé prvočíslo $p<n$ bolo v~zodpovedajúcom
súčine~(1) zastúpené s~párnym exponentom? Keďže v~druhom
činiteli $f_{p+1}$ z~(1) je prvočíslo~$p$ zastúpené buď raz (to vtedy, keď
$f_{p+1}=(p+1)!$), alebo zastúpené vôbec nie je
(ak je naopak $f_{p+1}=p+1$), "správne"
zastúpenia $p$ v~súčine~(1) môžeme zabezpečiť voľbou hodnoty $f_{p+1}$,
nech sú nasledujúce hodnoty $f_{p+2},\dots,f_{n}$ zadané
akokoľvek.\footnote{Platí to samozrejme aj pre prípadné prvočíslo
$p=n-1$, keď je činiteľ $f_{p+1}=f_{n}$ v~súčine~(1) posledný; vtedy
musíme prirodzene voliť $f_n=n!$.}

Z~predchádzajúcej úvahy už vyplýva konštrukcia požadovaného výberu
faktoriálov. Najskôr ľubovoľne zvolíme hodnoty $f_k\in\{k,k!\}$ pre
všetky také $k\le n$, pre ktoré číslo $k-1$ {\it nie je
prvočíslo}. Ostatné hodnoty $f_k$, teda hodnoty $f_{p+1}$, pričom $p$
je ľubovoľné prvočíslo menšie ako~$n$, potom budeme voliť
"odzadu", \tj. od najväčšieho takého $p$ po
najmenšie.\footnote{Posledná tak bude voľba $f_3$ zodpovedajúca
najmenšiemu prvočíslu $p=2$.} Vždy, keď bude pre niektoré
prvočíslo $p<n$ na rade voľba hodnoty $f_{p+1}$,
teda druhého činiteľa v~súčine~(1), budú už jeho ostatné
činitele určené, a~tak voľbu $f_{p+1}$ urobíme "správne" v~zmysle predchádzajúceho odseku.

Tým je konštrukcia druhej mocniny~$S'$ zavŕšená a~riešenie celej
úlohy hotové.

\odpoved
Hľadané $n\ge2$ sú práve všetky zložené čísla.




\návody
Aký najmenší násobok čísla 2\,016 je druhou mocninou prirodzeného
čísla?

Pre aké najmenšie prirodzené číslo $n$ platí $2\,015 \mid n!$?

Koľkými nulami končí číslo $2\,015!$?

Pre dané prirodzené číslo~$n$ a~prvočíslo~$p$ uvažujme najväčšie nezáporné
celé číslo~$k$, pre ktoré platí $p^k~\mid n$. Toto číslo~$k$ budeme
označovať $v_p(n)$ a~hovoriť mu {\it $p$-valuácia\/} čísla~$n$. Iný pohľad na vec
je, že $v_p(n)$ označuje exponent prvočísla~$p$ v~prvočíselnom rozklade
čísla~$n$. Pre ľubovoľné prirodzené čísla $a$, $b$ dokážte nasledujúce:
\item{a)} $v_p(ab) = v_p(a) + v_p(b)$,
\item{b)} $v_p(a^b) = bv_p(a)$.
\item{c)} Prirodzené číslo~$b$ je druhou mocninou práve vtedy, keď $v_p(b)$
je párne pre každé prvočíslo~$p$.
\item{d)} $a~\mid b$ práve vtedy, keď $v_p(a) \le v_p(b)$ pre každé prvočíslo~$p$.
\item{e)} Ak je $v_p(a) \ne v_p(b)$, tak platí $v_p(a+b) = \min(v_p(a),
v_p(b))$.

\D
Zistite, pre ktoré prirodzené
čísla $n \ge 2$ je možné za niektoré z~činiteľov súčinu
$$1 \cdot 2\cdot {3\cdot \dots \cdot n}$$
dopísať výkričník tak, aby
výsledný súčin alebo jeho dvojnásobok bol rovný {\it tretej\/}
mocnine prirodzeného čísla. [Hľadané sú práve tie zložené $n$,
pre ktoré je aj číslo $n-1$ zložené.]

Ukážte, že pre každé prirodzené číslo~$n$ a~ľubovoľné prvočíslo~$p$ platí vzorec
$$
v_p(n!)= \Big\lfloor\frac{n}{p}\Big\rfloor + \Big\lfloor\frac{n}{p^2}\Big\rfloor
+ \dots + \Big\lfloor\frac{n}{p^k}\Big\rfloor +\dots.
$$

Dokážte, že $$v_p(n!) = \frac{n - s_p(n)}{p-1},$$ pričom $s_p(n)$ je
ciferný súčet čísla~$n$ zapísaného v~sústave so základom~$p$. [Zapíšte
$n$ v~sústave so základom~$p$ ako $n = a_0 + a_1p + \dots + a_kp^k$,
použite výsledok~D1 a~pre určenie koeficientov pri každej z~mocnín~$p$
potom sčítajte vhodnú geometrickú postupnosť.]

Nájdite všetky prirodzené $n$, pre ktoré $2^{n-1} \mid n!$ [Použite
výsledok predchádzajúcej úlohy.]

Dokážte, že pre ľubovoľné prirodzené čísla $m$, $n$ je výraz
$$
\postdisplaypenalty10000
\frac{(2m)! (2n)!}{n! m! (m+n)!}
$$
vždy rovný celému číslu. [MMO 1972]

Dokážte, že pre ľubovoľné prirodzené čísla $m$, $n$ platí
$$
v_p\left( \binom{n+m}{m} \right) = \frac{s_p(n)+s_p(m)-s_p(m+n)}{p-1},
$$
pričom opäť $s_p(n)$ je ciferný súčet čísla~$n$
zapísaného v~sústave so základom~$p$.
Súvisí výsledok s~počtom "prenosov" pri písomnom sčítaní v~sústave
so základom~$p$?
\endnávod
}

{%%%%%   B-I-1
Aj keď rovnica v~zadaní obsahuje tri neznáme, podarí sa nám
ju jednoznačne vyriešiť vďaka tomu, že hľadáme riešenie iba
v~množine prirodzených čísel. Pokúsime sa z~oboch zlomkov oddeliť ich
celú časť (čo je v~prípade číselného zlomku jednoduché):
$$
% \label {eq: pqr1}
\frac {k+m+klm} {lm+1} = \frac {k~\left (1+lm \right)+m} {lm+1} = k+\frac {m} {lm+1}
,\quad \frac {2\,051} {404}= 5\,\frac {31} {404}.
\tag1
$$
Keďže $0 <m <lm+1$, je
$$
0 <\frac {m} {lm+1} <1,
$$
a~preto musí byť $k~= 5$. Z~rovností~(1) tak pre zlomkové časti oboch čísel
dostávame
$$
% \label {eq: pqr2}
\frac {m} {lm+1} = \frac {31} {404}.
\tag2
$$

Zlomok na pravej strane (2) je v~základnom tvare a~podobne
aj zlomok na ľavej strane (čísla $m$ a~$lm+1$ sú zjavne nesúdeliteľné).
Z~tejto rovnosti zlomkov tak vychádza rovnosť čitateľov aj menovateľov:
$m = 31$ a~$lm+1 = 404$, odkiaľ už ľahko dopočítame
$l = 13$. Súčin $klm$ tak môže nadobúdať jedinú
hodnotu $5 \cdot 13 \cdot 31 = 2\,015$.

\ineriesenie
Z~predchádzajúceho riešenia využijeme úvodný postup a~rovnicu~(2) prepíšeme
s~prevrátenými zlomkami ako
$$
\frac {lm+1} {m} = \frac {404} {31}.
$$
Teraz zopakujeme postup zo začiatku predošlého riešenia a~z~oboch zlomkov oddelíme ich
celú časť:
$$
\frac {lm+1} {m} = l+\frac 1m ,\quad \frac {404} {31}= 13\,\frac 1 {31}.
$$
Z~toho vidíme, že nutne $l = 13$ a~$m = 31$.

\ineriesenie
Zo zadanej rovnice vyjadríme neznámu~$k$ pomocou neznámych $l$ a~$m$,
v~prvom kroku sa pritom zbavíme zlomkov vynásobením oboma menovateľmi:
$$
\align
404 (k+m+klm) =& 2\,051 (lm+1),\\
404k (lm+1)+404m=&2\,051 (lm+1), \\
k=&\frac {2\,051 (lm+1)-404m} {404 (lm+1)}. %\label {eq: pqr3}
\tag3
\endalign
$$

Zlomok na pravej strane musí byť prirodzené číslo, skúmajme teda, či
oba činitele v~jeho menovateli (404 aj~$lm+1$) delia jeho čitateľa.
Čísla $2\,051 = 5 \cdot 404+31$ a~404 sú nesúdeliteľné
a~$404m$ je zrejme deliteľné číslom 404, preto 404 musí deliť $lm+1$.

Podobne čísla $lm+1$ a~$m$ sú nesúdeliteľné, teda $lm+1$ musí
v~čitateli deliť číslo~404. Ak sa dve prirodzené čísla delia
navzájom, musia byť rovnaké.\footnote{Ak prirodzené číslo $a$ delí
prirodzené číslo $b$, je $a\le b$.}
Dostávame tak rovnosť $lm+1 = 404$, ktorá po dosadení do~(3) dáva
$$
% \align
k=\frac {2\,051 \cdot 404-404m} {404 \cdot 404}
=\frac {2\,051-m} {404}. \tag4 %\label {eq: pqr4}.
% \endalign
$$
Z~rovnosti $lm+1 = 404$ však tiež vyplýva, že $m <404$. Navyše číslo $k$ je
prirodzené, takže $m$ môže byť iba zvyšok po delení čísla 2\,051
číslom~404, \tj. $m = 31$. Spätným dosadením do~(4) dostaneme $k~= 5$
a~z~rovnice $lm+1 = 31l+1 = 404$ vyjde $l = 403/31 = 13$. Jediné vyhovujúce
riešenie je $(k, l, m) = (5, 13, 31)$, a~teda $klm = 2\,015$.

\návody
V~prirodzených číslach vyriešte rovnicu $\frac 1{p+\frc
1q} = \frac {2015} {2016}$. [$p = 1$, $ q = 2\,015$]

Pripomeňte si dôležitý poznatok o~deliteľnosti celých čísel: ak delí
číslo~$x$ súčin~$yz$ a~ak sú pritom čísla $x$ a~$y$ nesúdeliteľné, tak
číslo~$x$ delí samo číslo~$z$. Využite potom toto pravidlo na zdôvodnenie
takéhoto záveru: ak pre prirodzené čísla $a$, $b$, $c$, $d$ sú
oba zlomky $\frac ab = \frac cd$ v~základnom tvare, platí $a~= c$ a~$b = d$.

Dokážte, že ak pre prirodzené čísla $a$, $b$, $k$, $l$ platí, že $ka$ delí
$b$ a~$lb$ delí $a$, tak $k= l = 1$ a~$a=b$. [Deliteľ nemôže byť
(v~absolútnej hodnote) väčší ako delenec, preto $ka \le b$ a~$lb \le a$, takže
$kla \le lb \le a$, teda $kl \le 1$ a~odtiaľ $k~= l = 1$. Nakoniec spätne
$a\le b \le a$, čo nastáva, len ak $a~= b$.]

\D
Nájdite aspoň jedno riešenie rovnice
$$\frac {k+m+klm} {lm+1} = \frac {2051} {404}$$ v~racionálnych číslach, pre ktoré
je hodnota súčinu $klm$ rovná~2\,016. [Dosadením
$klm = 2\,016$ a~$lm = 2\,016/k$ dostaneme voľbou $k~= 1$ jedno z~riešení
$(k,l,m) = (1, {2\,016 \cdot 404}/(1\,647 \cdot 2\,017)$, $ 2\,017 \cdot 1\,647/404)$.]
\endnávod
}

{%%%%%   B-I-2
Označme $z$ číslo, ktoré zakrýva ľavý horný roh doštičky. Celá
doštička musí ležať vnútri danej tabuľky, preto hodnoty~$z$ môžu byť
iba čísla vpísané v~prvých ôsmich riadkoch a~v~prvých ôsmich
stĺpcoch tabuľky (ak by bolo napríklad $z~= 10$, doštička by
prečnievala, teda by nemohla zakrývať 16~čísel tabuľky).

Prvých 8~riadkov tabuľky obsahuje čísla od~1
po~88, z~nich musíme ešte vylúčiť čísla v~posledných troch stĺpcoch.
Všimnime si, že čísla v~každom stĺpci dávajú po delení jedenástimi
taký istý zvyšok.
% To súvisí s~tým, že tabuľka má 11~stĺpcov.
Posledné tri stĺpce zľava (=~prvé tri sprava) tak obsahujú čísla,
ktoré po delení jedenástimi dávajú zvyšky 9, 10 a~0;
sú to čísla 9, 10, 11 (prvý riadok), 20, 21, 22 (druhý riadok), atď.
až~86, 87, 88 (ôsmy riadok).

Takto pripravení môžeme vypočítať súčet čísel, ktoré doštička zakryje.
Zakryté čísla sú $z$, $z+1$, $z+2$, $z+3$ (prvý riadok doštičky),
$z+11$, $ z+12$, $ z+13$, $ z+14$ (druhý riadok doštičky), $z+22$, $ z+23$, $ z+24$,
$z+25$ (tretí riadok doštičky) a~$z+33$, $ z+34$, $ z+35$, $ z+36$ (štvrtý riadok
doštičky) a~ich súčet je
$$16z+288 = 16 (z+18) = 4^2(z+18).$$

Ak je tento súčet druhou
mocninou nejakého celého čísla, musí byť $z+18$ druhou mocninou
nejakého celého čísla~$n$. Už vieme, že $1\le z\le88$, a~teda $19\le
z+18 = n^2\le18+88 = 116$. Tým zabezpečíme, že horný ľavý roh doštičky
položíme na políčko v~prvých ôsmich riadkoch. Pre prirodzené číslo~$n$, pričom
$19\le n^2\le116$, prichádzajú do úvahy hodnoty $n \in\{5, 6, 7, 8, 9, 10\}$.
Dopočítaním hodnôt $z= n^2-18$ dostávame zodpovedajúce $z\in\{7, 18, 31, 46, 63, 82\}$.

Musíme ešte preveriť, či niektoré z~týchto čísel
neležia v~posledných troch stĺpcoch tabuľky. Dopočítame preto zvyšky čísel po
delení jedenástimi a~zistíme, že musíme dodatočne vylúčiť hodnotu $z= 31$
so~zvyškom~9.

Doštičku možno položiť požadovaným spôsobom na
päť rôznych pozícií, ktoré charakterizuje číslo zakryté ľavým horným
rohom doštičky, a~to $z~\in \{7, 18, 46, 63, 82\}$. V~týchto prípadoch bude
súčet čísel zakrytých políčok $16 (z+18) \in \{16 \cdot 25, 16 \cdot 36,
16 \cdot 64,\allowbreak 16 \cdot 81, 16 \cdot 100\}$.

\ineriesenie
Ak položíme doštičku na tabuľku tak, že ľavý horný roh doštičky zakrýva
číslo~1, bude súčet zakrytých čísel
$$
s~= 1+2+3+4+12+13+14+15+23+24+25+26+34+35+36+37 = 304 = 16 \cdot 19.
$$
Aby doštička zostala ležať celá v~štvorcovej tabuľke, môžeme doštičku posunúť
nanajvýš o~8~stĺpcov doprava a~podobne nanajvýš o~8~riadkov nadol. Pri
každom posunutí doštičky doprava o~jeden stĺpec sa každé zakryté číslo
zväčší o~1, takže súčet čísel zakrytých doštičkou sa zvýší o~16.
Podobne zvážime, čo spôsobí posun doštičky o~jeden riadok
nadol~-- vtedy sa každé zakryté číslo zväčší o~11, a~súčet všetkých
zakrytých políčok sa teda zväčší o~$11 \cdot 16$.

Číslo~$s$ je teda deliteľné 16 a~každý pohyb doštičky
deliteľnosť~16 zachová, preto bude súčet čísel zakrytých doštičkou vždy
deliteľný~16. Ak má byť tento súčet druhou mocninou celého čísla,
bude to práve vtedy, ak bude aj jeho šestnástina druhou mocninou celého
čísla (keďže $16 = 4^2$). Stačí teda uvažovať iba šestnástiny
súčtov čísel zakrytých doštičkou.

Teraz vytvoríme tabuľku $8 \times 8$, do jej políčok vpíšeme
šestnástiny súčtov čísel zakrytých doštičkou s~ľavým horným políčkom
zakrývajúcim zodpovedajúce políčko danej tabuľky.
V~jej ľavom hornom rohu bude
číslo~19 ($=\frac 1 {16} \cdot 304$), pri pohybe doprava zväčšíme číslo
o~1 a~pri pohybe nadol o~11:
$$
\vbox{\let\\=\cr\everycr{\noalign{\hrule}}\offinterlineskip
\def\strut{\vrule width 0pt height1.1em depth.55em\relax}
\halign{\strut\vrule#&&\hbox to1.8em{\hss#\unskip\hss}\vrule\cr
&19&20&21&22&23&24&{\bf 25}&26 \\
&30&31&32&33&34&35&{\bf 36}&37 \\
&41&42&43&44&45&46&47&48 \\
&52&53&54&55&56&57&58&59 \\
&63&{\bf 64}&65&66&67&68&69&70 \\
&74&75&76&77&78&79&80&{\bf 81} \\
&85&86&87&88&89&90&91&92 \\
&96&97&98&99&{\bf 100}&101&102&103 \\
}}
$$
Nakoniec stačí spočítať, koľko z~týchto čísel je druhou mocninou
celého čísla. Takých čísel je práve päť a~sú zvýraznené polotučným písmom
(25, 36, 64, 81 a~100).

\návody
Do štvorcovej tabuľky veľkosti $5 \times 5$ sme vpísali prirodzené čísla
$1, 2, \dots, 25$ postupne po riadkoch zľava doprava a~zhora nadol.
Štvorcovou doštičkou veľkosti $2 \times 2$ sme všetkými možnými spôsobmi
zakryli štyri políčka. \\
\item{1.} Aký najmenší a~aký najväčší súčet môžu mať štyri zakryté čísla? [16, 88]
\item{2.} Koľkými spôsobmi môžeme takto doštičku položiť? [16]
\item{3.} Bude súčet štyroch zakrytých čísel vždy deliteľný štyrmi? [Áno]
\item{4.} Koľkokrát bude súčet zakrytých štyroch čísel druhou mocninou celého čísla?~[3-krát]

Do políčok štvorčekovej mriežky $11{\times}11$ sme postupne zľava doprava a~zhora nadol
zapísali čísla $1,2,\dots,121$. Štvorcovou doskou $3{\times}3$ sme všetkými
možnými spôsobmi zakryli presne deväť políčok. V~koľkých prípadoch bol súčet deviatich zakrytých
čísel druhou mocninou celého čísla? [62--B--S--2]

\D
V~jednom políčku šachovnice $8\times8$ je napísané "$\m$" a~v~ostatných políčkach~"$\p$".
V~jednom kroku môžeme zmeniť na opačné súčasne všetky štyri znamienka v~ktoromkoľvek štvorci~
$2\times2$ na šachovnici. Rozhodnite, či po určitom počte krokov môže
byť na šachovnici oboch znamienok rovnaký počet. [64--C--II--2]

V~každom políčku tabuľky $8\times8$ je napísané jedno nezáporné celé číslo tak, že každé
dve čísla, ktoré sú na políčkach súmerne združených podľa jednej či druhej uhlopriečky, sú
rovnaké. Súčet všetkých 64~čísel je $1\,000$, súčet 16~čísel na uhlopriečkach je~$200$. Dokážte, že
súčet čísel v~každom riadku aj stĺpci tabuľky je nanajvýš~$300$. Platí rovnaký záver aj pre číslo~$299$? [63--B--II--4]
\endnávod
}

{%%%%%   B-I-3
Majme také dve kružnice, ktoré spĺňajú predpoklady úlohy (\obr).
Zrejme stred~$S_1$ leží na osi uhla $BAC$ a~stred~$S_2$ na osi uhla $ABC$.
\insp{b65.1}%
Ďalej si uvedomme, že veľkosť polomeru~$r_1$ kružnice~$k_1$ je priamo
úmerná dĺžke úsečky~$AS_1$ a~podobne veľkosť~$r_2$ priamo úmerná
dĺžke úsečky~$BS_2$. Keď zväčšíme polomer jednej z~kružníc, musí sa
nutne polomer druhej kružnice zmenšiť.

Kružnica~$k_2$ nemôže mať polomer väčší ako najväčšia kružnica, ktorú
možno do trojuholníka $ABC$ vpísať. Takou kružnicou je zrejme kružnica~$k$
do trojuholníka $ABC$ vpísaná. A~naopak najmenší polomer bude mať kružnica~$k_2$,
ak zvolíme $k_1=k$. (Že v~oboch opísaných prípadoch pre $k_2=k$ aj pre $k_1=k$
existuje príslušná "vpísaná" kružnica $k_1$, resp.~$k_2$, je vcelku zrejmé.)

Stačí teda vypočítať polomer~$r$ kružnice~$k$ do trojuholníka $ABC$ vpísanej
a~polomer kružnice~$k_2$, ktorá sa dotýka kružnice~$k$ a~strán $AB$ a~$BC$
daného trojuholníka.

Polomer~$r$ vpísanej kružnice vypočítame napríklad zo vzorca $2S_{ABC} =ro$, pričom
$S_{ABC}$ označuje obsah trojuholníka~$ABC$ a~$o$ jeho obvod.\niedorocenky{\footnote{Iný postup
využívajúci pravouhlosť trojuholníka $ABC$ je predmetom dopĺňajúcej úlohy.}}
Obsah daného pravouhlého trojuholníka $ABC$
s~preponou~$AB$ je pri zvyčajnom označení dĺžok strán rovný $\frac12ab$.
Prepona v~trojuholníku $ABC$ má (v~centimetroch) podľa Pytagorovej vety veľkosť
$c = \sqrt {a^2+b^2} = \sqrt {3^2+4^2} = 5$. Maximálny polomer kružnice~$k_2$
je teda
$$
r = \frac {2S_{ABC}} o= \frac {ab} {a+b+c} = \frac {3 \cdot 4} {3+4+5} = 1.
$$

Pre výpočet polomeru~$r_2$ kružnice~$k_2$, ktorá sa dotýka kružnice~$k$
a~strán $AB$ a~$BC$, označme $D$ a~$E$ body, v~ktorých sa kružnice $k$ a~$k_2$ dotýkajú strany~$AB$,
a~$F$, $G$ dotykové body kružnice~$k$ postupne so stranami $BC$ a~$AC$ (\obr).
\insp{b65.2}%
Keďže daný trojuholník je pravouhlý, je $S_1FCG$ štvorec so stranou dĺžky $r=1$,
takže $|BF| = |BD| = 2$ a~podľa Pytagorovej vety $|BS_1|=\sqrt{5}$.
Z~podobnosti pravouhlých trojuholníkov $BES_2$ a~$BDS_1$ potom vyplýva
$$
{r_2\over|BS_2|} = {r\over|BS_1|},\quad\text{čiže}\quad
{r_2\over \sqrt{5} - r_2 - 1} = {1\over\sqrt{5}}.
$$
Po úprave tak pre hľadanú hodnotu neznámej~$r_2$ dostaneme lineárnu rovnicu
$$
r_2\bigl(\sqrt5+1\bigr) = \sqrt5-1,
$$
ktorú ešte zjednodušíme vynásobením $\sqrt5-1$. Zistíme tak, že
najmenšia možná hodnota polomeru kružnice~$k_2$ je rovná
$$
r_2 = \frac {3- \sqrt5} 2.
$$


\návody
Kružnice $k_1 (S_1; r_1)$ a~$k_2 (S_2; r_2)$ sa navzájom zvonka dotýkajú,
ich spoločnú vonkajšiu dotyčnicu označme $P_1P_2$, pričom $P_1 \in k_1$
a~$P_2 \in k_2$. Presvedčte sa, že platí
$(r_1+r_2)^2 = |P_1P_2|^2+(r_1-r_2)^2$. [Rovnica je Pytagorova veta pre
pravouhlý trojuholník s~preponou~$S_1S_2$.]

Kružnica vpísaná do trojuholníka $ABC$ sa dotýka jeho strán $BC$, $AC$,
$AB$ postupne v~bodoch $K$, $L$, $M$. Dokážte rovnosti
$|AL| = |AM| = \frac12(|AB|+|AC|-|BC|)$, $|BK| = |BM| = \frac12(|BC|+|AB|-|AC|)$
a~$|CK| = |CL| = \frac12(|AC|+|BC|-|AB|)$. [Body dotyku vpísanej kružnice so
stranami rozdeľujú hranicu trojuholníka na tri dvojice úsečiek rovnakých dĺžok.]

\D
Dokážte, že v~pravouhlom trojuholníku $ABC$ s~odvesnami dĺžok $a$, $b$
a~preponou dĺžky~$c$ je priemer vpísanej kružnice rovný $a+b-c$. [Ak sú
$D$ a~$E$ postupne body dotyku vpísanej kružnice so~stredom~$S$
s~odvesnami $BC$ a~$AC$, je $SDCE$ štvorec, takže
$|CD| = \frac12({a+b-c}) = |SD| = r$.]

Polomer vpísanej kružnice trojuholníka~$ABC$ je $r$. Zostrojme tri rôzne
dotyčnice vpísanej kružnice rovnobežné so stranami trojuholníka.
Polomery vpísaných kružníc troch malých "odrezaných" trojuholníkov
označme $r_A$, $r_B$, $r_C$ podľa vrcholov trojuholníka. Dokážte,
že $r_A+r_B+r_C = r$. [Z~podobnosti malého trojuholníka k~$ABC$ je
$r_A/r = (v_a-2r)/v_a$, pričom $v_a$ označuje veľkosť výšky z~vrcholu~$A$
v~trojuholníku $ABC$. Podobné rovnice platia aj~pre ostatné vrcholy,
takže ostáva ukázať, že $1/v_a+1/v_b+1/v_c = 1/r$. Tu využijeme vzorec
$ro = 2S= av_a= bv_b= c v_c$, pričom $o=a+b+c$.]
\endnávod
}

{%%%%%   B-I-4
Označme $n$ hľadané číslo a~nech $2^k$ je najvyššia mocnina
dvojky, ktorá číslo~$n$ delí. Ku každému nepárnemu deliteľu~$d$ čísla~$n$
(vrátane $d=1$)
môžeme priradiť práve $k$~rôznych párnych deliteľov $2d, 2^2d,\dots, 2^kd$.
Dostaneme tak všetky párne delitele čísla~$n$;
navyše rôznym nepárnym deliteľom priradíme rôzne párne
delitele (keďže z~rovnice $2^{k_1} d_1 = 2^{k_2} d_2$ pre prirodzené
čísla $k_1$, $k_2$, $d_1$, $d_2$, pričom $d_1$ a~$d_2$ sú nepárne,
vyplýva, že $d_1 = d_2$ a~$k_1 = k_2$). Vidíme tak, že ak má číslo~$n$ práve
$N$~nepárnych deliteľov, má práve $kN$~deliteľov párnych.

Podľa zadania má platiť $kN = N+3$, čiže $N (k-1) = 3$. Číslo~1 je nepárnym
deliteľom každého prirodzeného čísla, preto $N \ge 1$. Máme teda iba dve
možnosti:
\ite1.
$N = 1$ a~$k-1 = 3$.\hfil\break
V~tomto prípade má $n$ jediného nepárneho deliteľa,
a~je teda mocninou dvojky. Navyše najvyššia mocnina, ktorá ho delí, je
$2^k= 2^4 = 16$, teda $n = 16$. Párne delitele čísla 16 sú 2, 4, 8 a~16,
hľadaný podiel je
$$
\frac {2+4+8+16} {1} = 30.
$$
\ite2.
$N = 3$ a~$k-1 = 1$.\hfil\break
V~tomto prípade má $n$ tri nepárne delitele
a~najvyššia mocnina dvojky, ktorá ho delí, je $2^k~= 2^2 = 4$. Pokiaľ by malo
číslo~$n$ vo svojom prvočíselnom rozklade dve rôzne nepárne prvočísla
$p$ a~$q$, malo by aspoň štyri nepárne delitele 1, $p$, $q$ a~$pq$,
čo je spor. Číslo~$n$ je teda deliteľné jediným nepárnym prvočíslom~$p$.
Preto $n=4p^{\alpha}$ pre vhodné $\alpha \ge1$, takže číslo~$n$ má
celkom $\alpha+1$ nepárnych deliteľov $1,p, p^2,\dots,p^\alpha$.
Preto musí byť $\alpha = 2$. Pre $n = 4p^2$ je tak hľadaný podiel rovný
$$
\frac {2+2p+2p^2+4+4p+4p^2} {1+p+p^2} = \frac {(2+4) (1+p+p^2)} {1+ p+p^2} =6.
$$

Hľadaný podiel môže byť 30 (pre $n = 16$) alebo 6 (pre
$n = 4p^2$, pričom $p$ je ľubovoľné nepárne prvočíslo).


\ineriesenie
Rovnako ako v~predchádzajúcom riešení označme $n$ hľadané prirodzené číslo
a~najväčšiu mocninu dvojky, ktorá ho delí, označme $2^k$. Z~predošlého riešenia
už vieme, že všetky párne delitele čísla~$n$ môžeme rozdeliť na $k$-členné skupiny
$2d, 2^2d, \dots, 2^kd$, pričom $d$ je ľubovoľný nepárny deliteľ čísla~$n$.
Súčet párnych deliteľov v~každej z~vypísaných skupín sa dá vyjadriť ako
násobok príslušného~$d$:
$$
\align
2d+2^2d+\dots+2^kd=&(2+2^2+\dots+2^k)d=\\
=&\left(\frac{2^{k+1}-1}{2-1}-1\right)d=(2^{k+1}-2)d.
\endalign
$$
Taký istý činiteľ $2^{k+1}-2$ dostaneme pre každý nepárny deliteľ
čísla~$n$, preto súčet všetkých párnych deliteľov čísla~$n$ je vždy $(2^{k+1}-2)$-násobkom
súčtu všetkých jeho nepárnych deliteľov.

Ostáva nájsť možné hodnoty $k$ a~napokon
ukázať, že k~nim existuje zodpovedajúce číslo~$n$. Možné hodnoty $k$ určíme
rovnako ako v~predošlom riešení z~rovnice $N(k-1) = 3$, pričom $N$ je počet
nepárnych deliteľov čísla~$n$. Dostávame tak $k= 2$ ($N = 3$ a~hľadaný podiel
je $2^3-2 = 6$) a~$k= 4$ ($N = 1$ a~hľadaný podiel je $2^5-2 = 30$). Pre
$k= 2$ potom hľadáme násobok~4 s~tromi nepárnymi deliteľmi~-- tomu
vyhovuje napríklad $n = 4 \cdot 9 = 36$ s~tromi nepárnymi deliteľmi
1, 3 a~9~-- a~pre $k= 4$ zrejme vyhovuje $n = 16$
s~jediným nepárnym deliteľom.


\návody
Nájdite najmenšie prirodzené číslo, ktoré má práve tri delitele. Ako sa
zmení odpoveď, ak hľadáme najmenšie trojciferné nepárne číslo s~práve
tromi deliteľmi? [4; 121]

Nájdite všetky prirodzené čísla, ktoré majú rovnaký počet párnych
aj nepárnych deliteľov. [$2m$, pričom $m$ je ľubovoľné nepárne číslo.]

\D
Nech $m$ je prirodzené číslo, ktoré má 7~kladných deliteľov, a~$n$ je
prirodzené číslo, ktoré má 9~kladných deliteľov. Koľko deliteľov môže mať
súčin $m\cdot n$? [64--B--I--4]

Súčin všetkých kladných deliteľov prirodzeného čísla $n$ je $20^{15}$. Určte $n$. [64--B--II--1]
\endnávod
}

{%%%%%   B-I-5
Najskôr ukážeme, že $AD \parallel BC$. Keďže $|AB| = |CD|$,
sú obvodové uhly nad tetivami $AB$ a~$CD$ kružnice opísanej
šesťuholníku $ABCDEF$ zhodné (\obr), teda $|\angle ADB| = |\angle DBC|$;
to sú však striedavé uhly priečky~$BD$ priamok $AD$ a~$BC$, preto
$AD \parallel BC$.

Ostáva ukázať, že $GH \parallel AD$.
\insp{b65.3}%
Využitím zhodných obvodových uhlov nad tetivami $AB$ a~$CD$
pri vrcholoch $E$ a~$F$ dostávame
$$
|\angle GEH| = |\angle AEB| = |\angle CFD| = |\angle GFH|,
$$
čo znamená, že
body $E$, $F$, $G$ a~$H$ ležia na jednej kružnici, pretože
vrcholy zhodných uhlov $GEH$ a~$GFH$ ležia v~rovnakej polrovine s~hraničnou
priamkou~$GH$. Z~toho vyplýva, že
uhly $EFH$ a~$EGH$ nad jej tetivou~$EH$ sú zhodné. To spolu
so zhodnosťou uhlov $EFD$ a~$EAD$ nad tetivou~$ED$ pôvodnej kružnice (\obrr1)
vedie na zhodnosť súhlasných uhlov $EGH$ a~$EAD$
priečky~$AE$ priamok $GH$ a~$AD$, ktoré sú teda naozaj rovnobežné.
Tým je tvrdenie úlohy dokázané.



\návody
Dokážte, že tetivový lichobežník je
rovnoramenný. [Osi oboch rovnobežných základní lichobežníka prechádzajú stredom
opísanej kružnice, sú teda zhodné.]

Dokážte, že ak dve rôznobežné protiľahlé strany tetivového štvoruholníka $ABCD$
majú rovnakú dĺžku, je štvoruholník lichobežníkom.
[Ak sú zhodné strany $AB$ a~$CD$, uvažujme os~$o$ úsečky~$BC$,
tá prechádza stredom~$S$ opísanej kružnice. Rovnoramenné
trojuholníky $ABS$ a~$CDS$ sú zhodné, a~teda súmerne združené podľa osi~$o$.]

\D
Daná je tetiva~$AB$ kružnice~$k$ so stredom v~bode~$S$. Na úsečke~$AB$
zvoľme bod~$M$ a~priesečník kružnice opísanej trojuholníku $AMS$
s~kružnicou~$k$ označme~$C$. Dokážte, že uhly $MCS$ a~$MBS$ sú zhodné.
[Stačí využiť rovnosť uhlov v~rovnoramennom trojuholníku $ABS$
a~obvodové uhly nad $MS$ v~kružnici opísanej trojuholníku $AMS$.]

Vo vonkajšej oblasti kružnice~$k$ je daný bod~$A$. Všetky
lichobežníky, ktoré sú do kružnice~$k$ vpísané tak, že ich
predĺžené ramená sa pretínajú v~bode~$A$, majú spoločný
priesečník uhlopriečok. Dokážte. [47--A--III--5]
\endnávod
}

{%%%%%   B-I-6
Vzhľadom na definíciu čísel $x_1, x_2, \dots, x_6$ je zrejmé, že ľubovoľná permutácia
zvolených čísel $a$, $b$, $c$ sa prejaví jednak rovnakou permutáciou hodnôt $x_1$, $ x_2$, $x_3$,
jednak rovnakou permutáciou hodnôt $x_4$, $ x_5$, $ x_6$.
Stačí teda zistiť, koľko rôznych poradí možno dostať za predpokladu $a<b<c$.
Celkový počet možných poradí potom bude 6-krát väčší, keďže permutácií troch čísel
$a$, $b$, $c$ je práve $3!=6$.

Predpokladajme preto, že $a<b<c$, čiže $x_1<x_2<x_3$. Z~nerovností
$$
x_4 = \frac {2a^2} {b+c} <\frac {2a^2} {a+a} = a\quad \text {a} \quad
x_6 = \frac {2c^2} {a+b}> \frac {2c^2} {c+c} = c
$$
vyplýva $x_4 <x_1 <x_2 <x_3 <x_6$. Ostáva rozhodnúť, medzi ktorými dvoma
z~posledných piatich čísel môže ležať číslo~$x_5$, pretože to spĺňa nerovnosti
$$
x_5 = \frac {2b^2} {c+a}> \frac {2a^2} {c+b} = x_4 \quad \text {a} \quad
x_5 = \frac {2b^2} {c+a} <\frac {2c^2} {b+a} = x_6.
$$
Do úvahy tak prichádzajú štyri alternatívy
$$
x_4 < x_5 <x_1, \quad x_1 <x_5 <x_2, \quad x_2 <x_5 <x_3, \quad
x_3 <x_5<x_6;
$$
ukážeme, že sú všetky možné.

\ite1.
Pre $(a, b, c) = (1, 2, 8)$ dostávame
$$
x_4 = \tfrac 15 <x_5 = \tfrac 89 <x_1 = 1 <x_2 = 2 <x_3 = 8 <
x_6 = \tfrac {128} {3} = 42\,\tfrac 23.
$$
\ite2.
Pre $(a, b, c) = (1, 2, 4)$ dostávame
$$
x_4 = \tfrac 13 <x_1 = 1 <x_5 = \tfrac 85 <x_2 = 2 <x_3 = 4 <
x_6 = \tfrac {32} {3} = 10\,\tfrac 23.
$$
\ite3.
Pre $(a, b, c) = (2, 6, 9)$ dostávame
$$
x_4 = \tfrac 8 {15} <x_1 = 2 <x_2 = 6 <x_5 = \tfrac {72} {11} = 6\,\tfrac 6 {11} <
x_3 = 9 <x_6 = \tfrac {81} {4} = 20\,\tfrac 14.
$$
\ite4.
Pre $(a, b, c) = (1, 9, 12)$ dostávame
$$
x_4 = \tfrac 2 {21} <x_1 = 1 <x_2 = 9 <x_3 = 12 <x_5 = \tfrac {162} {13} = 12\,\tfrac
6 {13} <x_6 = \tfrac {144} {5} = 28\,\tfrac 45.
$$

Samozrejme, pre každú z~uvedených možností existuje veľa iných príkladov takých trojíc $a<b <c$.
Na príklade prvej trojice ešte stručne ukážeme, ako k~nej možno dospieť.

Ako vieme, prvá z~nerovností $x_4<x_5<x_1$ je splnená vždy, preto sa
budeme zaoberať iba druhou nerovnosťou, ktorú po prepísaní do premenných
$a$, $b$, $c$ vyriešime vzhľadom na~$c$:
$$
\frac{2b^2}{c+a}<a,\quad\text{čiže}\quad c>\frac{2b^2}{a}-a.
$$
Keď zvolíme napr. $b = 2a$, dostaneme podmienku $c>7a$ a~pre vyhovujúce
$c = 8a$ potom pri voľbe $a=1$ dostaneme práve trojicu $(a,b,c) = (1, 2, 8)$.


\odpoved
Existuje práve 24 rôznych poradí $(i_1, i_2, \dots, i_6)$.



\návody
Pre kladné reálne čísla $a\le b \le c$ dokážte nerovnosť $1/a~\ge 1/b \ge
1/c$. [Prvú nerovnosť vynásobíme $ab$ a~druhú $bc$.]

Pre kladné reálne čísla $a\le b \le c$ dokážte nerovnosť $1/a\ge
2/(b+c)$. [Vynásobte nerovnosť výrazom $a(b+c)$ a~využite
nerovnosti $a\le b$ a~$a\le c$.]

\D
Dokážte, že pre ľubovoľné kladné reálne čísla $a$, $b$, $c$ platí
$$
\frac {ab} {a^2-ab+b^2}+\frac {bc} {b^2-bc+c^2}+\frac {ca} {c^2-ca+a^2} \le 3.
$$
Určte, kedy nastane rovnosť. [64--B--S--3]

Sú dané reálne čísla $a$, $b$, $c$, pre ktoré platí $abc = 1$. Dokážte, že
najviac dve z~čísel
$$
2a-\frac 1b,\quad 2b-\frac 1c,\quad 2c-\frac 1a
$$
sú väčšie ako 1. [KMS, 3. zimná séria 2012/2013, úloha 7]
\endnávod
}

{%%%%%   C-I-1
Ľavú stranu danej rovnice rozložíme na súčin
podľa vzorca pre $A^2-B^2$. V~takto upravenej rovnici
$$
(p+q+r)(p-q-r)=637
$$
už ľahko rozoberieme všetky možnosti pre dva celočíselné činitele naľavo.
Prvý z~nich je väčší a~kladný, preto aj druhý musí byť kladný (lebo taký je
ich súčin), takže podľa rozkladu na súčin prvočísel čísla
$637=7^2\cdot13$ sa jedná o~jednu z~dvojíc $(637, 1)$,
$(91, 7)$ alebo $(49, 13)$. Prvočíslo~$p$ je zrejme aritmetickým
priemerom oboch činiteľov, takže sa musí rovnať jednému z~čísel
$\frac12(637+1)=319$, $\frac12(91+7)=49$,
$\frac12(49+13)=31$. Prvé dve z~nich však prvočísla nie sú
($319=11\cdot29$ a~$49=7^2$), tretie áno.
Takže nutne $p=31$ a~prislúchajúce rovnosti $31+q+r=49$ a~$31-q-r=13$
platia práve vtedy, keď $q+r=18$. Také dvojice
prvočísel $\{q,r\}$ sú iba $\{5, 13\}$ a~$\{7, 11\}$ (stačí
prebrať všetky možnosti, alebo si uvedomiť, že jedno z~prvočísel
$q$, $r$ musí byť aspoň $18:2=9$, nanajvýš však $18-2=16$).
Súčin $pqr$ tak má práve dve možné hodnoty, a~to $31\cdot5\cdot13=2\,015$
a~$31\cdot7\cdot11=2\,387$.


\návody
Určte všetky prirodzené čísla $a$ a~$b$, pre ktoré je rozdiel
$a^2-b^2$ druhou mocninou niektorého prvočísla. [$a=(p^2+1)/2$
a~$b=(p^2-1)/2$, pričom $p$ je ľubovoľné nepárne prvočíslo.]

\D
Nájdite všetky dvojice nezáporných celých čísel $a$, $b$, pre ktoré platí
$a^2+b+2=a+b^2$. [59--C--S--3]

Nájdite všetky dvojice prvočísel $p$ a~$q$, pre ktoré platí
$p+q^{2}=q+145p^{2}$. [55--C--II--4]
\endnávod
}

{%%%%%   C-I-2
Pre kontrolu dotyčnej podmienky stačí vedieť len to,
ktorým vrcholom kocky $ABCDEFGH$ sú pripísané čísla nepárne
a~ktorým čísla párne. Zaveďme preto znaky $N$ a~$P$ pre všetky
nepárne, resp. párne čísla a~riešme najskôr otázku, koľkými vyhovujúcimi spôsobmi
môžeme pripísať k~vrcholom kocky $ABCDEFGH$ štyri~$N$ a~štyri~$P$
(práve toľko ich totiž je medzi zadanými číslami
1, 3, 3, 3, 4, 4, 4,~4).

Uvedomme si, čo podmienka úlohy hovorí o~počte znakov~$N$
pripísaných vrcholom jednej a~tej istej steny kocky: počet týchto
$N$ je nanajvýš~2 (súčin troch pripísaných~$N$ by bol totiž nepárny,
teda v~rozpore s~danou podmienkou).
Keď však danú stenu kocky zvážime súčasne so stenou s~ňou
rovnobežnou (\tj. stenou protiľahlou), pri ktorej vrcholoch sú
tiež nanajvýš dve $N$, a~zohľadníme pritom, že pri ôsmich vrcholoch týchto dvoch stien
(teda pri všetkých ôsmich vrcholoch kocky) sú (všetky) štyri $N$,
dôjdeme k~záveru, že {\it pri vrcholoch každej steny sú práve dve $N$}
(a~teda aj dve~$P$). Naopak, každé také pripísanie štyroch~$N$ a~štyroch~$P$
zrejme vyhovuje požiadavkám úlohy.

Stojíme tak pred úlohou určiť počet tých pripísaní štyroch~$N$ a~štyroch~$P$
vrcholom kocky $ABCDEFGH$,
pri ktorých sú dve~$N$ a~dve~$P$ pri vrcholoch každej steny. Rozdelíme
ich na dve skupiny podľa toho, či existuje stena,
na ktorej sú obe~$N$ priradené vrcholom susedným (na kocke tak
vznikne aspoň jedna hrana "$NN$"), alebo naopak
vo všetkých stenách sú obe~$N$ priradené vrcholom
protiľahlým (všetky hrany kocky potom budú "$NP$"). Po jednom reprezentantovi
oboch skupín vidíme na \obr~--
pre lepší prehľad bez označenia vrcholov kocky písmenami. Ľahko
overíme (výklad tu vynecháme), že znaky v~krúžku pri reprezentantoch
oboch skupín už jednoznačne určujú znaky pri všetkých ostatných vrcholoch
kocky.
\insp{c65.1}%

Teraz už ľahko usúdime, že v~prvej skupine je práve šesť priradení~--
jednou hranou~"$NN$" je totiž, ako vieme, celé vyhovujúce priradenie
určené a~má práve dve hrany~"$NN$", ktoré sú pritom rovnobežné
a~neležia v~jednej stene; takých dvojíc hrán je pre kocku
$ABCDEFGH$ práve šesť. Naproti tomu v~druhej skupine sú iba dve rôzne
priradenia~-- pretože sa jedná o~priradenie bez hrany~"$NN$";
znakom $P$ alebo $N$ pri vrchole~$A$ danej kocky
sú totiž, ako vieme, určené znaky pri všetkých ďalších jej vrcholoch.
Existuje tak spolu $6+2=8$ vyhovujúcich priradení
štyroch~$N$ a~štyroch~$P$ vrcholom kocky $ABCDEFGH$.

V~ďalšej, jednoduchšej časti nášho postupu určíme, koľkými spôsobmi
môžeme štyri~$N$ a~štyri~$P$ (pevne pripísané vrcholom kocky)
zameniť konkrétnymi číslami 1, 3, 3, 3, 4, 4, 4, 4. Máme zrejme
práve štyri možnosti pre výber toho~$N$, ktoré zameníme číslom~1;
potom už zvyšné tri~$N$ musíme zameniť číslom~3, rovnako ako
všetky štyri~$P$ číslom~4. Počet spôsobov zámen znakov $N$ a~$P$
danými číslami je tak rovný~4.

Nakoniec uplatníme jednoduché kombinatorické {\it pravidlo
súčinu\/}: keďže existuje osem
vyhovujúcich pripísaní znakov $N$ a~$P$ k~vrcholom danej kocky
a~pri každom z~nich možno štyrmi spôsobmi zameniť znaky $N$ a~$P$ danými
číslami, je hľadaný počet vyhovujúcich pripísaní daných čísel
vrcholom danej kocky rovný $8\cdot4=32$.



\návody
Koľkými spôsobmi možno vrcholom štvorca $ABCD$ a~jeho stredu~$P$
pripísať čísla 1, 2, 3, 4, 5 tak, aby boli {\it napospol nepárne\/}
súčty čísel pri každej jeho strane aj oboch uhlopriečkach? Dokážete tento počet
určiť bez toho, aby ste vypísali všetky možnosti a~potom ich spočítali?
[24~spôsobov. Najskôr pripíšte daným piatim bodom tri znaky~$N$ a~dva znaky~$P$
pre nepárne, resp. párne čísla~-- to možno spraviť práve dvoma
vyhovujúcimi spôsobmi. Potom
uvážte, že znaky~$N$ možno zameniť danými číslami šiestimi spôsobmi
a~znaky~$P$ dvoma spôsobmi.]

\D
Určte, koľkými spôsobmi možno vrcholom pravidelného 9-uholníka $ABCDEFGHI$
priradiť čísla z~množiny $\{17,27,37,47,57,67,77,87,97\}$ tak, aby každé z~nich bolo
priradené inému vrcholu a~aby súčet čísel priradených každým trom susedným vrcholom bol
deliteľný tromi. [61--B--II--2]
\endnávod
}

{%%%%%   C-I-3
Daný výraz $V(x,y)$ upravme podľa vzorcov pre $(A\pm B)^2$:
$$
V(x,y)=\left(x^{2}-2xy+y^{2}\right)+\left(x^{2}+2x+1\right)+3=
\left(x-y\right)^{2}+\left(x+1\right)^{2}+3.
$$

\smallskip
a) Prvé dva sčítance v~poslednom súčte sú druhé mocniny, majú teda
nezáporné hodnoty. Minimum určite nastane v~prípade,
keď pre niektoré $x$ a~$y$ budú oba základy rovné nule (v~tom
prípade pre inú dvojicu základov už bude hodnota výrazu $V(x,y)$ väčšia).
Obe rovnosti $x-y=0$, $x+1=0$ súčasne
naozaj nastanú, a~to zrejme iba pre hodnoty $x=y=\m1$.
Dodajme (na to sa zadanie úlohy nepýta),
že $V_{\min}=V(\m1,\m1)=3$.

\odpoved
Daný výraz nadobúda svoju najmenšiu hodnotu iba pre
$x=y=\m1$.

\smallskip
b) Podľa úpravy z~úvodu riešenia platí
$$
V(x,y)=16\ \Leftrightarrow\ (x-y)^{2}+(x+1)^{2}+3=16
\ \Leftrightarrow\ (x-y)^{2}+(x+1)^{2}=13.
$$
Oba sčítance $(x-y)^2$ a~$(x+1)^2$ sú (pre celé nezáporné čísla
$x$ a~$y$) z~množiny $\{0, 1, 4, 9, 16,\dots\}$.
Jeden preto zrejme musí byť $4$ a~druhý~$9$. Vzhľadom na predpoklad
$x\ge0$ je základ $x+1$ mocniny $(x+1)^2$ kladný,
musí preto byť rovný $2$
alebo $3$ (a~nie $\m2$ či $\m3$).
V~prvom prípade, \tj. pre $x=1$, potom pre základ mocniny $(x-y)^2$
dostávame podmienku $1-y=\pm3$, teda $y=1\mp3$, čiže $y=4$
(hodnota $y=\m2$ je zadaním časti~b) vylúčená). V~druhom prípade,
keď $x=2$, dostaneme podobne z~rovnosti $x-y=2-y=\pm2$
dve vyhovujúce hodnoty $y=0$ a~$y=4$.

\odpoved
Všetky hľadané dvojice $(x,y)$ sú $(1, 4)$, $(2, 0)$ a~$(2, 4)$.



\návody
Pre ľubovoľné reálne čísla $x$, $y$ a~$z$ dokážte
nezápornosť hodnoty každého z~výrazov
$$
x^2z^2+y^2-2xyz,\
x^2+4y^2+3z^2-2x-12y-6z+13,\
2x^2+4y^2+z^2-4xy-2xz
$$
a~zistite tiež, kedy je dotyčná hodnota rovná nule.

\D
V~obore celých čísel vyriešte rovnicu $x^2+y^2+x+y=4$. [61--B--S--1]

Pre kladné reálne čísla $a$, $b$, $c$ platí $c^2 + ab = a^2 + b^2$. Dokážte, že potom platí aj
$c^2 + ab \le ac + bc$. [63--C--II--3]

Dokážte, že pre ľubovoľné nezáporné čísla $a$, $b$, $c$ platí
$(a+bc)(b+ac)\ge ab(c+1)^2$.
Zistite, kedy nastane rovnosť. [58--C--S--1]

Uvažujme výraz
$V(x)=\frpp{5x^{4}-4x^{2}+5}{x^{4}+1}$.
\item{a)} Dokážte, že pre každé reálne číslo~$x$ platí $V(x)\ge3$.
\item{b)} Nájdite najväčšiu hodnotu $V(x)$.
[58--C--II--1]

Dokážte, že pre ľubovoľné rôzne kladné čísla $a$, $b$ platí
$$
\frac{a+b}{2}<\frac{2(a^2+ab+b^2)}{3(a+b)}<\sqrt{\frac{a^2+b^2}{2}}.
$$
[58--C--I--6]
\endnávod
}

{%%%%%   C-I-4
Pre spoločnú hodnotu~$p$ oboch pomerov zo zadania platí
$$
|ED|=p|DF|\quad\text{a~zároveň}\quad|BE|=p|EA|. \tag1
$$
Pred vlastným riešením oboch úloh a) a~b) vyjadríme pomocou daného
čísla~$p$ skúmaný pomer obsahov trojuholníkov $ABC$ a~$ABD$.
Ten je rovný~-- keďže trojuholníky majú spoločnú stranu~$AB$~--
pomeru dĺžok ich výšok $CC_0$ a~$DD_0$ (\obr), ktorý je
\insp{c65.2}%
rovnaký ako pomer dĺžok úsečiek $BC$ a~$ED$,
a~to na základe podobnosti pravouhlých trojuholníkov $BCC_0$ a~$EDD_0$ podľa vety
$uu$ (uplatnenej vďaka $BC\parallel ED$).\footnote{V~prípade pravých uhlov $ABC$ a~$AED$ to platí
triviálne, lebo vtedy $B=C_0$ a~$E=D_0$.} Platí teda rovnosť
$$
\frac{S_{ABC}}{S_{ABD}}=\frac{|BC|}{|ED|}.
\tag2
$$

Vráťme sa teraz k~rovnostiam (1), podľa ktorých
$$
|EF|=(1+p)|DF|\quad\text{a}\quad|AB|=(1+p)|EA|,
$$
a~všimnime si, že trojuholníky $ABC$ a~$AEF$ majú spoločný uhol pri vrchole~$A$
a~zhodné uhly pri vrcholoch $C$ a~$F$ (pretože $BC\parallel EF$), takže
sú podľa vety $uu$ podobné. Preto pre dĺžky ich strán platí
$$
\frac{|AB|}{|AE|}=\frac{|BC|}{|EF|},\quad\text{čiže}\quad
1+p=\frac{|BC|}{(1+p)|DF|},\quad\text{odkiaľ}\quad
|BC|=(1+p)^2|DF|.
$$
Keď vydelíme posledný vzťah hodnotou $|ED|$, ktorá je rovná $p|DF|$ podľa
(1), získame podiel z~pravej strany (2) a~tým aj~hľadané vyjadrenie
$$
\frac{S_{ABC}}{S_{ABD}}=\frac{(1+p)^2}{p}.
\tag3$$

\smallskip
a) Algebraickou úpravou zlomku zo vzťahu (3)
$$
\frac{(1+p)^2}{p}=\frac{1+2p+p^2}{p}=2+p+\frac{1}{p}
$$
zisťujeme, že hodnota pomeru $S_{ABC}:S_{ABD}$ je pre akékoľvek dve
navzájom prevrátené hodnoty $p$ a~$1/p$ rovnaká, teda nielen pre
hodnoty $2/3$ a~$3/2$, ako sme mali ukázať.

\smallskip
b) Podľa vzťahu (3) je našou úlohou overiť pre každé $p>0$
nerovnosť
$$
\frac{(1+p)^2}{p}\ge4,\quad\text{čiže}\quad
(1+p)^2\ge4p.
$$
To je však zrejme ekvivalentné s~nerovnosťou $(1-p)^2\ge0$,
ktorá skutočne platí, nech je základ druhej mocniny akýkoľvek
(rovnosť nastane jedine pre $p=1$).

Dodajme,
že pre iný dôkaz bolo možné využiť aj vyššie uvedené "symetrické" vyjadrenie
$$
\frac{(1+p)^2}{p}=2+p+\frac{1}{p}
$$
a~uplatniť naň dobre známu nerovnosť $p+1/p\ge2$, ktorej
platnosť pre každé $p>0$ vyplýva napr. z~porovnania aritmetického
a~geometrického priemeru dvojice čísel $p$ a~$1/p$, nazývaného
AG-nerovnosť:
$$
\frac1{2}\Big(p+\frac{1}{p}\Big)\ge\sqrt{p\cdot\frac{1}{p}}=1,
\quad\text{pretože všeobecne}\quad\frac{a+b}{2}\ge\sqrt{a\cdot b}\quad
(\forall a,b\ge0).
$$


\návody
Vymyslite pravidlo, ako jednoducho vyjadriť pomer obsahov dvoch
trojuholníkov, ktoré sa zhodujú v~jednej strane či v~jednej výške.
Uplatnite ho potom na riešenie úloh N2 a~N3.

Uhlopriečky konvexného štvoruholníka $ABCD$ sa pretínajú
v~bode~$P$. Obsahy trojuholníkov $ABP$, $BCP$, $CDP$, $DAP$ označme postupne
$S_1$, $S_2$, $S_3$, $S_4$. Dokážte všeobecnú rovnosť $S_1\cdot
S_3=S_2\cdot S_4$ a~vysvetlite, prečo špeciálna rovnosť $S_2=S_4$
nastane práve vtedy, keď $AB\parallel CD$. [Pri prvej rovnosti prejdite
k~úmere $S_1:S_2=S_4:S_3$, pri druhej k~rovnosti obsahov trojuholníkov
$ABC$ a~$ABD$.]

Vnútri strán $BC$, $CA$, $AB$ daného trojuholníka $ABC$ sú zvolené
postupne body $K$, $L$, $M$ tak, že úsečky $AK$, $BL$, $CM$ sa
pretínajú v~jednom bode~$P$. Dokážte, že oba výrazy
$$
\frac{|BK|}{|KC|}\cdot\frac{|CL|}{|LA|}\cdot\frac{|AM|}{|MB|}\quad
\text{a}\quad
\frac{|PK|}{|AK|}+\frac{|PL|}{|BL|}+\frac{|PM|}{|CM|}
$$
sa rovnajú číslu 1.
[Pre prvý výraz vyjadrite vhodne pomery obsahov trojuholníkov $ABP$, $BCP$
a~$CAP$. Keď potom vyjadríte, akými sú časťami obsahu celého trojuholníka $ABC$,
a~tieto tri zlomky sčítate, dostanete tvrdenie o~hodnote druhého výrazu.]

\D
Označme $E$ stred základne $AB$ lichobežníka $ABCD$, v~ktorom platí
$|AB|:|CD|={3:1}$. Uhlopriečka~$AC$ pretína úsečky $ED$, $BD$ postupne
v~bodoch $F$, $ G$. Určte postupný pomer
$|AF|:|FG|:|GC|$. [64--C--I--4]

Označme $K$ a~$L$ postupne body strán $BC$ a $AC$ trojuholníka $ABC$,
pre ktoré platí $|BK|=\frac13|BC|$, $|AL|=\frac13|AC|$. Nech $M$ je
priesečník úsečiek $AK$ a~$BL$. Vypočítajte pomer obsahov trojuholníkov
$ABM$ a~$ABC$. [64--C--S--2]

Základňa~$AB$ lichobežníka $ABCD$ je trikrát dlhšia ako
základňa~$CD$. Označme $M$ stred strany~$AB$ a~$P$ priesečník
úsečky~$DM$ s~uhlopriečkou~$AC$. Vypočítajte pomer obsahov
trojuholníka $CDP$ a~štvoruholníka $MBCP$. [55--C--II--1]
\endnávod
}

{%%%%%   C-I-5
Aby sme sa mohli stručnejšie vyjadrovať, budeme
vyberať priamo {\it čísla}, a~nie {\it kartičky}.

Všimnime si najskôr, že pre súčet~$s$ ľubovoľných dvoch
daných čísel platí $11=5+6\le s\le55+54=109$.
Medzi číslami od 11 po 109 sú palindrómy práve {\it všetky\/}
násobky~11 a~navyše aj číslo~101.
Uvedomme si teraz, že deliteľnosť súčtu dvoch čísel daným číslom~$d$
(nám pôjde o~hodnotu $d=11$) závisí iba na zvyškoch
oboch sčítaných čísel po delení dotyčným~$d$. Toto užitočné pravidlo
uplatníme tak, že všetky dané čísla od~5 po~55
rozdelíme do skupín podľa ich zvyškov po delení číslom~11
a~tieto skupiny zapíšeme do riadkov tak,
aby súčet dvoch čísel z~rôznych skupín na rovnakom riadku bol deliteľný
číslom~11; o~význame zátvoriek na konci každého riadku budeme hovoriť
vzápätí.
$$
\alignedat3
&\phantom0\{5, 16, 27, 38, 49\},\qquad&&\{6, 17, 28, 39, 50\}\quad&&(5\ \text{čísel}), \\
&\phantom0\{7, 18, 29, 40, 51\}, &&\{15, 26, 37, 48\} &&(5\ \text{čísel}), \\
&\phantom0\{8, 19, 30, 41, 52\}, &&\{14, 25, 36, 47\} &&(5\ \text{čísel}), \\
&\phantom0\{9, 20, 31, 42, 53\}, &&\{13, 24, 35, 46\} &&(5\ \text{čísel}), \\
&\{10, 21, 32, 43, 54\}, &&\{12, 23, 34, 45\} &&(5\ \text{čísel})\\
&\span\span\{11, 22, 33, 44, 55\}\quad (1\ \text{číslo}).
\endaligned\\
$$

Na koniec každého riadku sme pripísali maximálny počet na ňom
zapísaných čísel, ktoré môžeme súčasne vybrať bez toho, aby súčet
dvoch z~nich bol násobkom čísla~11. Napríklad
v~treťom riadku máme päticu čísel so zvyškom~8 a~štvoricu čísel so
zvyškom 3. Je jasné, že nemôžeme súčasne vybrať po čísle z~oboch
týchto skupín (ich súčet by bol násobkom~11),
môžeme však vybrať súčasne všetkých päť čísel z~pätice (súčet
každých dvoch z~nich bude po delení~11 dávať taký istý zvyšok ako
súčet $8+8$, teda zvyšok~5). Dodajme ešte, že uvedená schéma
šiestich riadkov má pre nás ešte jednu obrovskú výhodu: súčet
žiadnych dvoch čísel z~{\it rôznych\/} riadkov nie je násobkom~11 (tým
totiž nie je ani súčet ich dvoch zvyškov).

Z~uvedeného rozdelenia všetkých daných čísel do šiestich riadkov vyplýva, že
vyhovujúcim spôsobom nemôžeme vybrať viac ako $5\cdot5+1=26$ čísel.
Keby sme však vybrali 26~čísel, muselo by medzi nimi byť aj jedno z~čísel~49 alebo~50
a~z~ďalších štyroch riadkov postupne čísla 51, 52, 53 a~54~--
potom by sme ale dostali palindróm $49+52$ alebo $50+51$.
A~tak sa nedá vybrať viac ako 25~čísel, pritom výber 25~čísel možný je:
z~prvých piatich riadkov vyberieme napríklad všetky čísla z~ľavých skupín s~výnimkou
čísla~52 a~k~tomu jedno číslo (napríklad~11) z~posledného riadku.
Potom súčet žiadnych dvoch vybraných čísel nebude deliteľný~11
(vďaka zaradeniu čísel do skupín), ani rovný
poslednému "kritickému" číslu, palindrómu~101 (preto sme
pri voľbe čísla~49 vylúčili~52).

\odpoved
Najväčší možný počet kartičiek, ktoré môžeme
požadovaným spôsobom vybrať, je rovný číslu~25.

\goodbreak
\ineriesenie
Medzi vybranými číslami môžu byť
\bulet
iba jedno číslo z~pätice $(11, 22, 33, 44, 55)$;
\bulet
nanajvýš jedno číslo z~každej z~20 nasledujúcich dvojíc
$(5, 6)$, $(7, 15)$, $(8, 14)$, $(9, 13)$, $(10, 12)$, $(16, 17)$,
$(18, 26)$, $(19, 25)$, $(20, 24)$, $(21, 23)$, $(27, 28)$, $(29, 37)$,
$(30, 36)$, $(31, 35)$, $(32, 34)$, $(38, 39)$, $(40, 48)$, $(41, 47)$,
$(42, 46)$ a~$(43, 45)$;\footnote{Tieto dvojice
so súčtami deliteľnými číslom~11 sme vytvorili
postupne zo zvyšných čísel tak, že k~najmenšiemu doposiaľ nezapísanému
číslu sme pripojili ďalšie najmenšie doposiaľ nezapísané
číslo, ktoré "dopĺňa" prvé číslo na nejaký násobok~11.
Takému postupu sa najmä v~matematickej informatike hovorí {\it
pažravý algoritmus}.}
\bulet
nanajvýš dve čísla zo štvorice $(49, 50, 51, 52)$
(pretože súčty $49+50$, $50+51$ a~$49+52$ sú palindrómy);
\bulet
obe zvyšné čísla 53 a~54.

Preto sa nedá požadovaným spôsobom vybrať viac ako $1+20+2+2=25$~čísel.
Vyhovujúci výber 25~čísel je možný: jedno číslo
z~pätice násobkov~11, menšie z~dvoch čísel z~každej z~20~dvojíc,
čísla $49$ a~$51$ zo štvorice a~napokon obe čísla 53 a~54.
Je však nutné vysvetliť, prečo súčet žiadnych
dvoch vybraných čísel nie je násobkom~11 (prečo nie je rovný~101,
je zrejmé hneď). Na to si stačí všimnúť,
že menšie čísla z~20~dvojíc dávajú po
delení jedenástimi postupne zvyšky, ktoré sa opakujú s~periódou
dĺžky~5 majúcou zloženie $(5, 7, 8, 9, 10)$, napokon posledné štyri
vybrané čísla majú postupne zvyšky 5, 7, 9 a~10,
takže súčet žiadnych dvoch zvyškov nami vybraných čísel
naozaj nie je násobkom~11.
(Zhodou okolností sa jedná o~rovnaký príklad vyhovujúceho výberu 25~čísel
ako v~prvom riešení.)


\návody
Z~množiny $\{1,2,3,\dots,99\}$ vyberte čo najväčší počet
čísel tak, aby súčet žiadnych dvoch vybraných čísel nebol násobkom
jedenástich. Vysvetlite, prečo
zvolený výber má požadovanú vlastnosť a~prečo žiadny
výber väčšieho počtu čísel nevyhovuje. [\hbox{58--C--I--5}]

\D
Z~množiny $\{1,2,3,\dots,99\}$ je vybraných niekoľko rôznych čísel tak, že súčet žiadnych troch z~nich nie je násobkom deviatich.
\item{a)} Dokážte, že medzi vybranými číslami sú najviac štyri deliteľné tromi.
\item{b)} Ukážte, že vybraných čísel môže byť 26.
[58--C--II--3]
\endnávod
}

{%%%%%   C-I-6
Poznamenajme predovšetkým, že vzhľadom na osovú
súmernosť podľa priamky~$AB$ je jedno, ktorý z~oboch priesečníkov
kružníc $k_1$ a~$k_2$ vyberieme za bod~$L$.

Hľadaný obsah trojuholníka $KLM$ vyjadríme nie pomocou dĺžok jeho
odvesien $KL$ a~$LM$, ale pomocou dĺžok jeho prepony~$KM$ a~k~nej
prislúchajúcej výšky~$LD$ (\obr{} vľavo), teda použitím
vzorca\footnote{Výpočet dĺžky odvesny~$LM$ bez medzivýpočtu výšky~$LD$
je totiž prakticky nemožný.}
$$
S_{KLM}=\frac{|KM|\cdot|LD|}{2}.
$$
\insp{c65.3}%

Na určenie vzdialeností bodu~$D$ od bodov $B$ a~$L$
uvažujme ešte stred~$S$ úsečky~$BL$ (\obrr1{} vpravo). Trojuholníky
$ASB$ a~$LDB$ sú oba pravouhlé so spoločným ostrým uhlom pri vrchole~$B$.
Sú preto podľa vety~$uu$ podobné, takže pre pomer ich strán platí
(počítame s~dĺžkami bez jednotiek, takže podľa zadania je
$|AB|=4$, $|BL|=2$, a~preto $|BS|=|BL|/2=1$)
$$
\frac{|BD|}{|BS|}=\frac{|BL|}{|BA|}=\frac{2}{4},
\quad\text{odkiaľ}\quad |BD|=\frac{1}{2}|BS|=\frac{1}{2}.
$$
Z~Pytagorovej vety pre trojuholník $LDB$ tak
vyplýva\footnote{Inou možnosťou pre výpočet výšky~$LD$ na rameno~$AB$
rovnoramenného trojuholníka $ABL$ je vypočítať jeho výšku~$AS$
na základňu~$BL$ (použitím Pytagorovej vety pre trojuholník $ABS$)
a~potom porovnať dvojaké vyjadrenie obsahu trojuholníka $ABL$
cez jeho výšky $AS$ a~$LD$.}
$$
|LD|=\sqrt{|BL|^2-|BD|^2}=\sqrt{4-\frac14}=\frac{\sqrt{15}}{2}.
$$
Z~rovnosti $|BD|=1/2$ už odvodíme aj dĺžku úseku~$KD$
prepony~$KM$ pravouhlého trojuholníka $KLM$:
$|KD|=|AB|-|AK|-|BD|=4-1-1/2=5/2$. Dĺžku druhého úseku~$DM$ teraz určíme
z~Euklidovej vety o~výške, podľa ktorej
$|LD|^2=|KD|\cdot|DM|$.
Dostaneme teda $|DM|=|LD|^2/|KD|=(15/4)/(5/2)=3/2$, čiže celá
prepona~$KM$ má dĺžku $|KM|=|KD|+|DM|=5/2+3/2=4$. Dosadením do
vzorca z~úvodu riešenia tak dôjdeme k~výsledku
$$
S_{KLM}=\frac{|KM|\cdot|LD|}{2}=
\frac{4\cdot\dfrac{\sqrt{15}}{2}}{2}=\sqrt{15}.
$$

\odpoved
Trojuholník $KLM$ má obsah $\sqrt{15}\cm^2$.

\ineriesenie
Keď narysujeme presne obe kružnice $k_1$, $k_2$ a~zodpovedajúci bod~$M$,
nadobudneme podozrenie, že $|KM|=|AB|$ a~bod~$L$ je taký bod Tálesovej kružnice~$k$
nad priemerom~$KM$ so stredom~$E$, ktorý leží na osi úsečky~$EB$ (\obr).
\insp{c65.4}%
Skutočne, pri opísanej voľbe bodu~$M$ a~konštrukcii bodu~$L$ bude platiť $|BL|=|EL|=2\cm$,
takže aby sme sa presvedčili, že sa jedná naozaj o~bod~$L$ zo zadania úlohy,
stačí overiť, že aj $|AL|=|AB|=4\cm$. Keďže (písané bez jednotiek)
$|EM|=2$, $|BM|=|AK|=1$, a~teda $|BD|=|ED|=\frac12$ a~$|AD|=\frac72$,
podľa Pytagorovej vety použitej postupne na pravouhlé trojuholníky $BDL$ a~$ADL$
pre takto zostrojený bod~$L$ máme
$$
|DL|^2=2^2-\Big({1\over2}\Big)^{\!2}=\frac{15}4,\qquad
|AL|^2=\Big({7\over2}\Big)^{\!2}+2^2-\Big({1\over2}\Big)^{\!2}=4^2.
$$
Tým je naša hypotéza overená. Obsah trojuholníka~$KLM$ už spočítame ľahko:
$$
S_{KLM}=\frac12{|KM|\cdot|LD|}=2|DL|\cm =\sqrt{15}\cm^2.
$$


\návody
Zopakujte si Euklidove vety o~odvesne a~o~výške pravouhlého
trojuholníka a~pripomeňte si ich dôkazy na základe podobnosti daného
trojuholníka s~dvoma menšími trojuholníkmi, ktoré vzniknú jeho rozdelením
pomocou výšky na preponu.

\D
Kružnice $k(S;6\cm)$ a~$l(O;4\cm)$ majú vnútorný dotyk v~bode~$B$.
Určte dĺžky strán trojuholníka $ABC$, pričom bod~$A$ je priesečník priamky~$OB$
s~kružnicou~$k$ a~bod~$C$ je priesečník kružnice~$k$ s~dotyčnicou z~bodu~$A$
ku kružnici~$l$. [59--C--S--2]

Pravouhlému trojuholníku $ABC$ s~preponou~$AB$ a~obsahom~$S$ je opísaná kružnica. Dotyčnica k~tejto kružnici v~bode~$C$ pretína dotyčnice vedené bodmi $A$ a~$B$ v~bodoch $D$ a~$E$. Vyjadrite dĺžku úsečky~$DE$ pomocou dĺžky~$c$ prepony a~obsahu~$S$. [58--C--II--4]
\endnávod
}

{%%%%%   A-S-1
Najskôr si všimnime, že $5^3-4^3 = 61$, $2^3-1^3 = 7$ a~$3^3-2^3 =19$
sú pekné prvočísla, takže 1, 7 a~9 patria medzi hľadané posledné cifry.
Ukážeme, že iné cifry na poslednom mieste byť nemôžu.

Zvoľme pekné prvočíslo~$p$ a~zapíšme ho ako $m^3 - n^3$, pričom $m>n$
sú prirodzené čísla. Podľa známeho vzorca potom
$$
p = m^3-n^3 = (m-n)(m^2+mn+n^2),
$$
a~keďže druhá zátvorka je väčšia ako~1, musí byť $m = n+1$
(inak by sa prvočíslo~$p$ dalo rozložiť na súčin dvoch činiteľov väčších ako~1).
Po dosadení vyjde
$$
p = 3n^2 + 3n+1.\tag1
$$
Keďže $3n^2 + 3n+1 > 6$, musí byť prvočíslo~$p$ nepárne a~rôzne od~5. Tým sme vylúčili $0$, $2$, $4$, $5$, $6$ a~$8$ ako možné
posledné cifry, a~ostáva tak už len vylúčiť cifru~3.

Na to stačí zistiť, aké zvyšky po delení piatimi dávajú
čísla tvaru $3n^2 + 3n+1$. Dosadením jednotlivých zvyškov 0, 1, 2, 3 a~4
do~\thetag1
postupne vyjdú
zvyšky 1, 2, 4, 2,~1, čím je zvyšok~3 vylúčený.

\odpoved
Posledné cifry pekných prvočísel sú 1, 7 a~9.

\poznamka
Odvodenie vzťahu \thetag1 nie je nevyhnutné. Po zistení
poznatku $m=n+1$ stačí totiž určiť možné cifry rozdielov
tretích mocnín dvoch po sebe idúcich prirodzených čísel. Na to
najskôr zostavíme tabuľku posledných cifier čísel $k$ a~$k^3$:
$$
\def\strut{\vrule width 0pt height1em depth.3em\relax}
\vbox{\halign{\hss$#$\hss\ \strut\vrule&&\enspace $#$\cr
k~& 0& 1& 2& 3& 4& 5& 6& 7& 8& 9\cr\noalign{\hrule}
k^3 & 0& 1& 8& 7& 4& 5& 6& 3& 2& 9\cr}}
$$

Vidíme, že posledné cifry rozdielov $(k+1)^3-k^3$ sú
$$
\let\,\
1-0,\,8-1,\,17-8,\,14-7,\,5-4,\,6-5,\,13-6,\,12-3,\,9-2,\,10-9,
$$
teda jedine cifry 1, 7 a~9. (Aj tak je ale nutné uviesť príklady
pekných prvočísel s~týmito poslednými ciframi.)


\nobreak\medskip\petit\noindent
Za úplné riešenie dajte 6 bodov.
Za nájdenie pekných prvočísel končiacich každou
z~cifier 1, 7 a~9 dajte spolu 1~bod. Akýkoľvek správny argument
vedúci k~vylúčeniu každej z~cifier $0$, $2$, $4$, $5$, $6$ a~$8$ oceňte
tiež jedným bodom. Ďalšie dva body dajte za odvodenie, že hľadané prvočísla
majú tvar $3n^2 + 3n+1$. Zvyšné dva body potom za dokončenie riešenia.
Pri postupe z~poznámky naopak 1~bod strhnite, ak riešiteľ neuvedie
zodpovedajúce príklady pekných prvočísel.
\endpetit
\bigbreak
}

{%%%%%   A-S-2
Pre dôkaz nerovnosti $ab\ge4$ dosaďme zo zadaných vzťahov.
Získame tak odhad
$$
ab = \Bigl(c+ \frac{1}{d}\Bigr)\Bigl(d+ \frac{1}{c}\Bigr)
= cd + 1 + 1 + \frac{1}{cd} \ge 4,
$$
pričom sme v~poslednej nerovnosti využili známy fakt, že pre kladné
čísla $x$ (teda aj pre $x=cd > 0$) platí $x + 1/x \ge 2$.

V~druhej časti úlohy budeme postupovať podobne. Dosadením za $a$ a~$b$ vyjde
$$
ab + cd = \Bigl( 2 + cd + \frac{1}{cd}\Bigr) + cd = 2 + 2cd +\frac{1}{cd}.
$$
Tentoraz využijeme nerovnosť $x+y \ge 2\sqrt{xy\mathstrut}$, ktorá platí pre
ľubovoľné nezáporné čísla~$x$,~$y$. Keď v~nej zvolíme $x = 2cd$, $y =1/cd$,
dostaneme
$$
2cd + \frac{1}{cd} \ge 2\sqrt{2}.
$$
Vidíme, že $ab + cd \ge 2(1+\sqrt{2})$. Aby sme sa presvedčili, že
sa jedná o~hľadané minimum, nájdeme prípustné hodnoty čísel
$a$, $b$, $c$, $d$, pre ktoré v~tejto nerovnosti nastane rovnosť.

Vyjdeme z~toho, že v~použitej nerovnosti nastáva rovnosť práve vtedy, keď
${x=y}$, čiže $2cd = 1/cd$, čo možno upraviť na $(cd)^2 = 1/2$. To
zabezpečíme napríklad voľbou
$c = 1$, $d = \sqrt{2}/2$ a~k~týmto hodnotám potom zo zadaných vzťahov
dopočítame ${a=1+\sqrt{2}}$, $b = 1+\sqrt{2}/2$. Zostrojená štvorica
tak spĺňa podmienky zo zadania a~zároveň pre ňu platí $ab+cd =
2(1+\sqrt{2})$. Môžeme si teda byť istí tým, že hodnota $2(1+\sqrt{2})$ je
hľadaným minimom výrazu $ab+cd$.


\nobreak\medskip\petit\noindent
Za úplné riešenie dajte 6 bodov.
Za riešenie prvej časti dajte dva body.
V~časti druhej dajte tri body za dôkaz nerovnosti $ab + cd \ge
2(1+\sqrt{2})$ a~jeden bod za zostrojenie štvorice hodnôt $a$, $b$, $c$, $d$, pre
ktorú je $ab + cd = 2(1+\sqrt{2})$.
Použité nerovnosti $x+1/x\ge2$ a~$x+y\ge2\sqrt{xy\mathstrut}$ (z~ktorých prvá
vyplýva z~druhej voľbou $y=1/x$) nie je nutné dokazovať, stačí ich buď označiť
za známe, alebo sa odvolať na nerovnosť medzi aritmetickým a~geometrickým
priemerom dvoch čísel.
\endpetit
\bigbreak}

{%%%%%   A-S-3
a) Označme $M$, $N$ postupne stredy ramien $BC$, $AD$. Ukážeme, že bod~$N$ leží
na kružnici s~priemerom~$BC$.

Dosadením danej rovnosti
do známeho vzťahu pre strednú priečku lichobežníka získame rovnosť
$$
|MN|=\frac{|AB|+|CD|}{2}=\frac12 |BC|.
$$
To znamená, že vzdialenosť bodu~$N$ od stredu~$M$ kružnice s~priemerom~$BC$
je rovná jej polomeru.
Bod~$N$ je teda bodom kružnice s~priemerom~$BC$.

\smallskip
b) Vzhľadom na zadanú podmienku existuje na strane~$BC$ bod~$E$ taký,
že $|BE|=|AB|$ a~$|EC|=|CD|$ (\obr). Ukážeme, že platí $|\uhol AED|=90^\circ$,
a~bod~$E$ potom bude oným hľadaným bodom na Tálesovej
kružnici nad priemerom~$AD$.
\insp{a65.6}%

To však vyplýva priamo z~rovnoramennosti trojuholníkov $ABE$, $ECD$
a~rovnobežnosti priamok $AB$ a~$CD$:
$$
\align
|\uhol AED| &= 180^\circ - |\uhol AEB| - |\uhol CED|= \\
&= \tfrac12 \bigl( (180^\circ - 2|\uhol AEB|) + (180^\circ - 2|\uhol CED|) \bigr)=\\
&= \tfrac12 ( |\uhol ABE| + |\uhol DCE|)= 90^\circ.
\endalign
$$
Tým je úloha vyriešená.

\poznamka
Ak začneme celé riešenie priamo dôkazom, že trojuholník $AED$ je pravouhlý,
môžeme si potom uvedomiť, že navzájom kolmé osi
jeho strán $AE$ a~$ED$ prechádzajú stredom~$N$ jemu opísanej kružnice. To však
znamená, že aj trojuholník $BCN$ je pravouhlý, takže kružnica nad priemerom~$BC$
prechádza stredom~$N$ strany~$AD$.

\nobreak\medskip\petit\noindent
Za úplné riešenie každej z~častí dajte 3 body.
V~časti~a) dajte jeden bod, ak žiak definuje bod~$N$ a~zároveň
prejaví úmysel ukázať, že leží na kružnici s~priemerom~$BC$. Zvyšné dva
body dajte, ak sa mu to podarí. Časť~b) obodujte analogicky.
V~prípade, že sa žiak pustí inou cestou, majte na pamäti, že pre riešenie
každej z~častí je nutné využiť podmienku zo zadania. Bez nej totiž ani
jeden zo záverov vo všeobecnosti neplatí.

\endpetit}

{%%%%%   A-II-1
Nech $s$ je súčet čísel na tabuli, $n$ je ich počet a~$p$ je
celá časť aritmetického priemeru. Potom platí rovnosť
$$
\frac{s}{n}=p+\frac{2\,016}{10\,000} = p + \frac{126}{625},
$$
z~ktorej roznásobením vyjde
$$
625(s-pn)= 126n.
$$
Z~toho vidíme, že $625$ delí ľavú stranu, takže musí deliť aj pravú. Avšak
čísla $126$ a~$625$ sú nesúdeliteľné, a~tak dokonca $625 \mid n$.
Určite je preto $n \ge 625$.

Ďalej využijeme rôznosť čísel na tabuli na zistenie, že
$$
p=\frac{s}{n}-\frac{126}{625} \ge \frac{1+2+\dots+n}{n}-\frac{126}{625}
= \frac{n(n+1)/2}{n}-\frac{126}{625}
\geq \frac{625+1}{2}-\frac{126}{625}>312.
$$
Celé číslo~$p$ je teda aspoň $313$ a~hodnota priemeru aspoň
$313{,}2016$.

Ostáva nájsť množinu čísel, ktorých aritmetický priemer by sa tejto hodnote
presne rovnal. Voľbou 625-prvkovej množiny $\{1, 2,\dots, 624, 751\}$
dostávame aritmetický priemer
$$
\frac{1+2+\dots+624+751}{625}=\frac{312 \cdot 625+751}{625}=313+\frac{126}{625}=313{,}2016.
$$

\odpoved
Najmenšia možná hodnota priemeru je $313{,}2016$.

\poznamka
Príklad vyhovujúcej množiny čísel s~aritmetickým
priemerom $313{,}2016$ nie je jediný. Ukážeme, ako jeden (vyššie
uvedený) príklad nájsť a~ako vyzerajú všetky ostatné.

Z~odvodeného odhadu
$$
p\ge\frac{n+1}{2}-\frac{126}{625}
$$
vyplýva, že hodnota $p=313$ je možná jedine pre $n=625$, keď
z~rovnosti $625(s-pn)=126n$ dosadením oboch hodnôt $p$ a~$n$
dostaneme $s=126+625\cdot313$. Taký súčet prvkov charakterizuje
všetky hľadané (ako vieme 625-prvkové) množiny prirodzených čísel.
Keďže $625\cdot313$ je súčet čísel množiny $\{1,2,\dots,625\}$,
vyhovujúcu množinu z~nej dostaneme, keď napríklad jej najväčší
prvok $625$ zväčšíme o~$126$ na hodnotu $625+126=751$. Dodajme, že na dosiahnutie
cieľa môžeme takú operáciu tiež spraviť s~ľubovoľným prvkom~$i$ uvedenej množiny
spĺňajúcim $i\ge500$ (pripomeňme, že čísla na tabuli boli rôzne), alebo
zodpovedajúco zväčšiť dve či viac čísel súčasne (tak možno napríklad dostať
vyhovujúcu množinu $\{1,2,\dots,499,501,\dots,625,626\}$).


\nobreak\medskip\petit\noindent
Za úplné riešenie dajte 6~bodov.
Dva body dajte za zostrojenie množiny
s~priemerom rovným $313{,}2016$ a~zvyšné štyri za odvodenie, že
nižší priemer sa dosiahnuť nedá. Zo štyroch bodov určených pre
druhú časť dajte dva za dôkaz, že počet čísel na tabuli je
aspoň $625$ (stačí aj vzťah $625 \mid n$), jeden za využitie rôznosti
čísel na dolný odhad priemeru číslom $(n+1)/2$ a~jeden bod za skombinovanie oboch výsledkov. Ak riešiteľ
nespomenie nesúdeliteľnosť čísel $126$ a~$625$, body nestrhávajte. Tiež
netrestajte, ak riešiteľom zostrojená množina nevyhovuje iba
vinou drobnej aritmetickej chyby (napríklad číslo $751$ z~nášho príkladu
je zamenené číslom~$750$).
\endpetit
\bigbreak}

{%%%%%   A-II-2
Nech $X$ a~$Y$ sú ľubovoľné dva body opísané v~zadaní. Uvažujme Tálesovu
kružnicu nad priemerom~$BY$ opísanú pravouhlému trojuholníku $BYX$.
Tá obsahuje bod~$X$ úsečky~$AE$ a~zároveň sa v~bode~$B$ dotýka priamky~$AB$.
Zo všetkých takých kružníc má zrejme najmenší priemer kružnica~$k$, ktorá má
s~úsečkou~$AE$ spoločný iba jeden bod, teda sa jej dotýka, a~je preto vpísaná do
rovnostranného trojuholníka~$AA'F$, pričom $A'$ je obraz bodu~$A$ v~stredovej
súmernosti podľa vrcholu~$B$ a~$F$ leží na polpriamke~$BC$ (pozri \obr, kde
je vykreslená aj jedna z~tých kružníc s~dotyčnicou~$AB$ v~bode~$B$, ktoré majú
menší polomer ako kružnica~$k$, a~tak k~úsečke~$AE$ "nedosahujú").
Keďže stred kružnice~$k$ je zároveň ťažiskom trojuholníka~$AA'F$
so stranami dĺžky~$2$, má jej priemer dĺžku $|BY| = \frac23 \sqrt{3}$
a~zodpovedajúci bod~$X$ je stredom strany~$AF$,
takže naozaj patrí úsečke~$AE$, pretože $|AX|=1<|AE|$.
\insp{a65.7}%

\odpoved
Najmenšia možná dĺžka úsečky~$BY$ je $\frac23 \sqrt{3}$.


\ineriesenie
Vezmime ľubovoľné prípustné body $X$ a~$Y$, označme $X_0$ pätu
kolmice z~bodu~$X$ na $AB$ a~položme $|AX| = 2x$. Trojuholník
$AXX_0$ má vnútorné uhly s~veľkosťami $30^\circ$, $60^\circ$
a~$90^\circ$, takže $|AX_0| = x$, $|X_0B| =
1-x$ a~$|XX_0| = \sqrt{3} x$. Podľa Pytagorovej vety pre trojuholník
$XX_0B$ potom platí $|BX|^2 = 4x^2 -2x+1$.
Navyše pravouhlé trojuholníky $XX_0B$ a~$BXY$ sú podobné
(uhly $X_0XB$ a~$YBX$ sú zhodné striedavé uhly priečky~$BX$
rovnobežiek $BY$ a~$XX_0$), a~my
tak môžeme vyjadriť dĺžku úsečky~$BY$ ako
$$
|BY| = |BX|\cdot \frac{|BX|}{|XX_0|} = \frac{4x^2 -2x+1}{\sqrt{3}x} =
\frac{\sqrt{3}}{3} \Bigl( 4x + \frac{1}{x} - 2 \Bigr).
$$
Podľa AG-nerovnosti navyše platí $4x + \frc{1}{x} \ge 4$, takže
$|BY| \ge \frac23 \sqrt{3}$. Keďže rovnosť nastáva v~prípade, že $4x=1/x$, \tj. $x =\frac12$, čiže $|AX|=2x=1<|AE|$, bod~$X$ je
vtedy naozaj bodom úsečky~$AE$. Hodnota $|BY| = \frac{2}{3} \sqrt{3}$ je preto dosiahnuteľná, a~teda je
zároveň hľadaným minimom.


\nobreak\medskip\petit\noindent
Za úplné riešenie dajte 6~bodov.
Za určenie bodov $X$ a~$Y$ takých, že
$|BY| = \frac23 \sqrt{3}$, dajte dva body, pritom v~prípade, že
riešiteľ úplne zabudne uviesť, prečo ním nájdený bod~$X$ leží
na úsečke~$AE$, dajte iba jeden bod. Zvyšné štyri
body dajte za dôkaz, že nižšia hodnota sa dosiahnuť nedá.
Ak postupuje riešiteľ výpočtom, dajte dva body za vyjadrenie
dĺžky úsečky~$BY$ pomocou akéhokoľvek parametra určujúceho
polohu bodu~$X$ (dĺžka~$AX$, uhol $ABX$ a~pod.).
\endpetit
\bigbreak
}

{%%%%%   A-II-3
Žiadne dve párne čísla nemôžu byť v~tej istej zo šiestich dvojprvkových
podmnožín (nazývajme ich ďalej "páry"), a~my sa
tak môžeme obmedziť na také rozdelenia množiny $\{1,2,\dots,12\}$
(nazývajme ich {\it párnonepárne}), v~ktorých sú páry tvorené vždy jedným
párnym a~jedným nepárnym číslom. Ďalšie obmedzenie je, že k~číslu~6 ani k~číslu~12
nesmieme priradiť čísla 3 a~9; k~číslu~10 nesmie byť
priradené číslo~5. Počet vyhovujúcich párnonepárnych
rozdelení určíme dvoma spôsobmi.

\smallskip\noindent{\it Prvý spôsob.}
Jednotlivým nepárnym číslam od 1 do 11 najskôr určíme potenciálnych
párnych "partnerov". Pre čísla 1, 7 a~11 tvoria množinu
$\{2, 4, 6, 8, 10, 12\}$, pre čísla 3 a~9 množinu $\{2, 4, 8, 10\}$ a~pre číslo 5 množinu $\{2, 4, 6, 8, 12\}$. Z~toho je vidno, že
na určenie počtu všetkých vyhovujúcich výberov párnych partnerov
nemôžeme priamo uplatniť pravidlo súčinu, nech už by sme
výber uskutočňovali postupne pre dané nepárne čísla v~akomkoľvek poradí.
Pre "nádejné" poradie $(5, 3, 9, 1, 7, 11)$ však pravidlo súčinu
zafunguje, ak najskôr rozlíšime, či je číslu~5 priradené číslo
z~$\{6, 12\}$, alebo číslo z~$\{2, 4, 8\}$.\footnote{K~tomuto
rozboru sme motivovaní prienikom určených množín $\{2, 4, 6, 8, 12\}$
a~$\{2, 4, 8, 10\}$.} Možností je teda $2 \cdot 4
\cdot 3 \cdot 3 \cdot 2 \cdot 1 = 144$ v~prípade prvom a~$3 \cdot
3 \cdot 2 \cdot 3 \cdot 2 \cdot 1 = 108$ v~druhom. Spolu tak máme
$144+108=252$ možností.

\smallskip\noindent{\it Druhý spôsob.}
Keďže pri párnonepárnom rozdelení musí byť každé zo šiestich nepárnych čísel
v~páre s~iným zo šiestich párnych čísel, je počet všetkých takých (vyhovujúcich
aj~nevyhovujúcich) rozdelení rovný $6!=720$. Spočítame teraz, koľko
z~týchto párnonepárnych rozdelení je nevyhovujúcich.

Všetky nevyhovujúce párnonepárne rozdelenia tvoria zrejme množinu
$\mm M_6\cup\mm M_{12}\cup\mm M_{10}$, pričom
$\mm M_6$ je množina tých párnonepárnych rozdelení množiny $\{1, 2,\dots, 12\}$,
v~ktorých je s~číslom~6 v~páre číslo~3 alebo~9,
$\mm M_{12}$ je množina párnonepárnych rozdelení, v~ktorých je
s~číslom~12 číslo~3 alebo~9 a~napokon $\mm M_{10}$ je
množina tých rozdelení,
v~ktorých je s~číslom~10 v~páre číslo~5. Veľkosť
zjednotenia $\mm M_6\cup\mm M_{12} \cup\mm M_{10}$ je podľa princípu inklúzie
a~exklúzie, ktorého platnosť možno pre tri množiny nahliadnuť pomocou
Vennovho diagramu (\obr), rovná
$$
\align
&|\mm M_6|+|\mm M_{12}|+|\mm M_{10}|-|\mm M_6\cap\mm M_{12}|-|\mm M_6\cap\mm M_{10}|-|\mm M_{12}\cap\mm M_{10}|
+|\mm M_6\cap\mm M_{12}\cap\mm M_{10}|= \\
&=2\cdot 5!+2\cdot 5!+5!-2\cdot 4!-2\cdot 4! -2\cdot 4!+2\cdot 3!=468.
\endalign
$$
Keďže všetkých párnonepárnych rozdelení je 720
a~nevyhovujúcich je~468, počet vyhovujúcich je $720 - 468 =252$.
\insp{a65.8}%


\nobreak\medskip\petit\noindent
Za úplné riešenie dajte 6~bodov.
Ak riešiteľ správne určí výsledok v~"kombinatorickom
tvare" (napr. ako $2 \cdot 4 \cdot 3 \cdot 3 \cdot 2 \cdot 1 + 3
\cdot 3 \cdot 2 \cdot 3 \cdot 2 \cdot 1$) a~dopustí sa chyby iba pri
jeho vyčíslení, dajte päť bodov. Ak predloží funkčný
spôsob, ako sa dopočítať k~výsledku (iný ako výpis všetkých
možností), a~riešenie nedokončí alebo v~jeho priebehu pochybí,
dajte nanajvýš tri body. Za pozorovanie o~párnonepárnych
rozdeleniach ani za výpis súdeliteľných dvojíc body neudeľujte.

\endpetit
\bigbreak
}

{%%%%%   A-II-4
Predpokladajme, že čísla $a$, $b$, $m$ spĺňajú podmienku
$$
\forall x\in\langle -1,1\rangle\: |f(x)|\le
m(x^2+1),\quad\text{pričom}\quad f(x)=x^2+ax+b.
$$
Najskôr ukážeme,
že aspoň jeden z~rozdielov $f(1) - f(0)$ a~$f(-1) - f(0)$ je
väčší alebo rovný jednej:
pre ľubovoľnú funkciu $f(x) = x^2 + ax+ b$ je totiž
$$
f(0)=b,\quad f(1)=1+a+b, \quad f(-1)=1-a+b,
$$
takže
$$
\max\bigl( f(1)-f(0), f(-1)-f(0)\big)= \max(1+a,1-a)=1+|a|\ge1.
$$

Predpoklad z~úvodu riešenia znamená, že $|f(1)| \le 2m$, $|f({-1})| \le 2m$ a~$|f(0)| \le m$.
Preto buď
$$
1 \le 1+|a|= f(1) - f(0) \le |f(1)| + |f(0)| \le 2m + m = 3m, \tag1
$$
alebo
$$
1 \le 1+|a|= f(-1) - f(0) \le |f(-1)| + |f(0)| \le 2m + m = 3m. \tag2
$$
V~oboch prípadoch dostávame odhad $m \ge \frac13$.

Pokúsime sa ukázať, že $m=\frac13$ spĺňa požiadavky úlohy. Pre také~$m$
v~(1) alebo (2) platí všade rovnosť, takže musí byť $a=0$, ${-f(0)}=|f(0)|$ a~$|f(0)|=m=\frac13$, čiže
$b=f(0)={-\frac13}$.
Zdôraznime, že pre nájdené hodnoty $m$, $a$, $b$
zatiaľ nič nevieme o~platnosti nerovnosti zo zadania úlohy
pre hodnoty $x\in\langle{-1}, 1\rangle$ rôzne od čísel ${-1}$, $0$ a~$1$,
o~ktorých sme doposiaľ vôbec neuvažovali.

Overme teda,
že nájdená funkcia $f(x) = x^2 - \frac13$ pre $m = \frac13$
podmienky úlohy spĺňa: Nerovnosť $|x^2-\frac13|\le\frac13(x^2+1)$ je
ekvivalentná s~nerovnosťami
$$
-\frac13(x^2+1)\le x^2-\frac13\le\frac13(x^2+1),\quad\text{čiže}\quad
-x^2-1\le3x^2-1\le x^2+1
$$
a~tie sú ekvivalentné s~nerovnosťami
$0\le x^2 \le 1$, ktoré sú
na intervale $\langle \m1, 1 \rangle$ zrejme splnené.

\odpoved
Hľadané najmenšie $m$ je rovné zlomku $\frac13$.


\nobreak\medskip\petit\noindent
Za úplné riešenie dajte 6~bodov.
Za nájdenie hodnôt parametrov $a$, $b$ pre
$m = \frc13$ dajte jeden bod, ďalší (druhý) bod potom dajte za priame overenie
vzťahu zo zadania. Za odvodenie {\it nutnej\/} podmienky $m \ge \frc13$ dajte štyri body.
Ak nahradí riešiteľ algebraické použitie trojuholníkovej nerovnosti
obdobnou geometrickou úvahou, body nestrhávajte. Naopak za úvahy
o~tvare grafu kvadratickej funkcie (napríklad mlčky využívajúce
jej konvexnosť) body neudeľujte, ak nie sú tieto úvahy sprevádzané
dodatočnými argumentmi.
\endpetit
\bigbreak}

{%%%%%   A-III-1
Zo súčinu 1., 3. a~5. zlomku vidíme, že ich hodnoty sa
rovnajú 1, takže platí
$$
a+b=c+d=e+f=p. \eqno{(1)}
$$
Z~tvaru 2. a~4. zlomku potom vyplýva
$$
f+a\deli d+e \qquad \text{a} \qquad d+e\deli b+c. \eqno{(2)}
$$
Z toho jednak vyplýva, že číslo $f+a$ nie je väčšie ako aritmetický priemer svojich násobkov,
$$
f+a\leq \tfrac13 ((f+a)+(d+e)+(b+c))=p, \eqno{(3)}
$$
a~zároveň
$$
f+a\deli (f+a)+(d+e)+(b+c)=3p.
$$
Číslo $f+a$ je teda deliteľom čísla $3p$ a~navyše leží
v~intervale $\langle 2,p \rangle$. Preto $f+a=p$ alebo $f+a=3$. Oba
prípady preskúmame osobitne.

\smallskip
(i) Nech $f+a=p$. Vzhľadom na (3) potom platí $f+a=d+e=b+c=p$,
čo spolu s~(1) dáva $p-1$ riešení v~tvare
$$
(a,b,c,d,e,f)=(a,p-a,a,p-a,a,p-a),\quad \text{pričom }a~\in
\{1,2,\dots ,p-1\}.
% \eqno{(R_1)}
$$

\smallskip
(ii) Nech $f+a=3$. V~tomto prípade je $\{a,f\}=\{1,2\}$.

Nech najskôr $a=1$ a~$f=2$. Podľa (1) potom $b=p-1$ a~$e=p-2$
a~relácie~(2) majú tvar
$$
3\deli d+(p-2) \qquad \text{a} \qquad d+(p-2)\deli (p-1)+c.
\eqno{(4)}
$$
Pri rozbore (4) rozlíšime, či $d=1$, alebo $d\ge 2$.

Pre $d=1$ je $c=p-1$ a~vzťahy (4) majú v~takom prípade tvar
$$
3\deli p-1 \qquad \text{a} \qquad p-1\deli 2(p-1).
$$
Zatiaľ čo pravá relácia platí vždy, ľavej relácii vyhovujú jedine prvočísla~$p$
tvaru ${p=3q+1}$ ($q$~je vhodné prirodzené číslo). Pre také prvočísla dostávame
s~využitím~(1) riešenie
$$
(a,b,c,d,e,f)=(1,p-1,p-1,1,p-2,2).
% \eqno{(R_2)}
$$

Pre $d\ge2$ najskôr ukážeme, že pravá relácia v~(4) je splnená
práve vtedy, keď platí $d+(p-2)=(p-1)+c$, čiže $d=c+1$. Zo vzťahu
$d\ge2$ totiž vyplýva $c=p-d\le p-2$,
a~tak číslo $(p-1)+c$ nemôže byť netriviálnym násobkom čísla $d+(p-2)$, lebo
$$
d+(p-2)\ge p \qquad \text{a} \qquad (p-1)+c\leq 2p-3<2p.
$$
Preto sa obe čísla rovnajú. Z~rovností $c+d=p$ a~$d=c+1$ potom máme
$c=\frac12(p-1)$ a~$d=\frac12(p+1)$. Keďže $d+(p-2)=\frac32(p-1)$,
je splnená aj ľavá relácia v~(4), a~dostávame tak ďalšiu vyhovujúcu šesticu
prirodzených čísel
$$
(a,b,c,d,e,f)=(1,p-1,\tfrac12(p-1),\tfrac12(p+1),p-2,2).
% \eqno{(R_3)}
$$

Ostáva posúdiť prípad $a=2$ a~$f=1$. V~tomto prípade potom platí
$b=p-2$ a~$e=p-1$, takže relácie~(2) majú tvar
$$
3\deli d+(p-1) \qquad \text{a} \qquad d+(p-1)\deli (p-2)+c.
\eqno{(5)}
$$
Keďže
$$
d+(p-1)\geq p \qquad \text{a} \qquad (p-2)+c < 2p,
$$
je pravá relácia v~(5) splnená práve vtedy, keď $d+(p-1)=(p-2)+c$, \tj.
práve vtedy, keď $c=d+1$. To spolu s~rovnosťou $c+d=p$ vedie na~$c=\frac12(p+1)$
a~$d=\frac12(p-1)$, takže aj ľavá relácia v~(5) platí, a~dostávame tak
poslednú vyhovujúcu šesticu prirodzených čísel
$$
(a,b,c,d,e,f)=(2,p-2,\tfrac12(p+1),\tfrac12(p-1),p-1,1).
% \eqno{(R_4)}
$$

\zaver
Všetky nájdené riešenia sú zrejme rôzne a~ich počet závisí od toho,
aký zvyšok po delení tromi dáva dané prvočíslo $p>3$:
Pre prvočísla~$p$ tvaru $p=3q+1$ existuje $p+2$ šestíc
a~pre prvočísla~$p$ tvaru $p=3q+2$ existuje $p+1$ šestíc.}

{%%%%%   A-III-2
Pri zvyčajnom označení strán a~vnútorných uhlov trojuholníka~$ABC$
označme v~nasledujúcom texte ešte $I$ stred kružnice vpísanej, $I_a$
stred kružnice pripísanej k~strane~$BC$ a~body dotyku spomenutých kružníc
so stranou~$BC$ označme postupne $D$ a~$E$. Keďže osi $BI$ a~$BI_a$
oboch susedných uhlov pri vrchole~$B$ sú navzájom kolmé, čo
samozrejme platí aj~pre osi $CI$ a~$CI_a$, ležia body $B$, $C$, $I$ a~$I_a$ na
kružnici s~priemerom~$II_a$. Z~toho zrejme vyplýva, že body $D$ a~$E$ ako
kolmé priemety oboch krajných bodov priemeru~$II_a$ na tetivu~$BC$ sú
súmerne združené podľa stredu strany~$BC$.

\smallskip\noindent
{\it 1. postup.}
Pravouhlé trojuholníky $BID$ a~$I_aBE$ sú podobné, lebo
oba uhly $BID$ a~$I_aBE$ dopĺňajú uhol $CBI$ do~90\st{} (\obr). Platí teda
$$
|BD|:|ID|=|I_aE|:|BE|,\quad\text{čiže}\quad
|BD| \cdot |BE| = |ID| \cdot |I_aE|
$$
a~vzhľadom na uvedenú symetriu aj
$$
|BD|+|BE|= |BD|+|CD|= |BC| = r+r_a = |ID|+|I_aE|.
$$
\removelastskip
\insp{a65.9}%

Z~oboch rovností tak vyplýva, že dvojice čísel $(|ID|, |EI_a|)$
a~$(|BD|, |BE|)$ sú korene tej istej kvadratickej rovnice, a~tak
je $|ID|=|BD|$ alebo $|ID|=|BE|$.


Zrejme $|ID| = |BD|$ práve vtedy, keď je trojuholník $BID$ pravouhlý rovnoramenný,
čiže $\beta = 90^\circ$. A~podobne $|ID| = |BE|$, čiže $|ID| = |CD|$
(opäť vďaka uvedenej symetrickej polohe bodov $D$ a~$E$) práve vtedy, keď
je pravouhlý rovnoramenný trojuholník~$CID$. V~tom prípade je
$\gamma = 90^\circ$.

Tak či tak je trojuholník $ABC$ pravouhlý.

\smallskip\noindent
{\it 2. postup.}
Os tetivy $BC$ kružnice~$k$ nad priemerom $II_a$ je
osou pásu medzi rovnobežkami
$ID$ a~$I_aE$ (opäť vďaka symetrickej polohe bodov $D$ a~$E$ na $BC$).
Ak označíme~$I'$ obraz bodu~$I$ v~tejto súmernosti (\obr),
je zrejme $|I'I_a|=|I'E|+|EI_a|=|ID|+|EI_a|=r+r_a$, takže podľa predpokladu
$|BC| = r+r_a = |I_aI'|$. Zhodným tetivám $BC$ a~$I'I_a$ jednej kružnice
prislúchajú zhodné obvodové uhly.
\insp{a65.10}%

Ako ľahko spočítame (pozri napr. \obrr2), je
$|\uhol BIC|=90\st-\frac12\beta+90\st-\frac12\gamma=90\st+\frac12\alpha$
a~$|\uhol I'CI_a|=|\uhol I'CB|+|\uhol BCI_a|=\frac12\beta+(90\st-\frac12\gamma)
=\beta+\frac12\alpha$. Preto buď $90\st+\frac12\alpha=\beta+\frac12\alpha$,
čiže $\beta=90\st$, alebo $(90\st+\frac12\alpha)+(\beta+\frac12\alpha)=180\st$,
čiže $\gamma=90\st$.
Tým je tvrdenie úlohy dokázané.

\smallskip\noindent
{\it 3. postup.}
Ak označíme $P$ obsah trojuholníka~$ABC$, $s$ polovicu jeho obvodu a~ak položíme $x=s-a$,
$y=s-b$, $z=s-c$, dajú sa známe vzorce pre obsah~$P$ zapísať
zjednodušene takto:
$$
P^2 = xyzs, \quad P= rs, \quad P= r_ax.
$$
Z toho vypočítame
$$
r^2 = \frac{xyz}{s} \qquad \text{a} \qquad r_a^2 = \frac{yzs}{x}.
$$
Zadanú podmienku $r+r_a=y+z$ umocníme a~pomocou predchádzajúceho prepíšeme ako
$$
x^2yz+yzs^2=y^2xs+z^2xs,
$$
čiže
$$
(zs-xy)(ys-xz) = 0.
$$
Po spätnej substitúcii na $a$, $b$, $c$ získame po chvíli ekvivalentných úprav
očakávané
$$
(a^2+b^2-c^2)(a^2-b^2+c^2)=0,
$$
a~tvrdenie úlohy tak vyplýva z~Pytagorovej vety.
}

{%%%%%   A-III-3
Uvažujme klub $K$ s~najmenším počtom členov (ak je takých klubov
viac, vyberme ktorýkoľvek). Jednému jeho členovi (nazývajme ho Jakub) dáme
oba nástroje a~ostatným členom kružidlo. Všetci zvyšní obyvatelia dostanú
pravítko. Tvrdíme, že také rozdelenie rysovacích potrieb vyhovuje
podmienkam úlohy.

Každý klub, ktorého je Jakub členom, je určite vybavený. Aj keď do
nejakého klubu Jakub nepatrí, tak má tento klub s~$K$ nejakého
spoločného člena, a~teda je vybavený aspoň kružidlom. Ak by v~tomto
klube nebolo žiadne pravítko, znamenalo by to, že je celý obsiahnutý v~$K$
a~má pritom aspoň o~jedného člena menej (neobsahuje Jakuba). To je
spor s~voľbou~$K$, a~vidíme tak, že je skutočne každý klub riadne
vybavený.

\poznamka
Nie je ťažké si uvedomiť, že bez možnosti dať
jednému obyvateľovi oboje by záver úlohy neplatil. Uveďme tu pre
zaujímavosť dva také (a~pritom veľmi odlišné) prípady.

Jeden obyvateľ je členom všetkých klubov a~zároveň má aj svoj
jednočlenný klub. Tento klub potom samozrejme nebude oboma nástrojmi vybavený.

Pre $2n+1$ obyvateľov povedzme, že každých $n+1$ z~nich tvorí klub.
Potom naozaj nie sú žiadne dva kluby disjunktné, a~pritom akokoľvek
rozdáme pravítka a~kružidlá, tak keďže od jedného nástroja sme
rozdali aspoň $n+1$ kusov, nájdeme klub vlastniaci iba tento
nástroj.
}

{%%%%%   A-III-4
Zadanú rovnosť šikovne upravíme a~odhadneme pomocou
známej nerovnosti ${a^2+b^2 }\ge 2ab$ takto:
$$
4a = (a+c)(b^2+ac)
= a(b^2+c^2) + c(a^2+b^2) \geq a(b^2+c^2) + 2abc
% = a(b^2+c^2) + c(\underbrace{ a^2+b^2}_{\geq 2ab}) \geq a(b^2+c^2) + 2cab
= a(b+c)^2.
$$
Z toho jednak vidíme, že $b + c \le 2$, a~tiež, že rovnosť nastane
práve vtedy, keď ${0 < a=b < 2}$ a~$c = 2-b > 0$. To je všetko.

\ineriesenie
Uvažujme kvadratickú rovnicu
$$
4t = (t+c)(b^2+tc)
$$
s~neznámou $t$. Vďaka vzťahu zo zadania vieme, že táto rovnica má koreň $t = a$.
Rovnicu upravíme na tvar
$$
ct^2+ (b^2+c^2-4)t + cb^2 = 0
$$
a~všimneme si, že musí platiť $b^2+c^2-4 <0$. Inak by totiž
ľavá strana bola pre ľubovoľné kladné~$t$ kladná, čo je v~spore s~faktom, že
rovnica má kladný koreň.

Skutočnosť, že uvedená rovnica má nezáporný diskriminant, zapíšeme takto:
$$
(2bc)^2 \le (4-b^2-c^2)^2.
$$
Keďže sú oba základy mocnín kladné, môžeme nerovnosť odmocniť
a~následne upraviť na tvar $(b+c)^2 \le 4$, čiže $b+c \le 2$.

Rovnosť $b+c = 2$ nastane práve vtedy, keď má spomenutá rovnica nulový
diskriminant, a~teda dvojnásobný koreň, ktorým ale musí byť číslo~$a$. Keďže je
však súčin koreňov (podľa Vi\`etových vzťahov) rovný $b^2$, musí byť nutne
$a=b$. Ľahko potom overíme, že trojice $(r,r,2-r)$ pre ľubovoľné $r \in (0,2)$
naozaj rovnosti zo zadania vyhovujú.
}

{%%%%%   A-III-5
Označme $E$ druhý priesečník priamky $AD$
s~kružnicou~$k$ trojuholníku $ABC$ opísanou. Z~mocnosti bodu $D$ ku kružnici~$k$
vyplýva, že $|DB| \cdot |DC|=|DA| \cdot |DE|$, čo porovnaním so zadanou
podmienkou $|DA|^2=|DB| \cdot |DC|$ dáva $|DA|=|DE|$. Bod~$E$ teda leží
na obraze~$p$ priamky $BC$ v rovnoľahlosti so stredom~$A$
a~koeficientom~$2$ (\obr).

Aj~naopak platí, že k~ľubovoľnému priesečníku priamky $p$ s~kružnicou~$k$ spätne zostrojíme
bod~$D$ na strane~$BC$, ktorý bude zrejme spĺňať rovnosť $|DA|^2={|DB| \cdot |DC|}$.
Aby bol taký bod určený jednoznačne, musí sa priamka~$p$ v~bode~$E$ kružnice~$k$ dotýkať.
\inspinsp{a65.11}{a65.12}%

Označme $S_b$ a~$S_c$ postupne stredy úsečiek $AC$ a~$AB$. V~rovnoľahlosti so stredom~$A$ a~koeficientom $\frac{1}{2}$ sa
body $A$, $B$, $C$, $E$ ležiace na kružnici~$k$
zobrazia na body $A$, $S_c$, $S_b$, $D$ ležiace na kružnici~$k'$ (\obr),
pritom obrazom priamky~$p$ bude dotyčnica~$BC$ kružnice~$k'$ v~bode~$D$.
Z~mocnosti bodov $B$, $C$ k~tejto kružnici
potom dostávame $|BD|^2=|BA| \cdot |BS_c|=\frac12 |BA|^2$ a~$|CD|^2=|CA|
\cdot |CS_b|=\frac12 |CA|^2$. Spolu tak pre obvod trojuholníka~$ABC$ platí
$$
|BC|+|AB|+|AC|=|BC|+\sqrt{2}(|BD|+|CD|)
=|BC|+\sqrt{2} \cdot|BC|=1+\sqrt{2},
$$
čo je teda (jediná možná) veľkosť obvodu trojuholníka $ABC$.

\ineriesenie
V~trojuholníku $ABC$ s~bodom $D$
vnútri strany $BC$ so stranami dĺžok $a$, $b$, $c$ označme $|BD| = m$,
$|DC| = n$ a~$|AD| = d$. Podľa Stewartovej vety\footnote{Stewartovu
vetu možno odvodiť použitím kosínusovej vety v~trojuholníkoch $BAD$ a~$CAD$
(v~oboch prípadoch voči uhlu pri vrchole~$D$) a~následným vylúčením výrazov
s~kosínusom.} potom platí
$$
b^2m + c^2n = a(d^2+mn).
$$
Prípad $d^2 = mn$ tak nastane práve vtedy, keď bude platiť
$$
b^2m + c^2n = 2amn.
$$

V~našom prípade, keď $a= 1$, zavedením $m = x$, $n = 1-x$ pre $x \in
(0,1)$ získame po úprave rovnicu
$$
P(x) = 2x^2 + (b^2-c^2-2)x + c^2 = 0.
$$
Keďže $P(0) = c^2 > 0$ a~$P(1) = b^2 > 0$, nemôže mať $P(x) = 0$ dva
rôzne korene, z~ktorých práve jeden leží v~intervale $(0,1)$. Jediná
možnosť, ako zabezpečiť jednoznačnosť bodu~$D$, je dvojnásobný koreň
v~intervale $(0,1)$. Podľa Vi\`etových vzťahov musí tento dvojnásobný koreň
spĺňať $x^2 = \frac12 c^2$, čím sa o~polohe bodu~$D$ dozvedáme, že
nutne $m\sqrt{2}= c$. Úplne analogicky možno odvodiť aj rovnosť $n\sqrt{2}=b$.
Obvod trojuholníka potom spočítame ako $a+b+c = 1 +(m+n)\sqrt{2} = 1 + \sqrt{2}$.
}

{%%%%%   A-III-6
Pripusťme, že vhodná východisková pozícia a~postupnosť 35~skokov existuje,
a~očíslujme políčka šachovnice podľa nasledujúcej schémy:
$$
\vbox{\offinterlineskip\everycr{\noalign{\hrule\adv\rrr}}\rrr=-1
\halign{\vrule height.8\sqwaiiiLXV depth.2\sqwaiiiLXV\ccc=\rrr\hbox to\sqwaiiiLXV{\hss\adv\ccc\number\ccc\hss}\vrule
#&&\hbox to\sqwaiiiLXV{\hss\adv\ccc\number\ccc\hss}#\vrule\cr
&&&&&\cr
&&&&&\cr
&&&&&\cr
&&&&&\cr
&&&&&\cr
&&&&&\cr
}}
$$

Ťahy dĺžky jedna vedú z~nepárneho čísla na párne a~naopak. Ťahy dĺžky dva
vždy vedú z~párneho čísla na {\it iné\/} párne číslo a~z~nepárneho čísla
na {\it iné\/} nepárne číslo. Ak navštívené políčka označíme $P_1$,
$P_2$, \dots, $P_{36}$, z~uvedeného vyplýva, že medzi štyrmi políčkami
$P_2$, $P_3$, $P_4$, $P_5$ je každé z~čísel zastúpené práve raz (na
$P_2$ a~$P_3$ sú rôzne čísla s~rovnakou paritou a~podobne na $P_4$,
$P_5$ s~druhou paritou). Z~rovnakých dôvodov je každé z~čísel zastúpené práve
raz vo štvoriciach políčok $P_{4k+2}$, $P_{4k+3}$, $P_{4k+4}$, $P_{4k+5}$ pre
každé $k\in\{0, 1,\dots, 7\}$. Medzi číslami na políčkach $P_2$, $P_3$, \dots,
$P_{33}$ je tak každé z~čísel zastúpené dokopy 8-krát.

Číslo~4 je na šachovnici iba 8-krát, preto žiadne z~čísel na $P_1$, $P_{34}$,
$P_{35}$, $P_{36}$ nemôže byť~4. Čísla na $P_{34}$ a~$P_{35}$ majú rovnakú paritu
a~sú rôzne (delí ich skok dĺžky~2). Keďže na nich nie je číslo~4, musia
byť obe nepárne. Potom na políčku~$P_{36}$ musí byť párne číslo a~rovnako tak
párne číslo vychádza aj~na políčko~$P_1$. Na oboch tak musí byť číslo~2.

Počiatočné políčko teda musí byť niektoré z~vyfarbených na ľavej šachovnici.
Tento argument možno ale zopakovať aj pre druhé očíslovanie vpravo,
ktoré je len "otočením" očíslovania prvého. Keďže žiadne políčko
nie je vyfarbené súčasne na oboch šachovniciach, došli sme ku sporu. Šachovnicu
tak žiadaným spôsobom prejsť nemožno, nech je počiatočné políčko zvolené akokoľvek.
$$
\centerline{%
\vbox{\offinterlineskip\everycr{\noalign{\hrule\adv\rrr}}\rrr=-1
\halign{\vrule height.8\sqwaiiiLXV depth.2\sqwaiiiLXV\ccc=\rrr\hbox to\sqwaiiiLXV{\adv\ccc\grey\hss\number\ccc\hss}\vrule
#&&\hbox to\sqwaiiiLXV{\adv\ccc\grey\hss\number\ccc\hss}#\vrule\cr
&&&&&\cr
&&&&&\cr
&&&&&\cr
&&&&&\cr
&&&&&\cr
&&&&&\cr
}}\hss
\vbox{\offinterlineskip
\everycr{\noalign{\hrule\global\advance\rrr-1 \ifnum\rrr=0 \global\rrr=4 \fi}}\rrr=2
\halign{\vrule height.8\sqwaiiiLXV depth.2\sqwaiiiLXV\ccc=\rrr\hbox to\sqwaiiiLXV{\adv\ccc\grey\hss\number\ccc\hss}\vrule
#&&\hbox to\sqwaiiiLXV{\adv\ccc\grey\hss\number\ccc\hss}#\vrule\cr
&&&&&\cr
&&&&&\cr
&&&&&\cr
&&&&&\cr
&&&&&\cr
&&&&&\cr
}}}
$$
}

{%%%%%   B-S-1
Najskôr zistíme, ako sa dá zapísať číslo~14 ako súčet štyroch
z~daných čísel. Keďže $4+3\cdot3<14<4\cdot4$, musí taký súčet obsahovať
aspoň dve a~nanajvýš tri štvorky. Dostávame tak práve dve
% , ako zapísať číslo~14 ako súčet štyri z~daných čísel:
možné vyjadrenia:
$14 = 4+4+3+3$ a~$14 = 4+4+4+2$.

Predpokladajme, že tabuľku $3 \times 3$ máme správne vyplnenú danými
číslami, a~skúmajme, čo pre ňu musí platiť. Vieme, že každý zo
štyroch štvorcov $2 \times 2$ musí obsahovať buď čísla 4, 4, 4, 2, alebo
čísla 4, 4, 3, 3. Sústreďme sa na štvorce $2 \times 2$, ktoré obsahujú
číslo~2; pre také štvorce máme iba nasledujúce štyri možnosti:
$$
\tabulka

&2&4

&4&4


\qquad
\tabulka

&4&2

&4&4


\qquad
\tabulka

&4&4

&2&4


\qquad
\tabulka

&4&4

&4&2


$$
% Všimnime si, že také štvorce $2 \times 2$ neobsahujú žiadnu trojku.

V~ktorom políčku tabuľky $3 \times 3$ môže byť zapísané
číslo~2? Ak by bolo zapísané v~strednom políčku, museli by byť
vo všetkých ostatných políčkach štvorky a~neostalo by miesto pre trojky:
$$
\let\quad\enspace
\tabulka

&&&

&&2&

&&&


\quad \longrightarrow \quad
\tabulka

&4&4&

&4&2&

&&&


\quad \longrightarrow \quad
\tabulka

&4&4&4

&4&2&4

&&&


\quad \longrightarrow \quad
\tabulka

&4&4&4

&4&2&4

&4&4&


\quad \longrightarrow \quad
\tabulka

&4&4&4

&4&2&4

&4&4&4


$$
Ak by bolo číslo~2 zapísané uprostred krajného riadka či stĺpca
tabuľky, v~susedných políčkach by museli byť štvorky, ktorých znova nemáme dostatok:
$$
\let\quad\enspace
\tabulka

&&2&

&&&

&&&


\quad \longrightarrow \quad
\tabulka

&4&2&

&4&4&\phantom {4}

&&&


\quad \longrightarrow \quad
\tabulka

&4&2&4

&4&4&4

&&&


$$
Preto sú (obe) čísla~2 zapísané v~rohoch tabuľky.

Ak by boli dvojky v~protiľahlých rohoch, znova nevystačíme so štvorkami:
$$
\let\quad\enspace
\tabulka

&&& 2

&&&

&2&&


\quad \longrightarrow \quad
\tabulka

&&& 2

&4&4&

&2&4&


\quad \longrightarrow \quad
\tabulka

&& 4&2

&4&4&4

&2&4&


$$
Zostala teda jediná možnosť pre polohu čísel~2~-- musia byť v~susedných rohoch.
Následne doplníme samé štvorky do dvoch štvorcov $2 \times 2$, ktoré obsahujú
tieto dvojky a~do zvyšných políčok nám nezostáva iná možnosť ako vpísať
zvyšné tri trojky.
$$
\let\quad\enspace
\tabulka

&2&&2

&&&

&&&


\quad \longrightarrow \quad
\tabulka

&2&4&2

&4&4&

&&&


\quad \longrightarrow \quad
\tabulka

&2&4&2

&4&4&4

&&&


\quad \longrightarrow \quad
\tabulka

&2&4&2

&4&4&4

&3&3&3


$$
Dostaneme tak prvé riešenie, v~ktorom je súčet čísel vo
všetkých štyroch štvorcoch ${2 \times 2}$ naozaj rovný~14. Ďalšie tri riešenia
dostaneme voľbou inej dvojice susedných rohov.
% otáčaním štvorca.
Existujú tak štyri možnosti, ako tabuľku vyplniť:
$$
\tabulka

&2&4&2

&4&4&4

&3&3&3


\qquad
\tabulka

&3&4&2

&3&4&4

&3&4&2


\qquad
\tabulka

&3&3&3

&4&4&4

&2&4&2


\qquad
\tabulka

&2&4&3

&4&4&3

&2&4&3


$$


\ineriesenie
Rovnako ako v~predošlom riešení ukážeme, že číslo~14 je možné
z~daných čísel získať ako súčet štyroch čísel iba ako $4+4+4+2$ alebo
$4+4+3+3$. Z~toho ďalej vyplýva, že v~každom štvorci $2 \times 2$,
ktorý obsahuje číslo~2 (alebo~3) už žiadna dvojka byť nemôže. V~prostrednom
štvorčeku (ktorý je súčasťou štyroch štvorcov $2 \times 2$) nemôže byť
ani číslo~2 (neostalo by miesto pre trojky), ani číslo~3 (neostalo by miesto
pre dvojky). V~prostrednom políčku musí teda byť číslo~4.

Z~predošlého riešenia vieme, že v~políčku susediacom stranou s~prostredným štvorčekom
nemôže byť číslo~2, lebo by oba štvorce $2 \times 2$ obsahujúce túto
dvojku obsahovali už iba štvorky, a~tých nemáme dostatok. Uprostred musí teda byť
číslo~4 a~v~jednom z~rohov číslo~2. Tým sú určené čísla v~zodpovedajúcom
štvorci $2 \times 2$ tabuľky:
$$
\let\quad\enspace
\tabulka

&2&&

&& 4&

&&&


\quad \longrightarrow \quad
\tabulka

&2&4&

&4&4&

&&&


$$

Zostalo nám päť nevyplnených políčok, ktoré majú obsahovať čísla 2, 3, 3,
3, 4. Číslo~3 sa musí vyskytovať v~aspoň jednom z~dvoch štvorcov $2 \times 2$,
ktoré majú s~už vyplneným štvorcom
% $2 \times 2$
dve spoločné políčka (a~to dvakrát).
Zvoľme jeden z~nich (pri voľbe druhého bude postup úplne rovnaký
a~situácia symetrická). Ostatné čísla potom môžeme vpísať už iba jediným
% spôsobom, ak prihliadneme na poznatok, že číslo~2 musí byť v~rohu a~nie
spôsobom, keďže číslo~2 nemôže byť
v~rovnakom štvorci $2 \times 2$ s~číslom~3:
$$
\let\quad\enspace
\tabulka

&2&4&3

&4&4&3

&&&


\quad \longrightarrow \quad
\tabulka

&2&4&3

&4&4&3

&2&&


\quad \longrightarrow \quad
\tabulka

&2&4&3

&4&4&3

&2&4&


\quad \longrightarrow \quad
\tabulka

&2&4&3

&4&4&3

&2&4&3


$$

Otočením o~násobok 90\st\ dostaneme ďalšie tri odlišné vyplnenia. Ľahko
nahliadneme, že sú to všetky možnosti, ktoré dostaneme inou voľbou
počiatočného rohu s~dvojkou a~voľbou priľahlého štvorca s~dvoma trojkami.
Tabuľku možno teda vyplniť štyrmi spôsobmi.

\ineriesenie
Označme čísla vo vyplnenej tabuľke podľa obrázka písmenami od~$a$
po~$i$ a~spíšme rovnice pre jednotlivé štvorce $2 \times 2$:
$$
\tabulka

&$a$&$b$&$c$

&$d$&$e$&$f$

&$g$&$h$&$i$


\qquad \longrightarrow \qquad
\aligned
a+b+d+e=&14\rlap, \qquad (1) \\
b+c+e+f=&14\rlap, \qquad (2) \\
d+e+g+h=&14\rlap, \qquad (3) \\
e+f+h+i=&14\rlap. \qquad (4)
\endaligned
$$
Riešme túto sústavu rovníc, keď vieme, že čísla od $a$ po $i$ sú
v~niektorom poradí dve dvojky, tri trojky a~štyri štvorky. Číslo~$e$ sa
nachádza vo všetkých štyroch rovniciach. Ak by bolo $e = 2$, mala by
sústava rovníc (1)--(4) tvar
$$
\align
a+b+d=&12, \\
b+c+f=&12, \\
d+g+h=&12, \\
f+h+i=&12,
\endalign
$$
pričom jediný spôsob, akým možno zo zvyšných čísel 2, 3, 3, 3, 4, 4,
4, 4 súčtom troch dostať číslo~12, je $4+4+4$, a~tak by museli všetky
ostatné čísla mať iba hodnotu~4, čo nie je možné.

Ak by bolo $e = 3$, mala by sústava rovníc (1)--(4) tvar
$$
\aligned
a+b+d=&11, \\
b+c+f=&11, \\
d+g+h=&11, \\
f+h+i=&11,
\endaligned
$$
pričom jediný spôsob, akým možno zo zvyšných čísel 2, 2, 3, 3, 4, 4,
4, 4 súčtom troch dostať číslo~11, je $4+4+3$, a~tak by museli všetky
ostatné čísla mať iba hodnoty 4 a~3, čo nie je možné (nemáme kam
umiestniť dvojky).

Zistili sme, že musí byť $e = 4$, a~tak budeme hľadať riešenie sústavy
$$
\align
a+b+d=&10,\\
b+c+f=&10,\\
d+g+h=&10,\\
f+h+i=&10,
\endalign
$$
pričom čísla $a$, $b$, $c$, $d$, $f$, $g$, $h$, $i$ sú v~niektorom poradí
čísla 2, 2, 3, 3, 3, 4, 4, 4. Ak tieto štyri rovnice sčítame, dostaneme
$$
\align
(a+b+d)+(b+c+f)+(d+g+h)+(f+h+i)=&40, \\
(a+b+c+d+f+g+h+i)+b+d+f+h=&40, \\
(2+2+3+3+3+4+4+4)+b+d+f+h=&40, \\
b+d+f+h=&15 . \tag5
\endalign
$$

Jediný spôsob, ako dostať číslo~15 ako súčet štyroch čísel z~množiny
$\{2, 3, 4\}$, je $4+4+4+3$. Preto jediné riešenie rovnice~(5)
je také, že jedno z~čísel $b$, $d$, $f$, $h$ je rovné
trom a~zvyšné sú štvorky. Pre každú zo štyroch možností už z~rovníc
(1)--(4) spolu s~$e = 4$ jednoznačne dopočítame
riešenia $a~= 10-b-d$, $c = 10-b-f$, $g = 10-d-h$, $i~= 10-f-h$,
% $$
% \align
% b = 3 \ \longrightarrow\ d = f = h = 4 &\ \longrightarrow\ a~= 3, \ c = 3, \ g = 2, \ i~= 2, \\
% d = 3 \ \longrightarrow\ b = f = h = 4 &\ \longrightarrow\ a~= 3, \ c = 2, \ g = 3, \ i~= 2, \\
% f = 3 \ \longrightarrow\ b = d = h = 4 &\ \longrightarrow\ a~= 2, \ c = 3, \ g = 2, \ i~= 3, \\
% h = 3 \ \longrightarrow\ b = d = f = 4 &\ \longrightarrow\ a~= 2, \ c = 2, \ g = 3, \ i~= 3,
% \endalign
% $$
ktoré zodpovedajú týmto tabuľkám:
$$
\tabulka

&3&3&3

&4&4&4

&2&4&2


\qquad
\tabulka

&3&4&2

&3&4&4

&3&4&2


\qquad
\tabulka

&2&4&3

&4&4&3

&2&4&3


\qquad
\tabulka

&2&4&2

&4&4&4

&3&3&3


$$

% Ako skúšku správnosti ešte pripomíname, že v~každom z~týchto štyroch
% riešení sú čísla od~$a$ po~$i$ permutáciou zadaných čísel 2, 2, 3, 3,
% 3, 4, 4, 4, 4.

\nobreak\medskip\petit\noindent
Za úplné riešenie dajte 6~bodov.
Ak riešiteľ postupuje prvým spôsobom, dajte 1~bod za nájdenie oboch možných rozkladov
$14 = 4+4+4+2 = 4+4+3+3$, druhý bod za vylúčenie dvojky v~strede štvorca
$3 \times 3$, tretí bod za vylúčenie dvojky "uprostred strany" štvorca
$3 \times 3$. Štvrtý bod dajte za zistenie, že dvojky nemôžu byť
v~protiľahlých rohoch, piaty bod za prvé nájdené riešenie a~posledný bod
za vypísanie všetkých štyroch riešení.

Ak riešiteľ postupuje druhým spôsobom, dajte 1~bod za nájdenie
$14 = 4+4+4+2 = 4+4+3+3$, 2~body za určenie čísla~4 v~strede štvorca $3 \times3$.
Štvrtý bod dajte za zistenie, že dvojky nemôžu byť "uprostred
strany", piaty bod za prvé nájdené riešenie a~posledný bod za vypísanie
všetkých štyroch riešení.

Ak riešiteľ postupuje tretím spôsobom, dajte
% 1 bod za zápis sústav (1)--(4), ďalšie 2 body
3~body za zistenie hodnoty~$e=4$.
Štvrtý bod dajte za sčítanie rovníc, piaty bod za objav
rovnice~(5) a~posledný bod za vypísanie všetkých štyroch
riešení. Ak riešiteľ neuvedie štyri riešenia, dajte nanajvýš 5~bodov.
Ak riešiteľ postupuje inak, snažte sa hodnotiť podobné míľniky
v~riešení v~súlade s~uvedenými riešeniami.

\endpetit
\bigbreak}

{%%%%%   B-S-2
Body $D$ aj~$F$ ležia na kružnici~$m$ so stredom~$B$, takže
trojuholník $BDF$ je rovnoramenný a~platí $|\uhol BFK| = |\uhol BDF|>45\st$,
pretože trojuholník $BDF$ je navyše ostrouhlý (\obr). To ale znamená, že
bod~$K$ musí ležať v~polrovine $oA$, pričom $o$ je os úsečky~$AB$,
pretože oblúk~$BK$ prislúcha obvodovému uhlu väčšiemu ako~$45\st$.

Bod~$L$, ktorý je vďaka podmienke $AB\parallel KL$ súmerne združený s~$K$ podľa~$o$,
bude preto ležať v~polrovine~$oB$, teda $KL$ a~$AB$ (a~teda aj~$DB$)
budú súhlasne orientované rovnobežné úsečky. Z~toho vyplýva
zhodnosť súhlasných uhlov $LKF$ a~$BDF$. Spolu tak dostávame
$|\uhol LKF| = |\uhol BDF|=|\uhol BFK|$.
\insp{b65.4}%

Priamky $FB$ a~$KL$ sú potom súmerne združené podľa osi úsečky~$FK$, ktorá
prechádza stredom~$C$ kružnice~$k$, a~je teda aj jej osou súmernosti. Preto
aj priesečníky $B$ a~$L$ týchto priamok s~kružnicou~$k$ sú
súmerne združené podľa tejto osi, takže štvoruholník $KLBF$ je rovnoramenný
lichobežník, čiže $|KL| = |BF|$. Spojením s~rovnosťou $|BF| = |BD|$
polomerov kružnice~$m$ tak dostávame požadovanú rovnosť
$$
|KL| = |BF| = |BD|.
$$


\nobreak\medskip\petit\noindent
Za úplné riešenie dajte 6~bodov.
Za uvedenie každej z~rovností uhlov $|\uhol BFK| = |\uhol BDF|$
a~$|\uhol LKF| = |\uhol BDF|$ dajte po 2~bodoch. Ďalšími dvoma bodmi odmeňte
riešiteľa za dôkaz toho, že $KLBF$ je rovnoramenný lichobežník, resp.
za to, že $|KL| = |BF|$.
Poznatok, že body $B$ a~$L$ ležia na rovnakej strane od priamky~$KF$,
možno považovať za očividný, a~tak môže byť v~riešení využitý mlčky.
\endpetit
\bigbreak
}

{%%%%%   B-S-3
Najskôr si všimnime, že zámenou hodnôt $a$ a~$b$ sa hodnota posledných
dvoch uvažovaných výrazov nezmení, zatiaľ čo poradie prvých dvoch sa zmení na opačné.
Čísla $a$ a~$b$ sú rôzne, môžeme preto predpokladať, že je $a<b$. Pre
$a>b$ tak výmenou čísel $a$ a~$b$ medzi sebou dostaneme už niektoré nájdené poradie,
v~ktorom budú vymenené hodnoty $1+a$ a~$1+b$.

Keďže čísla $a$ a~$b$ sú rôzne, leží ich aritmetický priemer medzi nimi, takže
pre $a<b$ je vždy
$$
1+a<1+\frac {a+b}{2}<1+b.
$$

Ostáva zistiť, na ktoré miesto sa dá (za uvedeného predpokladu) zaradiť hodnota posledného výrazu.
Napríklad pre $a=2$ a~$b=4$ máme zaradiť hodnotu $4{,}5$ medzi čísla $3<4<5$.
Vidíme teda, že jedno z~možných usporiadaní všetkých štyroch uvažovaných hodnôt je
$$
1+a<1+ \frac {a+b} 2 <\frac {a^2+b^2-2}{a+b-2} <1+b. \tag1
$$

Ako sme už uviedli skôr, prvá nerovnosť platí za predpokladu $a<b$ vždy.
Ukážeme, že ďalšie dve nerovnosti v~(1) tiež platia všeobecne pre ľubovoľné $a$, $b$, $1<a<b$.

Každú z~oboch nerovností vynásobíme kladným výrazom $a+b-2$.
Ekvivalentnými úpravami ľavej nerovnosti ďalej dostaneme
$$
\align
\frac{2+a+b}{2}\cdot
% \tfrac 12(a+b+2)
(a+b-2)<& a^2+b^2-2, \\
(a+b)^2-4<& 2 (a^2+b^2-2), \\
0<& a^2+b^2-2ab ,\\
0<& (a-b)^2,
\endalign
$$
čo je nerovnosť, ktorá pre $a\ne b$ platí vždy.

Úpravou pravej nerovnosti potom dostaneme %postupne
$$
\align
a^2+b^2-2<&(1+b)(a+b-2) , \\
a^2+b^2-2<& a+b-2+ab+b^2-2b, \\
0<& {-a}^2+ab+a-b ,\\
0<& a(b-a)-(b-a) = (b-a)(a-1),
\endalign
$$
čo vďaka nerovnostiam $1<a<b$ platí tiež.

Vzhľadom na symetriu spomenutú v~úvode tak dostávame dve možné usporiadania
uvažovaných hodnôt:
$$
\align
1+a~<1+ \frac {a+b} 2 <\frac {a^2+b^2-2}{a+b-2} <1+b, \quad& \text{keď} \ a<b, \\
1+b <1+ \frac {a+b} 2 <\frac {a^2+b^2-2}{a+b-2} <1+a, \quad& \text{keď} \ b<a.
\endalign
$$


\nobreak\medskip\petit\noindent
Za úplné riešenie dajte 6~bodov.
Za správnu pozíciu hodnoty $1+\frc {(a+b)}{2}$ medzi
číslami $1+a$ a~$1+b$ dajte 1~bod. Dôkaz každej z~ďalších dvoch nerovností
(1) oceňte dvoma bodmi. Posledný bod dajte za uvedenie oboch možných usporiadaní.
\endpetit
\bigbreak
}

{%%%%%   B-II-1
Najskôr zlomky v~zadanej rovnici prevrátime (v~oboch čitateľoch sú kladné čísla)
a~čiastočne vydelíme:
$$
\align
\frac {3kl+k+3} {3l+1}=&\frac {5lm+m+5} {lm+1}, \\
\frac {k~(3l+1)+3} {3l+1}=&\frac {5 (lm+1)+m} {lm+1}, \\
k+\frac {3} {3l+1}=&5+ \frac {m} {lm+1}.
\endalign
$$
Keďže pre prirodzené čísla $l$ a~$m$ platí $3 <3l+1$ aj~$m <lm+1$, ležia
hodnoty oboch zlomkov z~poslednej rovnice v~intervale $(0,1)$.
Dostávame tak
$$
k~= 5 \quad \text {a~zároveň} \quad \frac {3} {3l+1} = \frac {m} {lm+1}. \tag1
$$

Z~druhej rovnice po roznásobení vyplýva $3lm+3 = 3lm+m$, preto
$m = 3$, zatiaľ čo $l$~môže byť ľubovoľné.

Úlohe vyhovujú všetky trojice $(k, l, m)=(5, l, 3)$, kde $l$ je ľubovoľné prirodzené
číslo.


\nobreak\medskip\petit\noindent
Za úplné riešenie dajte 6 bodov.
Za odvodenie $k= 5$ dajte 3 body, za $m = 3$ dva body a~ďalší bod
dajte za zistenie, že hodnota $l$ môže byť ľubovoľná. Ak riešiteľ
uhádne iba konečný počet riešení $(k, l, m)$, napríklad iba jediné riešenie $(5,1,3)$, dajte 1
bod, ak uhádne a~overí riešenie $(5, l, 3)$ pre ľubovoľné prirodzené
číslo $l$, dajte ďalší bod.
\endpetit
\bigbreak
}

{%%%%%   B-II-2
Kružnica $k$ je Tálesovou kružnicou nad priemerom~$AB$, takže
trojuholník $ABF$ je pravouhlý s~pravým uhlom pri vrchole~$F$. Inými
\insp{b65.5}%
slovami, priamka~$AF$ je kolmá na polomer~$BF$ kružnice~$m$, a~preto sa
priamka~$AF$ dotýka kružnice~$m$ v~bode~$F$ (\obr).
Z~rovnosti úsekového uhla zovretého tetivou~$DF$ s~dotyčnicou~$AF$
a~obvodového uhla nad tou istou tetivou máme (ako už je vyznačené na
obrázku)
$$
|\uhol AFD| = |\uhol DEF|. %\tag1
$$
Zo súmernosti úsečky $EF$ podľa osi~$AB$ tak vyplýva
$$
\postdisplaypenalty 10000
|\uhol AFD| = |\uhol DEF| = |\uhol DFE|,
$$
čo znamená, že $FD$ je osou uhla $AFE$.


\ineriesenie
Označme $\beta$ veľkosť uhla $ABF$ a~dopočítajme veľkosti
uhlov $DFE$ a~$AFE$. Trojuholník $DBF$ je rovnoramenný, lebo
jeho ramená $BD$ a~$BF$ sú polomery kružnice~$m$, preto
$$
|\uhol DFB| = \frac {1} {2} \left (180^\circ- \beta \right) = 90^\circ- \frac {\beta} {2}.
% \tag2
$$
Keďže podobne aj~trojuholník $EBF$ je rovnoramenný s~osou~$BD$, platí
$$
|\uhol EFB| = 90^\circ- \beta. %\tag3
$$
Spojením oboch predchádzajúcich rovností tak dostávame
$$
|\uhol DFE| = |\uhol DFB|-|\uhol EFB| = \frac {\beta} {2}. %\tag4
$$

Z~vlastností Tálesovej kružnice~$k$ nad priemerom~$AB$ vieme, že uhol
$AFB$ je pravý. Pritom jeho časť uhol $EFB$ má, ako sme už zistili,
veľkosť $90\st-\beta$, takže jeho druhá časť uhol $AFE$ má veľkosť~$\beta$,
čo je presne dvojnásobok veľkosti uhla~$DFE$. Tým sme dokázali, že
priamka~$FD$ je osou uhla~$AFE$.

\ineriesenie
Nad oblúkom $AE$ kružnice~$k$ sa zhodujú uhly $ABE$ a~$AFE$ (\obr).
Oblúku~$DE$ kružnice~$m$ prislúcha obvodový uhol $DFE$ a~stredový
uhol~$DBE$. Spolu tak dostávame
$$
|\uhol DFE|=\frac12|\uhol DBE|=\frac12|\uhol ABE|=\frac12|\uhol AFE|,
$$
čo dokazuje, že $FD$ je osou uhla $AFE$.
\insp{b65.6}%

\nobreak\medskip\petit\noindent
Za úplné riešenie dajte 6 bodov.
Pri neúplnom riešení sa snažte úmerne oceniť užitočnosť dokázaných vlastností,
napr. v~prvom postupe dajte 3 body za objav rovnosti $|\uhol AFD| = |\uhol DEF|$
a~zvyšné 3 body za využitie súmernosti. Samotné úvahy o~symetrii
oceňte nanajvýš dvoma bodmi.
Ak študent urobí zásadný pokrok, ale nedokáže všetko spojiť a~dôjsť
k~požadovanému záveru, strhnite 1 bod.
\endpetit
\bigbreak
}

{%%%%%   B-II-3
Najskôr zistíme, ako vyzerajú čísla $n$, ktoré majú práve šesť deliteľov.

Ak je číslo $n$ deliteľné tromi rôznymi prvočíslami $p$, $q$, $r$,
má aspoň osem rôznych deliteľov: 1, $p$, $q$, $r$,
$pq$, $pr$, $qr$, $pqr$. Číslo $n$ môže teda mať nanajvýš dva
prvočíselné delitele $p$ a~$q$.

Ak je číslo $n$ deliteľné dvoma rôznymi prvočíslami a~jeden z~prvočiniteľov v~prvočíselnom rozklade čísla~$n$ je aspoň
v~tretej mocnine, teda keď $p^3q$ delí číslo~$n$, má opäť číslo $n$ aspoň
osem deliteľov: 1, $p$, $p^2$, $p^3$, $q$, $pq$, $p^2q$, $p^3q$. Ostáva rozobrať čísla deliteľné dvoma prvočíslami, každým nanajvýš v~druhej mocnine.
Číslo $n = pq$ má iba štyri delitele (1, $p$, $q$ a~$pq$), číslo $n = p^2q^2$
má deväť deliteľov (1, $p$, $p^2$, $q$, $q^2$, $pq$, $p^2q$, $q^2p$, $p^2q^2$), jedine
číslo tvaru $n = p^2q$ má práve šesť deliteľov (1, $p$, $p^2$, $q$, $pq$, $p^2q$).

Napokon ak je číslo~$n$ mocninou jediného prvočísla, $n = p^k$,
má $k+1$ deliteľov: $1, p,\allowbreak p^2, \dots, p^k$. V~tomto prípade tak
vyhovuje iba $k= 5$.

Zistili sme, že číslo $n$ so šiestimi deliteľmi má jeden z~tvarov $p^5$ alebo $p^2q$,
pričom $p$ a~$q$ sú rôzne prvočísla. Obe tieto možnosti ďalej preskúmame.

Ak $n = p^5$, dajú sa delitele čísla $n$ usporiadať podľa veľkosti:
$1<p<p^2<p^3<p^4<p^5$, takže má podľa zadania úlohy platiť
$p+p^4 = p(p^3+1) = 54 = 2 \cdot3^3$.
Prvočíslo $p$ je deliteľom čísla~54, preto $p\in\{2, 3\}$. Zároveň aj~bez
počítania vidíme, že $2(2^3+1)<2 \cdot3^3<3(3^3+1)$, takže ani jedno~$p\in\{2, 3\}$
požadovanému vzťahu nevyhovuje.

Ak $n = p^2q$, rozlíšime dva prípady, $p<q$ a~$q<p$.

Ak je $p<q$, je $p$ druhý najmenší deliteľ čísla~$n$ (najmenšia je
jednotka). Najväčším deliteľom čísla~$n$ je samo číslo~$p^2q$
a~druhým najväčším je $pq$, lebo
$pq$ je väčšie ako každý z~ostatných
štyroch deliteľov $1$, $p$, $q$ aj~$p^2$. Hľadáme teda
riešenie rovnice $p+pq = {p(1+q)} = 2\cdot3^3$. Navyše vieme, že $q$ je
nepárne (je väčšie ako prvočíslo~$p$), takže $1+q$ je párne, a~teda
musí byť $p = 3$. Potom $q = 17$, čo vedie na riešenie
$n = 3^2 \cdot17 = 153$.

Ak je $q<p$, dajú sa delitele čísla $n = p^2q$ usporiadať
podľa veľkosti: $1 <q <p <pq <p^2 <p^2q$. Hľadáme teda riešenie rovnice
$q+p^2 = 54$. Keďže $p$ aj~$q$ sú prvočísla, je $q\ge2$, $p\ge3$
a~tiež $p^2\le52$, čiže $p\le7$, preto $p \in \{3, 5, 7\}$.
Zároveň však vidíme, že $q= 54-p^2$ môže byť prvočíslo menšie ako~$p$
iba pre $p=7$, potom $q=5$ a~$n=49\cdot5=245$, čo je ďalšie riešenie.

\odpoved
Úloha má dve riešenia $153 = 3^2 \cdot 17$ (s~deliteľmi
$1 <\bold{3} <9 <17 <\bold{51} <153$) a~$245 = 7^2 \cdot 5$ (s~deliteľmi
$1 <\bold{5} <7 <35 <\bold{49} <245$).

\poznamka
Úvodný rozbor možno podstatne skrátiť, ak využijeme známe tvrdenie, že číslo~$n$
s~rozkladom na súčin prvočísel
$n = p_1^{\alpha_1} p_2^{\alpha_2}\dots p_m^{\alpha_m}$ má práve
$$
(\alpha_1+1)(\alpha_2+1)\dots(\alpha_m+1)
$$
deliteľov. Číslo 6 sa dá takto napísať iba dvoma spôsobmi $6 = 2 \cdot 3$,
ktorým zodpovedajú buď $m = 1$ a~$\alpha_1 = 5$, teda $n=p^5$, alebo $m = 2$
a~$\alpha_1=1$, $\alpha_2=2$, teda $n=p^2q$.


\ineriesenie
Najmenší deliteľ čísla $n$ je~1, druhý najmenší deliteľ je najmenší
prvočíselný deliteľ čísla~$n$~-- označme ho~$p$.
Najväčší deliteľ čísla $n$ je samotné číslo $n$ a~druhý najväčší
deliteľ je $n / p$. Máme teda riešiť rovnicu
$$
p+\frac {n} {p} = 54.
$$

Keďže číslo $n$ má šesť deliteľov, musí platiť $p <\frc np$, takže
$p <\frc {54} {2} = 27$. Stačí teda vyskúšať prvočísla $p$ menšie ako~27.
Pre každé také prvočíslo dopočítame $n = p (54-p)$ a~overíme počet
jeho deliteľov:
$$
\vbox{\offinterlineskip\let\\=\cr %\catcode`@\active\def@{\phantom0}
\halign{\strut\hss#\unskip\enspace \vrule &\enspace #\unskip\hss&\quad#\unskip\hss&\hss#\unskip\hss \\
$p$& $n = p (54-p)$ &delitele čísla $n$ &počet deliteľov \\
\noalign{\hrule}
2 & $104 = 2^3 \cdot 13$ &1, 2, 4, 8, 13, 26, 52, 104 &8 \\
3 & $\bold{153} = 3^2 \cdot 17$ &1, 3, 9, 17, 51, 153 &$\bold 6$ \\
5 & $\bold{245} = 5 \cdot 7^2$ &1, 5, 7, 35, 49, 245 &$\bold 6$ \\
7 & $329 = 7 \cdot 47$ &1, 7, 47, 329 &4 \\
11 & $473 = 11 \cdot 43$ &1, 11, 43, 473 &4 \\
13 & $533 = 13 \cdot 41$ &1, 13, 41, 533 &4 \\
17 & $629 = 17 \cdot 37$ &1, 17, 37, 629 &4 \\
19 & $665 = 5 \cdot 7 \cdot 19$ &1, 5, 7, 19, 35, 95, 133, 665 &8 \\
23 & $713 = 23 \cdot 31$ &1, 23, 31, 713 &4 \\
\noalign{\hrule}
}}
$$
Z~tabuľky vidíme, že práve šesť deliteľov majú iba čísla 153 a~245.


\nobreak\medskip\petit\noindent
Za úplné riešenie dajte 6 bodov.
Ak riešiteľ postupuje podľa prvého riešenia, dajte prvý bod za odvodenie
oboch možných prvočíselných rozkladov čísla $n$, ďalšie dva body dajte za vyriešenie
prípadu $n = p^5$ a~tri body za vyriešenie prípadu $n = p^2q$,
z~toho 1 bod za rozbor prípadov $p<q$ a~$p>q$ a~zostavenie oboch
rovníc a~po 1 bode za ich vyriešenie.
Ak riešiteľ postupuje podľa druhého riešenia, dajte 2~body
za odvodenie rovnice $n+\frc {n}{p} = 54$, za ohraničenie hodnoty $p$ zhora
dajte ďalšie dva body a~zvyšné dva body dajte za korektné preverenie
všetkých prípustných hodnôt $p$.

\endpetit
\bigbreak
}

{%%%%%   B-II-4
Adamovi na výhru stačí, aby rozdiel dvoch najmenších čísel, ktoré Braňo
vyberie, bol aspoň~23. Ak teda Adam napíše na tabuľu niektoré z~čísel
$$
a_1 = 100, \ a_2 = 123, \ a_3 = 146, \ \dots , \ a_{40} = 997 \
(= 100+39 \cdot 23), \tag1
$$
Braňo určite prehrá, pretože každé dve čísla z~postupnosti~(1) majú
rozdiel aspoň~23, lebo ich rozdiely sú násobkami čísla~23.
Vidíme teda, že v~prípade $k\le 40$ stačí Adamovi
na výhru napísať na tabuľu ľubovoľných $k$~čísel z~postupnosti
$a_1, a_2,\dots,a_{40}$.

Avšak Adamovi na~výhru stačí, aby len jeden z~rozdielov medzi dvoma
najmenšími a~medzi dvoma najväčšími z~vybraných štyroch čísel bol aspoň~23.
Ukážeme, že Adam dokáže vyhrať aj v~prípadoch $k=41$ a~$k=42$.

Doplňme k~číslam $a_1, a_2, \dots, a_{40}$ z~(1) ešte
$a_{41} = 998$ a~$a_{42} = 999$. Aj~pre $k\in\{41,42\}$ stačí,
keď Adam napíše na tabuľu čísla $a_1, \dots,a_{k}$.
Bez ohľadu na to, aké štyri čísla Braňo vyberie, budú dve najmenšie
nutne z~množiny $\{a_1, a_2, \dots, a_{40}\}$, ktorej každé dva prvky
sa líšia aspoň o~23, preto vyhrá Adam.

Teraz ukážeme, že pre $k\ge43$ môže vždy vyhrať Braňo bez ohľadu na to,
ktoré čísla Adam na tabuľu napísal. Označme ich $a_1, a_2, \dots, a_k$
od najmenšieho po najväčšie, teda $100\le a_1<a_2<\dots<a_k\le999$,
a~pozrime sa na rozdiely dvoch susedných čísel
$d_2 = a_2-a_1$, $d_3 = a_3-a_2$,~\dots, $d_{k} = a_k-a_{k-1}$.
Označme tri najmenšie z~nich ako $d_p = a_p-a_{p-1}$, $d_q = a_q-a_{q-1}$
a~$d_r = a_r-a_{r-1}$, pričom $2\le p<q<r\le k$ (taká voľba nemusí
byť jednoznačná, to však pre ďalšie úvahy nebude dôležité).
Zrejme potom platí $p+1\le q<r$, z~čoho $p<r-1$,
takže čísla $a_{p-1}$, $a_p$, $a_{r-1}$ a~$a_r$ spĺňajú nerovnosti
$$
a_{p-1}<a_p<a_{r-1}<a_{r},
$$
a~preto je $d_p$ rozdiel dvoch najmenších a~$d_r$ rozdiel dvoch najväčších z~nich.
Presvedčme sa, že Braňo vyhrá, keď vyberie tieto štyri čísla.

Na výhru Braňo potrebuje, aby bolo $d_p\le 22$ aj~$d_r\le 22$. Pripusťme, že
to tak nie je, čiže aspoň jedno z~čísel $d_p$, $d_q$, $d_r$ je najmenej~23 a~zvyšné dve
sú aspoň~1. Keďže $d_p$, $d_q$, $d_r$ sú tri najmenšie rozdiely, všetky ostatné
rozdiely~$d_i$ sú aspoň~23. Pre súčet všetkých rozdielov tak dostávame odhad
$$
\align
d_k+d_{k-1}+\dots
+ d_2&=(d_p+d_q+d_r)+\sum_{i\in\{2, \dots, k\} \setminus \{p, q, r\}}d_i \ge \\
& \ge(1+1+23)+(k-4) \cdot 23 = 2+23 \cdot (k-3). \tag2
\endalign
$$
Na druhej strane všetky čísla $a_i$ sú trojciferné, a~teda
$$
\align
d_k+d_{k-1}+\dots+d_2&=(a_k-a_{k-1})+(a_{k-1}-a_{k-2})+\dots+ (a_2-a_1) = \\
&=a_k-a_1\le999-100 = 899.
\endalign
$$
Spojením predchádzajúcich dvoch nerovností prichádzame ku sporu, lebo pre $k\ge 43$ je
$$
922 = 2+23 \cdot 40 \le 2+23 \cdot (k-3) \le d_k+d_{k-1}+\dots+d_2 \le 899.
$$

Dokázali sme, že opísaným výberom čísel $a_{p-1}$, $a_p$, $a_{r-1}$ a~$a_r$ si Braňo
zabezpečí výhru pre ľubovoľné $k\ge 43$.

\odpoved
Pre $k\le 42$ má vyhrávajúcu stratégiu Adam a~pre $k\ge 43$ Braňo.

\nobreak\medskip\petit\noindent
Za úplné riešenie dajte 6 bodov.
Ak riešiteľ iba ukáže, že hodnoty $k\le 40$, prípadne $k\le 41$
umožňujú Adamovi výhru, dajte 1 bod. Dva body dajte v~prípade, že riešiteľ uvedie
správnu hranicu $k\le 42$ aj s~uvedením čísel, ktoré má Adam napísať na
tabuľu (čísla $100, 123,\dots$ nie sú jedinou možnosťou).
Zvyšné 4 body dajte za dôsledný popis vyhrávajúcej stratégie pre
Braňa v~prípade $k\ge 43$; najťažšou časťou je dolný odhad súčtu
rozdielov, preto nerovnosť (2) alebo jej ekvivalent ohodnoťte 2
bodmi, zvyšný popis nanajvýš dvoma bodmi.
\endpetit
}

{%%%%%   C-S-1
V~rovnici zo zadania
$$
1\,000a+100b+10c+d=20(10a+b)+16(10c+d)
$$
majú neznáme cifry $a$ a~$b$ väčšie koeficienty na ľavej strane,
zatiaľ čo cifry $c$ a~$d$ na strane pravej. Preto rovnicu upravíme
na tvar $800a+80b=150c+15d$, ktorý po vydelení piatimi a~vyňatí
menších koeficientov oboch strán prepíšeme ako
$$
16(10a+b)={3(10c+d)}.
\tag1
$$
Z~toho vďaka nesúdeliteľnosti čísel 3 a~16 vyplýva, že $10c+d$ je dvojciferný
násobok čísla~16. Ten je však väčší ako~48, lebo $3\cdot48=144$,
zatiaľ čo $16(10a+b)\ge160$ (cifra~$a$ musí byť nenulová).
Ako hodnoty $10c+d$ tak prichádzajú do úvahy iba násobky~16
rovné 64, 80 a~96 -- čísla určujúce svojim zápisom cifry $c$ a~$d$.
Dosadením do rovnice \thetag1 dostaneme pre dvojciferné číslo $10a+b$
postupne hodnoty 12, 15 a~18.
% , čo sú vlastne dvojice cifier $a$ a~$b$.

\odpoved
Vyhovujú tri čísla $1\,264$, $1\,580$ a~$1\,896$.

\poznamka
Namiesto štyroch neznámych cifier $a$, $b$, $c$, $d$
možno na zápis rovnice zo zadania využiť zrejme priamo
obe dvojciferné čísla $x=\overline{ab}$ a~$y=\overline{cd}$. Rovnica potom bude mať tvar
$100x+y=20x+16y$, ktorý podobne ako v~pôvodnom postupe
upravíme na $80x=15y$, čiže $16x=3y$. Teraz namiesto vzťahu
$16\mid y$ môžeme využiť druhý podobný dôsledok $3\mid x$ a~uvedomiť si,
že z~odhadu $y\le99$ vyplýva $16x\le3\cdot99=297$, odkiaľ
$x\le18$, čo spolu s~odhadom $x\ge10$ vedie k~možným hodnotám
$x\in\{12,15,18\}$. Z~rovnice $16x=3y$ potom už
dopočítame $y=64$ pre $x=12$, $y=80$ pre $x=15$ a~$y=96$ pre $x=18$.

\nobreak\medskip\petit\noindent
Za úplné riešenie dajte 6~bodov, logicky správny postup
s~numerickými chybami (podľa ich miery) oceňte nanajvýš 4~bodmi.
Pri nedokončenom postupe dajte 1~bod za zápis zadanej rovnice
v~rozvinutom tvare (\tj. s~mocninami základu~10), za jeho úpravu
na súčinový tvar s~nesúdeliteľnými činiteľmi 16 a~3 potom tiež 1~bod.

\endpetit
\bigbreak}

{%%%%%   C-S-2
a) Ukážeme, že v~opísanej situácii mohol na začiatku pri stole
sedieť ľubovoľný počet ľudí väčší ako~2. Dvaja hráči to totiž byť nemohli
(to by v~prvom kole vypadli z~hry buď obaja, alebo žiadny
z~nich, rozdelenie dvoch hlasov je totiž buď $1:1$, alebo $2:0$).

Ak sú na začiatku hráči aspoň traja, tak v~prvom kole vypadne iba
jeden hráč~$A$, keď napríklad hráč~$A$ dá hlas sebe
a~všetci ostatní (sú najmenej dvaja) ho dajú tomu istému hráčovi~$B$, $B\ne A$
(teda aj hráč~$B$ dá hlas sebe). Nie je to samozrejme jediný spôsob
hlasovania s~požadovaným výsledkom.

b) Vysvetlíme, prečo jediný hráč v~hre nikdy zostať nemôže.
Opak by znamenal, že v~poslednom kole pred uvedenou situáciou, keď
v~hre bolo
povedzme $m$~hráčov, pričom $m>1$, by v~dôsledku ich
hlasovania vypadlo $m-1$~hráčov. Keďže pri tomto
hlasovaní bolo rozdaných práve $m$~hlasov a~$m-1$~hráčov
(tí, čo potom vypadli) dostalo práve jeden hlas,
musel aj zvyšný $m$-tý hráč dostať práve jeden (zvyšný) hlas,
a~teda tiež vypadnúť, a~to je spor.

\nobreak\medskip\petit\noindent
Za úplné riešenie časti~a) dajte 2~body, za
úplné riešenie časti~b) dajte 4~body. Za drobné argumentačné
nedostatky strhnite 1\,--\,2 body.

\endpetit
\bigbreak}

{%%%%%   C-S-3
Želaný vzťah medzi obvodmi trojuholníkov $ACD$ a~$SBC$ vyplynie, keď
pre dĺžky ich strán objavíme nerovnosti
$$
|AC|<2|SB|,\quad|AD|<2|SC|\quad\text{a}\quad|CD|<2|BC|.
$$
Prvé dve z~nich sú dôsledkom toho, že tetivy $AC$ a~$AD$ danej
kružnice sú kratšie ako jej priemer~$AB$ (\obr), tretia nerovnosť zapísaná v~tvare $\frac12|CD|<|BC|$ je nerovnosťou medzi dĺžkami odvesny
a~prepony dvoch zhodných pravouhlých trojuholníkov, na ktoré je trojuholník $BCD$
rozdelený priamkou~$AB$, ktorá je totiž (vďaka predpokladu ${AB\perp CD}$) osou tetivy~$CD$. Dodajme, že rovnako dobre možno využiť
aj trojuholníkovú nerovnosť $|CD|<|BC|+|BD|=2|BC|$.
\insp{c65.5}%

\ineriesenie
Označme $\al$ veľkosti vnútorných uhlov pri základni~$AC$ rovnoramenného
trojuholníka~$SAC$. Potom jeho vonkajší uhol pri vrchole~$S$, čiže uhol $CSB$,
má veľkosť~$2\al$, ktorú má aj uhol $CAD$, pretože
polpriamka~$AB$ je jeho osou (\obrr1).
Rovnoramenné trojuholníky $ACD$ a~$SCB$ sa
tak zhodujú vo vnútorných uhloch pri svojich hlavných
vrcholoch $A$ a~$S$, a~sú teda podobné. Preto
je pomer ich obvodov rovný pomeru dĺžok ich ramien, a~ten má
naozaj hodnotu menšiu ako~2, lebo ramená trojuholníka~$ACD$
sú kratšie ako priemer danej kružnice, zatiaľ čo ramená
trojuholníka~$SCB$ majú dĺžku jej polomeru.

\nobreak\medskip\petit\noindent
Za úplné riešenie dajte 6~bodov. Za odhad dĺžok $AC$, $AD$ pomocou
priemeru danej kružnice v~neúplnom riešení dajte 1~bod. Za
nerovnosť $|CD|<2|BC|$ dajte 1~bod, len ak je zdôvodnená;
ak chýba toto zdôvodnenie v~inak úplnom riešení, strhnite 1~bod.
Napokon 1~bod dajte aj za objav, že trojuholníky $ACD$ a~$SCB$ sú podobné.
Len za zmienku o~ich rovnoramennosti však žiadny bod neudeľujte.

\endpetit
\bigbreak
}

{%%%%%   C-II-1
Označme daný výraz $V$ a~upravme ho dvojakým použitím úpravy nazývanej {\it doplnenie na
štvorec\/}:
$$
V=3x^2-12xy+y^4=3(x-2y)^2-12y^2+y^4=3(x-2y)^2+(y^2-6)^2-36.
$$
Zrejme platí $(x-2y)^2\ge0$, takže najmenšiu hodnotu výrazu~$V$
{\it pri pevnom~$y$} dostaneme, keď položíme $x=2y$.
Ostáva preto nájsť najmenšiu možnú hodnotu mocniny $(y^2-6)^2$
s~nezápornou celočíselnou premennou~$y$. Keďže
$y^2\in\{0, 1, 4, 9, 16, 25,\dots\}$ a~číslo 6 padne medzi čísla 4 a~9
tejto množiny, platí pre každé celé číslo $y$ nerovnosť
$$
(y^2-6)^2\ge\min\bigl((4-6)^2,(9-6)^2\bigr)=\min\{4,9\}=4.
$$
Pre ľubovoľné celé čísla $x$ a~$y$ tak dostávame odhad
$$
V\ge 3\cdot0+4-36=\m32,
$$
pritom rovnosť $V=\m32$ nastáva pre $y=2$ a~$x=2y=4$.

\odpoved
Hľadaná najmenšia možná hodnota daného výrazu je
$\m32$.

\nobreak\medskip\petit\noindent
Za úplné riešenie dajte 6 bodov, z~toho 3 body za potrebnú úpravu
výrazu, 1~bod za určenie minima mocniny $(y^2-6)^2$ (stačí
konštatovať, že $2^2$ je druhá mocnina najbližšia k~číslu 6),
1~bod za určenie hľadanej hodnoty $\m32$ a~1~bod za uvedenie
dvojice $(x,y)$, pre ktorú sa táto hodnota dosahuje.

\endpetit
\bigbreak
}

{%%%%%   C-II-2
Rôzne farby musia mať nielen štyri hrany každej
steny kocky, ale aj každé tri hrany, ktoré vychádzajú z~rovnakého vrcholu kocky (keďže každé dve z~nich patria jednej stene).
Tento úvodný postreh budeme v~celom riešení mlčky opakovane
využívať.

Začneme ofarbením hrán steny $ABCD$,
farby jej hrán $AB$, $BC$, $CD$, $DA$ označíme postupne číslami
1, 2, 3, 4. Pre výber farby 1 máme 4 možnosti, pre výber farby~2
už iba 3 možnosti atď., takže počet všetkých ofarbení hrán steny $ABCD$
je rovný ${4\cdot3\cdot2\cdot1}=24$. Vyberieme jedno z~nich a~ďalej
budeme uvažovať o~možnostiach ofarbenia zvyšných ôsmich hrán kocky.

Podľa farby 1 hrany $AB$ a~farby~4 hrany~$AD$ vidíme, že hrana
$AE$ môže mať farbu 2 alebo 3. Rozoberme podrobne prvý prípad,
keď $AE$ má farbu 2. Poznáme teda ofarbenie piatich hrán, ako
je vyznačené na prvej kocke zo série na \obr,
\insp{c65.6}%
ktorý ukazuje, ako
postupne určiť (už jednoznačne dané) ofarbenie zvyšných siedmich hrán.
V~každom kroku nad šípkou uvádzame hranu, ktorej farbu práve
určujeme. Opíšeme prvý krok: hrana~$DH$ nemôže mať ani farby~3
a~4 (kvôli hranám $DC$ a~$DA$), ani farbu~2 (kvôli stene $ADHE$),
má teda farbu~1. Takto argumentujeme aj v~ďalších krokoch;
na poslednej kocke dostávame ofarbenie všetkých jej hrán,
ktoré naozaj vyhovuje podmienke úlohy.

Obdobným postupom možno získať (jediné) vyhovujúce ofarbenie všetkých
hrán kocky aj v druhom prípade, keď hrana $AE$ má farbu~3
(\obr). Podrobnú sériu sme vynechali, najmä preto, že
počiatočnú kocku z~\obrr1{} (vrátane určených farieb) možno previesť na
počiatočnú kocku z~\obrr2{}, keď ju najskôr zobrazíme
v~súmernosti podľa roviny $ACGE$ a~potom navzájom vymeníme farby
$1\leftrightarrow4$, $2\leftrightarrow3$ (a~označenie vrcholov
$B\leftrightarrow D$, $F\leftrightarrow H$).
\insp{c65.7}%

Overili sme, že každé z~24 možných ofarbení hrán steny $ABCD$
sa dá rozšíriť práve dvoma spôsobmi na vyhovujúce ofarbenie všetkých
12~hrán kocky. Preto je ich celkový počet rovný $2\cdot24=48$.


\ineriesenie
Opäť budeme opakovane využívat postreh
z~úvodu prvého riešenia, tentoraz však v~odlišnom postupe
založenom na poznatku, že {\it žiadne dve
rovnobežné hrany kocky nemôžu mať rovnakú farbu}. Stačí to
dokázať iba pre dve rovnobežné hrany, ktoré neležia v~jednej stene
kocky, bez ujmy na všeobecnosti napríklad pre hrany $AB$
a~$GH$. Keby naopak mali rovnakú farbu,
nemohla by ju mať už žiadna z~hrán $BC$, $BF$, $GC$, $GF$,
čiže by ju nemala žiadna z~hrán steny $BCGF$, a~to je spor.

Ak označíme farby hrán $AB$, $BC$, $CD$, $DA$ opäť postupne číslami
1, 2, 3, 4, tak podľa dokázaného poznatku
majú v~hornej stene $EFGH$ farbu~1 a~3 hrany $FG$ a~$EH$~--
nevieme však v~akom poradí;
obe možnosti sú vykreslené na \obr.
\insp{c65.8}%
V~akom poradí sú farby~2 a~4 zvyšných hrán $EF$ a~$GH$
hornej steny? Na ľavej kocke podľa farieb $AB$, $AD$ a~$EH$
vidíme, že $AE$ má farbu~2, takže v~hornej podstave má $EF$ farbu~4
a~$GH$ farbu~2. Podobne na pravej kocke má $BF$ farbu~4, takže
$EF$ má farbu~2 a~$GH$ farbu~4. Jednoznačné ofarbenie
zvyšných "zvislých" hrán oboch kociek je už jednoduché; pre
ľavú kocku dostaneme výsledné ofarbenie z~\obrr3, pre pravú
kocku ofarbenie z~\obrr2. Týmto
postupom znova prichádzame k~záveru, že hľadaný počet vyhovujúcich
ofarbení danej kocky $ABCDEFGH$ je rovný dvojnásobku počtu
výberov farieb pre hrany steny $ABCD$.

\nobreak\medskip\petit\noindent
Za úplné riešenie dajte 6 bodov. Za určenie počtu 24 všetkých
ofarbení jednej steny dajte 1 bod, rovnako tak ohodnoťte postreh
o~rôznych farbách ľubovoľných troch hrán so spoločným krajným bodom.
Pri neúplnom postupe z~prvého riešenia dajte 1~bod za rozbor dvoch
možností pre farbu (piatej) hrany~$AE$, určovanie farieb zvyšných
siedmich hrán môže byť v~úplnom riešení opísané iba pre
prvú z~týchto hrán a~pre ďalšie bez postupných obrázkov.
Pri neúplnom postupe z~druhého riešenia dajte 2~body
za zdôvodnenie poznatku o~farbách ľubovoľných dvoch rovnobežných hrán.

\endpetit
\bigbreak
}

{%%%%%   C-II-3
V~pravouholníku $APCD$ označme $c=|CD|=|AP|$ a~$v=|AD|=|CP|$
(\obr, pričom sme už
vyznačili ďalšie dĺžky, ktoré odvodíme v~priebehu riešenia).\footnote{Keďže
podľa zadania uhlopriečka~$BD$ pretína
výšku~$CP$, musí jej päta~$P$ ležať medzi bodmi $A$ a~$B$, takže
sa jedná o~"zvyčajný" lichobežník $ABCD$ s~dlhšou základňou~$AB$
a~kratšou základňou~$CD$.}
\insp{c65.9}%

Z~predpokladu $S_{APCD}=\frac12 S_{ABCD}$ vyplýva pre druhú
polovicu obsahu $ABCD$ vyjadrenie $\frac12S_{ABCD}=S_{PBC}$, takže
$S_{APCD}=S_{PBC}$ čiže
$cv=\frac12|PB|v$, odkiaľ vzhľadom na to, že $v\ne0$, vychádza $|PB|=2c$,
v~dôsledku čoho $|AB|=3c$.

Trojuholníky $CDK$ a~$PBK$ majú pravé uhly pri vrcholoch $C$, $P$
a~zhodné (vrcholové) uhly pri spoločnom vrchole $K$, takže sú
podľa vety $uu$ podobné, a~to s~koeficientom $|PB|:|CD|=2c:c=2$.
Preto tiež platí $|PK|:|CK|=2:1$, odkiaľ $|KP|=\frac23v$
a~$|CK|=\frac13 v$.

Posudzované obsahy trojuholníkov $ABC$ a~$BCK$
tak majú vyjadrenie
$$
S_{ABC}=\frac{|AB|\cdot|CP|}{2}=\frac{3cv}{2}\quad\text{a}\quad
S_{BCK}=\frac{|CK|\cdot|BP|}{2}=\frac{\frac13v\cdot2c}{2}=
\frac{cv}{3},
$$
preto ich pomer má hodnotu
$$
\frac{S_{BCK}}{S_{ABC}}=\frac{\frac13cv}{\frac32cv}=\frac29.
$$

\odpoved
Trojuholník $BCK$ zaberá $2/9$ obsahu
trojuholníka~$ABC$.


\nobreak\medskip\petit\noindent
Za úplné riešenie úlohy dajte 6 bodov, z~toho 1 bod za odvodenie
$|BP|=2|CD|$, 2 body za objav podobnosti trojuholníkov $CDK$
a~$PBK$ s~koeficientom~2, 1 bod za určenie $|CK|=\frac13|AD|$
a~zvyšné 2 body za zostavenie a~výpočet hľadaného pomeru.
Absenciu argumentu z~poznámky pod čiarou v~žiackych riešeniach
tolerujte.

\endpetit
\bigbreak
}

{%%%%%   C-II-4
Ak má prvý ťah Adam, môže Barbora hrať tak, aby bol výsledný
zlomok rovný jednej, čo podľa zadania prinesie Barbore výhru.
Taký zlomok vyjde, keď budú súčasne platiť obe rovnosti $a=c$
a~$b=d$, ktoré Barbora dosiahne ťahmi "súmerne združenými"
podľa zlomkovej čiary s~Adamovými ťahmi.

Ak začína Barbora, môže Adam hrať tak, aby vyšiel zlomok
s~menovateľom $10c+d$ deliteľným tromi, ktorého čitateľ $10a+b$
však deliteľný tromi nebude. Na to Adamovi stačí
po každom z~oboch Barboriných ťahov vhodne "doplniť" čitateľ či
menovateľ, napríklad podľa kritéria deliteľnosti tromi mu stačí
zabezpečiť, aby sa ciferný súčet $a+b$ čitateľa rovnal~10 a~aby
sa ciferný súčet $c+d$ menovateľa rovnal~9 alebo~12.
Adam tak vyhrá, pretože výsledný zlomok nebude
možné krátiť tromi, takže sa nebude rovnať žiadnemu zlomku
s~mocninou čísla~10 v menovateli, akým sa dá zapísať každé
číslo s~konečným počtom desatinných miest.

\nobreak\medskip\petit\noindent
Za úplné riešenie dajte 6 bodov, z~toho 3 body za popis vyhrávajúcej
Barborinej stratégie v~prípade, keď začína Adam, a~3 body za popis vyhrávajúcej
Adamovej stratégie v~prípade, keď začína Barbora. Obe opísané
vyhrávajúce stratégie nie sú samozrejme jediné možné, napríklad Adam
môže založiť svoju stratégiu na deliteľnosti jedenástimi namiesto
tromi, keď svojimi ťahmi dosiahne rovnosť $c=d$ a~nerovnosť $a\ne b$.

\endpetit
}

{%%%%%   vyberko, den 1, priklad 1
...}

{%%%%%   vyberko, den 1, priklad 2
...}

{%%%%%   vyberko, den 1, priklad 3
...}

{%%%%%   vyberko, den 1, priklad 4
...}

{%%%%%   vyberko, den 2, priklad 1
...}

{%%%%%   vyberko, den 2, priklad 2
...}

{%%%%%   vyberko, den 2, priklad 3
...}

{%%%%%   vyberko, den 2, priklad 4
...}

{%%%%%   vyberko, den 3, priklad 1
...}

{%%%%%   vyberko, den 3, priklad 2
...}

{%%%%%   vyberko, den 3, priklad 3
...}

{%%%%%   vyberko, den 3, priklad 4
...}

{%%%%%   vyberko, den 4, priklad 1
...}

{%%%%%   vyberko, den 4, priklad 2
...}

{%%%%%   vyberko, den 4, priklad 3
...}

{%%%%%   vyberko, den 4, priklad 4
...}

{%%%%%   vyberko, den 5, priklad 1
...}

{%%%%%   vyberko, den 5, priklad 2
...}

{%%%%%   vyberko, den 5, priklad 3
...}

{%%%%%   vyberko, den 5, priklad 4
...}

{%%%%%   trojstretnutie, priklad 1
...}

{%%%%%   trojstretnutie, priklad 2
...}

{%%%%%   trojstretnutie, priklad 3
...}

{%%%%%   trojstretnutie, priklad 4
...}

{%%%%%   trojstretnutie, priklad 5
...}

{%%%%%   trojstretnutie, priklad 6
...}

{%%%%%   IMO, priklad 1
Označme $|\uhol FAB|=|\uhol FBA|=\varphi$. Trojuholníky $AFB$ a~$ADC$ sú podľa vety $uu$ podobné, lebo platí $|\uhol FAB|=|\uhol DAC|=|\uhol FBA|=|\uhol DCA|=\varphi$. Preto pre ich pomery strán platí $$\frac{|BA|}{|AF|}=\frac{|CA|}{|AD|}.$$
No z rovnosti týchto pomerov a $|\uhol FAB|=|\uhol DAC|$ vyplýva, že aj trojuholníky $AFD$ a~$ABC$ sú podobné podľa vety $sus$. Preto $|\uhol AFD|=|\uhol ABC|=90^\circ+\varphi$.
Z rovnoramenného trojuholníka $ADE$ ľahko vypočítame, že $|\uhol AED|=180^\circ-2|\uhol DAE|=180^\circ-2\varphi$. Keď uvažujeme kružnicu so stredom v bode $E$ a polomerom $|EA|=|ED|$, tak obvodový uhol nad tetivou $AD$ bude $|\uhol AED|/2=90^\circ-\varphi$. Keďže $|\uhol AFD|=180^\circ-(90^\circ-\varphi)$ a~bod~$F$ leží v opačnej polrovine určenej priamkou $AD$ ako bod $E$, aj bod $F$ musí ležať na tejto kružnici. Preto $|EA|=|ED|=|EF|$. Trojuholník $AFE$ je rovnoramenný a~máme $|\uhol AFE|=2\varphi$.

Keďže súčet uhlov v trojuholníku $FAB$ je $180^\circ$, platí $|\uhol CFB |=|\uhol FBA |+|\uhol FAB |=2\varphi$. Vidíme, že $|\uhol CFB|= |\uhol AFE |$, a preto body $B,\ F,\ E$ ležia na jednej priamke.

Ďalej keďže $|\uhol DEA |+|\uhol EAM |=180^\circ$, tak $DE\parallel AM$, a preto aj body $X,\ D,\ E$ ležia na jednej priamke. Ľahko dopočítame, že $|\uhol DFC |=180^\circ-|\uhol DFA |=90^\circ-\varphi$, a potom z trojuholníka $CDF$ máme $|\uhol CDF |=180^\circ-|\uhol DFC |-|\uhol FCD |=90^\circ$.

Vidíme, že body $C$, $D$, $F$, $B$ ležia na Tálesovej kružnici nad priemerom $CF$ (\obr). Jej stred je $M$, a preto platí $|MD|=|MB|=|MF|$. Dopočítame aj $|\uhol FMD |=2|\uhol FCD |=2\varphi=|\uhol FAE |$. Z tohto vzťahu vyplýva, že $DEAM$ je rovnoramenný lichobežník, a preto $|MD|=|AE|$.
\insp{mmo65.1}%

Teraz sa pozrime na všetky dĺžkové vzťahy, ktoré sme ukázali. Vidíme, že platí $|MF|=|MD|$ a tiež $|EF|=|ED|$. Z tohto vyplýva, že body $E$ a $D$ sú osovo súmerné podľa priamky $EM$. Taktiež $|BM|=|MD|=|AE|=|XM|$, kde posledný vzťah vyplýva z toho, že $AMXE$ je rovnobežník. A napokon $|EB|=|BF|+|FE|=|AF|+|AE|=|AF|+|MD|=|AF|+|MF|=|AM|=|XE|$. Z posledných dvoch rovností vyplýva, že aj body $X$ a $B$ sú osovo súmerné podľa priamky $ME$.

Potom však aj priamky $FX$ a $DB$ sú osovo súmerné podľa priamky $ME$ a musia sa pretínať na $ME$, čo sme mali dokázať.
}

{%%%%%   IMO, priklad 2
Najprv ukážeme, že $n=9k$ vyhovuje pre každé prirodzené $k$. Pre $n=9$ to vieme urobiť takto:
$$
\vbox{\let\\=\cr\everycr{\noalign{\hrule}}\offinterlineskip
\def\strut{\vrule width 0pt height1em depth.3em\relax}
\halign{\vrule\strut\hbox to 1.3em{\hss$#$\hss}\vrule&&\hbox to 1.3em{\hss$#$\hss}\vrule\cr
I & I & I & M & M & M & O & O & O \\
M & M & M & O & O & O & I & I & I \\
O & O & O & I & I & I & M & M & M \\
I & I & I & M & M & M & O & O & O \\
M & M & M & O & O & O & I & I & I \\
O & O & O & I & I & I & M & M & M \\
I & I & I & M & M & M & O & O & O \\
M & M & M & O & O & O & I & I & I \\
O & O & O & I & I & I & M & M & M \\
}}
$$

Pre ostatné násobky 9 to spravíme tak, že tabuľku $9k\times 9k$ zložíme z $k\times k$ takýchto štvorcov $9\times 9$. Takto vytvorená tabuľka bude spĺňať podmienku o riadkoch a stĺpcoch, lebo každá tabuľka $9\times 9$ ju spĺňa. A tiež bude spĺňať aj podmienku o šikmých radoch, lebo prienik šikmého radu, ktorý ma počet políčok deliteľný troma, s nejakou tabuľkou $9\times 9$ je zasa šikmý rad, ktorého počet políčok je deliteľný troma. A preto v každej jeho časti je rovnako veľa políčok vyplnených $I$, $M$, $O$.

Teraz predpokladajme, že máme vyhovujúcu tabuľku veľkosti $n\times n$. Keďže tretina políčok z riadku je vyplnená písmenom $I$, zrejme $n=3k$ pre vhodné prirodzené $k$. Celá tabuľka sa potom skladá z $k\times k$ štvorcov $3\times 3$. Nazvime stredy týchto štvorcov $3\times 3$ {\it magické\/} políčka. Tiež budeme hovoriť, že {\it magické línie\/} sú riadky alebo stĺpce, ktoré obsahujú aspoň jedno magické políčko (a tým pádom ich obsahujú $k$). Spočítajme teraz dvojice $(l,p)$, kde $l$ je magická línia a $p$ je v nej ležiace políčko, na ktorom je napísané $I$. Toto číslo označíme $A$. Keďže magických línii je $2k$ a v každej je $k$ políčok s písmenom~$I$, máme $A=2k^2$.

Teraz spočítame dvojice $(s,p)$, kde $s$ je šikmý rad s počtom políčok deliteľným~3 a~$p$ je políčko v ňom, na ktorom je napísané $I$. Toto číslo označíme $B$. V šikmých radoch smerom hore doprava je spolu $3k^2$ políčok, tak za tieto šikmé rady napočítame $k^2$ takých dvojíc. Za tie opačným smerom tiež, a preto $B=6k^2$.
Keď však spočítame $A+B$, tak sme spočítali dvojice $(r,p)$, kde $r$ je buď šikmý rad s počtom políčok deliteľným~3, alebo magická línia a $p$ je políčko na nej, na ktorom je napísané $I$. A každé políčko na ktorom je napísané $I$ patrí do práve jednej takej dvojice, okrem magických políčok. Tie patria do štyroch. A keďže v celej tabuľke je tretina, \tj. $3k^2$ písmen $I$, tak $A+B=2k^2+3M$, kde $M$ je počet magických políčok na ktorých je napísané $I$. Z toho dostaneme, že $3M=4k^2$, z čoho vyplýva, že $k$ musí byť deliteľné 3, keďže $M$ je celé číslo. Preto $n$ musí byť deliteľné 9.

Záver je ten, že vyhovujú všetky prirodzené čísla $n$ deliteľné 9.}

{%%%%%   IMO, priklad 3
Na začiatok si uvedomíme, že $2S$ je vždy celé číslo, a preto má zmysel ukazovať, že je deliteľné $n$. Platí dokonca, že dvojnásobok obsahu ľubovoľného mnohouholníka s vrcholmi v mrežových bodoch je celé číslo. Pre trojuholníky to totiž platí, stačí im opísať obdĺžnik (s celočíselným obsahom) a od neho sa už len odčítajú pravouhlé trojuholníky, ktorých obsah je polovica celého čísla. A keďže každý mnohouholník sa dá rozdeliť na trojuholníky, tak to platí pre každý mnohouholník.

Najprv tvrdenie ukážeme pre trojuholník. Keďže druhé mocniny dĺžok jeho strán sú deliteľné $n$, nech sú to $na$, $nb$, $nc$. Potom dĺžky strán tohto trojuholníka sú $\sqrt{na}$, $\sqrt{nb}$, $\sqrt{nc}$, kde samozrejme $a$, $b$, $c$ sú prirodzené čísla. Potom z Herónovho vzorca pre obsah trojuholníka vieme, že
$$
\align
16S^2&=(\sqrt{na}+\sqrt{nb}+\sqrt{nc})\cdot(-\sqrt{na}+\sqrt{nb}+\sqrt{nc})\cdot\\
&\qquad\cdot(\sqrt{na}-\sqrt{nb}+\sqrt{nc})\cdot(\sqrt{na}+\sqrt{nb}-\sqrt{nc}) = \\
&=n^2(a+b+c)(-a+b+c)(a-b+c)(a+b-c).
\endalign
$$
Z toho vieme, že $n^2\mid16S^2$, ale keďže $n$ je nepárne, nutne $n^2\mid 4S^2$, a teda $n\mid 2S$.

Pre $k\ge 4$, tvrdenie dokážeme indukciou. Začneme tým, že zrejme tvrdenie stačí dokázať pre $n$, ktoré sú mocninou prvočísla. Potom to totiž bude platiť pre ľubovoľný súčin mocnín prvočísel, čo znamená, že to bude platiť pre ľubovoľné $n$. Nech teda $n=p^l$, kde $p$ je prvočíslo. Ak by pre niektorú uhlopriečku $k$-uholníka platilo, že druhá mocnina jej dĺžky je deliteľná $n$, tak ňou môžeme $k$-uholník rozdeliť na dva menšie, a tie by mali z indukčného predpokladu dvojnásobok obsahu deliteľný $n$. Preto predpokladajme, že taká uhlopriečka neexistuje.

Budeme hovoriť, že $v_p(a)=m$, ak $p^m\mid a$ a zároveň $p^{m+1}\nmid a$. Inými slovami to hovorí, koľkokrát sa $p$ nachádza v prvočíselnom rozklade $a$. $v_p(a)$ sa nazýva {\it $p$-valuácia~$a$}. Zoberme teraz takú stranu $k$-uholníka, že $v_p(|A_iA_{i+1}|^2)$ je minimálna. Bez ujmy na všeobecnosti nech je to strana $A_1A_2$. Teraz dokážeme jedno pomocné tvrdenie.

\Lema
Ak neexistuje uhlopriečka taká, že jej druhá mocnina je deliteľná $n=p^l$, tak pre všetky $2\le i\le k-1$ platí $v_p(|A_1A_i|^2)>v_p(|A_1A_{i+1}|^2)$.

\dokaz
Pre $i=2$ to platí triviálne, lebo zo zadania vieme, že $v_p(|A_1A_2|^2)\ge l$ a~z~predpokladu o uhlopriečkach máme $v_p(|A_1A_3|^2)<l$. Pre ostatné $i$ to ukážeme indukciou. Nech tvrdenie platí pre $i-1$, dokážeme ho pre $i$. Predpokladajme (sporom), že $v_p(|A_1A_i|^2)
\le v_p(|A_1A_{i+1}|^2)$.

Pozrime sa na štvoruholník $A_1A_{i-1}A_iA_{i+1}$. Zo zadania je tetivový, preto z Ptolemaiovej vety\footnote{Ptolemaiova veta hovorí, že ak $A$, $B$, $C$, $D$ ležia na kružnici v tomto poradí, tak $|AB|\cdot|CD|+{|AD|\cdot|CB|}=|AC|\cdot|BD|$.} platí
$$
|A_1A_{i-1}|\cdot|A_iA_{i+1}|+|A_iA_{i-1}|\cdot|A_1A_{i+1}|=|A_1A_i|\cdot|A_{i-1}A_{i+1}|.
$$
Po umocnení na druhú dostaneme
$$
\align
&|A_1A_{i-1}|^2\cdot|A_iA_{i+1}|^2+|A_iA_{i-1}|^2\cdot|A_1A_{i+1}|^2+\\
&\qquad+2\cdot|A_1A_{i-1}|\cdot|A_iA_{i-1}|\cdot|A_1A_{i+1}|\cdot|A_iA_{i+1}|=|A_1A_i|^2\cdot|A_{i-1}A_{i+1}|^2.
\endalign
$$
Keďže druhé mocniny dĺžok strán a uhlopriečok sú určite celé čísla, aj $2\cdot|A_1A_{i-1}|\cdot|A_iA_{i-1}|\cdot|A_1A_{i+1}|\cdot|A_iA_{i+1}|$ je celé číslo. Odhadnime teraz $p$-valuácie jednotlivých členov. Označme $v=v_p(|A_1A_i|^2)$. Potom
$$
\align
v_p(|A_1A_{i-1}|^2\cdot|A_iA_{i+1}|^2)&=v_p(|A_1A_{i-1}|^2)+v_p(|A_iA_{i+1}|^2)>v+l,\\
v_p(|A_iA_{i-1}|^2\cdot|A_1A_{i+1}|^2)&=v_p(|A_iA_{i-1}|^2)+v_p(|A_1A_{i+1}|^2)\ge l+v.
\endalign
$$
Využili sme indukčný predpoklad, predpoklad $v \le v_p(|A_1A_{i+1}|^2)$ a tiež to, že druhé mocniny dĺžok strán sú deliteľné $n$, a preto je ich $p$-valuácia aspoň $l$. Tak isto dostaneme odhad
$$
v_p(4\cdot|A_1A_{i-1}|^2\cdot|A_iA_{i-1}|^2\cdot|A_1A_{i+1}|^2\cdot|A_iA_{i+1}|^2)>2(v+l).
$$
Keďže $2\cdot|A_1A_{i-1}|\cdot|A_iA_{i-1}|\cdot|A_1A_{i+1}|\cdot|A_iA_{i+1}|$ je celé číslo, platí
$$
2\cdot|A_1A_{i-1}|\cdot|A_iA_{i-1}|\cdot|A_1A_{i+1}|\cdot|A_iA_{i+1}|>v+l.
$$
Vidíme, že $p$-valuácia každého člena na ľavej strane je aspoň $v+l$. Preto je aj $p$-valuácia celej ľavej strany aspoň $v+l$. Ale $A_{i-1}A_{i+1}$ je uhlopriečka a preto $v_p(|A_{i-1}A_{i+1}|^2)<l$. Preto dostávame
$$
v_p(|A_1A_i|^2\cdot|A_{i-1}A_{i+1}|^2)=v_p(|A_1A_i|^2)+v_p(|A_{i-1}A_{i+1}|^2)<v+l.
$$
To je ale spor, pretože $p$-valuácie oboch strán sa samozrejme musia rovnať. Tým je lema dokázaná.

\smallskip
Z lemy vyplýva, že $v_p(|A_1A_2|^2)>v_p(|A_1A_3|^2)>\dots>v_p(|A_1A_k|^2)$. To je ale spor s~predpokladom, že $|A_1A_2|^2$ má najmenšiu $p$-valuáciu zo všetkých druhých mocnín strán. Tak konečne dostávame spor aj s tým, že neexistuje uhlopriečka, ktorej druhá mocnina dĺžky je deliteľná $n$. Preto taká existuje a pre ten prípad sme už tvrdenie dokázali. Tým je celý dôkaz ukončený.}

{%%%%%   IMO, priklad 4
Označme $(a,b)$ najmenší spoločný deliteľ čísel $a$ a $b$.
Dokážeme nasledujúce pomocné tvrdenia:

\odsek{Tvrdenie 1}
Ak $n$ je prirodzené číslo, tak $(P(n),P(n+1))=1$.

\dokaz
Keďže číslo $n^2+n+1$, čiže $n(n+1)+1$, je nepárne, platí
$$
\align
(P(n),P(n+1)) &= (n^2+n+1,n^2+3n+3) =(n^2+n+1,2n+2) =\\
& =(n^2+n+1,2(n+1))
=(n^2+n+1,n+1)=\\
&=(n(n+1)+1,n+1)
=(1,n+1)
=1.
\endalign
$$

\odsek{Tvrdenie 2}
Ak $n\equiv2\pmod7$, tak $(P(n),P(n+2))=7$; ak $n\nequiv2\pmod7$, tak $(P(n),P(n+2))=1$.

\dokaz
Keďže $(2n+7)P(n)-(2n-1)P(n+2)=14$ a číslo $P(n)$ je nepárne, číslo $(P(n),P(n+2))$ je deliteľom $7$. Ak $q\in\{0,1,2,3,4,5,6\}$,
tak platí $$P(7k+q)\equiv(7k+q)^2+(7k+q)+1\equiv q^2+q+1\pmod7.$$ Zistime teda zvyšky po delení $7$ pre samotné zvyšky:
$$
\vbox{\let\\=\cr\offinterlineskip
\def\strut{\vrule width 0pt height1em depth.4em\relax}
\halign{\vrule\strut\ \hss$#$\hss\ \vrule&&\ \hss$#$\hss\ \vrule\cr
\noalign{\hrule}
q & P(q) & P(q)\pmod7 \\
\noalign{\hrule}
0 & 1 & 1 \\
1 & 3 & 3 \\
2 & 7 & 0 \\
3 & 13 & 6 \\
4 & 21 & 0 \\
5 & 31 & 3 \\
6 & 43 & 1 \\
\noalign{\hrule}
}}
$$
Z toho už vyplýva dokazované tvrdenie.

\odsek{Tvrdenie 3}
Ak $n\equiv1\pmod3$, tak $3$ delí $(P(n),P(n+3))$; ak $n\nequiv1\pmod3$, tak $(P(n),P(n+3))=1$.

\dokaz
Keďže $(n+5)P(n)-(n-1)P(n+3)=18$ a číslo $P(n)$ je nepárne, číslo $(P(n),P(n+3))$ je deliteľom $9$. Ak $q\in\{0,1,2\}$, tak platí
$$
P(3k+q)
\equiv(3k+q)^2+(3k+q)+1
\equiv q^2+q+1
\pmod3.
$$
Zistime teda zvyšky po delení $3$ pre samotné zvyšky:
$$
\vbox{\let\\=\cr\offinterlineskip
\def\strut{\vrule width 0pt height1em depth.4em\relax}
\halign{\vrule\strut\ \hss$#$\hss\ \vrule&&\ \hss$#$\hss\ \vrule\cr
\noalign{\hrule}
q & P(q) & P(q)\pmod3 \\
\noalign{\hrule}
0 & 1 & 1 \\
1 & 3 & 3 \\
2 & 7 & 1 \\
\noalign{\hrule}
}}
$$
Z toho už vyplýva dokazované tvrdenie.

\odsek{Tvrdenie 4}
Ak $n\equiv7\pmod{19}$ alebo ak $n\equiv11\pmod{19}$, tak $19$ delí $P(n)$.

\dokaz
Ak $q\in\{0,1,\dots,18\}$, tak platí $$P(19k+q)\equiv(19k+q)^2+(19k+q)+1\equiv q^2+q+1\pmod{19}.$$ A keďže platí
$P(7)=7^2+7+1=57$ a~$P(11)=11^2+11+1=133$ a $19$ delí $57$ aj $133$, tvrdenie je dokázané.

\smallskip
Predpokladajme, že existuje voňavá množina, ktorá má práve 5 prvkov,
nech sú to
$P(a)$,
$P(a+1)$,
$P(a+2)$,
$P(a+3)$,
$P(a+4)$.

Podľa tvrdenia 1 je $P(a+2)$ nesúdeliteľné s $P(a+1)$ aj s $P(a+3)$,
musí mať teda spoločného prvočíselného deliteľa s $P(a)$ alebo s $P(a+4)$,
\tj.
$(P(a),P(a+2))>1$
alebo
$(P(a+2),P(a+4))>1$.
Podľa tvrdenia 2 preto platí
$a\equiv2\pmod7$
alebo
$a+2\equiv2\pmod7$.
V ani jednom prípade však neplatí
$a+1\equiv2\pmod7$,
takže
$(P({a+1}),P({a+3}))=1$.

Číslo $P(a+1)$ teda nie je súdeliteľné s $P(a)$ ani s $P(a+2)$ ani s $P(a+3)$,
musí byť preto podľa zadania súdeliteľné s $P(a+5)$.
Analogicky musí byť $P(a+4)$ súdeliteľné s~$P(a)$.
Podľa tvrdenia 3 to však znamená, že
$a+1\equiv1\pmod3$
aj
$a\equiv1\pmod3$,
čo je spor.
Voňavá množina s 5 prvkami teda neexistuje.

Ani voňavá množina s menej než 5 prvkami neexistuje,
lebo sme ukázali, že $P({a+1})$ nie je súdeliteľné s $P(a)$, s $P(a+2)$ ani s $P(a+3)$.

Ukážeme, že voňavá množina so 6 prvkami existuje.
Podľa čínskej vety o~zvyškoch existuje $a$ také, že
$a\equiv7\pmod{19}$,
$a+1\equiv2\pmod7$
a
$a+2\equiv1\pmod3$,
a~to napríklad~$197$.
Podľa tvrdenia 2 sú čísla $P(a+1)$ a $P(a+3)$ deliteľné $7$,
podľa tvrdenia 3 sú čísla $P(a+2)$ a $P(a+5)$ deliteľné $3$
a podľa tvrdenia 4 sú čísla $P(a)$ a $P({a+4})$ deliteľné~$19$.
To však znamená, že množina
$\{P(a),P({a+1}),P({a+2}),P({a+3}),P({a+4}),P({a+5})\}$
je voňavá.

Najmenšie vyhovujúce $b$ je teda $6$.
}

{%%%%%   IMO, priklad 5
Keďže na oboch stranách rovnosti je $2016$ spoločných lineárnych dvojčlenov,
potrebujeme každý z nich vymazať aspoň z jednej strany.
Spolu ich teda treba vymazať aspoň $2016$.

Ukážeme, že tento počet je postačujúci.
Stačí totiž vymazať z ľavej strany všetky činitele tvaru $x-(4k+2)$ a $x-(4k+3)$
a z pravej všetky tvaru $x-(4k+1)$ a $x-(4k+4)$,
kde $k\in\{0,1,\dots,503\}$.
Treba teda ukázať, že rovnica
$$\prod_{i=0}^{503}(x-(4i+1))(x-(4i+4))
=\prod_{i=0}^{503}(x-(4i+2))(x-(4i+3))
\tag1$$
nemá žiaden reálny koreň.

Nech $x$ je ľubovoľné reálne číslo.
Rozoberme prípady:


\item{$\triangleright$}
Nech
$x\in\{1,2,\dots,2016\}$.
V takom prípade je jedna zo strán nulová, kým druhá nie.
Číslo $x$ teda nie je koreňom rovnice (1).

\item{$\triangleright$}
Nech
$4k+1<x<4k+2$
alebo
$4k+3<x<4k+4$,
kde $k\in\{0,1,\dots,503\}$.
Potom
$(x-(4k+1))(x-(4k+4))<0$,
ale ak $i\in\{0,1,\dots,503\}$ a $i\ne k$,
tak
$(x-(4i+1))(x-(4i+4))>0$,
takže celá ľavá strana rovnice (1) je záporná.
Avšak ak $i\in\{0,1,\dots,503\}$,
tak
$(x-(4i+2))(x-(4i+3))>0$,
takže celá pravá strana rovnice (1) je kladná.
Číslo $x$ teda nie je koreňom (1).

\item{$\triangleright$}
Nech
$x<1$
alebo
$x>2016$
alebo
$4k<x<4k+1$,
kde $k\in\{1,2,\dots,503\}$.
V~takom prípade môžeme (1) prepísať do tvaru
$$1
=\prod_{i=0}^{503}\frac{(x-(4i+1))(x-(4i+4))}{(x-(4i+2))(x-(4i+3))}
=\prod_{i=0}^{503}\left(1-\frac2{(x-(4i+2))(x-(4i+3))}\right).$$
Avšak ak $i\in\{0,1,\dots,503\}$,
tak
$(x-(4i+2))(x-(4i+3))>2$,
a teda
$$0<1-\frac2{(x-(4i+2))(x-(4i+3))}<1.$$
Súčin takýchto čísel je preto tiež menší než $1$.
Číslo $x$ teda nie je koreňom rovnice~(1).

\item{$\triangleright$}
Nech
$4k+2<x<4k+3$,
kde $k\in\{0,1,\dots,503\}$.
V takom prípade môžeme (1) prepísať do tvaru
$$
\align
1&=\frac{x-1}{x-2}\cdot\frac{x-2016}{x-2015}
\cdot\prod_{i=1}^{503}\frac{(x-4i)(x-(4i+1))}{(x-(4i-1))(x-(4i+2))}=\\
&=\frac{x-1}{x-2}\cdot\frac{x-2016}{x-2015}
\cdot\prod_{i=1}^{503}\left(1+\frac2{(x-(4i-1))(x-(4i+2))}\right).
\endalign
$$
Avšak ak $i\in\{1,2,\dots,503\}$,
tak
$(x-(4i-1))(x-(4i+2))>0$,
a teda
$$1+\frac2{(x-(4i-1))(x-(4i+2))}>1.$$
A keďže platí aj
$\frc{(x-1)}{(x-2)}>1$
a
$\frc{(x-2016)}{(x-2015)}>1$,
aj celý súčin je väčší než $1$.
Číslo $x$ teda nie je koreňom (1).

\smallskip\noindent
Rozobrali sme všetky prípady,
rovnica (1) teda nemá reálne riešenie.
To znamená, že minimálny počet lineárnych faktorov, ktoré treba vymazať z pôvodnej rovnice,
je 2016.
}

{%%%%%   IMO, priklad 6
Majme kružnicu $k$ dostatočne veľkú na to,
aby sa do nej zmestili všetky uvažované úsečky.
Každú z týchto úsečiek predĺžme na priamku $p_i$, kde $i\in\{1,2,\dots,n\}$,
tá nech pretína kružnicu $k$ v rôznych bodoch $A_i$ a $B_i$.
Tieto body označme za \uv{naše}.
Keďže každé dve úsečky sa pretínajú vnútri kružnice $k$,
pre každé $i$ z $\{1,2,\dots,n\}$ leží vnútri každého z oblúkov medzi bodmi $A_i$ a $B_i$
ďalších $n-1$ našich bodov.

\smallskip
a)
Každý z $2n$ našich bodov označme slovami \uv{dnu} a \uv{von} tak,
aby sa tieto dve slová po obvode kružnice $k$ pravidelne po jednom striedali.
Číslo $n-1$ je párne,
body $A_i$ a $B_i$ teda majú rôzne označenia --
pri jednom z nich je \uv{dnu} a pri druhom \uv{von}.
Tým je určená orientácia priamky $p_i$.
Bez ujmy na všeobecnosti tiež môžeme predpokladať,
že $A_i$ sú práve tie body, pri ktorých je značka \uv{dnu}.

Bohušovi teraz stačí pre každú úsečku priamky $p_i$ položiť žabu na ten jej krajný bod,
ktorý je bližšie k bodu $A_i$.
Tvrdíme, že žaby na rôznych priamkach $p_i$ a $p_j$ sa nikdy nestretnú na tom istom bode.
Ak by áno, musel by to byť ich priesečník, ktorý označme $P$.
\insp{mmo65.2}%

Uvedomme si že na oblúku $A_iA_j$ kružnice $k$ je nepárny počet našich bodov.
Každý z týchto bodov patrí nejakej priamke $p_h$, kde $h$ je rôzne od $i$ aj od $j$.
Táto priamka podľa zadania nemôže prechádzať bodom $P$,
musí preto pretínať práve jednu z úsečiek $PA_i$ a $PA_j$.
Na týchto úsečkách sa tak vytvorí spolu nepárny počet priesečníkov.
Zvyšné priamky nepretínajú uvažovaný oblúk $A_iA_j$ a neprechádzajú ani bodom $P$,
takže pretínajú buď obe úsečky $PA_i$ a $PA_j$, alebo ani jednu.
Na týchto úsečkách sa tak pridá istý párny počet priesečníkov.
Celkový počet priesečníkov na úsečkách $PA_i$ a $PA_j$ (okrem bodu $P$) je preto nepárny.
Ak sa však majú žaby stretnúť v bode $P$,
musí byť na každej z úsečiek $PA_i$ a $PA_j$ rovnaký počet priesečníkov,
a teda ich celkový počet musí byť párny, čo je spor.

\smallskip
b)
Majme ľubovoľné Bohušovo rozmiestnenie žiab.
Vzniknuté \uv{naše} body označme slovami \uv{dnu} a \uv{von} tak,
aby príslušná orientácia priamky zodpovedala smeru pohybu jej žaby.
Opäť bez ujmy na všeobecnosti môžeme predpokladať,
že $A_i$ sú práve tie body, pri ktorých je značka \uv{dnu}.

Uvedomme si, že tentokrát sa tieto dve slová po obvode kružnice $k$ pravidelne striedať nemôžu,
lebo medzi $A_i$ a $B_i$, ktoré majú rôzne označenia,
leží nepárny počet ($n-1$) našich bodov.
To znamená, že existujú dva naše body $A_i$ a $A_j$,
medzi ktorými už nie je žiaden iný náš bod.
Označme $P$ priesečník priamok $p_i$ a $p_j$.
\insp{mmo65.3}%

Zvyšné priamky pretínajú buď obe úsečky $PA_i$ a $PA_j$, alebo ani jednu.
Na týchto úsečkách tak vznikne rovnaký počet priesečníkov.
To však znamená, že žaby z priamok $p_i$ a $p_j$ sa stretnú v bode $P$.
}

{%%%%%   MEMO, priklad 1
...}

{%%%%%   MEMO, priklad 2
...}

{%%%%%   MEMO, priklad 3
...}

{%%%%%   MEMO, priklad 4
...}

{%%%%%   MEMO, priklad t1
...}

{%%%%%   MEMO, priklad t2
...}

{%%%%%   MEMO, priklad t3
...}

{%%%%%   MEMO, priklad t4
...}

{%%%%%   MEMO, priklad t5
...}

{%%%%%   MEMO, priklad t6
...}

{%%%%%   MEMO, priklad t7
...}

{%%%%%   MEMO, priklad t8
...} 