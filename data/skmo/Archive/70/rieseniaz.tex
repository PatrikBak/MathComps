{%%%%%   Z4-I-1
...}

{%%%%%   Z4-I-2
...}

{%%%%%   Z4-I-3
...}

{%%%%%   Z4-I-4
...}

{%%%%%   Z4-I-5
...}

{%%%%%   Z4-I-6
...}

{%%%%% Z5-I-1
\napad
Koľko vybrali za vstupenky pre dospelých?

\riesenie
Pán Krbec s~kocúrom Kokešom za detské vstupenky vyzbierali celkom ${210\cdot25} =5\,250$ grošov.
Za vstupenky pre dospelých utŕžili $5\,950-5\,250=700$.
Týchto vstupeniek teda predali $700:50=14$ kusov.
}

{%%%%% Z5-I-2
\napad
Ktoré kombinácie hodov sa vo svojich následkoch navzájom zrušia?

\riesenie
Keďže Marekovo družstvo skončilo 1 km východne od štartu, musela im aspoň raz padnúť dvojka.
Ostatné štyri hody museli byť také, že po splnení prislúchajúcich úloh bolo družstvo na pôvodnom mieste
(napr. 3 km na sever a~3 km na juh).
Všetky také štvorice hodených čísel sú -- až na poradie -- nasledujúce:
$$
1\,1\,2\,2,\quad 1\,2\,3\,4,\quad 1\,2\,5\,5,\quad 3\,3\,4\,4,\quad 3\,4\,5\,5,\quad 4\,4\,4\,6,\quad 5\,5\,5\,5.
$$
Spolu s~vyššie uvedenou dvojkou dostávame všetky možné súčty čísel, ktoré mohli tomuto družstvu padnúť:
$$
8,\quad 12,\quad 15,\quad 16,\quad 19,\quad 20,\quad 22.
$$

Možných trás, ktoré mohlo Marekovo družstvo prejsť, je nepreberné množstvo.
Napr. pätica hodov 2\,4\,4\,4\,6 zodpovedá trase 1~km na východ, trikrát 1~km na juh a~3~km na sever.
Zámenou poradia týchto hodov možno získať ďalších 19 možností, celkom 20 rôznych trás.
V~ostatných prípadoch to je obdobné, akurát sa môže stať, že rôzne poradia predstavujú iba rôzne spôsoby prejdenia tej istej trasy (napr. poradia 1\,2\,1\,2\,2, 2\,1\,1\,2\,2, 1\,2\,2\,1\,2).
Ak sa v~pätici hodených čísel vyskytuje 5 (státie na mieste), počty možných trás sa významne znižujú.
}

{%%%%% Z5-I-3
\napad
Zoraďte si psy podľa ich veku.

\riesenie
Pri každom psovi sledujeme štyri znaky, pri každom znaku rozlišujeme štyri možnosti.
Podľa informácií zo zadania môžeme začať priraďovať možnosti napr. podľa veku:
$$\begintable
vek\|2 roky|3 roky|4 roky|5 rokov\cr
rasa\||vlkodav||jazvečík\cr
meno\|Rubby|Bucki|Oddi|\cr
krajina\||||\endtable
$$
Týmto sa nepriamo dozvedáme, že vlkodav sa volal Bucki.
Z~textu navyše vieme, že vlkodav pochádzal z~Írska, identifikácia jedného z~vybraných psov je teda úplná.

Pri druhom čítaní si všímame, že Rubby nebol dalmatín, teda to bol ovčiak (z~predchádzajúceho vieme, že to nebol vlkodav ani jazvečík).
Ďalej ovčiak pochádzal z~Belgicka, teda identifikácia ďalšieho psa je úplná.

Medzi rasami, resp. menami teraz vieme doplniť poslednú chýbajúcu možnosť: Oddi -- dalmatín, resp. jazvečík -- Dunčo.

Posledná informácia zo zadania, ktorú sme zatiaľ nepoužili, hovorí, že Dunčo bol z~Dolnej Lehoty.
Dalmatín Oddi teda pochádzal z~Grécka a~výsledné priradenie vyzerá takto:
$$\begintable
vek\|2 roky|3 roky|4 roky|5 rokov\cr
rasa\|ovčiak|vlkodav|dalmatín|jazvečík\cr
meno\|Rubby|Bucki|Oddi|Dunčo\cr
krajina\|Belgicko|Írsko|Grécko|Dolná Lehota\endtable
$$
}

{%%%%% Z5-I-4
\napad
Koľko šťavy mamičke chýbalo k~naplneniu všetkých fliaš?

\riesenie
Mamičke zvýšilo nenaplnených 12 fliaš, každá s~objemom 500 ml.
Do nich by sa vošlo 6\,000~ml šťavy ($12\cdot500=6\,000$).

Rovnaké množstvo by sa vošlo do 8 veľkých fliaš ($6\,000:750=8$).
Keby mamička nalievala šťavu druhým spôsobom, zvýšilo by jej 8 prázdnych veľkých fliaš.

\poznamka
Jedna veľká fľaša má rovnaký objem ako jeden a~pol malej fľaše, čiže dve veľké fľaše majú rovnaký objem ako tri malé.
Podľa tohto návodu možno fľaše vhodne zamieňať a~vyhnúť sa počítaniu s~veľkými číslami:
12 malých fliaš má rovnaký objem ako 8 veľkých.
}

{%%%%% Z5-I-5
\napad
Určte najskôr obsah trojuholníka $KOZ$.

\riesenie
Obsah trojuholníka $KOZ$ je rovný rozdielu obsahu okolitého obdĺžnika a~troch rohových trojuholníkov, teda
$12-2-2-3=5\,(\Cm^2)$.
Keďže obsah štvoruholníka $KOZA$ má byť 4\,cm$^2$, teda menej ako obsah trojuholníka $KOZ$, musí bod $A$ ležať vnútri tohto trojuholníka.
Také body sú štyri:
\figure{z5-I-5a}%

Obsahy prislúchajúcich štvoruholníkov sú navzájom rôzne, teda nemôže vyhovovať viac ako jeden z~vyznačených bodov.
Podobným výpočtom ako na začiatku zisťujeme, že obsah 4\,cm$^2$ má štvoruholník $KOZA_1$.
\figure{z5-I-5b}%

\poznamka
Obsahy ostatných troch štvoruholníkov sú 1,5\,cm$^2$, 3\,cm$^2$ a~4,5\,cm$^2$.

Spôsob vykrojenia štvoruholníka z~daného trojuholníka je určený označením vrcholov a~súvisiacimi zvyklosťami.
Okrem štvoruholníka $KOZA_1$ existujú ďalšie štvoruholníky s~obsahom 4\,cm$^2$ a~s~vyznačenými bodmi ako vrcholmi:
$KOA_1Z$, $KOA_2Z$ a~$KA_3OZ$.
}

{%%%%%   Z5-I-6
\napad
Viete porovnať tretiu cifru s~ostatnými?

\riesenie
Číslo je tvorené párnymi ciframi, \tj. ciframi 0, 2, 4, 6, 8, nie nutne všetkými.

Vlastnosť s~prehadzovaním cifier znamená, že cifra na treťom mieste je menšia ako ktorákoľvek z~predchádzajúcich cifier a~súčasne väčšia ako ktorákoľvek z~nasledujúcich.

Na prvých dvoch miestach sú párne cifry, ktoré sú dvojnásobkami iných párnych cifier, \tj. niektoré z~0, 4, 8.

Medzi 0 a~0, ani medzi 4 a~2 sa nedá vložiť žiadne párne číslo, ktoré by bolo menšie ako prvé a~súčasne väčšie ako druhé.
Jediné mysliteľné číslo teda je 88644.
}

{%%%%%   Z6-I-1
\napad
Kam doskočila Pečienka?

\riesenie
Priamo zo zadania poznáme dĺžku skoku Rezňa, a~to 2\,730\,mm.
Skoky ostatných králikov z toho dopočítame, pričom všetko jednotne prevádzame na milimetre:
\begin{itemize}
\item Pečienka skočila o~1\,100\,mm menej ako Rezeň, \tj. skočila 1\,630\,mm,
\item Fašírka skočila o~150\,mm menej ako Pečienka, \tj. skočila 1\,480\,mm,
\item Guláš skočil o~200\,mm ďalej ako Fašírka, \tj. skočil 1\,680\,mm.
\end{itemize}

Králiky si v~súťaži vyskákali nasledujúce poradie:

1.~Rezeň, 2.~Guláš, 3.~Pečienka, 4.~Fašírka.
}

{%%%%%   Z6-I-2
\napad
Aké rozmery mohol mať štvorec?

\riesenie
Šachovnica má celkom 32 čiernych (a~32 bielych) políčok, čo musí byť súčet políčok novo vzniknutých čiernych útvarov.
Čierny štvorec teda môže mať rozmery nanajvýš $5\x5$ políčok.
Postupne rozoberieme všetky možnosti podľa veľkosti čierneho štvorca, určíme, koľko políčok ostáva na čierny obdĺžnik a~aké by mohol mať rozmery, a~nakoniec rozhodneme, či je možné také útvary umiestniť podľa uvedených požiadaviek:

$$\begintable
štvorec\|na obdĺžnik ostáva\|dá sa umiestniť\crthick
$1\x1$\|$1\x31$\|NIE\crthick
$2\x2$\|$1\x28$\|NIE\cr
$2\x2$\|$2\x14$\|NIE\cr
$2\x2$\|$4\x7$\|ÁNO\crthick
$3\x3$\|$1\x23$\|NIE\crthick
$4\x4$\|$1\x16$\|NIE\cr
$4\x4$\|$2\x8$\|ÁNO\crthick
$5\x5$\|$1\x7$\|ÁNO\endtable
$$

Všetky prípady, ktoré sa nedajú uspokojivo umiestniť, presahujú niektorým svojim rozmerom daný rámec šachovnice.
Vo vyhovujúcich prípadoch je možné čierne útvary umiestniť napr. takto:
\figure{z6-I-2}%


Pri každom prípade možno umiestnenie aspoň jedného z~čiernych útvarov meniť bez toho, aby došlo k~porušeniu niektorej z~požiadaviek.
Pri takých zmenách obvod bieleho útvaru buď zostáva rovnaký,
alebo sa zmenší (ak sa niektorý z~čiernych útvarov posunie do rohu šachovnice).

Obvod bieleho útvaru vo vyššie uvedených prípadoch pozostáva z~50, 36, resp. 56 strán políčok.
Najväčší je teda v treťom prípade a~je $56\cdot3=168$\,(cm).
}

{%%%%%   Z6-I-3
\napad
Je Ema všetko ovocie z~misy, alebo je prieberčivá?

\riesenie
Ema sa pri maškrtení dôsledne vyhýbala malinám:
ak v~jej výbere bola jedna malina, vracala ju späť do misy, ak vytiahla dve maliny, vymenila ich za jahodu.
Počet malín v~mise sa tak znižoval len po dvoch.

Keďže na začiatku bol v~mise nepárny počet malín, zostával nepárny aj po každej transakcii.
Teda posledným kusom ovocia v~mise bola malina.

\poznamka
Na rozdiel od malín sa počty jahôd v~mise menili po jednej (a~mohli sa ako znižovať, tak zvyšovať).
Parita pôvodného počtu jahôd nemá na riešenie úlohy vplyv.
}

{%%%%% Z6-I-4
\napad
Koľko úsečiek zostrojí Miki a~Niki pri štvorci, päťuholníku, resp. šesťuholníku?

\riesenie
Štvorec má dve uhlopriečky, päťuholník päť a~šesťuholník deväť:
\figure{z6-I-4}%


\begin{enumerate}
\item Rovnaký počet strán a~uhlopriečok má jedine päťuholník.
V~pondelok roboti rysovali päťuholníky.
\item Miki zostrojil 18 strán, teda nezostrojoval ani štvorce, ani päťuholníky (18 nie je deliteľné ani 4, ani 5), zostrojil tri šesťuholníky ($18:6=3$).
V troch šesťuholníkoch je 27 uhlopriečok ($3\cdot9=27$).
V~utorok Niki zostrojil 27 úsečiek.
\item Celkový počet úsečiek pri jednom štvorci je 6, pri jednom päťuholníku 10 a~pri jednom šesťuholníku 15.
Miki a~Niki zostrojili dokopy 70 úsečiek, teda nezostrojovali ani štvorce, ani šesťuholníky (70 nie je deliteľné ani 6, ani 15), zostrojili 7~päťuholníkov ($70:10=7$).
V stredu dostali za úlohu narysovať 7 päťuholníkov.
\end{enumerate}
}

{%%%%%   Z6-I-5
\napad
Kam má Petra vpisovať najväčšie čísla?

\riesenie
Čísla vo vrcholoch trojuholníka prispievajú do súčtov na dvoch stranách, zatiaľ čo ostatné čísla len na jednej.
Väčšie čísla vo vrcholoch trojuholníka preto budú dávať väčšie súčty.

Umiestnime trojicu najväčších čísel do vrcholov trojuholníka a~skúsime doplniť zvyšné čísla.
Úvodné umiestnenie možno spraviť celkom šiestimi spôsobmi, avšak podstatné je iba číslo v~hlavnom (najvyššom) vrchole -- prípadné doplnenia pre prehodené čísla vo zvyšných vrcholoch sú po dvojiciach osovo súmerné.
Postupne rozoberieme všetky možnosti.

\begin{itemize}
\item V~hlavnom vrchole 6:
\figure{z6-I-5a}%


Súčet vpísaných čísel na základni trojuholníka je 15, na ľavom ramene 13, na pravom ramene 14.
Čísla 1, 2, 3, 4, 5 potrebujeme umiestniť tak, aby kompenzovali rozdiely medzi existujúcimi súčtami na jednotlivých stranách.
To je možné spraviť napr. takto:
\figure{z6-I-5b}%


Súčet čísel na každej strane je 19.

\item V~hlavnom vrchole 7:
Podobnou úvahou ako v~predchádzajúcom prípade dostávame nasledujúce riešenie:
\figure{z6-I-5c}%


Súčet čísel na každej strane je 19.

\item V~hlavnom vrchole 8:
\figure{z6-I-5d}%


Súčet vpísaných čísel na základni trojuholníka je 13, na ľavom ramene 14, na pravom ramene 15.
Chýbajúce číslo na základni teda má byť o~1 väčšie ako súčet chýbajúcich čísel na ľavom ramene a~o~2 väčšie ako súčet čísel na pravom ramene.
Druhú požiadavku možno splniť nasledujúcim doplnením, ktoré je jednoznačné až na poradie novej dvojice čísel na pravom ramene:
\figure{z6-I-5e}%


Na ľavé rameno tak ostáva dvojica 3 a~4, ktorá však nevyhovuje prvej požiadavke.
V~hlavnom vrchole teda 8 byť nemôže.
\end{itemize}

Najväčší možný súčet, ktorý možno požadovaným spôsobom dostať, je 19.
Dva príklady možného vyplnenia sú vyššie.

\poznamka
Súčet daných čísel je $1+2+3+4+5+6+7+8=36$.
Pri umiestnení čísel 6, 7, 8 do vrcholov trojuholníka vychádza súčet súčtov čísel na jeho stranách $36+6+7+8=57$.
Súčet čísel na každej strane by tak mal byť $57:3=19$.
Tento postreh nám umožňuje rýchlo doplniť chýbajúce číslo na základni trojuholníka:
\begin{itemize}
\item pre 6 v~hlavnom vrchole vychádza $19-7-8=4$,
\item pre 7 v~hlavnom vrchole vychádza $19-6-8=5$,
\item pre 8 v~hlavnom vrchole vychádza $19-6-7=6$.
\end{itemize}
V~prvých dvoch prípadoch sa ľahko doplnia zvyšné čísla do prázdnych miest.
V~treťom prípade doplnenie nie je možné, lebo 6 by bola použitá dvakrát.
}

{%%%%%   Z6-I-6
\napad

Koľkokrát je Aničkino číslo väčšie ako Marienkino?

\riesenie
Aničkino číslo je desaťkrát väčšie ako Marienkino -- jedine v~tomto prípade je súčin Aničkinho čísla so sebou samým rovný stonásobku súčinu Marienkinho čísla so sebou samým ($10\cdot10=100$).
Polovica Aničkinho čísla je teda päťkrát väčšia ako číslo Marienkino.

Súčet Aničkinho a~Marienkinho čísla je o~18 väčší ako polovica Aničkinho čísla.
Preto je 18 súčtom polovice Aničkinho čísla s~číslom Marienkiným.
Podľa záveru predchádzajúceho odseku je tento súčet rovný šesťnásobku Marienkinho čísla.

Teda Marienkino číslo je 3 ($18:6=3$) a~Aničkino číslo je 30 ($10\cdot3=30$).

\poznamka
Ak Aničkino číslo označíme $a$ a~Marienkino číslo $m$, tak možno predchádzajúce úvahy zhrnúť nasledovne:
\begin{itemize}
\item $a\cdot a=100\cdot m\cdot m$, teda $a=10m$, teda $\frac12a=5m$.
\item $a+m=\frac12a+18$, teda $18=\frac12 a+m=6m$.
\item $m=18:6=3$ a~$a=10m=30$.
\end{itemize}

Po úvodnom postrehu ($a=10m$) je tiež možné postupne dosadzovať prirodzené čísla za $m$, vyjadrovať príslušný rozdiel ($a+m-\frac12a=\frac12a+m$) a~kontrolovať, či je alebo nie je rovný 18:
$$\begintable
$m$\|1|2|3|4|\dots\cr
$a$\|10|20|30|40|\dots\crthick
$\frac12a+m$\|6|12|18|24|\dots\endtable
$$
}

{%%%%%   Z7-I-1
\napad
Ako vyzerá desatinný rozvoj uvedeného čísla?

\riesenie
Desatinný rozvoj racionálneho čísla $\frac9{28}$ je
$$
0{,}32\overline{142857},
$$
pričom opakujúca sa časť, pozostávajúca zo šiestich cifier, je označená pruhom.

Šesť sa do tisíca vojde 166-krát a~zvýšia štyri ($1000=166\cdot6+4$).
Medzi desatinnou čiarkou a~opakujúcou sa časťou sú dve cifry.
Teda cifra na 1000.~mieste za desatinnou čiarkou je rovnaká ako druhá cifra z~opakujúcej sa časti, čo je cifra 4.

\poznamka
V~riešení je podstatný iba zvyšok po delení 1000 číslom 6, a~ten možno určiť bez úplného delenia nasledovne:
Najväčšie číslo deliteľné 6 (tzn. deliteľné 2 a~3), ktoré je menšie ako 1000, je 996.
Zvyšok po delení $1000:6$ je $1000-996=4$.
}

{%%%%%   Z7-I-2
\napad
Na koľko zlatých si bača cenil Kubovu mesačnú prácu?

\riesenie
Kubovi ostávalo do konca roku 5 mesiacov a~dostal zaplatené o~15 zlatých menej, ako by dostal po celom roku služby.
To znamená, že ak by bol vyplatený iba v~hotovosti, dostával by za každý mesiac 3 zlaté ($15:5=3$).

Za 7 mesiacov mal takto Kubo dostať 21 zlatých ($7\cdot3=21$).
Dostal ale 5 zlatých a~jednu ovcu, ovca má teda cenu 16 zlatých ($21-5=16$).

\napadd
Ako by mohla vyzerať spravodlivá odmena za sedemročnú službu?

\ineriesenie
Máme dva spôsoby vyjadrenia jednej odmeny:
\begin{itemize}
\item 20 zlatých a~1 ovca za 12 mesiacov,
\item 5 zlatých a~1 ovca za 7 mesiacov.
\end{itemize}
Ak tieto vyjadrenia prevedieme vzhľadom na rovnaký časový úsek, budeme vedieť určiť cenu ovce.
Odmenu za sedem rokov (\tj. $7\cdot12$ mesiacov) služby môžeme vyjadriť dvojakým spôsobom:
\begin{itemize}
\item 140 zlatých a~7 oviec (\tj. sedemkrát prvá možnosť),
\item 60 zlatých a~12 oviec (\tj. dvanásťkrát druhá možnosť).
\end{itemize}
V~prvom prípade je odmena o~80 zlatých väčšia a~o~5 oviec menšia ako v druhom.
Teda 5 oviec stojí 80 zlatých a~jedna ovca stojí 16 zlatých ($80:5=16$).

\poznamka
Ak zlaté označíme $z$ a~cenu ovce $o$, tak informácie zo zadania môžeme zapísať takto:
$$
5z+o=\tfrac7{12}(20z+o).
$$
Z toho možno rôznymi spôsobmi vyjadriť $o$ pomocou $z$.
Úvahy z~predchádzajúceho riešenia možno zhrnúť nasledovne:
$$\align
12(5z+o)&=7(20z+o), \\
60z+12o&=140z+7o, \\
5o&=80z, \\
o&=16z.
\endalign
$$

Zo zadania nie je vopred zrejmé, že cena ovce v zlatých je celočíselná.
S~týmto dodatočným predpokladom je možné výsledok nájsť postupným skúšaním možností.
}

{%%%%% Z7-I-3
\napad
Ak neviete, ako začať, uvažujte konkrétnu skupinu detí a~overte, či platia uvedené vlastnosti.

\riesenie
Keďže v~každej trojici je nejaký chlapec, sú v~celej skupine nanajvýš dve dievčatá.
Keďže v~každej štvorici je nejaké dievča, sú v~celej skupine nanajvýš traja chlapci.
Teda skupina pozostávajúca z dvoch dievčat a~troch chlapcov je najväčšia možná vyhovujúca skupina.

Keďže v~každej trojici je nejaký Adam, musia sa tak volať všetci traja chlapci.
Keďže v~každej trojici je nejaká Beáta, musia sa tak volať obe dievčatá.
}

{%%%%% Z7-I-4
\napad
Ktorá loď je rýchlejšia? A~koľkokrát?

\riesenie
Lode štartujú zo svojich prístavov súčasne, miesto prvého míňania zodpovedá ich rýchlostiam:
pomer prejdených vzdialeností od prístavov je rovný pomeru rýchlostí prislúchajúcich lodí.

Pomer rýchlostí modrej a~zelenej lode je $4:3$.
V rovnakom pomere sú vzdialenosti z prístavov Mumraj a~Zmätok (v~tomto poradí).
Vzdialenosť od Mumraja je 20~km, teda vzdialenosť od Zmätku je 15~km ($20:15=4:3$).

Trasa medzi prístavmi je dlhá 35~km ($20+15=35$).
}

{%%%%% Z7-I-5
\napad
Ako vyzerá riadok pod vrcholom?

\riesenie
Ak je vo vrchole 0, musia byť v~riadku pod ním dve rovnaké čísla.
Každá výborná pyramída musí mať najmenej tri riadky.

Výborná pyramída s~najmenším počtom riadkov a~s~najmenšími možnými číslami vyzerá (až na poradie čísel v treťom riadku) takto:
$$\thinsize=0pt
\thicksize=0pt\begintable
|||0|||\cr
||1||1||\cr
|0||1||2|\endtable
$$
Najväčšie číslo tejto pyramídy je 2, a~to je tiež odpoveď na druhú otázku zo zadania úlohy.

\poznamka
Najvyššie tri poschodia odčítacej pyramídy s~nulou vo vrchole vyzerajú (až na poradie čísel v treťom riadku) takto:
$$\thinsize=0pt
\thicksize=0pt\begintable
|||0|||\cr
||$a$||$a$||\cr
|$b-a$||$b$||$b+a$|\endtable
$$
}

{%%%%% Z7-I-6
\napad
Hľadajte zhodné uhly.

\riesenie
Veľkosť uhla $ABC$ je rovná súčtu veľkostí uhlov $ABD$ a~$CBD$, \tj. ${30\st+30\st}=60\st$.
V~trojuholníku $ABE$ majú dva vnútorné uhly veľkosť $60\st$, teda zvyšný uhol má tiež veľkosť $60\st$ a~trojuholník je rovnostranný.

Priamka $BD$ je osou vnútorného uhla rovnostranného trojuholníka $ABE$, teda je aj jeho výškou.
Preto body $A$ a~$E$ sú súmerné podľa priamky $BD$.
Úsečky $AD$ a~$ED$, resp. uhly $EAD$ a~$AED$ sú preto zhodné.
Veľkosť uhla $AED$ je rovná $20\st$.
\figure{z7-I-6a}%


\poznamka
Podľa údajov zo zadania možno zostrojiť trojuholník $ABC$ vrátane bodov $D$ a~$E$.
To je vhodný začiatok pre odhalenie potrebných súvislostí.
Riešenia založené na meraní z~obrázka však nemožno považovať za vyhovujúce.

Pri tejto príležitosti pripomíname, že uhol $20\st$ (rovnako ako veľa iných uhlov) sa nedá zostrojiť iba pomocou pravítka a~kružidla.
}

{%%%%%   Z8-I-1
\napad
Viete dopredu vylúčiť niektoré cifry na jednotlivých miestach?

\riesenie
Vlastnosť s~prehadzovaním cifier znamená, že každá cifra mysleného čísla je väčšia ako nasledujúca, resp. menšia ako predchádzajúca.
Vlastnosť so zväčšením a~zmenšením cifier znamená, že najväčšia cifra je menšia ako 9 a~najmenšia cifra je väčšia ako 0.
Celkom preto platí, že
\begin{enumerate}
\item\!cifra je menšia ako 9 a~väčšia ako 4,
\item\!cifra je menšia ako 8 a~väčšia ako 3,
\item\!cifra je menšia ako 7 a~väčšia ako 2,
\item\!cifra je menšia ako 6 a~väčšia ako 1,
\item\!cifra je menšia ako 5 a~väčšia ako 0.
\end{enumerate}

Vlastnosť s~deliteľnosťou štyrmi znamená, že posledné dvojčíslie zmenšeného čísla je deliteľné štyrmi.
To spolu s~predchádzajúcimi podmienkami znamená, že posledné dvojčíslie mysleného čísla môže byť niektoré z~nasledujúcich:
$$
51,\quad 43,\quad 31.
$$

Vlastnosť s~deliteľnosťou tromi znamená, že súčet cifier zväčšeného čísla je deliteľný tromi, teda súčet cifier mysleného čísla dáva po delení tromi zvyšok jedna.
Všetky možné mysliteľné čísla sú
$$
87643,\quad 76543,\quad 87631,\quad 86431,\quad 76531,\quad 65431.
$$
}

{%%%%%   Z8-I-2
\napad
V~ktorej debne bolo po druhom preložení najviac jabĺk?

\riesenie
Po treťom preložení bol v~každej debne rovnaký počet jabĺk, a~ten si označíme~$x$.
Postupne odzadu doplníme počty jabĺk v~jednotlivých debnách:
$$\begintable
\|1. debna|2. debna|3. debna\crthick
po 3. premiestnení\hfill\|$x$|$x$|$x$\cr
po 2. premiestnení\hfill\|$\frac12x$|$\frac12x$|$2x$\cr
po 1. premiestnení\hfill\|$\frac14x$|$\frac74x$|$x$\cr
pôvodne\hfill\|$\frac{13}8x$|$\frac78x$|$\frac12x$\endtable
$$

Počty jabĺk po každom preložení v~každej debne boli celočíselné.
Teda $x$ musí byť násobkom ôsmich a~$3x$ (súčet jabĺk vo všetkých debnách) musí byť násobkom 24.

Medzi číslami 151 až 189 je jediný násobok čísla 24, a~to 168.
Teda $x=168:3=56$ a~v~debnách pôvodne boli nasledujúce počty jabĺk:
$$\tfrac{13}8\cdot56=91,\quad \tfrac78\cdot56=49,\quad \tfrac12\cdot56=28.
$$

\poznamka
Ak by sme uvažovali odpredu, tak počty jabĺk v~debnách možno postupne vyjadriť takto:
$$\begintable
\|1. debna|2. debna|3. debna\crthick
pôvodne\hfill\|$a$|$b$|$c$\cr
po 1. premiestnení\hfill\|$a-b-c$|$2b$|$2c$\cr
po 2. premiestnení\hfill\|$2a-2b-2c$|$-a+3b-c$|$4c$\cr
po 3. premiestnení\hfill\|$4a-4b-4c$|$-2a+6b-2c$|$-a-b+7c$\endtable
$$
Rovnosť počtov jabĺk po treťom preložení vedie k~sústave rovníc, ktorá je pre riešiteľa v~tejto kategórii problematická.
Avšak s~predpokladmi celočíselnosti $a,b,c$ a~obmedzenosti súčtu $150<a+b+c<190$ má táto sústava jediné riešenie uvedené vyššie.
}

{%%%%%   Z8-I-3
\napad
Viete určiť obsah trojuholníka pomocou jeho obvodu a~polomeru kružnice vpísanej?

\riesenie
Trojuholník $ABC$ možno rozložiť na trojuholníky $ABS$, $BCS$ a~$ACS$.
Výška každého z~týchto trojuholníkov z~vrcholu $S$ je zhodná s~polomerom vpísanej kružnice.
Pomery ich obsahov sú preto rovnaké ako pomery dĺžok strán oproti vrcholu~$S$.
Podobne možno porovnávať tieto čiastočné trojuholníky s~celým trojuholníkom $ABC$
(ktorého obsah je rovný súčinu obvodu a~polomeru vpísanej kružnice).
\figure{z8-I-3}%


Keďže obsah štvoruholníka $ABCS$ je rovný štyrom pätinám obsahu trojuholníka $ABC$, ostáva na trojuholník $ACS$ jedna pätina obsahu trojuholníka $ABC$.
Teda dĺžka strany $AC$ je rovná pätine obvodu trojuholníka $ABC$, čo v~našom prípade je ${15:5}=3\,(\Cm)$.
Súčet dĺžok zvyšných dvoch strán je preto 12\,cm; do úvahy pripadajú nasledujúce dvojice dĺžok strán (uvedené v~cm, bez ohľadu na poradie):
$$
1,11;\quad 2, 10;\quad 3,9;\quad 4,8;\quad 5,7;\quad 6,6.
$$

Aby uvažované úsečky tvorili strany trojuholníka, musia byť splnené trojuholníkové nerovnosti.
Týmto požiadavkám vyhovujú iba nasledujúce trojice -- možné dĺžky strán trojuholníka $ABC$ v~cm:
$$
3,5,7;\quad 3,6,6.
$$
}

{%%%%%   Z8-I-4
\napad
Pátrajte najskôr po Jarkinej dennej mzde.

\riesenie
Za trojdňovú výplatu si Jarka kúpila jednu hru a~zvýšilo jej 49~€, teda za šesťdennú výplatu by si mohla kúpiť dve hry a~zvýšilo by jej 98~€.
Pritom za päť dní by zarobila tiež na dve hry, ale zvýšilo by jej iba 54~€.
Jarkina denná mzda preto bola 44~€ ($98-54=44$).

Z~prvého údaja dopočítame cenu jednej hry: $3\cdot44-49=83$~€.

\poznamka
Ak dennú výplatu označíme $v$ a~cenu hry $h$, tak možno predchádzajúce úvahy zhrnúť nasledovne:
\begin{itemize}
\item $3v=h+49$, teda $6v=2h+98$,
\item $5v=2h+54$, teda $v=98-54=44$,
\item $h=3v-49=3\cdot44-49=83$.
\end{itemize}
}

{%%%%%   Z8-I-5
\napad
Podľa akých pravidiel boli rozmiestnené prstene s~rôznymi kombináciami troch uvedených vlastností?

\riesenie
Každý 3.~prsteň bol zlatý, každý 4.~bol starožitný a~každý 10.~mal diamant.
Teda
\begin{itemize}
\item zlatých prsteňov bolo $120:3=40$,
\item starožitných prsteňov bolo $120:4=30$,
\item prsteňov s~diamantom bolo $120:10=12$.
\end{itemize}

Pri počítaní prsteňov s~viacerými vlastnosťami najskôr určíme, s~akou pravidelnosťou sa na výstave opakovali:
každý 12.~prsteň bol zlatý a~starožitný, každý 30.~bol zlatý s~diamantom a~každý 20.~bol starožitný s~diamantom (tu napr. 20 je najmenším spoločným násobkom čísel 4 a~10).
Teda
\begin{itemize}
\item zlatých starožitných prsteňov bolo $120:12=10$,
\item zlatých prsteňov s~diamantom bolo $120:30=4$,
\item starožitných prsteňov s~diamantom bolo $120:20=6$.
\end{itemize}
Ďalej každý 60.~prsteň mal všetky tri vlastnosti (60 je najmenším spoločným násobkom čísel 3, 4 a~10), teda
\begin{itemize}
\item zlaté starožitné prstene s~diamantom boli $120:60=2$.
\end{itemize}

Pri počítaní prsteňov s~niektorou z~uvedených vlastností musíme byť obozretní:
10~prsteňov bolo zlatých a~starožitných, 2~z~nich mali navyše diamant, teda zlatých starožitných prsteňov bez diamantu bolo $10-2=8$.
Podobne zlatých nestarožitných prsteňov s~diamantom bolo $4-2=2$ a~nezlatých starožitných prsteňov s~diamantom bolo $6-2=4$.

Zlatých prsteňov s~nejakou dodatočnou vlastnosťou bolo $2+8+2=12$, pritom zlatých prsteňov celkom bolo 40, teda zlatých nestarožitných prsteňov bez diamantu bolo $40-12=28$.
Podobne nezlatých starožitných prsteňov bez diamantu bolo $30-{(2+8+4)}=16$ a~nezlatých nestarožitných prsteňov s~diamantom bolo ${12-{(2+2+4)}}=4$.

Predchádzajúce počty a~vzťahy môžeme znázorniť pomocou Vennovho diagramu takto:
\figure{z8-I-5}%


Prsteňov s~niektorou z troch sledovaných vlastností (teda prsteňov, ktoré neboli falzifikáty) bolo $2+8+4+2+28+16+4=64$.
Falzifikátov preto bolo $120-64=56$.

\poznamka
Ak základné tri množiny prsteňov označíme $Z$, $S$ a~$D$, tak úvodnú časť predchádzajúceho riešenia možno zhrnúť nasledovne:
$$
\def|{\vert}
\gather
|Z|=40,\quad |S|=30,\quad |D|=12, \\
|Z\cap S|=10,\quad |Z\cap D|=4,\quad |S\cap D|=6, \\
|Z\cap S\cap D|=2.
\endgather
$$
V~ďalšej časti sme zisťovali počet prvkov zjednotenia $Z\cup S\cup D$ tak, že sme postupne vyjadrovali počty prvkov navzájom disjunktných%
\footnote[$^*$]{\!Disjunktné množiny sú množiny s~prázdnym prienikom, teda množiny bez spoločného prvku.}
podmnožín $(Z\cap S)\setminus(Z\cap S\cap D)$, $(Z\cap D)\setminus(Z\cap S\cap D)$ atď., ktoré sme potom sčítali.
Stručnejšie možno k~tomu istému výsledku dospieť nasledujúcim výpočtom:
$$\aligned
\def|{\vert}
|Z\cup S\cup D|
&=|Z|+|S|+|D|-|Z\cap S|-|Z\cap D|-|S\cap D|+|Z\cap S\cap D|= \\
&=40+30+12-10-4-6+2 =64.
\endaligned
$$
Tomuto vzťahu sa hovorí {\it princíp inklúzie a~exklúzie}.
Na jeho všeobecné zdôvodnenie (príp. ďalšie zovšeobecnenie) stačí overiť, že každú z~disjunktných častí Vennovho diagramu započítavame práve raz.
}

{%%%%%   Z8-I-6
\napad
Aký je súčet vnútorných uhlov v~trojuholníku?

\riesenie
Pomocou vyznačených uhlov možno vyjadriť všemožné ďalšie uhly v~danom mnohouholníku.
Takto začneme a~pokúsime sa zistiť niečo o~hľadanom súčte.
Vyznačené uhly a~zvyšné vrcholy mnohouholníka označíme nasledovne:
\figure{z8-I-6a}%


Uhol $LKO$ je vnútorným uhlom trojuholníka $BKD$, teda sa dá vyjadriť ako $180\st-\beta-\delta$.
Uhol $LKA$ je susedným uhlom k~uhlu $LKO$, resp. vonkajším uhlom trojuholníka $BKD$, teda je rovný $\beta+\delta$.
Podobne uhol $KLA$ je rovný $\gamma+\epsilon$ atď.

Súčet vnútorných uhlov v~trojuholníku $AKL$ je práve hľadaným súčtom vyznačených uhlov:
$$
\alpha+\beta+\gamma+\delta+\epsilon=180\st.
$$

\poznamky
V~predchádzajúcom riešení sme sa sústredili na vyjadrenie uhlov v~cípe hviezdy s~vrcholom $A$.
Taký istý výsledok dostávame v~ktoromkoľvek inom cípe.

Napriek tomu, že vyznačené uhly môžu byť veľmi rôznorodé, ich súčet je vždy rovnaký.
To je dôsledkom podobne nesamozrejmého tvrdenia o~súčte uhlov v~trojuholníku.
Toto tvrdenie bude určite v~pozadí akéhokoľvek iného riešenia úlohy.
Napr. je možné využiť súčet vnútorných uhlov všeobecného mnohouholníka:
$n$-uholník možno rozdeliť (rôznymi spôsobmi) na $n-2$ trojuholníkov, teda súčet jeho vnútorných uhlov je ${(n-2)}\cdot180\st$.

\ineriesenie
Súčet vnútorných uhlov päťuholníka $KLMNO$ je rovný $3\cdot180\st=540\st$.
Podobne ako v~riešení uvedenom vyššie možno vnútorné uhly pri vrcholoch $K$, $L$, $M$, $N$, $O$ vyjadriť postupne ako
$$
180\st-\beta-\delta,\quad
180\st-\gamma-\epsilon,\quad
180\st-\delta-\alpha,\quad
180\st-\epsilon-\beta,\quad
180\st-\alpha-\gamma.
$$
Celkom tak dostávame
$$\aligned
5\cdot180\st -2(\alpha+\beta+\gamma+\delta+\epsilon) &=540\st, \\
2(\alpha+\beta+\gamma+\delta+\epsilon) &=360\st, \\
\alpha+\beta+\gamma+\delta+\epsilon &=180\st.
\endaligned
$$
}

{%%%%%   Z9-I-1
\napad
Ktoré z~uvedených farieb nemôžu mať čísla 97 a~101?

\riesenie
Čísla si označíme počiatočnými písmenami podľa ich farby, teda $\check{c}$, $m$, $z$, $\check{z}$.
Informácie o~delení potom môžeme zapísať takto:
$$
\check{c}=m\cdot z+\check{z},\quad
m=z\cdot\check{z}.
$$

Keďže čísla majú byť navzájom rôzne, z~druhej rovnosti vyplýva, že $z$ ani $\check{z}$ nie je~1.
To znamená, že $m$ je číslo zložené.
Dosadením druhej rovnosti do prvej dostávame
$$
\check{c}=z^2\cdot\check{z}+\check{z} =(z^2+1)\cdot\check{z}.
$$
Z~predchádzajúceho vieme, že ako $z^2+1$, tak $\check{z}$ nie sú 1, teda aj $\check{c}$ je číslo zložené.
Vzhľadom na to, že obe čísla 97 a~101 sú prvočísla, nemôže byť žiadne z~nich ani modré, ani červené.

Jedno z~čísel 97 a~101 je preto žlté a~jedno zelené.
Zvyšné čísla dopočítame z~úvodných vzťahov:
$$\begintable
$\check{z}$|$z$|$m$|$\check{c}$\crthick
97|101|9\,797|989\,594\cr
101|97|9\,797|950\,410\endtable
$$

\poznamka
Úlohu možno riešiť aj postupným skúšaním možností:
dvojice známych čísel sa dosadia do úvodných rovností a~overí sa existencia zvyšnej dvojice čísel.
Napr. dosadenie $\check{c}=101$ a~$m=97$ určuje z~prvej rovnosti $z=1$ a~$\check{z}=4$, čo však nevyhovuje rovnosti druhej.
Týmto spôsobom by sa muselo prebrať 12 možností.

Akýkoľvek dodatočný postreh môže znížiť počet možností na overovanie.
Napr. (popri podmienkach uvedených v~predchádzajúcom riešení) platí, že $\check{c}$ je väčšie ako ktorékoľvek zo zvyšných čísel a~$m$ je väčšie ako $z$ a~$\check{z}$.
Preto nemôže byť $\check{c}=97$ a~touto možnosťou sa nie je nutné zaoberať.
}

{%%%%%   Z9-I-2
\napad
Existuje nejaká súvislosť medzi desatinnými rozvojmi čísel $\frac{x}{y}+1$ a~$\frac1y$?

\riesenie
Aby $\frac{x}{y}+1$ bolo periodické číslo s~jednocifernou periódou, musí byť aj $\frac{x}{y}$ periodické číslo s~jednocifernou periódou.
Keďže $\frac{x}{y}=x\cdot\frac1y$, musí tá istá podmienka platiť aj pre číslo $\frac1y$.
Medzi prirodzenými číslami 1 až 9 je táto podmienka splnená iba v troch prípadoch, ktoré postupne rozoberieme:
\begin{itemize}
\item Pre $y=3$ je $\frac1y=0{,}\overline{3}$.
Teda diskutovaná rovnosť je tvaru
$$
\frac{x}3+1=x+\frac13,
$$
čo po úprave dáva $x=1$.
\item Pre $y=6$ je $\frac1y=0{,}1\overline{6}$.
Teda $0{,}\overline{6}=10\cdot\frac16-1=\frac23$ a~diskutovaná rovnosť je tvaru
$$
\frac{x}6+1=x+\frac23,
$$
čo po úprave dáva $x=\frac25$.
To ale nie je celé číslo.
\item Pre $y=9$ je $\frac1y=0{,}\overline{1}$.
Teda $0{,}\overline{9}=9\cdot\frac19=1$ a~diskutovaná rovnosť je tvaru
$$
\frac{x}9+1=x+1,
$$
čo po úprave dáva $x=0$.
\end{itemize}
Úloha má dve riešenia: $x=1$, $y=3$ a~$x=0$, $y=9$.

\poznamky
Keďže $\frac19=0{,}\overline{1}$, je $\frac{y}9=0{,}\overline{y}$ a~diskutovanú rovnosť možno vyjadriť ako
$$
\frac{x}{y}+1=x+\frac{y}{9}.
$$
Pre každé jednociferné prirodzené číslo $y$ stačí doriešiť príslušnú lineárnu rovnicu a~overiť nezápornosť a~celočíselnosť $x$.
Takto dostaneme dve riešenia uvedené vyššie.

Predchádzajúcu rovnosť je možné ďalej upravovať, napr. takto:
$$\aligned
9x+9y&=9xy+y^2 \\
9(x+y-xy)&=y^2.
\endaligned
$$
Pre ľubovoľné celé čísla $x$ a~$y$ je výraz na pravej strane deliteľný 9.
Teda aj číslo $y^2$ musí byť deliteľné 9, takže číslo $y$ musí byť deliteľné 3.
Medzi číslami 1 až 9 stačí overovať iba $y=3$, 6 a~9.
Takto sme dospeli k~tomu istému obmedzeniu možností ako v~postupe uvedenom vyššie.
}

{%%%%%   Z9-I-3
\napad
Čo viete o~ťažisku rovnostranného trojuholníka?

\riesenie
V~rovnostrannom trojuholníku je ťažisko priesečníkom výšok.
Obrazy ťažiska v~osových súmernostiach podľa strán trojuholníka preto ležia na prislúchajúcich (predĺžených) ťažniciach, čiže výškach.
Preto trojice bodov $C,T,R$, resp. $A,T,N$ ležia na priamkach.
Priesečníky týchto priamok so stranami trojuholníka označíme $S$, resp.~$P$.
\figure{z9-I-3}%


Ťažisko delí ťažnicu v~pomere $2:1$, teda
$$
|CT|=2|TS|,\quad \text{resp.}\quad |AT|=2|TP|.
$$
V~osovej súmernosti sú vzdialenosti vzoru a~obrazu od osi rovnaké, teda
$$|TR|=2|TS|,\quad \text{resp.}\quad |TN|=2|TP|.
$$
Celkom z toho odvodzujeme, že dvojice úsečiek $CT$ a~$TR$, resp. $AT$ a~$TN$ sú zhodné.
Navyše (vrcholové) uhly $ATC$ a~$NTR$ sú zhodné, teda trojuholníky $TCA$ a~$TRN$ sú zhodné (podľa vety {\it sus\/}), preto majú rovnaký obsah.

Pomer obsahov trojuholníkov $ABC$ a~$TRN$ je rovnaký ako pomer obsahov trojuholníkov $ABC$ a~$ACT$.
Trojuholníky $ABC$ a~$ACT$ majú spoločnú stranu $AC$ a~zodpovedajúce výšky v~pomere $3:1$;
v~tom istom pomere sú aj ich obsahy.
Pomer obsahov trojuholníkov $ABC$ a~$TRN$ je $3:1$.

\poznamka
Predchádzajúce riešenie bolo založené na znalosti polohy ťažiska na ťažnici.
Aj bez tohto poznatku možno úlohu doriešiť s~použitím ďalších vlastností vyplývajúcich zo zadania, napr.:
\begin{itemize}
\item Trojuholník $ABC$ je tvorený tromi navzájom zhodnými trojuholníkmi $ABT$, $BCT$ a~$ACT$, príp. šiestimi trojuholníkmi, z~ktorých každý je zhodný s~trojuholníkom $TSB$.
\item Trojuholníky $TRB$ a~$TNB$ sú rovnostranné, navzájom zhodné.
\item Štvoruholník $TRBN$ je kosoštvorec, ktorý je uhlopriečkami rozdelený na štyri trojuholníky, z~ktorých každý je zhodný s~trojuholníkom $TSB$.
\end{itemize}
Odtiaľ pomer obsahov trojuholníkov $ABC$ a~$TRN$ je $6:2=3:1$.
}

{%%%%%   Z9-I-4
\napad
Ktoré z~novovzniknutých čísel bolo väčšie?

\riesenie
Obe nové čísla mali rovnaký počet cifier a~väčšie bolo trikrát väčšie ako to druhé.
Väčšie číslo teda nemohlo začínať jednotkou -- bolo to číslo Matovo.

Pôvodne napísané päťciferné číslo označíme $x$.
Patova úprava dáva číslo $100\,000+x$, Matova úprava dáva číslo $10x+1$
a~platí
$$\aligned
10x+1&=3(100\,000+x), \\
7x&=299\,999, \\
x&=42\,857.
\endaligned
$$

Na stene bolo pôvodne napísané dvakrát číslo 42\,857.

\poznamky
Ak by sme predpokladali, že Patovo nové číslo bolo väčšie ako Matovo, tak by sme dostali
$$\aligned
100\,000+x&=3(10x+1), \\
99\,997&=29 x.
\endaligned
$$
To však nevedie k~celočíselnému riešeniu (zvyšok po delení $99\,997:29$ je 5).

Úlohu je možné riešiť ako algebrogram.
Obe uvedené možnosti zodpovedajú postupne nasledujúcim zadaniam:
$$
\alggg{1&a&b&c&d&e\\\x&&&&&3}{a&b&c&d&e&1}
\hskip1cm
\alggg{a&b&c&d&e&1\\\x&&&&&3}{1&a&b&c&d&e}
$$
V~prvom prípade postupne nepriamo dopĺňame $e=7$, $d=5$, $c=8$, $b=2$, $a=4$, čo zodpovedá riešeniu uvedenému vyššie.
V druhom prípade postupne priamo dopĺňame $e=3$, $d=9$, $c=7$, $b=3$, $a=1$, čo však vedie ku sporu: prvá cifra vo výsledku vychádza 4 a~nie predpísaná 1.
}

{%%%%%   Z9-I-5
\napad
Môžu niektoré kolky byť súčasne vo všetkých troch kruhoch? Ak áno, koľko najviac?

\riesenie
Kolky v~kruhoch predstavujú prvky v~množinách: hľadáme tri množiny po 9 prvkoch, ktorých zjednotenie má 16 prvkov.
Vzťahy medzi množinami znázorníme nasledovne
(písmená $a$ až $g$ označujú počty prvkov v~prislúchajúcich podmnožinách):
\figure{z9-I-5a}%


Stačí sa sústrediť iba na $b$, $d$, $e$, $f$, lebo zvyšné tri neznáme sú týmito štyrmi úplne určené:
$$
a=9-b-e-d,\quad c=9-b-e-f,\quad g=9-d-e-f. \tag{1}
$$
Pomocou týchto vzťahov tiež dostávame obmedzenie
$$
a+b+c+d+e+f+g =27-b-d-f-2e=16,
$$
teda
$$
b+d+f+2e=11. \tag{2}
$$

Diskusiu môžeme viesť vzhľadom na spoločný prienik všetkých troch množín; z~obmedzenia (2) vyplýva, že $e$ nemôže byť väčšie ako 5.
Postupne rozoberieme všetky možné hodnoty $e$ a~pre každú z~nich nájdeme nezáporné celé čísla $b,d,f$, ktoré vyhovujú obmedzeniu (2) a~pre ktoré čísla v~(1) sú tiež nezáporné.
Trojice $b,d,f$ nás zaujímajú až na poradie, takže si ich pre poriadok vhodne usporiadame (zámena poradia vedie k~riešeniu, ktoré nie je podstatne odlišné od pôvodného):

Pre $e=5$ dostávame $b+d+f=1$, teda
$$\begintable
$b$|$d$|$f$\|$a$|$c$|$g$\crthick
1|0|0\|3|3|4\endtable
$$

Pre $e=4$ dostávame $b+d+f=3$, teda
$$\begintable
$b$|$d$|$f$\|$a$|$c$|$g$\crthick
3|0|0\|2|2|5\cr
2|1|0\|2|3|4\cr
1|1|1\|3|3|3\endtable
$$

Pre $e=3$ dostávame $b+d+f=5$, teda
$$\begintable
$b$|$d$|$f$\|$a$|$c$|$g$\crthick
5|0|0\|1|1|6\cr
4|1|0\|1|2|5\cr
3|2|0\|1|3|4\cr
3|1|1\|2|2|4\cr
2|2|1\|2|3|3\endtable
$$

Pre $e=2$ dostávame $b+d+f=7$, teda
$$\begintable
$b$|$d$|$f$\|$a$|$c$|$g$\crthick
7|0|0\|0|0|7\cr
6|1|0\|0|1|6\cr
5|2|0\|0|2|5\cr
5|1|1\|1|1|5\cr
4|3|0\|0|3|4\cr
4|2|1\|1|2|4\cr
3|3|1\|1|3|3\cr
3|2|2\|2|2|3\endtable
$$

Pre $e=1$ dostávame $b+d+f=9$, teda
$$\begintable
$b$|$d$|$f$\|$a$|$c$|$g$\crthick
7|1|1\|0|0|6\cr
6|2|1\|0|1|5\cr
5|3|1\|0|2|4\cr
5|2|2\|1|1|4\cr
4|4|1\|0|3|3\cr
4|3|2\|1|2|3\cr
3|3|3\|2|2|2\endtable
$$

Pre $e=0$ dostávame $b+d+f=11$, teda
$$\begintable
$b$|$d$|$f$\|$a$|$c$|$g$\crthick
7|2|2\|0|0|5\cr
6|3|2\|0|1|4\cr
5|4|2\|0|2|3\cr
5|3|3\|1|1|3\cr
4|4|3\|1|2|2\endtable
$$

Prípadné znázornenie jednotlivých riešení je očividné.
Pre ilustráciu uvádzame rozmiestnenie zodpovedajúce jedinému riešeniu v~prípade $e=5$:
\figure{z9-I-5b}%


\poznamky
Osem podstatne odlišných riešení možno tiež nájsť prostým (trpezlivým) skúšaním.
V~tomto duchu môže byť prehľadnejšie v~danej šesnásťprvkovej množine vyberať tri deväťprvkové podmnožiny tak, aby žiadny prvok nezostal na ocot.
Napr. vyššie znázornené riešenie zodpovedá nasledujúcemu výberu:
\figure{z9-I-5c}%


Všeobecný vzťah medzi počtami prvkov množín, ich prienikmi a~zjednotením popisuje tzv. {\it princíp inklúzie a~exklúzie}, pozri poznámky za riešením úlohy {\bf Z8--I--5}.
}

{%%%%%   Z9-I-6
\napad
V~akom vzťahu boli úsečky $AE$ a~$CE$ vzhľadom na priamku spájajúcu bod $B$ s~vrcholom ihlana?

\riesenie
Pre lepšiu prehľadnosť si situáciu zo zadania znázorníme:
\figure{z9-I-6a}%


Lomená čiara $AEC$ je najkratšia, keď sa po rozvinutí plášťa ihlana do roviny javí ako úsečka:
\figure{z9-I-6b}%


Úsečky $AV$ a~$CV$ sú zhodné a~rovnako tak úsečky $AB$ a~$CB$.
Teda po rozvinutí plášťa do roviny sú body $A$ a~$C$ súmerné podľa priamky $BV$, preto úsečka $AC$ je na túto priamku kolmá.
Dĺžka najkratšej možnej lomenej čiary $AEC$ je rovná dvojnásobku veľkosti výšky trojuholníka $ABV$ z~vrcholu $A$, a~tak ju aj určíme.

Trojuholník $ABV$ je rovnoramenný, veľkosť jeho základne poznáme, ramená sú preponami pravouhlých trojuholníkov, ktorých jedna odvesna je polovicou uhlopriečky podstavového štvorca a~druhá odvesna je výškou ihlana:
\figure{z9-I-6c}%


\noindent
Polovica uhlopriečky podstavového štvorca má (podľa Pytagorovej vety) veľkosť
$$
|AS|=\frac{\sqrt2}2\,|AB| \doteq 162{,}635\,\text{m}.
$$
Výška ihlana má (podľa informácie zo zadania o~obvodoch) veľkosť
$$
|SV|=\frac{4}{2\pi}\,|AB| \doteq 146{,}423\,\text{m}.
$$
Hrany prechádzajúce vrcholom ihlana majú (podľa Pytagorovej vety) veľkosť
$$
|AV|=\sqrt{|AS|^2+|SV|^2} \doteq 218{,}837\,\text{m}.
$$

Teraz poznáme všetky strany trojuholníka $ABV$.
Jeho výšku z~vrcholu $A$ môžeme vyjadriť pomocou výšky z~hlavného vrcholu $V$ a~dvojakého vyjadrenia obsahu tohto trojuholníka:
\figure{z9-I-6d}%


Trojuholník $ABV$ je súmerný podľa výšky idúcej hlavným vrcholom.
Táto výška má (podľa Pytagorovej vety) veľkosť
$$
|VP|=\sqrt{|AV|^2-\frac14|AB|^2} \doteq 186{,}184\,\text{m}.
$$
Obsah trojuholníka $ABV$ je rovný $\frac12|AB|\cdot|VP|=\frac12|BV|\cdot|AE|=\frac12|AV|\cdot|AE|$, teda výška z~vrcholu $A$ má veľkosť
$$
|AE|=\frac{|AB|\cdot|VP|}{|AV|} \doteq 195{,}681\,\text{m}.
$$

Dĺžka najkratšej možnej lomenej čiary $AEC$ je
$$
|AEC|=2|AE| \doteq 391{,}362\,\text{m},
$$
\tj. približne 391\,m a~36\,cm.


\poznamky
Úvodný postreh s~rozvinutým plášťom ihlana možno nahradiť nasledujúcou úvahou:
Trojuholníky $ABV$ a~$BCV$ sú zhodné, teda aj úsečky $AE$ a~$EC$ sú zhodné a~dĺžka lomenej čiary $AEC$ je rovná dvojnásobku dĺžky úsečky $AE$.
Tá je najkratšia možná, keď je na priamku $BE$ kolmá.
Stačí teda určiť výšku trojuholníka $ABV$ z~vrcholu~$A$.

V~uvedenom riešení úlohy môže vďaka priebežnému zaokrúhľovaniu dochádzať k~nežiaducemu hromadeniu chyby.
Preto sme radšej zaokrúhľovali na celé mm.
Všetky predchádzajúce veličiny je tiež možné vyjadriť všeobecne pomocou $|AB|$ a~dosadzovať až do konečného výrazu.
Takto postupne po úpravách dostávame:
$$
|AV|=\sqrt{\frac{\pi^2+8}{2\pi^2}}\cdot|AB|, \quad
|VP|=\sqrt{\frac{\pi^2+16}{4\pi^2}}\cdot|AB|, \quad
|AE|=\sqrt{\frac{\pi^2+16}{2\pi^2+16}}\cdot|AB|.
$$
Bežne používaná približná hodnota $\pi\doteq\frac{22}7$ vedie vo výsledku k~chybe medzi 2 a~3\,cm.
Porovnaním celkových rozmerov a~proporcií ihlana to vyzerá, že Jozef a~Mária boli na dovolenke v~Egypte pri~Cheopsovej (resp. Chufuovej či Veľkej) pyramíde.

K~vyjadreniu výšky trojuholníka možno tiež dospieť so znalosťami goniometrických funkcií, ich základného vzťahu ($(\sin\beta)^2+(\cos\beta)^2=1$) a~kosínusovej vety ($|AV|^2=|AB|^2+|BV|^2-2|AV|\cdot|BV|\cdot\cos\beta$).
Tieto znalosti v~danej kategórii nemôžeme predpokladať, avšak zvedaví riešitelia sa s~nimi môžu zoznámiť, príp. porovnať tento prístup s~ostatnými.
}

{%%%%%   Z4-II-1
...}

{%%%%%   Z4-II-2
...}

{%%%%%   Z4-II-3
...}

{%%%%%   Z5-II-1
Kvôli prehľadnosti najskôr vyjadríme, koľko ktorých úsekov sme v~jednotlivých prípadoch prešli:
$$
\begintable
prechádzka\|dlhšia strana|kratšia strana|uhlopriečka\crthick
1.\|1|3|3\cr
2.\|3|1|3\crthick
3.\|2|1|2\cr
4.\|1|2|2\endtable
$$
Zaujíma nás rozdiel dĺžok prechádzok. Rovnako dlhé úseky, ktoré sú v oboch prechádzkach, preto môžeme vynechať.
Prvá a druhá prechádzka majú spoločné 3 uhlopriečky, 1 dlhšiu a 1 kratšiu stranu. Líšia sa len v tom, že v prvej prechádzke je ešte navyše dvakrát kratšia strana a v druhej je navyše dvakrát dlhšia strana. Dĺžka 500 metrov je teda rozdiel medzi dlhšou a kratšou stranou krát dva. Rozdiel medzi kratšou a dlhšou stranou je teda 250 metrov.

Tretia a štvrtá prechádzka sa líšia len tým, že v tretej je navyše jedna dlhšia strana a v štvrtej je navyše jedna kratšia strana. Tento rozdiel už poznáme z predchádzajúceho odseku. Tretia prechádzka je o 250 metrov dlhšia než štvrtá.

%Trasu druhej prechádzky možno vzhľadom k~prechádzke prvej vyjadriť %pridaním dvoch dlhších strán a~odobraním dvoch kratších strán %obdĺžnika.
%Trasu tretej prechádzky možno vzhľadom k~prechádzke štvrtej vyjadriť %pridaním jednej dlhšej strany a~odobraním jednej kratšej strany %obdĺžnika.

%Inak povedané:
%druhá trasa je dlhšia ako trasa prvá o~dvojnásobok rozdielu dĺžok %strán obdĺžnika, tretia trasa je dlhšia ako trasa štvrtá o~(jeden) %rozdiel dĺžok strán obdĺžnika.
%Prvá trasa je podľa zadania o~500 metrov dlhšia ako druhá.
%Rozdiel dĺžok tretej a~štvrtej trasy je polovičný, teda 250 metrov.

%Pri tretej prechádzke sme prešli o~250 metrov viac ako pri prechádzke %štvrtej.

\hodnotenie
2~body za pozorovanie, že rozdiel dĺžok sa nezmení, ak v porovnávaných dĺžkach vynecháme rovnako dlhé úseky;
2~body za popis rozdielov dĺžok prvej a~druhej dvojice trás;
2~body za dopočítanie a~odpoveď.
\eres
}

{%%%%%   Z5-II-2
Z tretej a~štvrtej informácie vyplýva, že červený šuplík je uprostred.

Z~predošlého a~prvej informácie vyplýva, že zelený šuplík je hore a~modrý dole.

Z tretej informácie ďalej vyplýva, že mušľa je v~hornom zelenom šuplíku.

Zo štvrtej informácie ďalej vyplýva, že guľôčka je v~dolnom modrom šuplíku.

Pre mincu tak ostáva prostredný červený šuplík, čo je v~súlade s~druhou informáciou.

Teda poradie šuplíkov a~v~nich obsiahnutých predmetov vyzerá nasledovne:
$$
\begintable
zelený\hfill\|mušľa\hfill\cr
červený\hfill\|minca\hfill\cr
modrý\hfill\|guľôčka\hfill\endtable
$$
Minca je v prostrednom červenom šuflíku.

\hodnotenie
5~bodov za správne určenie šuplíka s mincou; stačí aj jedna z~odpovedí "Minca je v červenom šuflíku" alebo "Minca je v strednom šuflíku", pokiaľ je v~riešení zdôvodnená.
1~bod za overenie, či má úloha naozaj riešenie, teda dohľadanie farieb a obsahu všetkých šuplíkov.
%
%Pri neúplnom riešeni ?? Betka- mozno by som dala 4 + 2
\eres
}

{%%%%%   Z5-II-3
Keby žiaden z~mesiačikov perníčky neodmietol, rozobrali by si ich celkom
$$
1+2+3+4+5+6+7+8+9+10+11+12=78.
$$
V~takom prípade by zvýšilo 12 perníčkov ($90-78=12$).

Marienke ich ale zvýšilo 21, \tj. o~9 viac ($21-12=9$).
Dvaja mesiačikovia, ktorí si perníčky nevzali, ich teda mali mať dokopy 9.

Z toho dostávame nasledujúce štyri možné dvojice mesiačikov, ktorí mohli odmietnuť:
$$
\begintable
1 a 8\|január a august\hfill\cr
2 a 7\|február a júl\hfill\cr
3 a 6\|marec a jún\hfill\cr
4 a 5\|apríl a máj\hfill\endtable
$$

\hodnotenie
Body možno udeliť nasledovne (jednotlivé položky sa nesčítajú):

\item{$\triangleright$}1~bod za vyskúšanie aspoň jedného, aj keď nesprávneho, rozdelenia perníčkov;
\item{$\triangleright$}2~body za nájdenie 1 správneho rozdelenia perníčkov;
\item{$\triangleright$}3~body za nájdenie 2 správnych rozdelení perníčkov;
\item{$\triangleright$}4~body za nájdenie 3 správnych rozdelení perníčkov;
\item{$\triangleright$}5~bodov za nájdenie 4 správnych rozdelení perníčkov;
\item{$\triangleright$}6~bodov za nájdenie 4 správnych rozdelení perníčkov a zdôvodnenie prečo žiadne iné rozdelenia nevyhovujú.
\eres
}

{%%%%% Z6-II-1
Naznačíme možnosti, koľko orieškov mohla zjesť druhá a~tretia veverička, \tj. číslo 1277 rozložíme na dva sčítance, z~ktorých každý je rovný aspoň 103:
$$
1277 = 1174 + 103 = \dots = 640 + 637 = {639} + {638}.
$$
Z toho vyplýva, že prvá veverička zjedla aspoň 640 orieškov (zjedla viac ako ktorákoľvek z~ostatných veveričiek).

Na prvú a~štvrtú veveričku dokopy zvýšilo $2020 - 1277 = 743$ orieškov.
Opäť naznačíme možnosti, koľko orieškov mohla zjesť prvá a~štvrtá veverička:
$$
743 = {640} + {103} = 639 + 104 = \dots
$$
Z toho vyplýva, že prvá veverička zjedla nanajvýš 640 orieškov.

Prvá veverička zjedla práve 640 orieškov.

\hodnotenie
Po 2~bodoch za každý z~načrtnutých výpisov;
2~body za záver.
\endhodnotenie
}

{%%%%% Z6-II-2
Neznáme čísla v~dolnej vrstve pyramídy označíme postupne $a$, $b$, $c$.
V~druhej vrstve budú čísla $ab$, $bc$ a~v~hornej číslo $ab^2c$.

Úlohou je nájsť číslo $b$ tak, aby $ab^2c = 90$,
\tj. nájsť v~rozklade čísla 90 druhú mocninu celého čísla.
Prvočíselný rozklad čísla 90 je
$$
90= 2 \cdot 3^2 \cdot 5.
$$
Teda vo vyznačenom políčku môže byť buď číslo 1, alebo 3.

\hodnotenie
2~body za úvahu opisujúcu číslo vo vrchole pyramídy v tvare $ab^2c$;
2~body za rozklad čísla 90;
2~body za všetky riešenia.
\endhodnotenie

}

{%%%%% Z6-II-3
Pletivová časť dvierok pozostáva zo zhodných štvorcov a~ich polovíc.
Vnútorné hrany rámu sú tvorené uhlopriečkami týchto štvorcov, a~to tromi na kratšej strane a~štyrmi na dlhšej.
Ak je $a$ dĺžka uhlopriečky štvorca, má pletivová časť rozmery $3a \times 4a$.
\insp{z6-II-3a.eps}%

\noindent
To zodpovedá obsahu $432\cm^2$, teda
$$
12a^2=432,\quad
a^2=36,\quad
a=6.
$$

Vnútorné rozmery dvierok sú $18\cm \times 24\cm$, vonkajšie rozmery potom $28\cm \times 34\cm$ (na každej strane pridaných 5\cm).

\poznamka
Pletivovú časť dvierok tvorí $17$ zhodných (celých) štvorcov a~$14$ polovíc týchto štvorcov.
Dokopy teda $24$ štvorcov, ktoré majú obsah $432\cm^2$.
Na jeden štvorec tak pripadá $18\cm^2$.
Obsah štvorca s~uhlopriečkou dĺžky $a$ je rovný $\frac12 a^2$.
Uhlopriečku pletivového štvorca možno teda určiť úpravami:
$$
\frac12 a^2=18,\quad
a^2=36,\quad
a=6.
$$

\hodnotenie
2~body za pomocné rozklady a~úvahy;
2~body za pomocné výpočty;
2~body za záver a~kvalitu komentára.
\endhodnotenie
}

{%%%%% Z7-II-1
Keďže ako kyklopi, tak draci sú dvojnohí, všetkých týchto bytostí je celkom~17 ($34 : 2 = 17$).

Keby všetky bytosti boli kyklopi, mali by celkom 17 očí.
To je o~25 menej, než koľko ich je v skutočnosti ($42 - 17 = 25$).

Každý drak má o~5 očí viac ako ktorýkoľvek kyklop, teda medzi bytosťami je 5~drakov ($25 : 5 = 5$).
Zvyšných 12 bytostí sú kyklopi ($17 - 5 = 12$).

\ineriesenie
Úlohu možno riešiť aj skúšaním možností:
celkom je na ostrove 17 bytostí ($34:2=17$), medzi ktorými je nanajvýš 7 drakov ($42:6=7$).

V~nasledujúcej tabuľke uvádzame v~závislosti od počtu drakov ($d$) počet kyklopov ($k=17-d$) a~celkový počet ich očí ($6d+k=5d+17$), ktorý má byť 42:
$$\begintable
draci\|1|2|3|4|\bf 5|6|7\cr
kyklopi\|16|15|14|13|\bf 12|11|10\crthick
celkom očí\|22|27|32|37|42|47|52\endtable
$$
Jediné vyhovujúce riešenie je vyznačené silno.

\poznamka
S označením uvedeným v~opise predchádzajúcej tabuľky možno počet drakov určiť ako riešenie rovnice
$5d+17=42$ (čo tiež zodpovedá úvahám v~prvom riešení úlohy).

Skúšanie je možné založiť na inom princípe, pričom nie je nutné poznať počet bytostí na ostrove (zato uvažovať počty nôh).

\hodnoceni
2~body za čiastočné postrehy (napr. celkový počet bytostí či maximálny počet drakov);
3~body za doriešenie úlohy;
1~bod za úplnosť a~kvalitu komentára.
\eres
}

{%%%%% Z7-II-2
Doplnením chýbajúcich uhlopriečok šesťuholníka získame jeho rozdelenie na 12 navzájom zhodných trojuholníkov
(uhlopriečky prechádzajúce stredom šesťuholníka ho rozdeľujú na šesť zhodných rovnostranných trojuholníkov, zvyšné tri uhlopriečky predstavujú výšky v~týchto trojuholníkoch).
\insp{z7-II-2a.eps}%

Trojuholníky $C$, $D$, $G$ sú tri z~týchto základných trojuholníkov,
každý z~trojuholníkov $A$, $B$, $E$ je tvorený dvoma základnými trojuholníkmi a~štvoruholník $F$ tromi.

Obsah štvoruholníka $F$ je $1{,}8\cm^2$, teda obsah základného trojuholníka je $0{,}6\cm^2$.
Každý z~trojuholníkov $C$, $D$, $G$ má obsah $0{,}6\cm^2$ a~každý z~trojuholníkov $A$, $B$, $E$ má obsah $1{,}2\cm^2$.

\hodnoceni
3~body za pomocné delenie šesťuholníka a~porovnanie posudzovaných častí;
2~body za doriešenie úlohy;
1~bod za kvalitu komentára (zahŕňajúcu najmä zhodnosti pomocných trojuholníkov).
\eres
}

{%%%%% Z7-II-3
Jedným veľkým štvorkrokom (tromi dlhými krokmi vpred a~jedným krátkym vzad) sa Jozefína posunie o~135\,cm ($3 \cdot 60 - 45 = 135$).
Deväťdesiat krokov pozostáva z~22 veľkých štvorkrokov a~dvoch dlhých krokov ($90=22\cdot4+2$).
Jozefínin okruh teda meria 3\,090\,cm ($22\cdot135+2\cdot60=3\,090$).

Jedným malým štvorkrokom (tromi krátkymi krokmi vpred a~jedným dlhým vzad) sa Jozefína posunie o~75\,cm ($3 \cdot 45 - 60 = 75$).
Štyridsať takých štvorkrokov -- teda 160 krokov -- ju posunie o~3\,000\,cm ($40\cdot75=3\,000$), pričom takto určite neprekročí pôvodné miesto (pred posledným spiatočným krokom je vo vzdialenosti 3\,060\,cm).
Zvyšných 90\,cm prejde dvoma nasledujúcimi krokmi ($2\cdot45=90$).

Jozefína dotancuje na pôvodné miesto 162. krokom.

\poznamka
Pre priblíženie situácie uvádzame niekoľko vzdialeností (v~cm) prislúchajúcich tanečným krokom používaným v druhom prípade:
$$\begintable
krok\|1|2|3|4|5|6|\dots\cr
vzdialenosť\|45|90|135|75|120|165|\dots\endtable
$$
$$\begintable
krok\|\dots|158|159|160|161|162|163\cr
vzdialenosť\|\dots|3\,015|3\,060|3\,000|3\,045|3\,090|3\,135\endtable
$$
Z toho je tiež zrejmé, že rôznym krokom zodpovedajú rôzne vzdialenosti.

\hodnoceni
2~body za dĺžku Jozefíninho okruhu;
2~body za počet krokov v druhom prípade;
2~body za zrozumiteľnosť a~kvalitu komentára.
\eres
}

{%%%%% Z8-II-1
Uvažujeme počty samolepiek a~pečiatok vyhovujúce poslednej uvedenej podmienke, potom kontrolujeme ostatné požiadavky:

Jednu samolepku a~dve pečiatky možno získať za 17 bodov ($5+2\cdot6=17$).
Pri zmene 17 bodov len za samolepky by sa dva body nevyužili (zvyšok po delení $17:5$ je~2), čo nesúhlasí so zadaním.

Pre $s$ samolepiek máme $2s$ pečiatok a~$17s$ bodov.
Hľadáme najmenšie $s$ také, že $17s$ dáva po delení piatimi zvyšok jedna a~po delení šiestimi zvyšok tri:
$$\begintable
samolepky\|1|2|3\cr
pečiatky\|2|4|6\crthick
body\|17|34|\bf 51\crthick
zvyšok po del. 5\|2|4|1\cr
zvyšok po del. 6\|5|4|3\endtable
$$
Jaro mal pred menením najmenej 51 bodov.

\ineriesenie
Uvažujeme počty samolepiek a~pečiatok vyhovujúce prvým dvom podmienkam, potom kontrolujeme ostatné požiadavky:

Pri menení bodov len za samolepky by sa jeden bod nevyužil; možné počty sú (násobky piatich zväčšené o~jedna)
$$
6,\ 11,\ 16,\ {\bold{21}},\ 26,\ 31,\ 36,\ 41,\ 46,\ {\bold{51}},\ \dots
$$
Pri menení bodov len za pečiatky by sa tri body nevyužili; možné počty sú (násobky šiestich zväčšené o~tri)
$$
9,\ 15,\ {\bold{21}},\ 27,\ 33,\ 39,\ 45,\ {\bold{51}},\ \dots
$$
Hodnoty vyhovujúce obom požiadavkám súčasne sú vyznačené silno.

Bezo zvyšku možno 21 bodov zmeniť jedine za tri samolepky a~jednu pečiatku ($21=3\cdot5+1\cdot6$).
V~takom prípade by nebolo pečiatok dvojnásobné množstvo ako samolepiek.

Bezo zvyšku možno 51 bodov zmeniť buď za tri samolepky a~šesť pečiatok, alebo za deväť samolepiek a~jednu pečiatku ($51 =3\cdot5+6\cdot6 =9\cdot5+6$).
V~prvom prípade je pečiatok dvojnásobné množstvo ako samolepiek.

\penalty100
Jaro mal pred menením najmenej 51 bodov.

\poznamka
Vyššie diskutované podmienky pre neznámy počet bodov sú:
\begin{itemize}
\item zvyšok po delení piatimi je jedna,
\item zvyšok po delení šiestimi je tri,
\item zvyšok po delení sedemnástimi je nula.
\end{itemize}
Všetky prirodzené čísla vyhovujúce týmto trom podmienkam sú tvaru $51+{5\cdot6\cdot17\cdot k}=51+510k$, pričom $k$ je ľubovoľné nezáporné celé číslo.
Úloha súvisí s~tzv. {\it čínskou vetou o~zvyškoch\/}.

\hodnoceni
2~body za nájdenie systému preverovania možností (násobky 17 v~prvom riešení, správne zvyšky v druhom riešení);
2~body za doriešenie úlohy;
2~body za úplnosť a~kvalitu komentára.
\eres
}

{%%%%% Z8-II-2
Počty koláčov v~dolnom rade pyramídy označíme postupne $a$, $b$, $c$, $d$ a~z toho vyjadríme počty v~ostatných škatuliach:
\insp{z8-II-2a.eps}%

Súčet všetkých uvedených čísel je
$$
4a + 9b + 9c + 4d = 4(a+d) + 9(b + c) .
$$

Čísla v~škatuliach označených hviezdičkami sú $a$, $d$ a~$b+c$, čo sú (až na poradie) čísla 3, 5 a~6.
Celkový súčet je najmenší možný práve vtedy, keď číslo $b+c$ (\tj. číslo prispievajúce do súčtu najväčšou váhou) je najmenšie možné, teda 3.

V~označenej škatuli v druhom rade zdola boli tri koláče.

\poznamka
Úlohu možno riešiť aj rozborom všetkých možných rozmiestnení známych počtov koláčov.
Vzhľadom na zrejmé symetrie stačí uvažovať tri prípady:
\begin{itemize}
\item $a=3$, $d=5$, $b+c=6$, čo dáva celkový súčet 86,
\item $a=3$, $d=6$, $b+c=5$, čo dáva celkový súčet 81,
\item $a=5$, $d=6$, $b+c=3$, čo dáva celkový súčet 71.
\end{itemize}
Najmenší súčet nastáva v treťom prípade, čo súhlasí s~predchádzajúcimi závermi.

\hodnoceni
3~body za všeobecné doplnenie pyramídy, resp. rozbor možností;
3~body za doriešenie úlohy.
\eres
}

{%%%%% Z8-II-3
Kvôli prehľadnosti situáciu najskôr znázorníme:
\insp{z8-II-3.eps}%

Trojuholníky $ABC$ a~$DBC$ majú rovnakú výšku z~vrcholu $C$ a~protiľahlé strany ($AB$ a~$DB$) sú v~pomere $3:2$.
V rovnakom pomere sú tiež ich obsahy,
$$
S_{ABC}:S_{DBC}=3:2.
$$

Trojuholníky $BCD$ a~$ECD$ majú rovnakú výšku z~vrcholu $D$ a~protiľahlé strany ($BC$ a~$EC$) sú v~pomere $3:2$.
V rovnakom pomere sú tiež ich obsahy, čo spolu s~predchádzajúcim výsledkom dáva
$$
S_{ABC}:S_{ECD}=9:4.
$$

Trojuholníky $CDE$ a~$FDE$ majú rovnakú výšku z~vrcholu $E$ a~protiľahlé strany ($CD$ a~$FD$) sú v~pomere $3:2$.
V rovnakom pomere sú tiež ich obsahy, čo spolu s~predchádzajúcimi výsledkami dáva
$$
S_{ABC}:S_{DEF}=27:8.
$$

\hodnoceni
1~bod za znázornenie situácie;
po 1~bode za každé z~čiastočných pozorovaní;
2~body za záver a~kvalitu komentára.
\eres
}

{%%%%%   Z9-II-1
Označme $a$ dĺžku strany pôvodnej štvorcovej záhrady.
Po dokupovaní vznikol nový štvorcový pozemok, ktorého strana mala dĺžku $a+3$.
Pre výmery pozemkov podľa zadania platí
$$
(a+3)^2 = 2 a^2 + 9.
$$
Ekvivalentnými úpravami dostávame
$$
\aligned
a^2 + 6a + 9 &= 2 a^2 + 9, \\
6a &= a^2, \\
0&=a(a-6).
\endaligned
$$
Uvedená rovnica má dve riešenia: $a=0$ a~$a=6$.

Pôvodná záhrada mala nenulové rozmery, teda jej strana bola dlhá šesť metrov.

\poznamka
Síce nevieme, ako vyzerali dokúpené pozemky, ale výmery záhrad (a~ich pomocné delenia) môžeme znázorniť nasledovne:
\insp{z9-II-1.eps}%

Veľký štvorec je zložený z dvoch štvorcov a~dvoch zhodných obdĺžnikov, resp. z~jedného štvorca a~štyroch zhodných obdĺžnikov.
Takto názorne vyjavujeme výsledok $a=3+3=6$.

\hodnoceni
2~body za formuláciu podmienok zo zadania pomocou jednej neznámej;
2~body za ekvivalentné úpravy;
2~body za vylúčenie nulového riešenia a~záver.

Riešenia založené na grafickom znázornení hodnoťte podľa kvality sprievodného komentára.
\eres
}

{%%%%%   Z9-II-2
Označme $v$ vek vnučky a~$b$ vek babičky.
Vek babičky je násobkom veku vnučky, teda $b = kv$ pre nejaké prirodzené $k$.

Štvorciferné číslo zapísané vnučkou bolo $100b+v$, štvorciferné číslo napísané babičkou bolo $100v+b$, teda
$$
(100b+v)-(100v+b) = 7128.
$$
Po úpravách (a~dosadení $b=kv$) dostávame
$$
\aligned
99(kv-v) &= 7128,\\
%kv-v &= 72,\\
v(k-1) &= 72.
\endaligned
$$
Úlohou je nájsť $v$ a~$k$ tak, aby platila predchádzajúca rovnosť a~navyše $v>10$ a~$b=kv<100$.

Číslo 72 možno (až na poradie činiteľov) vyjadriť nasledujúcimi šiestimi spôsobmi:
$$
72=72\cdot1=36\cdot2=24\cdot3=18\cdot4=12\cdot6=9\cdot8.
$$
Postupne rozoberieme všetky možnosti vyhovujúce $v>10$ a~určíme zodpovedajúce $k$ a~$b=kv$:
$$
\begintable
$v$\|72|36|\bf 24|\bf 18|\bf 12\cr
$k$\|2|3|4|5|7\cr
$b$\|144|108|\bf 96|\bf 90|\bf 84%
\endtable
$$
Silno sú vyznačené vyhovujúce výsledky, \tj. tie, pre ktoré platí $b<100$.
Úloha má tri riešenia.

\ineriesenie
Označme $v$ vek vnučky a~$b$ vek babičky, ďalej $v=\overline{AB}$ a~$b=\overline{CD}$ dekadické zápisy týchto čísel.
Informáciu o~rozdiele štvorciferných čísel zo zadania vyjadríme pomocou algebrogramu:
$$
\alggg{&C&D&A&B\\-&A&B&C&D}{&7&1&2&8}
$$
Keďže vnučka je mladšia ako babička, dochádza na posledných dvoch miestach k~,,prechodu cez desiatku``:
$$
\alggg{&1&A&B\\-&&C&D}{&&2&8}
$$
Teda babička je o~72 rokov staršia ako vnučka.

Keďže vnučka má aspoň 11 rokov, má babička aspoň 83 rokov.
Keďže babička má nanajvýš 99 rokov, má vnučka nanajvýš 27 rokov.
V~týchto medziach stačí prebrať všetky dvojice $v$ a~$b=v+72$ a~overiť, či $b$ je násobkom $v$:
$$
\begintable
$v$\|11|\bf 12|13|14|15|16|17|\bf 18|19\crthick
$b$\|83|\bf 84|85|86|87|88|89|\bf 90|91\endtable
$$
$$
\begintable
$v$\|20|21|22|23|\bf 24|25|26|27\crthick
$b$\|92|93|94|95|\bf 96|97|98|99\endtable
$$
Vyhovujúce výsledky sú vyznačené silno; úloha má tri riešenia.

\hodnoceni
Po 1~bode za každé vyhovujúce riešenie;
3~body za úplnosť a~kvalitu komentára.
\eres
}

{%%%%%   Z9-II-3
Označme postupne $K$, $L$ a~$M$ počty známok, ktoré vlastní Karol, Miro a~Ľudo.
Podľa zadania platí
$$
K+M =101,\quad K+L =115,\quad M+L=110.
$$
Súčtom týchto troch rovností a~ďalšími úpravami postupne dostávame:
$$
\aligned
2K+2M+2L &=326, \\
K+M+L &=163, \\
L &=163-(K+M).
\endaligned
$$
Dosadením prvej z~úvodnej trojice rovníc zisťujeme, že Ľudo má celkom 62 známok ($L=163-101 =62$).

Z~týchto 62 známok má 12 rovnakých ako Karol a~7 rovnakých ako Miro.
Ľudo teda má 43 známok iných ako ostatní chlapci ($62-12-7=43$).

\ineriesenie
Označme postupne $k$, $l$ a~$m$ počty známok, ktoré vlastní iba Karol, iba Miro a~iba Ľudo.
Podľa zadania platí
$$
\aligned
(k + 5 + 12) + (m + 5 + 7) &= 101, \\
(k + 5 + 12) + (l + 7 + 12) &= 115, \\
(m + 5 + 7) + (l + 7 + 12) &= 110.
\endaligned
$$
Úpravami jednotlivých riadkov dostávame ekvivalentnú sústavu
$$
\aligned
k+m&=72, \\
k+l&=79, \\
m+l&=79.
\endaligned
$$

Z~posledných dvoch rovníc vyplýva, že $k=m$, z~prvej potom $k = m = 36$.
Z toho ďalej dopočítame $l=43$.
Ľudo má 43 známok iných ako ostatní chlapci.

\poznamka
Informácie o~počtoch známok možno znázorniť pomocou Vennovho diagramu takto:
\insp{z9-II-3a.eps}%

Pri vyjadrovaní súčtov známok sa hodnoty z~prislúchajúcich prienikov počítajú dvakrát.
Prehľadnejšie možno tento poznatok znázorniť nasledovne:
\insp{z9-II-3b.eps}%


\hodnoceni
3~body za vyjadrenie informácií zo zadania pomocou rovníc a~ich úpravy;
3~body za doriešenie sústavy a~záver.
\eres
}

{%%%%%   Z9-II-4
Lichobežník s~danými veľkosťami základne ($AB$), výšky ($AF$) a~ramien ($AD$ a~$BC$) nie je určený jednoznačne; môžu nastať nasledujúce možnosti:
\insp{z9-II-4.eps}%

Všetky tieto lichobežníky chápeme tak, že vznikli z~obdĺžnika $ABEF$ prikladaním, príp. odoberaním pravouhlých trojuholníkov $AFD$ a~$BEC$.
V~závislosti od veľkostí daných úsečiek sa vo všeobecnosti môže stať, že štvoruholník $ABCD$ je nekonvexný.%
\footnote[${}^{\dag}$]{Napr. v~prvom prípade by táto situácia nastala, ak by $|AB|<|FD|+|EC|$.}
To uvidíme, akonáhle dopočítame neznáme veľkosti úsečiek.

V~nasledujúcich výpočtoch nepíšeme jednotky (všade cm) a~dosadzujeme hodnoty zo zadania: $|AB|=30$, $|AF|=24$, $|AD|=25$ a~$|BC|=30$.
Podľa Pytagorovej vety v~trojuholníkoch $AFD$ a~$BEC$ dopočítame veľkosti zvyšných odvesien:
$$
\def|{\vert}
\align
|FD| &=\sqrt{|AD|^2-|AF|^2} =\sqrt{25^2-24^2} =7, \\
|EC| &=\sqrt{|BC|^2-|BE|^2} =\sqrt{30^2-24^2} =18.
\endalign
$$

V~prvom, resp. v druhom prípade vychádza
$$
\def|{\vert}
\align
|CD|&=|AB|-|FD|-|EC|=30-7-18=5, \\
|CD|&=|AB|+|FD|-|EC|=30+7-18=19.
\endalign
$$
V~oboch prípadoch je výsledný rozdiel kladný a~menší ako 30, teda sa jedná o~lichobežník, ktorého dlhšia základňa je $AB$.
Vo zvyšných dvoch prípadoch vychádza ako dlhšia základňa $CD$, takže sa týmito prípadmi nemusíme zaoberať.

V~prvom, resp. v druhom prípade obvod lichobežníka $|AB|+|BC|+|CD|+|DA|$ vychádza
$$
30+30+5+25=90,
\quad\text{resp.}\quad
30+30+19+25=104,
$$
a~to sú výsledky Adama a~Evy.

\poznamka
Pri ručnom počítaní veľkostí odvesien $FD$ a~$EC$ možno s~výhodou využiť nasledujúce úpravy:
$$
\aligned
\sqrt{25^2-24^2} &= \sqrt{(25-24)(25+24)} =\sqrt{49} =7,\\
\sqrt{30^2-24^2} &= \sqrt{(30-24)(30+24)} =\sqrt{6\cdot54} =\sqrt{18\cdot18} =18.
\endaligned
$$

\hodnoceni
2~body za rozbor možností;
2~body za pomocné výpočty;
2~body podľa kvality komentára.
\eres
}

{%%%%%   Z9-III-1
Z~prvých troch podmienok vyplýva, že prvé číslo je násobkom 24 (aby bolo deliteľné 8 a~6), druhé číslo je násobkom 8 a~tretie číslo je násobkom 6.
Keďže zmieňované delitele sú najväčšie možné, musia byť neznáme čísla tvaru
$$
24a,\quad 8b,\quad 6c,
\tag{1}
$$
pričom $a$, $b$, $c$ sú po dvoch nesúdeliteľné čísla.

Najmenší spoločný násobok takej trojice čísel je $24\cdot a\cdot b\cdot c$, čo má podľa zadania byť rovné $1680=24\cdot 2\cdot5\cdot7$.
Aby jedno z~čísel bolo štvrtou mocninou celého čísla,
musí to byť druhé, a~to v~prípade, že $b=2$
(prvé aj tretie číslo je deliteľné tromi, ale najmenší spoločný násobok všetkých troch čísel nie je deliteľný $3^4=81$).
Teda neznáme čísla sú tvaru
$$
24a,\quad 16,\quad 6c,
\tag{2}
$$
pričom čísla $a$, $c$ sú 5, 7 alebo 1, 35 (v~ľubovoľnom poradí).

Aby žiadne z~čísel nebolo väčšie ako 200, nemôže byť ani $a$, ani $c$ rovné 35.
Ostávajú tak dve možnosti: buď $a=5$ a~$c=7$, alebo $a=7$ a~$c=5$.
Oba prípady vyhovujú poslednej požadovanej podmienke, \tj. aby najväčšie z~hľadaných čísel bolo väčšie ako 100.
Neznáma trojica čísel je niektorá z~nasledujúcich
$$
\gathered
120,\quad 16,\quad 42, \\
168,\quad 16,\quad 30.
\endgathered
$$

\hodnotenie
2~body za odvodenie (1);
2~body za rozklad čísla 1680 a~odvodenie (2);
2~body za rozbor možností a~výsledok.
\endhodnotenie
}

{%%%%%   Z9-III-2
Každá zo strán trojuholníka $ACH$ je uhlopriečkou niektorej steny kocky.
Steny kocky sú navzájom zhodné štvorce, teda trojuholník $ACH$ je rovnostranný.

Vzťah medzi veľkosťami výšky $v$ a~strany $b$ rovnostranného trojuholníka je $v=\frac{\sqrt{3}}{2}b$.
Veľkosť strany trojuholníka $ACH$ je teda
$$
b =\frac{2\cdot12}{\sqrt{3}} =8\sqrt{3} \doteq 13{,}9\,(\Cm).
$$

Obsah trojuholníka so stranou veľkosti $b$ a~zodpovedajúcou výškou $v$ je $S=\frac12 b\cdot v$.
Obsah trojuholníka $ACH$ je teda
$$
S =\frac{12\cdot 8\sqrt3}{2} =48\sqrt3 \doteq 83{,}1\,(\Cm^2).
$$

Vzťah medzi veľkosťami strany $a$ a~uhlopriečky $b$ štvorca je $b=a\sqrt2$.
Veľkosť hrany kocky je teda
$$
a =\frac{8\sqrt3}{\sqrt2} =4\sqrt6 \doteq 9{,}8\,(\Cm).
$$

\hodnotenie
Po 1~bode za rozpoznanie rovnostrannosti trojuholníka $ACH$, vyjadrenie $b$, $S$ a~$a$;
2~body podľa kvality komentára.

\poznamky
Vzťahy $v=\frac{\sqrt{3}}{2}b$ a~$b=a\sqrt2$ možno považovať za známe, dajú sa odvodiť pomocou Pytagorovej vety.

Podľa spôsobu vyjadrenia výrazov s~odmocninami sa môžu drobne líšiť výsledky po zaokrúhľovaní.
Také rozdiely nemajú vplyv na hodnotenie úlohy.
\endhodnotenie
}

{%%%%%   Z9-III-3
Ak je číslo $b$ aritmetickým priemerom $a$ a~$c$, tak rozdiely $b-a$ a~$c-b$ sú rovnaké:
$$
\aligned
b&=\frac{a+c}2, \\
b+b&=a+c, \\
b-a&=c-b.
\endaligned
$$
Všetky úpravy sú ekvivalentné, teda platí aj opačné tvrdenie:
ak sú rozdiely $b-a$ a~$c-b$ rovnaké, tak je číslo $b$ aritmetickým priemerom $a$ a~$c$.

Zo zadania vyplýva, že rozdiely dvojíc susedných čísel sú rovnaké:
$$
b-a =c-b =d-c =e-d =f-e =g-f.
$$
Rozdiely $d-a$ a~$g-d$ sú preto tiež rovnaké (a~sú rovné trojnásobku rozdielu susedných čísel).
Číslo $d$ je teda aritmetickým priemerom čísel $a$ a~$g$.

\hodnotenie
3~body za rozpoznanie vzťahu medzi aritmetickým priemerom a~rozdielmi susedných čísel;
3~body za vlastné uplatnenie a~kvalitu komentára.

\poznamky
Úlohu možno riešiť formálne manipuláciou so vzťahmi $b=\frac{a+c}2$, $c=\frac{b+d}2$ a pod.
alebo aj názorne s~číselnou osou.
Také riešenie hodnoťte podľa úplnosti a~kvality prevedenia.

Postupnosti s~uvedenými vlastnosťami sa nazývajú aritmetické.
\endhodnotenie
}

{%%%%%   Z9-III-4
Rovnobežníky $ABGH$ a~$DEGH$ majú spoločnú stranu a~rovnakú výšku, majú teda rovnaký obsah.
To znamená, že pre obsahy prislúchajúcich trojuholníkov platí
$$
S_{ABI}+S_{BCI}+S_{CHI}+S_{CGH} = S_{DEF}+S_{CDF}+S_{CFG}+S_{CGH}.
\tag{1}
$$

Nezávisle od polohy bodu $I$ na úsečke $AH$ je obsah trojuholníka $BCI$ stále rovnaký.
Ak by bod $I$ splynul s~bodom $A$, tvoril by tento trojuholník spolu s~trojuholníkom $CGH$ polovicu rovnobežníka $ABGH$.
Z~podobného dôvodu tvoria aj trojuholníky $CDF$ a~$CGH$ polovicu obsahu rovnobežníka $DEGH$.
Celkom teda pre obsahy trojuholníkov platí
$$
S_{BCI}+S_{CGH} =
S_{ABI}+S_{CHI} =
S_{DEF}+S_{CFG} =
S_{CDF}+S_{CGH}.
\tag{2}
$$
Preto trojuholníky $BCI$ a~$CDF$ majú rovnaký obsah, a~to práve 7\,cm$^2$.

Tri zo štyroch trojuholníkov $ABI$, $CHI$, $DEF$ a~$CFG$ majú obsahy 3\,cm$^2$, 5\,cm$^2$ a~10\,cm$^2$.
Rozoberieme všetky možné súčty (2), z toho určíme obsah zvyšného trojuholníka z~uvedenej štvorice, obsah trojuholníka $CGH$ a~obsah mnohouholníka $ABCDEGH$:
$$
\begintable
súčet (2)\hfill\|$3+5=8$|$3+10=13$|$5+10=15$\cr
obsah zvyšného\hfill\|$8-10=-2$|$13-5=8$|$15-3=12$\cr
obsah $CGH$\hfill\|---|$13-7=6$|$15-7=8$\cr
obsah celého\hfill\|---|$14+26+6=46$|$14+30+8=52$\endtable
$$

V~oboch prípadoch je splnená podmienka, že okrem trojuholníkov s~obsahmi 7\,cm$^2$ nemá žiadna ďalšia dvojica trojuholníkov rovnaký obsah.
Mnohouholník $ABCDEGH$ má obsah buď 46\,cm$^2$, alebo 52\,cm$^2$.

\hodnotenie
1~bod za odvodenie vzťahu (1);
2~body za vzťahy (2) a~rozpoznanie trojuholníkov s~obsahmi 7\,cm$^2$;
3~body za rozbor možností, overenie podmienok a~určenie možných obsahov mnohouholníka $ABCDEGH$.
\endhodnotenie
}

