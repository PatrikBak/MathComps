{%%%%%   A-I-1
a) Pre prvočísla $p_1$, $p_2$, $p_3$, $p_4$, $p_5$
napísané na tabuli podľa zadania platí
$$
p_1p_2p_3p_4p_5=105(p_1+p_2+p_3+p_4+p_5).
\tag1
$$
Keďže číslo na pravej strane~\thetag1 je násobkom čísla $105=3\cdot5\cdot7$,
niektoré tri z~piatich prvočísel na ľavej strane \thetag1 sa rovnajú $3$, $5$ a $7$.
Bez ujmy na všeobecnosti tak môžeme predpokladať, že platí
$p_3=3$, $p_4=5$ a~$p_5=7$. Rovnosť~\thetag1 sa potom po dosadení a~vydelení
oboch strán číslom~105 zjednoduší na
$$
p_1p_2=p_1+p_2+15.
$$
Takú rovnicu s~neznámymi celými číslami $p_1$, $p_2$ štandardnou
úpravou prevedieme na súčinový tvar
$$
(p_1-1)(p_2-1)=16.
\tag2
$$
V~našej úlohe sú $p_1$, $p_2$ prvočísla, takže oba činitele
$p_1-1$ a~$p_2-1$ sú prirodzené čísla. Určite môžeme označenie voliť
tak, aby platilo $p_1\geqq p_2$, a teda aj $p_1-1\geqq p_2-1$.

Číslo~16 z~pravej strany \thetag2 možno rozložiť na súčin
dvoch prirodzených čísel $a=p_1-1$, $b=p_2-1$ spĺňajúcich podmienku $a~\ge b$
práve týmito spôsobmi: $16 \cdot 1$, $8 \cdot 2$ a~$4 \cdot 4$.
Týmto rozkladom postupne zodpovedajú dvojice $(p_1,p_2)$ rovné
$(17,2)$, $(9,3)$, $(5,5)$. Iba prostredná z~nich nie je
dvojicou prvočísel. Doplnením oboch krajných dvojíc o trojicu
$(3,5,7)$ z~úvodnej časti riešenia dostaneme jediné dve riešenia
časti a) úlohy, ktoré teraz zapíšeme kvôli prehľadnosti
ako pätice prvočísel v~neklesajúcom poradí: $2,3,5,7,17$ a
$3,5,5,5,7$.

\medskip
b) V~tomto prípade má pre prvočísla $p_1$, $p_2$, $p_3$, $p_4$,
$p_5$, $p_6$, $p_7$ napísané na tabuli platiť rovnosť
$$
p_1p_2p_3p_4p_5p_6p_7=105(p_1+p_2+p_3+p_4+p_5+p_6+p_7).
$$
Podobne ako v~časti~a) usúdime, že bez ujmy na všeobecnosti sú
$p_5$, $p_6$ a~$p_7$ prvočísla 3, 5 a~7 a že nám tak ostáva riešiť
rovnicu
$$
p_1p_2p_3p_4 = p_1+p_2+p_3+p_4+15.
\tag3
$$

Určite môžeme predpokladať, že~$p_1$ je najväčšie (prípadne jedno
z~najväčších) medzi číslami $p_1$, $p_2$, $p_3$, $p_4$.
Keďže sa jedná o~prvočísla, je každé z~nich aspoň 2.
Tým pádom môžeme odhadnúť ľavú aj pravú stranu rovnice~\thetag3 nasledovne:
$$
p_1 \cdot 2 \cdot 2 \cdot 2 \leq
p_1p_2p_3p_4 = p_1+p_2+p_3+p_4+15 \leq 4p_1+15.
$$
Pre upravené krajné výrazy tak máme nerovnosť
$8p_1 \leq 4p_1+15$, odkiaľ $p_1 \leq 3{,}75$,
teda pre prvočíslo $p_1$ platí $p_1\in\{2,3\}$.
Keďže sme za $p_1$ vybrali najväčšie z~prvočísel
$p_1$, $p_2$, $p_3$, $p_4$, je každé z~nich rovné 2 alebo 3.

Predchádzajúcim odsekom sme celý postup riešenia rovnice \thetag3
v~obore prvočísel zredukovali len na overenie,
ktoré zo štvoríc tvorených výlučne číslami 2 a~3
dotyčnej rovnici vyhovujú.
Jedná sa o~práve päť rôznych (neusporiadaných) štvoríc
$(2,2,2,2)$, $(2,2,2,3)$, $(2,2,3,3)$, $(2,3,3,3)$, $(3,3,3,3)$.
Dosadením sa ľahko presvedčíme, že vyhovuje jedine štvorica
$(2,2,2,3)$. Doplnením o úvodnú trojicu $(3,5,7)$ dostaneme
(jediné) riešenie časti~b) úlohy, ktorým je sedmica prvočísel
$2,2,2,3,3,5,7$.

\návody
Vyriešte súťažnú úlohu pre prípad troch prvočísel.
[Riešenie neexistuje. Súčin vyhovujúcich troch prvočísel je násobok čísla
$105=3\cdot 5 \cdot 7$, takže tieto prvočísla musia byť $3$, $5$ a~$7$.
Pritom však $3\cdot 5\cdot7\ne105\cdot(3+5+7)$.]

Vyriešte súťažnú úlohu pre prípad štyroch prvočísel.
[Riešenie neexistuje. Podobne ako v~N1 tri z~prvočísel musia byť $3$, $5$, $7$
a pre štvrté prvočíslo~$p$ má platiť rovnica
$3\cdot5\cdot 7\cdot p=105\cdot(3+5+4+p)$, ktorú zjavne žiadne prvočíslo~$p$
nespĺňa.]

Nájdite všetky trojice prvočísel také, že ich
súčin je sedemnásobkom ich súčtu.
[Úloha má jediné riešenie, trojicu 3, 5, 7. Podobne ako v~N1 dokážte,
že jedno z~prvočísel je~7. Pre zvyšné dve, označme ich $p$ a $q$,
má platiť $7pq=7(p+q+7)$,
čo upravíme postupne na $pq=p+q+7$, ďalej na $p(q-1)=(q-1)+8$
a napokon na $(p-1)(q-1)=8$. Pri označení takom, že $p \ge q$,
čiže $p-1 \ge q-1$, máme možnosti $(p-1,q-1)=(8,1)$ alebo
$(p-1,q-1)=(4,2)$. Iba druhá vedie na dvojicu prvočísel.]

Nájdite všetky celé čísla $x$ a $y$, pre ktoré $3xy=5x+7y+1$.
[Štyri riešenia $(x,y)$: $(\m4,1)$, $(2,\m11)$, $(3,8)$, $(15,2)$.
Po vynásobení tromi dostaneme rovnicu $9xy=15x+21y+3$. Teraz už ju môžeme
upraviť na súčinový tvar $(3x-a)(3y-b)=3a+3b$ s~vhodnými celými
číslami $a$ a $b$, konkrétne $(3x-7)(3y-5)=38$. Ostáva rozobrať všetky
možnosti rozkladu čísla 38 na súčin dvoch celých čísel.]


Nájdite všetky prirodzené čísla $x$, $y$ a $z$, pre ktoré platí
$xyz=x+y+z+2$.
[Až na poradie tri riešenia: $(1,2,5)$, $(1,3,3)$, $(2,2,2)$.
Ak je nejaké z~čísel $x$, $y$, $z$ rovné~1, napríklad~$x$,
dostaneme rovnicu $yz=y+z+3$ s~riešeniami $(2,5)$ a $(3,3)$
(podľa postupu z~N4). Ak je naopak $\min(x,y,z)\geqq2$ a
napríklad $z=\max(x,y,z)$, platí
$x+y+z+2\le 3z+2$ a súčasne $xyz \ge 4z$. Z toho vyplýva
$4z\le 3z+2$, čiže $z \le 2$, takže je teda nutne
$x=y=z=2$, a to je naozaj riešenie.]

\D
Vyriešte súťažnú úlohu pre prípad šiestich prvočísel.
[Riešenie neexistuje. Podobne ako v~N1 usúdime, že zo šiestich
prvočísel tri sú 3, 5 a 7, ostatné,
ktoré označíme $x$, $y$ a $z$, spĺňajú rovnicu
$xyz=x+y+z+15$. Ak je niektoré z~prvočísel $x$,~$y$,~$z$ rovné~2,
postupom z~riešenia N4 zistíme, že rovnica nemá prvočíselné riešenie.
V~opačnom prípade, keď $\min(x,y,z)\geqq3$ a napríklad
$z=\max(x,y,z)$, máme $9z \leq xyz=x+y+z+15\leq 3z+15$, čiže
$z\leq 2{,}5$, a to je spor.]

Vyriešte súťažnú úlohu pre prípad $n\ge8$ prvočísel.
[Riešenie neexistuje pre žiadne $n\ge8$.
Podobne ako v~N1 sú tri prvočísla~3, 5 a 7, zvyšné
označme $p_1,\ldots,p_k$, pričom $k=n-3$, teda $k \ge 5$.
Platí $p_1 \cdots p_k = p_1+\cdots+p_k+15$. Ak je napríklad
$p_1=\max(p_1,\ldots,p_k)$, tak
$2^{k-1}p_1 \leq p_1 \cdots p_k = p_1+\ldots+p_k+15 \leq k~\cdot p_1+15$,
teda
${p_1(2^{k-1}-k)} \leq 15$. Indukciou sa však ľahko ukáže, že pre
každé $k \ge 5$ platí ${2^{k-1}-k} \ge 11$, odkiaľ
${p_1(2^{k-1}-k)} \ge 2 \cdot 11 = 22$, čo odporuje skôr odvodenej
nerovnosti.]

Nájdite všetky prvočísla $x$, $y$, $z$ spĺňajúce rovnicu
${x^2+y^2+z^2 =x+y+z+18}$.
[Jediné riešenie $x=y=z=3$. Nech najskôr niektoré z~prvočísel
$x$, $y$ a $z$ je rovné 2, napr.~$z$. Potom platí $x^2+y^2=x+y+16$.
Ľahko overíme, že nemôže byť ani $x=2$, a teda ani $y=2$,
a preto $\min(x,y)\geqq3$. Vtedy $3x+3y \leq x^2+y^2=x+y+16$,
takže $x+y\leq 8$, teda stačí otestovať dvojice $(x,y)$
rovné $(3,3)$ a~$(5,3)$, ktoré však nevedú k~riešeniu.
Prejdeme k~druhému prípadu, keď $\min(x,y,z)\geqq3$. Vtedy
$3x+3y+3z \leq x^2+y^2+z^2=x+y+z+18$, takže $x+y+z \leq 9$,
teda nutne $x=y=z=3$, čo je naozaj riešenie úlohy.]

Nájdite všetky prvočísla $x$, $y$, $z$ spĺňajúce rovnicu $xyz=xy+yz+zx+x+y+z+35$.
[Jediné riešenie $x=y=z=5$. Ak je nejaké z~prvočísel rovné~2 alebo~3,
tak dosadením dostaneme rovnice podobné ako v~N4,
o~ktorých sa rovnakým postupom presvedčíme, že nemajú žiadne
prvočíselné riešenie. Predpokladajme preto ďalej, že
$x\geqq y\geqq z\geqq 5$. Potom
$$
5xy\leqq xyz=xy+yz+zx+x+y+z+35\leqq 3xy+3x+35,
$$
odkiaľ $x(2y-3)\leqq35$. Zároveň však z~$x\geqq5$ a
$2y-3\geqq2\cdot5-3=7$ vyplýva opačná nerovnosť $x(2y-3)\geqq35$,
takže nutne $x=y=5$, a preto tiež $z=5$, teda
jediné možné riešenie je $x=y=z=5$.\looseness=0]
\endnávod
}

{%%%%%   A-I-2
Zadaná rovnosť $|AD|=|CD|$ znamená, že bod $D$ je priesečník
strany~$BC$ s~osou $o_{AC}$ strany $AC$. Podobne vďaka
rovnosti $|AE|=|BE|$ je bod $E$ priesečník strany~$BC$
s~osou $o_{AB}$ strany $AB$. Priesečník $O$ týchto dvoch osí, ktorý
označíme $O$ (\obr), je stredom kružnice opísanej trojuholníku
$ABC$. Dodajme, že podľa zadania platí $D\ne E$, a~tak je bod
$O$ rôzny od bodov $D$ a $E$. Preto môžeme ďalej vravieť o~osiach
$o_{AC}$, $o_{AB}$ ako o~priamkach $DO$, resp. $EO$.
\inspsc{a70i_p.1}{0.8333}%

Zamerajme sa teraz na bod $F$. Tento je určený podmienkami
rovnobežnosti $FD \parallel AB$ a~$FE \parallel AC$
($F$ tak zrejme leží vnútri $\triangle ABC$, a teda
existuje $\triangle DEF$). Vzhľadom na $AB\perp EO$ a $AC\perp DO$
dostávame kolmosti vyznačené na obrázku: $FD\perp EO$ a~$FE\perp DO$.
Vyplýva z~nich, že bod~$O$ je priesečníkom dvoch výšok trojuholníka $DEF$.
Jeho tretia výška z~vrcholu $F$ teda leží na tej istej kolmici
na~priamku $BC$ ako stred $O$ opísanej kružnice. Inak povedané,
bod $F$ leží na osi strany $BC$, a preto platí rovnosť $|FB|=|FC|$,
ktorú sme mali dokázať.

\návody
Uvedomte si, ako sa dokazuje školský poznatok,
že osi strán ľubovoľného trojuholníka $ABC$ sa pretínajú
v~jednom bode.
[Označme~$O$ priesečník osí strán $AB$ a~$AC$.
Podľa vlastnosti osí úsečiek nutne platí
$|OA|=|OB|$ a~$|OA|=|OC|$, takže $|OB|=|OC|$, čo naopak znamená,
že bod $O$ leží na osi strany~$BC$.]

Dokážete použitím poznatku o~osiach strán z~úlohy N1 odvodiť
iný školský poznatok, že aj výšky ľubovoľného trojuholníka $ABC$
sa (ako priamky) pretínajú v~jednom bode?
[Trojuholník $ABC$ doplňte trikrát na rovnobežník $ABA'C$,
$BCB'A$, resp. $CAC'B$. Potom body $A$, $B$, $C$ sú
stredy strán trojuholníka $A'B'C'$. Osi jeho strán
sa pretínajú v~jednom bode (podľa úlohy N1)
a~ležia na nich výšky pôvodného trojuholníka $ABC$.]

Uvedomte si nasledujúcu ekvivalentnú formuláciu tvrdenia
o~existencii priesečníka výšok ľubovoľného trojuholníka~$ABC$:
Ak pre nejaký bod~$H$ roviny trojuholníka $ABC$ platí
$HB\perp AC$ a~$HC\perp AB$, tak buď $H=A$, alebo $HA\perp BC$.
[Bod $H$ je priesečníkom dvoch výšok trojuholníka $ABC$ a každá
z~relácií $H=A$, $HA\perp BC$ znamená, že bod~$H$ leží aj
na tretej výške z~vrcholu $A$.]

\D
Dokážte implikáciu z~úlohy N3 metódou obvodových uhlov,
aspoň pre prípad ostrouhlého trojuholníka $ABC$.
[Nech teda $HB\perp AC$, $HC\perp AB$.
Označme~$A'$ priesečník $AH$ a~$BC$, $B'$ priesečník~
$BH$ a~$AC$, a~$C'$ priesečník~$CH$ a~$AB$.
Štvoruholníky $AC'HB'$ a~$BC'B'C$ sú tetivové vďaka
pravým uhlom $\uhel AB'H$, $\uhel HC'A$, $\uhel BC'C$, $\uhel BB'C$.
Preto $|\uhel BAA'|=|\uhel C'AH|=|\uhel C'B'H|=|\uhel
C'B'B|=|\uhel C'CB|$. Trojuholníky $BAA'$, $BCC'$ sa preto
zhodujú v dvoch vnútorných uhloch, takže sa zhodujú aj v~treťom
z~nich: $|\uhel AA'B|=|\uhel BC'C|=90^\circ$. Odtiaľ už $HA\perp
BC$.]

V~trojuholníku~$ABC$ platí $|AB|\ne|AC|$.
Nech $S$ je taký bod osi uhla~$BAC$, pre
ktorý platí $|SB|=|SC|$. Dokážte, že bod~$S$ leží na kružnici
opísanej trojuholníku $ABC$.
[Pracovať priamo so zadaným bodom $S$ je ťažké, a~tak zvážime
pomocný bod $S'$, ktorým je priesečník $S'\ne A$ osi uhla $BAC$ s~kružnicou
opísanou. Ako je dobre známe, platí $|S'B|=|S'C|$ (vyplýva to zo
zhodnosti obvodových uhlov $S'AB$ a $S'AC$). Vďaka podmienke
$|AB|\ne|AC|$ je spoločný bod~$S$ osi uhla $BAC$ a osi strany~$BC$
jediný, a~tak platí $S'=S$.]

V~trojuholníku $ABC$ so stredom~$I$ kružnice vpísanej platí
$|AB|<|AC|$. Nech~$D$ je bod strany~$AC$ taký,
že $|AB|=|AD|$. Dokážte, že body $B$, $C$, $D$, $I$ ležia na
jednej kružnici.
[Polpriamka $AI$ je os uhla $BAC$, a~teda aj os uhla $BAD$,
ktorá je vďaka podmienke $|AB|=|AD|$ zároveň aj osou úsečky $BD$.
Bod~$I$ je teda pre trojuholník $BCD$ takým bodom osi uhla $BCD$, pre
ktorý platí $|IB|=|ID|$. Keďže navyše $|CB|\ne|CD|$ (lebo
$|CD|=|AC|-|AD|=|AC|-|AB|<|CB|$ podľa trojuholníkovej nerovnosti),
je možné použiť výsledok úlohy D2, podľa ktorého leží
bod $I$ na kružnici opísanej trojuholníku $BCD$. Iné riešenie: Stačí ukázať,
že oba uhly $BIC$ a $BDC$ majú veľkosť $90\st+\frac12\al$, kde ako zvyčajne $\al=|\uhel BAC|$.]
\endnávod
}

{%%%%%   A-I-3
Keďže $a$, $b$, $c$ sú navzájom rôzne kladné čísla,
sú také aj čísla $ab$, $bc$, $ca$, lebo napríklad
z~$ab=bc$ vyplýva $a=c$ (vďaka $b\ne0$). Vidíme tak, že
v skúmanej sedmici čísel $a+b$, $b+c$, $c+a$, $ab$, $bc$, $ca$, $abc$
sú aspoň 3~rôzne hodnoty. Dokážeme najskôr sporom,
že práve 3~hodnoty to nikdy byť nemôžu. Potom uvedieme príklad
skúmanej sedmice, ktorá je zložená iba zo 4~rôznych hodnôt.

Najskôr teda pripusťme, že v~niektorej sedmici sú práve 3~rôzne hodnoty.
Vieme, že sú to hodnoty troch súčinov $ab$, $bc$, $ca$,
a tak súčin $abc$ sa musí rovnať jednému z~nich. Znamená to,
že jedno z~čísel $a$, $b$, $c$ je rovné~1, lebo napríklad
z~$abc=ab$ vyplýva $c=1$.

Bez ujmy na všeobecnosti sa ďalej obmedzíme na prípad $c=1$. Dotyčnú
sedmicu s~tromi rôznymi hodnotami potom môžeme zredukovať na šesticu
s~rovnakou vlastnosťou, ktorá je zložená z~čísel
$$
a+b,\ a+1,\ b+1,\ ab, \ a,\ b
$$
(dosadili sme $c=1$ a vynechali číslo $abc$ rovné $ab$).
Keďže už sú vylúčené rovnosti $a=1$ a $b=1$, tri rôzne
hodnoty sú zastúpené ako v~prvej trojici $a+b$, $a+1$, $b+1$,
tak aj v~druhej trojici $ab$, $a$, $b$.
Obe čísla $a$, $b$ z~druhej trojice preto musia ležať
v~množine $\{a+b,a+1,b+1\}$. To možno dosiahnuť jedine tak,
že platí $a=b+1$ a súčasne $b=a+1$, a to je nemožné.
Dôkaz sporom je ukončený.

Ako sme sľúbili, v~druhej časti riešenia uvedieme príklad skúmanej
sedmice, ktorá je zložená zo 4~rôznych hodnôt. Podľa predchádzajúcich
pozorovaní sa vyplatí preskúmať situáciu, keď platí povedzme $c=1$ a
zároveň $b=a+1$. Vtedy máme
$$
(a+b,b+c,c+a)=(2a+1,a+2,a+1)\quad\hbox{a}\quad
(ab,bc,ca,abc)=\bigl(a^2+a,a+1,a,a^2+a\bigr).
$$
Stačí teda nájsť také kladné číslo $a\ne1$, aby v~pätici čísel
$$
a,\ a+1,\ a+2,\ 2a+1,\ a^2+a
$$
boli iba štyri rôzne hodnoty. Vzhľadom na zrejmé nerovnosti,
ktoré pre uvažované~$a$ medzi týmito piatimi číslami platia, sa
na splnenie požiadavky ponúkajú práve dve možnosti. Sú
vyjadrené rovnicami
$$
a^2+a=2a+1,\quad\hbox{resp.}\quad a^2+a=a+2.
$$
Obe naozaj vedú k~vyhovujúcim trojiciam, ktoré sú tvaru
$$
(a,b,c)=\biggl(\frac{\sqrt5+1}2,\frac{\sqrt5+3}2,1\biggr),
\quad\hbox{resp.}\quad
(a,b,c)=\bigl(\sqrt2,\sqrt2+1,1\bigr).
$$

\zaver
Najmenší možný počet rôznych čísel v skúmanej sedmici je rovný~4.

\poznamka
Ak nebudeme rozlišovať trojice $(a,b,c)$, ktoré sa líšia iba
poradím svojich prvkov, existujú ďalšie dve prípustné trojice,
pre ktoré sa v skúmanej sedmici nájdu iba 4 rôzne hodnoty. Prvá z~nich
je trojica
$$
(a,b,c)=\biggl(a,\frac{a}{a-1},\frac{a}{(a-1)^2}\biggr),
$$
pričom $a>0$ je jediný reálny koreň kubickej rovnice
$a^3-4a^2+4a-2=0$. Druhou vyhovujúcou trojicou je
$$
(a,b,c)=\biggl(a, \frac{a}{a^2-a-1}, \frac{a(a-1)}{a^2-a-1}\biggr),
$$
pričom~$a$ je väčší z~dvoch kladných koreňov rovnice $a^4-2a^3-2a^2+2a+2=0$.%
\footnote{Prvá trojica je približne
$(2{,}8393;1{,}5437;0{,}8393)$, druhá
$(2{,}3322;1{,}1069;1{,}4746)$.}


\ineriesenie
Ešte jedným spôsobom dokážeme, že v skúmanej sedmici
čísel $a+b$, $b+c$, $c+a$, $ab$, $bc$, $ca$, $abc$ musia byť aspoň
4~rôzne hodnoty. Využijeme pritom všeobecne užitočný obrat: vzhľadom
na symetrické zastúpenie čísel $a$, $b$, $c$
ich môžeme vopred usporiadať podľa veľkosti.

Budeme teda predpokladať, že pre (kladné) čísla $a$, $b$, $c$ platí
$a<b<c$ (podľa zadania sú rôzne). Potom zrejme tiež platí
$$
a+b<a+c<b+c\quad\hbox{a}\quad ab<ac<bc.
\tag1
$$
Vidíme, že medzi spolu šiestimi číslami zapísanými v~\thetag1 sa
nájdu aspoň 4~rôzne hodnoty, ak nenastane prípad,
keď usporiadané trojice $(a+b,a+c,b+c)$ a $(ab,ac,bc)$ splynú,
\tj. bude splnená sústava rovníc
$$\eqalign{
a+b&=ab,\cr
a+c&=ac,\cr
b+c&=bc.}
$$
Ukážeme, že to nie je možné. Odčítaním druhej rovnice od prvej
dostaneme po jednoduchej úprave $(b-c)(1-a)=0$.
To vzhľadom na $b\ne c$ znamená $a=1$. Po dosadení
do prvej rovnice ale dostaneme rovnicu $1+b=b$, ktorá nemá riešenie.

Tým je na úvod sľúbený dôkaz ukončený. Za povšimnutie stojí, že sme
v~ňom vôbec nepotrebovali posledné číslo $abc$ zo sedmice
$a+b$, $b+c$, $c+a$, $ab$, $bc$, $ca$, $abc$. Ukázali sme totiž,
že aspoň 4~rôzne hodnoty sa vždy nájdu už medzi jej prvými
šiestimi číslami.

Dodajme, že práve opísaný postup možno využiť aj pri hľadaní (všetkých)
sedmíc zložených zo 4~rôznych hodnôt. Vyplýva z~neho totiž,
že takú sedmicu dostaneme práve vtedy, keď trojprvkové
množiny $\{a+b,a+c,b+c\}$ a $\{ab,ac,bc\}$ sa budú zhodovať
{\it v dvoch hodnotách\/} a keď potom hodnota $abc$ bude rovná
{\it jednej zo štyroch hodnôt\/} z~oboch množín.%
\footnote{Všimnite si, že tieto podmienky sú v premenných $a$,
$b$, $c$ symetrické.} Teda napríklad dve usporiadané trojice $(a,b,c)$
nájdené v~závere prvého riešenia sú postupne riešeniami sústav rovníc
$$
\eqalign{
a+b&=ab,\cr
a+c&=bc,\cr
abc&=ab,}
\qquad\hbox{resp.}\qquad
\eqalign{
b+c&=ab,\cr
a+c&=bc,\cr
abc&=ab.}
$$
Podobne usporiadané trojice $(a,b,c)$ uvedené v~poznámke za prvým
riešením sú postupne riešeniami sústav rovníc
$$
\eqalign{
a+b&=ab,\cr
b+c&=ac,\cr
abc&=a+c,}
\qquad\hbox{resp.}\qquad
\eqalign{
a+b&=ac,\cr
b+c&=ab,\cr
abc&=a+c.}
$$

\návody
\titem
Vo všetkých úlohách budú $a$, $b$, $c$ navzájom
rôzne kladné reálne čísla.

Určte najmenší možný počet rôznych čísel medzi
číslami $a+b$, $b+c$, $c+a$, $a+b+c$.
[Jedná sa vždy o~štyri rôzne čísla. Určite možno predpokladať, že
$0<a<b<c$. Potom ale $a+b<a+c<b+c<a+b+c$.]

Určte najmenší možný počet rôznych čísel medzi
číslami $ab$, $bc$, $ca$, $abc$.
[Tri. Určite možno predpokladať, že
$0<a<b<c$. Potom ale $ab<ac<bc$, takže aspoň tri rôzne hodnoty
existujú vždy. Práve tri rôzne hodnoty to budú
práve vtedy, keď číslo $abc$ bude rovné jednému z~čísel $ab$, $ac$,
$bc$, \tj. práve keď bude $1\in\{a,b,c\}$.]

Určte najmenší počet rôznych čísel medzi číslami $a+1$, $b+1$,
$a$, $b$, $ab$. Je tento počet možný v~prípade, keď navyše
sú čísla $a$ a $b$ rôzne od 1?
[Tri a možný počet to je aj v~prípade $a\ne1\ne b$ .
Vzhľadom na symetriu zadania v~premenných $a$ a $b$ môžeme
predpokladať, že $a<b$. Keďže $b<b+1$, sú $a$, $b$, $b+1$
tri rôzne hodnoty. Aby boli práve tri v~celej pätici, musí platiť
$a+1=b$ a v~prípade $a\ne1\ne b$ ešte musí byť
$ab=b+1$. Obom rovnostiam vyhovujú čísla $a=\sqrt2$ a
$b=\sqrt2+1$, ktoré sú rôzne od 1 a pre ktoré sú v~pätici
naozaj tri rôzne hodnoty.]

Určte najmenší možný počet rôznych čísel medzi
číslami $ab$, $ac$, $a+b$, $a+c$.
[Tri. Keďže $ab \ne ac$ a aj $a+b\ne a+c$, máme aspoň 2
rôzne hodnoty. Pripusťme, že máme práve 2 rôzne hodnoty
v~celej štvorici. Potom máme dve možnosti: buď $ab=a+b$ a $ac=a+c$,
alebo $ab=a+c$ a $ac=a+b$. V~prvom prípade po odčítaní oboch rovností
dostaneme $(a-1)(b-c)=0$, odkiaľ $a=1$, a to je spor s~$ab=a+b$.
V~druhom prípade po podobnom odčítaním dostaneme $(a+1)(b-c)=0$,
a to je tiež spor. Preto vždy máme aspoň~3 rôzne hodnoty, a
tento počet neprekročíme, ak bude platiť $ab=a+b$, čo spĺňa
napríklad trojica $(a,b,c)=(3,\frac32,1)$. Dodajme, že postup
sme mohli zjednodušiť využitím symetrie zadania v~premenných
$b$ a $c$, vďaka ktorej môžeme predpokladať, že $b<c$. Vtedy platí
$ab<ac$ a $a+b<a+c$, takže stačí rozobrať iba prvý z dvoch vyššie
rozlíšených prípadov.]

\D
Určte najmenší možný počet rôznych čísel medzi
číslami $a+2b$, $b+2c$, $c+2a$.
[Dve. Úloha sa nezmení, keď trojicu $(a,b,c)$ zameníme ľubovoľnou
z~trojíc $(b,c,a)$ a $(c,a,b)$. Preto môžeme predpokladať,
že platí $a=\max\{a,b,c\}$. Potom $2a>b+c$, čiže $c+2a>b+2c$.
Tým pádom v~našej trojici máme aspoň dve rôzne hodnoty. Pre
nájdenie trojice s~práve dvoma rôznymi hodnotami
položme napríklad $c+2a=a+2b$. To dáva $a=2b-c$. Teda napríklad
pre $c=1$, $b=2$ vyjde $a=3$. Vtedy $(a+2b,b+2c,c+2a)=(7,4,7)$.]

Určte najmenší možný počet rôznych čísel medzi
číslami $a+b$, $b+c$, $c+a$, $ab+1$, $bc+1$, $ca+1$, $abc$.
[Štyri. Vďaka symetrii môžeme predpokladať, že $a>b>c$.
Potom $a+b>a+c>b+c$ a~$ab+1>ac+1>bc+1$, takže v~zadanej sedmici
sú vždy aspoň tri rôzne hodnoty. Keby boli práve tri, tak by
nutne platilo $a+b=ab+1$, $a+c=ac+1$ a~$b+c=bc+1$. To upravíme na
$(a-1)(b-1)=0$, $(a-1)(c-1)=0$ a $(b-1)(c-1)=0$. Nutne sa teda
dve z~čísel~$a$, $b$, $c$ rovnajú~1, a to je spor. V~našej sedmici
teda máme vždy aspoň~4 rôzne hodnoty, a tento počet neprekročíme, keď
napríklad zvolíme $a=1$ a budeme požadovať, aby platilo $bc=b+c$,
čo napríklad spĺňa dvojica $(b,c)=\left(3,\frac32\right)$.]

Určte najmenší možný počet rôznych čísel medzi
číslami
$$
\sqrt{\frac{a^2+b^2+c^2}{3}},\quad\frac{a+b+c}{3},\quad \root
3\of{abc},\quad \frac{3}{\frac1a + \frac1b+\frac1c}.
$$
[Štyri. Jednotlivé čísla sú rôzne druhy priemerov zostavené
pre tú istú trojicu čísel $a$, $b$, $c$. Označme zľava doprava ich hodnoty
$K$, $A$, $G$, $H$ podľa ich názvov {\it kvadratický}, resp.
{\it aritmetický}, resp. {\it geometrický}, resp. {\it harmonický} priemer.
Ako je známe, medzi týmito priemermi pre ľubovoľnú trojicu
kladných čísel $a$, $b$, $c$ platia nerovnosti $K \ge A~\ge G \ge H$,
pričom všetky nerovnosti sú ostré s~výnimkou prípadu, keď
platí $a=b=c$. (Dôkazy týchto nerovností možno nájsť v~brožúre
{\it A. Kufner: Nerovnosti a odhady}, dostupnej na
\pdfklink{www.dml.cz/handle/10338.dmlcz/403877}{https://www.dml.cz/handle/10338.dmlcz/403877}.)]

\endnávod
}

{%%%%%   A-I-4
Najskôr sa s~novým pojmom {\it superdeliteľ\/} bližšie zoznámime.
O~číslach a ich deliteľoch budeme v~celom riešení
predpokladať, že to sú celé kladné, \tj. prirodzené čísla.

Majme dané číslo $n>1$. Číslo $a$ je jeho deliteľ
práve vtedy, keď platí $n=ab$ pre vhodné číslo $b$,
ktoré potom je teda tiež deliteľom čísla $n$ (môže platiť
aj $a=b$). Určite pre také čísla $a$, $b$
(združené podmienkou $n=ab$) platí: čím väčšie je $a$,
tým menšie je~$b$. Keďže špeciálne
pre $a=n$ je $b=1$, najväčšiemu deliteľovi $a$
s~vlastnosťou $a<n$, \tj. superdeliteľovi $d$ čísla $n$,
bude zodpovedať najmenší deliteľ $b$ s~vlastnosťou $b>1$ --
a tým je určite {\it najmenší
prvočiniteľ}\footnote{Pripomeňme, že termín {\it
prvočiniteľ\/} znamená {\it prvočíselný deliteľ}.}~$p$ daného čísla $n$. S~jeho superdeliteľom $d$ je teda toto
prvočíslo $p$ zviazané rovnosťou $pd=n$. Na určenie superdeliteľa daného čísla
tak stačí nájsť jeho najmenší prvočiniteľ a tým potom dané číslo vydeliť.
Napríklad superdeliteľom každého párneho čísla $2k$ je číslo $2k:2=k$.

Teraz už sme pripravení posúdiť otázku, ako vyzerajú všetky
čísla $n$, ktorých superdeliteľom je dané číslo $d>1$. Podľa
predchádzajúceho výkladu to budú práve tie~$n$, ktoré sú tvaru
$n=pd$, pričom prvočíslo $p$ je volené tak, aby bolo
najmenším prvočiniteľom vzniknutého čísla~$n$, teda čísla~$pd$.
Čo podmienka \uv{$p$ je najmenším prvočiniteľom čísla $pd$} znamená?
Zrejme práve to, že prvočíslo $p$ neprevyšuje žiadneho
z~prvočiniteľov daného čísla~$d$. Vďaka
predpokladu $d>1$ nie je množina prvočiniteľov čísla $d$
prázdna,
a tak odvodenú podmienku spĺňa iba niekoľko prvých najmenších
prvočísel $p$ (všetky až po najväčší prvočiniteľ daného $d$
vrátane).\footnote{V prípade $d=1$ vyhovuje každé prvočíslo $p$,
a preto platí: Superdelitele 1 majú práve tie čísla,
ktoré sú prvočísla. Tých je nekonečne veľa, a tak bez
podmienky $d>1$ tvrdenie a) neplatí.}
Tým je časť a) úlohy vyriešená.

Podľa predchádzajúceho odseku pre súčet $s(d)$ všetkých čísel s~daným
superdeliteľom~$d$ platí vzorec
$$
s(d)=2d+3d+\cdots+p_ld=
(p_1+p_2+\cdots+p_l)d,
$$
pričom $2\!=\!p_1\!<\!p_2\!=\!3\!<\!p_3\!=\!5\!<\!\ldots\!<\!p_l$
je skupina prvých~$l$ prvočísel končiaca
najmenším prvočiniteľom $p_l$ daného čísla $d>1$. Úlohou časti
b) úlohy je preto rozhodnúť, či existuje nejaké nepárne číslo $d>1$,
pre ktoré platí relácia
$$
2\,020\mid(p_1+p_2+\cdots+p_l)d.
\tag1
$$
Existenciu takého čísla $d$ potvrdíme príkladom, ktorý nájdeme,
keď budeme postupne odvodzovať, aké vlastnosti musí každé
vyhovujúce číslo $d$ všeobecne mať.\footnote{Bez podmienky, že číslo
$d>1$ je nepárne, by taká úloha bola triviálna:
pre párne číslo $d=1\,010$ je $p_l=2$, a tak platí $s(d)=2d=2\,020$.}

Keďže nepárne číslo $d$ je nesúdeliteľné s~číslom 4,
ktoré je deliteľom čísla 2\,020, vyplýva z~\thetag1 relácia
$$
4\mid p_1+p_2+\cdots+p_l.
$$
Z toho vyplýva $l\geqq4$ (čísla $2$, $2+3=5$ a $2+3+5=10$ totiž
nie sú číslom 4 deliteľné). Pre najmenší prvočiniteľ $p_l$ čísla $d$
tak platí $p_l\geqq p_4=7$, a preto $5\nmid d$. Nepárne číslo~$d$ je
teda nesúdeliteľné s~deliteľom 20 čísla 2\,020. Preto z~\thetag1 vyplýva relácia
$$
20\mid p_1+p_2+\cdots+p_l.
\tag2
$$
Nebudeme tu vypisovať postupné dosadzovanie hodnôt $l=3,4,\dots$,
ktoré vedie k~zisteniu, že najmenšie číslo $l$ spĺňajúce reláciu \thetag2 je
$l=9$:
$$
p_1+p_2+\cdots+p_9=2+3+5+7+11+13+17+19+23=100.
\tag3
$$
Pokúsme sa už teraz nájsť vyhovujúce číslo $d$ medzi číslami
s~najmenším prvočiniteľom $p_9=23$. Po dosadení súčtu
\thetag3 do \thetag1 zistíme, že také čísla $d$ majú spĺňať podmienku
$$
2\,020\mid 100d,\quad\hbox{čiže (po krátení číslom 20)}\quad 101\mid 5d.
$$
Keďže 101 je prvočíslo, posledná relácia bude platiť
práve vtedy, keď číslo $d$ bude mať okrem (najmenšieho) prvočiniteľa $23$
aj prvočiniteľa 101. Za vyhovujúce číslo $d$ preto môžeme
zvoliť $d=23\cdot101=2\,323$. Tým je riešenie časti b) úlohy
ukončené.

\poznamka
Použitím počítača možno zistiť, že podmienku
$2\,020\mid s(d)$ spĺňajú aj niektoré (veľké) prvočísla $d$.
Okomentujme výsledok, že najmenšie z~nich je $d=10\,663$.
Z~nášho riešenia vyplýva, že pre každé prvočíslo $d$ platí vzorec
$$
s(d) = (2+3+5+\dots+d)\cdot d,
$$
v ktorom sa sčítajú všetky prvočísla od 2 do $d$ vrátane.
Pre $d=10\,663$ je tento súčet rovný
$6\,480\,160=3\,208\cdot2\,020$, a tak je hodnota
$s(10\,663)$ naozaj deliteľná číslom 2\,020.

Dodajme ešte, že reláciu
$2\,020\mid s(d)$ nespĺňa \uv{nádejné} prvočíslo $d=101$,
lebo súčet všetkých prvočísel od 2 do 101 je rovný číslu $1\,161$,
ktoré nie je deliteľné číslom~20, ako by sme podľa súčinu zo
vzorca pre $s(d)$ potrebovali.

\návody
\titem
Pripomeňme, že prvočiniteľom čísla $n$ nazývame každé
prvočíslo, ktoré číslo~$n$ delí.

Uvedomte si, že superdeliteľom daného čísla~$n>1$ je číslo $\frac np$,
pričom~$p$ je najmenší prvočiniteľ čísla~$n$.
[Ak je $d$ deliteľ čísla $n$, je jeho deliteľom aj číslo
$\frac nd$. Najmenšie dva delitele~$n$ sú 1~a~$p$, ktorým zodpovedajú
dva jeho najväčšie delitele~$\frac n1$ a~$\frac np$.]

Určte všetky prirodzená čísla, ktorých superdeliteľom je číslo~2.
[Jedine číslo $4$. Každé hľadané číslo $n$ je nutne párne, a tak
je číslo 2 jeho najmenší prvočiniteľ. Podľa výsledku~N1 teda
platí $2=\frac n2$, a preto $n=4$.]


Určte všetky prirodzené čísla, ktorých superdeliteľom je číslo~7.
[14, 21, 35 a 49. Každé hľadané $n$ je deliteľné siedmimi a podľa
výsledku N1 má platiť $7=\frac np$, pričom $p$ je najmenší
prvočiniteľ $n$, takže $p\leqq7$, čiže $p\in\{2,3,5,7\}$.]

\D
Nájdite všetky prirodzené čísla $n>1$ také, že
keď k~nim pripočítame ich superdeliteľa, dostaneme súčet~2\,020.
[$1\,515$ a $1\,919$. Ak je~$p$ najmenší prvočiniteľ čísla $n$, tak $n=dp$,
pričom~$d$ je superdeliteľ~$n$. Má platiť $dp+d=2020$, čiže
$d(p+1)=2020$, takže ostáva prebrať všetky delitele $d$ čísla
$2\,020$ s~prvočíselným rozkladom $2^2\cdot5\cdot101$.
Pre $d=1$ vychádza $p=2019$, čo nie je prvočíslo.
Keby bolo $d>1$ párne, bolo by párne aj číslo $n$, a tak
by bolo $p=2$, a teda $d(2+1)=2020$, čo nie je možné.
Ostávajú preto možnosti $d\in\{5,101,505\}$. Postupným dosadením
do $d(p+1)=2020$ zistíme, že riešenie dáva iba hodnota $d=101$,
ktorej zodpovedá $p=19$, a hodnota $d=505$, pre ktorú vychádza $p=3$.
(V~oboch prípadoch je naozaj nájdené~$p$ najmenším
prvočiniteľom súčinu $n=dp$.)]

Ktoré z~čísel $2,3,\ldots,20$ je superdeliteľom najväčšieho
počtu čísel?
[Číslo 19. Hľadáme to číslo~$n\in\{2,3,\ldots,20\}$, pre ktoré existuje čo
najviac prvočísel~$p$ takých,
že najmenší prvočiniteľ čísla $np$ je práve~$p$.
Keďže $n>1$, musí pre každé prvočíslo~$p$
s~uvedenou vlastnosťou platiť $p\leqq n$, a teda aj $p\leqq 20$,
takže takých prvočísel nemôže byť viac, ako je všetkých
prvočísel do~20, ktorých je 8. Aby ich bolo práve 8,
musel by aj súčin $19n$ mať najmenšieho prvočiniteľa 19, čo
z~uvažovaných čísel $n$ spĺňa jedine $n=19$. Toto číslo je
naozaj superdeliteľom ôsmich čísel $19\cdot 2$, $19\cdot 3$, $19\cdot
5$,\dots, $19\cdot19$.]

Ktoré prirodzené číslo najbližšie k~číslu~2\,020 má
svojho superdeliteľa medzi číslami $1,2,3,\ldots,45$?
[Číslo 2\,017. Číslo~1 je superdeliteľom každého prvočísla.
Najbližšie prvočíslo k~číslu~2\,020 je~2017. Výpočtom superdeliteľov
čísel $2\,018,2\,019,\ldots,2\,023$ zistíme, že žiadny
z~nich nepatrí medzi čísla zo zadania.]

Ktoré prirodzené číslo najbližšie k~číslu~2\,020 má
svojho superdeliteľa medzi číslami $2,3,\ldots,45$?
[Číslo 1\,849.
Najväčším číslom $n$ s~daným superdeliteľom
$d>1$ je číslo $n=dp$, pričom $p$ je najmenší prvočiniteľ čísla $d$.
Z toho vyplýva, že ak $1<d\leqq43$, pre každé číslo $n$ so
superdeliteľom $d$ platí $n\leqq d^2\leqq43^2=1\,849$, pritom číslo
1\,849 má za superdeliteľa prvočíslo 43. Ďalej najväčšie číslo so
superdeliteľom 44 je rovné $44\cdot2=88$, najväčšie číslo so
superdeliteľom 45 je rovné $45\cdot3=135$.]
\endnávod
}

{%%%%%   A-I-5
\let\la=\lambda
\let\ro=\varrho
\def\ve#1{\overrightarrow{#1}}
Náš postup založíme na využití tzv. Apollóniových kružníc. Uveďme
preto o~nich najskôr základné poučenie. Dôkazy uvedených
poznatkov aj niektoré využitie týchto kružníc sú vyložené v~časti
I~kapitoly 5 brožúry {\it S. Horák: Kružnice}, dostupnej na
\pdfklink{dml.cz/handle/10338.dmlcz/403589}{https://dml.cz/handle/10338.dmlcz/403589}.

Majme dané kladné reálne číslo $\la\ne1$ a dva rôzne body $P$ a
$Q$ v~rovine $\ro$. Potom platí, že množinou všetkých bodov $X\in\ro$,
$X\ne Q$, ktoré vyhovujú rovnici
$$
\frac{|PX|}{|QX|}=\la,
\tag1
$$
je istá kružnica. Hovoríme jej Apollóniova a my teraz
tiež opíšeme jej konštrukciu. Obmedzíme sa pritom na prípad $\la>1$, lebo
v~prípade $\la<1$ možno prehodením bodov $P$ a~$Q$ zmeniť
parameter~$\la$ rovnice~\thetag1 na hodnotu $1/\la>1$.

Konštrukciu Apollóniovej kružnice \thetag1 zahájime tak, že najskôr
určíme jej priesečníky s~priamkou $PQ$. Budú nimi jeden vnútorný
bod~$R$ úsečky $PQ$ a
jeden vnútorný bod~$S$ polpriamky opačnej k~polpriamke
$QP$.\footnote{Taká poloha bodu $S$ zodpovedá nášmu predpokladu
$\la>1$.} Zopakujme, že takto lokalizované body $R$ a $Q$
sú jednoznačne určené rovnosťami
$$
\frac{|PR|}{|QR|}=\frac{|PS|}{|QS|}=\la
$$
(pozri \obr{} pre hodnotu $\la=2$).
\inspsc{a70i_p.2}{0.8333}%
Potom platí, že Apollóniovou kružnicou \thetag1 je kružnica nad priemerom $RS$.

S ohľadom na súťažnú úlohu, ktorú ešte len začneme riešiť,
je na \obrr1{} vyfarbený kruh, ktorý Apollóniova
kružnica \thetag1 ohraničuje. Ukážeme, že vnútri tohto kruhu
za nášho predpokladu $\la>1$ leží každý bod $X'\in\ro$,
pre ktorý platí
$$
\frac{|PX'|}{|QX'|}>\la.
\tag2
$$
Naozaj, každý taký bod $X'$ bude spĺňať rovnicu tvaru~\thetag1,
v ktorej hodnotu~$\la$ zameníme za väčšiu hodnotu $\la'$ rovnú zlomku
${|PX'|}/{|QX'|}$. Bod $X'$ potom leží na novej Apollóniovej kružnici
pre určený parameter $\la'$, ktorá je vykreslená na \obrr1.
Je to kružnica nad priemerom $R'S'$, pritom vďaka nerovnosti
$\la'>\la$ sa ľahko vysvetlí, že body $R'$ a~$S'$ ležia postupne
vnútri úsečiek $RQ$ a $QS$. Preto nová kružnica pre parameter~$\la'$
leží vnútri kruhu ohraničeného pôvodnou kružnicou pre parameter
$\la$. Tým je sľúbený dôkaz ukončený.%
\footnote{Aj keď to nebudeme
ďalej potrebovať, dodajme, že aj naopak
{\it každý\/} bod $X'$ rôzny od $Q$, ktorý leží vnútri
dotyčného kruhu, spĺňa nerovnicu \thetag2. Naozaj, keby pre číslo
$\la'=|PX'|/|QX'|$ platilo $\la'<\la$, tak vďaka tomu, že zrejme
platí $\la'>1$,
by sme mohli zopakovať úvahu z tohto odseku s~dvojicou $(\la,\la')$
zamenenou za $(\la',\la)$ a dôjsť tak ku sporu.}

Po prevedenej príprave už môžeme prejsť k~vlastnej súťažnej úlohe
a podať jej vcelku krátke riešenie. Tvrdenie úlohy dokážeme sporom.
Budeme teda predpokladať, že zadaná nerovnosť neplatí. Potom pre
niektorý bod $X$ sú všetky tri podiely
$$
\frac{|XA|}{|XS_a|},\ \frac{|XB|}{|XS_b|},\ \frac{|XC|}{|XS_c|}
$$
väčšie ako 2. Podľa vyloženej teórie to znamená, že bod $X$ leží
vnútri troch kruhov, ktoré sú ohraničené Apollóniovými kružnicami
s~rovnicami
$$
\frac{|XA|}{|XS_a|}=2,\ \frac{|XB|}{|XS_b|}=2,\
\frac{|XC|}{|XS_c|}=2.
$$
Všeobecná poučka o~priemere Apollóniových kružníc pre naše tri
kružnice s~parametrom $\la=2$ znamená, že ich priemery sú
úsečky $GA'$, $GB'$ a $GC'$, pričom bod $G$ je ťažisko trojuholníka $ABC$ a~body $A'$, $B'$, $C'$ sú postupne
obrazy bodov $A$, $B$, $C$ v stredových súmernostiach podľa
$S_a$, $S_b$, $S_c$. Stredy týchto troch kružníc označíme postupne
$O_a$, $O_b$, $O_c$.
\insp{a70i_pictures.3}%

Ako obrázok napovedá, žiadny bod $X$ nemôže ležať vnútri všetkých
troch kruhov, ktoré zostrojené kružnice určujú. Aby sme túto
hypotézu dokázali (a~tak náš dôkaz sporom zavŕšili), uvedomíme si
niekoľko skutočností: bod $G$ je spoločným bodom všetkých troch kružníc a
leží vnútri trojuholníka s~vrcholmi v stredoch $O_a$, $O_b$ a $O_c$,
ktorý je na \obr{} vyfarbený. Vyplýva to z~toho, že tento trojuholník je
zrejme obrazom trojuholníka $ABC$ v stredovej súmernosti
podľa jeho ťažiska $G$, a tak oba trojuholníky majú toto ťažisko spoločné.
Konvexné uhly $O_a G O_b$, $O_a G O_c$ a $O_b G O_c$ pokrývajú celý trojuholník $O_aO_bO_c$ a~ich zjednotením je plný uhol. Teda aspoň dva z~nich musia byť tupé.\footnote{Keby dva uhly boli ostré, prípadne jeden z~nich pravý (oba pravé byť nemôžu, pretože $G$ je ťažisko, teda vnútorný bod trojuholníka $O_aO_bO_c$), zvyšný uhol by nebol konvexný.}
\inspsc{a70i_4.2}{0.8333}%

Predpokladajme, že uhly $O_a G O_b$, $O_a G O_c$ sú tupé (pozri \obrr1{} a~\obr{} pre ostrouhlý a~tupouhlý trojuholník, pre zvyšné možnosti sú úvahy obdobné). Potom prienik kruhov so stredmi $O_a$ a $O_b$ zrejme leží v~uhle $O_a G O_b$\fnote{V~tomto tupom uhle totiž leží prienik dvoch polrovín, ktoré sú určené dotyčnicami k~hraničným kružniciam v~spoločnom bode~$G$, v~ktorých dané dva kruhy po jednom ležia.} a~prienik kruhov so stredmi $O_a$ a $O_c$ leží v~uhle $O_a G O_c$. Prienik všetkých troch kruhov tak leží v~prieniku oboch spomenutých uhlov, teda na polpriamke $GO_a$. Avšak kruh so stredom v~bode $O_b$ pretína vďaka tupému uhlu~$O_a G O_b$ polpriamku $GO_a$ iba v~bode $G$. Teda prienik všetkých troch kruhov obsahuje jediný bod, a to bod $G$. Tým je naša hypotéza dokázaná a riešenie je tak dokončené.

%Situáciu, ktorú ostáva riešiť, opíšme všeobecne: Majme
%tri kružnice $k$, $l$, $m$ postupne sa stredmi $K$, $L$, $M$, ktoré
%majú spoločný bod $G$ vnútri trojuholníka $KLM$ (\obr).
%\insp{a70i_pictures.4}%
%Kruhy, ktoré kružnice $k$, $l$, $m$ ohraničujú, majú za prieniky
%po dvoch útvary, ktoré sú na obrázku vyfarbené. Každý z~týchto
%troch útvarov zrejme leží v~polrovine s~hraničnou priamkou, ktorá
%prechádza bodom $G$ rovnobežne s~príslušnou spojnicou stredov oboch
%kružníc a ktorá je na obrázku vykreslená rovnakou farbou ako
%daný útvar. Je jasné, že bod $G$ je jediný spoločný bod týchto
%troch polrovín\footnote{Z toho, ktoré z bodov $K$, $L$, $M$
%v jednotlivých troch polrovinách ležia, totiž vyplýva, že
%prienik dvoch polrovín je istý uhol, ktorý leží
%v~polrovine opačnej k~tretej z nich.},
%a teda jediný spoločný bod uvažovaných troch kruhov.
%Tým je celé riešenie ukončené.

\ineriesenie
Vyložíme iný postup, ktorý nevyužíva
Apollóniove kružnice. Namiesto nich budeme potrebovať pomocné
tvrdenie, ktoré teraz sformulujeme a vzápätí dokážeme:
{\sl Nech $G$ je ťažisko trojuholníka $ABC$ a~nech~$X$ je ľubovoľný bod
polroviny~$p_a$, ktorá obsahuje úsečku $GA$ a
ktorej hraničná priamka $k_a$ je kolmica
na túto úsečku vedená bodom~$G$. Potom platí nerovnosť
$$
\frac{|XA|}{|XS_a|}\leq 2.
\tag3
$$
}
Na dôkaz tohto tvrdenia využijeme bod $X'$, ktorý je kolmým priemetom bodu~$X$ na
polpriamku $GA$ (\obr).
\insp{a70i_pictures.5}%
Ak bod~$X'$ leží na úsečke~$AG$, spĺňa nerovnosť
$$
\frac{|X'A|}{|X'S_a|}\leq\frac{|GA|}{|GS_a|}=2.
\tag4
$$
V~opačnom prípade je $X'$ vnútorný bod polpriamky opačnej
k~polpriamke~$AG$. Potom ale platí $|X'A|<|X'S_a|$, a tak je ľavá
strana nerovnosti \thetag4 dokonca menšia ako~1. Nerovnosť \thetag4 preto
platí v~oboch prípadoch.

Použitím Pytagorovej vety a nerovnosti~\thetag4 upravenej na tvar
$|X'A|^2\leqq 4|X'S_a|^2$ dostaneme
$$
\frac{|XA|^2}{|XS_a|^2}=\frac{|XX'|^2+|X'A|^2}{|XX'|^2+|X'S_a|^2}
\leq
\frac{|XX'|^2 + 4|X'S_a|^2}{|XX'|^2 + |X'S_a|^2}\leq
\frac{4|XX'|^2 + 4|X'S_a|^2}{|XX'|^2 + |X'S_a|^2}=4.
$$
Po odmocnení krajných výrazov už dostaneme nerovnosť \thetag3.

Ak porovnáme nerovnosť \thetag3 z~práve dokázaného tvrdenia
s~nerovnosťou zo zadania súťažnej úlohy, vidíme, že pre jej nové
riešenie stačí dokázať, že každý bod roviny trojuholníka $ABC$ leží
v~aspoň jednej z~polrovín $p_a$, $p_b$, $p_c$, pričom $p_b$ a $p_c$
sú zrejmé analógie polroviny $p_a$. Využijeme to, že
polroviny $p_a$, $p_b$, $p_c$ sú tvorené práve tými bodmi~$X$,
ktoré postupne spĺňajú nerovnosti so skalárnymi súčinmi vektorov
$$
\ve{GA}\cdot\ve{GX}\geqq0,\quad
\ve{GB}\cdot\ve{GX}\geqq0,\quad \hbox{resp.}\quad
\ve{GC}\cdot\ve{GX}\geqq0.
$$
Našou úlohou je ukázať, že pre každý bod $X$ je aspoň jedna z~týchto troch nerovností splnená. Vyplýva to však okamžite
z toho, že súčet ich ľavých strán je rovný nule:
$$
\ve{GA}\cdot\ve{GX}+\ve{GB}\cdot\ve{GX}+\ve{GC}\cdot\ve{GX}=
\bigl(\ve{GA}+\ve{GB}+\ve{GC}\bigr)\cdot\ve{GX}=0,
$$
lebo pre ťažisko $G$ je súčet $\ve{GA}+\ve{GB}+\ve{GC}$ rovný
nulovému vektoru.

\poznamka
Použitím vektorovej algebry možno prehľadne vyložiť aj
prvú časť druhého riešenia. Z všeobecne platných vektorových
rovností
$$
\ve{AX}=\ve{GX}-\ve{GA}\quad\hbox{a}\quad
\ve{S_AX}=\ve{GX}-\ve{GS_A}=\ve{GX}+\tfrac12\ve{GA}
$$
vyplývajú vyjadrenia
$$
|AX|^2=|GX|^2+|GA|^2-2\ve{GA}\cdot\ve{GX},\quad\hbox{resp.}\quad
|S_AX|^2=|GX|^2+\tfrac14|GA|^2+\ve{GA}\cdot\ve{GX}.
$$
Za predpokladu $\ve{GA}\cdot\ve{GX}\geqq0$ z toho vyplýva
$|AX|^2\leqq4|S_AX|^2$, ako pre naše riešenie potrebujeme.

\ineriesenie
Označme $T$ ťažisko trojuholníka $ABC$ a uvažujme rovnoľahlosť~$\varkappa$ so stredom v~bode $T$ a~koeficientom $-\frac12$. Body $S_a$, $S_b$ a $S_c$ sú postupne obrazy bodov $A$, $B$ a $C$ v rovnoľahlosti $\varkappa$. Označme ďalej $X'$ obraz bodu $X$ v rovnoľahlosti~$\varkappa$. Z~vlastností rovnoľahlosti vyplýva $|XA|=2|X'S_a|$, $|XB|=2|X'S_b|$ a $|XC|=2|X'S_c|$. Použitím týchto rovností dokazovanú nerovnosť upravíme na ekvivalentný tvar
$$
\min \left \{\frac {|X'S_a|}{|XS_a|}, \frac {|X'S_b|}{|XS_b|}, \frac {|X'S_c|}{|XS_c|} \right\} \le 1.
\tag5
$$
V prípade $X= T\;(= X')$ táto nerovnosť zrejme platí. Nech ďalej $X\ne T$, potom určite $X\ne X'$. Označme $o$ os úsečky $XX'$ (\obr). Predpokladajme, že nerovnosť~\thetag5 pre niektorý bod~$X$ neplatí, dostaneme tak $|X'S_a|/|XS_a|>1$, $|X'S_b|/|XS_b|>1$ a~$|X'S_c|/|XS_c|>1$. Z toho vyplýva, že body $S_a$, $S_b$ a $S_c$ sú vnútornými bodmi polroviny~$oX$. Teda aj~všetky body trojuholníka $S_aS_bS_c$ sú vnútornými bodmi tejto polroviny. To je ale spor, keďže jeho ťažisko $T$,\fnote{Toto známe tvrdenie vyplýva napríklad z~uvažovanej rovnoľahlosti.} teda vnútorný bod, je zrejme bodom polroviny opačnej. Tým sme dostali spor s~predpokladom, že nerovnosť~\thetag5 neplatí.
\inspsc{a70i_4.1}{0.8333}%%


\návody
Zoznámte sa s~Apollóniovými kružnicami a ich konštrukciami.
Sú to množiny bodov významnej vlastnosti, ktorá je v~nasledujúcom
tvrdení určená parametrom $\lambda$ a bodmi $P$ a~$Q$: Majme dané
kladné reálne číslo $\lambda\ne1$ a dva rôzne body $P$ a
$Q$ v~rovine~$\varrho$. Potom platí, že množinou všetkých bodov $X\in\varrho$,
$X\ne Q$, ktoré vyhovujú rovnici
$$
\frac{|PX|}{|QX|}=\lambda,
\tag1$$
je istá kružnica. Táto (Apollóniova) kružnica je kružnica nad
priemerom $MN$, pričom $M$ a $N$ sú tie dva body priamky $PQ$, pre
ktoré rovnica \thetag1 po dosadení $X=M$, resp. $X=N$ sa zmení na platnú
rovnosť.
[Pozri časť I~kapitoly 5 brožúry {\it S.~Horák: Kružnice},
dostupnej na \pdfklink{dml.cz/handle/10338.dmlcz/403589}{https://dml.cz/handle/10338.dmlcz/403589}. Iný
podrobný výklad nájdete v~návodných úlohách k~úlohe
\pdfklink{63-A-I-6}{https://skmo.sk/dokument.php?id=992}.]

Majme dané reálne číslo $\lambda>1$ a dva rôzne body $P$ a
$Q$ v~rovine~$\varrho$. Dokážte, že množinou všetkých bodov $X\in\varrho$,
$X\ne Q$, ktoré vyhovujú nerovnici
$$
\frac{|PX|}{|QX|}>\lambda,
$$
je vnútro kruhu ohraničeného Apollóniovou kružnicou, ktorá je určená
rovnicou \thetag1 z~úlohy~N1.
[Uvážte dve Apollóniove kružnice
$$
\frac{|PX|}{|QX|}=\lambda_1>1\quad\hbox{a}\quad
\frac{|PX|}{|QX|}=\lambda_2>1.
$$
Prvá z~nich je kružnica nad priemerom
$M_1N_1$, druhá je kružnica nad priemerom $M_2N_2$, pričom $M_1N_1$
a $M_2N_2$ sú isté úsečky na priamke $PQ$. Tvrdenie úlohy N2
bude dokázané, keď vysvetlíme, prečo v~prípade $\lambda_1<\lambda_2$
ležia body $M_2$, $N_2$ vnútri úsečky $M_1N_1$.]

\D
V~rovine $\varrho$ je daná úsečka $BC$. Uvažujme body~$A\in\varrho$ mimo
priamky $BC$ také, že veľkosti výšok $v_b$, $v_c$ na strany
$AC$, $AB$ trojuholníka $ABC$ sú v~pomere $1:2$. Dokážte, že všetky
tieto body~$A$ ležia na jednej kružnici.
[Pre obsah~$S$ trojuholníka $ABC$ platí $2S=b \cdot v_b = c \cdot v_c$,
a teda $b:c=v_c:v_b=2:1$. Vyhovujúce body~$A$ teda ležia na
Apollóniovej kružnici $|CX|/|BX|=2$.]

V~rovine $\varrho$ je daná úsečka $BC$. Uvažujme body~$A\in\varrho$ mimo
priamky $BC$ také, že veľkosti ťažníc $t_b$, $t_c$
na strany $AC$, $AB$ trojuholníka $ABC$ sú v~pomere $1:2$.
Dokážte, že všetky tieto body~$A$ ležia na jednej kružnici.
[Pre ťažisko $G$ trojuholníka $ABC$ platí $|BG|=\frac23t_b$
a~$|CG|=\frac23t_c$, takže $|BG|:|CG|=1:2$. Všetky body $G$ teda
ležia na jednej Apollóniovej kružnici, preto všetky body $A$ ležia
na jej obraze v rovnoľahlosti so stredom v strede úsečky
$BC$, ktorá má koeficient 3.]

\MOarchiv63-A-II-2
[\pdfklink{63-A-II-2}{https://skmo.sk/dokument.php?id=997}]

\MOarchiv63-A-I-6
[\pdfklink{63-A-I-6}{https://skmo.sk/dokument.php?id=992}]

\endnávod
}

{%%%%%   A-I-6
Počet žiaroviek $n$ má v~zadaní úlohy hodnotu 70.
Zapíšeme celý postup tak, aby bolo jasné, že je použiteľný pre
každé $n\geqq5$.

V~prvej časti riešenia budeme predpokladať, že máme pripravené
prepínače tak, že s~nimi možno rozsvietiť ľubovoľnú štvoricu žiaroviek.
Potom pre každú štvoricu žiaroviek existuje séria
stlačení, ktorej aplikáciou nakoniec \uv{prepneme} (\tj. zmeníme stav
rozsvietenia či zhasnutia) práve tieto 4~žiarovky. V~ďalšom texte
budeme písať o~\uv{prepínaní danej skupiny žiaroviek} bez zdôrazňovania,
že ostatné žiarovky pritom zostanú neprepnuté. Ak budeme
uvažovať stlačenia prepínačov v~niekoľkých sériách za sebou, budeme
počítať zmeny stavov žiaroviek nie po jednotlivých stlačeniach,
ale po jednotlivých sériách.

Keďže možno sériou stlačení prepnúť ľubovoľnú {\it štvoricu\/} žiaroviek,
musí sa dať vhodnou sériou prepnúť aj ľubovoľná {\it dvojica\/} žiaroviek,
povedzme $(a,b)$ -- uvážime za tým účelom ďalšie tri rôzne\fnote{Tu
potrebujeme predpoklad $n\geqq5$. Ale pre $n=4$ je úloha
triviálna -- stačí nám jeden prepínač, ktorý zmení stav všetkých
štyroch žiaroviek.} žiarovky $c$, $d$, $e$ a
po prepnutí $(a,c,d,e)$ nasledovanom prepnutím $(b,c,d,e)$
dosiahneme zmenu stavu práve žiaroviek $a$, $b$ (tie boli
prepnuté každá raz, zatiaľ čo $c$, $d$, $e$ každá dvakrát,
ostatné žiarovky ani raz).

Keď možno (ako už vieme) sériou stlačení prepnúť ľubovoľnú
{\it dvojicu\/} žiaroviek, musí sa dať prepnúť
aj ľubovoľná {\it skupina s~párnym počtom\/}
žiaroviek -- stačí ich spárovať do dvojíc a prepínať po nich.
V~ďalšom odseku vysvetlíme, prečo takých skupín je práve
$2^{n-1}$.

Chceme dokázať, že každá $n$-prvková množina má práve
$2^{n-1}$ podmnožín s párnym počtom prvkov. Na to najskôr zvážime
nejakých $n-1$ prvkov danej množiny. Každý z~nich môžeme do
zostavovanej podmnožiny buď zahrnúť, alebo nezahrnúť, takže podľa
pravidla súčinu máme pre tieto voľby práve $2^{n-1}$ možností.
Počet takto vybraných prvkov je potom buď párne, alebo nepárne číslo.
Podľa toho posledný, doposiaľ neuvažovaný prvok pôvodnej množiny
k~zatiaľ vybraným prvkom priradíme, resp. nepriradíme. Dostaneme
tak $2^{n-1}$ rôznych podmnožín s~párnym počtom prvkov. Keďže
je zrejmé, že opísanou konštrukciou dostaneme {\it každú\/}
z~podmnožín s~párnym počtom prvkov, je dôkaz ukončený.

Z~doterajších úvah vyplýva, že pre každú vyhovujúcu skupinu
prepínačov musíme ich postupným stláčaním dosiahnuť aspoň
$2^{n-1}$ rôznych stavov rozsvietenia daných $n$ žiaroviek.
Zrejme nezáleží na tom, v~akom poradí v~danej sérii prepínače
stláčame, lebo pri každej žiarovke hrá úlohu iba to, koľkokrát
bola prepnutá. Navyše ani konkrétny počet prepnutí danej žiarovky
nie je podstatný, výsledok záleží iba na tom, či je tento počet
párny, alebo nepárny. Nemá preto zmysel uvažovať také série,
kde je niektorý prepínač stlačený viackrát.\fnote{Možno to tiež
vysvetliť konštatovaním, že dve stlačenia toho istého prepínača sa
navzájom rušia.} V jednej sérii pre každý prepínač tak budeme mať
iba dve možnosti -- nepoužiť, alebo použiť raz.
Pri použití $k$~prepínačov teda dosiahneme nanajvýš $2^k$ rôznych
stavov rozsvietenia žiaroviek. Ak máme preto nimi dosiahnuť
potrebných $2^{n-1}$ rôznych stavov, musí platiť nerovnosť
$2^k~\geqq 2^{n-1}$, čiže~$k~\geqq n-1$.
(Pre danú hodnotu $n=70$ to znamená, že $k\geqq69$.)

V~druhej časti riešenia ukážeme, že $k=n-1$ prepínačov na splnenie
zadaného cieľa stačí. Očíslujme si žiarovky $1, 2,\ldots, n$
a~prepínače $1, 2, \ldots, n-1$. Pre všetky
$i\in\{1,2,\ldots,n-1\}$ nastavme prepínač~$i$ tak, aby prepol
(práve) žiarovky~$i$ a~$n$. Dokážeme, že takým
nastavením prepínačov dokážeme rozsvietiť ľubovoľnú štvoricu
žiaroviek $(a,b,c,d)$, pričom $1\leqq a<b<c<d\leqq n$.

V~prípade $d<n$ úlohu splníme tak, že postupne stlačíme
prepínače $a$, $b$, $c$ a $d$ -- tým sa každá žiarovka zo
štvorice $(a,b,c,d)$ prepne práve raz, žiarovka~$n$ práve
štyrikrát a ostatné žiarovky sa neprepnú ani raz.

Ako vyriešiť úlohu vo zvyšnom prípade $d=n$? Vtedy stačí stlačiť postupne
prepínače $a$, $b$ a $c$ -- tým prepneme žiarovky $a$, $b$, $c$
práve raz, žiarovku~$n$ práve trikrát a~ostatné žiarovky
sa neprepnú ani raz.

\zaver
Najmenší počet vyhovujúcich prepínačov je~69.

\návody
\titem
Vo všetkých úlohách pracujeme s~pojmami \uv{žiarovka}
a~\uv{prepínač} vo význame zo súťažnej úlohy.

Určte počet možných stavov rozsvietenia žiaroviek
$1$, $2$, $3$ a $4$, ktoré možno dosiahnuť použitím prepínačov
$\{1,2,3\}$, $\{1,2,4\}$ a $\{3,4\}$ (napr. prepínač
$\{3, 4\}$ vždy prepne práve žiarovky~3 a~4).
[Štyri stavy (vrátane toho pôvodného). Uvedomme si, že
poradie stláčania prepínačov v~danej sérii
nemá vplyv na celkový výsledok. Tiež nemá zmysel žiadny
prepínač použiť v~jednej sérii viackrát.
Stačí teda nájsť výsledný stav pre každú z~$2^3$~podmnožín danej
množiny 3 prepínačov, ktoré môžeme stlačiť, a potom spočítať,
koľko týchto stavov je rôznych. Výhodné je pritom výsledné stavy
určovať takto: $\{1,2,3\}+\{3,4\}\sim\{1,2,4\}$. Z~toho príkladu
vidíme, že prepínač $\{1,2,4\}$ je možné v~zadaní úlohy vynechať.]

Dokážte, že ak máme~$n$ prepínačov, pričom~$n$ je
prirodzené číslo, tak postupným prepínaním môžeme dosiahnuť
nanajvýš~$2^n$~rôznych stavov rozsvietenia (vrátane pôvodného).
[Podľa úvahy z~riešenia N1 platí, že počet dosiahnuteľných stavov
neprevyšuje počet všetkých podmnožín danej množiny prepínačov.]

Dokážte, že ak by v~súťažnej úlohe bol cieľ pozmenený na
možnosť rozsvietiť každú trojicu žiaroviek, tak
pri jeho splnení by bolo možné rozsvietiť každú
jednotlivú žiarovku.
[Na rozsvietenie jedinej zvolenej žiarovky $a$ vyberieme tri ďalšie
žiarovky $b$, $c$, $d$ a aplikujeme postupne 3 série rozsvietenia,
a to pre trojice $(a,b,c)$, $(a,b,d)$ a~$(a,c,d)$.]

Koľko má daná $n$~prvková množina tých podmnožín, ktoré majú párny
počet prvkov?
[$2^{n-1}$ podmnožín. Postupujeme podobne ako pri známom dôkaze,
že počet {\it všetkých\/} podmnožín je~$2^n$: Prvých $n-1$ prvkov
môžeme do konštruovanej podmnožiny buď zaradiť, alebo nezaradiť,
takže máme $2^{n-1}$ možností. Po tejto procedúre máme vybraný
buď párny, alebo nepárny počet prvkov, a~tak podľa toho k~nim
nezaradíme, resp. zaradíme posledný $n$-tý prvok.
Takto dostaneme práve $2^{n-1}$~vyhovujúcich podmnožín.
Ostáva si uvedomiť, že opísaným postupom dostaneme {\it každú\/}
vyhovujúcu podmnožinu.]

\D
Riešte súťažnú úlohu so 70 žiarovkami s~pozmenenou podmienkou,
že máme byť schopní rozsvietiť každú $2k$-ticu žiaroviek, pričom
$k$ je dané prirodzené číslo z~intervalu $\langle 3,34\rangle$.
[Taká úloha je ekvivalentná so súťažnou úlohou. Nech ďalej
$(a,b)$ je ľubovoľná dvojica žiaroviek a $c_i$ označuje žiarovky rôzne
od $a$, $b$, pričom $c_i\ne c_j$ pre $i\ne j$. Ak možno rozsvietiť
$2k$-tice $(a,c_1,\ldots,c_{2k-1})$ a
$(b,c_1,\ldots,c_{2k-1})$, tak možno rozsvietiť aj dvojicu $(a,b)$,
a teda aj ľubovoľnú štvoricu (po dvoch dvojiciach).
Naopak, ak možno rozsvietiť štvorice $(a,c_1,c_2,c_3)$
a $(b,c_1,c_2,c_3)$, možno rozsvietiť aj dvojicu $(a,b)$,
takže potom možno rozsvietiť aj každú $2k$-tici (po $k$ dvojiciach).]

Riešte súťažnú úlohu so 70 žiarovkami s~pozmenenou podmienkou,
že máme byť schopní rozsvietiť každú $(2k-1)$-ticu žiaroviek, pričom
$k$ je dané prirodzené číslo z~intervalu $\langle 2,35\rangle$.
[70 prepínačov. Podobne ako v~riešení D1 úvahou o~dvoch $(2k-1)$-ticiach
$(a,c_1,\ldots,c_{2k-2})$ a $(b,c_1,\ldots,c_{2k-2})$ zistíme,
že možno rozsvietiť každú dvojicu $(a,b)$. Teraz úvahou o~jednej
$(2k-1)$-tici $(a,c_1,\ldots,c_{2k-2})$ a $k-1$ dvojiciach
$(c_1,c_2)$,\dots, $(c_{2k-3},c_{2k-2})$ zistíme, že možno
rozsvietiť ľubovoľnú žiarovku $a$, a teda aj každú z~$2^{70}$
množín žiaroviek, takže podľa výsledku úlohy N2
potrebujeme aspoň $70$ prepínačov. 70 prepínačov ale stačí --
jeden prepínač na každú žiarovku.]

\endnávod
}

{%%%%%   B-I-1
Hodnota skúmaného súčtu
$$
S=AB+CD+EF+GH+IJ
$$
závisí iba od toho, ktorých päť cifier sa nachádza v~zostavených
číslach na mieste desiatok, a~ktorých päť na mieste jednotiek.
Ak totiž určíme súčty čísel v~oboch spomenutých päticiach
$$
S_1=A+C+E+G+I\quad\text{a}\quad S_0=B+D+F+H+J,
$$
budeme zrejme mať $S=10S_1+S_0$. Podľa zadania úlohy sčítance
v~oboch súčtoch $S_1$ a~$S_0$ spolu tvoria všetkých 10
cifier od 0 po 9. Preto je súčet $S_1+S_0$ rovnaký ako súčet
$0+1+\dots+9$, ktorý je rovný 45. Platí teda $S_0=45-S_1$,
a~tak pre súčet~$S$ dostávame vyjadrenie
$$
S=10S_1+S_0=10S_1+(45-S_1)=9S_1+45=9\cdot(S_1+5).
$$

Pre riešenie našej úlohy stačí podľa posledného vzorca zistiť,
aké hodnoty môže nadobúdať súčet $S_1$. Je to súčet
cifier $A$, $C$, $E$, $G$, $I$ na miestach desiatok piatich dvojciferných čísel,
a~tak to podľa zadania môže byť súčet ľubovoľných piatich rôznych
{\it nenulových\/}
cifier -- zvyšnými piatimi nezastúpenými ciframi (vrátane nuly) potom
totiž môžeme akokoľvek obsadiť miesta jednotiek vytváraných čísel.
Pre súčet takých piatich cifier zrejme platí
$$
15 = 1+2+3+4+5 \leqq A+C+E+G+I \leqq 5+6+7+8+9 = 35.
$$
Presvedčíme sa, že súčet $S_1=A+C+E+G+I$ môže nadobúdať všetkých $21$
hodnôt v~rozsahu $15$ až $35$. Napríklad hodnotu $16$ dostaneme tak, že
v~pätici $1,2,3,4,5$ najväčšie číslo 5 zmeníme na číslo $6$. Takto môžeme
pokračovať v zväčšovaní súčtu~$S_1$ vždy o~1 tak dlho,
až dostaneme päticu $1,2,3,4,9$. Potom začneme o~jednotku opakovane
zväčšovať číslo 4, až dostaneme
päticu $1,2,3,8,9$. Podobne pokračujeme ďalej, až nakoniec dôjdeme k~pätici
$5,6,7,8,9$. Postupne tak naozaj dostaneme každú
celočíselnú hodnotu súčtu $S_1$ v~rozsahu $15$ až $35$ (ktorých
je 21).

Vďaka poslednému zisteniu podľa skôr odvodeného vzorca $S=9\cdot(S_1+5)$
prichádzame k~záveru, že skúmaný súčet $AB+CD+EF+GH+IJ$ môže nadobúdať
práve $21$ hodnôt, ktorými sú násobky deviatich v~rozsahu od
$180$ po $360$ (hraničnú hodnotu sme spočítali ako $9\cdot(15+5)=180$
a~$9\cdot(35+5)=360$).

\návody
Z dvoch rôznych cifier $A, B$ vytvoríme
dvojciferné čísla $AB$ a $BA$. Dokážte, že ich rozdiel je
deliteľný $9$.
[$AB-BA=(10\cdot A+B)-(10\cdot B+A)=9(A-B)$.]

Situáciu zo súťažnej úlohy \uv{zmenšíme}. Z~cifier $1$, $2$, $3$,
$4$ vytvoríme dvojciferné čísla $AB$, $CD$, pričom každú
cifru použijeme práve raz.\hfill\break
a) Nájdite najmenšiu možnú a najväčšiu možnú hodnotu $AB+CD$.\hfill\break
b) Zistite najmenší možný kladný rozdiel dvoch takto vytvorených
čísel $AB+CD$.\hfill\penalty-100
[a)~Minimum je $37$, maximum je~$73$. Minimum, resp. maximum
dostaneme, keď umiestnime najmenšie cifry 1 a 2 na pozície
desiatok, resp. na pozície jednotiek. b) 9. Každé vytvorené číslo
$AB+CD$ dáva po delení deviatimi rovnaký zvyšok ako číslo $A+B+C+D$
rovné $1+2+3+4=10$, dáva teda zvyšok 1. Preto je rozdiel každých dvoch
vytvorených čísel $AB+CD$ deliteľný deviatimi, teda hľadané minimum
je kladným násobkom čísla~9. Že to nie je viac ako~9,
ukazuje príklad $(12+43)-(12+34)=9$.]

Z~čísel $1$, $2$, $3$, $4$, $5$, $6$ vyberieme tri rôzne.
Zdôvodnite, že ich súčet môže nadobúdať
ktorúkoľvek celočíselnú hodnotu od $6$ do $15$.
[Nájsť najmenšiu a najväčšiu hodnotu nestačí. Je nutné opísať,
ako skonštruovať ktorýkoľvek prípustný súčet, napríklad $13$.
Opis môže vyzerať ako algoritmus, ktorý prejde všetky súčty
od $6$ do $15$. Keď začneme s~trojicou $1$, $2$, $3$, môžeme číslo
$3$ opakovane zväčšovať o~jedna tak dlho, až dostaneme
trojicu $1$, $2$, $6$. Potom začneme zväčšovať číslo $2$, až dostaneme
$1$, $5$, $6$. Nakoniec budeme zväčšovať číslo $1$, až dostaneme
$4$, $5$, $6$, čím celkovo prejdeme všetky čísla od 6 do 15.]

\DOP
\MOarchiv61-C-II-2
[\Ulink{https://skmo.sk/dokument.php?id=513}{61-C-II-2}]

\MOarchiv68-B-I-1
[\Ulink{https://skmo.sk/dokument.php?id=3042}{68-B-I-1}]
\endnávod
}

{%%%%%   B-I-2
Z~úpravy daného výrazu $V$ na súčin
$$
V=xy-x^3y-xy^3 = xy\cdot(1-x^2-y^2)
$$
najskôr vidíme, že ak je činiteľ $1-x^2-y^2$ záporný, je záporný aj výraz $V$. Preto ďalej stačí uvažovať iba také kladné $x$ a $y$, pre ktoré platí
$1-x^2-y^2>0$, ako sa stane napríklad pre
$x=y=\frac12$, všeobecnejšie pre ľubovoľné $x$, $y$ blízka nule.


Vďaka učinenému obmedzeniu môžeme hodnotu $V$ odhadnúť zhora tak, že
najskôr odhadneme súčin $xy$ podľa známej AG-nerovnosti, ktorá je dokázaná v návodnej úlohe~1. V~nej položíme $u=x^2$ a~$v=y^2$.
Dostaneme tak nerovnosť $xy\leqq\frac12(x^2+y^2)$. Z~nej po vynásobení
oboch strán kladným číslom $1-x^2-y^2$ dostaneme horný odhad
$$
V=xy\cdot(1-x^2-y^2)\leqq\frac12(x^2+y^2)\cdot\bigl(1-(x^2+y^2)\bigr).
\tag1
$$
Keď teraz označíme $t=x^2+y^2$, ostáva nám nájsť maximum kvadratickej
funkcie $f(t)=t(1-t)$ pre $t\in(0,1)$. Toto maximum je možné nájsť opätovným použitím AG\spojovnik{}nerovnosti. Tentoraz v~nej zvolíme (opäť kladné) hodnoty $u=t$ a~$v=1-t$, pričom $t=x^2+y^2$. Keďže $u+v=1$,
má AG-nerovnosť tvar $\sqrt{t(1-t)}\leqq\frac12$, z~ktorého po umocnení
na druhú dostaneme $t(1-t)\leqq\frac14$. Po dosadení $t=x^2+y^2$
tak vychádza
$$
(x^2+y^2)\cdot\bigl(1-(x^2+y^2)\bigr)\leqq\frac14.
$$
Z~odhadu \ref[2.1] teda vyplýva $V\leqq\frac12\cdot\frac14=\frac18$.

Je nájdené číslo $\frac18$ možnou hodnotou výrazu $V$? A~ak
áno, pre ktoré prípustné dvojice $x$, $y$ sa dosahuje? Odpoveď
nám poskytne predchádzajúci postup. Podľa neho dosiahneme rovnosť $V=\frac18$ práve vtedy, keď v~oboch uplatnených AG-nerovnostiach
nastane rovnosť.
Ako je známe, bude to tak práve vtedy, keď bude platiť $u=v$ pre obe
využité dvojice čísel $u$, $v$. V~našej situácii sa jedná o~rovnosti
$x^2=y^2$ a~$t=1-t$ pre $t=x^2+y^2$. Ekvivalentná podmienka
$x^2=y^2=\frac14$ zodpovedá
jedinej prípustnej dvojici $(x,y)=\bigl(\frac12,\frac12\bigr)$.

\Zav
Najväčšia možná hodnota daného výrazu je $\frac18$.
Výraz ju nadobúda pre jedinú dvojicu $(x,y)=(\frac12,\frac12)$.

\Pozn
Obe v~riešení využité nerovnosti
$$
xy\leqq\frac12(x^2+y^2)\quad\hbox{a}\quad t(1-t)\leqq\frac14
$$
možno bez odkazu na AG-nerovnosť ľahko dokázať úpravami \uv{na štvorec},
\tj. prepisom do ekvivalentných tvarov
$$
\frac12(x-y)^2\geqq0,\quad\hbox{resp.}\quad\Biggl(t-\frac12\Biggr)^2\geqq0.
$$
Z~nich tiež vyplývajú nutné a~postačujúce podmienky pre prípady rovnosti, \tj. $x=y$, resp. $t=\frac12$ (ako aj platnosť oboch
nerovností pre ľubovoľné reálne čísla $x$, $y$, $t$ bez ohľadu na
ich znamienka).

\návody
Dokážte, že pre ľubovoľné nezáporné reálne čísla $u$, $v$ platí nerovnosť
$\sqrt{uv}\leqq\frac12(u+v)$, pritom rovnosť v~nej nastane práve vtedy, keď $u=v$. [Zrejme platí $\bigl(\sqrt u~-\sqrt v\bigr)^2\geqq0$. Po roznásobení ľavej strany túto nerovnosť prepíšeme na tvar $u-2\sqrt{uv}+v\geqq0$, čo napokon upravíme na požadované $\sqrt{uv}\leqq\frac12(u+v)$. Rovnosť v~tejto nerovnosti nastane práve vtedy, keď $\sqrt u~-\sqrt v=0$, čo je ekvivalentné $\sqrt u~=\sqrt v$, teda aj $u=v$.
{\it Poznámka}. Keďže výraz na ľavej strane nerovnosti je {\it geometrickým\/} priemerom dvoch nezáporných reálnych čísel $u$, $v$ a výraz na pravej strane je ich {\it aritmetickým\/} priemerom, nazýva sa uvedená nerovnosť {\it nerovnosťou medzi aritmetickým a geometrickým priemerom\/} dvoch nezáporných reálnych čísel, skrátene {\it AG-nerovnosť}. Podobná rovnako pomenovaná nerovnosť platí aj pre $n$-tice nezáporných čísel.]

Nájdite najväčšiu hodnotu výrazu
a) $t(1-t)$, b) $uv(1-uv)$, c) $(u^2+v^2)(1-u^2-v^2)$. Vo všetkých
výrazoch písmená označujú ľubovoľné reálne čísla.
[a) $\frac14$. Ak sú obe čísla $t$ a $1-t$ kladné, napíšte pre ne
AG-nerovnosť. Rozmyslite si prípady, keď nie sú obe čísla kladné.
Alternatívne možnosťou je úprava na štvorec:
$t(1-t)=\frac14-(\frac12-t)^2$. b)~$\frac14$. Substitúcia $t=uv$ vedie na
prípad a). c) $\frac14$. Substitúcia $t=u^2+v^2$ vedie na prípad a).]

\DOP
Pre reálne čísla $a$, $b$ nájdite najväčšiu možnú hodnotu výrazu
$$
\frac{ab}{a^2+b^2}.
$$
[Maximum je $\frac12$ (pre $a=b$). Určite sa stačí obmedziť na prípad, keď
$a>0$ a ${b>0}$. Použite AG-nerovnosť pre dvojicu čísel $a^2$ a $b^2$.
Inak je možné vyjsť z~nerovnosti $({a-b})^2\geqq0$, upravenej
na tvar $2ab\leqq a^2+b^2$.]

\MOarchiv68-B-II-1
[\Ulink{https://skmo.sk/dokument.php?id=3125}{68-B-II-1}]

Dokážte, že pre ľubovoľné reálne
čísla $a$, $b$, $c$ platí $a^2+b^2+c^2\geqq ab+bc+ca$.
[Sčítaním troch nerovností $a^2+b^2\geq 2ab$, $b^2+c^2\geq 2bc$,
$c^2+a^2\geq 2ca$ získame nerovnosť, ktorú potom stačí vydeliť
dvoma.]

\MOarchiv68-B-I-4
[\Ulink{https://skmo.sk/dokument.php?id=3042}{68-B-I-4}]
\endnávod
}

{%%%%%   B-I-3
Výšky $AA'$ a~$BB'$ ležia vnútri trojuholníka $ABC$, lebo
je podľa zadania ostrouhlý. Preto pre pravouhlé
trojuholníky $AA'C$ a~$BB'C$ platí, že ich uhly pri vrcholoch $A$, resp. $B$
(vyznačené na \obr) majú tú istú veľkosť $\gamma'$, ktorá
dopĺňa zvyčajne značený uhol $\gamma$ pri vrchole $C$ do
$90^\circ$. Z~pravouhlého trojuholníka $BB'C$ s~bodom $A'$ vnútri
prepony $BC$ ďalej vyplýva, že jeho kolmý priemet $D$ je vnútorným
bodom odvesny~$BB'$. Preto má veľkosť $\gamma'$
aj uhol $A'BD$, čiže $CBD$.
\inspdfsirka{B70i_3a.pdf}{75mm}%

Podľa zadania je bod $E$ vnútorným bodom strany $AC$, ktorý leží
na kružnici opísanej trojuholníku $BDC$. Oblúk $BDC$ tejto kružnice
leží v~polrovine $BCA$, takže bod~$E$ leží na tomto oblúku,
a~to medzi bodmi $D$ a~$C$, lebo bod $D$ je vnútorným bodom
polroviny $ACB$. Preto je štvoruholník $BCED$
konvexný a~tetivový. Keďže má pri vrchole~$B$ ostrý uhol
veľkosti $\gamma'$,
má pri protiľahlom vrchole $E$ tupý uhol veľkosti
$180^{\circ}-\gamma'$. K~nemu vedľajší uhol $AED$ má teda
veľkosť $\gamma'$, ako je tiež vyznačené na \obrr1.
Ukážeme, že to už nám stačí na želaný dôkaz
zhodnosti vyfarbených úsečiek $DE$ a~$AA'$.

Priamky $AC$ a~$A'D$ sú rovnobežné, pretože sú obe kolmé
na výšku $BB'$. Označme~$d$ ich vzdialenosť. Dĺžky
oboch skúmaných úsečiek potom môžeme vyjadriť rovnakým zlomkom
$d/\sin\gamma'$ (ako je zrejmé z~\obr). Tým je riešenie úlohy
ukončené.
\inspdfsirka{B70i_3b.pdf}{29.1667mm}%

\Pozn
Bez použitia trigonometrie možno riešenie dokončiť
úvahou o~priesečníku~$X$ úsečiek $AA'$ a~$DE$. Vďaka zhodným uhlom
$EAX$, $AEX$ aj k~nim striedavým uhlom $XA'D$, $XDA'$ sú trojuholníky
$AEX$ a~$A'DX$ rovnoramenné so spoločným hlavným vrcholom $X$.
Platia preto rovnosti $|DX|=|A'X|$ a~$|XE|=|XA|$, ich sčítaním
už dostaneme potrebný výsledok $|DE|=|AA'|$.


\návody
Ostrouhlý rôznostranný trojuholník $ABC$ je svojimi výškami
rozdelený na šesť neprekrývajúcich sa trojuholníkov.
Zistite, či niektoré z~nich sú podobné. Ak áno, existujú
medzi nimi tri navzájom podobné trojuholníky?
[Každý trojuholník je podobný s~práve jedným z~ostatných piatich trojuholníkov.
Využite to, že sa jedná o~pravouhlé trojuholníky, ktorých ostré vnútorné uhly
pri stranách trojuholníka $ABC$ dopĺňajú jeho vnútorné uhly do
$90\st$.]

Na kružnici so stredom $O$ sú dané body $B$ a $C$ také, že
$|\uhel BOC|=120^\circ$. Zvoľme bod~$A$ na
dlhšom oblúku $BC$ a označme $|\uhel AOB|=\delta$.\hb
a) Zistite veľkosť uhla $BAC$, keď
$\delta=140^\circ$.\hb
b) Zistite, ako máme voliť uhol $\delta$, aby bol uhol $BAC$ čo najväčší.\hb
c) Na kratšom oblúku $BC$ zvolíme bod $A'$.
Zistite, ako máme voliť polohy bodov $A$, $A'$
(oba ležia na danej kružnici), aby súčet $|\uhel BAC|+|\uhel BA'C|$
bol čo najväčší.\hp
[V~rovnoramenných trojuholníkoch $BOC$, $COA$ a $AOB$ spočítajte
uhly, alebo ich vyjadrite v~závislosti od uhla $\delta$. V~a)
vyjde $|\uhel BAC|=60^\circ$, rovnako ako v~b) nezávisle na
voľbe~$\delta$. V~c) vyjde súčet $180^\circ$ nezávisle na
polohe bodu~$A$ alebo $A'$. Tvrdenie c) má známe zovšeobecnenie:
Štvoruholník je tetivový práve vtedy, keď súčet veľkostí jeho
protiľahlých uhlov je $180^\circ$.]

\DOP
V~ostrouhlom trojuholníku $ABC$
označme $A', B', C'$ päty jeho výšok a $H$ jeho ortocentrum
(priesečník výšok $AA'$, $BB'$, $CC'$). Nájdite všetkých $6$
tetivových štvoruholníkov s~vrcholmi v~bodoch $A$, $B$, $C$, $A'$, $B'$,
$C'$, $H$.
[Hľadajte pravé uhly a~Tálesove kružnice: Body $A'$, $B'$
ležia na Tálesovej kružnici nad priemerom $AB$, takže štvoruholník
$ABA'B'$ je tetivový a podobne $BCB'C'$ a $CAC'A'$. Body $B'$,
$C'$ ležia na Tálesovej kružnici nad priemerom $AH$, takže
štvoruholník $AB'HC'$ je tetivový a podobne $BC'HA'$ a $CA'HB'$.]

V~rovine sú dané kružnice $m$ a $n$, ktoré sa pretínajú v~bodoch
$K$ a~$L$. Na kružnici~$m$ ležia body $A$, $D$, $K$, $L$ a na
kružnici $n$ ležia body $B$, $C$, $K$, $L$ v~týchto poradiach,
pričom body $A$, $L$, $B$ ležia na priamke a body $C$, $K$, $D$
ležia na inej priamke v~týchto poradiach. Dokážte, že $AD \parallel BC$.
[Označme $|\uhel
LBC|=\beta$. Potom $|\uhel LKC|={180^\circ-\beta}$, $|\uhel LKD|=\beta$
a $|\uhel LAD|=180^\circ-\beta$. Z~rovnosti $|\uhel LBC|+|\uhel
LAD|=180^\circ$ vyplýva $AD\parallel BC$.]

Zvoľme ľubovoľné body $A'$, $B'$,
$C'$ vnútri strán $BC$, $CA$, $AB$ trojuholníka $ABC$. Dokážte,
že kružnice opísané trojuholníkom $AB'C'$, $BC'A'$, $CA'B'$ sa
pretínajú v~jednom bode.
[O priesečníku dvoch kružníc ukážeme, že leží na tretej kružnici. Označme $M$ napríklad priesečník kružníc opísaných trojuholníkom $AB'C'$ a $BC'A'$ a predpokladajme, že bod $M$ leží vnútri trojuholníka $ABC$. Z~tetivového štvoruholníka $AC'MB'$ vyplýva rovnosť $|\uhel CB'M|=|\uhel AC'M|$. Z~tetivového štvoruholníka $BA'MC'$ vyplýva rovnosť $|\uhel AC'M|=|\uhel BA'M|$. Spolu dostávame rovnosť $|\uhel CB'M|=|\uhel BA'M|$, takže štvoruholník
$CB'MA'$ je tetivový. Podobne sa rozoberú prípady, v ktorých bod $M$ leží zvonka trojuholníka $ABC$. Bod $M$ sa v literatúre označuje termínom {\it Miquelov bod}.]

\endnávod
}

{%%%%%   B-I-4
Všimnime si, že keď je dvojica $(x,y)$
riešením sústavy, je ním aj dvojica $(x,\m y)$. Riešenia teda typicky
\uv{prichádzajú po dvoch}. Napríklad keby bola dvojica $(x,y)=(2,8)$
riešením, bude ním aj dvojica ${(2,-8)}$.
Len v~prípade, keď ${y=-y}$, čiže $y=0$, nie sú také dve riešenia
$(x,y)$ a~$(x,\m y)$ navzájom rôzne.

Daná sústava tak môže mať nepárny počet riešení iba v~prípade,
keď pre ${y=0}$ existuje~$x$ spĺňajúce obe rovnice.
V~takom prípade sa prvá rovnica zjednoduší na $|{x+6}|=24$,
takže nutne je buď $x=18$, alebo $x={-30}$. Z~druhej rovnice potom vyplýva,
že dvojica $(18,0)$ je riešením pre
$k=18$ a~dvojica $({-30},0)$ je riešením pre $k=30$. V~oboch prípadoch
neexistuje žiadne iné riešenie $(x,y)$, pre ktoré je $y=0$, takže
sústava má nepárny počet riešení ako pre $k=18$, tak aj pre $k=30$,
pokiaľ však pre tieto dve $k$ nie je riešení {\it nekonečne veľa}.
Skutočnosť, že zadaná sústava má iba konečný počet riešení, vyplýva (aj pre všeobecné~$k$) z~toho, že každé z~nich je riešením jednej zo 16 sústav,
ktoré dostaneme, keď v~\uv{neurčitej} sústave
$$\eqalign{
\pm(x+6)\pm2y&=24,\cr
\pm(x+y)\pm(x-y)&=2k}
$$
akokoľvek vyberieme znamienka. Vysvetlíme teraz, že
každá z~týchto 16 sústav má práve jedno riešenie. Ľavá strana
druhej rovnice je totiž rovná jednému zo štyroch výrazov $\pm2x$ alebo
$\pm2y$, preto pri danom $k$ druhá rovnica vedie vždy
k~jednoznačnému určeniu jednej z~neznámych $x$ či $y$,
zatiaľ čo druhá neznáma môže mať ľubovoľnú hodnotu --
tú potom zrejme jednoznačne určíme z~prvej rovnice po
dosadení určenej hodnoty prvej neznámej. Preto má sústava rovníc zo
zadania úlohy vždy nanajvýš 16 riešení.\fnote{To je samozrejme
veľmi hrubý, aj keď v~daný moment dostatočný odhad. Podľa grafického
postupu, ktorý ďalej opíšeme, sa ľahko zistí,
že najväčší počet riešení je rovný číslu 8, a~to pre $k\in(10,12)$.}

\Zav
Hľadané hodnoty parametra $k$ sú práve dve, a~to
čísla 18 a~30.


\Jres
V~rovine zvolíme karteziánsku sústavu
súradníc $Oxy$ a~pozrieme sa, ako geometricky vyzerá
množina bodov $[x,y]$ prislúchajúcich prvej rovnici a~ako
množina bodov $[x,y]$ prislúchajúcich druhej rovnici. Prienik týchto dvoch
množín potom bude tvorený práve tými bodmi $[x,y]$, ktoré zodpovedajú
riešeniam zadanej sústavy.

Rozborom znamienok výrazov v absolútnej hodnote (pozri \obr) ľahko zistíme,
že pre prvú rovnicu sa jedná o~hranicu kosoštvorca so stredom
v~bode $[\m6,0]$,
ktorého vrcholy sú v~bodoch $[18,0]$, $[\m6,12]$, $[\m30,0]$
a~$[\m6,\m12]$. Napríklad v~oblasti, ktorá je vymedzená
nerovnosťami $x+6\leq0$ a~$y\geq0$, rovnica zodpovedá priamke ${-x-6+2y}=24$
a~vyhovujúce body tak vytvoria úsečku, ktorú táto priamka na danej
oblasti vytína. Analogicky postupujeme vo zvyšných troch oblastiach.
\inspsc{b70i_4.1}{0.8333}%

Pre druhú rovnicu najskôr zdôraznime, že ďalej budeme uvažovať iba
{\it kladné\/} hodnoty parametra $k$. V~prípade $k<0$ totiž zrejme žiadna
dvojica $(x,y)$ vyhovujúca druhej rovnici neexistuje,
takže počet riešení sústavy je rovný párnemu číslu 0. V~prípade
${k=0}$ je to rovnako tak, lebo z~druhej rovnice vtedy vyplýva
$(x,y)=(0,0)$, avšak táto dvojica nespĺňa prvú rovnicu.

Pre každé $k>0$ podobným rozborom ako pri prvej rovnici zistíme (\obr),
že množina bodov prislúchajúcich druhej rovnici je tvorená
hranicou štvorca so stredom v~počiatku, ktorého vrcholy sú v~bodoch $[k,k]$,
$[\m k,k]$, $[\m k,\m k]$ a~$[k,\m k]$.
\inspsc{b70i_4.2}{0.8333}%

Predstavme si teraz, že do prvého obrázka s~\uv{pevným} kosoštvorcom
začneme prikresľovať \uv{premenlivý} štvorec, ktorý sa bude meniť
podľa toho, ako parameter~$k$ bude prebiehať interval všetkých
kladných čísel (pozri \obr{} pre hodnotu $k=8$).
\inspsc{b70i_4.3}{0.8333}%

Našou úlohou je nájsť všetky tie hodnoty $k>0$, pri ktorých budú
mať hranice oboch štvoruholníkov nepárny počet spoločných bodov
(nazývajme ďalej {\it priesečníkov\/}). Oba útvary sú
súmerné podľa súradnicovej osi $x$, takže počet
priesečníkov nad osou $x$ bude rovnaký ako počet priesečníkov pod osou $x$.
Zaujímajú nás preto iba tie hodnoty parametra $k$, pre ktoré existuje
priesečník na osi $x$. To nastane iba pre $k=18$ a~$k=30$.
V~prípade $k=18$ budú v~celej rovine zrejme existovať celkom tri priesečníky,
v~prípade $k=30$ iba jeden.

Rovnako ako v~prvom riešení prichádzame
k~záveru, že $k=18$ a~$k=30$ sú jediné dve hľadané hodnoty.



\návody
V~karteziánskej sústave súradníc $Oxy$ znázornite množinu všetkých
bodov ${[x,y]}$, ktorých súradnice spĺňajú rovnicu
a) $|x|+|y|=7$, b) $|x-3|+|y|=7$, c) $|x|+2|y|=10$.
[a)~V~každom zo štyroch kvadrantov dostaneme rovnicu priamky,
celkovou množinou je hranica štvorca s~vrcholmi v~bodoch
${[7,0]}$, ${[0,7]}$, ${[-7,0]}$, ${[0,-7]}$. b) Hranica štvorca posunutá o~vektor
$(3,0)$ oproti štvorcu z~úlohy a). c) Hranica kosoštvorca
s~vrcholmi v~bodoch ${[10,0]}$, ${[0,5]}$, ${[-10,0]}$, ${[0,-5]}$.]

Rozmyslite si, ako v~karteziánskej sústave súradníc $Oxy$
vyzerá množina všetkých bodov ${[x,y]}$, ktorých súradnice
spĺňajú\hb
a) $x\leq y$ a zároveň $x\geq-y$,\hb
b) $x\leq y$ a zároveň $x\geq-y$ a zároveň $|x+y|+|x-y|=10$.\hp
[a) Jedná sa o~prienik dvoch polrovín s~hraničnými priamkami
$x=y$, resp. $x={-y}$. Výsledná množina je pravý uhol s~vrcholom
v~počiatku a~vnútorným bodom ${[0,1]}$.
b) Po odstránení absolútnych hodnôt dostanete rovnicu
priamky. Prienikom tejto priamky s~uhlom z~úlohy a) je úsečka
s~krajnými bodmi ${[\m5,5]}$ a ${[5,5]}$.]

Zdôvodnite, prečo pre ľubovoľnú hodnotu reálneho
parametra $k$ má sústava rovníc
$$\eqalign{
|x+6|+2|y|&=24,\cr
|x+6|+|y|&=k
}$$
párny počet riešení v~obore reálnych čísel.
[Obe zodpovedajúce množiny bodov (hranice kosoštvorca a štvorca)
sú súmerné podľa osi $x$. Teda pre každé riešenie $(x,y)$
tejto sústavy rovníc je aj dvojica $(x,{-y})$ jej riešením.
Ak teda pre každé riešenie $(x,y)$ platí $y\ne0$, má sústava
párny počet riešení (hranice štvorca a kosoštvorca nemôžu mať
nekonečne veľa spoločných bodov). Ak naopak má sústava
riešenie tvaru $(x,0)$, je nutne $k=24$. Potom existujú práve
dve riešenia $({-30},0)$ a $(18,0)$.]

\DOP
\MOarchiv66-B-I-2
[\Ulink{https://skmo.sk/dokument.php?id=2242}{66-B-I-2}]


Použitím grafickej metódy a ďalej potom
výpočtom určte všetky reálne riešenia sústavy rovníc
$$
\eqalign{
|x|+|y-1|&=1,\cr
|x - 1| + |y|&=p,
}$$
pričom $p$ je reálny parameter.
[13-A-II-3]

\endnávod
}

{%%%%%   B-I-5
Uvažovaná kolmica na stranu $DE$ vztýčená v~bode $D$ najskôr pretne
v~bode~$P$ uhlopriečku $CG$ a~potom \uv{opustí} daný sedemuholník
vo vnútornom bode $Q$ jeho strany $AB$ (\obr{} vľavo). Aby sme úsečky
v~dokazovanej rovnosti $|GP|=|EF|+|AQ|$ dostali k~sebe \uv{bližšie},
zameníme v~nej stranu $EF$ zhodnou stranou~$GA$. Budeme tak
dokazovať ekvivalentnú rovnosť
$$
|GP|=|GA|+|AQ|
\tag1$$
pre dĺžky troch strán štvoruholníka $AQPG$. Postup, ktorý pritom
zvolíme, nám zároveň dosvedčí, že naozaj
platí $|GP|>|AQ|$ (ako \obrr1{} napovedá).
\inspdfpdfsirka{B70i_51.pdf}{55mm}{B70i_52.pdf}{55mm}

Úsečky $GP$ a~$AQ$ sú v~prvom rade rovnobežné, lebo sú obe kolmé na
tú istú os súmernosti celého sedemuholníka, ktorou je spoločná
os strany $AB$ a~uhlopriečky~$CG$ (\obrr1{} vpravo). Iná os
súmernosti celého sedemuholníka, konkrétne os strany $DE$,
prechádza vrcholom $A$ a~je rovnobežná s~priamkou $PQ$. Keďže
navyše leží v~polrovine $PQG$, pretína táto os uhlopriečku $CG$
v~bode, ktorý leží medzi bodmi $P$, $G$
a~ktorý označíme $A'$ (\obrr1{} vľavo).
Vďaka $A'\in GP$ platí rovnosť $|GP|=|GA'|+|A'P|$,
navyše z~relácií $A'P\parallel AQ$ a~$PQ\parallel A'A$ vyplýva, že
štvoruholník $AQPA'$ je rovnobežník, a~teda $|A'P|=|AQ|$.
Rovnosti z~poslednej vety dokopy znamenajú, že
$$
|GP|=|GA'|+|AQ|.
$$
Týmto výsledkom je nerovnosť $|GP|>|AQ|$ potvrdená.
Významnejšie pre nás je však porovnanie s~rovnosťou \ref[rov5.1],
ktorú máme dokázať. Vidíme, že takú úlohu splníme,
akonáhle overíme rovnosť $|GA|=|GA'|$.
Rovnoramennosť trojuholníka $AA'G$ však vyplýva zo zhodnosti troch uhlov
vyznačených na \obrr1{} vľavo: uhly $QAA'$, $AA'G$ sú striedavé
uhly medzi rovnobežkami $AB$, $CG$ a~zhodnosť uhlov $QAA'$, $GAA'$
vyplýva z~toho, že polpriamka~$AA'$ ako os súmernosti celého
sedemuholníka rozpoľuje jeho vnútorný uhol $GAB$.

\návody
Rozmyslite si, že
pravidelný sedemuholník je osovo súmerný a každá jeho uhlopriečka
je rovnobežná s~niektorou z~jeho strán.
[Os ktorejkoľvek strany sedemuholníka je jeho osou
súmernosti. Rovnobežnosť vybranej uhlopriečky s~vhodnou
stranou dokážte z~osovej súmernosti.]

\DOP
Majme ostrouhlý trojuholník $ABC$ so
štandardne označenými uhlami. Predpokladajme, že platí
$|AB|<|AC|$. Na strane $AC$ zvoľme bod $D$ tak, aby platilo
$|\uhel CBD|=\frac12(\beta-\gamma)$. Dokážte, že $|AB|+|CD|=|AC|$.
[Dokážte, že $|AB|=|AD|$. Na to stačí overiť rovnosť
$|\uhel ABD|=|\uhel ADB|$.]


V~obdĺžniku $ABCD$ platí $|AB|=3|BC|$
a na strane $CD$ je zvolený bod $E$ tak, že $|BC|=|CE|$.
Dokážte, že $|\uhel BAC|+|\uhel BAE|=45^\circ$.
\inspscno{b70i_5n.1}{0.8333}%
[Obdĺžnik $ABCD$ s~uhlopriečkou $AC$ zobrazte v~osovej
súmernosti podľa $AB$. Bod~$C$ sa zobrazí na $C'$. Dokážte, že
trojuholník $AC'E$ je rovnoramenný a pravouhlý.
\inspscno{b70i_5n.2}{0.8333}%
]

\endnávod
}

{%%%%%   B-I-6
V~prvej časti riešenia vysvetlíme, prečo v~každom štvorci
$4\times4$, ktorý je časťou daného plánu $12\times12$,
musia byť zasiahnuté výstrelom aspoň dve políčka.

Uvažujme ľubovoľný taký štvorec $4\times4$. Loď, ktorú v~ňom
umiestnime ako na \obr{} do ľavého horného rohu štvorca,
neohrozí výstrel na ktorékoľvek z~ôsmich políčok, ktoré sú na
obrázku označené bodkou.
\insp{b70i_6.1}%


Keď zopakujeme túto úvahu pre lode umiestnené do zvyšných troch rohov
daného štvorca $4\times4$, označenie bodkou zrejme získa všetkých jeho 16 políčok.
(Na to iba stačí \obrr1{} pootočiť v~jednom smere
o~$90^{\circ}$, $180^{\circ}$ a~$270^{\circ}$.)

Tvrdenie z~prvej vety nášho riešenia je tak dokázané. Doplňme ho
(len pre zaujímavosť) o~zrejmé konštatovanie,
že dvoma vhodnými výstrelmi na štvorec $4\times4$ už v~ňom
akokoľvek umiestnenú loď aspoň raz zasiahneme.\fnote{Opisu
všetkých vyhovujúcich dvojíc zásahov štvorca $4\times4$
venujeme dopĺňajúcu úlohu D2.}

Teraz sa už budeme venovať celému štvorcu $12\times12$.
Ten zrejmým spôsobom rozdelíme na $9$ neprekrývajúcich
sa štvorcov $4\times4$. Ako už vieme, v~každom z~nich
musíme vystreliť na aspoň dve políčka,
takže v~celom štvorci $12\times12$ musíme vystreliť na aspoň
$9\cdot2=18$ políčok. Aj keď vieme, že pre každý štvorec $4\times4$
dva výstrely stačia, záver o~tom, že 18~výstrelov stačí pre celý štvorec
$12\times12$, predchádzajúca úvaha ešte zďaleka nedokazuje!
Loď možno totiž umiestniť tak, aby neležala celá v~žiadnom z~deviatich
zostrojených štvorcov $4\times4$.

Spomenutú hypotézu o~dostatočnom počte 18 výstrelov je nutné
zdôvodniť, najjednoduchšie (jedným) príkladom vyhovujúcej
voľby 18 zasiahnutých políčok, ktorú možno nájsť \uv{cestou pokusov
a~omylov}. Takých príkladov je veľa --
na \obr{} sú uvedené tri z~nich, ktoré sú navyše niečím zaujímavé.
\inspthree{b70i_6.21}{\quad}{b70i_6.22}{\quad}{b70i_6.23}{0.6667}%

Výstrely v~prvom príklade sú súmerné podľa oboch diagonál, v druhom
a~treťom podľa stredu celého štvorca.
Navyše v~prvých dvoch príkladoch je bez zásahu polovica riadkov
aj polovica stĺpcov. V treťom príklade nie sú zasiahnuté žiadne dve
políčka so spoločnou stranou. Naopak príklad, keď zasiahnuté
políčka tvoria deväť dvojíc sa spoločnou stranou, dostaneme
jednoduchou úpravou obrázka uprostred: \uv{rušivú} dvojicu zásahov
stredového štvorca $2\times2$ v~ňom dáme pod seba.

\Pozn
Opíšme predsa len postup, ako príklad vyhovujúcej
voľby 18 výstrelov pomerne ľahko zostrojiť a~ako súčasne overiť
jeho správnosť bez toho, aby sme pri skúške museli s~loďou
\uv{cestovať} po celom pláne $12\times12$.

Nazvime {\it stredovým\/} políčkom (umiestnenej) lode to políčko,
ktoré leží uprostred štvorca $3\times3$, v~ktorom sa loď nachádza. Je
zrejmé, že stredovými políčkami lodí môžu byť práve políčka vnútorného
štvorca $10\times10$ daného plánu $12\times12$. Tento štvorec
označíme~$\Cal C$. Dve (rôzne) políčka plánu nazveme {\it
susedné}, ak majú spoločný aspoň jeden vrchol.


Pri vyhovujúcej voľbe zasiahnutých políčok nemôže žiadne z~nich, ktoré
leží vo štvorci~$\Cal C$, zostať \uv{osamotené}, \tj. musí sa
vystreliť aj na niektoré z~ôsmich s~ním susediacich políčok.
Obmedzme preto našu konštrukciu iba na tie voľby 18 výstrelov,
ktorými bude zasiahnutých 9~dvojíc susedných políčok vo štvorci
$\Cal C$. Na \obr{} sú uvedené príklady dvoch takých dvojíc.
Vždy keď sa bude jednať o~dvojicu políčok so spoločnou stranou,
zasiahnutá bude (aspoň raz) každá loď so stredovým
políčkom z~útvaru zhodného s~vykresleným útvarom $\Cal U$.
Pri dvojiciach políčok s~jediným spoločným vrcholom budú zasiahnuté
všetky lode so stredovým políčkom z~útvaru zhodného
s~útvarom~$\Cal V$.
\insppretwo{\raise7.5mm\hbox{$\Cal U$:}\quad}{b70i_6.31}{\quad\quad \raise7.5mm\hbox{$\Cal V$:}\quad}{b70i_6.32}{0.8333}%

Z~predchádzajúcich úvah vyplýva, že našou úlohou je
{\it rozmiestniť deväť útvarov} do celého
plánu $12\times12$ tak, aby každý z~nich bol zhodný
s~$\Cal U$ alebo $\Cal V$ a~aby v~ich zjednotení
ležal celý vnútorný štvorec $\Cal C$. Príklady takých
rozmiestnení, ktoré zodpovedajú trom príkladom, ktoré sme uviedli
v~závere riešenia na \obrr2, vidíte na \obr. Do jednotlivých útvarov sme prikreslili aj dvojice prislúchajúcich zásahov.
\inspthree{b70i_6.41}{\quad}{b70i_6.42}{\quad}{b70i_6.43}{0.6667}%



\návody
Súťažnú úlohu \uv{zmenšíme}. Vyriešte
postupne tri úlohy, v ktorých plán $12\times12$
nahradíme plánom a) $4\times4$, b) $5\times5$, c) $6\times6$.
[V~úlohách a) aj b) stačia dva výstrely, v~c) štyri výstrely.
Plán $6\times6$ rozdeľte na neprekrývajúce sa štvorce $3\times3$.]

\DOP
Na pláne s~rozmermi $6 \times 6$
štvorčekov sa nachádza loď tvaru štvorca $2\times2$. Zdôvodnite,
že treba najmenej $9$ výstrelov, aby sme mali istotu, že
sme loď zasiahli.
[Plán $6\times6$ rozdeľte na $9$ neprekrývajúcich sa štvorcov
$2\times2$.]

Určte, koľko je všetkých dvojíc políčok daného štvorca $4\times4$,
ktorých zásahom dosiahneme s~istotou aj zásah lode zo zadania súťažnej
úlohy, ktorá je v~tomto štvorci akokoľvek umiestnená.
[Je ich 34. Všetky vyhovujúce dvojice políčok rozdeľte do troch skupín
podľa toho, koľko je v~nich políčok vnútorného štvorca $2\times 2$ (políčka
$A$, $B$, $C$, $D$ na obrázku vľavo). Potom dokážte: V~jednej skupine sú
všetky dvojice políčok z~kvarteta ${\Cal K}=\{A,B,C,D\}$.
V druhej skupine sú všetky dvojice tvorené vždy jedným políčkom
$X\in\Cal K$ a jedným z~piatich políčok $Y\notin\Cal K$, ktoré majú
s~$X$ aspoň jeden spoločný vrchol -- pre políčko $X=A$ sú
na obrázku vľavo vyznačené bodkou. V tretej skupine sú práve také
dvojice políčok, ktoré sú na obrázku vpravo označené
rôznymi číslami jednej parity.
$$
\epsfboxsc{b70i_6.611}{0.8333}\qquad\qquad\epsfboxsc{b70i_6.612}{0.8333}
$$
Hľadaný počet je tak rovný $6+4\cdot5+2\cdot2\cdot2=34$.]

Dokážte, že podmienku súťažnej úlohy
{\it je nemožné\/} splniť tak, že celý plán $12\times12$ rozdelíme na 9
štvorcov $4\times4$ a potom v~každom z~nich vystrelíme na
dve políčka, a to na rovnakých dvoch miestach vo všetkých 9 štvorcoch.
[Označíme rovnakým z~písmen $A$, $B$ tie zasiahnuté políčka, ktoré
sú na rovnakom mieste vo všetkých deviatich štvorcoch. Políčka $A$ tak
sú rozmiestnené v~istom štvorci $9\times9$, ako vidíme na
obrázku vľavo. Pre políčka $B$ ho to podobne.
$$
\epsfboxsc{b70i_6.621}{0.8333}\qquad\qquad\epsfboxsc{b70i_6.622}{0.8333}
$$
Uvažujme loď v~polohe, pri ktorej obklopuje políčko $A$ v strede
štvorca $9\times9$. Aby táto loď bola zasiahnutá, musí v~niektorom z~ôsmich
jej políčok stáť $B$.
Ak sa jedná o~jedno zo štyroch modrých políčok označených bodkou, tak označenie bodkou majú na obrázku vľavo všetky políčka~$B$ z~nášho štvorca $9\times9$, v ktorom teda môžeme nájsť štvorce $3\times3$ bez zásahu, dokonca aj bez bodiek
-- jeden zo štyroch takých je na rovnakom obrázku vľavo vyznačený.
Ak je uvažovaná loď zasiahnutá v~jednom zo štyroch
políčok v~jej rohoch, môžeme bez ujmy na všeobecnosti
predpokladať, že sa jedná o~ľavý horný roh (inak stačí štvorec na
obrázku vľavo pootočiť). V~uvažovanom štvorci $9\times9$ sa potom
nachádzajú práve 4~políčka $B$ -- pozri obrázok vpravo, v ktorom
sú navyše vyznačené dva z~ôsmich štvorcov $3\times3$ bez zásahu.]

\MOarchiv58-B-I-4
[\Ulink{https://skmo.sk/dokument.php?id=29}{58-B-I-4}]

\rocnik=70
Na pláne $5\times 5$ hráme hru lode. Zo štyroch políčok plánu je vytvorená
jedna loď majúca niektorý z~tvarov na obrázku.
$$\epsfboxsc{b70i_6.64}{0.8}$$
Môžeme sa spýtať na ľubovoľné políčko plánu, a~ak loď
zasiahneme, hra končí.
a)~Navrhnite osem políčok, na ktoré sa stačí spýtať, aby sme mali istotu zásahu lode. b)~Zdôvodnite, že sedem otázok vo všeobecnosti takú istotu nedáva.
%\MOarchiv58-B-II-2
[\Ulink{https://skmo.sk/dokument.php?id=35}{58-B-II-2}]
\endnávod
}

{%%%%%   C-I-1
Obe čísla $m$ a $n$ sú evidentne dvojciferné.
Nech teda $m=10a+b$ a $n=10c+d$, pričom $a$, $b$, $c$, $d$ sú cifry desiatkovej
sústavy ($a\ne 0\ne c$). Podľa zadania má platiť
$$
\align
m+s(n) &= (10a+b)+(c+d)=70,\tag1\\
n+s(m) &= (10c+d)+(a+b)=70.\tag{2}
\endalign$$
Odčítaním druhej rovnice od prvej dostaneme $9(a-c)=0$, a teda $a=c$.
Zámenou $c$ za $a$ obe rovnice \ref[1.1] a \ref[1.2] prejdú na rovnakú
rovnicu $11a+(b+d)=70$.
Vzhľadom na to, že $0\leq b+d\leq 18$, získame pre $a$ odhady
$52\leq 11a\leq 70$. Z toho bezprostredne vyplýva, že buď $a=5$,
alebo $a=6$. V~prvom prípade (pre $a=c=5$) potom z~podmienky $11a+(b+d)=70$
dostaneme $b+d=15$, \tj. $(b,d)\in\{(6,9),(7,8),(8,7),(9,6)\}$.
V druhom prípade (pre $a=c=6$) dostávame $b+d=4$, a~teda
$(b,d)\in\{(0,4),(1,3),(2,2),(3,1),(4,0)\}$. Skúška nie je nutná,
lebo máme zaručenú platnosť oboch rovníc \ref[1.1] a \ref[1.2].

\Zav
Úloha má 9 riešení, ktorými sú dvojice
$(m,n)\in\{(56,59),(57,58),(58,57),\penalty0(59,56),(60,64),(61,63),(62,62),
(63,61),(64,60)\}$.


\návody
Určte všetky dvojciferné čísla, ktoré sú rovné trojnásobku
svojho ciferného súčtu.
[Označme $n=\overline{ab}$, potom $n=10a+b=3(a+b)$, \tj. $7a=2b$.
Keďže čísla 2 a 7 sú nesúdeliteľné, je $b=7$ a $a=2$ ($a\ne 0$),
a teda $n=27$.]

Určte všetky dvojciferné čísla, ktoré sú rovné súčtu
svojej desiatkovej cifry a~druhej mocniny jednotkovej cifry.
[Pre hľadané číslo $n=\overline{ab}$ platí $n=10a+b=a+b^2$,
\tj. $9a=b(b-1)$. Vzhľadom na to, že čísla $b$ a $b-1$ sú nesúdeliteľné,
je buď $9\,|\,b$, alebo $9\,|\,(b-1)$. Zadaniu úlohy vyhovuje iba $n=89$.]

\MOarchiv69-C-S-1
[\Ulink{https://skmo.sk/dokument.php?id=3392}{69-C-S-1}]

V~roku 2000 Alena zistila, že jej vek je rovný súčtu
cifier roku, v~ktorom sa narodila. Koľko má rokov v~roku 2020?
[Súčet cifier roku jej narodenia je rovný nanajvýš 28 ($=1+9+9+9$).
Rok jej narodenia je teda štvorciferné číslo tvaru
$\overline{19xy}$, pre ktoré platí $2000-\overline{19xy}=1+9+x+y$.
Z toho po úprave máme $90=11x+2y$, odkiaľ zrejme $x=8$ a $y=1$.
Vek Aleny v~roku 2020 je teda 39 rokov.]

\DOP
Určte všetky štvorciferné čísla, ktoré sú štyrikrát menšie
ako číslo napísané rovnakými ciframi, avšak v~opačnom poradí.
[Označme $\overline{abcd}$ hľadané číslo. Potom platí
$4\cdot \overline{abcd}=\overline{dcba}$. Cifra $a$ musí byť párna
a súčasne $a\leq 2$. Teda $a=2$, a teda $d=8$. Jediným riešením je potom
číslo $2178$.]

\MOarchiv69-C-I-1
[\Ulink{https://skmo.sk/dokument.php?id=3388}{69-C-I-1}]

\endnávod
}

{%%%%%   C-I-2
Zvoľme ľubovoľný riadok (príp. stĺpec) štvorcovej
tabuľky $n\times n$, ktorá je vyplnená číslami $2$ a ${-1}$.
Označme v~ňom $d$ počet čísel $2$ a $p$ počet čísel ${-1}$.
Potom $d+p=n$ a~podľa zadania úlohy má platiť tiež $2d-p=0$.
Sčítaním týchto dvoch rovností dostaneme $3d=n$,
čo znamená, že číslo $n$ je {\it nutne\/} deliteľné tromi, \tj. $n=3k$,
pričom $k$ je prirodzené číslo.

Ukážeme, že táto podmienka je aj postačujúca. Pre ľubovoľné $n=3k$
rozdelíme tabuľku $n\times n$ zrejmým spôsobom na $k^2$ menších
štvorcových tabuliek $3\times 3$.
Ľahko sa vidí, že každú z~týchto menších tabuliek $3\times 3$ možno
vyplniť požadovaným spôsobom -- napríklad tak, že čísla $2$ napíšeme do
všetkých troch políčok jednej z~jej diagonál a do šiestich zvyšných políčok
tejto tabuľky napíšeme čísla ${-1}$. Celá štvorcová tabuľka $n\times n$ potom
spĺňa podmienku úlohy, keďže ju spĺňa každá z~$k^2$
vytvorených tabuliek $3\times3$.

\Zav
Daným spôsobom možno vyplniť práve tie tabuľky $n\times n$,
kde $n$ je prirodzené číslo, ktoré je deliteľné tromi.


\návody
Rozhodnite, či možno štvorcovú tabuľku $3\times 3$ vyplniť
prirodzenými číslami od 1 do 9 tak, aby každé číslo bolo použité práve
raz a súčet všetkých čísel v~každom riadku a v~každom stĺpci bol
deliteľný a) dvoma, b) troma.
[a) Nedá sa -- ak by súčet čísel už len v~každom riadku tabuľky
bol deliteľný dvoma, bol by aj súčet všetkých čísel v~tabuľke deliteľný dvoma.
Ich súčet je však rovný 45. b) Áno -- ľahko možno nájsť konkrétny príklad.]

Rozhodnite, pre ktoré prirodzené čísla $n$ možno štvorcovú
tabuľku $n\times n$ vyplniť číslami $1$ a ${-1}$ tak, aby súčet všetkých
čísel v~každom riadku a v~každom stĺpci bol rovný $0$.
[Pre každé párne číslo $n$ možno tabuľku $n\times n$ požadovaným spôsobom vyplniť
podľa vzoru čierno-bielej šachovnice. Pre nepárne $n$ sa to nedá, lebo
v~žiadnom riadku (stĺpci) nemôže byť rovnaký počet čísel ${-1}$ a $1$.]

Rozhodnite, či možno štvorcovú tabuľku $3\times 3$
vyplniť prirodzenými číslami od 1 do 9 tak, aby každé číslo bolo použité práve
raz a aby súčet všetkých čísel v~každom riadku, v~každom stĺpci
aj na oboch uhlopriečkach bol rovnaký (dostaneme tzv. {\it magický štvorec}).
[Keďže súčet daných deviatich čísel je rovný 45,
musí byť súčet všetkých čísel v~každom riadku a v~každom stĺpci rovný
$45:3=15$, čo je nepárne číslo. Nepárnych čísel je v~tabuľke
o~jedno viac ako párnych. Navyše $1+9=2+8=3+7=4+6=10$.
Ak je teda v~prostrednom políčku tabuľky číslo 5, ľahko
nájdeme konkrétny príklad možného vyplnenia tejto tabuľky.]

\DOP
\MOarchiv69-C-S-2
[\Ulink{https://skmo.sk/dokument.php?id=3392}{69-C-S-2}]

Koľkými spôsobmi možno do políčok tabuľky $2\times 3$ vpísať
čísla $\{1,2,3,4,5,6\}$ tak, aby každé bolo použité práve raz a aby
súčet všetkých čísel v~každom riadku a~v~každom stĺpci bol deliteľný troma?
[Matematický klokan 2018, kat. Junior, úloha 21 (48~možností).]

\endnávod
}

{%%%%%   C-I-3
Označme $V$, $W$ dotykové body kružnice vpísanej
trojuholníku $ABC$ postupne so stranami $AC$, $AB$ a $\rho$ jej polomer.
Ľahko sa vidí, že štvoruholník $CVIU$ je štvorec so
stranou dĺžky $\rho$. Z~osových súmerností podľa priamok $AI$ a
$BI$ tak pri zvyčajnom označení dĺžok odvesien trojuholníka $ABC$
vyplýva
$$
|AW|=|AV|=b-\rho\quad\hbox{a}\quad
|BW|=|BU|=a-\rho.
$$
Z toho dostávame vyjadrenie dĺžky prepony $AB$ v tvare
$$
|AB|=|AW|+|BW|=a+b-2\rho,
$$
takže podľa Pytagorovej vety platí rovnosť
$a^2+b^2=(a+b-2\rho)^2$.
\inspsc{svrcek69.1}{0.8333}%

Až teraz využijeme podmienku úlohy, podľa ktorej sa pravouhlé
trojuholníky $ACU$ a $BUI$ zhodujú v~ostrých uhloch pri vrcholoch $A$ a $B$
(vyznačených na \obr). Keďže sa (vo všeobecnosti) zhodujú
aj v~odvesnách $CU$ a $UI$, v~našej situácii sa jedná (podľa vety {\it
usu}) o~dva zhodné trojuholníky. To nás vedie k~rovnosti $|AC|=|BU|$,
čiže $b={a-\rho}$, teda $\rho=a-b$. Dosadením takého $\rho$
do vyššie upravenej rovnosti z~Pytagorovej vety dostaneme
$$
a^2+b^2=(3b-a)^2.
$$
Po roznásobení a jednoduchej úprave vyjde $b(4b-3a)=0$, odkiaľ $3a=4b$.

\Zav
Pre hľadaný pomer platí $|AC|:|BC|=b:a=3:4$.


\návody
Nech $D$, $E$, $F$ sú dotykové body kružnice
vpísanej trojuholníku $ABC$ postupne so stranami $BC$, $CA$, $AB$. Pomocou
ich dĺžok $a$, $b$, $c$ vyjadrite dĺžky úsekov, na ktoré
body $D$, $E$, $F$ rozdeľujú jednotlivé strany.
[Platí $|AE|=|AF|=s-a$, $|BF|=|BD|=s-b$, $|CD|=|CE|=s-c$, pričom $2s=a+b+c$.]

\MOarchiv61-C-I-2
[\Ulink{https://skmo.sk/dokument.php?id=453}{61-C-I-2}]

\MOarchiv61-C-S-2
[\Ulink{https://skmo.sk/dokument.php?id=457}{61-C-S-2}]

\MOarchiv69-C-S-3
[\Ulink{https://skmo.sk/dokument.php?id=3392}{69-C-S-3}]

\DOP
Daný je pravouhlý trojuholník $ABC$ s~preponou $AB$.
Kolmým priemetom kružnice vpísanej danému trojuholníku na preponu $AB$
je úsečka $MN$. Dokážte, že stred kružnice vpísanej trojuholníku $ABC$
je súčasne stredom kružnice opísanej trojuholníku $MNC$.
[Označme $I$ stred kružnice vpísanej a $U$, $V$, $W$
postupne jeho kolmé priemety na strany $BC$, $CA$, $AB$.
Trojuholníky $IMW$, $INW$ a $IVC$ sú zrejme zhodné rovnoramenné
pravouhlé trojuholníky (s~odvesnami veľkosti~$\varrho$ polomeru
vpísanej kružnice, lebo $|MN| =2\varrho$). Preto
$|IM|=|IN|=|IC|$.]

\endnávod
}

{%%%%%   C-I-4
Zlomky v~danom výraze majú zmysel, pretože ich
menovatele sú podľa zadania kladné čísla. Vďaka podmienke $a+b+c=1$ pre
prvý zlomok platí
$$\frac{a+bc}{a+b}=\frac{(a+b)+(bc-b)}{a+b}=1-b\cdot\frac{1-c}{a+b}=
1-b\cdot\frac{a+b}{a+b}=1-b.$$
Analogicky druhý, resp. tretí zlomok nadobúda postupne hodnoty $1-c$,
resp. $1-a$. Z~toho vyplýva
$$\frac{a+bc}{a+b}+\frac{b+ca}{b+c}+\frac{c+ab}{c+a}=(1-b)+(1-c)+(1-a)=3-(a+b+c)=2,$$
čo je za podmienok úlohy jediná možná hodnota daného výrazu.


\návody
Nech $a$, $b$, $c$ sú nenulové reálne čísla, ktorých súčet
je rovný 0. Určte, aké hodnoty môže nadobúdať výraz
$$\frac{(a-b)^2+(b-c)^2+(c-a)^2}{3(ab+bc+ca)}.$$
[Využite identitu $a^2+b^2+c^2=(a+b+c)^2-2(ab+bc+ca)$.
Hodnota daného výrazu je rovná ${-2}$.]

Nech $a$, $b$, $c$ sú nenulové reálne čísla, ktorých súčet
je rovný 0. Dokážte, že platí
$$\frac{1}{a+b}+\frac{1}{b+c}+\frac{1}{c+a}=\frac{a^2+b^2+c^2}{2abc}.$$
[Menovatele zlomkov na ľavej strane nahraďte číslami $\m c$,
$\m a$, $\m b$ a po sčítaní takto upravených zlomkov uplatnite
rovnakú identitu ako pri riešení N1.]

Nech $x$, $y$, $z$ sú kladné reálne čísla, ktorých
súčin je rovný 1. Dokážte, že platí rovnosť
$$\frac{1}{1+x+xy}+\frac{1}{1+y+yz}+\frac{1}{1+z+zx}=1.$$
[Prvý zlomok rozšírte $z$, druhý $xz$ a trikrát využite
podmienku $xyz=1$.]

\DOP
Ak reálne čísla $a$, $b$, $c$ spĺňajú rovnicu
$$a^3+b^3+c^3-3abc=0,$$
tak platí $a+b+c=0$ alebo $a=b=c$. Dokážte.
[18--B--I--1]

Určte najmenšiu možnú hodnotu výrazu $(1+a_1)(1+a_2)(1+a_3)$, ak sú $a_1$, $a_2$, $a_3$ kladné reálne čísla, ktorých súčin je rovný 1. [Dokážte a potom medzi sebou vynásobte nerovnosti $1+a_i\ge2\sqrt{a_i}$ pre $i=1,2,3$.]

\MOarchiv69-C-II-4
[\Ulink{http://www.matematickaolympiada.cz/media/6510497/c69-ii.pdf}{69-C-II-4}]

\endnávod
}

{%%%%%   C-I-5
Označme $S_a$, $S_b$ postupne stredy strán $BC$, $AC$
daného trojuholníka $ABC$ a $X$, $Y$ postupne priesečníky dvojíc
úsečiek $EF$ a $BC$, $EF$ a $AC$.
Napokon, nech $P$ je priesečník uhlopriečok rovnobežníka $TECF$.
Zhodnosť a rovnobežnosť jeho protiľahlých strán je vyznačená na
\obr.
\inspsc{svrcek65.1}{0.8333}%


Keďže $S_a$ je stredom strany $BC$ a $AE\parallel CF$, je
úsečka $TS_a$ strednou priečkou v~trojuholníku $BCF$ a jej dĺžka je
polovicou dĺžky úsečky $CF$, teda aj polovicou dĺžky úsečky $TE$.
Bod $S_a$ je teda stredom strany $TE$ v~trojuholníku $TEC$.
Z~toho vyplýva, že bod $X$ je jeho ťažiskom, lebo úsečky $CS_a$
a $EP$ sú jeho ťažnicami, ktoré sa pretínajú práve v~bode $X$.
Analogicky $Y$ je ťažiskom trojuholníka $CFT$, ktorý je navyše
súmerne združený s~trojuholníkom $TEC$ podľa stredu $S$. Z~vlastností dĺžok úsekov,
ktoré vytínajú ťažiská $X$ a $Y$ na ťažniciach $EP$ a $FP$ v zhodných
trojuholníkoch $TEC$ a $CFT$, už bezprostredne vyplýva tvrdenie úlohy.


\návody
\MOarchiv68-C-S-3
[\Ulink{https://skmo.sk/dokument.php?id=3047}{68-C-S-3}]

\MOarchiv68-C-I-2
[\Ulink{https://skmo.sk/dokument.php?id=3043}{68-C-I-2}]

\MOarchiv68-A-J-2
[8. CPS MO juniorov (2019). Pre dôkaz, že $P$ je stred
$BM$, uvažujte strednú priečku $BS$
v~trojuholníku $ADT$. Úsečka~$TP$ je strednou priečkou
v~trojuholníku $BMS$.]

\endnávod
}

{%%%%%   C-I-6
Každé číslo, ktoré je napísané na tabuli, má
jednoznačne určený prvočíselný rozklad
$n=2^a\cdot3^b\cdot5^c\cdot7^d$, pričom $a$, $b$, $c$, $d$
sú vhodné celé nezáporné čísla. Podľa parít čísel v štvorici
$(a,b,c,d)$ zavedieme \uv{typ} čísla $n$ ako štvoricu $(A,B,C,D)$, pričom každé
z~písmen je buď písmeno $P$ (znak pre párne číslo), alebo písmeno $N$
(znak pre nepárne číslo).
Teda napríklad číslo 60 s~rozkladom $2^2\cdot3^1\cdot5^1\cdot7^0$ je typu
$(P,N,N,P)$.

Uvedomme si, že prirodzené číslo je druhou mocninou práve vtedy, keď
v~jeho rozklade na súčin prvočísel má každé prvočíslo párny počet zastúpení.
Také zastúpenie zrejme majú prvočísla v~rozklade súčinu $A\cdot B$
práve tých čísel $A$ a $B$ z~tabule, ktoré sú toho istého typu.
Preto je podmienka úlohy pre súčiny dvojíc čísel splnená
práve vtedy, keď každé dve čísla napísané na tabuli majú rôzny typ.

Obe zadané úlohy budeme riešiť súčasne. V~prvej z~nich máme
zistiť, aký najväčší {\it počet\/} čísel môže byť na tabuli napísaný.
Odpoveď je podľa predchádzajúceho odseku rovná počtu tých zo
zavedených typov $(A,B,C,D)$, ktoré majú medzi číslami od 1 do 100
svoje zastúpenie. Budú to zrejme práve tie typy, ktorých
{\it najmenší zástupca\/} je číslo, ktoré neprevyšuje 100.

Všetkých typov $(A,B,C,D)$ je zrejme $2^4=16$.
Pre každý z~nich preto najskôr určíme {\it najmenšie\/}
číslo $k$~tohto typu. Bude ním určite číslo
$k=2^a\cdot3^b\cdot5^c\cdot7^d$, pričom najmenšie možné
mocnitele $a,b,c,d\in\{0,1\}$ vyberieme podľa štvorice znakov
$(A,B,C,D)$ -- pozri prvé dva stĺpce nižšie uvedenej tabuľky.
Chýbajú v~nej typy $(P,N,N,N)$ a~$(N,N,N,N)$, lebo pre ne sú
najmenší zástupcovia $k=3\cdot5\cdot7=105$ a~$k=2\cdot3\cdot5\cdot7=210$ väčší ako 100. Zisťujeme tak, že
najväčší možný počet čísel na tabuli je rovný $16-2=14$.
(Príkladom je 14 hodnôt~$k$ z~tabuľky.)
$$
\alignedat3
&(P,P,P,P) \qquad& k&=1 & K&=10^2=100 \\
&(N,P,P,P) & k&=2 & K&=2\cdot 7^2=98 \\
&(P,N,P,P) & k&=3 & K&=3\cdot5^2=75 \\
&(P,P,N,P) & k&=5 & K&=5\cdot4^2=80 \\
&(P,P,P,N) & k&=7 & K&=7\cdot3^2=63 \\
&(N,N,P,P) & k&=2\cdot 3=6 & K&=6\cdot4^2=96 \\
&(N,P,N,P) & k&=2\cdot 5=10 & K&=10\cdot3^2=90 \\
&(N,P,P,N) & k&=2\cdot 7=14 & K&=14\cdot2^2=56 \\
&(P,N,N,P) & k&=3\cdot 5=15 & K&=15\cdot2^2=60 \\
&(P,N,P,N) & k&=3\cdot 7=21 & K&=21\cdot2^2=84 \\
&(P,P,N,N) & k&=5\cdot 7=35 & K&=35 \\
&(N,N,N,P) & k&=2\cdot 3\cdot 5=30 & K&=30 \\
&(N,N,P,N) & k&=2\cdot 3\cdot 7=42 & K&=42 \\
&(N,P,N,N) & k&=2\cdot 5\cdot 7=70\qquad & K&=70 \\
\endalignedat
$$

Druhou úlohou je určenie najväčšieho možného {\it súčtu\/} čísel
na tabuli. Ten zrejme dostaneme tak, že ku každému zo 14 možných
typov nájdeme jeho najväčšieho zástupcu~$K$ medzi číslami od 1 do 100,
a týchto 14 hodnôt $K$ potom sčítame. Pri každom type však už poznáme jeho
najmenšieho zástupcu $k$, takže prvých desať\fnote{Jedenáste
číslo $12^2\cdot k$ nebudeme v~našom riešení potrebovať pre žiadny
zo 14 možných typov, vlastne už $10^2\cdot k$ je pre každé $k>1$
väčšie ako 100.} najmenších čísel
tohto typu má určite tvar
$$
k,\, 2^2\cdot k,\, 3^2\cdot k,\, 4^2\cdot k,\,\dots,\,10^2\cdot k.
$$
Číslo $K$ teda určíme ako najväčšie z~nich, ktoré ešte
neprevyšuje 100, a potom ho pre každý typ zapíšeme do tretieho stĺpca
tabuľky.

Ako sme už uviedli skôr, v~tabuľke je uvedených iba 14 zo 16
možných typov. Čísla zvyšných dvoch typov $(P,N,N,N)$ a $(N,N,N,N)$
na tabuli napísané byť nemôžu, lebo pre
súčin troch nutne zastúpených prvočísel 3, 5 a 7 platí
$3\cdot5\cdot7=105>100$. Odpoveď na časť a) úlohy tak znie:
Najväčší možný {\it počet\/} čísel napísaných na tabuli je rovný 14.

Pre čísla napísané na tabuli dosiahneme najväčší súčet, ak
sčítame {\it najväčšie\/} prípustné čísla z~každého zo 14 možných typov,
ktoré sme zapísali do pravého stĺpca tabuľky.
Z~toho vyplýva odpoveď na časť b) úlohy:
Najväčší možný {\it súčet\/} na tabuli napísaných čísel je
$$
100+98+75+80+63+96+90+56+60+84+35+30+42+70=979.
$$

\návody
\MOarchiv68-C-S-1
[\Ulink{https://skmo.sk/dokument.php?id=3047}{68-C-S-1}]

\MOarchiv68-C-II-2
[\Ulink{https://skmo.sk/dokument.php?id=3126}{68-C-II-2}]

\endnávod
}

{%%%%%   A-S-1
Bod~$F$ ako stred kružnice opísanej trojuholníku $ADE$ leží na osiach
jeho strán $AD$ a~$AE$. Os úsečky $AD$ vďaka rovnosti
$|CA|=|CD|$ prechádza bodom $C$ a je pritom kolmá na priamku $AB$,
takže je to priamka, na ktorej leží výška trojuholníka $ABC$ z~vrcholu~$C$ (\obr). Podobne z~rovnosti $|BE|=|BA|$ vyplýva, že na osi úsečky~$AE$
leží výška trojuholníka $ABC$ z~vrcholu~$B$. Spolu to znamená,
že spoločný bod~$F$ oboch osí je priesečníkom dvoch výšok trojuholníka $ABC$, a
preto ním prechádza aj jeho tretia výška z~vrcholu~$A$. Platí teda
$AF\perp BC$, ako sme mali dokázať.
\inspsc{a70s.1}{0.8333}%

\ineriesenie
Úlohu vyriešime, keď pri zvyčajnom označení $\beta=|\uhel ABC|$
dokážeme, že uhol $DAF$ má veľkosť $90^\circ-\beta$. To vďaka
rovnoramennému trojuholníku $ADF$ nastane práve vtedy, keď stredový uhol
$AFD$ bude mať veľkosť $2\beta$, čiže práve vtedy, keď obvodový uhol
$AED$ bude mať veľkosť $\beta$. Posledné je však ekvivalentné s tým,
že štvoruholník $BCED$ je tetivový. Overiť túto vlastnosť je už jednoduché:
Z rovnoramenných trojuholníkov $AEB$ a~$ADC$ vidíme, že veľkosť
$\alpha=|\uhel BAC|$ majú aj uhly $AEB$ a~$ADC$, takže úsečku $BC$ je z~bodov~$D$ a~$E$ vidno pod tým istým uhlom $180^\circ-\alpha$.

\nobreak\medskip\petit\noindent
Za úplné riešenie dajte 6~bodov, z toho: 2 body za dôkaz $BF\perp AC$,
2 body za dôkaz $CF\perp AB$ (jeden z dôkazov možno prehlásiť za analógiu
druhého) a 2 body za záver o~tretej výške trojuholníka. Len za hypotézu, že bod~$F$ je
priesečník výšok trojuholníka $ABC$, dajte 2 body.

Pri postupe, ktorý nepracuje s výškami trojuholníka $ABC$,
dajte 2~body za dôkaz, že $BCED$ je tetivový štvoruholník.
\endpetit
}

{%%%%%   A-S-2
Prvočísla na tabuli označíme~$p_1,p_2,\dots,p_n$.
Podľa zadania platí
$$
2020\cdot(p_1+p_2+\dots+p_n)=p_1p_2\dots p_n.
\tag1$$
Ľavá strana rovnosti~\thetag1 je deliteľná číslom
$2020=2\cdot2\cdot5\cdot101$. V~prvočíselnom rozklade na pravej strane~\thetag1 preto nutne vystupujú prvočísla 2, 2, 5, 101. Bez ujmy na
všeobecnosti predpokladajme, že $p_1=2$, $p_2=2$, $p_3=5$, $p_4=101$.
Keby na tabuli boli iba tieto 4~prvočísla (prípad $n=4$),
tak by rovnosť~\thetag1 zrejme neplatila ($2020 \cdot 110 \ne 2020$).
Nutne teda musí platiť $n\geqq 5$.

Dosadením už známych prvočísel $p_1=2$, $p_2=2$, $p_3=5$,
$p_4=101$ do rovnice~\thetag1 dostaneme po jej vydelení číslom~2020
rovnicu
$$
110+p_5+p_6+\ldots+p_n = p_5 \cdot p_6\cdot\dots\cdot p_n.
\tag2$$
Keďže hľadáme najmenšie možné $n\geqq5$, pri ktorom prvočísla
$p_5,p_6,\dots,p_n$ spĺňajúce rovnicu~\thetag2 existujú,
rozoberieme ďalej postupne najmenšie do úvahy prichádzajúce hodnoty
$n=5,6,\dots$, kým prvé vyhovujúce $n$ nenájdeme.

\item{$\triangleright$}{\it Prípad $n=5$}.
Rovnica~\ref[rov2.2] je tvaru $110+p_5=p_5$, takže ju
nespĺňa žiadne prvočíslo~$p_5$.

\item{$\triangleright$}{\it Prípad $n=6$}.
Príslušnú rovnicu
$$
p_5p_6=p_5+p_6+110
\tag3$$
upravíme na súčinový tvar
$$
(p_5-1)(p_6-1)=111.
$$
Oba činitele na ľavej strane sú tak nutne čísla nepárne,
takže čísla $p_5$ a $p_6$ sú obe párne.\fnote{Tento záver
pritom vyplýva aj priamo z~rovnice~\thetag3, podľa ktorej celé čísla
$p_5p_6$ a $p_5+p_6$ majú rovnakú paritu. Nastane
to jedine vtedy, keď sa jedná o~súčin a súčet dvoch párnych čísel.}
Jediné párne prvočísla $p_5=p_6=2$ však rovnici~\thetag3
nevyhovujú.

\item{$\triangleright$}{\it Prípad $n=7$}.
Rovnica~\thetag2 má tvar
$$
p_5p_6p_7=p_5+p_6+p_7+110.
\tag4$$
Zdôraznime, že našou úlohou je iba posúdiť, či niektorá
trojica prvočísel rovnici~\thetag4 vyhovuje. Uhádnuť trojicu
$p_5=p_6=p_7=5$ nie je ťažké:
$$
5^3=125=5+5+5+110.
$$
Tým je celé riešenie úlohy ukončené.
Dodajme ešte, že ďalšie
trojice prvočísel vyhovujúce rovnici~\thetag4 sú $(2,5,13)$ a
$(2,3,23)$.\fnote{Iné vyhovujúce trojice neexistujú, pozri
poznámky nižšie.} Preto v~prípade, že trojicu $(5,5,5)$ neuhádneme,
možno riešenie dokončiť tak, že do rovnice~\thetag4 skusmo
dosadíme najmenšie existujúce prvočíslo $p_5=2$. Pre prvočísla
$p_6$ a $p_7$ tak dostaneme rovnicu, pri ktorej po prevode na
súčinový tvar $(2p_6-1)(2p_7-1)=225$ určíme dokonca dve
prvočíselné riešenia $(3,23)$ a~$(5,13)$.

Zhrňme výsledok našich úvah:
Na tabuli bolo napísaných aspoň 7~prvočísel, príkladom možnej sedmice
(usporiadanej vzostupne) je $2,2,5,5,5,5,101$.

\zaver Najmenší možný počet prvočísel na tabuli je rovný 7.

\poznamky
Naznačme pre zaujímavosť, ako vyriešiť rovnicu~\ref[rov2.4]
úplne. Prvý krok sme už v~texte učinili popisom, ako vyriešiť
prípad, keď jedno z troch neznámych prvočísel je~2.
Podobne možno vyriešiť prípad, keď jedno z~prvočísel je 3,
v ktorom však dostaneme už iba skôr objevené riešenie~$(2,3,23)$
s~prvočíslom~2.

Ostáva posúdiť prípad, keď všetky tri prvočísla~$p_5$, $p_6$ a~$p_7$
sú aspoň~5. Pri vhodnom označení, keď $p_7$ je najväčšie z~nich,
platí
$$
5 \cdot 5 \cdot p_7 \leqq p_5p_6p_7 =p_5+p_6+p_7+110\leqq
3p_7+110,
$$
odkiaľ vyplýva $p_7\leqq5$. Nutne tak platí $p_5=p_6=p_7=5$, čo
je posledné hľadané riešenie.

Dá sa dokázať, že počet siedmich prvočísel na tabuli je jediný možný.

\nobreak\medskip\petit\noindent
Za úplné riešenie dajte 6~bodov, z~toho: 1~bod
za zdôvodnenie, že na tabuli je nutne štvorica prvočísel $(2,2,5,101)$;
1~bod za zdôvodnenie, že ich nemôže byť celkom~5; 2~body za
zdôvodnenie, že ich nemôže byť celkom~6; posledné 2~body za uvedenie
príkladu, že ich môže byť celkom~7. V~závere riešenia ale nie je
nutné vypisovať prvú určenú štvoricu spolu s~druhou
akokoľvek získanou trojicou, lebo zadanie úlohy to nevyžaduje.
\endpetit
}

{%%%%%   A-S-3
Úloha je v~premenných $a$, $b$, $c$ symetrická:
zmenou ich poradia dôjde iba ku zmene poradia skúmaných šiestich
čísel. V~prvej časti riešenia preto budeme predpokladať, že platí $a<b<c$.
Vďaka nezápornosti čísel $a$, $b$, $c$ potom platí tiež
$a^2<b^2<c^2$.

Z vypísaných nerovností ľahko vyplýva, že prvá trojica čísel $a+b$, $b+c$, $c+a$ a~druhá trojica čísel $a^2+b^2$, $b^2+c^2$,
$c^2+a^2$ sú usporiadané nasledovne:
$$
a+b<a+c<b+c\qquad\hbox{a}\qquad a^2+b^2<a^2+c^2<b^2+c^2.
\tag1$$
Vidíme, že medzi skúmanými šiestimi číslami sú vždy aspoň tri rôzne.
Ukážeme teraz, že len tri rôzne čísla to byť nemôžu.
Inak by totiž podľa~\thetag1 museli platiť rovnosti
$$
\eqalign{
a+b&=a^2+b^2,\cr
a+c&=a^2+c^2, \cr
b+c&=b^2+c^2.} \tag2
$$
Odčítaním prvej rovnosti od druhej, resp. od tretej rovnosti by sme
dostali
$$
\eqalign{
c-b&=(c-b)(c+b), \cr
c-a&= (c-a)(c+a).}
$$
Z toho po vydelení kladnými číslami $c-b$, resp. $c-a$ vyplýva,
že obe čísla $c+b$ a $c+a$ by sa rovnali 1, čo odporuje
nerovnostiam~\thetag1. Dokázali sme tak, že medzi skúmanými
šiestimi číslami sú vždy aspoň štyri rôzne.

\medskip
V~druhej časti riešenia nájdeme trojicu rôznych nezáporných
čísel $a$, $b$, $c$ tak, aby medzi skúmanými šiestimi číslami boli iba
štyri rôzne.

Podľa postupu z~prvej časti sa vyplatí voľbu čísel $a$, $b$, $c$
začať tak, aby bola splnená napríklad rovnosť $b+c=b^2+c^2$.
Jednoduchý spôsob, ako to dosiahnuť, je položiť $b=1$ a~$c=0$.
Potom bude šestica čísel v~poradí zo zadania úlohy vyzerať takto:
$$
a+1,\ 1,\ a,\ a^2+1,\ 1,\ a^2.
$$
Za nezáporné číslo $a$ už ale nemôžeme voliť ani 0, ani 1.
Nehodí sa ani žiadne $a$ medzi 0 a 1, lebo pre ne je v~šestici
päť rôznych čísel $a^2<a<1<a^2+1<a+1$.

Musíme teda voliť $a>1$. Vtedy bude platiť
$$
1<a<a^2<a^2+1\qquad\hbox{a zároveň}\qquad a<a+1<a^2+1.
$$
Vidíme, že v~našej šestici budú iba štyri rôzne čísla práve vtedy, keď
číslo $a>1$ bude spĺňať podmienku $a^2=a+1$. Jednoduchým výpočtom
zistíme, že sa jedná o~číslo
$$
a=\frac{1+\sqrt5}{2},
$$
známe pod názvom {\it zlatý rez\/} nielen v~matematike, ale aj v~umení.

\Zav
Najmenší možný počet rôznych čísel v~zadanej
šestici je rovný 4.

\Jres
Opíšme jeden z~ďalších spôsobov hľadania
trojíc $(a,b,c)$, pre ktoré sú v~zadanej šestici iba štyri rôzne
čísla. Taká situácia určite nastane, ak budú
splnené dve z troch rovníc sústavy~\thetag2. Vyberme si rovnice
$$
a+b=a^2+b^2\qquad\hbox{a}\qquad a+c=a^2+c^2
$$
a vzhľadom na symetriu predpokladajme, že $b<c$. Ak odčítame tieto
dve rovnice od seba, zistíme, že musí platiť $b+c=1$ (pozri text
prvého riešenia). Z toho vzhľadom na $b<c$ máme $b=\frac12-t$ a
$c=\frac12+t$, pričom $t\in(0,\frac12\rangle$.
Pri takej voľbe čísel $b$ a $c$ ďalej stačí
uvažovať iba prvú rovnicu $a+b=a^2+b^2$, z ktorej po dosadení
$b=\frac12-t$ vyjadríme kladné číslo $a$ pomocou parametra $t$.
Zapíšme ho kvôli prehľadnosti rovno na jeden riadok spolu
s~vyjadreniami čísel $b$ a $c$:
$$
a=\frac{1+\sqrt{2-4t^2}}{2},\quad
b=\frac12-t,\quad c=\frac12+t.
$$
Z možných hodnôt $t\in(0,\frac12\rangle$ sa nehodí iba $t=\frac12$, keď
vyjde $a=c=1$. Každá hodnota $t\in(0,\frac12)$ totiž určuje
trojicu rôznych kladných čísel $a$, $b$ a $c$, lebo pre ne
zrejme platí $0<b<c<1<a$.

Dodajme, že existenciu nájdenej nekonečnej množiny trojíc $(a,b,c)$
možno potvrdiť aj bez výpočtov. Stačí len dotyčnú dvojicu rovníc
$a+b=a^2+b^2$ a $a+c=a^2+c^2$ upraviť na tvar
$a(a-1)=b(1-b)=c(1-c)$ a využiť tvar paraboly, ktorý je grafom
funkcie $f(x)=x(1-x)$. Z \obr{} vidíme, že podmienke $f(b)=f(c)=\m f(a)$
naozaj vyhovuje nekonečne veľa trojíc
vždy navzájom rôznych nezáporných čísel $a$, $b$, $c$.
\inspsc{a70s.2}{0.8333}%

\nobreak\medskip\petit\noindent
Za úplné riešenie dajte 6~bodov, z~toho:
3~body za dôkaz, že medzi šiestimi skúmanými číslami sú vždy aspoň
4~rôzne; 3~body za príklad situácie, keď sú rôzne práve 4.

V~prípade neúplného dôkazu tvrdenia, že sa jedná o aspoň 4 rôzne čísla,
dajte: 1 bod za zmienku o~symetrii a~uvedenie oboch sérií nerovností
typu~\ref[rov3.1]; 1~bod za zostavenie sústavy~\ref[rov3.2]. Len za dôkaz tvrdenia,
že sa jedná vždy o aspoň 3 rôzne čísla, žiadny bod neudeľujte.

Za druhú časť riešenia strhnite 1 bod, ak postup nájdenia
správnej trojice $(a,b,c)$ (napríklad uhádnutím) vyžaduje
chýbajúce overenie, že v skúmanej šestici sú naozaj iba 4 rôzne
hodnoty. Na také overovanie stačí vypísať prislúchajúcich šesť hodnôt,
ak je ich porovnanie zrejmé.
\endpetit
}

{%%%%%   A-II-1
Keďže úloha je v~premenných $a$, $b$, $c$ symetrická,\fnote{Zmenou poradia čísel $a$, $b$, $c$
dôjde iba ku zmene poradia skúmaných šiestich čísel.} môžeme bez ujmy na všeobecnosti predpokladať, že platí $a<b<c$. Ľahko sa presvedčíme, že potom štyri zo skúmaných šiestich čísel
sú podľa veľkosti usporiadané takto:
%\label[rov1.1]
$$
b+2a < a+2b < a+2c < b+2c.
%\eqmark
\tag1
$$
Naozaj, prvá nerovnosť vyplýva z~$(a+2b)-(b+2a)=b-a>0$, druhá nerovnosť z~$(a+2c)-(a+2b)=2(c-b)>0$ a tretia
z~$(b+2c)-(a+2c)=b-a>0$.

Pozrime sa, ktorým zo štyroch čísel v~\ref[rov1.1] sa môžu rovnať zvyšné dve čísla~$c+2a$ a~$c+2b$. Za~tým účelom si povšimneme, že rovnako ľahko ako \ref[rov1.1] možno dokázať aj nerovnosti
%\label[rov1.2]
$$
b+2a < c+2a < a+2c\quad\hbox{a}\quad a+2b < c+2b < b+2c.
%\eqmark
\tag2
$$
Porovnaním s~\ref[rov1.1] prichádzame k~záveru, že jediné dve
možné rovnosti medzi skúmanými šiestimi číslami (za predpokladu
$a<b<c$) sú $c+2a=a+2b$ a~$c+2b=a+2c$.\fnote{Poznamenajme,
že aj pri zvolenom usporiadaní $a<b<c$ má doterajší výklad niekoľko variantov, ktoré spočívajú vo vzájomnej výmene čísel v~jednej alebo v~oboch z~dvojíc $(a+2b,c+2a)$ a $(a+2c,c+2b)$ v~nerovnostiach \ref[rov1.1] a \ref[rov1.2].}
Obe tieto rovnosti sú však zrejme ekvivalentné s tou istou rovnosťou~$b=\frac12(a+c)$, ktorá je
pri našom predpoklade splniteľná.\fnote{Rovnosť $y=\frac12(x+z)$
totiž pre navzájom rôzne čísla $x$, $y$, $z$ znamená, že $y$ leží medzi
$x$ a $z$.}

Výsledok našich úvah pre prípad $a<b<c$ možno zhrnúť nasledovne:
\item{$\triangleright$} Ak je $b=\frac12(a+c)$, sú medzi skúmanými číslami práve
štyri rôzne. Platí pre ne totiž
%\label[rov1.3]
$$
b+2a < a+2b = c+2a < a+2c = c+2b < b+2c.
%\eqmark
\tag3
$$
Stane sa tak napríklad pre $(a,b,c)=(1,2,3)$.

\item{$\triangleright$} Ak naopak platí $b\ne \frac12(a+c)$, tak všetkých šesť skúmaných
čísel je navzájom rôznych. Tak tomu bude napríklad pre $(a,b,c)=(1,2,4)$.

\Zav
Hľadaný počet rôznych čísel v~zadanej šestici
je rovný buď~4, alebo~6.

\Pozn
Po výpise nerovností \ref[rov1.1] možno otázku prípadnej rovnosti prvého \uv{zvyšného} čísla $c+2a$ niektorému z~čísel v~\ref[rov1.1] riešiť aj bez použitia nerovností \ref[rov1.2], a to priamym testovaním jednotlivých rovností
$$
c+2a=b+2a,\quad
c+2a=a+2b,\quad
c+2a=a+2c,\quad
c+2a=b+2c.
$$
Podobne aj pre druhé \uv{zvyšné} číslo $c+2b$ možno testovať rovnosti
$$
c+2b=b+2a,\quad
c+2b=a+2b,\quad
c+2b=a+2c,\quad
c+2b=b+2c.
$$
Niektoré z~týchto rovností (napr. $c+2a=b+2a$) sú vylúčené už tým, že čísla $a$, $b$, $c$ sú navzájom rôzne. Ostatné rovnosti (okrem tých
z~\ref[rov1.3]) sú v spore s~predpokladaným usporiadaním $a<b<c$ (napr. $c+2a=b+2c$ znamená, že $a=\frac12(b+c)$, teda číslo $a$ by ležalo medzi
číslami $b$ a $c$.)

\Jres
Pri skúmaní potenciálnych rovností medzi
jednotlivými číslami zo zadanej šestice môžeme dospieť k~nasledujúcej hypotéze:
{\sl Ak sa niektoré dve z~týchto šiestich čísel rovnajú,
potom jedno z~čísel~$a$, $b$, $c$ je aritmetickým priemerom
ostatných dvoch čísel.}

Uvedenú hypotézu možno dokázať mechanickým testovaním všetkých
${6\choose 2}=15$ možných rovností. Menej prácny postup
založíme na tom, že porovnanie ľubovoľného z~daných čísel~$x+2y$
s~ostatnými piatimi číslami zapíšeme rovnosťami
%\label[rov1.4]
$$
x+2y=x+2z,\
x+2y=y+2x,\
x+2y=y+2z,\
x+2y=z+2x,\
x+2y=z+2y,
%\eqmark
\tag4
$$
pričom $x$, $y$, $z$ označuje čísla $a$, $b$, $c$ v~niektorom poradí. Prvá, druhá a piata rovnosť v~\ref[rov1.4] odporujú tomu, že čísla $x$, $y$ a $z$ sú navzájom rôzne. Tretia rovnosť nastane práve vtedy, keď
$z=\frac12(x+y)$. Napokon štvrtá rovnosť je splnená práve vtedy, keď
$y=\frac12({x+z})$. Tým je dôkaz hypotézy ukončený.

Priamym dôsledkom dokázanej hypotézy je tvrdenie:
{\sl Ak žiadne z~čísel $a$, $b$, $c$ nie je aritmetickým priemerom
zvyšných dvoch čísel, je skúmaná šestica tvorená šiestimi rôznymi
číslami.\fnote{Že také trojice $(a,b,c)$ existujú, je úplne zrejmé, a~preto nie je nutné uvádzať ich príklad. Ani v~druhej časti tohto riešenia nebudeme uvádzať príklad trojice $(a,b,c)$ s~vlastnosťou $c=\frac12(a+b)$.}}

Zaoberajme sa teda situáciou, ktorú predchádzajúci záver
nezahŕňa: jedno z~čísel $a$, $b$, $c$ je rovné aritmetickému
priemeru zvyšných dvoch čísel. Vzhľadom na symetriu môžeme
predpokladať, že sa jedná o~číslo $c$, pre ktoré tak platí
$c=\frac12(a+b)$. Vtedy máme $a=c+d$ a $b=c-d$, pričom $d=\frac12(a-b)\ne0$.
Po dosadení takých hodnôt $a$, $b$ budú mať čísla zo zadanej šestice postupne vyjadrenia
$$
3c-d,\quad
3c+d,\quad
3c+d,\quad
3c-d,\quad
3c+2d,\quad
3c-2d.
$$
Vďaka tomu, že $d\ne0$, vidíme, že v~takej šestici sú práve
štyri navzájom rôzne čísla $3c\pm d$ a $3c\pm 2d$. Dospeli sme tak
k rovnakému záveru ako v~prvom riešení.

\Pozn
Opíšme iné možnosti, ako posúdiť prípad $c=\frac12(a+b)$
bez zavedenia pomocného čísla $d$. Po dosadení takej hodnoty $c$
získame pre šestice čísel zo zadania vyjadrenie
$$
a+2b,\quad
2a+b,\quad
b+2a,\quad
2b+a,\quad
\frac{5a+b}{2},\quad
\frac{5b+a}{2}.
$$
Čísla $a+2b$, $2a+b$ sú tu zastúpené každé dvakrát, takže
máme zistiť, koľko rôznych čísel môže byť v štvorici
$$
a+2b,\quad
2a+b,\quad
\frac{5a+b}{2},\quad
\frac{5b+a}{2}.
$$
Ukázať, že tu už sú všetky čísla navzájom rôzne,
možno rutinným testovaním
všetkých ${4\choose 2}$ potenciálnych rovností -- každá z~nich totiž
odporuje predpokladu $a\ne b$. Inou možnosťou je ukázať,
že v~prípade $a<b$ platí
$$
\frac{5a+b}{2} < 2a+b < a+2b < \frac{5b+a}{2},
$$
zatiaľ čo v~opačnom prípade $a>b$ platia aj vypísané ostré nerovnosti
naopak.

%\nobreak\medskip\petit\noindent
\schemaABC
Za úplné riešenie dajte 6~bodov. Ak sú opísané podmienky,
za ktorých je hľadaný počet 4, resp. 6,
nie je nutné uvádzať konkrétne príklady trojíc $(a,b,c)$,
ak je splniteľnosť týchto podmienok v~podanom riešení zrejmá.

V~prípade postupu podľa prvého vzorového riešenia
dajte: 3~body, ak riešiteľ pri zvolenom usporiadaní
nájde v~zadanej šestici štyri rôzne čísla (2~body) a
zdôvodní ich rôznosť (1~bod); 3~body za úplnú diskusiu
o~\uv{zvyšných} dvoch číslach zo šestice.
Ak riešiteľ nevylúči možnosť piatich rôznych čísel, môže získať
nanajvýš 3~body.

V~prípade postupu podľa druhého vzorového riešenia
dajte: 2 body za dôkaz uvedenej hypotézy; 1~bod za z toho vyplývajúci
záver, kedy sú v~zadanej šestici všetky čísla navzájom rôzne;
3~body za vyriešenie jedného z~prípadov typu $c=\frac12(a+b)$.
V~prípade drobnej chyby v dôkaze hypotézy strhnite 1~bod
(zabudnutie niektorého z~piatich prípadov).

Len za uhádnutie úplnej odpovedi dajte 1~bod, ale iba v prípade, že sú
oba počty 4 a~6 podložené buď konkrétnymi príkladmi trojíc $(a,b,c)$,
alebo podmienkami, o~ktorých je navyše ukázané, že sú na~dosiahnutie
počtu 4, resp. 6 postačujúce.
\endschema
}

{%%%%%   A-II-2
Superdeliteľa $d$ zloženého čísla $n$ dostaneme,
keď z~rozkladu čísla~$n$ na súčin prvočísel vyškrtneme najmenšieho
zastúpeného činiteľa (jedného, ak je ich viac).\fnote{Podrobnejšie
o~určovaní superdeliteľov pozri vzorové riešenie úlohy~4 z~domáceho kola.}
Preto platí $n=pd$, pričom $p$ je prvočíslo
s~vlastnosťou $p\leqq d$.\fnote{Fakt, že prvočíslo $p$
musí byť {\it najmenším\/} prvočiniteľom čísla $n$,
v~našom riešení potrebovať nebudeme.}

Nech $n_1$, $n_2$, $n_3$ je ľubovoľná hľadaná trojica. Podľa úvodného odseku pre superdeliteľa~$d_i$ čísla $n_i$ platí
rovnosť $n_i=p_id_i$, pričom $p_i$ je prvočíslo a pritom $p_i\leqq d_i$
pre každé $i=1,2,3$. Podľa zadania
úlohy má platiť súčasne $d_2d_3\mid p_1d_1$, $d_1d_3\mid p_2d_2$ a~$d_1d_2\mid p_3d_3$. Vynásobením týchto troch relácií dostaneme
$$
(d_1d_2d_3)^2\mid p_1p_2p_3d_1d_2d_3,\quad\hbox{čiže}\quad
d_1d_2d_3\mid p_1p_2p_3.
$$
Na druhej strane, vynásobením troch nerovností $p_i\leqq d_i$
dostaneme $p_1p_2p_3\leqq d_1d_2d_3$. Zo záverov dvoch
posledných viet vyplýva, že musí platiť rovnosť
$d_1d_2d_3=p_1p_2p_3$. Z~toho vďaka spôsobu odvodenia nerovnosti
$p_1p_2p_3\leqq d_1d_2d_3$ zisťujeme, že musí platiť rovnosť
$d_i=p_i$, čiže $n_i=p_i^2$ pre každé $i=1,2,3$.
Relácia $d_2d_3\mid p_1d_1$ potom znamená, že $p_2p_3\mid p_1^{2}$,
takže všetky tri prvočísla $p_i$ sa musia rovnať tomu istému prvočíslu~$p$.
Potom platí $d_i=p$ a $n_i=p^2$ pre každé $i=1,2,3$, takže podmienky
$d_id_j\mid n_k^2$ sú zrejme splnené v~podobe $p^2\mid p^2$.

\Zav
Riešeniami úlohy sú práve trojice
$\bigl(p^2,p^2,p^2\bigr)$, pričom $p$ je ľubovoľné prvočíslo.

\Jres
Keďže hľadané čísla $n_i$ ($i=1,2,3$) sú zložené, ich
superdelitele~$d_i$ sú väčšie ako 1. Vzhľadom na symetriu zadania
môžeme dokonca predpokladať, že platí $1<d_1\leqq d_2\leqq d_3$.

Vieme, že $d_1\mid n_1$ a že podľa zadania úlohy
tiež $d_2d_3\mid n_1$, a~teda aj $d_2\mid n_1$ a~$d_3\mid
n_1$. Všetky čísla $d_1$, $d_2$, $d_3$ a~$d_2d_3$ tak sú
deliteľmi čísla~$n_1$, pre ktoré navyše podľa nášho predpokladu
platí $$
d_1\leqq d_2\leqq d_3<d_2d_3\leqq n_1.
$$
Keďže však $d_1$ je superdeliteľ čísla $n_1$, z~posledných nerovností
vyplýva $d_1=d_2=d_3$ a~$d_2d_3=n_1$. Pri
označení $d$ spoločnej hodnoty čísel $d_i$ tak platí $n_1=d^2$.

Vďaka rovnostiam $d=d_i$ teraz z~relácií $d_1d_3\mid n_2$ a $d_1d_2\mid n_3$
vyplývajú závery $n_2=d^2$, resp. $n_3=d^2$, a to rovnakým postupom,
ako sme skôr za (slabšieho) predpokladu $d_1\leqq d_2\leqq d_3$ odvodili
záver $n_1=d^2$. Každá hľadaná trojica $(n_1,n_2,n_3)$ teda musí
mať tvar $\bigl(d^2,d^2,d^2\bigr)$; vyhovovať pritom budú zrejme práve tie
čísla $d>1$, ktoré sú superdeliteľmi čísla $d^2$. Stane sa tak
práve vtedy, keď $d$ bude prvočíslo. Naozaj: ak je~$d$ prvočíslo, tak
1, $d$ a $d^2$ sú jedinými deliteľmi čísla $d^2$, takže $d$ je
jeho superdeliteľom; ak je naopak číslo $d$ zložené, je deliteľné
niektorým prvočíslom $p<d$, takže $pd$ je deliteľom čísla $d^2$
s~vlastnosťou $d<pd<d^2$, čo odporuje tomu, že $d$ je superdeliteľom
čísla $d^2$. Dospeli sme tak k rovnakému záveru
ako pri prvom riešení.


\schemaABC
Za úplné riešenie dajte 6~bodov. Strhnite ale 1 bod, ak chýba
záverečná skúška (ak je pri danom postupe nutná).

Pri čiastočných riešeniach dajte: a) 3 body, ak riešiteľ usporiada tri superdelitele $d_i$ podľa veľkosti a
dokáže ich rovnosť; b) 2 body, ak riešiteľ uhádne výsledok a
overí, že trojice $(p^2,p^2,p^2)$ vyhovujú; c)~1~bod,
ak riešiteľ uhádne výsledok (bez jeho overenia) a/alebo
uvedie poznatok, že každé zložené číslo~$n$ je tvaru $n=pd$,
pričom~$d$ je superdeliteľ čísla $n$ a~$p$ je najmenší prvočiniteľ čísla $n$,
alebo iba prvočíslo s~vlastnosťou~$p\leqq d$
(možno sa pritom odvolať na vzorové riešenie úlohy 4 z~domáceho kola).
Zisky za časti a), b), c) sa pritom nesčítajú (berie sa najväčší z
nich), len za obe časti a) a b) prislúchajú dokopy 4 body.
\endschema
}

{%%%%%   A-II-3
V~celom texte budeme používať štandardné označenie~$\alpha$,
$\beta$, $\gamma$ veľkostí vnútorných uhlov trojuholníka $ABC$.
Kľúčovým bodom postupu bude zistenie, že bod~$F$ je
totožný so stredom kružnice opísanej trojuholníku~$ABC$, ktorý označíme $O$.
V~prvej časti riešenia tento poznatok dokážeme dvoma spôsobmi.

\smallskip
\noindent{\it 1. spôsob}. Vďaka predpokladu $|AD|=|CD|$ je os~$o_b$
strany~$AC$ totožná s~osou uhla $CDA$, a teda aj s~osou
uhla $EDA$.\fnote{Zo zadania úlohy totiž vyplýva, že bod
$E$ leží medzi bodmi $C$ a $D$.} Analogicky os~$o_c$ strany $AB$
je totožná s~osou uhla $AED$. Táto dvojaká rola oboch osí vedie k~záveru,
že stred $O$ kružnice opísanej trojuholníku $ABC$ je
súčasne stredom kružnice vpísanej trojuholníku $ADE$. Z toho vyplýva,
že polpriamka $AO$ je osou uhla $DAE$. Zároveň však platí $|OB|=|OC|$,
takže bod~$O$ je spoločným bodom osi uhla $DAE$ a~osi úsečky~$BC$.
Podľa zadania je ale ich spoločný bod jediný a má označenie
$F$, takže naozaj platí $F=O$.

\smallskip
\noindent{\it 2. spôsob}. Inou úvahou znova dokážeme, že polpriamka $AO$ je
os uhla $DAE$, odkiaľ už rovnako ako pri 1.~spôsobe vyplynie
záver o~rovnosti $F=O$. Podľa vety o~obvodovom a~stredovom uhle
platí $|\uhel AOB|=2\gamma$, takže z~rovnoramenného trojuholníka $ABO$
potom máme $|\uhel OBA|=|\uhel BAO|=90^\circ-\gamma$. Ďalej
v~rovnoramennom trojuholníku $ACD$ platí $|\uhel DAC|=|\uhel ACD|=\gamma$,
teda $|\uhel BAD|=|\uhel BAC|-|\uhel DAC|=\alpha-\gamma$.
Dokopy dostávame $|\uhel DAO|=|\uhel BAO|-|\uhel
BAD|=(90^\circ-\gamma)-(\alpha-\gamma)={90^\circ-\alpha}$.\fnote{Veľkosti uhlov $BAO$ a $BAD$ sme od seba odčítali v správnom
poradí, lebo vďaka zadaniu úlohy platí $90\st-\alpha>0$.}
Analogickou cestou cez trojuholníky $ACO$ a $ABE$ dostaneme rovnosť
$|\uhel EAO|={90^\circ-\alpha}$. Dokázaná zhodnosť uhlov $DAO$ a
$EAO$ potvrdzuje, že~polpriamka $AO$ je naozaj osou uhla $DAE$.
\inspsc{a70ii_4.1}{0.8333}%

V~druhej časti riešenia ostáva dokázať rovnosť
$|\uhel BAC|+|\uhel DFE|=180^\circ$, ktorú prepíšeme na tvar
$|\uhel DFE|=180\st-\alpha$. To sa bude dať spraviť obzvlášť ľahko, ak
okrem dokázanej rovnosti $F=O$ využijeme opäť osi $o_b$, $o_c$ úsečiek
$AC$, resp. $AB$, ktorých stredy označíme $S_b$, resp. $S_c$
ako na \obr. Okrem toho uvedieme dva postupy bez použitia týchto
osí.

\smallskip
\noindent{\it 1. spôsob}. Body~$S_b$, $F$, $D$ ležia v~tomto poradí
na osi~$o_b$, rovnako ako body~$S_c$, $F$,~$E$ na osi~$o_c$.
Okrem toho platí $FS_b \perp AC$ a~$FS_c \perp AB$,
takže zo štvoruholníka $AS_bFS_c$ vyplýva, že jeho vnútorný uhol
$S_bFS_c$ má veľkosť $180\st-\alpha$.
Rovnakú veľkosť má preto aj vrcholový uhol $DFE$, ako sme
potrebovali ukázať. Dodajme, že veľkosť $180\st-\alpha$ uhla $DFE$
možno vypočítať aj z~trojuholníka $DEF$,
v ktorom totiž vďaka osiam $o_b$, $o_c$ platia rovnosti
$|\uhel FDE|=90\st-\gamma$ a $|\uhel FED|=90\st-\beta$.

\smallskip
\noindent{\it 2. spôsob}. Všimnime si, že štvoruholník $BDFA$ je
tetivový, lebo podľa vety o~obvodovom a~stredovom uhle platí
$|\uhel AFB|=2\gamma$ a~tiež vonkajší uhol $ADB$ rovnoramenného
trojuholníka $ACD$ má veľkosť $2\gamma$. Analogicky je aj
štvoruholník $AFEC$ tetivový. Zo štvoruholníkov $BDFA$ a~$AFEC$ tak
vyplývajú rovnosti $|\uhel AFD|=180^\circ-\beta$
a~$|\uhel AFE|=180^\circ-\gamma$. Ich dosadením do
$|\uhel DFE|=360\st-|\uhel AFD|-|\uhel AFE|$ už získame
$|\uhel DFE|=\beta+\gamma=180^\circ-\alpha$.

\smallskip
\noindent{\it 3. spôsob}. Namiesto rovnosti $F=O$ využijeme už skôr
dokázaný poznatok o~tom, že bod $F$ je stredom
kružnice vpísanej trojuholníku $ADE$. Podľa známeho vzorca to znamená,
že $|\uhel DFE|=90^\circ+\frac12|\uhel DAE|$.\fnote{Vyplýva to~z výpočtu~$|\uhel DFE|=180^\circ-|\uhel EDF|-
|\uhel FED|=180^\circ-\frac12|\uhel EDA|-\frac12|\uhel
AED|=180^\circ-\frac12(|\uhel EDA|+|\uhel AED|)=
180^\circ-\frac12(180^\circ-|\uhel DAE|)=90^\circ+\frac12|\uhel DAE|$.}
Posledný uhol bude mať potrebnú veľkosť $180^\circ-\alpha$,
ak overíme, že $|\uhel DAE|=180^\circ-2\alpha$. To je jednoduché:
vďaka rovnoramenným trojuholníkom $ACD$, $ABE$ platí
$|\uhel CAD|=\gamma$, resp. $|\uhel BAD|=\beta$, odkiaľ už
$|\uhel DAE|=|\uhel CAD|+ |\uhel BAE|-|\uhel ABC|=\gamma+\beta-\alpha=180\st-2\alpha$.

\schemaABC
Za úplné riešenie dajte 6~bodov, z~toho: 4~body
za dôkaz, že bod~$F$ je totožný so stredom~$O$ kružnice opísanej
trojuholníku~$ABC$; 2~body za dôkaz uhlovej rovnosti zo záveru zadania.

V~prípade neúplných riešení dajte 1~bod za hypotézu
o~totožnosti bodov $F$ a~$O$. Ak riešiteľ dokáže uhlovú
rovnosť použitím nedokázanej hypotézy $F=O$, dajte celkom 3~body.

Žiadny bod neudeľujte ani za nedokázanú hypotézu, že $ABDF$ je
(rovnako ako $ACEF$) tetivový štvoruholník, ani za dôkaz, že
$ABDO$ a $ACEO$ sú tetivové štvoruholníky.
\endschema
}

{%%%%%   A-II-4
Očíslujme žiarovky $1, 2,\ldots,70$ v~poradí po obvode kruhu.
S~číslami žiaroviek budeme počítať \uv{cyklicky}, \tj. ako so zvyškovými
triedami modulo 70 (takže napr. ${68+3}$ je rovné~1). Písmenom $i$
budeme označovať ľubovoľné celé číslo od~1 do~70. Spojenie
\uv{rozsvietiť danú skupinu čísel} bude znamenať rozsvietiť
všetky žiarovky s~číslami z~tejto skupiny a~žiadne iné. V záverečnej
poznámke a pokynoch pre bodovanie budeme písať o \uv{párnych} a~\uv{nepárnych} žiarovkách podľa ich čísel.

Predpokladajme, že máme danú skupinu
prepínačov, ktorou sme schopní rozsvietiť každú štvoricu
$\{i, i+1, i+2, i+3\}$. Ukážeme, že v~takej skupine je
aspoň 68~prepínačov.

Keďže vieme rozsvietiť obe štvorice $\{i, i+1, i+2, i+3\}$ a
$\{i+1, i+2, i+3, i+4\}$, spojením oboch prislúchajúcich postupov
docielime rozsvietenie dvojice $\{i,i+4\}$. Tým pádom vieme
rozsvietiť každú z~dvojíc
$$
\{i, i+4\},\ \{i+4, i+8\},\ \{i+8, i+12\}, \ldots,\ \{i+68,i+72\},
$$
takže spojením prislúchajúcich postupov docielime rozsvietenie
dvojice $\{i, i+72\}$, \tj. dvojice $\{i,i+2\}$.

Zvoľme teraz za $d$ ľubovoľné prirodzené číslo menšie ako 35.
Keďže vieme rozsvietiť každú z~$d$ dvojíc
$$
\{i,i+2\},\ \{i+2, i+4\},\ \{i+4,i+6\},\ \ldots,\ \{i+2d-2, i+2d\},
$$
spojením prislúchajúcich postupov docielime rozsvietenie dvojice $\{i,i+2d\}$.

Získané dvojice $\{i,i+2d\}$ reprezentujú všetky dvojprvkové
množiny čísel (od~1 do~70) jednej parity.
Keďže sme schopní rozsvietiť každú z~nich, môžeme opakovaným
rozsvecovaním po dvojiciach rozsvietiť ľubovoľnú množinu
$M \subseteq \{1, \ldots, 70\}$, ktorá obsahuje párny počet
párnych čísel a~ zároveň párny počet nepárnych čísel (počítame aj prázdnu
množinu~$M$, ktorú sme tiež schopní \uv{rozsvietiť}).
Určíme počet všetkých takých množín~$M$.

Párne čísla od~1 do~70 tvoria 35-prvkovú množinu, ktorá má
práve $2^{34}$ podmnožín s párnym počtom prvkov.\fnote{Pozri
vzorové riešenie príkladu 6 z~domáceho kola, pričom bolo
dokázané, že každá $n$-prvková množina má práve $2^{n-1}$ podmnožín
s párnym počtom prvkov.} Rovnako tak
nepárne čísla od~1 do~70 tvoria 35-prvkovú množinu, ktorá má
práve $2^{34}$ podmnožín s párnym počtom prvkov. Hľadaný počet
množín $M$ z~predchádzajúceho odseku je preto
$2^{34} \cdot 2^{34} = 2^{68}$.

Keďže daná skupina prepínačov umožňuje dosiahnuť aspoň $2^{68}$
rôznych stavov rozsvietenia, je v~nej aspoň 68
prepínačov,\fnote{Vo vzorovom riešení
príkladu 6 z~domáceho kola bolo zdôvodnené, že použitím $k$ prepínačov možno
dosiahnuť nanajvýš $2^k$ rôznych stavov rozsvietenia.}
ako sme sľúbili dokázať.

Ostáva uviesť príklad skupiny 68 prepínačov, ktorými
možno rozsvietiť každú štvoricu $\{i, i+1, i+2, i+3\}$.
Za tým účelom označme ako $[p,q]$ prepínač, ktorý zmení
stav práve dvoch žiaroviek, konkrétne tých s~číslami $p$ a $q$.
Ukážeme, že skupina 68~takých prepínačov
$$
[1, 3],\ [2, 4],\ [3, 5],\ \ldots,\
[66, 68],\ [67, 69],\ [68, 70]
$$
nášmu zadaniu vyhovuje. Určite nimi rozsvietime
ktorúkoľvek dvojicu $\{i,i+2\}$ s dvoma výnimkami, keď $i=69$ a
$i=70$.

Použitím prepínačov
$[1, 3]$, $[3, 5]$, \dots, $[67, 69]$
rozsvietime vo výsledku dvojicu $\{1, 69\}$, \tj. dvojicu
$\{i,i+2\}$ pre $i=69$. Podobne použitím prepínačov
$[2,4]$, $[4, 6]$, \dots, $[68, 70]$
rozsvietime dvojicu $\{i,i+2\}$ pre $i=70$. Vieme teda rozsvietiť
každú dvojicu $\{i,i+2\}$, a tým pádom aj každú štvoricu
$\{i, i+1, i+2, i+3\}$ (rozsvietime po sebe dvojice $\{i,i+2\}$ a
$\{i+1,i+3\}$).

\Pozn
Pre inú konštrukciu vyhovujúcich 68 prepínačov možno
využiť príklad~6 domáceho kola, v ktorom sme vlastne
zostrojili $n-1$ prepínačov pre množinu $n$~žiaroviek tak,
aby nimi bolo možné rozsvietiť každú skupinu s párnym počtom žiaroviek
(nebolo podstatné, že v príklade bolo konkrétne $n=70$ párne). Podľa
toho si tak teraz pripravíme~34 prepínačov pre našich 35 párnych
žiaroviek a ~34 prepínačov pre našich 35 nepárnych žiaroviek. Každá
štvorica susedných žiaroviek je zrejme zložená z dvojice párnych a
dvojice nepárnych žiaroviek, je ju preto možné rozsvietiť.

\schemaABC
Za úplné riešenie dajte 6~bodov, z~toho: 3~body za dôkaz,
že ak možno rozsvietiť každú štvoricu susedných
žiaroviek, tak možno rozsvietiť ľubovoľnú skupinu žiaroviek
s~párnym počtom párnych žiaroviek a~párnym počtom nepárnych žiaroviek;
1~bod za dôkaz, že takých skupín žiaroviek je~$2^{68}$ (je možné
sa odvolať na tvrdenie z~domáceho kola o~počte
tých podmnožín danej množiny, ktoré majú párny počet prvkov);
1~bod za zdôvodnenie, že potrebujeme aspoň 68 prepínačov (je možné
sa odvolať na argumentáciu z~domáceho kola); 1~bod za
príklad vyhovujúcej skupiny 68 prepínačov (možno si pritom pomôcť
odvolaním sa na konštrukciu z~domáceho kola), ak je ale zdôvodnené, že navrhnutá skupina prepínačov je naozaj vyhovujúca.
\endschema
}

{%%%%%   A-III-1
Ukážeme, že vyhrávajúcu stratégiu má začínajúci hráč.
Nazvime ho A a~jeho protivníka označme~B. Opíšeme najskôr postup
hráča A, ktorým zabezpečí, že menovateľ výsledného zlomku bude
práve o~1~väčší ako jeho čitateľ.

Hráč~A pri svojom prvom ťahu vloží číslo~1 do menovateľa.
Kedykoľvek hráč~B pri svojom ťahu vyberie a vloží do zlomku číslo~$x$,
hráč~A pri následnom ťahu vyberie číslo~$2023-x$ a vloží ho do tej istej
časti zlomku, kam vložil hráč~B číslo $x$ (\tj. obe po sebe
vložené čísla $x$ a $2023-x$ budú buď v~čitateli, alebo v menovateli).

Vysvetlíme, prečo opísanú stratégiu môže hráč A~vždy uskutočniť.
Najskôr si uvedomíme, že po vložení čísla~1 do menovateľa
bude ako v~čitateli, tak v menovateli {\em párny počet voľných
políčok} (konkrétne 1010 v~oboch častiach zlomku). Táto vlastnosť
zostane zrejme zachovaná po každom ťahu hráča~A.
Preto sa nemôže stať, že by hráč~B pri niektorom svojom ťahu obsadil
posledné voľné políčko čitateľa alebo menovateľa.

Nemôže sa stať ani to, že by číslo, ktoré má hráč~A po niektorom ťahu
hráča~B vkladať, bolo už skôr vybrané,
lebo po úvodnom vložení čísla~1 možno
zvyšné čísla od~2 do~2021 rozdeliť na dvojice čísel s rovnakým
súčtom~2023:
$$
\{2, 2021\}, \{3, 2020\}, \{4, 2019\}, \ldots, \{1011, 1012\}.
$$
Znamená to, že pri každom svojom ťahu hráč~B nejakú z~týchto
dvojíc \uv{načne} a~hráč~A ju následne \uv{dokončí}.
Tým je sľúbené vysvetlenie ukončené.

Opísaná stratégia hráča A~zaručí, že po zaplnení všetkých políčok
bude hodnota čitateľa rovná $505\cdot2023$, zatiaľ čo hodnota
menovateľa bude $505\cdot2023+1$. Keďže pre taký zlomok
platí
$$
0<1-\frac{505\cdot2023}{505\cdot2023+1}
=\frac{1}{505\cdot2023+1}<\frac{1}{500\cdot2000}=10^{-6},
$$
hráč~A~bude víťazom.
}

{%%%%%   A-III-2
Označme $S\ne C$ druhý priesečník priamky $CI$
s~kružnicou $ABC$.\fnote{Pre stručnosť budeme pod \uv{kružnicou
$PQR$} rozumieť kružnicu opísanú trojuholníku $PQR$.}
Dokážeme, že priamka $CI$
sa dotýka kružnice $BMN$ práve v~bode $S$.

Je známe, že~bod $S$ je stred kružnice $AIB$. Tento výsledok
platí pre všeobecný trojuholník $ABC$ a uvedieme teraz jeho krátky dôkaz.
Keďže~$CS$ je os uhla~$ACB$, prislúchajú v~ňom ležiacim oblúkom~$SA$ a~$SB$
kružnice $ABC$ zhodné obvodové uhly, odkiaľ vyplýva $|SA|=|SB|$.
Druhú potrebnú rovnosť $|SB|=|SI|$ dokážeme cez uhly
trojuholníka $BIS$ pri vrcholoch $B$ a~$I$:
$$
|\angle BIS|=
|\angle BCI| + |\angle CBI| =
|\angle ACS| + |\angle ABI| =
|\angle ABS| + |\angle ABI| =
|\angle IBS|.
$$
Dokopy máme $|SA|=|SB|=|SI|$ a dôkaz je ukončený.

Keďže~$S$ je stred kružnice $AIB$, pre stredy $M$, $N$ jej
tetív $AB$, $BI$ platí $SM\perp MB$ a~$SN\perp NB$, takže body~$S$, $M$, $N$, $B$
ležia na (Tálesovej) kružnici nad priemerom~$BS$ (\obr). Stred kružnice
$BMN$ je teda stredom úsečky $BS$, a preto na dokončenie celého
riešenia stačí zdôvodniť, že $|\uhel CSB|=90^\circ$. Obvodový uhol
$CSB$ v~kružnici $ABC$ je však zhodný s~obvodovým uhlom $CAB$, a
ten je podľa zadania úlohy naozaj pravý.
\insp{a70iii.1}%

\Pozn
Rolu bodu $S$ v~riešení úlohy naznačuje fakt, že
uvažovaná kružnica $BMN$ je v rovnoľahlosti
${\Cal H}\left(B,\frac12\right)$ obrazom kružnice $AIB$, ktorej
stred je práve bod~$S$. Z~toho totiž hneď vyplýva kľúčový poznatok, že
úsečka $BS$ je priemerom kružnice $BMN$, ktorej sa má priamka
$CI$ dotýkať.}

{%%%%%   A-III-3
Čísla $a$, $b$, $c$ sú navzájom rôzne a nenulové,
takže sú navzájom rôzne ako čísla $a+b$, $b+c$, $c+a$, tak
aj čísla $ab$, $bc$, $ca$. Preto sú množiny z~prvej
vety zadania zapísané správne, \tj. sú naozaj trojprvkové.
Toto úvodné konštatovanie v~ďalších riešeniach opakovať nebudeme.

Keďže zadanie úlohy je v~číslach $a$, $b$, $c$ symetrické,
môžeme rovnosť posudzovaných trojprvkových množín vyjadriť
tromi principiálne
odlišnými sústavami rovníc\fnote{Rozlíšime pritom, koľko rovníc typu
$x+y=xy$ spĺňajú dvojice $\{x,y\}$ vybrané z~$\{a,b,c\}$. Buď to
sú všetky tri dvojice či práve jedna z~nich, alebo to nie je žiadna.}
$$
\eqalign{
a+b&=ab,\cr
b+c&=bc,\cr
c+a&=ca,}
\hskip2cm
\eqalign{
a+b&=ab,\cr
b+c&=ca,\cr
c+a&=bc,}
\hskip2cm
\eqalign{
a+b&=bc,\cr
b+c&=ca,\cr
c+a&=ab.}
$$
V~prípade prvej sústavy dostávame odčítaním jej druhej rovnice od
prvej vzťah $a-c=b(a-c)$. Z toho vzhľadom na predpoklad $a\ne c$
vyplýva $b=1$. To však odporuje prvej rovnici skúmanej sústavy.

Podobne v~prípade druhej sústavy dostávame odčítaním tretej
rovnice od druhej vzťah $b-a=c(a-b)$, takže vďaka $a\ne b$ máme
$c=-1$. Dosadením do sústavy dostaneme rovnice $a+b=ab$ a $a+b=1$,
odkiaľ $ab=1$. Čísla~$a$, $b$ sú preto korene rovnice~$x^2-x+1=0$.
Tá však reálne korene nemá.

Zostáva nám tak rozobrať prípad tretej sústavy. Označme jej
rovnice zhora nadol \thetag1, \thetag2, \thetag3. V~ich dôsledku platí
$$
ab^2\buildrel{(3)}\over= b(c+a)=bc+ab\buildrel{(1),\,(3)}\over=
(a+b)+(c+a)=2a+(b+c)\buildrel{(2)}\over=2a+ca=a(2+c).
$$
Porovnaním oboch krajných výrazov dostávame $ab^2=a(2+c)$,
odkiaľ vzhľadom na $a\ne0$ máme $b^2=2+c$, čiže
$c=b^2-2$.
Keďže skúmaná tretia sústava je cyklická,
platí tiež $a=c^2-2$ a~$b=a^2-2$. Množinová rovnosť zo záveru
zadania úlohy je tak dokázaná.

  \Jres
Ukážeme, že tretiu sústavu z~pôvodného
riešenia
$$
\align
a+b&=bc\tag1\\
b+c&=ca\tag2\\
c+a&=ab\tag3
\endalign
$$
možno posúdiť tak, že ju začneme riešiť bežnou eliminačnou
metódou. Z~\thetag3 vyjadríme $c=ab-a$ a dosadíme také $c$ do \thetag2.
Dostaneme po úprave $a^2-a=b(a^2-a-1)$. Všimnime si, že podľa
\thetag2 platí $a\ne1$, čo spolu s~$a\ne0$ dáva $a^2-a\ne0$.
Preto zo vzorca $a^2-a=b(a^2-a-1)$ vyplýva nerovnosť $a^2-a-1\ne0$
a~vyjadrenie $b={(a^2-a)}/{(a^2-a-1)}$. Po jeho dosadení do \thetag3
dostaneme $c=a/(a^2-a-1)$. Rovnicu \thetag1 tak môžeme prepísať na tvar
$$
a+\frac{a^2-a}{a^2-a-1}=\frac{a^2-a}{a^2-a-1}\cdot\frac{a}{a^2-a-1},
$$
odkiaľ po úpravách dostaneme rovnicu
$$
0=a^5-a^4-4a^3+3a^2+2a=a(a-2)(a^3+a^2-2a-1).
$$
Keďže $a\ne0$ a rovnosť $a=2$ by podľa \thetag2 znamenala~$b=c$,
prichádzame k~záveru, že číslo $a$ je nutne
koreňom mnohočlena
$$
Q(x)=x^3+x^2-2x-1.
$$
Keďže skúmaná sústava je cyklická, korene polynómu $Q(x)$
sú aj čísla $b$ a~$c$, takže platí rozklad
$Q(x)=(x-a)(x-b)(x-c)$.\fnote{Je možné overiť, že
korene $Q(x)$ sú tri navzájom rôzne reálne čísla, ale nie je
to nevyhnutné. Podľa zadania úlohy totiž vyhovujúce čísla $a$, $b$, $c$
existujú.} Na doriešenie úlohy preto stačí
dokázať, že čísla $a^2-2$, $b^2-2$, $c^2-2$ sú tiež navzájom
rôzne korene polynómu~$Q(x)$.

Najskôr zdôvodníme vzájomnú rôznosť: vzhľadom na symetriu
stačí vylúčiť napr. rovnosť $a^2-2=b^2-2$. Podľa nej by platilo
$a=b$ alebo $a=\m b$. Prvá rovnosť odporuje zadaniu priamo,
z~druhej rovnosti podľa \thetag1 vyplýva $bc=0$, čo je tiež v spore
so zadaním.

Dokázať, že čísla $a^2-2$, $b^2-2$, $c^2-2$ sú korene
mnohočlena $Q(x)$, znamená to isté čo overiť, že mnohočlen
$Q\bigl(x^2-2\bigr)$ je
deliteľný tromi rôznymi koreňovými činiteľmi $x-a$, $x-b$, $x-c$,
a teda aj ich súčinom, \tj. polynómom $Q(x)$. Je teda jasné,
ako riešenie dokončiť: najskôr vypočítame
$$
Q\bigl(x^2-2\bigr)=(x^2-2)^3+(x^2-2)^2-2(x^2-2)-1= x^6-5x^4+6x^2-1
$$
a potom sa použitím známeho algoritmu presvedčíme, že
delenie $Q\bigl(x^2-2\bigr):Q(x)$ vyjde bezo zvyšku.
Zapíšeme tu iba výsledok tohto delenia v~podobe rozkladu
$$
Q\bigl(x^2-2\bigr)=Q(x)\cdot(x^3-x^2-2x+1).
$$

\poznamky
Korene polynómu $Q(x)$ sú čísla
$2\cos\frac27\pi\doteq1{,}24698$, $2\cos\frac47\pi\doteq\m0{,}44504$
a~$2\cos\frac67\pi\doteq\m1{,}80194$. Také sú teda (až na
poradie) hodnoty ľubovoľnej trojice čísel $a$, $b$, $c$
vyhovujúcich zadaniu úlohy.

Naznačme odlišný dôkaz tvrdenia, že trojica $\{a,b,c\}$
koreňov odvodeného polynómu $Q(x)$ je zhodná s~trojicou
$\{a^2-2,b^2-2,c^2-2\}$ (nebude pritom nutné vopred zdôvodňovať,
že sa jedná aj o~tri rôzne čísla). Spomenutý fakt vyjadríme
pomocou Vietových vzorcov rovnosťami
$$\eqalign{
a+b+c&=\m1=(a^2-2)+(b^2-2)+(c^2-2),\cr
ab+bc+ca&=\m2=(a^2-2)(b^2-2)+(b^2-2)(c^2-2)+(c^2-2)(a^2-2),\cr
abc&=\hphantom{\m}1=(a^2-2)(b^2-2)(c^2-2).
}$$
Je teda nutné dokázať, že z troch ľavých rovností vyplývajú všetky
tri pravé rovnosti. Nebudeme sa touto algebraickou previerkou tu
zaoberať.

  \Jres
Nezávisle na prvých dvoch riešeniach opíšeme ešte jedno odvodenie
toho kubického polynómu $Q(x)=x^3-s x^2+r x-p$,
ktorého korene sú zadané (navzájom rôzne) čísla $a$, $b$, $c$. (Zvyšok
druhého riešenia potom opakovať nebudeme.) K~hľadaným hodnotám $s=\m1$,
$r=\m2$ a $p=1$ dôjdeme tak, že pre tieto neznáme čísla
určená Vietovými vzorcami
$$
s=a+b+c,\quad r=ab+bc+ca,\quad p=abc
$$
najskôr trojakým použitím rovnosti $\{a+b,b+c,c+a\}=\{ab,bc,ca\}$
získame sústavu troch rovníc, ktorú potom vyriešime.

Z~rovnosti súčtu prvkov v~oboch trojprvkových množinách
$$
(a+b)+(b+c)+(c+a)=ab+bc+ca
$$
máme prvú rovnicu $2s=r$. Sčítanie súčinov dvojíc čísel
v~každej z~oboch množín vedie k~druhému dôsledku
$$
(a+b)(b+c)+(b+c)(c+a)+(c+a)(a+b)=
ab\cdot bc+bc\cdot ca+ca\cdot ab,
$$
ktorý možno upraviť na tvar
$(a+b+c)^2+(ab+bc+ca)=abc(a+b+c)$, \tj. $s^2+r=p\cdot s$.
Napokon v~rovnosti súčinov trojíc čísel
$$
(a+b)(b+c)(c+a)=ab\cdot bc\cdot ca
$$
je ľavá strana rovná $(a+b+c)(ab+bc+ca)-abc=s\cdot r-p$ a pravá
je~$p^2$, takže platí $s\cdot r=p^2+p$. Výslednú sústavu
troch rovníc
$$
2s=r,\quad s^2+r=p\cdot s,\quad s\cdot r=p^2+p
$$
elimináciou $r=2s$ zredukujeme na sústavu
$$
\align
s^2+2s&=p\cdot s,\tag4\\
2s^2 &= p^2+p.\tag5
\endalign$$
V~prípade $s=0$ z~\thetag5 vyplýva $p^2+p=0$, takže $p\in\{0,\m1\}$.
Tejto situácii zodpovedajú polynómy $Q_1(x)=x^3$ a $Q_2(x)=x^3+1$,
žiadny z~nich však zrejme nemá tri rôzne reálne korene. Nutne teda
platí $s\ne0$.

Po vydelení \thetag4 číslom $s\ne0$ dostaneme $s+2=p$. Dosadením
takého $p$ do~\thetag5 dostaneme kvadratickú rovnicu
$0=s^2-5s-6=(s-6)(s+1)$. Do úvahy tak prichádzajú jediné
dve riešenia $(s,r,p)\in\{(6,12,8),(\m1,\m2,1)\}$. Prvému z~nich
zodpovedá mnohočlen $Q_3(x)=x^3-6x^2+12x-8=(x-2)^3$ s~iba jedným
reálnym koreňom, druhému riešeniu mnohočlen $Q_4(x)=x^3+x^2-2x-1=0$.
Hľadaný mnohočlen~$Q(x)$ je teda nutne rovný polynómu $Q_4(x)$.

\Pozn
Výpočet čísel $s$, $r$, $p$ môžeme zjednodušiť, keď
využijeme úvodný poznatok z~prvého riešenia, podľa ktorého čísla
$a$, $b$, $c$ po prípadnom preznačení (danom zmenou ich poradia)
spĺňajú sústavu rovností
$$
a+b=bc,\quad b+c=ca,\quad c+a=ab.
\tag6$$
Porovnaním rozdielov ich ľavých a prislúchajúcich pravých strán
dostaneme rovnosti
$$
a-c=c(b-a),\quad b-a=a(c-b),\quad c-b=b(a-c).
$$
Ak porovnáme teraz súčin troch ľavých strán so súčinom troch pravých
strán, potom po vydelení nenulovým číslom $(a-b)(b-c)(c-a)$
už dostaneme prvú z troch neznámych hodnôt, konkrétne $p=abc=1$.

Ak prepíšeme nanovo rovnosti \thetag6 na tvar
$$
a=b(c-1),\quad b=c(a-1),\quad c=a(b-1),
$$
potom podobnou procedúrou vynásobenia vzhľadom na $abc\ne0$
dostaneme rovnosť $1=(a-1)(b-1)(c-1)$.
Z~nej po roznásobení a dosadení hodnoty $abc=1$ vyjde pre neznáme
$s=a+b+c$ a $r=ab+bc+ca$
rovnica $r=s-1$. Tá spolu s~rovnicou $2s=r$, ktorú získame
sčítaním rovností \thetag6, už vedie k~určeniu hodnôt $s=\m1$ a~$r=\m2$.
}

{%%%%%   A-III-4
Podobne ako v~domácom kole využijeme poznatok, že pre
superdeliteľa~$d(m)$ celého čísla~$m>1$ platí vzorec
$d(m)=m/p$, pričom~$p$ je najmenší prvočiniteľ čísla~$m$.


Rozložme teda hľadané~$n$ na prvočinitele: $n=p_1p_2\dots p_k$,
pričom~$p_1\geqq p_2\geqq\dots\geqq p_k$ sú prvočísla.
Potom vďaka úvodnému poznatku môžeme rovnosť zo zadania prepísať na tvar
$$
p_1\dots p_k + p_1\dots p_{k-1} + p_1\dots p_{k-2}+\cdots+p_1p_2
+p_1+1 = 2021.
$$
Po odčítaní čísla 1 postupným vynímaním na ľavej strane dostaneme
$$
p_1(1+p_2(1+p_3(1+\cdots+p_{k-1}(1+p_k)\dots)))=2020.
\tag1$$
Vidíme, že $p_1\mid 2020=101\cdot 5\cdot 2\cdot 2$. Keďže
$p_1$ je prvočíslo, máme 3~možnosti:

\item{a)}$p_1=101$. Z~\thetag1 vyplýva $1+p_2(\dots)=20$, takže $p_2 \mid 19$,
a teda $p_2=19$ (a~$k=2$), čo vďaka $101>19$ vedie naozaj
k~riešeniu $n=101\cdot 19=1919$.

\item{b)}$p_1=5$. Z~\thetag1 vyplýva $1+p_2(\dots)=404$, takže $p_2 \mid 403$.
Podľa $5=p_1\geqq p_2$ však prvočíslo $p_2$ môže byť iba $2$, $3$ alebo~$5$,
nie je to teda deliteľ čísla $403=13\cdot31$.

\item{c)}$p_1=2$. Z~\thetag1 vyplýva $1+p_2(\dots)=1010$, takže $p_2 \mid 1009$.
To však odporuje tomu, že podľa $2=p_1\geqq p_2$ je $p_2=2$.


\Zav Jediné vyhovujúce číslo je $n=1919$.
}

{%%%%%   A-III-5
Vo všetkých riešeniach budeme označovať $\overline{T}$ reťazec, ktorý
vznikne z~reťazca~$T$ \uv{otočením} jeho znakov, \tj. ich zapísaním
v~opačnom poradí (od posledného znaku po prvý).
Okrem toho budeme používať vnútri
reťazcov okrúhle zátvorky, ktoré nebudeme považovať za znaky, ale iba
za vizuálne oddeľovače. Budeme tiež pracovať s~prázdnym
reťazcom, ktorý je redukciou každého reťazca $aa$ s~dvoma rovnakými
znakmi.

Pre ľubovoľný reťazec~$X$ zrejme platí, že reťazec~$\overline{X}X$
možno sériou redukcií previesť na prázdny reťazec, a to postupným
odoberaním dvojíc rovnakých znakov \uv{zo stredu}.

Nech $X$ a $Y$ sú ľubovoľné (nie nutne neprázdne) reťazce
a nech~$P$ je nejaký pekný reťazec, ktorý možno sériou redukcií
previesť na~$XY$. Potom reťazec $\overline{X}\,\overline{X}P$ je tiež
pekný a~možno ho previesť na reťazec $\overline{X}(\overline{X}X)Y$,
ktorý možno podľa predchádzajúceho odseku ďalej previesť na $\overline{X}Y$.

Dokázané tvrdenie o~dvojici reťazcov $T=XY$ a $T_1=\overline{X}Y$
môžeme vysloviť takto: Ak možno daný reťazec~$T$ získať sériou
redukcií z~pekného reťazca, platí to isté aj o~každom reťazci~$T_1$,
ktorý získame z~reťazca~$T$ otočením niektorého jeho \uv{začiatku}~$X$
(môže byť aj $X=T$). Uvedomme si,
že postupnosťou takých otočení možno z~daného
reťazca~$T$ vytvoriť ľubovoľnú permutáciu jeho znakov. Naozaj,
ľubovoľný znak z~$T$ možno najskôr jedným otočením presunúť na
počiatok reťazca (ak tam už pôvodne nestojí) a~potom ďalším
otočením (celého reťazca) na jeho koniec. Opakovaním tejto procedúry
dokážeme (postupne od konca) vytvoriť akúkoľvek permutáciu znakov
reťazca~$T$.

Uvažujme teraz podľa zadania úlohy ľubovoľný reťazec~$T$,
ktorý obsahuje každý svoj znak v~párnom počte.
V~takom reťazci $T$ možno zrejme preusporiadať
znaky tak, aby vyšiel nejaký úhľadný reťazec $T'$.
Ten je už sám o~sebe pekný, takže podľa predchádzajúceho odseku
možno sériami redukcií pekných reťazcov získať každý taký reťazec,
ktorý vznikne z~reťazca~$T'$ permutáciou jeho znakov.
Keďže jedna taká permutácia dáva východiskový reťazec~$T$,
je riešenie úlohy ukončené.

\Pozn
K~výsledku úlohy dodajme, že reťazce obsahujúce každý svoj znak v~párnom
počte sú práve tie, ktoré sa dajú získať sériou redukcií
z~nejakého pekného reťazca. Jednu implikáciu sme dokázali
vyriešením zadanej úlohy, druhá vyplýva okamžite z~toho, že každá redukcia
daného reťazca zachováva paritu všetkých počtov jeho jednotlivých znakov.

\Jres
O~reťazci hovoríme, že je \emph{párny},
ak obsahuje každý znak v~párnom počte. Máme teda dokázať, že každý
párny reťazec možno získať sériou redukcií z~nejakého pekného reťazca.
Dôkaz urobíme matematickou indukciou podľa dĺžky párneho reťazca.

Najmenšiu párnu dĺžku 0 má iba prázdny reťazec, pre ktorý tvrdenie
platí triviálne.

\def\rat{\rightarrowtail}%
V druhom indukčnom kroku uvažujme párny neprázdny reťazec~$T$
a~označme~$a$ jeho prvý znak.
Písmeno~$a$ sa v~$T$ nachádza
ešte na niektorej ďalšej pozícii, čo umožňuje zapísať~$T$ v tvare $T=aXaY$,
pričom $X$ a~$Y$ sú vhodné (nie nutne neprázdne) reťazce.
Keďže reťazec~$\overline{X}Y$ je párny a~kratší ako~$T$ (o~dva znaky),
podľa indukčného predpokladu možno $\overline{X}Y$ získať
sériou redukcií z~vhodného pekného reťazca~$A$,
čo zapíšeme symbolicky takto: $A\rat \overline{X}Y$.
Potom $A'=(aXaX)A$ je tiež pekný reťazec,
ktorý dokážeme zredukovať na východiskový reťazec $T$ spôsobom,
pri ktorom vhodné série redukcií zapíšeme opäť symbolicky:
$$
A'=(aXaX)A\rat(aXaX)(\overline{X}Y)=aXa(X\overline{X})Y\rat
aXaY=T.
$$
(Séria redukcií reťazca $X\overline{X}$ na prázdny je zrejmá,
pozri prvé riešenie). Tým je dôkaz indukciou ukončený.

\Jres
Pozrime sa na riešenú situáciu \uv{od konca}: Do zadaného
reťazca~$T$, ktorý obsahuje každý svoj znak v~párnom počte,
sa budeme snažiť opakovane vkladať dvojice rovnakých znakov,
kým nedostaneme pekný reťazec.\fnote{Dvojicu rovnakých znakov
budeme nielen vkladať medzi niektoré dva susedné znaky aktuálneho
reťazca, ale aj prípadne pripisovať za jeho posledný znak.}
Každá čiastočná séria vkladaní bude vyzerať nasledovne.

\def\sip{\rightarrow}
V~aktuálnom reťazci vyberieme niektorý úsek $a_1a_2\ldots a_n$
zložený z~niekoľkých susedných znakov $a_i$.
Do tohto úseku postupne vložíme dvojice
$a_1a_1$, $a_2a_2$,\dots, $a_na_n$ takto:
$$
\displaylines{
(a_1a_2\ldots a_n)\sip(a_1a_2\ldots a_n)a_1a_1\sip
(a_1a_2\ldots a_n)a_1a_2a_2a_1\sip\cr
\sip(a_1a_2\ldots a_n)a_1a_2a_3a_3a_2a_1\sip
\cdots\sip(a_1a_2\ldots a_n)(a_1a_2\ldots a_n) (a_n a_{n-1} \ldots a_1).
}$$
Vidíme, že sme východiskový úsek $C=a_1a_2\ldots a_n$ otočili na
$\overline{C}=a_n a_{n-1} \ldots a_1$ s~tým, že sa nám pred
$\overline{C}$ zľava objavili dve nové kópie $C$ v~podobe úhľadného
reťazca $CC$. Celý postup úpravy zvoleného úseku $C=a_1a_2\ldots a_n$
vyjadríme skráteným zápisom s~jednou šípkou: $C\sip CC\overline{C}$.

Teraz už sme pripravení opísať celkový algoritmus úprav počiatočného
reťazca~$T$. Rovnako ako v~druhom riešení ho najskôr zapíšeme v~tvare $T=xAxB$, pričom $x$ je prvý znak $T$ a~$A$, $B$ sú vhodné
(nie nutne neprázdne) reťazce. Použitím navrhnutej úpravy postupne pre
úseky $C=xA$ a $C=\overline{A}xx$ dostaneme
$$\eqalign{
T=&(xA)xB\sip(xA)(xA)(\overline{A}x)xB =
(xA)(xA)(\overline{A}xx)B \sip\cr
&\sip(xA)(xA)(\overline{A}xx)(\overline{A}xx)(xxA)B=
(xAxA)(\overline{A}xx\overline{A}xx)(xx)AB.
}
\tag1$$
Všetky úseky v~zátvorkách posledného reťazca sú úhľadné,
pritom koncový úsek $T'=AB$ je o~2 kratší ako~$T$
(a má všetky svoje znaky zastúpené v~párnom
počte). Ak nie je $T'$ prázdny reťazec, zopakujeme v druhom kroku
úpravy \thetag1 pre $T'$ namiesto~$T$, atď. Je zrejmé, že po konečnom počte krokov
dostaneme pekný reťazec, ako sme si na úvod tohto riešenia vytýčili.
}

{%%%%%   A-III-6
V~limitnom prípade, keď $X=A$, trojuholník $X_aX_bX_c$ degeneruje na
úsečku,
ktorá leží na priamke obsahujúcej výšku trojuholníka $ABC$ na stranu $BC$.
Podobne to funguje v~prípadoch $X=B$ a~$X=C$. Spolu to naznačuje,
že hľadaným spoločným bodom všetkých trojuholníkov $X_aX_bX_c$ bude ortocentrum~$H$
trojuholníka $ABC$. Dokážeme, že je to naozaj tak.

Označme najskôr $Y_a$, $Y_b$, $Y_c$ kolmé priemety bodu~$X$
postupne na strany $BC$, $CA$, $AB$. Bod~$X$ podľa zadania leží
vnútri ostrouhlého trojuholníka $ABC$, takže $Y_a$, $Y_b$, $Y_c$ sú
vnútorné body prislúchajúcich strán a~navyše $X$ je vnútorným bodom
trojuholníka $Y_aY_bY_c$. Posledný fakt zdôvodníme takto:
Štvoruholníky $AY_bXY_c$, $BY_cXY_a$,
a~$CY_aXY_b$ sú podľa Tálesovej vety tetivové, a~teda aj konvexné,
takže bod~$X$
nie je bodom žiadneho z~trojuholníkov $AY_bY_c$, $BY_cY_a$ a $CY_aY_b$,
teda leží vo \uv{zvyšku} trojuholníka $ABC$, ktorým je vnútro trojuholníka
$Y_aY_bY_c$. Keďže body $Y_a$, $Y_b$, $Y_c$
sú postupne stredmi úsečiek $XX_a$, $XX_b$,
$XX_c$, z rovnoľahlosti ${\Cal H}(X,2)$ vyplýva, že bod~$X$
leží aj vnútri trojuholníka $X_aX_bX_c$.

V~prípade, keď $X=H$, záver predchádzajúceho odseku už znamená, že bod~$H$
leží vnútri $\triangle X_aX_bX_c$. Ak je $X$ vnútorným bodom
jednej z~úsečiek $AH$, $BH$ alebo $CH$, napríklad úsečky $AH$,
tak~$H$ je vnútorným bodom úsečky~$XX_a$, takže tiež
leží vnútri $\triangle X_aX_bX_c$, ako sme mali dokázať. Ostáva
teda rozobrať prípad, keď bod~$X$ neleží na žiadnej z~úsečiek
$AH$, $BH$ ani $CH$. Vtedy bod $X$ leží vnútri jedného
z~trojuholníkov $BHC$, $CHA$ alebo $AHB$. Nech je to $\triangle BHC$
bez ujmy na všeobecnosti.
\insp{a70iii.2}%

Keďže $CH \parallel XX_c$, úsečka~$XX_c$ pretína úsečku~$BH$
v~niektorom bode~$Z_c$. Analogicky vďaka $BH \parallel XX_b$
úsečka~$XX_b$ pretína úsečku~$CH$ v~niektorom bode~$Z_b$.
Použité rovnobežnosti navyše znamenajú, že vzniknutý štvoruholník $XZ_bHZ_c$
je rovnobežník. Z toho vyplýva, že úsečky $Z_bZ_c$ a $XH$ majú
spoločný stred.

Uvažujme teraz opäť rovnoľahlosť ${\Cal H}(X,2)$. Tá zobrazí
úsečku~$Z_bZ_c$ na určitú úsečku, ktorej stred je podľa
záveru predchádzajúceho odseku práve bod~$H$. Ak teda ukážeme, že
oba krajné body tejto úsečky, čiže obrazy bodov $Z_b$ a $Z_c$,
ležia v~$\triangle XX_bX_c$, bude to znamenať, že v~$\triangle XX_bX_c$
leží aj bod $H$. Tým budeme s~celým riešením hotoví,
lebo trojuholník $XX_bX_c$ je časťou trojuholníka $X_aX_bX_c$ ($X$ je totiž,
ako vieme, jeho vnútorný bod).
Potrebná vlastnosť obrazov $Z_b$ a $Z_c$ v rovnoľahlosti
${\Cal H}(X,2)$ však vyplýva okamžite z~nerovností
$$
|XX_c|=2|XY_c|>2|XZ_c|\quad\hbox{a}\quad |XX_b|=2|XY_b|>2|XZ_b|.
$$

\Jres
Odlišným spôsobom overíme, že ortocentrum~$H$
je (vnútorným) bodom trojuholníka $X_aX_bX_c$. Najskôr vysvetlíme, prečo nám
na to stačí dokázať, že úsečka~$X_bX_c$ pretína úsečky $AH$, $AX$
v~ich vnútorných bodoch. To totiž bude vďaka symetrii zadania znamenať,
že podobne pretína úsečka $X_cX_a$ úsečky $BH$, $BX$ a~úsečka~$X_aX_b$
úsečky $CH$, $CX$, takže oba body $H$ a $X$ budú ležať vnútri
prieniku troch polrovín opačných k~polrovinám $X_bX_cA$, $X_cX_aB$
a~$X_aX_bC$. Týmto prienikom ale musí byť trojuholník $X_aX_bX_c$,
lebo v~ňom leží bod $X$, ako vieme z~prvého riešenia (dôkaz tohto
tvrdenia tu opakovať nebudeme).

Na dôkaz pretnutia úsečiek $X_bX_c$ a $AX$ využijeme rovnosti
$|AX|=|AX_b|=|AX_c|$, ktoré vyplývajú z~použitých osových súmerností
a podľa ktorých je bod~$A$ stredom kružnice opísanej trojuholníku
$XX_bX_c$. Keďže pre obvodový uhol $X_bXX_c$ v~tejto kružnici
platí
$$
|\uhel X_bXX_c|=180^{\circ}-|\uhel BAC|>90^{\circ},
$$
lebo podľa zadania $\al=|\uhel BAC|<90^{\circ}$,
pretína tetiva $X_bX_c$ polomer $AX$ v~jeho vnútornom bode, ako
sme chceli dokázať.

Pre dôkaz potrebného tvrdenia o~úsečkách $X_bX_c$ a $AH$
si najskôr všimneme, že konvexný uhol $X_cAX_b$ má vďaka konštrukcii bodov
$X_b$, $X_c$ veľkosť $2\al$ a úsečka $AH$ leží v~tomto uhle.
Ak preto dokážeme nerovnosť $|\uhel AX_cX_b|<|\uhel AX_cH|$, bude to
už znamenať, že úsečka $X_bX_c$ naozaj pretína úsečku $AH$
v~jej vnútornom bode.
\insp{a70iii.3}%

Ako už vieme, trojuholník $AX_bX_c$ je rovnoramenný a má pri hlavnom
vrchole $A$ uhol veľkosti $2\al$. Preto majú oba zvyšné
uhly pri vrcholoch $X_b$, $X_c$ veľkosť $90^{\circ}-\al$, akú má
zrejme aj uhol $ABH$. Preto je potrebná nerovnosť
$|\uhel AX_cX_b|<|\uhel AX_cH|$ ekvivalentná s~nerovnosťou
$|\uhel ABH|<|\uhel AX_cH|$, ktorú vo zvyšku riešenia dokážeme
nasledujúcim postupom.

Je známe, že bod $H$ leží na kružnici $k'$, ktorá je obrazom
kružnice $k$ opísanej trojuholníku $ABC$ v~osovej súmernosti podľa
priamky~$AB$, presnejšie na tom jej oblúku~$AB$, ktorý neobsahuje
obraz $C'$ vrcholu $C$.\fnote{Tento poznatok o~ostrouhlom trojuholníku $ABC$
možno pri zvyčajnom označení veľkostí uhlov dokázať takto:
vďaka $AH\perp BC$ platí $|\uhel BAH|=90^\circ-\be$,
podobne vďaka $BH\perp AC$ platí $|\uhel ABH|=90^\circ-\al$,
odkiaľ $|\uhel AHB|=180^\circ-(90^\circ-\al)-(90^\circ-\be)=
\al+\be=180^\circ-\ga$, zatiaľ čo $|\uhel AC'B|=|\uhel ACB|=\ga$;
navyše body $C'$ a $H$ ležia v~opačných
polrovinách vyťatých priamkou $AB$.} Keďže bod $X$ leží vnútri
$\triangle ABC$, leží jeho obraz $X_c$ vnútri $\triangle ABC'$, a
teda aj vnútri kruhového odseku kružnice~$k'$ určenej oblúkom $AC'B$.
Z toho už dostávame
$$
|\uhel ABH|=|\uhel AC'H|<|\uhel AX_cH|,
$$
ako sme potrebovali ukázať.

\Pozn
Z~dokázaného tvrdenia o~pretnutí úsečiek v~každej z dvojíc
$(AH,X_bX_c)$, $(BH,X_bX_c)$ a $(CH,X_cX_a)$ vyplýva nielen (v~riešení
využitý) záver, že bod $H$ leží vnútri prieniku troch polrovín
opačných k~$X_bX_cA$, $X_cX_aB$ a~$X_aX_bC$, ale aj
záver, že bod $H$ leží vnútri troch (konvexných) uhlov
$X_bAX_c$, $X_cBX_a$ a $X_aCX_b$.
Ani oba tieto závery však samy osebe neimplikujú, že bod $H$ leží
v~trojuholníku $X_aX_bX_c$. Nie je nimi totiž vylúčená situácia
z~\obrc2+\obr, na ktorom je neprázdnym prienikom troch spomenutých polrovín
a troch spomenutých uhlov žlto vyfarbená oblasť.
\insp{a70iii.4}%
Vylúčiť túto situáciu a súčasne dokázať, že z~tvrdenia
o~dvojiciach úsečiek $(AH,X_bX_c)$, $(BH,X_bX_c)$, $(CH,X_cX_a)$
vyplýva potrebný záver $H\in\triangle X_aX_bX_c$, môžeme nasledovne.

Uvažujme šesťuholník ${\Cal M}=AX_cBX_aCX_b$, ktorý dostaneme, keď
k~trojuholníku $ABC$ pozdĺž jeho strán \uv{prilepíme} zvonka trojuholníky $ABX_c$, $BCX_a$ a $CAX_b$, ktoré sú súmerne
združené postupne s~trojuholníkmi $ABX$, $BCX$ a $CAX$. Tento popis vzniku
šesťuholníka $\Cal M$ nám určuje jeho vnútro, podľa ktorého
teraz rozhodneme o~konvexnosti všetkých jeho vnútorných uhlov (presnejšie
máme na mysli, že veľkosti týchto uhlov sú menšie ako
$180^{\circ}$). O~uhloch pri vrcholoch $A$, $B$, $C$ to už vieme
z~riešenia. Uhly pri zvyšných vrcholoch $X_a$, $X_b$, $X_c$ sú však zhodné postupne
s~uhlami $AXB$, $BXC$, $CXA$, takže sú aj konvexné, keďže bod $X$ leží
vnútri trojuholníka $ABC$. Šesťuholník~$\Cal M$ je teda konvexný.
Z~toho, že úsečka $AH$ pretína úsečku~$X_bX_c$, tým pádom vyplýva,
že bod~$H$ leží v~polrovine $X_aX_bX_c$, lebo tá je vďaka
konvexnosti~$\Cal M$ polrovinou opačnou k~$X_bX_cA$. Analogickou
úvahou o~úsečkách $BH$ a $CH$ dokopy dostaneme, že bod $H$
leží vo všetkých troch polrovinách $X_aX_bX_c$, $X_bX_cX_a$ a~$X_cX_aX_b$,
teda naozaj platí $H\in\triangle X_aX_bX_c$.
}

{%%%%%   B-S-1
Ak je $x+y\geqq0$, tak podľa prvej zadanej rovnice platí
$x+y=1-z$, takže hľadaný súčet $x+y+z$ je rovný~1. K rovnakému záveru dôjdeme, keď bude platiť $y+z\geqq0$ alebo $z+x\geqq0$. Ostáva posúdiť prípad, keď všetky tri súčty $x+y$, $y+z$, $z+x$ sú záporné. Vtedy možno danú sústavu prepísať na tvar
%\label[rov1.2]
$$\eqalign{
- x-y&=1-z,\cr
- y-z&=1-x,\cr
- z-x&=1-y.}
\tag1$$
Sčítaním týchto troch rovníc dostaneme $x+y+z={-3}$. Súčet $x+y+z$ teda vždy nadobúda buď hodnotu $1$, alebo hodnotu ${-3}$. Ukážeme, že obe tieto hodnoty sú na riešeniach zadanej sústavy dosiahnuteľné.

Nájsť potrebné trojice $(x,y,z)$ bude jednoduché. Ako sa totiž
ukáže, stačí sa pritom obmedziť na trojice rovnakých čísel $x=y=z$. Vtedy sa zadaná sústava redukuje na jedinú rovnicu $|2x|=1-x$. Tá má zrejme dve riešenia $x={-1}$ a~$x=\frac13$. Tým zodpovedajú
riešenia ${(-1,-1,-1)}$ a~$(\frac13,\frac13,\frac13)$ pôvodnej
sústavy, ktoré už potvrdzujú, že oba súčty $x+y+z={-3}$
aj~$x+y+z=1$ sú dosiahnuteľné.\fnote{Trojice ${(-1,-1,-1)}$
a~$(\frac13,\frac13,\frac13)$ sme mohli rýchlejšie získať priamou voľbou $x=y=z$ v~rovniciach $x+y+z={-3}$, resp. $x+y+z=1$, a~potom sa dosadením presvedčiť, že to sú riešenia.}

\Pozn
Rovnosť $x+y+z={-3}$ spĺňa {\it jediné\/} riešenie
${(-1,-1,-1)}$ zadanej sústavy, keďže to je jediné riešenie sústavy \ref[rov1.1].\fnote{Ak totiž trojica $(x,y,z)$ spĺňa pôvodnú sústavu, nie však \ref[rov1.1],
musí byť jeden zo súčtov $x+y$, $y+z$, $z+x$ kladný, takže potom (ako vieme) platí $x+y+z=1$.}
Rovnosť $x+y+z=1$
spĺňa {\it nekonečne veľa\/} riešení $(x,y,z)$; všetky sú určené podmienkou

\medskip\centerline{$(x\leqq1)\land (y\leqq1)\land (z\leqq1)\land (x+y+z=1)$.\fnote{Množinou všetkých takých trojíc $(x,y,z)$ je v~karteziánskej sústave súradníc (rovnostranný) trojuholník s~vrcholmi $[1,1,{-1}]$, $[1,{-1},1]$ a $[{-1},1,1]$. Z~nerovníc $x\le 1$, $y\le 1$ totiž vyplýva $z=1-x-y\ge-1$ a podobne pre $x$, $y$. Vyhovujúce body tak ležia v~prieniku kocky s~vrcholmi $[\pm1,\pm1,\pm1]$ s~rovinou $x+y+z=1$, čo je vyššie spomenutý trojuholník.}}

\medskip\noindent
Prvé tri nerovnosti vyjadrujú, že pravé strany rovníc sústavy
sú nezáporné; ak sú splnené, tak z~rovnosti $x+y+z=1$
vyplýva, že sú nezáporné aj súčty $x+y$, $y+z$, $z+x$ (rovné postupne $1-z$, $1-x$, $1-y$), takže $(x,y,z)$ je riešením.

%\break
\Jres
Pri označení $s=x+y+z$ možno sústavu prepísať na tvar
%\label[rov1.3]
$$\eqalign{
|s-z|&=1-z,\cr
|s-x|&=1-x,\cr
|s-y|&=1-y.}
\tag2$$
Považujme ďalej číslo $s$ v~sústave \ref[rov1.2] za (nezávislý) parameter. Našou úlohou je zistiť, pre ktoré hodnoty $s$ existuje riešenie $(x,y,z)$ sústavy \ref[rov1.2] s~vlastnosťou $x+y+z=s$.

Ak je $s=1$, tak riešeniami sústavy \ref[rov1.2] sú (podľa definície absolútnej hodnoty) práve tie trojice $(x,y,z)$, pre ktoré platia nerovnosti $1-z\geqq0$, $1-x\geqq0$, $1-y\geqq0$. Príkladom takého riešenia s~vlastnosťou $x+y+z=s=1$ je napríklad trojica $(\frac13,\frac13,\frac13)$.

Ostáva posúdiť prípad, keď $s\ne1$. Vtedy $s-z\ne1-z$, takže
z~prvej rovnice v~\ref[rov1.2] vyplýva $s-z\ne|s-z|$, čiže $s-z<0$. Preto možno prvú rovnicu v~\ref[rov1.2] prepísať ako $-(s-z)=1-z$, odkiaľ $z=\frac12(s+1)$. Analogicky musí platiť $x=\frac12(s+1)$ a~$y=\frac12(s+1)$. Z~podmienky $x+y+z=s$ teraz zapísanej ako $\frac32(s+1)=s$ vyplýva, že je nutne $s={-3}$. Ľahko sa presvedčíme, že zodpovedajúca trojica $(-1,-1,-1)$ je riešením.

Dospeli sme k~rovnakému záveru ako v~prvom riešení: jediné dve možné hodnoty súčtu $x+y+z$ sú čísla $1$ a~${-3}$.

\Jres
Ešte jedným algebraickým postupom dokážeme, že $x+y+z\in\{\m3,1\}$.
(Otázku dosiahnuteľnosti oboch hodnôt už opakovane posudzovať nebudeme.)

Absolútnych hodnôt v~sústave sa zbavíme tak, že každú z~troch rovníc
umocníme na druhú. Keď ich následne sčítame, dostaneme
$$
(x+y)^2+(y+z)^2+(z+x)^2=(1-z)^2+(1-x)^2+(1-y)^2.
$$
Z toho po jednoduchej úprave dostaneme rovnicu
$$
(x+y+z)^2+2(x+y+z)-3=0,
$$
ktorá je kvadratickou rovnicou $s^2+2s-3=0$ pre skúmaný
súčet $s=x+y+z$. Jej korene sú $s={-3}$ a~$s=1$. Tým je sľúbený
dôkaz ukončený.

\schemaABC
Za úplné riešenie úlohy dajte 6~bodov, z~toho: 4~body za dôkaz, že súčet $s=x+y+z$ nemôže nadobúdať iné hodnoty ako $1$ alebo $-3$; 1~bod za príklad riešenia $(x,y,z)$ so súčtom $s=1$; 1~bod za uvedenie (jediného) riešenia $(-1,-1,-1)$ so súčtom $s=-3$.

Za neúplný dôkaz tvrdenia $s\in\{-3,1\}$ možno získať čiastkové body (napríklad pri neúplnom rozbore možných znamienok súčtov $x+y$, $y+z$, $z+x$), nanajvýš však 2~body, ak je v~postupe logická chyba (napríklad zabudnutie nejakého prípadu možných znamienok).

Ak sa riešiteľ snaží opísať množinu všetkých riešení danej sústavy, žiadne body navyše tým nezískava.
\endschema
}

{%%%%%   B-S-2
Úsečky $CF$ a~$DE$ z~dokazovanej rovnosti sú stranami trojuholníkov
$ADE$ a~$AFC$. Preto stačí dokázať, že tieto dva trojuholníky sú
zhodné, $\triangle ADE\cong\triangle AFC$.

Označme $\alpha=|\uhel BAC|$. Zdôraznime vopred, že vďaka zadanej podmienke $\alpha<60\st$, ktorá znamená, že $3\alpha<180\st$, budeme môcť počítať veľkosti (konvexných) uhlov so spoločným vrcholom $A$ podľa polôh ich ramien ako na \obr{}.\fnote{Použité osové súmernosti totiž zaručujú, že
ramená $AE$, $AB$, $AC$, $AD$, $AF$ ležia v~tomto poradí pri
otáčaní okolo vrcholu $A$.}

Osové súmernosti, ktoré určujú konštrukciu bodov $D$, $E$ a~$F$,
v prvom rade znamenajú, že platia rovnosti $|AD|=|AB|=|AF|$
a~$|AE|=|AC|$, ako sme dvoma farbami vyznačili na obrázku.
Ďalším dôsledkom je, že veľkosť $\alpha$ má nielen uhol $BAC$, ale majú ju aj uhly $CAD$ a~$EAB$, takže konvexný uhol $BAD$ má veľkosť $2\alpha$ a~konvexný uhol $EAD$ má veľkosť $3\alpha$. Posledným potrebným dôsledkom súmerností je, že veľkosť $2\alpha$ má nielen uhol $BAD$, ale aj uhol $DAF$, a~preto konvexný uhol $CAF$ má veľkosť $3\alpha$, teda rovnakú ako skôr uvedený uhol
$EAD$.
\insp{b70s_2.11}%

Z~odvodených rovností $|AD|=|AF|$, $|AE|=|AC|$ a~$|\uhel EAD|=|\uhel CAF|$ podľa vety $sus$ už vyplýva zhodnosť $\triangle ADE\cong\triangle AFC$, ktorú sme chceli dokázať.

Dodajme, že úvahy o~zhodných trojuholníkoch $ADE$, $AFC$ možno bez zmienky o~nich opísať úvahou o~otočení so stredom v~bode $A$ o~orientovaný uhol $EAC$. Podľa výsledkov z~tretieho odseku nášho riešenia je pri takom otočení obrazom úsečky $DE$ práve úsečka~$FC$.

\schemaABC
Za úplné riešenie úlohy dajte 6~bodov, z~toho:
1~bod za uvedenie rovnosti $|AE|=|AC|$; 1~bod za dôkaz rovnosti $|AD|=|AF|$; 1~bod za dôkaz rovností $|\uhel EAB|=|\uhel BAC|=|\uhel CAD|$; 1~bod za dôkaz $|\uhel DAF|=2|\uhel BAC|$ a zvyšné dva body za dokončenie dôkazu
použitím vety $sus$, kosínusovej vety či otočením so stredom v~$A$. Z~týchto posledných dvoch bodov môže jeden byť udelený pri neúplnom riešení, ak riešiteľ uvedie, že uvedená zhodnosť či otočenie stačia na dokončenie úlohy.
\endschema
}

{%%%%%   B-S-3
Predpokladajme, že čísla $a$, $b$, $c$ sú prirodzené a~že
$N=3^a\cdot7^b-10^c$ je kladné dvojciferné číslo. Číslo $N$
nie je deliteľné žiadnym zo štyroch najmenších prvočísel 2, 3, 5 a~7, lebo každým z~nich je deliteľné práve jedno z~čísel $3^a\cdot7^b$ a~$10^c$, takže ich rozdiel ním deliteľný nie je.

Piate najmenšie prvočíslo je 11. Z~rovnosti $11^2=121$ vyplýva, že každé zložené číslo, ktoré nie je deliteľné žiadnym
z~prvočísel 2, 3, 5, 7, má aspoň tri cifry.\fnote{Tento
(alebo jemu ekvivalentný) záver je v~úplnom riešení nutné objasniť. Možno sa pritom odvolať na známy školský poznatok, že každé zložené číslo $N$ má aspoň jedného prvočiniteľa $p$, ktorý spĺňa nerovnosť $p\leqq\sqrt{N}$.}
Naše dvojciferné číslo $N$ teda nie je zložené, je preto nutne prvočíslo, ako sme mali dokázať.

\Pozn
Je zrejmé, že dvojciferné čísla tvaru $3^a\cdot7^b-10^c$ naozaj existujú, napríklad $3\cdot7-10=11$, $3^3\cdot7-10^2=89$, $3\cdot7^3-10^3=29$ a pod.

\schemaABC
Za úplné riešenie úlohy dajte 6~bodov. V prípade neúplného riešenia dajte 3~body za pozorovanie, že $N$ nie je deliteľné prvočíslami z~množiny $\{2,3,5,7\}$; v~prípade, že riešiteľ spomenie nedeliteľnosť iba dvoma z~nich, dajte 1~bod, v~prípade, že ukáže nedeliteľnosť tromi prvočíslami, dajte 2~body (zmienku, že nie je deliteľné práve jedným z~prvočísel, typicky $N$ je nepárne, nebodujte). Ďalšie 3~body dajte za tvrdenie, že každé zložené dvojciferné číslo je deliteľné aspoň jedným z~prvočísel $2$, $3$, $5$ alebo $7$
(ak nie je toto konštatovanie ničím podložené, strhnite 1~bod).
\endschema
}

{%%%%%   B-II-1
a) Po roznásobení dostaneme ekvivalentnú nerovnosť
$4a^2+4b^2>a^2+b^2+3ab$, ktorú ešte upravíme na tvar
$3a^2+3b^2>3ab$, v~ktorom ju dokážeme.
Ak je $a\geqq b>0$, potom $3a^2\geqq 3ab$ a
$3b^2>0$. Sčítaním týchto dvoch nerovností už dostaneme
dokazovanú nerovnosť. Ak je naopak $b\geqq a$,
postupujeme analogicky (alebo sa len odvoláme na symetriu).

Inou možnosťou je napríklad vyjsť zo zrejmej nerovnosti
$3(a-b)^2\geqq 0$, ktorú upravíme na tvar $3a^2+3b^2\geqq 6ab$.
Určite platí $6ab>3ab$ (keďže obe čísla $a$, $b$ sú kladné),
a tak dokopy dostávame $3a^2+3b^2\geqq 6ab>3ab$ a sme hotoví.

\medskip
b) Zadanú nerovnosť ekvivalentne upravíme na tvar
$(k-1)(a^2+b^2)\geqq 3ab$.
Dosadením $a=b=1$ dostaneme $2(k-1)\geq3$, čiže $k\geqq5/2$,
a tak každé vyhovujúce číslo~$k$ je nutne aspoň $5/2$
(v~prípade $k<5/2$ zadaná nerovnosť neplatí napríklad
pre dvojicu $a=b=1$)\fnote{V úplnom riešení stačí uviesť jednu takú
dvojicu, aj keď požadovanú vlastnosť čísla $k<5/2$ vyvracia
napríklad aj každá iná dvojica sebe rovných kladných čísel $a$, $b$.}.
Pre $k=5/2$ pritom máme
$$
(k-1)(a^2+b^2) = \frac32(a^2+b^2)\geqq\frac32\cdot2ab=3ab,
$$
pričom sme využili všeobecne platnú nerovnosť $a^2+b^2\geq 2ab$,
ktorá je prepisom zrejmej nerovnosti $(a-b)^2\geqq0$.\fnote{Možno
sa tiež odvolať na AG-nerovnosť $\frac12(u+v)\geqq\sqrt{uv}$
pre kladné čísla $u=a^2$ a $v=b^2$.}
Došli sme k~záveru, že hľadané najmenšie $k$ je rovné $5/2$.


\schemaABC
Za úplné riešenie dajte 6 bodov, z~toho 2 body za časť a)
a 4 body za časť b).
%
V~časti b) dajte 1 bod za nájdenie hľadanej konštanty $k=5/2$ (aj
v~prípade, že je uhádnutá), ďalšie 2~body za dôkaz
nerovnosti pre $k=5/2$ a 1 bod za zdôvodnenie, že pre $k<5/2$
nerovnosť všeobecne neplatí.
\endschema

}

{%%%%%   B-II-2
Uvažujme jednociferné číslo $d$. Pozrieme sa,
koľkokrát číslo $d$ ako deliteľ prispieva
do súčtu $d_1+d_2+\dots+d_n$.
Napríklad $d$ rovné piatim prispieva jednotkou číslam $d_5, d_{10},
d_{15}, \dots$, ostatným číslam neprispieva. Z toho je jasné, že
každé $d$ prispieva do súčtu $d_1+d_2+\dots+d_n$ číslom
$\lfloor n/d\rfloor$, pričom $\lfloor x\rfloor$ označuje dolnú celú časť čísla
$x$. Sčítaním týchto príspevkov pre $d=1,2,\dots,9$ tak dostaneme
vyjadrenie
$$
d_1+d_2+\dots+d_n=\sum_{d=1}^9\left\lfloor
\frac{n}{d}\right\rfloor.
\tag1
$$
Ak teraz uplatníme zrejmú nerovnosť $\lfloor x\rfloor\leqq x$
pre čísla $x=n/d$, bude nám na vyriešenie úlohy stačit dokázať
druhú z~nerovností
$$
\frac{1}{n}\left(d_{1}+d_{2}+\dots+d_n\right)\leq
\frac{1}{n}\sum_{d=1}^9 \frac{n}{d}<3.
\tag2
$$
Tá po skrátení čísla $n$ prejde na tvar
$$
\frac11+\frac12+\dots+\frac19<3.
\tag3
$$
Namiesto rutinného výpočtu ľavej strany nerovnosti ($7129/2520$) si môžeme všimnúť, že platí
$$\displaylines{
\frac12+\frac13+\frac16=1,\cr
\frac14+\frac15<\frac24=\frac12,\cr
\frac17+\frac18+\frac19<\frac37<\frac12.
}$$
Sčítaním týchto troch nerovností a pričítaním jednotky k~obom
stranám už dostávame potrebnú nerovnosť \ref[rov:2.3]
dokonca takmer bez počítania.

\Pozn
Použitie funkcie $y=\lfloor x\rfloor$ v~predloženom riešení nie je nevyhnutné:
Príspevky deliteľov~$d$ do súčtu $d_1+d_2+\dots+d_n$ možno
zrejme priamo odhadnúť zhora zlomkami~$n/d$ a~potom ich sčítaním
rovno dospieť k~prvej z~nerovností~\ref[rov:2.2].

\schemaABC
Za úplné riešenie dajte 6 bodov, z~toho: 3 body za myšlienku, že pre
daný deliteľ $d$ sa pozrieme na jeho príspevok do súčtu
$d_1+d_2+\dots+d_n$ (jedná sa o~metódu počítania jedného množstva,
v~danom prípade platných relácií $d\mid k$ ($d=1,2,\dots,9$,
$k=1,2,\dots,n$), dvoma
spôsobmi); ďalšie 2 body za odvodenie prvej nerovnosti v~\ref[rov:2.2] (z~toho
1 bod za vyjadrenie \ref[rov:2.1], ak volí riešiteľ túto cestu); 1 bod za
akékoľvek správne overenie nerovnosti~\ref[rov:2.3].
\endschema
}

{%%%%%   B-II-3
Označme $P$ pätu kolmice $p$ z~bodu $D$ na~preponu
$AB$ uvažovaného pravouhlého trojuholníka $ABC$.\fnote{Na
priamke $p$ tak uvažujeme celkom 4~body, ktoré sú zrejme
v~poradí $E$, $P$, $D$, $F$, čo nie je v~riešení nutné
ani spomínať, tobôž nie dokazovať.}

Štvoruholník $AEBD$ je tetivový, takže podľa vlastnosti
obvodových uhlov platí $|\uhel ABE|=|\uhel ADE|$. Všimnime si ďalej,
že pravouhlé trojuholníky $ABC$ a $ADP$ majú spoločný ostrý vnútorný uhol
pri vrchole $A$, a tak majú zhodné ostré vnútorné uhly aj pri
vrcholoch $B$ a $D$. Dokopy dostávame
$$
|\uhel ABF|=|\uhel ABC|=|\uhel ADP|=|\uhel ADE|=|\uhel ABE|.
$$
Dokázaná zhodnosť uhlov $ABF$, $ABE$, čiže uhlov $PBF$, $PBE$
pre trojuholník $EBF$ znamená, že jeho výška $BP$ leží
na osi vnútorného uhla pri vrchole $B$ (\obr).
Tento trojuholník je teda rovnoramenný
a priamka~$BP$ je osou jeho základne $EF$.\fnote{Tento zo školskej
výučby známy záver, ktorý vyplýva zo zhodnosti trojuholníkov $BPE$ a $BPF$
podľa vety $usu$, nie je nutné v~riešení dokazovať.}
Keďže bod~$A$ na tejto osi leží tiež,
platí rovnosť $|AE|=|AF|$, ktorú sme mali dokázať.
(Navyše sme ukázali, že štvoruholník $AEBF$ je deltoid.)
\insp{b70ii_p.2}%

\Jres
Bod $D$ je ortocentrom ostrouhlého
trojuholníka $ABF$, lebo je priesečníkom jeho výšky $AC$ s~výškou
z~vrcholu $F$, ktorá totiž leží na priamke $p$. Z toho vyplýva
$BD\perp AF$, a~preto oba uhly $ABD$ a $AFD$ sú doplnky toho istého uhla
$BAF$ do $90^\circ$, teda sú zhodné. S~uhlom $ABD$
je navyše zhodný aj uhol $AED$, keďže sa jedná o~obvodové uhly
nad tetivou $AD$ kružnice opísanej štvoruholníku $AEBD$ (\obr).
Dokopy dostávame $|\uhel AED|=|\uhel AFD|$, čiže
$|\uhel AEF|=|\uhel AFE|$, čo znamená, že trojuholník $AEF$ je
rovnoramenný, \tj. naozaj platí $|AE|=|AF|$.
\insp{b70ii_p.3}%

\schemaABC
Za úplné riešenie úlohy dajte 6 bodov.

Pri postupe z~prvého riešenia dajte: 2 body za dôkaz rovnosti
$|\uhel ADE|=|\uhel ABE|$; 2 body za zdôvodnenie rovnosti
$|\uhel ADE|=|\uhel ABF|$; 2 body za úvahu vedúcu
od dokázaných rovností uhlov k~rovnosti $|AE|=|AF|$.

Pri postupe z~druhého riešenia dajte:
2 body za dôkaz rovnosti $|\uhel AED|=|\uhel ABD|$;
3 body za zdôvodnenie rovnosti $|\uhel ABD|=|\uhel AFD|$
(z~toho 1 bod za pozorovanie, že bod $D$ leží na dvoch výškach
trojuholníka $ABF$ a 1 bod za z toho vyplývajúci záver $BD\perp AF$);
1 bod za úvahu vedúcu od dokázaných rovností uhlov k~rovnosti
$|AE|=|AF|$. Po zistení, že bod $D$ je ortocentrom trojuholníka
$ABF$, možno riešenie tiež dokončiť použitím všeobecného poznatku,
že kružnice opísané trojuholníkom $ABF$ a $ABD$ sú súmerne združené
podľa priamky $AB$. Riešiteľ to môže prehlásiť za známe a
nedokazovať.

Body z~oboch postupov riešenia sa nesčítajú. V~prípade, že
riešiteľ naznačí obe cesty, ale ani jednu nedokončí, hodnoťte tú cestu,
na ktorej pozbiera viac bodov. Ak napríklad riešiteľ dokáže ako
rovnosť $|\uhel ADE|=|\uhel ABE|$, tak rovnosť $|\uhel AED|=|\uhel ABD|$, získava za to iba 2 body.
Ak k~tomu ešte napríklad dokáže, že $BD\perp AF$,
získava už 4 body.
\endschema

}

{%%%%%   B-II-4
a) Štvorec $9\times 9$ rozdelíme na 9 štvorcov
$3\times 3$. Do každého z~nich musíme určite vystreliť aspoň
dvakrát, \tj. celkom potrebujeme aspoň 18 výstrelov.
Príklad na \obr{} ukazuje, že 18 výstrelov stačí.
\insp{b70ii_p.4}%

b) Predpokladajme teraz, že máme k~dispozícii 18 výstrelov na hrací
plán $11\times 9$ tvorený 11 riadkami a 9 stĺpcami.
Riadky označme poradovými číslami 1 až 11.

Do ľubovoľných troch susedných riadkov sa vojdú tri neprekrývajúce sa lode,
na ktoré teda potrebujeme aspoň 6 výstrelov.
Takže 6 výstrelov potrebujeme ako na susedné riadky 6, 7 a 8,
tak aj na susedné riadky 9, 10 a 11. Na prvých päť riadkov nám
teda ostáva nanajvýš 6 výstrelov. Pritom v~riadkoch 1, 2 a 3 musí
byť aspoň 6~zásahov, rovnako ako v riadkoch 3, 4 a 5. Z toho
vyplýva, že v~prvých piatich riadkoch musí všetkých 6 zásahov ležať
v~3. riadku. Vždy tak niektoré jeho políčko zostane nezasiahnuté.
Umiestnime loď do riadkov 2, 3 a 4 tak, aby \uv{prechádzala}
\emph{vybraným} nezasiahnutým políčkom 3. riadka. Loď je potom zasiahnutá
nanajvýš raz (lebo zásahy v 2. aj 4.~riadku sme vylúčili),
pozri \obr. Tým sme dokázali, že $18$ výstrelov pre hrací plán
$11\times 9$ nestačí.
\insp{b70ii_p.5}%


\Jres
Opíšme odlišný postup pre časť b).
Znova predpokladajme, že máme k~dispozícii 18 výstrelov na hrací
plán $11\times 9$, tvorený 11 riadkami a 9 stĺpcami. Riadky opäť
označíme poradovými číslami 1 až~11. Rozdeľme plán na tri časti
s~rozmermi ${11\times 3}$. Vyznačili sme ich na \obr{}
spolu so všetkými 18~zásahmi, ktorých spôsob rozmiestnenia je,
ako ukážeme, zadaním úlohy vynútený. Na to budeme v~nasledujúcom odseku
potrebovať iba lode, z ktorých každá leží celá v~jednej z
troch vymedzených častí.\fnote{Toto konštatovanie
je dôležité pre dôkaz uvedený v poznámke za riešením.}
\insp{b70ii_p.7}%

Z celkového počtu 18 zásahov ich musí byť 6 v každej z
troch uvažovaných častí, lebo do akejkoľvek jej podčasti
$9\times3$ sa vojdú tri neprekrývajúce sa lode.
Ak navyše zvolíme
tieto podčasti v~riadkoch 3 až 11, resp. 1 až 9,
zistíme, že žiadny zásah sa nemôže vyskytovať v~riadkoch 1, 2, 10 a
11. Preto v~každej z troch uvažovaných častí musí byť po
2~zásahoch v~riadkoch 3 a 9 -- kvôli lodiam v~riadkoch 1 až 3,
resp. 9~až~11. Zvyšné 2~zásahy potom musia byť v~riadku 6 --
kvôli lodi v~riadkoch 4 až 6 a~lodi
v~riadkoch 6~až~8. Určené dvojice zásahov v~riadkoch 3, 6, 9
musia byť ale rozmiestnené ako na \obrr1{} -- kvôli lodiam v~trojiciach
riadkov 2 až 4, 5 až 7, 8 až 10.

Príklad lode vykreslenej na \obrr1{}
dokazuje, že 18 výstrelov na celý hrací plán $11\times9$ nestačí.

\Pozn
Využime poznatky z~druhého riešenia na krátky dôkaz,
že ani 19 výstrelov pre hrací plán $11\times9$ nestačí.

Rozdeľme opäť hrací plán na tri časti $11\times3$. Podľa druhého
riešenia musí byť všetkých 19~výstrelov rozdelených na tieto častí v~počtoch
6, 6 a 7. Dve časti so 6~zásahmi, ktorých nutné rozmiestnenie podľa druhého
riešenia poznáme, nemôžu byť susedné, lebo
v~opačnom prípade by sme podľa \obrr1{} našli loď
s~iba jedným zásahom. Časť so 7 zásahmi je teda prostredná a
v~každej z~oboch krajných častí je nutné rozmiestnenie 6 zásahov známe --
pozri \obr. Sú na
ňom navyše vykreslené 4 lode, ktoré vedú k~záveru, že
v~prostrednej časti by muselo byť aspoň 8 zasiahnutých políčok, a to
je spor.
\inspinsp{b70ii_p.8}{b70ii_p.6}%

Príklad z~\obr{} dosvedčuje, že 20 výstrelov na hrací plán
$11\times9$ už stačí.


\schemaABC
Za úplné riešenie dajte 6 bodov, z~toho 3 body za časť a) a 3 body
za časť b).

V~časti a) dajte 1 bod za úvahu, prečo 17 výstrelov nestačí, a 2
body za vyhovujúci príklad 18~výstrelov.

Za časť b) môžu získať body iba tí riešitelia, ktorí sa
rozhodnú zdôvodňovať, že 18 výstrelov nestačí (ale za sformulovanie
tejto hypotézy žiadny bod ešte neudeľujte).

Pri postupe z časti b) prvého riešenia,
založenom na úvahách o celých riadkoch, dajte:
1 bod za dôkaz existencie 5 susedných riadkov s nanajvýš 6 zásahmi;
ďalší 1 bod za zdôvodnenie, že medzi týmito 5~riadkami existuje riadok,
ktorý z nich nie je krajný a ktorého oba susedné riadky sú bez zásahu;
1 bod za určenie polohy lode, ktorá vedie ku sporu.

Pri postupe z druhého riešenia dajte: 2 body za odvodenie
nutného rozmiestnenia 6 zásahov v~každej z troch častí $11\times3$
(len za zistenie, že v každej časti je práve 6 zásahov, dajte 1 bod,
iba keď sú navyše vylúčené zásahy v prvých dvoch aj
posledných dvoch riadkoch);
1~bod za určenie polohy lode, ktorá vedie ku sporu.

Body z~oboch postupov riešenia sa nesčítajú, výsledkom je maximum z
oboch počtov. Pri iných neúplných postupoch je možné udeliť
napríklad 1 bod za odvodenie, že v niektorej časti $5\times3$
(jednej alebo aj viacerých) sú iba dva zásahy a oba ležia
v~prostrednom z jej piatich riadkov. Pri takých postupoch je možné
udeliť 2~body iba v prípadoch, keď je
jasné, že postup možno úspešne dokončiť a~riešiteľovi na to ostávalo
spraviť málo (rovnako ako pri pokynoch vyššie pre dva opísané
postupy). Za triviálnejšie zistenie
(napríklad: prvé dva aj posledné dva riadky sú bez zásahu)
žiadny bod neudeľujte.
\endschema

}

{%%%%%   C-S-1
Z~rovnice $2m+2s(n)=70$ upravenej na $m+s(n)=35$ vyplýva, že $m<35$,
a~teda $s(m)\leqq 11$ (rovnosť nastane pre $m=29$). Podľa
rovnice $n\cdot s(m)=70$ je preto číslo~$s(m)$ deliteľom čísla 70, ktorý je nanajvýš rovný 11, takže $s(m)\in\{1,2,5,7,10\}$.

\def\black{\pdfliteral{0 g}}
\def\Blue {\pdfliteral{0 0 1 rg}\aftergroup\black}
Teraz pre každú z~hodnôt $s(m)$ rovnú 1, 2, 5, 7, 10
dopočítame postupne hodnoty
$n=70/s(m)$, $s(n)$, $m=35-s(n)$
a zapíšeme ich do nasledujúcej tabuľky:
$$
\vbox{\offinterlineskip \def\mez{\unskip\hskip 14pt plus 1 fil}
\halign{\strut\vrule\mez#\mez\vrule&\mez\hfill # \mez&\mez\hfill # \mez&\mez\hfill # \mez&\mez\hfill # \mez&\mez\hfill # \mez\vrule\cr
\noalign{\hrule}
$s(m)$ & 1 & 2 & 5 & 7 & 10 \cr
$n$ & 70 & 35 & 14 & {\Blue\bf 10} & {\Blue\bf 7} \cr
$s(n)$ & 7 & 8 & 5 & 1 & 7 \cr
$m$ & 28 & 27 & 30 & {\Blue\bf 34} & {\Blue\bf 28} \cr
\noalign{\hrule}
}}
$$
Čísla z~každého stĺpca síce spĺňajú obe zadané
rovnice, nie je však zaručené, že $s(m)$ je naozaj ciferný súčet čísla $m$. Vidíme, že prvé tri prípady, keď $s(m)\in\{1,2,5\}$, nevedú k~žiadnemu riešeniu, lebo $s(28)\ne1$, $s(27)\ne2$ a $s(30)\ne5$. Zvyšné dva prípady, keď $s(m)=7$ a $s(m)=10$, vedú ku dvom riešeniam danej úlohy, keďže $s(34)=7$ a $s(28)=10$. Zodpovedajúce $m$ a $n$ sú v~tabuľke vytlačené tučne.

\Zav
Daná úloha má práve dve riešenia, ktorými sú
dvojice $(m,n)=(34,10)$ a~$(m,n)=(28,7)$.

\Pozn
Ak nerovnosť $s(m)\leqq11$ vopred neodvodíme, môžeme úlohu vyriešiť postupom z~uvedeného riešenia, keď pritom dotyčnú tabuľku rozšírime o~stĺpce pre hodnoty $s(m)\in\{14,35,70\}$, \tj. zvyšné tri delitele čísla 70:
$$
\vbox{\offinterlineskip \def\mez{\unskip\hskip 14pt plus 1 fil}
\halign{\strut\vrule\mez#\mez\vrule&\mez\hfill # \mez&\mez\hfill # \mez&\mez\hfill # \mez\vrule\cr
\noalign{\hrule}
$s(m)$ & 14 & 35 & 70 \cr
$n$ & 5 & 2 & 1 \cr
$s(n)$ & 5 & 2 & 1 \cr
$m$ & 30 & 33 & 34 \cr
\noalign{\hrule}
}}
$$
Tieto možnosti vylučujú nerovnosti $s(30)\ne14$, $s(33)\ne35$ a~$s(34)\ne70$.

\schemaABC
Za úplné riešenie úlohy dajte 6~bodov, z~toho: 2~body za
uvedenie všetkých (vopred nevylúčených) možností pre hodnoty $s(m)$ a
$n$ z~rozkladu $70=n\cdot s(m)$, 3~body za vylúčenie
všetkých nevyhovujúcich možností; 1~bod za nájdenie a overenie
oboch riešení vrátane uvedenia správnej odpovede.

Len za uhádnutie oboch vyhovujúcich dvojíc dajte 1 bod.
\endschema
}

{%%%%%   C-S-2
Označme $S$ priesečník uhlopriečok daného
rovnobežníka~$ABCD$, \tj. ich spoločný stred.
Bod $S$ je tiež stredom priečky $KL$,
ktorá je navyše rovnobežná so stranou~$AB$.\fnote{Tieto známe poznatky nie je nutné v riešení dokazovať. Napriek tomu uveďme, že vyplývajú z vlastností stredných priečok $SK$ a~$LS$ v~trojuholníkoch $ABC$ a~$ABD$.}

Všimnime si, že v pravouhlom trojuholníku $BDM$ je bod $S$ stredom
prepony $BD$ a~bod~$N$ je stredom odvesny $MB$ (\obr).
Úsečka $SN$ je teda jeho stredná priečka rovnobežná
s~odvesnou~$DM$, takže platí nielen $DM\perp AB$, ale aj $SN\perp AB$.
\insp{svrcek70.1}%

Teraz zo vzťahov $KL\parallel AB$ a $SN\perp AB$ dostávame $SN\perp KL$.
Priamka $SN$ kolmá na úsečku $KL$ prechádza jej stredom $S$, takže je to jej os. Z toho už vyplýva $|NK|=|NL|$, ako sme mali dokázať.
(Dodajme, že sme spravenú úvahu mohli zapísať ako úvahu o výške $NS$ trojuholníka $KLN$, ktorý je v jej dôsledku rovnoramenný.)


\schemaABC
Za úplné riešenie úlohy dajte 6~bodov, z~toho: 2 body za
zistenie, že úsečka $SN$ je strednou priečkou trojuholníka $BDM$;
2 body za dôkaz $SN\perp KL$; 1 bod za zdôvodnenie, že priamka $NS$ je os úsečky~$KL$ (prípadne úsečka $NS$ je výška k základni $KL$ rovnoramenného trojuholníka $KLN$); 1 bod za z toho
vyplývajúci potrebný záver.
\endschema
}

{%%%%%   C-S-3
Vzhľadom na podmienku $ab+bc+ca=1$ upravíme (podobne
ako v~úlohe~4 z~domáceho kola) najskôr prvý z troch sčítancov skúmaného výrazu
$$
\frac{a(b^2+1)}{a+b}=\frac{a(b^2+ab+bc+ca)}{a+b}=
\frac{a(b+a)(b+c)}{a+b}=a(b+c)=ab+ca.
$$
Analogicky zvyšné dva sčítance majú vyjadrenia
$$
\frac{b(c^2+1)}{b+c}=bc+ab \quad \hbox{a}\quad
\frac{c(a^2+1)}{c+a}=ca+bc.
$$
Dokopy tak pre súčet všetkých troch zadaných zlomkov dostávame
$$
\frac{a(b^2+1)}{a+b}+\frac{b(c^2+1)}{b+c}+\frac{c(a^2+1)}{c+a}=
(ab+ca)+(bc+ab)+(ca+bc)=2(ab+bc+ca)=2,
$$
pričom sme opäť využili rovnosť $ab+bc+ca=1$ zo zadania úlohy.

\Zav
Daný výraz nadobúda pre ľubovoľné kladné reálne čísla
$a$, $b$, $c$, ktoré spĺňajú podmienku $ab+bc+ca=1$,
vždy hodnotu 2.

\schemaABC
Za úplné riešenie úlohy dajte 6 bodov, z~toho: 3
body za potrebnú úpravu ktoréhokoľvek z troch sčítancov daného výrazu;
2 body za sčítanie troch upravených sčítancov (zlomkov); 1 bod
za uvedenie správnej odpovede.

Za nedokázanú hypotézu o~jedinej hodnote 2 daného výrazu dajte 1
bod, aj keď je podporená konkrétnymi numerickými výpočtami.
\endschema
}

{%%%%%   C-II-1
Najskôr si uvedomíme, že každé hľadané číslo $n$ musí byť
dvojciferné. Naozaj, pre jednociferné číslo $n$ platí
$n+p(n)=2n\leqq18$ a~zároveň zo zadanej rovnice $n+p(n)=70$ zrejme
vyplýva dokonca $n\leqq70$.

Dosaďme teraz dvojciferné $n=\overline{ab}$ s~prvou cifrou
$a\leqq7$ do rovnice a rovno ju upravme na súčinový tvar
$$
n+p(n)=(10a+b)+ab=70,\quad\hbox{po úprave}\quad
(a+1)(b+10)=80.
$$
Keďže $2\leqq a+1\leqq 8$ a súčasne $10\leqq b+10\leqq19$, možno
číslo $80=2^4\cdot5$ rozložiť na súčin dvoch činiteľov
$a+1$, $b+10$ iba dvoma spôsobmi, a to ako $5\cdot16$ alebo
$8\cdot10$ (podľa toho, v ktorom činiteli je zastúpený
prvočiniteľ~5).
V~prvom prípade je $a=4$ a $b=6$, v druhom $a=7$ a $b=0$.

\Zav Jediné dve vyhovujúce čísla sú $n=46$ a $n=70$.

\Pozn
Rovnicu $(10a+b)+ab=70$ možno riešiť tiež tak, že jednu
z~neznámych $a$,~$b$ vyjadríme lomeným
výrazom pomocou druhej a potom urobíme prislúchajúce delenie dvojčlenov
(so zvyškom), napríklad
$$
b=\frac{70-10a}{a+1}=\m10+\frac{80}{a+1}.
$$
Teraz už stačí iba zistiť, pre ktoré $a\in\{1,2,\dots,7\}$ platí
$(a+1)\mid 80$ a súčasne zodpovedajúce $b$ leží v~$\{0,1,\dots,9\}$.

Dodajme, že rovnicu $(10a+b)+ab=70$ s neznámymi ciframi
$a$, $b$ možno riešiť ešte kratším postupom. Upravíme ju na tvar
$b(a+1)=10(7-a)$, podľa ktorého je aspoň jedno z čísel $b$, $a+1$
deliteľné piatimi. Musí teda nastať niektorý z prípadov $b=0$, $b=5$,
$a=4$ či $a=9$. Zvyšok riešenia je už jednoduchý.

\Jres
Ak je $n\leqq39$, tak $n+p(n)\leqq39+3\cdot9=66$, takže pre každé
hľadané~$n$ musí platiť $n\geqq40$. Na druhej strane, z~rovnice
$n+p(n)=70$ vyplýva $n\leqq70$, pritom hodnota $n=70$ zrejme
vyhovuje. Ostáva tak posúdiť dvojciferné čísla $n$
s~desiatkovou cifrou 4, 5 a 6.
Ich jednotkovú cifru označme~$b$ ako predtým.
\item{$\triangleright$} $40\leqq n\leqq49$: Má platiť $(40+b)+4b=70$, čiže $5b=30$, odkiaľ $b=6$.
\item{$\triangleright$} $50\leqq n\leqq59$: Má platiť $(50+b)+5b=70$, čiže
$6b=20$, avšak $6$ nedelí 20.
\item{$\triangleright$} $60\leqq n\leqq69$: Má platiť $(60+b)+6b=70$, čiže
$7b=10$, avšak 7 nedelí 10.

\noindent
Došli sme k tomu istému záveru ako v prvom riešení:
vyhovujú iba čísla $46$ a $70$.


\schemaABC
Za úplné riešenie úlohy dajte 6 bodov, z~toho: 1 bod za prevod zadanej
rovnice na rovnicu pre dve neznáme cifry čísla $n$ (pritom
konštatovanie, že $n$ je dvojciferné, možno považovať za zrejmé, \tj.
môže byť uvedené bez zdôvodnenia); 4 body za úplné vyriešenie tejto
rovnice (pritom za drobné chyby či neúplnosť strhnite 1--2 body);
1 bod za uvedenie záverečnej odpovede.

K hodnoteniu riešení podobných tomu z~druhého riešenia, keď sa
riešiteľ rozhodne pre postupné testovanie dvojciferných čísel
(po desiatkach alebo dokonca jednotlivo): Ak také testovanie, ktoré
musí byť podľa pravidiel MO písomne doložené, nie je úplné či obsahuje
numerické chyby, dajte nanajvýš 5~bodov.

Za nájdenie {\it oboch\/} hľadaných čísel $n$ bez zdôvodnenia, prečo
iné neexistujú, dajte 1~bod. Tento bod možno pripočítať
iba k~1~bodu za prevod zadanej rovnice na rovnicu pre dve
neznáme cifry.
\endschema
}

{%%%%%   C-II-2
V~prvej časti riešenia dokážeme, že každé vyhovujúce číslo $n$ musí
byť deliteľné šiestimi. Trikrát pritom využijeme zrejmý poznatok
z~riešenia 2.~úlohy domáceho kola:
{\sl Ak je súčet niekoľkých čísel rovný nule a pritom každé
z~nich je rovné $2$ či~$\m1$, je celkový počet týchto
čísel deliteľný tromi.}

Predpokladajme teraz, že tabuľka $n\times n$ je vyplnená
požadovaným spôsobom. Vďaka úvodnému poznatku je podľa
podmienky (i) číslo $n$ (rovné počtu čísel v~jednom riadku tabuľky)
deliteľné tromi. Že je číslo $n$ deliteľné tiež dvoma,
dokážeme úvahou o~súčtoch $S_b$ a~$S_c$ čísel na všetkých bielych,
resp. všetkých čiernych políčkach tabuľky.

Súčet $S_b+S_c$ všetkých čísel z~tabuľky dostaneme,
keď sčítame všetkých $n$ súčtov čísel v~jednotlivých riadkoch. Tie sú podľa
podmienky (i) všetky rovné nule, a preto platí $S_b+S_c=0$.
Podľa podmienky (ii) však zároveň platí $S_b=S_c$, čo dokopy
dáva, že obe čísla $S_b$ a $S_c$ sú rovné nule. To podľa
úvodného poznatku znamená, že číslo 3 je deliteľom ako počtu
všetkých bielych políčok, tak počtu všetkých čiernych políčok. Tieto dva počty
sú však buď rovnaké (ak je $n$ párne), alebo sa líšia o~1 (ak je $n$
nepárne). Keďže sa však dva násobky čísla 3 nemôžu líšiť o~1,
prichádzame k~záveru, že počty bielych a čiernych políčok sú rovnaké,
takže číslo $n$ je skutočne párne, ako sme sľúbili ukázať.
Prvá časť riešenia je hotová: číslo $n$ je deliteľné dvoma aj tromi,
a teda aj šiestimi.

V druhej časti riešenia ukážeme, že pre každé číslo $n$, ktoré je
deliteľné šiestimi, možno tabuľku $n\times n$ vyplniť požadovaným spôsobom.

Pre číslo $n=6k$, pričom $k$ je ľubovoľné prirodzené číslo,
rozdeľme tabuľku ${6k\times 6k}$ disjunktným spôsobom na
$k^2$ menších tabuliek $6\times 6$. Určite stačí ukázať, že každú
z~týchto tabuliek $6\times 6$ možno vyplniť číslami $2$ a $\m1$
požadovaným spôsobom, vo výsledku totiž vždy dostaneme
vyhovujúce vyplnenie celej tabuľky $6k\times 6k$.
Príklad jedného z~možných vyplnení tabuľky $6\times 6$
vidíte na \obr{} (nezáleží zrejme na tom,
ktoré políčka sú ofarbené čierno, a ktoré bielo).
\insp{c70ii_p.4}%

\Pozn
V~prvej časti nášho riešenia sme využili
iba tú časť podmienky~(i), ktorá sa týka nulovosti súčtu
všetkých čísel v~každom riadku.

\schemaABC
Za úplné riešenie úlohy dajte 6 bodov, z~toho:
1~bod za konštatovanie, že číslo $n$ je deliteľné tromi (možno sa
pritom odvolať na výsledok z~domáceho kola);
1~bod za zdôvodnenie, prečo sú súčty čísel
na všetkých čiernych aj na všetkých bielych políčkach oba rovné nule;
2~body za dôkaz, že $n$ je párne číslo;
2~body za konštrukciu príkladu vyplnenia pre každé $n=6k$
(z~toho 1 bod za príklad pre $n=6$).
\endschema

}

{%%%%%   C-II-3
Venujme sa najskôr dvojici úsečiek $BC$ a $FG$. Zo zadania bodov $F$ a $G$
vyplýva $|AB|:|AF|=|AC|:|AG|=4:3$, takže trojuholníky $ABC$ a~$AFG$
sú podľa vety $sus$ podobné v~pomere $4:3$. Z toho vyplýva, že
aj $|BC|:|FG|=4:3$ a že $BC\parallel FG$ (vďaka zhodným súhlasným uhlom,
ktoré v~ďalších podobných situáciách už spomínať nebudeme).
Z~praktických dôvodov ďalej zvolíme jednotku dĺžky tak,
aby platilo $|BC|=4$ a $|FG|=3$.

Označme teraz $P$ priesečník priamky $DF$ a rovnobežky zo zadania
úlohy. Z~\obr{} je zrejmé, že trojuholníky $BDF$ a $APF$
sú podľa vety $uu$ podobné. Ich pomer podobnosti je $1:3$, lebo
podľa zadania platí $|AF|=3\,|BF|$. Z~podobnosti trojuholníkov
$BDF$ a~$APF$ vzhľadom na $|BD|=2$ teda vyplýva
$|AP|=3\,|BD|=3\cdot 2=6$.

Označme ďalej $P'$ priesečník priamky $GE$ a rovnobežky zo zadania
úlohy. Podľa~\obr{} vidíme, že aj trojuholníky $FGE$ a $AP'E$
sú podľa vety $uu$ podobné; ich pomer podobnosti je
pritom $|FE|:|AE|=1:2$, teda $|AP'|=2\cdot |FG|=2\cdot 3=6$.
\inspinspblizko{c70ii_p.11}{c70ii_p.12}%

Zistili sme, že pre body $P$ a $P'$, ktoré ležia
na uvažovanej rovnobežke v~jednej polrovine vyťatej priamkou $AB$,
platí $|AP|=|AP'|$, a~preto nutne $P=P'$.
Na danej rovnobežke teda leží aj priesečník priamok $DF$ a~$GE$,
ako sme mali dokázať.

\poznamkac1.
Namiesto úvahy o~druhom priesečníku $P'$ je možné dokázať, že
priesečník~$P$ leží na polpriamke opačnej k~polpriamke $EG$. Na to
stačí overiť rovnosť $|\uhel FEG|=|\uhel AEP|$. Tá ale vyplýva
z~vyšrafovaných trojuholníkov $FEG$ a $AEP$, ktoré sú totiž podobné podľa vety $sus$,
lebo sa zhodujú v~uhloch pri vrcholoch $F$, $A$ a oba pomery
$|FE|:|AE|$, $|FG|:|AP|$ sú rovné pomeru $1:2$.

Postupovať pri riešení možno tiež tak, že uvážime iba priesečník~$P'$,
z~podobných trojuholníkov $FEG$, $AEP'$ odvodíme rovnosť $|AP'|=6$ a
dokážeme rovnosť $|\uhel BFD|=|\uhel AFP'|$ z~vyšrafovaných trojuholníkov $BFD$ a $AFP'$,
ktoré sú totiž podľa vety $sus$ podobné,
lebo sa zhodujú v~uhloch pri vrcholoch $B$,~$A$ a
oba pomery $|BF|:|AF|$, $|BD|:|AP'|$ sú rovné pomeru
$1:3$.

\poznamkac2.
Polohu priesečníka priamok $DF$ a~$GE$ nám prezradí, keď si
vopred načrtneme rovinnú trojuholníkovú mriežku, ktorej uzly delia
každú stranu trojuholníka $ABC$ na štyri zhodné úseky.
\insp{c70ii_p.2}%

Obr.\,\obrnum{} síce priamo napovedá, že priamky $DF$ a $GE$ prechádzajú
obe tým uzlom, ktorý je \uv{šiesty naľavo} od uzla $A$, samotný
náčrt však sa nedá považovať za úplné riešenie úlohy.
Je nutné dokázať to, čo na obrázku vidíme. Presnejšie povedané,
je nutné zdôvodniť, že uvažované spojnice deliacich bodov
na stranách trojuholníka $ABC$ ho rozdeľujú na 16 menších,
navzájom zhodných trojuholníkov,\fnote{Prvým krokom takého zdôvodnenia
môže byť úvaha o dvojici úsečiek
$BC$ a $FG$ z úvodného odseku riešenia.} ktoré už potom
možno doplniť do výslednej mriežky, ktorá bude \uv{zjemnením} mriežky tvorenej
kópiami trojuholníka $ABC$.

\schemaABC
Za úplné riešenie úlohy dajte 6 bodov, z~toho 2 body za
zavedenie aspoň jedného z priesečníkov $P$ a~$P'$.
Zvyšné 4 body dajte podľa úplnosti úvah o~dvojiciach podobných
trojuholníkov, ktoré sú pri zvolenom postupe potrebné,
z toho 1~bod za dôkaz oboch relácií $|BC|:|FG|=4:3$ a $BC\parallel
FG$. Tento bod možno udeliť aj v~prípade, keď žiadny z bodov $P$, $P'$
zavedený nie je.

Za náčrt trojuholníkovej mriežky s vykreslenými polpriamkami $DF$ a $GE$
(ako je uvedený v~poznámke~2) dajte 4 body,
k tomu možno pripočítať 1--2 body podľa miery úplnosti,
s~akou je existencia takej mriežky zdôvodnená.
\endschema
}

{%%%%%   C-II-4
Použitím podmienky $a^2+b^2=1$ upravíme najskôr prvý zlomok zo
zadaného výrazu:
$$
\frac{a^2(a+b^3)}{b-b^3}=\frac{a^2(a+b^3)}{b(1-b^2)}=
\frac{a^2(a+b^3)}{b\cdot a^2}=\frac{a+b^3}{b}=\frac{a}{b}+b^2.
$$
Podobne pre druhý zlomok platí
$$
\frac{b^2(b+a^3)}{a-a^3}=\frac{b}{a}+a^2.
$$
Sčítaním vyjadrení oboch zlomkov tak pre zadaný výraz $V$ dostaneme:
$$
V=\left(\frac{a}{b}+b^2\right)+\left(\frac{b}{a}+a^2\right)=
\left(\frac{a}{b}+\frac{b}{a}\right)+(a^2+b^2)=
\left(\frac{a}{b}+\frac{b}{a}\right)+1,
$$
pričom sme opäť využili podmienku $a^2+b^2=1$.

Všimnime si, že pre súčet zlomkov v~posledných okrúhlych zátvorkách
platí\fnote{Na dôkaz
tohto výsledku možno tiež využiť AG-nerovnosť $\frac12(u+v)\geqq\sqrt{uv}$
pre hodnoty $u=a/b$ a $v=b/a$. Tiež je možné vďaka podmienke
$a^2+b^2=1$ využiť rovnosť $a/b+b/a=1/(ab)$ a potom maximalizovať
súčin $ab$, napríklad použitím uvedenej AG-nerovnosti pre $u=a^2$ a
$v=b^2$.}
$$
\frac{a}{b}+\frac{b}{a}=\frac{a^2+b^2}{ab}=
\frac{\bigl(a-b\bigr)^2+2ab}{ab}=
\frac{\bigl(a-b\bigr)^2}{ab}+2\geqq2,\quad\hbox{lebo}\quad
\frac{\bigl(a-b\bigr)^2}{ab}\geqq0.
$$
Z odvodeného vyjadrenia výrazu $V$ teda vyplýva
dolný odhad $V\geqq2+1=3$. Keďže navyše v~použitej nerovnosti $a/b+b/a\geqq2$
nastane rovnosť práve vtedy, keď platí ${a=b}$, rovnosť $V=3$ nastane
práve vtedy, keď kladné čísla $a$,~$b$ spĺňajú obe podmienky ${a^2+b^2}=1$ a $a=b$,
\tj. práve vtedy, keď $a=b=\frac12\sqrt{2}$.
Číslo~3 je teda hľadaná najmenšia hodnota
daného výrazu $V$ a je dosiahnutá pre jedinú dvojicu uvažovaných
čísel $a$,~$b$ (určených v~závere predchádzajúcej vety).

\schemaABC
Za úplné riešenie úlohy dajte 6 bodov, z~toho: 2 body za potrebnú úpravu
aspoň jedného z dvoch zadaných zlomkov;
1 bod za úpravu ich súčtu
na tvar vhodný pre minimalizáciu
súčtu $a/b+b/a$ či maximalizáciu súčinu $ab$;
2 body za zdôvodnenie dolného odhadu číslom 3
(možno sa pritom odvolať na AG-nerovnosť ako v~poznámke pod čiarou);
1 bod za určenie (nie iba uhádnutie),
kedy je najmenšia hodnota dosiahnutá.

Pri neúplnom riešení za uhádnutie hľadaného minima 3 dajte 1~bod iba
v~prípade, keď je pre túto hodnotu uvedená zodpovedajúca dvojica
$a=b=\frac12\sqrt{2}$. Tento bod možno pripočítať k prvým 2 alebo 3
bodom udeleným podľa pokynov z predchádzajúceho odseku. Bez úplného dôkazu
nerovnosti $V\geqq3$ aj pri iných postupoch možno získať nanajvýš 4
body.
\endschema
}

{%%%%%   vyberko, den 1, priklad 1
...}

{%%%%%   vyberko, den 1, priklad 2
...}

{%%%%%   vyberko, den 1, priklad 3
...}

{%%%%%   vyberko, den 1, priklad 4
...}

{%%%%%   vyberko, den 2, priklad 1
...}

{%%%%%   vyberko, den 2, priklad 2
...}

{%%%%%   vyberko, den 2, priklad 3
...}

{%%%%%   vyberko, den 2, priklad 4
...}

{%%%%%   vyberko, den 3, priklad 1
...}

{%%%%%   vyberko, den 3, priklad 2
...}

{%%%%%   vyberko, den 3, priklad 3
...}

{%%%%%   vyberko, den 3, priklad 4
...}

{%%%%%   vyberko, den 4, priklad 1
...}

{%%%%%   vyberko, den 4, priklad 2
...}

{%%%%%   vyberko, den 4, priklad 3
...}

{%%%%%   vyberko, den 4, priklad 4
...}

{%%%%%   vyberko, den 5, priklad 1
...}

{%%%%%   vyberko, den 5, priklad 2
...}

{%%%%%   vyberko, den 5, priklad 3
...}

{%%%%%   vyberko, den 5, priklad 4
...}

{%%%%%   trojstretnutie, priklad 1
...}

{%%%%%   trojstretnutie, priklad 2
...}

{%%%%%   trojstretnutie, priklad 3
...}

{%%%%%   trojstretnutie, priklad 4
...}

{%%%%%   trojstretnutie, priklad 5
...}

{%%%%%   trojstretnutie, priklad 6
...}

{%%%%%   IMO, priklad 1
...}

{%%%%%   IMO, priklad 2
...}

{%%%%%   IMO, priklad 3
...}

{%%%%%   IMO, priklad 4
...}

{%%%%%   IMO, priklad 5
...}

{%%%%%   IMO, priklad 6
...}

{%%%%%   MEMO, priklad 1
...}

{%%%%%   MEMO, priklad 2
...}

{%%%%%   MEMO, priklad 3
...}

{%%%%%   MEMO, priklad 4
...}

{%%%%%   MEMO, priklad t1
...}

{%%%%%   MEMO, priklad t2
...}

{%%%%%   MEMO, priklad t3
...}

{%%%%%   MEMO, priklad t4
...}

{%%%%%   MEMO, priklad t5
...}

{%%%%%   MEMO, priklad t6
...}

{%%%%%   MEMO, priklad t7
...}

{%%%%%   MEMO, priklad t8
...} 