{%%%%%   Z4-I-1
...}
\podpis{...}

{%%%%%   Z4-I-2
...}
\podpis{...}

{%%%%%   Z4-I-3
...}
\podpis{...}

{%%%%%   Z4-I-4
...}
\podpis{...}

{%%%%%   Z4-I-5
...}
\podpis{...}

{%%%%%   Z4-I-6
...}
\podpis{...}

{%%%%%   Z5-I-1
Pán Krbec s~kocúrom Kokešom predávali na hrade Kulíkov vstupenky.
V~sobotu predali 210 detských vstupeniek po 25 grošov a~tiež nejaké vstupenky pre dospelých po 50 grošov.
Celkom za ten deň utŕžili 5\,950 grošov.
Koľko predali vstupeniek pre dospelých?
}
\podpis{Marie Krejčová}

{%%%%% Z5-I-2
Deti na tábore hádzali hracou kockou a~podľa výsledkov plnili nasledujúce úlohy:
\insp{70-z5-i-2}
Po piatich hodoch bolo Marekovo družstvo 1~km východne od štartu.
\begin{enumerate}
\item Akú trasu mohlo Marekovo družstvo prejsť? Naznačte aspoň štyri možnosti.
\item Aký mohol byť celkový súčet všetkých čísel, ktoré tomuto družstvu padli? Určte všetky možnosti.
\end{enumerate}
}
\podpis{Eva Semerádová}

{%%%%% Z5-I-3
Pán režisér Alík potreboval do televíznej rozprávky štyri psy.
Dostal ponuku z~Grécka, Belgicka, Írska a~z~Dolnej Lehoty.
Vybral ovčiaka, dalmatína, vlkodava a~jazvečíka, každého z~inej krajiny, s~rôznym menom a~rôznym vekom.
\begin{itemize}
\item Najstarší zo psov bol jazvečík, mal 5~rokov.
\item Bucki bol z~nich druhý najmladší.
\item Vlkodav pochádzal z~Írska.
\item Pes z~Dolnej Lehoty sa volal Dunčo.
\item Oddi oslávil včera svoje štvrté narodeniny.
\item Ovčiak pochádzal z~Belgicka.
\item Rubby nebol dalmatín.
\item Vlkodav mal tri roky.
\item Najmladší z~vybraných psov bol Rubby, mal dva roky.
\end{itemize}
Zistite, ako sa každý zo štyroch psov volal, odkiaľ pochádzal, akej bol rasy a~koľko mal rokov.
}
\podpis{Libuše Hozová}

{%%%%% Z5-I-4
Mamička uvarila domácu ríbezľovú šťavu a~nalievala ju do fliaš.
Fľaše mala dvojaké: malé s~objemom 500~ml a~veľké s~objemom 750~ml.
Nakoniec jej zvýšilo 12 malých fliaš prázdnych, ostatné fľaše boli úplne naplnené.
Potom mamička zistila, že mohla šťavu nalievať tak, aby jej zvýšili prázdne iba veľké fľaše a~všetky ostatné boli úplne naplnené.
Koľko prázdnych fliaš by jej v~takom prípade zvýšilo?
}
\podpis{Michaela Petrová}

{%%%%% Z5-I-5
V~štvorčekovej sieti so štvorčekmi s~rozmermi $1\,\cm\times1\,\cm$ sú vyznačené tri mrežové body $K$, $O$ a~$Z$.
Určte mrežový bod~$A$ tak, aby obsah štvoruholníka $KOZA$ bol 4\,cm$^2$.
\insp{Z5-I-5.eps}%
}
\podpis{Eva Semerádová}

{%%%%% Z5-I-6
Myslím si päťciferné číslo tvorené párnymi ciframi.
Keď prehodím cifru na treťom mieste s~akoukoľvek inou, číslo sa zmenší.
Ďalej prezradím, že prvá cifra je dvojnásobkom poslednej a~druhá cifra je dvojnásobkom predposlednej.
Aké číslo si myslím?
}
\podpis{Martin Mach}

{%%%%% Z6-I-1
Králiky Pečienka, Fašírka, Rezeň a~Guláš súťažili v skoku do diaľky.
Pečienka skočila o~15\,cm ďalej ako Fašírka, ktorá skočila o~2\,dm menej ako Guláš.
Rezeň skočil 2\,730\,mm, teda o~1\,m a~1\,dm ďalej ako Pečienka.
Určte poradie a~dĺžky skokov všetkých králikov.
}
\podpis{Svetlana Bednářová}

{%%%%% Z6-I-2
Vzal som klasickú čierno-bielu šachovnicu, ktorá bola tvorená $8\times8$ štvorcovými políčkami so stranami dĺžky 3\,cm.
Políčka som v~danom rámci preskupil tak, že vznikol jeden čierny obdĺžnik, jeden čierny štvorec a~jeden súvislý biely útvar.
Jednotlivé políčka sa aj po preskupení dotýkali celými stranami.
Čierne útvary sa nedotýkali (ani rohom) a~každý z~nich mal aspoň jednu stranu spoločnú s~okrajom šachovnice.
Určte najväčší možný obvod bieleho útvaru a~nakreslite, ako by v~takom prípade mohol vyzerať.
}
\podpis{Martin Mach}

{%%%%% Z6-I-3
Mamička dala do misy 56~jahôd a~39~malín a~zaniesla ich Eme, ktorá si čítala.
Ema si čítanie spríjemnila maškrtením, a~to tak, že si postupne brala po dvoch náhodných kusoch ovocia:
\begin{itemize}
\item Keď vytiahla dve maliny, vymenila ich u~mamičky za jednu jahodu a~tú vrátila do misy.
\item Keď vytiahla dve jahody, jednu zjedla a~druhú vrátila do misy.
\item Keď vytiahla jednu jahodu a~jednu malinu, zjedla jahodu a~malinu vrátila do misy.
\end{itemize}
Takto nejakú chvíľu maškrtila, až v~mise zostal jediný kus ovocia.
Rozhodnite (a~vysvetlite), či to bola jahoda, alebo malina.
}
\podpis{Libuše Hozová}

{%%%%% Z6-I-4
Ctibor naprogramoval dva spolupracujúce rysovacie roboty Mikiho a~Nikiho.
Miki vie zostrojovať štvorce, pravidelné päťuholníky a~pravidelné šesťuholníky.
Počas jedného dňa však rysuje iba navzájom zhodné mnohouholníky.
Niki do všetkých Mikiho mnohouholníkov dopĺňa všetky uhlopriečky.
\begin{enumerate}
\item V~pondelok zostrojil Miki rovnaký počet úsečiek ako Niki.
Aké mnohouholníky rysovali?
\item V~utorok zostrojil Miki 18~úsečiek.
Koľko ich zostrojil Niki?
\item V~stredu zostrojili Miki a~Niki dokopy 70~úsečiek. Koľko mnohouholníkov im dal Ctibor rysovať?
\end{enumerate}
}
\podpis{Michaela Petrová}

{%%%%% Z6-I-5
Petra vpisovala do krúžkov čísla 1, 2, 3, 4, 5, 6, 7, 8 tak, že každé bolo použité práve raz a~že súčet čísel na každej strane trojuholníka bol rovnaký.
Aký najväčší súčet mohla takto dostať?
Uveďte príklad možného vyplnenia.
\insp{Z6-I-5.eps}
}
\podpis{Alžbeta Bohiniková}

{%%%%% Z6-I-6
Anička a~Marienka majú každá svoje obľúbené prirodzené číslo.
Ak vynásobíme Aničkino číslo samo so sebou, vyjde nám stokrát väčšie číslo, ako keď vynásobíme Marienkino číslo samo so sebou.
Ak sčítame Aničkino a~Marienkino obľúbené číslo, získame číslo o~18 väčšie, ako je polovica Aničkinho čísla.
Určte Aničkino a~Marienkino obľúbené číslo.
}
\podpis{Eva Semerádová}

{%%%%% Z7-I-1
Určte, ktorá cifra je na 1000. mieste za desatinnou čiarkou v~desatinnom rozvoji čísla~$\frac9{28}$.
}
\podpis{Marie Krejčová}

{%%%%% Z7-I-2
Kubo sa dohodol s~bačom, že sa mu bude starať o~ovce.
Bača Kubovi sľúbil, že po roku služby dostane dvadsať zlatých a~k~tomu jednu ovcu.
Lenže Kubo dal výpoveď práve vtedy, keď uplynul siedmy mesiac služby.
Aj tak ho Bača spravodlivo odmenil a~zaplatil mu päť zlatých a~jednu ovcu.
Na koľko zlatých si bača cenil jednu ovcu?
}
\podpis{Libuše Hozová}

{%%%%% Z7-I-3
Pre skupinu detí platí, že v~každej trojici detí zo skupiny je chlapec menom Adam a~v~každej
štvorici je dievča menom Beáta.
Koľko nanajvýš detí môže byť v~takej skupine a~aké sú v~tom prípade ich mená?
}
\podpis{Jaroslav Zhouf}

{%%%%% Z7-I-4
Medzi prístavmi Mumraj a~Zmätok pendlujú po rovnakej trase dve lode.
V~prístavoch trávia zanedbateľný čas, hneď sa otáčajú a~pokračujú v~plavbe.
Ráno v~rovnakom okamihu vypláva modrá loď z~prístavu Mumraj a~zelená loď z~prístavu Zmätok.
Prvýkrát sa lode míňajú 20~km od prístavu Mumraj a~po nejakom čase sa stretnú priamo v~tomto prístave.
To už modrá loď stihla uplávať trasu medzi prístavmi štyrikrát, zatiaľ čo zelená loď iba trikrát.
Aká dlhá je trasa medzi prístavmi Mumraj a~Zmätok?
}
\podpis{František Steinhauser}

{%%%%% Z7-I-5
Odčítacia pyramída je pyramída tvorená nezápornými celými číslami, z~ktorých každé je rozdielom dvoch najbližších čísel z~predchádzajúceho riadka (čítané odspodu nahor).
Tu je príklad odčítacej pyramídy:
\Image{70-z7-i-5}{1.4}
Význačné číslo je najväčšie číslo odčítacej pyramídy.
Výborná pyramída je odčítacia pyramída, ktorá má vo vrchole 0 a~aspoň jeden riadok tvorený navzájom rôznymi číslami.
\begin{enumerate}
\item Koľko najmenej riadkov musí mať výborná pyramída?
\item Ktoré najmenšie význačné číslo môže byť obsiahnuté vo výbornej pyramíde s~najmenším počtom riadkov?
\end{enumerate}
}
\podpis{Katarína Jasenčáková}

{%%%%% Z7-I-6
V~trojuholníku $ABC$ leží na strane $AC$ bod~$D$ a~na strane~$BC$ bod~$E$.
Veľkosti uhlov $ABD$, $BAE$, $CAE$ a~$CBD$ sú postupne $30\st$, $60\st$, $20\st$ a~$30\st$.
Určte veľkosť uhla $AED$.
\insp{Z7-I-6.eps}

Poznámka: obrázok je iba ilustračný.
}
\podpis{Alžbeta Bohiniková}

{%%%%% Z8-I-1
Myslím si päťciferné číslo, ktoré nie je deliteľné tromi ani štyrmi.
Ak každú cifru zväčším o~jedna, získam päťciferné číslo, ktoré je deliteľné tromi.
Ak každú cifru o~jedna zmenším, získam päťciferné číslo deliteľné štyrmi.
Ak prehodím ľubovoľné dve cifry, číslo sa zmenší.
Aké číslo si môžem myslieť?
Nájdite všetky možnosti.
}
\podpis{Martin Mach}

{%%%%% Z8-I-2
Na záhrade stáli tri debny s~jablkami.
Spolu bolo jabĺk viac ako 150, avšak menej ako 190.
Potom Marienka premiestnila z~prvej debny do dvoch ďalších debien jablká tak, že sa ich počet v~každej z~týchto dvoch debien oproti predošlému stavu zdvojnásobil.
Obdobným spôsobom Marta premiestnila jablká z~druhej debny do prvej a~tretej.
Nakoniec Štefka podľa rovnakých pravidiel premiestnila jablká z~tretej debny do prvej a~druhej.
Keď prišiel na záhradu Vojto, začudoval sa, že v~každej debne bol rovnaký počet jabĺk.
Koľko jabĺk bolo v~jednotlivých debnách pôvodne?
}
\podpis{Libuše Hozová}

{%%%%% Z8-I-3
V~trojuholníku $ABC$ je bod~$S$ stredom vpísanej kružnice.
Obsah štvoruholníka $ABCS$ je rovný štyrom pätinám obsahu trojuholníka $ABC$.
Dĺžky strán trojuholníka $ABC$ vyjadrené v~centimetroch sú všetky celočíselné a~obvod trojuholníka $ABC$ je 15\,cm.
Určte dĺžky strán trojuholníka $ABC$.
Nájdite všetky možnosti.
}
\podpis{Eva Semerádová}

{%%%%% Z8-I-4
Jarka bola na brigáde s~nemennou dennou mzdou.
Za tri dni si zarobila toľko eur, že si kúpila stolovú hru a~ešte jej 49~€ zvýšilo.
Keby strávila na brigáde päť dní, mohla by si kúpiť dve také stolové hry a~ešte by jej zvýšilo 54~€.
Koľko eur stála stolová hra?
}
\podpis{Karel Pazourek}

{%%%%% Z8-I-5
Pán Strieborný usporiadal výstavu.
Vystavoval 120 prsteňov, ktoré ležali na stoloch pozdĺž stien sály a~tvorili tak jednu veľkú kružnicu.
Prehliadka začínala pri vchodových dverách v~označenom smere.
Každý tretí prsteň v~rade bol zlatý, každý štvrtý prsteň bol starožitný a~každý desiaty prsteň mal diamant.
Prsteň, ktorý nemal žiadnu z~týchto troch vlastností, bol falzifikát.
Koľko bolo na výstave zlatých prsteňov, ktoré boli starožitné a~zároveň mali diamant?
Koľko vystavil pán Strieborný falzifikátov?
}
\podpis{Libuše Hozová}

{%%%%% Z8-I-6
Body $A$, $B$, $C$, $D$ a~$E$ sú vrcholmi nepravidelnej päťcípej hviezdy, pozri obrázok.
Určte súčet vyznačených uhlov.
\insp{Z8-I-6.eps}

Poznámka: obrázok je iba ilustračný.
}
\podpis{Libuše Hozová}

{%%%%% Z9-I-1
Slávka si napísala farebnými fixkami štyri rôzne prirodzené čísla: červené, modré, zelené a~žlté.
Keď červené číslo vydelí modrým, dostane ako neúplný podiel zelené číslo a~žlté predstavuje zvyšok po tomto delení.
Keď vydelí modré číslo zeleným, vyjde jej delenie bezo zvyšku a~podielom je číslo žlté.
Slávka prezradila, že dve z~jej štyroch čísel sú 97 a~101.
Určte ostatné Slávkine čísla a~priraďte jednotlivým číslam farby.
Nájdite všetky možnosti.
}
\podpis{Michaela Petrová}

{%%%%% Z9-I-2
Nájdite všetky dvojice nezáporných celých čísel~$x$ a~jednociferných prirodzených čísel~$y$, pre ktoré platí
$$
\frac{x}{y}+1=x{,}\overline{y}.
$$
Zápis na pravej strane rovnosti označuje periodické číslo.
}
\podpis{Karel Pazourek}

{%%%%% Z9-I-3
V~rovnostrannom trojuholníku $ABC$ je bod~$T$ jeho ťažiskom, bod~$R$ je obrazom bodu~$T$ v~osovej súmernosti podľa priamky~$AB$ a~bod~$N$ je obrazom bodu~$T$ v~osovej súmernosti podľa priamky~$BC$.
Určte pomer obsahov trojuholníkov $ABC$ a~$TRN$.
}
\podpis{Eva Semerádová}

{%%%%% Z9-I-4
Na stene bolo napísané dvakrát to isté päťciferné číslo.
Pat pred jeden zápis čísla pripísal jednotku, Mat pripísal jednotku za ten druhý zápis čísla.
Tým dostali dve šesťciferné čísla, z~ktorých jedno bolo trikrát väčšie ako druhé.
Ktoré päťciferné číslo bolo pôvodne napísané na stene?
}
\podpis{Libuše Hozová}

{%%%%% Z9-I-5
Na ihrisku sú nakreslené tri rovnako veľké kruhy.
Rozostavte 16~kolkov tak, aby v~každom kruhu stálo 9~kolkov.
Nájdite aspoň osem podstatne odlišných rozostavení, \tj. takých rozostavení, pri ktorých sa nerozlišujú kolky ani kruhy.
}
\podpis{Libuše Hozová}

{%%%%% Z9-I-6
Jozef a~Mária objavili na dovolenke pravidelný ihlan, ktorého podstavou bol štvorec so~stranou 230\,m
a~ktorého výška bola rovná polomeru kruhu s~rovnakým obvodom ako má podstavný štvorec.
Mária označila vrcholy štvorca $ABCD$.
Jozef vyznačil na priamke spájajúcej bod~$B$ s~vrcholom ihlana taký bod~$E$, že dĺžka lomenej čiary $AEC$ bola najkratšia možná.
Určte dĺžku lomenej čiary $AEC$ zaokrúhlenú na celé centimetre.
}
\podpis{Marie Krejčová, František Steinhauser}

{%%%%%   Z4-II-1
...}
\podpis{...}

{%%%%%   Z4-II-2
...}
\podpis{...}

{%%%%%   Z4-II-3
...}
\podpis{...}

{%%%%% Z5-II-1
Cesty v~našom parku vedú po stranách a~uhlopriečkach dvoch rovnakých obdĺžnikov:
\insp{z5-II-1a.eps}%

Počas víkendu sme sa parkom prešli štyrmi spôsobmi.
V~každom prípade sme každou vyznačenou cestou prešli práve raz:
\insp{z5-II-1b.eps}%

V~prvom prípade sme prešli o~500 metrov menej ako v~druhom prípade.
O~koľko metrov viac sme prešli v~treťom prípade ako vo štvrtom?
}
\podpis{Eva Semerádová}

{%%%%% Z5-II-2
Skrinka má tri šuplíky umiestnené nad sebou. V~nich sú tri predmety: guľôčka, mušľa a~minca. Každý šuplík obsahuje jeden predmet.
Vieme, že:
\itemitem{$\bullet$} zelený šuplík je vyššie ako modrý šuplík,
\itemitem{$\bullet$} minca je vyššie ako guľôčka,
\itemitem{$\bullet$} červený šuplík je nižšie ako mušľa,
\itemitem{$\bullet$} guľôčka je nižšie ako červený šuplík.

V ktorom šuplíku je minca?
}
\podpis{Libuše Hozová}

{%%%%% Z5-II-3
Ako poďakovanie za ich pomoc upiekla Maruška dvanástim mesiačikom deväťdesiat perníčkov.
Keď im ich dávala, vzal si každý toľko perníčkov, aké je poradové číslo jeho mesiaca v~roku.
Dvaja mesiačikovia s~vďakou odmietli a~žiadny perníček si nevzali. Maruške tak v~košíčku zvýšilo 21 perníčkov.
Určte všetky možné dvojice mesiačikov, ktorí mohli odmietnuť.
}
\podpis{Michaela Petrová}

{%%%%% Z6-II-1
Štyri veveričky zjedli dokopy 2020 orieškov, každá najmenej 103 orieškov.
Prvá veverička zjedla viac orieškov ako ktorákoľvek z~ostatných troch veveričiek.
Druhá a~tretia veverička zjedli dokopy 1277 orieškov.
Koľko orieškov zjedla prvá veverička?
}
\podpis{Libuše Hozová}

{%%%%% Z6-II-2
V~súčinovej pyramíde je v~každom políčku jedno kladné celé číslo, ktoré je súčinom čísel z~dvoch susediacich políčok z~nižšej vrstvy.
Vo vrchole trojvrstvovej súčinovej pyramídy je číslo~90.
Aké číslo môže byť vo vyznačenom políčku?
Určte všetky možnosti.
\ifobrazkyvedla\else\insp{z6-II-2.eps}\fi%
}
\podpis{Alžbeta Bohiniková}

{%%%%% Z6-II-3
Dvierka králikárne sú vyrobené z~dreveného rámu a~drôteného pletiva so štvorcovými okami.
Latky rámu sú široké 5\,cm.
Niektoré mrežové body pletiva sa nachádzajú presne na vnútorných hranách rámu ako na obrázku.
Vnútorná (pletivová) časť dvierok má obsah $432\cm^2$.
Určte vonkajšie rozmery (\tj. šírku a~výšku) celých dvierok.
\ifobrazkyvedla\else\insp{z6-II-3.eps}\fi%
}
\podpis{Svetlana Bednářová}

{%%%%% Z7-II-1
Na rozprávkovom ostrove žijú draci a~kyklopi.
Všetci draci sú červení, trojhlaví a~dvojnohí.
Všetci kyklopi sú hnedí, jednohlaví a~dvojnohí.
Kyklopi majú jedno oko uprostred čela, draci majú na každej hlave dve oči.
Dokopy majú kyklopi a~draci 42 očí a~34 nôh.
Koľko drakov a~koľko kyklopov žije na ostrove?
}
\podpis{Michaela Petrová}

{%%%%% Z7-II-2
Pravidelný šesťuholník je štyrmi svojimi uhlopriečkami rozdelený na šesť trojuholníkov a~jeden štvoruholník ako na obrázku.
Obsah štvoruholníka $F$ je $1{,}8\cm^2$.
Určte obsahy trojuholníkov $A$, $B$, $C$, $D$, $E$ a~$G$.
\ifobrazkyvedla\else\insp{z7-ii-2.eps}\fi%
}
\podpis{Eva Semerádová}

{%%%%% Z7-II-3
Bludička Jozefína tancuje pri močiari, pričom používa kroky dvojakej dĺžky -- krátke merajú $45\cm$, dlhé $60\cm$.
Časom si vyšliapala oválny chodník, po ktorom za dlhých nocí tancuje stále dokola.
Ak opakuje tri dlhé kroky dopredu a~jeden krátky vzad, tak deväťdesiatym krokom dotancuje presne tam, kde začínala.
Ak opakuje tri krátke kroky dopredu a~jeden dlhý vzad, tak jej tiež vychádza krok presne tam, kde začínala.
Koľkým krokom dotancuje Jozefína na pôvodné miesto v druhom prípade?
}
\podpis{Michaela Petrová}

{%%%%% Z8-II-1
Deti na tábore získavali za splnené úlohy body.
Tie bolo možné ďalej meniť: päť bodov za samolepku, šesť bodov za pečiatku.
Jaro si prepočítal, že keby chcel len samolepky, zvýšil by mu jeden bod nevyužitý.
Keby si vybral len pečiatky, nevyužil by tri body.
Nakoniec dokázal uplatniť všetky svoje body, pričom získal dvojnásobné množstvo pečiatok ako samolepiek.
Koľko najmenej bodov mohol mať Jaro pred menením?
}
\podpis{Eva Semerádová}

{%%%%% Z8-II-2
Eliška umiestňovala koláče do škatúľ, z ktorých potom postavila pyramídu ako na obrázku.
Pritom každá škatuľa vo vyššom rade obsahovala toľko koláčov ako dve susediace škatule pod ňou dokopy.
V troch škatuliach označených hviezdičkami bolo tri, päť a~šesť koláčov.
Eliška si všimla, že keby označené škatule akokoľvek zamenila a~podľa predchádzajúceho pravidla upravila počty koláčov v~ostatných škatuliach, celkový počet koláčov by sa nezmenšil.
Koľko koláčov bolo v~označenej škatuli v druhom rade zdola?
\ifobrazkyvedla\else\insp{z8-ii-2.eps}\fi%
}
\podpis{Libuše Hozová}

{%%%%% Z8-II-3
Pre všeobecný trojuholník $ABC$ sú dané body $D$, $E$, $F$:
\itemitem{$\bullet$} bod $D$ je v tretine úsečky $AB$, bližšie k~bodu $A$,
\itemitem{$\bullet$} bod $E$ je v tretine úsečky $BC$, bližšie k~bodu $B$,
\itemitem{$\bullet$} bod $F$ je v tretine úsečky $CD$, bližšie k~bodu $C$.

Určte pomer obsahov trojuholníkov $ABC$ a~$DEF$.
}
\podpis{Alžbeta Bohiniková}

{%%%%% Z9-II-1
Babička mala štvorcovú záhradu. Dokúpila niekoľko susedných pozemkov, čím získala zasa štvorcový pozemok, ktorého strana bola o~tri metre dlhšia ako strana pôvodnej záhrady.
%Takto výmeru svojho pozemku viac ako zdvojnásobila, a~to o~deväť štvorcových metrov.
Takto bola výmera pozemku o deväť štvorcových metrov väčšia ako dvojnásobok pôvodnej výmery.
Aká dlhá bola strana pôvodnej záhrady?
}
\podpis{Katarína Buzáková}

{%%%%% Z9-II-2
Babička ešte nemá 100 rokov, vnučka má viac ako 10~rokov a~vek babičky je násobkom veku vnučky.
Keď vnučka napísala vek babičky a~zaň vek svoj, dostala štvorciferné číslo.
Keď babička napísala vek vnučky a~zaň vek svoj, dostala iné štvorciferné číslo.
Rozdiel týchto dvoch štvorciferných čísel je 7128.
Koľko rokov môže mať babička a~koľko vnučka?
Uveďte všetky možnosti.
}
\podpis{Libuše Hozová}

{%%%%% Z9-II-3
Karol, Miro a~Ľudo porovnávali svoje zbierky známok.
Keď kontrolovali počty, zistili, že
Karol a~Miro majú dokopy 101~známok,
Karol a~Ľudo 115 známok,
Miro a~Ľudo 110.
Keď overovali, čo by mohli meniť, zistili, že žiadnu známku nemajú všetci rovnakú, ale že
Karol a~Miro majú 5 známok rovnakých,
Karol a~Ľudo 12 rovnakých,
Miro a~Ľudo 7.
Koľko známok má Ľudo iných ako ostatní chlapci?}
\podpis{Miroslava Farkas Smitková}

{%%%%% Z9-II-4
Pán učiteľ chcel po Adamovi a~Eve, aby vypočítali obvod lichobežníka, ktorého dlhšia základňa merala 30\,cm, výška 24\,cm a~ramená 25\,cm a~30\,cm.
Adamovi vyšiel iný obvod ako Eve, aj tak však pán učiteľ oboch pochválil za správne riešenia.
Určte výsledky Adama a~Evy.}
\podpis{Libuše Hozová}

{%%%%% Z9-III-1
Pre tri neznáme prirodzené čísla platí, že
\begin{itemize}
\iitem najväčší spoločný deliteľ prvého a~druhého je 8,
\iitem najväčší spoločný deliteľ druhého a~tretieho je 2,
\iitem najväčší spoločný deliteľ prvého a~tretieho je 6,
\iitem najmenší spoločný násobok všetkých troch čísel je 1680,
\iitem najväčšie z~čísel je väčšie ako 100, ale nie je väčšie ako 200,
\iitem jedno z~čísel je štvrtou mocninou celého čísla.
\end{itemize}
O~ktoré čísla ide? Určte všetky možnosti.
}
\podpis{Eva Semerádová}

{%%%%% Z9-III-2
Trojuholník $ACH$ je určený tromi vrcholmi kocky $ABCDEFGH$, pozri obrázok.
Výška tohto trojuholníka na stranu $CH$ má veľkosť 12\,cm.
Vypočítajte obsah trojuholníka $ACH$ a~veľkosť hrany danej kocky.
\ifobrazkyvedla\else\inspdf{z9-iii-2_mensi.pdf}\fi
}
\podpis{Marie Krejčová}

{%%%%% Z9-III-3
Daná je postupnosť siedmich čísel $a$, $b$, $c$, $d$, $e$, $f$, $g$.
Každé z~čísel $b$, $c$, $d$, $e$, $f$ je aritmetickým priemerom susedných dvoch čísel.
Ukážte, že číslo $d$ je aritmetickým priemerom čísel $a$ a~$g$.}
\podpis{Karel Pazourek}

{%%%%% Z9-III-4
Dané sú rovnobežníky $ABGH$ a~$DEGH$, ktorých vrcholy $A$, $B$, $D$ a~$E$ ležia na jednej priamke.
Bod $C$ je priesečníkom úsečiek $BG$ a~$DH$,
bod $I$ leží na úsečke $AH$ a~bod $F$ leží na úsečke $EG$.
Mnohouholník $ABCDEGH$ pozostáva zo siedmich trojuholníkov, pričom medzi trojuholníkmi $ABI$, $BCI$, $CHI$, $DEF$, $CDF$ a~$CFG$ je jeden s~obsahom 3\,cm$^2$, jeden s~obsahom 5\,cm$^2$, dva s~obsahom 7\,cm$^2$ a~jeden s~obsahom 10\,cm$^2$.
Okrem trojuholníkov s~obsahmi 7\,cm$^2$ nemá žiadna ďalšia dvojica z~uvedených siedmich trojuholníkov rovnaký obsah.
Rozhodnite, či možno s~istotou určiť trojuholníky s~obsahmi 7\,cm$^2$.
Ďalej určte obsah mnohouholníka $ABCDEGH$; nájdite všetky možnosti.

{\it Poznámka}: Obrázok je len ilustračný.
\ifobrazkyvedla\else\inspdf{z9-iii-4_mensi.pdf}\fi
}
\podpis{Eva Semerádová}


