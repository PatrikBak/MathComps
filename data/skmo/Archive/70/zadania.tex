{%%%%% A-I-1
Na tabuli sú napísané (nie nutne rôzne) prvočísla, ktorých súčin je 105-krát väčší ako ich súčet. Určte všetky napísané prvočísla, ak ich je
\ite a) päť;
\ite b) sedem.}
\podpis{Tomáš Jurík, Jaromír Šimša}

{%%%%% A-I-2
V~ostrouhlom trojuholníku $ABC$ ležia na strane~$BC$ body $D$ a~$E$ tak, že $D$ je medzi $B$ a~$E$, $|AD| = |CD|$ a~$|AE| = |BE|$. Bod~$F$ je taký bod, že $FD \parallel AB$ a~$FE \parallel AC$. Dokážte, že $|FB| = |FC|$.
}
\podpis{Patrik Bak}

{%%%%% A-I-3
Ak sú $a$, $b$, $c$ navzájom rôzne kladné reálne čísla, aký je
najmenší možný počet rôznych čísel medzi číslami
$a+b$, $b+c$, $c+a$, $ab$, $bc$, $ca$, $abc$?}
\podpis{Patrik Bak}

{%%%%% A-I-4
Najväčšieho deliteľa $d$ prirodzeného čísla~$n>1$ s~vlastnosťou $d<n$ nazveme
jeho {\it superdeliteľom}.
\ite a)
Dokážte, že každé prirodzené číslo $d>1$ je superdeliteľom iba konečného počtu čísel.
\ite b)
Označme $s(d)$ súčet všetkých čísel, ktorých superdeliteľom
je dané číslo~${d>1}$. Rozhodnite, či existuje nepárne
číslo~$d>1$ také, že $s(d)$ je násobkom čísla~2\,020.\endgraf
}
\podpis{Michal Rolínek}

{%%%%% A-I-5
V~trojuholníku $ABC$ označme $S_a$, $S_b$, $S_c$ postupne stredy jeho
strán $BC$, $CA$, $AB$. Dokážte, že pre ľubovoľný bod~$X$ rôzny
od bodov $S_a$, $S_b$, $S_c$ platí
$$
\min \left \{\frac {|XA|}{|XS_a|}, \frac {|XB|}{|XS_b|},
\frac {|XC|}{|XS_c|} \right\} \le 2.
\belowdisplayskip0pt
$$}
\podpis{Patrik Bak}

{%%%%% A-I-6
Majme 70 zhasnutých žiaroviek.
Pre ľubovoľnú skupinu žiaroviek vieme pripraviť prepínač,
ktorý zmení stav každej žiarovky z tejto skupiny (zhasne
zasvietené a rozsvieti zhasnuté) a~ostatné žiarovky neovplyvní.
Aký je najmenší počet prepínačov, pomocou ktorých je
možné rozsvietiť ľubovoľnú štvoricu žiaroviek (pričom ostatné budú
zhasnuté)?}
\podpis{Martin Melicher}

{%%%%% B-I-1
Z~cifier $0$ až $9$ vytvoríme dvojciferné čísla $AB$, $CD$, $EF$,
$GH$, $IJ$, pričom každú cifru použijeme práve raz. Zistite,
koľko rôznych hodnôt môže nadobúdať súčet $AB+CD+EF+GH+IJ$ a~ktoré
hodnoty to sú.
(Zápisy typu $07$ nepovažujeme za dvojciferné čísla.)}
\podpis{Jaroslav Zhouf}

{%%%%% B-I-2
Aká je najväčšia možná hodnota výrazu $xy-x^3y-xy^3$, ak sú
$x$, $y$ kladné reálne čísla?
Pre ktoré $x$, $y$ sa táto hodnota dosahuje?}
\podpis{Mária Dományová, Patrik Bak}

{%%%%% B-I-3
V~ostrouhlom trojuholníku $ABC$ sú $AA'$ a~$BB'$ jeho výšky. Kolmý priemet
bodu~$A'$ na výšku~$BB'$ označme~$D$. Predpokladajme, že kružnica
prechádzajúca bodmi $B$, $C$, $D$ pretína stranu~$AC$ v~jej vnútornom
bode~$E$.
Dokážte, že $|DE|=|AA'|$.}
\podpis{Patrik Bak}

{%%%%% B-I-4
Zistite, pre ktoré hodnoty reálneho parametra $k$ má sústava rovníc
$$
\align
|x+6|+2|y|&=24,\\
|x+y|+|x-y|&=2k
\endalign
$$
nepárny počet riešení v~obore reálnych čísel.}
\podpis{Pavel Calábek}

{%%%%% B-I-5
Daný je pravidelný sedemuholník $ABCDEFG$. Kolmica vedená bodom~$D$ na priamku~$DE$
pretína priamky $CG$ a~$AB$ postupne v~bodoch $P$ a~$Q$. Dokážte,
že $|AQ|+|EF|=|GP|$.}
\podpis{Marián Poturnay}

{%%%%% B-I-6
Na pláne s~rozmermi $12 \times 12$ štvorčekov sa nachádza loď tvorená
ôsmimi políčkami pozdĺž obvodu štvorca $3 \times 3$ (na \obr{} je vyznačená
sivou farbou). Na koľko najmenej políčok treba vystreliť, aby sme
s~istotou zasiahli loď aspoň raz?
\insp{b70.1}}
\podpis{Jozef Rajník}

{%%%%% C-I-1
Určte všetky dvojice $(m,n)$ prirodzených čísel, pre ktoré platí
$$
m+s(n)=n+s(m)=70,
$$
pričom $s(a)$ označuje ciferný súčet prirodzeného čísla~$a$.}
\podpis{Jaroslav Švrček}

{%%%%% C-I-2
Určte, pre ktoré prirodzené čísla~$n$ možno tabuľku $n\times n$ vyplniť
číslami $2$ a~$\m1$ tak, aby súčet všetkých čísel v~každom
riadku a~v~každom stĺpci bol rovný~0.}
\podpis{Ján Mazák}

{%%%%% C-I-3
V~pravouhlom trojuholníku $ABC$ s~preponou~$AB$ označme postupne
$I$ a~$U$ stred kružnice jemu vpísanej a~dotykový bod tejto kružnice
s~odvesnou~$BC$. Určte, aký je pomer $|AC|:|BC|$, ak sú
uhly $C\!AU$ a~$CBI$ zhodné.}
\podpis{Jaroslav Zhouf}

{%%%%% C-I-4
Určte, aké hodnoty môže nadobúdať výraz
$$\frac{a+bc}{a+b}+\frac{b+ca}{b+c}+\frac{c+ab}{c+a},$$
ak sú $a$, $b$, $c$ kladné reálne čísla so súčtom~1.}
\podpis{Michal Rolínek, Pavel Calábek}

{%%%%% C-I-5
Daný je trojuholník $ABC$ s~ťažiskom~$T$. Na priamkach $AT$ a~$BT$ sú
zvolené postupne body $E$ a~$F$ tak, že štvoruholník $TECF$
je rovnobežník. Dokážte, že úsečky $AC$ a~$BC$ delia úsečku~$EF$ na tri
zhodné časti.}
\podpis{Tomáš Jurík}

{%%%%% C-I-6
Na tabuli je napísaných niekoľko prirodzených čísel od~1 do~100,
pričom žiadne z~nich nie je deliteľné dvojciferným prvočíslom
a~súčin žiadnych dvoch z~nich nie je druhou mocninou prirodzeného čísla.
\ite a)
Určte najväčší možný počet čísel na tabuli.
\ite b)
Určte najväčší možný súčet čísel na tabuli.}
\podpis{Jaromír Šimša}

{%%%%% A-S-1
Daný je ostrouhlý trojuholník $ABC$. Vnútri jeho strán $AB$ a $AC$ ležia
postupne body $D$ a $E$ tak, že $|CD|=|CA|$ a $|BE|=|BA|$. Označme
$F$ stred kružnice opísanej trojuholníku $ADE$. Dokážte, že $AF\perp BC$.
}
\podpis{Patrik Bak, Josef Tkadlec}

{%%%%% A-S-2
Na tabuli je napísaných niekoľko (nie nutne rôznych) prvočísel tak, že
ich súčin je 2020-krát väčší ako ich súčet. Určte
ich najmenší možný počet.
}
\podpis{Patrik Bak}

{%%%%% A-S-3
Ak sú $a$, $b$, $c$ navzájom rôzne nezáporné reálne čísla, aký je
najmenší možný počet rôznych čísel medzi číslami
$a+b$, $b+c$, $c+a$, $a^2+b^2$, $b^2+c^2$, $c^2+a^2$?}
\podpis{Patrik Bak}

{%%%%% A-II-1
Koľko rôznych čísel môže byť medzi číslami
$$
a+2b,\quad a+2c,\quad b+2a,\quad b+2c,\quad c+2a,\quad c+2b,
$$
ak sú $a$, $b$, $c$ navzájom rôzne reálne čísla? Nájdite všetky možnosti.
}
\podpis{Josef Tkadlec}

{%%%%% A-II-2
Určte všetky trojice zložených prirodzených čísel, z~ktorých každé
je deliteľné súčinom superdeliteľov zvyšných dvoch čísel.
(Superdeliteľom čísla $n>1$ je jeho najväčší deliteľ $d$
s~vlastnosťou $d<n$.)}
\podpis{Jaromír Šimša}

{%%%%% A-II-3
V~ostrouhlom trojuholníku $ABC$ sú $D$ a $E$ vnútorné body strany~$BC$,
pritom~$D$ leží medzi $B$ a~$E$, $|AD|=|CD|$ a~$|AE|=|BE|$.
Predpokladajme, že os uhla $DAE$ má s~osou úsečky~$BC$ jediný
spoločný bod, ktorý označíme~$F$. Dokážte rovnosť
$|\uhel BAC|+|\uhel DFE|=180\st$.
}
\podpis{Patrik Bak}

{%%%%% A-II-4
Okolo kruhu je usporiadaných 70 zhasnutých žiaroviek. Pre ľubovoľnú skupinu
žiaroviek sme schopní pripraviť prepínač,
ktorý zmení stav každej žiarovky z~tejto skupiny (zhasne
rozsvietené a~rozsvieti zhasnuté) a~ostatné žiarovky neovplyvní.
Aký je najmenší počet prepínačov, pomocou ktorých je
možné rozsvietiť ľubovoľnú štvoricu susedných žiaroviek (pričom ostatné budú
zhasnuté)?}
\podpis{Martin Melicher}

{%%%%% A-III-1
Zlomok s~1010 políčkami v~čitateli a 1011 políčkami v~menovateli slúži ako hrací plán pre hru dvoch hráčov.
$$
\frac{\square+\square+\dots+\square}
{\square+\square+\dots+\square+\square}
$$
Hráči sa striedajú v~ťahoch. V~každom ťahu hráč vyberie jedno
z~čísel $1,2, \dots, 2021$ a vloží ho do ľubovoľného prázdneho
políčka. Každé číslo pritom môže byť použité iba raz.
Začínajúci hráč vyhráva, ak sa hodnota zlomku po zaplnení
všetkých políčok líši od čísla $1$ o~menej ako $10^{\m6}$. V~opačnom
prípade vyhráva druhý hráč. Rozhodnite, ktorý z~hráčov má vyhrávajúcu
stratégiu.}
\podpis{Pavel Šalom}

{%%%%% A-III-2
Označme~$I$ stred kružnice vpísanej pravouhlému trojuholníku $ABC$
s~pravým uhlom pri vrchole~$A$. Ďalej označme $M$ a~$N$ stredy
úsečiek $AB$ a~$BI$. Dokážte, že priamka~$CI$ je dotyčnicou kružnice
opísanej trojuholníku $BMN$.
}
\podpis{Patrik Bak, Josef Tkadlec}

{%%%%% A-III-3
Navzájom rôzne nenulové reálne čísla $a$, $b$, $c$ spĺňajú
množinovú rovnosť
$$
\{a+b,b+c,c+a\}=\{ab,bc,ca\}.
$$
Dokážte, že platí aj rovnosť
$$
\postdisplaypenalty=10000
\{a,b,c\}=\bigl\{a^2-2,b^2-2,c^2-2\bigr\}.
$$
}
\podpis{Josef Tkadlec}

{%%%%% A-III-4
Nájdite všetky prirodzené čísla $n$, pre ktoré platí rovnosť
$$
n+d(n)+d(d(n))+\cdots=2021,
$$
pričom $d(0)=d(1)=0$ a pre $k>1$ je $d(k)$ superdeliteľ čísla~$k$
(\tj. jeho najväčší deliteľ~$d$ s~vlastnosťou $d<k$).
}
\podpis{Tomáš Bárta}

{%%%%% A-III-5
Reťazec znakov nazveme {\it úhľadným}, keď má párnu dĺžku a jeho
prvá polovica je zhodná s~druhou polovicou (napr. $abab$).
Reťazec nazveme {\it pekným}, ak ho možno rozdeliť na niekoľko
úhľadných reťazcov (ako $abcabcdede\mskip-1mu f\!f$ na $abcabc$, $dede$ a $f\!f$).
{\it Redukciou\/} reťazca nazveme operáciu, pri ktorej z reťazca
zotrieme dva rovnaké susedné znaky (napr. reťazec $abbac$ možno
zredukovať na $aac$ a ten ďalej na $c$). Dokážte, že ľubovoľný
reťazec obsahujúci každý svoj znak v~párnom počte možno
získať sériou redukcií z~vhodného pekného reťazca.
}
\podpis{Martin Melicher}

{%%%%% A-III-6
Daný je ostrouhlý trojuholník $ABC$. Pre každý jeho vnútorný bod $X$
označme $X_a$,~$X_b$,~$X_c$ jeho obrazy v~osových súmernostiach
postupne podľa priamok $BC$,~$CA$,~$AB$. Dokážte, že všetky
trojuholníky $X_aX_bX_c$ majú spoločný bod.
}
\podpis{Josef Tkadlec}

{%%%%% B-S-1
Pre reálne čísla $x$, $y$, $z$ platí
$$\align
|x+y|&=1-z,\\
|y+z|&=1-x,\\
|z+x|&=1-y.
\endalign$$
Zistite, aké všetky hodnoty môže nadobúdať súčet $x+y+z$. Pre
každý vyhovujúci súčet uveďte príklad prislúchajúcich čísel
$x$, $y$, $z$.}
\podpis{Mária Dományová, Patrik Bak}

{%%%%% B-S-2
Uvažujme trojuholník $ABC$, v ktorom je $|\uhel BAC|<60\st$.
Obraz bodu $B$ v osovej súmernosti podľa priamky $AC$ označme~$D$,
obraz $C$ podľa $AB$ označme~$E$ a obraz $B$ podľa $AD$ označme
$F$. Dokážte, že $|CF|=|DE|$.
}
\podpis{Patrik Bak}

{%%%%% B-S-3
Dokážte, že ak pre nejaké kladné celé čísla $a$, $b$, $c$ je
$3^a\cdot7^b-10^c$ kladné dvojciferné číslo, tak je to prvočíslo.}
\podpis{Josef Tkadlec}

{%%%%% B-II-1
\ite a) Dokážte nerovnosť $4(a^2+b^2)>(a+b)^2+ab$ pre všetky dvojice
kladných reálnych čísel $a$, $b$.
\ite b) Nájdite najmenšie reálne číslo $k$ také, aby nerovnosť
$k(a^2+b^2)\geqq(a+b)^2+ab$ platila pre všetky dvojice kladných
reálnych čísel $a$, $b$.
}
\podpis{Ján Mazák}

{%%%%% B-II-2
Pre každé kladné celé číslo $k$ označme $d_k$ počet jednociferných
deliteľov čísla~$k$. Dokážte, že pre každé kladné celé číslo $n$
platí
$$
\frac {d_{1}+d_{2}+\cdots+d_{n}}{n}<3.
$$
}
\podpis{Josef Tkadlec}

{%%%%% B-II-3
Daný je pravouhlý trojuholník $ABC$ s pravým uhlom pri vrchole $C$.
Nech $D$ je ľubovoľný vnútorný bod odvesny $AC$ a $p$ kolmica z bodu $D$
na preponu $AB$. Označme $E\ne D$ bod priamky $p$ taký, že
body $A$, $B$, $D$, $E$ ležia na kružnici. Označme ešte~$F$
priesečník priamok $p$ a $BC$. Dokážte, že $|AE|=|AF|$.
}
\podpis{Jaroslav Švrček}

{%%%%% B-II-4
Na hracom pláne s rozmermi $9\times9$ štvorčekov je umiestnená
loď tvorená ôsmimi štvorčekmi po obvode štvorca $3\times3$ (na
obrázku je vyznačená sivou farbou).
\ite a) Na koľko štvorčekov musíme vystreliť, aby sme mali istotu,
že loď zasiahneme aspoň na dvoch rôznych miestach? O~prvom zásahu sa
pritom nedozvieme.
\ite b) Stačí rovnaký počet výstrelov aj pre hrací plán s veľkosťou~$11\times 9$?
\insp{b70.1}
}
\podpis{Tomáš Bárta}

{%%%%% C-S-1
Určte všetky dvojice $(m,n)$ prirodzených čísel, pre ktoré platí
$$
2m+2s(n)=n\cdot s(m)=70,
$$
pričom $s(a)$ označuje ciferný súčet prirodzeného čísla $a$.
}
\podpis{Jaroslav Švrček}

{%%%%% C-S-2
Daný je rovnobežník $ABCD$, v~ktorom $K$, $L$ sú postupne stredy
strán $BC$,~$AD$. Nech päta $M$ kolmice z~bodu $D$
na priamku $AB$ leží vnútri strany $AB$ daného rovnobežníka a
nech $N$ je stred úsečky $MB$. Dokážte, že $|NK|=|NL|$.
}
\podpis{Vojtěch Zlámal}

{%%%%% C-S-3
Nech $a$, $b$, $c$ sú kladné reálne čísla, pre ktoré platí $ab+bc+ca=1$.
Určte, aké hodnoty nadobúda výraz
$$
\frac{a(b^2+1)}{a+b}+\frac{b(c^2+1)}{b+c}+\frac{c(a^2+1)}{c+a}.
$$
}
\podpis{Josef Tkadlec, Patrik Bak}

{%%%%% C-II-1
Určte všetky prirodzené čísla $n$, pre ktoré platí
$$
n+p(n)=70,
$$
pričom $p(n)$ označuje {\it súčin\/} všetkých cifier čísla $n$.
}
\podpis{Jaroslav Švrček}

{%%%%% C-II-2
Určte, pre ktoré prirodzené čísla $n$ možno štvorcovú tabuľku
$n\times n$, ktorej políčka sú ofarbené ako políčka
šachovnice, vyplniť číslami $2$ a $\m1$ tak, že súčasne platí:
\ite (i) súčet všetkých čísel v~každom riadku aj v~každom stĺpci tabuľky
je rovný 0,
\ite (ii) súčet čísel na všetkých čiernych políčkach tabuľky sa rovná
súčtu čísel na všetkých jej bielych políčkach.\endgraf
}
\podpis{Martin Melicher}

{%%%%% C-II-3
Daný je trojuholník $ABC$, v~ktorom $D$, $E$ sú postupne stredy
strán $BC$,~$AB$. Nech $F$ je stred úsečky $BE$
a $G$ vnútorný bod strany $AC$, pre ktorý platí $|AG|=3\,|CG|$.
Dokážte, že priesečník priamok $DF$ a $GE$ leží na tej rovnobežke
s~priamkou~$BC$, ktorá prechádza bodom $A$.
}
\podpis{Patrik Bak}

{%%%%% C-II-4
Nech $a$, $b$ sú ľubovoľné kladné reálne čísla, pre ktoré platí $a^2+b^2=1$.
Nájdite najmenšiu možnú hodnotu výrazu
$$
\frac{a^2(a+b^3)}{b-b^3}+\frac{b^2(b+a^3)}{a-a^3}
$$
a určte, pre ktoré uvažované dvojice $a$, $b$ je táto hodnota dosiahnutá.}
\podpis{Tomáš Bárta}

{%%%%%   vyberko, den 1, priklad 1
[A0]
Nájdite všetky mnohočleny~$P$ s~reálnymi koeficientami také, že pre každé reálne číslo~$x$ platí
$$
(x+1)P(x-1) = (x-1)P(x).
\belowdisplayskip0pt
$$}
\podpis{...}

{%%%%%   vyberko, den 1, priklad 2
[C0]
Určte najväčší možný počet veží, ktoré je možné umiestniť na povrch šachovnicovej kocky rozmerov $100 \times 100 \times 100$, pričom žiadne dve veže sa neohrozujú. (Veža ohrozuje políčka nachádzajúce sa v~rovnakom riadku alebo stĺpci, pričom tieto riadky a~stĺpce pokračujú do všetkých stien, viď \obr.)
\Image{c0.pdf}{0.8}%
}
\podpis{...}

{%%%%%   vyberko, den 1, priklad 3
[G0]
Je daný ostrouhlý trojuholník~$ABC$. Predpokladajme, že~$X$ a~$Y$ sú také body, že $BX$ a~$CY$ sú dotyčnice kružnice opísanej~$ABC$, $|AB|=|BX|$, $|AC|=|CY|$ a~$X,Y,A$ ležia v~rovnakej polrovine vzhľadom na priamku~$BC$. Dokážte, že ak~$I$ je stred kružnice vpísanej trojuholníku~$ABC$, tak $|\angle BAC| + |\angle XIY| = 180^\circ$.
}
\podpis{...}

{%%%%%   vyberko, den 1, priklad 4
[N0]
Nájdite všetky kladné celé čísla~$n$ s~počtom kladných deliteľov rovným~$\sqrt{n+1}$.
}
\podpis{...}

{%%%%%   vyberko, den 2, priklad 1
[A1]
Nech $a_1,a_2,\ldots,a_n$ je postupnosť reálnych čísel spĺňajúcich $a_1+a_2+\cdots+a_n=0$. Pre každé celé~$i$ spĺňajúce $1 \leq i\leq n$ označme $b_i=a_1+a_2+\cdots+a_i$. Predpokladajme, že pre každé celé~$i$ spĺňajúce $1 \leq i \leq n-2$ platí $b_i(a_{i+2}-a_{i+1}) \ge 0$. Dokážte, že
$$
\max_{1 \leq l \leq n} |a_l| \ge
\max_{1 \leq m \leq n} |b_m|.
\belowdisplayskip0pt
$$
}
\podpis{...}

{%%%%%   vyberko, den 2, priklad 2
[N2]
Pre každé prvočíslo~$p$ existuje kráľovstvo $p$-Landia obsahujúce~$p$ ostrovov očíslovaných~$1,2\ldots,p$. Dva rôzne ostrovy s~číslami~$m$ a~$n$ sú spojené mostom práve vtedy, keď $p$~delí $(n^2-m+1)(m^2-n+1)$. Dokážte, že pre nekonečne veľa prvočísel~$p$ existujú dva ostrovy v~$p$-Landii, ktoré nie sú spojené reťazcom mostov.
}
\podpis{...}

{%%%%%   vyberko, den 2, priklad 3
[C3]
V~rovine je daných~1000 bodov, z ktorých žiadne tri neležia na jednej priamke. Pre každé tri z~nich zapíšeme obsah trojuholníka nimi tvoreného. Dokážte, že k~týmto číslam vieme zvoliť znamienka tak, aby ich súčet bol~0.
}
\podpis{...}

{%%%%%   vyberko, den 3, priklad 1
[C1]
Nech~$n$ je kladné celé číslo. Nájdite počet permutácií $a_1,a_2,\ldots,a_n$ čísel $1,2,\ldots,n$ takých, že platí
$$
a_1 \leq 2a_2 \leq 3a_3 \leq \ldots \leq n a_n.
\belowdisplayskip0pt
$$
}
\podpis{...}

{%%%%%   vyberko, den 3, priklad 2
[A2]
Predpokladajme, že kladné reálne čísla $a$, $b$, $c$, $d$ spĺňajú $(a+c)(b+d) = ac + bd$. Nájdite najmenšiu možnú hodnotu výrazu
$$
\frac ab + \frac bc + \frac cd + \frac da.
\belowdisplayskip0pt
$$
}
\podpis{...}

{%%%%%   vyberko, den 3, priklad 3
[G3]
Body $D$, $E$, $F$ sú päty výšok po rade z~vrcholov $A$, $B$, $C$ ostrouhlého trojuholníka~$ABC$ s~ortocentrom~$H$. Nech~$E'$ a~$F'$ sú postupne obrazy bodov~$E$ a~$F$ v~osovej súmernosti podľa priamky~$AD$. Predpokladajme, že priamky $BF'$ a~$CE'$ sa pretínajú v~bode~$X$, zatiaľ čo priamky~$BE'$ a~$CF'$ sa pretínajú v~bode~$Y$. Dokážte, že priamky $AX$, $YH$ a~$BC$ sa pretínajúce v~jednom bode.
}
\podpis{...}

{%%%%%   vyberko, den 4, priklad 1
[N1]
Je dané kladné celé číslo~$k$. Dokážte, že existuje prvočíslo~$p$ také, že si vieme zvoliť po dvoch rôzne čísla $a_1,a_2,\ldots,a_{k+3}$ z~množiny~$\{1,2,\ldots,p-1\}$ tak, že $p \mid a_ia_{i+1}a_{i+2}a_{i+3}-i$ pre všetky $i=1,2,\ldots,k$.
}
\podpis{...}

{%%%%%   vyberko, den 4, priklad 2
[G2]
Je daný konvexný štvoruholník~$ABCD$ spĺňajúci $|\angle ABC|>90^\circ$, $|\angle CDA|>90^\circ$, a~$|\angle DAB|=|\angle BCD|$. Označme~$E$ a~$F$ obrazy bodu~$A$ v~osovej súmernosti postupne podľa priamok~$BC$ a~$CD$. Predpokladajme, že úsečky $AE$ a~$AF$ pretínajú priamku $BD$ po rade v~bodoch~$K$ a~$L$. Dokážte, že kružnice opísané trojuholníkom~$BEK$ a~$DFL$ sa navzájom dotýkajú.
}
\podpis{...}

{%%%%%   vyberko, den 4, priklad 3
[A3]
Nájdite všetky funkcie $f\colon {\Bbb Z} \to {\Bbb Z}$ také, že
$$
f^{a^2+b^2}(a+b)=af(a)+bf(b)
\qquad \hbox{pre každé $a,b \in {\Bbb Z}$.}
$$
(Symbol $f^n$ označuje $n$-tú iteráciu~$f$, teda $f^0(x)=x$, $f^{n+1}(x)=f(f^n(x))$ pre všetky celé $n \ge 0$.)
}
\podpis{...}

{%%%%%   vyberko, den 5, priklad 1
[G1]
Nech~$ABC$ je rovnoramenný trojuholník spĺňajúci $|BC|=|CA|$. Bod~$D$ je ľubovoľný bod vnútri úsečky~$AB$ taký, že $|AD|<|DB|$. Nech~$P$ a~$Q$ sú dva body postupne vnútri strán~$BC$ a~$AC$ spĺňajúce $|\angle DPB|=|\angle DQA|=90^\circ$. Predpokladajme, že os úsečky~$PQ$ pretína úsečku~$CQ$ vo vnútornom bode~$E$, a~že kružnice opísané trojuholníkom~$ABC$ a~$CPQ$ sa pretínajú v~bode~$F \ne C$. Dokážte, že ak body $P,E,F$ ležia na jednej priamke, tak $|\angle ACB|=90^\circ$.
}
\podpis{...}

{%%%%%   vyberko, den 5, priklad 2
[C2]
Fibonacciho postupnosť~$F_0,F_1,F_2,\ldots$ je definovaná induktívne ako $F_0=0$, $F_1=1$ a~$F_{n+1}=F_{n}+F_{n-1}$ pre $n \ge 1$. Pre každé celé číslo~$n \ge 2$ určte najmenšiu možnú veľkosť množiny~$S$ celých čísel takých, že pre každé $k=2,3,\ldots,n$ existujú nejaké čísla $x,y \in S$ také, že $x-y=F_k$.
}
\podpis{...}

{%%%%%   vyberko, den 5, priklad 3
[N3]
Nech~${\Cal S}$ je~$n$-prvková množina aspoň troch kladných celých čísel takých, že súčet žiadnych dvoch rôznych prvkov nie je rovný nijakému tretiemu. Dokážte, že prvky~${\Cal S}$ vieme usporiadať ako $a_1,a_2,\ldots,a_n$ tak, že pre každé $i=2,3,\ldots,n-1$ platí, že $a_i$ nedelí $a_{i-1}+a_{i+1}$.
}
\podpis{...}

{%%%%%   vyberko, den 2, priklad 4
...}
\podpis{...}

{%%%%%   vyberko, den 3, priklad 4
...}
\podpis{...}

{%%%%%   vyberko, den 4, priklad 4
...}
\podpis{...}

{%%%%%   vyberko, den 5, priklad 4
...}
\podpis{...}

{%%%%%   trojstretnutie, priklad 1
Nájdite všetky štvorice kladných celých čísel~$(a, b, c, d)$ spĺňajúcich $\hbox{NSD}(a, b, c, d) = 1$ a~súčasne
$$
a \mid b + c,\quad
b \mid c + d,\quad
c \mid d + a,\quad
d \mid a + b.
$$
}
\podpis{Vítězslav Kala, Česká rep.}

{%%%%%   trojstretnutie, priklad 2
Vpísaná kružnica~$\omega$ ostrouhlému trojuholníku~$ABC$ sa dotýka strany~$BC$ v bode~$D$. Označme~$I_a$ stred kružnice pripísanej trojuholníku~$ABC$ oproti vrcholu~$A$ a~$M$ stred úsečky~$DI_a$. Dokážte, že kružnica opísaná trojuholníku $BMC$ sa dotýka kružnice~$\omega$.
}
\podpis{Patrik Bak, Slovensko}

{%%%%%   trojstretnutie, priklad 3
Pre ľubovoľné dva konvexné mnohouholníky $P_1$ a~$P_2$ s navzájom rôznymi vrcholmi označme $f(P_1, P_2)$ celkový počet vrcholov, ktoré ležia na strane druhého mnohouholníka. Pre každé kladné celé číslo~$n \ge 4$ určte
$$
\max\{ f(P_1, P_2) \mid
       \hbox{$P_1$ a $P_2$ sú konvexné $n$-uholníky}
    \}.
$$
(O mnohouholníku povieme, že je {\it konvexný}, ak veľkosť každého jeho vnútorného uhla je ostro menšia ako~$180^\circ$.)
}
\podpis{Josef Tkadlec, Česká rep.}

{%%%%%   trojstretnutie, priklad 4
Určte počet usporiadaných 2021-tíc kladných celých čísel, ktoré obsahujú číslo $3$ a~každé dve susedné čísla sa líšia najviac o~1.
}
\podpis{Walther Janous, Rakúsko}

{%%%%%   trojstretnutie, priklad 5
Postupnosť $a_1, a_2, a_3, \dots$ spĺňa $a_1 = 1$ a pre všetky $n \ge 2$, platí
$$
a_n = \begin{cases}
			a_{n-1} + 3 &
			    \text{ak } n-1 \in \{a_1, a_2,\ldots ,a_{n-1}\};\cr
			a_{n-1} + 2 &
			    \text{inak.}
	  \end{cases}
$$
Dokážte, že pre všetky kladné celé čísla~$n$ platí
$$
a_n < n\cdot(1 + \sqrt2).
\belowdisplayskip0pt
$$
}
\podpis{Dominik Burek, Poľsko}

{%%%%%   trojstretnutie, priklad 6
Nech $ABC$ je ostrouhlý trojuholník a predpokladajme, že body $A$, $A_b$, $B_a$, $B$, $B_c$, $C_b$, $C$, $C_a$, a~$A_c$ ležia v tomto poradí na jeho obvode. Nech $A_1\ne A$ je druhý priesečník kružníc opísaných trojuholníkom $AA_bC_a$ a~$AA_cB_a$. Analogicky, $B_1\ne B$ je druhý priesečník kružníc opísaných trojuholníkom $BB_cA_b$ a~$BB_aC_b$, a $C_1\ne C$ je druhý priesečník kružníc opísaných trojuholníkom $CC_aB_c$ a~$CC_bA_c$.
Predpokladajme, že body $A_1$, $B_1$ a~$C_1$ sú všetky rôzne, ležia vnútri trojuholníka $ABC$ a neležia na jednej priamke.
Dokážte, že priamky $AA_1$, $BB_1$, $CC_1$ a kružnica opísaná trojuholníku $A_1B_1C_1$ sa všetky pretínajú v~jednom bode.
}
\podpis{Josef Tkadlec, Česká rep., Patrik Bak, Slovensko}

{%%%%%   IMO, priklad 1
Je dané celé číslo $n>100$. Ivan napísal každé z čísel $n,n+1,\ldots,2n$ na jednu kartičku.
Potom týchto $n+ 1$ kariet zamiešal a rozdelil na dve kôpky. Dokážte, že aspoň jedna z týchto kôpok
obsahuje dve kartičky, ktorých súčet čísel je druhá mocnina celého čísla.}
\podpis{Austrália}

{%%%%%   IMO, priklad 2
Dokážte, že nerovnosť
$$
\sum_{i=1}^n \sum_{j=1}^n \sqrt{|x_i-x_j|} \leq
\sum_{i=1}^n \sum_{j=1}^n \sqrt{|x_i+x_j|}
$$
platí pre všetky reálne čísla $x_1,\ldots,x_n$.}
\podpis{Kanada}

{%%%%%   IMO, priklad 3
Nech~$D$ je vnútorný bod ostrouhlého trojuholníka~$ABC$ spĺňajúceho $|AB|>|AC|$ taký, že
$|\angle DAB|= |\angle CAD|$. Bod~$E$ leží na úsečke~$AC$ a spĺňa $|\angle ADE|= |\angle BCD|$. Bod $F$ leží na úsečke $AB$
a spĺňa $|\angle FDA|= |\angle DBC|$. Bod $X$ leží na priamke $AC$ a~spĺňa $|CX|= |BX|$. Nech $O_1$ a $O_2$ sú
postupne stredy kružníc opísaných trojuholníkom $ADC$ a $EXD$. Dokážte, že priamky $BC$, $EF$, $O_1O_2$
sa pretínajú v jednom bode.}
\podpis{Ukrajina}

{%%%%%   IMO, priklad 4
Nech $\Gamma$ je kružnica so stredom $I$ a nech $ABCD$ je konvexný štvoruholník taký, že každá
z úsečiek $AB$, $BC$, $CD$ a $DA$ sa dotýka $\Gamma$. Nech $\Omega$ je kružnica opísaná trojuholníku $AIC$. Časť
polpriamky $BA$ za bodom $A$ pretína $\Omega$ v $X$ a časť polpriamky $BC$ za bodom $C$ pretína $\Omega$ v $Z$. Časti
polpriamok $AD$ a $CD$ za bodom $D$ pretínajú $\Omega$ postupne v bodoch $Y$ a $T$. Dokážte, že
$$
|AD|+ |DT |+ |T X|+ |XA|= |CD|+ |DY |+ |Y Z|+ |ZC|.
\belowdisplayskip0pt
$$}
\podpis{Poľsko}

{%%%%%   IMO, priklad 5
Dve veveričky, Chip a Dale, nazbierali 2021 orieškov na zimu. Dale očísloval oriešky
číslami od 1 do 2021 a vykopal 2021 dierok do kruhu v zemi okolo ich obľúbeného stromu. Ďalšie ráno
si Dale všimol, že Chip umiestnil do každej dierky jeden orech nedbajúc na číslovanie. Nešťastný Dale
sa rozhodol oriešky preusporiadať postupnosťou 2021 krokov. V $k$-tom kroku Dale vymení pozície
orechov susedných s orechom $k$. Dokážte, že existuje taká hodnota $k$, že v $k$-tom kroku Dale vymení
orechy $a$ a $b$ také, že $a < k < b$.}
\podpis{Španielsko}

{%%%%%   IMO, priklad 6
Nech $m >2$ je dané celé číslo, $A$ je daná konečná množina (nie nutne kladných) celých
čísel a $B_1, B_2, B_3, \ldots , B_m$ sú podmnožiny $A$. Predpokladajme, že pre každé $k = 1, 2, \ldots , m$ je súčet
prvkov množiny $B_k$ rovný $m_k$. Dokážte, že $A$ obsahuje aspoň $m/2$ prvkov.}
\podpis{Rakúsko}

{%%%%%   MEMO, priklad 1
Nájdite všetky reálne čísla $A$ také, že každá postupnosť nenulových reálnych čísel $x_1,x_2,\dots$ spĺňajúca
$$
x_{n+1}=A-\frac{1}{x_n}
$$
pre každé celé číslo $n\ge1$ obsahuje iba konečne veľa záporných členov.}
\podpis{Česká rep.}

{%%%%%   MEMO, priklad 2
Dané sú kladné celé čísla $m$ a $n$. Niektoré štvorčeky tabuľky $m \times n$ sú zafarbené načerveno. Postupnosť $2r \ge 4$ po dvoch rôznych červených štvorčekov $a_1, a_2, \dots, a_{2r}$ nazývame \emph{strelecký cyklus}, ak pre každé $k\in\{1,\dots, 2r\}$ ležia štvorčeky $a_k$ a~$a_{k+1}$ na diagonále, ale štvorčeky $a_k$ a $a_{k+2}$ neležia na diagonále (pritom $a_{2r+1}=a_1$ a $a_{2r+2}=a_2$). V~závislosti od $m$ a~$n$ určte najväčší možný počet červených štvorčekov v~tabuľke $m \times n$, v~ktorej sa nenachádza strelecký cyklus.

(\emph{Poznámka.} Dva štvorčeky ležia na diagonále, ak priamka prechádzajúca ich stredmi zviera so stranami tabuľky uhol $45^\circ$.)}
\podpis{Jozef Rajník, Slovensko}

{%%%%%   MEMO, priklad 3
Vo vnútri strany $BC$ ostrouhlého trojuholníka $ABC$ je daný bod $D$. Body $E$ a~$F$ ležia v~polrovine určenej priamkou $BC$ obsahujúcej bod $A$ tak, že priamka $DE$ je kolmá na priamku $BE$ a priamka $DE$ sa dotýka kružnice opísanej trojuholníku $ACD$, zatiaľ čo priamka $DF$ je kolmá na priamku $CF$ a priamka $DF$ sa dotýka kružnice opísanej trojuholníku $ABD$. Dokážte, že body $A$, $D$, $E$ a $F$ ležia na jednej kružnici.}
\podpis{Patrik Bak, Slovensko}

{%%%%%   MEMO, priklad 4
Je dané celé číslo $n\ge3$. Veverička Zagi sedí vo vrchole pravidelného $n$-uholníka. Zagi plánuje výlet pozostávajúci z $n-1$ skokov takých, že v~$i$-tom skoku skočí o~$i$ hrán v~smere hodinových ručičiek, kde $i\in \{1,\dots,n-1\}$. Dokážte, že ak Zagi po $\lceil{\frac n2}\rceil$ skokoch navštívil $\lceil{\frac n2}\rceil+1$ rôznych vrcholov, tak po $n-1$ skokoch navštívi všetky vrcholy.}
\podpis{Česká rep.}

{%%%%%   MEMO, priklad t1
Nájdite všetky funkcie $f\colon\Bbb R\to\Bbb R$ také, že nerovnosť
$$
f(x^2)-f(y^2)\le (f(x)+y)(x-f(y))
$$
je splnená pre všetky reálne čísla $x$ a $y$.
}
\podpis{Rakúsko}

{%%%%%   MEMO, priklad t2
Pre kladné celé číslo $n$ povieme, že polynóm $P(x)$ s~reálnymi koeficientmi je \emph{$n$-krásny}, ak má rovnica $P(\lfloor x \rfloor) = \lfloor P(x) \rfloor$ presne $n$ reálnych riešení. Ukážte, že pre každé kladné celé číslo $n$
\ite a) existuje $n$-krásny polynóm;
\ite b) každý $n$-krásny polynóm má stupeň aspoň $(2n+1)/3$.}
\podpis{Chorvátsko}

{%%%%%   MEMO, priklad t3
Dané sú kladné celé čísla $n$, $b$ a $c$. Skupina $n$ pirátov si chce férovo rozdeliť poklad. Poklad pozostáva z~$c \cdot n$ rovnakých mincí rozmiestnených v~$b \cdot n$ vreciach, z~ktorých je aspoň $n-1$ na začiatku prázdnych. Kapitán Jack si prezrie obsah jednotlivých vriec a~spraví postupnosť ťahov. V~každom ťahu môže premiestniť ľubovoľný počet mincí z~niektorého vreca do jedného prázdneho vreca. Dokážte, že bez ohľadu na to, ako sú mince na začiatku rozmiestnené, môže Jack spraviť nanajvýš $n-1$ ťahov tak, že potom môže rozdeliť vrecia medzi pirátov tak, aby každý pirát dostal $b$ vriec a~$c$ mincí.}
\podpis{Josef Tkadlec, Česká rep.}

{%%%%%   MEMO, priklad t4
Dané je prirodzené číslo~$n$. Dokážte, že v pravidelnom $6n$-uholníku môžeme nakresliť $3n$ uhlopriečok takých, že ich koncové body sú všetky po dvoch rôzne, a rozdeliť nakreslené uhlopriečky do $n$ trojíc tak, že:
\ite $\bullet$ uhlopriečky v každej trojici sa pretínajú v jednom vnútornom bode $6n$-uholníka a
\ite $\bullet$ všetkých týchto $n$ priesečníkov je rôznych.\endgraf
}
\podpis{Maďarsko}

{%%%%%   MEMO, priklad t5
Nech $AD$ je priemer kružnice opísanej ostrouhlému trojuholníku $ABC$. Rovnobežky so stranami $AB$ a $AC$ prechádzajúce bodom $D$ pretínajú priamky $AC$ a $AB$ postupne v bodoch $E$ a $F$. Priamky $EF$ a $BC$ sa pretínajú v bode $G$. Dokážte, že priamky $AD$ a $DG$ sú navzájom kolmé.}
\podpis{Rakúsko}

{%%%%%   MEMO, priklad t6
Nech $M$ je stred strany $BC$ trojuholníka $ABC$. Bod $X$ je zvolený na polpriamke $AB$ tak, že platí $2|\angle CXA|=|\angle CMA|$. Bod $Y$ je zvolený na polpriamke $AC$ tak, že platí $2|\angle AYB|=|\angle AMB|$. Priamka $BC$ pretína kružnicu opísanú trojuholníku $AXY$ v bodoch $P$ a $Q$ tak, že body $P$, $B$, $C$ a $Q$ ležia v tomto poradí na priamke $BC$. Dokážte, že $|PB|=|QC|$.}
\podpis{Poľsko}

{%%%%%   MEMO, priklad t7
Nájdite všetky dvojice kladných celých čísel $(n,p)$ také, že $p$ je prvočíslo a
$$
1+2+\dots + n  = 3\cdot (1^2 + 2^2 + \dots+ p^2).
$$}
\podpis{Chorvátsko}

{%%%%%   MEMO, priklad t8
Dokážte, že existuje nekonečne veľa kladných celých čísel $n$ takých, že $n^2$ napísané v sústave so základom $4$ obsahuje iba cifry $1$ a $2$.}
\podpis{Martin Melicher, Slovensko}

{%%%%%   CPSJ, priklad 1
Uvažujme tabuľku~$2 \times 2$. V~každom jej políčku je napísané kladné celé číslo. Ak sčítame súčin čísel v~prvom stĺpci, súčin čísel v~druhom stĺpci, súčin čísel v~prvom riadku a~súčin čísel v~druhom riadku, tak dostaneme $2021$.
\item{a)} Určte všetky možné hodnoty súčtu všetkých štyroch čísel v~tabuľke.
\item{b)} Nájdite počet tabuliek spĺňajúcich zadanie, ktoré obsahujú štyri navzájom rôzne čísla.\endgraf
}
\podpis{Tomáš Jurík, Patrik Bak, Slovensko}

{%%%%%   CPSJ, priklad 2
Je daný ostrouhlý trojuholník~$ABC$. Nech~$D$ a~$E$ sú päty kolmíc postupne z~bodov~$B$ a~$C$ na os vonkajšieho uhla~$BAC$. Označme~$F$ priesečník úsečiek~$BE$ a~$CD$. Dokážte, že priamka $AF$ je kolmá na priamku~$DE$.}
\podpis{Patrik Bak, Slovensko}

{%%%%%   CPSJ, priklad 3
{\it Krížom\/} nazveme útvar pozostávajúci zo šiestich jednotkových štvorčekov zobrazených na \obr{} (a~každý iný útvar, ktorý možno dostať jeho otočením).
\insp{cpsj70.1}%
Nájdite najväčší možný počet krížov, ktoré môžu byť vystrihnuté z~papiera rozmerov~$6 \times 11$ rozdeleného na jednotkové štvorčeky (každý kríž musí pozostávať z~práve šiestich z~nich).}
\podpis{Tomasz Cieśla, Poľsko}

{%%%%%   CPSJ, priklad 4
Nájdite najmenšiu možnú hodnotu, ktorú môže nadobúdať výraz
$$
x^4+y^4-x^2y-xy^2,
$$
v~ktorom~$x$ a~$y$ sú kladné reálne čísla spĺňajúce $x+y \le 1$.}
\podpis{Mária Dományová, Slovensko}

{%%%%%   CPSJ, priklad 5
Nech $ABCDEFG$ je pravidelný 7-uholník. Priamky $AB$ a~$CE$ sa pretínajú v~bode~$P$. Určte $|\angle PDG|$.}
\podpis{\L{}ukasz Bożyk, Tomasz Przyby\l{}owski, Poľsko}

{%%%%%   CPSJ, priklad t1
Mějme lichoběžník $ABCD$ se základnami $AB$ a $CD$, splňujícími $|AB|>|CD|$. Označme $M$ střed úsečky $AB$. Nechť bod $P$ leží uvnitř $ABCD$ a platí $|AD|=|PC|$ a~$|BC|=|PD|$. Dokaž, že pokud $|\angle CMD|=90^\circ$, pak čtyřúhelníky $AMPD$ a $BMPC$ mají stejný obash.}
\podpis{\L{}ukasz Bożyk, Poľsko}

{%%%%%   CPSJ, priklad t2
Mějme dána čísla $x_i \in \{-1, 1\}$ pro $i = 1, 2, \ldots, n$, splňující
$$
x_1x_2+x_2x_3+\ldots+x_{n-1}x_n+x_nx_1=0.
$$
Dokaž, že $n$ je dělitelné $4$.}
\podpis{Jaroslav Švrček, Česká rep.}

{%%%%%   CPSJ, priklad t3
Wyznacz liczb\ę{} par $(a,b)$ dodatnich liczb ca\l{}kowitych o~tej w\l{}asności, że najwi\ę{}kszy wspólny dzielnik liczb $a$ i~$b$ jest równy
$$
1\cdot 2\cdot 3\cdot\ldots\cdot 50,
$$
a najmniejsza wspólna wielokrotność liczb $a$ i~$b$ jest równa
$$
1^2\cdot 2^2\cdot 3^2\cdot\ldots\cdot 50^2.
$$
}
\podpis{Martin Melicher, Slovensko}

{%%%%%   CPSJ, priklad t4
Wyznacz najmniejsz\ą{} dodatni\ą{} liczb\ę{} ca\l{}kowit\ą{} $n$ o~tej w\l{}asności, że w~zbiorze
$$
\{70, 71, 72, \ldots, 70 + n\}
$$
można wskazać dwie różne liczby, których iloczyn jest kwadratem liczby ca\l{}kowitej.}
\podpis{Jaromír Šimša, Česká rep.}

{%%%%%   CPSJ, priklad t5
Nájdite všetky trojice reálnych čísel $(x,y,z)$ spĺňajúce sústavu rovníc
$$
\aligned
\frac{x}{y}+\frac{y}{z}+\frac{z}{x}&=\frac{x}{z}+\frac{z}{y}+\frac{y}{x},\\
x^2+y^2+z^2&=xy+yz+zx+4.
\endaligned
$$
}
\podpis{Jaroslav Švrček, Česká rep.}

{%%%%%   CPSJ, priklad t6
Nech~$s(n)$ označuje ciferný súčet kladného celého čísla~$n$. Použitím šiestich rôznych cifier sme vytvorili tri dvojciferné čísla $p$, $q$, $r$ také, že
$$
p\cdot q \cdot s(r) = p\cdot s(q)\cdot r = s(p)\cdot q\cdot r.
$$
Nájdite všetky takéto čísla $p$, $q$, $r$.
}
\podpis{Pavel Šalom, Česká rep.}

{%%%%%   EGMO, priklad 1
Číslo 2021 je \emph{bombastické}. Platí, že ak pre nejaké kladné celé číslo~$m$ je ľubovoľný prvok množiny $\{m,2m+1,3m\}$ bombastický, tak potom sú všetky tieto prvky bombastické. Rozhodnite, či číslo $2021^{2021}$ je bombastické.
}
\podpis{Austrália}

{%%%%%   EGMO, priklad 2
Nájdite všetky funkcie $f\colon\Bbb{Q}\to\Bbb{Q}$ také, že rovnica
$$
f(xf(x)+y)=f(y)+x^2
$$
je splnená pre všetky racionálne čísla~$x$ a~$y$.
}
\podpis{Patrik Bak, Slovensko}

{%%%%%   EGMO, priklad 3
Nech~$ABC$ je tupouhlý trojuholník s~tupým uhlom pri vrchole~$A$. Os vonkajšieho uhla pri vrchole~$A$ pretína výšky trojuholníka~$ABC$ vo vrcholoch~$B$ a~$C$ postupne v~bodoch~$E$ a~$F$. Nech~$M$ a~$N$ sú body ležiace postupne na úsečkách~$EC$ a~$FB$ také, že $|\angle EMA| = |\angle BCA|$ a~$|\angle ANF| = |\angle ABC|$. Dokážte, že body $E$, $F$, $N$, $M$ ležia na jednej kružnici.
}
\podpis{Ukrajina}

{%%%%%   EGMO, priklad 4
Máme daný trojuholník~$ABC$ so stredom kružnice vpísanej~$I$. Nech~$D$ je bod ležiaci na jeho strane~$BC$. Priamka prechádzajúca bodom~$D$ kolmá na~priamku~$BI$ pretína priamku~$CI$ v~bode~$E$. Analogicky, priamka prechádzajúca bodom~$D$ kolmá na priamku~$CI$ pretína priamku~$BI$ v~bode~$F$. Dokážte, že obraz bodu~$A$ v~osovej súmernosti podľa priamky~$EF$ leží na priamke~$BC$.
}
\podpis{Austrália}

{%%%%%   EGMO, priklad 5
Je daná rovina so špeciálnym bodom~$O$, ktorý nazveme \emph{počiatok}. Nech~$P$ je množina~2021 bodov v~tejto rovine takých, že
\item{(i)} žiadne tri body z~$P$ neležia na priamke;
\item{(ii)} žiadna dva body z~$P$ neležia na priamke prechádzajúcej počiatkom.

Trojuholník s~vrcholmi z~$P$ nazveme \emph{tučný} ak~$O$ leží vnútri neho. Nájdite maximálny počet tučných trojuholníkov.
}
\podpis{Rakúsko}

{%%%%%   EGMO, priklad 6
Rozhodnite, či existuje také nezáporné celé číslo~$a$, že rovnica
$$
\left\lfloor\frac{m}{1}\right\rfloor + \left\lfloor\frac{m}{2}\right\rfloor +
\left\lfloor\frac{m}{3}\right\rfloor + \cdots +
\left\lfloor\frac{m}{m}\right\rfloor = n^2 + a
$$
má viac ako milión rôznych riešení~$(m,n)$, kde~$m$ a~$n$ sú kladné celé čísla.
}
\podpis{Rakúsko}

{%%%%%   vyberko C, den 1, priklad 1
 Do tabuľky $2021 \times 2021$ je vpísaných $2021^2$ reálnych čísel tak, že súčet všetkých čísel v~tabuľke je nezáporný. Rozhodnite, či je vždy možné preusporiadať stĺpce tejto tabuľky tak, aby súčet čísel na hlavnej diagonále bol nezáporný. (Hlavná diagonála spája ľavý horný a~pravý dolný roh tabuľky.)
}
\podpis{...}

{%%%%%   vyberko C, den 1, priklad 2
Je daný rovnobežník $ABCD$. Nech $E$, $F$, $G$, $H$ sú postupne stredy jeho strán $AB$, $BC$, $CD$, $DA$. Priamky $BH$ a~$AC$ sa pretínajú v~bode~$I$, priamky $BD$ a~$EC$ v~bode~$J$, priamky $AC$ a~$DF$ v~bode~$K$, priamky $AG$ a~$BD$ v~bode~$L$. Dokážte, že štvoruholník $IJKL$ je rovnobežník.
}
\podpis{...}

{%%%%%   vyberko C, den 1, priklad 3
Nájdite všetky usporiadané dvojice kladných celých čísel $(a,b)$ takých, že $a-b$ je prvočíslo a~$ab$ je druhá mocnina celého čísla.}
\podpis{...}

{%%%%%   vyberko C, den 1, priklad 4
Predpokladajme, že štyri po dvoch rôzne reálne čísla $a$, $b$, $c$, $d$ spĺňajú
$$
(a^2+b^2-1)(a+b) = (b^2+c^2-1)(b+c) = (c^2+d^2-1)(c+d).
$$
Určte všetky možné hodnoty súčtu $a+b+c+d$ a~pre každú možnosť uveďte príklad štvorice vyhovujúcich čísel~$a$, $b$, $c$, $d$ s~týmto súčtom.
}
\podpis{...}

{%%%%%   vyberko C, den 1, priklad 5
Predpokladajme, že v~trojuholníku $ABC$ so stredom~$I$ kružnice vpísanej platí vzťah $|AC|+|AI|=|BC|$. Určte pomer $|\angle BAC| :|\angle CBA|$.
}
\podpis{...} 