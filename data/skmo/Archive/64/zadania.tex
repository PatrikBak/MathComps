{%%%%%   A-I-1
Dané je prirodzené číslo~$n$. Štvorec so stranou dĺžky $n$ je rozdelený na $n^2$ jednotkových štvorčekov. Za vzdialenosť dvoch štvorčekov považujeme vzdialenosť ich stredov. Určte počet dvojíc štvorčekov, ktorých vzdialenosť je $5$.}
\podpis{Jaroslav Zhouf}

{%%%%%   A-I-2
Daný je trojuholník $ABC$, v~ktorom je $BC$ najkratšia strana. Jej stred označme $M$. Na stranách $AB$ a~$AC$ určíme postupne body $X$ a~$Y$ tak, aby platilo $|BX|=|BC|=|CY|$. Priesečník priamok $CX$ a~$BY$ označme $Z$. Dokážte, že priamka~$ZM$ prechádza stredom kružnice pripísanej ku strane~$BC$ daného trojuholníka.}
\podpis{Michal Rolínek}

{%%%%%   A-I-3
Nájdite všetky celé čísla $k\ge2$, pre ktoré existuje $k$-prvková množina $\mm M$ celých kladných čísel taká, že súčin všetkých čísel z~$\mm M$ je deliteľný súčtom ľubovoľných dvoch (rôznych) čísel z~$\mm M$.}
\podpis{Jaromír Šimša}

{%%%%%   A-I-4
Predpokladajme, že pre reálne čísla $x$, $y$, $z$ platí
$$
15(x + y + z) = 12(xy + yz + zx) = 10(x^2 + y^2 + z^2)
$$
a~že aspoň jedno z~nich je rôzne od nuly.
\ite a) Dokážte rovnosť $x + y + z = 4$.
\ite b) Nájdite najmenší interval $\langle a,b\rangle$, v~ktorom ležia všetky tri čísla z~ľubovoľnej trojice $(x,y,z)$ vyhovujúcej predpokladom úlohy.\endgraf}
\podpis{Jaromír Šimša}

{%%%%%   A-I-5
V~danom trojuholníku $ABC$ označme $D$ bod dotyku kružnice vpísanej so stranou~$BC$. Kružnica vpísaná do trojuholníka $ABD$ sa dotýka strán $AB$ a~$BD$ v~bodoch $K$ a~$L$. Kružnica vpísaná do trojuholníka $ADC$ sa dotýka strán $DC$ a~$AC$ v~bodoch $M$ a~$N$. Dokážte,
že body $K$, $L$, $M$, $N$ ležia na jednej kružnici.}
\podpis{Josef Tkadlec}

{%%%%%   A-I-6
Nech $a$, $b$ sú dané navzájom nesúdeliteľné prirodzené čísla.
Postupnosť $(x_n)_{n=1}^{\infty}$ prirodzených
čísel je zostavená tak, že pre každé $n>1$ platí $x_n=ax_{n-1}+b$.
Dokážte, že v~ľubovoľnej takej postupnosti každý člen
$x_n$ s~indexom $n>1$ delí nekonečne veľa jej ďalších členov. Platí toto tvrdenie aj pre $n=1$?}
\podpis{Jaromír Šimša}

{%%%%%   B-I-1
V~obore reálnych čísel vyriešte sústavu rovníc
$$\eqalign{
|x-5|+|y-9|&=6,\cr
|x^2-9|+|y^2-5|&=52.
}$$}
\podpis{Pavel Calábek}

{%%%%%   B-I-2
Drak má $n$~hláv, po jednej na každom z~$n$~krkov usporiadaných do kruhu.
Rytier dokáže jedným úderom seknúť $k$~susedných krkov a~hlavy na
nich sťať. Ak drakovi po údere zostane aspoň jedna hlava, môže
si nechať niektorú z~chýbajúcich hláv dorásť. Dokážte, že ak pre
dané čísla $n$ a~$k$ môže rytier draka zbaviť všetkých hláv bez ohľadu na
to, ako mu dorastajú, dokáže to urobiť najviac tromi údermi.
}
\podpis{Ján Mazák}

{%%%%%   B-I-3
V~trojuholníku $ABC$ označme $U$ stred strany $AB$ a~$V$ stred
strany~$AC$. V~polrovine opačnej k~polrovine $BCA$ uvažujme ľubovoľný
rovnobežník $BCDE$. Označme~$X$ priesečník priamok $UD$ a~$VE$. Dokážte,
že priamka~$AX$ delí rovnobežník $BCDE$ na dve časti s~rovnakým obsahom.}
\podpis{Michal Rolínek}

{%%%%%   B-I-4
Nech $m$ je prirodzené číslo, ktoré má 7~kladných deliteľov, a~$n$ je
prirodzené číslo, ktoré má 9~kladných deliteľov. Koľko deliteľov môže mať
súčin $m\cdot n$?}
\podpis{Eva Patáková}

{%%%%%   B-I-5
Nech $S$ je stred prepony $AB$ pravouhlého trojuholníka $ABC$, ktorý nie je
rovnoramenný. Označme~$D$ pätu výšky z~vrcholu~$C$ a~$R$ priesečník osi
vnútorného uhla pri vrchole~$C$ s~preponou~$AB$. Určte veľkosti
vnútorných uhlov tohto trojuholníka, ak platí $|SR|=2|DR|$.}
\podpis{Jaroslav Švrček}

{%%%%%   B-I-6
Dokážte, že pre ľubovoľné kladné reálne čísla $a$, $b$, $c$ platí
$$
\frac{1}{a^2+ab+b^2}+\frac{1}{b^2+bc+c^2}+\frac{1}{c^2+ca+c^2}\le\frac1{a^2}+\frac1{b^2}+\frac1{c^2}.
$$
Určte, kedy nastáva rovnosť.}
\podpis{Jaroslav Švrček}

{%%%%%   C-I-1
Určte všetky dvojice $(x, y)$ reálnych čísel, ktoré vyhovujú
sústave rovníc
$$
\align
\sqrt{(x+4)^2}=&4-y,\\
\sqrt{(y-4)^2}=&x+8.
\endalign
$$}
\podpis{Jaroslav Švrček}

{%%%%%   C-I-2
Peter má zvláštne hodinky s~tromi ručičkami~-- prvá z~nich obehne
kruhový ciferník za minútu, druhá za 3~minúty a~tretia za 15~minút. Na
začiatku sú všetky ručičky v~rovnakej polohe. Určte, za ako dlho
budú ručičky rozdeľovať ciferník na tri zhodné časti. Nájdite všetky
riešenia.}
\podpis{Tomáš Jurík}

{%%%%%   C-I-3
Simona a~Lenka hrajú hru. Pre dané celé číslo~$k$ také, že $0 \le
k\le 64$, vyberie Simona $k$~políčok šachovnice $8\times 8$ a~každé
z~nich označí krížikom. Lenka potom šachovnicu nejakým spôsobom vyplní
tridsiatimi dvoma dominovými kockami. Ak je počet kociek pokrývajúcich dva
krížiky nepárny, vyhráva Lenka, inak vyhráva Simona. V~závislosti od~$k$
určte, ktoré z~dievčat má vyhrávajúcu stratégiu.}
\podpis{Michal Rolínek}

{%%%%%   C-I-4
Označme $E$ stred základne $AB$ lichobežníka $ABCD$, v~ktorom platí
$|AB|:|CD|={3:1}$. Uhlopriečka~$AC$ pretína úsečky $ED$, $BD$ postupne
v~bodoch $F$, $ G$. Určte postupný pomer
$$
|AF|:|FG|:|GC|.
$$}
\podpis{Jaroslav Zhouf}

{%%%%%   C-I-5
Rozdiel dvoch prirodzených čísel je $2\,010$ a~ich
najväčší spoločný deliteľ je $2\,014$-krát menší ako ich
najmenší spoločný násobok. Určte všetky také dvojice čísel.}
\podpis{Jaromír Šimša}

{%%%%%   C-I-6
Nájdite najmenšie prirodzené číslo $n$ také, že v~zápise iracionálneho
čísla $\sqrt n$ nasledujú bezprostredne za desatinnou čiarkou dve
deviatky.}
\podpis{Josef Tkadlec}

{%%%%%   A-S-1
Určte počet ciest dĺžky~$14$, ktoré vedú po hranách siete na \ifobrazkyvedla{}obrázku\else{}\obr{}\fi{}
z~bodu~$A$ do bodu~$B$. Dĺžka každej hrany je rovná $1$.
\ifobrazkyvedla{}\else\insp{a64s.1}\fi%
}
\podpis{Pavel Novotný}

{%%%%%   A-S-2
Daný je rovnobežník $ABCD$, pričom $|AB|=2|BC|$. Určte
všetky priamky, ktoré delia daný rovnobežník na dva dotyčnicové štvoruholníky.}
\podpis{Jaroslav Švrček}

{%%%%%   A-S-3
Určte všetky dvojice $(p,q)$ celých čísel takých, že $p$
je celočíselným násobkom čísla~$q$ a~kvadratická rovnica
$x^2+px+q=0$ má aspoň jeden celočíselný koreň.}
\podpis{Jaroslav Švrček}

{%%%%%   A-II-1
Daný je trojuholník $ABC$ s~tupým uhlom pri vrchole~$C$. Os $o_1$
úsečky~$AC$ pretína stranu~$AB$ v~bode~$K$, os $o_2$ úsečky~$BC$ pretína
stranu~$AB$ v~bode~$L$. Priesečník osí $o_1$ a~$o_2$ označme~$O$. Dokážte,
že stred kružnice vpísanej do trojuholníka $KLC$ leží na kružnici opísanej
trojuholníku $OKL$.}
\podpis{Radek Horenský}

{%%%%%   A-II-2
Nájdite všetky dvojice prvočísel $(p,q)$ také, že hodnota výrazu
$p^2+5pq+4q^2$ je druhou mocninou celého čísla.}
\podpis{Pavel Calábek}

{%%%%%   A-II-3
Pre kladné reálne čísla $a$, $b$, $c$ platí
$$
ab + bc + ca = 16,\quad a\ge3.
$$
Nájdite najmenšiu možnú hodnotu výrazu $2a + b + c$.}
\podpis{Michal Rolínek}

{%%%%%   A-II-4
Majme $n$ bodov v~rovine, $n\ge3$, pričom žiadne tri z~nich neležia na jednej priamke.
Uvažujme vnútorné uhly všetkých trojuholníkov s~vrcholmi
v~daných bodoch a~veľkosť najmenšieho z~nich označme~$\varphi$.
Pre dané~$n$ nájdite najväčšie možné~$\varphi$.}
\podpis{Stanislava Sojáková}

{%%%%%   A-III-1
Nájdite všetky štvorciferné čísla $n$ také, že súčasne platí:
\ite{i)} číslo $n$ je súčinom troch rôznych prvočísel;
\ite{ii)} súčet najmenších dvoch z~týchto prvočísel je rovný rozdielu
najväčších dvoch z~nich;
\ite{iii)} súčet všetkých troch prvočísel je rovný druhej mocnine iného
prvočísla.
}
\podpis{Radek Horenský}

{%%%%%   A-III-2
Pre dané prirodzené číslo~$n$
určte počet ciest dĺžky $2n+2$ z~bodu $[0,0]$ do bodu $[n,n]$, ktoré
žiadnym bodom neprechádzajú viackrát. Cestou dĺžky $2n+2$
z~bodu $[0,0]$ do bodu~$[n,n]$ rozumieme $(2n+2)$-člennú postupnosť
$$
\postdisplaypenalty=10000
(A_0A_1,A_1A_2,A_2A_3,\dots,A_{2n+1}A_{2n+2})
$$
úsečiek spájajúcich dva susedné mrežové body, pričom $A_0=[0,0]$,
$A_{2n+2}=[n,n]$.
\insp{a64iii.1}
}
\podpis{Pavel Novotný}

{%%%%%   A-III-3
V~ľubovoľnom trojuholníku $ABC$, v~ktorom ťažnica z~vrcholu~$C$ nie je kolmá na stranu~$CA$ ani na stranu~$CB$, označme $X$ a~$Y$ priesečníky osi tejto ťažnice s~priamkami $CA$ a~$CB$.
Nájdite všetky také trojuholníky $ABC$, pre ktoré body $A$, $B$, $X$, $Y$ ležia
na jednej kružnici.}
\podpis{Ján Mazák}

{%%%%%   A-III-4
V~obore reálnych čísel vyriešte sústavu rovníc
$$
\align
a(b^2 + c) &= c(c + ab),\\
b(c^2 + a) &= a(a + bc),\\
c(a^2 + b) &= b(b + ca).
\endalign
$$
}
\podpis{Michal Rolínek}

{%%%%%   A-III-5
Daný je trojuholník $ABC$, ktorého každé dve strany sa líšia aspoň o~dĺžku~$d>0$.
Označme~$T$ jeho ťažisko, $I$~stred kružnice vpísanej a~$\rho$ jej polomer.
Dokážte, že
$$
\postdisplaypenalty=10000
S_{AIT} + S_{BIT} + S_{CIT} \ge \frac23 d\rho,
$$
pričom $S_{XY\!Z}$ označuje obsah trojuholníka $XY\!Z$.
}
\podpis{Michal Rolínek}

{%%%%%   A-III-6
Dané je prirodzené číslo $n>2$. Určte najväčšie celé
číslo~$d$, pre ktoré platí nasledujúce tvrdenie: Z~ľubovoľnej
$n$-prvkovej množiny celých čísel možno vybrať tri rôzne neprázdne
podmnožiny tak, že súčet prvkov každej z~nich je celočíselným násobkom
čísla~$d$. (Vybrané podmnožiny môžu mať spoločné prvky.)
}
\podpis{Jaromír Šimša}

{%%%%%   B-S-1
Predpokladajme, že prirodzené číslo~$a$ má 15~kladných deliteľov.
Koľko ich môže mať prirodzené číslo~$b$,
ak najmenší spoločný násobok čísel $a$ a~$b$ má 20~kladných deliteľov?}
\podpis{Jaromír Šimša}

{%%%%%   B-S-2
Označme $P$ priesečník uhlopriečok konvexného štvoruholníka $ABCD$.
Vypočítajte jeho obsah, ak obsahy trojuholníkov $ABC$, $BCD$ a~$DAP$ sú
postupne $8\cm^2$, $9\cm^2$, $10\cm^2$.}
\podpis{Pavel Novotný}

{%%%%%   B-S-3
Dokážte, že pre ľubovoľné kladné reálne čísla $a$, $b$, $c$ platí
$$
\frac{ab}{a^2-ab+b^2}+\frac{bc}{b^2-bc+c^2}+\frac{ca}{c^2-ca+a^2}
\le3.
$$
Určte, kedy nastáva rovnosť.}
\podpis{Jaromír Šimša}

{%%%%% B-II-1
Súčin všetkých kladných deliteľov prirodzeného čísla $n$ je $20^{15}$. Určte $n$.}
\podpis{Pavel Calábek}

{%%%%% B-II-2
Určte najmenšiu hodnotu výrazu
$$
V=x^2+\frac2{1+2x^2},
$$
pričom $x$ je ľubovoľné reálne číslo. Pre ktoré $x$ výraz~$V$ túto hodnotu nadobúda?}
\podpis{Jaroslav Švrček}

{%%%%% B-II-3
Dokážte, že priesečník výšok a~ťažisko daného ostrouhlého trojuholníka
$ABC$ majú rovnakú vzdialenosť od strany~$AB$ práve vtedy, keď pre
vnútorné uhly $\alpha$, $\beta$ pri vrcholoch $A$, $B$ platí rovnosť
$\tg\alpha\cdot\tg\beta=3$.}
\podpis{Jaromír Šimša}

{%%%%% B-II-4
Na tabuli je zoznam čísel $1,2,3,4,5,6$ a~"rovnica"
$$
\frac\square\square x^2+\frac\square\square x+\frac\square\square=0.
$$
Marek s~Tomášom hrajú nasledujúcu hru. Najskôr Marek vyberie
ľubovoľné číslo zo zoznamu, napíše ho do jedného z~prázdnych políčok
v~"rovnici" a~číslo zo zoznamu zotrie. Potom Tomáš vyberie niektoré zo
zvyšných čísel, napíše ho do iného prázdneho políčka a~v~zozname
ho zotrie. Nato Marek urobí to isté a~nakoniec Tomáš doplní tri zvyšné
čísla na tri zvyšné voľné políčka v~"rovnici". Marek vyhrá,
ak vzniknutá kvadratická rovnica s~racionálnymi koeficientmi bude
mať dva rôzne reálne korene, inak vyhrá Tomáš. Rozhodnite, ktorý
z~hráčov môže vyhrať nezávisle na postupe druhého hráča.}
\podpis{Pavel Calábek}

{%%%%%   C-S-1
V obore reálnych čísel vyriešte sústavu rovníc
$$
\align
|{1-x}|&=y+1, \\
|{1+y}|&=z-2, \\
|{2-z}|&=x-x^2.
\endalign
$$}
\podpis{Jaroslav Švrček}

{%%%%%   C-S-2
Označme $K$ a~$L$ postupne body strán $BC$ a $AC$ trojuholníka $ABC$,
pre ktoré platí $|BK|=\frac13|BC|$, $|AL|=\frac13|AC|$. Nech $M$ je
priesečník úsečiek $AK$ a~$BL$. Vypočítajte pomer obsahov trojuholníkov
$ABM$ a~$ABC$.}
\podpis{Pavel Novotný}

{%%%%%   C-S-3
Nájdite najmenšie prirodzené číslo $n$ s~ciferným súčtom~8, ktoré sa rovná
súčinu troch rôznych prvočísel, pričom rozdiel dvoch najmenších z~nich je~8.}
\podpis{Tomáš Jurík}

{%%%%% C-II-1
Celé čísla od 1 do 9 rozdelíme ľubovoľne na tri skupiny po troch a~potom čísla
v~každej skupine medzi sebou vynásobíme.
\ite a) Určte tieto tri súčiny, ak viete, že dva z~nich sa rovnajú a~sú
menšie ako tretí súčin.
\ite b) Predpokladajme, že jeden z~troch súčinov, ktorý označíme~$S$, je menší ako
dva ostatné súčiny (ktoré môžu byť rovnaké). Nájdite najväčšiu možnú
hodnotu~$S$.\endgraf}
\podpis{Jaromír Šimša}

{%%%%% C-II-2
V~jednom políčku šachovnice $8\times8$ je napísané "$\m$" a~v~ostatných políčkach~"$\p$".
V~jednom kroku môžeme zmeniť na opačné súčasne všetky štyri znamienka v~ktoromkoľvek štvorci~
$2\times2$ na šachovnici. Rozhodnite, či po určitom počte krokov môže
byť na šachovnici oboch znamienok rovnaký počet.}
\podpis{Jaromír Šimša}

{%%%%% C-II-3
Daný je lichobežník $ABCD$ so základňami $AB$, $CD$, pričom $2|AB|=3|CD|$.
\ite a) Nájdite bod~$P$ vnútri lichobežníka tak, aby obsahy trojuholníkov $ABP$
a~$CDP$ boli v~pomere $3:1$ a~aj obsahy trojuholníkov $BCP$ a~$DAP$ boli v~pomere $3:1$.
\ite b) Pre nájdený bod~$P$ určte postupný pomer obsahov trojuholníkov $ABP$, $BCP$, $CDP$ a~$DAP$.\endgraf}
\podpis{Jaroslav Zhouf}

{%%%%% C-II-4
Hovoríme, že kladné reálne číslo je {\it copaté}, ak nie je prirodzené
a~vo svojom dekadickom zápise obsahuje za desatinnou čiarkou iba konečne veľa
nenulových cifier.
\ite a) Nájdite dve copaté čísla $a$, $b$ také, že $a\cdot b =2\,015$.
\ite b) Rozhodnite, či existujú tri copaté čísla $a$, $b$, $c$ také,
že čísla $a\cdot b$, $b\cdot c$ a~$c\cdot a$ sú všetky prirodzené.\endgraf}
\podpis{Josef Tkadlec}

{%%%%%   vyberko, den 1, priklad 1
Nájdite všetky celočíselné riešenia rovnice
$$
(m^2-n^2)^2=16n+1.
$$}
\podpis{Dominik Csiba, Michal Kopf:2007 Croatia TST 1.1}

{%%%%%   vyberko, den 1, priklad 2
Nájdite všetky funkcie $f\colon \Bbb R\to\Bbb R$ také, že pre všetky $x,y\in\Bbb R$ platí
$$
f\left(f(x)^2 + f(y)\right) = xf(x) + y.
$$}
\podpis{Dominik Csiba, Michal Kopf:2012 Kyrgyzstan NO 4}

{%%%%%   vyberko, den 1, priklad 3
Na tabuľke veľkosti $1\times n$ ($n\ge2$) sa dvaja hráči striedajú v~ťahoch, v~ktorých vpisujú krížiky a~krúžky do prázdnych políčok tabuľky. Prvý hráč vpisuje krížiky, druhý krúžky. Nie je dovolené, aby sa v~tabuľke nachádzali dva rovnaké znaky na susedných políčkach. Hráč, ktorý už nemá ťah, prehráva. Ktorý z~hráčov má víťaznú stratégiu?}
\podpis{Dominik Csiba, Michal Kopf:2009 Bosnia and Herzegovina TST 2.1}

{%%%%%   vyberko, den 1, priklad 4
Na strane~$BC$ trojuholníka $ABC$ leží bod~$M$ taký, že ťažisko trojuholníka
$ABM$ leží na kružnici opísanej trojuholníku $ACM$ a zároveň ťažisko trojuholníka $ACM$ leží na kružnici opísanej trojuholníku $ABM$. Dokážte, že
ťažnice trojuholníkov $ABM$ a~$ACM$ z~bodu~$M$ sú rovnako dlhé.}
\podpis{Dominik Csiba, Michal Kopf:2008 Croatia TST 3}

{%%%%%   vyberko, den 2, priklad 1
Nech $n$, $k$ sú dané prirodzené čísla. Na stole je $2n$ listov papiera a~na každom je napísané číslo~$1$. V~jednom kroku vyberieme dva papiere a~ak na nich sú čísla $a$, $b$, tak na oba z~nich napíšeme $a + b$. Dokážte, že po $nk$ krokoch bude súčet čísel na všetkých papieroch aspoň $n\cdot2^{k+1}$.}
\podpis{Martin Vodička:Shortlist 2014, C2}

{%%%%%   vyberko, den 2, priklad 2
Je daný ostrouhlý trojuholník $ABC$, pričom $|AC|>|AB|$. Stred jeho opísanej kružnice označme~$O$. Os uhla $BAC$ pretína opísanú kružnicu tomuto trojuholníku v~bode $M\ne A$. Nech $\Gamma$ je kružnica s~priemerom~$AM$. Osi uhlov $AOB$ a~$AOC$ pretínajú $\Gamma$ v~bodoch $P$ a~$Q$. Na priamke~$PQ$ je zvolený bod~$R$ tak, že $|MR| = |AR|$.
Dokážte, že $AR\parallel BC$. (Os uhla chápeme ako polpriamku.)}
\podpis{Martin Vodička:Shortlist 2014, G3}

{%%%%%   vyberko, den 2, priklad 3
Nech $a_1<a_2<\dots <a_n$ sú po dvoch nesúdeliteľné prirodzené čísla také, že $a_1$ je prvočíslo a~$a_1\ge n+2$. Na reálnej osi na intervale $\mm I=\langle0,a_1a_2\dots a_n\rangle$ vyznačíme všetky celé čísla, ktoré sú deliteľné aspoň jedným z~čísel $a_1,a_2,\dots, a_n$. Tieto body rozdelia~$\mm I$ na niekoľko menších úsečiek. Dokážte, že ak sčítame druhé mocniny ich dĺžok, tak dostaneme číslo deliteľné $a_1$.}
\podpis{Martin Vodička:Shortlist 2014, N6}

{%%%%%   vyberko, den 3, priklad 1
Majme pravouhlý trojuholník $ABC$ s~pravým uhlom pri vrchole~$B$. Nech $BD$ je výška z~vrcholu~$B$ na stranu~$AC$ (bod~$D$ leží na $AC$). Ďalej označme $P$, $Q$ a~$I$ postupne stredy kružníc vpísaných trojuholníkom $ABD$, $CBD$ a~$ABC$. Ukážte, že stred kružnice opísanej trojuholníku $PIQ$ leží na priamke~$AC$.}
\podpis{Filip Hanzely, Miroslav Psota:India 2015, national round}

{%%%%%   vyberko, den 3, priklad 2
Nájdite všetky funkcie $f\colon\Bbb{R}\setminus\{0,1\} \to \Bbb{R}$ také, že pre všetky $x$ z~definičného oboru platí
$$
f(x)+f\left(\frac{1}{1-x}\right) = 1+\frac{1}{x(1-x)}.
$$
}
\podpis{Filip Hanzely, Miroslav Psota:Flanders MO, 2006}

{%%%%%   vyberko, den 3, priklad 3
Nech $a$, $b$, $c$ sú prirodzené čísla. Dokážte, že ak
$$
(abc)^n \mid \bigl((a^n-1)(b^n-1)(c^n-1)+1\bigr)^3
$$
pre každé prirodzené číslo~$n$, tak potom nutne $a=b=c$.}
\podpis{Filip Hanzely, Miroslav Psota:Kazachstan 2013, national round grade 10}

{%%%%%   vyberko, den 3, priklad 4
Mucha a~$k$ pavúkov stoja v~nejakých mrežových bodoch mriežky $2015 \times 2015$. Mucha a~pavúky sa hýbu v~ťahoch: najprv ide mucha, a~buď ostane na svojom mieste, alebo sa pohne do susedného mrežového bodu. Potom idú naraz všetky pavúky, a~každý z~nich môže ostať na svojom mieste alebo sa pohnúť do susedného bodu (v~jednom bode môže byť aj viac pavúkov). Pavúky a~mucha poznajú vždy pozície všetkých ostatných.
\ite a) Nájdite najmenšie $k$ také, že $k$~pavúkov dokáže vždy chytiť v~konečnom čase muchu, bez ohľadu na ich počiatočné rozmiestnenie.
\ite b) Odpovedzte na rovnakú otázku pre trojrozmernú mriežku $2015\times 2015\times 2015$.\endgraf
\noindent
(Mrežové body sú susedné práve vtedy, keď majú práve jednu rôznu súradnicu, a~rozdiel v~danej súradnici je~$1$. Pavúky chytia muchu, ak je aspoň jeden z~nich v~rovnakom bode ako mucha.)}
\podpis{Filip Hanzely, Miroslav Psota:Srbsko 2012, national round}

{%%%%%   vyberko, den 4, priklad 1
Slovom budeme označovať konečnú postupnosť písmen z~nejakej abecedy. Slovo nazveme periodické, ak sa dá rozložiť na aspoň dve rovnaké podslová (napr. $ababab$ a~$abcabc$ sú periodické, zatiaľ čo $ababa$ a~$aabb$ nie sú). Dokážte, že ak slovo má takú vlastnosť, že po prehodení akýchkoľvek dvoch susedných písmen je periodické, tak potom toto slovo má všetky písmená rovnaké.}
\podpis{Viktor Lukáček:}

{%%%%%   vyberko, den 4, priklad 2
Zistite, či existuje nekonečná postupnosť $x_1,x_2,\dots$ prirodzených čísel, ktorá obsahuje práve $10^{2016}$ rôznych čísel (napr. postupnosť $1, 2, 3, 1, 2, 3, 1, \dots$ obsahuje tri rôzne čísla), pričom pre všetky $n\in\Bbb N$ platí
$$
x_{n+2} = \nsd(x_n,x_{n+1}) + 2016.
$$
}
\podpis{Viktor Lukáček:}

{%%%%%   vyberko, den 4, priklad 3
Nech $a$, $b$, $c$ sú kladné reálne čísla. Dokážte nerovnosť
$$
\frac{(b+c-a)^2}{(b+c)^2+a^2} + \frac{(c+a-b)^2}{(c+a)^2+b^2} + \frac{(a+b-c)^2}{(a+b)^2+c^2}\ge\frac35.
$$}
\podpis{Viktor Lukáček:}

{%%%%%   vyberko, den 4, priklad 4
Nech $ABC$ je trojuholník so stredom opísanej kružnice v~bode~$O$ a~nech $D$ je priesečník osi uhla $BAC$ a~úsečky~$BC$. Ďalej nech $M$ je taký bod, že $MC\perp BC$ a~$MA\perp AD$ a~nech priamky $BM$ a~$OA$ sa pretínajú v~bode~$P$. Dokážte, že priamka~$BC$ je dotyčnicou ku kružnici, ktorá má stred v~bode~$P$ a~prechádza bodom~$A$.}
\podpis{Viktor Lukáček:}

{%%%%%   vyberko, den 5, priklad 1
Nájdite všetky dvojice $(x,y)$ kladných celých čísel, ktoré sú riešením rovnice
$$
\root3\of{7x^2-13xy+7y^2}=|x-y|+1.
$$
}
\podpis{Peter Novotný:Shortlist 2014, N2}

{%%%%%   vyberko, den 5, priklad 2
Pre danú $n$-ticu reálnych čísel $(x_1,x_2,\dots,x_n)$ definujeme jej {\it cenu\/} ako maximum z~čísel
$$
|x_1|,\quad |x_1+x_2|,\quad |x_1+x_2+x_3|,\quad\dots,\quad|x_1+x_2+\dots+x_n|.
$$
Daných je $n$ čísel (nie nutne rôznych). Dorota a~Lucia ich chcú usporiadať do $n$-tice s~nízkou cenou. Dôsledná Dorota vyskúša všetky možné $n$-tice a~nájde najmenšiu možnú cenu~$D$. Lenivá Lucia zvolí $x_1$ tak, že hodnota $|x_1|$ je najmenšia možná; zo zvyšných čísel zvolí $x_2$ tak, že hodnota $|x_1+x_2|$ je najmenšia možná, atď. Inými slovami, v~$i$-tom kroku Lucia spomedzi zvyšných čísel vyberie $x_i$ tak, aby hodnota $|x_1+x_2+\dots+x_i|$ bola najmenšia možná. Ak má v~niektorom kroku na výber viac čísel dávajúcich najmenšiu hodnotu, zvolí jedno z~nich náhodne. Takto napokon dostane $n$-ticu s~cenou~$L$. Nájdite najmenšiu možnú konštantu~$c$ takú, že pre každé prirodzené číslo~$n$, pre každých $n$ reálnych čísel a~pre každú možnú $n$-ticu, ktorú Lucia vie dostať, platí $L\le cD$.
}
\podpis{Peter Novotný:Shortlist 2014, A3}

{%%%%%   vyberko, den 5, priklad 3
Daný je trojuholník $ABC$. Vnútri strán $BC$, $CA$ a~$AB$ sú postupne zvolené body $K$, $L$ a~$M$, pričom priamky $AK$, $BL$ a~$CM$ sa pretínajú v~jednom bode. Dokážte, že spomedzi trojuholníkov $ALM$, $BMK$ a~$CKL$ vieme vždy vybrať dva tak, že súčet polomerov im vpísaných kružníc je väčší alebo rovný polomeru kružnice vpísanej trojuholníku $ABC$.
}
\podpis{Peter Novotný:Shortlist 2014, G2}

{%%%%%   vyberko, den x, priklad x
...}
\podpis{...}

{%%%%%   vyberko, den x, priklad x
...}
\podpis{...}

{%%%%%   trojstretnutie, priklad 1
Na kružnici s~polomerom $r$ ležia rôzne body $A$, $B$, $C$, $D$, $E$ v~tomto poradí, pričom $|AB|=|CD|=|DE|>r$. Dokážte, že trojuholník, ktorého vrcholy sú ťažiská trojuholníkov $ABD$, $BCD$ a~$ADE$, je tupouhlý.}
\podpis{Tomáš Jurík}

{%%%%%   trojstretnutie, priklad 2
Systém množín $\Cal F$ sa nazýva {\it skvelý\/}, ak je splnená nasledujúca podmienka: Pre každú trojicu množín $X_1,X_2,X_3\in\Cal F$ je aspoň jedna z~množín
$$
(X_1\setminus X_2) \cap X_3,\qquad (X_2\setminus X_1)\cap X_3
$$
prázdna. Dokážte, že ak $\Cal F$ je skvelý systém pozostávajúci z~niektorých podmnožín danej konečnej množiny~$U$, tak ${|\Cal F|\le|U|+1}$.
}
\podpis{Micha\l{} Pilipczuk}

{%%%%%   trojstretnutie, priklad 3
Reálne čísla $x$, $y$, $z$ spĺňajú rovnicu
$$
\frac{1}{x}+\frac{1}{y}+\frac{1}{z}+x+y+z=0
$$
a~žiadne z~nich neleží v~otvorenom intervale $(\m1,1)$. Nájdite najväčšiu možnú hodnotu súčtu $x+y+z$.}
\podpis{Jaromír Šimša}

{%%%%%   trojstretnutie, priklad 4
Podivná kalkulačka má len dve tlačidlá, na každom je napísané dvojciferné pri\-ro\-dze\-né číslo. Na začiatku je na displeji číslo $1$. Vždy, keď stlačíme tlačidlo s~číslom~$N$, kalkulačka zmení zobrazené číslo $X$ na číslo $X\cdot N$ alebo $X+N$. Pritom násobenie a~sčítanie sa strieda, začína sa násobením. (Napríklad ak na 1. tlačidle je číslo $10$ a~na 2. tlačidle je číslo $20$ a~stlačíme postupne 1., 2., 1. a 1. tlačidlo, dostaneme postupne výsledky $1\cdot10=10$, $10+20=30$, $30\cdot10=300$, $300+10=310$.) Rozhodnite, či existujú konkrétne hodnoty dvojciferných čísel na tlačidlách také, že dokážeme zobraziť nekonečne veľa čísel končiacich štvorčíslím
\item{a)} 2015,
\item{b)} 5813.}
\podpis{Michal Rolínek, Peter Novotný}

{%%%%%   trojstretnutie, priklad 5
Daný je ostrouhlý trojuholník $ABC$, ktorý nie je rovnostranný. Označme $O$ stred jeho opísanej kružnice a~$H$ jeho priesečník výšok. Kružnica~$k$ prechádza bodom~$B$ a~dotýka sa priamky~$AC$ v~bode~$A$. Kružnica~$l$ má stred na polpriamke~$BH$ a~dotýka sa priamky~$AB$ v~bode~$A$. Kružnice $k$ a~$l$ sa pretínajú v~bode~$X$ ($X\ne A$). Dokážte, že $|\uhol HXO|=180{^\circ}-|\uhol BAC|$.}
\podpis{Josef Tkadlec}

{%%%%%   trojstretnutie, priklad 6
Nech $n$ je dané párne prirodzené číslo. Na tabuli je napísaných $n$ reálnych čísel. V~jednom kroku zvolíme ľubovoľné dve čísla, zotrieme ich a {\it každé\/} z~nich nahradíme ich súčinom. Dokážte, že bez ohľadu na to, s~akou $n$-ticou začneme, je možné dostať po konečnom počte krokov na tabuli $n$ rovnakých čísel.}
\podpis{Peter Novotný}

{%%%%%   IMO, priklad 1
Konečnú množinu $\mn S$ pozostávajúcu z~bodov roviny nazývame {\it vyvážená}, ak pre ľubovoľné dva rôzne body $A$, $B$ z~množiny~$\mn S$ existuje v~$\mn S$ taký bod~$C$, že $|AC|=|BC|$.
Množinu $\mn S$ nazývame {\it bezstredová}, ak pre žiadne tri rôzne body $A$, $B$, $C$ z~množiny $\mn S$ neexistuje v~$\mn S$ taký bod~$P$, že $|PA|=|PB|=|PC|$.
\item{a)} Dokážte, že pre každé prirodzené číslo $n\ge3$ existuje vyvážená množina obsahujúca práve~$n$~bodov.
\item{b)} Určte všetky prirodzené čísla $n\ge3$, pre ktoré existuje vyvážená bezstredová množina obsahujúca práve $n$ bodov.\endgraf
 }
\podpis{Holandsko}

{%%%%%   IMO, priklad 2
Určte všetky trojice $(a,b,c)$ kladných celých čísel, pre ktoré je každé z~čísel
$$
ab-c, \quad bc-a, \quad ca-b
$$
mocninou čísla $2$.}
\podpis{Srbsko}

{%%%%%   IMO, priklad 3
Daný je ostrouhlý trojuholník $ABC$, pričom $|AB|>|AC|$. Označme $\Gamma$ jeho opísanú kružnicu,
$H$ priesečník výšok a~$F$ pätu výšky z~vrcholu~$A$. Stred strany~$BC$ označme~$M$.
Nech $Q$ je taký bod kružnice $\Gamma$, že $|\angle HQA|=90^\circ$. Ďalej nech $K$ je taký bod kružnice $\Gamma$, že $|\angle HKQ|=90^\circ$. Predpokladajme, že body $A$, $B$, $C$, $K$ a $Q$ sú všetky rôzne a ležia na kružnici~$\Gamma$ v~tomto poradí.
Dokážte, že kružnice opísané trojuholníkom $KQH$ a~$FKM$ sa navzájom dotýkajú.}
\podpis{Ukrajina}

{%%%%%   IMO, priklad 4
Trojuholník $ABC$ má opísanú kružnicu~$\Omega$, ktorej stred označme~$O$. Nech kružnica~$\Gamma$ so stredom~$A$ pretína úsečku~$BC$ v~bodoch $D$ a~$E$, pričom body $B$, $D$, $E$, $C$ sú všetky rôzne a~ležia na priamke~$BC$ v~tomto poradí. Kružnice $\Gamma$ a~$\Omega$ sa pretínajú v~bodoch $F$ a~$G$, pričom body $A$,~$F$,~$B$,~$C$,~$G$ ležia na kružnici $\Omega$ v~tomto poradí.
Označme $K$ ďalší priesečník kružnice opísanej trojuholníku $BDF$ s~úsečkou~$AB$.
Podobne označme $L$ ďalší priesečník kružnice opísanej trojuholníku $CGE$ s~úsečkou~$CA$.
Predpokladajme, že priamky $FK$ a~$GL$ sú rôzne a~pretínajú sa v~bode~$X$. Dokážte, že $X$ leží na priamke~$AO$.}
\podpis{Grécko}

{%%%%%   IMO, priklad 5
Označme $\Bbb R$ množinu všetkých reálnych čísel. Určte všetky funkcie $f\colon\Bbb R\to\Bbb R$ také, že rovnosť
$$
f\bigl(x+f(x+y)\bigr)+f(xy) = x+f(x+y)+yf(x)
$$
platí pre všetky reálne čísla $x$, $y$.}
\podpis{Albánsko}

{%%%%%   IMO, priklad 6
Postupnosť $a_1, a_2, \dots$ celých čísel spĺňa nasledujúce podmienky:
\item{i)}
$1\le a_j\le 2015$ pre všetky $j\ge 1$;
\item{ii)}
$k+a_k\ne l+a_l$ pre všetky $1\le k<l$.
\endgraf\noindent
Dokážte, že existujú kladné celé čísla $b$ a~$N$ také, že nerovnosť
$$
\left|\sum_{j=m+1}^{n}(a_j-b)\right|\le 1007^2
$$
platí pre všetky celé čísla $m$, $n$ spĺňajúce $n> m\ge N$.}
\podpis{Austrália}

{%%%%%   MEMO, priklad 1
Nájdite všetky surjektívne funkcie $f\colon\Bbb N\to\Bbb N$ také, že pre všetky kladné celé čísla $a$, $b$ platí práve jedna z~rovností
$$
\align
f(a) & = f(b), \\
f(a + b) & = \min\{ f(a) , f(b)\}.
\endalign
$$
\poznamka
Symbolom $\Bbb N$ označujeme množinu všetkých kladných celých čísel. Funkcia $f \colon X \to Y$ sa nazýva surjektívna, ak pre každé $y \in Y$ existuje $x \in X$ také, že $f(x)=y$.
}
\podpis{Švajčiarsko}

{%%%%%   MEMO, priklad 2
Nech $n\ge 3$ je celé číslo. {\it Vnútorná uhlopriečka\/} jednoduchého $n$-uholníka je uhlopriečka, ktorá je celá vnútri tohto $n$-uholníka. Označme $D(\Cal P)$ počet všetkých vnútorných uhlopriečok jednoduchého $n$-uholníka $\Cal P$ a $D(n)$ najmenšiu možnú hodnotu $D(\Cal Q)$, kde $\Cal Q$ je jednoduchý $n$-uholník. Dokážte, že žiadne dve vnútorné uhlopriečky v~$\Cal P$ sa nepretínajú (okrem možných spoločných koncov) práve vtedy, keď $D(\Cal P)=D(n)$.

\poznamka
Jednoduchý $n$-uholník je sám seba nepretínajúci mnohouholník s~$n$ vrcholmi. Mnohouholník nie je nutne konvexný.
}
\podpis{Slovinsko}

{%%%%%   MEMO, priklad 3
Nech $ABCD$ je tetivový štvoruholník. Označme $E$ priesečník priamok rovnobežných s~$AC$ a~$BD$ prechádzajúcich bodmi $B$ a~$A$. Priamky $EC$ a~$ED$ pretínajú kružnicu opísanú trojuholníku $AEB$ znova v~bodoch $F$ a~$G$. Dokážte, že body $C$, $D$, $F$ a~$G$ ležia na jednej kružnici.}
\podpis{Slovensko, Patrik Bak}

{%%%%%   MEMO, priklad 4
Nájdite všetky dvojice kladných celých čísel $(m, n)$, pre ktoré existujú nesúdeliteľné celé čísla $a$,~$b$ väčšie ako jedna také, že
$$
\frac{a^m + b^m}{a^n + b^n}
$$
je celé číslo.}
\podpis{Chorvátsko}

{%%%%%   MEMO, priklad t1
Dokážte, že pre všetky kladné reálne čísla $a$, $b$, $c$ také, že $abc = 1$, platí nerovnosť
$$
\frac{a}{2b + c^2} + \frac{b}{2c + a^2} + \frac{c}{2a + b^2} \le \frac{a^2 + b^2 + c^2}{3}.
$$}
\podpis{Chorvátsko}

{%%%%%   MEMO, priklad t2
Nájdite všetky funkcie $f\colon\Bbb R\setminus\{0\}\to\Bbb R\setminus\{0\}$ také, že
$$
f(x^2yf(x)) + f(1)= x^2f(x)+f(y)
$$
platí pre všetky nenulové reálne čísla $x$ a~$y$.}
\podpis{Slovensko, Patrik Bak}

{%%%%%   MEMO, priklad t3
V~rade stojí $n$~žiakov na pozíciách $1$ až $n$. Kým sa učiteľ pozerá preč, niektorí žiaci zmenia svoje pozície. Keď sa učiteľ pozrie späť, žiaci stoja znova v~rade. Ak žiak, ktorý stál pôvodne na pozícii~$i$, je teraz na pozícii~$j$, hovoríme, že sa posunul o~$|i-j|$ krokov. Určte najväčší súčet krokov, o~ktorý sa mohli posunúť všetci žiaci.}
\podpis{Slovinsko}

{%%%%%   MEMO, priklad t4
Nech $N$ je kladné celé číslo. V každom z $N^2$ štvorčekov tabuľky $N\times N$ je nakreslená jedna uhlopriečka. Nakreslené uhlopriečky rozdelia tabuľku  $N\times N$ na $K$ oblastí. Pre každé $N$ určte najmenšiu a najväčšiu možnú hodnotu $K$. (Na \obr{} je príklad pre $N = 3$ a~$K = 7$.)
\insp{memo.1}%
}
\podpis{Chorvátsko}

{%%%%%   MEMO, priklad t5
Nech $ABC$ je ostrouhlý trojuholník, pričom $|AB| > |AC|$. Dokážte, že existuje bod~$D$ s~nasledujúcou vlastnosťou: Ak $X$ a~$Y$ sú také dva rôzne body vnútri trojuholníka $ABC$, že body $B$, $C$, $X$ a~$Y$ ležia na jednej kružnici a~platí
$$
|\uhol AXB| -|\uhol ACB| =|\uhol CYA| - |\uhol CBA|,
$$
tak priamka~$XY$ prechádza bodom~$D$.}
\podpis{Slovensko, Patrik Bak}

{%%%%%   MEMO, priklad t6
Nech $I$ je stred vpísanej kružnice trojuholníka $ABC$, pričom $|AB| > |AC|$. Označme $D$ priesečník priamky~$AI$ so stranou~$BC$. Predpokladajme, že bod~$P$ leží na úsečke~$BC$ a~platí $|PI|=|PD|$. Označme $J$ obraz bodu~$I$ v~osovej súmernosti podľa osi strany $BC$. Nech $Q$ je druhý priesečník kružníc opísaných trojuholníkom $ABC$ a~$APD$. Dokážte, že $|\uhol BAQ|=|\uhol CAJ|$.}
\podpis{Slovensko, Patrik Bak}

{%%%%%   MEMO, priklad t7
Nájdite všetky usporiadané dvojice kladných celých čísel $(a,b)$ takých, že platí
$$
a! + b! = a^b + b^a.
$$}
\podpis{Švajčiarsko}

{%%%%%   MEMO, priklad t8
Nech $n\ge 2$ je celé číslo. Určte počet kladných celých čísel $m$ takých, že $m\le n$ a~$ m^2 + 1$ je deliteľné číslom  $n$.}
\podpis{Chorvátsko}

{%%%%%   CPSJ, priklad 1
V~pravouhlom trojuholníku $ABC$ s~kratšou odvesnou~$AC$ má prepona~$AB$ dĺžku $12$. Označme $T$ jeho ťažisko a~$D$ pätu výšky
z~vrcholu~$C$. Určte veľkosť jeho vnútorného uhla pri vrchole~$B$, pre ktorú má trojuholník $DTC$ najväčší možný obsah.}
\podpis{Ján Mazák}

{%%%%%   CPSJ, priklad 2
Rozhodnite, či sa dajú každému vrcholu pravidelného 30-uholníka priradiť po jednom čísla $1,2,\dots ,30$ tak, aby súčet čísel
priradených ľubovoľným dvom susedným vrcholom bol druhou mocninou niektorého prirodzeného čísla.}
\podpis{\L{}ukasz Bożyk}

{%%%%%   CPSJ, priklad 3
Pre reálne čísla $x$, $y$ platí $x^2+y^2\le 2$. Dokážte, že tieto čísla spĺňajú nerovnosť
$$
xy+3\ge 2x+2y.
$$}
\podpis{Peter Novotný}

{%%%%%   CPSJ, priklad 4
Nech $E$, $F$ sú postupne stredy odvesien $BC$, $AC$ pravouhlého trojuholníka $ABC$ a~$D$ je päta jeho výšky z~vrcholu~$C$.
Ďalej nech $P$ označuje priesečník osi jeho vnútorného uhla pri vrchole~$A$ a~priamky~$EF$. Dokážte, že $P$ je stredom kružnice vpísanej
trojuholníku $CDE$.}
\podpis{Jaroslav Švrček}

{%%%%%   CPSJ, priklad 5
Určte všetky prirodzené čísla $n>1$ s~vlastnosťou: Pre každé $d>1$, ktoré je deliteľom čísla $n$, je $d-1$
deliteľom $n-1$.}
\podpis{\L{}ukasz Bożyk}

{%%%%%   CPSJ, priklad t1
Nech $I$ je stred kružnice vpísanej trojuholníku $ABC$ a~$M$ je stred jeho strany~$BC$. Ak platí $|AI|=|MI|$, tak
medzi stranami trojuholníka $ABC$ existujú dve, z~ktorých jedna má dvojnásobnú dĺžku ako druhá. Dokážte.
}
\podpis{Tomasz Cieśla}

{%%%%%   CPSJ, priklad t2
Zo šachovnice s~rozmermi $8\times8$ sme odstránili prostredný štvorec s~rozmermi $2\times2$.
\ite a) Koľko najviac dám možno umiestniť na zvyšných 60 políčok tak, aby sa žiadne dve neohrozovali?
\ite b) Koľko najmenej dám sa dá umiestniť na šachovnicu tak, aby nimi bolo ohrozených všetkých 60 políčok?
\endgraf\noindent
(Dáma ohrozuje políčko, na ktorom stojí a~tiež každé políčko, na ktoré sa vie dostať jedným ťahom bez toho, aby prešla ponad niektoré zo štyroch odstránených políčok.)
}
\podpis{Peter Novotný}

{%%%%%   CPSJ, priklad t3
Różne punkty $A$ i $D$ leż\ą{} po tej samej stronie prostej $BC$, przy czym $|AB| = |BC| = |CD|$ oraz proste $AD$ i $BC$ s\ą{} prostopad\l{}e. Niech $E$ b\ę{}dzie punktem przeci\ę{}cia prostych $AD$ i $BC$. Udowodnij, że
$$
\bigl| |BE| - |CE|\bigr| < |AD|\sqrt3.
\postdisplaypenalty=10000
$$}
\podpis{\L{}ukasz Bożyk}

{%%%%%   CPSJ, priklad t4
Wyznacz wszystkie takie pary $(a,b)$ liczb ca\l{}kowitych dodatnich, że
$$
a+b+(D(a,b))^2=n(a,b)= 2\cdot n(a-1,b),
$$
gdzie $n(a,b)$ oznacza najmniejsz\ą{} wspóln\ą{} wielokrotność, a $D(a, b)$ -- najwi\ę{}kszy wspól\-ny
dzielnik liczb $a$, $b$.}
\podpis{Tomáš Jurík}

{%%%%%   CPSJ, priklad t5
Určete nejmenší reálnou konstantu $p$, pro kterou platí nerovnost
$$
\sqrt{ab}-\frac{2ab}{a+b} \leq p\left(\frac{a+b}2-\sqrt{ab}\right)
$$
s libovolnými kladnými reálnými čísly $a$, $b$.
}
\podpis{Jaromír Šimša}

{%%%%%   CPSJ, priklad t6
Vrcholům krychle připíšeme po jednom čísla $1,2,3,\dots ,8$ a každé její hraně pak přiřadíme součin čísel připsaných jejím
dvěma krajním bodům. Určete největší možnou hodnotu součtu čísel přiřazených všem dvanácti hranám krychle.
}
\podpis{Jaromír Šimša}
