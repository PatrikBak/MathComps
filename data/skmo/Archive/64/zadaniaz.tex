{%%%%%   Z4-I-1
...}
\podpis{...}

{%%%%%   Z4-I-2
...}
\podpis{...}

{%%%%%   Z4-I-3
...}
\podpis{...}

{%%%%%   Z4-I-4
...}
\podpis{...}

{%%%%%   Z4-I-5
...}
\podpis{...}

{%%%%%   Z4-I-6
...}
\podpis{...}

{%%%%%   Z5-I-1
Chlapci si medzi sebou menili známky, guľôčky a~loptičky.
Za 8~guľôčok je 10~známok, za 4~loptičky je 15~známok.
Koľko guľôčok je za jednu loptičku?
}
\podpis{Marie Krejčová}

{%%%%%   Z5-I-2
Žabí princ sa zúčastnil skokanskej súťaže, pri ktorej sa skákalo po
kameňoch rozmiestnených ako na obrázku. Bolo dovolené skákať len na
najbližšie kamene východným alebo južným smerom. Každý skok na východ
bol ocenený dvoma bodmi, každý skok na juh bol ocenený piatimi bodmi.
Žabí princ získal 14~bodov.
Určte všetky možné cesty, kadiaľ mohol skákať.
\insp{z5-I-2.eps}%
}
\podpis{Eva Patáková}

{%%%%%   Z5-I-3
Z~čísla $215$ môžeme vytvoriť štvorciferné číslo tým, že medzi jeho cifry
vpíšeme akúkoľvek ďalšiu cifru. Takto sme vytvorili dve štvorciferné
čísla, ktorých rozdiel bol~$120$.
Aké dve štvorciferné čísla to mohli byť?
Určte aspoň jedno riešenie.}
\podpis{Libor Šimůnek}

{%%%%%   Z5-I-4
Nájdite najväčšie číslo také, že
\begin{itemize}
\item žiadna cifra sa v~ňom neopakuje,
\item súčin každých dvoch cifier je nepárny,
\item súčet všetkých cifier je párny.
\end{itemize}}
\podpis{Martin Mach}

{%%%%%   Z5-I-5
Na obrázku je štvorec rozdelený na 25 štvorčekov.
Vyfarbite štvorčeky piatimi farbami tak, aby platilo:
\begin{itemize}
\item každý štvorček je vyfarbený jednou farbou,
\item v~žiadnom riadku ani v~žiadnom stĺpci nie sú dva štvorčeky rovnakej farby,
\item na žiadnej z~oboch uhlopriečok nie sú dva štvorčeky rovnakej farby,
\item žiadne dva rovnako zafarbené štvorčeky sa nedotýkajú stranou ani vrcholom.
\end{itemize}
~\insp{z5-I-5.eps}%
}
\podpis{Michaela Petrová}

{%%%%%   Z5-I-6
Na medaile, ktorá má tvar kruhu s~priemerom 20\,cm,
je narysovaný snehuliak tak, že sú
splnené nasledujúce požiadavky:
\begin{itemize}
  \item snehuliak je zložený z~troch kruhov ako na obrázku,
  \item priemery všetkých kruhov vyjadrené v~cm sú celočíselné,
  \item priemer každého väčšieho kruhu je o~2\,cm väčší ako priemer
    kruhu predchádzajúceho.
\end{itemize}
Určte výšku čo najväčšieho snehuliaka s~uvedenými vlastnosťami.
\insp{z5-I-6.eps}%
}
\podpis{Lenka Dedková}

{%%%%%   Z6-I-1
Erika a~Peter dostali kocku, ktorá mala každú stenu rozdelenú na štyri
rovnaké štvorce, pozri obrázok.
Peter tvrdil, že sa dajú do všetkých štvorcov vpísať čísla $1$ alebo $2$ tak, aby na
každej zo šiestich stien bol iný súčet.
Erika naopak tvrdila, že to možné nie je.
Rozhodnite, kto z~nich mal pravdu.
\insp{z6-I-1.eps}%
}
\podpis{Erika Novotná}

{%%%%%   Z6-I-2
Janíčko a~Walter zbierali autíčka.
Walter mal autíčka uložené v~skrinke na troch poličkách.
Najviac autíčok stálo na hornej poličke, na prostrednej ich bolo o~tri menej
ako na hornej a~na spodnej poličke ich bolo o~tri menej ako na prostrednej.
Pritom na jednej z~týchto poličiek bolo 15~autíčok.
Keď si Janíčko zbierku prezrel, povedal Walterovi: "Myslel som si, že keď mám viac ako 20~autíčok, tak ich mám veľa.
Teraz ale vidím, že ty máš dvakrát viac autíčok ako ja!"
Koľko autíčok mal vo svojej zbierke Janíčko?
}
\podpis{Libuše Hozová}

{%%%%%   Z6-I-3
Pán Karfiól má obdĺžnikovú záhradu rozdelenú na 9~pravouholníkových
záhonov ako na obrázku.
Pri piatich záhonoch sú zapísané veľkosti ich obvodov v~metroch.
Určte obvod celej záhrady pána Karfióla.
\insp{z6-I-3.eps}%
}
\podpis{Libuše Hozová}

{%%%%%   Z6-I-4
Katka, Barbora a~Adela sa dohadovali, ktoré dvojciferné číslo je najkrajšie.
Katka vravela, že to je to jej, pretože je deliteľné štyrmi, a~keď ho
napíše pospiatky, dostane iné dvojciferné číslo, ktoré je tiež deliteľné
štyrmi.
Barbora tvrdila, že je to určite to jej, pretože jedna z~jeho cifier je
násobkom druhej.
Adela o~svojom obľúbenom čísle prezradila, že sa dá rozložiť na súčin štyroch
prvočísel.
Nakoniec kamarátky zistili, že hovoria všetky o~tom istom čísle.
Určte, ktoré číslo to bolo.}
\podpis{Lenka Dedková}

{%%%%%   Z6-I-5
Určte, koľko rôznych riešení má nasledujúci algebrogram.
Každé písmeno zodpovedá jednej cifre od~$0$ do~$5$,
rôzne písmená zodpovedajú rôznym cifrám, rovnaké rovnakým.
$$
\begin{array}{cccc}
K & O & S & A \\
S & A & K & O \\
\hline
B & A & B & A \\
\end{array}
$$
}
\podpis{Karel Pazourek}

{%%%%%   Z6-I-6
Skauti na výlete hrali hru. V~lese bolo rozmiestnených 8~stanovísk prepojených špagátmi tak,
ako na obrázku.
Na každom stanovisku sa vydávalo jedno písmenko, prípadne pomlčka.
Stanoviská sa dajú pozdĺž špagátov prebehnúť tak, že získané
znaky tvoria reťazec
$$
\text{ANANAS--KOKOS--MANGO}.
$$
Priraďte jednotlivým stanoviskám zodpovedajúce znaky.
\insp{z6-I-6.eps}%
}
\podpis{Martin Mach}

{%%%%%   Z7-I-1
Ľuboš, Martin a~ich kamarátka Erika šetria na hračku.
Ľuboš a~Martin prispeli do spoločnej pokladničky rovnakým množstvom eur,
Erika prispela inou sumou.
Keby Erika prispela len tretinou z~toho, čo do pokladničky dodala, celkom
by mali polovicu zo sumy, ktorá je v~pokladničke teraz.
Koľkokrát viac eur do pokladničky dodala Erika ako Ľuboš?}
\podpis{Eva Patáková}

{%%%%%   Z7-I-2
Lenka sa bavila tým, že vyťukávala na kalkulačke čísla.
Používala iba cifry od~$2$ do~$9$ (\obr) a~čoskoro si všimla, že niektoré zápisy boli
osovo súmerné.
Určte počet všetkých nanajvýš trojciferných čísel s~uvedenými vlastnosťami.
\insp{z7-I-2.eps}%
}
\podpis{Lenka Dedková}

{%%%%%   Z7-I-3
Podľa projektu bude dno bazénu pokryté kamienkami troch farieb tak, ako ukazuje
obrázok (dno je navyše rozdelené na 25~zhodných pomocných štvorcov).
Cena kamienkov
na jednotku plochy sa pri jednotlivých farbách líši. Projektant počítal cenu
kamienkov použitých na takto pokryté dno a~na jeho prekvapenie sa za každý
druh kamienkov utratí rovnaká suma. Ďalej spočítal, že keby celú plochu
pokryl tými najlacnejšími kamienkami, boli by náklady 1\,700~€.
Zistite, aké by boli náklady, keby celé dno nechal pokryť tými
najdrahšími kamienkami.
\insp{z7-I-3.eps}%
}
\podpis{Libor Šimůnek}

{%%%%%   Z7-I-4
Body $N$, $O$, $P$ a~$Q$ sú vzhľadom na trojuholník $KLM$ zadané nasledujúcim
spôsobom:
\begin{itemize}
\item body $N$ a~$O$ sú postupne stredy strán $KM$ a~$KL$,
\item vrchol~$M$ je stredom úsečky~$NP$,
\item bod~$Q$ je priesečníkom priamok $LM$ a~$OP$.
\end{itemize}
\noindent
Určte, aký je pomer dĺžok úsečiek $MQ$ a~$ML$.}
\podpis{Libuše Hozová}

{%%%%%   Z7-I-5
Na starom hrade býva drak a~väzní tam princeznú.
Jano išiel princeznú oslobodiť, na hrade objavil troje dvier s~nasledujúcimi
nápismi.
\itemitem{I:} "Jaskyňa za dverami III~je prázdna."
\itemitem{II:} "Princezná je v~priestore za dverami I."
\itemitem{III:} "Pozor! Drak je v~jaskyni za dverami II."

Dobrá víla Janovi prezradila, že na dverách, za ktorými je princezná, je
nápis pravdivý, pri drakovi je nepravdivý a~na dverách prázdnej jaskyne môže byť
napísaná pravda aj lož.
Jano má na oslobodenie princeznej iba jeden pokus. Ktoré dvere má otvoriť?
}
\podpis{Marta Volfová}

{%%%%%   Z7-I-6
Matej má dve kartičky, na každú z~nich napísal jedno dvojciferné číslo.
Ak zaradí menšie číslo za väčšie, dostane štvorciferné číslo, ktoré je deliteľné
štyrmi a~deviatimi.
Ak zaradí naopak väčšie číslo za menšie, dostane štvorciferné číslo, ktoré je
deliteľné piatimi a~šiestimi.
Koľko dvojíc kartičiek mohol Matej vyrobiť tak, aby platili vyššie uvedené
podmienky? Určte všetky možnosti.}
\podpis{Michaela Petrová}

{%%%%%   Z8-I-1
Písmenkový logik je hra pre dvoch hráčov, ktorá má nasledujúce pravidlá:
\begin{enumerate}
  \item Prvý hráč si myslí slovo zložené z~piatich písmen, v~ktorom sa žiadne písmeno
    neopakuje.
  \item Druhý hráč napíše nejaké slovo z~piatich písmen.
  \item Prvý hráč odpovie dvoma číslami --
    prvé číslo udáva, koľko písmen napísaného slova sa zhoduje s~mysleným
    slovom, \tj. stoja zároveň na správnom mieste;
    druhé číslo udáva, koľko písmen napísaného slova je obsiahnutých v~myslenom
    slove, ale nestoja na správnom mieste.
  \item Kroky 2 a~3 sa opakujú, kým druhý hráč myslené slovo neuhádne.
\end{enumerate}
Záznam jednej hry dvoch kamarátov vyzeral nasledovne:
$$
\begin{array}{ccc}
\text{SONET} & 1 & 2 \\
\text{MUDRC} & 0 & 2 \\
\text{PLAST} & 0 & 2 \\
\text{KMOTR} & 0 & 4 \\
\text{ATOLY} & 1 & 1 \\
\text{DOGMA} & 0 & 2 \\
\end{array}
$$
V~nasledujúcom ťahu bolo myslené slovo uhádnuté.
Určte, ktoré slovo to bolo.
}
\podpis{Marta Volfová}

{%%%%%   Z8-I-2
Súčet všetkých deliteľov istého nepárneho čísla je~$78$.
Určte, aký je súčet všetkých deliteľov dvojnásobku tohto neznámeho čísla.}
\podpis{Karel Pazourek}

{%%%%%   Z8-I-3
V~lichobežníku $KLMN$ platí, že
\begin{itemize}
\item strany $KL$ a~$MN$ sú rovnobežné,
\item úsečky $KL$ a~$KM$ sú zhodné,
\item úsečky $KN$, $NM$ a~$ML$ sú navzájom zhodné.
\end{itemize}
Určte veľkosť uhla $KNM$.}
\podpis{Libuše Hozová}

{%%%%%   Z8-I-4
Adam má plnú krabicu guľôčok, ktoré sú veľké alebo malé, čierne alebo biele.
Pomer počtu veľkých a~malých guľôčok je $5:3$.
Medzi veľkými guľôčkami je pomer počtu čiernych a~bielych guľôčok $1:2$,
medzi malými guľôčkami je pomer počtu čiernych a~bielych $1:8$.
Aký je pomer počtu všetkých čiernych a~všetkých bielych guľôčok?}
\podpis{Michaela Petrová}

{%%%%%   Z8-I-5
Priemer známok, ktoré mali na vysvedčení žiaci 8.A z~matematiky, je presne~$2{,}45$.
Ak by sme nepočítali jednotku a~trojku súrodencov Michala a~Aleny,
ktorí do triedy prišli pred mesiacom, bol by priemer presne $2{,}5$.
Určte, koľko žiakov má 8.A.}
\podpis{Monika Dillingerová}

{%%%%%   Z8-I-6
Pejko dostal od svojho pána kváder zložený z~navzájom rovnakých kociek cukru,
ktorých bolo najmenej 1\,000 a~nanajvýš 2\,000.
Pejko kocky cukru odjedal po jednotlivých vrstvách~-- prvý deň odjedol jednu
vrstvu spredu, druhý deň jednu vrstvu sprava a~tretí deň jednu vrstvu zhora.
Pritom v~týchto troch vrstvách bol zakaždým rovnaký počet kociek.
Zistite, koľko kociek mohol mať darovaný kváder.
Určte všetky možnosti.}
\podpis{Erika Novotná}

{%%%%%   Z9-I-1
Milena nazbierala do košíka posledné spadnuté orechy a~zavolala na partiu chlapcov,
nech sa o~ne podelia.
Dala im ale podmienku:
prvý si vezme 1~orech a~desatinu zvyšku, druhý si vezme 2~orechy a~desatinu
nového zvyšku, tretí si vezme 3~orechy a~desatinu ďalšieho zvyšku
a~tak ďalej.
Takto sa podarilo rozobrať všetky orechy a~pritom každý dostal rovnako veľa.
Určte, koľko Milena nazbierala orechov a~koľko sa o~ne delilo chlapcov.
}
\podpis{Marta Volfová}

{%%%%%   Z9-I-2
Lenka sa bavila tým, že vyťukávala na kalkulačke čísla, pričom
používala iba cifry od~$2$ do~$9$ (\obr).
Zápisy niektorých čísel mali tú vlastnosť, že ich obraz v~osovej
alebo stredovej súmernosti bol opäť zápisom nejakého čísla.
Určte počet všetkých nanajvýš trojciferných čísel s~uvedenými vlastnosťami.
\insp{z9-I-2.eps}%
}
\podpis{Lenka Dedková}

{%%%%%   Z9-I-3
Darček je zabalený do krabice, ktorej rozmery v~cm sú $40\times30\times6$.
Krabica je previazaná špagátom ako na obrázku.
Určte, koľko najmenej cm špagátu je treba na previazanie krabice, ak na
uzol a~mašľu stačí 20\,cm.
\insp{z9-I-3.eps}%
}
\podpis{Marie Krejčová}

{%%%%%   Z9-I-4
Peter, Martin a~Juro triafali do zvláštneho terča, ktorý mal iba tri
políčka s~hodnotami 12, 18 a~30 bodov.
Všetci chlapci hádzali rovnakým počtom šípok,
všetky šípky trafili do terča
a~výsledky každých dvoch chlapcov sa líšili v~jedinom hode.
Petrov priemerný bodový výsledok bol o~dva body lepší ako Martinov
a~ten bol o~jeden bod lepší ako priemer Jurov.
Určte, koľkými šípkami hádzal každý z~chlapcov.}
\podpis{Erika Novotná}

{%%%%%   Z9-I-5
Jaro si kúpil nové nohavice, ale boli príliš dlhé.
Ich dĺžka bola vzhľadom k~Jarovej výške v~pomere $5:8$.
Mamička mu nohavice skrátila o~4\,cm, čím sa pôvodný pomer zmenšil o~4\,\%.
Určte, aký vysoký je Jaro.}
\podpis{Libuše Hozová}

{%%%%%   Z9-I-6
Neznáme číslo je deliteľné práve tromi rôznymi prvočíslami.
Keď tieto prvočísla zoradíme vzostupne, platí nasledujúce:
\begin{itemize}
\item Rozdiel druhého a~prvého prvočísla je polovicou rozdielu tretieho a~druhého prvočísla.
\item Súčin rozdielu druhého a~prvého prvočísla s~rozdielom tretieho a~druhého prvočísla
je násobkom~$17$.
\end{itemize}
\noindent
Určte najmenšie číslo, ktoré má všetky vyššie uvedené vlastnosti.}
\podpis{Karel Pazourek}

{%%%%%   Z4-II-1
...}
\podpis{...}

{%%%%%   Z4-II-2
...}
\podpis{...}

{%%%%%   Z4-II-3
...}
\podpis{...}

{%%%%%   Z5-II-1
V~kúzelníckom bazáre si kúzelníci medzi sebou vymieňali kúzelnícke klobúky, paličky
a~plášte. Za 4~paličky je 6~plášťov a~za 5~paličiek je 5~klobúkov. Koľko plášťov je za 5~paličiek
a~1~klobúk?}
\podpis{Veronika Hucíková}

{%%%%%   Z5-II-2
Juro mal tri zhodné obdĺžniky. Najskôr ich k~sebe priložil tak ako na \ifobrazkyvedla{}obrázku\else\obr{}\fi{}
a~dostal obdĺžnik, ktorý mal obvod $20\cm$. Potom ich k~sebe priložil inak a~dostal obdĺžnik
s~iným obvodom. Aký obvod mal tento obdĺžnik?
\ifobrazkyvedla\vskip.0\baselineskip~\else\insp{z5-ii-2.eps}\fi
}
\podpis{Erika Novotná}

{%%%%%   Z5-II-3
Z~čísla $215$ môžeme vytvoriť štvorciferné číslo tak, že medzi jeho cifry vpíšeme
akúkoľvek ďalšiu cifru. Takto sme vytvorili dve štvorciferné čísla, ktorých súčet bol $4\,360$.
Aké dve štvorciferné čísla to mohli byť? Určte všetky možnosti.}
\podpis{Libor Šimůnek}

{%%%%%   Z6-II-1
Fabián má štyri kartičky, na každú z~nich napísal jedno celé kladné číslo
menšie ako 10. Čísla napísal štyrmi rôznymi farbami, pričom platí:
%%Každé číslo napsal inou farbou a~platí pre ne:
\begin{itemize}
\iitem Súčin zeleného a žltého čísla je zelené číslo.
\iitem Modré číslo je rovnaké ako červené číslo.
\iitem Súčin červeného a modrého čísla je dvojciferné číslo zapísané zelenou
a žltou cifrou (v~tomto poradí).
\end{itemize}
Určte tieto štyri čísla.
}
\podpis{Michaela Petrová}

{%%%%%   Z6-II-2
Na deň detí otvorili v~ZOO bludisko so šiestimi stanoviskami, na ktorých sa
rozdávali cukríky. Na jednom stanovisku sa pri každom vstupe rozdávalo 5~cukríkov, na dvoch
stanoviskách sa rozdávali 3~cukríky a na troch stanoviskách 1~cukrík.
Juraj najskôr vstúpil na stanovisko označené \ifobrazkyvedla\else{}na \obr{} \fi{}šípkou a~pokračoval tak, že
každou cestičkou prešiel nanajvýš raz.
Určte, koľko najviac cukríkov mohol Juraj dostať.
\ifobrazkyvedla\else\insp{z6-ii-2.eps}\fi%
}
\podpis{Erika Novotná}

{%%%%%   Z6-II-3
Jano mal štyri zhodné trojuholníky.
Skladal z~nich rôzne útvary, a~to tak, že trojuholníky k~sebe prikladal
stranami rovnakej dĺžky.
Najskôr zložil útvar z~troch trojuholníkov ako na \ifobrazkyvedla{}prvom obrázku\else{}na \obr{}\fi{}, ktorý mal obvod
43\,cm.
Potom útvar rozobral a~zložil iný útvar z~troch trojuholníkov, ktorý mal obvod
35\,cm.
Nakoniec zo všetkých štyroch trojuholníkov zložil ďalší útvar ako na \ifobrazkyvedla{}druhom obrázku\else{}na \obr{}\fi{},
a~ten mal obvod 46\,cm.
Určte dĺžky strán trojuholníkov.
\insp{64-z6-ii-3}%
}
\podpis{Eva Patáková}

{%%%%%   Z7-II-1
Nájdite všetky prirodzené čísla od 90 do 150 také, že ciferný
súčet ich ciferného súčtu je rovný 1.}
\podpis{Eva Patáková}

{%%%%%   Z7-II-2
Siedme triedy z~našej školy súťažili v~zbieraní vrchnákov z~PET fliaš.
Trieda~A nazbierala štvrtinu toho, čo triedy B a~C dokopy,
trieda~B nazbierala pätinu toho, čo triedy A a~C dokopy,
a~trieda C nazbierala 570 vrchnákov.
Určte, koľko vrchnákov nazbierali tieto tri triedy spolu.
}
\podpis{Marta Volfová}

{%%%%%   Z7-II-3
Prebieha rekonštrukcia námestia v~tvare štvorca so stranou 20~metrov. Keby
bolo celé vydláždené lacnejšou svetlou dlažbou, boli by náklady na materiál
10\,000€. Keby bolo celé námestie pokryté drahšou tmavou dlažbou, stál by
materiál 30\,000€. Architekt však v~centrálnej časti námestia navrhol svetlú
štvorcovú časť, ktorá bude olemovaná pruhom tmavej dlažby o~šírke 2~metre,
a~vo vonkajšej časti námestia bude rovnaká svetlá dlažba ako uprostred,
pozri obrázok. Podľa tohto návrhu budú náklady na materiál tmavej časti
rovnaké ako na celkový materiál svetlých častí.
Určte:
\begin{itemize}
\iitem koľko stojí materiál na vydláždenie námestia podľa tohto projektu,
\iitem aká dlhá je strana svetlého štvorca v~centrálnej časti námestia.
\end{itemize}
\insp{z7-II-3.eps}
}
\podpis{Libor Šimůnek}

{%%%%%   Z8-II-1
V~rovnostrannom trojuholníku $ABC$ leží na strane~$BC$ bod~$F$.
Obsah trojuholníka $ABF$ je trikrát väčší ako obsah trojuholníka $ACF$
a~rozdiel obvodov týchto dvoch trojuholníkov je 5\,cm.
Určte dĺžku strany trojuholníka $ABC$.
}
\podpis{Eva Patáková}

{%%%%%   Z8-II-2
Traja hudobníci Janek, Mikeš a~Vávra si zvyčajne rozdelia spoločný honorár
v~pomere $4:5:6$, najmenej dostane Janek a~najviac Vávra. Tentoraz Vávra
nehral dobre, a~tak sa svojho podielu vzdal. Janek navrhol, že si Vávrovu časť
rozdelia s~Mikešom na polovice. Mikeš však trval na tom, aby si aj túto časť
rozdelili nerovnomerne ako zvyčajne, teda v~pomere $4:5$. Mikeš by totiž
podľa Jankovho návrhu dostal o 4€ menej ako podľa svojho. Určte výšku
spoločného honorára.}
\podpis{Libor Šimůnek}

{%%%%%   Z8-II-3
Keď jeden rozmer kvádra zdvojnásobíme, druhý rozmer kvádra predelíme dvoma
a~tretí rozmer zväčšíme o~6\,cm, dostaneme kocku, ktorá má rovnaký
povrch ako pôvodný kváder. Určte rozmery tohto kvádra. }
\podpis{Michaela Petrová}

{%%%%%   Z9-II-1
Myslím na niekoľko bezprostredne po sebe idúcich prirodzených čísel. Keby sme
z~nich vyškrtli čísla $70$, $82$ a~$103$, aritmetický priemer čísel by sa nezmenil. Keby sme
namiesto toho vyškrtli čísla $122$ a~$123$, aritmetický priemer by sa zmenšil presne o~$1$. Na
ktoré prirodzené čísla myslím?}
\podpis{Libor Šimůnek}

{%%%%%   Z9-II-2
Po sebe idúce prirodzené čísla postupne pričítame a~odčítame podľa nasledujúceho
návodu:
$$
1 + 2 - 3 - 4 + 5 + 6 - 7 - 8 + 9 + 10 - 11 - 12 + \dots
$$
Určte, aká bude hodnota takéhoto výrazu, ak jeho posledný člen je $2\,015$. }
\podpis{Libuše Hozová}

{%%%%%   Z9-II-3
Anička dostala na narodeniny tortu v~tvare obdĺžnika. Pomocou dvoch priamych rezov
si odkrojila kúsok, ktorý je na \ifobrazkyvedla{}obrázku\else\obr{}\fi{} vyznačený sivou. Určte, akú časť torty si
Anička odkrojila.
\ifobrazkyvedla\vskip\baselineskip~\else\insp{z9-ii-3.eps}\fi
}
\podpis{Alžbeta Bohiníková}

{%%%%%   Z9-II-4
Istý obdĺžnik mal svoje rozmery vyjadrené v~decimetroch celými číslami. Potom rozmery
trikrát zmenil. Najskôr jeden svoj rozmer zdvojnásobil a~druhý zmenil tak, aby mal
rovnaký obsah ako na začiatku. Potom jeden rozmer zväčšil o~1~dm a~druhý zmenšil o~4~dm,
pričom mal stále taký istý obsah ako na začiatku. Nakoniec svoj kratší rozmer zmenšil
o~1~dm, dlhší ponechal bezo zmeny. Určte pomer strán posledného obdĺžnika.}
\podpis{Erika Novotná}

{%%%%%   Z9-III-1
Peter, Martin a~Juraj triafali do zvláštneho terča, ktorý mal iba tri
políčka s~navzájom rôznymi hodnotami.
Každý z~chlapcov hádzal celkom desaťkrát a~vždy trafil do terča.
Bodový zisk z~prvých ôsmich hodov mali všetci traja chlapci rovnaký.
Pri posledných dvoch hodoch trafil Juraj dvakrát políčko s~najnižšou možnou
hodnotou, Martin dvakrát políčko so strednou hodnotou a~Peter dvakrát políčko s~najvyššou hodnotou.
Aritmetický priemer všetkých Martinových hodov bol o~1 väčší ako Jurajov priemer
a~Petrov priemer bol o~1 väčší ako Martinov priemer.
Určte všetky možné hodnoty políčok na terči, ak viete, že jedna z~nich bola 12.
}
\podpis{Erika Novotná}

{%%%%%   Z9-III-2
V~trojuholníku $ABC$ ležia na strane~$AB$ body $E$ a~$F$.
Obsah trojuholníka $AEC$ je $1\cm^2$,
obsah trojuholníka $EFC$ je $3\cm^2$
a~obsah trojuholníka $FBC$ je $2\cm^2$.
Bod~$T$ je ťažiskom trojuholníka $AFC$ a~bod~$G$ je priesečníkom priamok
$CT$ a~$AB$.
Bod~$R$ je ťažiskom trojuholníka $EBC$ a~bod~$H$ je priesečníkom priamok
$CR$ a~$AB$.
Určte obsah trojuholníka $GHC$.
}
\podpis{Eva Patáková}

{%%%%%   Z9-III-3
Určte, aká je posledná cifra súčinu všetkých párnych prirodzených čísel,
ktoré sú menšie ako 100 a~ktoré nie sú násobkami~10.
}
\podpis{Marta Volfová}

{%%%%%   Z9-III-4
Daný je obdĺžnik $ABCD$, ktorého kratšia strana je $AB$.
Určte, pre ktoré body~$P$ na priamke~$AD$ platí, že os uhla $BPD$ prechádza
bodom~$C$.
Svoje tvrdenie zdôvodnite a~popíšte, ako by ste všetky také body
zostrojili.
}
\podpis{L. Růžičková}

