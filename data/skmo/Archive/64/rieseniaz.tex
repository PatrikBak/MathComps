{%%%%%   Z4-I-1
...}

{%%%%%   Z4-I-2
...}

{%%%%%   Z4-I-3
...}

{%%%%%   Z4-I-4
...}

{%%%%%   Z4-I-5
...}

{%%%%%   Z4-I-6
...}

{%%%%%   Z5-I-1
\napad
Predstavte si, že máte napr. 100 známok. Za čo by ste mohli niektoré z~nich vymeniť?

\riesenie
V~oboch porovnaniach "hodnôt" jednotlivých predmetov sa vyskytujú známky,
budeme ich teda považovať za akési spoločné "platidlo":

Ak za 8 guľôčok je 10 známok, tak za trojnásobné množstvo
guľôčok musí byť trojnásobné množstvo známok, teda za 24~guľôčok je 30~
známok. Ak za 4~loptičky je 15~známok, tak za dvojnásobné množstvo
loptičiek musí byť dvojnásobné množstvo známok, teda za 8~loptičiek je 30~známok.

Ak máme 30 známok, môžeme ich vymeniť buď za 24 guľôčok, alebo za 8 loptičiek.
Osem loptičiek má teda rovnakú hodnotu ako 24 guľôčok,
za jednu loptičku je osemkrát menej guľôčok, teda 3 guľôčky.

\poznamka
K~rovnakému výsledku sa možno dostať rôznymi spôsobmi, napr. takto:

Za 8 guľôčok je 10 známok, teda za 4 guľôčky je 5~známok  a~za 12~guľôčok je
15~známok.
Za rovnaký počet známok možno vymeniť aj 4 loptičky, takže 4~loptičky majú rovnakú
hodnotu ako 12~guľôčok.
Za jednu loptičku sú 3~guľôčky.
}

{%%%%%   Z5-I-2
\napad
Skúšajte skákať podľa uvedených pravidiel a~vylučujte nevyhovujúce možnosti.

\riesenie
V~závislosti od počtu skokov, ktoré mohol žabí princ spraviť na juh, určíme
počet skokov na východ tak, aby súčet získaných bodov bol 14:
\begin{itemize}
\item na juh nemusel skočiť ani raz, 14 bodov možno získať siedmimi skokmi na
východ;
\item keby skočil na juh raz, potom by musel niekoľkými skokmi na východ
získať ${14-5}=7$~bodov, čo nie je možné (skokmi len na východ by získal
párny počet bodov);
\item keby skočil na juh dvakrát, tak by na východ musel skočiť tiež dvakrát
($10+4=14$);
\item keby skočil na juh viac ako dvakrát, tak by získal viac ako 14~bodov.
\end{itemize}

Žabí princ teda skákal buď sedemkrát na východ, alebo dvakrát na východ
a~dvakrát na juh. V~prvom prípade sa dá skákať jediným spôsobom:
\insp{z5-I-2a.eps}%

V~druhom prípade mohol skákať ktoroukoľvek z~nasledujúcich možností:
\insp{z5-I-2b.eps}%
}

{%%%%%   Z5-I-3
\napad
Je možné, aby vpisované cifry boli v~oboch prípadoch na rovnakom
mieste?

\riesenie
Novo vytvorené štvorciferné číslo je buď typu $2{*}15$, alebo
typu $21{*}5$ (hviezdičkou označujeme vpisované neznáme cifry).
Keby boli obe nové čísla rovnakého typu, bol by v~prvom prípade
($2{*}15$) rozdiel takých čísel násobkom~100, v~druhom prípade ($21{*}5$)
by bol násobkom~10, avšak nie väčším ako~90.
Rozdiel však má byť~120, takže nové čísla musia byť rôzneho typu.

V~ľavom stĺpci uvažujeme prípad, keď väčšie z~čísel je číslo typu $2{*}15$,
v~pravom stĺpci naopak:
$$
\alggg{&2&*&1&5\\-&2&1&*&5}{&&1&2&0}
\hskip2cm
\alggg{&2&1&*&5\\-&2&*&1&5}{&&1&2&0}
$$
Oba prípady doriešime odzadu ako algebrogram, čím dostaneme dve možné riešenia:
$$
\alggg{&2&\bfm 3&1&5\\-&2&1&\bfm 9&5}{&&1&2&0}
\hskip2cm
\alggg{&2&1&\bfm 3&5\\-&2&\bfm 0&1&5}{&&1&2&0}
$$}

{%%%%%   Z5-I-4
\napad
Z~akých cifier môže pozostávať číslo, pre ktoré platí druhá podmienka zo zadania?

\riesenie
Súčin dvoch čísel je nepárny práve vtedy, keď sú obe čísla nepárne.
Preto hľadané číslo môže pozostávať len z~nepárnych cifier.

Súčet niekoľkých nepárnych čísel je párny práve vtedy, keď je počet sčítancov párny.
Preto hľadané číslo pozostáva z~párneho počtu cifier.

Nepárnych cifier je päť a~cifry sa nemajú opakovať, preto každé číslo
vyhovujúce uvedeným podmienkam je buď dvojciferné, alebo štvorciferné.
Najväčším z~nich je číslo zapísané pomocou štyroch najväčších cifier zoradených
zostupne, teda číslo $9753$.
}

{%%%%%   Z5-I-5
\bgroup
\def\ctr#1{\hbox to16pt{\hfil#1\hfil}}
\def\tstrut{\vrule height 11pt depth 5pt width 0pt}
\def\x{$\times$}
\noncenteredtables
\napad
Skúste ako prvý vyfarbiť prostredný štvorček.

\res
Zo zadania vieme, že v~každom riadku, v~každom stĺpci a~na každej uhlopriečke bude každá z~piatich farieb práve raz.
Spolu teda bude každá z~piatich farieb v~celom štvorci zastúpená práve
päťkrát.
Navyše musíme mať na pamäti ešte podmienku, že dva štvorčeky rovnakej farby
nesmú mať spoločný ani vrchol.

Prostredný štvorček leží na oboch uhlopriečkach, takže po jeho vyfarbení
budeme môcť uplatniť podmienky na čo možno najväčší počet ostatných štvorčekov.
Začneme tým, že prostredný štvorček vyfarbíme napr. namodro (M) a~všetky
políčka, ktoré nemôžu byť vyfarbené rovnakou farbou, označíme krížikom:
$$
\begintable
\x||\x||\x\cr
|\x|\x|\x|\cr
\x|\x|\bf M|\x|\x\cr
|\x|\x|\x|\cr
\x||\x||\x\endtable
$$

V~každom riadku a~v~každom stĺpci, kde ešte nie je modrý štvorček, si môžeme vybrať z~dvoch možností.
Budeme postupovať po riadkoch zhora nadol~--
v~každom riadku vyfarbíme jedno políčko a~označíme krížikom všetky ostatné
políčka v~štvorci, ktoré nemôžu byť modré.
Takto postupne dostávame:
$$
\begintable
\x|\bf M|\x|\x|\x\cr
\x|\x|\x|\x|\cr
\x|\x|M|\x|\x\cr
|\x|\x|\x|\cr
\x|\x|\x||\x\endtable
\quad
\begintable
\x|M|\x|\x|\x\cr
\x|\x|\x|\x|\bf M\cr
\x|\x|M|\x|\x\cr
|\x|\x|\x|\x\cr
\x|\x|\x||\x\endtable
\quad
\begintable
\x|M|\x|\x|\x\cr
\x|\x|\x|\x|M\cr
\x|\x|M|\x|\x\cr
\bf M|\x|\x|\x|\x\cr
\x|\x|\x||\x\endtable
\quad
\begintable
\x|M|\x|\x|\x\cr
\x|\x|\x|\x|M\cr
\x|\x|M|\x|\x\cr
M|\x|\x|\x|\x\cr
\x|\x|\x|\bf M|\x\endtable
$$

Budeme pokračovať vo vyfarbovaní napr. zelenou farbou (Z).
Z~rovnakého dôvodu ako vyššie chceme začať s~takým políčkom, ktorého
vyfarbenie ovplyvní čo možno najviac ostatných políčok.
Preto volíme nejaké políčko blízko stredu štvorca, napr. druhé políčko v~treťom riadku:
$$
\begintable
|M|||\cr
\x|\x|\x||M\cr
\x|\bf Z|M|\x|\x\cr
M|\x|\x||\cr
|\x||M|\endtable
$$
Ďalej postupujeme podobne ako v~predchádzajúcom prípade, akurát dávame prednosť
tým riadkom, v~ktorých sa javí jediná možnosť vyfarbenia:
$$
\begintable
|M|\x|\x|\x\cr
\x|\x|\x|\bf Z|M\cr
\x|Z|M|\x|\x\cr
M|\x|\x|\x|\cr
\x|\x||M|\endtable
\quad
\begintable
\bf Z|M|\x|\x|\x\cr
\x|\x|\x|Z|M\cr
\x|Z|M|\x|\x\cr
M|\x|\x|\x|\cr
\x|\x||M|\x\endtable
\quad
\begintable
Z|M|\x|\x|\x\cr
\x|\x|\x|Z|M\cr
\x|Z|M|\x|\x\cr
M|\x|\x|\x|\bf Z\cr
\x|\x|\bf Z|M|\x\endtable
$$

Podľa rovnakých zásad pokračujeme s~ďalšou farbou, napr. červenou~(Č).
Začíname napr. na štvrtom políčku v~treťom riadku:
$$
\begintable
Z|M||\x|\cr
||\x|Z|M\cr
\x|Z|M|\bf Č|\x\cr
M||\x|\x|Z\cr
||Z|M|\endtable
\quad
\begintable
Z|M||\x|\x\cr
|\x|\x|Z|M\cr
\x|Z|M|Č|\x\cr
M|\bf Č|\x|\x|Z\cr
\x|\x|Z|M|\endtable
\quad
\begintable
Z|M|\bf Č|\x|\x\cr
\bf Č|\x|\x|Z|M\cr
\x|Z|M|Č|\x\cr
M|Č|\x|\x|Z\cr
\x|\x|Z|M|\bf Č\endtable
$$

Ako ďalšie vyfarbíme napr. druhý štvorček v~druhom riadku, a~to napr.
hnedou farbou~(H):
$$
\begintable
Z|M|Č||\cr
Č|\bf H|\x|Z|M\cr
\x|Z|M|Č|\cr
M|Č||\x|Z\cr
|\x|Z|M|Č\endtable
\quad
\begintable
Z|M|Č||\x\cr
Č|H|\x|Z|M\cr
\x|Z|M|Č|\bf H\cr
M|Č||\x|Z\cr
|\x|Z|M|Č\endtable
\quad
\begintable
Z|M|Č|\bf H|\x\cr
Č|H|\x|Z|M\cr
\x|Z|M|Č|H\cr
M|Č|\bf H|\x|Z\cr
\bf H|\x|Z|M|Č\endtable
$$

Teraz vidíme, že aj posledné nevyfarbené políčka spĺňajú všetky
podmienky zo zadania. Ak ich teda vyfarbíme nejakou ďalšou farbou,
napr. ružovou~(R), dostávame jedno z~možných riešení:
$$
\begintable
Z|M|Č|H|R\cr
Č|H|R|Z|M\cr
R|Z|M|Č|H\cr
M|Č|H|R|Z\cr
H|R|Z|M|Č\endtable
$$

\poznamka
I~keď to z~predchádzajúceho opisu riešenia nie je úplne zrejmé, v~skutočnosti má
táto úloha iba dve riešenia (až na voľbu použitých farieb), a~tie sú navyše
zrkadlovo prevrátené.
Pokúste sa preskúmať, prečo to tak je.
\egroup
}

{%%%%%   Z5-I-6
\goodbreak\napad
O~koľko cm je priemer najväčšieho kruhu väčší ako priemer najmenšieho?

\riesenie
Najmenší je horný kruh predstavujúci hlavu snehuliaka.
Prostredný kruh je o~2\,cm väčší ako najmenší kruh a~spodný kruh je o~4\,cm
väčší ako najmenší kruh.
Celková výška snehuliaka je teda rovná trojnásobku priemeru najmenšieho kruhu a~k~tomu 6\,cm.

Aby bola nad aj pod snehuliakom ešte nejaká medzera, musí byť výška snehuliaka
menšia ako 20\,cm.
To znamená, že trojnásobok priemeru najmenšieho kruhu musí byť menší ako 14\,cm,
takže priemer najmenšieho kruhu musí byť menší alebo rovný 4\,cm ($14:3$ je~4,
zvyšok~2).
Najväčší snehuliak so všetkými požadovanými vlastnosťami je teda vysoký
$$
3\cdot 4+6=18\,(\Cm).
$$
}

{%%%%%   Z6-I-1
\napad
Aký je najväčší možný súčet na jednej stene?

\riesenie
Najmenší možný súčet, ktorý možno na stene kocky vytvoriť, je~4, a~to vtedy,
keď je vo všetkých štvorcoch napísané číslo~1.
Naopak, najväčší možný súčet je~8, a~to vtedy, keď je všade napísané číslo~2.
Na jednotlivých stenách kocky môžu byť iba súčty medzi týmito dvoma
hodnotami, tzn. 4, 5, 6, 7 alebo~8.

Uvedeným spôsobom je teda možné vytvoriť nanajvýš päť rôznych súčtov, avšak
kocka má stien šesť.
Nech už sú čísla napísané akokoľvek, musí sa aspoň jeden súčet opakovať.
Pravdu preto mala Erika.
}

{%%%%%   Z6-I-2
\napad
Mohlo stáť 15 Walterových autíčok na hornej poličke?

\res
Zo zadania nevieme, na ktorej poličke bolo oných 15 Walterových autíčok.
Preto musíme rozlišovať nasledujúce tri prípady:
%\begin{enumerate}\alphatrue
\item{a)}
Keby bolo 15 autíčok na hornej poličke, bolo by na prostrednej 12 a~na dolnej
9, \tj.~dokopy $15+12+9=36$.
Janíčko by potom mal 18 autíčok, čo je menej ako~20.
\item{b)}
Keby bolo 15 autíčok na prostrednej poličke, bolo by na hornej 18 a~na dolnej
12, \tj.~dokopy $18+15+12=45$.
Tento súčet je však nepárny, čo nie je možné (Janíčko by mal mať 22{,}5 autíčka).
\item{c)}
Keby bolo 15 autíčok na dolnej poličke, bolo by na hornej 21 a~na prostrednej
18, \tj.~dokopy $21+18+15=54$.
Janíčko by potom mal 27~autíčok.
%\end{enumerate}

\noindent
Jediný prípad, ktorý nie je v spore so žiadnou uvedenou informáciou, je c).
Janíčko mal vo svojej zbierke 27 autíčok.
}

{%%%%%   Z6-I-3
\napad
Pokúste sa zložiť obvod celej záhrady z~úsečiek, ktoré ohraničujú záhony, ktorých obvody poznáme.

\riesenie
Pre každý zo záhonov so známymi obvodmi platí, že jeho strany
sú zhodné (príp. dokonca splývajú) s~úsečkami vyznačenými na obvode celej
záhrady.
Navyše obvod celej záhrady môže byť zložený práve z~týchto úsečiek, avšak určite
nie zo všetkých~-- napr. je možné použiť všetky úsečky okrem tých, ktoré
ohraničujú záhon s~obvodom 4 metre.
\insp{z6-I-3a.eps}%

To znamená, že obvod celej záhrady môže byť vyjadrený pomocou daných obvodov
takto:
$$
6+6+8+12-4=28.
$$
Záhrada pána Karfióla má teda obvod 28 metrov.

\poznamka
Ak predpokladáme, že prostredný záhon je štvorcový, tak je jednoduché určiť
dĺžky všetkých strán všetkých záhonov a~odtiaľ spočítať obvod celej záhrady.
Tento predpoklad však nie je priamo uvedený v~zadaní, preto by sa s~ním pracovať
nemalo.
\niedorocenky{Riešenia založené na tomto predpoklade hodnoťte nanajvýš stupňom "dobre".}
}

{%%%%%   Z6-I-4
\napad
Hľadajte dvojciferné čísla, ktoré majú všetky uvedené vlastnosti.

\riesenie
Podľa Katky má byť hľadané číslo deliteľné štyrmi, je to teda jedno z~nasledujúcich čísel:
$$
12,\ 16,\ 20,\ 24,\ 28,\ 32,\ 36,\ 40,\ 44,\ 48,\ 52,\ 56,\ 60,\ 64,\ 68,\
72,\ 76,\ 80,\ 84,\ 88,\ 92,\ 96.
$$
Súčasne má platiť, že toto číslo napísané pospiatky dáva iné dvojciferné
číslo deliteľné štyrmi.
Tejto podmienke vyhovujú iba čísla
$$
48,\ 84.
$$
(Čísla 40, 44, 80 a~88 napísané pospiatky sú tiež deliteľné štyrmi, ale
44 a~88 nedávajú iné čísla, 40 a~80 nedávajú dvojciferné čísla.)

Podľa Barbory je hľadané číslo také, že jedna jeho cifra je násobkom
druhej.
Tejto podmienke vyhovuje ako 48, tak 84.

Podľa Adely možno hľadané číslo rozložiť na súčin štyroch prvočísel.
Prvočíselný rozklad oboch zatiaľ vyhovujúcich čísel je
$$
48=2\cdot2\cdot2\cdot2\cdot3,\quad
84=2\cdot2\cdot3\cdot7.
$$
Číslo 48 má vo svojom rozklade päť prvočísel, číslo 84 štyri prvočísla.
Kamarátky hovorili o~čísle~84.
}

{%%%%%   Z6-I-5
\napad
Čo sa dá povedať o~cifrách prislúchajúcich písmenám $A$ a~$O$?

\riesenie
Piatim rôznym písmenám máme priradiť rôzne cifry od 0 po 5.
To znamená, že v~uvedenom algebrograme nemusíme uvažovať prechod
cez desiatku (súčet najväčších dvoch povolených cifier je $4+5=9$).

Najskôr si všimnime druhý a~štvrtý stĺpec:
Aby sme dostali $A+O=A$, musí nutne byť
$$O=0.
$$
Z~prvého a~tretieho stĺpca vidíme, že $K+S=B$.
Pritom $K$ a~$S$ zodpovedajú rôznym cifrám od 1 po 5 (nula už je obsadená)
a~ich súčet má byť rovný inej cifre od 1 po 5.
Vypíšeme systematicky všetky možnosti:
$$
\begintable
K\|1|1|1|2\|2|3|4|3\cr
S\|2|3|4|3\|1|1|1|2\crthick
B\|3|4|5|5\|3|4|5|5\endtable
$$
Máme celkom 8 možností, ako môže vyzerať trojica $S$, $K$, $B$.
Zo šiestich možných cifier sú v~tejto chvíli obsadené štyri, a~to 8 rôznymi
spôsobmi.
Písmeno~$A$ potom môže zodpovedať ktorejkoľvek zo zvyšných dvoch cifier.
Počet všetkých možných riešení algebrogramu je teda
$$
8\cdot2=16.
$$
}

{%%%%%   Z6-I-6
\napad
Môže sa niektoré z~písmen vydávať na viacerých stanoviskách?

\riesenie
Najskôr sa uistíme, že vo výslednom reťazci sa nachádza práve toľko rôznych znakov,
koľko je stanovísk, \tj. osem.
Kvôli jednoduchšiemu vyjadrovaniu si jednotlivé stanoviská očíslujeme:
\insp{z6-I-6a.eps}%

Niektoré stanoviská susedia s~dvoma (1, 3, 7, 8), iné s~troma ďalšími
stanoviskami (2, 4, 5, 6).
Určitú informáciu o~rozmiestnení znakov na stanoviskách získame, keď pre každý
znak v~reťazci zistíme, s~koľkými ďalšími znakmi susedí.
S~dvoma znakmi susedia štyri znaky:
\begin{itemize}
\item N susedí s~A~a~G,
\item K~susedí s~-- a~O,
\item M susedí s~-- a~A,
\item G susedí s~N a~O.
\end{itemize}
\noindent
S~troma znakmi susedia zvyšné štyri znaky:
\begin{itemize}
\item A~susedí s~N, S~a~M,
\item S~susedí s~A, -- a~O,
\item O~susedí s~K, S~a~G,
\item -- susedí s~S, K~a~M.
\end{itemize}
Z~toho vyplýva, že žiadny zo znakov A, S, O,~-- nemôže byť na žiadnom zo
stanovísk 1, 3, 7, 8,
tieto znaky teda musia byť nejako rozmiestnené na stanoviskách 2, 4, 5, 6.
Medzi týmito stanoviskami vidíme, že iba stanovisko 5 susedí so všetkými
ostatnými.
Rovnakú vlastnosť medzi znakmi A, S, O,~-- má iba písmeno~S.
Písmeno S~sa preto určite vydávalo na 5.~stanovisku.
\insp{z6-I-6b.eps}%

Znaky N, K, M, G sú nejako rozmiestnené na stanoviskách 1, 3, 7, 8.
Medzi týmito stanoviskami spolu navzájom susedia iba stanoviská 7 a~8.
Rovnakú vlastnosť medzi písmenami N, K, M, G majú iba písmená N a~G.
Preto sa písmeno~N vydávalo na 7.~stanovisku a~G na 8.~stanovisku, alebo
naopak. Budeme uvažovať prvý prípad.
\insp{z6-I-6c.eps}%

Na stanovisku 4 môže byť jedine znak, ktorý susedí s~S a~súčasne s~N;
jedinou možnosťou je~A.
Z~podobných dôvodov môže byť na stanovisku~6 jedine~O.
Posledné neobsadené stanovisko zo skupiny 2, 4, 5, 6 je teraz~2, na ktoré
ostáva jediný možný znak, a~to~--.

Na posledné dve stanoviská 1 a~3 ostávajú písmená M a~K.
Ich jediné možné umiestnenie je opäť určené podľa susedov
(M~susedí s~A, K~susedí s~O):
\insp{z6-I-6d.eps}%

Vidíme, že v~tomto priradení naozaj platia všetky vyššie uvedené susedské
vzťahy.
Teraz je nutné ukázať, že stanoviskami je možné pozdĺž špagátov prebehnúť tak,
že získané písmená tvoria onen reťazec.
Riešením je nasledujúca cesta:
$$
4\ 7\ 4\ 7\ 4\ 5\ 2\ 3\ 6\ 3\ 6\ 5\ 2\ 1\ 4\ 7\ 8\ 6.
$$

\poznamka
Až na voľbu poradia písmen N a~G na 7. a~8. stanovisku bolo priradenie
ostatných znakov ku stanoviskám určené jednoznačne.
Úloha má teda dve riešenia, ktoré sú osovo súmerné.

Značnú časť uvedeného riešenia možno samozrejme nahradiť skúšaním;
podstatné je nájsť nejaké priradenie znakov stanoviskám a~opísať
zodpovedajúcu cestu.
V~takom prípade však nerozpoznáme, že viac riešení vlastne nie je.
}

{%%%%%   Z7-I-1
\napad
Akú časť celkovej sumy tvoria dve tretiny Erikinho príspevku?

\riesenie
Zo zadania vyplýva, že súčet príspevkov Ľuboša a~Martina a~tretiny príspevku
Eriky tvoria polovicu celkovej sumy.
To znamená, že dve tretiny Erikinho príspevku tvoria druhú polovicu celkovej
sumy.
Preto jedna tretina Erikinho príspevku je rovná polovici z~polovice~-- teda
štvrtine~-- celkovej sumy.
Erika teda prispela tromi štvrtinami
a~Martin spolu s~Ľubošom štvrtinou celkovej sumy.
Keďže Martin a~Ľuboš prispeli rovnako, každý z~nich dodal polovicu zo
štvrtiny~-- teda osminu~-- celkovej sumy.

Erika dodala tri štvrtiny a~Ľuboš jednu osminu celkovej sumy.
Keďže tri štvrtiny sú to isté ako šesť osmín, vidíme, že Erika dodala do
spoločnej pokladničky šesťkrát viac eur ako Ľuboš.

\poznamky
Príspevok Ľuboša, Martina a~Eriky označíme postupne $l$, $m$ a~$e$, celkovú
sumu označíme $c=l+m+e$.
Pri tomto označení možno predchádzajúce riešenie zapísať nasledujúcim spôsobom:
$$
\aligned
l+m+\frac{e}3=\frac23 e&=\frac{c}2,\\
\frac{e}3&=\frac{c}4,
\quad\text{čiže}\quad
e=\frac34 c.
\endaligned
$$
Z~toho vyplýva, že
$$
l+m=\frac{c}4.
$$
Keďže $l=m$, dostávame
$$
2l=\frac{c}4,
\quad\text{čiže}\quad
l=\frac{c}8.
$$
Spolu teda vidíme, že
$$
e=\frac34 c=\frac68 c=6l.
$$

Pomocné grafické znázornenie je na nasledujúcom obrázku.
\insp{z7-I-1a.eps}%
}

{%%%%%   Z7-I-2
\napad
Určte osobitne počet čísel súmerných podľa vodorovnej, resp. podľa zvislej osi.

\riesenie
Jediné cifry od~2 po~9, ktoré sú súmerné podľa vodorovnej osi, sú
cifry 3 a~8.
Všetky nanajvýš trojciferné čísla zložené len z~týchto cifier tak
zodpovedajú zadaniu.
Takých čísel je celkom~14:
\begin{itemize}
\item 3, 8,
\item 33, 88, 38, 83
\item 333, 888, 338, 383, 388, 833, 838, 883.\insp{z7-I-2a.eps}%
\end{itemize}


Podľa zvislej osi je súmerná jedine cifra~8.
Navyše sú podľa zvislej osi navzájom súmerné cifry 2 a~5.
Z~týchto troch cifier možno zostaviť 7 nanajvýš trojciferných čísel vyhovujúcich
zadaniu:
\begin{itemize}
\item 8,
\item 88, 25, 52,
\item 888, 285, 582.
\end{itemize}

Čísla 8, 88 a~888 sú súmerné ako podľa zvislej, tak aj podľa vodorovné osi,
do celkového počtu čísel vyhovujúcich zadaniu ich teda počítame iba raz:
Počet nanajvýš trojciferných osovo súmerných čísel je
$$
14+7-3=18.
$$
}

{%%%%%   Z7-I-3
\napad
Vyjadrite obsahy jednotlivých jednofarebných plôch.

\riesenie
Obsah každého jednofarebného útvaru možno spočítať pomocou sčítania a~odčítania
obsahov vhodných pravouholníkov a~pravouhlých trojuholníkov.
\insp{z7-I-3a.eps}%

Pre ilustráciu uvádzame výpočet obsahu svetlosivého trojuholníka, jednotkou je obsah pomocného štvorca:
$$
3\cdot4-\frac{1\cdot4}2-\frac{2\cdot3}2-\frac{3\cdot1}2
=12-2-3-1{,}5=5{,}5.
$$

\noindent
Obsahy jednotlivých jednofarebných plôch sú:
\begin{itemize}
\item biela: $5+3{,}5=8{,}5$,
\item svetlosivá: $5{,}5+5=10{,}5$,
\item tmavosivá: $3+3=6$.
\end{itemize}
Celková cena kamienkov použitých na každej z~týchto troch plôch bola podľa zadania
rovnaká, najlacnejší materiál je teda svetlosivý a~najdrahší tmavosivý.

Pokrytie celej plochy, \tj. všetkých 25 pomocných štvorcov, kamienkami najlacnejšej svetlosivej
farby by stálo 1\,700€.
Jeden štvorec by potom stál
$$
1\,700:25=68\ (\text\euro).
$$
Plocha, ktorá bola naozaj pokrytá svetlosivo, teda stála
$$
10{,}5\cdot68=714\ (\text\euro).
$$

Podľa zadania na rovnakú sumu vyšiel aj materiál na pokrytie 6 štvorcov
kamienkami najdrahšej tmavosivej farby.
Jeden taký štvorec vyjde na
$$
714:6=119\ (\text\euro).
$$
Keby bolo týmito kamienkami pokryté celé dno bazénu, náklad na materiál by bol
$$
25\cdot119=2\,975\ (\text\euro).
$$

\poznamka
Záverečné počítanie v~uvedenom riešení je možné skrátiť nasledovne:

Podľa zadania možno za rovnakú sumu pokryť 6 tmavosivých a~10{,}5 svetlosivých
štvorcov, \tj. 1{,}75-krát viac svetlosivých štvorcov ako tmavosivých
($10{,}5:6=1{,}75$).
Tmavosivý materiál je teda 1{,}75-krát drahší ako svetlosivý.
Pokrytie celého dna tmavosivou farbou by preto stálo
$$
1,75\cdot1\,700=2\,975\ (\text\euro).
$$
}

{%%%%%   Z7-I-4
\napad
Pripomeňte si definíciu a~vlastnosti stredných priečok v~trojuholníku.

\riesenie
Najskôr podľa zadania určíme polohu bodov $N$, $O$, $P$ a~$Q$ vzhľadom na všeobecný trojuholník $KLM$, pozri obrázok.
\insp{z7-I-4a.eps}%

Body $N$ a~$O$ sú stredmi úsečiek $KM$ a~$KL$.
Preto je úsečka~$NO$ strednou priečkou v~trojuholníku $KLM$ rovnobežnou so stranou~$ML$.
Platí teda $|ML|=2|NO|$.

Bod~$M$ je stredom úsečky~$NP$ a~úsečka~$MQ$ je rovnobežná s~$NO$.
Preto je úsečka~$MQ$ strednou priečkou v~trojuholníku $NPO$ a~platí
$|NO|=2|MQ|$.

Spolu zisťujeme, že platí
$$
|ML|=2|NO|=4|MQ|,
$$
čiže pomer dĺžok úsečiek $MQ$ a~$ML$ je rovný
$$
|MQ|:|ML|=1:4.
$$
}

{%%%%%   Z7-I-5
\napad
Vžite sa do Janovej situácie a~rozhodnite, za ktorými dverami princezná byť
nemôže.

\riesenie
Podľa informácií od dobrej víly vieme, že nápis na dverách, za ktorými je
princezná, je pravdivý.
Preto princezná nemôže byť za dverami~II, a~je teda buď za dverami~I, alebo III.

Keby princezná bola za dverami~I, bola by pravda, že jaskyňa za dverami~III je
prázdna.
V~takom prípade by drak musel byť za dverami~II a~na týchto dverách by tak
mal byť nepravdivý nápis.
Nápis ale tvrdí, že princezná je za dverami~I, čo by však bola pravda.
Preto princezná nemôže byť ani za dverami~I.

Princezná je teda ukrytá v~priestore za dverami III.
Podľa nápisu na týchto dverách vieme, že drak je za dverami~II a~zodpovedajúci
nápis je naozaj nepravdivý.
Jaskyňa za dverami I~je prázdna a~nápis na oných dverách preto môže byť akýkoľvek.

Ak to Jano s~princeznou myslí naozaj vážne, mal by otvoriť dvere III.

\poznamka
Keď nevieme, ako začať, vždy je možné vypísať všetky možnosti rozmiestnení
draka a~princeznej do jednotlivých jaskýň (celkom 6~možností) a~v~každom
z~týchto prípadov skontrolovať, či nápisy na dverách súhlasia s~radou od
dobrej víly.
Ako jediný bezosporný prípad vyjde ten, ktorý sme práve odvodili.
}

{%%%%%   Z7-I-6
\napad
Čo možno povedať o~deliteľnosti čísel napísaných na jednotlivých kartičkách?

\riesenie
Väčšie číslo označíme $\overline{AB}$, menšie $\overline{cd}$.
Prvá podmienka v~zadaní hovorí, že:
\begin{itemize}
\item Číslo $\overline{ABcd}$ je deliteľné štyrmi, takže posledné dvojčíslie
$\overline{cd}$ je deliteľné štyrmi.
\item Číslo $\overline{ABcd}$ je deliteľné deviatimi, čo znamená, že ciferný súčet
$A+B+c+d$ je deliteľný deviatimi.
\end{itemize}
\noindent
Z~druhej podmienky vieme, že:
\begin{itemize}
\item Číslo $\overline{cdAB}$ je deliteľné piatimi, takže cifra $B$ je 0 alebo~5.
\item Číslo $\overline{cdAB}$ je deliteľné šiestimi, takže je súčasne deliteľné troma a~dvoma.
Aby bolo deliteľné dvoma, musí byť cifra $B$ párna, čo spolu s~predchádzajúcim
dôsledkom znamená, že $B=0$.
(Aby bolo $\overline{cdAB}$ deliteľné troma, musí byť ciferný súčet $c+d+A+B$
deliteľný troma. Z~prvej podmienky však vieme, že je tento súčet deliteľný
deviatimi, čo je silnejšia požiadavka.)
\end{itemize}

Spolu tak dostávame, že väčšie číslo $\overline{AB}$ je deliteľné desiatimi, menšie číslo
$\overline{cd}$ je deliteľné štyrmi a~ciferný súčet $A+B+c+d=A+c+d$ je deliteľný
deviatimi.

Predtým, ako začneme preverovať všetky možnosti vyhovujúce uvedeným podmienkam,
si môžeme všimnúť nasledujúce:
Najväčšie dvojciferné číslo $\overline{AB}$ deliteľné desiatimi je 90.
Najväčšie dvojciferné číslo $\overline{cd}$ deliteľné štyrmi a~menšie ako maximálne možné $\overline{AB}$ je 88.
Pre ľubovoľné $\overline{cd}$ platí, že nenulová cifra~$A$ taká, že súčet $A+c+d$ je deliteľný
deviatimi, je určená jednoznačne.
Navyše tento súčet môže byť nanajvýš $9+8+8=25$, takže aby bol deliteľný
deviatimi, musí byť buď 9, alebo~18.

Budeme teda postupne vypisovať všetky dvojciferné násobky štyroch, ktoré
sú menšie ako~90, a dosadzovať ich za $\overline{cd}$;
pre každé z~týchto čísel určíme číslo $\overline{AB}=\overline{A0}$ tak, aby súčet $A+c+d$ bol~9
alebo 18;
skontrolujeme, či je číslo $\overline{cd}$ menšie ako $\overline{AB}$:
\bgroup
\def\ctr#1{\hfil\ #1\ \hfil}
$$
\tablewidth=\hsize
\begintable
$\overline{cd}$\|12|16|20|24|28|32|36|40|44|48|52|56|60|64|68|72|76|80|84|88\cr
$\overline{AB}$\|60|20|70|30|80|40|90|50|10|60|20|70|30|80|40|90|50|10|60|20\crthick
$\overline{cd}\!<\!\overline{AB}$\|a|a|a|a|a|a|a|a|n|a|n|a|n|a|n|a|n|n|n|n\endtable
$$
\egroup
Matej mohol vyrobiť nanajvýš 12 rôznych dvojíc kartičiek.
}

{%%%%%   Z8-I-1
\napad
Najskôr určte, ktoré písmená sa v~hľadanom slove vyskytujú, resp.
nevyskytujú.

\riesenie
Najskôr určíme, ktoré písmená sa v~myslenom slove vyskytujú, potom
rozhodneme, na ktorých miestach.

Zo záznamu hry vidíme, že v~slove SONET sa vyskytujú tri a~v slove MUDRC dve
hľadané písmená.
Vzhľadom na to, že tieto dve slová nemajú žiadne písmeno spoločné, hľadaná
pätica písmen sa nachádza len medzi písmenami
$$
\text{S, O, N, E, T, M, U, D, R, C}.
$$
Preto dve z~hľadaných písmen, ktoré sú súčasne v~slove PLAST, sú práve
písmená S~a~T.
Z~rovnakého dôvodu sú štyri z~hľadaných písmen, ktoré sú súčasne v~slove KMOTR, práve písmená M, O, T a~R.
Hľadaná pätica písmen je
$$
\text{S, M, O, T, R}.
$$

Všetky ostatné písmená v~uvedenom zázname hry odstránime a~budeme uvažovať
o~poradí písmen v~hľadanom slove.
\bgroup
\def\p#1{\clap{#1}\phantom{M}}
\thinsize=0pt
\thicksize=0pt
\def\ctr#1{\quad#1\quad\hfil}
$$\begintable
\p{S}\p{O}\p{\_}\p{\_}\p{T}|1|2\cr
\p{M}\p{\_}\p{\_}\p{R}\p{\_}|0|2\cr
\p{\_}\p{\_}\p{\_}\p{S}\p{T}|0|2\cr
\p{\_}\p{M}\p{O}\p{T}\p{R}|0|4\cr
\p{\_}\p{T}\p{O}\p{\_}\p{\_}|1|1\cr
\p{\_}\p{O}\p{\_}\p{M}\p{\_}|0|2\endtable
$$
\egroup
V~piatom riadku sú iba dve písmená, z~ktorých jedno je na správnom mieste.
Písmeno~O to byť nemôže, pretože na rovnakom mieste sa vyskytuje aj v~riadku
predchádzajúcom, kde však na správnom mieste nie je žiadne z~napísaných písmen.
Preto je na správnom mieste písmeno~T a~môžeme začať dopĺňať hľadané slovo:
$$
\def\p#1{\clap{#1}\phantom{M}}
\text{\p{\_}\p{T}\p{\_}\p{\_}\p{\_}.}
$$
Jedno z~písmen v~prvom riadku je tiež na správnom mieste.
Písmeno~O to byť nemôže, pretože druhé miesto je už obsadené.
Písmeno~T to tiež byť nemôže, pretože na rovnakom mieste sa vyskytuje aj v~treťom
riadku, kde na správnom mieste nie je žiadne z~písmen.
Preto musí byť na správnom mieste písmeno~S:
$$
\def\p#1{\clap{#1}\phantom{M}}
\text{\p{S}\p{T}\p{\_}\p{\_}\p{\_}.}
$$
Z~druhého a~štvrtého riadku vieme, že písmeno~R nemôže byť na štvrtom ani na
piatom mieste.
Prvé dve miesta sú už obsadené, preto R musí byť na treťom mieste:
$$
\def\p#1{\clap{#1}\phantom{M}}
\text{\p{S}\p{T}\p{R}\p{\_}\p{\_}.}
$$
Z~posledného riadku vieme, že písmeno~M nemôže byť na štvrtom mieste.
Prvé tri miesta sú obsadené, preto je M na piatom mieste a~pre O ostáva miesto
štvrté.
Hľadané slovo je
$$
\text{STROM}.
$$
}

{%%%%%   Z8-I-2
\napad
Ako sa líšia všetky delitele pôvodného čísla a~jeho dvojnásobku?

\riesenie
Neznáme číslo označíme $N$.
Ľubovoľný deliteľ čísla $2 N$ má tvar $c\cdot d$, pričom $c$ je nejaký
deliteľ čísla 2 a~$d$ je nejaký deliteľ čísla $N$.
Jediné delitele čísla 2 sú 1 a~2,
delitele čísla $N$ sú 1, $N$, príp. ešte nejaké ďalšie čísla
$$
d_1,d_2,\dots,d_k,
$$
o~ktorých predpokladáme, že sú navzájom rôzne.
Delitele čísla $2N$ teda patria do množiny
$$
\{1,d_1,d_2,\dots,d_k,N, 2, 2d_1, 2d_2,\dots, 2d_k, 2N\} \eqno (1)
$$
a~podmienka zo zadania znamená
$$
1+d_1+d_2+\cdots+d_k+N=78.
$$

Keďže číslo $N$ je nepárne, musia byť nepárne aj všetky jeho delitele
$1$, $d_1$, $d_2$, \dots, $d_k$, $N$.
Naopak, všetky čísla $2, 2d_1, 2d_2,\dots, 2d_k, 2N$ sú párne, takže prvky
množiny (1) sú navzájom rôzne čísla.
Súčet všetkých deliteľov čísla $2N$ je preto rovný
$$
1+d_1+\cdots+d_k+N+2+2d_1+\cdots+2d_k+2N
=3\cdot(1+d_1+\cdots+d_k+N)
=3\cdot78
=234.
$$

\ineriesenie
Každé číslo je deliteľné sebou samým a~číslom 1.
Preto neznáme číslo, ktoré má súčet všetkých svojich deliteľov rovný 78, musí byť menšie ako 78.
Vyskúšaním všetkých nepárnych čísel od 1 do 77 zistíme, že túto vlastnosť má iba číslo~45.
Dvojnásobkom je číslo~90 a~jeho delitele sú:
$$
1,\ 2,\ 3,\ 5,\ 6,\ 9,\ 10,\ 15,\ 18,\ 30,\ 45,\ 90.
$$
Súčet týchto deliteľov je 234.
}

{%%%%%   Z8-I-3
\napad
Aký je súčet veľkostí uhlov $KNM$ a~$KLM$?

\riesenie
Zo zadania vyplýva, že lichobežník $KLMN$ je rovnoramenný a~navyše je
uhlopriečkou~$KM$ rozdelený na dva rovnoramenné trojuholníky.
Súčet veľkostí vnútorných uhlov v~každom z~týchto trojuholníkov je rovný
$180\st$;
súčet veľkostí vnútorných uhlov v~danom lichobežníku je rovný
$2\cdot180\st=360\st$.
Veľkosť hľadaného uhla $KNM$ označíme~$\alpha$.
Pomocou neznámej $\alpha$ budeme vyjadrovať veľkosti ostatných uhlov v~lichobežníku, kým nebudeme schopní túto neznámu určiť.

Keďže lichobežník $KLMN$ je rovnoramenný, sú vnútorné uhly pri vrcholoch $N$ a~$M$, resp. $K$ a~$L$ zhodné, \tj.
$$
\alpha=|\uhel KNM|=|\uhel LMN|,
\quad\text{resp.}\quad
|\uhel LKN|=|\uhel KLM|.
$$
Súčet veľkostí všetkých týchto uhlov je $360\st$, preto je súčet ktorýchkoľvek
dvoch nezhodných uhlov rovný $180\st$, teda
$$
|\uhel LKN|=|\uhel KLM|=180\st-\alpha.
$$
Keďže trojuholník $KLM$ je rovnoramenný, sú vnútorné uhly pri základni
zhodné, tzn.
$$
|\uhel KML|=|\uhel KLM|=180\st-\alpha.
$$
Uhol $KMN$ je rozdielom uhlov $LMN$ a~$KML$, platí teda
$$
|\uhel KMN|=\alpha-(180\st-\alpha)=2\alpha-180\st.
$$
Keďže trojuholník $KMN$ je rovnoramenný, sú vnútorné uhly pri základni
zhodné, teda
$$
|\uhel MKN|=|\uhel KMN|=2\alpha-180\st.
$$
\insp{z8-I-3a.eps}%

Súčet veľkostí vnútorných uhlov v~trojuholníku $KMN$ je $180\st$, dostávame
tak rovnicu s~neznámou $\alpha$, ktorú doriešime:
$$
\aligned
5\alpha-360\st&=180\st,\\
5\alpha&=540\st,\\
\alpha&=108\st.
\endaligned
$$
Veľkosť uhla $KNM$ je $108\st$.

\poznamka
Pri označení ako na nasledujúcom obrázku môžeme všetky vyššie uvedené
podmienky sformulovať takto:
$$
\alpha+2\gamma=\delta+2\beta=180\st,
\quad
\beta+\gamma=\alpha,
\quad
\gamma+\delta=\beta.
$$
\insp{z8-I-3b.eps}%

S~týmito podmienkami môžeme pracovať veľmi rôznorodo a~niekedy môže byť výhodné
vyjadriť veľkosti niektorých ďalších uhlov.
Pre kontrolu uvádzame veľkosti všetkých vyznačených uhlov:
$$
\alpha=108\st,
\quad
\beta=72\st,
\quad
\gamma=\delta=36\st.
$$

\ineriesenie
Lichobežník $KLMN$ sa dá doplniť na päťuholník $KPLMN$ tak, aby
$$
|LP|=|PK|=|KN|=|NM|=|ML|.
$$
Zo zadania vyplýva, že päťuholník $KPLMN$ je pravidelný.
(Keďže $|KL|=|KM|$, sú trojuholníky $LPK$ a~$KNM$ zhodné.
Lichobežníky $KLMN$ a~$KMLP$ sú teda tiež zhodné, a~preto sú
všetky vnútorné uhly v~päťuholníku $KPLMN$ rovnaké.)
\insp{z8-I-3c.eps}%

Uhlopriečky $KM$ a~$KL$ delia tento päťuholník na tri trojuholníky.
Súčet veľkostí vnútorných uhlov v~každom z~týchto trojuholníkov je rovný
$180\st$;
súčet veľkostí vnútorných uhlov v~päťuholníku $KPLMN$ je rovný
$3\cdot180\st=540\st$.
Keďže sú všetky vnútorné uhly zhodné, má každý z~nich veľkosť
$$
\frac{540\st}{5}=108\st.
$$
Veľkosť uhla $KNM$ je $108\st$.

\poznamka
Obe uvedené riešenia boli založené na vete o~súčte vnútorných uhlov v~trojuholníku.
Z~toho sme určili súčet vnútorných uhlov v~špecifickom štvoruholníku, resp.
päťuholníku.
Všeobecne platí, že
súčet veľkostí vnútorných uhlov v~ľubovoľnom $n$-uholníku je rovný
$(n-2)\cdot180\st$.
}

{%%%%%   Z8-I-4
\napad
Najskôr určte pomer počtu veľkých čiernych a~malých čiernych guľôčok.

\riesenie
Pomer počtu veľkých čiernych a~veľkých bielych guľôčok je $1:2$.
To znamená, že medzi veľkými guľôčkami tvoria 1~diel čierne a~2~diely biele
guľôčky.
Z~toho o.\,i. vyplýva, že počet všetkých veľkých guľôčok je násobkom~3.

Pomer počtu malých čiernych a~malých bielych guľôčok je $1:8$.
To znamená, že medzi malými guľôčkami tvoria 1~diel čierne a~8~dielov biele
guľôčky; počet všetkých malých guľôčok je teda násobkom~9.

Keby boli diely, ktorými sme porovnávali farebné guľôčky medzi veľkými a~malými,
rovnaké, bol by pomer počtu všetkých veľkých a~všetkých malých guľôčok rovný
$3:9=1:3$.
Aby bol tento pomer $5:3$, musí byť jeden diel, ktorý sme používali pre veľké
guľôčky, päťkrát väčší ako diel, ktorý sme používali pre malé guľôčky.
Inými slovami, veľkých čiernych guľôčok musí byť päťkrát viac ako malých
čiernych guľôčok.

V~závislosti od počtu malých čiernych guľôčok (1~diel) sú všetky ostatné počty
vyjadrené v~nasledujúcej tabuľke:
\bgroup
\def\ctr#1{\quad#1\quad\hfil}
$$
\begintable
guľôčky\|veľké|malé\|celkom\crthick
čierne\|\ 5 dielov|\ 1 diel\|\ 6 dielov\cr
biele\|10 dielov|\ 8 dielov\|18 dielov\crthick
celkom\|15 dielov|\ 9 dielov\|
\endtable
$$
Pomer počtu všetkých čiernych a~všetkých bielych guľôčok je teda rovný
$$
6:18=1:3.
$$

\poznamka
Každý pomer môžeme podľa potreby rozšíriť.
Pre ľubovoľné prirodzené čísla $a$, $b$, $c$ napr. platí
$$
\frac12=\frac{a}{2a},
\quad
\frac18=\frac{b}{8b},
\quad
\frac53=\frac{5c}{3c}.
$$
Čísla $a$, $b$, $c$ predstavujú práve diely, pomocou ktorých sme porovnávali
jednotlivé typy guľôčok v~predchádzajúcom riešení:
$$
\begintable
guľôčky\|veľké|malé\|celkom\crthick
čierne\|\hfil$a$|\hfil$b$|\hfil$a+b$\cr
biele\|\hfil$2a$|\hfil$8b$|\hfil$2a+8b$\crthick
celkom\|\hfil$5c$|\hfil$3c$|
\endtable
$$
\egroup
V~tomto duchu môžeme uvedené riešenie formulovať tak, že hľadáme čísla
$a$, $b$, $c$, aby platilo
$$
3a=5c,\quad 9b=3c.
$$
Z~toho vidíme, že $c$ musí byť nutne násobkom troch, čiže $c=3d$ (pričom
$d$ je akékoľvek prirodzené číslo).
Zvyšné dve čísla sú potom rovné
$$
a=5d,\quad b=d.
$$
Po doplnení do tabuľky dostávame taký istý záver ako vyššie.
}

{%%%%%   Z8-I-5
\napad
Vyjadrite informácie zo zadania pomocou počtu žiakov v~triede a~súčtu
ich známok z~matematiky.

\riesenie
Priemernú známku si môžeme predstaviť tak, že každý žiak získal práve túto známku
(aj keď známka 2{,}45 alebo 2{,}5 sa samozrejme nedáva).
Počet všetkých detí v~8.A označíme~$n$.
Z~toho dostávame, že súčet všetkých známok je jednak $2{,}45n$,
jednak $1+3+2{,}5(n-2)$.
Riešime tak rovnicu:
$$
\aligned
2{,}45n&=4+2{,}5n-5,\\
0{,}05n&=1,\\
n&=20.
\endaligned
$$
Trieda 8.A~má 20 žiakov.

\poznamka
Ak súčet všetkých známok označíme~$z$, tak podľa zadania platí
$$
\frac{z}n=2{,}45
\quad\text{a}\quad
\frac{z-4}{n-2}=2{,}5,
$$
čiže
$$
z=2{,}45n
\quad\text{a}\quad
z=2{,}5(n-2)+4.
$$
Porovnaním dvojakého vyjadrenia toho istého čísla~$z$ dostaneme rovnicu ako vyššie.
}

{%%%%%   Z8-I-6
\napad
Vyjadrite počty postupne zjedených kociek v~závislosti od rozmerov darovaného
kvádra.

\riesenie
Rozmery kvádra (v~kockách cukru) označíme $x$, $y$ a~$z$ tak, aby predná
stena mala rozmery $x\times z$, bočná stena $y\times z$ a~horná stena
$x\times y$.

\begin{itemize}
\item Prvý deň Pejko odjedol jednu vrstvu spredu, zjedol teda $x\cdot z$ kociek a~rozmery kvádra potom boli $x\times(y-1)\times z$.
\item Druhý deň odjedol jednu vrstvu sprava, zjedol teda $(y-1)\cdot z$ kociek a~rozmery kvádra sa
zmenšili na $(x-1)\times(y-1)\times z$.
\item Tretí deň odjedol jednu vrstvu zhora, zjedol $(x-1)\cdot(y-1)$ kociek.
\end{itemize}

Každý deň Pejko zjedol rovnaký počet kociek, platí teda
$$
x\cdot z=(y-1)\cdot z=(x-1)\cdot(y-1).
$$
Keďže $x$, $y$, $z$ sú prirodzené čísla (a~teda $z\ne0$),
z~prvej rovnosti dostávame
$$
x=y-1.
$$
Z~toho vidíme, že $y-1\ne0$, teda z~druhej rovnosti dostávame
$$
z=x-1=y-2.
$$

Rozmery pôvodného kvádra teda boli $(y-1)\times y\times(y-2)$ a~podľa zadania
má platiť
$$
1\,000\le(y-1)\cdot y\cdot(y-2)\le2\,000.
$$
Postupným skúšaním sa rýchlo presvedčíme, že jediné riešenia vyhovujúce
týmto dvom nerovnostiam zodpovedajú buď hodnote $y=12$, alebo $y=13$.
Darovaný kváder mohol mať nasledujúci počet kociek:
$$
11\cdot12\cdot10=1\,320
\quad\text{alebo}\quad
12\cdot13\cdot11=1\,716.
$$
}

{%%%%%   Z9-I-1
\napad
Uvažujte od konca: koľko si vzal posledný chlapec a~koľko predposledný?

\riesenie
Poradové číslo posledného chlapca označme~$n$.
Tento chlapec si vzal $n$~orechov a~desatinu vzniknutého zvyšku a~potom už nič nezostalo.
Nulový teda musel byť už zvyšok po odobratí $n$~orechov.

Predposledný chlapec, s~poradovým číslom $n-1$, si vzal $n-1$ orechov a~desatinu vzniknutého zvyšku.
Aby mal tiež $n$~orechov ako posledný chlapec, musí byť táto desatina zvyšku rovná práve 1~orechu.
Spomenutý zvyšok je teda 10~orechov.

Na posledného $n$-tého chlapca ostalo z~týchto 10~orechov~9, neznáma~$n$ je
teda~9.
Chlapcov bolo 9 a~každý si vzal rovnako ako posledný z~nich, \tj. 9~orechov.
Celkový počet orechov bol $9\cdot9=81$.
Pre kontrolu uvádzame, ako si chlapci orechy postupne rozoberali:
\begin{itemize}
\item 1. chlapec: $1+(81-1):10=1+8=9$, ostane 72,
\item 2. chlapec: $2+(72-2):10=2+7=9$, ostane 63,
\item 3. chlapec: $3+(63-3):10=3+6=9$, ostane 54,
\item atď.,
\item 8. chlapec: $8+(18-8):10=8+1=9$, ostane 9,
\item 9. chlapec: 9.
\end{itemize}
Milena nazbierala 81~orechov, o~ktoré sa delilo 9~chlapcov.

\inynapad
Uvažujte od začiatku: koľko mohlo, príp. nemohlo byť na začiatku orechov?

\ineriesenie
Prvý chlapec si vzal 1~orech a~desatinu zvyšku, čo znamená, že celkový
počet orechov bol
$$
10x+1,
$$
pričom $x$ je neznáme prirodzené číslo.
Pri tomto označení si prvý chlapec vzal $1+x$ orechov a~nový zvyšok bol
$10x+1-1-x=9x$.

Druhý chlapec si vzal 2~orechy a~desatinu nového zvyšku, čo znamená, že
tento zvyšok bol
$$
9x=10y+2,\eqno (1)
$$
pričom $y$ je neznáme prirodzené číslo.
Pri tomto označení si druhý chlapec vzal $2+y$ orechov,
nový zvyšok bol $10y+2-2-y=9y$ atď.

Keďže si prvý a~druhý chlapec odobrali rovnaký počet orechov, platí
$$
1+x=2+y,
\quad\text{čiže}\quad
y=x-1.
$$
Dosadením do rovnice (1) a~jednoduchou úpravou získame $x$:
$$
\aligned
9x&=10x-10+2,\\
x&=8.
\endaligned
$$
Celkový počet orechov bol $80+1=81$ a~každý chlapec si vzal $1+8=9$ orechov.
Kontrolu u~všetkých chlapcov urobíme rovnako ako v~predchádzajúcej časti.
}

{%%%%%   Z9-I-2
\napad
Rozdeľte vyhovujúce čísla do skupín podľa typu uvažovanej súmernosti.

\riesenie
Čísla vyhovujúce podmienkam zo zadania rozdelíme do troch skupín podľa
toho, či uvažujeme súmernosť podľa vodorovnej osi, súmernosť podľa zvislej
osi alebo súmernosť stredovú.

1)
Jediné cifry, ktoré sú súmerné podľa vodorovnej osi alebo ktorých
obraz vzhľadom na takú súmernosť je tiež cifrou, sú cifry 3, 8, 2
a~5:
\insp{z9-I-2a.eps}%

Určíme, koľko je všetkých nanajvýš trojciferných čísel zostavených len z~týchto cifier:
Jednociferné čísla sú 4.
Dvojciferných čísel je $4\cdot4=16$ (pred každú z~cifier 2, 3, 5,~8
môžeme pripísať akúkoľvek z~týchto štyroch cifier).
Trojciferných čísel je $4\cdot16=64$ (pred každé dvojciferné číslo,
ktorých sme napočítali 16, môžeme pripísať ľubovoľnú z~uvažovaných štyroch
cifier).
Celkový počet čísel v~tejto skupine je
$$
4+16+64=84.
$$

2)
Jediné cifry, ktoré sú súmerné podľa zvislej osi alebo ktorých
obraz vzhľadom na takú súmernosť je tiež cifrou, sú cifry 8, 2
a~5:
\insp{z9-I-2b.eps}%

Počet všetkých nanajvýš trojciferných čísel zostavených len z~týchto cifier
určíme podobne ako v~predchádzajúcej časti~--
najskôr počet jednociferných čísel, potom dvojciferných a~nakoniec
trojciferných:
$$
3+3\cdot3+3\cdot3\cdot3=3+9+27=39.
$$


3)
Jediné cifry, ktoré sú stredovo súmerné alebo ktorých
obraz vzhľadom na takú súmernosť je tiež cifrou, sú cifry 2, 5, 8,
6 a~9:
\insp{z9-I-2c.eps}%

Celkový počet čísel v~tejto skupine určíme rovnako ako vyššie:
$$
5+5\cdot5+5\cdot5\cdot5=5+25+125=155.
$$

Všetky čísla zo skupiny~2) sú zahrnuté ako v~skupine~1), tak v~skupine~3).
Okrem týchto čísel nemajú tieto tri skupiny žiadne ďalšie spoločné prvky.
Celkový počet čísel vyhovujúcich zadaniu teda dostaneme tak, že k~počtu čísel zo skupiny~1) pričítame počet čísel skupiny~3) zmenšený o~počet
čísel skupiny~2):
$$
84+155-39=200.
$$

\poznamky
Pre ilustráciu uvádzame možné obrazy čísla 825, ktoré patria do všetkých
diskutovaných skupín:
\insp{z9-I-2d.eps}%

Pokiaľ by sme uvažovali súmernosti vzhľadom k~všeobecným osiam, resp. stredom,
budú výsledné obrazy iba pootočením, resp. posunutím niektorých obrazov
uvažovaných vyššie.
Na počet čísel vyhovujúcich zadaniu nemá tento všeobecnejší predpoklad
žiadny vplyv.
\insp{z9-I-2e.eps}%
}

{%%%%%   Z9-I-3
\napad
Rozložte krabicu do roviny.

\riesenie
Špagát vždy obopína krabicu tak, aby mal čo najmenšiu možnú dĺžku.
Preto po rozložení krabice do roviny tvoria stopy po špagáte časti priamok.
Ďalej si uvedomme, že krabica je previazaná jedným kusom špagátu,
a~preto musí byť aspoň v~jednom bode kríženia špagát pretočený, aby zmenil
smer.

1)
Najskôr predpokladajme, že všetky body kríženia sú v~stredoch bočných
stien.
V~takom prípade môže rozložená krabica s~vyznačenými stopami po špagáte
vyzerať takto:
\insp{z9-I-3a.eps}%

Stopy v~navzájom zhodných stenách sú navzájom zhodné,
navyše stopy v~jednotlivých stenách pozostávajú z~navzájom zhodných
úsečiek.
Preto si môžeme ich merania uľahčiť vhodným premiestnením ich častí.
Súčet dĺžok silno vyznačených úsečiek na predchádzajúcom obrázku je rovnaký ako
na obrázku nasledujúcom:
\insp{z9-I-3b.eps}%

Tento útvar je zložený z~ôsmich zhodných úsečiek, ktorých súčet dĺžok je
rovný
$$
8\cdot|AB|.
$$
Úsečka~$AB$ je preponou v~pravouhlom trojuholníku s~odvesnami dĺžok
$20+3=23$\,(cm) a~$15+3=18$\,(cm).
Podľa Pytagorovej vety platí
$$
|AB|=\sqrt{23^2+18^2}=\sqrt{853}\doteq29{,}2\,(\Cm).
$$
Na previazanie krabice týmto spôsobom teda postačí špagát s~dĺžkou
$$
8\cdot\sqrt{853}+20\doteq253{,}6\,(\Cm).
$$

2)
Teraz predpokladajme, že body kríženia špagátu nie sú v~stredoch bočných
stien, avšak sú naďalej rovnako ďaleko od hornej aj dolnej steny.
V~tomto prípade nie sú stopy po špagáte tak súmerné ako vyššie, naďalej však
platí, že tieto stopy v~hornej a~dolnej stene sú rovnaké
(presnejšie povedané, v~priemete kolmom na tieto steny stopy po špagáte splývajú).
Po rozložení krabice do roviny a~prípadnom preložení môže meraný útvar
vyzerať napr.~takto:
\insp{z9-I-3c.eps}%

Z~uvedeného vyplýva, že štvoruholníky $ABCD$ a~$AB'C'D'$ sú osovo
súmerné, a~preto sú vyznačené uhly v~blízkosti vrcholu~$A$ zhodné.
Z~rovnakého dôvodu sú zhodné aj zodpovedajúce uhly v~blízkosti každého
ďalšieho vrcholu.
Z~toho vyplýva, že štvoruholníky $ABCD$ a~$AB'C'D'$ sú rovnobežníky,
ktorých strany sú rovnobežné s~uhlopriečkami obdĺžnikov, ktoré predstavujú
hornú a~dolnú stenu krabice.

Vyznačený útvar teda pozostáva z~dvoch štvoríc navzájom zhodných úsečiek,
ktorých súčet dĺžok je rovný
$$
4\cdot(|BA|+|AD|)=
4\cdot(|BA|+|AD'|)=
4\cdot|BD'|.
$$
Úsečka~$BD'$ je však zhodná s~analogickou úsečkou na obrázku v~predchádzajúcom
prípade.
Jej dĺžka je teda taká istá a~dĺžka špagátu potrebného na previazanie krabice
sa nezmení.

3)
Vo všeobecnom prípade neplatí, že stopy po špagáte v~hornej a~dolnej stene sú
rovnaké.
Naďalej však platí, že tieto stopy sú rovnobežné s~uhlopriečkami hornej a~dolnej steny.
Po rozložení krabice do roviny a~prípadnom preložení môže meraný útvar
vyzerať nasledovne:
\insp{z9-I-3d.eps}%

V~tomto prípade rovnobežníky $ABCD$ a~$AB'C'D'$ nie sú zhodné.
Vyznačený útvar pozostáva zo štyroch dvojíc navzájom zhodných úsečiek, ktorých
súčet dĺžok je rovný
$$
2\cdot(|BA|+|AD|+|B'A|+|AD'|)=
2\cdot(|BD'|+|DB'|).
$$
Ako úsečka $BD'$, tak úsečka $DB'$ sú však zhodné s~analogickými úsečkami
na obrázkoch v~predchádzajúcich prípadoch.
Ich dĺžky sú teda také isté a~dĺžka špagátu potrebného na~previazanie krabice
sa ani v~tomto prípade nezmení.

\poznamka
Špagát na obrázku v~zadaní úlohy je znázornený približne ako v~prípade~1).
Tento predpoklad však nie je v~texte priamo uvedený, preto
by sa s~ním pracovať nemalo.
Riešenie zaoberajúce sa iba týmto prípadom hodnoťte nanajvýš stupňom "dobre".
}

{%%%%%   Z9-I-4
\napad
Určte, ako sa líšili celkové bodové výsledky niektorých dvoch chlapcov.

\riesenie
Najlepší výsledok mal Peter, stredný Martin a~najhorší Juro.
Výsledky každých dvoch chlapcov sa líšili v~jedinom hode,
preto musel Peter v~tomto hode trafiť políčko s~hodnotou~30, Martin políčko s~hodnotou~18 a~Juro~12.
Odtiaľ vidíme, že Petrov celkový výsledok bol o~12~bodov väčší ako celkový
súčet Martinov a~ten bol o~6~bodov väčší ako súčet Jurov.

Súčasne zo zadania vieme, že Petrov priemerný výsledok bol o~2~body lepší ako
Martinov a~ten bol o~1~bod lepší ako priemer Jurov.
Z~toho vyplýva, že každý z~chlapcov hádzal 6-krát ($12:6=2$, resp. $6:6=1$).

\poznamka
Hľadaný počet hodov označíme $n$ a~celkový bodový výsledok
Jura, Martina a~Petra označíme postupne $J$, $M$ a~$P$.
Pri tomto označení možno predchádzajúce riešenie zapísať nasledovne:
$$
\aligned
P=M+12,
&\quad\text{resp.}\quad
M=J+6,\\
\frac{P}n=\frac{M}n+2,
&\quad\text{resp.}\quad
\frac{M}n=\frac{J}n+1.
\endaligned
$$
Z~toho vyplýva, že $12:n=2$, resp. $6:n=1$, a~teda $n=6$.
}

{%%%%%   Z9-I-5
\napad
Určte pomer dĺžky skrátených nohavíc vzhľadom k~Jarovej výške.

\riesenie
Pôvodný pomer dĺžky nohavíc a~výšky Jara sa po skrátení nohavíc zmenšil
z~$5:8$ na
$$
\frac58-\frac4{100}\cdot\frac58=\frac{96}{100}\cdot\frac58=\frac{12}{20}=\frac35.
$$
Výšku Jara a~dĺžku nohavíc (v~cm) označíme postupne $j$ a~$n$.
Podľa zadania teda platí
$$
\frac{n}{j}=\frac58 \quad\text{a}\quad \frac{n-4}j=\frac35. \eqno (1)
$$
Z~každej rovnice môžeme vyjadriť neznámu~$n$:
$$
n=\frac58j
\quad\text{a}\quad
n=\frac35j+4.
$$
Z~toho zostavíme novú rovnicu s~neznámou~$j$, ktorú doriešime:
$$
\aligned
\frac58j&=\frac35j+4,\\
25j&=24j+160,\\
j&=160.
\endaligned
$$
Jaro je vysoký 160\,cm.

\poznamka
Rovnice v~(1) tvoria sústavu dvoch rovníc o~dvoch neznámych, ktorá je
ekvivalentná so sústavou
$$
\aligned
8n-5j&=0,\\
5n-3j&=20.\\
\endaligned
$$
}

{%%%%%   Z9-I-6
\napad
Vyjadrite súčin z~druhej podmienky pomocou dvoch najmenších prvočísel.

\riesenie
Uvedené prvočísla označíme vzostupne $a$, $b$, $c$.
Zo zadania vieme, že ${b-a}=\frac12(c-b)$, čiže
$$
c-b=2(b-a),\eqno (1)
$$
a~že číslo $(b-a)(c-b)$ je násobkom~17.
Z~toho vyplýva, že číslo
$$
2(b-a)^2\eqno (2)
$$
je násobkom 17.
Keďže 17 nie je násobkom 2, je táto podmienka splnená práve vtedy, keď
rozdiel $b-a$ je násobkom~17.

Aby neznáme číslo bolo najmenšie možné, musia byť aj prvočísla $a$, $b$, $c$ najmenšie
možné.
Najmenšie prvočíslo je~2.
Ak by bolo $a=2$, tak najmenšie prvočíslo~$b$, pre ktoré je rozdiel
$b-a$ násobkom~17, by bolo $b=19$.
Číslo~$c$ je potom jednoznačne určené rovnosťou~(1):
$$
c=2\cdot 17+19=53,
$$
a~to je tiež prvočíslo.
Teda najmenšie číslo s~vyššie uvedenými vlastnosťami je
$$
2\cdot19\cdot53=2\,014.
$$
}

{%%%%%   Z4-II-1
...}

{%%%%%   Z4-II-2
...}

{%%%%%   Z4-II-3
...}

{%%%%%   Z5-II-1
Za 5 paličiek je 5 klobúkov.
Za 1 paličku je teda 1 klobúk, a~preto 5~paličiek a~1~klobúk majú rovnakú hodnotu
ako 6 paličiek.

Ďalej vieme, že za 4 paličky je 6 plášťov.
Za 2 paličky sú teda 3 plášte, a~preto 6~paličiek má rovnakú hodnotu ako 9 plášťov.

Celkom teda vidíme, že 5 paličiek a~1 klobúk majú rovnakú hodnotu ako 9 plášťov.

\hodnotenie
2~body za odvodenie vzťahu 5~paličiek + 1~klobúk = 6~paličiek;
3~body za odvodenie vzťahu 6~paličiek = 9~plášťov;
1~bod za odpoveď.
\endhodnotenie
}

{%%%%%   Z5-II-2
Z~vytvoreného útvaru vidíme, že dlhšia strana každého z~troch Jurových
obdĺžnikov je rovnako dlhá ako dve strany kratšie.
Vysoký obdĺžnik na obrázku má obvod zložený z~troch dlhších a~štyroch kratších strán
malých obdĺžnikov.
Keďže súčet troch dlhších strán je rovnako veľký ako súčet šiestich kratších,
je obvod veľkého obdĺžnika rovný $6+4=10$ kratším stranám malého obdĺžnika.
Keďže vysoký obdĺžnik má obvod $20\cm$, je kratšia strana malého obdĺžnika
rovná $20:10=2$\,(cm).
Malé obdĺžniky teda majú rozmery $2\cm\times4\cm$.

Dané tri obdĺžniky sa dajú k~sebe podľa uvedených pravidiel priložiť ešte
nasledujúcimi spôsobmi:
\insp{z5-II-2a.eps}%

\ite a)
V~prvom prípade je obvod veľkého obdĺžnika rovný súčtu 6 kratších a~2
dlhších strán malého obdĺžnika, \tj. $6\cdot2+2\cdot4=20$\,(cm).
\ite b)
V druhom prípade je obvod veľkého obdĺžnika rovný súčtu 6 dlhších a~2
kratších strán malého obdĺžnika, \tj. $6\cdot4+2\cdot2=28$\,(cm).

\noindent
V~prvom prípade je obvod rovnaký ako obvod obdĺžnika zo zadania, v druhom
prípade je iný.
Novovzniknutý Jurov obdĺžnik mal obvod $28\cm$.

\hodnotenie
2~body za objav, že malý obdĺžnik má jeden rozmer dvakrát väčší ako druhý;
2~body za vyčíslenie rozmerov malého obdĺžnika;
1~bod za obdĺžnik s~iným obvodom;
1~bod za jeho obvod.

Diskusia možnosti a) nie je povinnou súčasťou riešenia.
\endhodnotenie
}

{%%%%%   Z5-II-3
Novovzniknuté štvorciferné číslo je buď typu $2\,{*}15$, alebo typu $2\,1{*}5$.
V~úlohe vytvorené čísla mohli byť buď a) obe prvého typu, alebo b) obe druhého typu, alebo c) každé iného typu.

a) V~tomto prípade končí súčet $2\,{*}15+2\,{*}15$ číslom $30$, čo je v~rozpore so
zadaným súčtom $4\,360$.
V~úlohe vytvorené čísla preto nemohli byť obe typu $2\,{*}15$.

b) Tento prípad riešime ako algebrogram:
$$
\alggg{2&1&*&5\\2&1&*&5}{4&3&6&0}
$$
V stĺpci stoviek si všimneme, že po sčítaní $1+1$ je vo výsledku zapísaná
cifra~$3$, v~stĺpci desiatok preto určite došlo k~prechodu cez desiatku.
V~stĺpci jednotiek sčítame $5+5=10$ a~vo výsledku je na mieste desiatok
zapísaná cifra~$6$.
Preto miesta označené hviezdičkami ukrývajú cifry so súčtom $15$.
Môže sa jednať buď o~súčet $9+6$, alebo $8+7$.
Tento algebrogram má teda dve riešenia.

c) Tento prípad riešime ako algebrogram:
$$
\alggg{2&*&1&5\\2&1&*&5}{4&3&6&0}
$$
Podobnými úvahami ako vyššie zisťujeme, že tento algebrogram má jediné
riešenie: na miesto desiatok patrí cifra $4$, na miesto stoviek $2$.

\smallskip
Úloha má celkom tri riešenia:
hľadanou dvojicou čísel sú buď čísla $2\,195$ a~$2\,165$, alebo čísla $2\,185$ a~$2\,175$,
alebo čísla $2\,215$ a~$2\,145$.

\hodnotenie
3~body za riešenie $2\,195+2\,165$ a~$2\,185+2\,175$ (z~toho po 1~bode za každé riešenie a~1~bod za komentár);
2~body za riešenie $2\,215+2\,145$;
1~bod za vylúčenie možnosti $2\,*15+2\,*15$.
\endhodnotenie
}

{%%%%%   Z6-II-1
Fabián písal iba prirodzené čísla.
Z~prvej podmienky vyplýva, že keď vynásobíme zelené číslo žltým, nezmeníme
jeho hodnotu.
To znamená, že žlté číslo je~1.
Z~druhej podmienky vyplýva, že modrou a~červenou napísal rovnaké číslo.
Z~tretej podmienky potom vyplýva, že keď vynásobíme dve červené (resp. dve
modré) čísla, dostaneme ako súčin dvojciferné číslo končiace jednotkou
(žlté číslo je 1).
Prejdeme teda všetky možnosti:
$$
\begin{gathered}
1\cdot1=1,\quad
2\cdot2=4,\quad
3\cdot3=9,\quad
4\cdot4=16,\\
5\cdot5=25,\quad
6\cdot6=36,\quad
7\cdot7=49,\quad
8\cdot8=64,\quad
9\cdot9=81.
\end{gathered}
$$
Uvedenej podmienke vyhovuje jediná, a~to posledná možnosť.
Fabián napísal červenou číslo~9, modrou číslo~9, zelenou číslo~8 a~žltou číslo~1.

\hodnotenie
2~body za určenie žltého čísla, z~toho 1 bod za zdôvodnenie;
3~body za prešetrenie všetkých možností a~za nájdenie súčinu $9\cdot9=81$;
1~bod za správne určenie zvyšných troch čísel.

Ak riešiteľ neuvedie všetky možnosti súčinu (má iba výsledný súčin
$9\cdot9=81$), nezdôvodní, že viac riešení nie je, a~ani nevysvetlí, že prešiel
všetky možnosti a~iné riešenie nenašiel, dajte nanajvýš 4~body.

Priradenie farieb jednotlivým číslam nie je nutnou súčasťou riešenia, preto ho
nijako bodovo nepostihujte.
\endhodnotenie
}

{%%%%%   Z6-II-2
Najviac cukríkov môže Juraj dostať, keď každou cestičkou prejde práve raz
a~na stanoviskách, ktorými prechádza najčastejšie, sa rozdáva najviac cukríkov.
Prejsť bludiskom uvedeným spôsobom je možné, pozri napr. cestu
$$
\text{A---B---C---D---E---C---A---E---F---A}
$$
podľa označenia ako na obrázku.
\insp{z6-II-2a.eps}%

Existujú aj iné možné cesty, ktoré však nie je nutné rozoberať.
Podstatné je, že vieme určiť, koľkokrát sa dá do každého stanoviska vstúpiť.
To nezávisí od zvolenej cesty, ale iba od počtu cestičiek, ktoré do daného
stanoviska vedú.
Do stanovísk B, F a~D vedú dve cestičky, do stanovísk A, C a~E vedú štyri
cestičky.
Pritom do stanoviska~A sa vstupuje zvonku, ostatnými stanoviskami sa dá iba
prejsť (tzn. koľkokrát sa vstúpi, toľkokrát sa vystúpi).
Platí teda, že
\begin{itemize}
\item do stanoviska~A môže Juraj vstúpiť trikrát (raz zvonku, raz
prechádza a~naposledy tu končí),
\item do každého zo stanovísk C a~E môže Juraj vstúpiť dvakrát (\tj.~dvakrát prechádza),
\item do každého zo stanovísk B, D a~F môže Juraj vstúpiť iba raz
(\tj.~raz prechádza).
\end{itemize}
Juraj teda mohol získať najviac cukríkov, ak by sa na stanovisku~A
rozdávalo 5~cukríkov, na stanoviskách C a~E po 3~cukríkoch a~na zvyšných
stanoviskách po 1~cukríku.
V~takom prípade by Juraj získal na jednom stanovisku trikrát po 5~cukríkoch,
na dvoch stanoviskách dvakrát po 3~cukríkoch a na troch stanoviskách po 1~cukríku, \tj.
$$
3\cdot5+2\cdot3+2\cdot3+3\cdot1=30
$$
cukríkov.

\hodnotenie
2~body za určenie nejakej cesty, ktorá prechádza každou cestičkou práve raz;
2~body za vyčíslenie navštevovanosti jednotlivých stanovísk;
1~bod za priradenie počtu rozdávaných cukríkov jednotlivým stanoviskám;
1~bod za zodpovedajúci celkový zisk cukríkov.
\endhodnotenie
}

{%%%%%   Z6-II-3
\def\c{\check c}
Navzájom zhodné strany v~použitých trojuholníkoch môžeme rozlíšiť
farebne~-- použijeme napr. červenú, zelenú a~modrú farbu,
zodpovedajúce dĺžky v~centimetroch označíme $\c$, $z$ a~$m$:
\insp{z6-II-3c.eps}%

V~obvode posledného útvaru je strana každej farby zastúpená práve dvakrát.
Tento obvod je rovný
$$
2\cdot(\c+z+m)=46,
$$
preto je obvod jedného trojuholníka rovný 23\,cm.
V~obvode prvého útvaru sú tri červené strany, jedna zelená a~jedna modrá,
čo je to isté ako dve červené strany a~strany obvodu jedného trojuholníka.
Tento obvod je rovný
$$
3\c+z+m=2\c+(\c+z+m)=43.
$$
Obvod jedného trojuholníka je 23\,cm, preto
$2\c=43-23=20\,(\Cm)$ a~dĺžka jednej strany trojuholníka je $\c=10$\,(cm).

Druhý útvar (ktorý nie je v~zadaní zobrazený) je zložený z~troch trojuholníkov
podľa rovnakých pravidiel ako prvý, napr. takto:
\insp{z6-II-3d.eps}%

Obvod tohto útvaru pozostáva z~jednej červenej, jednej zelenej a~troch modrých
strán, čo je to isté ako strany obvodu jedného trojuholníka a~dve modré
strany navyše.
Tento obvod je podľa zadania rovný
$$
\c+z+3m=(\c+z+m)+2m=35.
$$
Rovnako ako vyššie určíme dĺžku druhej zo strán trojuholníka:
$2m=35-23=12\,(\Cm)$, teda $m=6\,(\Cm)$.
Zo znalosti obvodu a~dvoch strán v~trojuholníku vyjadríme dĺžku strany tretej:
$z=23-10-6=7\,(\Cm)$.

Dĺžky strán Janových trojuholníkov sú 10\,cm, 6\,cm a~7\,cm.
(Výsledok je v~súlade s~trojuholníkovou nerovnosťou.)

\hodnotenie
2~body za určenie obvodu trojuholníka;
2~body za určenie dĺžky jednej strany;
po 1~bode za určenie dĺžok zvyšných dvoch strán.

\poznamka
Pri druhom útvare možno uvažovať ešte iné spôsoby zloženia~--
v~každom prípade sa v~obvode útvaru objavuje vždy jedna strana trojuholníka
viackrát.
Pri uvedenom označení by nanajvýš vyšli dĺžky strán $z$ a~$m$ naopak, čo je
ekvivalentné inému označeniu na začiatku.
V~riešení úlohy nie je nutné všetky možnosti rozoberať.
\endhodnotenie
}

{%%%%%   Z7-II-1
Hľadané číslo je nanajvýš trojciferné, preto je jeho ciferný súčet
nanajvýš dvojciferné číslo. Aby bol ciferný súčet tohto čísla
rovný~1, musí byť ciferný súčet hľadaného čísla 1 alebo~10. Jediné
prirodzené číslo od~90 do~150 s~ciferným súčtom 1 je číslo~100.
Všetky prirodzené čísla od~90 do~150 s~ciferným súčtom~10 sú 91,
109, 118, 127, 136, 145. Úloha má celkom sedem riešení:
$$
91,\quad 100,\quad 109,\quad 118,\quad 127,\quad 136,\quad 145.
$$

\hodnotenie
Po 1~bode za odhalenie možných ciferných súčtov 1 a~10;
zvyšné 4~body rozdeľte podľa počtu nájdených riešení
(4~body za všetkých 7~riešení, 3~body za 6 či 5~riešení, 2~body za 4 či 3~riešenia,
1~bod za 2 či 1~riešenie).
\endhodnotenie
}

{%%%%%   Z7-II-2
Trieda~A nazbierala štvrtinu toho, čo triedy B a~C dokopy;
to znamená, že trieda~A nazbierala pätinu celkového počtu vrchnákov.
Trieda~B nazbierala pätinu toho, čo triedy A~a~C dokopy;
to znamená, že trieda~B nazbierala šestinu celkového počtu vrchnákov.
Triedy A~a~B teda dokopy nazbierali
$$
\frac15+\frac16=\frac6{30}+\frac5{30}=\frac{11}{30}
$$
celkového počtu vrchnákov.

Trieda C tak nazbierala $\frac{19}{30}$ celkového počtu vrchnákov, čo bolo 570~kusov.
Z~toho vyplýva, že tridsatinu celkovej zbierky tvorilo $570:19=30$ vrchnákov.
Celkom teda všetky triedy dokopy nazbierali $30\cdot30=900$ vrchnákov.

Pomocné grafické znázornenie je na nasledujúcom obrázku:
\insp{z7-II-2.eps}%

\hodnotenie
2~body za poznatok, že trieda~A nazbierala pätinu celkového počtu vrchnákov;
1~bod za obdobný poznatok pre triedu~B;
2~body za poznatok, že trieda~C nazbierala $\frac{19}{30}$ celkového počtu;
1~bod za doriešenie úlohy.
\endhodnotenie
}

{%%%%%   Z7-II-3
a)
Svetlá dlažba je trikrát lacnejšia ako tmavá.
Náklady na svetlú a~tmavú časť námestia sú rovnaké, preto je svetlá časť
plochy trikrát väčšia ako tmavá.
Tmavá časť tak tvorí jednu štvrtinu námestia, náklady na ňu sú
$30\,000:4=7\,500$~(\euro).
Náklady na svetlú časť sú rovnaké.
Dlažba na celé námestie podľa daného projektu teda stojí
$$
2\cdot7\,500=15\,000\ (\text{\euro}).
$$

b)
Celé námestie má plochu $20\cdot20=400\,(\text m^2)$.
Z~predchádzajúcej časti vieme, že tmavšia plocha tvorí štvrtinu námestia, teda
$400:4=100\,(\text m^2)$.
Túto plochu môžeme vyjadriť ako súčet štyroch rovnakých štvorcov a~štyroch
rovnakých obdĺžnikov, pozri obrázok.
\insp{z7-II-3a.eps}%

Každý zo zmienených štvorcov má obsah $2\cdot2=4\,(\text m^2)$,
zvyšná časť tmavej plochy má obsah $100-4\cdot4=84\,(\text m^2)$.
Obsah každého zo zmienených obdĺžnikov je preto rovný $84:4=21\,(\text m^2)$.
Jedna strana obdĺžnika meria 2\,m, dĺžka druhej strany je teda
$$
21:2=10{,}5\,(\text m),
$$
a~to je zároveň dĺžka strany svetlého štvorca v~centrálnej časti námestia.

\hodnotenie
2~body za úlohu a), z~toho 1~bod za vysvetlenie;
4~body za úlohu b), z~toho 3~body za vysvetlenie
(napr. 1~bod za obsah tmavej plochy, 1~bod za nápad oddeliť štvorce v~rohoch,
1~bod za obsah pomocného obdĺžnika).
\endhodnotenie
}

{%%%%%   Z8-II-1
Trojuholníky $ABF$ a~$ACF$ majú spoločnú výšku zo spoločného vrcholu~$A$,
preto sú obsahy týchto trojuholníkov v~rovnakom pomere ako dĺžky strán,
ktoré sú vrcholu~$A$ protiľahlé.
Platí teda $|BF|=3|CF|$.
\insp{z8-II-1.eps}%

Strana $AF$ je obom trojuholníkom spoločná a~strany $AB$ a~$AC$ sú
zhodné, pretože trojuholník $ABC$ je rovnostranný.
Z~daného vzťahu pre obvody týchto trojuholníkov vyplýva
$$
|BF|=|CF|+5\cm.
$$
Z~predchádzajúceho odseku vieme, aký je vzťah medzi veľkosťami strán $BF$
a~$CF$, takže môžeme určiť veľkosť jednej z~týchto úsečiek:
$$
\aligned
3|CF|&=|CF|+5\cm,\\
|CF|&=2{,}5\cm.
\endaligned
$$
Dĺžka strany trojuholníka $ABC$ je teda rovná
$$
|BC|=|BF|+|CF|=4|CF|=10\cm.
$$

\hodnotenie
2~body za určenie pomeru dĺžok úsečiek $BF$ a~$CF$;
1~bod za zistenie, že dĺžky strán $AF$, $AB$, resp. $AC$ do rozdielu obvodov neprispievajú;
3~body za doriešenie úlohy.
\endhodnotenie
}

{%%%%%   Z8-II-2
Zadaný postupný pomer $4:5:6$ rozšírime tak, aby sme mohli jeho tretí člen
vhodne rozdeliť ako podľa Jankovho návrhu (delenie v~pomere $1:1$), tak podľa
Mikešovho návrhu (delenie v~pomere $4:5$).
Pomer preto rozšírime tak, aby jeho tretí člen bol deliteľný dvoma a~súčasne
deviatimi, rozšírime ho teda troma:
$$
12:15:18.
$$

Suma pôvodne určená pre Vávru sa skladá z~18 rovnakých dielov, z~ktorých Janek
navrhuje prideliť 9 sebe a~9 Mikešovi, zatiaľ čo Mikeš navrhuje dať 8 dielov Jankovi
a~10 sebe.
Podľa Jankovho návrhu by tak Mikeš dostal o~1~diel menej ako podľa svojho vlastného návrhu.
Tento diel zodpovedá 4€.
Spoločný honorár pre hudobníkov bol tvorený 45 takýmito dielmi ($12+15+18=45$), celkom teda dostali
$$
45\cdot4=180\ (\text{\euro}).
$$

\hodnotenie
2~body za výsledok; 4~body za postup riešenia.
\endhodnotenie
}

{%%%%%   Z8-II-3
Vyjdeme od konca, tzn. z~rozmerov výslednej kocky spätne odvodíme rozmery
pôvodného kvádra.
Ak označíme dĺžku hrany kocky v~centimetroch $x$, tak jeden rozmer
kvádra je $\frac12x$ (dvojnásobok tejto dĺžky je dĺžka hrany kocky),
druhý rozmer kvádra je $2x$ (polovica tejto dĺžky je dĺžka hrany kocky)
a~tretí rozmer je $x-6$ (táto dĺžka zväčšená o~6\,cm je dĺžka hrany kocky).

Povrch výslednej kocky je $6x^2$, zatiaľ čo povrch pôvodného kvádra je rovný
$$
2\left(\frac{x}2\cdot2x+\frac{x}2\cdot(x-6)+2x\cdot(x-6)\right)
=2x^2+x^2-6x+4x^2-24x
=7x^2-30x.
$$
Keďže oba povrchy sú si rovné, dostávame rovnicu
$$
\aligned
6x^2&=7x^2-30x,\\
0&=x^2-30x=x\cdot(x-30).
\endaligned
$$
Keďže $x$ nemôže byť nula ($x$ je dĺžka hrany kocky), musí byť
$x=30$\,(cm).

Rozmery pôvodného kvádra teda sú:
$$
\frac{30}2=15\,(\cm),\quad
2\cdot30=60\,(\cm),\quad
30-6=24\,(\cm).
$$

\hodnotenie
1~bod za vyjadrenie rozmerov kvádra pomocou dĺžky hrany kocky;
po 1~bode za vyjadrenie povrchu každého z~telies;
2~body za vyriešenie rovnice (a~to aj~v~prípade, že riešiteľ nevysvetlí
nenulovosť $x$);
1~bod za výpočet rozmerov pôvodného kvádra.
\endhodnotenie
}

{%%%%%   Z9-II-1
Keď sa škrtnutím trojice čísel množiny nezmení jej aritmetický priemer, má
táto trojica čísel rovnaký aritmetický priemer ako celá množina.
Aritmetický priemer celej množiny čísel je teda
$$
\frac{70+82+103}3=85.
$$

Počet čísel celej množiny označme $n$.
Po škrtnutí čísel $122$ a~$123$ zvýši $n-2$ čísel a~podľa zadania sa aritmetický priemer zmenší o~$1$, teda bude $84$.
Vynásobením aritmetického priemeru čísel a~počtu týchto čísel dostaneme
ich súčet.
Na základe toho zostavíme nasledujúcu rovnicu, ktorú vyriešime:
$$
\aligned
85n&=84(n-2)+122+123,\\
85n&=84n+77,\\
n&=77.
\endaligned
$$
Neznámu množinu čísel tvorí 77 bezprostredne po sebe idúcich čísel.

Prostredné číslo je rovné aritmetickému priemeru, pred ním a~za ním je 38 čísel,
pretože $77=2\cdot38+1$.
Najmenšie číslo množiny je teda $85-38=47$ a~najväčšie $85+38=123$.
Hľadanými číslami sú prirodzené čísla od $47$ po $123$ vrátane.

\ineriesenie
Rovnakým spôsobom zistíme, že aritmetický priemer množiny čísel pred škrtaním je $85$.
Vyškrtnuté čísla $122$ a~$123$ potom vyjadríme pomocou tohto priemeru: $122=85+37$,
$123=85+38$.
V~súčte čísel, ktoré zostanú po vyškrtnutí čísel $122$ a~$123$, "chýba"
práve súčet $37+38$ k~tomu, aby tieto čísla mali aritmetický priemer $85$.
Zo zadania vieme, že aritmetický priemer týchto zvyšných čísel je o~$1$ menší ako
$85$, týchto čísel preto musí byť práve $37+38$, \tj. 75.
Počet čísel pred škrtaním bol $75+2$, \tj. 77.
Ďalej pokračujeme ako v~predchádzajúcom riešení.

\hodnotenie
2~body za priemer myslených čísel ($85$);
3~body za počet myslených čísel (77);
1~bod za myslené čísla ($47$ až $123$).
\endhodnotenie
}

{%%%%%   Z9-II-2
Súčty dvojíc susediacich čísel s~opačnými znamienkami sú buď $-1$, alebo $1$.
Tieto hodnoty sa navyše pravidelne striedajú.
V~uvažovanom výraze sa teda vždy niekoľko susediacich čísel zruší:
$$
\aligned
&1+2-3-4+5+6-7-8+9+10-11-12+13+\cdots\\
=&[1+(2-3)]+[(-4+5)+(6-7)]+[(-8+9)+(10-11)]+[(-12+13)+\cdots\\
=&[1-1]+[1-1]+[1-1]+[1-\cdots\\
=&0+0+0+\cdots
\endaligned
$$

Prvú nulu dostávame ako súčet prvej trojice čísel, ostatné nuly dostávame
ako súčty po sebe idúcich štvoríc čísel.
Súčet uvedeného výrazu by teda bol rovný nule, ak by končil číslom 3, 7,
11, 15 atď.
Všeobecnejšie súčet takého výrazu je rovný nule práve vtedy, keď končí
číslom, ktoré má po delení štyrmi zvyšok 3.
Keďže číslo $2\,015$ má práve túto vlastnosť, je hodnota zadaného výrazu
rovná nule.

\poznamka
Úvodné postrehy je možné zúročiť rôznymi spôsobmi, ktoré môžu viesť k~podrobnejšej diskusii o~znamienkach pri číslach na konci daného výrazu.
Z~uvedeného riešenia vyplýva, že tento výraz končí takto:
$$
\cdots+[(-2\,012+2\,013)+(2\,014-2\,015)].
$$

Sčítance daného výrazu je možné zoskupovať aj napr. takto:
$$
[1+2-3-4]+[5+6-7-8]+[9+10-11-12]+\cdots
=-4-4-4-\cdots
$$
Takých štvoríc možno utvoriť nanajvýš 503 ($2\,015=503\cdot 4+3$).
Celkový súčet je podľa tohto návodu vyjadrený nasledovne:
$$
503\cdot(-4)+2\,013+2\,014-2\,015
=-2\,012+2\,013+2\,014-2\,015 =0.
$$

\hodnotenie
1~bod za objav, že súčty dvojíc susedných čísel s~opačnými znamienkami sú
striedavo $-1$ a~1;
2~body za objav, že súčty po sebe idúcich štvoríc, resp. prvej trojice
čísel sú rovné 0;
3~body za správne určenie a~zdôvodnenie celkového súčtu.
\endhodnotenie}

{%%%%%   Z9-II-3
Obdĺžnik je stredovo súmerný podľa stredu, ktorý je priesečníkom
uhlopriečok.
Podľa tohto stredu sú tiež súmerné každé dva body, ktoré ležia na
protiľahlých stranách obdĺžnika a~sú v rovnakých pomeroch vzhľadom na~zodpovedajúce si vrcholy.
Takými dvojicami bodov sú aj body určujúce oba rezy zo zadania.
Tieto rezy a~uhlopriečky obdĺžnika sa teda pretínajú v~jednom spoločnom bode.
Jedna z~týchto uhlopriečok rozdeľuje sivo vyznačený štvoruholník na dva trojuholníky:
\insp{z9-II-3a.eps}%

Jedna strana ľavého trojuholníka je práve tretinou hornej strany
obdĺžnika a~výška na túto stranu je polovicou pravej strany obdĺžnika.
Tento trojuholník teda zaberá
$$
\frac12\cdot\frac13\cdot\frac12=\frac1{12}
$$
obsahu celého obdĺžnika.
Jedna strana pravého trojuholníka je práve pätinou pravej strany
obdĺžnika a~výška na túto stranu je polovicou hornej strany obdĺžnika.
Tento trojuholník teda zaberá
$$
\frac12\cdot\frac15\cdot\frac12=\frac1{20}
$$
obsahu celého obdĺžnika.
Obsah sivého štvoruholníka je rovný
$$
\frac1{12}+\frac1{20}=\frac{5+3}{60}=\frac2{15}
$$
obsahu celého obdĺžnika.
Zjedená časť teda tvorí $\frac2{15}$ celej torty.

\poznamka
Ak veľkosť hornej strany obdĺžnika označíme $a$ a~veľkosť pravej strany
$b$, tak môžeme predchádzajúce úvahy vyjadriť takto:
$$
\frac12\cdot\frac{a}3\cdot\frac{b}2+
\frac12\cdot\frac{b}5\cdot\frac{a}2=
\cdots
=\frac{2ab}{15}.
$$
Obdĺžnik má obsah $ab$, zjedená časť teda tvorí $\frac2{15}$ celej torty.

\ineriesenie
Na úvod opäť ukážeme, že sa rezy pretínajú v strede obdĺžnika.
Týmto bodom prechádzajú aj osi obdĺžnika, ktoré ho rozdeľujú na štyri zhodné
obdĺžniky.
Jeden z~týchto obdĺžnikov je rezmi rozdelený na tri časti pozostávajúce zo sivého
štvoruholníka a~dvoch bielych pravouhlých trojuholníkov.
\insp{z9-II-3b.eps}%

Obsah sivej časti môžeme vyjadriť tak, že od obsahu štvrtinového obdĺžnika
odčítame obsahy oboch bielych trojuholníkov.
Vzhľadom na predošlé označenie sú obsahy týchto trojuholníkov rovné
$$
\frac12\cdot\left(\frac{a}2-\frac{a}3\right)\cdot\frac{b}2=\frac{ab}{24}
\quad\text{a}\quad
\frac12\cdot\Bigl(\frac{b}2-\frac{b}5\Bigr)\cdot\frac{a}2=\frac{3ab}{40}.
$$
Obsah sivej plochy teda vychádza
$$
\frac{ab}4-\frac{ab}{24}-\frac{3ab}{40}=\frac{(30-5-9)ab}{120}=\frac{2ab}{15}.
$$
Opäť prichádzame k~záveru, že zjedená časť tvorí $\frac2{15}$ torty.

\hodnotenie
1~bod za zdôvodnenie, že sa rezy pretínajú v strede obdĺžnika
(možno dokázať s~využitím stredovej súmernosti, podobnosti či iných osobitných úvah);
2~body za obsahy pomocných trojuholníkov;
2~body za výsledok;
1~bod podľa kvality a~úplnosti komentára.
\endhodnotenie
}

{%%%%%   Z9-II-4
Označme dĺžky strán obdĺžnika v~decimetroch $x$ a~$y$; obsah obdĺžnika teda bol $x\cdot y$.
Predpokladajme, že pri prvej zmene sa zväčšovala strana s~dĺžkou $x$.
Po prvej zmene mal obdĺžnik rovnaký obsah, musel mať teda rozmery $2x$ a~$\frac{y}2$.
Po druhej zmene mohol mať buď rozmery a) $2x+1$ a~$\frac{y}2-4$, alebo b) $2x-4$ a~$\frac{y}2+1$.
V~každom prípade mal obdĺžnik aj po druhej zmene rovnaký obsah ako pôvodne,
teda $x\cdot y$.
Rozoberme obe možnosti:

a) V~tomto prípade platí
$$
x\cdot y=(2x+1)\left(\frac{y}2-4\right).
$$
Po úprave dostávame $0=-8x+\frac{y}2-4$, teda $\frac{y}2-4=8x$.
Po druhej zmene mal obdĺžnik rozmery $2x+1$ a~$8x$.
Keďže $x$ je prirodzené číslo, platí $8x>2x+1$ a~pri tretej zmene musel
obdĺžnik skrátiť svojou stranu $2x+1$ na $2x$.
Pomer strán výsledného obdĺžnika je v~tomto prípade rovný $8x:2x=4:1$.

b) V~tomto prípade platí
$$
x\cdot y=(2x-4)\left(\frac{y}2+1\right).
$$
Po úprave dostávame $0=2x-2y-4$, teda $2x-4=2y$.
Po druhej zmene mal obdĺžnik rozmery $2y$ a~$\frac{y}2+1$.
Keďže $y$ je prirodzené číslo, platí $2y>\frac{y}2+1$ a~pri tretej zmene
musel obdĺžnik skrátiť svoju stranu $\frac{y}2+1$ na $\frac{y}2$.
Pomer strán výsledného obdĺžnika je aj v~tomto prípade rovný
$2y:\frac{y}2=4:1$.

\hodnotenie
Po 3~bodoch za rozbor každej z~možností:
1~bod za vyjadrenie vzťahu medzi $x$ a~$y$ (napr. $\frac{y}2-4=8x$);
1~bod za vyjadrenie rozmerov obdĺžnika po ich druhej zmene (napr. $2x+1$
a~$8x$);
1~bod za označenie kratšej strany a~vyjadrenie výsledného pomeru.

Ak riešiteľ označí pri niektorej možnosti jeden rozmer za kratší bez akejkoľvek
úvahy, strhnite pri hodnotení tejto možnosti 1~bod.
\endhodnotenie
}

{%%%%%   Z9-III-1
Aritmetický priemer všetkých Martinových hodov bol o~1 väčší ako priemer Jurajov
a~každý z~chlapcov hádzal spolu desaťkrát.
Preto bol Martinov celkový súčet o~10 väčší ako súčet Jurajov.
Tieto celkové súčty sa pritom líšili iba o~súčty posledných dvoch zásahov
-- Juraj trafil dvakrát políčko s~najmenšou možnou hodnotou, Martin dvakrát
políčko so strednou hodnotou.
Preto bolo políčko so strednou hodnotou o~5 väčšie ako políčko s~najmenšou hodnotou.

Podobným spôsobom možno zdôvodniť, že políčko s~najväčšou hodnotou bolo o~5 väčšie
ako políčko so strednou hodnotou.
Jedna z~hodnôt týchto troch políčok bola 12, nevieme však, či to bola tá najmenšia,
stredná alebo najväčšia.
Do úvahy prichádzajú nasledujúce tri možnosti hodnôt políčok na terči:
\begin{itemize}
\item 12, 17, 22,
\item 7, 12, 17,
\item 2, 7, 12.
\end{itemize}

\poznamka
Súčet prvých ôsmich hodov ktoréhokoľvek z~chlapcov označme $S$ a~neznáme hodnoty
na terči postupne $j$, $m$ a~$p$, pričom $j<m<p$.
Pri tomto označení možno začiatok predchádzajúceho riešenia zapísať nasledovne:
$$
\begin{aligned}
\frac{S+2m}{10}&=\frac{S+2j}{10}+1,\\
2m&=2j+10,\\
m&=j+5.
\end{aligned}
$$
Podobným spôsobom možno zdôvodniť, že $p=m+5$.

\hodnotenie
4~body za objav a~zdôvodnenie závislostí medzi hodnotami políčok na terči;
2~body za uvedenie všetkých troch možností hodnôt na terči.
\endhodnotenie
}

{%%%%%   Z9-III-2
Podmienkam zo zadania vyhovuje jediné usporiadanie bodov $E$ a~$F$ na úsečke~$AB$, pozri obrázok.
Obsah trojuholníka $ABC$ je rovný súčtu obsahov trojuholníkov $AEC$, $EFC$
a~$FBC$, \tj.
$$
S_{ABC}
=S_{AEC}+S_{EFC}+S_{FBC}
=1+3+2=6\,(\Cm^2).
$$
Ten istý obsah možno vyjadriť ako súčet obsahov trojuholníkov $AGC$, $GHC$ a~$HBC$.
Obsah prvého a~tretieho trojuholníka odvodíme zo zadania, potom
ľahko určíme obsah trojuholníka~$GHC$.
\insp{z9-III-2.eps}%

Úsečka $CG$ je ťažnicou trojuholníka $AFC$, a~tá delí tento trojuholník na
dva trojuholníky s rovnakým obsahom.
Pritom obsah trojuholníka $AFC$ je súčtom obsahov trojuholníkov $AEC$ a~$EFC$, ktoré poznáme.
Platí teda
$$
S_{AGC}
=\frac12(S_{AEC}+S_{EFC})
=\frac12(1+3)=2\,(\Cm^2).
$$
Podobným spôsobom možno zdôvodniť, že obsah trojuholníka $HBC$ je rovný
$$
S_{HBC}
=\frac12(S_{EFC}+S_{FBC})
=\frac12(3+2)
=2{,}5\,(\Cm^2).
$$
Obsah trojuholníka $GHC$ je preto rovný
$$
S_{GHC}
=S_{ABC}-S_{AGC}-S_{HBC}
=6-2-2{,}5=1{,}5\,(\Cm^2).
$$


\poznamka
Všetky diskutované trojuholníky majú spoločnú výšku zo spoločného vrcholu~$C$.
Preto sú pomery obsahov ktorýchkoľvek dvoch trojuholníkov rovnaké ako pomery
veľkostí strán, ktoré sú protiľahlé vrcholu~$C$.
Na určenie obsahu trojuholníka $GHC$ preto stačí určiť pomer dĺžky strany~$GH$
vzhľadom k~dĺžke strany niektorého z~trojuholníkov so známym obsahom:

Ak napr. označíme $|AE|=a$, tak $|EF|=3a$ a~$|FB|=2a$.
Keďže $CG$ je ťažnicou trojuholníka $AFC$, je bod~$G$ stredom úsečky~$AF$.
Veľkosť tejto úsečky je $|AF|=|AE|+|EF|=4a$, teda $|AG|=\frac12|AF|=2a$.
Podobne sa zdôvodní, že $|HB|=\frac12(|EF|+|FB|)=2{,}5a$.
Veľkosť úsečky~$GH$ je preto rovná
$$
|GH|=|AB|-|AG|-|HB|=(6-2-2{,}5)a=1{,}5a.$$
Odtiaľ vyplýva, že
$S_{GHC}=1{,}5\cdot S_{AEC}=1{,}5\cm^2$.

\hodnotenie
1~bod za usporiadanie bodov $E$ a~$F$ na úsečke $AB$ (stačí náčrt);
2~body za zistenie, že úsečka $CG$, resp. $CH$ delí trojuholník $AFC$, resp.
$EBC$ na dva trojuholníky s~rovnakým obsahom;
3~body za odvodenie hľadaného obsahu.
\endhodnotenie
}

{%%%%%   Z9-III-3
Posledná cifra súčinu je daná len poslednými ciframi činiteľov.
Pri riešení úlohy preto budeme vo výpočtoch uvažovať iba posledné cifry.
Podľa zadania násobíme desať štvoríc činiteľov a~v~každej z~nich sú činitele
končiace ciframi 2, 4, 6 a~8.
Súčin $2\cdot4\cdot6\cdot8$ má na mieste jednotiek cifru~4.
Pre zistenie poslednej cifry výsledku násobenia desiatich takých štvoríc
stačí vynásobiť desať štvoriek a~pri násobení sledovať iba poslednú
cifru:

Súčin $4\cdot4$ končí cifrou 6, preto namiesto násobenia desiatich štvoriek stačí
vynásobiť päť šestiek.
Číslo~6 násobené sebou samým dá opäť číslo končiace cifrou~6, a~preto súčin
piatich šestiek, a~teda aj~desiatich štvoriek končí cifrou~6.
Hľadaná cifra je~6.

\ineriesenie
Opäť uvažujeme v~súčinoch len posledné cifry.
Máme zadaných desať činiteľov končiacich cifrou~2, desať končiacich cifrou~4,
desať končiacich cifrou 6 a~desať končiacich cifrou~8.
Postupne budeme uvažovať o~každých desiatich činiteľoch:

Pri násobení
$2\cdot2\cdot2\cdot2\cdot2\cdot2\cdot2\cdot2\cdot2\cdot2$ sledujeme iba
cifry na mieste jednotiek a~zistíme, že výsledok končí cifrou 4.
Podobne zistíme, že
súčin $4\cdot4\cdot4\cdot4\cdot4\cdot4\cdot4\cdot4\cdot4\cdot4$ končí
cifrou~6, súčin $6\cdot6\cdot6\cdot6\cdot6\cdot6\cdot6\cdot6\cdot6\cdot6$
končí cifrou~6 a~súčin
$8\cdot8\cdot8\cdot8\cdot8\cdot8\cdot8\cdot8\cdot8\cdot8$ končí cifrou 4.
Poslednú cifru hľadaného súčinu určíme vynásobením práve zistených
cifier, tzn. $4\cdot6\cdot6\cdot4$.
Tento súčin končí cifrou 6, teda hľadaná cifra je 6.

\hodnotenie
1~bod za objav, že stačí uvažovať iba posledné cifry činiteľov;
4~body za medzivýsledky a~ich zdôvodnenie
(napr. 2~body za poslednú cifru súčinu $2\cdot4\cdot6\cdot8$ a~2~body za
poslednú cifru mocniny $4^{10}$);
1~bod za hľadanú cifru 6.
\endhodnotenie
}

{%%%%%   Z9-III-4
Striedavé uhly určené priečkou~$PC$ dvoch rovnobežných priamok $AD$ a~$BC$ sú
zhodné.
Pritom bod~$P$ určite leží na polpriamke~$DA$.
(Keby totiž ležal na opačnej polpriamke, tak by bod~$C$ ležal mimo uhol
$BPD$ a~os tohto uhla by nemohla bodom~$C$ prechádzať.)
Preto sú uhly $DPC$ a~$PCB$ zhodné.

Podľa zadania je priamka~$PC$ osou uhla $BPD$, preto sú aj uhly $DPC$
a~$CPB$ zhodné.
Trojuholník $CPB$ má teda dva zhodné vnútorné uhly pri vrcholoch $C$ a~$P$,
preto je tento trojuholník rovnoramenný s~ramenami $BC$ a~$BP$.

Z~uvedeného vyplýva, že bod~$P$ je priesečníkom polpriamky~$DA$ a~kružnice so
stredom v~bode~$B$ a~polomerom~$|BC|$.
Vzhľadom na podmienku $|AB|<|BC|$ táto kružnica polpriamku~$DA$ pretína, a~to
v~dvoch rôznych bodoch (z~ktorých jeden je vnútorným bodom úsečky~$AD$).
\insp{z9-III-4.eps}%

\hodnotenie
2~body za objav a~zdôvodnenie zhodnosti uhlov $DPC$ a~$PCB$;
2~body za objav a~zdôvodnenie vzťahu $|BP|=|BC|$;
2~body za doriešenie úlohy vrátane diskusie o~počte riešení.
\endhodnotenie
}

