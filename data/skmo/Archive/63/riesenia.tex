{%%%%%   A-I-1
Hľadáme  $n=p\cdot q\cdot r$ s~prvočíslami $p\leqq q\leqq r$, ktoré
spĺňajú rovnosť
$$
(p+1)(q+1)(r+1)=pqr+963.
\tag1$$
Jej pravá strana je v~prípade najmenšieho prvočísla $p=2$
nepárne číslo, takže potom i~činitele $q+1$ a~$r+1$ zo súčinu na ľavej strane
musia byť nepárne čísla. Pre prvočísla $q$, $r$ to znamená, že
$q=r=2$, ale trojica $p=q=r=2$ rovnosti \thetag1 nevyhovuje.
Platí teda $p\geqq3$.

Ukážeme teraz, že nutne platí $p=3$. V~opačnom prípade sú všetky tri
prvočísla $p$, $q$, $r$ väčšie ako~$3$, a~preto ani súčin
$pqr$, a~teda ani pravá strana~\thetag1 nie sú čísla deliteľné tromi
(číslo $963$ je totiž násobkom troch). Z~podmienky, že súčin $(p+1)(q+1)(r+1)$
z~ľavej strany~\thetag1 nie je deliteľný tromi, ale vyplýva, že žiadne
z~prvočísel $p$, $q$, $r$ nemôže dávať po delení tromi zvyšok~$2$. A~keďže
zvyšok~$0$ je v~uvažovanom prípade vylúčený, môžu
prvočísla $p$, $q$, $r$ dávať pri delení tromi jedine zvyšok~$1$.
Odtiaľ ďalej dostávame, že
rozdiel $(p+1)(q+1)(r+1)-pqr$ dáva po delení tromi rovnaký zvyšok
ako číslo $2\cdot2\cdot2-1\cdot1\cdot1=7$, teda zvyšok~$1$. To je
ale spor, lebo podľa~\thetag1 platí $(p+1)(q+1)(r+1)-pqr=963$.
Rovnosť $p=3$ je tak dokázaná.

Po dosadení $p=3$ do \thetag1 dostaneme rovnosť $4(q+1)(r+1)=3qr+963$,
ktorú postupne upravíme na "súčinový" tvar:
$$
\align
4qr+4q+4r+4&=3qr+963,\\
qr+4q+4r&=959,\\
(q+4)(r+4)&=975.
\endalign
$$

Číslo $975$ má prvočíselný rozklad $975=3\cdot5^2\cdot13$, takže
vzhľadom na nerovnosti $3\leqq q\leqq r$ čiže
$7\leqq q+4\leqq r+4$ pre menší činiteľ z~odvodeného rozkladu
musí platiť $q+4\le\sqrt{975}<32$, preto $q+4\in\{13, 15, 25\}$.
To spĺňa jediné prvočíslo
$q=11$, pre ktoré $q+4=15$. Pre druhý činiteľ tak máme
$r+4=5\cdot13=65$, odtiaľ $r=61$, čo je naozaj prvočíslo.
Hľadané číslo $n=p\cdot q\cdot r$ je teda jediné a~má hodnotu
$$
n=3\cdot11\cdot61=2\,013.
$$

\návody
Číslo $n$ je súčinom štyroch prvočísel. Ak každé z~týchto prvočísel
zväčšíme o~$1$ a~vzniknuté štyri čísla vynásobíme, dostaneme
číslo o~$2\,886$ väčšie ako pôvodné číslo~$n$. Určte všetky také~$n$.
[59--B--II--4]

Zistite, kedy pre tri prvočísla
$p$, $q$, $r$ má rozdiel ${(p+1)(q+1)(r+1)}-pqr$
hodnotu, ktorá po delení šiestimi dáva zvyšok~$3$.
[Jedno z~prvočísel $p$, $q$, $r$ sa rovná
trom, druhé je tvaru $6k-1$, kde $k\in\Bbb N$, a~tretie je
ľubovoľné nepárne prvočíslo. (Úvaha~A: Keby
bol súčin $pqr$ párny, musel by byť súčin $(p+1)(q+1)(r+1)$ nepárny,
takže by muselo byť $p=q=r=2$.
Úvaha B: Keby nebol súčin $pqr$ deliteľný tromi, nebol by taký
ani súčin  $(p+1)(q+1)(r+1)$, takže čísla $p$, $q$, $r$ by
dávali po delení číslom~$3$ zvyšok~$1$ a~uvažovaný rozdiel by tak
dával zvyšok~$2$. Z~úvah A a~B už  ľahko vyplýva vyššie uvedená odpoveď.)]

Nájdite všetky dvojice prvočísel $p$, $q$, pre ktoré platí
\item{a)} $p+q^{2}=q+145p^{2}$,\quad [55--C--II--4]
\item{b)} $p+q^2=q+p^3$.\quad        [55--B--II--1]

Určte, pre ktoré trojice navzájom rôznych prvočísel $p$, $q$, $r$
platí súčasne:
$p\mid q+r$,\quad $q\mid r+2p$,\quad$r\mid p+3q$.\quad[55--A--III--5]

Určte všetky trojice $(p,q,r)$ prvočísel, pre ktoré platí:
$(p+1)(q+2)(r+3)=4pqr$.\quad[60--A--III--2]

\endnávod
}

{%%%%%   A-I-2
Keďže v~dokazovanej nerovnosti vystupuje minimum z~dvoch
kladných čísel a~funkcia $y=x^2$ je na množine kladných čísel
rastúca, je našou úlohou overiť dvojicu nerovností
$$
(x+y+z)\Bigl(\frac{1}{x}+\frac{1}{y}+\frac{1}{z}\Bigr)\leqq
\Bigl(\frac{x}{y}+\frac{y}{z}+\frac{z}{x}\Bigr)^{\!2}
\enspace\text{a}\enspace
(x+y+z)\Bigl(\frac{1}{x}+\frac{1}{y}+\frac{1}{z}\Bigr)\leqq
\Bigl(\frac{y}{x}+\frac{x}{z}+\frac{z}{y}\Bigr)^{\!2}
\tag1
$$
a~zistiť, kedy v~{\it aspoň jednej\/} z~nich nastane rovnosť
(práve vtedy totiž nastane rovnosť aj v~pôvodnej nerovnosti).
Druhú nerovnosť v~\thetag1 ale zrejme dostaneme z~prvej,
keď trojicu $(x,y,z)$ zameníme trojicou $(y,x,z)$. Stačí preto overiť,
že pre ľubovoľnú
trojicu $(x,y,z)$ kladných čísel platí prvá nerovnosť
$$
(x+y+z)\Bigl(\frac{1}{x}+\frac{1}{y}+\frac{1}{z}\Bigr)\leqq
\Bigl(\frac{x}{y}+\frac{y}{z}+\frac{z}{x}\Bigr)^{\!2},
$$
a~zistiť, kedy v~nej nastane rovnosť. Po roznásobení oboch strán dostaneme
$$
3+\frac{x}{y}+\frac{x}{z}+\frac{y}{x}+\frac{y}{z}+\frac{z}{x}+\frac{z}{y}
\leqq
\frac{x^2}{y^2}+\frac{y^2}{z^2}+\frac{z^2}{x^2}+
      2\Big(\frac{x}{z}+\frac{y}{x}+\frac{z}{y}\Big).
$$
Takú nerovnosť bude zrejme výhodné zapísať v~nových
(opäť kladných) premenných $a=x/y$, $b=y/z$, $c=z/x$. Dostaneme
tak ekvivalentnú nerovnosť
$$
3+a+\frac{1}{c}+\frac{1}{a}+b+c+\frac{1}{b}\leqq
a^2+b^2+c^2+2\Big(\frac{1}{a}+\frac{1}{b}+\frac{1}{c}\Big),
$$
ktorú ešte upravíme na tvar so súčtom troch hodnôt toho istého výrazu
$$
\Big(a^2-1-a+\frac{1}{a}\Big)+\Big(b^2-1-b+\frac{1}{b}\Big)+
\Big(c^2-1-c+\frac{1}{c}\Big)\geqq0.
\tag2$$
Vďaka tomu, že $a$, $b$, $c$ sú kladné čísla a~že uvedený výraz má
vyjadrenie
$$
t^2-1-t+\frac{1}{t}=(t^2-1)-\frac{t^2-1}{t}=\frac{(t^2-1)(t-1)}{t}=
\frac{(t-1)^2(t+1)}{t},
$$
upravená nerovnosť \thetag2 platí a~rovnosť v~nej nastane práve vtedy, keď
$a=b=c=1$, čo pre pôvodné premenné $x$, $y$, $z$
znamená práve to, že $x=y=z$.
Táto podmienka je ale rovnaká i~pre rovnosť v~druhej nerovnosti~\thetag1
(ak chceme byť dôslední, má tvar $y=x=z$). Preto i~rovnosť
v~pôvodnej dokazovanej nerovnosti nastane jedine v~prípade, keď
$x=y=z$.

Dodajme, že na dôkaz \thetag1 sme vlastne ani nevyužili vzťah
$abc=1$, ktorý novo zavedené premenné $a$, $b$, $c$ spĺňajú.

\návody
\titem
Pripomeňme najskôr známe tvrdenie o~nerovnosti medzi aritmetickým
a~geometrickým priemerom (stručne nazývanej AG-nerovnosťou):
{\sl Pre
ľubovoľné nezáporné reálne čísla $a_1,a_2,\dots,a_n$ platí
$$
\frac{a_1+a_2+\dots+a_n}{n}\geqq\root{n}\of{a_1a_2\dots a_n};
$$
pritom rovnosť nastane jedine v~prípade $a_1=a_2=\dots=a_n$.}

Pre ľubovoľné kladné reálne čísla $x$, $y$, $z$ dokážte
nerovnosti
$$
\min\Bigl(\frac{x}{y}+\frac{y}{z}+\frac{z}{x},
\frac{y}{x}+\frac{z}{y}+\frac{x}{z}\Bigr)\geqq3\quad\text{a}\quad
(x+y+z)\Bigl(\frac{1}{x}+\frac{1}{y}+\frac{1}{z}\Bigr)\geqq9.
$$
[Prvá nerovnosť je dôsledkom toho, že nerovnosť $a+b+c\geqq3$
platí pre každú trojicu kladných čísel $a$, $b$, $c$
spĺňajúcich podmienku $abc=1$ (využite pre takú trojicu AG-nerovnosť).
Druhú nerovnosť dostaneme, keď medzi sebou vynásobíme dve AG-nerovnosti zapísané pre trojice čísel $(x,y,z)$
a~$(1/x, 1/y, 1/z)$.]

Pre ľubovoľné kladné reálne čísla $x$, $y$, $z$ dokážte
$$
\Bigl(x+{1\over y}\Bigr)\Bigl(y+{1\over z}\Bigr)\Bigl(z+{1\over x}\Bigr)\ge8
$$
a~zistite, kedy nastane rovnosť. [55--B--S--1]

Dôsledkom nerovnosti zo súťažnej úlohy je (slabšia) nerovnosť
$$
(x+y+z)\Bigl(\frac{1}{x}+\frac{1}{y}+\frac{1}{z}\Bigr)\leqq
\Bigl(\frac{x}{y}+\frac{y}{z}+\frac{z}{x}\Bigr)
\Bigl(\frac{y}{x}+\frac{z}{y}+\frac{x}{z}\Bigr).
$$
Spravte jej priamy dôkaz (pre ľubovoľné $x,y,z>0$),
skôr ako začnete riešiť súťažnú úlohu.
[Roznásobením zátvoriek na ľavej strane dostaneme výraz $3+A+B$, kde
$A=x/y+y/z+z/x$ a~$B=y/x+x/z+z/y$ sú činitele z~pravej strany.
Máme tak dokázať nerovnosť $3+A+B\leqq AB$;
v~súťažnej úlohe ide o~silnejšiu nerovnosť
$3+A+B\leqq\bigl(\min(A,B)\bigr)^2$.
Nerovnosť $3+A+B\leqq AB$ upravíme na súčinový tvar
$(A-1)(B-1)\geqq4$ a~všimneme si, že podľa návodnej úlohy~1
platí $A\geqq3$ a~$B\geqq3$, čiže $A-1\geqq2$ a~$B-1\geqq2$.
Vynásobením posledných dvoch nerovností dostaneme potrebné.]

Pri riešení súťažnej úlohy je možné (ak nie dokonca nevyhnutné)
uplatniť častý užitočný obrat, kedy určitú nerovnosť zdôvodníme
sčítaním niekoľkých (v~danom prípade troch) analogických nerovností.
Využite to na~určenie najmenšej hodnoty výrazu
$$
V=a^2+b^2+c^2+2\Bigl(\frac{1}{a}+\frac{1}{b}+\frac{1}{c}\Bigr),
$$
kde $a$, $b$, $c$ sú ľubovoľne kladné reálne čísla.
[$V_{\min}=9$  pre $a=b=c=1$. Podľa AG-nerovnosti použitej na~trojicu čísel $t^2$, $1/t$, $1/t$ pre každé $t\geqq3$
platí $t^2+2/t\geqq3$. Sčítaním takých nerovností pre $t=a$, $t=b$
a~$t=c$ dostaneme potrebný odhad $V\geqq9$.]

Určte všetky trojice $(x,y,z)$ reálnych čísel, pre  ktoré platí
$$
x^2+y^2+z^2\leq 6+\min\Bigl(x^2-\frac{8}{x^4}, y^2-\frac{8}{y^4}, z^2-\frac{8}{z^4}\Bigr).
$$
[53--A--III--1]
\endnávod
}

{%%%%%   A-I-3
Ukážeme najskôr dvoma spôsobmi, že priamka~$MI$, teda
kolmica na priamku~$CI$ v~bode~$I$, je dotyčnicou ku kružnici
$ABI$ (tak budeme označovať kružnice prechádzajúce tromi danými bodmi).
Prvý postup založíme na známom poznatku, že
$AIC$ a~$BIC$ sú tupé uhly veľkostí $90\st+\frac12\be$, resp.
$90\st+\frac12\al$\niedorocenky{ (návodná úloha~1)}. Preto skúmaná kolmica~$MI$
na priamku~$CI$ zviera
s~úsečkami $AI$ a~$BI$ ostré uhly $\frac12\be$, resp.
$\frac12\al$,\footnote{Z~určených uhlov medzi priamkou~$MI$
a~úsečkami $AI$ a~$BI$ vyplýva, že bod~$M$ (ktorého existencia sa
v~zadaní úlohy predpokladá) skutočne existuje práve vtedy, keď platí
$\frac12\al\ne\frac12\be$ čiže $\al\ne\be$. Vzhľadom na
symetriu zadania môžeme predpokladať, že platí $\al>\be$; bod~$M$
potom leží~-- ako na našom obrázku~-- na predĺžení strany~$AB$ za
vrchol~$A$ a~$|\uhel IMA|=\frac12\al-\frac12\be$.}
teda uhly zhodné s~obvodovými uhlami $IBA$, resp.
$IAB$ v~kružnici $ABI$ (\obr).
To už podľa vety o~zhodnosti obvodových
a~úsekových uhlov znamená práve to, že priamka~$MI$ je dotyčnicou
kružnice~$ABI$. Rovnaký záver vyplýva okamžite aj z~poznatku, že
stredom kružnice $ABI$ je stred toho oblúka~$AB$ kružnice $ABC$, ktorý
neobsahuje vrchol~$C$ a~ktorým prechádza priamka~$CI$
(os vnútorného uhla pri vrchole~$C$ trojuholníka $ABC$\niedorocenky{, návodná úloha~2}).
\insp{a63.1}%

Z~dokázaného dotyku priamky~$MI$ s~kružnicou $ABI$ vyplýva, že bod~
$M$ leží na priamke~$AB$ mimo úsečky~$AB$ a~má ku kružnici $ABI$
kladnú mocnosť~$m$, ktorá má dvojaké vyjadrenie $m=|MI|^2=|MA|\cdot|MB|$.
Bod~$M$ preto leží i~vo vonkajšej oblasti kružnice $ABC$ (lebo
úsečka~$AB$ je jej tetiva) a~má k~nej takú istú mocnosť $m=|MA|\cdot|MB|$.
Tá istá hodnota $m=|MI|^2$ je ale menšia ako $|MC|^2$, čo vyplýva
z~pravouhlého trojuholníka $CMI$. Nerovnosť $|MC|^2>m$ tak znamená, že polpriamka
$MC$ má s~kružnicou $ABC$ okrem bodu~$C$ spoločný ešte jeden bod~$N$, ktorý navyše leží medzi bodmi $M$ a~$C$  (lebo
z~rovnosti $|MC|\cdot|MN|=m$ vyplýva nerovnosť $|MN|<|MC|$).
Prvá časť tvrdenia úlohy je tak dokázaná.

Našou druhou úlohou je ukázať, že uhol $CNI$ je pravý. K~tomu na
dokázanú rovnosť $|MC|\cdot|MN|=|MI|^2$ môžeme uplatniť
nasledovné "obrátenie" Euklidovej vety o~odvesne~$MI$
pravouhlého trojuholníka $CMI$. Päta jeho výšky z~vrcholu~$I$ na preponu~$CM$
je taký bod~$X$ úsečky~$CM$, ktorého poloha je
(vďaka Euklidovej vete) jednoznačne určená rovnosťou
$|MC|\cdot|MX|=|MI|^2$. Preto $X=N$ a~dôkaz je hotový.
Bez použitia Euklidovej vety je možné argumentovať takto: Keďže priamka~$MI$ sa v~bode~$I$ dotýka Tálesovej kružnice zostrojenej nad
priemerom~$CI$, má bod~$M$ aj k~tejto kružnici mocnosť $m=|MI|^2$,
a~preto na nej~-- vďaka rovnosti $m=|MC|\cdot|MN|$~-- leží i~bod~$N$,
takže uhol $CNI$ je podľa Tálesovej vety naozaj pravý.

\návody
Pomocou vnútorných uhlov $\al$, $\be$, $\ga$ všeobecného trojuholníka $ABC$
s~vpísanou kružnicou so~stredom~$I$ vyjadrite veľkosti uhlov $AIB$, $AIC$,
$BIC$. [Veľkosti sú postupne $90\st+\frac12\ga$,
$90\st+\frac12\be$, $90\st+\frac12\al$. Vyplýva to zo súčtov
vnútorných uhlov v~jednotlivých trojuholníkoch $AIB$, $AIC$,
$BIC$, lebo polpriamky $AI$, $BI$ a~$CI$ sú osami vnútorných
uhlov trojuholníka~$ABC$.]

\titem
Pred riešením nasledujúcej úlohy si zopakujte učebnicové poznatky
o~stredových, obvodových a~úsekových uhloch.

Pre všeobecný trojuholník $ABC$ s~vpísanou kružnicou so~stredom~$I$
dokážte, že polpriamka~$CI$ pretne oblúk~$AB$
kružnice opísanej v~takom bode~$S$, ktorý má od bodov
$A$, $B$, $I$ rovnakú vzdialenosť. [Rovnosť $|SA|=|SB|$ vyplýva
z~rovnosti obvodových uhlov $ACS$ a~$BCS$; rovnosť $|SA|=|SI|$ je
dôsledkom toho, že v~trojuholníku $AIS$ majú oba vnútorné uhly pri vrcholoch
$A$ a~$I$ takú istú veľkosť rovnú $\frac12\al+\frac12\ga$ čiže
$90\st-\frac12\be$.]

\titem
Zopakujte si ďalej učebnicový poznatok o~všetkých sečniciach danej
kružnice prechádzajúcich daným bodom,  ktorý je vyjadrený veličinou
nazývanou {\it mocnosť bodu ku kružnici}. Dobré poučenie a~veľa
ukážok použitia nájdete v~článku T. Nedevová: {\it Mocnost bodu ke
kružnici}, Rozhledy matematicko-fyzikální, roč.~87\,(2012), č.~2,
str.~9--17.

V~rovine je daný pravouhlý lichobežník $ABCD$ s~dlhšou základňou~$AB$ a~pravým uhlom pri vrchole~$A$. Označme $k_1$ kružnicu
zostrojenú nad stranou~$AD$ ako nad priemerom a~$k_2$ kružnicu
prechádzajúcu vrcholmi $B$, $C$ a~dotýkajúcu sa priamky~$AB$.
Ak majú kružnice $k_1$, $k_2$ vonkajší dotyk v~bode $P$, je priamka~$BC$ dotyčnicou kružnice opísanej trojuholníku $CDP$. Dokážte.
[52--B--II--4]

Do kružnice~$k$ je vpísaný štvoruholník $ABCD$, ktorého uhlopriečka~$BD$ nie je priemerom. Dokážte, že priesečník priamok, ktoré sa
kružnice~$k$ dotýkajú v~bodoch $B$ a~$D$, leží na priamke~$AC$
práve vtedy, keď platí $|AB|\cdot|CD|=|AD|\cdot|BC|$. [51--A--II--3]

Je daný rovnobežník $ABCD$ s~tupým uhlom $ABC$. Na jeho uhlopriečke
$AC$ v~polrovine $BDC$ zvoľme bod~$P$ tak, aby platilo $|\uhel
BPD|=|\uhel ABC|$. Dokážte, že priamka $CD$ je dotyčnicou kružnice
opísanej trojuholníku $BCP$ práve vtedy, keď úsečky $AB$ a~$BD$ sú zhodné.
[59--A--II--2]

V~trojuholníku $ABC$, ktorý nie je rovnostranný, označme $K$ priesečník osi
vnútorného uhla~$BAC$ so stranou~$BC$ a~$L$ priesečník osi vnútorného uhla
$ABC$ so stranou~$AC$. Ďalej označme $S$ stred kružnice vpísanej,
$O$~stred kružnice opísanej a~$V$ priesečník výšok trojuholníka $ABC$.
Dokážte, že nasledujúce dve tvrdenia sú ekvivalentné:
\item{a)} Priamka~$KL$ sa dotýka kružníc opísaných trojuholníkom $ALS$, $BVS$ a~$BKS$.
\item{b)} Body $A$, $B$, $K$, $L$ a~$O$ ležia na jednej kružnici.
[55--A--III--3]

Sú dané kružnice $k$, $l$, ktoré sa pretínajú v~bodoch $A$, $B$. Označme $K$, $L$
postupne dotykové body ich spoločnej dotyčnice zvolené tak, že bod~$B$ je vnútorným
bodom trojuholníka $AKL$. Na kružniciach $k$ a~$l$ zvoľme postupne body $N$ a~$M$ tak,
aby bod~$A$ bol vnútorným bodom úsečky $MN$. Dokážte, že
štvoruholník $KLMN$ je tetivový práve vtedy, keď priamka~$MN$ je dotyčnicou kružnice opísanej
trojuholníku $AKL$. [60--A--I--3]

Označme $I$ stred kružnice vpísanej trojuholníku $ABC$. Kružnica, ktorá
prechádza vrcholom~$B$ a~dotýka sa priamky~$AI$ v~bode~$I$, pretína strany $AB$, $BC$ postupne
v~bodoch $P$, $Q$. Priesečník priamky~$QI$ so stranou~$AC$ označme~$R$.
Dokážte, že platí $$|AR|\cdot|BQ|=|PI|^2.$$ [62--A--I--5]
\endnávod
}

{%%%%%   A-I-4
Je zrejmé, že pre každé prirodzené číslo~$k$ platia rovnosti
$l(2k)=l(k)$ {a} $l(2k-1)=2k-1$. Vďaka nim je možné hodnoty
$l(n)$ sčítať po skupinách čísel~$n$ ležiacich vždy medzi dvoma
susednými mocninami čísla~$2$, presnejšie určovať súčty
$$
s(n)=l\bigl(2^{n-1}+1\bigr)+l\bigl(2^{n-1}+2\bigr)+l\bigl(2^{n-1}+3\bigr)
+\dots+l\bigl(2^n\bigr)
$$
postupne pre jednotlivé $n=1, 2, 3, \dots$. Pre názornosť najskôr
uveďme postup určenia konkrétneho súčtu
$$
s(4)=l(9)+l(10)+l(11)+l(12)+l(13)+l(14)+l(15)+l(16).
$$
Vklad jeho sčítancov $l(2k-1)$ je rovný
$$
l(9)+l(11)+l(13)+l(15)=9+11+13+15=\frac42\cdot(9+15)=48
$$
(naznačili sme použitie vzorca pre súčet niekoľkých členov
aritmetickej postupnosti), vklad sčítancov $l(2k)$
má hodnotu
$$
l(10)+l(12)+l(14)+l(16)=l(5)+l(6)+l(7)+l(8).
$$
To je ale "predošlý" súčet $s(3)$, ktorý sme už skôr mohli
určiť zo súčtu $s(1)=1$ podobným, teraz stručnejšie zapísaným postupom:
$$
\postdisplaypenalty 10000
s(3)=l(5)+l(6)+l(7)+l(8)=5+7+s(2)=12+3+s(1)=15+1=16.
$$
Teraz už ľahko dopočítame $s(4)=48+s(3)=64$.

Vykonaný výpočet nás privádza k~hypotéze, že pre každé~$n$
platí $s(n)=4^{n-1}$. Dokážeme ju matematickou indukciou. Ak platí
$s(n)=4^{n-1}$ pre určité $n$ (ako to je pre $n=1, 2, 3$), potom
pre súčet $s(n+1)$ podľa nášho postupu dostaneme
$$\align
s(n+1)&=l\bigl(2^{n}+1\bigr)+l\bigl(2^{n}+2\bigr)+l\bigl(2^{n-1}+3\bigr)+\dots
+l\bigl(2^{n+1}\bigr)=\\
&=\bigl[(2^{n}+1\bigr)+(2^{n-1}+3\bigr)+\dots+\bigl(2^{n+1}-1\bigr)\bigr]
+s(n)=\\
&=\frac{2^{n-1}}{2}\bigl(2^{n}+1+2^{n+1}-1\bigr)+4^{n-1}=
2^{n-2}\cdot3\cdot2^n+4^{n-1}=4^n.
\endalign$$
(Využili sme to, že počet nepárnych čísel od $2^{n}+1$ do $2^{n+1}-1$
vrátane je rovný~$2^{n-1}$.) Dôkaz indukciou je hotový.

Podľa dokázaného vzorca $s(n)=4^{n-1}$ pre súčet zo zadania úlohy
platí
$$
\align
l(1)+&l(2)+l(3)+\dots+l\bigl(2^{2013}\bigr)=l(1)+s(2)+s(3)+\dots+s(2\,013)=\\
&=1+1+4+4^2+4^3+\dots+4^{2012}=1+\frac{4^{2013}-1}{3}=\frac{4^{2013}+2}{3}.
\endalign
$$

\poznamka
Za pozornosť stojí, že vzorec $s(n)=4^{n-1}$
z~podaného riešenia je špeciálnym prípadom celkom prekvapivého vzťahu
$$
l(k+1)+l(k+2)+l(k+3)+\dots+l(2k)=k^2,
\tag1
$$
ktorý sa dá pre každé prirodzené číslo $k$ dokázať bez použitia indukcie
nasledujúcou elegantnou úvahou:
Všetkých $k$ sčítancov na ľavej strane \thetag1 sú zrejme čísla
z~$k$-prvkovej množiny $\{1, 3, 5, \dots, 2k-1\}$ a~sú po dvoch rôzne,
lebo podiel žiadnych dvoch čísel z~množiny príslušných argumentov
$\{k+1, k+2, \dots, 2k\}$ nie je mocninou čísla~$2$.
Preto je (až na poradie sčítancov) na ľavej strane~\thetag1
súčet $1+3+5+\dots+(2k-1)$, ktorý má skutočne hodnotu $k^2$.

\návody
Určte, pre ktoré celé kladné $n$ má číslo $l(n)$ päťkrát
menej deliteľov ako samotné číslo $n$ (počítame všetky kladné delitele).
[Čísla $n$ deliteľné $2^4$, ale nie $2^5$.]

Ukážte, že pre každé celé kladné $n$ platí
$n+1\leqq l(n)+l(n+1)\leqq\frac12(3n+2)$. [Využite to, že pre párne $n$
platí $1\leqq l(n)\leqq\frac12n$ a~$l(n+1)=n+1$ a pre nepárne $n$
je $l(n)=n$ a~$1\leqq l(n+1)\leqq\frac12(n+1)$.]

Dokážte, že ak vyberieme $n+1$ rôznych čísel z~množiny
$\mm M=\{1, 2,\dots, 2n\}$, bude niektoré vybrané číslo deliteľom iného
vybraného čísla. [Hodnota $l(k)$ pre ľubovoľné $k\in\mm M$ je
jedno z~$n$ nepárnych čísel $1, 3,\dots, 2n-1$, takže platí $l(a)=l(b)$
pre niektoré dve z~vybraných čísel $a>b$. Podiel $a:b$ je potom
mocninou čísla~$2$.]
\endnávod
}

{%%%%%   A-I-5
Namiesto plochy $3\times10$ uvažujme všeobecnú plochu $3\times2n$,
kde $n$ je prirodzené číslo, a~označme $a_n$ počet všetkých
spôsobov vydláždenia tejto plochy dlaždicami
$2\times1$ (ktorých potrebujeme $3n$~kusov).\footnote{Určite je zrejmé,
prečo uvažujeme iba plochy $3\times k$ s~párnym parametrom~$k$.}
Hoci je našou úlohou nájsť z~čísel~$a_n$ iba jediné, totiž
číslo $a_5$, nie je uvažované zovšeobecnenie samoúčelné.
Ukáže sa totiž, že každú jednotlivú hodnotu~$a_n$ bude možné
vypočítať podľa jednoduchého vzorca, ak budeme poznať dve
predchádzajúce hodnoty $a_{n-1}$ a~$a_{n-2}$.\footnote{Spomenutý
jednoduchý vzorec nájdete pod číslom \thetag2 až v~poznámke
za skončeným riešením. V~ňom totiž budeme počítať
rekurentne čísla $a_n$ súčasne s~istými vhodnými číslami~$b_n$,
bez ktorých by výklad našej rekurentnej metódy bol menej prehľadný.}
Tak je možné zo "začiatočných" hodnôt $a_1$, $a_2$ vypočítať najprv $a_3$,
potom $a_4$ atď. až po hľadané~$a_n$.
Takému postupu (bežnému v~rade kombinatorických situácií\niedorocenky{,
pozri návodné úlohy}) hovoríme {\it rekurentná metóda\/} alebo
tiež {\it procedúra rekurzie}.

Hodnotu $a_1=3$ je síce ľahké určiť nakreslením všetkých
možných vydláždení plochy $3\times2$ (\obr), ale podobný postup
na určenie hodnoty $a_2=11$ by bol dosť prácny. Namiesto toho ešte
zavedieme i~pre naše ďalšie rekurentné úvahy vhodné čísla
$b_n$: každé číslo~$b_n$ bude označovať
počet všetkých spôsobov neúplného vydláždenia plochy
$3\times(2n-1)$ dlaždicami $2\times1$ v~počte $3n-2$ kusov,
pri ktorom ostane jedno nepokryté políčko $1\times1$
v~{\it konkrétne zvolenom\/} rohu celej plochy, povedzme vpravo dole.
Vďaka osovej súmernosti vyjde rovnaký počet $b_n$ spôsobov
i~pri požiadavke, aby nepokryté pole ostalo v~pravom hornom rohu.
\insp{a63.2}%

K~hodnote $a_1=3$ pripojme zrejmú hodnotu $b_1=1$ a~prejdime
k~vlastnej rekurentnej metóde. Odvodíme pri nej, že pre každé
celé $n>1$ platia rovnosti
$$
b_n=a_{n-1}+b_{n-1}\quad\text{a}\quad
a_n=a_{n-1}+2b_n.
\tag1$$
Prvú rovnosť z~\thetag1 dokážeme tak, že všetky vydláždenia plochy
$3\times(2n-1)$ s~"odseknutým" rohom $1\times1$ vpravo dole
rozdelíme do dvoch disjunktných skupín podľa toho, či je pravý
horný roh $1\times1$ pokrytý zvislou dlaždicou (vydláždenie typu~A na \obr),
alebo vodorovnou dlaždicou (vydláždenie typu~B, pri ktorom je
vynútená poloha ďalších dvoch, spolu teda troch vodorovných
dlaždíc, nakreslených na \obrr1). Počet
vydláždení typu~A je zrejme rovný počtu vydláždení
zvyšnej plochy $3\times(2n-2)$,
% (jež je z~původní plochy na obrázku vymezena čárkovanou čarou),
teda číslu $a_{n-1}$. Podobne počet
vydláždení typu~B je rovný počtu vydláždení plochy
$3\times(2n-3)$ s~odseknutým pravým dolným rohom $1\times1$, teda
číslu~$b_{n-1}$. Tým je prvá rovnosť z~\thetag1 dokázaná.
\insp{a63.3}%

Druhú rovnosť z~\thetag1 overíme podobne so stručnejším komentárom:
všetky vydláždenia plochy
$3\times2n$ rozdelíme do troch disjunktných skupín~-- vydláždenia
typov C, D a~E znázornených na \obr. Všetkých vydláždení typu~C je
zrejme $a_{n-1}$, počty vydláždení typov D a~E sa rovnajú
tomu istému číslu~$b_n$ (to sme zdôraznili už skôr).
Tým je i~dôkaz druhej rovnosti z~\thetag1 skončený.
\insp{a63.4}%

Máme všetko pripravené na výpočet hľadaného čísla $a_5$. Z~hodnôt
$a_1=3$, $b_1=1$ a~rovností \thetag1 postupne dostaneme
$$
\gather
b_2=a_1+b_1=4,\ a_2=a_1+2b_2=11,\ b_3=a_2+b_2=15,\ a_3=a_2+2b_3=41,\\
b_4=a_3+b_3=56,\ a_4=a_3+2b_4=153,\ b_5=a_4+b_4=209,\ a_5=a_4+2b_5=571.
\endgather
$$

\odpoved
Hľadaný počet spôsobov vydláždenia plochy $3\times 10$
je rovný $571$.

\poznamka
Ako sme sľúbili v~úvode riešenia, ukážeme teraz,
že skúmané počty $a_n$ všetkých vydláždení plochy $3\times2n$
dlaždicami $2\times1$ vyhovujú rekurentnej rovnici
$$
a_{n+2}=4a_{n+1}-a_{n}\quad\text{pre každé}\ n\geqq1,
\tag2
$$
takže ich môžeme počítať "samostatne", teda bez súčasného výpočtu
čísel~$b_n$, ktorými sme si pomohli v~podanom riešení.
Odvodenie \thetag2 urobíme algebraicky,
totiž vylúčením čísel $b_n$ zo vzťahov~\thetag1. Podľa nich platí
$$
\align
a_{n+2}&=a_{n+1}+2b_{n+2}=a_{n+1}+2(a_{n+1}+b_{n+1})=\\
&=3a_{n+1}+2b_{n+1}=3a_{n+1}+(a_{n+1}-a_{n})=4a_{n+1}-a_{n}.
\endalign
$$
Dodajme ešte, že zo základov diskrétnej matematiky je známe, že
každá postupnosť čísel $(a_n)_{n=1}^{\infty}$,
ktorá vyhovuje \thetag2, má vyjadrenie
$a_n=C_1\la_1^{n}+C_2\la_2^{n}$, kde $\la_{1, 2}=2\pm\sqrt3$ sú
korene kvadratickej rovnice $\la^2=4\la-1$ a~$C_{1, 2}$ sú
ľubovoľné konštanty. Tie je možné pre našu situáciu určiť z~hodnôt
$a_1=3$ a~$a_2=11$ a~dostať sa tak ku konečnému vzorcu pre počet
$a_n$ všetkých vydláždení plochy $3\times2n$ s~všeobecným~$n$ v~tvare
$$
a_n=\frac{3+\sqrt3}{6}\cdot\bigl(2+\sqrt3\bigr)^n+
    \frac{3-\sqrt3}{6}\cdot\bigl(2-\sqrt3\bigr)^n.
$$

\ineriesenie
Opíšeme ešte jednu rekurentnú metódu na určovanie počtov $a_n$
všetkých spôsobov vydláždenia plochy $3\times2n$ dlaždicami
$2\times1$. Bude nás teraz zaujímať, či také vydláždenie
je "zlepené" z~vydláždení dvoch plôch $3\times k$ a~$3\times(2n-k)$
pre vhodné $k=1, 2,\dots, 2n-1$, ktoré musí byť zrejme párne. Obe
plochy vzniknú z~pôvodnej plochy $3\times2n$
jedným priamym rezom dĺžky~$3$; pýtame sa teda, či pri niektorom
takom reze žiadnu uloženú dlaždicu nerozpolíme.
Pokiaľ sa to nestane, teda pokiaľ pri
každom takom reze aspoň jednu dlaždicu rozpolíme,
povieme, že pôvodné vydláždenie plochy $3\times2n$ je
{\it celistvé}.

Je zrejmé, že každé z~troch vydláždení plochy $3\times2$ je
celistvé (\obrr3). Vysvetlíme, prečo pri každom $n\geqq2$
existujú práve dve celistvé vydláždenia plochy $3\times2n$. Pri
jej ľavom okraji musia byť umiestnené dlaždice jedným z~dvoch
spôsobov nakreslených na \obr. Pokrytie prvého stĺpca $3\times1$
totiž nemôže byť zabezpečené tromi vodorovnými dlaždicami, ale
jednou zvislou a~jednou vodorovnou dlaždicou, ktoré sú v~oboch
možných situáciách vyfarbené na sivo. Tieto dve dlaždice vynucujú
vodorovné umiestnenie ďalších troch dlaždíc s~vpísanou cifrou $1$ (inak
by skúmané vydláždenie nebolo celistvé, bolo by totiž zlepené
z~vydláždení plôch $3\times2$ a~$3\times(2n-2)$). Ak je $n=2$, sme
s~celým rozborom hotoví; v~prípade $n\geqq3$ je vynútené
vodorovné umiestnenie ďalších troch dlaždíc s~vpísanou cifrou~$2$.
Opakovaním tejto úvahy nakoniec získame (jediné) dve
celistvé vydláždenia celej plochy $3\times2n$ pre ľubovoľné dané $n\geqq2$.
\insp{a63.5}%

Zaoberajme sa teraz dláždeniami plochy $3\times2n$, ktoré nie sú celistvé.
Každé z~nich je teda zlepením vydláždení dvoch plôch $3\times2k$
a~$3\times(2n-2k)$ pre vhodné $k=1, 2,\dots,n-1$.
Také $k$ ale nemusí byť pre dané
vydláždenie jediné, a~tak vyberieme vždy "najväčšie"
vyhovujúce $k$; bude to práve také $k$, pri ktorom už je vyššie
spomenuté vydláždenie "pravej" plochy $3\times(2n-2k)$ celistvé
(pre najväčšie $k=n-1$ je to splnené automaticky),
pričom vydláždenie "ľavej" plochy $3\times2k$ je ľubovoľné.

Práve opísaným spôsobom sme všetky "necelistvé" vydláždenia
plochy $3\times2n$ s~daným $n\geqq2$ rozdelili
do $n-1$ disjunktných skupín. Ich celkový počet je
rovný, ak počítame po skupinách,
$$
a_1\cdot2+a_2\cdot2+\dots+a_{n-2}\cdot2+a_{n-1}\cdot3,
$$
lebo prvý činiteľ v~$k$-tom sčítanci
udáva vždy počet (všetkých) vydláždení "ľavej" plochy $3\times2k$
a~druhý činiteľ udáva počet celistvých vydláždení
"pravej" plochy ${3\times(2n-2k)}$. Ak pridáme k~tomuto súčtu
ešte sčítanec $2$ za dve celistvé vydláždenia celej plochy
$3\times2n$, dostaneme pre každé $n\geqq2$ hľadaný rekurentný vzťah
$$
a_n=2(a_1+a_2+\dots+a_{n-2})+3a_{n-1}+2.
\tag3
$$
Odtiaľ sa od hodnoty $a_1=3$ postupne dostaneme až
k~hľadanej hodnote~$a_5$, a~tak skončíme alternatívne riešenie celej
zadanej úlohy:
$$
\gather
a_2=3a_1+2=11,\ a_3=2a_1+3a_2+2=41,\ a_4=2(a_1+a_2)+3a_3+2=153,\\
a_5=2(a_1+a_2+a_3)+3a_4+2=571.
\endgather
$$

\poznamka
Ukážme ešte, že pomerne zložito zapísanú rekurentnú
závislosť \thetag3 je možné podobne ako v~prvom riešení zjednodušiť na tvar rovnice \thetag2
uvedenej v~prvej poznámke. Naozaj, podľa \thetag3
pre ľubovoľné  $n\geqq1$ platí
$$\align
a_{n+2}&=2(a_1+a_2+\dots+a_n)+3a_{n+1}+2,\\
a_{n+1}&=2(a_1+a_2+\dots+a_{n-1})+3a_n+2
\endalign$$
a~po odčítaní druhej rovnosti od prvej vychádza
$$
a_{n+2}-a_{n+1}=3a_{n+1}-a_n\quad\text{čiže}\quad
a_{n+2}=4a_{n+1}-a_n.
$$

\návody
Nájdite rekurentný vzťah pre určovanie počtov $s(n)$ spôsobov
zdolania schodiska o~$n$~schodoch, ak sú dovolené iba kroky
o~jeden, dva alebo tri schody. [$s(n)=s(n-1)+s(n-2)+s(n-3)$ pre každé
$n>3$.]

Ukážte, že počty spôsobov vydláždenia plochy $2\times n$
dlaždicami $2\times1$ pre jednotlivé $n=1, 2, 3,\dots$ tvoria
známu Fibonacciho postupnosť
$$
1,\ 2,\ 3,\ 5,\ 8,\ 13,\ 21,\ 34,\ 55,\ \dots,
$$
v~ktorej je každý člen (počnúc tretím) rovný súčtu dvoch
predchádzajúcich členov. [Všetky vydláždenia plochy $2\times n$ pre
dané $n\geqq3$ rozdeľte do dvoch skupín podľa toho, či pravý
okraj dláždenia je tvorený jednou zvislou alebo dvoma vodorovnými
dlaždicami. Vysvetlite, prečo v~každej z~oboch skupín je práve toľko
vydláždení, koľko je všetkých vydláždení plôch $2\times(n-1)$,
resp. $2\times(n-2)$.]

Označme $p(n)$
počet všetkých $n$-ciferých čísel zložených len z~cifier
$1$, $2$, $3$, $4$, $5$, v~ktorých
sa každé dve susedné cifry líšia aspoň o~$2$.
Dokážte, že pre každé prirodzené číslo~$n$ platí
$$
5\cdot2{,}4^{n-1}\le p(n)\le 5\cdot2{,}5^{n-1}.
$$
[62--A--I--3]

Pre ľubovoľné prirodzené číslo $n$ zostavme
z~písmen $A$, $B$ všetky možné "slová" dĺžky~$n$.
Rozdeľme ich do dvoch skupín $S_n$ a~$L_n$ podľa toho, či je v~danom slove
párny alebo nepárny počet "slabík" $BA$ (za párny
považujeme i~počet $0$). Napríklad slová
$\underline{BA}BBB\underline{BA}$ a~$AAAAAAB$ patria obe
do skupiny $S_7$, ale slová $AAB\underline{BA}BB$
a~$\underline{BA}\,\underline{BA}A\underline{BA}$ patria obe
do skupiny $L_7$. Určte, pre ktoré $n$ majú skupiny
$S_n$ a~$L_n$ rovnaký počet prvkov. [56--A--I--3]

Pre ľubovoľné prirodzené číslo $n$ zostavme z~písmen $A$~a~$B$
všetky možné "slová" dĺžky~$n$ a~označme $p_n$ počet tých z~nich,
ktoré neobsahujú ani trojicu $AAA$ po sebe idúcich písmen $A$
ani dvojicu $BB$ po sebe idúcich písmen $B$. Určte,
pre ktoré prirodzené čísla~$n$ platí, že obe čísla
$p_n$ a~$p_{n+1}$ sú párne. [53--A--II--2]

Pre ľubovoľné prirodzené číslo $n$ zostavme z~písmen $A$~a~$B$ všetky možné
"slová" dĺžky~$n$ a~označme $p_n$ počet tých z~nich,
ktoré neobsahujú ani štvoricu $AAAA$ po sebe idúcich písmen $A$
ani trojicu $BBB$ po sebe idúcich písmen~$B$. Určte hodnotu výrazu
$$\frac{p_{2004}-p_{2002}-p_{1999}}{p_{2001}+p_{2000}}.
$$
[53--A--III--2]
\endnávod
}

{%%%%%   A-I-6
Zvolíme akýkoľvek bod~$X$ roviny $ABC$
a~zostrojíme jeho obrazy $X_a$, $X_b$, $X_c$ v~osových súmernostiach
podľa priamok $BC$, $CA$, $AB$ (\obr). Aby sme mohli
posúdiť otázku, kedy dostaneme rovnostranný trojuholník $X_aX_bX_c$,
dokážeme najprv, že pre vzájomné vzdialenosti jeho vrcholov platia
všeobecne (\tj. bez ohľadu na voľbu bodu~$X$) vzorce
$$
|X_aX_b|=2|XC|\sin\ga,\quad
|X_aX_c|=2|XB|\sin\be,\quad
|X_bX_c|=2|XA|\sin\al,
\tag1$$
v~ktorých $\al$, $\be$, $\ga$ je zvyčajné označenie
vnútorných uhlov trojuholníka $ABC$.
\insp{a63.6}%

Stačí dokázať iba prvú rovnosť v~\thetag1. Tá je zrejmá
v~prípade $X=C$, lebo vtedy platí  $X_a=X_b$ $(=X)$. V~prípade $X\ne C$
je $XC$ priemerom Tálesovej kružnice z~\obrr1, na ktorej ležia
vyznačené kolmé priemety $P_a$, $P_b$ bodu~$X$ na priamky $BC$, $CA$.
Keďže tetive~$P_aP_b$ prislúchajú obvodové uhly $\ga$
a~$180\st-\ga$, zo sínusovej vety vyplýva rovnosť $|P_aP_b|=|XC|\sin\ga$.
Vzhľadom na to, že úsečka~$X_aX_b$ je zrejme obrazom
tetivy~$P_aP_b$ v~rovnoľahlosti so stredom v~bode~$X$
a~koeficientom~$2$, platí $|X_aX_b|=2|P_aP_b|$, čím sú
rovnosti~\thetag1 dokázané.

Zo vzorcov \thetag1 vyplýva, že našou úlohou je nájsť
práve tie body~$X$ roviny $ABC$, pre ktoré platí
$$
2|XA|\sin\al=2|XB|\sin\be=2|XC|\sin\ga>0
$$
(práve vtedy je totiž trojuholník $X_aX_bX_c$ rovnostranný). Inak
vyjadrené, hľadáme body~$X$, ktorých vzdialenosti od vrcholov $A$,
$B$, $C$ sú kladné a~spĺňajú úmeru
$$
|XA|:|XB|:|XC|=\frac{1}{\sin\al}:\frac{1}{\sin\be}:\frac{1}{\sin\ga}
=\frac{1}{|BC|}:\frac{1}{|AC|}:\frac{1}{|AB|}
$$
(kvôli ďalšiemu výkladu sme využili sínusovú vetu a~prešli od sínusov
uhlov k~dĺžkam protiľahlých strán v~trojuholníku $ABC$).
Práve také body~$X$ zostrojíme ako spoločné body nasledujúcich
troch Apollóniových kružníc, presnejšie množín bodov~$X$ zadaných rovnicami
$$
k_a\: \frac{|XB|}{|XC|}=\frac{|AB|}{|AC|},\quad
k_b\: \frac{|XA|}{|XC|}=\frac{|AB|}{|BC|},\quad
k_c\: \frac{|XA|}{|XB|}=\frac{|AC|}{|BC|};
\tag2$$
pritom je zrejmé, že ľubovoľným priesečníkom dvoch týchto kružníc
bude prechádzať i~tretia kružnica.\footnote{Niektoré z~týchto množín
(jedna alebo tri) môžu byť priamkami namiesto kružnicami.
Budeme sa tomu venovať neskoršie pri diskusii o~počte riešení.}
Z~rovností v~\thetag2 vidno, že $A\in k_a$,
$B\in k_b$ a~$C\in k_c$. To uľahčuje praktickú konštrukciu
týchto troch kružníc: ak sú $AK$, $BL$, $CM$ priečky trojuholníka $ABC$,
na ktorých ležia osi jeho vnútorných uhlov (\obr),
leží na kružnici~$k_a$ nielen bod~$A$, ale i~bod~$K$
(v~dôsledku dobre známeho pomeru $|KB|:|KC|=|AB|:|AC|$), takže
stred kružnice~$k_a$ môžeme zostrojiť ako priesečník osi úsečky~$AK$
s~priamkou~$BC$ (ak neplatí $|AB|=|AC|$,
kedy $k_a$ je os strany~$BC$). Podobne využitím osí úsečiek $BL$, $CM$
určíme stredy kružníc $k_b$, $k_c$.
Na \obrr1{} vidíme situáciu, keď kružnice $k_a$, $k_b$,
$k_c$ majú dva spoločné body $X$, $Y$, a~naša úloha tak má dve
riešenia.\footnote{Ako uvidíme o~chvíľu, dve riešenia zadanej úlohy budú
existovať vždy, keď daný trojuholník $ABC$ nebude rovnostranný.
Netriviálnosť tohto poznatku nepriamo podporuje i~skutočnosť,
že obe riešenia $X$ a~$Y$ na \obrr1{} ležia mimo trojuholníka $ABC$.}
Sú na ňom nakreslené i~zodpovedajúce rovnostranné trojuholníky
$X_aX_bX_c$ a~$Y_aY_bY_c$. Nie je náhoda, že ich vrcholy ležia
po jednom na zodpovedajúcich kružniciach $k_a$, $k_b$ a~$k_c$,
lebo stredy týchto kružníc ležia na osiach uvedených súmerností.
\insp{a63.7}%

Aj keď je zadaná úloha {\it konštrukčne\/} vyriešená, musíme
ešte posúdiť otázku, koľko má úloha riešení, teda
otázku počtu spoločných bodov Apollóniových kružníc
$k_a$, $k_b$, $k_c$ (ako vieme, stačí vziať dve z~nich).
Odpoveď podáme v~nasledujúcej časti; pomôžeme si pritom opäť
poznatkom o~existencii priečok daného trojuholníka $ABC$, ktoré sú zároveň
tetivami skúmaných kružníc; zachováme aj ich označenie $AK$,
$BL$, $CM$ z~\obrr1.

\diskusia
\item{a)}
Ak je trojuholník $ABC$ {\it rovnostranný}, sú $k_a$, $k_b$, $k_b$ osi jeho
strán a~úloha tak má jediné riešenie, ktorým je stred daného
rovnostranného trojuholníka.
\item{b)}
Ak je trojuholník $ABC$ {\it rovnoramenný\/} a~platí napríklad
$|AB|\ne|AC|=|BC|$, je $k_c$ os jeho základne~$AB$, ktorá pretne
kružnicu $k_a$ v~dvoch bodoch, lebo pretína jej tetivu~$AK$,
a~teda i~oba jej oblúky~$AK$.
Pre rovnoramenný trojuholník $ABC$, ktorý nie je rovnostranný, má preto
úloha vždy dve riešenia.
\item{c)}
Ak je trojuholník $ABC$ {\it rôznostranný\/} a~napríklad $AB$ je jeho
najdlhšia strana (ako na \obrr1),
majú kružnice $k_a$, $k_b$, ako ukážeme, dva spoločné body.
%Využijeme k tomu tětivou $BL$ kružnice $k_b$.
Keďže kružnica~$k_a$ je určená rovnicou $|XB|/|XC|=|AB|/|AC|>1$,
leží bod~$B$ v~jej vonkajšej oblasti a~bod~$C$ v~jej
vnútornej oblasti. Preto vo vnútornej oblasti~$k_a$ leží aj celá úsečka~$AC$
(s~výnimkou bodu~$A$), teda aj jej vnútorný bod~$L$.
Kružnica~$k_a$ tak pretína tetivu~$BL$ kružnice~$k_b$, a~preto
sa pretínajú aj obe kružnice. Pre rôznostranný trojuholník $ABC$ má preto úloha
vždy dve riešenia.

\návody
\titem
Súťažná úloha sa vzťahuje k~študijnej téme tohto ročníka MO
kategórie~A, ktorým sú {\it Apollóniove kružnice}, čiže
kružnice, ktoré spájame s~nasledovným tvrdením.
{\endgraf}
{\sl Predpokladajme, že v~rovine~$\pi$ sú dané dva rôzne body $A$, $B$
a~že je ešte dané kladné číslo~$\la\ne1$. Potom množina
$$
\mm M=\{X\in\pi;\,|AX|:|BX|=\la\}
$$
je kružnica so stredom na priamke~$AB$.}
Práve táto množina $\mm M$ sa nazýva Apollóniovou kružnicou
(prislúchajúcou daným bodom $A$ a~$B$ a~danému pomeru~$\la$).
{\endgraf}
Dôkaz uvedeného tvrdenia
predkladáme nižšie ako návodnú úlohu~1
(doplnenú riešením). Dodajme ešte, že v~prípade pomeru $\la=1$
je $\mm M$ zrejme os úsečky~$AB$.
Pre podrobnejšie zoznámenie sa s~témou odporúčame
text na str.~11--14. brožúrky L.~Boček, J.~Zhouf:
{\it Máte rádi kružnice?}, Prometheus, Praha, 1995.

Dokážte vyššie uvedené tvrdenie o~množine $\mm M$ bodov v~rovine.
[Najskôr ukážeme, že pre ľubovoľný bod $X\in\mm M$, ktorý neleží
na priamke~$AB$, platí:
{\sl Osi vnútorného a~vonkajšieho uhla pri vrchole~$X$
trojuholníka $ABX$ pretínajú priamku~$AB$ postupne v~bodoch $P$ a~$Q$,
ktoré nezávisia od výberu bodu~$X$ a~ktoré samotné patria do množiny~$\mm M$}
(sú to jediné dva body z~$\mm M$, ktoré na priamke~$AB$ ležia).
Skutočne, zo sínusovej vety pre
trojuholníky $APX$ a~$BPX$ vyplýva $|AP|:|AX|=\sin\frac12\ga:\sin\phi$
a~$|BP|:|BX|=\sin\frac12\ga:\sin(180\st-\phi)$, kde
$\ga=|\uhel AXB|$ a~$\phi=|\uhel APX|$,
odkiaľ $|AP|:|BP|=|AX|:|BX|=\la$. Podobne sa odvodí i~druhá
úmera $|AQ|:|BQ|=\la$. Z~dokázaného tvrdenia o~bodoch $P$ a~$Q$
vyplýva, že každý bod množiny~$\mm M$ nutne leží na Tálesovej
kružnici zostrojenej nad (fixným) priemerom~$PQ$. Ostáva ukázať
opačné tvrdenie o~tom, že každý bod~$X$ tejto kružnice patrí do
množiny~$\mm M$, že teda spĺňa úmeru $|AX|:|BX|=\lambda$. Na to podľa
predošlého stačí predpokladať, že $X$ neleží na priamke~$AB$,
a~dokázať rovnosť veľkostí uhlov $\ga_1=|\uhel PXA|$
a~$\ga_2=|\uhel PXB|$. Urobíme to v~prípade $\la>1$, keď
uvažované body ležia na priamke v~poradí $A$, $P$, $B$, $Q$
(prípad $0<\la<1$ sa posúdi analogicky, alebo sa
prehodením bodov $A$, $B$ prejde od pomeru $\la$ k~pomeru~$1/\la$.)
Keďže uhly $AXQ$, $BXQ$ majú postupne veľkosti
$90\st+\ga_1$, $90\st-\ga_2$, zo sínusovej vety vyplývajú rovnosti
$$
\la=\frac{|AP|}{|BP|}=\frac{|AX|\sin\ga_1}{|BX|\sin\ga_2}
\quad\text{a}\quad
\la=\frac{|AQ|}{|BQ|}=\frac{|AX|\cos\ga_1}{|BX|\cos\ga_2},
$$
z~ktorých vychádza $\tg\ga_1=\tg\ga_2$, čiže $\ga_1=\ga_2$. Celý
dôkaz tvrdenia o~množine~$\mm M$ je hotový.]

V~rovine~$\pi$ sú dané štyri rôzne body $A$, $B$, $C$, $D$
neležiace na jednej priamke. Zostrojte všetky body $X\in\pi$,
pre ktoré platí $\triangle ABX\sim \triangle CDX$.
[Koeficient $\la=|AB|:|CD|$ podobnosti poznáme;
preto hľadané body~$X$ zostrojíme ako priesečníky dvoch Apollóniových
kružníc: prvá z~nich prislúcha bodom $A$, $C$ a~pomeru $\la$,
druhá bodom $B$, $D$ a~tomu istému pomeru $\la$.]

V~rovine $\pi$ sú dané štyri rôzne body $A$, $B$, $C$, $D$
ležiace v~tomto poradí na jednej priamke.
Zostrojte všetky body $X\in\pi$, pre ktoré sú uhly
$AXB$, $BXC$, $CXD$ navzájom zhodné. [Hľadané body $X$ sú
priesečníky dvoch Apollóniových kružníc: prvá z~nich prislúcha bodom
$A$, $C$ a~prechádza bodom~$B$ (ten teda určuje príslušný deliaci
pomer), druhá prislúcha bodom $B$, $D$
a~prechádza bodom~$C$.]
\endnávod
}

{%%%%%   B-I-1
Označme~$S$ skúmanú hodnotu, \tj. súčet čísel na stranách 63-uholníka.
Ak priradíme každému vrcholu 63-uholníka číslo~$1$, dostaneme
$S = 63$, pretože ku každej jeho strane bude pripísané číslo~$1$.
Pritom je zrejmé, že ku každému zvolenému očíslovaniu vrcholov možno dôjsť
postupnou zmenou~$1$ na~$\m1$.

Teraz skúmajme, ako sa bude meniť hodnota~$S$, keď zmeníme hodnotu
v~niektorom vrchole 63-uholníka z~$1$ na $\m1$. Hodnoty v~susedných vrcholoch
označme $a$ a~$b$ (na ich poradí nezáleží). Spravenou zmenou zrejme zmeníme
zodpovedajúce čísla na dvoch stranách, ktoré z~tohto vrcholu vychádzajú (a~na žiadnych
iných). Pri výpočte hodnoty~$S$ sa tak zmenia hodnoty práve dvoch sčítancov.

Ak $a = b= 1$, zmenší sa súčet~$S$ o~$4$
(pred zmenou prispievali čísla $a$, $b$ hodnotou $a + b = 1 \cdot
1 +1 \cdot 1 = 2$ a~po zmene bude ich príspevok $1 \cdot (\m1) + (\m1) \cdot
1 = \m2$). Podobne rozoberieme aj ostatné možnosti:
% --- vždy měníme hodnotu čísla~$b$ na opačnou:
$$
\vbox{\let\\=\cr\everycr{\noalign{\hrule}}\offinterlineskip
 \def\@{\phantom{+}}\def\q{\omit\,\vrule}
 \halign{\vrule#\strut&&\hss\enspace#\unskip\enspace\hss\vrule\cr
% &$a$  & $b$ & $c$  &\q& změna $ab+bc$     &rozdíl v~hodnotě $S$ \\
% &@1   & @1  & @1   &\q&$@2\rightarrow -2$ & $-4$\\
% &@1   & @1  &$-1$  &\q&$@0\rightarrow @0$ & @0\\
% &$-1$ & @1  &$-1$  &\q&$-2\rightarrow +2$ & +4\\
% &@1   &$-1$ & @1   &\q&$-2\rightarrow +2$ & +4\\
% &@1   &$-1$ &$-1$  &\q&$@0\rightarrow @0$ & @0\\
% &$-1$ &$-1$ &$-1$  &\q&$@2\rightarrow -2$ & $-4$\\
&$a$  & $b$   &\q& zmena $a+b$     &rozdiel v~hodnote $S$ \\
&\@1   & \@1    &\q&$\@2\rightarrow -2$ & $-4$\\
&\@1   &$-1$   &\q&$\@0\rightarrow \@0$ & \@0\\
&$-1$ &$\@1$   &\q&$\@0\rightarrow \@0$ & \@0\\
&$-1$ &$-1$   &\q&$-2\rightarrow \@2$ & \@4\\
}}
$$
Vidíme teda, že zmena jedného čísla vo vrchole
nemení zvyšok čísla~$S$ po delení štyrmi. V~úvode sme zistili, že
jedna z~dosiahnuteľných hodnôt je $S = 63$, ktorá dáva po
delení štyrmi zvyšok~$3$, preto aj najmenšia nezáporná hodnota musí dávať rovnaký
zvyšok~$3$, takže najmenšiu možnú hodnotu súčtu~$S$ budeme hľadať medzi číslami
$\{3,7,11,\dots\}$.

Ako sa ľahko presvedčíme,  dá sa dosiahnuť hodnota $S = 3$:
stačí do vrcholov 63-uholníka umiestniť nasledujúcich 63~čísel v~uvedenom poradí
$$
\underbrace{1,1,-1,-1,1,1,-1,-1,\dots,1,1,-1,-1}_{\text{60 čísel}},1,1,1
$$
a~dostaneme $S = 3$. Ten istý súčet možno dosiahnuť aj inou
voľbou čísel.


\návody
Do vrcholov pravidelného 63-uholníka vpíšeme ľubovoľným spôsobom
32~jednotiek a~31~núl, pričom do každého vrcholu vpíšeme jedno
číslo. Ku každej jeho strane pripíšeme súčin čísel v~jej vrcholoch
a~všetky čísla pri jednotlivých stranách sčítame. Nájdite najmenšiu
hodnotu, ktorú môže nadobúdať tento súčet. [Súčet je nezáporný,
a~keďže jednotiek je viac ako núl, na obvode budú vedľa seba aspoň dve
jednotky, preto je každý súčet aspoň~$1$ a~takýto súčet možno dosiahnuť
pravidelným striedaním čísel $1$ a~$0$ vo vrcholoch strán: začneme v~ľubovoľnom
vrchole jednotkou a~skončíme jednotkou v~susednom vrcholu z~opačnej strany.]

Každému vrcholu pravidelného $n$-uholníka priradíme číslo~$\m1$.
V~jednom kroku je dovolené zmeniť dve susedné čísla na opačné. Zistite,
pre aké hodnoty~$n$ je možné opakovaním krokov dosiahnuť, aby boli
všetky čísla~$+1$. [Pre párne $n$ to je možné, pre nepárne~$n$ to možné
nie je. Pri zmene dvoch čísel sa nezmení zvyšok súčtu všetkých
čísel po delení štyrmi a~rozdiel medzi súčtom všetkých čísel
v~požadovanej a~v~počiatočnej pozícii je~$2n$.]

\D
Na každej stene kocky je napísané práve jedno celé číslo. V~jednom kroku
zvolíme ľubovoľné dve susedné steny kocky a~čísla na nich napísané
zväčšíme o~$1$. Určte nutnú a~postačujúcu podmienku pre očíslovanie stien kocky
na začiatku, aby po konečnom počte vhodných krokov mohli byť na
všetkých stenách kocky rovnaké čísla.
[60--A--I--5]

V~každom vrchole pravidelného 2008-uholníka leží jedna minca.
Vyberieme dve mince a~premiestnime každú z~nich do susedného vrcholu
tak, že jedna sa posunie v~smere
a~druhá proti smeru chodu hodinových ručičiek.
Rozhodnite, či je možné týmto spôsobom všetky mince postupne
presunúť:
 a) na 8~kôpok po 251 minciach,
 b) na 251 kôpok po 8~minciach.   [58--A--I--5]

V~každom z~vrcholov pravidelného $n$-uholníka $A_1A_2\dots A_n$ leží určitý počet mincí: vo vrchole~$A_k$ je to práve $k$~mincí, $1\le k\le n$. Vyberieme dve mince a~preložíme každú z~nich do susedného vrcholu tak, že jedna sa posunie v~smere a~druhá proti smeru chodu hodinových ručičiek. Rozhodnite, pre ktoré $n$ možno po konečnom počte takých preložení dosiahnuť, že pre ľubovoľné $k$, $1\le k\le n$, bude vo vrchole~$A_k$ ležať $n+1-k$~mincí.
   [58--A--III--5]
\endnávod
}

{%%%%%   B-I-2
Zrejme $x$ ani $y$ nesmie byť nula, pretože sa nachádzajú v~menovateli
zlomkov v~zadanej nerovnici. V~zadaní nie je zmienka o~znamienku
čísel $x$ a~$y$, preto nesmieme pri úpravách zabúdať na to, že tieto
čísla môžu byť aj záporné.

Pokúsime sa najskôr zjednodušiť výrazy v~zadaní. Prenásobením
nerovnice kladným číslom $x^2y^2$ sa ekvivalentne zbavíme menovateľov:
$$
\align
(x+y)\frac{x+y}{xy}\ge&\biggl(\frac{x^2+y^2}{xy}\biggr)^{\!\!2},
                                      \qquad\big|\cdot x^2y^2\\
xy(x+y)^2\ge& (x^2+y^2)^2.
\endalign
$$
Teraz je vidno, že roznásobením výrazov v~získanej nerovnici
dostaneme člen $2x^2y^2$ na oboch stranách~-- ten v~prvom kroku zrušíme
a~nerovnicu ďalej upravíme na súčinový tvar:
$$
\align
x^3y+2x^2y^2+xy^3\ge& x^4+2x^2y^2+y^4,\\
        x^3y+xy^3\ge& x^4+y^4,\\
                0\ge& x^4-x^3y+y^4-xy^3,\\
                0\ge&x^3(x-y)-y^3(x-y),\\
                0\ge& (x^3-y^3)(x-y).    \tag1
\endalign
$$
Na pravej strane poslednej nerovnice máme súčin dvoch dvojčlenov, a~keďže ho
porovnávame s~nulou, stačí skúmať znamienka jednotlivých zátvoriek.

Ak $x \ge y$, je aj $x^3 \ge y^3$ a~podobne pre $x \le y$ platí, že
$x^3 \le y^3$ (je to dôsledok toho, že funkcia tretej mocniny je
na celom obore reálnych čísel rastúca). Výrazy v~zátvorkách majú teda rovnaké znamienka pre
ľubovoľné hodnoty $x$ a~$y$, preto platí opačná nerovnosť $(x^3 - y^3) (x-y) \ge0$.
Nerovnica~\thetag1 tak môže byť splnená iba v~prípade, keď je jedna
zo zátvoriek nulová. Rovnice $x = y$ a~$x^3 = y^3$ sú ekvivalentné, a~preto
nerovnosť~\thetag1 platí práve vtedy, keď $x = y$.

Riešením sú všetky dvojice reálnych čísel $(x, y)$, kde $x = y \ne 0$.
Pri úpravách sme použili iba ekvivalentné úpravy, preto nie je skúška správnosti nutná.
% rozhodně však neuškodí: jistota, že jsme se při úpravách nespletli, je také důležitá...
% překvapit, že zkouška správnosti~--- tedy dosazení $y = x$ do zadání~--- je
% úspěšná.

\poznamka
Súčin na pravej strane \thetag1 sme mohli upraviť aj pomocou %známého
vzorca ${x^3 - y^3}=(x-y)(x^2+xy+y^2)$ na súčin dvoch nezáporných mnohočlenov:
$$
(x^3-y^3)(x-y)=(x-y)^2(x^2+xy+y^2).
$$
Trojčlen v~druhej zátvorke má totiž ako kvadratická funkcia
jednej neznámej (napr.~$x$) pre ľubovoľné~$y$ nekladný diskriminant
$D(y)=y^2-4y^2=-3y^2\le0$. Preto rovnosť $x^2+xy+y^2=0$ nastane
iba v~prípade $x=y=0$.

\návody
Určte všetky dvojice $(x, y)$ kladných reálnych čísel, ktoré vyhovujú
nerovnici $x/y + y/x \le 2$. [Po odstránení zlomkov upravíme nerovnicu
na $(x-y)^2 \le 0$, ktorá platí len pre $x = y$.]

Určte všetky dvojice $(x, y)$ reálnych čísel, ktoré vyhovujú nerovnici
$x/y + y/x \le 2$. [Riešením sú všetky dvojice $(x, y)$ také, že
$x = y \ne 0$ alebo $xy < 0$.]

Určte všetky dvojice $(x, y)$ reálnych čísel, ktoré vyhovujú nerovnici
$x^2 +4 y^2 \le 4xy$. [Riešením sú všetky dvojice $(x, y)$ také, že
$x = 2y$.]

Dokážte, že pre ľubovoľné reálne čísla $x$, $y$ platí nerovnosť
$x^4 + y^4 \ge x^3y + xy^3$. [Úprava na súčin.]

\D
Dokážte, že pre ľubovoľné kladné čísla $a$, $b$ platí nerovnosť
$$
\root{3}\of{\frac{{\vphantom b}a}b}+\root{3}\of{\frac ba}
\le\root{3}\of{2(a+b)\left(\frac 1a+\frac 1b\right)}.
$$
Zistite, kedy nastáva rovnosť.  [49--A--II--3]
\endnávod
}

{%%%%%   B-I-3
Ak niektorý z~takto zostrojených bodov $E$ a~$F$ splynie s~vrcholom~$C$,
je tvrdenie úlohy triviálne. Ďalej teda budeme mlčky predpokladať, že to tak nie je
a~že používané uhly majú zmysel.
\insp{b63.1}%

Najskôr predpokladajme, že body $E$ a~$F$ ležia postupne na
{\it úsečkách} $BC$ a~$AC$. Keďže $|BD|=|BE|$ a~bod~$I$ leží
na osi uhla $ABC$, sú trojuholníky $DBI$ a~$EBI$ zhodné podľa vety
{\it sus}. Podobne to platí aj~pre trojuholníky $DAI$ a~$FAI$, a~preto
platí (\obr)
$$
|\angle IDB|=|\angle IEB|\quad \text{a} \quad |\angle IFA|=|\angle IDA|. \tag1
$$
% \twocpictures

Tvrdenie úlohy, že body $C$, $E$, $F$ a~$I$ ležia na jednej
kružnici, je v~takom prípade ekvivalentné s~tým, že súčet
veľkostí uhlov $CFI$ a~$CEI$ je $180^\circ$. S~využitím rovností~\thetag1
dostávame
$$
\postdisplaypenalty 10000
|\angle CFI|=180^\circ-|\angle IFA|=180^\circ-|\angle IDA|
=|\angle IDB|=|\angle IEB|=180^\circ-|\angle CEI|,
$$
teda súčet protiľahlých uhlov v~štvoruholníku $CFIE$ je naozaj $180^\circ$.


Ak je jeden z~bodov $E$, $F$ vnútorným a~druhý
vonkajším bodom strán $BC$ a~$AC$, môžeme bez ujmy na všeobecnosti predpokladať,
že bod~$E$ leží na úsečke~$BC$ a~bod~$F$ leží na polpriamke opačnej
k~polpriamke~$CA$. Pre takú polohu bodov $C$, $E$, $F$ a~$I$ nám stačí ukázať
rovnosť uhlov $IFC$ a~$IEC$. Znova využijeme
rovnosti~\thetag1 a~dostaneme podobne (\obr)
$$
|\angle IFC|=|\angle IFA|=|\angle IDA|=180^\circ-|\angle IDB|
=180^\circ-|\angle IEB|=|\angle IEC|,
$$
odkiaľ vyplýva, že uhly $IFC$ a~$IEC$ sú zhodné.
\insp{b63.2}%

Tretia možnosť, že by oba body $E$ aj~$F$ ležali
zvonka prislúchajúcich strán trojuholníka $ABC$, zrejme nemôže nastať. V~tom
prípade by totiž muselo pre jednotlivé dĺžky platiť
$$
|AB|=|AD|+|BD|=|BE|+|AF|\ge |BC|+|AC|,
$$
čo odporuje trojuholníkovej nerovnosti pre strany trojuholníka~$ABC$.


\návody
Označme $I$ stred kružnice vpísanej do trojuholníka $ABC$ a~$K$, $L$, $M$ postupne
jej body dotyku so stranami $BC$, $AC$, $AB$. Dokážte,
že štvoruholníky $AMILO$, $BKIM$ a~$CLIK$ sú tetivové. [Štvoruholníky
majú dva protiľahlé uhly pravé.]

Označme $I$ stred kružnice vpísanej do trojuholníka $ABC$ a~$K$, $L$, $M$ postupne
jej body dotyku so stranami $BC$, $AC$, $AB$. Dokážte,
že štvoruholníky $AMILO$, $BKIM$ a~$CLIK$ sú deltoidy. [Osi vnútorných
uhlov trojuholníka $ABC$ delia každý štvoruholník na dva zhodné
pravouhlé trojuholníky.]

Dané kružnice $k$ a~$l$ sa pretínajú v~dvoch bodoch $B$ a~$C$. Priamka~$p$
sa dotýka kružnice~$k$ v~bode~$A$. Priamky $AB$ a~$AC$ pretínajú
kružnicu~$l$ postupne v~bodoch $D \ne B$ a~$E \ne C$. Dokážte, že priamky
$p$ a~$DE$ sú rovnobežné. [Využite úsekový uhol pri vrchole~$A$
a~obvodové uhly pri vrcholoch $C$ a~$D$ nad tetivou~$BE$. Je treba
dôsledne rozobrať všetky rôzne polohy bodov $B$, $C$, $D$ a~$E$ na
kružnici~$l$.]

\D
Nech $P$ je bod na strane~$BC$ trojuholníka $ABC$.
% so stredom kružnice vpísanej v~bode~$I$.
Skonštruujme postupne tieto body: $Q$ na strane~$AB$
tak, aby $|BQ|=|BP|$, $R$ na strane~$AC$ tak, aby $|AR|=|AQ|$, $P'$ na
strane~$BC$ tak, aby $|CP'|=|CR|$, $Q'$ na strane~$AB$ tak, aby
$|BQ'|=|BP'|$, $R'$ na strane~$AC$ tak, aby
$|AR'|=|AQ'|$. Dokážte, že $|CP|=|CR'|$, a~to, že body
$P$, $Q$, $R$, $P'$, $Q'$ a~$R'$ ležia na jednej
kružnici.
[Z~voľby jednotlivých bodov vyplývajú pre stred~$I$ kružnice vpísanej rovnosti
$|IP|=|IQ|=|IR|=|IP'|=|IQ'|=|IR'|$,
takže všetky uvedené body ležia na kružnici so stredom~$I$. A~keďže $CI$
je osou rovnoramenného trojuholníka $PIR'$, je aj trojuholník $CPR'$ rovnoramenný, takže
$|CP|=|CR'|$.]
% [Ak $|CP|= x $, tak pri standardním označení dostaneme
% $|AR'|=b -x$, a~preto aj~$|CR'|= x $. Keďže $|IP|=|IR'|$,
% $|IR|=|IQ|$ a~$|IQ'|=|IP'|$, stačí bez újmy na obecnosti
% ukázat, že bod $I$ ležia na osi úsečky $PP'$. To vyplývá zo
% vzdialenosti bodov $B$, $C$, $P$, $P'$ a~päty výšky z~bodu $I$ na
% stranu $BC$.]
\endnávod
}

{%%%%%   B-I-4
Označme cifry Daninho čísla postupne $a$, $b$, $c$. Informáciu
o~zvyškoch po delení siedmimi zo zadania môžeme prepísať na rovnice
$$
\align
100a+10b+c =&7x+2,\tag1 \\
100b+10a+c =&7y+3,\tag2 \\
100a+10c+b =&7z+5.\tag3
\endalign
$$
Cifry $a$ a~$b$ nesmú byť nulové, pretože ako prvé, tak aj druhé číslo
v~zadaní je trojciferné; preto $a, b \in \{1, 2, \dots, 9\}$,
$c \in \{0, 1, 2, \dots, 9\}$ a~$x$, $y$ a~$z$ sú celé čísla.

Teraz sa pokúsime postupne zistiť zvyšok cifier $a$, $b$, $c$ po delení
siedmimi. To nám dá pre samotné cifry nanajvýš dve možnosti.
Keď sa pozrieme na koeficienty v~rovniciach
\thetag1~--~\thetag3, vidíme, že vhodným odčítaním sa dokážeme zbaviť
dvoch cifier $a$ aj~$c$ naraz~-- keď od desaťnásobku
rovnice~\thetag2 odčítame rovnicu~\thetag3. Výsledok postupne upravíme tak,
aby sme zistili zvyšok cifry~$b$ po delení siedmimi:
$$
\align
10(100b+10a+c)-(100a+10c+b) =&10(7y+3)-(7z+5),\\
999b =&70y-7z+25,\\ %\qquad \big|-994b \qquad(994=7\cdot 142)\qquad \\
5b =&70y-7z-7\cdot 142b+7\cdot 3+4 ,\\ %\qquad \big|\cdot 3\\
15b =&3\,(70y-7z-7\cdot 142b+7\cdot 3)+3\cdot 4 ,\\ %\qquad \big|-14b\\
b =&3\,(70y-7z-7\cdot 142b+7\cdot 3)-7\cdot 2b + 12.
\endalign
$$
Keďže na pravej strane v~poslednej rovnici sú všetky členy okrem
čísla~$12$ deliteľné siedmimi, dáva $b$ rovnaký zvyšok po delení
siedmimi ako číslo~$12$, a~jediná vyhovujúce cifra~$b$ je tak $b = 5$.

Odčítaním rovníc \thetag3 a~\thetag1 dostaneme rovnicu
$9c - 9b = 7 (z-x) +3$, odkiaľ po dosadení $b = 5$ dostávame
$$
\align
9c-9\cdot 5 =&7(z-x)+3,\\ %\qquad \big|+45-7c\\
2c =& 7\,(z-x-c) + 48,\\ %\qquad \big|\cdot 4\qquad (48=7\cdot 6+6)\\
8c =& 4\cdot 7(z-x-c+6) + 4\cdot 6,\\ %\qquad \big|-7c\\
c =& 4\cdot 7(z-x-c+6) -7c + 7\cdot 3 + 3.
\endalign
$$
Z~deliteľnosti jednotlivých členov siedmimi tak dostávame $c = 3$.

Nakoniec dosadením $b = 5$ a~$c = 3$ napríklad do~prvej rovnice ľahko
dopočítame hodnotu~$a$:
$$
\aligned
100a+10b+c =& 7x+2,\\
100a+53 =& 7x+2\qquad (53=7\cdot 8-3,\ 98=7\cdot 14),\\
2a =& 7x+2-7\cdot 8+3-7\cdot 14a,\\ %\qquad\big|\cdot 4,\\
a~=& 4 (7x-7\cdot 8-7\cdot 14a) + 4 \cdot 5 - 7a,
\endaligned
$$
odkiaľ vyplýva, že cifra $a$ dáva rovnaký zvyšok po delení
siedmimi ako číslo $20 = 7 \cdot2 +6$, a~preto $a = 6$.

Dana teda napísala na papier číslo~$653$, takže číslo vzniknuté
prehodením prvej a~poslednej cifry je $356=7\cdot50+6$ a~dáva po delení
siedmimi zvyšok~$6$.

\ineriesenie
Predošlé riešenie môžeme zapísať prehľadnejšie pomocou kongruencií.\footnote{Využijeme
pritom iba základné vlastnosti kongruencií, ktoré v~texte spomenieme bez dôkazu. Ďalšími
zdrojmi pre štúdium tejto problematiky môžu byť napr. Alois Apfelbeck:
{\it Kongruence}, ŠMM č.~21,
alebo dokumenty
\hfil\break
{\tt thales.doa.fmph.uniba.sk/cincura/public/Element\%20-teoria\%20-cisel.pdf}
či
\hfil\break
{\tt mks.mff.cuni.cz/library/KongruenceMS/KongruenceMS.pdf}}
Hovoríme, že dve prirodzené čísla
$k$ a~$l$ sú kongruentné modulo~$m$, ak dávajú po delení číslom~$m$
rovnaký zvyšok, \tj. ak je číslo $k-l$ deliteľné číslom~$m$. Uvedenú kongruenciu
zapisujeme ako $k \equiv l \pmod m$.

Teraz pri rovnakom označení ako v~prvom riešení môžeme rovnice
\thetag1~--~\thetag3 s~využitím vzťahov $100x \equiv 2x \pmod 7$
a~$10x \equiv 3x \pmod 7$ zapísať ako
$$
\align
(100a+10b+c\equiv)\quad 2a+3b+c \equiv& 2\pmod 7,\tag 4\\
(100b+10a+c\equiv)\quad 2b+3a+c \equiv& 3\pmod 7,\tag 5\\
(100a+10c+b\equiv)\quad 2a+3c+b \equiv& 5\pmod 7.\tag 6
\endalign
$$

S~kongruenciami môžeme prevádzať podobné ekvivalentné úpravy ako
s~obyčajnými rovnicami~-- napríklad vynásobiť kongruenciu celým číslom
alebo odčítať dve kongruencie. Budeme postupovať podobne ako v~predošlom
riešení a~pokúsime sa získať kongruenciu iba pre cifru~$b$. Od
trojnásobku kongruencie~\thetag5 odčítame kongruenciu~\thetag6 a~dostaneme
$$
\align
3(2b+3a+c)-(2a+3c+b)\equiv& 3\cdot 3-5\pmod 7,\\
5b \equiv & 4 \pmod 7 , \tag7\\
15b \equiv &12 \pmod 7,\\
b \equiv &5 \pmod 7,
\endalign
$$
odkiaľ máme $b = 5$, pretože to je jediná cifra dávajúca zvyšok~$5$
po delení siedmimi. Kongruenciu~\thetag7 sme vynásobili tromi, aby sme
na ľavej strane dostali číslo, ktoré je kongruentné s~$1$ modulo~$7$.
Dá sa ukázať, že ak čísla $k$ a~$m$ sú nesúdeliteľné,
existuje vždy násobok čísla~$k$, ktorý je kongruentný s~$1$ modulo~$m$.

Odčítaním kongruencií \thetag6 a~\thetag4 dostaneme
$$
\align
(2a+3c+b)-(2a+3b+c) \equiv &5-2 \pmod 7,\\
2c \equiv &3+2b\pmod 7
\endalign
$$
a~po dosadení $b=5$ vyjde
$$
\align
2c \equiv &13\equiv 6\pmod 7,\\ %\qquad\big|\cdot 4\\
8c \equiv &24\pmod 7,\\
c \equiv & 3\pmod 7.
\endalign
$$
Jediná cifra dávajúca zvyšok~$3$ po delení siedmimi je $c = 3$,
pričom sme prvú kongruenciu vynásobili číslom~$4$, aby sme dostali $ 2 \cdot
4 =8\equiv 1 \pmod 7$. Nakoniec dosadením $b = 5$ a~$c = 3$ napríklad do
štvornásobku kongruencie~\thetag4 dostaneme
$$
\postdisplaypenalty-500
\align
4(2a+3b+c) \equiv &4\cdot 2\pmod 7,\\
8a+12b+4c \equiv &1\pmod 7,\\
a+5b+4c \equiv &1\pmod 7,\\
a+25+12 \equiv &1\pmod 7,\\
a~\equiv &6\pmod 7.
\endalign
$$
riešením sústavy kongruencií sme
dospeli k~rovnakej trojici cifier $a$, $b$, $c$ ako v~predošlom riešení
so sústavou rovníc.


\návody
Pri delení číslom~$99$ je zvyšok ľubovoľného trojciferného čísla
rovnaký ako zvyšok čísla, ktoré vznikne z~pôvodného čísla
čítaním odzadu. Presvedčte sa!
[$(100a +10 b +10 c) - (100c +10 b + a) = 99 (a-c))$]

Dokážte, že čísla $\overline{aba}$ a~$\overline{bab}$ dávajú rovnaký
zvyšok po delení siedmimi. (Výraz $\overline{klm}$ označuje zápis
trojciferného čísla s~ciframi $k$, $l$, $m$ v~desiatkovej sústave.)
[$\overline {aba} - \overline {bab} = (101a +10 b) - (101b +10 a) = 7 \cdot 13 (a-b)$]

Isté štvorciferné prirodzené číslo je deliteľné siedmimi. Ak zapíšeme jeho číslice v~opačnom poradí, dostaneme väčšie
štvorciferné číslo, ktoré je tiež deliteľné siedmimi. Navyše po delení číslom~$37$ dávajú obe spomenuté štvorciferné čísla rovnaký zvyšok. Určte pôvodné štvorciferné číslo. [58--A--II--1]

\D
Klárka mala na papieri napísané trojciferné číslo. Keď ho správne
vynásobila deviatimi, dostala štvorciferné číslo, ktoré začínalo rovnakou
číslicou ako pôvodné číslo, prostredné dve číslice sa rovnali
a~posledná číslica bola súčtom číslic pôvodného čísla.
Ktoré štvorciferné číslo mohla Klárka dostať?  [57--C--I--6]

Janko má tri kartičky, na každej je iná nenulová cifra. Súčet
všetkých trojciferných čísel, ktoré možno z~týchto kartičiek
zostaviť, je číslo o~$6$ väčšie ako trojnásobok jedného z~nich.
Aké cifry sú na kartičkách?
[61--C--II--2]

Nájdite všetky štvormiestne čísla $\overline{abcd}$ (v~desiatkovej
sústave), pre ktoré platí rovnosť
$$
\overline{abcd}+1=(\overline{ac}+1)(\overline{bd}+1).
$$
[49--A--III--6]
\endnávod
}

{%%%%%   B-I-5
Označme $k_v (S_v; r_v)$ kružnicu vpísanú do hľadaného trojuholníka $ABC$
a~$k_p (S_p; r_p)$ kružnicu pripísanú k~jeho strane~$BC$. Stredy $S_v$ a~$S_p$
ležia na osi uhla $BAC$, ktorej priesečník so stranou~$BC$
ešte označíme~$S$. Priamky $AB$, $AC$ a~$BC$ sú
spoločnými dotyčnicami kružníc $k_v$ a~$k_p$, ktoré sú teda rovnoľahlé
podľa stredov $A$ a~$S$ (\obr).
\insp{b63.3}%
Bod~$S$ je pritom stredom vnútornej rovnoľahlosti, v~ktorej si zodpovedajú
aj body $T$ a~$U$ dotyku kružníc $k_v$ a~$k_p$ s~úsečkou~$BC$.
Podľa zadania je ale $T\ne U$
(predpokladá sa totiž existencia uhla $ATU$), takže stred~$S$
vnútornej rovnoľahlosti kružníc $k_v$, $k_p$ je tým bodom
úsečky~$TU$, ktorý ju delí v~pomere $r_v:r_p$, a~ten je podľa
vonkajšej rovnoľahlosti rovný pomeru $|AS_v|:|AS_p|$. Platí teda
$$
\frac{|ST|}{|SU|}=\frac{|AS_v|}{|AS_p|}.
$$

Označme teraz $A'$ kolmý priemet bodu~$A$ na priamku $BC$ a~$R_v$
a~$R_p$ kolmé priemety stredov $S_v$ a~$S_p$ na priamku~$AA'$. Keďže $A$,
$S_v$, $S$, $S_p$ je poradie bodov na jednej priamke, majú ich
kolmé priemety na priamky $BC$ a $AA'$ poradie $A'$, $T$, $S$, $U$,
respektíve $A$, $R_v$, $A'$, $R_p$ (\obrr1).\footnote{Všimnime si,
že z~poradí bodov $A'$, $T$, $U$ a kolmosti $AA'\perp TU$ vyplýva,
že uhol $ATU$ je tupý (bez tejto podmienky by úloha nemala riešenie).}
Z~pravouholníkov $S_vR_vA'T$
a~$S_pR_pA'U$ zrejme vyplýva
$$
\frac{|A'T|}{|A'U|}=\frac{|R_vS_v|}{|R_pS_p|}=\frac{|AS_v|}{|AS_p|}. % \tag2
$$
Porovnaním odvodených rovností dostaneme úmeru
$|ST|:|SU|=|A'T|:|A'U|$. Podľa nej neznámy bod~$S$ delí zadanú
úsečku~$TU$ v~pomere určenom bodom~$A'$, ktorého polohu
na polpriamke~$UT$ zvonka úsečky~$UT$ poznáme. A~akonáhle zostrojíme bod~$S$, môžeme zostrojiť
aj stred~$S_v$, ktorý leží na priamke~$AS$ a~na kolmici na priamku~$TU$ vedenej bodom~$T$.
Vrcholy $B$ a~$C$ potom získame ako priesečníky dotyčníc z~vrcholu~$A$
ku kružnici~$k_v(S_v;r_v=|S_vT|)$ s~priamkou~$TU$.

Na zostrojenie bodu~$S$ využijeme napr. nasledujúci postup (\obr):
\insp{b63.4}%
V~polrovine $TUA$ zvoľme nejaký bod~$Y$ a~k~nemu vnútri úsečky~$A'Y$ zostrojme
bod~$X$ tak, aby úsečky $TX$, $UY$ boli rovnoľahlé podľa stredu~$A'$.
Označme $X'$ bod súmerne združený s~bodom~$X$ podľa stredu~$T$.
Priesečník priamok $TU$ a~$X'Y$ je potom hľadaným bodom~$S$, lebo je stredom
rovnoľahlosti úsečiek $TX'$ a~$UY$, takže podľa oboch spomenutých rovnoľahlostí platí
$$
\frac{|ST|}{|SU|}=\frac{|TX'|}{|UY|}=\frac{|TX|}{|UY|}=\frac{|A'T|}{|A'U|}.
$$

Vďaka podmienke tupého uhla $ATU$ padne bod~$S_v$ vyššie
uvedenou konštrukciou na priamku~$AS$ do polohy
medzi bodmi $A$ a~$S$, teda celá úloha bude mať jediné riešenie
(ak neberieme do úvahy možnosť vymeniť označenie vrcholov $B$ a~$C$).


\ineriesenie
Použijeme rovnaké označenie ako v~predošlom riešení.
V~rovnoľahlosti so stredom v~bode~$A$ a~koeficientom~$r_p/r_v$ sa bod $T \in k_v$
\insp{b63.5}%
zobrazí na bod $T'\in k_p$ (\obr). Z~vlastnosti použitej
rovnoľahlosti vieme, že dotyčnica~$t$ ku kružnici~$k_p$ vedená bodom~$T'$
je rovnobežná s~dotyčnicou ku kružnici~$k_v$ vedenou bodom~$T$, čo je priamka~$BC$. Priamka~$S_pU$ je teda kolmá
nielen na priamku~$TU$, ale aj na dotyčnicu~$t$, takže úsečka~$T'U$ je
priemerom kružnice~$k_p$. Odtiaľ vyplýva nasledujúca
{\it konštrukcia}:
\ite1.
Bod~$T'$ je priesečníkom priamky~$AT$ a~kolmice z~bodu~$U$ na
priamku~$UT$. Keďže uhol $ATU$ je tupý, leží bod~$T'$ v~opačnej
polrovine určenej priamkou~$TU$ ako bod~$A$.
\ite2.
Kružnica $k_p$ je kružnica s~priemerom~$UT'$.
\ite3.
Body $B$ a~$C$ sú priesečníky dotyčníc z~bodu~$A$ ku kružnici~$k_p$ s~priamkou~$TU$.

Úloha má jediné riešenie.


\návody
\titem
Zopakujte si učebnicové poznatky
o~rovnoľahlosti dvoch kružníc a~jej použití pri konštrukcii spoločných
dotyčníc.

Do pásu určeného dvoma rovnobežkami $p \parallel q$, $p \ne q$, vpíšme
kružnicu tak, že sa dotýka oboch priamok. Dokážte, že polomer
kružnice je polovicou vzdialenosti priamok $p$ a~$q$. [Priemer kružnice,
ktorý je kolmý na priamku~$p$, má dĺžku rovnú vzdialenosti rovnobežiek $p$ a~$q$.]

Daný je trojuholník $ABC$. Dokážte, že bod~$A$, stred
kružnice trojuholníku $ABC$ vpísanej a~stred kružnice pripísanej ku strane~$BC$
ležia na jednej priamke. [Tá priamka je osou uhla $BAC$.]

Na úsečke~$AB$ zostrojte bod~$X$ tak, aby platilo $|AX|:|BX|= p$, pričom $p>0$
je dané číslo.
[Na kolmici na priamku~$AB$ bodom~$A$ zostrojte bod~$C$ tak, aby $|AC|= p$,
a~na kolmici na priamku~$AB$ bodom~$B$ zostrojte bod~$D$ tak, aby
$|BD|= 1$, pričom body $C$ a~$D$ ležia v~opačných polrovinách určených
priamkou~$AB$. Bod~$X$ je priesečník $AB$ a~$CD$.]

\D
Daný je lichobežník $ABCD$, $AB \parallel CD$. Dokážte, že priesečník~$P$
jeho uhlopriečok $AC$ a~$BD$ leží na spojnici stredov strán $AB$ a~$CD$.
[Priesečník uhlopriečok je stred rovnoľahlosti oboch základní, takže ich
stredy si musia navzájom zodpovedať.]

Označme~$r$ polomer kružnice vpísanej trojuholníku $ABC$. Jej dotyčnice
rovnobežné so stranami daného trojuholníka z~neho vytínajú
tri menšie podobné trojuholníky, polomery im vpísaných
kružníc označíme $r_a$, $r_b$ a~$r_c$. Dokážte rovnosť
$r_a + r_b + r_c = r$.
[Tibor Fonód, Milan Maxian: {\it Geometrické perličky}, úloha~3.10.
Pri vhodnom označení polomerov bude príslušný pomer podobnosti $r_a/r=(v_a-2r)/v_a$,
pričom $v_a$ označuje veľkosť výšky trojuholníka $ABC$ na stranu~$a$, a~podobne pre strany $b$ a~$c$.
Požadovanú rovnosť potom dostaneme z~nasledujúcich rovností pre obsah~$S$ trojuholníka $ABC$:
$2S = a\cdot v_a = b \cdot v_b = c \cdot v_c = r (a+ b + c)$.]
\endnávod}

{%%%%%   B-I-6
Jediným kritériom, či dokážeme zostrojiť trojuholník so stranami
daných dĺžok, je trojuholníková nerovnosť. Na zostrojenie trojuholníka
stačí, aby {\it najväčšia dĺžka bola menšia ako súčet
zvyšných dvoch}. Ak platí táto nerovnosť, platia aj ostatné
trojuholníkové nerovnosti a~trojuholník sa dá zostrojiť. Ak to
neplatí, trojuholník sa zostrojiť nedá, pretože neplatí
trojuholníková nerovnosť.
{\it Trojuholník sa nedá
z~troch strán daných dĺžok zostrojiť práve vtedy, keď najväčšia dĺžka je
aspoň taká ako súčet zvyšných dvoch}.

Predpokladajme, že $r$ je hľadané najmenšie číslo.
To znamená, že vieme
tyč rozlomiť na štyri časti dĺžky nanajvýš~$r$ tak, že zo žiadnych troch
týchto častí sa nedá zložiť trojuholník. Najdlhšia z~častí musí mať
dĺžku~$r$, pretože inak by $r$ nebolo najmenšie číslo s~požadovanou vlastnosťou.
Dĺžky ostatných častí označme $x$, $y$, $z$ tak,
že $x \le y \le z\le r$ a~$x + y + z+ r = 1$,
a~predpokladajme stále,
že zo žiadnych troch takých
častí sa nedá zostrojiť trojuholník, čo teraz vyjadríme príslušnými
nerovnosťami.

Z~trojice dĺžok $(x, y, z)$
nie je možné zostrojiť trojuholník práve vtedy, keď $x + y \le z$. Keďže
$z \le r$, vyplýva odtiaľ aj príslušná nerovnosť $x + y \le r$ pre trojicu
$(x, y, r)$, takže ani z~týchto troch dĺžok nie je možné zostrojiť trojuholník.
Podobne pre trojicu dĺžok $(y, z, r)$ platí, že z~nich nemožno zostrojiť trojuholník práve vtedy, keď
$y + z\le r$, a~keďže $x \le y$, vyplýva odtiaľ príslušná nerovnosť $x + z\le r$
aj pre trojicu dĺžok $(x, z, r)$.

Zo žiadnych troch častí teda nedokážeme zložiť trojuholník práve vtedy, keď budú okrem
rovnosti $x + y + z+ r = 1$ splnené aj obe podmienky
$$
x+y\le z\qquad\text{a}\qquad y+z\le r. \tag1
$$
Keď do prvej nerovnosti z~\thetag1 dosadíme $x+y=1-z-r$, dostaneme
$$
\align
1-z-r \le& z, \\
1 -r \le&2z. \tag2
\endalign
$$

Keďže $y \ge x$, má druhá nerovnosť z~\thetag1 nasledujúci dôsledok:
$$
r \ge y+z = \frac{y+y+z}{2}+\frac z2\ge
\frac{x+y+z}{2}+\frac z2 = \frac {1-r}2 +\frac z2.
% \tag3
$$
V~získanej nerovnosti ešte pomocou nerovnosti~\thetag2
odhadneme~$z$, takže dostaneme
$$
r\ge \frac {1-r}2 +\frac z2 \ge \frac{1-r}2 + \frac{1-r}4\ge \frac34(1-r),
$$
z~ktorej už porovnaním pravej a~ľavej strany vyplýva požadovaný odhad
hodnoty~$r$:
$$
4r\ge 3(1-r) \quad\text{čiže}\quad
r \ge \frac 37.
$$

Ostáva ukázať, že existuje rozlámanie tyče dĺžky~$1$ na štyri časti
dĺžky nanajvýš $3/7$ tak, že zo žiadnych troch týchto častí sa potom nedá
zložiť trojuholník~-- vyhovujú napríklad dĺžky $(1/7, 1/7, 2/7, 3/7)$.

\ineriesenie
K~hodnote $r = 3/7$ sa dá dôjsť aj intuitívnym prístupom, najmä pokiaľ
sa najskôr pokúsime vyriešiť úlohu pre rozlámanie tyče na tri časti.
Podobne ako v~predošlom riešení označme dĺžky jednotlivých častí
ako $x \le y \le z\le r$. Úlohu si zjednodušíme tak, že sa obmedzíme na
prípad $y = x$. Aby sa z~dĺžok $(x, y, z)$ nedal zostrojiť
trojuholník, musí platiť $z \ge x + y = 2x$; vezmime teda $z = 2x$. Napokon,
aby sa nedal zostrojiť trojuholník ani z~dĺžok $(y, z, r)$, stačí, keď bude platiť $r=
z+ y = 2x + x = 3x$. Odtiaľ vychádza $x + y + z~+ r = x + x +2x +3x = 7x = 1$,
a~teda $r = 3x = 3/7$.

Naozaj, zo žiadnej trojice z~dĺžok $(1/7, 1/7, 2/7, 3/7)$ sa trojuholník nedá zostrojiť.

Ostáva ešte ukázať, že táto hodnota~$r$ je najmenšia. Inými slovami,
keď rozlámeme tyč dĺžky~$1$ na ľubovoľné štyri časti, pričom
každá z~nich bude mať dĺžku menšiu ako~$3/7$, tak sa z~niektorých troch
častí trojuholník zložiť dá. Označme dĺžky jednotlivých
častí ako $a \le b \le c \le d < 3/7$, pričom $a + b + c + d = 1$. Skúmajme dve
možnosti pre hodnotu~$a$.

Ak by platilo $a < 1/7$, dostali by sme z~nerovnosti $d < 3/7$
a~rovnosti $1 ={a+ b + c + d}$, že $1-1/7-3/7 < b + c$. V~tom prípade by ale bolo
%\vadjust{\goodbreak}%
$d < 3/7 < b + c$, a~preto by sa z~dĺžok $b \le c \le d$ dal zostrojiť trojuholník.
Ak by platilo $1/7 \le a$, bolo by $1/7 \le a\le b$.
Keby za týchto podmienok ani jedna z~trojíc $(a, b, c)$, $(a, c, d)$
nespĺňala trojuholníkové nerovnosti,
muselo by platiť $2/7 \le a+ b \le c$ a~následne $3/7 = 1/7 +2/7 \le a+ c \le d$,
čo je v spore s~tým, že $d < 3/7$.

\návody
Nájdite najmenšie reálne číslo~$r$ také, že tyč s~dĺžkou~$1$ sa dá
rozlámať na tri časti dĺžky nanajvýš~$r$ tak, aby sa z~nich nedal
zložiť trojuholník.
[$r = 1/2$, najdlhšia časť musí byť aspoň polovica celej dĺžky]

Dokážte, že v~ľubovoľnom štvorstene existuje taký vrchol, že z~hrán,
ktoré z~neho vychádzajú, sa dá zostrojiť trojuholník. [Nech
najdlhšia hrana v~štvorstene $ABCD$ je $AB$. Ak sa z~hrán pri~vrchole~$A$
nedá zostrojiť trojuholník, tak $|AC|+|AD|\le|AB|$.
Z~trojuholníkových nerovností $|AC|+|BC|>|AB|$ a~$|AD|+|BD|>|AB|$
dostaneme, že z~hrán vychádzajúcich z~vrcholu~$B$ sa trojuholník zostrojiť dá,
pretože $|BC|+|BD|>2|AB|-|AC|-|AD|\ge|AB|$.]

\D
Na zabudnutej tabuli v~Rasťovej ešte zabudnutejšej tmavej komnate je
nakreslených päť už skoro zabudnutých úsečiek. Z~každej trojice
z~týchto úsečiek vieme zložiť trojuholník. Dokážte, že vieme vybrať
tri úsečky tak, že trojuholník, ktorý z~nich vznikne, je ostrouhlý.
[KMS 2008/2009, 3.~zimná séria, úloha~8]

Vyriešte zadanú úlohu pre lámanie tyče na päť (a~prípadne viac)
častí. [Zovšeobecnením úvahy v~druhom riešení dostaneme pre päť
častí hodnotu $r = 5/12$ a~lámanie na $n$ častí vedie na vzorec
$r = \frac {F_n}{F_1 + \cdots + F_n}$, pričom $F_n$ je $n$-té Fibonacciho číslo.]
\endnávod}

{%%%%%   C-I-1
Sčítaním oboch rovníc zistíme, že $b = 2$. Dosadením za~$b$ do niektorej z~nich
vyjde $c = \m a$. Platí teda $V = (a- 2)^2 + (2 + a)^2 + (\m2a)^2$.
Po umocnení a~sčítaní zistíme, že
$V = 6a^2 + 8\ge8$. Rovnosť nastane práve vtedy, keď $a = 0$, $ b = 2$ a~$c = 0$.

Hľadaná najmenšia hodnota výrazu $V$ je teda rovná~$8$.

\návody
Určte najmenšiu hodnotu výrazu $V = 5 + ( x - 2)^2 $, $x\in\Bbb R$.
  Pre ktoré~$x$ ju výraz nadobúda?

Určte najmenšiu možnú hodnotu výrazu $W = 9 - ab$,
  kde $a$, $b$ sú reálne čísla spĺňajúce podmienku $a + b = 6$.
  Pre ktoré hodnoty $a$, $b$ je $W$ minimálne?
  [$W=(a-3)^2\ge0$]

Určte najmenšiu možnú hodnotu výrazu $Y = 12 - ab$,
  kde $a$, $b$ sú reálne čísla spĺňajúce podmienku $a + b = 6$.
  Pre ktoré hodnoty $a$, $ b$ je $Y$ minimálne?
  [$Y=3+W\ge3$]

Určte najväčšiu možnú hodnotu výrazu $K = 5 + ab$,
  kde $a$, $b$ sú reálne čísla spĺňajúce podmienku $a + b = 8$.
  Pre ktoré hodnoty $a$, $b$ je $K$ maximálne?
  [$K=5+8a-a^2=\m(a-4)^2+21\le21$, $a=b=4$]

Nech $a$, $b$, $c$, $d$ sú také reálne čísla, že $a + d = b + c$.
  Dokážte nerovnosť $(a - b)(c - d ) + (a~- c)(b - d ) + (d - a~)(b - c)\ge0$.
   [C--54--I--1]

Pre kladné reálne čísla $a$, $b$, $c$, $d$ platí
$$
a~+ b = c + d,\quad ad = bc,\quad ac + bd = 1.
$$
Akú najväčšiu hodnotu môže mať súčet $a + b + c + d$? [C--62--I--2]
\endnávod
}

{%%%%%   C-I-2
Vrchol $B$ je určený polpriamkou~$AT$ a~kolmicou~$p$
na výšku~$AP$ v~bode~$P$ (\obr), na ktorej leží strana~$BC$. Pritom bod~$T$
musí byť vnútorným bodom úsečky~$AB$. Stred~$S$ kružnice
\insp{c63.1}%
vpísanej trojuholníku $ABC$ potom dostaneme ako priesečník kolmice~$q$
na priamku~$AT$ v~bode~$T$ s~osou uhla ohraničeného
priamkou~$p$ a~polpriamkou~$BA$. Jej polomer bude mať veľkosť~$|ST|$.

Ostáva zostrojiť vrchol~$C$ hľadaného trojuholníka $ABC$. Ten bude ležať jednak na priamke~$p$,
jednak na druhej dotyčnici vpísanej kružnice z~vrcholu~$A$, ktorá je súmerne združená
so stranou~$AB$ podľa priamky~$AS$.
Stačí teda zostrojiť
bod~$U$ dotyku strany~$AC$ s~kružnicou vpísanou ako obraz bodu~$T$
v~uvedenej osovej súmernosti.

Odtiaľ vyplýva {\it konštrukcia\/}:

\ite1. $p\: P \in p$ a~$p\perp AP$;
\ite2. $B\: B \in AT \cap p$, bod~$B$ musí ležať na polpriamke~$AT$ za bodom~$T$;
\ite3. $q\: T \in q$ a~$ q\perp AT$;
\ite4. $u_1, u_2$: dve (navzájom kolmé) osi rôznobežiek $AB$, $p$;
\ite5. $S_1, S_2\: S_1 \in  q\cap u_1$, $S_2\in q\cap  u_2$;
\ite6. $U_1, U_2$: obrazy bodu~$T$ v~súmernostiach podľa priamok $AS_1$ a~$AS_2$;
\ite7. $C_1, C_2$: priesečníky priamky~$p$ s~polpriamkami $AU_1$ a~$AU_2$;
\ite8. trojuholníky $ABC_1$ a~$ABC_2$.

\diskusia
Bod~$B$ konštruovaný
v~2.~kroku existuje, len ak uhol $PAT$ je ostrý (inak ani polpriamka~$AT$
nepretne priamku~$p$) a~zároveň bod~$T$ leží vnútri polroviny~$pA$, čo je
ekvivalentné s~tým, že aj uhol $APT$ je ostrý.
Body $S_1$, $S_2$ existujú vždy a~sú rôzne, lebo ležia v~opačných polrovinách
určených priamkou~$AB$.
Kružnica vpísaná leží celá v~trojuholníku $ABC$, a~teda i~v~páse určenom priamkou~$p$
a~priamkou s~ňou rovnobežnou, ktorá prechádza vrcholom~$A$,
takže
stred~$S$ vpísanej kružnice
musí padnúť do pásu tvoreného priamkou~$p$ a~priamkou~$p'$ s~ňou rovnobežnou, ktorá rozpoľuje
výšku~$AP$. V~takom prípade
dotyčnica ku kružnici $(S;|ST|)$
(súmerne združená s~dotyčnicou~$AB$ podľa priamky~$AS$) určite pretne priamku~$p$
v~hľadanom vrchole~$C$.

Diskusiu zhrnieme takto:
Ak pre vnútorné uhly trojuholníka $APT$ platí $|\uhol PAT|\ge90\st$ alebo
$|\uhol APT|\ge90\st$, nemá úloha riešenie. Ak platí
$|\uhol PAT|<90\st$ a~zároveň $|\uhol APT|<90\st$, je počet riešení~0 až~2
podľa toho, koľko zo zostrojených bodov $S_1$ a~$S_2$ leží
medzi rovnobežkami $p$ a~$p'$.


\návody
Zostrojte trojuholník, ak sú dané body dotyku jeho strán s~kružnicou
tomuto trojuholníku vpísanou.

V~trojuholníku $ABC$ označme postupne $P$, $Q$, $R$ päty
výšok z~vrcholov $A$, $B$, $C$. Ďalej postupne označme $T$, $U$, $V$
 body dotyku kružnice vpísanej so stranami $BC$, $CA$, $AB$.
Zostrojte trojuholník $ABC$, ak je dané:
\item{a)} $A$, $C$, $V$,
\item{b)} $A$, $U$, $R$,
\item{c)} $A$, $P$, $Q$,
\item{d)} $A$, $B$, $R$.
\endgraf [V~a) i~b) vieme zostrojiť vpísanú kružnicu; v~c) zostrojíme~$AB$ ako
priemer kružnice určenej danými bodmi. Úloha~d) nemá riešenie, pokiaľ $R$ neleží na priamke~$AB$.
Ak $R$ leží na priamke~$AB$, má úloha nekonečne veľa riešení.]
\endnávod
}

{%%%%%   C-I-3
Nech $n = pqr $, $p < q < r$. Rovnosť $(p + 1)(q + 1)r = pqr + 915$ ekvivalentne upravíme
na tvar $(p + q + 1)\cdot r = 915 = 3\cdot5\cdot61$, z~ktorého vyplýva, že prvočíslo~$r$ môže nadobudnúť len niektorú
z~hodnôt $3$, $5$ a~$61$. Pre $r = 3$ ale z~poslednej rovnice dostávame $(p + q + 1)\cdot3 = 3\cdot5\cdot61$,
čiže $p + q = 304$. To je spor s~tým, že $r$ je najväčšie. Analogicky zistíme, že
nemôže byť ani $r = 5$. Je teda $r = 61$ a~$p + q = 14$. Vyskúšaním všetkých možností pre $p$ a~$q$
vyjde $p = 3$, $q = 11$, $r = 61$ a~$n = 3\cdot11\cdot61 = 2\,013$.


\návody
Určte všetky prvočísla $p$, $q$, pre ktoré platí $p + q = 14$.

Číslo $n$ je súčinom dvoch rôznych prvočísel. Ak zväčšíme
menšie z~nich o~$1$ a~druhé ponecháme, ich súčin sa zväčší o~$7$. Určte
číslo~$n$. [Výsledok: $n\in\{14, 21, 35\}$.]

Číslo $n$ je súčinom dvoch prvočísel. Ak zväčšíme jedno z~nich
o~$1$ a~druhé o~$1$ zmenšíme, ich súčin ostane nezmenený. Určte číslo~$n$.
[Výsledok: $n = 6$.]

Číslo $n$ je súčinom dvoch prvočísel. Ak zväčšíme každé z~nich
o~$1$, ich súčin sa zväčší o~$35$. Určte číslo~$n$.
[Výsledok: $n\in\{93,145, 253, 289\}$.]
\endnávod
}

{%%%%%   C-I-4
Platí $|AK| = |DL|$ a~$|AD| = |DC| = 2 |AK|$ (\obr), takže pravouhlé trojuholníky
$AKD$ a~$DLC$ sú zhodné podľa vety {\it sus}. Okrem toho sú trojuholníky $MLD$
a~$AKD$ podobné podľa vety {\it uu},
lebo $|\uhol LDM| =|\uhol KDA|$ a~$|\uhol DLM| =|\uhol DLC| =|\uhol AKD|$.
Analogicky sa dá overiť i~podobnosť trojuholníkov $MDC$
a~$AKD$. Z~podobnosti trojuholníkov $AKD$, $MLD$ a~$MDC$ vyplýva, že
$|MD| = 2|ML| = 2\cm$ a~$|MC| = 2|MD| = 4\cm$. Obsahy útvarov $MLD$, $MDC$ a~$AKML$ sú
$$
S_{MLD} ={1\cdot2\over2} = 1\cm^2,\qquad S_{MDC} ={2\cdot4\over2} = 4\cm^2
$$
a~$$
S_{AKML} = S_{AKD} - S_{MLD} = S_{DLC} - S_{MLD} = S_{MDC} = 4\cm^2.
$$
Nakoniec pomocou Pytagorovej vety dostávame
$S_{ABCD} = |DC|^2 = |DM|^2 + |CM|^2 =20\cm^2$, takže
$$
S_{KBCM} = S_{ABCD} - (S_{MLD} + S_{MDC} + S_{AKML}) =11\cm^2.
$$

\zaver
 Obsahy trojuholníkov $MLD$, $MDC$ a~štvoruholníkov
$AKML$, $KBCM$ sú postupne $1\cm^{2}$, $4\cm^{2}$, $4\cm^{2}$  a~$11\cm^{2}$.
\ifrocenka\insp{c63.2}\else\inspinsp{c63.2}{c63.3}\fi%

\návody
Dva zhodné pravouhlé trojuholníky $ABC$ a~$DEB$ sú
umiestnené podľa \obr{} a~platí $|BD| = 10\cm$, $|CD| = 20\cm$.
\item{a)} Určte dĺžky strán trojuholníka $ABC$.  [$10$, $30$, $10\sqrt{10}$]
\item{b)} Dokážte, že trojuholníky $DBF$, $ABC$ a~$BEF$ sú navzájom podobné.
\item{c)} Určte dĺžky strán trojuholníkov $DBF$ a~$BEF$.
        [$10$, $3\sqrt{10}$, $\sqrt{10}$; 30, $9\sqrt{10}$, $3\sqrt{10}$]
\item{d)} Určte obsahy trojuholníkov $ABC$, $DBF$ a~$BEF$. [$150$, $15$, $135$]
        \vadjust{\nobreak}%
\item{e)} Určte obsah štvoruholníka $AFDC$. [$135$]

Dva zhodné pravouhlé trojuholníky $ABC$ a~$DEB$ sú
umiestnené podľa \obrr1. Trojuholník $BEF$ má obsah $30\cm^2$.
Určte obsah štvoruholníka $AFDC$. [$30$]

Dokážte vety:
\item{a)} Ak majú dva trojuholníky rovnakú výšku, potom pomer ich obsahov sa
        rovná pomeru dĺžok príslušných základní.
\item{b)} Ak majú dva trojuholníky zhodné základne, potom pomer ich obsahov
        sa rovná pomeru príslušných výšok.

V~rovnoramennom pravouhlom trojuholníku $ABC$ s~preponou~$BC$ je $|AB| = 12\cm$.
Označme~$K$~stred strany $AB$ a~$L$ taký bod strany~$BC$, pre ktorý
platí $|CL|:|LB| = 1:2$.
   Určte obsahy útvarov, ktoré vzniknú rozrezaním trojuholníka
$ABC$ pozdĺž úsečiek $KC$ a~$AL$. [Nakreslite si
obrázok, označte~$M$ priesečník úsečiek $KC$ a~$AL$,
dokreslite úsečku~$BM$ a~pomocou viet z~predošlej úlohy vypočítajte
najprv obsahy všetkých piatich trojuholníkov, ktoré majú spoločný vrchol~$M$.]

V~danom rovnobežníku $ABCD$ je bod~$E$ stred strany~$BC$ a~bod~$F$ leží
vnútri strany~$AB$. Obsah trojuholníka $AFD$ je $15\cm^2$ a~obsah
trojuholníka $FBE$ je $14\cm^2$. Určte obsah štvoruholníka $FECD$.  [C--57--S--2]
\endnávod
}

{%%%%%   C-I-5
Keďže $96 = 3\cdot32 = 3\cdot2^5$, budeme dokazovať deliteľnosť súčtu
$S = n^4 + 2n^2 + 2\,013$ dvoma nesúdeliteľnými číslami $3$ a~$32$.

Deliteľnosť tromi: Pretože číslo $2\,013$ je deliteľné tromi, stačí dokázať
deliteľnosť tromi zmenšeného súčtu
$$
S~- 2\,013 = n^4 + 2n^2= n^2 (n^2 + 2).
$$

V~prípade $3\mid n$ je všetko jasné, v~opačnom prípade je $n = 3k\pm1$
pre vhodné celé~$k$, takže platí $3\mid n^2 + 2$, lebo $n^2 + 2 = 3(3k^2 + 2k + 1)$.

Deliteľnosť číslom $32$: Keďže $2\,016 = 32\cdot63$, stačí dokázať deliteľnosť
číslom~$32$ zmenšeného súčtu
$$
S~- 2\,016 = n^4 + 2n^2-3= (n^2 + 1)^2-2^2= (n^2 + 3)(n^2 - 1).
$$
Predpokladáme, že $n$ je nepárne, teda $n = 2k + 1$ pre vhodné celé $k$, preto platí
$$
n^2 + 3 = (2k + 1)^2 + 3 = 4(k^2 + k~+ 1)
\quad\hbox{a}\quad
n^2 - 1 = (2k + 1)^2 - 1 = 4k (k~+ 1).
$$
Odtiaľ vyplýva, že $32\mid (n^2 + 3)(n^2 - 1)$, lebo číslo $k (k~+ 1)$ je párne.

\poznamka
Deliteľnosť číslom~$32$ sa dá dokazovať i~bez vykonaného
algebraického rozkladu trojčlena $n^4 + 2n^2-3$, z~ktorého po dosadení $n = 2k + 1$
roznásobením dostaneme
$$
n^4 + 2n^2-3 = 16k^4 + 32k^3 + 32k^2 + 16k = 16k(k^3 + 2k^2 + 2k + 1).
$$

Pre párne $k$ je deliteľnosť takto upraveného výrazu číslom~$32$
zrejmá. Pre nepárne $k$ je zase párny súčet $k^3 + 1$, takže je párny i~druhý činiteľ
$k^3 + 2k^2 + 2k + 1$.

\návody
Dokážte, že pre každé prirodzené $n$ je číslo $n^3 + 2n$ deliteľné tromi.

Dokážte, že pre každé nepárne číslo $n$ je číslo $n^2 -1$ deliteľné ôsmimi.

Dokážte, že pre všetky celé kladné čísla $n$ je rozdiel $n^6 - n^2$ deliteľný šesťdesiatimi.

Určte všetky kladné celé čísla~$m$, pre ktoré je rozdiel $m^6 - m^2$ deliteľný číslom~$120$.
[C--55--I--1]

Určte všetky celé čísla $n$, pre ktoré je $2n^3 - 3n^2 + n + 3$ prvočíslo. [C--62--I--5]
\endnávod
}

{%%%%%   C-I-6
Poslední štyria hráči odohrali medzi sebou 6~partií,
takže počet bodov, ktoré dokopy získali, je aspoň~6. Hráč, ktorý
skončil na 2.~mieste, teda získal aspoň 6~bodov.
Keby získal viac ako~6, teda aspoň 6{,}5~bodov, musel by
najlepší hráč (vďaka podmienke rôznych počtov) získať všetkých 7~možných
bodov; porazil by tak i~hráča na 2.~mieste, ktorý by v~dôsledku toho získal menej
ako 6{,}5 bodov, a~to je spor. Hráč v~poradí
druhý preto získal práve 6~bodov. Presne toľko ale získali dokopy
i~poslední štyria, a~tak mohli tieto body získať len zo vzájomných partií, čo znamená,
že prehrali všetky partie s~hráčmi z~prvej polovice výsledného poradia.
Hráč, ktorý skončil na 6.~mieste, preto prehral partiu s~hráčom, ktorý
skončil na 4.~mieste.


\návody
Šachového turnaja, v ktorom každý s~každým odohral jednu partiu, sa
zúčastnilo $n$~hráčov. Koľko partií bolo odohratých? Koľko bodov
získali všetci dokopy, ak za
víťazstvo získal hráč 1~bod, za remízu pol bodu a~za prehru žiadny bod?
[$\frac12 n(n - 1)$]

Šachového turnaja sa podľa pravidiel z~predošlej úlohy zúčastnili 4~hráči.
Víťaz turnaja získal rovnaký počet bodov ako zvyšní traja hráči dokopy.
\item{a)} Aký najväčší a~aký najmenší počet bodov mohol mať? [Získal práve 3~body.]
\item{b)} Koľko partií mohlo skončiť remízou, ak na konci turnaja mali
všetci účastníci rôzne počty bodov? [Buď 0, alebo 1, alebo 2.]

Hokejového turnaja sa zúčastnili štyri družstvá, pričom každé
odohralo s~každým práve jedno stretnutie. Počet gólov vstrelených v~každom
stretnutí delí celkový počet gólov vstrelených v~turnaji, pritom
v~žiadnych dvoch stretnutiach ich nepadol rovnaký počet. Koľko najmenej mohlo
v~turnaji padnúť gólov? [C--55--S--1]

Tomáš, Jakub, Martin a~Peter organizovali na námestí zbierku pre
dobročinné účely. Za chvíľu sa pri nich postupne zastavilo päť
okoloidúcich. Prvý dal Tomášovi do jeho pokladničky 3~Sk, Jakubovi 2~Sk,
Martinovi 1~Sk a~Petrovi nič. Druhý dal jednému z~chlapcov 8~Sk a~ostatným
trom nedal nič. Tretí dal dvom chlapcom po 2~Sk a~dvom nič. Štvrtý
dal dvom chlapcom po 4~Sk a~dvom nič. Piaty dal dvom chlapcom po 8~Sk
a~dvom nič. Potom chlapci zistili, že každý z~nich vyzbieral inú
čiastku, pričom tieto tvoria štyri po sebe idúce prirodzené čísla. Ktorý
z~chlapcov vyzbieral najmenej a~ktorý najviac korún? [C--58--I--1]
\endnávod
}

{%%%%%   A-S-1
\letterspacefont\sprm \tenrm 180
\def\rddots{\mathinner{\mkern0mu
             \raise-1pt\hbox{.}\mkern-2mu
             \raise3.5pt\hbox{.}\mkern-2mu
             \raise8pt\hbox{.}\mkern0mu }}%
Zadané číslo má vyjadrenie
$$
\align
\bigl(10^{2n-1}+&10^{2n-2}+\dots+10^{n+1}\bigr)+2\cdot10^n+
8\cdot\bigl(10^{n-1}+10^{n-2}+\dots+10^{2}\bigr)+96=\\
&=10^{n+1}\cdot\frac{10^{n-1}-1}{9}+2\cdot10^n+8\cdot10^2\cdot\frac{10^{n-2}-1}{9}+96=\\
&=\frac{10^{2n}-10^{n+1}+18\cdot10^n+800\cdot10^{n-2}-800+9\cdot96}{9}=\\
&\hfill =\frac{10^{2n}+16\cdot10^n+64}{9}=\Bigl(\frac{10^n+8}{3}\Bigr)^{\!2}.
\endalign
$$
Získali sme druhú mocninu celého čísla, pretože číslo $10^n+8$ je
deliteľné tromi~-- má totiž ciferný súčet rovný~$9$. (Namiesto toho
možno tiež konštatovať, že keby zlomok $\dfrac{10^n+8}{3}$ nebol celým
číslom, nebola by celým číslom ani jeho druhá mocnina, a~to by
bol spor.)

\ineriesenie
Ak objavíme experimentovaním niekoľko prvých rovností
(neuškodí uvedomiť si, že vzorec funguje aj pre $n=2$)
$$
1\,296=36^2,\quad 112\,896=336^2,\quad 11\,128\,896=3\,336^2,\quad \dots,
$$
napadne nás určite hypotéza, že pre každé $n\ge2$ bude platiť
$$
\underbrace{1\dots1}_{n-1}2 \underbrace{8\dots8}_{n-2}96=
\underbrace{33\dots3}_{n-1}6^2.
$$
Jej dôkaz urobíme použitím algoritmu písomného násobenia:
$$
\def\@{\phantom0}
\vbox{\sprm \textfont0\sprm \setbox1\hbox{0}\dimen1=\wd1
\halign{\strut\hss$#$&$#$\hss\cr
333&\dots3336\cr
\times333&\dots3336\cr \noalign{\nointerlineskip\hbox{\hskip6\dimen1\vrule width 11.5\dimen1 height.4pt}\nointerlineskip}
2000&\dots0016\cr
10000&\dots008 \cr
100000&\dots08 \cr
1000000&\dots8 \cr
\rddots\,\@\@\@\@\@\@& \rddots \cr
1\dots 000008& \cr
10\dots 00008\@& \cr
100\dots 0008\@\@& \cr \noalign{\hrule}
111\dots 112888&\dots8896\cr
}}
$$
Obe rovnaké násobené čísla sú $n$-ciferné, v~každom z~$n$~riadkov
medzi oddeľujúcimi linkami je $(n+1)$-ciferné číslo. Z~toho
ľahko určíme, ako stoja pod sebou cifry týchto následne
sčítaných čísel, a~teda aj počty rovnakých cifier vo výsledku.

\nobreak\medskip\petit\noindent
Za úplné riešenie dajte 6~bodov. Pri druhom postupe dajte 2~body
za určenie základu $33\dots36$ druhej mocniny
a~4~body za schému jej písomného výpočtu so všeobecným~$n$.
Ak riešiteľ tieto výpočty písomne urobí aspoň pre $n=3$
a~potom spomenie analógiu bez podrobnejšieho opisu, dajte
celkom 5~bodov. Ak chýbajú písomné výpočty a~riešiteľ sa namiesto nich
odvolá na (zakázanú) kalkulačku,
neudeľujte žiadny bod.

\endpetit
\bigbreak
}

{%%%%%   A-S-2
V~prvej časti riešenia budeme predpokladať, že
$|\uhel ABC|+|\uhel ACM|=90\st$.
Pri označení $\phi=|\uhel ACM|$ a~$\psi=|\uhel BCM|$
(\obr) je potom splnená nielen rovnosť $|\uhel ABC|=90\st-\phi$,
ale aj rovnosť $|\uhel BAC|=90\st-\psi$, ako vyplýva zo súčtu
uhlov v~trojuholníku $ABC$:
$$
\align
|\uhel BAC|=&180\st-|\uhel ABC|-|\uhel ACB|=\\
=&180\st-(90\st-\phi)-(\phi+\psi)=90\st-\psi.
\endalign
$$
\removelastskip
\insp{a63.8}%

Zo sínusovej vety pre trojuholníky $ACM$ a~$BCM$ tak vyplýva
$$
\frac{\sin(90\st-\psi)}{\sin\phi}=\frac{|CM|}{|AM|}=
\frac{|CM|}{|BM|}=\frac{\sin(90\st-\phi)}{\sin\psi}.
$$
Z~porovnania krajných zlomkov vzhľadom na vzorec
$\sin(90\st-\om)=\cos \om$ vyplýva rovnosť
$\sin\phi\cos\phi=\sin\psi\cos\psi$, čiže
$\sin2\phi=\sin2\psi$. Keďže uhly $\phi$ a~$\psi$ sú
ostré, oba uhly $2\phi$ a~$2\psi$ ležia v~intervale od $0\st$ po $180\st$.
Podľa známych vlastností funkcie sínus rovnosť
$\sin2\phi=\sin2\psi$ tak znamená, že buď $2\phi=2\psi$, alebo
$2\phi+2\psi=180\st$. V~prvom prípade ($\phi=\psi$) je trojuholník $ABC$
rovnoramenný so základňou~$AB$, v~druhom prípade
($\phi+\psi=90\st$) je pravouhlý s~preponou~$AB$.
Tým je dokázaná prvá (náročnejšia)
z~dvoch implikácií, z~ktorých je zložená ekvivalencia
zo zadania úlohy.

Pri dôkaze druhej (jednoduchšej) implikácie najskôr predpokladajme, že
trojuholník $ABC$ je pravouhlý s~preponou~$AB$. Podľa Tálesovej vety vtedy
platí $|MB|=|MC|$, a~tak sú zhodné uhly $MCB$ a~$MBC$ (čiže
$ABC$), odkiaľ už vyplýva
$$
|\uhel ABC|+|\uhel ACM|=|\uhel MCB|+|\uhel ACM|=|\uhel ACB|=90\st.
$$
Ostáva dokázať rovnakú rovnosť aj za predpokladu,
že trojuholník $ABC$ je rovnoramenný so základňou~$AB$. Vtedy
však sú trojuholníky $ACM$ a~$BCM$ zhodné a~majú pri vrchole~$M$ pravý
uhol, takže platí
$$
|\uhel ABC|+|\uhel ACM|=|\uhel MBC|+|\uhel BCM|=
180\st-|\uhel BMC|=90\st.
$$
Tým je aj dôkaz druhej implikácie ukončený a~celá úloha je vyriešená.

\ineriesenie
Zostrojme kružnicu~$k$ opísanú
danému trojuholníku $ABC$ a~jeho ťažnicu~$CM$ predĺžme za bod~$M$
na tetivu~$CC'$ kružnice~$k$ (\obr). Zo zhodnosti
obvodového uhla $ABC'$ s~obvodovým uhlom $ACC'$ (čiže uhlom $ACM$)
vyplýva, že súčet uhlov $ABC$ a~$ACM$ zo zadania úlohy má rovnakú
veľkosť ako uhol $CBC'$. Podľa Tálesovej vety je táto veľkosť
rovná $90\st$ práve vtedy, keď tetiva~$CC'$ kružnice~$k$ je jej
priemerom. To nastane práve vtedy, keď stred $S$ kružnice $k$ bude
ležať na polpriamke~$CM$. Pre takú situáciu rozlíšime prípady
$S=M$ a~$S\ne M$. Prvý prípad podľa Tálesovej vety
nastane práve vtedy, keď bude trojuholník $ABC$ pravouhlý s~preponou~$AB$.
Druhý prípad ($C$, $M$ a~$S$ sú tri rôzne body ležiace na jednej priamke) nastane
práve vtedy, keď bude priamka~$MS$, ktorá je osou úsečky~$AB$,
prechádzať bodom~$C$, teda práve vtedy, keď
bude trojuholník $ABC$ rovnoramenný so základňou~$AB$ (a~pritom uhol $ACB$
nebude pravý). Tým je dokázaná celá ekvivalencia zo zadania úlohy.
\inspinsp{a63.9}{a63.10}%

\poznamka
V~predchádzajúcom postupe bolo možné namiesto tetivy~$CC'$
kružnice~$k$ využiť jej dotyčnicu~$t$ v~bode~$C$ (\obr). Zo
zhodnosti obvodového uhla $ABC$ s~vyznačeným úsekovým uhlom medzi
tetivou~$AC$ a~dotyčnicou~$t$ totiž vyplýva, že súčet uhlov $ABC$ a~$ACM$
je rovný $90\st$ práve vtedy, keď je dotyčnica~$t$ kolmá na polpriamku~$CM$,
teda práve vtedy, keď na tejto polpriamke leží stred~$S$ kružnice~$k$.


\nobreak\medskip\petit\noindent
Za úplné riešenie dajte 6~bodov, z~toho 5~bodov za prvú (náročnejšiu)
implikáciu a~1~bod za oba prípady druhej (jednoduchšej) implikácie.

\endpetit
\bigbreak
}

{%%%%%   A-S-3
Hľadaný počet úsečiek, ktoré prechádzajú vnútornými bodmi valca,
určíme tak, že od celkového počtu zostrojených úsečiek odčítame
jednak počet tých úsečiek, ktoré ležia na plášti valca, jednak
počet tých úsečiek, ktoré ležia v~niektorej z~oboch podstáv valca.

Výpočet urobíme pre prípad valca s~obvodom podstavy~$x$
a~výškou~$y$.
Spájané body sú teda na valci rozložené na $x$~úsečkách
po $y+1$ exemplároch, takže ich počet je
$x(y+1)$. Pre počet $P_0$ všetkých zostrojených úsečiek
preto platí vzorec
$$
P_0=\binom{x(y+1)}{2}=\frac{x(y+1)(xy+x-1)}{2},
$$
počet $P_1$ úsečiek ležiacich na plášti má vyjadrenie
$$
P_1=x\cdot\binom{y+1}{2}=\frac{x(y+1)y}{2}
$$
a~napokon počet $P_2$ úsečiek v~oboch podstavách je daný vzorcom
$$
P_2=2\cdot\binom{x}{2}=x(x-1).
$$
Z~toho už pre hľadaný počet $P$ úsečiek,
ktoré prechádzajú vnútornými bodmi valca, dostaneme
vzorec
$$
\align
P&=P_0-P_1-P_2=\frac{x(y+1)(xy+x-1)}{2}-\frac{x(y+1)y}{2}-x(x-1)=\\
&=\frac{x(x-1)(y^2+2y-1)}{2}.
\endalign
$$

Pre zodpovedajúci počet $Q$ úsečiek,
ktoré prechádzajú vnútornými bodmi druhého valca s~obvodom podstavy~$y$ a~výškou~$x$,
zrejme platí analogický vzorec
$$
Q=\frac{y(y-1)(x^2+2x-1)}{2}.
$$

Pre porovnanie oboch počtov $P$ a~$Q$ upravíme ich rozdiel $P-Q$
(s~vedomím, že ten bude násobkom dvojčlena $x-y$,
pretože pre $x=y$ zrejme platí $P=Q$):
$$
\align
2(P-Q)=&(x^2-x)(y^2+2y-1)-(y^2-y)(x^2+2x-1)=\\
=&(x^2y^2-xy^2+2x^2y-2xy-x^2+x)- \\
&-(x^2y^2-x^2y+2xy^2-2xy-y^2+y)=\\
=&3xy(x-y)-(x-y)(x+y)+(x-y)=\\
=&(x-y)(3xy-x-y+1).
\endalign
$$

V~prípade $x>y$ ako väčšie vyjde číslo~$P$, pretože ukážeme, že výraz
$3xy-x-y+1$ je kladný:
z~$y\ge2$ máme $3xy\ge6x$, a~preto
$$
3xy-x-y+1\ge5x-y+1>4x+1>0.
$$

\poznamka
Opíšme kratší spôsob určenia hľadaného počtu úsečiek, a~to opäť
pre valec s~obvodom podstavy~$x$ a~výškou~$y$.

Kolmým priemetom každej započítanej úsečky do podstavy je jedna
z~$\frac12x(x-1)$ úsečiek, ktoré spájajú $x$~bodov na hraničnej
kružnici. Do jednej z~týchto úsečiek sa vždy premietne
$(y+1)^2-2=y^2+2y-1$ započítaných úsečiek, pretože $y+1$ je počet
spájaných bodov s~rovnakým priemetom a~od súčinu $(y+1)(y+1)$ treba odčítať
číslo~$2$ za dve spojnice ležiace v~podstavách.
Hľadaný počet $P$ je teda rovný
$$
P=\frac{x(x-1)(y^2+2y-1)}{2}.
$$

\nobreak\medskip\petit\noindent
Za úplné riešenie dajte 6~bodov, z~toho 4~body za určenie
počtu úsečiek s~bodmi vnútri valca a~ďalšie 2~body potom za zdôvodnenie,
ktorý z~oboch počtov je väčší. Prvé 4~body sa dajú čiastkovo udeľovať
takto: po 1~bode za určenie čísel $P_0$, $P_1$, $P_2$ a~1~bod
za správne výsledné dopočítanie~$P$.

\endpetit
\bigbreak
}

{%%%%%   A-II-1
Využime zvyčajný zápis $n=p_1^{\al_1}p_2^{\al_2}\dots p_k^{\al_k}$
prvočíselného rozkladu hľadaného čísla~$n$,
v~ktorom $p_1<p_2<\dots<p_k$ sú všetky prvočísla deliace $n$
a~exponenty $\al_i$ sú kladné celé čísla.
Z~podmienky úlohy vyplýva, že $p_1=2$ (inak by najväčším nepárnym
deliteľom čísla~$n$ bolo samo $n$ a~dostali by sme nerovnosť $n>3n$,
ktorá nemôže platiť) a~že $k\ge2$ (inak by $n$ bolo mocninou čísla~$2$).
Najväčším nepárnym deliteľom čísla~$n$ je zrejme
číslo $p_2^{\al_2}\dots p_k^{\al_k}$, jeho najmenším nepárnym
deliteľom (väčším ako $1$) je určite prvočíslo~$p_2$. Rovnica
vyjadrujúca podmienku úlohy má preto zápis
$$
2^{\al_1}p_2^{\al_2}\dots
p_k^{\al_k}=3p_2^{\al_2}\dots p_k^{\al_k}+5p_2,\quad\text{čiže}\quad
\bigl(2^{\al_1}-3\bigr)p_2^{\al_2-1}\dots p_k^{\al_k}=5.
$$
(V~prípade $k=2$ je ľavá strana upravenej rovnice
rovná $\bigl(2^{\al_1}-3\bigr)p_2^{\al_2-1}$.)
Keďže číslo~$5$ má jediné dva delitele, platí $2^{\al_1}-3\in\{1,5\}$,
takže $\al_1=2$ alebo $\al_1=3$.

(i) Prípad $\al_1=2$. Upravená rovnica prejde na tvar
$$
p_2^{\al_2-1}\dots p_k^{\al_k}=5,
$$
takže je splnená práve vtedy, keď je buď $k=2$, $p_2=5$ a~$\al_2-1=1$,
alebo $k=3$, $\al_2-1=0$, $p_3=5$ a~$\al_3=1$~--
vtedy ale prvočíslo $p_2$
nemôže byť ľubovoľné, lebo z~$2<p_2<p_3=5$ vyplýva $p_2=3$.
Prvej možnosti tak zodpovedá jediné riešenie $n=2^2\cdot5^2=100$, druhej
možnosti jediné riešenie $n=2^2\cdot3^1\cdot5^1=60$.

(ii) Prípad $\al_1=3$. Teraz dostávame po úprave rovnicu
$$
p_2^{\al_2-1}\dots p_k^{\al_k}=1,
$$
ktorá znamená, že $k=2$ a~$\al_2-1=0$ (na prvočíslo $p_2$
tentoraz žiadne obmedzenie okrem nerovnosti $p_2>2$ neexistuje).
Tomuto prípadu tak zodpovedá nekonečne veľa riešení $n=2^3\cdot p_2^1=8p_2$.

\odpoved
Všetky vyhovujúce celé kladné čísla $n$ sú:
$n=60$, $n=100$ a~$n=8p$, pričom $p$ je ľubovoľné nepárne prvočíslo.

\poznamka
Celý postup zapíšeme úspornejšie, keď namiesto "úplného"
prvočíselného rozkladu hľadaného čísla~$n$ využijeme jeho
vyjadrenie $n=2^{\al}\cdot p\cdot l$, v~ktorom $2^{\al}$ je najvyššia mocnina
čísla $2$, ktorá delí~$n$, $p$ je najmenšie nepárne prvočíslo deliace $n$
a~$l$ je (nepárne) číslo, ktoré nemá žiadneho prvočíselného deliteľa menšieho ako
$p$ (môže byť aj~$l=1$, v~ostatných prípadoch však $l\ge p$).
Potom je našou úlohou riešiť rovnicu
$$
n=2^{\al}pl=3pl+5p,\quad\text{čiže}\quad
\bigl(2^{\al}-3\bigr)l=5.
$$
Odtiaľ máme buď $l=1$ a~$2^{\al}-3=5$, alebo $l=5$ a~$2^{\al}-3=1$.
V~prvom prípade vychádza $\al=3$, a~teda vyhovuje každé $n=8p$,
pričom $p$ je ľubovoľné nepárne prvočíslo; v~druhom prípade je $l=5$ a~$\al=2$, takže $n=20p$, pričom
ale z~$5=l\ge p$ vyplýva $p\in\{3,5\}$, a~preto vyhovujú iba
čísla $n=60$ a~$n=100$.

\ineriesenie
Zo zadania vyplýva, že medzi hľadaným číslom~$n$ a~jeho najväčším
nepárnym deliteľom~$L$ platia nerovnosti $n>3L$ a~$n\le3L+5L=8L$.
Keďže podiel $n:L$ je mocnina čísla $2$, z~nerovností
$3<n:L\le8$ vyplýva, že buď $n=4L$, alebo $n=8L$.

Začneme s~rozborom prípadu $n=8L$. Zo spôsobu, akým sme
odvodili nerovnosť $n\le8L$, vyplýva, že číslo~$L$ je nielen najväčším, ale
aj najmenším netriviálnym nepárnym deliteľom čísla~$n$, a~preto je
prvočíslom. Dostávame prvú skupinu hľadaných čísel~$n$, ktoré majú
tvar $n=8L$, pričom $L$ je ľubovoľné nepárne prvočíslo.

V~prípade $n=4L$ je najmenší netriviálny nepárny deliteľ čísla $n$
také prvočíslo~$p$, ktorého päťnásobok je rovný číslu
$n-3L=L$. Z~rovnosti $5p=L$ máme $n=4L=20p$, a~preto $5\mid n$, takže $p\le5$, čiže $p\in\{3,5\}$. Zodpovedajúce riešenia sú $n=60$
a~$n=100$.


\nobreak\medskip\petit\noindent
Za úplné riešenie dajte 6~bodov, z~toho 2~body za vhodné vyjadrenie
oboch dotyčných deliteľov a~zostavenie rovnice vyjadrujúcej podmienku
úlohy, ďalší 1~bod za úpravu na rovnicu medzi istým
súčinom a~číslom 5. Podľa úplnosti upravenej rovnice
potom dajte zostávajúce tri body. Pri postupe z~druhého riešenia dajte
1~bod za nerovnosť $n>3L$, 2~body za nerovnosť
$n\le 8L$, 1~bod za dôsledok o~možných tvaroch $n=4L$, $n=8L$
a~po 1~bode za ich rozbor.

\endpetit
\bigbreak
}

{%%%%%   A-II-2
V~prvej časti riešenia predpokladajme,
že $X$ je ľubovoľný bod, ktorý má požadovanú vlastnosť.
Zrejme musí ležať vo vonkajšej oblasti každej z~oboch kružníc.
Body $S_1$, $S_2$ a~$X$ sú potom vrcholmi trojuholníka, ktorého
strany $S_1X$, $S_2X$ sú preťaté postupne
kružnicami $k_1$, $k_2$ v~bodoch $Y_1$ a~$Y_2$, ktoré ležia
na jednej rovnobežke s~priamkou $S_1S_2$ (\obr).
Preto aj body $Y_1$, $Y_2$, $X$ sú vrcholmi
trojuholníka, ktorý je podobný trojuholníku $S_1S_2X$ podľa vety~$uu$, teda
pre ich strany platí úmera
$$
\frac{|XY_1|}{|XS_1|}=\frac{|XY_2|}{|XS_2|},
\tag1
$$
ktorú vďaka rovnostiam
$$
|XY_1|=|XS_1|-r_1\quad\text{a}\quad|XY_2|=|XS_2|-r_2
\tag2
$$
prevedieme na úmeru pre dĺžky úsečiek $XS_1$ a~$XS_2$:
$$
\align
\frac{|XS_1|-r_1}{|XS_1|}&=\frac{|XS_2|-r_2}{|XS_2|},\\
\intext{takže}
% \intext{která je ekvivalentní s rovností}
\frac{|XS_1|}{|XS_2|}&=\frac{r_1}{r_2}.\tag3
\endalign
$$
\insp{a63.11}%

Množinu bodov v~rovine s~danými bodmi $S_1$ a~$S_2$, ktoré majú vlastnosť~\thetag3,
poznáme: pre $r_1=r_2$ je to os úsečky~$S_1S_2$ a~pre $r_1\ne r_2$ je
to {\it Apollóniova kružnica}. Tá je určená svojim priemerom~$H_1H_2$
na priamke~$S_1S_2$, na ktorej sú $H_1$ a~$H_2$ jediné dva body~$X$
s~vlastnosťou~\thetag3. Z~tej navyše vyplýva, že sa jedná o~stredy rovnoľahlostí
kružníc $k_1$ a~$k_2$.

\smallskip
V~druhej časti riešenia budeme naopak predpokladať, že bod~$X$ je
ľubovoľný bod Apollóniovej kružnice určenej rovnicou~\thetag3, ktorý je
rôzny od jej priesečníkov $H_1$, $H_2$ s~priamkou $S_1S_2$.
Vzhľadom na~predpoklady úlohy ležia body $H_1$, $H_2$ vo vonkajšej oblasti
oboch kružníc, takže tam leží aj príslušná Apollóniova kružnica, pretože
jej priemer obsahuje priemer jednej z~daných kružníc (tej s~menším polomerom)
a~s~priemerom druhej kružnice je disjunktná.

Body $S_1$, $S_2$ a~$X$ sú tak vrcholmi trojuholníka, pričom
$|XS_1|>r_1$ a~$|XS_2|>r_2$.
Existujú teda body $Y_1\in k_1$, $Y_2\in k_2$
ležiace vnútri strán $S_1X$, $S_2X$ trojuholníka $S_1S_2X$. Preto
pre ne tiež platia rovnosti \thetag2, vďaka ktorým možno prejsť tentoraz
od rovnice \thetag3 k~rovnici \thetag1. Jej platnosť znamená,
že trojuholníky $S_1S_2X$ a~$Y_1Y_2X$ sú podobné podľa vety $sus$,
a~preto sú úsečky $S_1S_2$ a~$Y_1Y_2$ rovnobežné. Body $Y_1$,
$Y_2$ tak majú od priamky~$S_1S_2$ rovnaké vzdialenosti, čo
dokazuje požadovanú vlastnosť bodu~$X$.

\odpoved
Hľadanou množinou bodov~$X$ je Apollóniova kružnica
určená rovnicou \thetag3, z~ktorej sú vylúčené oba
jej priesečníky s~priamkou~$S_1S_2$. V~prípade $r_1=r_2$ je
hľadanou množinou os úsečky~$S_1S_2$ bez jej stredu.

\poznamka
Potrebnú vlastnosť Apollóniovej kružnice možno odvodiť priamo z~rovnosti~\thetag3.
Z~rovnosti, ktorá je pred rovnosťou \thetag3 a~ktorá je s~ňou v~skutočnosti
ekvivalentná, pre ľubovoľný taký bod~$X$
vyplýva, že oba výrazy $|XS_1|-r_1$ a~$|XS_2|-r_2$ majú rovnaké
znamienko. A~keďže podľa predpokladov úlohy je
$$
(|XS_1|-r_1)+(|XS_2|-r_2)>|S_1S_2|-(r_1+r_2)>0,
$$
je $|XS_1|>r_1$ a~$|XS_2|>r_2$, čo znamená, že nájdená Apollóniova
kružnica leží v~prieniku vonkajších oblastí oboch daných kružníc.


\nobreak\medskip\petit\noindent
Za úplné riešenie dajte 6~bodov, z~toho 4~body za dôkaz, že
každý vyhovujúci bod~$X$ leží na Apollóniovej kružnici a~2~body
za obrátené tvrdenie. Zo zadania úlohy je ihneď zrejmé, že všetky
vyhovujúce body~$X$ musia ležať v~prieniku vonkajších oblastí oboch
kružníc $k_1$ a~$k_2$. Poznatok, že v~tomto prieniku leží aj nájdená
Apollóniova kružnica, by nemal v~úplnom riešení chýbať.
Za absenciu tohto poznatku strhnite 1~bod.
Tiež strhnite 1~bod, ak riešiteľ v~priebehu riešenia alebo
na záver nespomenie situáciu, keď $r_1=r_2$.

\endpetit
\bigbreak
}

{%%%%%   A-II-3
Ak je napr. $x=0$, dostávame sústavu $0=yz^2=2y^2z$, takže aj jedna
z~hodnôt $y$, $z$ je nulová a~druhá môže byť ľubovoľná. Podobne
vyriešime aj prípady $y=0$ a~$z=0$. Dostávame tak tri skupiny
riešení $(t,0,0)$, $(0,t,0)$ a~$(0,0,t)$, pričom $t$ je ľubovoľné reálne číslo.
Všetky ostatné riešenia už spĺňajú podmienku $xyz\ne0$, ktorej platnosť
budeme vo zvyšku riešenia predpokladať.

Úpravou rovnice
$x(y^2+2z^2)=y(z^2+2x^2)$ dostaneme
$(2x-y)(z^2-xy)=0$. Rozlíšime teda, ktorý z~dvoch činiteľov je
rovný nule.

(i) $2x-y=0$. Z~pôvodnej sústavy po dosadení $y=2x$ zostane jediná rovnica
$$
% 2x(2x^2+z^2)=
2x(2x^2+z^2)=9x^2z,
$$
odkiaľ po delení číslom $x\ne0$ dostávame %vychádza jediná rovnica
$$
4x^2+2z^2-9xz=0,\quad\text{čiže}\quad (x-2z)(4x-z)=0.
$$
Prípadu (i) teda zodpovedajú skupiny riešení $(2t,4t,t)$ a~$(t,2t,4t)$,
pričom $t$ je ľubovoľné reálne číslo.

(ii) $z^2-xy=0$. Čiastočným dosadením $z^2=xy$ dostaneme jedinú rovnicu
$$
% xy(2x+y)=
xy(2x+y)=z(x^2+2y^2),
$$
ktorá je (vďaka tomu, že $x^2+2y^2>0$) ekvivalentná s~rovnicou
$$
z=\frac{xy(2x+y)}{x^2+2y^2}.
$$

Teraz ale musíme zistiť, kedy také $z$ spĺňa podmienku
$z^2=xy$, ktorá po dosadení nájdeného vzorca pre $z$ získava tvar
$$
\frac{x^2y^2(2x+y)^2}{(x^2+2y^2)^2}=xy.
$$
Po vydelení číslom $xy\ne0$ a~odstránení zlomku dostaneme podmienku
$$
xy(2x+y)^2=(x^2+2y^2)^2,\quad\text{čiže}\quad
(4y-x)(x^3-y^3)=0.
$$
Posledná rovnosť platí práve vtedy, keď buď $x=4y$, alebo $x^3=y^3$, \tj.
$x=y$.\footnote{Posledný záver platí vďaka tomu, že funkcia
$x\mapsto x^3$ je na reálnom obore rastúca, a~teda prostá.}
Po dosadení do vzorca pre $z$ dostaneme v~prvom prípade $z=2y$,
v~druhom $z=x$. Prípadu (ii) teda zodpovedajú skupiny riešení
$(4t,t,2t)$ a~$(t,t,t)$, pričom $t$ je ľubovoľné reálne číslo.

\odpoved
Všetky riešenia danej sústavy sú $(t,0,0)$, $(0,t,0)$, $(0,0,t)$,
$(t,t,t)$, $(4t,t,2t)$, $(2t,4t,t)$ a~$(t,2t,4t)$, pričom $t$ je ľubovoľné reálne číslo.

\poznamka
Ukážeme spôsob, ako sa v~podanom riešení vyhnúť
rozboru náročnejšieho prípadu (ii). Vďaka cyklickej symetrii má
zadaná sústava za dôsledok nielen prvú z~nasledujúcich troch rovníc
(ktorú sme vyššie odvodili), ale aj ďalšie dve analogické rovnice
$$
(2x-y)(z^2-xy)=0,\quad
(2y-z)(x^2-yz)=0,\quad
(2z-x)(y^2-zx)=0.
\tag1$$
Prípad $2x-y=0$ sme vyššie rozobrali, prípady $2y-z=0$
a~$2z-x$ sú analogické a~vypísaním riešení pre tieto tri prípady
dostaneme všetky riešenia uvedené v~odpovedi s~výnimkou $(t,t,t)$.
Ak nenastane žiadny z~týchto troch prípadov, musia byť splnené rovnice
$$
z^2-xy=x^2-yz=y^2-zx=0.
\tag2$$
Ukážeme, že ich v~obore reálnych čísel spĺňajú iba trojice
$(x,y,z)=(t,t,t)$. To je jednoduchý dôsledok algebraickej identity
$$
(x-y)^2+(y-z)^2+(z-x)^2=2(z^2-xy)+2(x^2-yz)+2(y^2-zx),
$$
ktorej pravá strana je podľa \thetag2 rovná nule, takže sa rovná nule
aj základ každej z~troch druhých mocnín na ľavej strane (ktorá by inak
mala kladnú hodnotu).
Dodajme, že sústavu~\thetag2 možno vyriešiť aj kratšou úvahou: ak platí rovnosť~\thetag2,
majú výrazy $x^3$, $y^3$, $z^3$ tú istú hodnotu $xyz$, a~tak platí $x=y=z$
podľa poznámky pod čiarou.

\ineriesenie
%vytřizrakové
Nebudeme opakovať úvodnú úvahu pôvodného
riešenia a~rovno budeme hľadať len tie riešenia, ktoré spĺňajú podmienku
$xyz\ne0$.

Po vydelení výrazov v~zadanej sústave nenulovým číslom $xyz$
dostaneme
$$
\frac{y}{z}+\frac{2z}{y}=\frac{z}{x}+\frac{2x}{z}=
\frac{x}{y}+\frac{2y}{x},
\tag3$$
čo je rovnosť troch hodnôt funkcie $f(s)=s+2/s$ v~nenulových
bodoch $s_1=y/z$, $s_2=z/x$ a~$s_3=x/y$. Zistíme preto, kedy pre
nenulové čísla $s$ a~$t$ platí $f(s)=f(t)$. Z~vyjadrenia
$$
\postdisplaypenalty 10000
f(s)-f(t)=s+\frac{2}{s}-t-\frac{2}{t}=\frac{(s-t)(st-2)}{st}
$$
vidíme, že želaná situácia nastane, len ak $s=t$ alebo $st=2$.

Sústava rovníc \thetag3 je preto splnená práve vtedy, keď pre zavedené
čísla $s_1$, $s_2$, $s_3$ platí: každé dve z~nich sa rovnajú alebo je
ich súčin rovný číslu $2$. Ak je ale taký súčin rovný~$2$,
tretie číslo je vďaka rovnosti $s_1s_2s_3=1$ rovné
$\frac12$ a~prvé dve čísla (majúce~súčin~$2$) teda ležia v~množine
$\{\frac12,4\}$, takže sú rôzne, a~preto
$(s_1,s_2,s_3)$ je niektorou permutáciou
trojice $(\frac12,\frac12,4)$. Ľahko možno overiť, že týmto
(spolu trom) permutáciám zodpovedajú riešenia
$(4t,t,2t)$, $(2t,4t,t)$ a~$(t,2t,4t)$ pôvodnej sústavy
(nebudeme to tu rozpisovať). Ešte jednoduchšie je dokončenie zvyšného
prípadu $s_1=s_2=s_3$: vďaka rovnosti $s_1s_2s_3=1$ je spoločná
hodnota čísel $s_i$ rovná $1$ a~zodpovedajúce riešenia zrejme sú
$(t,t,t)$ (opäť je $t$ ľubovoľné reálne číslo).


\nobreak\medskip\petit\noindent
Za úplné riešenie dajte 6~bodov, z~toho 2~body za odvodenie
súčinového tvaru aspoň jednej z~rovníc \thetag1 alebo za úpravu
sústavy na tvar \thetag3 doplnenú zmienkou o~súvislosti s~funkciou
$f(s)=s+2/s$.

\endpetit
\bigbreak
}

{%%%%%   A-II-4
Permutácie kurtov v~jednotlivých kolách aj permutácie samotných
kôl posúdime nakoniec; najskôr {\it družstvá pevne označíme číslami\/}
1, 2, 3, 4, 5, 6 a~podľa toho
{\it päticu kôl ľubovoľného rozpisu jednoznačne preusporiadame}. Zápas družstva $x$ proti družstvu $y$ budeme označovať ako pár $(x,y)$ (máme pritom na pamäti, že na poradí čísel v~páre nezáleží).

Za kolá 1 a~2 prehlásime kolá postupne s~pármi $(1,2)$ a~$(1,3)$;
ak je pritom v~kole~1 pár $(3,a)$ a~v~kole~2 pár $(2,b)$,
sú $a$,~$b$ dve {\it rôzne\/} čísla z~$\{4,5,6\}$, inak by nám
totiž pre tretí zápas v~každom z~oboch kôl zvýšila tá istá dvojica.
Za kolá 3, 4 a~5 potom prehlásime kolá
postupne s~pármi $(1,a)$, $(1,b)$ a~$(1,c)$, pričom
$c\in\{4,5,6\}\setminus\{a,b\}$. Máme teda jednoznačne určené poradie
všetkých kôl s~neúplným rozpisom
$$
\matrix
1\:&(1,2),&(3,a),\\
2\:&(1,3),&(2,b),\\
3\:&(1,a),&\\
4\:&(1,b),&\\
5\:&(1,c),&
\endmatrix
$$
ktorý sa dá zrejme jediným spôsobom doplniť na úplný rozpis:
$$
\matrix
1\:&(1,2),&(3,a),&(b,c),\\
2\:&(1,3),&(2,b),&(a,c),\\
3\:&(1,a),&(2,c),&(3,b),\\
4\:&(1,b),&(2,a),&(3,c),\\
5\:&(1,c),&(2,3),&(a,b).
\endmatrix
$$
Keďže $(a,b,c)$ je ľubovoľná permutácia trojice $(4,5,6)$, je
počet takto zapísaných rozpisov práve $3!=6$, v~každom takom
rozpise potom môžeme kolá usporiadať $5!$ spôsobmi a~nakoniec v~každom
z~piatich kôl priradiť párom kurty práve $3!=6$ spôsobmi. Hľadaný
počet je teda rovný číslu
$$
6\cdot5!\cdot6^5=5!\cdot6^6=2^9\cdot3^7\cdot5=5\,598\,720.
$$
% pow(2,9)*pow(3,7)*5
% $ans = 5598720

\ineriesenie
Označme družstvá číslami 1, 2, 3, 4, 5 a~6 a~zostavme najskôr
neusporiadaný rozpis turnaja tak, že
jednotlivé kolá "očíslujeme" súpermi družstva číslo~1
a~preskúmame, koľkými spôsobmi sa dá k~týmto dvojiciam
doplniť zápas družstva číslo~2, pričom súpera dvojky
vyberáme už len zo štvorprvkovej množiny $\{3,4,5,6\}$:
$$
\matrix
1\:&(1,2),&\\
2\:&(1,3),&(2,a),\\
3\:&(1,4),&(2,b),\\
4\:&(1,5),&(2,c),\\
5\:&(1,6),&(2,d).
\endmatrix
$$

Na dopĺňané páry $(2,a)$, $(2,b)$, $(2,c)$, $(2,d)$ máme nasledujúce obmedzenia:

\smallskip
\item{$\triangleright$}
Družstvo~2 nemôže mať v~žiadnom kole rovnakého súpera ako družstvo~1.

\item{$\triangleright$}
Obe družstvá 1 a~2 nemôžu mať v~dvoch rôznych kolách striedavo
tých istých súperov, pretože potom by nám na tretí zápas v~oboch kolách
zvýšila tá istá dvojica.

\smallskip
Pri splnení týchto dvoch podmienok je potom zrejme jednoznačne určená
zvyšná dvojica v~každom z~kôl, v~ktorých proti sebe nehrajú 1 a~2.
Po ich určení nám zostanú práve dve dvojice súperov, ktoré sa
musia stretnúť v~kole, v~ktorom spolu hrajú 1 a~2.

Ostáva už len určiť počet permutácií
štvorprvkovej množiny $\{3,4,5,6\}$ súperov družstva~2, ktoré spĺňajú
uvedené dve podmienky.

Počet permutácií $(a,b,c,d)$ spĺňajúcich prvú podmienku, teda nerovnosti
$a\ne3$, $b\ne4$, $c\ne5$ a~$d\ne6$, nájdeme metódou inklúzie a~exklúzie:
vhodných permutácií je
$$
4!-\left(4\cdot3!-\binom42\cdot 2!+4-1\right)=9.
$$

Medzi nimi sú práve tri permutácie $(a,b,c,d)$, ktoré nevyhovujú druhej
podmienke:
jedná sa zrejme o~permutácie $(4,3,6,5)$, $(5,6,3,4)$ a~$(6,5,4,3)$ a~žiadne iné.

Všetkých možných rozpisov je tak celkom~6, v~každom kole máme však $3!=6$
možností, ako priradiť párom súperov jednotlivé kurty, a~celkom 5!~možných usporiadaní
jednotlivých kôl, dokopy teda
$$
6\cdot6^5\cdot5!=5!\cdot6^6=2^9\cdot3^7\cdot5=5\,598\,720 \text{ možností}.
$$

\poznamka
Namiesto použitia princípu inklúzie a~exklúzie %možno
môžeme všetky permutácie $(a,b,c,d)$
vyhovujúce obom podmienkam vhodným systematickým postupom priamo nájsť
a~vypísať. Jedná sa o~permutácie
$(4,5,6,3)$,
$(4,6,3,5)$,
$(5,3,6,4)$,
$(5,6,4,3)$,
$(6,3,4,5)$,
$(6,5,3,4)$.

Naopak, k~počtu permutácií vyhovujúcich prvej, nie však druhej podmienke môžeme dospieť
nasledujúcou úvahou: máme šesť možností, ako vybrať prestriedavanú dvojicu súperov,
pritom také dvojice sú v~každej z~počítaných permutácií zrejme dve.


\nobreak\medskip\petit\noindent
Za úplné riešenie dajte 6~bodov.
Dajte 2 až 3~body, ak riešiteľ opíše nejaký zmysluplný spôsob,
ako vytvoriť všetky možné rozpisy pre konanie turnaja, aj keď sa
mu nakoniec nepodarí z~toho odvodiť správny počet možností.

\endpetit
}

{%%%%%   A-III-1
Rozlíšime, či hľadané $n$ je nepárne alebo párne.

(i) Nech $n$ je nepárne, potom aj všetky $d_i$ sú nepárne.
Z~rovnosti $11d_5+8d_7=3n$ vyplýva $d_7\mid11d_5$ a~tiež
$d_5\mid8d_7$ čiže $d_5\mid d_7$. Z~$d_5\mid d_7\mid11d_5$
vzhľadom na $d_7>d_5$ máme $d_7=11d_5$ a~po dosadení do rovnosti
$11d_5+8d_7=3n$ dostaneme $99d_5=3n$, čiže $33d_5=n$. Vidíme, že
štyri čísla $1$, $3$, $11$ a~$33$ sú delitele čísla~$n$,
a~to dokonca jediné delitele menšie ako $50$, pretože pre piaty deliteľ
$d_5$ už podľa zadania platí $d_5=d_3+50>50$. Máme teda $d_1=1$,
$d_2=3$, $d_3=11$, $d_4=33$, $d_5=d_3+50=61$, a~preto
$n=33d_5={33\cdot61}=2\,013$. Toto číslo naozaj vyhovuje, lebo
jeho najmenšie delitele sú v~predchádzajúcej vete vypísané správne, navyše
platí $d_6=61\cdot3$ a~$d_7=61\cdot11$, takže je naozaj
splnené $d_7=11d_5$, teda i~všetko požadované.

(ii) Nech $n$ je párne. Z~rovnosti $11d_5+8d_7=3n$ potom vyplýva
$2\mid d_5$, takže aj $2\mid d_5-50=d_3$. Keďže $d_1=1$
a~$d_2=2$, nemôže byť $d_3=3$, takže je buď $d_3=4$, alebo $d_3=2t$,
pričom $t>2$. Posledné však možné nie je (číslo $t<d_3$ by chýbalo vo
výpise najmenších deliteľov čísla~$n$), a~preto je nutne $d_3=4$.
Potom je ale $d_5=d_3+50=54$, a~teda $3\mid n$, napriek tomu, že $3$ nie je
medzi najmenšími deliteľmi čísla~$n$. Žiadne vyhovujúce párne~$n$ preto
neexistuje.

\odpoved
Úloha má jediné riešenie $n=2\,013$.

\ineriesenie
Pre delitele $d_5<d_7$ máme $d_5=n/x$ a~$d_7=n/y$, pričom $x>y$
sú opäť kladné delitele čísla~$n$. Dosadením do $11d_5+8d_7=3n$
dostaneme po vydelení $n$ rovnicu $11/x+8/y=3$, ktorú v~obore
všetkých celých kladných čísel štandardne vyriešime,
napríklad úpravou na súčinový tvar:
$$
8x=y(3x-11)\ \Leftrightarrow\ 8(3x-11)+88=3y(3x-11)\ \Leftrightarrow\ (3x-11)(3y-8)=88.
$$
Z~prvej upravenej rovnice máme $3x-11>0$, takže z~poslednej aj
$3y-8>0$; vzhľadom na $x\ge y+1$ teda platí $3x-11\ge
3y-8>0$. Pre usporiadanú dvojicu činiteľov
z~rozkladu čísla~$88$ preto dostávame možnosti
$$
(3x-11,3y-8)\in\{(88, 1),(44, 2),(22, 4),(11, 8)\}.
$$
Podľa zvyškov modulo~$3$ však vyhovujú iba dvojice $(88, 1)$
a~$(22, 4)$, ktorým zodpovedajú páry $(x,y)$ rovné $(33, 3)$, resp.
$(11, 4)$. Pre prvý z~nich máme $d_5=n/33$ (a~$d_7=n/3$),
takže $1$, $3$, $11$ a~$33$ sú delitele čísla~$n$, odkiaľ rovnako ako
v~prvom riešení dôjdeme k~hľadanému $n=2\,013$. Pre $(x,y)=(11, 4)$
čiže $d_5=n/11$
a~$d_7=n/4$ má číslo~$n$ delitele $1$, $2$, $4$, $11$, $22$, $44$, čo je v~spore s~nerovnosťou $d_5>50$.
}

{%%%%%   A-III-2
Podľa zadania ležia body $Y$ a~$Z$ v~tej istej polrovine s~hraničnou
priamkou~$AB$. Zostrojme obraz~$Y'$ bodu~$Y$ v~súmernosti podľa priamky~$AB$.
Vzhľadom na predpokladanú podobnosť sú uhly $XAZ$ a~$BYX$
zhodné (\obr), takže je aj $|\uhol BAZ|=|\uhol BY'Z|$.
Z~tejto rovnosti ale podľa vety o~obvodových uhloch vyplýva,
že na kružnici~$k$ opísanej trojuholníku $ABZ$ leží nielen bod~$Y$, ale aj bod~$Y'$.
Priamka~$AB$ ako os tetivy~$YY'$ tak prechádza
stredom~$O$ kružnice~$k$, takže tetiva~$AB$ je jej priemerom.
Kružnica $k$ je teda vzhľadom na voľbu bodu~$Z$ nezávislá.
Stred~$M$ jej tetivy~$YZ$ preto nutne leží vo vnútornej oblasti kružnice~$k$.
Z~pravých uhlov $OMZ$ a~$OMY$ (\obr) vidíme, že (menšie) uhly $AMO$ a~$BMO$
sú ostré, takže bod~$M$ nutne leží v~prieniku vonkajších oblastí
Tálesových kružníc nad priemermi $AO$ a~$BO$. Ukážeme, že obe nutné
podmienky na polohu bodu~$M$ už spolu vymedzujú hľadanú množinu.
\inspinspblizko{a63.12}{a63.13}%

Nech $M$ je ľubovoľný bod vo vnútornej oblasti kružnice~$k$, pre ktorý
sú oba uhly $AMO$ a~$BMO$ ostré (\tj. bod~$M$ leží zvonka oboch kruhov s~priemermi
$AO$ a~$BO$). Potom zrejme kolmica na priamku~$OM$ v~bode~$M$ pretína
kružnicu~$k$ v~polrovine~$ABM$, takže
na jednej z~Tálesových polkružníc nad priemerom~$AB$ vytína tetivu,
ktorej krajné body môžeme označiť $Y$ a~$Z$ tak, aby
$A$, $B$, $Y$, $Z$ bolo poradím bodov na kružnici~$k$. Ak je $Y'$
obraz bodu~$Y$ na druhej polkružnici v~súmernosti podľa priemeru~$AB$,
tak pre priesečník~$X$ úsečiek $AB$ a~$Y'Z$ platí, že trojuholníky $XBY$ a~$XZA$
sú podobné podľa vety~$uu$. Dôkaz je hotový.

\zaver
Hľadanou množinou je vnútro kruhu s~priemerom~$AB$ a~stredom~$O$
bez oboch kruhov s~priemermi $AO$ a~$BO$ (\obr).
\insp{a63.14}%
}

{%%%%%   A-III-3
Dotyčných hrán je $7\cdot8=56$ zvislých a~rovnaký počet
vodorovných, celkom ich je $56\cdot2=112$.
Pri ľubovoľnom rozrezaní šachovnice vznikne 32~obdĺžničkov $2\times1$, preto
sa každé také rozrezanie týka práve $112-32=80$~hrán,
a~prispeje tak do celkového súčtu číslom~$80$. Výsledný súčet
je teda deliteľný číslom~$80$, takže jeho dekadický zápis končí nulou.
}

{%%%%%   A-III-4
Pre každé $k\in\{1,2,\dots,n\}$ označme $p_k$ počet divákov
v~$k$-tom rade. Podmienka úlohy pre dané $i$ a~$j$ znamená, že počet
známostí medzi divákmi z~$i$-teho a~$j$-teho radu je rovný~$jp_i$.
Prehodením úloh čísel $i$ a~$j$ zistíme, že ten istý
počet sa má rovnať číslu~$ip_j$. Musí teda platiť
$jp_i =ip_j$ čiže $p_i:p_j=i:j$. Prichádzame tak
k~záveru, že počty $p_k$ divákov v~jednotlivých radoch nutne spĺňajú
podmienku
$$
p_1:p_2:\dots:p_n=1:2:\dots:n.
$$

Ukážeme, že je to aj podmienka postačujúca, teda že pri počtoch
$p_k=kd$ pre vhodné celé~$d$ sa rozsadení diváci mohli poznať tak,
aby podmienka zo zadania úlohy bola splnená. Pre $d=1$ to tak
určite bude, keď sa budú navzájom poznať všetci diváci v~kine;
pre všeobecné~$d$ stačí, aby diváci boli rozdelení na $d$~skupín
navzájom sa poznajúcich divákov a~aby diváci z~každej takej
skupiny boli rozsadení do jednotlivých radov rovnako ako v~prípade~$d=1$.

Našou úlohou je preto zistiť, pre ktoré $n\ge4$ existuje celé
kladné číslo~$d$ vyhovujúce rovnici
$$
d+2d+\dots+nd=234,\quad\text{čiže}\quad dn(n+1)=468.
$$
Hľadáme teda všetky delitele čísla $468=2^2\cdot3^2\cdot13$, ktoré
sú tvaru $n(n+1)$. Keďže $22\cdot23>468$, musí platiť $n<22$,
teda $n\in\{4,6,9,12,13,18\}$. Vidíme, že vyhovuje jedine $n=12$
(s~príslušným $d=3$).

\odpoved
Opísaná situácia mohla nastať jedine pri rozsadení
divákov do 12~radov.
}

{%%%%%   A-III-5
Uvažované kružnice opísané trojuholníkom $APC$ a~$BQC$ označme postupne $l_A$, $l_B$
a~všimnime si napríklad druhú z~nich (\obr, uhly trojuholníka $ABC$ označujeme zvyčajným spôsobom).
\insp{a63.15}%

Vysvetlime, že naozaj platí, čo na obrázku vidíme. Predovšetkým bod~$Q$
zrejme leží v~polrovine $BCA$, lebo pre body~$X$ tamojšieho oblúka~$BC$
kružnice~$l_B$ sa uhol $AXB$ zväčšuje z~(ostrého) uhla~$\gamma$
na (tupý) uhol $180\st-\b/2$, takže nadobúda niekde vnútri oblúka hodnotu~$90\st$.
Keďže úsekový uhol prislúchajúci tomuto oblúku~$BC$ kružnice~$l_B$
je rovný~$\b/2$, je $|\uhol BQC|=180\st-\b/2$. Odtiaľ
$|\uhol AQB|+|\uhol BQC|=270\st-\b/2>180\st$.
Preto bod~$Q$ leží v~polrovine $ACB$, čiže aj vnútri trojuholníka $ABC$,
konvexný uhol $AQC$ má teda veľkosť $90\st+\b/2$.
Ak označíme~$I$ stred kružnice vpísanej trojuholníku~$ABC$,
bude mať uhol~$AIC$ takú istú veľkosť:
$|\uhol AIC|=180\st-\a/2-\gamma/2=90\st+\b/2$. To však znamená,
že bod~$Q$ leží na rovnakom oblúku~$AC$ kružnice $k_B$ opísanej trojuholníku $ACI$
ako bod~$I$, takže priamka~$AQ$ je chordálou
kružníc $k$ a~$k_B$.

Druhá uvažovaná priamka~$BP$ je analogicky
chordálou kružníc $k$ a~$k_A$ (ktorá je opísaná trojuholníku $BCI$). Priesečník oboch priamok
$AQ$ a~$BP$ má teda rovnakú mocnosť ku kružniciam $k_A$ a~$k_B$,
preto leží na ich chordále, ktorou je
priamka~$CI$, teda os uhla~$ACB$. Dôkaz je hotový.

\poznamka
Ešte jedným spôsobom vysvetlíme, prečo bod~$Q$ leží v polrovine~$BCA$.
Priesečník~$Q$ kružníc $k$ a~$l_B$ zrejme musí ležať v~polrovine $ABC$
v~uhle ohraničenom dotyčnicami oboch kružníc v~bode~$B$. Pritom
vrchol~$C$ ostrouhlého trojuholníka~$ABC$ aj stred~$S_B$ kružnice~$l_B$ zrejme
ležia zvonka kruhu ohraničeného kružnicou~$k$. V~trojuholníku $BS_BC$ leží síce časť
kružnice~$k$, ale s~výnimkou bodov $B$ a~$C$ tam určite neleží žiadny iný
bod kružnice~$l_B$. Z~toho je zrejmé, že bod~$Q$ musí ležať v~polrovine~$BCA$.
}

{%%%%%   A-III-6
Jednoduchým dosadením zistíme, že dokazovaná nerovnosť prejde
na rovnosť, ak platí aspoň jedna z~rovností $a=0$, $b=0$ alebo
$a=b$. Z~ďalšieho postupu vyplynie, že sú to jediné prípady rovnosti.

Vzhľadom na symetriu budeme predpokladať, že platí $a>b>0$,
a~postupnými ekvivalentnými úpravami dokážeme ostrú nerovnosť
zo zadania, ktorú rovno zapíšeme zbavenú zlomkov:
$$
a\sqrt{a^2+1}\sqrt{ab+1}+b\sqrt{b^2+1}\sqrt{ab+1}>
(a+b)\sqrt{a^2+1}\sqrt{b^2+1}.
$$
Teraz roznásobíme pravú stranu zastúpeným súčtom $a+b$
a~po preskupení výrazov vyjmeme spoločné činitele:
$$
a\sqrt{a^2+1}\bigl(\sqrt{ab+1}-\sqrt{b^2+1}\bigr)>
b\sqrt{b^2+1}\bigl(\sqrt{a^2+1}-\sqrt{ab+1}\bigr)
$$
Na ľavej aj pravej strane vystupujú rozdiely odmocnín, rozšírime ich
súčtami tých istých odmocnín na zlomky:
$$
a\sqrt{a^2+1}\cdot\frac{b(a-b)}{\sqrt{ab+1}+\sqrt{b^2+1}}>
b\sqrt{b^2+1}\cdot\frac{a(a-b)}{\sqrt{a^2+1}+\sqrt{ab+1}}.
$$
Obe strany teraz môžeme vydeliť kladným číslom $ab(a-b)$; po
následnom odstránení zlomkov dostaneme nerovnosť
$$
\sqrt{a^2+1}\bigl(\sqrt{a^2+1}+\sqrt{ab+1}\bigr)>
\sqrt{b^2+1}\bigl(\sqrt{b^2+1}+\sqrt{ab+1}\bigr),
$$
ktorej platnosť už vyplýva z~porovnania odmocnín na rovnakých miestach
oboch strán (vďaka predpokladu $a>b$ totiž platí
$\sqrt{a^2+1}>\sqrt{b^2+1}$).

\ineriesenie
Tentoraz z~nášho postupu vylúčime iba prípady $a=0$ a~$b=0$,
v~ktorých je však dokazovaná nerovnosť triviálna.
Budeme teda predpokladať, že čísla $a$ a~$b$ sú kladné
a~zapíšeme pre ne Cauchyho nerovnosť
$$
\biggl(\frac{a}{u}+\frac{b}{v}\biggr)(au+bv)\geqq(a+b)^2
$$
s~kladnými koeficientmi $u=\sqrt{b^2+1}$ a~$v=\sqrt{a^2+1}$:
$$
\biggl(\frac{a}{\sqrt{b^2+1}}+\frac{b}{\sqrt{a^2+1}}\biggr)
\Bigl(a\sqrt{b^2+1}+b\sqrt{a^2+1}\Bigr)\geqq(a+b)^2.
\tag1
$$
Druhý činiteľ z~ľavej strany \thetag1 odhadneme {\it zhora\/} podľa
inej Cauchyho nerovnosti takto:
$$
\align
a\sqrt{b^2+1}&+b\sqrt{a^2+1}=\sqrt{a}\sqrt{ab^2+a}+
\sqrt{b}\sqrt{a^2b+b}\leqq\\
\leqq&\sqrt{a+b}\sqrt{ab^2+a+a^2b+b}=\sqrt{a+b}
\sqrt{(a+b)(ab+1)}=(a+b)\sqrt{ab+1}.
\endalign
$$
Pre prvý činiteľ z~ľavej strany \thetag1 tak dostávame
$$
\frac{a}{\sqrt{b^2+1}}+\frac{b}{\sqrt{a^2+1}}\geqq
\frac{(a+b)^2}{a\sqrt{b^2+1}+b\sqrt{a^2+1}}\geqq
\frac{a+b}{\sqrt{ab+1}},
$$
čo sme mali dokázať. Keďže v~prvej vypísanej
Cauchyho nerovnosti nastáva rovnosť práve vtedy, keď platí $u=v$,
čiže $\sqrt{b^2+1}=\sqrt{a^2+1}$, \tj. $a=b$, je posledná
rovnosť tretím a~posledným prípadom (okrem spomenutých
prípadov $a=0$ a~$b=0$), kedy v~dokazovanej nerovnosti nastáva
rovnosť.

\ineriesenie
Vylúčime z~úvah zrejmé prípady $a=0$, $b=0$, $a=b$
a~dokazovanú ostrú nerovnosť ekvivalentne upravíme na tvar
$$
\frac{a}{a+b}\cdot\frac{1}{\sqrt{b^2+1}}+
\frac{b}{a+b}\cdot\frac{1}{\sqrt{a^2+1}}
>\frac{1}{\sqrt{ab+1}}.
$$
Ľavá strana je zrejme ľavou stranou Jensenovej nerovnosti
$$
pf(\a)+qf(\b)>f(p\a+q\b) \tag2
$$
s~kladnými koeficientmi $p=a/(a+b)$ a~$q=b/(a+b)$ (ktoré, ako vieme,
musia spĺňať podmienku $p+q=1$) pre funkciu $f(x)=1/\sqrt{x}$
v~bodoch $\a=b^2+1$ a~$\b=a^2+1$. Keďže funkcia~$f$ je na obore
kladných čísel rýdzo konvexná\footnote{Graf funkcie $y=x^{\m\frac12}$
je dobre známy.}
a~body $\a$, $\b$ sú rôzne vďaka predpokladu $a\ne b$,
Jensenova nerovnosť~\thetag2 platí.
Ostáva sa teda presvedčiť, že aj jej pravá strana sa rovná pravej strane
upravenej nerovnosti z~úvodu riešenia. To je jednoduché:
$$
\align
f(p\a+q\b)&=f\Bigl(\frac{a}{a+b}(b^2+1)+\frac{b}{a+b}(a^2+1)\Bigr)=\\
&=f\biggl(\frac{a+ab^2+b+a^2b}{a+b}\biggr)=f(ab+1)
=\frac{1}{\sqrt{ab+1}}.
\endalign
$$
}

{%%%%%   B-S-1
Os reálnych čísel rozdelíme podľa nulových bodov, teda podľa čísel,
pre ktoré sú hodnoty výrazov s~absolútnou
hodnotou v~danej rovnici rovné nule. Výrazu $|x+1|$ zodpovedá $x=\m1$ a~výrazu
$|2^{x}-1|$ zodpovedá $x = 0$. Dostávame tak nasledujúce tri možnosti:

1) V~prípade $ x \le -1 $ vyjde $|x +1|= \m(x +1)$
a~$|2^{x} -1|= \m(2^{x} -1) = 1-2^x$. Rovnica zo zadania má potom tvar
$$
2^{- (x +1)} -2^{x} = 1 +1-2^x, \quad\text{čiže}\quad 2^{-(x +1)}= 2
$$
a~má jediné riešenie $x=\m2$.
Spätným dosadením sa presvedčíme, že je to naozaj riešenie danej
rovnice.

2) V~prípade $\m1 < x \le 0 $ vyjde $|x +1|= x +1$
a~$|2^{x} -1|= \m(2^{x}-1) = 1-2^x$. Rovnica zo zadania má potom tvar
$$
2^{x +1} -2^{x} =1 +1-2^x, \quad\text{čiže}\quad 2^{x +1}= 2
$$
a~má jediné riešenie $x=0$.
Spätným dosadením sa presvedčíme, že je to naozaj riešenie danej
rovnice.

3) Nakoniec pre $0 < x$ vyjde $|x +1|= x +1$
a~$|2^{x} -1|= 2^{x} -1$ a~po dosadení do danej rovnice dostaneme
$$
2^{x +1} -2^{x}= 1 +2^x - 1, \quad\text{čiže}\quad 2^{x +1} = 2\cdot2^x,
$$
čo je identita, ktorá platí pre ľubovoľné reálne číslo~$x$. Všetky
$x>0$ sú teda riešením danej rovnice.

\odpoved
Riešeniami danej rovnice sú $x = \m2$ a~ľubovoľné $x\ge0$.

\poznamka
Namiesto kontroly riešenia dosadením do danej rovnice stačí overiť, že nájdené riešenie
padne do vyšetrovaného intervalu.

\ineriesenie
Rozoberieme dva prípady:

Ak $x +1 \ge 0$, tak $|x +1|= x+1$ a~danú rovnicu možno zjednodušiť
na rovnicu
$$
2^{x +1} -2^x = 1 + |2^x - 1|,\quad\text{čiže}\quad 2^x - 1 = |2^x - 1|.
$$
Tá je splnená práve vtedy, keď
$2^x - 1 \ge 0$, čo platí práve vtedy, keď $x \ge 0$. V~tomto prípade sú riešeniami
všetky $x$ také, že $x +1\ge 0$ a~zároveň $x\ge 0$, čiže všetky
nezáporné čísla~$x$.

Ak naopak $x +1 < 0$, je tiež $2^x - 1 < 0$ (lebo $x<0$), takže daná rovnica dostane tvar
$$
2^{-(x +1)} -2^{x} = 1 - (2^x - 1),\quad\text{čiže}\quad 2^{-(x +1)} = 2.
$$
Táto rovnica má jediné riešenie $x = \m2$ a~to podmienku $x+1<0$ spĺňa.

Riešením danej rovnice je množina $\langle 0,\infty)\cup\{\m2\}$.

\nobreak\medskip\petit\noindent
Za úplné riešenie dajte 6~bodov.
Ak riešiteľ postupuje pomocou nulových bodov,
dajte za ich určenie 1~bod. Za vyriešenie prípadu $ \m1 \le
x $ dajte 2~body, správne vyriešený prípad $ x > 0 $ ohodnoťte 2~bodmi
a~za posledný prípad $ \m1 < x \le 0 $ dajte 1~bod. Ak riešenie rozoberá iba dve
možnosti ako v~druhom riešení, dajte za každú časť 3~body. Za
uhádnutie všetkých riešení dajte 1~bod, za uhádnutie len niektorých (i~keď nekonečne veľa)
body nedávajte.

\endpetit
\bigbreak
}

{%%%%%   B-S-2
Na~vyriešenie časti a) stačí uviesť príklad množiny, ktorá neobsahuje žiadne celé číslo,
pričom súčet ľubovoľných dvoch jej prvkov je celé číslo.
Množina $\mm M = \{1/2, 3/2,{\dots}, 4027/2\}$ je
jedným z~príkladov takej 2014-prvkovej množiny.

Označme $a$, $b$, $c$ ľubovoľné tri čísla danej množiny $\mm M$.
Ukážeme, že dvojnásobok čísla $a$ je celé číslo. Čísla $a + b$
aj $b + c$ sú podľa zadania celé, preto aj ich rozdiel $a-c$ je celé
číslo. Zo zadania vieme, že aj číslo $a + c$ je celé, preto
aj~súčet $(a + c)+(a-c)=2a$ je celé číslo. Dvojnásobok každého
čísla v~množine~$\mm M$ teda musí byť celé číslo, a~preto množina~$\mm M$
nemôže obsahovať žiadne iracionálne číslo.

\ineriesenie
Časť a) vyriešime rovnako ako v~predchádzajúcom riešení.

Pripusťme, že sa v~množine~$\mm M$ nájde iracionálne číslo~$a$, a~označme $\alpha$, $0<\alpha<1$,
jeho desatinnú časť. Všetky ostatné čísla z~množiny~$\mm M$ (a~takých je tam 2013)
musia mať desatinnú časť $1-\alpha$, lebo súčet každého z~nich s~číslom~$a$ dáva celé číslo.
Ale súčet každých dvoch čísel s~kladnou desatinnou časťou $1-\alpha$ tiež musí dať celé číslo.
To je možné jedine v~prípade, že $1-\alpha=1/2$ čiže $\alpha=1/2$, čo je však v~spore s~tým,
že číslo~$a$ je iracionálne.
Množina~$\mm M$, ktorá by obsahovala iracionálne číslo, teda neexistuje.



\nobreak\medskip\petit\noindent
Za úplné riešenie dajte 6 bodov.
Za nájdenie vyhovujúcej množiny pre časť~a) dajte 2~body. Za
zistenie, že aj rozdiel dvoch čísel z~množiny~$\mm M$ je celé číslo,
dajte 2~body. Za korektné dokončenie časti~b) dajte zvyšné 2~body.
V~prípade, že študent postupuje podľa druhého riešenia, v~časti~b)
získava 2~body za zistenie, že všetky ostatné čísla majú
rovnakú desatinnú časť, a~zvyšné 2~body za zistenie, že táto
necelá časť môže byť jedine $1/2$.
Štyri body dajte aj za priamy dôkaz silnejšieho tvrdenia, že každé číslo z~$\mm M$
je buď celé, alebo má desatinnú časť rovnú~$1/2$.
\endpetit
\bigbreak
}

{%%%%%   B-S-3
V~danom trojuholníku $ABC$ označme $X$, $Y$, $Z$ body dotyku vpísanej kružnice s~jeho
stranami a~$x=|AY|=|AZ|$, $y=|BX|=|BZ|$, $z=|CX|=|CY|$
zhodné úseky dotyčníc k~vpísanej kružnici z~jednotlivých vrcholov (\obr).
\insp{b63.6}%
Ak označíme zvyčajným spôsobom $a$, $b$, $c$ dĺžky jednotlivých strán, platí
$$
a=y+z,\quad b=z+x,\quad c=x+y.
$$
Sčítaním týchto troch rovníc dostaneme (pomocou~$s$ ako zvyčajne označujeme
polovičný obvod trojuholníka)
$$
2s=a+b+c=2x+2y+2z,
$$
takže nám vyjde
$$
x+y+z=s,\quad x=s-a,\quad y=s-b,\quad z=s-c. \tag1
$$

Pozrime sa teraz na pripísanú kružnicu trojuholníku $ABC$, ktorá sa dotýka jeho
strany~$BC$ v~bode~$P$ a~polpriamok $AB$ a~$AC$ v~bodoch $R$ a~$Q$ (\obr).
Zo zhodnosti úsekov príslušných dotyčníc k~tejto kružnici máme
$$
|AR|=|AQ|,\quad |BR|=|BP|,\quad |CP|=|CQ|,
$$
odkiaľ vychádza
$$
\align
2|AR|=|AR|+|AQ|=&|AB|+|BR|+|AC|+|CQ|=\\
=&|AB|+|BP|+|AC|+|CP|=a+b+c=2s,
\endalign
$$
čiže $|AR|=|AQ|=s$. Z~tejto rovnosti ale vyplýva, že
$|BP|=|BR|=s-c$, čo je podľa~\thetag1 zároveň dĺžka~$z$ úsečky~$CX$, teda $|BP|=|CX|$.
To znamená, že body $P$ a~$X$ sú súmerne združené podľa stredu úsečky~$BC$.
\insp{b63.7}%

Analogicky by sme odvodili rovnosti $|BK|=s$ a~$|CL|=s$
pre body dotyku $K$ a~$L$ kružníc pripísaných stranám $CA$ a~$AB$ (\obrr1)
trojuholníka $ABC$ s~priamkou~$a$. Z~týchto posledných rovností však vidíme, že
$|BL|=s-a=|CK|$, teda aj body $K$ a~$L$ sú súmerne združené podľa stredu úsečky~$BC$.

Body $K$ a~$L$ sú známe (z~troch daných bodov na priamke sú to tie dva krajné),
poznáme teda aj stred~$S$ strany~$BC$ (je to stred úsečky~$KL$)
a~bod~$X$ nájdeme ako obraz tretieho daného bodu~$P$
v~stredovej súmernosti podľa stredu~$S$.

\nobreak\medskip\petit\noindent
Za úplné riešenie dajte 6~bodov, z~toho 3~body za určenie stredu strany~$BC$
pomocou súmerne združených bodov dotyku kružníc pripísaných ostatným dvom stranám.
Tri body tiež dajte za poznatok, že aj body dotyku pripísanej a~vpísanej
kružnice na jednej strane trojuholníka sú súmerne združené podľa
stredu strany. Len za odvodenie všetkých vzdialeností bodov dotyku od
vrcholov $B$ a~$C$ bez nájdenia konštrukcie dajte 4~body.
\endpetit
}

{%%%%%   B-II-1
Odčítaním druhej rovnice od prvej a~tretej od druhej dostaneme dve rovnice
$$
\align
(x - y) (x + y - 6) = & 0,\\
(y - z) (y + z~- 6) = & 0,
\endalign
$$
ktoré spolu s~ľubovoľnou z~troch daných rovníc tvoria sústavu s~danou sústavou
ekvivalentnú. Pre splnenie získaných dvoch rovníc pritom máme štyri možnosti:

Ak $x = y = z$, vyjde dosadením do ktorejkoľvek zo zadaných rovníc
$y^2 +12 y - 85 = 0$ a~odtiaľ $y = 5$ alebo $y = -17$.

Ak $x = y$, $ z~= 6 - y$, dostaneme z~prvej zadanej rovnice $y^2 +36 = 85$,
a~teda $y = 7$ alebo $y = -7$.

Ak $x = 6 - y$, $z = y$, dostaneme z~poslednej zadanej rovnice opäť
$y^2 +36 = 85$, a~teda $y = 7$ alebo $y = -7$.

Ak $x = z~= 6 - y$, dostaneme z~druhej zadanej rovnice $y^2 +6 (12 - 2y) = 85$
čiže $y^2 -12y-13=0$ a~odtiaľ $y = -1$ alebo $y = 13$.

\odpoved
Sústava rovníc má osem riešení, a~to $(5, 5, 5)$, $(-17, -17, -17)$,
$(7, 7, -1)$, $({-7}, {-7}, 13)$, $(-1, 7, 7)$, $(13, -7, -7)$, $(7, -1, 7)$,
$(-7, 13, -7)$.


\nobreak\medskip\petit\noindent
Za úplné riešenie dajte 6 bodov.
Za uhádnutie riešenia $x = y = z~\in \{5, -17 \}$ dajte 2~body, alebo len
1~bod, keď riešiteľ nájde iba jedno z~týchto riešení. Rozobranie ostatných dvoch
možností odmeňte dokopy 2~bodmi (je to rovnaká kvadratická rovnica)
a~poslednú možnosť $x = z= 6 - y$ ohodnoťte tiež dvoma bodmi. Ak študent
rozoberie všetky štyri prípady, ale zabudne na riešenia, ktoré vzniknú
zámenou poradia, dajte iba 5 bodov.

\endpetit
\bigbreak
}

{%%%%%   B-II-2
Označme prvočísla napísané na tabuli ako $p_1 < p_2 < \cdots < p_k$.
Z~týchto čísel je vďaka predpokladu $k \ge 3$ možné vytvoriť $k$~rôznych čísel
$$
p_1 + p_2 - 7 < p_1 + p_3 - 7 < \cdots < p_1 + p_k - 7 < p_ {k-1} + p_k - 7, \tag1
$$
ktoré všetky musia byť medzi číslami napísanými na tabuli. Preto sa postupne rovnajú
prvočíslam $p_1 < p_2 < \cdots < p_k$. Presnejšie, najmenšie z~nich sa rovná
$p_1$, \tj. $p_1 + p_2 -7 = p_1$, a~teda $p_2 = 7$. Následne pre druhé najmenšie
číslo v~nerovniciach~\thetag1 platí $p_1 + p_3 -7 = p_2 = 7$, a~teda
$p_1 + p_3 = 14$. Pre prvočíslo $p_1 < p_2 = 7$ máme iba tri možnosti
$p_1 \in \{2, 3, 5 \}$, ktorým zodpovedajú hodnoty $p_3 = 14 - p_1 \in \{12, 11, 9 \}$,
z~ktorých iba $p_3 = 11$ je prvočíslo, takže $p_1=3$. Predpokladajme, že na tabuli je
napísané ešte ďalšie prvočíslo~$p_4$. Potom tretie najmenšie číslo
v~nerovniciach~\thetag1 musí byť $11 = p_3 = p_1 + p_4 - 7$, z~čoho vzhľadom
na rovnosť $p_1 = 3$ vyplýva $p_4 = 15$, čo prvočíslo nie je.

\odpoved
Na tabuli mohli byť iba tri prvočísla $3$, $7$ a~$11$.

Dodajme, že záver $k=3$ sa dá inak zdôvodniť poznámkou, že rovnakú $k$-ticu
prvočísel v~skupine~\thetag1 musíme dostať aj vtedy, keď zameníme posledné z~čísel za
$p_2+p_k-7$; musí teda platiť $p_{k-1}=p_2$ čiže $k=3$.

\ineriesenie
Označme prvočísla napísané na tabuli ako $p_1 < p_2 < \cdots < p_k$. Ak si
Janko vyberie dve najmenšie a~dve najväčšie prvočísla, ich súčet
zmenšený o~$7$ je na tabuli, a~preto $p_1 + p_2 -7 \ge p_1$ a~$p_k + p_ {k-1} -7 \le
p_k$, z~čoho vyplýva $7 \le p_2 < p_ {k- 1} \le 7$. Aby sme nedostali spor, musí byť
$k \le 3$. Podľa zadania sú na tabuli aspoň tri prvočísla, teda na tabuli
sú presne tri prvočísla $p_1 < p_2 < p_3$. Keďže
$p_1 + p_2 < p_1 + p_3 < p_2 + p_3$, musí platiť
$$
p_1 + p_2 - 7 = p_1, \qquad p_1 + p_3 - 7 = p_2, \qquad p_2 + p_3 - 7 = p_3.
$$
Z~prvej (a~poslednej) rovnice vychádza $p_2 = 7$ a~z~prostrednej $p_1 + p_3 = 14$,
pričom $p_1 < 7 = p_2$. Vyskúšaním všetkých možností $p_1 \in \{2, 3, 5 \}$ dostaneme
jediné riešenie $p_1 = 3$ a~$p_3 = 11$.


\nobreak\medskip\petit\noindent
Za úplné riešenie dajte 6 bodov,
len za vyriešenie situácie s~tromi prvočíslami dajte 3 body.
Ďalšie 3 body dajte, ak študent preukáže, že na tabuli mohli byť
nanajvýš 3 prvočísla. Za uhádnutie riešenia $\{3, 7, 11 \}$ dajte 2 body.

\endpetit
\bigbreak
}

{%%%%%   B-II-3
Označme päty výšok z~vrcholov $A$ a~$C$ na strany daného trojuholníka postupne
$K$ a~$L$ (\obr). Z~Euklidovej vety o~odvesne v~pravouhlom trojuholníku $BCD$
vieme, že $|BD|^2 =|BK|\cdot|BC|$. Podobne pre pravouhlý trojuholník
$ABE$ máme $|BE|^2 =|BL|\cdot|BA|$. Trojuholníky $ACK$ a~$ACL$ sú
pravouhlé s~preponou $AC$, a~preto body $K$ a~$L$ ležia na kružnici
s~priemerom~$AC$. Mocnosť bodu~$B$ k~tejto kružnici je $|BK|\cdot|BC|={|BL|\cdot|BA|}$,
a~tak spojením s~dôsledkami Euklidových viet dostávame
$|BD|^2 =|BK|\cdot|BC|={|BL|\cdot|BA|}=|BE|^2$, a~teda $|BE|=|BD|$.
\insp{b63.8}%

\ineriesenie
Pri označení piat výšok ako v~prvom riešení (\obrr1)
z~Pytagorových viet v~trojuholníkoch $BAK$, $BDK$, $CAK$ a~$CDK$ dostaneme
$$
\align
| BD|^2 -|BA|^2 = & (|DK|^2 +|BK|^2) - (|AK|^2 +|BK|^2) =|DK|^2 -|AK|^2 =\\
= & (|DK|^2 +|CK|^2) - (|AK|^2 +|CK|^2) =|CD|^2 -|CA|^2.
\endalign
$$
Navyše z~pravouhlého trojuholníka $BCD$ vieme, že
$|CD|^2 =|BC|^2 -|BD|^2$. Dosadením do predchádzajúcej rovnosti po úprave
dostaneme
$$
\align
|BD|^2 = &|BA|^2 +|CD|^2 -|CA|^2 =|BA|^2 + (|BC|^2 -|BD|^2) -|CA|^2, \\
2|BD|^2 = &|BA|^2 +|BC|^2 -|CA|^2, \\
|BD| = & \sqrt {\frac {|BA|^2 +|BC|^2 -|CA|^2} {2}}.
\endalign
$$
Veľkosť $|BE|$ dostaneme zo symetrie zámenou bodov $C \leftrightarrow A$
a~$D \leftrightarrow E$, takže
$$
|BE|= \sqrt {\frac{|BC|^2 +|BA|^2 -|AC|^2} {2}} =|BD|.
$$

\ineriesenie
Označme $H$ priesečník výšok trojuholníka $ABC$. Otočme
trojuholník $BCD$ v~priestore okolo priamky $BC$ do polohy $BCD'$
tak, aby rovina $BHD'$ bola kolmá na rovinu $ABC$, teda tak,
aby kolmý priemet priamky~$BD'$ do roviny $ABC$ splýval
s~výškou z~vrcholu~$B$ v~trojuholníku $ABC$ (\obr). Keďže $DA$ je výškou
trojuholníka $ABC$, je rovina~$AHD'$ kolmá na rovinu $ABC$, takže
priamka~$HD'$ (priesečnica rovín $BHD'$ a~$AHD'$) je kolmá na rovinu $ABC$.
\insp{b63.9}%

Priamka~$BD'$ je kolmá na priamku~$AC$ aj na priamku~$CD'$ (uhol $BD'C$ sa zhoduje
s~pravým uhlom $BDC$ nad priemerom~$BC$)~-- je teda kolmá na rovinu $ACD'$. Potom je
však pravý aj uhol $AD'B$. Bod~$D'$ leží v~rovine $CHD'$
kolmej na rovinu $ABC$, pričom priamka $CH=CE$ je výškou
% z~vrcholu~$C$
trojuholníka $ABC$. Presne tieto vlastnosti má aj bod~$E'$ trojuholníka $BAE'$,
ktorý vznikne otočením pravouhlého trojuholníka $BAE$ okolo priamky~$BA$ tak, aby
rovina~$BHE'$ bola kolmá na rovinu $ABC$. Body $D'$ a~$E'$ teda splývajú,
a~preto $|BD|=|BD'|=|BE'|=|BE|$.

\ineriesenie
Označme $H$ priesečník výšok trojuholníka $ABC$ a~$K$, $M$ a~$L$ päty výšok postupne
z~vrcholov $A$, $B$ a~$C$. Rovnako ako v~druhom
riešení opakovaným využitím Pytagorovej vety dostávame
$$
\align
|BD|&^2 -|BE|^2 =\\
= & (|BD|^2 -|BH|^2) + (|BH|^2 -|BE|^2 ) =\\
= & \bigl((|KD|^2 +|BK|^2) - (|BK|^2 +|HK|^2) \bigr) +
\bigl((|HL|^2 +|BL|^2 ) - (|BL|^2 +|LE|^2 ) \bigr) =\\
= & (|KD|^2 -|HK|^2) +(|HL|^2 -|LE|^2) =\\
= & \bigl((|KD|^2 +|CK|^2 ) - (|CK|^2 +|HK|^2 ) \bigr) +
\bigl((|HL|^2 +|AL|^2 ) - (|AL|^2 +|LE|^2 ) \bigr) =\\
= & (|CD|^2 -|CH|^2 ) + (|AH|^2 -|AE|^2) =\\
= &|CD|^2 + |AH|^2 -|CH|^2 -|AE|^2 =\\
= &|CD|^2 + (|AM|^2 +|MH|^2) -(|MH|^2 +|CM|^2) -|AE|^2 =\\
= &|CD|^2 + |AM|^2 -|CM|^2 -|AE|^2 =\\ %\noalign{\goodbreak}
= &|CD|^2 + (|AB|^2 -|BM|^2 ) - (|BC|^2 -|BM|^2) -|AE|^2 =\\
= & (|BC|^2 -|BD|^2 ) + |AB|^2 -|BC|^2 - (|AB|^2 -|BE|^2) =\\
= & {-|BD|^2 +|BE|^2},
\endalign
$$
z~toho vyplýva $2|BD|^2 = 2|BE|^2$, a~teda $|BE|=|BD|$.


\nobreak\medskip\petit\noindent
Za úplné riešenie dajte 6~bodov.
Ak študent postupuje podľa prvého riešenia, tak za využitie Euklidovej
vety o~odvesne dajte 3~body a~za využitie mocnosti bodu ku kružnici nad
priemerom~$BC$ tiež 3~body.

\endpetit
\bigbreak
}

{%%%%%   B-II-4
Riadky a~stĺpce uvažovanej tabuľky očíslujme zhora nadol, resp. zľava doprava
číslami $1, 2, \dots, 8$.

Najskôr ukážeme, že súčet čísel v~každom riadku a~stĺpci je nanajvýš~$300$.
Uvedomme si, že zložením oboch súmerností podľa uhlopriečok vznikne
stredová súmernosť podľa stredu danej tabuľky. To teda znamená, že
pre každé $i \in\{ 1, 2, 3, 4\}$ budú
riadky $i$, $9-i$ a~stĺpca $i$, $9-i$ obsahovať štyri
zhodné osmice čísel. Súčet dvoch diagonálnych čísel
z~tejto osmice je nanajvýš $200:2 = 100$, pretože
každé z~týchto čísel je v~súčte všetkých 16~(nezáporných) čísel
na oboch uhlopriečkach započítané dvakrát. Súčet šiestich
čísel z~uvažovanej osmice, ktoré neležia na žiadnej z~uhlopriečok, je nanajvýš $800:4 = 200$,
pretože v~súčte $1\,000-200 = 800$ všetkých 48~nediagonálnych
(nezáporných) čísel je každé číslo započítané štyrikrát. Preto
súčet všetkých ôsmich čísel v~žiadnom riadku ani stĺpci neprevyšuje
$100 +200 = 300$.

Ostáva nájsť príklad tabuľky, pre ktorú záver s~číslom $299$
neplatí. Ak zapíšeme číslo $50$ do štyroch rohových políčok, číslo $100$
do ôsmich políčok krajných riadkov a~stĺpcov, ktoré susedia s~rohovými
políčkami, a~nuly do ostatných políčok, dostaneme vyhovujúcu tabuľku,
ktorá má v~krajných riadkoch a~stĺpcoch súčet $300$, teda viac ako posudzovaných $299$.
(Iný z~mnohých kontrapríkladov je opísaný v~závere druhého riešenia.)


\ineriesenie
Označme niektoré čísla v~tabuľke podľa schémy vľavo.
Týmito číslami už vieme vďaka symetrii podľa uhlopriečok vyplniť
celú tabuľku (schému vpravo):
$$
\centerline{\offinterlineskip \everycr{\noalign{\hrule}}\let\\=\cr
\dimen1=1.6em
\vbox{\halign{\strut\vrule#&&\hbox to\dimen1{\hss#\unskip\hss}\vrule\cr
&$ a_1 $ & $ b_1 $ & $ b_2 $ & $ b_3 $ & $ c_3 $ & $ c_2 $ & $ c_1 $ & $ d_1 $ \\
& & $ a_2 $ & $ b_4 $ & $ b_5 $ & $ c_5 $ & $ c_4 $ & $ d_2 $ & \\
& && $ a_3 $ & $ b_6 $ & $ c_6 $ & $ d_3 $ && \\
& &&& $ a_4 $ & $ d_4 $ &&& \\
&&&&&&&& \\
&&&&&&&& \\
&&&&&&&& \\
&&&&&&&& \\
}}\hfil
\vbox{\halign{\strut\vrule#&&\hbox to\dimen1{\hss#\unskip\hss}\vrule\cr
&$ a_1 $ & $ b_1 $ & $ b_2 $ & $ b_3 $ & $ c_3 $ & $ c_2 $ & $ c_1 $ & $ d_1 $ \\
&$ b_1 $ & $ a_2 $ & $ b_4 $ & $ b_5 $ & $ c_5 $ & $ c_4 $ & $ d_2 $ & $ c_1 $ \\
&$ b_2 $ & $ b_4 $ & $ a_3 $ & $ b_6 $ & $ c_6 $ & $ d_3 $ & $ c_4 $ & $ c_2 $ \\
&$ b_3 $ & $ b_5 $ & $ b_6 $ & $ a_4 $ & $ d_4 $ & $ c_6 $ & $ c_5 $ & $ c_3 $ \\
&$ c_3 $ & $ c_5 $ & $ c_6 $ & $ d_4 $ & $ a_4 $ & $ b_6 $ & $ b_5 $ & $ b_3 $ \\
&$ c_2 $ & $ c_4 $ & $ d_3 $ & $ c_6 $ & $ b_6 $ & $ a_3 $ & $ b_4 $ & $ b_2 $ \\
&$ c_1 $ & $ d_2 $ & $ c_4 $ & $ c_5 $ & $ b_5 $ & $ b_4 $ & $ a_2 $ & $ b_1 $ \\
&$ d_1 $ & $ c_1 $ & $ c_2 $ & $ c_3 $ & $ b_3 $ & $ b_2 $ & $ b_1 $ & $ a_1 $ \\
}}}
$$
Označme súčty čísel v~týchto skupinách zodpovedajúcim veľkým
písmenom, \tj. $A = a_1 + \cdots + a_4$, $B = b_1 + \cdots + b_6$,
$C = c_1 + \cdots + c_6$, $D = d_1 + \cdots + d_4$. Zo zadania tak máme
$2 (A~+ D) = 200$ a~$2 (A~+ D) +4 (B + C) = 1\,000$, z~čoho úpravou dostaneme $A + D = 100$
a~$B + C = 200$.

Vzhľadom na symetriu čísel v~tabuľke teraz stačí ukázať, že tvrdenie platí pre
každý z~prvých štyroch riadkov. Všetky čísla sú nezáporné
a~v~jednom riadku sa nevyskytujú dve rovnako označené čísla, preto ich
súčet je nanajvýš
$$
a_1 + \cdots + a_4 + b_1 + \cdots + b_6 + c_1 + \cdots + c_6 + d_1 + \cdots + d_4 = A~+ B + C + D = 300.
$$

Vyhovujúcu tabuľku, pre ktorú záver s~číslom $299$ neplatí, dostaneme
napríklad pre hodnoty $a_1 = 100$, $b_1 = 200$ a~ostatné čísla nulové.


\nobreak\medskip\petit\noindent
Za úplné riešenie dajte 6 bodov.
Nájdenie kontrapríkladu pre číslo $299$ odmeňte 2 bodmi.

Ak študent postupuje podľa prvého riešenia, dajte 2 body za ohraničenie súčtu
dvoch diagonálnych čísel v~jednom riadku alebo stĺpci číslom $100$. Zdôvodnenie,
že každé z~nediagonálnych čísel sa
v~tabuľke vyskytuje práve štyrikrát a~že ich súčet v~jednej osmici je
nanajvýš $200$, ohodnoťte ďalšími 2~bodmi.

Ak postupuje študent podľa druhého riešenia, dajte 1~bod za rozdelenie čísel
podľa súmernosti na oblasti $A$, $B$, $C$ a~$D$ a~za výpočet
$A + D = 100$ a~$B + C = 200$ dajte ďalší bod. Zostávajúce dva body dajte za
podrobné zdôvodnenie, prečo nie je možné, aby sa v~jednom riadku alebo
stĺpci vyskytovali dve čísla z~jednej oblasti $B$, resp.~$C$.


\endpetit
\bigbreak}

{%%%%%   C-S-1
Pre daný výraz $V$ platí
$$
V~= a(b + d) + c(b + d ) = (a~+ c)(b + d ).
$$
Podobne môžeme upraviť aj obe dané podmienky:
$$
2(a + c) - 5(b + d ) = 4\qquad \text{a}\qquad 3(a + c) + 4(b + d ) = 6. \tag1
$$
Ak teda zvolíme substitúciu $m = a~+ c$ a~$n = b + d $, dostaneme riešením sústavy \thetag1
$m = 2$ a~$n = 0$.
Pre daný výraz potom platí $V = mn = 0$.

\zaver
Za daných podmienok nadobúda výraz~$V$ iba hodnotu $0$.

\ineriesenie
Podmienky úlohy si predstavíme ako sústavu
rovníc s~neznámymi $a$, $b$ a~parametrami $c$, $d$. Vyriešením tejto sústavy
(sčítacou alebo dosadzovacou metódou) vyjadríme
$a = 2 - c$, $b = \m d$ ($c, d\in\Bbb R$) a~po dosadení do výrazu~$V$ dostávame
$$
V~= (2 - c)(-d ) - dc + cd + d (2 - c) = 0.
$$

\nobreak\medskip\petit\noindent
Za úplné riešenie dajte 6~bodov. Pri postupe z~prvého riešenia dajte 2~body za rozklad výrazu~$V$ na súčin, 2~body za úpravu podmienok na sústavu \thetag1, 1~bod za jej
vyriešenie a~1~bod za výpočet hodnoty~$V$.

\endpetit
\bigbreak
}

{%%%%%   C-S-2
Pre uvažované súčiny $a$ a~$b$ určite platí $a\cdot b=1\cdot2\cdot\dots\cdot10
= 2^8\cdot3^4\cdot5^2\cdot7$. Aspoň
jedno z~čísel $a$, $b$ je preto deliteľné $2^4$, aspoň jedno deliteľné
$3^2$, aspoň jedno deliteľné $5$ a~práve jedno deliteľné~$7$. Pre najmenší
spoločný násobok $n$ čísel $a$, $b$ preto platí $n\ge2^4\cdot3^2\cdot5\cdot7 = 5\,040$,
pritom rovnosť tu nastane práve vtedy, keď ani jedno z~čísel
$a$, $b$ nebude deliteľné žiadnym z~čísel $2^5$, $3^3$ a~$5^2$.

Ak zvolíme napríklad $a = 2\cdot3\cdot4\cdot5\cdot6 = 720$
a~$b = 1\cdot7\cdot8\cdot9\cdot10 = 5\,040$, bude najmenší spoločný násobok oboch
čísel práve~$5\,040$.
Tým je ukázané, že $5\,040$ je naozaj najmenšia zo všetkých možných hodnôt~$n$.

I~keď bolo úlohou nájsť iba jeden príklad, pre úplnosť uvedieme všetky
rozdelenia s~minimálnou hodnotou $n=5\,040$:
$$
\vbox{\let\\=\cr
\halign{&\quad\hss\strut#\unskip\hss\quad\cr \noalign{\hrule height.8pt}
Prvá skupina čísel & Druhá skupina čísel \\ \noalign{\hrule height.4pt}
2, 3, 4, 5, 6 & 1, 7, 8, 9, 10 \\
3, 5, 6, 8 & 1, 2, 4, 7, 9, 10 \\
2, 5, 8, 9 & 1, 3, 4, 6, 7, 10 \\
1, 2, 3, 4, 5, 6 & 7, 8, 9, 10 \\
1, 3, 5, 6, 8 & 2, 4, 7, 9, 10 \\
1, 2, 5, 8, 9 & 3, 4, 6, 7, 10 \\
%
2, 3, 4, 5, 6, 7 & 1, 8, 9, 10 \\
3, 5, 6, 7, 8 & 1, 2, 4, 9, 10 \\
2, 5, 7, 8, 9 & 1, 3, 4, 6, 10 \\
1, 2, 3, 4, 5, 6, 7 & 8, 9, 10 \\
1, 3, 5, 6, 7, 8 & 2, 4, 9, 10 \\
1, 2, 5, 7, 8, 9 & 3, 4, 6, 10 \\ \noalign{\hrule height.8pt}
}}
$$

Nájsť ich nie je ťažké, keď si uvedomíme, že čísla $1$ a~$7$ môžeme dať do ľubovoľnej
z~oboch skupín, zatiaľ čo v~tej istej skupine spolu nemôžu byť $4$ s~$8$, $5$ s~$10$,
$3$ s~$9$ ani $6$ s~$9$; s~$8$ spolu môže byť práve jedno z~párnych čísel $2$, $6$ a~$10$. Získame
tak iba tri základné rozdelenia (prvé tri riadky tabuľky), z~ktorých možno
každé štyrmi spôsobmi doplniť číslami $1$ a~$7$.

\poznamka
Úlohu možno vyriešiť aj bez výpočtu súčinu $a\cdot b$. Deliteľnosť
$n$ číslami $3^2$, $5$ a~$7$ vyplýva z~ich priameho zastúpenia medzi
rozdeľovanými číslami, deliteľnosť číslom~$2^4$ z~jednoduchej úvahy o~rozdelení
všetkých piatich párnych čísel: ak nie je číslo~$8$ vo svojej skupine ako párne jediné,
je všetko jasné, v~opačnom prípade sú v~rovnakej skupine čísla $2$, $4$ a~$6$
(aj~$10$, ale to už ani nepotrebujeme).

\nobreak\medskip\petit\noindent
Za správne zdôvodnené riešenie dajte 6~bodov, z~toho 1~bod za rozklad
súčinu $10!$ na súčin prvočísel, 3~body za odhad najmenšieho spoločného
násobku a~2~body za aspoň jedno správne rozdelenie čísel (hoci len uhádnuté).
\endpetit
\bigbreak
}

{%%%%%   C-S-3
Trojuholník $AIX$ je rovnoramenný, pretože $|\uhel IAX| = |\uhel IAC| = |\uhel AIX|$
(prvá rovnosť vyplýva z~podmienky, že bod~$I$ leží na osi uhla~$BAC$,
druhá potom z~vlastností striedavých uhlov, \obr). Preto $|AX| = |IX|$.
Analogicky zistíme, že $|BY| = |YI|$. Keďže úsečky $IX$ a~$IY$
zvierajú (rovnako ako s~nimi rovnobežné úsečky $CA$ a~$CB$) pravý uhol, podľa
Pytagorovej vety pre pravouhlý trojuholník $XIY$ platí
$$
|AX|^2+ |BY|^2= |IX|^2+ |YI|^2=|XY|^2,
$$
čo sme mali dokázať.
\insp{c63.4}%

\nobreak\medskip\petit\noindent
Za úplné riešenie dajte 6~bodov, z~toho 4~body za zdôvodnenie rovností $|AX| = |IX|$
a~$|BY| = |YI|$.
\endpetit
}

{%%%%%   C-II-1
Zostavíme a~vyriešime rovnicu pre neznáme cifry $a$, $b$, $c$, ktorú
vďaka tvaru zadaných
čísel môžeme zapísať rovno pre jedinú neznámu $x = 100a + 10b + c$:
$$
\align
{60\,000 +10x + 3\over 30\,000 +10x + 6}=& {63\over 36} = {7 \over 4} ,\\
40x + 240\,012 =& 70x + 210\,042, \\
30x =& 29\,970, \\
x =& 999.
\endalign
$$

\zaver
Nájdenému $x$ zodpovedá trojica cifier $a = b = c = 9$. Úloha má jediné riešenie.



\nobreak\medskip\petit\noindent
Za úplné riešenie dajte 6~bodov, z~toho 3~body za zostavenie správnej
rovnice a~3~body za jej vyriešenie a~záver. Len za uhádnutie výsledku (bez
vysvetlenia, že je jediný možný) dajte 1~bod.

\endpetit
\bigbreak
}

{%%%%%   C-II-2
Každý hráč odohral po jednej partii so zvyšnými štyrmi. Bolo teda
odohraných celkom $\frac12\cdot 5 \cdot 4 = 10$
partií, takže každý hráč získal práve 2~body.
Sú len tri možnosti, ako získať odohraním štyroch partií 2~body,
a~podľa toho obsahovala celková tabuľka nanajvýš tri rovnocenné
skupiny hráčov. Tieto skupiny, $A$, $B$ a~$C$, uvádzame v~poradí, v~ktorom by sa
v~konečnej tabuľke umiestnili:

Skupina~$A$ obsahuje všetkých hráčov, ktorí majú po dvoch výhrach a~dvoch
prehrách. Skupina~$B$ pozostáva z~hráčov s~jednou výhrou, jednou prehrou
a~dvoma remízami. Skupina~$C$ obsahuje hráčov so štyrmi remízami.

Vojto a~Tomáš sú jediní víťazi, preto nepatria do skupiny~$C$. Nepatria
ani do skupiny~$B$, pretože v~opačnom prípade by s~nimi museli všetci traja hráči zo skupiny~$C$
s~horším výsledkom remizovať (a~každý hráč skupiny~$B$ má
len dve remízy).

Z~toho vyplýva, že Vojto a~Tomáš majú po dvoch výhrach a~dvoch prehrách
a~skupina~$C$ je prázdna. Zvyšní traja hráči tak majú po jednej výhre, jednej
prehre a~dvoch remízach, ktoré museli uhrať navzájom medzi sebou.

\zaver
Peter a~Martin spolu remizovali.

\ineriesenie
Využijeme (nadbytočný) údaj, že Vojto porazil Petra:
Keby mali Vojto a~Tomáš po jednej výhre, jednej prehre a~dvoch remízach,
musel by aj Peter patriť medzi víťazov turnaja. Jediný v~poradí nižší
celkový výsledok sú totiž štyri remízy, Peter však jednu partiu
prehral, a~tak musel aj jednu vyhrať.
Vojto a~Tomáš majú preto po dvoch výhrach a~dvoch prehrách. Ak
Peter prehral s~Vojtom, musel poraziť Tomáša. (Nemohol mať dve prehry,
keďže bol v~poradí nižšie ako Tomáš a~Vojto. Ani nemohol s~Tomášom,
ktorý žiadnu remízu nemá, remizovať.) Potrebný druhý bod získal dvoma
remízami~-- s~Martinom a~nepomenovaným piatym hráčom.

\zaver
Peter a~Martin spolu remizovali.



\nobreak\medskip\petit\noindent
Za úplné riešenie dajte 6 bodov.

\endpetit
\bigbreak
}

{%%%%%   C-II-3
Vzhľadom na podmienku $c^2 + ab = a^2 + b^2$
stačí dokázať nerovnosť $a^2 + b^2 \le ac + bc$.
Tá je ekvivalentná so vzťahom $(a^2 + b^2 )^2 \le c^2 (a~+ b)^2$,
ktorý vzhľadom na danú podmienku prepíšeme na tvar
$$
(a^2 + b^2 )^2 \le (a^2 + b^2 - ab)(a + b)^2 . \tag1
$$
Po roznásobení a~zlúčení rovnakých členov zistíme, že máme dokázať
nerovnosť
$$
0 \le a^3b + ab^3 - 2a^2b^2 = ab(a - b)^2,
$$
ktorá pre kladné čísla $a$, $b$ zrejme platí. Vzhľadom na to, že všetky úpravy boli
ekvivalentné, môžeme celý postup obrátiť. Nerovnosť je tak dokázaná.

\ineriesenie
Bez ujmy na všeobecnosti predpokladajme, že $0 < b \le a$ (dané vzťahy sa výmenou
čísel $a$ a~$b$ nemenia). Nerovnosť $c^2 + ab\le ac + bc$ je ekvivalentná
s~nerovnosťou $(a- c)(c-b) \ge 0$, takže
stačí dokázať, že $b \le c \le a$. Platí
$$
c^2 = b^2 + a^2 - ab = b^2 + a~(a~- b) \ge b^2 ,
$$
teda $b \le c$. Analogicky zistíme, že
$$
c^2 = a^2 + b^2 - ab = a^2 + b(b - a~) \le a^2,
$$
a~odtiaľ $c \le a$. Tým je dôkaz prevedený.

\nobreak\medskip\petit\noindent
Za úplné riešenie dajte 6 bodov, z~toho 2 body za prevedenie pôvodnej
nerovnosti na súčinový tvar, alebo 4 body za úpravu na tvar~\thetag1.

\endpetit
\bigbreak
}

{%%%%%   C-II-4
Hľadaný obsah trojuholníka $ECD$ označme~$S$. Uhol $DEC$ je striedavý
s~uhlami $ADE$ a~$ECB$, odtiaľ $AD \parallel EC$ a~$ED \parallel BC$ (\obr).
\insp{c63.5}%
Trojuholníky $EDA$ a~$EDC$ majú spoločnú stranu~$ED$, pomer ich obsahov je teda rovný
pomeru prislúchajúcich výšok. Ak navyše postupne označíme $P$, $Q$ a~$R$ kolmé priemety
vrcholov $A$, $B$ a~$C$ na priamku~$DE$ a~označíme $v=|AP|$, $w=|BQ|=|CR|$,
dostaneme z~podobných pravouhlých trojuholníkov $AEP$ a~$BEQ$ úmeru
$$
{18\over S} = {v\over w} = {|AE| \over |EB|}.
$$

Analogicky pre trojuholníky $ECD$ a~$ECB$ zistíme, že
$$
{8\over S} = {|EB| \over |AE|}.
$$
(V~\obrr1{} sú prislúchajúce priemety iba naznačené, ale jedná sa o~ten istý výpočet ako
v~predošlom odseku, len v~ňom zameníme zodpovedajúce body $A \leftrightarrow B$,
$C \leftrightarrow D$ a~prislúchajúce obsahy trojuholníkov $AED$ a~$BEC$.)
Dokopy teda je $S:8=18:S$ čiže $S^2=144$, takže trojuholník $ECD$ má obsah $S=12\cm^2$.


\ineriesenie
Rovnako ako v~prvom riešení zistíme, že $AD \parallel EC$ a~$ED \parallel BC$.
Z~toho vyplýva podobnosť trojuholníkov $AED$ a~$EBC$. Ak označíme $k$ príslušný pomer
podobnosti (\obr), platí pre obsahy dotyčných trojuholníkov
$$
18 = \tfrac12 ab \sin\gamma ,\quad
S~= \tfrac12 ka \cdot b \sin\gamma , \quad
8 = \tfrac12 ka \cdot kb \sin\gamma ,
$$
takže zrejme platí $18 \cdot 8 = S^2$. Pre obsah
trojuholníka $ECD$ tak dostávame $S=12\cm^2$.
\insp{c63.6}%


\nobreak\medskip\petit\noindent
Za úplné riešenie dajte 6~bodov, z~toho 2~body za zdôvodnenie vzťahov $AD \parallel EC$
a~$ED \parallel BC$.

\endpetit
\bigbreak
}

{%%%%%   vyberko, den 1, priklad 1
...}

{%%%%%   vyberko, den 1, priklad 2
...}

{%%%%%   vyberko, den 1, priklad 3
...}

{%%%%%   vyberko, den 1, priklad 4
...}

{%%%%%   vyberko, den 2, priklad 1
...}

{%%%%%   vyberko, den 2, priklad 2
...}

{%%%%%   vyberko, den 2, priklad 3
...}

{%%%%%   vyberko, den 2, priklad 4
...}

{%%%%%   vyberko, den 3, priklad 1
...}

{%%%%%   vyberko, den 3, priklad 2
...}

{%%%%%   vyberko, den 3, priklad 3
...}

{%%%%%   vyberko, den 3, priklad 4
...}

{%%%%%   vyberko, den 4, priklad 1
...}

{%%%%%   vyberko, den 4, priklad 2
...}

{%%%%%   vyberko, den 4, priklad 3
...}

{%%%%%   vyberko, den 4, priklad 4
...}

{%%%%%   vyberko, den 5, priklad 1
...}

{%%%%%   vyberko, den 5, priklad 2
...}

{%%%%%   vyberko, den 5, priklad 3
...}

{%%%%%   vyberko, den 5, priklad 4
...}

{%%%%%   trojstretnutie, priklad 1
V~prvej časti riešenia uvažujme ľubovoľný trojuholník $ABC$
s~vnútornými uhlami $\al$, $\be$, $\ga$. Vďaka sínusovej vete ho prípadnou
podobnosťou môžeme zmeniť tak, aby dĺžky jeho strán boli priamo
kladné čísla $a$, $b$, $c$ rovné sínusom jeho vnútorných uhlov, teda
$$
a=\sin\al,\quad b=\sin\be\quad\text{a}\quad c=\sin\ga.
$$
Podľa kosínusovej vety pre tieto čísla platí rovnosť, ktorú budeme
hneď upravovať, až dostaneme rovnosť zo zadania úlohy:
$$\align
a^2&=b^2+c^2-2bc\cdot(\pm\sqrt{1-a^2}),\\
2bc\cdot(\pm\sqrt{1-a^2})&=b^2+c^2-a^2,\quad/^{2}\\
4b^2c^2(1-a^2)&=\bigl(b^2+c^2-a^2\bigr)^{2},\\
4b^2c^2-4a^2b^2c^2&=a^4+b^4+c^4-2a^2b^2-2a^2c^2+2b^2c^2,\\
a^4+b^4+c^4+4a^2b^2c^2&=2\bigl(a^2b^2+a^2c^2+b^2c^2\bigr).
\endalign$$

V~druhej časti riešenia predpokladajme, že $a$, $b$, $c$ sú pevné
kladné čísla spĺňajúce zadanú
rovnicu. Opačným postupom ako v~prvej časti riešenia ju upravíme
na tvar
$$
4b^2c^2\bigl(1-a^2\bigr)=\bigl(b^2+c^2-a^2\bigr)^2.
$$
Z~toho vyplýva, že číslo $a$ (kladné podľa zadaného predpokladu)
je nanajvýš rovné $1$. Existuje preto $\al\in(0,\pi)$ také, že
$\sin\al=a$ a~že upravenú rovnicu možno po odmocnení zapísať
v~tvare
$$
2bc\cos\al=b^2+c^2-a^2.
$$
Ak teda zostrojíme podľa vety $sus$ trojuholník $ABC$,
v~ktorom $|AC|=b$, $|AB|=c$ a $|\uhel BAC|=\al$,
bude v~ňom vďaka kosínusovej vete podľa predchádzajúcej rovnosti platiť
$|BC|=a$, takže potom z~ďalších dvoch podobných dôsledkov zadanej rovnice
$$
4a^2c^2\bigl(1-b^2\bigr)=\bigl(a^2+c^2-b^2\bigr)^2\quad\text{a}\quad
4a^2b^2\bigl(1-c^2\bigr)=\bigl(a^2+b^2-c^2\bigr)^2
$$
vyplýva, že pre vnútorné uhly $\al$, $\be$, $\ga$
takého trojuholníka $ABC$ bude platiť nielen
$\sin\al=a$ (pozri vyššie), ale aj $\cos^2\be=1-b^2$ a~$\cos^2\ga=1-c^2$, čiže
$\sin\be=b$ a $\sin\ga=c$.

\ineriesenie
\podla{Patrika Baka}
Uvedenú rovnosť možno ekvivalentne upraviť na tvar
$$
4a^2b^2c^2=(a+b+c)(a+b-c)(a+c-b)(b+c-a).
\tag1
$$

Nech pre kladné čísla $a$, $b$, $c$ platí táto rovnosť. Vzhľadom na symetriu môžeme bez ujmy na všeobecnosti predpokladať, že $a\ge b\ge c$.
Prvé tri činitele na pravej strane v~\thetag1 sú podľa našich predpokladov kladné a~keďže ľavá strana je tiež kladná, musí byť kladný aj štvrtý činiteľ. Čísla $a$, $b$, $c$ teda spĺňajú trojuholníkové nerovnosti a~existuje trojuholník $ABC$ so stranami $a$, $b$, $c$. Pre jeho obsah~$S$ podľa Herónovho vzorca platí
$$
16S^2=(a+b+c)(a+b-c)(a+c-b)(b+c-a),
\tag2
$$
odkiaľ spojením s~\thetag1 máme $4a^2b^2c^2=16S^2$, čiže $abc=2S$. Ak označíme $r$ polomer kružnice opísanej trojuholníku $ABC$, podľa známeho vzťahu platí $r=abc/(4S)$, a~preto $r=\frac12$. Zároveň však $r=a/(2\sin\alpha)$, a~teda $a=\sin\alpha$. Analogicky dostaneme $b=\sin\beta$ a~$c=\sin\gamma$.

Pre dôkaz opačnej implikácie predpokladajme, že trojuholník $ABC$ má strany a~uhly spĺňajúce $a=\sin\alpha$, $b=\sin\beta$ a~$c=\sin\gamma$. Potom použitím označenia a~známych vzťahov z~predošlého odseku dostaneme $r=\frac12$ a~$abc=2S$, odkiaľ po umocnení a~použití vzorca \thetag2 dostaneme želanú rovnosť \thetag1.

}

{%%%%%   trojstretnutie, priklad 2
Ukážeme, že hľadanou podmienkou je rovnosť $x_1=b$. V~priebehu
riešenia využijeme jednoduchý poznatok,
že všeobecný člen~$x_n$ skúmaných postupností má vyjadrenie
$$
x_n=a^{n-1}x_1+\bigl(a^{n-2}+a^{n-3}+\dots+a+1\bigr)b.
\tag1$$

V~prvej časti riešenia odvodíme rovnosť $x_1=b$ z~predpokladu,
že pre každé $n\geqq1$ platí $x_n\mid x_{2n}$ (využijeme teda len
časť zo všetkých implikácií zo zadania úlohy). Keďže vzorec (1) pre
skrátenú postupnosť $x_n,x_{n+1},x_{n+2},\dots$ vedie na~vyjadrenie
$$
x_{2n}=a^{n}x_n+\bigl(a^{n-1}+a^{n-2}+\dots+a+1\bigr)b,
$$
pre podiel $x_{2n}/x_n$ z~posledného vzťahu vyplýva
$$
\frac{x_{2n}}{x_n}=a^n+
\frac{\bigl(a^{n-1}+a^{n-2}+\dots+a+1\bigr)b}{x_n}.
\tag2$$
Špeciálne pre $n=1$ tak dostávame nutnú podmienku $x_1\mid b$.
Potrebnú podmienku $x_1=b$ odvodíme, keď pripustíme, že
platí $x_1<b$, a z~toho dôjdeme k sporu. Uvažujme čísla
$$
y_n=\bigl(a^{n-1}+a^{n-2}+\dots+a+1\bigr)b
$$
z~čitateľa zlomku na pravej strane (2), ktoré ako čísla $x_n$
spĺňajú rekurentný vzťah $y_{n+1}=ay_{n}+b$. Podľa vzorca (1)
potom v~prípade $x_1<b$ platí pre každé $n$ nerovnosť
$x_n<y_n$, pričom podľa (2) sú všetky podiely
$y_n/x_n$ celočíselné a tvoria klesajúcu postupnosť, pretože
$$
\frac{y_n}{x_n}-\frac{y_{n+1}}{x_{n+1}}=
\frac{y_n}{x_n}-\frac{ay_n+b}{ax_n+b}=
\frac{b(y_n-x_n)}{x_n(ax_n+b)}>0.
\tag3$$
Dostali sme nekonečnú klesajúcu postupnosť celých čísel
väčších ako 1, a to je avizovaný spor. Rovnosť $x_1=b$ je tak dokázaná.

\smallskip
V~druhej jednoduchšej časti riešenia ukážeme, že v~prípade $x_1=b$
platia všetky implikácie zo zadania úlohy. Využijeme pritom opäť
vzorec (1), podľa ktorého v~prípade $a=1$ po dosadení $x_1=b$
dostaneme pre každé $n$ vyjadrenie $x_n=nb$, zatiaľ čo v~prípade
$a>1$ vychádza vzorec
$$
x_n=\dfrac{\bigl(a^n-1\bigr)b}{a-1}.
$$
Všetky dokazované implikácie tak sú v~prípade $a=1$ úplne triviálne,
zatiaľ čo v~prípade $a>1$ sú zrejmým dôsledkom známeho pravidla
$$
m\mid n\quad\Longrightarrow\quad a^m-1\mid a^n-1,
$$
ktoré pre $n=km$ vyplýva z~algebraického vzorca
$$
a^{km}-1=\bigl(a^{m}-1\bigr)
\bigl(a^{(k-1)m}+a^{(k-2)m}+\dots+a+1\bigr).
$$

\poznamka
Tvrdenie, že všetky zlomky z~pravých strán (2),
teda zlomky
$$
\frac{y_n}{x_n}=
\frac{\bigl(a^{n-1}+a^{n-2}+\dots+a+1\bigr)b}
{a^{n-1}x_1+\bigl(a^{n-2}+a^{n-3}+\dots+a+1\bigr)b},
$$
sú celočíselné jedine v~prípade $x_1=b$, možno odvodiť aj
z~poznatku, že také zlomky majú pre $n\to\infty$ limitu
$$
\dfrac{ab}{(a-1)x_1+b}
$$
(rutinný výpočet tu neuvádzame),
a preto kvôli ich celočíselnosti musí
pre všetky dostatočne veľké $n$ platiť rovnosť
$y_{n+1}/x_{n+1}=y_n/x_n$, ktorá podľa úpravy~(3) platí, len keď
je $y_n=x_n$, a teda $x_1=b$.
}

{%%%%%   trojstretnutie, priklad 3
\podla{Patrika Baka}
Pripomeňme si najskôr jedno známe tvrdenie o~{\it špirálovej podobnosti} (\tj. o~zobrazení, ktoré je zložením rovnoľahlosti a~otočenia s~rovnakým stredom).

\odsek{Tvrdenie}
Majme úsečky $XY$ a~$X'Y'$, ktoré nie sú rovnobežné. Prienik priamok $XY$ a~$X'Y'$ označme $T$.
Potom stredom špirálovej podobnosti, ktorá úsečku~$XY$ zobrazí na úsečku~$X'Y'$, je priesečník kružníc opísaných trojuholníkom $XTX'$ a~$YTY'$ rôzny od bodu~$T$.\footnote{Tvrdenie neuvádzame v~presnej forme: V~špeciálnych prípadoch sa môže stať, že bod $T$ splýva s~niektorým z~krajných bodov daných úsečiek, prípadne uvedené kružnice nemajú iný spoločný bod ako~$T$.}
\insp{cps.1}%

\dokaz
Uvedieme dôkaz len pre konfiguráciu ako na \obr. Priesečník kružníc, o~ktorom dokazujeme, že je stredom špirálovej podobnosti, označme~$S$.
Stačí ukázať podobnosť trojuholníkov $XSY$ a~$X'SY'$. Tá vyplýva jednoducho z~obvodových uhlov:
$$
\gather
|\angle XYS|=|\angle TYS| = |\angle TY'S|=|\angle X'Y'S|, \\
|\angle SXY| = 180\st - |\angle SXT| = 180\st - |\angle SX'T| = |\angle SX'Y'|.
\endgather
$$

\smallskip
Prejdime k~riešeniu samotnej úlohy. Keďže súčet vnútorných uhlov v~štvoruholníku je $360\st$, zo zadania vyplýva $|\angle BAD|+|\angle BCD|=90^{\circ}$ a~$|\angle MBC|=|\angle NDC|=45^{\circ}$. Potom ${|\angle BCM| = 90^{\circ} -|\angle BCD|=|\angle BAD|}$ a~podobne dostaneme ${|\angle DCN|=|\angle BAD|}$. Takže trojuholníky $BMC$ a~$DNC$ sú podobné (\obr).
Päty kolmíc z~bodu~$C$ na priamky $AB$ a~$AD$ označme postupne $P$ a~$Q$.
Aj trojuholníky $BPC$ a~$DQC$ sú podobné, preto $|BM|:|BP|=|DN|:|DQ|$.
\insp{cps.2}%

Uvažujme špirálovú podobnosť $\Cal S$, ktorá zobrazí úsečku $BM$ na $DN$. Podľa tvrdenia z~úvodu je jej stredom práve bod~$K$. Avšak vzhľadom na odvodený pomer zobrazuje~$\Cal S$ úsečku $BP$ na $DQ$. Preto (opäť podľa tvrdenia z~úvodu) bod~$K$ leží na kružnici opísanej trojuholníku $APQ$, čo je Tálesova kružnica nad priemerom~$AC$, takže uhol $AKC$ je pravý.}

{%%%%%   trojstretnutie, priklad 4
Označme $D$ obraz bodu~$B$ v~stredovej súmernosti so stredom v~$P$ (potom $ABCD$ je rovnobežník).
Z rovnosti $2|AN|\cdot |AB|\cdot |CL| = |AC|\cdot |AK|\cdot |CL|$ máme ${|AN|\cdot |AB|} = \frac12|AC|\cdot |AK|$, takže štvoruholník $KPBN$ je tetivový. Podobne aj štvoruholník $LPBM$ je tetivový.
Keďže štvoruholník $KLMN$ je tiež tetivový, sú priamky $KN$, $BP$ a $LM$ chordálami dvojíc kružníc opísaných uvedeným tetivovým štvoruholníkom.
Preto priesečník $X$ priamok $KN$ a $ML$ leží na priamke~$BD$ (\obr).
\insp{cps.3}%

Z~tetivovosti uvedených štvoruholníkov vyplýva zhodnosť uhlov $CBD$ a~$KLX$; podobne uhly $CDB$, $ABD$ a $LKX$ sú zhodné.
Podľa vety $uu$ je trojuholník $KLX$ podobný s trojuholníkom $DBC$. Uhly $XSP$ a $XPS$ musia byť rovnaké, lebo sú to uhly, ktoré zvierajú zodpovedajúce si ťažnice so stranami v~podobných trojuholníkoch (to platí ako pre konfiguráciu, keď bod~$S$ leží na úsečke~$AP$, tak pre konfiguráciu, keď leží na úsečke~$CP$, argumentácia je takmer rovnaká). Preto bod~$X$ leží na osi úsečky~$PS$.
}

{%%%%%   trojstretnutie, priklad 5
\podla{Miroslava Psotu}
V~riešení budeme pracovať s~Pascalovým trojuholníkom, teda so zápisom kombinačných čísel do nasledovnej (nekonečnej) schémy:
$$
\gathered
\tbinom00\\
\tbinom10\ \tbinom11\\
\tbinom20\ \tbinom21\ \tbinom22\\
\tbinom30\ \tbinom31\ \tbinom32\ \tbinom33\\
\dots
\endgathered
$$
Vieme, že každé číslo v~tejto schéme (okrem čísel $1$ na začiatku a~na konci každého riadka) je rovné súčtu dvoch čísel bezprostredne nad ním. Ak v~Pascalovom trojuholníku každé číslo nahradíme jeho zvyškom po delení dvoma, \tj. nulou alebo jednotkou podľa parity čísla, uvedené pravidlo spôsobí, že ak vedľa seba budú dva rovnaké zvyšky, pod nimi bude $0$ a~ak vedľa seba budú dva rôzne zvyšky, pod nimi bude $1$. Prvých niekoľko riadkov vyzerá takto (kvôli prehľadnosti nuly znázorňujeme bodkami):
$$
\hbox{\epsfbox{pascal63.1}}
$$

Nech $T_{j}$ označuje trojuholník vytvorený z~prvých $2^j$ riadkov takéhoto "zvyškového" Pascalovho trojuholníka (\tj. riadkov pochádzajúcich z~kombinačných čísel $\binom il$ pre $i=0,1,\dots,2^j-1$).
Ľahko možno nahliadnuť a~matematickou indukciou dokázať, že $T_{j+1}$ sa skladá z~troch kópií $T_{j}$ vo vrcholoch a~"otočeného" trojuholníka zloženého zo samých núl v~strede.

Predpokladajme, že $n$ spĺňa podmienky zadania.
Voľbou $m=0$ dostaneme, že číslo $\binom{n}{k}$ musí byť nepárne pre každé $k\le n$. To znamená, že v~príslušnom riadku Pascalovho trojuholníka sú len zvyšky $1$, čomu vyhovujú len hodnoty $n=2^j-1$ pre $j=0,1,\dots$ (vyplýva to z~vyššie uvedenej konštrukcie trojuholníkov $T_j$, formálne môžeme opäť použiť indukciu).
\insp{cps.4}%

Naopak, ak $n=2^j-1$, tak podmienka zo zadania je splnená. Zvyšky kombinačných čísel $\binom{n-k}{m}$ a~$\binom{n-m}{k}$ sa totiž nachádzajú v~$T_j$ na pozíciách, ktoré sú súmerne združené vzhľadom na os uhla pri ľavom dolnom vrchole trojuholníka~$T_j$. Stačí teda ukázať, že $T_{j}$ je symetrický podľa tejto osi. Na to znova použijeme indukciu: Nech $T_{j}$ je symetrický podľa uvedenej osi (pre malé hodnoty $j$ to zjavne platí).
Pozrime sa na trojuholník~$T_{j+1}$. Keďže je rovnostranný, jeho os uhla je zároveň osou uhla ľavej dolnej kópie~$T_{j}$ aj vnútorného trojuholníka zloženého z~núl (\obr).
Na kópiách $T_{j}$ je tak symetrický podľa indukčného predpokladu a~prostredný trojuholník je podľa osi symetrický tiež, keďže sa skladá zo samých núl.}

{%%%%%   trojstretnutie, priklad 6
\podla{Zhen Ning Davida Liu} Nech $\Cal M$ je najväčší podsystém navzájom disjunktných množín systému~$\Cal{F}$. Označme $\mm S$ zjednotenie všetkých množín z~$\Cal M$. Ďalej označme ${\mm S'=\{1,2,\dots,n\}\setminus\mm S}$, teda množinu tých prvkov, ktoré sa nenachádzajú v~žiadnej množine z~$\Cal M$.
Sporom ukážeme, že $|\Cal{M}| \ge \lfloor\frac13n\rfloor - 1$, čiže počet prvkov v~$\mm S$ je aspoň ${3(\lfloor\frac13n\rfloor - 1)} \ge n-5$.

Ak sa prvky $x$ a~$y$ nachádzajú v~$\mm S'$, tak všetky také prvky~$z$, pre ktoré množina $\mm A=\{x,y,z\}$ je prvkom $\Cal{F}$, musia ležať v~$\mm S$. Inak by sme totiž mohli zobrať množinu~$\mm A$ a~pridať ju do systému $\Cal{M}$, čím by sme zväčšili jeho veľkosť.

% Takýchto prvkov je ${\lfloor\dfrac{n}{3}\rfloor - 1}$.
Nech $|\Cal{M}| < \lfloor\frac13n\rfloor - 1$. Potom $|\mm S'|\ge 6$, teda spomedzi prvkov množiny~$\mm S'$ vieme vybrať tri disjunktné dvojprvkové množiny $\{x_1, y_1\}$, $\{x_2, y_2\}$ a $\{x_3,y_3\}$.
Ukážeme, že pre niektoré dva rôzne indexy $i,j\in\{1,2,3\}$ existujú množiny
$$
\mm A=\{s_1, s_2, s_3\}\in \Cal{M},\qquad \mm B=\{x_i,y_i,s_1\}\in\Cal{F},\qquad \mm C=\{x_j,y_j,s_2\}\in\Cal{F}.
\tag1
$$
Potom v~systéme~$\Cal{M}$ môžeme množinu~$\mm A$ nahradiť množinami $\mm B$ a~$\mm C$, čím ho zväčšíme.

Nazývajme {\it dobrými trojicami\/} také množiny $\mm A\in\Cal{F}$, že pre niektoré $i\in\{1,2,3\}$ a~$s\in\mm S$ platí $\mm A=\{x_i,y_i,s\}$.
Ak žiadna trojica množín tvaru \thetag1 neexistuje, tak prvky každej jednej množiny z~$\Cal{M}$ tvoria dobré trojice buď iba s~jednou dvojicou $\{x_i, y_i\}$, alebo iba jeden prvok tejto množiny je v~dobrej trojici s~niektorými dvojicami $\{x_i, y_i\}$.
V~každom prípade na jednu množinu z~$\Cal{M}$ prislúchajú nanajvýš tri dobré trojice.

Z~pohľadu dvojíc prvkov $\{x_i, y_i\}$ je však podľa zadania počet dobrých trojíc aspoň $3(\lfloor\frac13n\rfloor - 1)$, pretože, ako sme poznamenali na úvod, každý prvok, ktorý s~niektorou dvojicou $\{x_i, y_i\}$ tvorí množinu z~$\Cal{F}$, leží v~$\mm S$.
Z~toho dostávame odhady
$$
3|\Cal{M}|\ge \text{počet dobrých trojíc}\ge 3(\lfloor\tfrac13n\rfloor - 1),
$$
čo je v~spore s~predpokladom, že $|\Cal{M}| < \lfloor\frac13n\rfloor - 1$.}

{%%%%%   IMO, priklad 1
Označme $\mm B$ množinu tých prirodzených čísel~$n$, pre ktoré je splnená prvá nerovnosť zo zadania, teda takých $n$, pre ktoré $a_n<(a_0+a_1+\cdots+a_n)/n$. Ekvivalentnou úpravou tejto podmienky dostaneme
$$
(a_n - a_{n-1}) + (a_n - a_{n-2}) + \cdots + (a_n - a_1) < a_0.
\tag1
$$
Množina~$\mm B$ má nasledovné tri vlastnosti:
\item{$\triangleright$}
Je neprázdna, pretože zrejme $1\in\mm B$.
\item{$\triangleright$}
Každý z~$n-1$ sčítancov na ľavej strane \thetag1 je aspoň~$1$, takže ak $n>a_0$, tak $n\notin\mm B$. Preto $\mm B$ je konečná.
\item{$\triangleright$}
Pri zväčšení $n$ o~$1$ pribudne na ľavej strane \thetag1 kladný sčítanec $a_{n+1}-a_n$ a~navyše sa o~túto hodnotu zväčší každý zo zvyšných sčítancov.
Ľavá strana \thetag1 sa tak s~rastúcim $n$ zväčšuje. Teda ak nejaké číslo patrí do $\mm B$, patrí tam aj každé od neho menšie prirodzené číslo.
\endgraf\noindent
Z~uvedeného vyplýva, že $\mm B=\{1, 2, 3, \dots, n_0\}$ pre nejaké prirodzené číslo $n_0$.

Všimnime si, že druhá nerovnosť zo zadania $\frc{(a_0+a_1+\cdots+a_n)}{n}  \le a_{n+1}$ je ekvivalentná s~podmienkou
$$
a_0 \le (a_{n + 1} - a_{n}) + (a_{n+1} - a_{n-1}) + \cdots + (a_{n+1} - a_1),
$$
teda s~tým, že $n+1 \not\in\mm B$. Také $n$, ktoré spĺňa obe nerovnosti zo zadania, \tj. také, že $n\in\mm B$ a~súčasne $n+1 \not\in\mm B$, je zrejme iba číslo~$n_0$.
}

{%%%%%   IMO, priklad 2
Hľadaná najväčšia hodnota je $k=\lfloor \sqrt{n-1} \rfloor$. V~prvej časti riešenia dokážeme, že ak $n>l^2$, tak každá šťastná konfigurácia obsahuje prázdny štvorec ${l\times l}$. V~druhej časti potom ukážeme, že ak $n\le l^2$, tak existuje šťastná konfigurácia neobsahujúca žiadny prázdny štvorec $l\times l$. Z~týchto dvoch tvrdení zrejme vyplýva uvedený záver.

\smallskip
Nech teda $n > l^2$. Uvažujme ľubovoľnú šťastnú konfiguráciu. Zoberme blok $l$ za sebou idúcich riadkov, medzi ktorými sa nachádza aj riadok obsahujúci vežu v~prvom stĺpci. Spolu je v~tomto bloku práve $l$ veží. Následne z~neho odstráňme prvých $n-l^2$ stĺpcov. Podľa nášho predpokladu je $n-l^2 \ge 1$, takže sme uvedenou operáciou odobrali z~bloku aspoň jednu vežu -- tú, ktorá sa nachádza v~prvom stĺpci. Zvyšných $l\times l^2$ políčok sa dá rozdeliť na $l$~štvorcov s~rozmermi $l\times l$ (\obr). Keďže je v~nich spolu nanajvýš $l-1$ veží, aspoň jeden z~nich neobsahuje vežu.
\insp{mmo.1}%

\smallskip
V~druhej časti predpokladajme, že $n\le l^2$. Najskôr zostrojíme šťastnú konfiguráciu bez prázdnych štvorcov pre prípad $n=l^2$ a~potom ukážeme, ako ju modifikovať pre menšie hodnoty $n$.

Označme riadky a~stĺpce šachovnice s~rozmermi $l^2\times l^2$ číslami od $0$ po $l^2-1$. Políčko v~$s$-tom stĺpci a~$r$-tom riadku budeme označovať súradnicami $(s,r)$. Rozložme veže na políčka so súradnicami
$(il+j,jl+i)$ pre všetky dvojice $i,j\in\{0,1,\dots,l-1\}$ (na \obr{} je zobrazená táto konfigurácia pre $l=4$; stĺpce sú číslované zľava doprava, riadky zdola nahor).
Keďže každé číslo od $0$ po $l^2-1$ sa dá práve jedným spôsobom zapísať v~tvare $il+j$, kde $0 \le i,j \le l-1$, v~každom stĺpci sa nachádza práve jedna veža a~rovnaký záver platí pre riadky.\footnote{Uvedená jednoznačnosť zápisu vyplýva z~faktu, že ak $s=il+j$, tak $j$ je zvyšok čísla~$s$ po delení číslom~$l$ a~$i$ je celočíselný podiel tohto delenia.} Preto táto konfigurácia je šťastná.
\insp{mmo.2}%

Dokážeme, že v~každom štvorci $l\times l$ tejto konfigurácie sa nachádza aspoň jedna veža. Uvažujme ľubovoľný taký štvorec a~zoberme $l$ po sebe idúcich stĺpcov, ktoré ho obsahujú. Povedzme, že prvý z~nich má číslo $pl+q$, kde $0 \le p,q \le l-1$ (platí teda $pl+q \le l^2-l$). Potom čísla riadkov, v~ktorých sa nachádzajú veže z~týchto stĺpcov, sú $ql+p$, $(q+1)l + p$, \dots, $(l-1)l + p$, $p+1$, $l + (p+1)$, \dots, $(q-1)l+(p+1)$. Ak ich zapíšeme od najmenšieho po najväčšie, dostaneme postupnosť
$$
p+1,\ l + (p+1),\ \dots,\ (q-1)l+(p+1),\ ql+p,\ (q+1)l + p,\ \dots,\ (l-1)l + p.
\tag1
$$

Ľahko nahliadneme, že najmenší člen tejto postupnosti má hodnotu nanajvýš $l-1$ (ak totiž $p=l-1$, tak $q=0$ a~postupnosť začína členom $ql+p=l-1$), najväčší člen má hodnotu aspoň $l(l-1)$ a~rozdiel medzi ľubovoľnými po sebe idúcimi členmi neprevyšuje $l$. Preto niektorý z~$l$ po sebe idúcich riadkov prechádzajúcich uvažovaným štvorcom $l\times l$ má číslo uvedené v~\thetag1, a~teda veža z~tohto riadku leží v~danom štvorci.

Ostáva zostrojiť šťastnú konfiguráciu bez prázdnych štvorcov v~prípade $n < l^2$. Za tým účelom zväčšíme šachovnicu na rozmery $l^2 \times l^2$ a~vyplníme ju ako vyššie. Následne odoberieme pridané riadky a~stĺpce. Môže sa stať, že v~niektorých riadkoch a~stĺpcoch nebudú veže. Avšak riadkov bez veží bude práve toľko ako stĺpcov bez veží, takže postupným doplnením veží na priesečníky prázdnych riadkov a~stĺpcov vytvoríme šťastnú konfiguráciu. Keďže neexistoval prázdny štvorec $l\times l$ v~šachovnici rozšírenej na rozmer $l^2 \times l^2$, nenájdeme ho ani v~tejto výslednej konfigurácii.
}

{%%%%%   IMO, priklad 3
Označme $Q$ priesečník priamky~$AB$ s~kolmicou na priamku~$SC$ vedenou bodom~$C$. Potom
$$
|\uhol SQC| = 90^{\circ} - |\uhol CSB| = 180^{\circ} - |\uhol CHS|,
$$
teda štvoruholník $SQCH$ je tetivový. Jeho opísanou kružnicou je Tálesova kružnica nad priemerom~$SQ$, ktorej stred ležiaci na priamke~$AB$ označme~$K$ (\obr). Analogicky stred kružnice opísanej trojuholníku $THC$, ktorý označíme~$L$, leží na priamke~$AD$.

Na vyriešenie úlohy stačí ukázať, že priesečník osí úsečiek $SH$ a~$TH$ leží na priamke~$AH$. Keďže $|KH|=|KS|$, je os úsečky~$SH$ zároveň osou uhla $AKH$; podobne os úsečky~$TH$ je osou uhla $ALH$. Aby sme dokázali, že osi uhlov $AKH$ a~$ALH$ delia úsečku~$AH$ v~rovnakom pomere, stačí podľa známeho tvrdenia o~osi uhla v~trojuholníku\footnote{Os vnútorného uhla trojuholníka delí protiľahlú stranu v~pomere strán priľahlých. Tento poznatok možno odvodiť jednoducho napr. použitím sínusovej vety pre trojuholníky, na ktoré delí os uhla pôvodný trojuholník.} dokázať, že
$$
\frac{|AK|}{|KH|} = \frac{|AL|}{|LH|}.
\tag1
$$
\insp{mmo.3}%

Ak body $A$, $H$, $C$ ležia na jednej priamke, uvedená rovnosť vyplýva zo symetrie (štvoruholník $ABCD$ je vtedy deltoid). V~opačnom prípade uvažujme kružnicu~$k$, ktorá cez ne prechádza. Keďže uhly pri vrcholoch $B$ a~$D$ sú pravé, je štvoruholník $ABCD$ tetivový, preto
$$
|\uhol BAC| = |\uhol BDC| = 90^{\circ} - |\uhol ADH| = |\uhol HAD|.
$$
Nech $N\ne A$ je priesečník kružnice~$k$ s~osou uhla $CAH$. Z~práve odvodenej rovnosti uhlov potom dostávame, že $AN$ je zároveň osou uhla $BAD$.
Zrejme body $H$ a~$C$ sú súmerne združené podľa priamky~$KL$ a~$|HN|=|NC|$. Z~toho vyplýva, že bod~$N$ aj stred kružnice~$k$ ležia na priamke~$KL$.
Navyše z~rovnakého tvrdenia o~osi uhla, ktoré sme použili pred chvíľou, máme $\frc{|KN|}{|NL|} = \frc{|AK|}{|AL|}$. Uvedené skutočnosti znamenajú, že $k$ je Apollóniovou kružnicou prislúchajúcou bodom $K$ a~$L$, odkiaľ už priamo dostaneme rovnosť \thetag1.
}

{%%%%%   IMO, priklad 4
Označme $S$ priesečník priamok $BM$ a~$CN$. Podľa zadania majú trojuholníky $PBA$ a~$QAC$ veľkosti uhlov rovnaké ako trojuholník $ABC$ (budeme ich označovať štandardne $\alpha$, $\beta$, $\gamma$), sú teda navzájom podobné. Odtiaľ
$$
\frac{|BP|}{|PM|}=\frac{|BP|}{|PA|}=\frac{|AQ|}{|QC|}=\frac{|NQ|}{|QC|}.
$$
Pritom $|\uhol BPM|=|\uhol NQC|=180-\alpha$, pretože sú to uhly susedné k~uhlom pri vrcholoch $P$ a~$Q$ uvedených podobných trojuholníkov (\obr). Trojuholníky $BPM$ a~$NQC$ majú zhodné uhly pri vrcholoch $P$ a~$Q$ aj pomery dĺžok strán, ktoré tieto uhly zvierajú, takže sú podobné. Preto $|\uhol BMP|=|\uhol NCQ|$ a~aj trojuholníky $BPM$ a~$BSC$ sú podobné. Z~toho dostávame $|\uhol CSB|=|\uhol BPM|=180^\circ-\alpha$, teda štvoruholník $ABSC$ je tetivový.
\insp{mmo.4}%
%Nech $S$ je priesečník $BM$ a~$NC$ a~nech kružnica opísaná trojuholníku $ABC$ pretína druhý krát priamky $AP$ a~$AQ$ postupne v~bodoch $K$ a~$L$.
%
%Všimnime si rovnosti $|\uhol LBC| = |\uhol LAC| = |\uhol CBA|$ a~$|\uhol KCB| = |\uhol KAB| = |\uhol BCA|$. Z~nich vyplýva, že priamky $BL$ a~$CK$ sa pretínajú v~bode~$X$ symetrickom s~$A$ vzhľadom k~priamke~$BC$. Pretože $|AP| = |PM|$
%a~$AQ=QN$, leží $X$  na $MN$. Teraz použitím Pascalovej vety na šesťuholník $ALBSCK$ dostaneme, že $S$ leží na kružnici opísanej trojuholníku $ABC$, čím sme ukončili dôkaz.
}

{%%%%%   IMO, priklad 5
Dokážeme všeobecnejšie tvrdenie: Pre každé kladné celé číslo~$N$ sa mince s~celkovou hodnotou nanajvýš $N-\frac{1}{2}$ dajú rozdeliť na $N$ častí tak, aby každá časť mala celkovú hodnotu nanajvýš~$1$ (pripúšťame aj "prázdne" časti s~hodnotou $0$).

Uvažujme teda ľubovoľnú kolekciu mincí s~hodnotou nanajvýš $N-\frac{1}{2}$. Pred samotným delením na časti kolekciu mierne modifikujeme. Pokiaľ niekoľko mincí dokopy má celkovú hodnotu $\frc{1}{k}$, nahradíme ich jednou mincou tejto hodnoty. Ak sa bude dať zmenená kolekcia rozdeliť požadovaným spôsobom, tak sa samozrejme dá rozdeliť aj pôvodná. Po každej modifikácii počet mincí v~kolekcii klesá, takže raz tento proces musí skončiť -- vtedy už nevieme vykonať žiadne opísané "zlučovanie" mincí. Z~toho vyplýva, že pre každé párne $k$ existuje len jedna minca s~hodnotou $\frc{1}{k}$ (inak by sme dve mince s~touto hodnotou vedeli nahradiť jednou mincou s~hodnotou $1/\frac k2$) a~pre každé nepárne $k>1$ existuje najviac $k-1$ mincí s~hodnotou $\frc{1}{k}$ (inak by sme $k$ takých mincí vedeli nahradiť mincou s~hodnotou $1$).

Každá minca s~hodnotou~$1$ musí sama tvoriť jednu z~$N$~častí. Ak máme $d$~mincí s~hodnotou~$1$, stačí ich odobrať a~v~tvrdení nahradiť $N$ hodnotou $N-d$. Môžeme teda predpokladať, že v~kolekcii nemáme žiadne mince s~hodnotou~$1$.

Za týchto predpokladov rozdeľme mince nasledovne: Pre každé $k = 1, 2, \dots, N$ dajme všetky mince s~hodnotami $\frc{1}{(2k-1)}$ a~$\frc{1}{(2k)}$ do skupiny $C_k$. Celková hodnota skupiny $C_k$ bude nanajvýš
$$
(2k-2)\cdot \frac{1}{2k-1} + \frac{1}{2k} = 1 - \left(\frac{1}{2k-1} - \frac{1}{2k}\right) < 1.
$$

Ostáva rozdeliť mince s~hodnotami menšími ako $\frc{1}{(2N)}$. Budeme ich pridávať po jednej opakovaním nasledujúceho kroku. Zoberme ľubovoľnú mincu čo zostala. Celková hodnota už rozdelených mincí je maximálne $N - \frac{1}{2}$, takže existuje časť s~celkovou hodnotou nanajvýš
$$
\frac{N - \frac{1}{2}}N = 1 - \frac{1}{2N},
$$
do ktorej je možné našu mincu pridať bez prekročenia stanoveného limitu.

\poznamka
Algoritmus rozdelenia mincí s~hodnotami aspoň $1/(2N)$ sa dá modifikovať: napr. do časti $C_k$ môžeme dať všetky mince s~hodnotami $1/[2^s(2k-1)]$ pre všetky celé čísla $s\ge0$. Ľahko možno nahliadnuť, že ich hodnota neprekročí~$1$.
}

{%%%%%   IMO, priklad 6
Nech $\mm P$ je ľubovoľná množina $n$~priamok vo všeobecnej polohe. Označme $\mm F$ množinu konečných oblastí prislúchajúcich množine~$\mm P$. Zoberme maximálnu (vzhľadom na inklúziu) podmnožinu $\mm Q\subseteq \mm P$ takú, že po zafarbení priamok z~$\mm Q$ namodro žiadna oblasť z~$\mm F$ nebude mať celú hranicu modrú. Položme $k=|\mm Q|$. Stačí dokázať, že ${k \ge \sqrt{n}}$.

Zafarbime priamky z~$\mm P \setminus\mm Q$ načerveno. Priesečníky dvoch modrých priamok budeme nazývať {\it modré\/} body a~priesečníky modrých priamok s~červenými priamkami {\it dvojfarebné\/} body. Modrých bodov je zrejme $\binom{k}{2}$ a~červených priamok $n-k$.

Uvažujme ľubovoľnú červenú priamku~$p$. Nech $\Cal A\in\mm F$ je taká oblasť, ktorá má práve jednu červenú stranu a~tá leží na $p$ (ak by taká oblasť neexistovala, mohli by sme priamku~$p$ pridať do $\mm Q$, čo je v~spore s~maximálnosťou množiny $\mm Q$). Označme $C$, $D$, $M_1$, $M_2$, \dots, $M_l$ vrcholy oblasti~$\Cal A$ v~smere hodinových ručičiek, pričom $C, D\in p$. Potom body $C$, $D$ sú dvojfarebné, zatiaľ čo body $M_1$, $M_2$, \dots, $M_l$ sú modré. Budeme hovoriť, že priamka~$p$ je {\it pridružená\/} k~bodu $M_1$ {\it cez\/} bod $D$ (\obr).
\insp{mmo.5}%

Všimnime si, že pre každú dvojicu tvorenú dvojfarebným bodom~$D$ a modrým bodom~$M$ môže byť k~$M$ cez~$D$ pridružená nanajvýš jedna červená priamka, pretože existuje nanajvýš jedna oblasť~$\Cal A$, ktorá má body $D$ a~$M$ umiestnené na svojom obvode v~smere hodinových ručičiek. Ukážeme, že k~žiadnemu modrému bodu nie sú pridružené viac ako dve červené priamky. Z~toho dostaneme odhad
$$
n-k \le 2\binom{k}{2},
$$
ktorý je ekvivalentný so želanou nerovnosťou $n\le k^2$.

Predpokladajme sporom, že tri rôzne priamky $p_1$, $p_2$, $p_3$ sú pridružené k~modrému bodu~$M$ postupne cez rôzne dvojfarebné body $D_1$, $D_2$, $D_3$. Bodom~$M$ prechádzajú dve modré priamky, ktoré určujú štyri polpriamky. Zrejme každý z~bodov~$D_i$ leží na jednej z~nich (každý na inej) a~je najbližším bodom k~$M$ spomedzi všetkých priesečníkov ležiacich na tejto polpriamke. Bez ujmy na všeobecnosti môžeme predpokladať, že $D_2$ a~$D_3$ ležia na navzájom opačných polpriamkach.
\insp{mmo.6}%
Uvažujme oblasť~$\Cal A$, vzhľadom na ktorú je $p_1$ pridružená k~$M$ cez $D_1$. Jej tri po sebe idúce vrcholy sú $D_1$, $M$ a~jeden z~vrcholov $D_2$, $D_3$, povedzme~$D_2$. Keďže oblasť~$\Cal A$ má iba jednu červenú stranu, musí to byť strana $D_2D_1$, \tj. $\Cal A$ je trojuholník $D_2D_1M$ (\obr). To je však spor s~tým, že priamky $p_1$ a~$p_2$ sú rôzne.
}

{%%%%%   MEMO, priklad 1
Ukážeme, že jedinými riešeniami spĺňajúcimi podmienky zo zadania sú funkcie tvaru $f(x) = 2x + a$, kde $a$ je ľubovoľné reálne číslo.

Označme $f(1) = c$. Po dosadení $x = 1$ do rovnosti zo zadania dostávame
$$
\align
0 &= 2 + f(y) - f(y+1),  \\
f(y+1) &= f(y) + 2
\endalign
$$
pre všetky $y\in\Bbb R$. Použitím matematickej indukcie v~oboch smeroch dostávame
$$
f(y+n) = f(y) + 2n \tag1
$$
pre všetky $y\in\Bbb R$ a $n\in \Bbb Z$.

Dosadením $y=1$ do rovnosti zo zadania dostávame
$$
cx + f(cx) - xf(c) - f(x) = 2x + c - (f(x) + 2),
$$
čo môžeme upraviť na tvar
$$
f(cx) = x(f(c) - c + 2) + c - 2.
$$
Ak $c\ne 0$, môžeme nahradiť $x$ neznámou $z/c$, z~čoho už vidno, že $f$ je lineárna funkcia. Z~rovnosti \thetag1 však vidíme, že smernica funkcie~$f$ musí byť $2$. Preto funkcia $f$ môže byť jedine tvaru $f(x) = 2x + a$.

Predpokladajme teraz, že $c = 0$. Z~rovnice \thetag1 dostávame
$$
f(n) = f(1 + (n-1)) = f(1) + 2(n-1) = c + 2n - 2 = 2n - 2
$$
pre akékoľvek celé číslo~$n$. Dosadením celého čísla $y = n$ do rovnice zo zadania dostávame
$$
\align
(2n-2)x + f((2n-2)x) - (4n-6)x - f(nx) &= 2x + (2n-2) - f(x) - 2n, \\
f((2n-2)x) - f(nx) + f(x) &= (2n-2)x -2. \tag2
\endalign
$$
V~prípade $n=0$ sa rovnosť zjednoduší na tvar
$$
f(-2x) + f(x) = -2x + 4.
$$
Dosadením $-2x$ za $x$ a odčítaním od predchádzajúcej rovnosti dostávame
$$
f(4x) - f(x) = (f(4x) + f(-2x)) - (f(-2x) + f(x)) = (4x-4) - (-2x-4) = 6x.\tag3
$$
Dosadením $n = \m1$ do \thetag2 a $x = \m x$ do \thetag3 dostávame
$$
\align
f(-4x) - f(-x) + f(x) &= -4x -2,    \\
f(-4x) - f(-x) &= - 6x.
\endalign
$$
Odčítaním posledných dvoch rovností dostávame $f(x) = 2x - 2$, teda opäť funkciu tvaru $f(x) = 2x + a$.

Dosadením $f(x) = 2x + a$ do rovnice zo zadania jednoducho overíme, že takáto funkcia je naozaj riešením pre ľubovoľné $a\in\Bbb R$.
}

{%%%%%   MEMO, priklad 2
Najskôr ukážeme, že všetky triangulárne čísla sú deliteľné tromi. Nech $n$ je triangulárne číslo. Uvažujme takú dvojfarebnú trianguláciu pravidelného $n$-uholníka, ktorá spĺňa podmienky zo zadania. Celkový počet strán bielych trojuholníkov triangulácie označme $b$. Keďže trojuholníky majú tri strany a~žiadne dva biele trojuholníky nemajú spoločnú stranu, číslo~$b$ musí byť deliteľné tromi.

Pozrime sa teraz bližšie na ľubovoľný vrchol~$A$. Trojuholníky pri vrchole~$A$ sú striedavo biele a~čierne, pričom prvý a~posledný trojuholník musí byť čierny, aby bola splnená podmienka zo zadania. Preto sú strany bielych trojuholníkov zároveň uhlopriečkami $n$-uholníka.

Rovnako dostaneme, že počet strán $c$ čiernych trojuholníkov je deliteľný tromi. Každá uhlopriečka $n$-uholníka použitá pri triangulácii je súčasne stranou bieleho aj čierneho trojuholníka a~navyše všetky strany $n$-uholníka sú stranami čiernych trojuholníkov, preto $c = n + b$. Keďže $c$ a~$b$ sú deliteľné tromi, musí byť tromi deliteľné aj $n$.

\smallskip
V~druhej časti dokážeme, že pre ľubovoľné prirodzené číslo $k$ a~pre $n=3k$ existuje dvojfarebná triangulácia spĺňajúca podmienky zo zadania, a~to nielen pre pravidelný, ale pre ľubovoľný konvexný $n$-uholník.
\insp{memo.1}%

Dôkaz prevedieme indukciou vzhľadom na $k$. Pre $k=1$ máme len jeden trojuholník, ktorý po ofarbení čiernou farbou tvorí trianguláciu spĺňajúcu podmienky zo zadania. Nech teda tvrdenie platí pre nejaké $k$. Uvažujme ľubovoľný $3(k+1)$-uholník $P = A_1A_2\dots A_{3(k+1)}$. Na $3k$-uholník $A_1A_2\dots A_{3k}$ použijeme indukčný predpoklad, \tj. predpokladáme, že v~ňom je vytvorená dvojfarebná triangulácia spĺňajúca podmienky zadania. Z~prvej časti riešenia vieme, že trojuholník so stranou~$A_1A_{3k}$ je v~nej ofarbený čiernou farbou. Do triangulácie mnohouholníka~$P$ pridáme uhlopriečky $A_1A_{3k+2}$ a~$A_{3k}A_{3k+2}$, trojuholník $A_1A_{3k}A_{3k+2}$ ofarbíme bielou farbou a~trojuholníky $A_{3k}A_{3k+1}A_{3k+2}$ a~$A_1A_{3k+2}A_{3k+3}$ ofarbíme čiernou farbou (\obr). Takto vznikne dvojfarebná triangulácia $3(k+1)$-uholníka, ktorá (ako možno ľahko nahliadnuť) spĺňa podmienky zo zadania.}

{%%%%%   MEMO, priklad 3
Nech $H$ je priesečník kolmice na $AE$ vedenej bodom~$E$ a~priamky~$AI$. Naším cieľom bude ukázať, že bod~$H$ je stredom kružnice pripísanej k~strane~$BI$ trojuholníka $BIG$ a~teda leží na osi uhla $BGI$.
\insp{memo.2}%

Osová súmernosť podľa priamky~$AI$ zobrazí bod~$E$ na bod~$B$ a~teda $AI$ je osou uhla $BIE$ a~zároveň uhol $HBA$ je zhodný s~uhlom $HEA$, čiže pravý (\obr). Označme $|\uhol ABC|=\beta$. Zo zadania vieme, že $|\uhol IBG| = \beta$ a~$|\uhol ABI| = \beta/2$. Preto $|\uhol IBH| = 90^\circ-|\uhol GBI|/2$, z~čoho vyplýva, že $H$ leží na osi vonkajšieho uhla pri vrchole $B$ v~trojuholníku $BIG$. Z~toho už dostávame dokazované tvrdenie.}

{%%%%%   MEMO, priklad 4
\def\bibinom#1#2{\left(\!\!\binom{#1}{#2}\!\!\right)}
\def\tbibinom#1#2{\bigl(\!\!\binom{#1}{#2}\!\!\bigr)}
Dvojice $(n,k)$ spĺňajú podmienky zo zadania práve vtedy, keď $(n,k) = (2,1)$, $k\in\{0,n\}$ alebo $n$ a~$k$ sú párne.

Najskôr ukážeme, že dvojice popísané vyššie skutočne spĺňajú podmienky zo zadania. Zrejme $\tbibinom{2}{1} = 2$ a~pre všetky nezáporné celé čísla $n$ máme
$$
\bibinom{n}{0} = \bibinom{n}{n} = \frac{n!!}{n!!\cdot 0!!} = 1.
$$
Napokon, ak $n$ a~$k$ sú párne čísla, označme $n = 2n'$ a $k = 2k'$. Potom
$$
\bibinom{n}{k} = \frac{(2n')!!}{(2k')!!(2(n'-k'))!!} = \frac{2^{n'}n'!}{2^{k'}k'!2^{n'-k'}(n'-k')!} = \binom{n'}{k'},
$$
čo očividne je celé číslo.

Ostáva ukázať, že žiadne ďalšie dvojice zadaniu nevyhovujú.
Nech $n$ a~$k$ sú také celé nezáporné čísla, že $\tbibinom{n}{k}$ je celé číslo. Rozlíšime dva prípady v~závislosti od parity~$n$.

Nech $n$ je nepárne číslo. V~takom prípade $k = 0$ alebo $k=n$. V~opačnom prípade je totiž jedno z~čísel $k$, $n-k$ párne kladné celé číslo a~preto aj menovateľ rozpísaného výrazu $\tbibinom{n}{k}$ je párny, zatiaľ čo čitateľ je nepárny.

Nech $n$ je párne číslo. Na úvod sme ukázali, že v~prípadoch $n=0$ a~$n=2$ môže byť $k$ ľubovoľné nezáporné celé číslo, ktoré spĺňa $k\le n$. Predpokladajme preto, že $n\ge 4$. Označme $n = 2m$ a~nech $k$ je nepárne. Zo symetrie vyplýva $\tbibinom{n}{k} = \tbibinom{n}{n-k}$, takže sa stačí obmedziť na prípady, keď $1 \le k \le m$. Pre $j \in \{1,2,\dots,m-k\}$ je splnená nerovnosť $k+j < k+2j$ a~preto
$$
\prod_{j = 1}^{m-k}(k+j) \leq \prod_{j=1}^{m-k}(k+2j),
$$
pričom rovnosť nastáva pre $m=k$. Ďalej použijeme aj nerovnosť $(k-1)!!\le k!!$, v~ktorej nastáva rovnosť pre $k=1$. Vynásobením posledných dvoch nerovností dostávame
$$
(k-1)!!\frac{m!}{k!} \leq k!! \frac{(2m-k)!!}{k!!} = (2m-k)!!,
$$
pričom rovnosť nastáva práve vtedy, keď $m=k=1$, čo nastáva pre $n=2$. V~tejto časti však predpokladáme, že $n\ge 4$ a~preto v~poslednej nerovnosti rovnosť nikdy nenastáva. Prenásobením $k!!$ a~použitím rovnosti $(k-1)!!k!!=k!$ dostávame $m! < k!!(2m-k)!!$, čo znamená, že podiel
$$
\frac{m!}{k!!(2m-k)!!}
$$
nemôže byť celé číslo. Keďže menovateľ je nepárny, podiel
$$
\frac{2^mm!}{k!!(2m-k)!!} = \frac{n!!}{k!!(2m-k)!!} = \bibinom{n}{k}
$$
taktiež nemôže byť celé číslo. Preto v~prípade, že $n\ge 4$ je párne a~$0< k < n$ je nepárne, riešenie neexistuje.
}

{%%%%%   MEMO, priklad t1
Ukážeme, že najmenšia možná hodnota je $\frac{17}{6}$.

Ak $a=x=1$ a~$b=y=\frac{1}{2}$, potom
$$
	\frac{1}{a+x} = \frac{1}{2}, \quad
		\frac{1}{a+y}= \frac{2}{3}, \quad
		\frac{1}{b+x} = \frac{2}{3},  \quad
		\frac{1}{b+y} = 1.
$$
Podmienky sú splnené a~navyše súčet týchto štyroch zlomkov je
$$
\frac{1}{2}+\frac{2}{3}+\frac{2}{3}+1=\frac{17}{6}
$$
a~teda $\frac{17}{6}$ je naozaj možná hodnota výrazu za splnenia predpokladov.


Nech $a$, $b$, $x$ a~$y$ sú kladné reálne čísla spĺňajúce zadané nerovnosti. Z~prvej a~štvrtej podmienky máme
$$
a+b+x+y=(a+x)+(b+y)\le 2+1=3.
$$
Z~nerovnosti medzi aritmetickým a~harmonickým priemerom dvoch kladných reálnych čísel dostávame
$$
\frac{2}{\frac{1}{a+y}+\frac{1}{b+x}}\le\frac{(a+y)+(b+x)}{2}=\frac{a+b+x+y}{2}\le\frac{3}{2}.
$$
Z~toho úpravou obdržíme nerovnosť
$$
\frac{1}{a+y}+\frac{1}{b+x}\ge\frac{4}{3}.
$$
Pridaním prvej a~štvrtej nerovnosti dostávame
$$
\frac{1}{a+x}+\frac{1}{a+y}+\frac{1}{b+x}+\frac{1}{b+y}\ge \frac{1}{2}+\frac{4}{3}+1=\frac{17}{6}.
$$
}

{%%%%%   MEMO, priklad t2
Ukážeme, že jediné funkcie spĺňajúce podmienky zo zadania sú funkcie v~tvare
$$
f(x) = \left\{
\aligned
  x,& \quad\text{ak $x\ge 0$,} \\
 kx,& \quad\text{ak $x\le 0$,}
\endaligned
\right.
$$
pričom $k$ je ľubovoľné reálne číslo spĺňajúce $k\ge1$.

Nech $f$ je funkcia vyhovujúca zadaniu. Dosadením $x=y$ dostávame
$$
0\ge (f(x^2)-x^2)(f(x)-x). \tag1
$$
Špeciálne pre $x=1$ a~$x=0$ máme $f(0) = 0$ a $f(1) = 1$. Po dosadení $y=1$ do pôvodnej nerovnosti dostávame
$$
2xf(x) \ge f(x^2) + x^2 \tag2
$$
pre všetky reálne čísla~$x$.

Najskôr ukážeme, že $f(x) = x$ pre $x\ge 0$. Uvažujme dva prípady.
\item{$\triangleright$}
Predpokladajme, že $f(x) < x$. Z~nerovnosti \thetag2 dostávame $0 > f(x^2) - x^2$. Použitím tejto nerovnosti dostávame $0 < (f(x^2)-x^2)(f(x)-x)$, čo je v~spore s~nerovnosťou~\thetag1. Preto $f(x) \ge x$ pre všetky $x\ge 0$.
\item{$\triangleright$}
Predpokladajme, že $f(x^2) > x^2$. Z~nerovnosti \thetag2 dostávame $0 < f(x)-x$. Použitím tejto nerovnosti dostávame $0 < (f(x^2)-x^2)(f(x)-x)$, čo je opäť v~spore s~nerovnosťou \thetag1. Preto $f(x^2)\le x^2$ pre všetky $x\ge0$. Dosadením $\sqrt{x}$ za $x$ dostávame $f(x)\le x$ pre všetky $x\ge0$.
\endgraf\noindent
Z~uvedených prípadov vidíme, že jediné riešenie pre $x\ge0$ je $f(x) = x$.

Pozrime sa teraz na prípad, keď $x$ a~$y$ sú záporné čísla. Z~nerovnosti zo zadania a~použitím $f(x) = x$ pre $x\ge0$ dostávame nerovnosť $x^2y + xyf(x) \ge x^2f(y) + x^2y$. Po vydelení tejto nerovnosti záporným číslom $x^2y$ dostávame
$$
\frac{f(x)}{x} \le \frac{f(y)}{y}.
$$
V~ostatnej nerovnosti musí nastať rovnosť vďaka symetrii. Preto $f(x) = kx$ pre nejaké pevne zvolené reálne číslo~$k$. Po dosadení $x=\m1$ a $y =1$ do pôvodnej nerovnosti dostávame podmienku $k\ge1$. Preto riešením pre $x\le 0$ môže byť len funkcia tvaru $f(x) = kx$ pre pevne zvolené $k\ge1$.

Dosadením tohto riešenia do pôvodnej nerovnosti ľahko nahliadneme, že je splnená.
}

{%%%%%   MEMO, priklad t3
Predpokladajme, že políčka šachovnice sú ofarbené striedavo čiernou a~bielou farbou tak, že políčko, na ktorom mravec začína, je čierne. Najskôr ukážeme, že {\smc memo}rované sú práve tie pravouholníky, ktoré majú všetky rohové políčka biele. V~druhej časti určíme počet pravouholníkov s~bielymi rohmi.

Zaveďme súradnicovú sústavu na šachovnici tak, že čierne políčko, na ktorom mravec začína, má súradnice $(1,1)$, jednotková dĺžka je zhodná s~dĺžkou strany jedného políčka a~$x$-ová os je rovnobežná so spodnou stranou šachovnice. Keďže mravec začína a~končí na čiernom políčku, počet čiernych políčok, ktoré mravec počas presunu navštívi, je o~1 viac ako počet bielych políčok, ktoré navštívi. Preto na šachovnici ostane nepárny počet nenavštívených políčok a~bielych bude medzi nimi o~1 viac ako čiernych. Z~toho vyplýva, že {\smc memo}rovaný pravouholník musí mať strany nepárnej dĺžky a~rohy bielej farby.

Ďalej ukážeme, že každý pravouholník spĺňajúci vyššie popísané podmienky je {\smc memo}rovaný. Budeme postupovať indukciou vzhľadom na $K+L$. Pre $K = L = 1$ pozostávajú oba prípustné pravouholníky len z~jedného políčka. Pre oba jednoducho nájdeme cestu, ktorou mravec mohol ísť. Predpokladajme preto, že $K+L\ge3$. Uvažujme ľubovoľný prípustný pravouholník. Nech jeho ľavý dolný roh má súradnice $(a,b)$. Ak $a\ge 3$, tak mravec môže na začiatku svojej cesty prejsť ľavé dva stĺpce, čím sa šachovnica zmenší o~dva stĺpce a~mravec sa bude nachádzať na políčku $(3,1)$. Na túto situáciu môžeme použiť indukčný predpoklad. Preto môžeme predpokladať, že $a<3$. Analogicky môžeme predpokladať, že $b<3$. Keďže políčko $(a,b)$ musí mať bielu farbu, ostávajú prípady $(a,b)\in\{(1,2),(2,1)\}$. Vďaka symetrii sa môžeme obmedziť na prípad $a = 2$ a~$b = 1$. Podobnou úvahou ukážeme, že pre súradnice pravého horného rohu uvažovaného pravouholníka stačí uvažovať možnosti $(2L-1,2K)$ a~$(2L,2K-1)$. Aby však pravouholník mal strany nepárnej dĺžky, pravý horný roh musí byť $(2L,2K-1)$. V~takomto prípade ľahko nahliadneme, že stačí, aby mravec prešiel ľavý stĺpec z~políčka $(1,1)$ na políčko $(1,2K)$ a~potom horný riadok z~políčka $(1,2K)$ na políčko $(2L,2K)$.

Ostáva určiť počet pravouholníkov s~bielymi rohmi. Každý taký pravouholník je jednoznačne určený jeho ľavým dolným rohom $(a,b)$ a~pravým horným rohom $(c,d)$. Navyše čísla $a$, $b$, $c$ a~$d$ musia spĺňať $1\le a\le c\le 2L$, $1\le b\le d\le 2K$, čísla $a$ a~$c$ majú rovnakú paritu a~čísla $b$ a~$d$ majú opačnú paritu ako číslo~$a$. V~prípade, že číslo~$a$ je nepárne, existuje $\binom{L+1}{2}$ možností pre dvojice $(a,c)$ a~nezávisle na tom $\binom{K+1}{2}$ možností pre párne dvojice $(b,d)$. V~prípade, že $a$ je párne, dostaneme rovnako veľa prípustných pravouholníkov. Preto počet všetkých {\smc memo}rovaných pravouholníkov je
$$
2\binom{K+1}{2}\binom{L+1}{2} = \frac{K(K+1)L(L+1)}{2}.
$$
}

{%%%%%   MEMO, priklad t4
Každému šťastnému obyvateľovi Happy City priraďme číslo $\m1$ a~každému nešťastnému $\p1$. Takéto priradenie nám umožňuje jednoducho popísať zmenu v~šťastí obyvateľa: Ak sa obyvateľ~$A$ s~číslom~$a$ usmeje na obyvateľa~$B$ s~číslom~$b$, nové číslo obyvateľa~$B$ bude $ab$.

Uvažujme situáciu, keď obyvatelia majú priradenú postupnosť čísel $u_1$, $u_2$, \dots, $u_{2014}$. Na konci dňa sa postupnosť priradených čísel zmení na $v_1$, $v_2$, \dots, $v_{2014}$, pričom $v_i = u_1u_2\dots u_i = v_{i-1}u_i$. Z~toho dostávame $u_i = v_iv_{i-1}$.

Označme $a_1$, $a_2$, \dots, $a_{2014}$ postupnosť čísel priradených vo štvrtok večer, $b_1$, $b_2$, \dots, $b_{2014}$ v~stredu večer, $c_1$, $c_2$, \dots, $c_{2014}$ v~utorok večer, $d_1$, $d_2$, \dots, $d_{2014}$ v~pondelok večer a~$e_1$, $e_2$, \dots, $e_{2014}$ v~pondelok ráno. Pre jednoduchosť zápisu definujme $a_i = b_i = c_i = d_i = e_i = 1$ pre všetky celé čísla $i\le 0$. Pre všetky $i\in\{1,2,\dots,2014\}$ dostávame
$$
\align
b_i &= a_ia_{i-1},                                       \\
c_i &= b_ib_{i-1} = a_ia_{i-1}^2a_{i-2} = a_ia_{i-2},    \\
d_i &= c_ic_{i-1} = a_ia_{i-1}a_{i-2}a_{i-3},            \\
e_i &= d_id_{i-1} = a_ia_{i-4}.
\endalign
$$

Nech $x_j$ označuje počet jednotiek v~postupnosti $a_j$, $a_{j+4}$, $a_{j+8}$, \dots Použitím tohto označenia vieme zapísať počet nešťastných obyvateľov vo štvrtok večer ako $x_1 + x_2 + x_3 + x_4 = 14$.

Pozrime sa teraz na to, koľko obyvateľov mohlo mať v~pondelok ráno priradené číslo~$\m1$. Keďže $e_i = a_ia_{i-4}$, môže $e_i$ nadobúdať hodnotu $\m1$ len v~prípade, že práve jedno z~čísel $a_i$, $a_{i-4}$ je rovné~$1$. Preto pre $j=1,2,3,4$, počet členov postupnosti $e_j$, $e_{j+4}$, $e_{j+8}$, \dots rovných $\m1$ môže byť maximálne $2x_j + 1$, pričom $2x_j$ členov prislúcha nešťastným obyvateľom vo štvrtok večer a~jeden obyvateľ navyše je v~prípade, že $a_j = \m1$ a~$j-4 \le 0$. Preto v~pondelok ráno mohlo byť maximálne $2x_1 + 1 +2x_2 + 1 + 2x_3 + 1 + 2x_4 + 1 = 32$ šťastných obyvateľov.

Posledným krokom riešenia je ukázať situáciu, v~ktorej bolo v~pondelok ráno 32~šťastných obyvateľov a~ktorá je v~súlade so zadaním. Uvažujme prípad $a_8 = a_{16} =\dots=a_{8\cdot 14} = 1$ a~$a_i = \m1$ pre všetky ostatné $i$. To znamená, že vo štvrtok večer bolo práve 14 nešťastných a~teda 2000 šťastných obyvateľov. Jednoducho môžeme overiť, že takúto situáciu vo štvrtok večer môžeme dostať zo situácie v~pondelok ráno, pri ktorej $e_1 = e_2 = e_3 = \m1$, $e_4 = e_8 = \dots = e_{4\cdot 29} = \m1$ a~$e_i = 1$ pre ostatné~$i$. V~tomto prípade je počet šťastných obyvateľov v~pondelok ráno $3+29 = 32$.
}

{%%%%%   MEMO, priklad t5
Zrejme bod~$X$ leží medzi bodmi $A$ a~$Y$ (\obr). Veľkosti vnútorných uhlov trojuholníka $ABC$ označme ako zvyčajne $\alpha$, $\beta$, $\gamma$. Platí
$$
|\uhol EXI|=|\uhol DEC|-|\uhol XAE|=\left(90^\circ-\tfrac12{\gamma}\right)-\tfrac12{\alpha}=\tfrac12{\beta}=|\uhol DBI|,
$$
odkiaľ dostávame, že štvoruholník $BDXI$ je tetivový, a~teda
$$
|\uhol AXB|=|\uhol IXB|=|\uhol IDB|= 90^\circ=|\uhol AZB|.
$$
Z~toho vyplýva, že aj štvoruholník $ABZX$ je tetivový. Odtiaľ $|\uhol ZXY|=\beta$ a~keďže $|\uhol DXY|=|\uhol EXI|=\frac12\beta$, dokázali sme, že bod~$D$ leží na osi uhla $ZXY$. Podobne možno dokázať, že $D$ leží na osi uhla $XYZ$. Z~toho už vyplýva dokazované tvrdenie.
\insp{memo.3}%

\poznamka
Alternatívny spôsob, ako dokončiť riešenie úlohy po dokázaní, že štvoruholník $ABZX$ je tetivový, je nasledovný: Vieme, že $|\uhol DZX|=\frac12{\alpha}$ a~podobne $|\uhol YZD|=\frac12{\alpha}$. Takže $ZD$ je os uhla $YZX$. Ostáva len overiť, že $|\uhol YDX|=90^\circ+\frac{1}{2}|\uhol YZX|$. Vieme, že $|\uhol YZX|=\alpha$ a
$$
|\uhol YDX|=|\uhol FDB|+|\uhol CDE|=\left(90^\circ-\tfrac12{\beta}\right)+\left(90^\circ-\tfrac12{\gamma}\right)=90^\circ+\tfrac12{\alpha}.
$$}

{%%%%%   MEMO, priklad t6
Bez ujmy na všeobecnosti predpokladajme, že $|AB|\le|AC|$. Označme $\omega$ kružnicu pripísanú k~strane~$BC$ trojuholníka $ABC$ a~$E$ priesečník priamky~$AD$ a~kružnice~$\omega$, ktorý je ďalej od $D$. Rovnoľahlosť so stredom~$A$, ktorá zobrazí $k$ na $\omega$, zobrazí bod~$D$ na bod~$E$. Navyše priamka~$EK$ je kolmá na stranu~$BC$ a~pretína ju v~bode~$J$, ktorý je dotykovým bodom $\omega$ s~$BC$. Keďže bod~$K$ je stredom $JE$ a~-- ako vieme \footnote{Je známe, že pri štandardnom označení dĺžok strán trojuholníka $ABC$ platí $|BD|=|CJ|=s-b$, pričom $s=\frac12(a+b+c)$.} -- bod~$M$ je stredom $DJ$, sú priamky $MK$ a~$DE$ rovnobežné (\obr).
\insp{memo.4}%

Veďme bodom~$N$ rovnobežku s~$BC$ a~jej priesečník s~$DE$ označme~$F$. Potom štvoruholník $DFNM$ je rovnobežník. Aplikovaním Tálesovej vety na pravouhlý trojuholník $MKJ$ dostávame $|JN|=|MN|=|DF|$. Z~toho vyplýva, že $DFNJ$ je rovnoramenný lichobežník, a~keďže $|BD|=|JC|$, aj $BFNC$ je rovnoramenný lichobežník. Z~toho dôvodu je tiež tetivovým štvoruholníkom a~stačí ukázať, že body $B$, $C$, $F$, $L$ ležia na jednej kružnici. Urobíme to tak, že dokážeme rovnosť $|DB| \cdot |DC| = {|DL| \cdot |DF|}$.

Nech $I$ je stred kružnice vpísanej trojuholníku $ABC$ a~$|\uhol ABC|=\beta$. Pravouhlé trojuholníky $BDI$ a~$KJB$ majú ostré uhly s~veľkosťami $\frac12\beta$ a~$90^\circ-\frac12\beta$, sú teda podobné. Preto
$$
|DI| \cdot |JK| = |DB| \cdot |JB| = |DB| \cdot |DC|
$$
a~ostáva overiť, že $|DI| \cdot |JK| = |DL| \cdot |DF|$.

Keďže $AE\parallel MK$ a~$ID\parallel JK$, dostávame $|\uhol IDL|=|\uhol JED|=|\uhol JKM|$, preto rovnoramenné trojuholníky $ILD$ a~$NKJ$ sú podobné a~tým je dokázané, že  $|DI|\cdot |JK| =|DL|\cdot |JN|$. Už skôr sme ukázali, že $|JN|=|DF|$, z~čoho dostávame požadovanú rovnosť.
}

{%%%%%   MEMO, priklad t7
Označme $S_n$ hľadaný najmenší možný súčet prvkov priemerovej $n$-prvkovej množiny. Dokážeme, že $S_n=n+\frac{1}{2}n(n+1)D_n$, pričom $D_1=D_2=2$ a~pre $n\ge3$ je $D_n$ najmenším spoločným násobkom prvkov množiny $\{1,\dots,n-1\}$.

\smallskip
Príkladom priemerovej množiny, ktorej súčet prvkov nadobúda uvedenú hodnotu, je množina
$$
\mm A_n=\{1+j\cdot D_n\mid j=0,1,\dots,n-1\}.
$$
Naozaj, nech $1+j_1D_n,\dots,1+j_kD_n$ je nejakých jej $k$ rôznych prvkov. Ak $1\le k<n$, tak ich aritmetický priemer
$$
\frac{(1+j_1D_n)+\dots+(1+j_kD_n)}{k}=1+(j_1+\dots+j_k)\cdot\frac{D_n}{k}
$$
je celé číslo, pretože $\frc{D_n}{k}$ je celé číslo. Ak $k=n$, tak ich aritmetický priemer je
$$
\frac{(1+j_1D_n)+\dots+(1+j_kD_n)}{k}=1+(n-1)\cdot\frac{D_n}{2},\tag1
$$
čo je celé číslo, pretože $D_n$ je vždy párne. Súčet prvkov množiny~$\mm A_n$ je $n$-násobkom výrazu \thetag1, čiže práve $n+\frac{1}{2}n(n-1)D_n$.

\smallskip
Uvažujme teraz ľubovoľnú $n$-prvkovú priemerovú množinu $\mm A=\{a_1,\dots,a_n\}$, pričom $a_1<a_2<\dots<a_n$. Dokážeme, že súčet jej prvkov je aspoň $n+\frac{1}{2}n(n-1)D_n$.

Pre $n=1$ a~$n=2$ to triviálne platí, preto predpokladajme, že $n\ge3$. Použijeme nasledovné pomocné tvrdenie: Ak prirodzené čísla $i$, $j$ spĺňajú $1\le i<j\le n$, tak $a_i\equiv a_j\pmod{D_n}$.

\dokaz
Uvažujme ľubovoľné celé číslo~$k$ spĺňajúce $1\le k<n$ a~nech $r_1,\dots,r_{k-1}$ je ľubovoľných $k-1$ rôznych indexov z~množiny $\{1,2,\dots,n\}-\{i,j\}$. Keďže $\mm A$ je priemerová, súčty $a_i+a_{r_1}+\dots+a_{r_{k-1}}$ a~$a_j+a_{r_1}+\dots+a_{r_{k-1}}$ sú deliteľné číslom~$k$ a~teda je ním deliteľný aj ich rozdiel $a_i-a_j$. To dokazuje, že $a_i-a_j$ je násobkom každého z~čísel $1, 2, \dots, n-1$ a~teda tiež násobkom ich najmenšieho spoločného násobku $D_n$.

\smallskip
Z~uvedeného tvrdenia máme $a_1\equiv a_2\equiv \dots\equiv a_n\pmod{D_n}$, preto existujú celé čísla $0=j_1<j_2<\dots < j_n$ také, že $a_i=a_1+j_i\cdot D_n$. Odtiaľ dostávame
$$
\align
a_1+a_2+\dots+a_n&= na_1+(j_1+j_2+\dots+j_n)D_n \ge \\
 &\ge n+(0+1+\dots+(n-1))D_n=n+\frac{1}{2}n(n-1)D_n,
\endalign
$$
čo sme chceli dokázať.}

{%%%%%   MEMO, priklad t8
Ukážeme, že jediným riešením je štvorica $(x,y,z,t)=(1,1,3,1)$.

Predpokladajme, že $(x,y,z,t)$ je jedno z~riešení. V~prípade, že $z=t=1$, dostávame upravenú rovnicu $20^x+14^{2y}=x+2y+1$. Pritom $20^x>2^x\ge x+1$ a~$14^{2y}>2y+1$. Sčítaním a~využitím vyššie upravenej rovnice dostávame $x+2y+1> (x+1)+(2y+1)$, čo je spor. Preto $zt>1$.

Ak by $x$ bolo párne, mali by sme
$$
20^x+14^{2y}\equiv (-1)^x+(-1)^{2y}\equiv 2\pmod{3}.
$$
Zároveň, aby bola splnená rovnosť (keďže ľavá strana je vždy párna), muselo by aj $z$ byť párne. V~tom prípade by však pravá strana bola štvorcom a~jej zvyšok po delení troma by nemohol byť 2.

Preto $x$, a~teda aj $z$, sú nepárne. Z~toho dostávame
$$
20^x+14^{2y}\equiv (-1)^x+(-1)^{2y}\equiv (-1)+1\equiv 0\pmod{3},
$$
čiže $x+2y+z$ je deliteľné tromi. Keďže $zt>1$, máme
$$
0\equiv 20^x+196^{y}\equiv 2^x+7^{y}\pmod{9},
$$
z~čoho vyplýva
$$
2^{x+2y}\equiv 2^x\cdot4^y\equiv(-7^y)\cdot4^y\equiv -28^y\equiv -1\pmod{9}.
$$
Vyšetrením zvyškov mocnín čísla~$2$ po delení deviatimi ľahko nahliadneme, že na platnosť predošlej kongruencie je nutné, aby $x+2y$ bolo deliteľné tromi, čiže aj $z=(x+2y+z)-(x+2y)$ musí byť deliteľné tromi.

Poznamenajme, že $x$ ani $y$ nemôžu byť deliteľné tromi. V~opačnom prípade by totiž museli byť obe deliteľné tromi a~dostali by sme netriviálne riešenie Fermatovej rovnice $a^3+b^3=c^3$, ktoré, ako vieme, neexistuje. K~rovnakému záveru možno dospieť aj bez použitia uvedeného poznatku, a~to analýzou rovnice modulo 13: Ak by bolo $\frc{x}{3}$ nepárne celé číslo, mali by sme
$$
20^x+14^{2y}\equiv \left(20^3\right)^{\frc x3}+1^{2y}\equiv 5^{\frc{x}{3}}+1\equiv 6\ \text{alebo} \ 9\pmod{13}.
$$
Na druhej strane, štvorec čísla môže po delení číslom~$13$ dávať iba zvyšky $0$, $1$, $5$, $8$ a~$12$.

Najskôr predpokladajme, že $x\ne y$ a~označme $a=\min\{x,y\}$. Všimnime si, že $2^{2x}$ je najväčšou mocninou dvoch, ktorá delí $20^x$, zatiaľ čo $2^{2y}$ je najväčšou mocninou dvoch, ktorá delí $14^{2y}$. Z~toho vyplýva, že $2^{2a}$ je najväčšou mocninou dvoch, ktorá delí $20^x+14^{2y}$. Vzhľadom na to, že $3\mid z$, je nutne aj $a$ deliteľné tromi, čo je v~spore s~predošlým zistením, že žiadne z~čísel $x$, $y$ nie je deliteľné tromi. Nutne teda $x=y$.

Využitím všetkých doterajších záverov dostávame
$$
(3x+z)^{zt}=20^x+14^{2x}=4^x(5^x+49^x)=4^x\cdot54\cdot A,
$$
pričom
$$
A= 5^{x-1}-49\cdot5^{x-2}+49^2\cdot5^{x-3}-\dots+49^{x-1},
$$
čiže
$$
A\equiv(-1)^{x-1}-1\cdot(-1)^{x-2}+\dots+1\equiv1+1+\dots+1\equiv x\not\equiv 0\pmod{3}.
$$
Číslo $(3x+z)^{zt}=4^x\cdot54\cdot A=2^{2x+1}\cdot3^3\cdot A$ nie je deliteľné $3^4$. Preto $zt<4$ a~keďže $3\mid z$, nutne $z=3$ a~$t=1$.

Ak by bolo $x>1$, dostali by sme
$$
27(x+1)^3=(3x+3)^3=20^x+14^{2x}>14^{2x}=14^x\cdot14^x>14^2\cdot8^x>27\cdot(2^x)^3\ge27(x+1)^3,
$$
čo je očividný spor. Takže $x=y=1$ a~jediným kandidátom na riešenie je štvorica $(1,1,3,1)$. Táto štvorica naozaj vyhovuje -- hodnota na oboch stranách zadanej rovnice je pre ňu $216$.} 