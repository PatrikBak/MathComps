{%%%%%   A-I-1
Číslo~$n$ je súčinom troch (nie nutne rôznych) prvočísel.
Keď zväčšíme každé z~nich o~$1$, zväčší sa ich súčin o~$963$.
Určte pôvodné číslo~$n$.}
\podpis{Pavel Novotný}

{%%%%%   A-I-2
Pre ľubovoľné kladné reálne čísla $x$, $y$, $z$ dokážte
nerovnosť
$$
(x+y+z)\Bigl(\frac{1}{x}+\frac{1}{y}+\frac{1}{z}\Bigr)\leqq m^2,\quad
\text{pričom}\ m=\min\Bigl(\frac{x}{y}+\frac{y}{z}+\frac{z}{x},
\frac{y}{x}+\frac{z}{y}+\frac{x}{z}\Bigr).
$$
Zistite tiež, kedy v~dokázanej nerovnosti nastane rovnosť.}
\podpis{Jaroslav Švrček, Jaromír Šimša}

{%%%%%   A-I-3
Označme $I$ stred kružnice vpísanej do daného trojuholníka
$ABC$. Predpokladajme, že kolmica na priamku~$CI$
vedená bodom~$I$ pretína priamku~$AB$ v~bode~$M$.
Dokážte, že kružnica opísaná trojuholníku $ABC$ pretína
úsečku~$CM$ v~jej vnútornom bode~$N$ a~že priamky $NI$ a~$MC$
sú navzájom kolmé.}
\podpis{Peter Novotný}

{%%%%%   A-I-4
Označme $l(n)$ najväčšieho nepárneho deliteľa čísla~$n$. Určte
hodnotu súčtu
$$
l(1)+l(2)+l(3)+\dots+l(2^{2013}).
$$}
\podpis{Michal Rolínek}

{%%%%%   A-I-5
Koľkými rôznymi spôsobmi možno vydláždiť plochu $3\times 10$
dlaždicami $2\times1$, ak je dovolené klásť ich v~oboch navzájom
kolmých smeroch?}
\podpis{Stanislava Sojáková}

{%%%%%   A-I-6
V~rovine daného trojuholníka $ABC$ určte všetky body, ktorých
obrazy v~osových súmernostiach podľa priamok $AB$, $BC$, $CA$
tvoria vrcholy rovnostranného trojuholníka.}
\podpis{Pavel Calábek}

{%%%%%   B-I-1
Každému vrcholu pravidelného 63-uholníka priradíme jedno z~čísel $1$ alebo $\m1$.
Ku každej jeho strane
pripíšeme súčin čísel v~jej vrcholoch a~všetky čísla pri jednotlivých stranách
sčítame. Nájdite najmenšiu možnú nezápornú hodnotu takého súčtu.}
\podpis{Pavel Calábek}

{%%%%%   B-I-2
Určte všetky dvojice $(x, y)$ reálnych čísel, pre ktoré platí nerovnosť
$$
(x + y)\Bigl(\frac1x+\frac1y\Bigr)\ge\Bigl(\frac xy+\frac yx\Bigr)^{\!2}.
$$}
\podpis{Jaroslav Švrček}

{%%%%%   B-I-3
Nech $D$ je ľubovoľný vnútorný bod strany $AB$ trojuholníka $ABC$. Na polpriamkach $BC$ a~$AC$
zvoľme postupne body $E$ a~$F$ tak, aby platilo $|BD| = |BE|$ a~$|AD| = |AF|$.
Dokážte, že body $C$, $E$, $F$ a~stred~$I$ kružnice vpísanej trojuholníku $ABC$ ležia
na jednej kružnici.}
\podpis{Jaroslav Švrček}

{%%%%%   B-I-4
Dana napísala na papier trojciferné číslo, ktoré po delení siedmimi dáva zvyšok~$2$.
Prehodením prvých dvoch cifier vzniklo trojciferné číslo, ktoré po delení siedmimi dáva
zvyšok~$3$. Číslo, ktoré vznikne prehodením posledných dvoch cifier pôvodného čísla, dáva po delení siedmimi
zvyšok~$5$. Aký zvyšok po delení siedmimi bude mať číslo, ktoré vznikne prehodením prvej
a~poslednej cifry Daninho čísla?}
\podpis{Pavel Novotný}

{%%%%%   B-I-5
V~rovine sú dané body $A$, $T$, $U$ tak, že uhol $ATU$ je tupý.
Zostrojte trojuholník $ABC$, v ktorom $T$, $U$ sú postupne body dotyku strany~$BC$
s~kružnicou trojuholníku vpísanou a~pripísanou.
(Pripísanou kružnicou tu rozumieme kružnicu, ktorá sa okrem strany~$BC$
dotýka aj polpriamok opačných k~polpriamkam $BA$ a~$CA$.)}
\podpis{Šárka Gergelitsová}

{%%%%%   B-I-6
Nájdite najmenšie reálne číslo $r$ také, že tyč s~dĺžkou~$1$ možno rozlámať na štyri
časti dĺžky nanajvýš~$r$ tak, že sa zo žiadnych troch týchto častí nedá zložiť trojuholník.}
\podpis{Ján Mazák}

{%%%%%   C-I-1
Určte, akú najmenšiu hodnotu môže nadobúdať výraz $V = (a - b)^2 + (b - c)^2 + (c - a)^2$,
ak reálne čísla $a$, $b$, $c$ spĺňajú dvojicu podmienok
$$
\align
a~+ 3b + c &= 6, \\
-a + b - c &= 2.
\endalign
$$}
\podpis{Jaroslav Švrček}

{%%%%%   C-I-2
V~rovine sú dané body $A$, $P$, $T$ neležiace na jednej priamke. Zostrojte trojuholník $ABC$ tak,
aby $P$ bola päta jeho výšky z~vrcholu~$A$ a~$T$ bod dotyku strany~$AB$ s~kružnicou
jemu vpísanou.
Uveďte diskusiu o~počte riešení vzhľadom na polohu daných bodov.}
\podpis{Pavel Leischner}

{%%%%%   C-I-3
Číslo~$n$ je súčinom troch rôznych prvočísel. Ak zväčšíme dve menšie z~nich o~$1$ a~najväčšie
ponecháme nezmenené, zväčší sa ich súčin o~$915$. Určte číslo~$n$.}
\podpis{Pavel Novotný}

{%%%%%   C-I-4
Vo štvorci $ABCD$ označme $K$ stred strany~$AB$ a~$L$ stred strany~$AD$. Úsečky
$KD$ a~$LC$ sa pretínajú v~bode~$M$ a~rozdeľujú štvorec na dva trojuholníky
a~dva štvoruholníky. Vypočítajte ich obsahy, ak úsečka~$LM$ má dĺžku 1\,cm.}
\podpis{Leo Boček}

{%%%%%   C-I-5
Dokážte, že pre každé nepárne prirodzené číslo~$n$ je súčet $n^4 + 2n^2 + 2\,013$
deliteľný číslom~$96$.}
\podpis{Jaromír Šimša}

{%%%%%   C-I-6
Šachového turnaja sa zúčastnilo 8~hráčov a~každý s~každým odohral jednu partiu.
Za víťazstvo získal hráč 1~bod, za remízu pol bodu, za prehru žiadny bod. Na konci
turnaja mali všetci účastníci rôzne počty bodov. Hráč, ktorý skončil na 2.~mieste,
získal rovnaký počet bodov ako poslední štyria dokopy. Určte výsledok partie
medzi 4. a~6.~hráčom v~celkovom poradí.}
\podpis{Vojtech Bálint}

{%%%%%   A-S-1
Dokážte, že pre každé celé číslo $n\ge3$ je $2n$-ciferné číslo
s~dekadickým zápisom
$$
\underbrace{1\dots1}_{n-1}2\underbrace{8\dots8}_{n-2}96
$$
druhou mocninou niektorého celého čísla.}
\podpis{Vojtech Bálint}

{%%%%%   A-S-2
Označme $M$ stred strany~$AB$ ľubovoľného trojuholníka $ABC$. Dokážte, že rovnosť
$|\uhol ABC|+|\uhol ACM|=90\st$ platí práve vtedy, keď je trojuholník $ABC$ rovnoramenný
so základňou~$AB$ alebo pravouhlý s~preponou~$AB$.}
\podpis{Pavel Novotný}

{%%%%%   A-S-3
Dĺžky strán pravouholníka sú celé čísla $x$ a~$y$ väčšie ako~$1$.
V~pravouholníku vyznačíme rozdelenie na $x\cdot y$ jednotkových štvorcov
a~potom z~neho zvinutím a~zlepením dvoch protiľahlých strán
zhotovíme plášť rotačného valca.
Každé dva vrcholy jednotkových štvorcov na plášti spojíme úsečkou.
Koľko z~týchto úsečiek prechádza vnútornými bodmi tohto valca?
V~prípade $x>y$ rozhodnite, kedy bude tento počet väčší~--
keď bude obvod podstavy valca rovný~$x$, alebo~$y$?}
\podpis{Vojtech Bálint}

{%%%%%   A-II-1
Nájdite všetky celé kladné čísla, ktoré nie sú mocninou čísla~$2$
a~ktoré sa rovnajú súčtu trojnásobku svojho najväčšieho nepárneho
deliteľa a~päťnásobku svojho najmenšieho nepárneho deliteľa väčšieho
ako~$1$.}
\podpis{Tomáš Jurík}

{%%%%%   A-II-2
V~rovine sú dané dve kružnice $k_1(S_1,r_1)$ a~$k_2(S_2,r_2)$,
pričom $|S_1S_2|>r_1+r_2$. Nájdite množinu všetkých bodov~$X$, ktoré
neležia na priamke $S_1S_2$ a~majú tú vlastnosť, že úsečky $S_1X$,
$S_2X$ pretínajú postupne kružnice $k_1$, $k_2$ v~bodoch, ktorých
vzdialenosti od priamky~$S_1S_2$ sa rovnajú.
}
\podpis{Jaromír Šimša}

{%%%%%   A-II-3
Nájdite všetky trojice reálnych čísel $x$, $y$ a~$z$, pre ktoré
platí
$$
x\bigl(y^2+2z^2\bigr)=y\bigl(z^2+2x^2\bigr)=z\bigl(x^2+2y^2\bigr).
$$
}
\podpis{Michal Rolínek}

{%%%%%   A-II-4
Volejbalového turnaja sa zúčastnilo šesť družstiev, každé hralo proti každému
práve raz. V~jednotlivých piatich kolách prebiehali
v~tom istom čase vždy tri zápasy na troch kurtoch 1, 2 a~3.
Koľko bolo možností pre rozpis takého turnaja?
Rozpisom rozumieme tabuľku $3\times5$, v~ktorej
pre $i\in\{1,2,3\}$ a~$j\in\{1,2,3,4,5\}$
je na pozícii~$(i,j)$ uvedená dvojica družstiev (bez určenia poradia), ktoré hrali
proti sebe v~$j$-tom kole na kurte číslo~$i$. Namiesto dekadického
zápisu výsledného čísla stačí uviesť jeho rozklad na súčin
prvočísel.}
\podpis{Martin Panák}

{%%%%%   A-III-1
Nech $n$ je celé kladné číslo. Označme všetky jeho kladné
delitele $d_1,d_2,\dots,d_k$ tak, aby platilo
$d_1<d_2<\dots<d_k$ (čiže $d_1=1$ a~$d_k=n$). Určte všetky
také hodnoty~$n$, pre ktoré platí $d_5-d_3=50$ a~$11d_5+8d_7=3n$.}
\podpis{Matúš Harminc}

{%%%%%   A-III-2
V~rovine, v~ktorej je daná úsečka~$AB$, uvažujme trojuholníky $XYZ$
také, že $X$ je vnútorným bodom úsečky~$AB$, trojuholníky $XBY$ a~$XZA$
sú podobné ($\triangle XBY\sim\triangle XZA$)
a~body $A$, $B$, $Y$, $Z$ ležia v~tomto poradí na
kružnici. Nájdite množinu stredov všetkých úsečiek~$YZ$.}
\podpis{Michal Rolínek, Jaroslav Švrček}

{%%%%%   A-III-3
Majme šachovnicu $8\times8$ a~ku každej "hrane", ktorá
oddeľuje dve jej políčka, napíšme prirodzené číslo, ktoré udáva počet
spôsobov, ako možno celú šachovnicu
rozrezať na obdĺžničky
$2\times1$ tak, aby dotyčná hrana bola
súčasťou rezu.
Určte poslednú cifru súčtu všetkých takto napísaných čísel.}
\podpis{Michal Rolínek}

{%%%%%   A-III-4
Do kina prišlo 234 divákov. Určte, pre ktoré $n\ge4$ sa mohlo stať,
že divákov bolo možné rozsadiť do $n$~radov tak, aby každý divák
v~$i$-tom rade sa poznal práve s~$j$~divákmi v~$j$-tom rade pre
ľubovoľné $i,j\in\{1,2,\dots,n\}$, $i\ne j$.
(Vzťah známosti je vzájomný.)}
\podpis{Tomáš Jurík}

{%%%%%   A-III-5
Daný je ostrouhlý trojuholník $ABC$. Označme $k$ kružnicu s~priemerom~$AB$. Kružnica, ktorá sa dotýka osi uhla $BAC$ v~bode~$A$
a~prechádza bodom~$C$, pretína kružnicu~$k$ v~bode~$P$, $P\ne A$.
Kružnica, ktorá sa dotýka osi uhla $ABC$ v~bode~$B$
a~prechádza bodom~$C$, pretína kružnicu $k$ v~bode $Q$, $Q\ne B$.
Dokážte, že priesečník priamok $AQ$ a~$BP$ leží na osi uhla $ACB$.}
\podpis{Peter Novotný}

{%%%%%   A-III-6
Pre ľubovoľné nezáporné reálne čísla $a$ a~$b$ dokážte nerovnosť
$$
\frac{a}{\sqrt{b^2+1}}+\frac{b}{\sqrt{a^2+1}}\ge\frac{a+b}{\sqrt{ab+1}}
$$
a~zistite, kedy nastáva rovnosť.}
\podpis{Tomáš Jurík, Jaromír Šimša}

{%%%%%   B-S-1
V~obore reálnych čísel vyriešte rovnicu $$2^{|x+1|} - 2^x = 1 + |2^x - 1|.$$}
\podpis{Vojtech Bálint}

{%%%%%   B-S-2
Množina~$\mm M$ obsahuje 2014 rôznych reálnych čísel. Súčet každých dvoch rôznych čísel
z~množiny~$\mm M$ je celé číslo.
\ite a) Rozhodnite, či existuje taká množina~$\mm M$, ktorá neobsahuje žiadne celé číslo.
\ite b) Rozhodnite, či existuje taká množina~$\mm M$, ktorá obsahuje iracionálne číslo.}
\podpis{Ján Mazák}

{%%%%%   B-S-3
Na priamke~$a$, na ktorej leží strana~$BC$ trojuholníka $ABC$, sú dané body dotyku
všetkých troch jemu pripísaných kružníc (body $B$ a~$C$ nie sú známe).
Nájdite na tejto priamke bod dotyku kružnice vpísanej.}
\podpis{Šárka Gergelitsová}

{%%%%%   B-II-1
V~obore reálnych čísel vyriešte sústavu rovníc
$$
\align
x^2 + 6(y + z) &= 85, \\
y^2 + 6(z + x) &= 85, \\
z^2 + 6(x + y) &= 85.
\endalign
$$
}
\podpis{Pavel Novotný}

{%%%%%   B-II-2
Janko napísal na tabuľu niekoľko rôznych prvočísel (aspoň tri). Keď sčítal
ľubovoľné dve z~nich a~tento súčet zmenšil o~$7$, bolo výsledné číslo medzi napísanými.
Ktoré čísla mohli na tabuli byť?}
\podpis{Tomáš Jurík}

{%%%%%   B-II-3
Nad stranami $BC$ a~$AB$ ostrouhlého trojuholníka $ABC$ sú zvonka zostrojené
polkružnice $k$ a~$l$. Označme postupne $D$ a~$E$ priesečníky výšok z~vrcholov $A$ a~$C$
s~polkružnicami $k$ a~$l$ (výškami rozumieme priamky). Dokážte, že platí $|BE| = |BD|$.}
\podpis{Veronika Hucíková}

{%%%%%   B-II-4
V~každom políčku tabuľky $8\times8$ je napísané jedno nezáporné celé číslo tak, že každé
dve čísla, ktoré sú na políčkach súmerne združených podľa jednej či druhej uhlopriečky, sú
rovnaké. Súčet všetkých 64~čísel je $1\,000$, súčet 16~čísel na uhlopriečkach je~$200$. Dokážte, že
súčet čísel v~každom riadku aj stĺpci tabuľky je nanajvýš~$300$. Platí rovnaký záver aj pre číslo~$299$?}
\podpis{Jaromír Šimša}

{%%%%%   C-S-1
Určte, aké hodnoty môže nadobúdať výraz $V = ab + bc + cd + da$, ak reálne čísla $a$,
$b$, $c$, $d$ spĺňajú dvojicu podmienok
$$
\align
2a - 5b + 2c - 5d &= 4,\\
3a + 4b + 3c + 4d &= 6.
\endalign
$$}
\podpis{Jaroslav Švrček}

{%%%%%   C-S-2
Čísla $1, 2, \dots, 10$ rozdeľte na dve skupiny tak, aby najmenší spoločný násobok súčinu
všetkých čísel prvej skupiny a~súčinu všetkých čísel druhej skupiny bol čo najmenší.}
\podpis{Ján Mazák}

{%%%%%   C-S-3
Daný je trojuholník $ABC$ s~pravým uhlom pri vrchole~$C$. Stredom~$I$ kružnice
trojuholníku vpísanej vedieme rovnobežky so stranami $CA$ a~$CB$, ktoré pretnú preponu
postupne v~bodoch $X$ a~$Y$. Dokážte, že platí $|AX|^2+ |BY|^2= |XY|^2$.}
\podpis{Michal Rolínek}

{%%%%%   C-II-1
Nájdite všetky trojice (nie nutne rôznych) cifier $a$, $b$, $c$ také, že päťciferné čísla
$\overline{6abc3}$ a~$\overline{3abc6}$ sú v~pomere $63 : 36$.}
\podpis{Jaromír Šimša}

{%%%%%   C-II-2
Šachového turnaja sa zúčastnilo 5~hráčov a~každý s~každým odohral jednu partiu. Za prvenstvo
získal hráč 1~bod, za remízu pol bodu, za prehru žiadny bod. Poradie hráčov na turnaji sa určuje
podľa počtu získaných bodov. Jediným ďalším kritériom rozhodujúcim o~konečnom umiestnení
hráčov v~prípade rovnosti bodov je počet výhier (kto má viac výhier, je na tom v~umiestnení lepšie).
Na turnaji získali všetci hráči rovnaký počet bodov. Vojto porazil Petra a~o~prvé miesto sa delil
s~Tomášom.
Ako dopadla partia medzi Petrom a~Martinom?}
\podpis{Martin Panák}

{%%%%%   C-II-3
Pre kladné reálne čísla $a$, $b$, $c$ platí $c^2 + ab = a^2 + b^2$. Dokážte, že potom platí aj
$c^2 + ab \le ac + bc$.}
\podpis{Michal Rolínek}

{%%%%%   C-II-4
Daný je konvexný štvoruholník $ABCD$ s~bodom $E$ vnútri strany~$AB$ tak, že platí
$|\uhel ADE| = |\uhel DEC| = |\uhel ECB|$. Obsahy trojuholníkov $AED$
a~$CEB$ sú postupne $18\cm^2$ a~$8\cm^2$. Určte obsah trojuholníka $ECD$.}
\podpis{Ján Mazák}

{%%%%%   vyberko, den 1, priklad 1
Nech $\mm S$ je konečná množina kladných celých čísel, ktorá má nasledujúcu vlastnosť. Ak $x$ je prvok množiny~$\mm S$, potom aj každý kladný deliteľ čísla~$x$ patrí do množiny~$\mm S$. Neprázdna podmnožina~$\mm T$ množiny~$\mm S$ sa nazýva {\it dobrá}, ak pre každé $x$, $y\in\mm T$, $x>y$ je podiel $x/y$ mocninou prvočísla. Neprázdna podmnožina~$\mm T$ množiny~$\mm S$ sa nazýva {\it zlá}, ak pre žiadne $x$, $y\in\mm T$, $x>y$ nie je podiel $x/y$ mocninou prvočísla. Jednoprvkové podmnožiny sú zároveň dobré aj zlé. Nech $k$ je veľkosť najväčšej dobrej podmnožiny množiny~$\mm S$. Ukážte, že $k$ je taktiež najmenší možný počet po dvoch disjunktných zlých podmnožín množiny~$\mm S$, ktorých zjednotením je $\mm S$.
}
\podpis{Tomáš Kocák, Matúš Stehlík:Balkan Mathematical Olympiad 2011, Problem 3}

{%%%%%   vyberko, den 1, priklad 2
Daný je ostrouhlý trojuholník $ABC$. Nech $B_1$ je bod na strane~$AC$ taký, že $B_1B$ je osou ostrého uhla $ABC$. Kolmica z~bodu~$B_1$ na stranu~$BC$ pretne kratší oblúk~$BC$ kružnice opísanej trojuholníku $ABC$ v~bode~$K$. Kolmica z~bodu~$B$ na $AK$ pretína $AC$ v~bode~$L$. Priamka~$B_1B$ pretína oblúk~$AC$ v~bode~$M$ rôznom od $B$. Dokážte, že body $K$, $L$, $M$ ležia na jednej priamke.
}
\podpis{Tomáš Kocák, Matúš Stehlík:Russian Olympiad 2007, Grade 9, Day 1, Problem 4}

{%%%%%   vyberko, den 1, priklad 3
Nájdite všetky funkcie $f\colon\Bbb R\to\Bbb R$ také, že
$$
f(f(x)-y) = f(x) +  f(f(y)-f(-x))+x
$$
pre všetky $x$, $y\in\Bbb R$.}
\podpis{Tomáš Kocák, Matúš Stehlík:Poland 2008 second round, Problem 3}

{%%%%%   vyberko, den 2, priklad 1
Body $A_1$, $B_1$ a~$C_1$ ležia po rade vo vnútri strán $BC$, $CA$ a~$AB$ trojuholníka $ABC$.
Označme po rade $A_0$, $B_0$ a~$C_0$ priesečníky $BB_1 \cap CC_1$, $CC_1 \cap AA_1$
a~$AA_1 \cap BB_1$. Dokážte, že ak štyri trojuholníky $CB_1A_0$, $AC_1B_0$, $BA_1C_0$ a~$A_0B_0C_0$  majú rovnaký obsah rovný jednej,
tak aj tri štvoruholníky $AB_0A_0B_1$, $BC_0B_0C_1$ a~$CA_0C_0A_1$ majú rovnaký obsah. Nájdite
jeho veľkosť.
}
\podpis{Richard Kollár:}

{%%%%%   vyberko, den 2, priklad 2
Označme $\alpha$, $\beta$, $\gamma$ uhly trojuholníka $ABC$. Dokážte, že ak platí
$$
\frac{\sin \alpha  + \sin \beta + \sin \gamma}{\cos \alpha + \cos \beta + \cos \gamma} = \sqrt{3},
$$
tak potom má jeden z~uhlov trojuholníka $ABC$ veľkosť $60^{\circ}$.
}
\podpis{Richard Kollár:}

{%%%%%   vyberko, den 2, priklad 3
Dokážte, že každé prirodzené číslo~$k$ možno jediným spôsobom vyjadriť v~tvare
$$
k= b_1 \cdot 1! + b_2 \cdot 2! + \dots + b_n \cdot n!,
$$
kde $b_j$, $j = 1, \dots, n$, sú celé čísla, pre ktoré platí
$0 \le b_j \le j$, $b_n \ne 0$.}
\podpis{Richard Kollár:}

{%%%%%   vyberko, den 2, priklad 4
O~postupnosti $a_1, a_2, \dots, a_{2014}$ je známe, že
$$
|a_1| = 1 \qquad \text{a} \qquad |a_{k+1}| = |a_k + 1| \quad \text{pre každé $k = 1, \dots, 2013$.}
$$
Nájdite najmenšiu možnú hodnotu výrazu $|a_1 + a_2 + \dots + a_{2013}|$.
}
\podpis{Richard Kollár:}

{%%%%%   vyberko, den 3, priklad 1
Pre každé prirodzené číslo $k>1$ nájdite najmenšie prirodzené číslo $m>1$ také, že existuje polynóm $P(x)$ s celočíselnými koeficientmi a vlastnosťami:
\item{$\bullet$} $P(x)-1$ má aspoň jeden celočíselný koreň,
\item{$\bullet$} $P(x)-m$ má presne $k$ celočíselných koreňov.
}
\podpis{Michal Hagara, Tomáš Jurík:cinska olympiada 2001, problem 7 (Mathematical olympiad 2000-2001)}

{%%%%%   vyberko, den 3, priklad 2
Na priamke $AC$ trojuholníka $ABC$ zvolíme body $M$ a $N$ tak, aby $|AM|=|AB|$, $|CN|=|BC|$ a body sú na priamke v poradí $M$, $A$, $C$, $N$. Kružnice opísané trojuholníkom $BCM$ a $ABN$ sa pretínajú v bodoch $B$ a $K$. Dokážte, že $BK$ je osou uhla $ABC$.
}
\podpis{Michal Hagara, Tomáš Jurík:Bielorusko 97/98 http://oi.sk/rocenky/rocenka49.pdf}

{%%%%%   vyberko, den 3, priklad 3
Pre nezáporné čísla $a$, $b$, $c$ so súčtom $5$ nájdite maximum výrazu $a^4b+b^4c+c^4a$.
}
\podpis{Michal Hagara, Tomáš Jurík:http://www.fen.bilkent.edu.tr/~cvmath/Problem/0711a.pdf}

{%%%%%   vyberko, den 3, priklad 4
Na nekonečnej šachovnici máme vyznačených 100 políčok, po ktorých sa môže pohybovať veža. Medzi každou dvojicou vyznačených políčok sa dá dostať konečným počtom ťahov vežou. (Vežou môžeme ťahať medzi ľubovoľnými dvoma vyznačenými políčkami v~rovnakom riadku alebo stĺpci.) Dokážte, že vieme vyznačené políčka rozdeliť na 50~dvojíc tak, že políčka každej dvojice ležia v~rovnakom riadku alebo stĺpci.
}
\podpis{Michal Hagara, Tomáš Jurík:ruska olympiada 2001, problem 14 (Mathematical olympiad 2000-2001)}

{%%%%%   vyberko, den 4, priklad 1
Radikálom prirodzeného čísla $N$ (ktorý sa označuje $\text{rad}(N)$) sa nazýva súčin všetkých prvočíselných deliteľov čísla $N$, ktoré sa berú v~prvej mocnine. Napríklad $\text{rad}(120) = 2 \cdot 3 \cdot 5 = 30$. Existuje taká trojica po dvoch nesúdeliteľných prirodzených čísel $A$, $B$, $C$, že platí $A + B = C$ a tiež $C > 1000 \cdot\text{rad}(ABC)$? }
\podpis{Filip Hanzely, Martin Kollár:Moskovská MO 2014}

{%%%%%   vyberko, den 4, priklad 2
Na stranách $AD$ a $CD$ rovnobežníka $ABCD$ so stredom $S$ zvolíme postupne také body $P$, $Q$ aby platilo $|\uhol ASP| = |\uhol CSQ| = |\uhol ABC|$. Dokážte, že
\ite a) uhly $ABP$ a $CBQ$ sú zhodné,
\ite b) priamky $AQ$ a $CP$ sa pretínajú na kružnici opísanej trojuholníku
$ABC$.}
\podpis{Filip Hanzely, Martin Kollár:Moskovská MO 2014}

{%%%%%   vyberko, den 4, priklad 3
Pre reálne po dvoch rôzne $a$, $b$, $c$ dokážte nerovnosť
$$\left|\frac{a+b}{a-b}\right|+\left| \frac{a+c}{a-c} \right| + \left| \frac{c+b}{c-b} \right| \ge 2$$
a zistite kedy nastáva rovnosť.}
\podpis{Filip Hanzely, Martin Kollár:Romanian Stars of Mathematics competition 2012}

{%%%%%   vyberko, den 4, priklad 4
Máme štvorcovú tabuľku $2014 \times 2014$, ktorá je ako torus. Vpíšeme do nej čísla od $1$ po $2014^{2}$, každé práve raz. Nájdite najväčšie také $M$, že vždy existuje dvojica susedných políčok (hranou) líšiaca sa aspoň o~$M$.}
\podpis{Filip Hanzely, Martin Kollár:IMAR Mathematical Competition 2011(Rumunsko)}

{%%%%%   vyberko, den 5, priklad 1
Nájdite všetky funkcie $f\colon\Bbb N\to\Bbb N$ také, že pre všetky $a,b\in \Bbb N$ platí
$$
a^2+f(b)\mid af(a)+b.
$$
(Symbol $\Bbb N$ označuje množinu všetkých kladných celých čísel.)}
\podpis{Filip Sládek, Martin Vodička:Shortlist 2013, N1}

{%%%%%   vyberko, den 5, priklad 2
Daný je trojuholník $ABC$ taký, že $|\angle ABC|>|\angle BCA|$. Nech $P$, $Q$ sú také dva rôzne body na priamke~$AC$, že $|\angle QBA|=|\angle PBA|=|\angle BCA|$ a~$A$ leží medzi $P$ a~$C$. Predpokladajme, že vnútri úsečky~$BQ$ existuje bod~$D$ taký, že $|PB|=|PD|$. Označme $R$ priesečník polpriamky~$AD$ a~kružnice opísanej trojuholníku $ABC$ rôzny od $A$. Dokážte, že $|QB|=|QR|$.}
\podpis{Filip Sládek, Martin Vodička:Shortlist 2013, G4}

{%%%%%   vyberko, den 5, priklad 3
Šialený vedec zostrojil armádu robotov. Problém je v~tom, že niektoré dvojice robotov sa nenávidia (nenávisť je vzájomná). Vždy však s~robotmi vie urobiť jednu z~nasledujúcich dvoch operácií:
\item{(i)} Ak nejaký robot nenávidí nepárny počet robotov, vedec ho môže zničiť.
\item{(ii)} Vedec môže zdvojnásobiť armádu tak, že každý robot~$R$ sa rozdelí na dvoch robotov $R_1$ a~$R_2$. Pre každú dvojicu pôvodných robotov $R$, $Q$, ktorí sa nenávideli, sa budú nenávidieť roboti $R_1$, $Q_1$ aj roboti $R_2$ a~$Q_2$. Roboti $R_1$ a~$R_2$ sa tiež nenávidia pre každého pôvodného robota~$R$. To sú všetky dvojice robotov, ktoré sa budú nenávidieť po zdvojnásobení.

Dokážte, že vedec vie po konečnom počte operácií dostať armádu robotov, v~ktorej neexistuje dvojica robotov, ktorá sa nenávidí.}
\podpis{Filip Sládek, Martin Vodička:Shortlist 2013, C3}

{%%%%%   vyberko, den 1, priklad 4
...}
\podpis{...}

{%%%%%   vyberko, den 5, priklad 4
...}
\podpis{...}

{%%%%%   trojstretnutie, priklad 1
Dokážte, že kladné reálne čísla $a$, $b$, $c$ spĺňajú rovnicu
$$
a^4+b^4+c^4+4a^2b^2c^2=2\bigl(a^2b^2+a^2c^2+b^2c^2\bigr)
$$
práve vtedy, keď existuje trojuholník $ABC$ s~vnútornými uhlami
$\alpha$, $\beta$, $\gamma$ takými, že
$$
\sin\alpha=a,\quad \sin\beta=b\quad\text{a}\quad\sin\gamma=c.
$$
}
\podpis{Jaromír Šimša}

{%%%%%   trojstretnutie, priklad 2
Pre dané kladné celé čísla $a$, $b$, $x_1$ zostavíme
postupnosť čísel $(x_n)_{n=1}^{\infty}$ spĺňajúcich
vzťah $x_n=ax_{n-1}+b$ pre každé $n\ge2$. Určte podmienku na zadané čísla
$a$, $b$ a~$x_1$, ktorá je nutná a postačujúca na to, aby pre
všetky indexy $m$, $n$ platila implikácia ${m\mid n}\Rightarrow x_m\mid
x_n$.
}
\podpis{Jaromír Šimša}

{%%%%%   trojstretnutie, priklad 3
Daný je konvexný štvoruholník $ABCD$, pričom $|\angle ABC| = |\angle ADC| = 135^{\circ}$. Na polpriamkach $AB$, $AD$ sú postupne zvolené také body $M$, $N$, že $|\angle MCD| = |\angle NCB| = 90^{\circ}$. Kružnice opísané trojuholníkom $AMN$ a~$ABD$ sa druhýkrát pretínajú v~bode $K\ne A$. Dokážte, že priamky $AK$ a~$KC$ sú navzájom kolmé.
}
\podpis{Irán}

{%%%%%   trojstretnutie, priklad 4
Daný je trojuholník $ABC$, pričom $P$ je stred strany $AC$. Kružnica $k$ pretína úsečky $AP$, $CP$, $BC$, $AB$ postupne v~ich vnútorných bodoch $K$, $L$, $M$, $N$.
Označme $S$ stred $KL$. Nech platí
$$
2\cdot|AN|\cdot |AB|\cdot |CL| = 2\cdot|CM|\cdot |BC| \cdot |AK| = |AC|\cdot |AK| \cdot |CL|.
$$
Dokážte, že ak $P\ne S$, tak priesečník priamok $KN$ a~$ML$ leží na osi úsečky~$PS$.
}
\podpis{Ján Mazák}

{%%%%%   trojstretnutie, priklad 5
Určte všetky kladné celé čísla $n$ spĺňajúce nasledujúcu podmienku: pre každé nezáporné celé čísla $k$, $m$ také, že $k+m \le n$, dávajú čísla ${n-k} \choose {m}$ a~${n-m} \choose {k}$ rovnaký zvyšok po delení dvoma.
}
\podpis{Poľsko}

{%%%%%   trojstretnutie, priklad 6
Nech $n\ge6$ je celé číslo a~$\Cal{F}$ je systém $3$-prvkových podmnožín množiny $\{ 1, 2, \ldots, n\}$ spĺňajúci nasledujúcu podmienku: pre každé $1 \le i < j \le n$ existuje aspoň $\lfloor \frac13n\rfloor - 1$ podmnožín $\mm A \in \Cal{F}$ takých, že $i, j \in\mm A$. Dokážte, že pre niektoré celé číslo $m\ge1$ existujú navzájom disjunktné podmnožiny $\mm A_1, \mm A_2, \dots, \mm A_m \in \Cal{F}$ také, že
$$
|\mm A_1 \cup \mm A_2 \cup \dots \cup \mm A_m | \ge n-5.
$$
}
\podpis{Poľsko}

{%%%%%   IMO, priklad 1
Nech $a_0<a_1<a_2<\dots$ je nekonečná postupnosť kladných celých čísel.
Dokážte, že existuje práve jedno celé číslo $n\ge1$ také, že
$$
  a_n < \frac{a_0+a_1+\cdots+a_n}n \le a_{n+1}.
$$}
\podpis{Rakúsko}

{%%%%%   IMO, priklad 2
Nech $n\ge2$ je celé číslo. Uvažujme šachovnicu s~rozmermi $n\times n$ pozostávajúcu z~$n^2$~jednotkových štvorcových políčok.
Konfiguráciu $n$~veží na tejto šachovnici nazývame {\it šťastná}, ak každý riadok a~každý stĺpec obsahuje práve jednu vežu. Nájdite najväčšie kladné celé číslo $k$ také, že pre každú šťastnú konfiguráciu $n$~veží existuje štvorec s~rozmermi $k \times k$, ktorý neobsahuje vežu na žiadnom zo svojich $k^2$~políčok.}
\podpis{Chorvátsko}

{%%%%%   IMO, priklad 3
V~konvexnom štvoruholníku $ABCD$ platí $|\angle ABC|=|\angle CDA|=90^\circ$.
Bod~$H$ je pätou kolmice z~bodu~$A$ na priamku~$BD$.
Body $S$, $T$ ležia postupne na stranách $AB$, $AD$,
pričom bod~$H$ je vnútorným bodom trojuholníka $SCT$ a~platí
$$
|\angle CHS|-|\angle CSB|=90^\circ,\quad |\angle THC|-|\angle DTC|=90^\circ.
$$
Dokážte, že priamka~$BD$ sa dotýka kružnice opísanej trojuholníku $TSH$.}
\podpis{Irán}

{%%%%%   IMO, priklad 4
Na strane~$BC$ daného ostrouhlého trojuholníka $ABC$
ležia body $P$, $Q$, pričom $|\angle PAB|=|\angle BCA|$ a~$|\angle CAQ|=|\angle ABC|$.  Body $M$ a~$N$ ležia postupne na priamkach $AP$ a~$AQ$ tak, že bod $P$ je stredom úsečky $AM$ a~bod~$Q$ je stredom úsečky~$AN$. Dokážte, že priamky $BM$ a~$CN$ sa pretínajú na kružnici opísanej trojuholníku $ABC$.}
\podpis{Gruzínsko}

{%%%%%   IMO, priklad 5
Banka v~Kapskom Meste razí mince s~hodnotou~$\frac1n$ pre každé kladné celé číslo~$n$. Majme konečnú kolekciu takýchto mincí (nie nutne s~rôznymi hodnotami), ktorá má celkovú hodnotu nanajvýš $99+\frac12$. Dokážte, že túto kolekciu možno rozdeliť na 100 alebo menej častí tak, aby každá časť mala celkovú hodnotu nanajvýš~$1$.}
\podpis{Luxembursko}

{%%%%%   IMO, priklad 6
Hovoríme, že priamky v~rovine sú vo {\it všeobecnej polohe}, ak žiadne dve nie sú rovnobežné a~žiadne tri neprechádzajú jedným bodom. Množina priamok vo všeobecnej polohe rozdeľuje rovinu na oblasti, z~ktorých niektoré majú konečný obsah; nazývame ich {\it konečné oblasti\/} prislúchajúce danej množine priamok. Pre každé dostatočne veľké~$n$ dokážte, že v~ľubovoľnej množine $n$~priamok vo všeobecnej polohe je možné zafarbiť aspoň $\sqrt{n}$
priamok modrou farbou tak, že žiadna z~prislúchajúcich konečných oblastí nebude mať celú hranicu modrú.}
\podpis{Rakúsko}

{%%%%%   MEMO, priklad 1
Nájdite všetky funkcie $f\colon\Bbb R\to\Bbb R$ také, že
$$
xf(y)+f\bigl(xf(y)\bigr)-xf\bigl(f(y)\bigr)-f(xy)=2x+f(y)-f(x+y)
$$
platí pre všetky $x,y\in\Bbb R$.}
\podpis{Litva}

{%%%%%   MEMO, priklad 2
Daný je pravidelný $n$-uholník, ktorý rozrežeme na $n-2$ trojuholníkov pomocou $n-3$ rezov pozdĺž uhlopriečok, ktoré sa navzájom nepretínajú vo vnútri $n$-uholníka. Nech {\it dvojfarebná triangulácia\/} je také rozrezanie $n$-uholníka, v~ktorom každý trojuholník je ofarbený čiernou alebo bielou farbou a~každé dva trojuholníky so spoločnou stranou majú rôzne farby. Celé číslo $n\ge 3$ nazývame {\it triangulárne}, ak pre pravidelný $n$-uholník existuje dvojfarebná triangulácia taká, že pre každý vrchol~$A$ daného $n$-uholníka je počet čiernych trojuholníkov s~vrcholom~$A$ väčší ako počet bielych trojuholníkov s~vrcholom~$A$. Nájdite všetky triangulárne čísla.
}
\podpis{Chorvátsko}

{%%%%%   MEMO, priklad 3
Daný je trojuholník $ABC$ so stredom~$I$ kružnice vpísanej, pričom $|AB|<|AC|$. Označme $E$ bod na strane~$AC$ taký, že $|AE|=|AB|$. Nech $G$ je bod na priamke~$EI$ taký, že $|\uhol IBG|=|\uhol CBA|$ a~bod~$I$ leží medzi bodmi $E$ a~$G$.
Dokážte, že priamka~$AI$, kolmica na $AE$ v~bode~$E$ a~os uhla $BGI$ sa pretínajú v~jednom bode.
}
\podpis{Chorvátsko}

{%%%%%   MEMO, priklad 4
Pre celé čísla $n\ge k\ge 0$ definujme {\it bibinomický koeficient\/} $\displaystyle\left(\!\!\binom{n}{k}\!\!\right)$ ako
$$
\left(\!\!\binom{n}{k}\!\!\right)=\frac{n!!}{k!!(n-k)!!}.
$$
Nájdite všetky dvojice $(n,k)$ celých čísel $n\ge k\ge 0$ také, že prislúchajúci bibinomický koeficient je celé číslo.

(\uv{Dvojitý} faktoriál $n!!$ je definovaný ako súčin všetkých párnych čísel od $2$ po $n$, ak $n$ je párne, a~ako súčin všetkých nepárnych čísel od $1$ po $n$, ak $n$ je nepárne. Napríklad $0!!=1$, $4!!=2\cdot4=8$ a $7!!=1\cdot3\cdot5\cdot7=105$.)
}
\podpis{Rakúsko}

{%%%%%   MEMO, priklad t1
Určte najmenšiu možnú hodnotu výrazu
$$
\frac{1}{a+x} + \frac{1}{a+y} + \frac{1}{b+x} + \frac{1}{b+y},
$$
kde $a$, $b$, $x$ a~$y$ sú kladné reálne čísla spĺňajúce nerovnosti
$$
		\frac{1}{a+x} \ge \frac{1}{2}, \qquad
		\frac{1}{a+y} \ge \frac{1}{2}, \qquad
		\frac{1}{b+x} \ge \frac{1}{2},  \qquad
		\frac{1}{b+y} \ge 1.
$$}
\podpis{Maďarsko}

{%%%%%   MEMO, priklad t2
Nájdite všetky funkcie $f\colon\Bbb R\to\Bbb R$ také, že
$$
	xf(xy)+xy f(x)\ge f(x^2)f(y)+x^2y
$$
platí pre všetky $x,y\in\Bbb R$.}
\podpis{Česká rep., Pavel Calábek}

{%%%%%   MEMO, priklad t3
Nech $K$ a~$L$ sú prirodzené čísla. Na šachovnici pozostávajúcej z~$2K\times 2L$ políčok sa pohybuje mravec. Začína v~ľavom dolnom políčku a~presúva sa do pravého horného políčka. V~každom kroku sa presunie na susedné políčko vo vodorovnom alebo zvislom smere a~každé z~políčok pri svojom presune navštívi nanajvýš raz. V~niektorých prípadoch nenavštívené políčka tvoria pravouholník -- taký nazývame {\it memorovaný}.
Určte počet všetkých rôznych memorovaných pravouholníkov. (Pravouholníky sú rôzne, pokiaľ nepozostávajú z~tých istých políčok.)	
}
\podpis{Rakúsko}

{%%%%%   MEMO, priklad t4
V~Happy City je 2014 obyvateľov označených $A_1,A_2,\dots,A_{2014}$. V~každom okamihu dňa je každý obyvateľ buď {\it šťastný\/} alebo {\it nešťastný}. Nálada obyvateľa~$A$ sa zmení (z~nešťastnej na šťastnú alebo opačne) práve vtedy, ak sa usmeje nejaký iný šťastný obyvateľ na obyvateľa~$A$. Na začiatku v~Happy City bolo $N$ šťastných obyvateľov. V~pondelok počas dňa sa udialo nasledovné: obyvateľ $A_1$ sa usmial na obyvateľa $A_2$, potom sa obyvateľ $A_2$ usmial na obyvateľa $A_3$, atď. Nakoniec sa obyvateľ $A_{2013}$ usmial na obyvateľa $A_{2014}$. Okrem toho sa nikto iný neusmial na nikoho iného. Rovnako to prebehlo aj v~utorok, stredu a~vo štvrtok. Na konci dňa vo štvrtok bolo v~Happy City presne 2000 šťastných obyvateľov. Určte najväčšiu možnú hodnotu~$N$.
}
\podpis{Litva}

{%%%%%   MEMO, priklad t5
Daný je trojuholník $ABC$, pre ktorý platí $|AB| < |AC|$. Kružnica vpísaná trojuholníku $ABC$ so stredom~$I$ sa dotýka strán $BC$, $CA$ a~$AB$ postupne v~bodoch $D$, $E$ a~$F$. Priamka~$AI$ pretína priamky $DE$ a~$DF$ postupne v~bodoch $X$ a~$Y$. Označme $Z$ pätu výšky z~bodu~$A$ na stranu~$BC$.
Dokážte, že $D$ je stred kružnice vpísanej trojuholníku $XYZ$.
}
\podpis{Nemecko}

{%%%%%   MEMO, priklad t6
Kružnica~$k$ vpísaná trojuholníku $ABC$ sa dotýka strany~$BC$ v~bode~$D$. Nech $L\ne D$ je priesečník priamky~$AD$ a~kružnice~$k$ a~nech $K$ je stred kružnice pripísanej k~strane~$BC$. Označme $M$ a~$N$ postupne stredy úsečiek $BC$ a~$KM$. Dokážte, že body $B$, $C$, $N$ a~$L$ ležia na jednej kružnici.}
\podpis{Slovensko, Patrik Bak}

{%%%%%   MEMO, priklad t7
Konečná množina~$\mm A$ kladných celých čísel sa nazýva {\it priemerová}, ak pre každú jej neprázdnu podmnožinu je aritmetický priemer jej prvkov taktiež kladné celé číslo. Ináč povedané, $\mm A$ je priemerová, ak $(a_1+\dots+a_k)/k$ je celé číslo pre každé $k\ge 1$, kde $a_1,\dots,a_k\in \mm A$ sú navzájom rôzne. Pre ľubovoľné celé číslo~$n$ určte najmenší možný súčet prvkov priemerovej $n$-prvkovej množiny.}
\podpis{Rakúsko}

{%%%%%   MEMO, priklad t8
Určte všetky štvorice $(x,y,z,t)$ kladných celých čísel, pre ktoré platí
$$
	20^x+14^{2y}=(x+2y+z)^{zt}.
$$}
\podpis{Litva}

{%%%%%   CPSJ, priklad 1
V~rovine sú dané kružnice $k$, $l$, ktoré sa pretínajú v~bodoch $C$ a~$D$. Kružnica~$k$ prechádza stredom~$L$ kružnice~$l$. Uvažujme ľubovoľnú priamku prechádzajúcu bodom~$D$, ktorá pretína kružnice $k$, $l$ postupne v~bodoch $A$, $B$, pričom $D$ leží vnútri úsečky~$AB$. Dokážte, že $|AB|=|AC|$.
}
\podpis{Česká rep.}

{%%%%%   CPSJ, priklad 2
V~obore reálnych čísel vyriešte rovnicu $a+b+4=4\sqrt{a\sqrt{b}}$.
}
\podpis{Poľsko}

{%%%%%   CPSJ, priklad 3
K~dispozícii máme 10 rovnakých dlaždíc; jedna z~nich je zobrazená na obrázku nižšie. Dlaždice možno otáčať v~rovine ale nesmieme ich preklopiť na opačnú stranu. Plochu $7\times 7$ chceme pokryť týmito dlaždicami takým spôsobom, že práve jeden jednotkový štvorček je pokrytý dvoma dlaždicami a~všetky ostatné jednotkové štvorčeky sú pokryté práve jednou dlaždicou. Určte všetky jednotkové štvorčeky, ktoré môžu byť pokryté dvoma dlaždicami.
}
\podpis{Poľsko}

{%%%%%   CPSJ, priklad 4
Číslo $a_n$ vzniklo napísaním všetkých čísel $1,2,...,n$ za sebou (napríklad $a_3=123$, $a_{11}=1234567891011$ a~pod.). Pre aký najmenší index $t$ platí $11 \mid a_t$?
}
\podpis{Patrik Bak}

{%%%%%   CPSJ, priklad 5
Daný je štvorec. Rezmi v~podobe priamok ho rozdelíme na $n$ mnohouholníkov.
Aký je najväčší možný súčet vnútorných uhlov všetkých mnohouholníkov?
}
\podpis{Ján Mazák}

{%%%%%   CPSJ, priklad t1
Množinu $\{1,2,3,\dots,63\}$ sme rozdelili na tri neprázdne disjunktné množiny, vypočítali sme súčin všetkých čísel v~každej množine a~napokon sme určili najväčšieho spoločného deliteľa uvedených troch súčinov. Aký najväčší možný výsledok sme mohli dostať?
}
\podpis{Peter Novotný}

{%%%%%   CPSJ, priklad t2
V~rovnobežníku $ABCD$ platí $|\angle BAD|<90^\circ$ a $|AB|>|BC|$. Os uhla $BAD$ pretína úsečku~$CD$ v~bode~$P$ a~polpriamku opačnú k~$CB$ v~bode~$Q$. Dokážte, že stred kružnice opísanej trojuholníku $CPQ$ je rovnako vzdialený od bodov $B$ a~$D$.
}
\podpis{Patrik Bak}

{%%%%%   CPSJ, priklad t3
Znajdź wszystkie liczby ca\l{}kowite $n$ o~tej w\l{}asności, że
$$
|n^3-4n^2+3n-35|\quad\text{oraz}\quad |n^2+4n+8|
$$
s\ą{} liczbami pierwszymi.}
\podpis{Česká rep.}

{%%%%%   CPSJ, priklad t4
Punkt $M$ jest środkiem boku $AB$ trójk\ą{}ta ostrok\ą{}tnego $ABC$. Okr\ą{}g o~środku w~punkcie $M$ przechodz\ą{}cy przez punkt $C$ przecina proste $AC$ i~$BC$ po raz drugi odpowiednio w~punktach $P$ i~$Q$. Punkt $R$ należ\ą{}cy do odcinka $AB$ jest taki, że trójk\ą{}ty $APR$ i~$BQR$ maj\ą{} równe pola. Wykaż, że proste $PQ$ i~$CR$ s\ą{} prostopad\l{}e.}
\podpis{Poľsko}

{%%%%%   CPSJ, priklad t5
Na tabuli je na počátku napsáno číslo $1$. Pokud je na tabuli napsáno číslo $a$, pak na ni můžeme napsat rovněž přirozené číslo $b$ takové, že $a+b+1$ je dělitelem čísla $a^2+b^2+1$. Rozhodněte, zda se na tabuli po jistém čase může objevit kterékoliv předem dané přirozené číslo. Svou odpověď zdůvodněte.
}
\podpis{Poľsko}

{%%%%%   CPSJ, priklad t6
Určete největší a nejmenší hodnotu zlomku
   $$F={{y-x}\over{x+4y}},$$
jestliže reálná čísla $x$ a $y$ vyhovují rovnici $x^2y^2+xy+1=3y^2$.
}
\podpis{Česká rep.}
