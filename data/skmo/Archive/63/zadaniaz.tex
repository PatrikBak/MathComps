{%%%%%   Z4-I-1
...}
\podpis{...}

{%%%%%   Z4-I-2
...}
\podpis{...}

{%%%%%   Z4-I-3
...}
\podpis{...}

{%%%%%   Z4-I-4
...}
\podpis{...}

{%%%%%   Z4-I-5
...}
\podpis{...}

{%%%%%   Z4-I-6
...}
\podpis{...}

{%%%%%   Z5-I-1
Medzi dvoma tyčami je napnutá šnúra dlhá 3{,}8\,m, na ktorú chce
mamička zavesiť vypraté vreckovky.
Všetky vreckovky majú tvar štvorca so stranou 40\,cm.
Na šnúre však už visia dve vreckovky rovnakého tvaru od susedky
a~tie chce mamička nechať na svojich miestach.
Pritom ľavý roh jednej z~týchto vreckoviek je 60\,cm od ľavej tyče a~ľavý
roh tej druhej je 1{,}3\,m od pravej tyče. Koľko najviac vreckoviek môže mamička na šnúru zavesiť?
Vreckovky sa vešajú natiahnuté za dva susedné rohy tak, aby sa žiadne dve neprekrývali.}
\podpis{Martin Mach}

{%%%%%   Z5-I-2
Vojto má dve rovnaké sklíčka tvaru rovnostranného trojuholníka, ktoré sa líšia
iba svojou farbou -- jedno je červené, druhé modré.
Keď sa sklíčka položia cez seba, vznikne útvar fialovej farby.
Uveďte príklad prekrývania sklíčok, pri ktorom mohol Vojto dostať:
\begin{enumerate}
  \item fialový trojuholník,
  \item fialový štvoruholník,
  \item fialový päťuholník,
  \item fialový šesťuholník.
\end{enumerate}
}
\podpis{Erika Novotná}

{%%%%%   Z5-I-3
Palindróm je také číslo, ktoré je rovnaké, či už ho čítame spredu alebo
zozadu. (Napr. číslo $1881$ je palindróm.)
Nájdite taký dvojciferný a~trojciferný palindróm, aby ich súčet
bol štvorciferný palindróm.}
\podpis{Marta Volfová}

{%%%%%   Z5-I-4
Eve sa páčia čísla deliteľné šiestimi, Zdenke čísla obsahujúce aspoň jednu šestku
a~Jane čísla, ktorých ciferný súčet je $6$.
\begin{enumerate}
  \item Ktoré dvojciferné čísla sa páčia všetkým trom dievčatám?
  \item Ktoré dvojciferné čísla sa páčia práve dvom dievčatám?
\end{enumerate}
}
\podpis{Michaela Petrová}

{%%%%%   Z5-I-5
Doplňte do prázdnych krúžkov na \obr{} prirodzené čísla tak, aby súčet čísel na
každej strane trojuholníka bol rovnaký a~aby súčet všetkých šiestich čísel bol~100.
\insp{z63.1}%
}
\podpis{Libor Šimůnek}

{%%%%%   Z5-I-6
Recepčná v~hoteli si vykladala karty a~dostala nasledujúcu postupnosť:
$$
5,\ 9,\ 2,\ 7,\ 3,\ 6,\ 8,\ 4.
$$
Presunula dve susedné karty na iné miesto tak, že
táto dvojica opäť susedila, a~to v~rovnakom poradí.
Tento krok urobila celkom trikrát,
kým neboli karty usporiadané vzostupne podľa svojej hodnoty.
Zistite, ako recepčná postupovala.}
\podpis{Libuše Hozová}

{%%%%%   Z6-I-1
V~továrni na výrobu plyšových hračiek majú dva stroje.
Prvý vyrobí štyroch zajacov za rovnaký čas, za ktorý vyrobí druhý päť
medveďov.
Aby bolo ich ovládanie jednoduchšie, oba stroje sa spúšťajú a~vypínajú
naraz spoločným vypínačom.
Navyše sú stroje nastavené tak, že prvý po spustení najskôr vyrobí troch
ružových zajacov, potom jedného modrého, potom zasa troch ružových atď.
Druhý po spustení najskôr vyrobí štyroch modrých medveďov, potom jedného ružového,
potom opäť štyroch modrých atď.
Po istom čase bolo na týchto dvoch strojoch vyrobených celkom 220 modrých
hračiek.
Koľko bolo vtedy vyrobených ružových zajacov?
}
\podpis{Michaela Petrová}

{%%%%%   Z6-I-2
Juro, Mišo, Peter, Filip a~Samo skákali do diaľky.
Samo skočil 135\,cm,
Peter skočil o~4\,cm viac ako Juro,
Juro o~6\,cm menej ako Mišo
a~Mišo o~7\,cm menej ako Filip.
Navyše Filipov skok bol presne v~polovici medzi Petrovým a~Samovým.
Zistite, koľko cm skočili jednotliví chlapci.}
\podpis{Monika Dillingerová}

{%%%%%   Z6-I-3
Koľko musíme napísať cifier, ak chceme vypísať všetky prirodzené čísla od~1
do~2013?}
\podpis{Marta Volfová}

{%%%%%   Z6-I-4
Správne vyplnená tabuľka na \obr{} má obsahovať šesť prirodzených čísel,
pričom v~každom sivom políčku má byť súčet čísel z~dvoch bielych políčok, ktoré s~ním
susedia.
\insp{z63.3}%

Určte čísla správne vyplnenej tabuľky, ak viete, že súčet prvých dvoch čísel
zľava je $33$, súčet prvých dvoch čísel sprava je~$28$ a~súčet všetkých šiestich
čísel je~$64$.}
\podpis{Libor Šimůnek}

{%%%%%   Z6-I-5
Adam dostal od deda drevené kocky.
Všetky boli rovnaké a~mali hranu dlhú 4\,cm.
Rozhodol sa, že z~nich bude stavať komíny, a~to také:
\begin{itemize}
  \item aby boli použité všetky kocky,
  \item aby komín pri pohľade zhora vyzeral ako "dutý obdĺžnik" alebo "dutý
        štvorec" ohraničený jedným radom kociek (podobne ako na \obr),
  \item aby ani v~najvyššej vrstve žiadna kocka nechýbala.\insp{z63.2}%
\end{itemize}
\noindent
Adam zistil, že komín vysoký 16\,cm, 20\,cm aj 24\,cm sa podľa týchto pravidiel určite dá z~jeho kociek postaviť.
\begin{enumerate}
  \item Aký najmenší počet kociek mohol Adam dostať od deda?
  \item Aký vysoký je najvyšší komín, ktorý môže Adam s~týmto najmenším počtom kociek
        postaviť podľa uvedených pravidiel?
\end{enumerate}
}
\podpis{Michaela Petrová}

{%%%%%   Z6-I-6
Na \obr{} je sieť zložená z~20~zhodných obdĺžnikov, do ktorej sme zakreslili
tri útvary a~vyfarbili ich.
Obdĺžnik označený písmenom~$A$ a~šesťuholník označený písmenom~$B$ majú
zhodné obvody, a~to 56\,cm.
Vypočítajte obvod tretieho útvaru označeného písmenom~$C$.\insp{z63.4}%
}
\podpis{Libor Šimůnek}

{%%%%%   Z7-I-1
Na lavičke v~parku sedia vedľa seba Anička, Barborka, Cilka, Dominik a~Edo.
Anička má 4~roky, Edo má 10~rokov, súčin vekov Aničky, Barborky a~Cilky
je $140$, súčin vekov Barborky, Cilky a~Dominika je $280$ a~súčin vekov Cilky,
Dominika a~Eda je~$560$.
Koľko rokov má Cilka?}
\podpis{Libuše Hozová}

{%%%%%   Z7-I-2
K~starej mame prišli na prázdniny vnuci --
päť rôzne starých bratov.
Stará mama im povedala, že pre nich má celkom 60 € ako vreckové,
ktoré si majú rozdeliť tak, aby:
\begin{itemize}
  \item najstarší dostal najviac,
  \item každý mladší dostal o~určitú čiastku menej ako jeho
    starší vekom najbližší súrodenec,
  \item táto čiastka bola stále rovnaká,
  \item najmladší dostal sumu, ktorá sa dá vyplatiť v~jednoeurovkách
        a~ktorá nie je menšia ako 5 €, ale nie je väčšia ako 8 €.
\end{itemize}
\noindent
Určte všetky možnosti, ako si mohli vnuci vreckové rozdeliť.
}
\podpis{Marta Volfová}

{%%%%%   Z7-I-3
Juro, Mišo, Peter, Filip a~Samo skákali do diaľky.
Samo skočil 135\,cm,
Peter skočil o~4\,cm viac ako Juro
a~Mišo o~7\,cm menej ako Filip.
Navyše Filipov skok bol presne v~polovici medzi tým Petrovým a~Samovým
a~najkratší skok meral 127\,cm.
Zistite, koľko cm skočili jednotliví chlapci.}
\podpis{Monika Dillingerová}

{%%%%%   Z7-I-4
V~hostinci U~troch prasiatok obsluhujú Pašík, Rašík a~Sašík.
Pašík je nečestný, takže každému hosťovi pripočíta
k~celkovej cene 6~grajciarov.
Rašík je poctivec, každému vyúčtuje presne to, čo zjedol a~vypil.
Sašík je dobrák, takže každému hosťovi dá zľavu z~celkovej ceny vo výške
20\,\%.
Prasiatka sa na seba tak podobajú, že žiadny hosť nepozná, ktoré práve obsluhuje.
Koza Lujza zašla v~pondelok, v~utorok aj v~stredu do tohto hostinca na
čučoriedkovú buchtu.
Napriek tomu, že vedela, že v~pondelok bol Rašík chorý
a~neobsluhoval, utratila za svoju pondelkovú, utorkovú aj stredajšiu buchtu
dokopy rovnako, ako keby ju vždy obsluhoval Rašík.
Koľko grajciarov účtuje Rašík za jednu čučoriedkovú buchtu?
Nájdite všetky možnosti.
(Ceny uvádzané v~jedálnom lístku sa v~tieto dni nemenili.)
}
\podpis{Michaela Petrová}

{%%%%%   Z7-I-5
Mamička delí čokoládu, ktorá má $6\times 4$ rovnakých dielikov,
svojim trom deťom.
Ako môže mamička čokoládu rozdeliť na práve tri časti s~rovnakým obsahom
tak, aby jeden útvar bol trojuholník, jeden štvoruholník a~jeden päťuholník?}
\podpis{Erika Novotná}

{%%%%%   Z7-I-6
Keď Cézar stojí na psej búde a~Dunčo na zemi, je Cézar o~70\,cm vyšší ako
Dunčo.
Keď Dunčo stojí na psej búde a~Cézar na zemi, je Dunčo o~90\,cm vyšší ako
Cézar.
Aká vysoká je psia búda?}
\podpis{Libuše Hozová}

{%%%%%   Z8-I-1
Po okružnej linke v~meste ide električka, v~ktorej je 300 cestujúcich.
Na každej zastávke sa odohrá jedna z~nasledujúcich situácií:
\begin{itemize}
  \item ak je v~električke aspoň 7 cestujúcich, tak ich 7 vystúpi,
  \item ak je v~električke menej ako 7 cestujúcich, tak 5 nových cestujúcich
    pristúpi.
\end{itemize}
\noindent
Vysvetlite, prečo v~istom okamihu v~električke neostane žiadny cestujúci.
Potom zistite, koľko by malo byť na začiatku v~električke cestujúcich,
aby sa električka nikdy nevyprázdnila.
}
\podpis{Ján Mazák}

{%%%%%   Z8-I-2
Mamička delí čokoládu, ktorá má $6\times 4$ rovnakých dielikov,
svojim štyrom deťom.
Ako môže mamička čokoládu rozdeliť na práve štyri časti s~rovnakým
obsahom tak, aby jeden útvar bol trojuholník, jeden štvoruholník, jeden
päťuholník a~jeden šesťuholník?}
\podpis{Erika Novotná}

{%%%%%   Z8-I-3
Zmeňte v~každom z~troch čísel jednu cifru tak, aby bol príklad na odčítanie bez chyby:
$$
\begin{array}{rrrr}
 & 7 & 2 & 4 \\
- & 3 & 0 & 7 \\
\hline
 & 1 & 8 & 8 \\
\end{array}
$$
Nájdite všetky riešenia.}
\podpis{Michaela Petrová}

{%%%%%   Z8-I-4
Trojuholníky $ABC$ a~$DEF$ sú rovnostranné s~dĺžkou strany 5\,cm.
Tieto trojuholníky sú položené cez seba tak, aby strany jedného
trojuholníka boli rovnobežné so stranami druhého a~aby prienikom týchto
dvoch trojuholníkov bol šesťuholník (na \obr{} označený ako $GHIJKL$).
\insp{z63.5}%
Je možné určiť obvod dvanásťuholníka $AGEHBIFJCKDL$ bez toho,
aby sme poznali presnejšie informácie o~polohe trojuholníkov?
Ak áno, spočítajte ho; ak nie, vysvetlite prečo.}
\podpis{Eva Patáková}

{%%%%%   Z8-I-5
Zákazník privážajúci odpad do zberných surovín je povinný zastaviť naloženým
autom na váhe a~po vykládke odpadu znova.
Rozdiel nameraných hmotností tak zodpovedá privezenému odpadu.
Pat a~Mat spravili chybu.
Pri vážení naloženého auta sa na váhu priplietol Pat a~pri vážení vyloženého
auta sa tam namiesto Pata ocitol Mat.
Vedúci zberných surovín si tak zaznamenal rozdiel 332\,kg.
Následne sa na prázdnu váhu postavili spolu vedúci a~Pat, potom samotný
Mat a~váha ukázala rozdiel 86\,kg.
Ďalej sa spolu zvážili vedúci a~Mat, potom samotný Pat a~váha ukázala rozdiel
64\,kg.
Koľko v~skutočnosti vážil privezený odpad?
}
\podpis{Libor Šimůnek}

{%%%%%   Z8-I-6
V~dome máme medzi dvoma poschodiami dve rôzne schodiská.
Na každom z~týchto schodísk sú všetky schody rovnako vysoké.
Jedno zo schodísk má každý schod vysoký 10\,cm, druhé má o~11~schodov menej
ako to prvé.
Behom dňa som išiel päťkrát nahor a~päťkrát nadol, pričom som si medzi
týmito dvoma schodiskami vyberal náhodne.
Celkom som na každom zo schodísk zdolal rovnaký počet schodov.
Aký je výškový rozdiel medzi poschodiami?
}
\podpis{Martin Mach}

{%%%%%   Z9-I-1
Peter si myslí dvojciferné číslo.
Keď toto číslo napíše dvakrát za sebou, vznikne štvorciferné číslo deliteľné deviatimi.
Keď to isté číslo napíše trikrát za sebou, vznikne šesťciferné číslo deliteľné ôsmimi.
Zistite, aké číslo si môže Peter myslieť.}
\podpis{Erika Novotná}

{%%%%%   Z9-I-2
Daný je rovnoramenný lichobežník s~dĺžkami strán $|AB|=31\cm$, $|BC|=26\cm$
a~$|CD|=11\cm$.
Na strane~$AB$ je bod~$E$ určený pomerom vzdialeností $|AE|:|EB|=3:28$.
Vypočítajte obvod trojuholníka $CDE$.}
\podpis{Lenka Dedková}

{%%%%%   Z9-I-3
Podlahu tvaru obdĺžnika so~stranami 360\,cm a~540\,cm máme pokryť (bez medzier)
zhodnými štvorcovými dlaždicami. Môžeme si vybrať z~dvoch typov
štvorcových dlaždíc, ktorých strany sú v~pomere $2:3$.
V~oboch prípadoch sa dá pokryť celá plocha jedným typom dlaždíc bez pílenia.
Menších dlaždíc by sme potrebovali o~30 viac ako väčších.
Určte, ako dlhé sú strany dlaždíc.}
\podpis{Karel Pazourek}

{%%%%%   Z9-I-4
V~pravouholníku $ACKI$ sú vyznačené dve rovnobežky so susednými
stranami a~jedna uhlopriečka (\obr).
Pritom trojuholníky $ABD$ a~$GHK$ sú zhodné.
Určte pomer obsahov pravouholníkov $ABFE$ a~$FHKJ$.
\insp{z63.6}%
}
\podpis{Vojtěch Žádník}

{%%%%%   Z9-I-5
Eva riešila experimentálnu úlohu Fyzikálnej olympiády.
Dopoludnia od 9:15 robila v~trojminútových odstupoch 4~merania.
Získané hodnoty zapisovala do tabuľky, ktorú si pripravila v~počítači:
$$
\begin{array}{c|c|c}
\text{hodín} & \text{minút} & \text{hodnota} \\
\hline
9 & 15 & \\
9 & 18 & \\
9 & 21 & \\
9 & 24 & \\
\end{array}
$$
Popoludní v~experimente pokračovala.
Tentoraz urobila v~trojminútových odstupoch 9~meraní a~hodnoty zapisovala
do podobnej tabuľky.
Omylom do počítača zadala, aby sa zobrazil súčet deviatich čísel
z~prostredného stĺpca.
Tento zbytočný výpočet vyšiel~$258$.
Ktoré čísla boli v~danom stĺpci?
}
\podpis{Libor Šimůnek}

{%%%%%   Z9-I-6
V~hostinci U~troch prasiatok obsluhujú Pašík, Rašík a~Sašík.
Pašík je nečestný, takže každému hosťovi pripočíta
k~celkovej cene 10~grajciarov.
Rašík je poctivec, každému vyúčtuje presne to, čo zjedol a~vypil.
Sašík je dobrák, takže každému hosťovi dá zľavu z~celkovej ceny vo výške
20\,\%.
Prasiatka sa na seba tak podobajú, že žiadny hosť nepozná, ktoré práve obsluhuje.
Baránok Vendelín si v~pondelok objednal tri koláčiky a~džbánok džúsu
a~zaplatil za to 56~grajciarov.
Bol spokojný, takže hneď v~utorok zjedol päť koláčikov, vypil k~nim tri džbánky
džúsu a~platil 104~grajciarov.
V~stredu zjedol osem koláčikov, vypil štyri džbánky
džúsu a~zaplatil 112 grajciarov.
\begin{enumerate}
\item Kto obsluhoval Vendelína v~pondelok, kto v~utorok a~kto v~stredu?
\item Koľko grajciarov účtuje Rašík za jeden koláčik a~koľko za jeden
džbánok džúsu?
\end{enumerate}
(Všetky koláčiky sú rovnaké, rovnako tak všetky džbánky džúsu.
Ceny uvádzané v~jedálnom lístku sa v~uvedených dňoch nemenili.)
}
\podpis{Michaela Petrová}

{%%%%%   Z4-II-1
...}
\podpis{...}

{%%%%%   Z4-II-2
...}
\podpis{...}

{%%%%%   Z4-II-3
...}
\podpis{...}

{%%%%%   Z5-II-1
Na kraji cesty je za sebou niekoľko rovnako dlhých parkovacích miest. Jeden autobus stál na piatom až siedmom mieste zľava, druhý autobus zaberal ôsme až desiate miesto sprava, inak bolo parkovisko prázdne. Neskôr tu zaparkovali ďalšie štyri autobusy a žiadny ďalší autobus už sa na parkovisko nezmestil. Určte, koľko najviac parkovacích miest mohlo byť na parkovisku, ak každý autobus zaberá presne tri parkovacie miesta.}
\podpis{Eva Patáková}

{%%%%%   Z5-II-2
Ľuboš rozdelil obdĺžnik jednou čiarou na dva menšie obdĺžniky. Obvod veľkého obdĺžnika je 76\,cm, obvody menších obdĺžnikov sú 40\,cm a 52\,cm. Určte rozmery veľkého obdĺžnika.}
\podpis{Libor Šimůnek}

{%%%%%   Z5-II-3
Juraj a Peter spustili stopky a niekedy v priebehu prvých 15 sekúnd od spustenia každý z nich začal tlieskať. Juraj po svojom prvom tlesknutí tlieskal pravidelne každých 7 sekúnd, Peter po svojom prvom tlesknutí tlieskal pravidelne každých 13 sekúnd. V 90. sekunde tleskli obaja naraz. Určte všetky možnosti, kedy mohol začať tlieskať Juraj a kedy Peter.}
\podpis{Lenka Dedková}

{%%%%%   Z6-II-1
Keď kráčame pozdĺž plota zo severu na juh, sú rozstupy medzi jeho stĺpikmi zozačiatku zhodné. Od určitého stĺpika sa rozstup zmenší na 2{,}9 metra a~taký zostáva až po južný koniec plota. Medzi 1. a~16. stĺpikom (počítané od severu) sa rozstupy nemenia a vzdialenosť medzi týmito dvoma stĺpikmi je 48 metrov. Vzdialenosť medzi 16. a 28. stĺpikom je 36 metrov. Koľký stĺpik má od svojich susedných stĺpikov rôzne rozstupy?}
\podpis{Libor Šimůnek}

{%%%%%   Z6-II-2
Ivana, Majka, Lucka, Saša a Zuzka pretekali v~čítaní tej istej knihy.
Za jednu hodinu stihla Lucka prečítať 32~strán, čo bolo presne v~strede
medzi počtami strán, ktoré stihli prečítať Saša a~Zuzka.
Ivana prečítala o~5~strán viac ako Zuzka a~Majka prečítala o~8~strán menej ako
Saša. Ivanin výsledok bol presne v~strede medzi Majkiným a~Zuzkiným.
Určte, koľko strán prečítali jednotlivé dievčatá.}
\podpis{Monika Dillingerová}

{%%%%%   Z6-II-3
Na \ifobrazkyvedla{}obrázku\else\obr{}\fi{} je znázornených niekoľko pravouholníkov s~niekoľkými spoločnými vrcholmi.
Ich dĺžky strán zapísané v~centimetroch sú celé čísla.
Obsah pravouholníka $DRAK$ je rovný $44\cm^2$,
obsah pravouholníka $DUPE$ je rovný $64\cm^2$ a~obsah mnohouholníka
$DUPLAK$ je rovný $92\cm^2$.
Určte dĺžky strán mnohouholníka $DUPLAK$.
\ifobrazkyvedla\else\insp{z6-ii-3.eps}\fi
}
\podpis{Monika Dillingerová}

{%%%%%   Z7-II-1
Tabuľka na \ifobrazkyvedla{}obrázku\else{}\obr{}\fi{} má obsahovať sedem prirodzených čísel, pričom v~každom
sivom políčku má byť súčin čísel z~dvoch bielych políčok, ktoré s~ním susedia.
Čísla v~bielych políčkach sú navzájom rôzne a~súčin čísel v~sivých políčkach
je rovný $525$. Aký je súčet čísel v~sivých políčkach? Nájdite všetky možnosti.
\insp{z7-ii-1.eps}%
}
\podpis{Eva Patáková}

{%%%%%   Z7-II-2
Ivana, Majka, Lucka, Saša a~Zuzka pretekali v~čítaní tej istej knihy.
Za jednu hodinu stihla Lucka prečítať 32 strán, čo bolo presne v~strede
medzi počtami strán, ktoré stihli prečítať Saša a~Zuzka.
Ivana prečítala o~5~strán viac ako Zuzka a~Majka prečítala o~8~strán menej ako
Saša. Žiadne dve dievčatá neprečítali rovnaký počet strán a~najhorší výsledok bol 27~strán.
Určte, koľko strán prečítali jednotlivé dievčatá.}
\podpis{Monika Dillingerová}

{%%%%%   Z7-II-3
Po náraze kamienka praskla sklenená tabuľa tak, že vznikli štyri menšie
trojuholníky so spoločným vrcholom v~mieste nárazu.
Pritom platí, že:
\begin{itemize}
\iitem sklenená tabuľa mala tvar obdĺžnika, ktorý bol 8\,dm široký a~6\,dm vysoký,
\iitem trojuholník vpravo mal trikrát väčší obsah ako trojuholník vľavo,
\iitem trojuholník vľavo mal dvakrát menší obsah ako trojuholník dole.
\end{itemize}
Určte vzdialenosti bodu nárazu od strán obdĺžnika.}
\podpis{Erika Novotná}

{%%%%%   Z8-II-1
Angela, Barbora, Jano, Vlado a~Matúš súťažili v~hode
papierovým lietadielkom.
Každý hádzal raz a~súčet dĺžok ich hodov bol 41~metrov.
Matúš hodil najmenej, čo bolo o 90\,cm menej ako hodila Angela,
a~tá hodila o~60\,cm menej ako Vlado.
Jano hodil najďalej a~trafil lietadielkom do pásky označujúcej celé metre.
Ak by súťažili iba Matúš, Vlado a~Angela, priemerná dĺžka hodu by
bola o~20\,cm kratšia.
Určte dĺžky hodov všetkých menovaných detí.
}
\podpis{Lenka Dedková}

{%%%%%   Z8-II-2
Daný je štvoruholník $ABCD$, pozri \ifobrazkyvedla{}obrázok\else\obr{}\fi{}.
Bod~$T_1$ je ťažiskom trojuholníka $BCD$, bod~$T_2$ je ťažiskom trojuholníka
$ABD$ a~body $T_1$ a~$T_2$ ležia na úsečke~$AC$.
Dĺžka úsečky $T_1T_2$ je 3\,cm a~bod~$D$ má od úsečky~$AC$ vzdialenosť 3\,cm.
Určte obsah štvoruholníka $ABCD$.
\ifobrazkyvedla\else\insp{z8-ii-2.eps}\fi
}
\podpis{Eva Patáková}

{%%%%%   Z8-II-3
V~meste rekordov a~kuriozít postavili pyramídu z~kociek.
V~hornej vrstve je jedna kocka a~počty kociek v~jednotlivých vrstvách sa
smerom nadol zväčšujú vždy o~dve (niekoľko horných vrstiev stavby je znázornených
na \ifobrazkyvedla{}obrázku\else\obr{}\fi{}).
Prvá, teda najspodnejšia vrstva má čiernu farbu, druhá sivú, tretia bielu,
štvrtá opäť čiernu, piata sivú, šiesta bielu a~takto sa farby pravidelne
striedajú až k~hornej vrstve.
Určte, koľko má pyramída vrstiev, ak viete, že čiernych kociek je použitých o~55 viac ako bielych.
\ifobrazkyvedla\else\insp{z8-ii-3.eps}\fi
}
\podpis{Libor Šimůnek}

{%%%%%   Z9-II-1
Jana mala za domácu úlohu vypočítať súčin dvoch šesťciferných čísel. Pri prepisovaní druhého z~nich z~tabule zabudla odpísať jednu cifru a~tak namiesto celého šesťciferného čísla napísala iba päťciferné číslo $85522$. Keď bola doma, zistila svoj omyl. Pamätala si však, že číslo, ktoré zle odpísala, bolo deliteľné tromi. Rozhodla sa, že sa pokúsi určiť, aké mohlo byť pôvodné číslo. Zistite, koľko takých šesťciferných čísel existuje.}
\podpis{Monika Dillingerová}

{%%%%%   Z9-II-2
Renáta zostrojila lichobežník $PRST$ so základňami $PR$ a~$ST$, v~ktorom súčasne platí:
\begin{itemize}
\iitem lichobežník $PRST$ nie je pravouhlý,
\iitem trojuholník $TRP$ je rovnostranný,
\iitem trojuholník $TRS$ je pravouhlý,
\iitem jeden z~trojuholníkov $TRS$ a~$TRP$ má obsah $10\cm^2$.
\end{itemize}
Určte obsah druhého z~týchto dvoch trojuholníkov; nájdite všetky možnosti.
}
\podpis{Michaela Petrová}

{%%%%%   Z9-II-3
Lenka mala papierový kvietok s~ôsmimi okvetnými lístkami. Na každom lístku bola napísaná práve jedna cifra a~žiadna z~cifier sa na žiadnom inom lístku neopakovala. Keď sa Lenka s~kvietkom hrala, uvedomila si niekoľko vecí:
\begin{itemize}
\iitem Z~kvietka bolo možné odtrhnúť štyri lístky tak, že súčet na nich napísaných čísel by bol rovnaký ako súčet čísel na neodtrhnutých lístkoch.
\iitem Tiež bolo možné odtrhnúť štyri lístky tak, že súčet na nich napísaných čísel by bol dvakrát väčší ako súčet čísel na neodtrhnutých lístkoch.
\iitem Dokonca bolo možné odtrhnúť štyri lístky tak, že súčet na nich napísaných čísel by bol štyrikrát väčší ako súčet čísel na neodtrhnutých lístkoch.
\end{itemize}
Určte, aké cifry mohli byť napísané na okvetných lístkoch.
}
\podpis{Erika Novotná}

{%%%%%   Z9-II-4
V~hostinci U~temného lesa obsluhujú obrie dvojčatá Pravdoslav a~Krivomír. Pravdoslav je poctivý a~účtuje vždy presne, Krivomír je nečestný a~ku každému koláču a~každému džbánku medoviny vždy pripočíta dva grajciare. Raz tento hostinec navštívilo sedem trpaslíkov, ktorí si sadli k~dvom stolom. Trpaslíci zaplatili za štyri koláče u~jedného obra rovnako veľa ako za tri džbánky medoviny u~druhého. Inokedy trpaslíci platili za štyri džbánky medoviny u~Pravdoslava o~21~grajciarov viac ako za tri koláče u~Krivomíra. Určte, koľko stál jeden koláč a~koľko jeden džbánok medoviny. Všetky ceny sú v~celých grajciaroch a~medzi oboma návštevami sa tieto ceny nijako nemenili.}
\podpis{Michaela Petrová, Monika Dillingerová}

{%%%%%   Z9-III-1
Hviezdičky v~schéme predstavujú 16 bezprostredne po sebe idúcich
prirodzených násobkov čísla tri
napísaných zľava doprava od najmenšieho po najväčší.
Pritom čísla v~prvom rámiku majú rovnaký súčet ako čísla v~druhom rámiku.
Určte najmenšie z~týchto 16 čísel.
$$
\boxed{*\,,\ *\,,\ *\,,\ *\,,\ *\,,\ *},\ *\,,\ *\,,\ *\,,\ *\,,\ *\,,\ \boxed{*\,,\ *\,,\ *\,,\ *\,,\ *}
$$
}
\podpis{Libor Šimůnek}

{%%%%%   Z9-III-2
V~rovnostrannom trojuholníku $ABC$ je vpísaný rovnostranný trojuholník
$DEF$.
Vrcholy $D$, $E$ a~$F$ ležia na stranách $AB$, $BC$ a~$AC$ tak, že strany
trojuholníka $DEF$ sú kolmé na strany trojuholníka $ABC$ (tak ako na \obr).
Ďalej platí, že úsečka $DG$ je ťažnicou v~trojuholníku $DEF$ a~bod~$H$ je
priesečníkom priamok $DG$ a~$BC$.
Určte pomer obsahov trojuholníkov $HGC$ a~$DBE$.
\ifobrazkyvedla\else\insp{z9-iii-2.eps}\fi
}
\podpis{Eva Patáková}

{%%%%%   Z9-III-3
Danka mala papierový kvietok s~desiatimi okvetnými lístkami.
Na každom lístku bola napísaná práve jedna cifra a~žiadna z~cifier sa
na žiadnom inom lístku neopakovala.
Danka odtrhla dva lístky tak, že súčet čísel na zvyšných lístkoch bol
násobkom deviatich.
Potom odtrhla ďalšie dva lístky tak, že súčet čísel na zvyšných lístkoch
bol násobkom ôsmich.
Nakoniec odtrhla ďalšie dva lístky tak, že súčet čísel na zvyšných
lístkoch bol násobkom desiatich.
Určte, aké mohli byť súčty čísel po každom odtrhávaní; nájdite všetky
možnosti.}
\podpis{Erika Novotná}

{%%%%%   Z9-III-4
Štyri dievčatá hrali na sústredení niekoľko zápasov.
Keď sčítame počty výhier dvoch dievčat dokopy (pre všetky možné dvojice dievčat), dostaneme čísla
8, 10, 12, 12, 14 a 16.
Určte, koľko výhier vybojovalo každé z~dievčat.}
\podpis{Marta Volfová}

