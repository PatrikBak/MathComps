{%%%%%   Z4-I-1
...}

{%%%%%   Z4-I-2
...}

{%%%%%   Z4-I-3
...}

{%%%%%   Z4-I-4
...}

{%%%%%   Z4-I-5
...}

{%%%%%   Z4-I-6
...}

{%%%%%   Z5-I-1
\napad
Pomôžte si náčrtom s~vyznačenými danými vzdialenosťami.

\riesenie
Celú situáciu môžeme pre názornosť prekresliť napríklad takto:
\insp{z5-i-1.eps}%

Susedkine vreckovky delia šnúru na 3 rôzne dlhé časti, do ktorých
môže mamička vešať svoje vreckovky. Medzi ľavým rohom jednej vreckovky
a~ľavou tyčou je $60\cm$. Medzi ľavým rohom druhej vreckovky a~pravou
tyčou je $130\cm$ a~vreckovka je široká $40\cm$; to znamená, že medzi jej
pravým rohom a~pravou tyčou je vzdialenosť $90\cm$. Keďže je šnúra
dlhá $380\cm$, medzi vreckovkami ostáva $150\cm$ voľnej šnúry.

Medzi ľavú vreckovku a~ľavú tyč sa vojde iba jedna mamičkina vreckovka
(dve vreckovky vedľa seba by zabrali $80\cm$ šnúry a~voľných je tu
iba $60\cm$). Medzi susedkine vreckovky môže mamička zavesiť tri svoje
(tie zaberú $120\cm$, keď sa zavesia tesne vedľa seba, na~pridanie
štvrtej by bola treba medzera dlhá $160\cm$). Medzi pravú vreckovku
a~pravú tyč sa vojdú dve ďalšie mamičkine vreckovky. Celkom teda
mamička môže na šnúru privesiť $1+3+2=6$ svojich vreckoviek.

\poznamka
Ak bude riešiteľ uvažovať vzdialenosť $1{,}3\,\text{m}$ od pravej tyče k~pravému
rohu vreckovky, vyjde mu tiež výsledok "6~vreckoviek". Také riešenie
však nemôže byť uznané ako správne.
}

{%%%%%   Z5-I-2
\napad
Ak vám nestačí predstavivosť, vystrihnite si dva také trojuholníky
z~papiera a~skúšajte ich klásť rôzne cez seba.

\riesenie
Každý z~fialových mnohouholníkov sa dá realizovať mnohými rôznymi spôsobmi;
uvádzame niekoľko možných príkladov.

a) fialový trojuholník:
\insp{z5-i-2a.eps}%

b) fialový štvoruholník:
\insp{z5-i-2b.eps}%

c) fialový päťuholník:
\insp{z5-i-2c.eps}%

d) fialový šesťuholník:
\insp{z5-i-2d.eps}%
}

{%%%%%   Z5-I-3

\napad
Akú cifru má štvorciferný palindróm na prvom mieste?

\riesenie
Súčtom dvojciferného a~trojciferného čísla môžeme získať nanajvýš číslo $1\,098$.
Ak je súčtom štvorciferné číslo, musí byť jeho prvá cifra~$1$.
Prvá cifra trojciferného sčítanca musí byť $9$, lebo keby bol tento
sčítanec menší ako $900$, bol by súčet menší ako $1\,000$.

V~našom prípade sú všetky čísla palindrómy, preto poznáme aj posledné
cifry trojciferného sčítanca a~výsledného súčtu:
$$
{**} + {9{*}9} = 1\,{**}1.
$$
Odtiaľ vyplýva, že druhá cifra pri prvom sčítanci musí byť $2$ -- dvojciferný
palindróm je~$22$:
$$
22+9{*}9=1\,{**}1.
$$
Najmenší možný palindróm na pravej strane je $1\,001$; ten je súčtom $22$ a~$979$.
Ďalšie štvorciferné palindrómy ($1\,111$, $1\,221$, \dots) sa takto vyjadriť nedajú,
pretože sú väčšie ako $1\,098$.
Jediné možné riešenie úlohy je:
$$
22+979=1\,001.
$$
}

{%%%%%   Z5-I-4
\napad
Určte, ktoré dvojciferné čísla sa páčia jednotlivým dievčatám.

\riesenie
Eve sa páčia čísla, ktoré sa dajú bezo zvyšku deliť šiestimi;
všetky dvojciferné násobky čísla $6$ sú:
$$
12,\ 18,\ 24,\ 30,\ 36,\ 42,\ 48,\ 54,\ 60,\ 66,\ 72,\ 78,\ 84,\ 90,\ 96.
$$
Dvojciferné čísla, ktoré sa páčia Zdenke, sú:
$$
16,\ 26,\ 36,\ 46,\ 56,\ 60,\ 61,\ 62,\ 63,\ 64,\ 65,\ 66,\ 67,\ 68,\ 69,\ 76,\ 86,\ 96.
$$
Dvojciferné čísla, ktoré sa páčia Jane, sú:
$$
15,\ 24,\ 33,\ 42,\ 51,\ 60.
$$
Pre lepšiu predstavu všetky uvedené čísla zapíšeme do tabuľky:
\bgroup
\def\ctr#1{\hfil\ #1\ \hfil}
$$\begintable
Eva\hfill\|12|||18|24||30||36|42||48||54||60\cr
Zdenka\|||16|||26|||36||46||||56|60\cr
Jana\hfill\||15|||24|||33||42|||51|||60\endtable
$$
$$
\begintable
Eva\hfill\||||||66||||72||78|84||90|96\cr
Zdenka\|61|62|63|64|65|66|67|68|69||76|||86||96\cr
Jana\hfill\||||||||||||||||\endtable
$$
\egroup

Z~toho vidíme, že všetkým trom dievčatám sa páči iba číslo $60$.
Práve dvom dievčatám sa páčia čísla $24$, $36$, $42$, $66$ a~$96$.
}

{%%%%%   Z5-I-5
\napad
Na začiatku dodržte iba prvú podmienku, teda vyplňte trojuholník tak, aby súčet na všetkých jeho stranách bol rovnaký.

\riesenie
Najmenšie číslo, ktoré máme doplniť, bude na strane s~číslami $24$ a~$20$, lebo zo
známych čísel dávajú práve tieto dve najväčší súčet. Skúsme do prázdneho
políčka na tejto strane doplniť najmenšie možné prirodzené číslo, teda~$1$.
V~takom prípade by súčet na strane trojuholníka bol $24+1+20=45$.
Do zvyšných prázdnych políčok by potom patrili čísla $45-15-20=10$ a~$45-15-24=6$.

Súčet všetkých šiestich čísel by v~tomto prípade bol $24+1+20+10+15+6=76$, čo je
o~$24$ menej ako požadovaných $100$.
Každé z~doplnených čísel preto musíme zväčšiť o~$24:3=8$.
Do prázdnych políčok patria čísla $1+8=9$, $10+8=18$ a~$6+8=14$:
\insp{z5-i-5a.eps}%

\ineriesenie
Čísla v~prázdnych políčkach majú dať súčet $100-24-20-15=41$. Každú stranu trojuholníka zobrazíme zvlášť:
\insp{z5-i-5b.eps}%

Všetkých deväť čísel tejto schémy dáva súčet $41+2\cdot(15+24+20)=159$.
V~každom riadku schémy, resp. na každej strane trojuholníka, má teda
byť súčet $159:3=53$.
Čísla v~prázdnych políčkach sú
$53-24-20=9$, $53-15-20=18$, $53-15-24=14$.
}

{%%%%%   Z5-I-6
\napad
Skúšajte presúvať hracie karty opísaným spôsobom.

\riesenie
Všimnime si, že keď presunieme dvojicu $(3, 6)$ medzi čísla $2$ a~$7$,
dostaneme dve dvojice po sebe idúcich čísel, ktoré sú navyše
zoradené podľa veľkosti:
$$
5,\ 9,\ 2,\ \text{\bf3},\ \text{\bf6},\ 7,\ 8,\ 4.
$$
Medzi čísla $3$ a~$6$ potrebujeme vložiť čísla $4$ a~$5$. Tie ale nie sú pri sebe,
presunieme teda dvojicu $(5, 9)$ za číslo $4$. Tým jednak dostaneme číslo
$5$ hneď vedľa čísla $4$, jednak číslo $9$ bude na konci postupnosti:
$$
2,\ 3,\ 6,\ 7,\ 8,\ 4,\ \text{\bf5},\ \text{\bf9}.
$$
V~poslednom kroku presunieme dvojicu $(4, 5)$ medzi čísla $3$ a~$6$:
$$
2,\ 3,\ \text{\bf4},\ \text{\bf5},\ 6,\ 7,\ 8,\ 9.
$$

\poznamka
Je samozrejme možné zameniť prvý a~druhý krok; postup recepčnej teda nie je
jednoznačný.
}

{%%%%%   Z6-I-1
\napad
Zistite, koľko je vyrobených modrých hračiek v~momente, keď prvý stroj dokončí
štvrtého zajaca.

\riesenie
Dobe, za ktorú prvý stroj vyrobí 4~hračky a~druhý stroj 5~hračiek, budeme
hovoriť "cyklus". Behom jedného cyklu sa teda vyrobia:
3~ružové zajace, 1~modrý zajac, 4~modré medvede a~1 ružový medveď,
\tj. celkom 4~ružové a~5~modrých hračiek.

To znamená, že 220 modrých hračiek sa vyrobí za $220:5=44$ úplných cyklov.
Keďže za jeden cyklus prvý stroj vyrobí 3~ružové zajace, za 44 cyklov
vyrobí $44\cdot3=132$ ružových zajacov.
V~každom cykle sa najskôr vyrábajú ružové zajace a~až potom modré.
Preto v~okamihu, keď oba stroje vyrobili 220 modrých hračiek, bolo na prvom stroji
vyrobených 132 ružových zajacov.
}

{%%%%%   Z6-I-2
\napad
Porovnajte dĺžky Jurovho, Mišovho, Petrovho a~Filipovho skoku.

\riesenie
Zatiaľ neuvažujme Samov skok a~dĺžky skokov ostatných chlapcov zoraďme od
najkratšieho po najdlhší. Zo zadania je zrejmé, že najmenej skočil Juro:
Peter skočil o~4\,cm viac ako Juro, Mišo skočil o~6\,cm viac ako Juro,
Filip skočil o~7\,cm viac ako Mišo, teda o~13\,cm viac ako Juro.
\insp{z6-i-2a.eps}%

Keďže Filipov skok má byť presne v~polovici medzi Petrovým a~Samovým
a~keďže Peter skočil menej ako Filip, musel Samo skočiť najviac zo
všetkých chlapcov. A~keďže Filip skočil o~9\,cm viac ako Peter ($13-4=9$),
musel zároveň skočiť o~9\,cm menej ako Samo. Z~toho určíme, že
Filip skočil 126\,cm ($135-9=126$), Juro skočil 113\,cm
($126-13=113$), Peter skočil 117\,cm ($113+4=117$) a~Mišo skočil
119\,cm ($113+6=119$).
\insp{z6-i-2b.eps}%
}

{%%%%%   Z6-I-3
\napad
Počítajte systematicky.

\riesenie
Musíme napísať všetky jedno-, dvoj- a~trojciferné čísla a~časť
štvorciferných čísel. Postupne počítame počty napísaných čísel a~cifier:
$$
\begintable
napísané čísla\|počet čísel|počet cifier\crthick
1, \dots, 9\|\hfill9|\hfill9\cr
10, \dots, 99\|\hfill90|\hfill180\cr
100, \dots, 999\|\hfill900|\hfill2\,700\cr
1\,000, \dots, 2\,013\|\hfill1\,014|\hfill4\,056\crthick
celkom\|\hfill2\,013|\hfill6\,945\endtable
$$
Na~zapísanie čísel od 1 do 2\,013 treba 6\,945 cifier.
}

{%%%%%   Z6-I-4
\napad
Aký súčet dáva tretie a~štvrté číslo?

\riesenie
Súčet tretieho a~štvrtého prirodzeného čísla je $64-33-28=3$, sú to
teda čísla 1 a~2. Najskôr predpokladajme, že číslo~1 je na tretej
pozícii a~číslo~2 na štvrtej. Na prvej pozícii by potom bolo $(33-1):2=16$
a~na šiestej $(28-2):2=13$.

Ešte preverme možnosť, že číslo~1 je na štvrtej pozícii a~číslo~2 na
tretej. Na prvej pozícii by potom muselo byť $(33-2):2=15,5$, čo
nepripúšťa zadanie, lebo v~tabuľke majú byť prirodzené čísla. Tabuľku
môžeme teda vyplniť jediným spôsobom:
\insp{z6-i-4r.eps}%
}

{%%%%%   Z6-I-5
\napad
Rozhodnite, či mohol Adam dostať napríklad 100 kociek.

\riesenie
Najskôr spočítame, koľko vrstiev mali jednotlivé Adamove komíny:
% \begin{enumerate}
\par 1. komín: $16:4=4$,
\par 2. komín: $20:4=5$,
\par 3. komín: $24:4=6$.
% \end{enumerate}¨

Keďže všetky vrstvy boli úplné, musí byť počet kociek deliteľný štyrmi,
piatimi a~šiestimi súčasne.
Najmenší spoločný násobok týchto troch čísel je 60, počet kociek preto musí
byť nejakým násobkom šesťdesiatich.
Overíme, či mohol Adam dostať 60 kociek:

1. komín má 4~vrstvy, čo by znamenalo 15~kociek v~jednej vrstve ($60:4=15$).
To ale nie je možné, pretože v~každej vrstve je párny počet kociek
("had" z~kociek sa dá vždy rozdeliť na 2 rovnaké časti, pozri obrázok).
\insp{z6-i-5r.eps}%

Adam teda dostal viac ako 60 kociek; ďalšia možnosť je $2\cdot60=120$:

1. komín má 4~vrstvy, tzn. 30 kociek v~jednej vrstve ($120:4=30$).
To sa dá splniť napr. takto:
\insp{z6-i-5a.eps}%

2. komín má 5~vrstiev, tzn. 24 kociek v~jednej vrstve ($120:5=24$).
To sa dá splniť napr. takto:
\insp{z6-i-5b.eps}%

3. komín má 6 vrstiev, tzn. 20 kociek v~jednej vrstve ($120:6=20$).
To sa dá splniť napr. takto:
\insp{z6-i-5c.eps}%

Tým sme dokázali, že najmenší počet kociek, ktorý mohol Adam dostať, je 120.

Teraz ešte zistíme výšku najvyššieho komína, ktorý sa dá zo 120 kociek
postaviť podľa Adamových pravidiel. Na jednu vrstvu potrebujeme najmenej
8~kociek:
\insp{z6-i-5d.eps}%

V~takom prípade bude mať komín 15 úplných vrstiev ($120:8=15$).
Kocky sú vysoké 4\,cm, takže najvyšší komín, ktorý sa dá postaviť zo 120
kociek, meria 60\,cm ($15\cdot4=60$).
}

{%%%%%   Z6-I-6
\napad
Určte, o~koľko vodorovných a~o~koľko zvislých dielikov sa líšia obvody útvarov
$A$ a~$B$.

\riesenie
Dĺžky strán obdĺžnikov, z~ktorých sa skladá sieť, označme $x$ a~$y$ ($x$ pre
vodorovnú stranu a~$y$ pre zvislú). Obvod útvaru~$A$ je rovný $6x+2y$,
obvod útvaru~$B$ je $4x+6y$ a~obvod útvaru~$C$ je $2x+6y$.

V~obvode útvaru~$A$ sú započítané o~2~dĺžky~$x$ viac
a~o~4~dĺžky~$y$ menej ako v~obvode útvaru~$B$. Obvody $A$ a~$B$ sú podľa zadania rovnaké, teda $2x$ je rovné $4y$. Z toho vidíme, že $x$ je dvakrát
väčšie ako $y$, teda že $x=2y$. Obvod útvaru~$A$ tak môžeme vyjadriť ako
$14y$, odtiaľ $y=56:14=4$\,(cm). Obvod útvaru~$C$ je $10y=10\cdot4=40$\,(cm).
}

{%%%%%   Z7-I-1
\napad
Skúste najskôr určiť, koľko rokov má Dominik.

\riesenie
Pre zjednodušenie budeme vek každého dieťaťa označovať malým začiatočným písmenom jeho
mena.
Podľa zadania platí:
$$
a=4,\ abc=140,\ bcd=280,\ cde=560,\ e=10.
$$
Z~druhej a~tretej rovnosti vyplýva, že $2abc=bcd$, tzn. $d=2a=8$.
Odtiaľ a~z~posledných dvoch rovností dostávame $c=560:(8\cdot10)=7$.
Cilka teda má 7 rokov.

\poznamka
S~uvedenými vzťahmi sa dá samozrejme manipulovať rôzne.
Napr. z~prvej a~druhej rovnosti vyplýva, že $bc=140:4=35$.
Z~tretej rovnosti potom vyplýva, že $d=280:35=8$.
Odtiaľ a~z~posledných dvoch rovností dostávame $c=560:(8\cdot10)=7$.
}

{%%%%%   Z7-I-2
\napad
Zvoľte vhodne neznáme a~pomocou nich vyjadrite sumy pre jednotlivých vnukov.

\riesenie
Najmladší súrodenec môže dostať
$$5,\ 6,\ 7,\ \text{alebo 8}; \eqno (1)
$$
túto hodnotu označíme $p$ (\euro{} písať nebudeme).
Druhý najmladší súrodenec dostane viac ako najmladší o~čiastku, ktorú
označíme~$o$.
Prostredný súrodenec tak dostane o~$2o$ viac ako najmladší atď.
To znamená, že súrodenci postupne dostanú
$$
p,\ p+o,\ p+2o,\ p+3o,\ p+4o. \eqno (2)
$$
Celkom si takto rozdelia $5p+10o$, čo má byť 60.

Ak $p=5$, tak $5p=25$ a~$10o$ musí byť $60-25=35$.
Z~toho vyplýva, že $o=3{,}50$.
Vnuci si v~tomto prípade peniaze rozdelili nasledovne
(v~poradí od najmladšieho):
$$5,\quad 8{,}50,\quad 12,\quad 15{,}50,\quad 19.
$$
Podobne sa dajú určiť všetky zvyšné možnosti:
$$
\begintable
$p$|$o$\|\multispan{5}\hfil odpoveď\hfil\crthick
5|3{,}50\|5, \novb 8{,}50,\novb 12,\novb 15{,}50,\novb 19\hfill\cr
6|3 \|6, \novb 9, \novb 12,\novb 15, \novb 18\hfill\cr
7|2{,}50\|7, \novb 9{,}50,\novb 12,\novb 14{,}50,\novb 17\hfill\cr
8|2 \|8, \novb 10, \novb 12,\novb 14, \novb 16\hfill\endtable
$$

\poznamka
Nie je náhoda, že prostredný súrodenec dostane v~každom prípade rovnakú
sumu.
To si možno všimnúť hneď na začiatku, pretože súrodencov je nepárny počet
a~rozdiel medzi susednými sumami je stále rovnaký.

Ak by sme sumu, ktorú má dostať prostredný vnuk, označili $s$, tak
namiesto \thetag2 píšeme
$$
s-2o,\ s-o,\ s,\ s+o,\ s+2o.
$$
Súčet všetkých týchto súm je $5s$, a~to má byť $60$.
Z~toho vyplýva, že $s=12$.
Všetky možné rozdelenia vreckového sa dajú určiť tak, že zistíme, pre ktoré
$o$ je $12-2o$ niektoré z~čísel \thetag1.
}

{%%%%%   Z7-I-3
\napad
Koho skok bol najkratší?

\riesenie
Najskôr určíme, koho skok meral 127\,cm.
Určite to nebol skok Samov (ten skočil 135\,cm), ani Petrov (skočil viac ako
Juro), ani Filipov (skočil viac ako Mišo). Sú teda dve možnosti: Najmenej
skočil Juro alebo Mišo.

Najskôr preveríme možnosť, že najmenej skočil Juro:
V~takom prípade by Peter skočil 131\,cm ($127+4=131$).
Keďže Filipov skok bol presne v~polovici medzi Petrovým a~Samovým skokom,
musel by merať 133\,cm ($135-133=133-131=2$).
Potom by Mišo skočil 126\,cm ($133-7=126$), čo ale nie je možné, pretože by to
bolo menej ako Jurov najkratší skok.
\insp{z7-i-3a.eps}%

Musela teda nastať druhá situácia, čiže najmenej skočil Mišo:
Potom Filip skočil 134\,cm ($127+7=134$).
Z~toho vyplýva, že Peter skočil 133\,cm ($135-134=1=134-133$).
A~napokon Juro skočil 129\,cm ($133-4=129$).
\insp{z7-i-3b.eps}%
}

{%%%%%   Z7-I-4
\napad
Ktoré dve prasiatka určite kozu Lujzu obsluhovali?

\riesenie
Aby celková trojdňová Lujzina útrata bola rovnaká, musel niektorý deň obsluhovať
Sašík (účtoval si menej, ako naozaj mala zaplatiť) a~niektorý deň Pašík
(účtoval si viac, ako naozaj mala zaplatiť). Zvyšný tretí deň mohlo
obsluhovať ktorékoľvek z~troch prasiatok (nevieme o~tom, že by Rašík bol chorý
aj v~ďalších dňoch).
Postupne rozoberieme všetky tri možnosti:
\begin{enumerate}
\item Obsluhoval raz Sašík, raz Pašík a~raz Rašík:
To znamená, že
6~grajciarov, ktoré Lujza zaplatila Pašíkovi navyše,
predstavuje zľavu 20\,\%, ktorú dostala od Sašíka.
Pätina ceny buchty je teda 6~grajciarov, čo znamená, že Rašík by v~tomto
prípade účtoval $5\cdot6=30$ grajciarov.
\item Obsluhoval raz Sašík a~dvakrát Pašík:
Pašíkovi Lujza zaplatila navyše $2\cdot6=12$ grajciarov.
Týchto 12 grajciarov je zároveň 20\,\% zľava, ktorú dostala od Sašíka.
Rašík by teda za buchtu v~tomto prípade účtoval $5\cdot12=60$ grajciarov.
\item Obsluhoval dvakrát Sašík a~raz Pašík:
Lujza dostala dvakrát 20\,\% zľavu z~rovnakej sumy.
To je rovnaké, ako keby raz zaplatila plnú čiastku a~raz dostala
zľavu 40\,\%.
Táto zľava zodpovedá 6~grajciarom, ktoré zaplatila navyše Pašíkovi.
Dve pätiny Rašíkovej ceny sú 6~grajciarov, pätina je teda 3~grajciare.
Rašík by si v~tomto prípade účtoval $5\cdot3=15$ grajciarov.
\end{enumerate}

Rašík za jednu čučoriedkovú buchtu účtuje 15, 30, alebo 60 grajciarov.
}

{%%%%%   Z7-I-5
\napad
Skúste najskôr rozdeliť obdĺžnik na požadované útvary bez podmienky
rovnosti obsahov. Potom pozmeňte svoje delenie tak, aby obsahy útvarov
boli rovnaké.

\riesenie
Rozdelíme obdĺžnik na požadované útvary, zatiaľ bez ohľadu na rovnosť obsahov.
Tu je niekoľko možností:
\insp{z7-i-5a.eps}%

Teraz skúsime modifikovať delenie tak, aby obsahy útvarov boli rovnaké.
Celý obdĺžnik pozostáva z~24~dielikov, preto každý z troch útvarov musí mať obsah
$24:3=8$ dielikov.
Pri menení útvarov stačí zabezpečiť, aby dva z~týchto útvarov mali
obsah 8~dielikov, obsah tretieho potom bude nutne taký istý.

Pre ukážku upresníme tretie z~vyššie uvedených delení
-- všetky vrcholy uvažujeme v~mrežových bodoch:
\insp{z7-i-5b.eps}%

Obsah štvoruholníka je práve $1+2+5=8$ dielikov.
Trojuholník má zrejme taký istý obsah, našli sme teda jedno z~mnohých možných
riešení.
Pre inšpiráciu uvádzame niekoľko ďalších:
\insp{z7-i-5c.eps}%

\poznamka
Uvedené riešenia využívajú iba mrežové body siete, čo však nie je nutné.
Existujú samozrejme riešenia, kde zodpovedajúce vrcholy nie sú v~mrežových
bodoch.
V~takých prípadoch však môže byť zdôvodnenie rovnosti obsahov
komplikovanejšie.
}

{%%%%%   Z7-I-6
\napad
Určte, ktorý zo psov je vyšší a~o~koľko.

\riesenie
Keby bol Cézar rovnako vysoký ako Dunčo, boli by obe hodnoty v~zadaní
rovnaké. Rozdiel medzi nameranými hodnotami je $90-70=20$\,(cm), čo
znamená, že jeden zo psov je o~10\,cm vyšší ako druhý. Väčšia hodnota
zodpovedá situácii, keď Dunčo stojí na búde, čo znamená, že Dunčo je
o~10\,cm vyšší ako Cézar.

Cézar na psej búde je o~70\,cm vyšší ako Dunčo a~Dunčo je o~10\,cm
vyšší ako Cézar. Teda Cézar na psej búde je o~80\,cm vyšší ako samotný
Cézar; búda je vysoká 80\,cm.
\insp{z7-i-6.eps}%

\ineriesenie
Cézar na búde je o~70\,cm vyšší ako Dunčo na zemi a~Dunčo na búde je o~90\,cm vyšší ako Cézar na zemi.
Keby sme na Cézara stojaceho na búde postavili ešte ďalšiu rovnako vysokú
búdu a~na ňu Dunča, bude táto zostava o~$70+90=160$\,(cm) vyššia ako keby
stál Cézar na Dunčovi.
To znamená, že dve búdy sú vysoké 160\,cm, búda je teda vysoká 80\,cm.

\poznamka
Výšku psej búdy označíme $b$, výšku Cézara označíme~$c$ a~výšku Dunča
označíme~$d$ (všetko v~cm).
Informácie zo zadania pri tomto označení zapíšeme takto:
$$
\aligned
b+c&=d+70,\\
b+d&=c+90.
\endaligned
$$
Uvedené riešenie potom možno interpretovať nasledovne.

Výškový rozdiel Dunča a~Cézara je možné určiť odčítaním oboch rovníc:
$$
\aligned
d-c&=c-d+20,\\
2(d-c)&=20,\\
d-c&=10.
\endaligned
$$
Z~toho vyplýva, že $d=c+10$ a~z~prvej rovnice potom dostávame
$b=10+70=80$.

Naopak sčítaním oboch rovníc dostaneme:
$$
\aligned
2b+c+d&=c+d+160,\\
2b&=160,\\
b&=80.
\endaligned
$$
}

{%%%%%   Z8-I-1
\napad
Skúste uvažovať menšie počty cestujúcich.

\riesenie
Nejaký čas budú cestujúci z~električky len vystupovať.
Po 42~zastaveniach zostane v~električke $300-42\cdot7=6$ ľudí.
Podľa uvedených pravidiel sa počet cestujúcich bude ďalej vyvíjať nasledovne:
$$
\dots 6,\ 11,\ 4,\ 9,\ 2,\ 7,\ 0.
$$
V~električke teda neostane žiadny cestujúci.

Teraz si predstavme všeobecnú situáciu, keď je v~električke neznámy počet ľudí.
Títo budú postupne vystupovať, kým ich nebude v~električke menej ako 7.
To znamená, že bez ohľadu na to, koľko ľudí bolo v~električke na začiatku,
časom bude ich počet rovný niektorému z~čísel 0, 1, 2, 3, 4, 5 alebo 6.
Stačí teda uvažovať iba tieto možnosti.

V~predchádzajúcom odseku sme zistili, že ak v~električke zostanú 6, 4,
alebo 2 cestujúci, tak sa električka nakoniec vyprázdni.
Ak v~električke zostane 5 cestujúcich, tak sa ich počet bude ďalej meniť
takto:
$$
\dots,\ 5,\ 10,\ 3,\ 8,\ 1,\ 6,\ \dots
$$
Túto situáciu už poznáme; električka sa nakoniec vyprázdni.
V~postupnosti sa objavujú aj čísla 3 a~1, čím sme vyčerpali všetky
možnosti.
Bez ohľadu na počiatočný počet cestujúcich sa električka nakoniec vždy
vyprázdni.

\poznamka
Skúste si rozmyslieť, ako riešenie úlohy ovplyvňujú čísla 7 a~5 zo zadania.
Vyprázdnila by sa električka vždy
aj pre inú dvojicu čísel?
}

{%%%%%   Z8-I-2
\napad
Skúste najskôr rozdeliť obdĺžnik na požadované útvary bez podmienky
rovnosti obsahov. Potom pozmeňte svoje delenie tak, aby obsahy útvarov
boli rovnaké.

\riesenie
Rozdelíme obdĺžnik na požadované útvary, zatiaľ bez ohľadu na rovnosť
obsahov. Tu je jedna z~možností:
\insp{z8-i-2a.eps}%

Teraz skúsime modifikovať delenie tak, aby obsahy útvarov boli rovnaké.
Celý obdĺžnik pozostáva z~24~dielikov, preto každý zo štyroch útvarov musí mať
obsah $24:4=6$~dielikov. Pri menení útvarov stačí zabezpečiť, aby tri
z~týchto útvarov mali obsah 6~dielikov, obsah štvrtého potom bude nutne
taký istý.

Upresnenie vyššie uvedeného delenia môže byť nasledujúce~-- všetky
vrcholy uvažujeme v~mrežových bodoch:
\insp{z8-i-2b.eps}%

Trojuholník aj štvoruholník zrejme majú obsah 6~dielikov.
Obsah päťuholníka je $2+2+2=6$ dielikov.
Našli sme teda jedno z~mnohých možných riešení.
Pre inšpiráciu uvádzame niekoľko ďalších:
\insp{z8-i-2c.eps}%

\poznamka
Uvedené riešenia využívajú iba mrežové body siete, čo však nie je nutné.
Existujú riešenia, kde zodpovedajúce vrcholy nie sú v~mrežových
bodoch.
V~takých prípadoch však môže byť zdôvodnenie rovnosti obsahov
komplikovanejšie.
}

{%%%%%   Z8-I-3
\napad
Je možné odčítanie v~ráde desiatok opraviť len zmenou cifier v~ráde jednotiek?

\riesenie
Príklad nie je správne, a~to ani v~ráde jednotiek.
Keď skontrolujeme odčítanie v~ráde desiatok a~v~ráde stoviek,
zistíme, že chyba je v~oboch rozdieloch väčšia ako 1. To znamená, že taký
rozdiel nie je možné opraviť iba zmenou cifier o~rád nižšie.
Ak máme spraviť tri zmeny, musí byť v~každom stĺpci práve jedna.
Preto stačí, keď budeme uvažovať nasledujúcich šesť možností:
$$
\alggg{&7&2&*\\-&3&*&7}{&*&8&8}\quad
\alggg{&7&2&*\\-&*&0&7}{&1&*&8}\quad
\alggg{&7&*&4\\-&3&0&*}{&*&8&8}\quad
\alggg{&*&2&4\\-&3&0&*}{&1&*&8}\quad
\alggg{&7&*&4\\-&*&0&7}{&1&8&*}\quad
\alggg{&*&2&4\\-&3&*&7}{&1&8&*}
$$
Pre každú z~týchto možností postupne odzadu doplníme správne cifry.
Napr. v~prvom prípade môžeme postupovať nasledovne:
\begin{itemize}
\item správna cifra na mieste jednotiek musí byť 5, pretože jedine $1{\bold5}-7=8$
(v~ráde desiatok pripočítame 1),
\item správna cifra na mieste desiatok musí byť 3, pretože jedine $12-{\bold3}-1=8$
(v~ráde stoviek pripočítame 1),
\item správna cifra na mieste stoviek musí byť 3, pretože $7-3-1={\bold3}$.
\end{itemize}

Podobným spôsobom rozoberieme ostatné možnosti.
Úloha má celkom šesť riešení:
$$
\alggg{&7&2&{\bold5}\\-&3&{\bold3}&7}{&{\bold3}&8&8}\quad
\alggg{&7&2&{\bold5}\\-&{\bold6}&0&7}{&1&{\bold1}&8}\quad
\alggg{&7&{\bold9}&4\\-&3&0&{\bold6}}{&{\bold4}&8&8}\quad
\alggg{&{\bold4}&2&4\\-&3&0&{\bold6}}{&1&{\bold1}&8}\quad
\alggg{&7&{\bold9}&4\\-&{\bold6}&0&7}{&1&8&{\bold7}}\quad
\alggg{&{\bold5}&2&4\\-&3&{\bold3}&7}{&1&8&{\bold7}}
$$

\poznamka
Úvodný postreh nie je úplne samozrejmý, ale pri systematickom postupe by naň mal časom prísť každý.
Uvedomte si, že bez tohto poznatku by diskusia musela zahŕňať celkom
$3\cdot3\cdot3=27$ možností.
}

{%%%%%   Z8-I-4
\napad
Koľko je na obrázku rovnostranných trojuholníkov?
Skúste ich nejako využiť.

\riesenie
Na obrázku je spolu 8~rovnostranných trojuholníkov: dva veľké ($ABC$, $DEF$)
a~šesť malých ($LAG$, $GEH$, $HBI$, $IFJ$, $JCK$, $KDL$).
Veľké trojuholníky sú tak priamo zadané.
Zdôvodnenie, že malé trojuholníky sú rovnostranné, ukážeme iba pre
trojuholník $LAG$:
Uhol pri vrchole~$A$ má veľkosť $60\st$, pretože je vnútorným uhlom
rovnostranného trojuholníka $ABC$.
Priamky $LG$ a~$CB$ sú podľa zadania rovnobežné, uhly $ALG$ a~$ACB$ sú
teda súhlasné.
Oba tieto uhly majú veľkosť $60\st$, pretože uhol $ACB$ je vnútorným
uhlom v~rovnostrannom trojuholníku $ABC$.
Dva z~troch vnútorných uhlov v~trojuholníku $LAG$ majú veľkosť $60\st$, preto je
tento trojuholník rovnostranný.

Obvod dvanásťuholníka môžeme vyjadriť ako obvod dvoch veľkých
trojuholníkov zmenšený o~obvod šesťuholníka $GHIJKL$.
Z~rovnostrannosti malých trojuholníkov vyplýva, že úsečky $GL$ a~$AL$,
$KJ$ a~$KC$ sú po dvojiciach zhodné.
Dĺžka lomenej čiary $GLKJ$ je teda rovnaká ako dĺžka úsečky~$AC$, \tj. 5\,cm.
Podobne odvodíme, že aj dĺžka lomenej čiary $JIHG$ je 5\,cm.
Z~toho poznáme obvod šesťuholníka $GHIJKL$;
obvod dvanásťuholníka $AGEHBIFJCKDL$ je preto rovný
$$
6\cdot5-2\cdot5=20\,(\Cm),
$$
a~to bez ohľadu na vzájomnú polohu veľkých trojuholníkov, ktorá vyhovuje zadaniu.


\ineriesenie
Dvanásťuholník $AGEHBIFJCKDL$ je tvorený stranami šiestich malých trojuholníkov,
pričom z~každého malého trojuholníka sa takto uplatnia dve z~jeho troch strán.
Všetky tieto trojuholníky sú rovnostranné, preto je obvod dvanásťuholníka
rovný $\frac23$ súčtu obvodov malých trojuholníkov.
Súčet obvodov šiestich malých trojuholníkov je však rovnaký ako súčet obvodov dvoch
veľkých trojuholníkov;
obvod dvanásťuholníka $AGEHBIFJCKDL$ je preto rovný
$$
\frac23\cdot6\cdot5=20\,(\Cm),
$$
a~to bez ohľadu na vzájomnú polohu veľkých trojuholníkov, ktorá vyhovuje zadaniu.

\poznamky
Predchádzajúce úvahy je možné zjednodušiť objavom, že malé trojuholníky sú po
dvojiciach zhodné ($LAG$ a~$IFJ$, $GEH$ a~$JCK$, $HBI$ a~$KDL$).
Tento poznatok je však nutné zdôvodniť, čo ukážeme pre trojuholníky
$LAG$ a~$IFJ$:
V~úvode sme zdôvodnili, že oba trojuholníky sú rovnostranné.
Pritom výška oboch trojuholníkov je rovnaká -- je to výška rovnostranného
trojuholníka so stranou 5\,cm zmenšená o~vzdialenosť rovnobežiek $IJ$ a~$GL$.
Tieto trojuholníky sú teda naozaj zhodné.
}

{%%%%%   Z8-I-5
\napad
Skúste si hmotnosti a~ich rozdiely znázorniť graficky.

\riesenie
Z~rozdielov zmeraných pri vážení s~vedúcim vyplýva, že Pat je ťažší ako Mat.
Hmotnosti pri týchto váženiach znázorníme graficky:
\insp{z8-i-5.eps}%

Rozdiel hmotnosti Pata a~Mata je zvýraznený tučnou prerušovanou čiarou. Na
schéme vidíme, že dvojnásobok tohto rozdielu je rovný
$86-64=22$\,(kg). Pat je teda o~11\,kg ťažší ako Mat. Pri vážení
odpadu Pat postavením sa na váhu zväčšil meraný rozdiel, Mat ho
postavením sa na váhu zmenšil. Nameraná hodnota je teda o~toľko väčšia ako
hmotnosť odpadu, o~koľko je Pat ťažší ako Mat. Skutočná hmotnosť
odpadu je $332-11=321$\,(kg).

\ineriesenie
Hmotnosti Pata, Mata, vedúceho a~odpadu označme postupne $p$, $m$,
$v$, $x$. Pri prvom vážení bolo na váhe auto, odpad a~Pat, pri druhom
vážení auto a~Mat. Rozdiel zmeraných hmotností môžeme teda vyjadriť:
$$
x+p-m=332. \eqno (1)
$$
Ďalšie dva rozdiely zmeraných hmotností vyjadríme podobne:
$$\eqalignno{
v+p-m&=86, &(2) \cr
v+m-p&=64. &(3)
}
$$
Odčítaním rovníc \thetag2 a~\thetag3 dostaneme:
$$\aligned
2(p-m)&=86-64,\\
p-m&=11.
\endaligned
$$
Dosadením tohto rozdielu do rovnice \thetag1 dostaneme:
$$\aligned
x+11&=332,\\
x&=321.
\endaligned
$$
Hmotnosť vyvezeného odpadu je 321\,kg.

\poznamka
Všimnite si, že hmotnosť vedúceho sa dá určiť na rozdiel od hmotnosti Pata či
Mata jednoznačne:
napr. sčítaním rovníc \thetag2 a~\thetag3 dostaneme
$2v=86+64=150$, teda $v=75$\,(kg).
}

{%%%%%   Z8-I-6
\napad
Najskôr určte, koľkokrát som mohol ísť po jednotlivých schodiskách.

\riesenie
Keďže druhé schodisko má menej schodov ako prvé, musel som po ňom ísť
viackrát ako po prvom.
Behom dňa som teda po schodiskách mohol ísť takto:
\begin{itemize}
\item 1-krát po prvom a~9-krát po druhom,
\item 2-krát po prvom a~8-krát po druhom,
\item 3-krát po prvom a~7-krát po druhom,
\item 4-krát po prvom a~6-krát po druhom.
\end{itemize}
Označíme počet schodov na prvom schodisku~$x$, na druhom schodisku
ich je $x-11$. Na každom zo schodísk som spolu zdolal rovnaký počet
schodov. Táto podmienka dáva pre každú z~vyššie uvedených možností
rovnicu s~neznámou~$x$, ktorú vyriešime. V~prvom prípade dostávame:
$$\aligned
x&=9x-99,\\
8x&=99,\\
x&=\frac{99}8.
\endaligned
$$
Počet schodov je prirodzené číslo, takže tento prípad nastať nemohol.
V~druhom prípade dostávame rovnicu
$$
2x=8x-88,
$$
ktorej riešením je $x=\frac{88}6$, takže táto možnosť tiež nevyhovuje.
V~treťom prípade dostávame
$$
3x=7x-77,
$$
s~riešením $x=\frac{77}4$, čo je opäť nevyhovujúce.
V~štvrtom prípade dostávame
$$
4x=6x-66,
$$
s~riešením $x=\frac{66}2=33$, čo je jediná vyhovujúca možnosť.

Prvé schodisko má 33~schodov, každý schod je vysoký 10\,cm,
výškový rozdiel medzi poschodiami je 3,3\,m.

\poznamka
Pokiaľ uvažujeme, že po prvom schodisku som išiel behom dňa spolu
$a$-krát a~po druhom spolu $b$-krát, tak predchádzajúce požiadavky môžeme
formulovať takto:
$$
a+b=10\ \text{a}\ ax=bx-11b.
$$
Z~druhej rovnice vyjadríme $x$:
$$
x=\frac{11b}{b-a}.
$$
Aby $x$ bolo kladné, musí byť $b>a$, čo spolu s~podmienkou $a+b=10$ dáva
práve štyri možnosti uvedené vyššie.
Aby $x$ bolo prirodzené číslo, musí byť $11b$ deliteľné $b-a$. Keďže
11~je prvočíslo a~$b-a$ nemôže byť ani~11, ani~1, musí byť $b$
deliteľné $b-a$. Z~uvedených štyroch možností tejto podmienke vyhovuje
iba dvojica $a=4$ a~$b=6$.
}

{%%%%%   Z9-I-1
\napad
Ako sa zadané informácie o~deliteľnosti prejavia na deliteľnosti hľadaného dvojciferného čísla?

\riesenie
Číslo, ktoré vzniklo dvojnásobným zápisom mysleného čísla, je podľa zadania
deliteľné deviatimi, preto musí byť aj jeho ciferný súčet deliteľný deviatimi.
Avšak ciferný súčet takto vzniknutého čísla je dvojnásobkom ciferného súčtu
mysleného čísla.
Preto je ciferný súčet mysleného čísla nutne deliteľný deviatimi.

Keď číslo vzniknuté trojnásobným zápisom mysleného čísla je deliteľné ôsmimi,
musí byť deliteľné aj štyrmi.
Jeho posledné dvojčíslie teda musí byť deliteľné štyrmi a~týmto dvojčíslím je
práve myslené číslo.

Zistili sme, že Petrovo myslené číslo je deliteľné deviatimi a~súčasne
štyrmi, je teda deliteľné 36.
Jediné dvojciferné čísla deliteľné 36 sú 36 a~72.
Overme, ktoré z~týchto dvoch možností vyhovuje podmienke v~zadaní o~deliteľnosti ôsmimi:
\begin{itemize}
\item 363636 nie je deliteľné 8,
\item 727272 je deliteľné 8.
\end{itemize}
\noindent
Peter si môže myslieť jedine číslo 72.

\poznamka
V~riešení používame tieto dobre známe kritériá o~deliteľnosti:
\begin{itemize}
\item číslo je deliteľné štyrmi práve vtedy, keď jeho posledné dvojčíslie je
deliteľné štyrmi,
\item číslo je deliteľné ôsmimi práve vtedy, keď jeho posledné trojčíslie je
deliteľné ôsmimi,
\item číslo je deliteľné deviatimi práve vtedy, keď jeho ciferný súčet je
deliteľný deviatimi.
\end{itemize}
}

{%%%%%   Z9-I-2
\napad
Použite opakovane Pytagorovu vetu.

\riesenie
Pätu kolmice na stranu~$AB$ z~bodu~$C$, resp. $D$ označíme $C'$, resp.
$D'$.
\insp{z9-i-2a.eps}%

\noindent
Keďže $ABCD$ je lichobežník, platí $|C'D'|=|CD|=11\cm$, a~keďže je
navyše rovnoramenný,
$$
|AD'|=|BC'|=\frac{|AB|-|C'D'|}2=\frac{31-11}2=10\,(\Cm).
$$
Pomocou Pytagorovej vety v~pravouhlom trojuholníku $BC'C$ (ktorý je
zhodný s~trojuholníkom $AD'D$) vypočítame výšku lichobežníka:
$$
|CC'|=|DD'|=\sqrt{|BC|^2-|BC'|^2}=\sqrt{26^2-10^2}=24\,(\Cm).
$$
Na strane~$AB$ je určený bod~$E$ tak, že $|AE|:|EB|=3:28$.
Keďže $3+28=31$ a~$|AB|=31\cm$, je $|AE|=3\cm$ a~$|EB|=28\cm$.
Odtiaľ odvodzujeme, že
$$
\aligned
\vert ED'\vert &=\vert AD'\vert -\vert AE\vert =10-3=7\,(\Cm),\\
\vert EC'\vert &=\vert EB\vert -\vert BC'\vert =28-10=18\,(\Cm).
\endaligned
$$
\insp{z9-i-2b.eps}%

\noindent
Pomocou Pytagorovej vety v~pravouhlom trojuholníku $ED'D$, resp. $EC'C$
vypočítame dĺžku strany~$ED$, resp. $EC$:
$$
\aligned
\vert ED\vert &=\sqrt{\vert ED'\vert ^2+\vert DD'\vert ^2}=\sqrt{7^2+24^2}=25\,(\Cm),\\
\vert EC\vert &=\sqrt{\vert EC'\vert ^2+\vert CC'\vert ^2}=\sqrt{18^2+24^2}=30\,(\Cm).
\endaligned
$$
Teraz môžeme vypočítať obvod trojuholníka $CDE$:
$$
o=|CD|+|DE|+|EC|=11+25+30=66\,(\Cm).
$$
}

{%%%%%   Z9-I-3
\napad
Vyjadrite počet použitých dlaždíc v~závislosti od dĺžky ich strany.

\riesenie
Označme $m$ dĺžku strany malej dlaždice a~$v$~dĺžku strany veľkej dlaždice (v~cm). Ak pokryjeme podlahu menšími dlaždicami, ku kratšej
strane ich bude priliehať $\frac{360}m$ a~k~dlhšej strane $\frac{540}m$,
celkom ich teda bude $\frac{360}m\cdot\frac{540}m$.
Podobne celkový počet väčších dlaždíc bude $\frac{360}v\cdot\frac{540}v$.
Podľa zadania tak musí platiť
$$
\frac{360}m\cdot\frac{540}m=\frac{360}v\cdot\frac{540}v+30.\eqno (1)
$$

Keďže $m:v=2:3$, môžeme písať $m=2x$ a~$v=3x$ pre novú
neznámu~$x$.
Dosadením do predchádzajúcej rovnice a~ďalšími úpravami dostávame:
$$
\aligned
\frac{180}{x}\cdot\frac{270}{x}&=\frac{120}{x}\cdot\frac{180}{x}+30, \\
180\cdot270&=120\cdot180+30x^2,\\
180\cdot150&=30x^2,\\
900&=x^2.
\endaligned
$$
Čísla $m$, $v$ a~$x$ sú kladné, teda $x=30$ a~$m=60$, $v=90$\,(cm).

Strany dlaždíc sú dlhé 60\,cm a~90\,cm.


\inynapad
Určte priamo zo zadaného pomeru $2:3$ pomer počtov jednotlivých dlaždíc.

\ineriesenie
Keďže pomer dĺžok strán dlaždíc je $2:3$, pomer ich obsahov je
$4:9$.
Označíme $p_m$ počet malých dlaždíc a~$p_v$ počet veľkých dlaždíc.
Keďže oba typy dlaždíc pokryjú celý obdĺžnik, musí platiť
$$
p_m:p_v=9:4.\eqno(2)
$$
Zo zadania vyplýva, že $p_m=p_v+30$, čo spolu s~predchádzajúcim pomerom dáva:
$$
\aligned
\frac94p_v&=p_v+30,\\
\frac54p_v&=30,\\
p_v&=24.
\endaligned
$$

Aby sme vedeli určiť dĺžku strany veľkej dlaždice, musíme zistiť, ako
tieto dlaždice pokrývajú obdĺžnikovú podlahu.
Dĺžky strán obdĺžnika sú v~pomere $360:540=2:3$, v~rovnakom pomere musia byť
aj počty dlaždíc priliehajúcich k~zodpovedajúcim stranám.
Zo všetkých možných rozkladov celkového počtu veľkých dlaždíc,
$$
24=1\cdot24=2\cdot12=3\cdot8=4\cdot6,
$$
tejto podmienke vyhovuje práve posledný rozklad.
Dĺžka strany veľkej dlaždice je teda $\frac{360}4=\frac{540}6=90$\,(cm).
Dĺžka strany malej dlaždice je potom $\frac23\cdot90=60$\,(cm).

\poznamky
Ak v~prvom riešení namiesto pomocnej neznámej~$x$ vyjadríme napr.
$m=\frac23v$, tak po dosadení do \thetag1 dostávame rovnicu s~neznámou~$v$.
Po úpravách dostaneme $v=90$ a~$m=\frac23\cdot90=60$\,(cm).

Ak v~druhom riešení namiesto $p_m=\frac94p_v$ vyjadríme $p_v=\frac49p_m$,
tak po dosadení do \thetag2 dostávame rovnicu s~neznámou~$p_m$.
Po jednoduchej úprave dostaneme $p_m=54$, jediný vyhovujúci rozklad je
$54=6\cdot9$, takže $m=\frac{360}6=\frac{540}9=60$\,(cm).
}

{%%%%%   Z9-I-4
\napad
Bod~$F$ leží na uhlopriečke~$IC$.

\riesenie
Zo zadania vyplýva, že trojuholník $ABD$ je zhodný napr. aj s~trojuholníkom
$IJF$.
Zodpovedajúca zhodnosť je osová súmernosť podľa priamky, ktorá je kolmá
na~úsečku $AI$ a~prechádza jej stredom.
Podľa tejto priamky sú súmerné aj uhlopriečky pravouholníka
$ACKI$, čo znamená, že bod~$F$ leží na uhlopriečke~$IC$.
\insp{z9-i-4a.eps}%

Obsah pravouholníka $ABFE$ teda môžeme vyjadriť ako obsah trojuholníka
$IAC$ zmenšený o~obsahy trojuholníkov $IEF$ a~$FBC$.
Podobne môžeme obsah pravouholníka $FHKJ$ vyjadriť ako obsah trojuholníka
$CKI$ zmenšený o~obsahy trojuholníkov $FJI$ a~$CHF$.
Trojuholníky $IAC$ a~$CKI$, $IEF$ a~$FJI$, $FBC$ a~$CHF$ sú však po
dvojiciach zhodné, takže majú po dvojiciach rovnaké obsahy.
Z~toho vyplýva, že pravouholníky $ABFE$ a~$FHKJ$ majú rovnaký obsah.

\inynapad
Trojuholníky $ABD$ a~$GFD$ sú podobné.

\ineriesenie
Zo zadania vyplýva, že trojuholníky $ABD$ a~$GFD$ majú po dvojiciach zhodné vnútorné uhly,
takže sú podobné a~zodpovedajúce strany sú teda v~rovnakom pomere:
$$
|AB|:|BD|=|GF|:|FD|,
$$
čiže
$$
|AB|\cdot|FD|=|GF|\cdot|BD|.
$$
Keďže $|AB|=|EF|$ a~$|BD|=|FJ|$, môžeme predchádzajúcu rovnosť interpretovať ako
rovnosť obsahov pravouholníkov $EFDM$ a~$GFJN$.
\insp{z9-i-4b.eps}%

%\hfuzz.564-pt
Pravouholník $ABFE$ pozostáva z~pravouholníkov $EFDM$ a~$ABDM$,
pravouholník $FHKJ$ pozostáva z~pravouholníkov $GFJN$ a~$GHKN$,
pričom pravouholníky $ABDM$ a~$GHKN$ sú zrejme zhodné.
Z~toho vyplýva, že pravouholníky $ABFE$ a~$FHKJ$ majú rovnaký obsah.

\poznamka
Druhú časť prvého riešenia možno nájsť ako samostatné tvrdenie v~Euklidových
Základoch (43.~tvrdenie v~I.~knihe).
Tento poznatok sa používa napr. pri konštrukcii pravouholníka, ktorý má danú
jednu stranu a~rovnaký obsah ako iný pravouholník (príp. trojuholník či
všeobecný mnohouholník).
}

{%%%%%   Z9-I-5
\napad
Dokážte, že počas experimentu musela minútová ručička dosiahnuť ku dvanástke.

\riesenie
Ak počas experimentu nedosiahla minútová ručička ku dvanástke,
môžeme čísla v~stĺpci minút označiť takto:
$$
x,\ x+3,\ x+6,\ x+9,\ x+12,\ x+15,\ x+18,\ x+21,\ x+24.
$$
Ich súčet je $9x+108$, čo má byť rovné 258:
$$
\aligned
9x+108&=258,\\
9x&=150.
\endaligned
$$
To však nie je možné, pretože $x$ je prirodzené číslo a~150 nie je násobkom 9.

Preto musela minútová ručička v~priebehu experimentu dosiahnuť ku dvanástke.
Od tohto okamihu sú všetky čísla oproti predchádzajúcej postupnosti zmenšené o~60.
Ak počet meraní od dosiahnutia dvanástky označíme~$z$, tak súčet čísel
v~danom stĺpci je rovný
$$
9x+108-60z=258,\eqno (1)
$$
pričom $z$ je prirodzené číslo od 1 do 8.
Po úpravách dostávame:
$$
\aligned
9x&=150+60z,\\
3x&=50+20z.
\endaligned
$$
Prirodzené číslo na ľavej strane je násobkom troch.
Aby bolo číslo na pravej strane tiež násobkom troch, musí byť $z$ rovné 2, 5 alebo 8.
Pre každú hodnotu $z$~vypočítame $x$ a~urobíme diskusiu:
\begin{itemize}
\item Pre $z=2$ dostaneme $x=30$. Pokiaľ by experiment začal v~30.~minúte,
skončil by v~54.~minúte a~k~prechodu cez dvanástku by vôbec nedošlo.
To je v~rozpore s~predpokladom $z=2$.
\item Pre $z=5$ dostaneme $x=50$. Pokiaľ by experiment začal v~50.~minúte,
skončil by v~14.~minúte:
$$
50,\ 53,\ 56,\ 59,\ 2,\ 5,\ 8,\ 11,\ 14.\eqno (2)
$$
Po dosiahnutí dvanástky by sa tak uskutočnilo
päť meraní, čo je v~súlade s~predpokladom $z=5$.
\item Pre $z=8$ dostaneme $x=70$. Avšak $x$ označuje minúty po celej hodine a~môže
nadobúdať iba hodnoty 0 až 59.
\end{itemize}
Z~uvedených riešení rovnice \thetag1 vyhovuje iba druhá možnosť.
Čísla, ktoré boli napísané v~stĺpci minút, sú uvedené v~riadku \thetag2.
}

{%%%%%   Z9-I-6
\napad
Koľko toho Vendelín zjedol a~vypil v~pondelok a~v~utorok dokopy?

\riesenie
Všimnime si, že Vendelín zjedol a~vypil v~stredu to isté, čo v~pondelok
a~v~utorok dokopy. Lenže v~pondelok a~v~utorok dokopy utratil
$56+104=160$ grajciarov, zatiaľ čo v~stredu platil 112 grajciarov, čo je
o~48~grajciarov menej. Diskutujme, kto mohol obsluhovať v~stredu:
\begin{itemize}
\item Keby to bol poctivec Rašík, musel by hostinec počas pondelka a~utorka
oklamať Vendelína práve o~48~grajciarov. Pri dvoch plateniach však možno
zákazníka oklamať maximálne o~20~grajciarov. Možnosť, že v~stredu
obsluhoval Rašík, preto zavrhujeme.
\item Keby to bol Pašík, bola by skutočná stredajšia cena $112-10=102$
grajciarov. Vendelína by museli počas prvých dvoch dní oklamať dokonca o~58~grajciarov.
Možnosť, že v~stredu obsluhoval Pašík, tiež zavrhujeme.
\item Keby to bol Sašík, predstavovala by útrata 112 grajciarov 80\,\%
skutočnej ceny.
To znamená, že skutočná stredajšia cena by bola $112:0{,}8=140$
grajciarov. Hostinec by musel počas pondelka a~utorka oklamať Vendelína o~20~grajciarov.
To sa mohlo stať len vtedy, ak ho v~oba dni obsluhoval
nečestný Pašík. Táto možnosť vyhovuje.
\end{itemize}
Poznáme teda odpoveď na prvú otázku: V~pondelok a~utorok obsluhoval Pašík,
v~stredu Sašík. Skutočné ceny (\tj. ceny podľa Rašíka) tak boli:
\begin{itemize}
\item pondelok (3 koláčiky, 1 džbánok): $56-10=46$ grajciarov,
\item utorok (5 koláčikov, 3 džbánky): $104-10=94$ grajciarov,
\item streda (8 koláčikov, 4 džbánky): $112:0{,}8=140$ grajciarov.
\end{itemize}
Z~Vendelínovej stredajšej objednávky môžeme odvodiť, že 2~koláčiky a~1~džbánok
džúsu stojí $140:4=35$ grajciarov.
Porovnaním s~Vendelínovou pondelkovou objednávkou zisťujeme, že skutočná cena
jedného koláčika je $46-35=11$ grajciarov.
Podľa pondelkovej objednávky určíme aj skutočnú cenu jedného džbánku:
$46-3\cdot11=13$ grajciarov.
Tým máme odpoveď na druhú otázku: Rašík účtuje za jeden koláčik 11~grajciarov
a~za jeden džbánok džúsu 13~grajciarov.
}

{%%%%%   Z4-II-1
...}

{%%%%%   Z4-II-2
...}

{%%%%%   Z4-II-3
...}

{%%%%%   Z5-II-1
Máme určiť, koľko najviac miest môže byť na parkovisku, musíme teda
zistiť, ako najmenej úsporne môžu byť autobusy zaparkované.
Uvažujeme také rozmiestnenie, aby medzery medzi autobusmi boli najväčšie
možné a~súčasne také, aby sa do týchto medzier žiadny ďalší autobus
nezmestil. Zo zadania vyplýva, že maximálna možná medzera pozostáva z~dvoch
parkovacích miest.

Znázorníme, ako boli autobusy zaparkované na začiatku:
\insp{z5-ii-1a.eps}%

Pred prvým autobusom sú štyri voľne miesta.
Tam sa zmestí jeden autobus (a~to dvojakým spôsobom).

Za druhým autobusom je sedem voľných miest.
Tam by sa mohli zmestiť dva autobusy; ak je však jeden autobus zaparkovaný
ako na nasledujúcom obrázku, tak sa tam už žiadny ďalší nezmestí:
\insp{z5-ii-1b.eps}%

Zvyšné dva autobusy budú zaparkované medzi pôvodnými dvoma
autobusmi v~strednej časti, ktorej dĺžku nepoznáme.
Najmenej úsporné rozmiestnenie je také, aby medzera medzi každými dvoma
autobusmi bola práve dve parkovacie miesta:
\insp{z5-ii-1c.eps}%

Vidíme, že medzi pôvodne vyznačenými parkovacími miestami môže byť nanajvýš 12
ďalších miest.
Parkovisko môže mať najviac $7+12+10=29$ parkovacích miest.

\hodnotenie
1~bod za umiestnenie jedného autobusu na ľavý koniec;
2~body za umiestnenie jedného autobusu na pravý koniec;
2~body za umiestnenie zvyšných dvoch autobusov medzi dva pôvodné (z~toho 1~bod za
zdôvodnenie);
1~bod za výsledný počet parkovacích miest.
\endhodnotenie
}

{%%%%%   Z5-II-2
Spoločnú stranu dvoch menších obdĺžnikov označíme~$a$
a~tieto dva obdĺžniky prekreslíme oddelené od seba.
\insp{z5-ii-2r.eps}%

Súčet obvodov dvoch menších obdĺžnikov je $40+52=92$\,(cm),
čo je o~16\,cm viac ako obvod pôvodného obdĺžnika, lebo $92-76=16$.
Týchto 16\,cm zodpovedá dvom stranám~$a$, strana $a$ má preto dĺžku
$16:2=8$\,(cm).

Rovnakú dĺžku majú aj dve strany pôvodného obdĺžnika, súčet
dĺžok jeho zvyšných dvoch strán je $76-2\cdot8=60$\,(cm), každá z~nich je teda
dlhá $60:2=30$\,(cm).
Rozmery pôvodného veľkého obdĺžnika sú 30\,cm a~8\,cm.

\hodnotenie
4~body za kratšiu stranu obdĺžnika (z~toho 2~za zdôvodnenie);
2~body za dlhšiu stranu obdĺžnika (z~toho 1~za zdôvodnenie).
\endhodnotenie
}

{%%%%%   Z5-II-3
Postupne od konca vypíšeme, v~koľkej sekunde tlieskal Juraj
a~v~koľkej Peter.
\begin{itemize}
\item Juraj: 90, 83, 76, 69, 62, 55, 48, 41, 34, 27, 20, {\bf13}, {\bf6}.
\item Peter: 90, 77, 64, 51, 38, 25, {\bf12}.
\end{itemize}
Čas prvého tlesknutia má byť menší alebo rovný 15 sekundám.
Juraj preto mohol začať tlieskať v~šiestej alebo trinástej sekunde, Peter musel
začať tlieskať v~dvanástej sekunde.

\ineriesenie
Rovnaký výsledok možno dosiahnuť pomocou delenia so zvyškom.
\begin{itemize}
\item
Keďže $90:7$ je $12$, zvyšok $6$, znamená to, že Juraj mohol začať tlieskať v~6.~sekunde (a~do 90.~sekundy tleskol ešte dvanásťkrát).
\item
Keďže $90:13$ je $6$, zvyšok $12$, znamená to, že Peter mohol začať tlieskať v~12.~sekunde (a~do 90.~sekundy tleskol ešte šesťkrát).
\end{itemize}
\noindent
Postupne vypisujeme ďalšie tlesknutia oboch chlapcov a~zvýrazňujeme tie, ktoré
vyhovujú podmienke v~zadaní:
\begin{itemize}
\item Juraj: {\bf6}, {\bf13}, 20, \dots
\item Peter: {\bf12}, 25, \dots
\end{itemize}


\hodnotenie
Po 3~bodoch za výpis a~odpoveď u~každého z~chlapcov.
Ak riešiteľ zabudne pri Jurajovi druhé riešenie, strhnite jeden bod.
\endhodnotenie
}

{%%%%%   Z6-II-1
V~úseku medzi 1. a~16. stĺpikom je 15 medzier a~podľa zadania sa ich šírka
nemení. Každá z~nich preto meria $48:15=3{,}2$ metra, a~je teda o~0{,}3 metra
väčšia ako medzera pri~južnom konci plota.

Keby všetky medzery medzi 16. a~28.~stĺpikom merali
zadaných 2{,}9~metra, bola by celková vzdialenosť medzi týmito
stĺpikmi $12\cdot2{,}9=34{,}8$ metra.
Podľa zadania je však táto vzdialenosť 36~metrov, čo je o~1{,}2~metra viac.
Tento rozdiel možno vyjadriť ako $4\cdot0{,}3$~metra, teda práve štyri medzery
v~úseku medzi týmito stĺpikmi sú o~0{,}3~metra dlhšie ako zadaných 2{,}9~metra.
Jedná sa o~medzery medzi 16. a~20. stĺpikom.

Hľadaný stĺpik, ktorý má od svojich
susedných stĺpikov rôzne rozostupy, je 20. v~poradí.

\hodnotenie
2 body za výpočet veľkosti severnej medzery~3{,}2\,m;
3 body za počet medzier s~veľkosťou 3{,}2\,m medzi 16. a~28.~stĺpikom
(resp. celkový počet medzier s~veľkosťou 2{,}9\,m);
1~bod za výsledok.
\endhodnotenie
}

{%%%%%   Z6-II-2
Rozhodnime, ktorá z~nasledujúcich možností mohla nastať: buď prečítala
Saša menej strán ako Zuzka, alebo naopak. Ak by Saša prečítala menej
ako Zuzka, znamenalo by to, že Ivana prečítala najviac.
Potom by ale jej výkon nebol v~strede medzi žiadnymi dvoma dievčatami.
Preto musela Zuzka prečítať menej ako Saša.

Ivana prečítala o~5~strán viac ako Zuzka a~súčasne bol jej výsledok
v~strede medzi Zuzkou a~Majkou. Preto Majka prečítala o~5~strán viac
ako Ivana a~zo zadania ďalej vieme, že Saša prečítala o~8~strán viac ako
Majka. Sčítaním rozdielov medzi týmito štyrmi dievčatami zisťujeme, že
Saša prečítala o~$5+5+8=18$ strán viac ako Zuzka.
\insp{z6-II-2.eps}%

Lucka prečítala 32~strán, čo bolo v~strede medzi Zuzkou a~Sašou.
Zuzka teda prečítala $32-9=23$ strán a~Saša prečítala $32+9=41$ strán.
Z~toho dopočítame ostatné výsledky:
Majka prečítala $41-8=33$ strán a~Ivana prečítala $23+5=28$ strán.

\hodnotenie
2 body za zdôvodnenie, že Saša prečítala viac ako Zuzka;
2 body za zdôvodnenie, že rozdiel medzi nimi bol 18;
2 body za výsledky dievčat.
\endhodnotenie
}

{%%%%%   Z6-II-3
Z~údajov, ktoré poznáme, vieme dopočítať obsah pravouholníka $DRLE$:
$$
S_{DRLE}%=S_{DRAK}+S_{DUPE}-S_{DUPLAK}
=44+64-92=16.
$$
(Všetky veličiny sú buď v~cm, alebo $\Cm^2$ a~tieto jednotky ďalej
neuvádzame.)
Dĺžky strán všetkých pravouholníkov sú celočíselné a~súčasne má platiť:
$$
|DR|\cdot|DE|=16,\quad
|DR|\cdot|DK|=44,\quad
|DU|\cdot|DE|=64.
$$
V~nasledujúcej tabuľke
vypíšeme všetky možnosti, ako vyjadriť~16 ako súčin dvoch kladných celých
čísel.
Pri každej z~týchto možností vyjadríme veľkosti ostatných strán,
$$
|DK|=44:|DR|,\quad |DU|=64:|DE|,
% \quad |LA|=28:|DR|,\quad |PL|=48:|DE|,
$$
a~budeme kontrolovať, či dostaneme celé čísla.
Ak áno, dopočítame veľkosti zvyšných strán mnohouholníka:
$$
|LA|=|DK|-|DE|,\quad
|LP|=|DU|-|DR|.
$$
$$
\def\tstrut{\vrule height 11pt depth 5pt width 0pt}
\def\@{\phantom{0}}
\begintable
$\vert DR\vert$\|\@1|\@2|\@4|\@8|16\cr
$\vert DE\vert$\|16|\@8|\@4|\@2|\@1\crthick
$\vert DK\vert$\|44|22|11|$-$|$-$\cr
$\vert DU\vert$\|\@4|\@8|16|32|64\crthick
$\vert LA\vert$\|28|14|\@7|$-$|$-$\cr
$\vert PL\vert$\|\@3|\@6|12|24|48\endtable
$$

Úloha má tri riešenia, ktoré vychádzajú z~prvých troch stĺpcov predchádzajúcej
tabuľky.

\hodnotenie
1 bod za obsah pravouholníka $DRLE$;
3 body za možné riešenia;
2 body za zdôvodnenie, že riešení nie je viac.

\poznamka
Vypíšeme všetky možnosti, ako vyjadriť obsahy pravouholníkov $DRAK$
a~$DU\!PE$ ako súčiny dvoch kladných celých čísel:
$$
\aligned
\vert DR\vert \cdot\vert DK\vert &=1\cdot44=2\cdot22=4\cdot11=11\cdot4=22\cdot2=44\cdot1,\\
\vert DU\vert \cdot\vert DE\vert &=1\cdot64=2\cdot32=4\cdot16=8\cdot8=16\cdot4=32\cdot2=64\cdot1.
\endaligned
$$
Zo zadania ďalej vieme, že
$$
|DR|<|DU|,\quad |DK|>|DE|,\quad
|DR|\cdot|DK|+|DU|\cdot|DE|-|DR|\cdot|DE|=92.
$$
Systematickým preberaním všetkých možností sa dajú nájsť všetky tri vyššie
uvedené riešenia bez toho, aby bol vyjadrený obsah pravouholníka $DRLE$.
V~takom prípade dajte
3~body za možné riešenia a~3~body za zdôvodnenie, že riešení nie je viac.
\endhodnotenie
}

{%%%%%   Z7-II-1
Označíme čísla v~bielych políčkach a~pomocou nich vyjadríme, ako vyzerajú
súčiny v~sivých políčkach:
\insp{z7-II-1a.eps}%

Súčin čísel v~sivých políčkach je pri zavedenom označení rovný
$a\cdot b\cdot b\cdot c\cdot c\cdot d$, a~ten má byť rovný 525.
Jediný spôsob, ako vyjadriť toto číslo ako súčin šiestich prirodzených čísel, pričom činiteľ $1$ nie je použitý viac ako dvakrát, je
$$
525=1\cdot1\cdot3\cdot5\cdot5\cdot7.
$$
Porovnaním s~predchádzajúcim vyjadrením zisťujeme, že 1 a~5 musia byť vo
vnútorných bielych políčkach, zatiaľ čo 3 a~7 v~krajných.
Dve zo štyroch možných vyplnení tabuľky sú na nasledujúcom obrázku.
Zvyšné dve vyplnenia sú osovo súmerné s~uvedenými, takže dávajú taký istý
súčet čísel v~sivých políčkach.
\insp{z7-II-1b.eps}%

Možné súčty čísel v~sivých políčkach sú $3+5+35=43$ a~$7+5+15=27$.

\hodnotenie
1 bod za rozklad čísla 525 na 6 prirodzených čísel;
2 body za nájdenie jedného vyplnenia;
2 body za nájdenie druhého vyplnenia;
1 bod za určenie správnych súčtov.
\endhodnotenie
}

{%%%%%   Z7-II-2
Uvažujme, ktoré dievča mohlo prečítať najmenej strán.
Lucka to byť nemohla, pretože jej výsledok bol iný ako~27.
Saša to byť nemohla, pretože Majka prečítala o~8~strán menej.
Ak by to bola Zuzka, tak by Ivana musela prečítať $27+5=32$ strán.
Tento prípad však nastať nemohol, pretože každé z~dievčat prečítalo iný počet strán.
Ivana to tiež byť nemohla, pretože Zuzka prečítala o~5~strán menej ako ona.
Takže najmenej strán musela prečítať Majka.

Ak Majka prečítala 27 strán, tak Saša prečítala $27+8=35$ strán,
čo je o~3~viac ako Lucka. Zuzka tak musela prečítať o~3~strany menej
ako Lucka, teda $32-3=29$ strán. Posledné z~dievčat, Ivana, prečítala
$29+5=34$ strán.

\hodnotenie
1 bod za zistenie, že Ivana ani Saša neprečítali 27~strán;
2 body za zistenie, že Zuzka neprečítala 27~strán;
3 body za výsledky dievčat.

\poznamka Riešenie je možné znázorňovať graficky podobne ako v~domácom kole.
\endhodnotenie
}

{%%%%%   Z7-II-3
Trojuholník vpravo a~trojuholník vľavo majú rovnako dlhú jednu stranu,
a~to stranu dlhú 6\,dm, ktorá je súčasne stranou obdĺžnika.
Keďže trojuholník vpravo má trikrát väčší obsah ako ten vľavo, musia byť
veľkosti ich výšok na uvedenú stranu v~rovnakom pomere.
Súčet týchto výšok je rovný šírke obdĺžnika, ktorá je 8\,dm.
Výška ľavého trojuholníka, \tj. vzdialenosť bodu nárazu od ľavej strany
obdĺžnika, je teda 2\,dm a~výška pravého trojuholníka, \tj. vzdialenosť bodu
nárazu od pravej strany obdĺžnika, je 6\,dm.

Z~uvedeného vyplýva, že trojuholník vľavo má obsah 6\,dm$^2$
($\frac{6\cdot2}2=6$).
Trojuholník dole má obsah dvakrát väčší, teda 12\,dm$^2$.
Strana dolného trojuholníka, ktorá je súčasne stranou obdĺžnika, je dlhá
8\,dm.
To znamená, že zodpovedajúca výška tohto trojuholníka,
\tj. vzdialenosť bodu nárazu od dolnej strany obdĺžnika, musí byť 3\,dm
($\frac{8\cdot3}2=12$).
Výška obdĺžnika je 6\,dm, teda vzdialenosť bodu nárazu od hornej strany
obdĺžnika je tiež 3\,dm ($6-3=3$).
\insp{z7-II-3.eps}%

\ineriesenie
Rozdeľme obdĺžnik priamkami prechádzajúcimi bodom nárazu
na štyri menšie pravouholníky ako na obrázku.
\insp{z7-II-3a.eps}%

Každý z~týchto pravouholníkov je prasklinami rozdelený na dva zhodné
trojuholníky.
Z toho vyplýva, že súčet obsahov ľavého a~pravého trojuholníka
je taký istý ako súčet obsahov horného a~dolného trojuholníka.
Tento súčet je teda polovicou obsahu celého obdĺžnika, \tj. 24\,dm$^2$.
Z~druhej podmienky v~zadaní vyplýva, že
obsah trojuholníka vľavo musí byť 6\,dm$^2$ a~obsah toho vpravo 18\,dm$^2$.
Z~tretej podmienky v~zadaní vyplýva, že obsah trojuholníka dole je 12\,dm$^2$.
Obsah trojuholníka hore je teda taký istý.

U~všetkých trojuholníkov poznáme obsah a~jednu stranu.
Zodpovedajúce výšky, \tj. vzdialenosti bodu nárazu od jednotlivých
strán obdĺžnika, možno teraz určiť rovnako ako v~druhej časti predchádzajúceho
riešenia.

\hodnotenie
3~body za vzdialenosti bodu nárazu od ľavej a~pravej strany obdĺžnika;
3~body za vzdialenosti bodu nárazu od dolnej a~hornej strany obdĺžnika.
Správne riešenie bez komentára ohodnoťte nanajvýš 3~bodmi.
\endhodnotenie
}

{%%%%%   Z8-II-1
Dĺžky hodov v~metroch označíme začiatočnými písmenami súťažiacich.
(Všetky nasledujúce výsledky sú tiež v~metroch a~túto jednotku vo
výpočtoch neuvádzame.)
Súčet dĺžok všetkých ich hodov bol 41 metrov,
$$
a+b+j+v+m=41, \eqno (1)
$$
a~priemerná dĺžka ich hodov bola
$$
\frac{a+b+j+v+m}5=8{,}2.
$$
Ak by súťažili iba Matúš, Vlado a~Angela, priemerná dĺžka hodu by
bola o~20\,cm kratšia, tzn. presne 8\,m:
$$
\frac{a+v+m}3=8.
$$
Po úprave dostávame
$$
a+v+m=24. \eqno (2)
$$

Zo zadania vieme, že Matúš hodil najmenej, čo bolo o~90\,cm menej, ako hodila
Angela, a~tá hodila o~60\,cm menej ako Vlado:
$$
a=m+0{,}9,\quad
v=a+0{,}6=m+1{,}5. \eqno (3)
$$
Dosadením do rovnice (2) dostávame
$$
\aligned
m+0{,}9+m+1{,}5+m&=24,\\
m&=7{,}2.
\endaligned
$$
Dosadením do rovníc (3) dopočítame dĺžky hodov Angely a~Vlada:
$$
a=7{,}2+0{,}9=8{,}1,\quad
v=7{,}2+1{,}5=8{,}7.
$$

Podľa rovnice (2) nahradíme v~rovnici (1) súčet $a+v+m$ číslom 24,
upravíme a~získame rovnicu
$$
b+j=17.
$$
Zo zadania ďalej vieme, že Jano hodil najďalej a~trafil sa lietadielkom do pásky
označujúcej celé metre.
Porovnaním so zatiaľ najdlhším vypočítaným hodom zisťujeme, že Jano hodil
aspoň 9~metrov.
Keby Jano hodil práve 9~metrov, Barbora by hodila $17-9=8$ metrov.
Keby Jano hodil 10~metrov (alebo viac), hodila by Barbora 7~metrov (alebo
menej).
To však nie je možné, pretože najmenej zo všetkých hodil Matúš.
Úloha má teda jednoznačné riešenie, a~to
$$
a=8{,}1,\quad b=8,\quad j=9,\quad v=8{,}7,\quad m=7{,}2.
$$

\hodnotenie
2~body za zistenie, že $a+v+m=24$;
2~body za vypočítanie dĺžok hodov Angely, Vlada a~Matúša;
2~body za vypočítanie dĺžok hodov Barbory a~Jana vrátane zdôvodnenia, že sa jedná o~jediné riešenie.
\endhodnotenie
}

{%%%%%   Z8-II-2
Označme $E$ priesečník uhlopriečok štvoruholníka $ABCD$.
Body $T_1$ a~$T_2$ sú ťažiskami trojuholníkov $BCD$ a~$ABD$,
úsečky $CE$ a~$AE$ sú teda ťažnicami v~týchto trojuholníkoch,
a~preto je bod~$E$ stredom úsečky~$BD$.
Keďže $|DE|=|EB|$, sú si rovné obsahy trojuholníkov $DEC$ a~$EBC$ a~tiež
obsahy trojuholníkov $DEA$ a~$EBA$.
Z~toho vyplýva, že trojuholníky $ACD$ a~$ACB$ majú rovnaký obsah.
Zo zadania poznáme veľkosť výšky trojuholníka $ACD$ z~vrcholu~$D$,
na vyjadrenie jeho obsahu potrebujeme určiť dĺžku úsečky~$AC$.

Z~vlastnosti ťažísk vieme, že
$$
|CE|=3\cdot|T_1E|,\quad |AE|=3\cdot|T_2E|.
$$
Veľkosť úsečky $AC$ je
$$
|AC|=|AE|+|EC|=3\cdot(|T_2E|+|ET_1|)=3\cdot|T_2T_1|=3\cdot3=9\,(\Cm).
$$
Veľkosť výšky trojuholníka $ACD$ z~vrcholu $D$ je 3\,cm,
obsah trojuholníka je teda rovný
$$
S_{ACD}=\frac{9\cdot3}2=\frac{27}2\,(\Cm^2).
$$
Obsah štvoruholníka $ABCD$ je rovný
$$
S_{ABCD}=2\cdot S_{ACD}=27\,(\Cm^2).
$$

\hodnotenie
3~body za určenie dĺžky úsečky~$AC$;
2~body za zdôvodnenie rovnosti obsahov trojuholníkov $ACD$ a~$ACB$;
1~bod za určenie hľadaného obsahu.
\endhodnotenie 
}

{%%%%%   Z8-II-3
V~prvých troch vrstvách počítaných odspodu je počet čiernych kociek o~4
väčší ako počet bielych. Tvrdenie platí aj pre každú ďalšiu trojicu vrstiev
s~čiernymi kockami na spodu.
V~zadaní nie je uvedené,
\itemitem{a)} či možno stavbu rozdeliť bezo zvyšku na také trojice,
\itemitem{b)} či je nad hornou trojicou ešte jedna vrstva, a~síce čierna,
\itemitem{c)} či sú nad hornou trojicou ešte dve vrstvy, čierna a~sivá.

V~prípade~a) by rozdiel medzi počtami čiernych a~bielych kociek musel byť násobkom
štyroch, v~prípade~b) by tento rozdiel musel byť násobkom štyroch zväčšeným o~jedna
a~v~prípade~c) by musel byť násobkom štyroch zväčšeným o~tri.

Keď zadaný rozdiel 55 vydelíme 4, dostaneme 13 a~zvyšok~3.
Z~toho vidíme, že z~uvedených možností platí~c).
Pyramída má celkom $13\cdot3+2=41$ vrstiev.

\ineriesenie
Riešenie rozdelíme na tri časti. V~časti a) budeme predpokladať, že vrchná
vrstva je biela, v~časti~b), že vrchná vrstva je čierna, a~v~časti~c), že
vrchná vrstva je sivá. V~každej časti riešenia budeme do tabuľky postupne
pridávať zväčšujúce sa vrstvy.
Zadanie uvádza, že najväčšia vrstva je čierna, preto pri každej čiernej vrstve
zaznamenáme rozdiel medzi počtami čiernych a~bielych kociek v~doposiaľ zapísaných
vrstvách.
Tabuľku prestaneme vypisovať, akonáhle bude tento rozdiel rovný 55 alebo bude väčší.

a) Horná kocka biela:
\bgroup
\def\ctr#1{\hfil\ #1\ \hfil}
$$
\begintable
vrstva zhora\hfill\|\ 1|\ 2|\ 3\|\ 4|\ 5|\ 6\|\dots\dots\|37|38|39\|40|41|42\cr
kociek vo vrstve\hfill\|\ 1|\ 3|\ 5\|\ 7|\ 9|11\|\dots\dots\|73|75|77\|79|81|83\cr
farba\hfill\|b|s|č\|b|s|č\|\dots\dots\|b|s|č\|b|s|č\crthick
rozdiel\hfill\|\multispan{3}\hfil 4\ \|\multispan{3}\hfil 8\ \|\dots\dots\|\multispan{3}\hfil 52\ \|\multispan{3}\hfil 56\ \endtable
$$

b) Horná kocka čierna:
$$
\begintable
vrstva zhora\hfill\|\ 1\|\ 2|\ 3|\ 4\|\ 5|\ 6|\ 7\|\dots\dots\|38|39|40\|41|42|43\cr
kociek vo vrstve\hfill\|\ 1\|\ 3|\ 5|\ 7\|\ 9|11|13\|\dots\dots\|75|77|79\|81|83|85\cr
farba\hfill\|č\|b|s|č\|b|s|č\|\dots\dots\|b|s|č\|b|s|č\crthick
rozdiel\hfill\|\ 1\|\multispan{3}\hfil 5\ \|\multispan{3}\hfil 9\ \|\dots\dots\|\multispan{3}\hfil 53\ \|\multispan{3}\hfil 57\ \endtable
$$

c) Horná kocka sivá:
$$
\begintable
vrstva zhora\hfill\|\ 1|\ 2\|\ 3|\ 4|\ 5\|\ 6|\ 7|\ 8\|\dots\|36|37|38\|39|40|41\cr
kociek vo vrstve\hfill\|\ 1|\ 3\|\ 5|\ 7|\ 9\|11|13|15\|\dots\|71|73|75\|77|79|81\cr
farba\hfill\|s|č\|b|s|č\|b|s|č\|\dots\|b|s|č\|b|s|č\crthick
rozdiel\hfill\|\multispan{2}\hfil 3\ \|\multispan{3}\hfil 7\ \|\multispan{3}\hfil 11\ \|\dots\|\multispan{3}\hfil 51\ \|\multispan{3}\hfil 55\ \endtable
$$
\egroup

K~rozdielu 55 sme došli iba v~tabuľke c), podľa ktorej má stavba 41 vrstiev.

\hodnotenie
$2+2$ body za vylúčenie možností a) a~b); 2 body za správny počet vrstiev.
\endhodnotenie
}

{%%%%%   Z9-II-1
Celé číslo je deliteľné tromi práve vtedy, keď je jeho ciferný súčet
deliteľný tromi.
Súčet cifier odpísaného čísla je
$$
8+5+5+2+2=22,
$$
čo nie je číslo deliteľné tromi.
Vynechaná cifra teda nebola $0$ a~ciferný súčet pôvodného čísla musí byť
väčší.
Najbližšie väčšie čísla deliteľné tromi sú $24$, $27$, $30$, $33$ atď.
Keby bol ciferný súčet $24$, $27$, resp. $30$, bola by vynechanou cifrou $2$, $5$,
resp. $8$.
(Ciferný súčet $33$ a~viac sa nedá dostať doplnením jedinej cifry.)

Teraz treba určiť, na ktoré miesta možno tieto cifry doplniť tak, aby
vzniklo zakaždým iné číslo.
To najjednoduchšie zistíme vypísaním všetkých možností:
\begin{itemize}
\item 285522, 825522, 852522, 855222,
\item 585522, 855522, 855252, 855225,
\item 885522, 858522, 855822, 855282, 855228.
\end{itemize}
Celkom teda existuje 13 možností.

\hodnotenie
2~body za zistenie vynechanej cifry a~zdôvodnenie;
3~body za vypísanie všetkých možných čísel (alebo za správne vysvetlenie, koľko ich je, aj bez ich vypísania);
1~bod za počet možností.
\endhodnotenie
}

{%%%%%   Z9-II-2
Z~druhej podmienky vieme, že trojuholník $TRP$ je rovnostranný,
preto všetky jeho vnútorné uhly majú veľkosť~$60\st$.
Uhly $TRP$ a~$STR$ sú striedavé, preto je aj veľkosť uhla $STR$ rovná~$60\st$.

Z~tretej podmienky vieme, že trojuholník $TRS$ je pravouhlý.
Z~predchádzajúceho odseku vieme, že pravý uhol nemôže byť pri vrchole~$T$,
zároveň z~prvej podmienky vyplýva, že pravý uhol nemôže byť ani pri vrchole~$S$.
Takže pravý uhol je pri vrchole~$R$.
Ďalej označme $K$ stred prepony~$ST$ trojuholníka $TRS$.
Keďže je tento trojuholník pravouhlý, leží vrchol~$R$ na kružnici so
stredom~$K$ a~polomerom $|KS|=|KT|$.
Platí teda
$$
|KS|=|KT|=|KR|.
$$
Trojuholník $TRK$ je teda rovnoramenný so základňou~$RT$.
Navyše z~predchádzajúceho vieme, že uhol $RTK$ má veľkosť~$60\st$,
preto aj druhý uhol pri základni má veľkosť~$60\st$.
Trojuholník $TRK$ je teda rovnostranný a~navyše zhodný s~rovnostranným
trojuholníkom $TRP$.
\insp{z9-ii-2a.eps}%

Trojuholníky $TRP$ a~$TRK$ majú rovnaký obsah, pretože sú zhodné.
Trojuholníky $TRK$ a~$SRK$ majú rovnaký obsah, pretože strany $KT$ a~$KS$
sú rovnako dlhé a~výška na tieto strany je spoločná.
To znamená, že trojuholník $TRS$ má dvakrát väčší obsah ako trojuholník~$TRP$,
$$
S_{TRS}=2S_{TRP}. \tag1
$$
Zo štvrtej podmienky vieme, že obsah jedného z~týchto dvoch trojuholníkov je
$10\cm^2$:
\begin{itemize}
\item Ak $S_{TRS}=10\cm^2$, tak $S_{TRP}=5\cm^2$.
\item Ak $S_{TRP}=10\cm^2$, tak $S_{TRS}=20\cm^2$.
\end{itemize}

\ineriesenie
Rovnakým spôsobom ako v~predchádzajúcom riešení určíme vnútorné uhly trojuholníka $TRS$.
Bod~$T$ zobrazíme v~osovej súmernosti podľa osi~$RS$, symetrický bod
označíme~$I$.
Všetky vnútorné uhly trojuholníka $TIS$ majú veľkosť $60\st$, trojuholník je
teda rovnostranný.
\insp{z9-ii-2b.eps}%

Strana~$TI$ je dvojnásobkom strany~$TR$, trojuholníky
$TRP$ a~$TIS$ sú teda podobné s~pomerom podobnosti $1:2$.
Preto sú ich obsahy v~pomere $1:4$,
$$
S_{TIS}=4S_{TRP}.
$$
Trojuholník $TRS$ tvorí polovicu trojuholníka $TIS$,
jeho obsah je teda dvakrát väčší ako obsah trojuholníka $TRP$.
Tak prichádzame ku vzťahu~\thetag1 a~úlohu uzavrieme rovnako ako v~predchádzajúcom
riešení.


\hodnotenie
Po 1~bode za určenie veľkostí vnútorných uhlov $RTS$ a~$TRS$;
3~body za zdôvodnenie rovnosti \thetag1;
1~bod za výsledné obsahy $5\cm^2$ a~$20\cm^2$.
\endhodnotenie
}

{%%%%%   Z9-II-3
Z~prvej podmienky vieme, že lístky sa dajú rozdeliť na dve skupiny tak, že súčet
jednej skupiny čísel je rovnaký ako súčet druhej.
Z~toho vyplýva, že súčet všetkých čísel napísaných na lístkoch musí
byť deliteľný dvoma.
Podobnou úvahou z~druhej podmienky odvodzujeme, že súčet všetkých
napísaných čísel je deliteľný tromi; z~poslednej podmienky vyplýva, že tento
súčet musí byť deliteľný aj piatimi.
Súčet všetkých napísaných čísel je preto deliteľný~$30$.

Súčet všetkých použiteľných cifier je $0+1+\cdots+8+9=45$, a~preto
je vyššie diskutovaný súčet rovný práve $30$.
Na lístkoch teda neboli použité také dve cifry, ktorých súčet bol
rovný $15$.
To mohli byť buď cifry $7$ a~$8$, alebo $6$ a~$9$.
V~oboch prípadoch musíme overiť, či je možné otrhať lístky s~ostatnými
ciframi tak, aby platili podmienky zo zadania.
\begin{itemize}
\item
Overme možnosť, keď na kvietku neboli cifry $7$ a~$8$:

Po trhaní podľa prvej podmienky mohli zostať napr. cifry $0+1+5+9=15$,

po trhaní podľa druhej podmienky mohli zostať napr. cifry $0+1+3+6=10$,

po trhaní podľa tretej podmienky mohli zostať cifry $0+1+2+3=6$.
\item Overme možnosť, keď na kvietku neboli cifry $6$ a~$9$:

Po trhaní podľa prvej podmienky mohli zostať napr. cifry $0+2+5+8=15$,

po trhaní podľa druhej podmienky mohli zostať napr. cifry $0+1+2+7=10$,

po trhaní podľa tretej podmienky mohli zostať cifry $0+1+2+3=6$.
\end{itemize}
Na okvetných lístkoch teda mohli byť buď všetky cifry okrem $7$ a~$8$, alebo
všetky okrem $6$ a~$9$.

\hodnotenie
2~body za odvodenie toho, že celkový súčet napísaných čísel je deliteľný $30$;
2~body za určenie oboch dvojíc cifier, ktoré mohli byť vynechané;
2~body za kontrolu, či je možné otrhať lístky podľa zadania.

Za riešenie obsahujúce iba jednu možnosť bez princípu deliteľnosti $30$ dajte nanajvýš 3~body.
\endhodnotenie
}

{%%%%%   Z9-II-4
Skutočnú cenu (cenu u~Pravdoslava) jedného džbánku medoviny
označme~$m$ a~skutočnú cenu jedného koláča označme~$k$. Pri prvej
návšteve mohli trpaslíci platiť koláče u~Pravdoslava a~medovinu
u~Krivomíra, alebo naopak. Tomu zodpovedajú dve rôzne vyjadrenia:
$$
4k=3(m+2),
\tag{1a}
$$
alebo
$$
4(k+2)=3m.
\tag{1b}
$$
Podľa informácie o~druhej návšteve vieme, že platí
$$
4m=3(k+2)+21.
\tag2
$$
Riešením sústavy rovníc (1a) a~(2) je dvojica
$$k=15,\ m=18,
$$
riešením sústavy rovníc (1b) a~(2) je dvojica
$$k=7,\ m=12.
$$

Úloha má teda dve riešenia: Buď stál koláč 15~grajciarov a~džbánok medoviny 18~grajciarov, alebo koláč 7~grajciarov a~džbánok medoviny 12~grajciarov.

\hodnotenie
Po 1~bode za sformulovanie každej z~podmienok (1a), (1b) a~(2);
po 1~bode za doriešenie každej zo sústav rovníc;
1~bod za výsledok.
\endhodnotenie
}

{%%%%%   Z9-III-1
Porovnajme posledné, teda najväčšie čísla oboch rámikov: najväčšie číslo
druhého rámika leží o~10~miest vpravo od najväčšieho čísla prvého rámika,
je teda o~$30$ väčšie. Rovnakým porovnaním zistíme, že druhé najväčšie čísla
v~rámikoch sa líšia tiež o~$30$, rovnako tak tretie najväčšie, štvrté najväčšie
aj piate najväčšie čísla. Súčet všetkých piatich čísel druhého rámika je teda
o~$5\cdot30=150$ väčší ako súčet piatich najväčších čísel prvého rámika. Aby
boli v~rámikoch rovnaké súčty, musí byť zvyšné číslo prvého rámika
práve~$150$. Najmenšie číslo tejto postupnosti je teda~$150$.

\ineriesenie
Najskôr hľadajme 16 po sebe bezprostredne idúcich prirodzených čísel
spĺňajúcich podmienku o~rovnakých súčtoch. Ak nájdené čísla vynásobíme
tromi, vynásobia sa tromi aj súčty v~rámikoch a~ich rovnosť teda zostane
zachovaná. Týmto postupom dôjdeme k~žiadanému výsledku a~pritom si
zjednodušíme výpočty, lebo budeme v~úvode pracovať s~menšími číslami.

Najmenšie číslo v~postupnosti bezprostredne po sebe idúcich prirodzených
čísel označíme~$x$ a~vyjadríme súčet čísel v~prvom rámiku:
$$
x+x+1+x+2+x+3+x+4+x+5=6x+15.
$$
Prvé číslo v~druhom rámiku je pri danom označení rovné $x+11$.
Vyjadríme súčet čísel v~druhom rámiku:
$$
x+11+x+12+x+13+x+14+x+15=5x+65.
$$
Riešením rovnice $6x+15=5x+65$ dostaneme $x=50$.
Vynásobením tromi dostaneme prvé číslo postupnosti násobkov troch,
týmto číslom je teda $50\cdot3=150$.

\hodnotenie
2 body za výsledok; 4 body za vysvetlenie postupu.
\endhodnotenie
}

{%%%%%   Z9-III-2
Zo symetrie celého útvaru (príp. z~vety {\it usu}) vyplýva,
že trojuholníky $BED$ a~$CFE$ sú zhodné.
Budeme teda určovať pomer obsahov trojuholníkov $HGC$ a~$CFE$.

Trojuholník $DEF$ je rovnostranný, preto v~ňom ťažnice a~výšky splývajú,
ťažnica~$DG$ je teda kolmá na úsečku~$EF$. Podľa zadania sú kolmé aj
úsečky $EF$ a~$AC$, preto sú priamky $AC$ a~$DG$, resp. $FC$ a~$GH$
rovnobežné.
Keďže $DG$ je ťažnicou trojuholníka $DEF$, leží bod~$G$ v~strede úsečky~$EF$.
Z~toho vyplýva, že $GH$ je strednou priečkou trojuholníka $CFE$.
\insp{z9-iii-2a.eps}%

Keďže trojuholníky $HGC$ a~$HGE$ majú spoločnú stranu~$HG$ a~výšky
prislúchajúce k~tejto strane sú rovnaké (menovite $|FG|=|GE|$),
majú tieto trojuholníky rovnaký obsah.
Keďže trojuholníky $HGE$ a~$CFE$ sú podobné a~zodpovedajúce strany
sú v~pomere $1:2$, sú obsahy týchto trojuholníkov v~pomere $1:4$.
(Strana~$GH$ je dvakrát menšia ako strana~$FC$ a~výška trojuholníka
$HGE$ na stranu~$GH$ je dvakrát menšia ako výška trojuholníka $CFE$ na stranu~$FC$;
obsah prvého trojuholníka je teda štyrikrát menší ako obsah druhého.)

Pomer obsahov trojuholníkov $HGC$ a~$BED$ je rovný $1:4$.

\poznamka
Ak označíme dĺžku úsečky~$BE$ ako $a$, môžeme v~závislosti na tejto veličine
vyjadriť obsahy trojuholníkov $HGC$ a~$BED$.

Z~rovnobežnosti priamok $AC$ a~$DH$ vyplýva, že trojuholník $DBH$
je rovnostranný a~trojuholník $BED$ je jeho polovicou.
Preto $|BD|=|BH|=2a$ a~veľkosť $DE$ vypočítame pomocou
Pytagorovej vety v~trojuholníku $BED$:
$$|DE|=\sqrt{(2a)^2-a^2}=a\sqrt3.
$$
Trojuholník $DEF$ je rovnostranný, teda $|EF|=|DE|=a\sqrt3$.
Bod $G$ je v~strede úsečky~$EF$, teda
$$|GF|=\frac{a\sqrt3}2.
$$
Trojuholníky $BED$ a~$CFE$ sú zhodné, teda $|CF|=|BE|=a$.
Úsečka~$GH$ je strednou priečkou trojuholníka $CFE$, teda
$$|GH|=\frac{a}2.
$$
Teraz môžeme vyjadriť obsahy skúmaných trojuholníkov:
$$
\aligned
S_{BED}&=\frac12|BE|\cdot|DE| =\frac12a^2\sqrt3,\\
S_{HGC}&=\frac12|GH|\cdot|GF| =\frac18 a^2\sqrt3.
\endaligned
$$

\hodnotenie
1~bod za zistenie rovnobežnosti priamok $AC$ a~$DG$;
2~body za určenie $|GH|=\frac12|FC|$;
3~body za vyjadrenie hľadaných obsahov, resp. ich pomeru.
\endhodnotenie
}

{%%%%%   Z9-III-3
Súčet všetkých čísel na okvetných lístkoch je $0+1+\cdots+9=45$.
Danka mohla pri každom trhaní odtrhnúť súčet najmenej $1$ (keby odtrhla
lístky $0$ a~$1$) a~najviac~$17$ (keby odtrhla lístky $8$ a~$9$).
\begin{itemize}
\item Po prvom trhaní musel na kvietku zostať súčet z~intervalu $28$ až~$44$
($45-17=28$ a~$45-1=44$).
Medzi týmito číslami je jediný násobok deviatich, a~síce~$36$.
\item Po druhom trhaní musel na kvietku zostať súčet z~intervalu $35$~až~$19$
($36-17=19$ a~$36-1=35$). Medzi týmito číslami sú dva násobky ôsmich, a~síce
$32$ a~$24$.
\item Podobne stanovíme možné súčty po treťom trhaní:
\itemitem{--}
Ak po druhom trhaní zostal súčet $32$, musel po treťom trhaní zostať
súčet z~intervalu $15$ až $31$ ($32-17=15$ a~$32-1=31$).
Medzi týmito číslami sú dva násobky desiatich, a~síce $30$ a~$20$.
\itemitem{--}
Ak po druhom trhaní zostal súčet $24$, musel po treťom trhaní zostať
súčet z~intervalu $7$ až $23$ ($24-17=7$ a~$24-1=23$).
Medzi týmito číslami sú dva násobky desiatich, a~síce $20$ a~$10$.
\end{itemize}
\noindent
Po jednotlivých trhaniach teda mohli zostať nasledujúce trojice súčtov:
$$
(36,32,20),\ (36,32,30),\ (36,24,20),\ (36,24,10).
$$
Pri~každej z~týchto možností musíme overiť, či sa dá naozaj zrealizovať,
\tj. či sa dajú dvojice lístkov odtrhnúť tak, aby žiadna cifra nebola
použitá dvakrát.
V~nasledujúcej schéme píšeme nad šípky príklad, aké lístky mohli byť
v~jednotlivých krokoch odtrhnuté:
$$
\aligned
45\buildrel0,9\over\longrightarrow
36\buildrel1,3\over\longrightarrow
32\buildrel4,8\over\longrightarrow
20\\
45\buildrel4,5\over\longrightarrow
36\buildrel1,3\over\longrightarrow
32\buildrel0,2\over\longrightarrow
30\\
45\buildrel2,7\over\longrightarrow
36\buildrel4,8\over\longrightarrow
24\buildrel1,3\over\longrightarrow
20\\
45\buildrel2,7\over\longrightarrow
36\buildrel4,8\over\longrightarrow
24\buildrel5,9\over\longrightarrow
10
\endaligned
$$

Vidíme, že všetky uvedené možnosti sa dajú zrealizovať, úloha má štyri
riešenia.

\hodnotenie
2~body za nájdenie štyroch možností;
2~body za overenie, že každá z~možností sa dá zrealizovať;
2~body za postup, ktorý vylučuje zabudnutie ďalšej možnosti,
príp. za zdôvodnenie, že žiadna ďalšia možnosť neexistuje.
\endhodnotenie
}

{%%%%%   Z9-III-4
Označme počty výhier jednotlivých dievčat $a$, $b$, $c$, $d$.
Keby niektoré dve dievčatá mali rovnaký počet výhier, museli by medzi súčtami
v~zadaní byť aspoň dve dvojice rovnakých čísel (keby platilo napr. $a=b$,
platilo by $a+c=b+c$ a~tiež $a+d=b+d$).
To však nie je pravda, teda čísla $a$, $b$, $c$, $d$ sú navzájom rôzne.

Ak zoradíme tieto čísla vzostupne $a<b<c<d$, tak platí
$$
a+b < a+c < a+d < b+d < c+d
$$
a~súčasne
$$
a+c < b+c < b+d.
$$
Vzhľadom na to, že sú v~zadaní dve čísla s~rovnakou hodnotou, musí byť
$a+d=b+c$.
Podľa zadania zostavíme sústavu rovníc:
$$
\aligned
a+b&=8,\\
a+c&=10,\\
b+c&=12,\\
a+d&=12,\\
b+d&=14,\\
c+d&=16.
\endaligned
$$
Z~1. a~2. rovnice vyplýva, že $c=b+2$.
Po dosadení do 3. rovnice dostávame $b+b+2=12$, teda $b=5$.
Ak dosadíme za $b$ do 1., 3., resp. 5.~rovnice, získame aj hodnoty ostatných
neznámych: $a=3$, $c=7$, resp. $d=9$.

Pri riešení sústavy sme nepoužili 4.~a~6.~rovnicu, ich platnosť preto
musíme overiť dosadením vypočítaných hodnôt:
$3+9=12$, $7+9=16$.

Jednotlivé dievčatá vybojovali 3, 5, 7 a~9 výhier.

\hodnotenie
2~body za zostavenie sústavy rovníc a~zdôvodnenie, že
až na označenie a~usporiadanie neznámych je táto sústava určená jednoznačne;
4~body za vyriešenie sústavy
(ak nie je prevedená skúška pri rovniciach, ktoré neboli použité pri riešení,
strhnite 1~bod).

\poznamka
Na~vyriešenie úlohy nie je nutné zostavovať a~riešiť sústavu rovníc,
celý postup možno vyjadriť slovne.
Hodnotenie takého riešenia je obdobné vyššie uvedenému.
\endhodnotenie
}

