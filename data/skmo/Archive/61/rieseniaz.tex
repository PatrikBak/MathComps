{%%%%%   Z4-I-1
...}

{%%%%%   Z4-I-2
...}

{%%%%%   Z4-I-3
...}

{%%%%%   Z4-I-4
...}

{%%%%%   Z4-I-5
...}

{%%%%%   Z4-I-6
...}

{%%%%%   Z5-I-1
\napad
Ktorou cestou Bonifác určite nešiel?

\riesenie
Najskôr určíme, ktorými cestami išli jednotliví kamaráti. Na to potrebujeme vedieť, na ktoré svetové strany vedú jednotlivé cesty. Cesta pozdĺž plnej čiary vedie iba na sever a východ. Čiarkovaná cesta vedie na sever, východ a západ.
Bodkovaná cesta mieri postupne na všetky svetové strany.
Jediná cesta, ktorá nevedie nikdy západným smerom, je tá vyznačená plnou čiarou -- patrí teda Servácovi. Tadiaľ Bonifác určite nešiel.
Zo zvyšných dvoch ciest na juh nemieri tá čiarkovaná -- po nej teda šiel Pankrác.
Potom Bonifác musel ísť po bodkovanej čiare.

Vieme, že Pankrác prešiel 500\,m. Teraz spočítajme, po koľkých úsečkách (\tj. stranách štvorčeka štvorcovej siete) šiel:
$$\align
&7\text{(východ)} + 2\text{(sever)} + 4\text{(západ)} + 1\text{(sever)} +
6\text{(západ)} + 2\text{(sever)} +\\
+ &4\text{(východ)} +1\text{(sever)} + 4\text{(východ)} + 1\text{(sever)} +
7\text{(východ)} + 1\text{(sever)} = 40.
\endalign
$$
Takisto zistime, po koľkých úsečkách išiel Bonifác:
$$\align
&1\text{(západ)} + 1\text{(sever)} + 3\text{(západ)} + 3\text{(sever)} +
3\text{(východ)} + 2\text{(sever)} +\\
+ &1\text{(východ)} + 4\text{(sever)} + 3\text{(východ)} + 3\text{(jih)} +
1\text{(východ)} + 6\text{(juh)} +\\
+ &2\text{(východ)} + 8\text{(sever)} + 2\text{(východ)} + 1\text{(sever)} + 4\text{(východ)} + 2\text{(juh)} = 50.
\endalign
$$
Ak 40 úsečiek meria 500\,m, tak 10 úsečiek meria $500:4 = 125$\,(m).
Takže 50 úsečiek meria $500+125=625$\,(m).
Bonifác prešiel v~labyrinte 625 metrov.
}

{%%%%%   Z5-I-2
\napad
Začnite súčinom.

\riesenie
Začneme súčinom:
Z~čísel 1, 2 a~3 potrebujeme vybrať dve také, aby ich súčin bol 6. Do úvahy prichádza
jediná možnosť  -- 2 a~3. Keďže v treťom riadku už dvojka je, môžeme do príslušného políčka tohto riadku dopísať jedine trojku (\obr).
Je zrejmé, že v~prvom políčku tretieho stĺpca a~v druhom políčku
tretieho riadku môžu byť jedine jednotky (\obr).
\inspinsp{z61.20}{z61.21}%


Teraz napr. súčet: Súčet dvoch čísel má byť 4 a jeden zo sčítancov je 1, takže druhý
musí byť 3 (\obr). Všimnime si teraz rozdiel: Rozdiel dvoch čísel má byť 1, jedným z~týchto čísel je
1, takže druhé musí byť 2 (\obr).
\inspinsp{z61.22}{z61.23}%

Zostáva doplniť posledné čísla: V~prvom riadku chýba číslo 3, v druhom
riadku chýba číslo  1 (\obr).
Ešte overíme, že podiel práve doplnených čísel je naozaj 3 a~že v~každom stĺpci a~riadku je každé z~čísel 1, 2 a~3 práve raz.
\insp{z61.24}%

\poznamka
Samozrejme sa dá postupovať rôznymi spôsobmi, v~každom prípade si však
rýchlo všimneme, že v~zadaní je podstatne viac informácií, ako je potrebné na jednoznačné vyriešenie úlohy.
Pokiaľ sa napr. prednostne sústredíme na požiadavku, aby v každom stĺpci a riadku bolo každé z čísel 1, 2, 3 práve raz, tak stačí už len jedna
zo štyroch zmienených informácií -- viete zistiť ktorá?
Súčasne niektoré zmienené informácie sú splnené vždy  -- viete zistiť ktoré?
}

{%%%%%   Z5-I-3
\napad
Vymyslite vhodný systém, podľa ktorého budete jednotlivé možnosti vypisovať.

\riesenie
Pre prehľadnosť zostavíme tabuľku. Hviezdička znamená, že daný chlebíček
obsahuje príslušnú prísadu, prázdne políčko znamená, že chlebíček túto
prísadu neobsahuje.
$$ \begintable
\|šunka|syr|vajíčko|paprika\|\crthick
1 prísada\|*|||\|1\cr
\||*||\|2\cr
\|||*|\|3\cr
\||||*\|4\crthick
2 prísady\|*|*||\|5\cr
\|*||*|\|6\cr
\|*|||*\|7\cr
\||*|*|\|8\cr
\||*||*\|9\cr
\|||*|*\|10\crthick
3 prísady\|*|*|*|\|11\cr
\|*|*||*\|12\cr
\|*||*|*\|13\cr
\||*|*|*\|14\endtable
$$
Pretože sme tabuľku tvorili systematicky a~vyčerpali sme všetky možnosti,
vidíme, že Julka môže pripraviť až 14 chlebíčkov tak, aby boli splnené jej požiadavky.
}

{%%%%%   Z5-I-4
\napad
Skúste počítať po vrstvách.

\riesenie
Stavbu rozdelíme štyrmi vodorovnými rezmi na päť vrstiev. Prostredná vrstva je
na \obr{} vľavo, skladá sa zo~16 kociek. Ostatné štyri vrstvy vyzerajú
všetky tak, ako vidno na \obrr1{} vpravo, a~každá z~nich sa skladá z~24
kociek. Na celú stavbu bolo teda použitých $16 + 4\cdot 24 = 112$ kociek.
\insp{z61.31}%

\inynapad
Koľko kociek chýba v tuneloch?

\ineriesenie
Predstavme si, že by stavba bola vyhotovená bez "tunelov" a~tie by boli prerazené až
dodatočne. Pôvodne sa teda skladala z~$5\cdot 5\cdot 5 = 125$ kociek.
Prerazením prvého tunela stavba stratila 5 kociek, prerazením ďalších dvoch tunelov stratila po 4~kockách. Konečný počet kociek je teda
$125 - 5 - 4 - 4 = 112$.
}

{%%%%%   Z5-I-5
\napad
Aký bol vekový rozdiel najmladšieho a najstaršieho brata?

\riesenie
Najstarší brat bol od najmladšieho starší o 9~rokov ($6\cdot 1{,}5=9$).
Najstarší brat bol štyrikrát starší ako najmladší, takže
rozdiel 9 rokov musí zodpovedať trojnásobku veku najmladšieho brata.
V~čase zakliatia mal teda najmladší 3 roky ($9:3=3$).
Ďalší bratia mali postupne 4 a~pol, 6, 7 a~pol, 9, 10 a~pol, 12 rokov.
}

{%%%%%   Z5-I-6
\napad
Mohla by Janka mať napr. 6 alebo 8 uzáverov?

\riesenie
To, že  Janka zložila ohradu tvaru rovnostranného trojuholníka, znamená, že
počet jej uzáverov musel byť násobkom čísla 3.
Podobne, štvorcovú ohradu mohla postaviť len vtedy, keď počet uzáverov bol
násobkom čísla 4.
Počet uzáverov teda musel byť súčasne násobkom čísla 3 i~4, \tj. napr. 12,
24, 36, \dots{} (ľubovoľný násobok čísla 12).
}

{%%%%%   Z6-I-1
\napad
V každom kruhu má byť rovnako veľa fazuliek.

\riesenie
Spoločnú časť prvých dvoch kruhov nazveme $A$, spoločnú časť druhého a~tretieho kruhu nazveme $B$.
Z~druhého kruhu zostane po odtrhnutí časti $A$ zvyšok, v ktorom musí byť rovnaký počet fazuliek, ako vo zvyšku, ktorý vznikne po odtrhnutí časti $A$ z~prvého kruhu (\obr).
V~zadaní sa uvádza, že v tejto zostávajúcej časti je 110 fazuliek a vďaka tomu
vieme, že do časti $B$ musíme dať
$110 - 68 = 42$
fazuliek.
\insp{z61.50}%

Teraz už vieme koľko je fazuliek v treťom kruhu (\obr):
$42 + 87 = 129$.
\inspinsp{z61.51}{z61.52}%

Rovnaký počet fazuliek musí byť v prvom aj druhom kruhu. Využitím prvého kruhu zistíme, že v časti $A$ je
$129 - 110 = 19$ fazuliek (\obr).

V sivých plochách je postupne zľava 19 a~42 fazuliek.
}

{%%%%%   Z6-I-2
\napad
Využite, že z uvedených zvieratiek majú klepetá jedine kraby.

\riesenie
Pre prehľadnosť si údaje o~jednotlivých hračkách zaznamenáme do tabuľky:
$$\begintable
\|nohy|krídla|klepetá|hlavy\crthick
vážka\|6|4|0|1\cr
pštros\|2|2|0|1\cr
krab\|8|0|2|1\endtable
$$
Je zrejmé, že klepetá majú jedine kraby. Pretože klepiet je 22 a~každý krab má klepetá dve, musí byť krabov
$22:2 = 11$.
Tieto kraby majú dokopy $11\cdot 8 = 88$ nôh.
Na vážky a~pštrosy potom zostáva $118 -88 = 30$ nôh.

Vážky a~pštrosy teda majú dokopy 30 nôh a~22 krídiel.
Aby sme určili počty jednotlivých hračiek, všimneme si nasledujúcu tabuľku:
$$\begintable
\|nohy|krídla\crthick
jedna vážka\|6|4\cr
jeden pštros\|2|2\cr
dva pštrosy\|4|4\endtable
$$
Vidíme, že dva pštrosy majú dokopy rovnako veľa krídiel ako jedna vážka, ale
majú o~2~nohy menej.
Môžeme si to predstaviť tak, že z dvoch pštrosov "vyrobíme" jednu vážku tak,
že im "pridáme" ešte dve nohy.

Podľa krídiel máme 11 pštrosov ($22 : 2 = 11$).
Tie by ale mali len 22 nôh ($11\cdot 2 = 22$).
Ostáva nám teda 8 nôh ($30 -22 =8$), na ktorých budeme "prerábať pštrosy na vážky".
Vždy dve nohy premenia dva pštrosy na jednu vážku, vážky sú preto $8 :2 = 4$.

Štyri vážky majú dokopy 24 nôh ($4\cdot 6 =24$) a~16 krídel ($4\cdot 4
=16$).
Na pštrosy tak ostáva 6 nôh ($30 - 24 = 6$) a~6 krídel ($22 - 16 = 6$).
Sú teda 3 ($6 : 2 = 3$).
Predchádzajúce úvahy môžeme schematicky znázorniť nasledovne (symbol \`\i\'\i\
predstavuje dve krídla a~dve nohy, teda určujúce prvky jedného pštrosa):
\bgroup
\thinsize=0pt
\thicksize=0pt
\def\tstrut{\vrule height 0pt depth 0pt width 0pt}
$$\begintable
\`\i\'\i|\`\i\'\i|\`\i\'\i|\`\i\'\i|\`\i\'\i|\`\i\'\i|\`\i\'\i|\`\i\'\i|\`\i\'\i|\`\i\'\i|\`\i\'\i\cr
|||\multispan{2}\hfil$\underbrace{}$\hfil|\multispan{2}\hfil$\underbrace{}$\hfil|\multispan{2}\hfil$\underbrace{}$\hfil|\multispan{2}\hfil$\underbrace{}$\hfil\cr
|||\multispan{2}\hfil\i\i\hfil|\multispan{2}\hfil\i\i\hfil|\multispan{2}\hfil\i\i\hfil|\multispan{2}\hfil\i\i\hfil\endtable
$$
\egroup

Do hračkárstva priviezli 11 krabov, 4 vážky a~3 pštrosy.
Keďže každé z~týchto zvierat má jednu hlavu, dokopy majú 18 hláv ($11 + 4 + 3 = 18$).


\ineriesenie
Rovnako ako v predchádzajúcom riešení určíme, že priviezli 11 krabov a~že vážky a~pštrosy majú dokopy 30 nôh  a~22 krídel.

Keďže každá vážka má 6 nôh, môže ich byť nanajvýš 5
a~jednotlivé možnosti postupne preberieme.
Keby vážka bola jedna, ostávalo by na pštrosy $30-6=24$ nôh a~$22-4=18$
krídel.
Aby súhlasili počty nôh, muselo by byť pštrosov 12, ale aby súhlasili
počty krídel, muselo by ich byť 9 -- jedna vážka preto byť nemôže.

Ostatné prípady rozpisovať nebudeme, diskusiu zhrnieme nasledujúcou tabuľkou. Záver je rovnaký ako v predchádzajúcom riešení.
$$\begintable
vážok|zostatok nôh|zostatok krídel|pštrosov\crthick
1|24|18|---\cr
2|18|14|---\cr
3|12|10|---\cr
\bf 4|6|6|\bf 3\cr
5|0|2|---\endtable
$$
}

{%%%%%   Z6-I-3
\napad
Zistite, koľko kociek nemá zafarbenú ani jednu stenu.

\riesenie
Stavbu rozdelíme štyrmi vodorovnými rezmi na päť vrstiev tak, ako na \obr{}. Prostredná vrstva sa skladá zo~16 kociek, ostatné vždy z~24 kociek.
Celkový počet kociek je $16 + 4\cdot 24 = 112$.
Kocky, ktoré nemajú zafarbenú ani jednu stenu, sme na obrázku označili čiernou bodkou -- je ich 8.
Ostatné kocky majú zafarbenú aspoň jednu stenu a~je ich teda
$112 - 8 = 104$.
\insp{z61.32}%

\inynapad
Počítajte po vrstvách.

\ineriesenie
Pracujeme s~\obrr1.
V~spodnej vrstve majú všetky kocky, ktorých je 24,
zafarbenú aspoň jednu stenu. V druhej vrstve je 8 kociek s~jednou zafarbenou stenou a~12 s dvoma zafarbenými stenami. V~prostrednej vrstve majú všetky kocky dve ofarbené steny.
Štvrtá vrstva je rovnaká ako druhá a piata je rovnaká ako prvá.
Kociek, ktoré majú zafarbenú aspoň jenu stenu, sme teda dokopy napočítali
$$
2\cdot 24 + 2\cdot (8 + 12) + 16 = 48+40+ 16 = 104.
$$
}

{%%%%%   Z6-I-4
\napad
Začnite súčinom 48.

\riesenie
Začneme súčinom 48:
Potrebujeme číslo 48 rozložiť na súčin troch čísel tak, aby činitele boli
len 1, 2, 3 alebo 4.
To sa dá len jedným spôsobom, $48=3\cdot4\cdot4$, a~činitele môžu byť v~zodpovedajúcej oblasti doplnené jedine tak, ako na \obr.
Do druhého políčka prvého stĺpca doplníme číslo 2, ktoré v~tomto stĺpci chýba (\obr).
\inspinsp{z61.60}{z61.61}%


Teraz sa zaoberajme súčinom 6, ktorý sa dá získať z~daných čísel len ako
$6=1\cdot2\cdot3$.
To znamená, že v~druhom stĺpci budú okrem už napísaného čísla 4 ešte
čísla 1 a~3 na prostredných dvoch políčkach (zatiaľ nevieme, v~akom poradí).
Takže v~prvom políčku druhého stĺpca musí byť číslo~2 (\obr).
\inspinsp{z61.62}{z61.63}%

Rozdiel 1: Rozdiel dvoch čísel má byť 1, jedno z~čísel %%(menšenec nebo menšiteľ)
je 2, takže druhé číslo musí byť 1 alebo 3.
Keďže číslo 1 už je  v~prvom riadku napísané, musí byť v treťom políčku
tohto riadku číslo 3.

Do štvrtého políčka prvého riadku tak musíme doplniť číslo  4, ktoré jediné
v~tomto riadku ešte napísané nie je (\obr).

Uvažujme teraz inak.
Zatiaľ sme doplnili trikrát číslo 4, takže ešte jedno ostáva.
Pretože v~prvom, treťom a~štvrtom riadku štvorky sú, bude tá posledná v druhom riadku.
Takisto, pretože štvorka je napísaná  v~prvom, druhom a~štvrtom stĺpci,
chýba v treťom stĺpci.
To znamená, že posledná, štvrtá štvorka, musí byť v druhom riadku tretieho
stĺpca (\obr).
\inspinsp{z61.64}{z61.65}%

Súčet 9: V~tejto oblasti chýba posledné číslo, a~to musí byť $9 - 4 - 4 = 1$ (\obr).

V druhom políčku druhého riadku musí byť číslo 3, ktoré tu ako jediné zatiaľ napísané nie je.
To znamená, že v treťom políčku druhého stĺpca bude číslo  1; buď preto, že
v~tomto stĺpci chýba, alebo preto, že chýba v oblasti so súčinom 6 (\obr).
\inspinsp{z61.66}{z61.67}%

Ďalej môžeme uvažovať rovnako ako pri dopĺňaní posledného čísla 4.
Doplnili sme trikrát číslo~1, ktoré zatiaľ nie je vo štvrtom riadku a~treťom
stĺpci.
Rovnako tak doplníme i~posledné číslo 3, ktoré chýba len v treťom riadku a~vo štvrtom stĺpci (\obr).

Teraz už chýbajú len dve čísla 2. Ľahko overíme, že po ich doplnení do prázdnych políčok
spĺňajú všetky zapísané čísla požadované podmienky (\obr).
\insp{z61.68}%

\poznamka
Samozrejme, dá sa postupovať rôznymi spôsobmi, v každom prípade si však rýchlo všimneme, že v~zadaní je podstatne viac informácií, ako je treba na doriešenie úlohy.
Pokiaľ sa napríklad prednostne zameriame na požiadavku, aby v~každom stĺpci a~riadku bolo každé z~čísel 1, 2, 3, 4 práve raz,
tak stačia už len tri zo šiestich ďalej uvedených informácií -- ktoré tri by napríklad stačili?
}

{%%%%%   Z6-I-5
\napad
Najskôr zistite, kto dostal hasičské auto.

\riesenie
Petrove výpovede o chlapcoch sú:
\begin{enumerate}
  \item  Ondro dostal hasičské auto,
  \item  Kubo nedostal hasičské auto,
  \item  Maťo nedostal Merkur.
\end{enumerate}
Keby hasičské auto dostal Ondro, boli by prvé dve výpovede pravdivé.
Ale pravdivá má byť len jedna výpoveď, takže hasičské auto Ondro dostať
nemohol.

Keby hasičské auto dostal Maťo, boli by opäť dve výpovede pravdivé a to druhá a~tretia.

Hasičské auto teda musel dostať Kubo.
Prvé a druhé tvrdenie je preto nepravdivé a~pravdivé musí byť tretie, že Maťo
nedostal Merkur.
Maťo nedostal ani hasičské auto (to dostal Kubo), takže musel dostať
vrtuľník.

Darčeky boli rozdelené takto:
Kubo dostal hasičské auto, Maťo vrtuľník a~Ondro stavebnicu Merkur.
}

{%%%%%   Z6-I-6
\napad
Dve z troch častí musia byť trojuholníky.

\riesenie
Ihrisko zložené z~18 rovnakých štvorcov treba rozdeliť na tri rovnako veľké časti.
Veľkosť jednej časti bude potom $18:3=6$ štvorcov.
Uhlopriečka delí ihrisko na dva rovnaké trojuholníky s~obsahom $18:2=9$
štvorcov.
To znamená, že pokiaľ máme dostať tri časti s obsahom 6 štvorcov, musí byť
jedna z~deliacich čiar "pod" a~druhá "nad" touto uhlopriečkou (\obr).
Dve z~takto vzniknutých častí sú teda trojuholníky a~jedna štvoruholník.
Teraz stačí určiť čiary tak, aby trojuholníky mali obsah 6 štvorcov. Potom zostávajúci štvoruholník bude mať takisto požadovaný obsah.
\insp{z61.70}%

Pozrime sa najskôr na trojuholník vľavo.
Tento trojuholník je polovicou obdĺžnika, ktorého zvislá strana je dlhá tri dieliky. Obsah tohto obdĺžnika má byť $2\cdot 6=12$ štvorcov, takže jeho druhá strana
musí byť dlhá $12:3=4$ dieliky -- môžeme nakresliť prvú deliacu čiaru.

Postupujme podobne pri druhom -- pravom trojuholníku.
Tento trojuholník má byť polovicou obdĺžnika s~obsahom 12 štvorcov,
ktorého vodorovná strana je dlhá 6 dielikov.
Jeho zvislá strana musí byť potom dlhá $12:6=2$ dieliky
-- môžeme nakresliť druhú deliacu čiaru.

Dievčatá by mali rozdeliť ihrisko tak ako na \obr.
\insp{z61.71}%
}

{%%%%%   Z7-I-1
\napad
Začnite druhou podmienkou.
%%Kejchal = Kýchal, Dřímal = Spachtoš, Stydlín= Plaško, Prófa = Vedko, Štístko = Smejko,
%%Šmudla = Kýblik, Rejpal=Dudroš

\riesenie
Z~druhej podmienky vyplýva, že Spachtošov džbánik má objem 3~litre a~Plaškov 9~litrov (platí $3\cdot 3 =9$, a~keby mal Spachtoš džbánik iný, musel by byť Plaškov džbánik aspoň 12-litrový).

Teraz zo štvrtej podmienky vyplýva, že Kýblikov džbánik je o~3~litre väčší ako
Smejkov.
Spoločne s~treťou podmienku tak vieme, že Smejko, Vedko a~Kýblik majú
postupne džbániky s~objemami buď 4, 6 a~7, alebo 5, 7 a~8 litrov.

Z~prvej podmienky potom vyplýva, že jediné možnosti, ako mali trpaslíci džbániky
rozdelené, sú:
$$\begintable
3|4|5|6|7|8|9\crthick
Spachtoš|Smejko|Kýchal|Vedko|Kýblik|Dudroš|Plaško\cr
Spachtoš|Smejko|Dudroš|Vedko|Kýblik|Kýchal|Plaško\cr
Spachtoš|Dudroš|Smejko|Kýchal|Vedko|Kýblik|Plaško\endtable
$$
Keď overíme poslednú (piatu) podmienku, zistíme, že prvé dve vyznačené
možnosti nevyhovujú ($6+7\ne 8+5+4$), zatiaľ čo tretia áno ($7+8=4+5+6$).
Kýchal s~Kýblikom teda dohromady prinesú $6+8=14$ litrov vody.
}

{%%%%%   Z7-I-2
\napad
Vhodne obrázok rozdeľte.

\riesenie
V štvorci $ABCD$ vyznačíme obe uhlopriečky a~spojnice stredov protiľahlých
strán. Štyri takto doplnené úsečky sa pretínajú v~jedinom bode $S$ a~rozdeľujú
štvorec bezo zvyšku na osem zhodných  trojuholníkov. Jeden z~nich sme
v~\obr{} označili $STC$.
\insp{z61.81}%

Týchto osem trojuholníkov sa zhoduje aj~vo svojich sivo vyfarbených častiach, a~preto zadanú podmienku o~obsahoch môžeme využiť pre každý tento trojuholník zvlášť. V~prípade trojuholníka $STC$ tak platí, že jeho sivá a~biela plocha,
teda trojuholníky $FTC$ a~$SFC$, majú rovnaký obsah. Oba trojuholníky
majú výšku $TC$. Aby mali rovnaký obsah, musia byť rovnaké aj veľkosti strán
kolmých na túto výšku, teda $|FT|=|SF|$.
Dĺžka úsečky $SF$ je polovičná vzhľadom na dĺžku
uvedenú v~zadaní, teda je $6\cm$. Veľkosť úsečky $ST$ je potom $6 + 6 = 12$\,(cm)
a~veľkosť strany štvorca $ABCD$ je $2\cdot 12 = 24$\,(cm).

\ineriesenie
Vo všetkých sivo zafarbených rovnoramenných trojuholníkoch vyznačíme výšku kolmú
na základňu. Tým rozdelíme pôvodné trojuholníky na osem zhodných  pravouhlých
trojuholníkov, ktoré vnútri štvorca $ABCD$ premiestnime tak, ako na
\obr{}.
\insp{z61.82}%

V štvorci $ABCD$ sme dostali dva zhodné sivé obdĺžniky a~jeden biely obdĺžnik.
Strany týchto troch obdĺžnikov, na \obrr1{} zvislé, majú rovnakú dĺžku. Veľkosť strany bieleho obdĺžnika, ktorá je na obrázku
vodorovne, je zadaných $12\cm$. Aby sivé plochy a~biela plocha mali rovnaký
obsah, musia mať vodorovné strany oboch sivých obdĺžnikov dokopy dĺžku
takisto 12\,cm. Veľkosť strany štvorca $ABCD$ je teda $24\cm$.
}

{%%%%%   Z7-I-3
\napad
Zistite v akých všetkých vzťahoch je na konci číslo v strede a číslo s ním susedné.

\riesenie
Čísla označíme vzostupne podľa veľkosti  ako 1. až 7.
Ich rozmiestnenie sa postupne menilo takto:
\bgroup
\thinsize=0pt
\thicksize=0pt
$$\begintable
1.|2.|3.|4.|5.|6.|7.\cr
7.|2.|3.|4.|5.|6.|1.\cr
6.|7.|2.|3.|4.|5.|1.\cr
6.|2.|3.|7.|4.|5.|1.\endtable
$$
\egroup

Takže číslo v strede by mohlo byť dvojnásobkom tretieho čísla v rade alebo štvrtého čísla v rade.

Rozoberme najskôr možnosť, že 7. číslo  je dvojnásobkom 3. čísla v rade. Keďže čísla v~pôvodnom rade boli po sebe idúce, tak rozdiel medzi 3. a 7. číslom je štyri. Tento rozdiel musí byť zároveň aj tretie číslo, keďže 7. číslo je dvakrát 3. číslo. Keďže tretie číslo je 4, tak na začiatku mohli stáť v rade čísla 2, 3, 4, 5, 6, 7, 8.

Pri druhej možnosti je rozdiel medzi 4. číslom a 7. číslom rovný 3 a toto číslo je súčasne štvrté číslo z radu. Keďže štvrté číslo je 3, tak v~rade mohli stáť na začiatku aj čísla 0, 1, 2, 3, 4, 5, 6.

Úloha má teda dve riešenia: 2, 3, 4, 5, 6, 7, 8 alebo 0, 1, 2, 3, 4, 5, 6.
}

{%%%%%   Z7-I-4
\napad
Rozložte zadané dĺžky na súčin prvočísel.

\riesenie
Rozložíme dĺžky všetkých hrán na súčiny prvočísel:
$$\gather
12 = 2\cdot 2\cdot 3,\
18 = 2\cdot 3\cdot 3,\
20 = 2\cdot 2\cdot 5, \\
24 = 2\cdot 2\cdot 2\cdot 3,\
30 = 2\cdot 3\cdot 5,\
33 = 3\cdot 11,\
70 = 2\cdot 5\cdot 7.
\endgather
$$
V~týchto súčinoch sa nachádzajú činitele 7 a~11, teda výsledný objem musí
byť násobkom čísla 77. Činitele 7 a~11 sú v~zadaných dĺžkach zahrnuté
každý iba raz, a~to v~hranách 33 a~70. Rozhodnime, či tieto dĺžky môžu patriť dvom rôznym kvádrom.

Keby hrany 33 a~70 patrili rôznym kvádrom, musel by kváder s~hranou 33 mať
ďalšiu hranu
rovnú násobku sedem, kváder s~hranou 70 by musel mať ďalšiu hranu rovnú
násobku jedenástich a~posledný kváder by musel mať medzi svojimi hranami násobok
siedmich a~násobok jedenástich. Práve sme predpokladali existenciu aspoň troch
hrán, ktoré nie sú uvedené v~zadaní, ale v~ňom pritom chýbajú iba dve. Tým
sme ukázali, že hrany 33 a~70 patria rovnakému kvádru.

Obe dĺžky hrán, ktoré nie sú v~zadaní uvedené, musia byť násobkami čísla 77 a~patriť k zostávajúcim kvádrom. Zvyšnú
hranu nášho kvádra preto musíme hľadať medzi zadanými hranami. Všimnime si
hrany 18 a~24. Podľa prvej odvodíme, že výsledný objem je násobkom deviatich
(\tj. $3\cdot 3$), podľa druhej je zároveň násobok ôsmich
(\tj. $2\cdot 2\cdot 2$).
V~súčinoch zodpovedajúcich hranám 33 a~70 sa nachádzajú činitele 2 a~3
každý práve raz. Tretia hrana uvažovaného kvádra preto musí mať vo svojom
rozklade súčin $2\cdot 2\cdot 3$. V~zadaní tak môžeme vybrať buď hranu 12,
alebo 24.

Uvažujme najskôr o~možnosti, že jeden z~kvádrov má hrany 33, 70 a~12, teda
že kvádre majú objem $2\cdot 2\cdot 2\cdot 3\cdot 3\cdot 5\cdot 7\cdot 11$.
Pre druhý kváder vyberieme hranu 24 a~vidíme, že ten už nesmie mať v~dĺžke
žiadnej ďalšej hrany činiteľ 2. V~zadaní však ostávajú iba také dĺžky, preto možnosť s~kvádrom s~hranami 33, 70 a~12 musíme zavrhnúť.

Teraz uvažujme o~možnosti, že jeden z~kvádrov má hrany 33, 70 a~24, teda že
kvádre majú objem
$$
2\cdot 2\cdot 2\cdot 2\cdot 3\cdot 3\cdot 5\cdot 7\cdot 11.
$$
Hrany zostávajúcich dvoch kvádrov ľahko určíme, pokiaľ sa držíme
poznatku, že kváder musí mať v~dĺžkach svojich hrán raz činiteľ 5 a~práve dvakrát činiteľ 3:
druhý kváder má hrany
$$
30 = 2\cdot 3\cdot 5,\ 12 = 2\cdot 2\cdot 3,\ 154 = 2\cdot 7\cdot 11
$$
a~hrany tretieho kvádra sú
$$
20 = 2\cdot 2\cdot 5,\ 18 = 2\cdot 3\cdot 3,\ 154 =  2\cdot 7\cdot 11.
$$

Dĺžky zostávajúcich dvoch hrán, ktoré nie sú uvedené v~zadaní, sú zhodne $154\cm$.
}

{%%%%%   Z7-I-5
\napad
Nemusíme poznať veľkosti zvyšných dvoch vnútorných uhlov, aby sme úlohu doriešili.

\riesenie
Uvažujme trojuholník $ABC$ s~uhlom 50\st\ pri~vrchole $A$;
neznáme uhly pri~vrcholoch $B$ a~$C$ označíme $\beta$ a~$\gamma$.
Priesečník osí vnútorných uhlov označíme $O$, uhol $BOC$ označíme $\omega$ a~uhol k~nemu susedný $\psi$ (\obr).
\insp{z61.11}%

Súčet vnútorných uhlov v~ľubovoľnom trojuholníku je 180\st.
Preto v~trojuholníkoch $ABC$ a~$OBC$ platí
$$\align
50\st+\beta+\gamma&=180\st, \\
\omega+\frac\beta2+\frac\gamma2&=180\st.
\endalign
$$
Z~druhej rovnosti a~z~toho, že $\omega$ a~$\psi$ sú susedné uhly, vyplýva
$$
\psi=\frac\beta2+\frac\gamma2.
$$
Z~prvej rovnosti vyjadríme
$$
\frac\beta2+\frac\gamma2=\frac{130\st}2=65\st,
$$
teda odchýlka osí zostávajúcich dvoch vnútorných uhlov je 65\st.

\poznamka
Odpoveď, že osi  zvierajú uhol $\omega=180\st-65\st=115\st$, považujeme tiež za správnu.
}

{%%%%%   Z7-I-6
\napad
Zamerajme sa na to, ako vyzerajú jednotlivé dvojčíslia, obzvlášť to posledné.

\riesenie
Posledná cifra nemôže byť 0 ani 5:
keby to tak bolo, tak by podľa piatej podmienky prvé a~druhé
dvojčíslie končilo buď 0 alebo 5, takže cifra 0 alebo 5 by bola v~kóde obsiahnutá
viackrát, čo odporuje prvej podmienke.

S~týmto poznatkom spolu s~ostatnými podmienkami zo zadania začneme
vypisovať všetky možné dvojčíslia, ktoré sa môžu vyskytovať na konci kódu.
Navyše, aby bola splnená piata a~prvá podmienka, má zmysel uvažovať len také
dvojčíslia, ktoré majú  aspoň dva rôzne násobky menšie ako 100.
Všetky vyhovujúce možnosti sú uvedené v~ľavom stĺpci nasledujúcej
tabuľky.
Pravý stĺpce  obsahuje všetky ich dvojciferné násobky, ktoré
prípadne môžu tvoriť prvé  a~druhé dvojčíslie hľadaného kódu.
$$\begintable
14|28, 42, 56, 70, 84, 98\hfill\cr
16|32, 48, 64, 80, 96\hfill\cr
18|36, 54, 72, 90\hfill\cr
27|54, 81\hfill\cr
29|58, 87\hfill\endtable
$$
Pokiaľ vyradíme všetky dvojčíslia, ktoré nevyhovujú tretej alebo štvrtej
podmienke zo zadania, zostávajú len:
$$\begintable
14|70\hfill\cr
16|96\hfill\cr
18|36, 72, 90\hfill\cr
27|81\hfill\cr
29|58\hfill\endtable
$$
Odtiaľ je zrejmé, že posledné dvojčíslie musí byť 18.
Aby bola splnená druhá podmienka, musí byť jedno zo zostávajúcich dvojčíslí 90, a~aby bola splnená štvrtá podmienka, musí byť 90 ako prvé dvojčíslie.
Z rovnakého dôvodu nemôže byť druhé dvojčíslie 72. Zostáva už len 36.
Výsledný kód teda môže byť jedine číslo $903618$
a~kontrolou všetkých podmienok zo zadania zistíme, že to tak skutočne je.

\poznamka
Popri úvodnom poznatku, že 0 nemôže byť posledná cifra, sa dá využiť aj to, že 0 nemôže byť na prvom ani na treťom mieste. (Inak by prvé alebo
druhé dvojčíslie predstavovalo jednociferné číslo, teda podľa  piatej podmienky
by i~posledné dvojčíslie muselo byť jednociferné číslo a~na piatom mieste by
musela byť zase 0.)
Preto je 0  buď na druhom alebo štvrtom mieste.
Z tretej podmienky potom vyplýva, že párne cifry môžu byť len na
párnych a~nepárne na nepárnych miestach.
Ďalšia diskusia sa potom trošku zjednoduší.
}

{%%%%%   Z8-I-1
\napad
Všetky medzivýsledky musia byť prirodzené čísla.

\riesenie
Počet všetkých riešiteľov prvého kola označme~$x$.
Počet úspešných riešiteľov prvého kola (a~teda počet všetkých riešiteľov druhého
kola) je 14\,\% z~$x$, teda $0{,}14x$.
Počet úspešných riešiteľov druhého kola (a~teda počet všetkých riešiteľov tretieho
kola) je 25\,\% z~$0{,}14x$, \tj. $0{,}25\cdot 0{,}14x = 0{,}035x$.
Počet úspešných riešiteľov tretieho kola (a~teda aj počet víťazov) je 8\,\%
z~$0{,}035x$, \tj. $0{,}08\cdot 0{,}035x = 0{,}0028x$.

Keďže všetky výpočty sú presné (bez zaokrúhľovania), musia byť čísla
$x$, $0{,}14x$, $0{,}035x$ a~$0{,}0028x$ prirodzené.
Začneme posledným z~nich:
$$
0{,}0028x=\frac{28}{10\,000}x=\frac7{2\,500}x,
$$
číslo~$x$ teda musí byť násobkom čísla $2\,500$.
Keďže hľadáme najmenšie riešenie, budeme postupne skúšať násobky $2\,500$, kým nebudú všetky spomínané čísla prirodzené:
$$\begintable
$x$\|$0{,}14x$|$0{,}035x$|$0{,}0028x$\|záver\crthick
2\,500\|350|87{,}5|7\|nevyhovuje\cr
5\,000\|700|175|14\|vyhovuje\endtable
$$
Najmenší počet súťažiacich, ktorí sa mohli zúčastniť prvého kola, je 5\,000.
Víťazov by v~takom prípade bolo 14.


\ineriesenie
Počet všetkých riešiteľov prvého kola označme~$x$.
Počet úspešných riešiteľov prvého kola (a~teda počet všetkých riešiteľov druhého
kola) je 14\,\% z~$x$, teda
$$
\frac{14}{100}x=\frac7{50}x.
$$
Počet úspešných riešiteľov druhého kola (a~teda počet všetkých riešiteľov tretieho
kola) je 25\,\% z~predchádzajúceho počtu, \tj.
$$
\frac{25}{100}\cdot\frac7{50}x=\frac{7}{200}x.
$$
Počet úspešných riešiteľov tretieho kola (a~teda aj počet víťazov) je 8\,\%
z~predchádzajúceho počtu, \tj.
$$
\frac8{100}\cdot\frac7{200}x=\frac7{2500}x.
$$

%%Protože všechny výpočty jsou přesné (bez zaokrouhlování), musejí být
%%Všechna čísla $x$, $\frac7{50}x$, $\frac{7}{200}x$  a~$\frac7{2500}x$
Všetky vyššie uvedené výrazy musia byť prirodzené čísla,
číslo~$x$ teda musí byť spoločným násobkom čísel $50$, $200$ a~$2\,500$.
Keďže nás zaujíma najmenší možný počet súťažiacich v~prvom kole súťaže,
hľadáme najmenší spoločný násobok uvedených čísel, čo je $5\,000$.

Najmenší počet súťažiacich v~prvom kole je teda 5\,000 a~počet víťazov by
v~tomto prípade bol
$$
\frac7{2500}\cdot5000=14.
$$
}

{%%%%%   Z8-I-2
\napad
Uvedená konštrukcia je osovo súmerná.

\riesenie
Trojuholník $ABC$ je súmerný podľa osi~$CS$, preto aj deliace priamky musia
byť osovo súmerné podľa tejto osi.
Výsledné časti potom budú tvoriť dve dvojice osovo súmerných trojuholníkov
a~jeden (osovo súmerný) štvoruholník s~vrcholom~$C$.
Označme priesečníky dvoch deliacich priamok s~jedným ramenom $X$ a~$Y$ ako na \obr.
\insp{z61.12}%

Podľa zadania majú byť obsahy trojuholníkov $ASX$ a~$XSY$ a~dvojnásobok obsahu
trojuholníka $YSC$ rovnaké.
Tieto tri trojuholníky však majú rovnakú výšku zo spoločného vrcholu~$S$,
takže ich obsahy sú v~uvedenom pomere práve vtedy, keď pre protiľahlé
strany platí
$$
|AX|=|XY|=2|YC|.
$$
Súčasne vieme, že
$$
|AC|=|AX|+|XY|+|YC|=20\cm.
$$
Z~uvedeného vyplýva, že $5|YC|=20\cm$, \tj. $|YC|=4\cm$ a~$|AX|=|XY|=8\cm$.
Deliace priamky vytínajú na ramenách trojuholníka úsečky dlhé $4$ a~$8\cm$.
}

{%%%%%   Z8-I-3
\napad
Určte, ako môžu byť umiestnené cifry $2$ a~$4$; pre každý prípad zvlášť potom
diskutujte zvyšné podmienky.

\riesenie
Päťciferné palindrómy, v~ktorých sa cifra~$2$ nachádza bezprostredne za
cifrou~$4$, sú práve nasledujúce:
$$
42{*}24,\quad {*}424{*},\quad {*}242{*},\quad 24{*}42.
$$
Pre tieto prípady stačí teraz diskutovať deliteľnosť dvanástimi.
Číslo je deliteľné dvanástimi práve vtedy, keď je deliteľné tromi a~zároveň
štyrmi, \tj. práve vtedy, keď jeho ciferný súčet je deliteľný tromi
a~zároveň posledné dvojčíslie je deliteľné štyrmi.

Číslo $24$ je deliteľné štyrmi, preto sú palindrómy typu $42{*}24$
vždy deliteľné štyrmi, a~preto sme zaujímame len o~deliteľnosť tromi.
Známe cifry majú ciferný súčet~$12$, ktorý deliteľný tromi je, preto
hviezdička uprostred musí zastupovať násobok troch -- $0$, $3$, $6$ alebo $9$.

Palindrómy typu ${*}424{*}$ sú deliteľné štyrmi práve vtedy, keď posledná
cifra je $0$, $4$ alebo $8$. Keďže sa jedná o~palindróm, rovnaká cifra bude aj na
začiatku, preto variant s~nulou nevyhovuje. Po doplnení štvoriek je ciferný
súčet $18$, po doplnení osmičiek $26$. Takže deliteľný tromi je len palindróm
$44244$.

Palindrómy typu ${*}242{*}$ sú deliteľné štyrmi práve vtedy, keď posledná
cifra je $0$, $4$ alebo $8$.
Rovnako ako v~predošlom prípade vylúčime cifru~$0$ a~určíme ciferné súčty:
pre štvorky je to $16$, pre osmičky $24$. Deliteľný tromi je len palindróm
$82428$.

Keďže číslo $42$ nie je deliteľné štyrmi, palindrómy typu $24{*}42$
nemôžu byť deliteľné štyrmi, teda ani dvanástimi.
Zadaným podmienkam teda vyhovujú práve čísla
$$
42024,\,\, 42324,\,\, 42624,\,\, 42924,\,\, 44244,\,\, 82428.
$$
}

{%%%%%   Z8-I-4
\napad
Zamerajte sa na vzťah medzi číslami na navzájom rovnobežných bočných stenách.

\riesenie
Čísla, ktorá vidíme pred roztočením kruhu, označme $a$, $b$, $c$, pričom
$c$ je číslo na hornej stene. Po otočení o~$90\st$ stratíme z~nášho pohľadu
stenu s~číslom~$a$ a~objaví sa stena s~ňou rovnobežná. Podľa zadania sa
súčet viditeľných čísel zmení z~$42$ na $34$, teda zmenší sa o~$8$. Číslo, ktoré sa ukázalo po otočení
je preto o~$8$ menšie ako $a$, \tj. $a-8$.

Podobne uvažujeme o~ďalšej otočke o~$90\st$. Pri nej stratíme z~pohľadu stenu
s~číslom~$b$ a~súčet viditeľných čísel sa zmení z~$34$ na $53$, teda zväčší sa
o~$19$. Na poslednej bočnej stene sa preto objaví číslo $b+19$.
\insp{z61.13}%

Ešte po ďalšom otočení o~$90\st$ tak vidíme steny s~číslami $b + 19$, $a$,
$c$. Zo zadania vieme, že $a+ b + c = 42$, teda $a+ b + 19 + c = 42 + 19 =
61$. Súčet $61$ je riešením prvej časti úlohy.

Teraz budeme riešiť druhú časť úlohy. Pred roztočením kruhu vidíme tri steny so súčtom~$42$,
po otočení o~$180\st$ vidíme iné dve bočné steny a~stále rovnakú hornú
stenu s~číslom~$c$, tentoraz ide o~súčet~$53$. Teda súčet čísel na týchto
piatich stenách je rovný $42 + 53 - c$. Na kocke je podľa zadania zospodu
napísané číslo~$6$, súčet všetkých jej čísel je teda rovný $6 + 42 + 53 - c$,
\tj. $101 - c$. Ak máme určiť najväčšiu možnú hodnotu tohto výrazu, dosadíme
za $c$ najmenšiu prípustnú hodnotu~$1$. Potom vidíme, že súčet čísel na
kocke mohol byť najviac $100$.

\bfodsek{Iné riešenie druhej časti}
Z~vyššie uvedeného riešenia použijeme \obr{} s~jeho popisom. Súčet všetkých čísel
na kocke je
$$
a+ b + (a- 8) + (b + 19) + c + 6 = 2a + 2b + c + 17.
$$
Pritom platí, že všetky neznáme sú prirodzené čísla a~$a\ge9$, aby aj hodnota $a- 8$ bola prirodzené číslo. Poslednou podmienkou je $a+ b + c =
42$. Aby sme pri danom súčte $a+ b + c$ získali čo najväčšiu hodnotu výrazu
$2a + 2b + c + 17$, musíme za $c$ zvoliť čo najmenšiu prípustnú hodnotu, \tj.~$1$,
pretože ostatné neznáme sú vo výraze zastúpené vo svojich násobkoch. Súčet $a+ b$ potom
nadobúda hodnotu~$41$ a~súčet všetkých šiestich čísel na kocke tak môže byť najviac
$$
2a + 2b + c + 17 = 2\cdot 41 + 1 + 17 = 100.
$$
}

{%%%%%   Z8-I-5
\napad
Bonifácov vek možno určiť veľmi ľahko.

\riesenie
Z~tretej rovnice vyplýva, že $B$ je prirodzené číslo menšie ako $16$ práve vtedy,
keď $S-P=1$, čiže $S=P+1$; potom nutne $B=8$.
Dosadíme tieto poznatky do druhej rovnice a~určíme~$P$:
$$
\align
P+1&=2(8-P),\\
P+1&=16-2P,\\
3P&=15,\\
P&=5.
\endalign
$$
Odtiaľ $S=5+1=6$ a~ľahko overíme, že trojica $B=8$, $P=5$ a~$S=6$ vyhovuje
aj rovnici prvej:   $5=\frac52(8-6)$.
Pankrác má teda~5, Servác~6 a~Bonifác 8~rokov.

\poznamka
S~rovnicami zo zadania možno manipulovať rôznymi spôsobmi, no bez obmedzenia
$P,S,B<16$ by úloha nemala riešenie určené jednoznačne -- nájdete nejaké
ďalšie?

\ineriesenie
Zo zadania vyplýva, že  $P$, $S$ a~$B$ sú kladné čísla práve vtedy, keď
$B>S>P>0$.
Rovnako ako v~predchádzajúcom riešení určíme, že z~tretej rovnice vyplýva
$B=8$ a~$P=S-1$.
Navyše z~druhej rovnice je zrejmé, že $S$ je párne číslo.
Spolu teda vidíme, že riešením úlohy môže byť jedine niektorá z~nasledujúcich
trojíc čísel:
$$\begintable
$B$|$S$|$P$\crthick
8|6|5\cr
8|4|3\cr
8|2|1\endtable
$$
Dosadením do prvej a~druhej rovnice zistíme, že jediným riešením je trojica
$B=8$, $S=6$ a~$P=5$.
}

{%%%%%   Z8-I-6
\napad
Určte, ako mohla Janka obdĺžnik rozdeliť; pre jednotlivé možnosti potom
vyjadrite zadaný rozdiel obvodov pomocou dĺžok deliacich čiar.

\riesenie
Všetky veličiny v~texte sú vyjadrené v~centimetroch, jednotky ďalej uvádzať
nebudeme.
Dĺžky strán Jankinho obdĺžnika označíme $x$ a~$y$, podľa zadania sú to
prirodzené čísla.

Najskôr zistíme, ako mohla Janka svoj obdĺžnik rozdeliť.
Typovo máme len dve možnosti uvedené na \obr{} (ktorý je len
schematický, \tj. rozhodne nepredpokladáme, že $x>y$).
\insp{z61.14}%

Obvod pôvodného obdĺžnika je $2(x + y) = 22$, teda $x + y = 11$.
Súčet obvodov troch nových obdĺžnikov je vždy väčší ako pôvodný obvod,
a~to práve o~dvojnásobok súčtu dĺžok deliacich úsečiek, ktoré sú na \obrr1{}
vyznačené prerušovane.
Tento rozdiel má byť rovný~$18$.

\smallskip
I.
Obe deliace úsečky majú rovnakú dĺžku~$y$.
Potom musí platiť $4y = 18$, odtiaľ $y = 4{,}5$.
To však nie je možné, pretože $4{,}5$ nie je celé číslo.
Týmto spôsobom teda Janka obdĺžnik nerozdelila.

\smallskip
II.
Dve deliace úsečky, ktoré ležia na jednej priamke, majú súčet dĺžok~$y$.
Dĺžku tretej deliacej úsečky označíme~$z$.
Potom musí platiť $2y + 2z = 18$, teda
$
y + z~= 9.
$
To spolu s~podmienkou $x + y = 11$ znamená, že rozmer~$x$ je o~$2$ väčší ako
rozmer~$z$.
Tento poznatok naznačíme do \obr.
\insp{z61.141}%

Teraz preveríme, ktorý z~nových obdĺžnikov môže mať rozmery $2\times 6$ a~ako
môže byť umiestnený -- celkom máme tri možnosti ako na \obr.
\insp{z61.142}%

Pomocou skôr odvodených vzťahov medzi $x$, $y$ a~$z$
%%$x+y=11$, $y+z=9$, příp. $x=z+2$,
vyjadríme rozmery obdĺžnika v~jednotlivých prípadoch:

a) Ak $y=6$, tak $x=5$ a~$z=3$. %Obdélník měl rozměry 6 cm a~5 cm.

b) Ak $z=6$, tak $x=8$ a~$y=3$. %Obdélník měl rozměry 8 cm a~3 cm.

c) Ak $z=2$, tak $x=4$ a~$y=7$. %Obdélník měl rozměry 7 cm a~4 cm.

Jankin obdĺžnik mohol mať rozmery $5\times6$, $8\times3$ alebo $4\times7$.

\poznamka
Pri rovnakom označení ako vyššie z~požiadavky, aby Jankin obdĺžnik obsahoval
obdĺžnik $2\times6$, vyplýva, že $x,y\ge2$.
Keďže $x$ a~$y$ sú prirodzené čísla a~$x+y=11$, rozmery Jankinho obdĺžnika
by mohli byť $2\times9$, $3\times8$, $4\times7$ alebo $5\times6$.
\insp{z61.143}%

Teraz možno postupne preberať tieto štyri prípady, teda
umiestniť obdĺžnik $2\times6$,
diskutovať možné dodatočné delenia (\obr)
a~kontrolovať požiadavku o~obvodoch.
Takto rýchlo zistíme, že jediné možnosti, ako mohla Janka svoj obdĺžnik
rozdeliť, sú práve vyššie uvedené možnosti a), b), c).
}

{%%%%%   Z9-I-1
\napad
Všimnite si, o~koľko sa líšia čísla na predaných žltých a~červených
vstupenkách.

\riesenie
Označme počet žltých vstupeniek~$n$.
Na prvej žltej vstupenke bolo číslo~$1$, na druhej~$2$, atď., na poslednej žltej
vstupenke bolo číslo~$n$.
Na prvej červenej vstupenke bolo číslo $n+1$, na druhej červenej $n+2$, atď.,
na poslednej červenej bolo číslo~$2n$.

Všimnime si, že prvá červená vstupenka má číslo o~$n$ väčšie ako prvá žltá.
Rovnako druhá červená vstupenka má číslo o~$n$ väčšie ako druhá žltá;
to isté platí pre všetky takéto dvojice vstupeniek, ktorých je celkom~$n$.
Súčet čísel na červených vstupenkách je preto o~$n^2$ väčší ako súčet
čísel na žltých vstupenkách.
Zo zadania vieme, že $n^2=1\,681$, teda $n=41$.

Pokladníčka v~ten deň predala 41~žltých a~41~červených vstupeniek, celkom teda
82~vstupeniek.
}

{%%%%%   Z9-I-2
\napad
Ktoré cifry môžu tvoriť poslednú a~ktoré prvú štvoricu?

\riesenie
Najskôr nájdime všetky štvorice tlačidiel, ktorých stredy tvoria štvorec --
sú to tlačidlá s~nasledujúcimi ciframi:
$$\align
1,\,2,\,4,\,5\quad\quad&1,\,3,\,7,\,9 \\
2,\,3,\,5,\,6\quad\quad&2,\,4,\,6,\,8 \\
4,\,5,\,7,\,8\quad\quad&5,\,7,\,9,\,0 \\
5,\,6,\,8,\,9\quad\quad
\endalign
$$

Štvorice v~ľavom stĺpci však využiť nemôžeme, lebo vedľa štvorca, ktorý
príslušné tlačidlá tvoria, by sme žiadny ďalší štvorec zo zvyšných tlačidiel
nezostavili.
Keďže telefónne číslo je deliteľné piatimi, musí končiť $5$ alebo $0$;
preto posledné štyri cifry telefónneho čísla sú $5$, $7$, $9$, $0$ (ich
poradie diskutujeme neskôr).
Keďže sme už použili cifry $7$ a~$9$, prvé štyri cifry telefónneho čísla
musia byť $2$, $4$, $6$, $8$ (v~tomto poradí, sú zoradené podľa veľkosti).

Doteraz nepoužité cifry, ktoré môžu byť uprostred telefónneho čísla, sú $1$
a~$3$.
Telefónne číslo má byť deliteľné tromi, určme teda možné ciferné súčty.
Súčet všetkých cifier na klávesnici je $45$.
Ak by v~telefónnom čísle bola $1$, \tj. telefónne číslo by obsahovalo
všetky cifry okrem~$3$, bol by ciferný súčet $45-3=42$.
Ak by v~telefónnom čísle bola~$3$, bol by ciferný súčet $45-1=44$.
Číslo~$42$ je deliteľné~$3$, číslo~$44$ nie je, prostredná cifra je teda~$1$.

Keďže sme nevynechali žiadnu z~požiadaviek v~zadaní, hľadané telefónne
číslo je
$$
246\,81{*}\,{*}{*}{*},
$$
pričom posledné štyri cifry sú $5$, $7$, $9$, $0$ v~neznámom poradí, vieme len, že
na poslednom mieste musí byť $5$ alebo $0$.
Aby sme zistili počet všetkých možných Kláriných telefónnych čísel, nebudeme
ich všetky vypisovať, len si touto predstavou pomôžeme:
Poslednú cifru možno zvoliť dvoma spôsobmi, predposlednú cifru potom
vyberáme spomedzi troch zvyšných cifier, cifru pred ňou už len spomedzi dvoch
zvyšných a~na posledné nevyplnené miesto nám vždy zvýši jediná cifra.
Dostávame dokopy
$$
2\cdot3\cdot2=12
$$
možných poradí na posledných štyroch miestach, a~teda aj 12 možných
Kláriných telefónnych čísel.
}

{%%%%%   Z9-I-3
\napad
Zostavte sústavu rovníc. Pred samotným riešením sústavy si všimnite, že
niektoré rovnice sú prebytočné, \tj. dajú je odvodiť z~ostatných.

\riesenie
Pôvodný počet veveričiek na smreku označíme~$s$, na buku~$b$ a~na jedli~$j$.
Zo zadania môžeme zostaviť sústavu štyroch rovníc o~týchto troch neznámych:
$$
\align
s+ b + j &= 34, \\
b + 7 &= j + s- 7, \\
j - 5 &= s- 7, \\
b + 7 + 5 &= 2j.
\endalign
$$
Všimnime si, že sčítaním tretej a~štvrtej rovnice dostaneme rovnicu druhú.
Pre vyriešenie sústavy rovníc preto stačí vybrať ľubovoľné dve z~týchto troch
rovníc a~doplniť ich prvou rovnicou.
Takto dostaneme sústavu troch rovníc o~troch neznámych.
Ukážeme si riešenie s~prvou, druhou a~štvrtou rovnicou:
$$
\align
s+ b + j &= 34, \\
b + 7 &= j + s- 7, \\
b + 7 + 5 &= 2j.
\endalign
$$
Sčítame prvé dve rovnice a~upravíme:
$$
\align
s+ 2b + j +7 &= 27 + j + s, \\
    2b &= 20, \\
      b &= 10.
\endalign
$$
Dosadíme tento výsledok do poslednej rovnice a~vyjadríme~$j$:
$$
\align
10+7+5 &= 2j, \\
22 &= 2j, \\
j &=11.
\endalign
$$
Všetko dosadíme do prvej rovnice a~vyjadríme~$s$:
$$
\align
s+10+11&=34, \\
s&=13.
\endalign
$$
Na smreku pôvodne sedelo 13, na buku 10 a~na jedli 11~veveričiek.


\inynapad
Určte, koľko veveričiek sedelo na buku vo chvíli, keď ich tam bolo rovnako veľa
ako na oboch ihličnatých stromoch.

\ineriesenie
Po prvom preskákaní bola na buku presne polovica veveričiek, čiže 17.
Keďže ich na buk 7 priskočilo, sedelo pôvodne na buku $17 - 7 = 10$
veveričiek.

Po druhom preskákaní bolo na buku $17 + 5 = 22$ veveričiek.
Vieme, že na konci bolo na buku dvakrát viac veveričiek ako na jedli na začiatku,
teda na začiatku sedelo na jedli $22 : 2 = 11$ veveričiek.

Celkom bolo veveričiek 34, preto bolo pôvodne na smreku $34 - 10 - 11 = 13$
veveričiek.
Práve sme došli k~pôvodným počtom veveričiek na všetkých stromoch, ale vôbec sme
nepracovali so zadanou informáciou, že po všetkých preskokoch bolo na jedli
rovnako veľa veveričiek ako na smreku.
Musíme overiť, či táto informácia nie je v~rozpore s~predchádzajúcimi výpočtami (v~opačnom prípade by úloha nemala žiadne riešenie):
$11 - 5 = 13 - 7$, teda naše výsledky zodpovedajú celému zadaniu.

Na smreku pôvodne sedelo 13, na buku 10 a~na jedli 11 veveričiek.

\poznamka
Uvedené riešenia predstavujú rôzne pohľady na ten istý problém.
Porovnajte hlavne, ako sme odhalili prebytočnosť jednej zo zadaných
informácií v~prvom a~ako v~druhom prípade.
Vedeli by ste túto prebytočnosť zistiť aj inak?
}

{%%%%%   Z9-I-4
\napad
Rozdeľte päťuholník na trojuholníky so spoločným vrcholom v~strede
kružnice opísanej a~zistite ich vlastnosti.

\riesenie
Najprv spočítajme veľkosť stredového uhla pravidelného dvanásťuholníka
$ABCDEFGHIJKL$, \tj. uhla s~vrcholom v~strede $S$~opísanej kružnice a~ramenami
prechádzajúcimi susednými vrcholmi.
Súčet všetkých dvanástich stredových uhlov je $360\st$, takže veľkosť jedného
stredového uhla je $360\st : 12 = 30\st$.
\insp{z61.101}%

Veľkosti stredových uhlov päťuholníka $ACFHK$ sú
$$
\def\vb{\vert}
\gather
|\uhol ASC|=|\uhol FSH|=|\uhol KSA|=2\cdot 30\st=60\st, \\
|\uhol CSF|=|\uhol HSK|=3\cdot 30\st=90\st.
\endgather
$$
Trojuholníky $ASC$, $FSH$, $KSA$ sú teda rovnostranné a~trojuholníky
$CSF$ a~$HSK$ sú rovnoramenné pravouhlé (\obr).
Preto
$$
\def\vb{\vert}
\gather
|AC|=|FH|=|KA|=r=6\cm, \\
|CF|=|HK|=r\sqrt2=6\sqrt2\cm.
\endgather
$$
Obvod päťuholníka $ACFHK$ je tak rovný $3\cdot 6+2\cdot6\sqrt2=18+12\sqrt2\doteq 34{,}97\,(\Cm)$.
}

{%%%%%   Z9-I-5
\napad
Určte vzťah medzi počtom výrobkov predaných pred koncertom a~za ne získanou
čiastkou. Uvedomte si, že všetky neznáme majú byť celé čísla.

\riesenie
Počet výrobkov, ktoré žiaci predali pred koncertom, označme~$v$~a~sumu
v~centoch, ktorú za ne získali, označme~$k$.
Podľa zadania je podiel $k/v$ celé číslo a~platí
$$
\frac{k+2\,505}{v+7}=130.
$$
Z~tejto rovnice vyjadríme neznámu~$k$ pomocou neznámej~$v$:
$$
\align
k+2\,505&=130v+130\cdot7, \\
k&=130v-1\,595.
\endalign
$$
Získaný výraz dosadíme do zlomku $k/v$ a~následne ho čiastočne
vydelíme:
$$
\frac{130v-1\,595}{v}=130-\frac{1\,595}{v}.
$$

Neznáma~$v$ je prirodzené číslo, a~keďže práve uvedený výraz má byť rovný
celému číslu, musí byť $v$ deliteľom čísla $1\,595$.
Pritom číslo $1\,595 = 5\cdot 11\cdot29$ má práve nasledujúce delitele:
$$
1,\ 5,\ 11,\ 29,\ 55,\ 145,\ 319,\ 1\,595.
$$
Podľa zadania zostalo po predaji $v$~výrobkov z~celkových 60 minimálne~7,
teda $v\le53$.
Keď do rovnice $k=130v-1595$ dosadíme za $v$ $1$, $5$ či $11$, dostaneme $k$ záporné, čo odporuje zadaniu.
Pre $v=29$ je tržba $k$ kladná, takže jediná prípustná hodnota pre $v$~je $29$.
Celkovo sa predalo $29 + 7 = 36$ výrobkov a~nepredaných zostalo $60 - 36 = 24$.

\inynapad
Ak sa hovorí o~priemernej cene, skúste si situáciu zjednodušiť
predstavou, že všetci zaplatili zhodne práve túto cenu.

\ineriesenie
Pre potreby riešenia môžeme situáciu zjednodušiť a~predstaviť si, že každý
zákazník nastupujúci pred koncertom zaplatil rovnakú celočíselnú sumu.
Tá sa rovnala priemernej cene vypočítanej žiakmi pred koncertom.

Po koncerte prišla sedmica zákazníkov, ktorá za každého skoršieho zákazníka
doplatila zaplatenú čiastku do 130 centov a~sama za seba zaplatila $7\cdot
130$ centov.
Celkové doplatenie ceny teda zodpovedalo čiastke $2\,505 - 7\cdot 130  = 1\,595$~(centov).

Túto sumu musíme rozložiť na súčin, pričom jedným činiteľom bude počet
zákazníkov pred koncertom a~druhým činiteľom počet centov, ktoré za každého
takého zákazníka doplatila sedmice neskorších zákazníkov.
O~prvom činiteli vieme zo zadania, že musí byť menší alebo rovný
$53$, a~o~druhom vieme, že musí byť menší ako $130$.
Číslo $1\,595$ možno rozložiť na súčin dvoch prirodzených čísel práve týmito
spôsobmi:
$$
1\cdot 1\,595,\ 5\cdot 319,\ 11\cdot 145,\ 29\cdot 55.
$$
Uvedeným podmienkam vyhovuje jedine rozklad $29\cdot 55$.
Celkovo sa teda predalo $29 + 7 = 36$ výrobkov a~nepredaných ostalo
$60 - 36 = 24$.
}

{%%%%%   Z9-I-6
\napad
Použite vhodne Pytagorovu vetu, príp. opačnú vetu.

\riesenie
Obdĺžnik predstavujúci pôdorys záhrady označíme $ABCD$,
broskyňa na jednej z~jeho uhlopriečok je zastúpená bodom~$X$.
Povedzme, že dva susedné rohy zo zadania sú $A$, $B$ a~platí
$|AX| = 5$, $|BX| = 12$, $|AB| = 13$.
(Všetky dĺžky sú v~metroch a~jednotky ďalej nepíšeme.)
Tieto čísla tvoria pytagorejskú trojicu, čiže platí $5^2+12^2=13^2$.
Preto je trojuholník $AXB$ pravouhlý s~preponou~$AB$, \tj. s~pravým uhlom pri vrchole~$X$.

Bod~$X$ môže ležať buď na uhlopriečke~$AC$ alebo na uhlopriečke~$BD$,
budeme diskutovať obe možnosti.
V~každom prípade vzdialenosť bodu~$X$ od druhého vrcholu na uhlopriečke
označíme~$x$ a~neznámu dĺžku strany obdĺžnika označíme~$y$.
Zo zadaných informácií určíme~$y$, plocha záhrady (v~štvorcových metroch) potom bude rovná $13y$.
\insp{z61.15}%
\insp{z61.151}%

\smallskip
I.
Bod~$X$ leží na uhlopriečke~$AC$ (\obr).
Podľa Pytagorovej vety pre pravouhlé trojuholníky $ABC$ a~$BXC$ zostavíme
sústavu dvoch rovníc o~dvoch neznámych:
$$\align
(5+x)^2 &= 13^2 + y^2, \\
y^2 &= 12^2 + x^2.
\endalign
$$
Do prvej rovnice dosadíme za $y^2$ a~vypočítame~$x$:
$$
\align
(5+x)^2&=13^2+12^2+x^2, \\
25+10x+x^2&=169+144+x^2, \\
10x&=288, \\
x&=\frac{144}{5}.
\endalign
$$
Dosadíme za $x$ do druhej rovnice a~výraz upravíme:
$$
y^2 =12^2+\Big(\frac{144}{5}\Big)^2
=144+\frac{20\,736}{25} =\frac{24\,336}{25}.
$$
Odtiaľ vyplýva, že $y=156/5$
a~obsah obdĺžnika $ABCD$, \tj. plocha záhrady (v~štvorcových metroch),
je v~tomto prípade
$$
13\cdot\frac{156}{5}=\frac{2\,028}{5}=405{,}6.
$$


\smallskip
II.
Bod~$X$ leží na uhlopriečke~$BD$ (\obr).
Podľa Pytagorovej vety pre pravouhlé trojuholníky $DAB$ a~$DXA$ zostavíme
sústavu dvoch rovníc o~dvoch neznámych:
$$
\align
(12+x)^2 &= 13^2 + y^2, \\
y^2 &= 5^2 + x^2.
\endalign
$$
Do prvej rovnice dosadíme za $y^2$ a~vypočítame~$x$:
$$
\align
(12+x)^2&=13^2+5^2+x^2, \\
144+24x+x^2&=169+25+x^2, \\
24x&=50, \\
x&=\frac{25}{12}.
\endalign
$$
Dosadíme za $x$ do druhej rovnice a~výraz upravíme:
$$
y^2 =5^2+\Big(\frac{25}{12}\Big)^2
=25+\frac{625}{144} =\frac{4\,225}{144}.
$$
Odtiaľ vyplýva, že $y=65/12$
a~obsah obdĺžnika $ABCD$, \tj. plocha záhrady (v~štvorcových metroch),
je v~tomto prípade
$$
13\cdot\frac{65}{12}=\frac{845}{12}\doteq 70{,}42.
$$

\poznamka
Všimnite si výpočet dĺžky~$x$.
Dvojitým použitím Pytagorovej vety sme v~prvom prípade odvodili, že
$5x=144$, čomu zodpovedá $|AX|\cdot|XC|=|XB|^2$.
Uvedený výpočet v~podstate dokazuje, že táto rovnosť platí v~ľubovoľnom
pravouhlom trojuholníku $ABC$, kde $X$ je päta výšky na preponu~$AC$.
Toto tvrdenie je známe ako Euklidova veta o~výške.

\inynapad
Všimnite si podobné trojuholníky.

\ineriesenie
Pri rovnakom označení ako vyššie môžeme jednotlivé možnosti diskutovať
nasledovne.

\smallskip
I.
Bod~$X$ leží na uhlopriečke~$AC$.
Trojuholníky $ABC$ a~$AXB$ sú oba pravouhlé a~majú rovnaký vnútorný uhol pri spoločnom vrchole~$A$.
Tieto trojuholníky sú teda podobné, a~preto platí
$$
\frac{|BC|}{|AB|}=\frac{|XB|}{|AX|},
\qquad\text{čiže}\qquad
\frac{y}{13}=\frac{12}{5}.
$$
Odtiaľ $y=156/5$ a~záver je rovnaký ako v~predošlom riešení.

\smallskip
II.
Bod~$X$ leží na uhlopriečke~$BD$.
Trojuholníky $DAB$ a~$AXB$ sú oba pravouhlé a~majú rovnaký vnútorný uhol pri spoločnom vrchole~$B$.
Tieto trojuholníky sú teda podobné, a~preto platí
$$
\frac{|DA|}{|AB|}=\frac{|AX|}{|XB|},
\qquad\text{čiže}\qquad
\frac{y}{13}=\frac{5}{12}.
$$
Odtiaľ $y=65/12$ a~záver je rovnaký ako v~predošlom riešení.
}

{%%%%%   Z4-II-1
...}

{%%%%%   Z4-II-2
...}

{%%%%%   Z4-II-3
...}

{%%%%%   Z5-II-1
Prejdeme postupne všetky možnosti; pri každej uvedieme rozmery takého
obdĺžnika v~"dlaždiciach" a~ďalej obvod v~"dlaždiciach" (resp. v "stranách
dlaždíc"):
\begin{itemize}
  \item rozmery: $1\times 24$, obvod: $2\cdot(1+24)=50$,
  \item rozmery: $2\times 12$, obvod: $2\cdot(2+12)=28$,
  \item rozmery: $3\times 8$, obvod: $2\cdot(3+8)=22$,
  \item rozmery: $4\times 6$, obvod: $2\cdot(4+6)=20$.
\end{itemize}
Je zrejmé, že najmenší obvod vyjadrený v~centimetroch/metroch musí byť
zároveň najmenší aj pri vyjadrení v~"dlaždiciach".
Najmenší obvod dostávame vo štvrtom prípade~-- 20 dlaždíc.
Ak je strana jednej dlaždice dlhá $40\cm$, tak 20~takých strán meria
spolu $20\cdot40=800$\,(cm).
Obvod vydláždeného obdĺžnika meral 8~metrov.

\hodnotenie
1~bod za myšlienku riešenia;
po 1~bode za každú správne uvedenú a~rozobranú možnosť;
1~bod za výsledok.

Ak riešiteľ neuvedie všetky možnosti a~nevysvetlí, prečo niektoré z~nich
vynechal (napr. pri prvej možnosti, že je obvod príliš dlhý v~porovnaní
s~ostatnými obvodmi), môže získať najviac 4~body, a~to aj vtedy, ak uvedie
správny výsledok.
\endhodnotenie
}

{%%%%%   Z5-II-2
Vypísaním jednociferných čísel žiadnu nulu nezískame. Dvojciferné čísla obsahujúce nulu sú:
$$
\thinsize=0pt
\thicksize=0pt
\begintable
10, 20, \dots, 90\hskip.6cm|9 núl\hfill|\hskip2.6cm
%%ponekud nesystemove reseni, ale nevim, jak jinak
\endtable
$$
Trojciferné čísla obsahujúce nulu sú:
$$
\postdisplaypenalty 10000
\begintable
100\hfill                 |2 nuly\hfill|(celkom zatiaľ 11 núl)\cr
101, 102, \dots, 109\hfill|9 núl\hfill |(celkom zatiaľ 20 núl)\cr
110, 120, \dots, 190\hfill|9 núl\hfill |(celkom zatiaľ 29 núl)\cr
200\hfill                 |2 nuly\hfill|(celkom zatiaľ 31 núl)\cr
201, 202, 203, 204\hfill  |4 nuly\hfill|(celkom zatiaľ 35 núl)%
\endtable
$$
Cipúšik napísal ako posledné číslo $204$. (Číslo $205$ už napísať nemohol, lebo to už by obsahovalo 36. nulu.)


\ineriesenie
Postupne vypisujeme tie čísla, ktoré obsahujú aspoň jednu nulu. V~zátvorke za
každým z~nich je uvedené, koľko núl celkom sme už napísali, keď sme
dopísali príslušné číslo.

% \noindent
 10 (1),     20 (2),     30 (3),     40 (4),     50 (5),     60 (6),     70 (7),     80 (8),
 90 (9),    100 (11),   101 (12),   102 (13),   103 (14),   104 (15),   105 (16),   106 (17),
107 (18),   108 (19),   109 (20),   110 (21),   120 (22),   130 (23),   140 (24),   150 (25),
160 (26),   170 (27),   180 (28),   190 (29),   200 (31),   201 (32),   202 (33),   203 (34),
204 (35), 205 (36), \dots

Cipúšikovo posledné napísané číslo bolo $204$.

\hodnotenie
1~bod za určenie počtu núl v~číslach 1 až 99;
4~body za počty núl v~trojciferných číslach;
1~bod za poznatok, že 35. nula je v~čísle 204.
Ak riešiteľ nejakú nulu prehliadne, strhnite bod, ak by takých
prehliadnutí bolo viac, dajte najviac 4~body.
\endhodnotenie
}

{%%%%%   Z5-II-3
Medzi deviatimi lastovičkami je len 8 medzier.

$\bullet$ Vzdialenosť susedných lastovičiek je $720:8=90$\,(cm).

$\bullet$ Keby si do každej z~ôsmich medzier sadli 3 nové lastovičky, pribudlo by spolu
$8\cdot3=24$ lastovičiek.
Na drôte by potom sedelo $9+24=33$ lastovičiek.

\hodnotenie
1~bod za poznatok o~počte medzier;
2~body za prvú časť;
3~body za druhú časť úlohy.
\endhodnotenie
}

{%%%%%   Z6-II-1
Vieme, že každá z~dievčat omaľovala tri steny svojej kocky.
To možno urobiť iba dvoma spôsobmi:
\begin{enumerate}
  \item Niektoré dve protiľahlé steny sú omaľované, potom je omaľovaná ešte
    jedna stena medzi nimi; uvažujme napr. hornú, dolnú a~jednu bočnú stenu.
  \item Žiadne dve protiľahlé steny nie sú omaľované, teda tri omaľované steny
    majú spoločný vrchol; uvažujme napr. hornú, prednú a~bočnú stenu.
\end{enumerate}

Ak sú počty omaľovaných malých kociek v~jednotlivých prípadoch rôzne, museli
dievčatá omaľovať kocky rôznymi spôsobmi. Pri každom spôsobe omaľovania zistíme,
koľko malých kociek má omaľovanú aspoň jednu stenu.
Budeme postupovať po vrstvách, vrstvy počítame zdola:
\begin{enumerate}
  \item V~prvej a~piatej vrstve má aspoň jednu stenu omaľovanú všetkých 25 malých kociek,
    v~druhej, tretej a~štvrtej vrstve je omaľovaných vždy 5 malých kociek;
    celkom je v~tomto prípade omaľovaných $25+5+5+5+25=65$ malých kociek.
  \item V~prvých štyroch vrstvách má aspoň jednu stenu omaľovanú vždy 9
    malých kociek, v~piatej vrstve je omaľovaných všetkých 25 malých kociek;
    celkom je v~tomto prípade omaľovaných $9+9+9+9+25=61$ malých kociek.
\end{enumerate}
Hľadaný rozdiel počtov omaľovaných malých kociek je $65-61=4$.

\hodnotenie
1~bod za určenie možných spôsobov omaľovania;
4~body za určenie počtov omaľovaných malých kociek v~jednotlivých prípadoch;
1~bod za rozdiel.

\poznamka
Rozdiel možno zistiť aj bez určenia celkových počtov omaľovaných malých kociek v~jednotlivých
prípadoch. Pri takomto riešení hodnoťte druhú časť úlohy 1 až 5 bodmi podľa kvality
zdôvodnenia (prvý bod je rovnaký).
\endhodnotenie
}

{%%%%%   Z6-II-2
Jednociferných nepárnych čísel je päť.
Dvojciferných čísel je celkom ${99-9}=90$, z~toho nepárnych je 45.
S~týmito číslami mohol Jurko očíslovať $5+45=50$ strán a~použil by pri tom $5+2\cdot45=95$ cifier.
Ostáva ešte použiť $125-95=30$ cifier.
Tieto presne zodpovedajú prvým desiatim trojciferným nepárnym číslam od $101$ do $119$.
Jurko celkom očísloval $5+45+10=60$ strán, zošit má teda 60 listov.

\smallskip
Cifra $1$ je na mieste jednotiek celkom 12-krát (raz v~každej
desiatke: $1$, $11$, \dots, $91$, $101$, $111$).
Na mieste desiatok sa cifra $1$ vyskytuje 10-krát (päťkrát medzi číslami $11$ a~$19$ a~päťkrát medzi $111$ a~$119$).
Na mieste stoviek je cifra $1$ tiež 10-krát (trojciferných čísel bolo napísaných práve desať).
Jurko napísal celkom $12+10+10=32$ jednotiek.

\hodnotenie
Po 3~bodoch za každú časť úlohy.
(Úlohu možno riešiť aj postupným vypisovaním a~počítaním cifier.)
\endhodnotenie
 }

{%%%%%   Z6-II-3
Najskôr zistíme, ako dlho bežali Vlci.
Šiesty člen predával sviečku siedmemu v~okamihu, keď mala horieť ešte 3~minúty.
Sám ju teda niesol tiež 3 minúty
a~od piateho člena ju preberal v~okamihu, keď mala horieť ešte $3+3=6$ minút.
Piaty člen ju teda niesol 6~minút a~od štvrtého člena ju preberal v~okamihu, keď mala horieť ešte $6+6=12$ minút.
Takto môžeme postupovať až k~prvému členovi:
$$
\def\tstrut{\vrule width0pt height12pt depth5pt}
\begintable
člen č.|niesol sviečku (min)\crthick
7|0\cr 6|3\cr 5|6\cr 4|12\cr 3|24\cr 2|48\cr 1|96%
\endtable
$$
Pochod Vlkom trval $96+48+24+12+6+3=189$ minút.
(Alternatívne možno spočítať ako $96+96-3=189$.)

Líškam pochod trval $120+57=177$ minút.

Líšky teda boli v~cieli prvé, a~to o~$189-177=12$ minút skôr ako Vlci.

\hodnotenie
1~bod za postreh, že každý člen niesol sviečku tak dlho, ako dlho mala
ešte horieť, keď ju predával ďalšiemu;
2~body za vyplnenú tabuľku alebo jej obdobu;
1~bod za určenie času pochodu Vlkov;
po 1~bode za odpoveď na každú otázku zo zadania.

Za riešenie, v~ktorom je iba nezdôvodnený záver, že v~cieli boli prvé Líšky,
nedávajte žiadny bod.
\endhodnotenie
}

{%%%%%   Z7-II-1
Karol musel vyhrať toľkokrát, aby mu po odčítaní 12~bodov za 6~prehier zostalo
ešte 9~bodov.
Na výhrach teda musel získať $9+12=21$ bodov, čomu zodpovedá 7~výhier.
Peter teda sedemkrát prehral.
Zo zadania ešte vieme, že dvakrát remizoval a~šesťkrát vyhral,
má teda celkom $6\cdot3+2\cdot0-7\cdot2=4$ body.

Chlapci odohrali $6+2+7=15$ partií, vedie Karol.

\hodnotenie
4~body za správnu úvahu o~počte výhier Karola;
1~bod za zdôvodnený poznatok, že vedie Karol;
1~bod za stanovenie počtu odohraných partií.

\poznamka
Súťažiaci nemusia určovať Petrov bodový zisk. Poznatok, že vedie Karol, môžu
totiž zdôvodniť aj porovnaním siedmich Karlových výhier so šiestimi Petrovými
výhrami.
\endhodnotenie
}

{%%%%%   Z7-II-2
Označme vnútorné uhly v~trojuholníku $\alpha$, $\beta$ a~$\gamma$ (\obr).
\insp{z61ii.71}%

Súčet vnútorných uhlov v~ľubovoľnom trojuholníku je $180\st$.
Preto aj v~trojuholníku $ASC$ platí
$$
\frac\alpha2+\frac\gamma2+120\st=180\st,
$$
odkiaľ dopočítame
$$\align
\frac\alpha2+\frac\gamma2&=60\st, \\
\alpha+\gamma&=120\st.
\endalign
$$
V~trojuholníku $ABC$ platí $\alpha+\beta+\gamma=180\st$,
odkiaľ teraz vieme vyjadriť uhol $\beta$:
$$
\beta=180\st-(\alpha+\gamma)=180\st-120\st=60\st.
$$

Podobne v~trojuholníku $BSC$:
$$\align
\frac\gamma2+\frac\beta2+130\st&=180\st,\\
\frac\gamma2+\frac\beta2&=50\st, \\
\gamma+\beta&=100\st.
\endalign
$$
Keďže $\beta=60\st$, musí byť $\gamma=100\st-60\st=40\st$ a~napokon
$\alpha=120\st-40\st=80\st$ (alternatívne $\alpha=180\st-100\st=80\st$).

\hodnotenie
2~body za určenie súčtu $\alpha+\gamma=120\st$, príp. $\gamma+\beta=100\st$;
4~body za určenie veľkostí všetkých vnútorných uhlov.
\endhodnotenie
}

{%%%%%   Z7-II-3
Najskôr určíme počet všetkých párnomilných čísel:
\begin{itemize}
  \item Ak je cifra $1$ na mieste stoviek, môže byť na mieste desiatok ľubovoľná
    párna cifra ($0$, $2$, $4$, $6$, $8$) a~to isté platí pre miesto jednotiek;
    to je celkom $5\cdot5=25$ možností.
  \item Ak je cifra $1$ na mieste desiatok, môže byť na mieste stoviek ľubovoľná
    nenulová párna cifra ($2$, $4$, $6$, $8$) a~na mieste jednotiek ľubovoľná párna
    cifra; to nám dáva ďalších $4\cdot5=20$ možností.
  \item Ak je cifra $1$ na mieste jednotiek, môže byť na mieste stoviek ľubovoľná
    nenulová párna cifra a~na mieste desiatok ľubovoľná párna cifra;
    to nám dáva ďalších $4\cdot5=20$ možností.
\end{itemize}
Celkový počet párnomilných čísel je $25+20+20=65$.

\smallskip
Nepárnomilné čísla obsahujú okrem cifry $2$ iba nepárne cifry.
Môžeme uvažovať rovnako ako pri počítaní párnomilných čísel, avšak tentoraz
možno v~každom prípade napočítať $5\cdot5=25$ možností (na každom voľnom mieste
môže byť ľubovoľná nepárna cifra $1$, $3$, $5$, $7$, $9$).
Celkový počet nepárnomilných čísel je $3\cdot25=75$.

\hodnotenie
4~body za určenie počtu párnomilných čísel;
2~body za určenie počtu nepárnomilných čísel.
\endhodnotenie
}

{%%%%%   Z8-II-1
Počet predstavení, na ktorých bolo vstupné zdražené o~6~€, označíme $a$.
Filoména za ne oproti svojmu predpokladu utratila o~$6a$~€ viac.
Po zlacnení bolo vstupné nižšie ako na začiatku sezóny, a~to o~$8{,}5-6=2{,}5$~€.
Počet predstavení s~týmto vstupným označíme $b$.
Filoména za ne zaplatila o~$2{,}5b$~€ menej, ako pôvodne plánovala.
Celková vydaná suma zodpovedá presne plánu, preto musí platiť
$$
6a=2{,}5b.
$$
Neznáme $a$, $b$ sú prirodzené čísla.
Rovnicu upravíme na tvar, z~ktorého bude zrejmý pomer týchto čísel:
$$
\frac{a}{b}=\frac{2{,}5}{6}=\frac{5}{12}.
$$
Neznáma~$a$ teda musí byť násobkom piatich a~neznáma~$b$ musí byť zodpovedajúcim
násobkom dvanástich ako v~tabuľke:
$$
\begintable
$a$\|5|10|15|\dots\cr
$b$\|12|24|36|\dots
\endtable
$$
Keďže podľa zadania nesmie súčet $a+b$ presiahnuť $30$, do úvahy prichádza len
prvá možnosť. Predstavení so zmenenou cenou vstupného bolo celkom $5+12=17$.
Predstavení so vstupným v~pôvodnej výške tak bolo $30-17=13$.

\hodnotenie
4~body za poznatok, že počty predstavení $a$ a~$b$ sú v~pomere $5:12$
(z~toho 2~body udeľte podľa úplnosti komentára);
1~bod za diskusiu, že riešenie je jediné;
1~bod za správnu a~jasne formulovanú odpoveď.

\poznamka
Dá sa zvoliť aj dlhší postup, pri ktorom do rovnice $6a-2{,}5b=0$ dosadzujeme za
$a$ postupne prirodzené čísla a~vždy skúmame, či $b$ vychádza tiež prirodzené číslo.
Aj za takúto prácu možno udeliť plný počet bodov, ak je v~nej ukázané, že úloha má jediné riešenie.
\endhodnotenie
}

{%%%%%   Z8-II-2
Ak je polovica čísla deliteľná tromi, tak číslo musí byť deliteľné číslami $2$
a~$3$ súčasne. Z~podobných dôvodov musí byť deliteľné aj číslami $3$ a~$4$ a~tiež číslami $4$ a~$11$.
Najmenší spoločný násobok všetkých týchto čísel je súčin
$3\cdot4\cdot11=132$; hľadané číslo musí byť násobkom čísla $132$.
Polovica hľadaného čísla je teda násobkom čísla $66$,
zostáva už len preskúmať zvyšok po delení siedmimi:
$$
\begintable
polovica hľadaného čísla|zvyšok po delení siedmimi\crthick
66|3\cr 132|6\cr 198|2\cr 264|5%
\endtable
$$
Najmenší násobok čísla $66$, ktorý po delení siedmimi dáva zvyšok $5$, je $264$.
Hľadané číslo je teda $2\cdot264=528$.

\hodnotenie
1~bod za zjednodušenie zadania (napr. že prvá informácia znamená deliteľnosť
číslami $2$ a~$3$);
3~body za nájdenie najmenšieho spoločného násobku $132$ a~interpretáciu, že hľadané
číslo je násobkom tohto čísla, alebo jej obdobu (pojem "násobok" nemusí byť
explicitne uvedený);
2~body za nájdenie čísla $528$ (z~postupu musí byť zrejmé vylúčenie všetkých
menších čísel).
\endhodnotenie
}

{%%%%%   Z8-II-3
Keďže $XY$ je stredná priečka trojuholníka $ABC$, musí byť $Y$ stred strany~$BC$,
a~teda úsečka~$AY$ nie je iba výška, ale aj ťažnica.
Priesečník ťažníc $CX$ a~$AY$ je ťažiskom trojuholníka $ABC$, tento bod označíme $T$.

Každá ťažnica rozdelí trojuholník na dva trojuholníky s~rovnakým obsahom (takto
vzniknuté trojuholníky majú spoločnú výšku na rovnako dlhé strany):
$AY$ je ťažnica trojuholníka $ABC$, preto
$$
S_{ABY}=S_{ACY}=\frac12S_{ABC}=\frac{24}2=12\,(\Cm^2),
$$
a~podobne $XY$ je ťažnica trojuholníka $ABY$, preto
$$
S_{AXY}=S_{BXY}=\frac12S_{ABY}=\frac{12}2=6\,(\Cm^2).
$$

Trojuholník $AXY$, ktorého obsah poznáme, pozostáva z~trojuholníkov $AXT$ a~$YXT$.
Tieto dva trojuholníky majú spoločnú výšku z~bodu~$X$, ich obsahy sú
teda v~rovnakom pomere, v~akom je pomer dĺžok strán $AT$ a~$TY$,
\tj. v~pomere $2:1$ (ťažisko delí ťažnicu v~pomere $2:1$).
Obsah trojuholníka $YXT$ je teda tretinový vzhľadom na obsah trojuholníka $AXY$, preto
$$
S_{YXT}=\frac13S_{AXY}=\frac63=2\,(\Cm^2).
$$

\hodnotenie
2~body za poznatok (vrátane zdôvodnenia), že $AY$ je ťažnica v~trojuholníku
$ABC$;
po 1~bode za obsahy trojuholníkov $ABY$ a~$AXY$;
2~body za výsledný obsah trojuholníka $YXT$ (vrátane jeho odvodenia alebo
zdôvodnenia).
\endhodnotenie
}

{%%%%%   Z9-II-1
Dĺžku strany~$AB$ označme~$a$ a~veľkosť výšky kosodĺžnika $ABCD$ na túto
stranu označme~$v$; obsah kosodĺžnika je rovný~$av$.
Trojuholníky $DGF$ a~$BEF$ sú podobné podľa vety {\it uu\/}
(uhly $DFG$ a~$BFE$ sú vrcholové, uhly $FEB$ a~$FGD$ striedavé).
Ďalej označme $v_1$ veľkosť výšky trojuholníka $DGF$ na stranu~$DG$
a~$v_2$ veľkosť výšky trojuholníka $BEF$ na stranu~$BE$.

a) Bod~$F$ je v~jednej pätine uhlopriečky~$DB$.
Pomer podobnosti trojuholníkov $DGF$ a~$BEF$ je v~tomto prípade $1:4$.
Pre zodpovedajúce si výšky týchto trojuholníkov platí $v_1+v_2=v$.
Z~uvedeného vyplýva, že $v_1=\frac15v$ a~$v_2=\frac45v$.
Všimnime si, že $|DG|=|AE|$, a~teda že pre dĺžky zodpovedajúcich si strán
týchto trojuholníkov platí $|DG|+|BE|=a$.
Preto $|DG|=\frac15a$ a~$|BE|=\frac45a$ a~obsahy vyfarbených trojuholníkov sú
$$\align
S_{DGF}&=\frac12\left(\frac15a\cdot\frac15v\right)=\frac{1}{50}av,\\
S_{BEF}&=\frac12\left(\frac45a\cdot\frac45v\right)=\frac{16}{50}av.
\endalign
$$
Vyfarbená časť kosodĺžnika má v~tomto prípade obsah
$$
S_{DGF}+S_{BEF}=\frac{17}{50}av.
$$

b) Bod~$F$ je v~dvoch pätinách uhlopriečky~$DB$.
Pomer podobnosti trojuholníkov $DGF$ a~$BEF$ je v~tomto prípade $2:3$.
Podobne ako v~predchádzajúcom odseku odvodíme, že
$S_{DGF}=\frac12\left(\frac25a\cdot\frac25v\right)=\frac{4}{50}av$
a~$S_{BEF}=\frac12\left(\frac35a\cdot\frac35v\right)=\frac{9}{50}av$.
Vyfarbená časť kosodĺžnika má v~tomto prípade obsah
$$
S_{DGF}+S_{BEF}=\frac{13}{50}av.
$$

Zo zadania vieme, že rozdiel $\frac{17}{50}av-\frac{13}{50}av=\frac2{25}av$ je
práve $1\cm^2$.
Z~toho vyplýva, že $av=\frac{25}2\cm^2$.
Obsah kosodĺžnik $ABCD$ je $12{,}5\cm^2$.

\ineriesenie
Úsečky $AB$ a~$BC$ rozdelíme na pätiny.
Rovnobežky s~týmito úsečkami vedené vzniknutými bodmi rozdelia kosodĺžnik $ABCD$
na 25~zhodných kosodĺžničkov.
Ak vyznačíme ešte aj rovnobežky s~uhlopriečkou~$DB$, bude kosodĺžnik $ABCD$
rozdelený na 50~zhodných trojuholníčkov ako na \obr{}.
\insp{z61ii.91}%

Keď je bod~$F$ v~jednej pätine uhlopriečky~$DB$, tak
vyfarbená časť kosodĺžnika pozostáva zo 17 trojuholníčkov.
Keď je bod~$F$ v~dvoch pätinách uhlopriečky~$DB$,
vyfarbená časť kosodĺžnika pozostáva z~13 trojuholníčkov (\obr).
% \figure101
\insp{z61ii.93}%

Rozdiel $1\cm^2$ zodpovedá obsahu 4 trojuholníčkov.
Obsah kosodĺžnika $ABCD$ je teda rovný
$50\cdot\frac14=\frac{25}2=12{,}5\,(\Cm^2)$.

\hodnotenie
1~bod za nájdenie vlastnosti, pomocou ktorej možno porovnať obsahy vyfarbených
trojuholníkov (napr. podobnosť, rozdelenie na trojuholníčky a pod.);
3~body za vyjadrenie vzťahu, z~ktorého možno presne určiť rozdiel obsahov
v~daných situáciách (napr. rovnice vyplývajúce z~podobnosti trojuholníkov,
zistenie, o~koľko trojuholníčkov sa vyfarbené časti líšia a pod.);
2~body za dopočítanie obsahu kosodĺžnika $ABCD$.
\endhodnotenie
}

{%%%%%   Z9-II-2
Označme $x$ rozdiel hmotností dvoch susedných trpaslíkov.
Prvý trpaslík váži 5\,kg, druhý $5-x$, tretí $5-2x$, \dots,  $n$-tý trpaslík
váži $5-(n-1)\cdot x$~(kg).
Súčet hmotností 76. až 80. trpaslíka je
$$
(5-75x)+(5-76x)+(5-77x)+(5-78x)+(5-79x)
%=5\cdot 5-(75+76+77+78+79)\cdot x
=25-385x.
$$
Súčet hmotností 96. až 101. trpaslíka je
$$
(5-95x)+(5-96x)+(5-97x)+(5-98x)+(5-99x)+(5-100x)
%=6\cdot 5-(95+96+97+98+99+100)\cdot x
=30-585x.
$$
Dostávame teda rovnicu, z~ktorej vypočítame $x$:
$$\align
25-385x&=30-585x, \\
200x&=5, \\
x&=0{,}025\text{\ (kg)}.
\endalign
$$
Najľahší trpaslík váži $5-100\cdot 0{,}025=2{,}5$~(kg).


\ineriesenie
Označme $x$ rozdiel hmotností dvoch susedných trpaslíkov.
Prvý trpaslík váži 5\,kg, druhý $5-x$, tretí $5-2x$, \dots,  101.~trpaslík
váži $5-100x$~(kg).

Rozdiel hmotností 76. a~96. trpaslíka je $20x$.
Rovnaký rozdiel je aj medzi 77. a~97., 78. a~98., 79. a~99., 80. a~100.
trpaslíkom.
Celková hmotnosť 76. až 80. trpaslíka je teda o~$100x$ väčšia ako celková
hmotnosť 96. až 100. trpaslíka.
Aby 76. až 80. trpaslík vážili dokopy presne toľko ako 96. až 101.
trpaslík, musí byť hmotnosť 101. trpaslíka rovná $100x$.

Získavame teda rovnicu, z~ktorej jednoducho vypočítame $x$:
$$\align
5-100x&=100x, \\
200x&=5, \\
x&=0{,}025\text{\ (kg)}.
\endalign
$$
Najľahší trpaslík váži $100\cdot 0{,}025=2{,}5$~(kg).

\hodnotenie
4~body za zostavenie rovnice umožňujúcej výpočet rozdielu hmotností dvoch
susedných trpaslíkov;
2~body za úpravy rovnice a~výsledok úlohy.

\poznamka
Ani v~jednom uvedenom riešení súťažiaci nemusí nutne vypočítať, že
$x=0{,}025$\,kg.
Stačí mu napr. zistenie, že $100x=2{,}5$\,kg.
\endhodnotenie
}

{%%%%%   Z9-II-3
Postupujeme úvahou "odzadu".

$\frac1{11}$ zvyšku cesty po druhom dni je rovná 4\,km, teda celý zvyšok bol
$11\cdot 4=44$~(km).
Keby druhý deň prešli (ako plánovali) o~2\,km viac, bol by zvyšok po druhom
dni o~2\,km menší, teda len 42\,km, a~tvoril by polovicu toho, čo po prvom dni ostávalo do cieľa.
Po prvom dni ostávalo do cieľa 84\,km.

Keby prvý deň prešli podľa plánu celú $\frac13$ trasy, teda o~4\,km viac,
bol by zvyšok o~4\,km menší, \tj. len 80\,km, a~predstavoval by $\frac23$ celej
trasy.
$\frac13$ trasy bola teda 40\,km a~celá 120\,km.

Predošlé zistenia zhrnieme, pričom spočítame prejdené vzdialenosti v~jednotlivých
dňoch.
Prvý deň chceli prejsť 40\,km, ale prešli len 36\,km; ostávalo im do cieľa
84\,km.
Druhý deň chceli prejsť polovicu zvyšku, \tj. 42\,km, ale prešli len 40\,km;
ostávalo im 44\,km.
Tretí deň prešli $\frac{10}{11}$ zvyšku, \tj. 40\,km, a~ešte 4\,km; boli teda
v~cieli.

V~jednotlivých dňoch prešli postupne 36\,km, 40\,km a~44\,km, dokopy 120\,km.

\ineriesenie
Postupujeme pomocou algebry "odpredu"; dĺžku celej trasy v~km označíme~$x$.

{\openup\jot
Prvý deň prešli $\frac13x-4$, zvýšilo im $\frac23x+4$.

Druhý deň prešli $\frac12\cdot(\frac23x+4)-2=\frac13x$, zvýšilo im
$\frac13x+4$.

Tretí deň prešli
$\frac{10}{11}\cdot(\frac13x+4)+4=\frac{10}{33}x+\frac{40}{11}+4$.

Dĺžka celej trasy teda bola
$$
x=\frac13x-4+\frac13x+\frac{10}{33}x+\frac{40}{11}+4.
$$


Po úpravách dostávame
$$\align
\frac1{33}x&=\frac{40}{11}, \\
x&=120.
\endalign
$$}

Po dosadení do vyššie uvedených výrazov zisťujeme, že v~jednotlivých dňoch
prešli postupne 36\,km, 40\,km a~44\,km, dokopy teda 120\,km.

\hodnotenie
Po 1~bode za dĺžku trasy prvý, druhý a~tretí deň a~dĺžku celej trasy;
zvyšné 2~body podľa úplnosti komentára.
\endhodnotenie
}

{%%%%%   Z9-II-4
Označme $n$ počet všetkých ľudí, ktorí navštívili výstavu, ďalej
$p$ počet návštevníkov, ktorým sa páčila prvá časť výstavy,
a~$d$ počet návštevníkov, ktorým sa páčila druhá časť výstavy.
Hľadáme nejaký vzťah medzi $n$ a~$p$, z~ktorého už ľahko odvodíme odpoveď
na otázku.

Vyjadríme počet návštevníkov, ktorým sa páčili obe časti:
z~prvej podmienky to je $0{,}96p$, z~druhej podmienky $0{,}6d$.
Samozrejme platí
$$\align
0{,}96p&=0{,}6d, \\
96p&=60d, \\
8p&=5d, \\
1{,}6p&=d.
\endalign
$$
Odtiaľ môžeme vyjadriť počet ľudí v~jednotlivých skupinách pomocou~$p$.
Počet ľudí, ktorým sa páčila
\begin{itemize}
  \item prvá časť, ale nie druhá časť, je $p-0{,}96p=0{,}04p$,
  \item druhá časť, ale nie prvá časť, je $d-0{,}6d=0{,}4d=0{,}4\cdot1{,}6p=0{,}64p$,
  \item prvá aj druhá časť, je samozrejme $0{,}96p$.
\end{itemize}
\noindent
Sčítaním zistíme, koľkým ľuďom sa páčila aspoň jedna časť výstavy:
$$
0{,}04p+0{,}64p+0{,}96p=1{,}64p.
$$

Podľa tretej podmienky v~zadaní vieme, že $0{,}59n$ návštevníkom sa nepáčila ani
jedna časť výstavy; teda $0{,}41n$ návštevníkom sa aspoň jedna časť výstavy
páčila.
Tento počet zároveň podľa predošlého zodpovedá $1{,}64p$.
Zostavíme rovnicu, ktorú ďalej upravíme:
$$
\align
0{,}41n&=1{,}64p,\\
41n&=164p,\\
n&=4p,\\
0{,}25n&=p.
\endalign
$$
Odtiaľ vyplýva, že
25\,\% všetkých návštevníkov uviedlo, že sa im páčila prvá časť výstavy.

%\poznamka
Vzťahy medzi skúmanými počtami návštevníkov môžeme znázorniť napr.
diagramom na \obr.
\insp{z61ii.94}%

\hodnotenie
2~body za vyjadrenie $d=1{,}6p$ či analogický vzťah;
2~body za vyjadrenie počtu návštevníkov, ktorým sa páčila aspoň jedna časť,
pomocou počtu návštevníkov, ktorým sa páčila prvá časť (alebo za analogický
poznatok);
2~body za výsledných 25\,\%.
\endhodnotenie
}

{%%%%%   Z9-III-1
Vypočítame dĺžku~$u$ uhlopriečky sivého štvorca, pričom vyjdeme zo vzťahu pre
výpočet obsahu štvorca $S=\frac12u^2$:
$$
u=\sqrt{2S}=\sqrt{2\cdot72}=\sqrt{144}=12\,(\Cm).
$$
Jeden nepomenovaný vrchol sivého štvorca označíme~$U$ (\obr).
\insp{z61iii.60}%

Z~bodu~$U$ spustíme kolmicu na stranu~$AB$ a~potom na stranu~$DA$.
Ich päty označíme $P$ a~$Q$.
Uhol $XAB$ je vnútorným uhlom pravouhlého rovnoramenného
trojuholníka $ABX$ a~má teda veľkosť $45\st$.
Odtiaľ vyplýva, že pravouholník $APUQ$ je štvorec.
Na \obrr1{} ukazujeme tri ďalšie analogicky zostrojené štvorce.
Dĺžku strany týchto štvorcov označíme~$s$.
Z~\obrr1{} je zrejmé, že  $|AB|=2s+12$\,(cm) a~$|BC|=2s$.
Tieto výrazy dosadíme do pomeru uvedeného v~zadaní.
Úpravami vzniknutej rovnice získame $2s$:
$$\align
\frac{|AB|}{|BC|}&=\frac75,\\
\frac{2s+12}{2s}&=\frac75,\\
10s+60&=14s,\\
2s&=30\,(\Cm).
\endalign
$$
Dĺžka strany~$AB$ je $30+12=42$\,(cm) a~dĺžka strany~$BC$ je $30\cm$.

\ineriesenie
V~zadanom obrázku posunieme trojuholník $CDY$ tak, aby sa bod~$Y$ zobrazil do
bodu~$X$ (\obr).
Posunieme ho teda o~dĺžku úsečky~$XY$ a~rovnako ako v~predchádzajúcom riešení určíme, že táto dĺžka je $12\cm$.
\insp{z61iii.62}%

Vzhľadom na to, že trojuholníky $ABX$ a~$C'D'Y'$  sú pravouhlé
rovnoramenné, vzniknutý pravouholník $ABC'D'$ musí byť štvorec.
Úsečka~$AB$ predstavuje podľa zadania 7~dielov a~úsečka~$BC$ 5~dielov.
Úsečka~$BC'$ je o~$12\cm$ dlhšia ako $BC$ a~predstavuje tiež 7~dielov.
Dva diely sú teda $12\cm$, jeden diel je potom $6\cm$.
Dĺžka strany~$AB$ je $7\cdot 6=42$\,(cm) a~dĺžka strany~$BC$ je $5\cdot6=30$\,(cm).

\poznamka
Dĺžku uhlopriečky sivého štvorca možno určiť aj nasledovne.
Priesečník uhlopriečok %%šedého čtverce označíme $S$~a~tento bod
zobrazíme v~osovej súmernosti podľa jednotlivých strán štvorca (\obr).
Získané štyri body tvoria vrcholy štvorca, ktorý má dvojnásobný obsah ako
sivý štvorec, \tj. $2\cdot 72=144\,(\Cm^2)$.
Dĺžka strany vzniknutého štvorca, a~teda aj dĺžka úsečky~$XY$, je rovná
$\sqrt{144}=12$\,(cm).
\insp{z61iii.61}%

\hodnotenie
1~bod za výpočet uhlopriečky sivého štvorca;
3~body za poznatok a~zdôvodnenie, že rozdiel dĺžok strán $AB$ a~$BC$ je rovný dĺžke tejto uhlopriečky;
2~body za výpočet strán $AB$ a~$BC$.
\endhodnotenie
}

{%%%%%   Z9-III-2
Najmenšie z~desiatich napísaných čísel označme~$p$.
Čísla na kartičkách potom boli:
$$
p,\ p+1,\ p+2,\ p+3,\ \dots,\ p+9.
$$
Súčet čísel na všetkých desiatich kartičkách bol $10p+45$.

Najskôr predpokladajme, že sa stratila kartička s~číslom~$p$.
Potom by platila nasledujúca rovnosť:
$$\align
(10p+45)-p&=2\,012,\\
9p&=1\,967.
\endalign
$$
Z~predchádzajúceho riadku je zrejmé, že $p$ by v~tomto prípade nebolo prirodzené
číslo. Predpoklad, že sa stratila kartička s~číslom~$p$, preto zavrhujeme.

Teraz predpokladajme, že sa stratila kartička s~číslom $p+1$.
Podobne zavrhneme aj túto možnosť:
$$\aligned
(10p+45)-(p+1)&=2\,012,\\
9p&=1\,968.
\endaligned
$$
Rovnako odmietneme aj predpoklad, že sa stratila kartička s~číslom $p+2$:
$$\align
(10p+45)-(p+2)&=2\,012,\\
9p&=1\,969.
\endalign
$$
Stratiť sa nemohla ani kartička s~číslom $p+3$:
$$\align
(10p+45)-(p+3)&=2\,012,\\
9p&=1\,970.
\endalign
$$
Až za predpokladu, že sa stratila kartička s~číslom $p+4$, dôjdeme
k~celočíselnej hodnote~$p$:
$$\align
(10p+45)-(p+4)&=2\,012,\\
9p&=1\,971,\\
p&=219.
\endalign
$$

Ostáva nám diskutovať ďalších päť možností, teda straty kartičiek s~číslami
$p+5$ až $p+9$.
V~predchádzajúcich diskusiách dostávame vždy rovnicu, ktorá má na ľavej strane iba číslo~$9p$.
Čísla na pravej strane týchto rovníc sa postupne zväčšujú~o~$1$.
Práve sme dostali na pravej strane číslo deliteľné deviatimi, a~preto sa
v~nasledujúcich piatich diskusiách číslo deliteľné deviatimi vyskytovať nemôže.
Úloha má tak jediné riešenie: stratila sa kartička s~číslom
$p+4=219+4=223$.

\ineriesenie
Najmenšie z~desiatich napísaných čísel označme~$p$.
Čísla na kartičkách potom boli:
$$
p,\ p+1,\ p+2,\ p+3,\ \dots,\ p+9.
$$
Číslo na stratenej kartičke označme $p+c$, pričom $c$ predstavuje jednociferné
prirodzené číslo alebo nulu.
Súčet čísel na všetkých desiatich kartičkách bol $10p+45$.
Súčet čísel na deviatich kartičkách, ktoré zvýšili po strate, bol $10p+45-(p+c)$, po
úprave $9p+45-c$.
Dostávame tak rovnicu, ktorú upravíme nasledujúcim spôsobom:
$$\align
9p+45-c&=2\,012,\\
9p&=1\,967+c.
\endalign
$$
Súčet na pravej strane tejto rovnice musí byť deliteľný deviatimi.
Pri delení $1\,967:9$ dostaneme výsledok $218$ a~zvyšok $5$.
Platí teda $1\,967=9\cdot 218+5$.
Z~tohto rozkladu je zrejmé, že súčet $1967+c$ je deliteľný deviatimi
jedine pre $c=4$.
Neznáma $p$ je potom rovná $219$ a~číslo na stratenej kartičke bolo
$p+c=219+4=223$.

\hodnotenie
1~bod za vyjadrenie súčtu čísel na desiatich kartičkách, \tj. napr. $10p+45$;
2~body za výsledok;
3~body za popis postupu.

Ak súťažiaci postupuje ako my v~prvom uvedenom riešení a~po nájdení
celočíselného $p$ prestane bez akéhokoľvek zdôvodnenia preverovať ďalšie
možnosti, strhnite 1~bod.

\poznamka
Súťažiaci môžu úlohu riešiť aj "odhadom" pomocou aritmetického priemeru
čísel na deviatich zvyšných kartičkách: $2\,012:9=223$, zvyšok $5$.
Skusmo vypíšu napr.~deväť po sebe idúcich prirodzených čísel takých, aby
$223$ bolo uprostred: $219+220+\cdots+227=2\,007$.
Následne uvažujú, ako možno ľavú stranu "zväčšiť" o~5\dots{}
Ak v~takejto práci nie je zdôvodnené, že úloha má skutočne iba jedno riešenie,
dajte za ňu maximálne 5~bodov.
\endhodnotenie
}

{%%%%%   Z9-III-3
Súčet všetkých vpísaných čísel je $1+2+3+4+5+6+7+8+9=45$.
V~obrázku sú práve tri štvorpolíčkové trojuholníky a~súčet štvoríc čísel
vpísaných do týchto trojuholníkov je dohromady $3\cdot23=69$.
V~tomto súčte sú však čísla na sivých políčkach na \obr{} započítané dvakrát
(každé patrí do dvoch štvorpolíčkových trojuholníkov), ostatné čísla jedenkrát.
Súčet čísel na troch sivých políčkach preto musí byť $69-45=24$.
\insp{z61iii.70}%

Keďže najväčší možný súčet troch vpísaných čísel je práve $9+8+7=24$,
v~sivých políčkach musia byť čísla $7$, $8$ a~$9$.
Zo zadania vieme, že $7$ má byť v~spodnom riadku,
v~krajných políčkach druhého riadku potom musia byť $8$ a~$9$.
\insp{z61iii.71}%

Pre číslo $2$ tak zostáva jediné voľné miesto (\obr).
V~hornom štvorpolíčkovom trojuholníku teraz chýba jediné číslo, ktoré tým pádom
vieme doplniť: $23-8-9-2=4$.
\insp{z61iii.72}%

Číslo $6$ musí byť na niektorom sivom políčku na \obr, patrí
teda do ľavého štvorpolíčkového trojuholníka.
Keby tento trojuholník obsahoval číslo~$8$, tak posledné voľné políčko
by obsahovalo číslo $23-8-7-6=2$, čo nie je možné (číslo~$2$ je už umiestnené v~druhom riadku).
Ľavý štvorpolíčkový trojuholník preto má vo svojom hornom políčku číslo~$9$
a~v~poslednom neobsadenom políčku je $23-9-7-6=1$.
V~sivých políčkach na \obrr1{} sú preto čísla $1$ a~$6$, ktoré môžeme umiestniť dvoma spôsobmi.

Na zatiaľ neobsadených miestach v~pravom štvorpolíčkovom trojuholníku môžu byť
jedine čísla $3$ a~$5$;
súčet v~tomto trojuholníku vychádza naozaj $8+7+3+5=23$.
Čísla $3$ a~$5$ môžeme doplniť opäť dvoma spôsobmi.
Úloha má teda celkom $2\cdot2=4$ riešenia (\obr).
\insp{z61iii.73}%

\hodnotenie
3~body za vysvetlenie, kde sa nachádzajú čísla $7$, $8$ a~$9$;
1~bod za doplnenie čísel $2$ a~$4$;
1~bod za doplnenie čísel $6$ a~$1$;
1~bod za správny počet riešení.

Ak riešiteľ nájde náhodne (bez bodovateľného vysvetlenia) jedno správne
riešenie, dajte 1~bod; za dve riešenia 2~body; za tri a~štyri riešenia 3~body.
\endhodnotenie
}

{%%%%%   Z9-III-4
Prekliate číslo nazveme $\overline{ABC}$.
Vojto však namiesto neho pripočítal dvojciferné čísla $\overline{AB}$,
$\overline{AC}$, $\overline{BC}$, všeobecne ich budeme označovať ako ??.
Ďalej označíme $****$ súčet čísel, ktoré Vojto dokázal sčítať bez
preklepu.
Platia schematicky vyjadrené sčítania:
$$
\alggg{*&*&*&*\\ &&?&?}{2&2&2&4}\hskip1cm
\alggg{*&*&*&*\\ &&?&?}{2&1&9&8}\hskip1cm
\alggg{*&*&*&*\\ &&?&?}{2&2&0&4}
$$
Prvý a~tretí uvedený súčet má rovnakú cifru na mieste jednotiek, v~týchto
prípadoch teda na miesto označené ?? patria čísla $\overline{AC}$ a~$\overline{BC}$.
Zvyšné číslo $\overline{AB}$ tak patrí k~súčtu $2\,198$.
Súčet $2\,224$ je o~$26$ väčší ako $2\,198$, a~preto nemohol vzniknúť sčítaním
čísla $****$ a~čísla $\overline{AC}$, ktoré sa od čísla $\overline{AB}$ líši
len cifrou na mieste jednotiek.
Čísla $\overline{AB}$, $\overline{AC}$, $\overline{BC}$ tak máme jednoznačne
priradené k~súčtom:
$$
\alggg{*&*&*&*\\ &&B&C}{2&2&2&4}\hskip1cm
\alggg{*&*&*&*\\ &&A&B}{2&1&9&8}\hskip1cm
\alggg{*&*&*&*\\ &&A&C}{2&2&0&4}
$$
Podľa prvého a~tretieho súčtu platí: $B=A+2$.
Podľa druhého a~tretieho súčtu platí: $C=B+6$, po dosadení predošlého
vzťahu dostaneme: $C=A+2+6=A+8$.
Písmeno $A$ je podľa zadania nenulové jednociferné číslo, a~aby aj $C$ vychádzalo
jednociferné, môže byť $A$ jedine $1$.
Potom $B=3$ a~$C=9$.
Prekliate číslo bolo teda $139$.
Súčet $****$ vypočítame napríklad takto: $2\,224-39=2\,185$.
Správne mal Vojtovi vyjsť súčet $2\,185+139=2\,324$.

\poznamka
Pre vyriešenie úlohy nie je nutné stanovovať cifry $B$ a~$C$.
K~záveru môžeme dôjsť bez znalosti týchto cifier, teda bez znalosti
prekliateho čísla, a~síce takto: $2\,224+\overline{A00}= 2\,224+100=2\,324$.

\hodnotenie
2~body za priradenie pozícií vynechaných cifier k~jednotlivým súčtom;
3~body za výpočet vynechaných cifier (v~určitom type postupu môže stačiť
len cifra $A$, ako je naznačené v~poznámke);
1~bod za správny výsledný súčet.
\endhodnotenie
}

