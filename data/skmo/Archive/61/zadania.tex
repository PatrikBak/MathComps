{%%%%%   A-I-1
Označme $n$ súčet všetkých desaťciferných čísel, ktoré majú vo svojom
dekadickom zápise
každú z~cifier $0,1,\dots,9$. Zistite zvyšok po delení čísla~$n$
sedemdesiatimi siedmimi.}
\podpis{Pavel Novotný}

{%%%%%   A-I-2
Na stretnutí bolo niekoľko ľudí. Každí dvaja, ktorí sa nepoznali, mali medzi
ostatnými prítomnými práve dvoch spoločných známych. Účastníci $A$ a~$B$ sa
poznali, ale nemali ani jedného spoločného známeho. Dokážte, že $A$ aj $B$
mali medzi prítomnými rovnaký počet známych.
Ukážte tiež, že na stretnutí mohlo byť práve šesť osôb.}
\podpis{Vojtech Bálint}

{%%%%%   A-I-3
Označme $S$ stred kružnice vpísanej, $T$ ťažisko a~$V$ priesečník výšok
daného rovnoramenného trojuholníka, ktorý nie je rovnostranný.
\ite a) Dokážte, že bod~$S$ je vnútorným bodom úsečky~$TV$.
\ite b) Určte pomer dĺžok strán daného trojuholníka, ak je bod~$S$
stredom úsečky~$TV$.\endgraf}
\podpis{Jaromír Šimša}

{%%%%%   A-I-4
Nech $p$, $q$ sú dve rôzne prvočísla, $m$, $n$ prirodzené čísla a~súčet
$$
\frac{mp-1}{q}+\frac{nq-1}{p}
$$
je celé číslo. Dokážte, že platí nerovnosť
$$
\frac{m}{q}+\frac{n}{p}>1.
$$}
\podpis{Jaromír Šimša}

{%%%%%   A-I-5
Dané sú dve zhodné kružnice $k_1$, $k_2$ s~polomerom rovným vzdialenosti ich stredov.
Ich priesečníky označme $A$ a~$B$.
Na kružnici~$k_2$ zvoľme bod~$C$ tak, že úsečka~$BC$ pretne kružnicu~$k_1$
v~bode rôznom od~$B$, ktorý označíme~$L$. Priamka~$AC$ pretne kružnicu~$k_1$
v~bode rôznom od~$A$, ktorý označíme~$K$. Dokážte, že priamka, na ktorej leží
ťažnica z~vrcholu~$C$ trojuholníka $KLC$, prechádza pevným bodom nezávislým od
polohy bodu~$C$.}
\podpis{Tomáš Jurík}

{%%%%%   A-I-6
Nájdite najväčšie reálne číslo~$k$ také, že nerovnosť
$$
{2(a^2+kab+b^2)\over (k+2)(a+b)}\ge \sqrt{ab}
$$
platí pre všetky dvojice kladných reálnych čísel $a$, $b$.}
\podpis{Ján Mazák}

{%%%%%   B-I-1
Medzi všetkými desaťcifernými číslami deliteľnými jedenástimi, v~ktorých sa
žiadna cifra neopakuje, nájdite najmenšie a~najväčšie.}
\podpis{Jaroslav Zhouf}

{%%%%%   B-I-2
Daný je pravouhlý trojuholník $ABC$ s~pravým uhlom pri
vrchole~$C$, ktorého obsah označme~$P$. Nech $F$ je päta výšky z~vrcholu~$C$
na preponu~$AB$. Na kolmiciach na priamku~$AB$, ktoré prechádzajú vrcholmi $A$ a~$B$,
v~polrovine opačnej k~polrovine $ABC$ uvažujme postupne
body $D$ a~$E$, pre ktoré platí $|AF|=|AD|$ a~$|BF|=|BE|$. Obsah
trojuholníka $DEF$ označme~$Q$. Dokážte, že platí $P\ge Q$, a~zistite, kedy
nastáva rovnosť.}
\podpis{Jaroslav Švrček}

{%%%%%   B-I-3
Nájdite všetky dvojice reálnych čísel $x$, $y$, ktoré
vyhovujú sústave rovníc
$$
\align
x\cdot{\lfloor{y}\rfloor}&=7,\\
y\cdot{\lfloor{x}\rfloor}&=8.
\endalign
$$
(Zápis $\lfloor{a}\rfloor$ označuje {\it dolnú celú časť\/} čísla~$a$, t. j. najväčšie
celé číslo, ktoré neprevyšuje~$a$.)}
\podpis{Pavel Novotný}

{%%%%%   B-I-4
Dané sú dve rôznobežky $a$, $c$ prechádzajúce bodom~$P$ a~bod~$B$, ktorý na nich neleží. Zostrojte pravouholník $ABCD$ s~vrcholmi
$A$, $C$ a~$D$ postupne na priamkach $a$, $c$ a~$PB$.}
\podpis{Jaromír Šimša}

{%%%%%   B-I-5
V~istom meste majú vybudovanú sieť na šírenie klebiet, v~ktorej si
každý klebetník vymieňa informácie s~tromi klebetnicami a~každá
klebetnica si vymieňa informácie s~tromi klebetníkmi.
Inak sa klebety nešíria.
\ite a) Dokážte, že klebetníkov a~klebetníc je rovnako veľa.
\ite b) Predpokladajme, že sieť na šírenie klebiet je súvislá (klebety od
ľubovoľného klebetníka a~ľubovoľnej klebetnice sa môžu dostať ku všetkým
ostatným). Dokážte, že aj keď sa jeden klebetník z~mesta odsťahuje, zostane sieť
súvislá.\endgraf}
\podpis{Ján Mazák}

{%%%%%   B-I-6
Anna a~Boris hrajú kartovú hru. Každý z~nich má päť
kariet s~hodnotami $1$ až $5$ (z~každej jednu). V~každom z~piatich kôl obaja vyložia
jednu kartu a~kto má vyššie číslo, získa bod. V~prípade kariet s~rovnakými číslami
nezíska bod nikto. Použité karty sa do hry nevracajú.
Ten, kto získa na konci viac bodov, vyhral.
Koľko percent zo všetkých možných priebehov takej hry skončí výhrou Anny?}
\podpis{Tomáš Jurík}

{%%%%%   C-I-1
Nájdite všetky trojčleny $p(x)=ax^2+bx+c$, ktoré dávajú po delení
dvojčlenom $x+1$ zvyšok~$2$ a~po delení dvojčlenom $x+2$ zvyšok~$1$, pričom $p(1)=61$.}
\podpis{Jaromír Šimša}

{%%%%%   C-I-2
Dĺžky strán trojuholníka sú v~metroch vyjadrené celými číslami. Určte ich,
ak má trojuholník obvod 72\,m a ak je najdlhšia strana trojuholníka rozdelená
bodom dotyku vpísanej kružnice v~pomere $3:4$.}
\podpis{Pavel Leischner}

{%%%%%   C-I-3
Nájdite všetky trojice prirodzených čísel $a$, $b$, $c$, pre ktoré platí
množinová rovnosť
$$
\bigl\{(a,b),(a,c),(b,c),[a,b],[a,c],[b,c]\bigr\}=
\{2,3,5,60,90,180\},
$$
pričom $(x,y)$ a~$[x,y]$ označuje postupne najväčší spoločný deliteľ
a~najmenší spoločný násobok čísel $x$ a~$y$.}
\podpis{Tomáš Jurík}

{%%%%%   C-I-4
Reálne čísla $a$, $b$, $c$, $d$ vyhovujú rovnici $ab+bc+cd+da=16$.
\ite a) Dokážte, že medzi číslami $a$, $b$, $c$, $d$ sa nájdu dve so súčtom
najviac~$4$.
\ite b) Akú najmenšiu hodnotu môže mať súčet $a^2+b^2+c^2+d^2$?}
\podpis{Ján Mazák}

{%%%%%   C-I-5
Daný je rovnoramenný trojuholník so základňou dĺžky~$a$
a~ramenami dĺžky~$b$. Pomocou nich vyjadrite polomer~$R$
kružnice opísanej a~polomer~$r$ kružnice vpísanej tomuto trojuholníku. Potom ukážte,
že platí $R\ge 2r$, a~zistite, kedy nastane rovnosť.}
\podpis{Leo Boček}

{%%%%%   C-I-6
Na hracej ploche $n\times n$ tvorenej bielymi štvorcovými políčkami sa Monika
a~Tamara striedajú v~ťahoch jednou figúrkou pri nasledujúcej hre.
Najskôr Monika položí figúrku na ľubovoľné políčko a~toto políčko
zafarbí namodro. Ďalej vždy hráčka, ktorá je na ťahu, urobí
s~figúrkou {\it skok\/} na políčko, ktoré je doposiaľ biele, a~toto políčko zafarbí
namodro. Pritom pod {\it skokom\/} rozumieme bežný ťah šachovým jazdcom, \tj.
presun figúrky o~dve políčka zvislo alebo vodorovne a~súčasne o~jedno
políčko v~druhom smere. Hráčka, ktorá je na rade a~už nemôže urobiť ťah,
prehráva. Postupne pre $n=4,5,6$ rozhodnite,
ktorá z~hráčok môže hrať tak,
že vyhrá nezávisle na ťahoch druhej hráčky.}
\podpis{Pavel Calábek}

{%%%%%   A-S-1
V~obore reálnych čísel vyriešte sústavu rovníc
$$
\align
y + 3x &= 4x^3,\\
x + 3y &= 4y^3.
\endalign
$$
}
\podpis{Pavel Calábek}

{%%%%%   A-S-2
V~rovine uvažujme lichobežník $ABCD$ so základňami $AB$ a~$CD$ a~označme $M$
stred jeho uhlopriečky~$AC$. Dokážte, že ak majú
trojuholníky $ABM$ a~$ACD$ rovnaké obsahy, tak sú priamky $DM$ a~$BC$ rovnobežné.}
\podpis{Jaroslav Švrček}

{%%%%%   A-S-3
Nájdite všetky prirodzené čísla~$n$, pre ktoré je súčin $(2^n+1)(3^n+2)$ deliteľný číslom~$5^n$.}
\podpis{Ján Mazák}

{%%%%%   A-II-1
Označme $S_n$ súčet všetkých $n$-ciferných čísel,
ktorých dekadický zápis obsahuje iba cifry $1$, $2$, $3$, každú aspoň raz.
Nájdite všetky celé čísla $n\ge3$, pre ktoré je číslo~$S_n$ deliteľné siedmimi.}
\podpis{Pavel Novotný}

{%%%%%   A-II-2
Dané je celé číslo~$a$ väčšie ako $1$. Nájdite aritmetickú
postupnosť s~prvým členom~$a$, ktorá obsahuje práve dve z~čísel
$a^2$, $a^3$, $a^4$, $a^5$ a~má čo najväčšiu diferenciu.
(Nepredpokladáme, že diferencia je nutne celočíselná.)}
\podpis{Jaromír Šimša}

{%%%%%   A-II-3
Do kružnice je vpísaný šesťuholník $ABCDEF$, v~ktorom platí $AB\perp BD$, $|BC|=|EF|$.
Predpokladajme, že priamky $BC$, $EF$
pretínajú polpriamku~$AD$ postupne v~bodoch $P$, $Q$. Označme $S$ stred
uhlopriečky $AD$ a~$K$, $L$ stredy kružníc vpísaných trojuholníkom $BPS$, $EQS$.
Dokážte, že trojuholník $KLD$ je pravouhlý.}
\podpis{Tomáš Jurík}

{%%%%%   A-II-4
Predpokladajme, že pre kladné reálne čísla $a$, $b$, $c$, $d$ platí
$$
ab+cd=ac+bd=4\qquad\text{a}\qquad ad+bc=5.
$$
Nájdite najmenšiu možnú hodnotu súčtu $a+b+c+d$ a~zistite,
ktoré vyhovujúce štvorice $a$, $b$, $c$, $d$ ju dosahujú.}
\podpis{Jaromír Šimša}

{%%%%%   A-III-1
Nájdite všetky celé čísla~$n$, pre ktoré je $n^4-3n^2+9$ prvočíslo.}
\podpis{Aleš Kobza}

{%%%%%   A-III-2
Zistite, aký je najväčší možný obsah trojuholníka $ABC$, ktorého ťažnice majú dĺžky spĺňajúce nerovnosti $t_a\le2$, $t_b\le3$, $t_c\le4$. }
\podpis{Pavel Novotný}

{%%%%%   A-III-3
Dokážte, že medzi ľubovoľnými 101 reálnymi číslami existujú dve čísla $u$ a~$v$, pre ktoré platí
$$
100\,|u-v|\cdot|1-uv|\le(1+u^2)(1+v^2).
$$}
\podpis{Pavel Calábek}

{%%%%%   A-III-4
Vnútri rovnobežníka $ABCD$ je daný bod~$X$. Zostrojte priamku, ktorá prechádza bodom~$X$ a~rozdeľuje daný rovnobežník na dve časti, ktorých obsahy sa
navzájom líšia čo najviac.}
\podpis{Vojtech Bálint}

{%%%%%   A-III-5
V~skupine 90~detí má každé aspoň 30~kamarátov (kamarátstvo je vzájomné). Dokážte, že ich možno rozdeliť do troch 30-členných skupín tak, aby každé dieťa malo vo svojej skupine aspoň jedného kamaráta.}
\podpis{Ján Mazák}

{%%%%%   A-III-6
V~obore reálnych čísel riešte sústavu rovníc
$$
\align
x^4+y^2+4 &= 5yz,\\
y^4+z^2+4 &= 5zx,\\
z^4+x^2+4 &= 5xy.
\endalign
$$
}
\podpis{Jaroslav Švrček}

{%%%%%   B-S-1
V~obore celých čísel vyriešte rovnicu
$$
x^2+y^2+x+y=4.
$$}
\podpis{Jaroslav Švrček}

{%%%%%   B-S-2
Daný je pravouhlý trojuholník $ABC$ s~pravým uhlom pri vrchole~$C$. Nech $F$
je päta výšky z~vrcholu~$C$ na preponu~$AB$. Na kolmiciach na priamku~$AB$, ktoré prechádzajú
vrcholmi $A$ a~$B$, sú v~polrovine opačnej k~polrovine $ABC$ zvolené postupne body $D$ a~$E$, pre
ktoré platí $|AF|=|AD|$ a~$|BF|=|BE|$. Označme ďalej $R$ stred úsečky~$DE$. Dokážte, že
platí nerovnosť $|RF|\ge |CF|$ a~zistite, kedy nastane rovnosť.}
\podpis{Jaroslav Švrček}

{%%%%%   B-S-3
V~istom meste majú vybudovanú súvislú sieť na šírenie klebiet (klebety od
ľubovoľného klebetníka a~ľubovoľnej klebetnice sa môžu dostať ku všetkým
ostatným). V~nej si každý klebetník vymieňa informácie s~dvoma klebetnicami a~každá klebetnica si vymieňa
informácie s~troma klebetníkmi. Predpokladajme, že v~uvedenej sieti sa nájde taký muž aj taká žena,
že po prípadnom odsťahovaní ktorejkoľvek z~týchto dvoch osôb prestane byť sieť súvislou.
Nájdite najmenší možný počet členov tejto siete.}
\podpis{Pavel Calábek}

{%%%%%   B-II-1
Daných je $2\,012$ kladných čísel menších ako $1$, ktorých súčet je~$7$. Dokážte, že tieto čísla sa dajú
rozdeliť na štyri skupiny tak, aby súčet čísel v~každej skupine bol aspoň~$1$.}
\podpis{Vojtech Bálint}

{%%%%%   B-II-2
Určte, koľkými spôsobmi možno vrcholom pravidelného 9-uholníka $ABCDEFGHI$
priradiť čísla z~množiny $\{17,27,37,47,57,67,77,87,97\}$ tak, aby každé z~nich bolo
priradené inému vrcholu a~aby súčet čísel priradených každým trom susedným vrcholom bol
deliteľný tromi.}
\podpis{Jaroslav Švrček}

{%%%%%   B-II-3
Pravouhlému trojuholníku $ABC$ je vpísaná kružnica, ktorá sa dotýka prepony~$AB$
v~bode~$K$. Úsečku~$AK$ otočíme o~$90\st$ do polohy $AP$ a~úsečku $BK$ otočíme
o~$90\st$ do polohy $BQ$ tak, aby body $P$, $Q$ ležali v~polrovine opačnej k~polrovine
$ABC$.
\ite a) Dokážte, že obsahy trojuholníkov $ABC$ a~$PQK$ sú rovnaké.
\ite b) Dokážte, že obvod trojuholníka $ABC$ nie je väčší ako obvod
trojuholníka $PQK$. Kedy nastane rovnosť obvodov?\endgraf}
\podpis{Jaroslav Zhouf}

{%%%%%   B-II-4
Nájdite všetky reálne čísla $x$, $y$, ktoré vyhovujú sústave rovníc
$$
\align
{{x}\cdot{\Bigl\lfloor{\frac{y}{x}}\Bigr\rfloor}}&=5,\\
 {y \cdot{\Bigl\lfloor{\frac{x}{y}}\Bigr\rfloor}}&=-6.
\endalign
$$
(Zápis $\lfloor{a}\rfloor$ označuje {\it dolnú celú časť\/} čísla~$a$, \tj. najväčšie
celé číslo, ktoré neprevyšuje~$a$.)}
\podpis{Pavel Calábek}

{%%%%%   C-S-1
Nájdite všetky dvojice prirodzených čísel $a$, $b$, pre ktoré platí
rovnosť množín
$$
\{a\cdot[a,b],\ b\cdot(a,b)\}=\{45,180\},
$$
pričom $(x,y)$ označuje najväčší spoločný deliteľ a~$[x,y]$ najmenší spoločný násobok čísel $x$ a~$y$.}
\podpis{Tomáš Jurík}

{%%%%%   C-S-2
Označme $S$ stred základne~$AB$ daného rovnoramenného trojuholníka $ABC$.
Predpokladajme, že kružnice vpísané trojuholníkom $ACS$, $BCS$ sa
dotýkajú priamky~$AB$ v~bodoch, ktoré delia základňu~$AB$ na tri
zhodné diely. Vypočítajte pomer $|AB|:|CS|$.}
\podpis{Jaromír Šimša}

{%%%%%   C-S-3
Reálne čísla $p$, $q$, $r$, $s$ spĺňajú rovnosti
$$
p^2+q^2+r^2+s^2=4\qquad\text{a}\qquad pq+rs=1.
$$
Dokážte, že niektoré dve z~týchto štyroch čísel sa líšia najviac o~$1$
a~niektoré dve sa líšia najmenej o~$1$.}
\podpis{Jaromír Šimša}

{%%%%%   C-II-1
Pre všetky reálne čísla $x$, $y$, $z$ také, že $x<y<z$, dokážte nerovnosť
$$
x^2-y^2+z^2>(x-y+z)^2.
$$
}
\podpis{Jaromír Šimša}

{%%%%%   C-II-2
Janko má tri kartičky, na každej je iná nenulová cifra. Súčet
všetkých trojciferných čísel, ktoré možno z~týchto kartičiek
zostaviť, je číslo o~$6$ väčšie ako trojnásobok jedného z~nich.
Aké cifry sú na kartičkách?}
\podpis{Tomáš Jurík}

{%%%%%   C-II-3
Nech $E$ je stred strany~$CD$ rovnobežníka $ABCD$, v~ktorom
platí $2|AB|=3|BC|$. Dokážte, že ak sa dá do štvoruholníka $ABCE$ vpísať
kružnica, dotýka sa táto kružnica strany~$BC$ v~jej strede.}
\podpis{Ján Mazák}

{%%%%%   C-II-4
Na tabuli je napísaných prvých $n$ celých kladných čísel. Marína
a~Tamara sa striedajú v~ťahoch pri nasledujúcej hre. Najskôr
Marína zotrie jedno z~čísel na tabuli. Ďalej vždy hráčka, ktorá
je na ťahu, zotrie jedno z~čísel, ktoré sa od predchádzajúceho
zotretého čísla ani nelíši o~$1$, ani s~ním nie je súdeliteľné. Hráčka,
ktorá je na ťahu a~nemôže už žiadne číslo zotrieť, prehrá.
Pre $n=6$ a~pre $n=12$ rozhodnite, ktorá z~hráčok môže hrať tak,
že vyhrá nezávisle na ťahoch druhej hráčky.}
\podpis{Pavel Calábek}

{%%%%%   vyberko, den 1, priklad 1
Nech $W$ je vnútorný bod trojuholníka $ABC$. Bodom $W$ vedieme priamky $p_1$, $p_2$, $p_3$
rovnobežné so stranami trojuholníka $AB$, $BC$ a~$CA$, ktoré pretínajú strany trojuholníka
$ABC$ postupne v~bodoch~$K$ ($p_1 \cap CA$), $N$ ($p_1 \cap BC$), $L$ ($p_2 \cap AB$), $O$ ($p_2 \cap CA$), $M$ ($p_3 \cap BC$)
a $P$ ($p_3 \cap AB$). Uhlopriečky $KB$, $LC$ a~$MA$ lichobežníkov $ABNK$, $BLOC$ a~$CMPA$  delia trojuholník
$ABC$ na sedem častí, z~ktorých štyri sú trojuholníky. Dokážte, že súčet obsahov troch z~týchto trojuholníkov, ktoré ležia pri stranách trojuholníka $ABC$, je rovný obsahu štvrtého (vnútorného).}
\podpis{Dominik Csiba, Richard Kollár:Kvant M534, riesenie 10/1979}

{%%%%%   vyberko, den 1, priklad 2
Dané je prirodzené číslo $n\ge2$. Množina $\mm M$ uzavretých intervalov má tieto vlastnosti:
\itemitem Pre každý interval $\langle u, v \rangle \in M$ platí, že $u$ aj $v$ sú prirodzené čísla, $1 \le u < v \le n$.
\itemitem Pre každé dva rôzne intervaly $I \in\mm M$ a~$J \in\mm M$ platí $I \subset J$, alebo $J \subset I$, alebo $I \cap J = \emptyset$.
\endgraf
\noindent
Určte najväčší možný počet prvkov množiny $\mm M$.}
\podpis{Dominik Csiba, Richard Kollár:30-A-I-3}

{%%%%%   vyberko, den 1, priklad 3
Nájdite najmenšie reálne číslo $k$ také, že platí: ak je daný ľubovoľný trojuholník $ABC$ so stranami $a \le b \le c$, tak existuje
\ite a) rovnoramenný,
\ite b) pravouhlý rovnoramenný

trojuholník $XYZ$, ktorý obsahuje trojuholník $ABC$, a pre ktorého obsah platí
$$
S_{XYZ} \le kb^2.
$$

Ako sa zmení výsledok v~časti b), ak predpokladáme, že trojuholník $ABC$ je ostrouhlý?
}
\podpis{Dominik Csiba, Richard Kollár:34-A-I-5}

{%%%%%   vyberko, den 1, priklad 4
Dané sú kladné celé čísla $n$ a~$k$, $n > k$. Dokážte, že existujú celé kladné čísla $c_1, c_2, \dots, c_n$ také, že nerovnosť
$$
k(c_1 + \cdots + c_k) + (n-k) (c_{k+1} + \cdots +c_n) \le
p(c_1 + \cdots + c_p) + (n-p) (c_{p+1} + \cdots + c_n)
$$
platí pre všetky $p \in \{ 1, 2, \dots, n-1\}$.}
\podpis{Dominik Csiba, Richard Kollár:34-A-II-2}

{%%%%%   vyberko, den 2, priklad 1
Dané sú dve kružnice, ktoré majú vnútorný dotyk v~bode~$M$ a~priamka, ktorá sa dotýka vnútornej kružnice v~bode~$P$ a~pretína vonkajšiu kružnicu v~bodoch $Q$ a~$R$. Dokážte, že uhly $QMP$ a~$RMP$ sú zhodné.
}
\podpis{Tomáš Kocák:British mathematical olympiad - Round 1 13th January 1993}

{%%%%%   vyberko, den 2, priklad 2
Nájdite najväčšie prirodzené číslo~$k$ pre ktoré platí: Množina prirodzených čísel sa dá rozdeliť na $k$ navzájom disjunktných podmnožín $\mm A_1,\mm A_2,\dots, \mm A_k$ takých, že pre všetky prirodzené čísla $n\ge15$ a~všetky $i\in\{1,2,\dots,k\}$ existujú dva rôzne prvky z~množiny~$\mm A_i$, ktorých súčet je $n$.
%{\it Zadanie bude zverejnené po IMO 2012.}
}
\podpis{Tomáš Kocák:Shortlist 2011, C4}

{%%%%%   vyberko, den 2, priklad 3
Nech $n$ je dané prirodzené číslo. Nájdite všetky funkcie $f\colon\Bbb Z\to\Bbb Z$ také, že pre všetky celé čísla $x$ a~$y$ platí
$$
f(x+y+f(y))=f(x)+ny.
$$
}
\podpis{Tomáš Kocák:China team selection test 1 2012 day 2}

{%%%%%   vyberko, den 3, priklad 1
Nájdite minimum výrazu $a^4+b^4+c^4-3abc$ pre reálne čísla $a$, $b$, $c$ spĺňajúce podmienky $a\ge 1$ a $a+b+c=0$.
}
\podpis{Tomáš Jurík, Martin Kollár:Problem of the month, July/August 2006, Bilkent University, http://www.fen.bilkent.edu.tr/~cvmath/Problem/0609a.pdf}

{%%%%%   vyberko, den 3, priklad 2
Nech $ABCDE$ je tetivový päťuholník. Označme $a$, $b$, $c$, $d$ postupne vzdialenosti priamok $BC$, $CD$, $DE$ a $BE$ od bodu $A$. Vyjadrite $d$ pomocou $a$, $b$, $c$.}
\podpis{Tomáš Jurík, Martin Kollár:KS SKMO cca 1999, zborník poľských a~rakúskych MO}

{%%%%%   vyberko, den 3, priklad 3
Šachovnica $8\times 8$ je bez medzier pokrytá 32-mi kameňmi domina. Potom k~šachovnici pridáme na koniec prvého riadka deviate políčko. Je dovolené vziať ľubovoľný kameň domina a~umiestniť ho na ľubovoľné dve prázdne susedné políčka upravenej šachovnice. Dokážte, že takými ťahmi sa dajú všetky kamene domina uložiť vodorovne.}
\podpis{Tomáš Jurík, Martin Kollár:Moskovska olympiada 2004, http://olympiads.mccme.ru/mmo/2004/mmo2004.htm, klass 8, problem 6}

{%%%%%   vyberko, den 3, priklad 4
Dokážte, že pre každé prirodzené číslo~$d$ existuje také prirodzené číslo~$n$, ktoré je deliteľné číslom~$d$, a~aj číslo, ktoré vznikne vyškrtnutím vhodnej nenulovej číslice z~čísla~$n$, je deliteľné číslom~$d$.}
\podpis{Tomáš Jurík, Martin Kollár:Moskovska olympiada 2004, http://olympiads.mccme.ru/mmo/2004/mmo2004.htm, klass 11, problem 3}

{%%%%%   vyberko, den 4, priklad 1
Nech $x$, $y$ a~$z$ sú kladné celé čísla také, že $\frac1x-\frac1y=\frac1z$.
Nech $d$ je najväčším spoločným deliteľom $x$, $y$ a~$z$. Dokážte, že obe čísla $dxyz$ a $d(y-x)$
sú štvorcami celých čísel.}
\podpis{Peter Csiba, Jakub Santer, Matúš Stehlík:UK 1998, knizka Mathematical olympiads 1998/1999 problems and solutions from around the world, strana 157}

{%%%%%   vyberko, den 4, priklad 2
Nájdite všetky konečné množiny~$\mm A$ nezáporných reálnych čísel, pre ktoré platia obe podmienky:
\ite $\bullet$ množina~$\mm A$ obsahuje aspoň 4~čísla;
\ite $\bullet$ pre ľubovoľne 4~rôzne čísla $a,b,c,d \in\mm A$ je číslo $ab + cd$ tiež prvkom množiny~$\mm A$.
}
\podpis{Peter Csiba, Jakub Santer, Matúš Stehlík:Bulharsko 1998 - Mathematical olympiads 1998/1999 problems and solutions from around the world, strana 17}

{%%%%%   vyberko, den 4, priklad 3
Tri rôzne body $A$, $B$ a~$C$ ležia na priamke v~tomto poradí. Nech $k$ je kružnica prechádzajúca cez $A$ a~$C$, ktorej stred neleží na priamke~$AC$. Dotyčnice ku $k$ v~bodoch $A$ a~$C$ sa pretínajú v~bode~$P$. Úsečka~$PB$ pretína kružnicu~$k$ v~bode~$Q$. Dokážte, že priesečník osi uhla $AQC$ s~priamkou~$AC$ je rovnaký nezávisle na výbere kružnice~$k$.}
\podpis{Peter Csiba, Jakub Santer, Matúš Stehlík:Shortlist 2003, G2}

{%%%%%   vyberko, den 4, priklad 4
Do triedy chodí konečný počet dievčat a chlapcov. {\it Živá\/} skupina chlapcov je taká, že každé dievča pozná aspoň jedného chlapca zo skupiny. Podobne, {\it živá\/} skupina dievčat je taká, že každý chlapec pozná aspoň jedno dievča zo skupiny. Dokážte, že počet živých skupín chlapcov má rovnakú paritu ako počet živých skupín dievčat. (Poznanie sa je vzájomné, ak Fero pozná Aničku, tak aj Anička pozná Fera).}
\podpis{Peter Csiba, Jakub Santer, Matúš Stehlík:Romanian Masters (2012/1), http://www.artofproblemsolving.com/Forum/viewtopic.php?p=2617494&sid=0bd380e3c01a28a54e9b344cafed794a#p2617494}

{%%%%%   vyberko, den 5, priklad 1
Majme pevne dané prirodzené číslo~$n$. Nájdite všetky $n$-tice celých čísel $ (a_1,\dots,a_n)$, ktoré splňujú obidve nasledovné podmienky:
\ite $\bullet$ $a_1 + a_2 + \cdots + a_n \ge n^2$,
\ite $\bullet$ $a_1^2 + a_2^2 + \cdots + a_n^2 \le n^3 + 1$.
}
\podpis{Filip Sládek:China Western Olympiad 2002 Problem 6, Mathematical_Olympiad_in_China-Problems_and_Solutions}

{%%%%%   vyberko, den 5, priklad 2
Nech $ABC$ je rovnoramenný trojuholník so základňou $AB$. Ďalej nech $M$ je stred $AB$ a~$P$ je bod vnútri trojuholníka $ABC$ taký, že $|\uhol PAB| = |\uhol PBC|$. Dokážte, že
$$
|\uhol APM| + |\uhol BPC| = 180^{\circ}.
$$}
\podpis{Filip Sládek:Poland 2000 Problem 2, Andreescu - Contests Around the World 2000 - 2001}

{%%%%%   vyberko, den 5, priklad 3
Nech  $n$ je nepárne prirodzené číslo. Nájdite všetky funkcie $f\colon\Bbb Z \to \Bbb Z$ také, aby pre všetky celé čísla $x$ a~$y$ platilo $f(x) - f(y) \mid x^n - y^n$.
%{\it Zadanie bude zverejnené po IMO 2012.}
}
\podpis{Filip Sládek:Shortlist 2011, N3}

{%%%%%   vyberko, den 2, priklad 4
...}
\podpis{...}

{%%%%%   vyberko, den 5, priklad 4
...}
\podpis{...}

{%%%%%   trojstretnutie, priklad 1
Pre dané kladné celé číslo~$n$ označme $\tau(n)$ počet kladných deliteľov čísla~$n$ a~$\varphi(n)$ počet kladných celých čísel, ktoré nie sú väčšie ako $n$ a~sú s~$n$ nesúdeliteľné.  Nájdite všetky $n$, pre ktoré je niektoré z~troch čísel $n$, $\tau(n)$, $\varphi(n)$ aritmetickým priemerom zvyšných dvoch.}
\podpis{Peter Novotný}

{%%%%%   trojstretnutie, priklad 2
Nájdite všetky funkcie $f:\Bbb R\to \Bbb R$ také, že rovnosť
$$
f(x+f(y))-f(x)=(x+f(y))^4-x^4
$$
platí pre všetky $x,y\in\Bbb R$.}
\podpis{Kamil Duszenko}

{%%%%%   trojstretnutie, priklad 3
Daný je tetivový štvoruholník $ABCD$ s~opísanou kružnicou~$\omega$. Označme postupne $I$, $J$, $K$ stredy kružníc vpísaných do trojuholníkov $ABC$, $ACD$, $ABD$. Nech $E$ je stred oblúka~$DB$ kružnice~$\omega$ obsahujúceho bod~$A$. Priamka~$EK$ pretína kružnicu~$\omega$ v~bode~$F$ ($F\ne E$). Dokážte, že body $C$, $F$, $I$, $J$ ležia na jednej kružnici.}
\podpis{Kamil Duszenko}

{%%%%%   trojstretnutie, priklad 4
Daný je pravouhlý trojuholník $ABC$ s~preponou~$AB$ a~bod~$P$ ležiaci vnútri kratšieho oblúka~$AC$ kružnice opísanej trojuholníku $ABC$. Kolmica na priamku~$CP$ prechádzajúca bodom~$C$ pretína priamky $AP$, $BP$ postupne v~bodoch $K$, $L$. Dokážte, že pomer obsahov trojuholníkov $BKL$ a~$ACP$ nezávisí od polohy bodu~$P$.}
\podpis{Tomáš Jurík}

{%%%%%   trojstretnutie, priklad 5
Mesto Mar del Plata má tvar štvorca $W\!SEN$ a~je rozdelené $2(n+1)$ ulicami na $n\times n$ blokov, pričom $n$ je dané párne číslo (ulice vedú aj po obvode štvorca). Každý blok má rozmer $100\times100$ metrov. Všetky ulice v~Mar del Plata sú jednosmerné, majú rovnaký smer v~celej svojej dĺžke a~susedné rovnobežné ulice majú vždy opačný smer. Ulicou~$WS$ sa jazdí v~smere z~$W$ do $S$ a~ulicou $WN$ sa jazdí z~$W$ do $N$. V~bode~$W$ štartuje polievacie auto. Chce sa dostať do bodu~$E$ a~poliať pritom čo najviac ciest. Aká je dĺžka najdlhšej trasy, ktorú môže prejsť, ak po žiadnom 100-metrovom úseku nechce ísť viac ako raz? (Na \obr{} je pre $n=6$ znázornený plán mesta a~jedna z~možných -- nie však najdlhších -- trás polievacieho auta. Poz. tiež http://goo.gl/maps/JAzD.)
\insp{cps.1}%
}
\podpis{Peter Novotný}

{%%%%%   trojstretnutie, priklad 6
Kladné reálne čísla $a$, $b$, $c$, $d$ spĺňajú podmienky
$$
abcd=4,\qquad a^2+b^2+c^2+d^2 = 10.
$$
Určte najväčšiu možnú hodnotu výrazu $ab+bc+cd+da$.}
\podpis{Ján Mazák}

{%%%%%   IMO, priklad 1
Daný je trojuholník $ABC$. Označme $J$ stred kružnice pripísanej k~strane~$BC$. Táto kružnica sa dotýka strany~$BC$ v~bode~$M$ a~priamok $AB$, $AC$ postupne v~bodoch $K$, $L$. Priamky $LM$ a~$BJ$ sa pretínajú v~bode~$F$, priamky $KM$ a~$CJ$ v~bode~$G$. Nech $S$ je priesečník priamok $AF$, $BC$ a~$T$ priesečník priamok $AG$, $BC$. Dokážte, že $M$ je stredom úsečky~$ST$.}
\podpis{Grécko}

{%%%%%   IMO, priklad 2
Dané je celé číslo $n\ge3$. Nech $a_2a_3, \dots,a_n$ sú kladné reálne čísla spĺňajúce $a_2a_3\cdots a_n=1$.
Dokážte, že
$$
(1+a_2)^2(1+a_3)^3\cdots(1+a_n)^n > n^n.
$$
}
\podpis{Austrália}

{%%%%%   IMO, priklad 3
Hru "Myslím si číslo" s~povoleným klamaním hrajú dvaja hráči $A$ a~$B$. Pravidlá hry závisia od dvoch kladných celých čísel $k$ a~$n$, ktoré poznajú obaja hráči.

Na začiatku hry hráč~$A$ zvolí dve celé čísla $x$ a~$N$, pričom $1\le x\le N$. Číslo~$x$ hráč~$A$ uchová v~tajnosti a~pravdivo prezradí hráčovi~$B$ číslo~$N$. Hráč~$B$ sa potom pokúša zistiť informácie o~čísle~$x$ kladením otázok hráčovi~$A$ nasledovným spôsobom: každá jeho otázka pozostáva z~voľby ľubovoľnej množiny kladných celých čísel~$\mm S$ (môže to byť aj rovnaká množina, akú použil v~niektorej predošlej otázke) a~opýtania sa hráča~$A$, či číslo~$x$ patrí do~$\mm S$. Hráč~$B$ môže položiť toľko otázok, koľko len chce. Hráč~$A$ musí na každú z~otázok hráča~$B$ okamžite odpovedať {\it áno\/} alebo {\it nie}, môže však klamať, koľko sa mu zachce. Jediným obmedzením je, že medzi každými $k+1$ po sebe idúcimi odpoveďami hráča~$A$ musí byť aspoň jedna odpoveď pravdivá.

Potom, ako hráč~$B$ položí toľko otázok, koľko chce, určí množinu~$\mm X$ obsahujúcu nanajvýš $n$ kladných celých čísel.
Ak $x$ patrí do $\mm X$, hráč~$B$ vyhral; inak prehral. Dokážte nasledujúce tvrdenia:
\ite 1.
Ak $n\ge 2^k$, tak hráč~$B$ má víťaznú stratégiu.
\ite 2.
Pre každé dostatočne veľké číslo~$k$ existuje celé číslo $n\ge1{,}99^k$ také, že $B$ nemá víťaznú stratégiu.
\endgraf
}
\podpis{Kanada}

{%%%%%   IMO, priklad 4
Nájdite všetky funkcie $f\colon\Bbb Z\to\Bbb Z$ také, že pre
všetky celé čísla $a$, $b$, $c$ spĺňajúce $a+b+c=0$ platí
$$
f(a)^2 + f(b)^2 + f(c)^2  =  2f(a)f(b) + 2f(b)f(c) + 2f(c)f(a).
$$
}
\podpis{Južná Afrika}

{%%%%%   IMO, priklad 5
V~danom pravouhlom trojuholníku $ABC$ s~pravým uhlom pri vrchole~$C$ označme $D$ pätu výšky z~vrcholu~$C$. Nech $X$ je ľubovoľný vnútorný bod úsečky~$CD$. Označme $K$ taký bod na úsečke~$AX$, že $|BK|=|BC|$. Podobne označme $L$ taký bod na úsečke~$BX$, že $|AL|=|AC|$.
Priesečník priamok $AL$ a~$BK$ označme~$M$. Dokážte, že $|MK|=|ML|$.}
\podpis{Česká rep., Josef Tkadlec}

{%%%%%   IMO, priklad 6
Určte všetky kladné celé čísla~$n$, pre ktoré existujú nezáporné celé čísla $a_1,a_2,\dots,a_n$ také, že
$$
\frac{1}{2^{a_1}}+\frac{1}{2^{a_2}}+\cdots+\frac{1}{2^{a_n}}  =
\frac{1}{3^{a_1}}+\frac{2}{3^{a_2}}+\cdots+\frac{n}{3^{a_n}}  = 1.
$$
}
\podpis{Srbsko}

{%%%%%   MEMO, priklad 1
Nájdite všetky funkcie $f\colon\Bbb R^+\to \Bbb R^+$ také, že rovnosť
$$
f(x+f(y))=yf(xy+1)
$$
platí pre všetky $x,y\in\Bbb R^+$. (Symbol $\Bbb R^+$ označuje množinu všetkých kladných reálnych čísel.)}
\podpis{Chorvátsko}

{%%%%%   MEMO, priklad 2
Dané je kladné celé číslo~$N$. Množina $\mm S \subseteq \{1,2,\dots,N \}$ sa nazýva {\it prípustná}, ak neobsahuje také tri navzájom rôzne čísla $a$, $b$, $c$, že $a$ delí $b$ a~súčasne $b$ delí $c$. Určte najväčší možný počet prvkov, ktorý môže mať prípustná množina~$\mm S$.}
\podpis{Maďarsko}

{%%%%%   MEMO, priklad 3
Daný je lichobežník $ABCD$, pričom $AB\parallel CD$, $|AB|>|CD|$ a~priamka~$BD$ je osou uhla $ADC$.
Priamka prechádzajúca bodom~$C$ rovnobežná s~$AD$ pretína úsečky $BD$ a~$AB$ postupne v~bodoch $E$ a~$F$. Označme $O$ stred kružnice opísanej trojuholníku $BEF$. Predpokladajme, že $|\uhol ACO| = 60^\circ$. Dokážte, že
$$
|CF| = |AF| + |FO|.
$$}
\podpis{Chorvátsko}

{%%%%%   MEMO, priklad 4
Postupnosť $\{a_n\}_{n=0}^\infty$ je definovaná vzťahmi $a_0=2$, $a_1=4$ a
$$
a_{n+1}=\frac{a_na_{n-1}}{2}+a_n+a_{n-1}\quad \text{pre všetky kladné celé čísla $n$.}
$$
Určte všetky prvočísla~$p$, pre ktoré existuje kladné celé číslo $m$ také, že $p$ je deliteľom $a_m-1$.}
\podpis{Švajčiarsko}

{%%%%%   MEMO, priklad t1
Nájdite všetky trojice $(x,y,z)$ reálnych čísel také, že
$$
\aligned
2x^3+1 &= 3zx,\\
2y^3+1 &= 3xy,\\
2z^3+1 &= 3yz.
\endaligned
$$
}
\podpis{Česká rep., Jaroslav Švrček}

{%%%%%   MEMO, priklad t2
Nech $a$, $b$, $c$ sú kladné reálne čísla, pre ktoré platí $abc=1$. Dokážte, že
$$
\postdisplaypenalty=10000
\sqrt{9+16a^2}+\sqrt{9+16b^2}+\sqrt{9+16c^2} \ge 3+4(a+b+c).
$$
}
\podpis{Nemecko}

{%%%%%   MEMO, priklad t3
Nech $n$ je kladné celé číslo. Uvažujme slová dĺžky $n$ zložené z~písmen množiny $\{M,E,O\}$. Označme $a$ počet tých slov, ktoré obsahujú párny počet (môže byť aj nulový) blokov $ME$ a~párny počet (môže byť aj nulový) blokov $MO$. Podobne označíme $b$ počet tých slov, ktoré obsahujú nepárny počet blokov $ME$ a~nepárny počet blokov $MO$.
Dokážte, že $a>b$.}
\podpis{Poľsko}

{%%%%%   MEMO, priklad t4
Nech $p > 2$ je prvočíslo. Pre ľubovoľnú permutáciu $\pi = (\pi(1),\pi(2),\dots,\pi(p))$ množiny $\mm S = \{1,2,\dots,p\}$ označme $f(\pi)$ počet tých čísel spomedzi
$$
    \pi(1), \ \pi(1)+\pi(2), \ \dots, \ \pi(1)+\pi(2)+\dots+\pi(p),
$$
ktoré sú deliteľné číslom $p$.
Určte priemernú hodnotu čísla $f(\pi)$, ak uvažujeme všetky permutácie $\pi$ množiny $\mm S$.}
\podpis{Maďarsko}

{%%%%%   MEMO, priklad t5
Nech $K$ je stred strany~$AB$ daného trojuholníka $ABC$. Nech $L$
a~$M$ sú také body ležiace postupne na stranách $AC$ a~$BC$, pre ktoré platí
$|\uhol CLK|=|\uhol KMC|$. Dokážte, že kolmice na strany
$AB$, $AC$ a~$BC$ prechádzajúce postupne bodmi $K$, $L$ a~$M$ sa pretínajú v~jednom bode.}
\podpis{Poľsko}

{%%%%%   MEMO, priklad t6
Nech $ABCD$ je konvexný štvoruholník, ktorého žiadne dve strany nie sú rovnobežné, pričom $|\uhol ABC|= |\uhol CDA|$.
Predpokladajme, že priesečníky dvojíc osí uhlov pri susedných vrcholoch štvoruholníka $ABCD$ tvoria štvoruholník $EFGH$. Nech $K$ je priesečník uhlopriečok štvoruholníka $EFGH$. Dokážte, že priesečník priamok $AB$ a~$CD$ leží na kružnici opísanej trojuholníku $BKD$.}
\podpis{Chorvátsko}

{%%%%%   MEMO, priklad t7
Nájdite všetky trojice $(x,y,z)$ celých kladných čísel také, že
$$
\aligned
x^y+y^x&=z^y, \\
x^y+2012&=y^{z+1}.
\endaligned
$$
}
\podpis{Litva}

{%%%%%   MEMO, priklad t8
Pre ľubovoľné celé kladné číslo $n$ označme $\tau(n)$ počet kladných deliteľov čísla~$n$.
Zistite, či existujú celé kladné čísla $a$ a $b$ také, že
$\tau(a) = \tau(b)$ a $\tau(a^2)=\tau(b^2)$, ale $\tau(a^3) \ne \tau(b^3)$.}
\podpis{Česká rep., Michal Rolínek}

{%%%%%   CPSJ, priklad 1
Nech $P$ je bod ležiaci vnútri trojuholníka $ABC$. Body $K$, $L$, $M$
sú obrazmi bodu~$P$ v~stredovej súmernosti postupne podľa stredov strán
$BC$, $CA$, $AB$. Dokážte, že priamky $AK$, $BL$, $CM$ sa pretínajú v~jednom bode.}
\podpis{Jaroslav Švrček}

{%%%%%   CPSJ, priklad 2
Nájdite všetky trojice prvočísel $(a,b,c)$ spĺňajúcich
rovnosť
$$
a^2+ab+b^2=c^2+3.
$$}
\podpis{Poľsko}

{%%%%%   CPSJ, priklad 3
Na kružnici so stredom~$O$ sú zvolené štyri rôzne body $A$, $B$, $C$, $D$, pričom
$$
|\uhol AOB|=|\uhol BOC|=|\uhol COD|=60^\circ.
$$
Nech $P$ je ľubovoľný bod ležiaci na kratšom oblúku~$BC$ danej kružnice.
Body $K$, $L$, $M$ sú päty kolmíc spustených z~bodu~$P$ postupne
na priamky $AO$, $BO$, $CO$. Dokážte, že
\ite a) trojuholník $KLM$ je rovnostranný,
\ite b) obsah trojuholníka $KLM$ nezávisí od polohy bodu~$P$ na kratšom oblúku~$BC$.\endgraf}
\podpis{Poľsko}

{%%%%%   CPSJ, priklad 4
Dokážte, že ak zvolíme ľubovoľných 51~vrcholov pravidelného 101-uholníka, tak niektoré tri
zo zvolených bodov budú vrcholmi rovnoramenného trojuholníka.}
\podpis{Jaromír Šimša}

{%%%%%   CPSJ, priklad 5
Nech $a$, $b$, $c$ sú kladné celé čísla spĺňajúce $a^2+b^2=c^2$. Dokážte, že
číslo ${1\over2}(c-a)(c-b)$ je druhou mocninou celého čísla.}
\podpis{Poľsko}

{%%%%%   CPSJ, priklad t1
Na tabuli je napísaných niekoľko rôznych reálnych čísel.
Vieme, že hodnota súčinu ľubovoľných dvoch rôznych čísel z tabule je
tiež napísaná na tabuli.
Určte, koľko najviac čísel môže byť napísaných na tabuli.}
\podpis{Ján Mazák}

{%%%%%   CPSJ, priklad t2
Na kružnici~$k$ sú dané body $A$ a~$B$, pričom $AB$ nie je priemerom
kružnice~$k$.
Bod~$C$ sa pohybuje po dlhšom oblúku~$AB$ kružnice~$k$ tak, že trojuholník $ABC$ je ostrouhlý.
Nech $D$ je päta výšky z~vrcholu~$A$ na stranu~$BC$ a~$E$ je päta výšky z~$B$ na~$AC$.
Ďalej nech $F$ je päta kolmice z~bodu~$D$ na priamku~$AC$ a~$G$ je päta kolmice z~$E$ na~$BC$.
\ite a) Dokážte, že priamky $AB$ a~$FG$ sú rovnobežné.
\ite b) Určte množinu stredov~$S$ úsečiek~$FG$ prislúchajúcim ku všetkým prípustným polohám bodu~$C$.\endgraf}
\podpis{Ján Mazák}

{%%%%%   CPSJ, priklad t3
Udowodnij, że jeśli $n$ jest dodatni\ą{} liczb\ą{} ca\l{}kowit\ą{}, to liczba $2(n^2 + 1) - n$ nie jest
kwadratem liczby ca\l{}kowitej.}
\podpis{Poľsko}

{%%%%%   CPSJ, priklad t4
Dany jest romb $ABCD$, w którym $\uhol BAD = 60\st$. Punkt $P$ leży wewn\ą{}trz rombu, przy
czym spe\l{}nione s\ą{} równości $BP = 1$, $DP = 2$, $CP = 3$. Wyznacz d\l{}ugość odcinka $AP$.}
\podpis{Poľsko}

{%%%%%   CPSJ, priklad t5
Určete všechny trojice $(a,k,m)$ kladných celých čísel, které vyhovují rovnici
$$k+a^k=m+2a^m.$$
}
\podpis{Poľsko}

{%%%%%   CPSJ, priklad t6
Šachovnicovou desku $8\times8$ máme pokrýt pomocí rovinných útvarů stejných
jako na \obr{} (každý z útvarů můžeme otočit o~$90\st$) tak, že se žádné
dva nepřekrývají ani nepřesahuji přes okraj šachovnice. Určete, jaký největší
možný počet polí této šachovnice můžeme uvedeným způsobem pokrýt.
\insp{61-cpsj-t6}%
}
\podpis{Poľsko}
