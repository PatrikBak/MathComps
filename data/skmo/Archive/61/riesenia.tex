{%%%%%   A-I-1
Najskôr vypočítame hodnotu čísla~$n$, potom už jeho zvyšok po delení číslom~$77$ určíme jednoducho. V~zadaní opísané desaťciferné čísla nebudeme sčitovať priamo. Hľadaný súčet ľahšie nájdeme tak, že zistíme, koľkokrát sa ktorá cifra nachádza vo všetkých číslach na mieste jednotiek, desiatok, stoviek, atď. Následne určíme, aký je "príspevok" každej cifry do celkového súčtu~$n$.

Ak je cifra~$1$ na mieste jednotiek, potom na zvyšných deväť pozícií môžeme zvyšných deväť cifier rozmiestniť ľubovoľne, akurát na prvú pozíciu nesmieme dať cifru~$0$. Na prvú pozíciu teda môžeme umiestniť niektorú z~ôsmich rôznych cifier (každú okrem~$0$), následne na druhú pozíciu niektorú z~ôsmich rôznych cifier (každú okrem tej, ktorú sme dali na prvé miesto), na tretiu pozíciu niektorú zo siedmich rôznych cifier (každú okrem cifier na prvých dvoch pozíciách), atď. Cifra~$1$ sa preto na mieste jednotiek nachádza $8\cdot8\cdot7\cdot6\cdot5\cdot4\cdot3\cdot2\cdot1=8\cdot8!$-krát.\footnote{K~rovnakému výsledku by sme dospeli aj inou úvahou: Ak by sme cifru~$0$ pripustili aj na prvej pozícii, všetkých čísel končiacich cifrou $1$ by bolo~$9!$. Nevyhovujúcich čísel s~nulou na prvej pozícii je 8!. Vyhovujúcich čísel je teda $9!-8!=9\cdot8!-8!=8\cdot8!$.}

Rovnakou úvahou zistíme, že cifra~$1$ sa nachádza $8\cdot8!$-krát aj na mieste desiatok, stoviek, tisísok, atď. Len na prvej pozícii sa nachádza až $9!$-krát, pretože vtedy na zvyšných deväť pozícií môžeme zvyšných deväť cifier rozmiestniť ľubovoľne -- nemáme obmedzenie pre cifru~$0$.

Takže príspevok cifry~$1$ do celkového súčtu je
$$
\gathered
8\cdot8!+8\cdot8!\cdot10+8\cdot8!\cdot100+\cdots+8\cdot8!\cdot10^8+9!\cdot10^9=\\
=8\cdot8!\cdot111\,111\,111+9\cdot8!\cdot10^9=8!\cdot9\,888\,888\,888.
\endgathered
$$

Cifra~$2$ sa zrejme nachádza vo všetkých číslach na jednotlivých pozíciách rovnako veľakrát ako cifra~$1$, takže jej príspevok do celkového súčtu je dvojnásobný. Príspevok cifry~$3$ je trojnásobný, príspevok cifry~$4$ je štvornásobný, atď. Počet výskytov cifry~$0$ na jednotlivých pozíciách je síce iný ako pri ostatných cifrách, ale nemusíme ho určovať, keďže príspevok cifry~$0$ do súčtu je nulový. Spolu teda máme
$$
n=8!\cdot9\,888\,888\,888\cdot(1+2+\cdots+9)=45\cdot8!\cdot9\,888\,888\,888.
\tag1
$$

Hľadaný zvyšok by sme už teraz mohli určiť vyčíslením hodnoty~$n$ a~následným delením číslom~$77$. My sa však vyhneme zdĺhavému násobeniu a~deleniu veľkých čísel. Z~vyjadrenia \thetag1 vidíme, že $n$ je deliteľné siedmimi (pretože činiteľ $8!$ je násobok siedmich). Keďže $77=7\cdot11$, zvyšok $n$ po delení sedemdesiatimi siedmimi musí byť násobkom siedmich.

Ešte určíme zvyšok čísla~$n$ po delení jedenástimi. Triviálne platí $45\equiv1\pmod{11}$ a~ľahko vypočítame, že $8!=40\,320\equiv5\pmod{11}$. Určenie zvyšku čísla $9\,888\,888\,888$ môžeme urýchliť známym tvrdením: Číslo s~dekadickým zápisom $\overline{a_ka_{k-1}\dots a_1a_0}$ dáva po delení jedenástimi rovnaký zvyšok ako číslo $a_0-a_1+a_2-\cdots+(-1)^ka_k$. Takže dostávame
$$
9\,888\,888\,888\equiv 8-8+8-8+\cdots+8-9=-1\equiv10\pmod{11}.
$$
Spolu máme
$$
n=45\cdot8!\cdot9\,888\,888\,888\equiv 1\cdot5\cdot10=50\equiv6\pmod{11}
$$
(využili sme vlastnosť, že súčin činiteľov dáva rovnaký zvyšok ako súčin zvyškov činiteľov).

Zvyškom po delení čísla $n$ sedemdesiatimi siedmimi je teda číslo z~množiny $\{6,17,28,39,50,61,72\}$, ktoré je deliteľné siedmimi,
teda číslo $28$.

\ineriesenie
Pripusťme, že na prvom mieste môže byť aj nula. Potom medzi všetkými číslami,
ktoré majú vo svojom dekadickom zápise každú z~cifier $0$, $1$, \dots, $9$, sa každá
cifra vyskytuje na každom z~desiatich miest $9!$-krát. Preto je súčet všetkých
takých čísel
$$
s_1=9!(0+1+2+\cdots+9)\cdot(1+10+\dots+10^9)=9!\cdot45\cdot\frac{10^{10}-1}9=9!\cdot5\cdot(10^{10}-1).
$$
Musíme ale odčítať súčet tých (navyše zarátaných) desaťciferných čísel, ktoré
začínajú nulou. To sú vlastne deväťciferné čísla, ktoré majú vo svojom dekadickom
zápise každú z~cifier $1$, $2$, \dots, $9$. Analogicky ako v~pred chvíľou odvodíme, že ich súčet je
$$
s_2=8!(1+2+\cdots+9)\cdot(1+10+\dots+10^8)=8!\cdot45\cdot\frac{10^9-1}9=8!\cdot5\cdot(10^9-1).
$$

Zrejme $7\mid s_1$ a~$7\mid s_2$, takže aj $7\mid s_1-s_2=n$. Ďalej s~využitím známeho rozkladu $(10^{2k+1}+1)=(10+1)(10^{2k}-10^{2k-1}+\cdots+1)$ máme
$$
\alignedat2
s_1&=9!\cdot5(10^5-1)(10^5+1)\equiv9!\cdot5(10^5-1)\cdot0\equiv0 & &\quad\pmod{11},\\
s_2&=8!\cdot5\cdot(10^9-1)=(8\cdot7)\cdot(6\cdot5\cdot4)\cdot3\cdot2\cdot5\cdot(10^9+1-2)\equiv & &\\
   &\equiv 1\cdot(-1)\cdot30\cdot(-2)\equiv5 & &\quad\pmod{11},
\endalignedat
$$
čiže $n=s_1-s_2\equiv6\pmod{11}$. Záver je rovnaký ako pri prvom riešení.

\návody
Určte počet všetkých desaťciferných čísel, ktoré majú vo svojom dekadickom zápise každú z~cifier $0$, $1$, \dots, $9$. [$10!-9!$]

Učiteľ si myslí číslo. Žiakom prezradil, že jeho číslo končí cifrou~$6$ a~dáva po delení trinástimi zvyšok~$9$. Určte, aký zvyšok dáva učiteľovo číslo po delení číslom $65$. [Učiteľovo číslo $n$ spĺňa $n\equiv1\pmod5$, $n\equiv9\pmod{13}$. Keďže $65=5\cdot13$, hľadaný zvyšok musí byť z~množiny $\{9, 9+13, 9+26, 9+39, 9+52\}$. Z~týchto čísel jedine $9+52=61$ dáva správny zvyšok po delení piatimi.]

Dokážte, že číslo s~dekadickým zápisom $\overline{a_ka_{k-1}\dots a_1a_0}$ dáva po delení jedenástimi rovnaký zvyšok ako číslo $a_0-a_1+a_2-\cdots+(-1)^ka_k$. [Ak $i$ je párne, tak $10^i=(10^i-1)+1=9\dots9+1=99\cdot10\dots101+1\equiv1\pmod{11}$. Ak $i$ je nepárne, tak $10^i=10\cdot10^{i-1}\equiv10\cdot1\equiv-1\pmod{11}$.]

\D
Dokážte, že ak čísla $a$, $b$ dávajú po delení číslom~$d$ postupne zvyšky $u$, $v$, tak zvyšky čísel $ab$, $uv$ po delení $d$ sú rovnaké.

Dokážte, že zvyšky čísel $1$, $10$, $10^2$, $10^3$, \dots{} po delení ľubovoľným nepárnym prvočíslom rôznym od $5$ tvoria periodickú postupnosť.
\endnávod
}

{%%%%%   A-I-2
Účastníkov stretnutia budeme znázorňovať plnými krúžkami a~to, že sa dvaja ľudia poznajú, budeme znázorňovať spojením príslušných krúžkov čiarou\footnote{Takému znázorneniu sa hovorí {\it graf}; účastníci sú {\it vrcholy\/} a~známosti sú {\it hrany\/} grafu}. Zadanú situáciu zakreslíme na \obr.
\insp{a61.1}%

Množinu známych účastníka $A$ rôznych od $B$ označme $\mm M_A$ a~množinu známych účastníka~$B$ rôznych od $A$ označme $\mm M_B$.
Ani jeden človek z~$\mm M_A$ sa nepozná s~$B$, lebo $A$ a~$B$ nemajú spoločného známeho. Takže každý z~$\mm M_A$ má s~$B$ práve dvoch spoločných známych: jeden z~nich je $A$, druhý sa nachádza medzi zvyšnými známymi účastníka~$B$, teda v~$\mm M_B$.
Pritom žiadni dvaja ľudia $X$, $Y$ z~$\mm M_A$ nemôžu poznať toho istého človeka~$Z$ v~$\mm M_B$. V~opačnom prípade by sa totiž $Z$ nepoznal s~$A$ (lebo $A$, $B$ nemajú spoločných známych) a~zároveň by $X$, $Y$, $B$ tvorili trojicu ich spoločných známych (\obr{}a), čo je v~rozpore so zadaním.
\inspnspab{a61.2}{a61.3}{\quad}%

Zhrňme ešte raz poznatky odvodené v~predošlom odseku: Každý človek z~$\mm M_A$ pozná niekoho z~$\mm M_B$ a~žiadni dvaja ľudia z~$\mm M_A$ nepoznajú toho istého v~$\mm M_B$ (\obrr1b). Z~toho vyplýva, že v~množine~$\mm M_B$ je aspoň toľko ľudí ako v~množine~$\mm M_A$, \tj. $|\mm M_B|\ge|\mm M_A|$. Tou istou úvahou (po zámene úloh $A$ a~$B$) vieme samozrejme dokázať, že $|\mm M_A|\ge|\mm M_B|$. Nutne teda $|\mm M_A|=|\mm M_B|$, čiže $A$ a~$B$ majú medzi prítomnými rovnaký počet známych.

\smallskip
V~druhej časti riešenia už len vymyslíme a~znázorníme nejaké vyhovujúce rozloženie známostí medzi šesticou osôb, z~ktorých dve osoby sú $A$ a~$B$, poznajú sa a~nemajú spoločných známych. Z~prvej časti už vieme, že tieto osoby musia mať rovnaký počet známych. Stačí teda napríklad nakresliť spojené osoby $A$, $B$, každú z~nich spojiť s~dvoma ďalšími osobami a~medzi štvoricou osôb dokresliť čiary tak, aby bolo splnené zadanie. Ľahko objavíme vyhovujúce rozloženie ako na \obr{}a (ktoré možno prekresliť napr. do tvaru na \obrr1b).
\inspinspab{a61.4}{a61.5}%

\ineriesenie
Nech $\mm M_A$, $\mm M_B$ sú tie isté množiny ako v~prvom riešení. Budeme používať pojmy z~teórie grafov. Nech $\mm M_A$ obsahuje $m$~vrcholov a~$\mm M_B$ obsahuje $n$~vrcholov. Keďže každý vrchol z~$\mm M_A$ má s~$B$ spoločného známeho~$A$ a~zároveň nie je známym vrcholu~$B$, musí mať s~$B$ práve jedného známeho v~$\mm M_B$ (aby bola splnená podmienka zo zadania, že majú práve dvoch spoločných známych). Takže z~každého vrcholu v~$\mm M_A$ vychádza práve jedna hrana do $\mm M_B$. Spolu preto vychádza z~$\mm M_A$ do $\mm M_B$ práve $m$~hrán. Analogicky z~$\mm M_B$ vychádza do $\mm M_A$ práve $n$~hrán. Sú to však tie isté hrany, takže nutne $m=n$.

\návody
Na stretnutí bolo niekoľko ľudí. Každí dvaja, ktorí sa nepoznali, mali medzi
ostatnými prítomnými práve {\it jedného\/} spoločného známeho. Nikto sa nepoznal s~každým. Účastníci $A$ a~$B$ sa
poznali, ale nemali ani jedného spoločného známeho. Dokážte, že na stretnutí bola osoba, ktorá nepoznala $A$ ani $B$.
[Ak by $A$ nemal okrem $B$ žiadneho známeho, musel by každý poznať $B$, čo nevyhovuje zadaniu. Takže $A$ má okrem $B$ aspoň jedného známeho~$X$. Podobne $B$ má okrem $A$ známeho $Y$. Pritom $X$ a~$Y$ sa nemôžu poznať, preto musia mať spoločného známeho $Z$, ktorý nepozná $A$ ani $B$.]

\D
Dokážte, že rozloženie na \obrr1a je jediné vyhovujúce rozloženie so šiestimi osobami.

Na stretnutí bolo niekoľko ľudí. Každí dvaja, ktorí sa nepoznali, mali medzi
ostatnými prítomnými práve {\it troch\/} spoločných známych. Účastníci $A$ a~$B$ sa
poznali, ale nemali ani jedného spoločného známeho. Dokážte, že $A$ aj $B$
mali medzi prítomnými rovnaký počet známych. [Pri podobných úvahách ako v~druhom riešení (z~každého vrcholu z~$M_A$ vychádzajú práve dve hrany do $M_B$) dostaneme $2m=2n$, čiže $m=n$.]

V~skupine $n$~žiakov sa spolu niektorí kamarátia. Vieme, že každý má medzi ostatnými
aspoň štyroch kamarátov. Učiteľka chce žiakov rozdeliť na dve nanajvýš štvorčlenné skupiny
tak, že každý bude mať vo svojej skupine aspoň jedného kamaráta. a)~Ukážte, že v~prípade $n = 7$ sa dajú žiaci požadovaným spôsobom vždy rozdeliť. b)~Zistite, či možno žiakov takto vždy rozdeliť aj v~prípade $n = 8$.
\vpravo{[60--C--I--4]}
\endnávod
}

{%%%%%   A-I-3
Označme vrcholy daného trojuholníka písmenami $A$, $B$, $C$ tak, aby $BC$ bola základňa. Veľkosti strán a~uhlov trojuholníka budeme označovať štandardným spôsobom, \tj. $|BC|=a$, $|AC|=|AB|=b$, $\beta=\gamma=90\st-\frac12\alpha$. Nech $H$ je stred základne~$BC$.

\smallskip
a) Veďme vrcholom~$B$ výšku, os uhla a~ťažnicu trojuholníka $ABC$ a~ich priesečníky s~priamkou~$CA$ označme postupne $P$, $D$ a~$M$. Všetky tri ležia na polpriamke $CA$ (body $D$, $M$ dokonca na úsečke~$CA$). Zrejme stačí dokázať, že bod~$D$ je vnútorným bodom úsečky~$MP$. Rozoberieme dva prípady.

Ak $b>a$ (\obr{}a), tak zrejme $\beta>60\st$. Odtiaľ máme
$$
|\uhol CBP|=90\st-\beta<90\st-60\st=30\st<\tfrac12\beta=|\uhol CBD|,
$$
takže $|CP|<|CD|$. Os uhla delí stranu trojuholníka v~pomere priľahlých strán, preto $|CD|/|AD|=a/b<1$, odkiaľ $|CD|<\frac12|CA|=|CM|$. Spolu teda $|CP|<|CD|<|CM|$, čiže $D$ leží vnútri úsečky~$MP$.

Ak $a>b$ (\obrr1b), tak $\beta<60\st$ a~analogicky dostávame $|\uhol CBP|=90\st-\beta>90\st-60\st=30\st>\tfrac12\beta=|\uhol CBD|$, $|CD|/|AD|=a/b>1$, takže $|CP|>|CD|>\frac12|CA|=|CM|$, čiže aj v~tomto prípade $D$ leží vnútri úsečky~$MP$.
\inspinspab{a61.6}{a61.7}%

\smallskip
b)
Najskôr vyjadríme dĺžky úsečiek $TH$, $SH$, $VH$ pomocou dĺžok strán trojuholníka a~pomocou dĺžky $|AH|$, ktorú označme~$v$ (\obr). Z~Pytagorovej vety v~trojuholníku $ABH$ máme $v^2=b^2-\frac14a^2$, takže neskôr budeme vedieť za $v$ dosadiť vyjadrenie len pomocou dĺžok $a$, $b$.

Ťažisko~$T$ delí ťažnicu $AH$ v~pomere $2:1$, takže $|TH|=\frac13v$.

Úsečka $SH$ je polomerom vpísanej kružnice, takže jej dĺžku vypočítame zo známeho vzorca $S_{ABC}=\varrho\cdot s$ pre obsah trojuholníka $ABC$, pričom $s$ označuje polovicu obvodu. Dostávame
$$
|SH|=\varrho=\frac{S_{ABC}}{s}=\frac{\frac12av}{\frac12(a+2b)}=\frac{av}{a+2b}.
$$

Trojuholníky $BVH$ a~$ABH$ sú podobné, pretože sú oba pravouhlé a~$|\uhol VBH|=90\st-\beta=\frac12\alpha=|\uhol BAH|$. Pre prislúchajúce dĺžky strán preto máme $|VH|:|BH|=|BH|:|AH|$, odkiaľ
$$
|VH|=\frac{|BH|^2}{|AH|}=\frac{a^2}{4v}.
$$

Ak je $S$ stredom úsečky $TV$, platí $|TS|=|SV|$. Pritom
$$
|TS|=\bigl||TH|-|SH|\bigr|,\qquad |SV|=\bigl||SH|-|VH|\bigr|
\tag1
$$
a~z~časti~a) vieme, že rozdiely v~absolútnych hodnotách v~\thetag1 sú buď oba kladné, alebo oba záporné. Rovnosť $|TS|=|SV|$ je teda ekvivalentná s~rovnosťou
$$
|TH|-|SH|=|SH|-|VH|,\qquad\text{čiže}\qquad |TH|+|VH|=2|SH|.
$$
Dosadením odvodených dĺžok po ekvivalentných úpravách dostávame
$$
\align
\frac13v+\frac{a^2}{4v} &= \frac{2av}{a+2b},\\
4v^2(a+2b)+3a^2(a+2b) &= 24av^2,\\
3a^2(a+2b) &= 4v^2(5a-2b),\\
3a^2(a+2b) &= 4(b^2-\tfrac14a^2)(5a-2b),\\
3a^2(a+2b) &= (2b+a)(2b-a)(5a-2b),\\
3a^2 &= (2b-a)(5a-2b),\\
3a^2 &= 12ab-5a^2-4b^2,\\
2a^2-3ab+b^2 &= 0,\\
(2a-b)(a-b) &= 0.\\
\endalign
$$
Keďže podľa zadania je $a\ne b$, bod~$S$ je stredom úsečky~$TV$ práve vtedy, keď $2a=b$, teda keď pomer dĺžok strán trojuholníka je $1:2:2$.
\inspinsp{a61.8}{a61.9}%

\ineriesenie
a) Keďže $S$, $T$, $V$ sú vnútorné body polpriamky~$HA$ (body $S$, $T$ sú dokonca vnútorné body úsečky~$HA$), stačí
ukázať, že oba rozdiely $|HS|-|HT|$ a $|HV|-|HS|$ sú nenulové a~majú rovnaké znamienko.

Bez ujmy na všeobecnosti nech $a=2$, \tj. $|BH|=|HC|=1$. Z~pravouhlých trojuholníkov $BSH$, $BAH$ a~$BVH$
dostávame (\obr)
$$
|HS|=\tg\frac{\beta}{2},\quad
|HT|=\frac{|HA|}{3}=\frac{\tg\beta}{3}\quad\text{a}\quad
|HV|=\tg(90\st-\beta)=\frac{1}{\tg\beta}.
$$
Vďaka vzťahu $\tg2x=\dfrac{2\tg x}{1-\tg^2x}$ pri označení
$t=\tg\frac12\beta$ máme
$$
\align
|HS|-|HT|&=t-\frac{2t}{3(1-t^2)}=\frac{t(1-3t^2)}{3(1-t^2)},\\
|HV|-|HS|&=\frac{1-t^2}{2t}-t=\frac{1-3t^2}{2t}.
\endalign
$$
Keďže $\beta$ je ostrý uhol rôzny od $60\st$, platí $\frac12\beta\in(0\st,45\st)$ a~$\frac12\beta\ne30\st$,
odkiaľ
pre hodnotu $t=\tg\frac12\beta$ vyplývajú vzťahy $0<t<1$ a $3t^2\ne1$,
takže oba skúmané rozdiely sú buď kladné (ak $\beta<60\st$), alebo
záporné (ak $\beta>60\st$).

\smallskip
b)
Podiel rozdielov z~časti~a) má vyjadrenie
$$
\frac{|HS|-|HT|}{|HV|-|HS|}=
\frac{t(1-3t^2)}{3(1-t^2)}\cdot\frac{2t}{1-3t^2}=
\frac{2t^2}{3(1-t^2)}.
$$
V~obore $(0,1)$ riešime rovnicu
$$
\frac{2t^2}{3(1-t^2)}=1,
$$
ktorá tam má zrejme jediný koreň
$$
t=\sqrt{\tfrac35},\quad\text{odkiaľ}\quad
\tg\beta=\frac{2t}{1-t^2}=\sqrt{15}.
$$
Teda v~trojuholníku $BHA$ okrem $|BH|=1$ platí $|HA|=\tg\beta=\sqrt{15}$,
takže podľa Pytagorovej vety máme $|BA|=\sqrt{1+15}=4$.
Strany trojuholníka $ABC$ sú teda v~pomere $2:4:4$, čiže $1:2:2$.

\ineriesenie
a)
Označme vrcholy daného trojuholníka rovnako ako v~prvom riešení, ďalej $O$
stred opísanej kružnice, $B_0$~pätu výšky z~vrcholu~$B$ a~$B_1$ stred strany~$AC$.
Keďže os uhla pri vrchole~$B$ pretína oblúk~$CA$ opísanej
kružnice v~jeho strede~$M$ (\obr), je zrejmé, že vďaka podmienke $O\ne V$ (ktorá je
ekvivalentná s~tým, že daný trojuholník nie je rovnostranný) pretne
táto os stranu~$AC$ vnútri úsečky~$B_0B_1$, a~teda bod~$S$ leží vnútri úsečky~$TV$.
\insp{a61.12}%

\smallskip
b)
Využijeme známu vlastnosť troch základných bodov trojuholníka: ťažiska~$T$, stredu~$O$ opísanej
kružnice a~priesečníka~$V$ výšok. Uvedené tri body ležia totiž na priamke
v~ľubovoľnom trojuholníku, pričom ťažisko~$T$ vždy
delí úsečku~$OV$ v~pomere $1:2$.\footnote{Uvedená vlastnosť jednoducho vyplýva z~toho, že
trojuholník $A_1B_1C_1$ tvorený strednými priečkami daného trojuholníka $ABC$ je jeho obrazom
v~rovnoľahlosti so stredom v~ťažisku a~koeficientom~$\frac12$. A~keďže os každej zo strán
trojuholníka $ABC$ je zároveň výškou trojuholníka $A_1B_1C_1$, je stred~$O$ kružnice opísanej danému trojuholníku $ABC$
zároveň priesečníkom výšok trojuholníka $A_1B_1C_1$, takže bod~$O$ je v~uvedenej rovnoľahlosti obrazom
bodu~$V$ a~$|TO|=\frac12|TV|$.} Ak teda stred~$S$ kružnice vpísanej trojuholníku $ABC$ rozpoľuje
úsečku~$TV$, delia body $T$ a~$S$ (v~tomto poradí) orientovanú úsečku~$OV$ na tri zhodné časti.

Pre kolmé priemety $B_1$, $S_0$ a~$B_0$ bodov $O$, $S$ a~$V$ na stranu~$AC$ (\obr) preto platí
$$
CB_1=CB_0+3(CS_0-CB_0)=3CS_0-2CB_0.
$$
Ak uvedenú rovnosť chápeme ako rovnosť orientovaných úsečiek (alebo vektorov)
na priamke~$CA$, vyhneme sa nutnosti rozlišovať, či je uhol pri vrchole~$A$ väčší alebo menší
ako~$60\st$, pretože ako už vieme, na poradie spomenutých bodov to nemá vplyv.
\insp{a61.13}%

Do odvodenej rovnosti teraz môžeme dosadiť $\frac12b$ za $CB_1$, $\frac12a$ za $CS_0$
($|CS_0|=|CA_1|$ je dĺžka úsekov oboch dotyčníc z~vrcholu~$C$ ku kružnici vpísanej
trojuholníku $ABC$) a~napokon z~dvoch podobných pravouhlých trojuholníkov $BCB_0$ a~$ACA_1$
(zhodujú sa v~spoločnom uhle $BCA$) vychádza $CB_0=\frac12a^2/b$. Dostávame tak
$$
{b\over2}={3a\over2}-{a^2\over b},\quad\text{čiže}\quad b^2-3ab+2a^2=0.
$$
Poslednú rovnosť možno prepísať ako $(b-a)(b-2a)=0$, a~keďže $a\ne b$, musí byť $b=2a$.

\ineriesenie
Časť~a) už nebudeme znovu dokazovať, použijeme rovnaký postup aj označenie ako v~predošlom
riešení.

\smallskip
b)
Najskôr ukážeme, že v~trojuholníku, v~ktorom je pri vrchole~$A$ uhol väčší ako~$60\st$, nemôže
os uhla stredom úsečky~$VT$ vôbec prechádzať.
Na to využijeme známu vlastnosť priesečníka výšok: jeho obraz~$V'$ v~osovej súmernosti
podľa strany~$AC$ leží na kružnici~$k$ trojuholníku $ABC$ opísanej (\obr). Keďže za uvedeného predpokladu
ležia body $T$ a~$O$ (v~tomto poradí) na polpriamke~$AO$ až za bodom~$V$,
ležia zrejme obrazy $T'$, $O'$ bodov $T$, $O$
v~uvedenej osovej súmernosti vo vonkajšej oblasti kružnice~$k$. V~jej vonkajšej oblasti
leží však aj obraz~$B''$ vrcholu~$B$ v~stredovej súmernosti podľa stredu úsečky~$VT$:
bod~$B''$ je totiž priesečníkom polpriamok $TT'$ a~$CO'$, pretože
$CO'$ je zároveň kolmá na~$BC$ ($AOCO'$ je kosoštvorec) a~je tak
obrazom priamky~$AO$ v~rovnoľahlosti so stredom~$B$
a~koeficientom~$2$. Uhlopriečka~$BB''$ rovnobežníka $BTB''V$ (na ktorej leží ťažnica trojuholníka $BTV$)
preto určite pretne kružnicu~$k$ vnútri pásu rovnobežiek $BV$ a~$TT'$.
Os uhla pri vrchole~$B$ však pretína kružnicu~$k$ v~strede~$M$ príslušného oblúka~$AC$
a~priamka~$OM$ leží mimo spomenutého pásu.
\insp{a61.14}%

Predpokladajme teda, že v~danom trojuholníku je pri vrchole~$A$ uhol menší ako~$60\st{}$
a~že stred~$S$ vpísanej kružnice rozpoľuje úsečku~$VT$.
Označme $Q$ stred úsečky~$BT$, $S_0$~bod dotyku vpísanej kružnice so stranou~$AC$.
Teda $|SA_1|=|SS_0|$. Bod~$S_0$ leží zrejme na Tálesovej kružnici
s~priemerom~$QB_1$ a~zároveň bod~$A_1$ leží na Tálesovej kružnici s~priemerom~$BT$
(\obr), takže $|S_0T|=|TQ|=|QA_1|$. Trojuholníky $STS_0$ a~$SQA_1$
sa zhodujú v~dvoch stranách a~v~uhle oproti jednej z~nich.
Oba trojuholníky preto majú zhodné opísané kružnice a~pre ich uhly
pri vrcholoch $T$, resp.~$Q$ oproti zhodným tetivám $SS_0$ a~$SA_1$ zodpovedajúcich
kružníc platí, že sú buď zhodné, alebo doplnkové (\tj. ich súčet dáva $180\st$).
\insp{a61.16}%

Ak sú oba uhly zhodné, ležia body $A_1$, $S_0$, $T$, $Q$ na kružnici, a~keďže
$|QA_1|=|TS_0|$, je $A_1S_0TQ$ lichobežník alebo pravouholník, takže úsečka~$A_1S_0$ je
rovnobežná s~$QT$, a~je teda strednou priečkou trojuholníka $BCB_1$.
Trojuholník $A_1S_0C$ je rovnoramenný, preto
je rovnoramenný aj trojuholník $BB_1C$. Odtiaľ ihneď vyplýva, že $b=2a$.

Ukážeme teraz, že vďaka predpokladu $\a<60\st$ nemôže druhá možnosť nastať, teda
že súčet oboch uhlov $STS_0$ a~$SQA_1$ je menší ako $180\st$.
Na \obrr1{} sú vyznačené jednoduchým oblúčikom uhly, ktoré sa zhodujú s~uhlom $SQT$
(všade sa jedná o~súhlasné uhly alebo o~uhly pri základni rovnoramenného trojuholníka).
Podobne dva oblúčiky vyznačujú uhly zhodné s~uhlom $STQ$. Z~trojuholníka $STQ$ navyše vyplýva, že
$|\uhol SQT|+|\uhol STQ|=\beta>60\st$. Z~kombinácie uhlov pri bodoch $Q$, $T$ priamky~$BB_1$
vidíme, že na vyšetrovaný súčet uhlov $STS_0$ a~$SQA_1$ ostáva len $2\cdot180\st-3\b<180\st$.
Tým je požadovaná vlastnosť dokázaná.

\návody
Dokážte, že v~každom trojuholníku delí os uhla protiľahlú stranu v~pomere strán priľahlých. [Ak $D$ je priesečník strany $CA$ a osi uhla $CBA$, tak pomer~$q$ obsahov trojuholníkov $BCD$ a~$BAD$ možno vyjadriť dvoma spôsobmi: $q=|BC|:|BA|$ (lebo výšky spustené z~$D$ majú rovnakú veľkosť) a~zároveň $q=|CD|:|AD|$.]

V~rovnoramennom trojuholníku so základňou dĺžky~$a$ a~ramenami dĺžky~$b$ vyjadrite veľkosť polomeru vpísanej kružnice. [$\varrho=\frac12a\sqrt{4b^2-a^2}/(a+2b)$]

Dokážte platnosť súčtového vzorca $\tg(x+y)=(\tg x+\tg y)/(1-\tg x\cdot\tg y)$.

\D
Na odvesnách dĺžok $a$, $b$ pravouhlého trojuholníka ležia postupne stredy dvoch kružníc $k_a$, $k_b$. Obe kružnice sa dotýkajú prepony a~prechádzajú vrcholom oproti prepone. Polomery uvedených kružníc označme $\rho_a$, $\rho_b$. Určte najväčšie kladné reálne číslo~$p$ také, že nerovnosť
$$
\frac{1}{\rho_a}+\frac{1}{\rho_b} \ge p\Bigl(\frac{1}{a}+\frac{1}{b}\Bigr)
$$
platí pre všetky pravouhlé trojuholníky.
\vpravo{[58--A--II--2]}
\endnávod
}

{%%%%%   A-I-4
Aby sa dala jednoduchšie využiť podmienka celočíselnosti, súčet z~prvej vety zadania prepíšeme na jeden zlomok do tvaru
$$
\frac{mp-1}{q}+\frac{nq-1}{p}=\frac{p(mp-1)+q(nq-1)}{pq}.
$$
Čitateľ posledného zlomku je násobkom jeho menovateľa, takže je deliteľný ako prvočíslom~$p$, tak aj prvočíslom~$q$. Keďže prvý sčítanec čitateľa je násobkom~$p$, musí ním byť aj druhý sčítanec, teda $p\mid q(nq-1)$. Z~nesúdeliteľnosti prvočísel $p$, $q$ odtiaľ dostávame $p\mid nq-1$. Podobne máme $q\mid p(mp-1)$, čiže $q\mid mp-1$.

Prirodzené číslo $mp+(nq-1)$ (ktoré možno zapísať aj v~tvare $(mp-1)+nq$) je preto deliteľné ako číslom~$p$, tak aj číslom~$q$, a~teda aj (vďaka nesúdeliteľnosti $p$, $q$) súčinom~$pq$. Pre jeho veľkosť potom platí odhad
$$
mp+nq-1\ge pq,\quad\text{odkiaľ}\quad mp+nq>pq.
$$
Po vydelení oboch strán číslom $pq$ dostaneme nerovnosť, ktorú
sme mali dokázať.

\poznamka
Kľúčové tvrdenie $pq\mid mp+nq-1$ možno dokázať aj inak. Keďže $p\mid nq-1$ a~$q\mid mp-1$, nutne $pq\mid(nq-1)(mp-1)=mnpq+1-mp-nq$. Odtiaľ $pq\mid 1-mp-nq$.

\návody
Nájdite niekoľko štvoríc $m$, $n$, $p$, $q$ vyhovujúcich predpokladom zadania. [Vyhovuje ľubovoľná štvorica, pre ktorú sú oba zlomky $(mp-1)/q$, $(nq-1)/p$ celé čísla.]

Nech $a$, $b$, $c$ sú prirodzené čísla. Dokážte, že ak sú $a$, $b$ nesúdeliteľné a~$a\mid bc$, tak $a\mid c$. [Keďže $(a,b)=1$, existujú celé čísla $x$, $y$ také, že $ax+by=1$. Keďže $a\mid bc$, existuje $k$ také, že $ak=bc$. Odtiaľ $aky=bcy=c(1-ax)$, čiže $c=a(ky+cx)$, teda $a\mid c$.]

Pre celé čísla $a$, $b$, $c$, $d$ platí $b\mid a+c$, $a\mid b+d$. Dokážte, že $ab\mid ad+bc+cd$. [$ab\mid(a+c)(b+d)=ab+(ad+bc+cd)$]

\D
Určte všetky celé kladné čísla $m$, $n$ také, že $n$ delí $2m-1$
a~$m$ delí $2n-1$.
\vpravo{[59--A--II--3]}

Určte všetky dvojice $(m,n)$ kladných celých čísel, pre ktoré je číslo $4(mn+1)$ deliteľné číslom $(m+n)^2$.
\vpravo{[60--A--II--3]}

Nájdite všetky trojice navzájom rôznych prvočísel $p$, $q$,
$r$ spĺňajúce nasledujúce podmienky:
$$
\align
p&\deli q+r,\\
q&\deli r+2p,\\
r&\deli p+3q.
\endalign
$$
\vpravo{[55--A--III--5]}
\endnávod
}

{%%%%%   A-I-5
Označme $S_1$, $S_2$ stredy kružníc $k_1$, $k_2$. Nech $P$ je taký bod kružnice $k_1$, že $PS_2$ je jej priemerom. Ukážeme, že hľadaným pevným bodom je~$P$, \tj. dokážeme, že stred úsečky $KL$ leží s~bodmi $P$, $C$ na jednej priamke.

Aby úsečka~$BC$ preťala kružnicu~$k_1$, musí bod~$C$ ležať vnútri kratšieho oblúka~$AQ$ kružnice~$k_2$, pričom $S_1Q$ je priemerom~$k_2$. Bod~$L$ je potom vnútorným bodom kratšieho oblúka~$AB$ a~bod~$K$ vnútorným bodom kratšieho oblúka~$PA$ kružnice~$k_1$ (\obr).
\insp{a61.10}%

Keďže kružnice majú rovnaké polomery, sú trojuholníky $S_1S_2A$, $S_1S_2B$ rovnostranné a~veľkosť stredového uhla $BS_1A$ je $120\st$. Príslušný obvodový uhol $BPA$ má preto veľkosť $60\st$. Navyše body $A$, $B$ sú súmerne združené podľa priamky $PS_2$, takže $|PA|=|PB|$ a~trojuholník $ABP$ je rovnostranný\footnote{To ihneď vyplýva aj z~toho, že $P$, $B$, $S_2$, $A$ sú štyri zo šiestich vrcholov pravidelného šesťuholníka vpísaného do $k_1$.}. Všetky obvodové uhly nad zhodnými tetivami $PA$, $PB$, $AB$ majú teda veľkosť $60\st$ (ak vrchol leží na dlhšom oblúku), resp. $120\st$ (ak vrchol leží na kratšom oblúku). Pri tetive~$AB$ to platí aj pre obvodové uhly na kružnici~$k_2$, keďže obe kružnice sú zhodné.

Z~uvedeného dostávame
$$
|\uhol ACB|=60\st,\qquad |\uhol PLB|=60\st,\qquad|\uhol PKA|=120\st.
$$
Z~rovnosti prvých dvoch uhlov vyplýva rovnobežnosť priamok $PL$ a~$KC$ a~z~toho, že súčet prvého a~tretieho uhla je $180\st$, vyplýva rovnobežnosť priamok $PK$ a~$LC$. Štvoruholník $PLCK$ je teda rovnobežník, z~čoho už priamo vyplýva dokazované tvrdenie (uhlopriečky rovnobežníka sa rozpoľujú, takže priamka~$PC$ prechádza stredom úsečky~$KL$).

\ineriesenie
Označme body rovnako ako v~prvom riešení. Odlišným spôsobom dokážeme, že $PLCK$ je rovnobežník.
\insp{a61.11}%

Body $A$, $B$ sú súmerne združené podľa priamky $PS_2$, preto s~využitím vlastností obvodových a~stredových uhlov dostávame
$$
|\uhol PLB|=|\uhol PS_2B|=\frac12|\uhol AS_2B|=|\uhol ACB|,
$$
odkiaľ vyplýva $PL\parallel KC$. Štvoruholník $PLAK$ je teda lichobežník, a~keďže je tetivový (má opísanú kružnicu $k_1$), musí byť rovnoramenný. Jeho uhlopriečky $KL$, $PA$ sú teda zhodné, a~keďže z~vyššie spomenutej súmernosti máme $|PA|=|PB|$, platí tiež $|KL|=|PB|$. Štvoruholník $KPBL$ je tetivový a~jeho protiľahlé strany $KL$, $PB$ sú zhodné, takže to tiež musí byť rovnoramenný lichobežník\footnote{Vyplýva to napríklad z~rovnosti obvodových uhlov $PLB$, $KPL$ nad zhodnými tetivami $PB$, $KL$, alebo jednoducho zo súmernosti podľa osi úsečky~$BL$.} (\obr). Z~toho už dostávame $PK\parallel LC$.

\poznamka
Zadané tvrdenie platí, aj keď pripustíme, že kružnice $k_1$, $k_2$ majú rôzne polomery, pričom $S_2$ leží na $k_1$; v~druhom uvedenom riešení sme nikde nevyužili zhodnosť kružníc. V~takom prípade však bod~$K$ môže ležať aj mimo kratšieho oblúka~$PA$, takže treba osobitne rozlíšiť dva prípady.

\návody
Dokážte, že každý tetivový lichobežník je rovnoramenný. [Ak $PQRS$ je tetivový lichobežník so základňou $PQ$, tak zo striedavých uhlov máme $|\uhol QPR|=|\uhol SRP|$. Teda obvodové uhly nad tetivami $QR$, $PS$ majú rovnakú veľkosť a~tetivy $QR$, $PS$ musia byť zhodné. Iný spôsob: Os každej tetivy prechádza stredom kružnice, preto os strany~$PQ$ je totožná s~osou strany~$RS$ (sú rovnobežné a~prechádzajú spoločným bodom) a~podľa tejto osi sú úsečky $PS$, $QR$ súmerne združené, čiže zhodné.]

Dokážte, že v~štvoruholníku sa obe uhlopriečky rozpoľujú práve vtedy, keď je to rovnobežník. [Ak sa v~štvoruholníku $ABCD$ uhlopriečky rozpoľujú v~bode $S$, tak trojuholníky $ABS$, $CDS$ sú zhodné a~zo striedavých uhlov $AB\parallel CD$, analogicky $BC\parallel AD$. Ak $ABCD$ je rovnobežník s~priesečníkom uhlopriečok~$S$, tak zo striedavých uhlov vyplýva zhodnosť trojuholníkov $ABS$, $CDS$, \tj. zhodnosť úsečiek $BS$, $SD$, resp. $AS$, $SC$.]

\D
Daný je tetivový štvoruholník $ABCD$. Dokážte, že spojnica priesečníkov výšok trojuholníka $ABC$ s~priesečníkom výšok trojuholníka $ABD$ je rovnobežná s~priamkou~$CD$.
\vpravo{[58--A--I--2]}

Je daná kružnica~$k$ s~tetivou~$AC$, ktorá nie je priemerom. Na jej dotyčnici vedenej bodom~$A$ zvolíme bod $X\ne A$ a~označíme $D$ priesečník kružnice~$k$ s~vnútrom úsečky~$XC$ (ak existuje). Trojuholník $ACD$ doplníme na lichobežník $ABCD$ vpísaný do kružnice~$k$. Určte množinu priesečníkov priamok $BC$ a~$AD$ prislúchajúcich všetkým takým lichobežníkom.
\vpravo{[59--A--III--4]}
\endnávod
}

{%%%%%   A-I-6
Pre $k=2$ sa dá nerovnosť po vykrátení upraviť na tvar $\frac12(a+b)\ge\sqrt{ab}$, čo je známa nerovnosť medzi aritmetickým a~geometrickým priemerom platná pre ľubovoľné kladné $a$, $b$. Hľadané najväčšie $k$ je teda určite aspoň~$2$. Skúmajme ďalej danú nerovnosť len za predpokladu $k\ge2$.

Ekvivalentnou úpravou (keďže $k+2>0$) dostaneme
$$
2(a^2+kab+b^2)\ge(k+2)(a+b)\sqrt{ab}
$$
a~po vydelení oboch strán členom $b^2$ máme
$$
2\left(\frac{a^2}{b^2}+k\frac ab +1\right)\ge(k+2)\left(\frac ab+1\right)\sqrt{\frac ab}.
$$
Označme $\sqrt{a/b}=x$. Zrejme $x$ môže nadobudnúť ľubovoľnú kladnú hodnotu. Ďalej sa preto stačí zaoberať nerovnosťou
$$
2(x^4+kx^2 +1)\ge(k+2)(x^2+1)x
$$
a~hľadať najväčšie $k$ také, že je splnená pre každé kladné~$x$. Po jednoduchých úpravách smerujúcich k~osamostatneniu $k$ dostávame
$$
\align
k\bigl((x^2+1)x-2x^2\bigr) &\le 2\bigl(x^4+1-(x^2+1)x\bigr),\\
k(x^3-2x^2+x) &\le 2(x^4-x^3-x+1),\\
kx(x^2-2x+1) &\le 2\bigl(x^3(x-1)-(x-1)\bigr),\\
kx(x-1)^2 &\le 2(x-1)^2(x^2+x+1).
\endalign
$$
pre $x=1$ je posledná nerovnosť splnená vždy. Pre $x\ne1$ nerovnosť vydelíme kladným výrazom $x(x-1)^2$ a~získame priamo ohraničenie pre $k$:
$$
k\le \frac{2(x^2+x+1)}x=2+2\left(x+\frac1x\right).
\tag1
$$
Pre kladné $x$ je $x+1/x\ge2$, pričom rovnosť platí jedine pre $x=1$. Pre $x\ne1$ výraz $x+1/x$ nadobúda všetky hodnoty z~intervalu $(2,\infty)$. Teda pravá strana v~\thetag1 nadobúda všetky hodnoty z~intervalu $(6,\infty)$. Z~toho je jasné, že najväčšie $k$ také, že \thetag1 platí pre všetky kladné $x\ne1$, je $k=6$.

\ineriesenie
Zadanú nerovnosť ekvivalentne upravíme:
$$
\align
\frac2{k+2}\cdot\frac{(a+b)^2+(k-2)ab}{a+b} &\ge \sqrt{ab},\\
\frac2{k+2}\left(\frac{a+b}{\sqrt{ab}}+(k-2)\frac{\sqrt{ab}}{a+b}\right) &\ge 1.\tag2
\endalign
$$
Označme $x=(a+b)/\sqrt{ab}$. Potom $x\ge2$, pretože $\frac12(a+b)\ge\sqrt{ab}$. Úpravou \thetag2 za podmienky $k+2>0$ získame
$$
\align
\frac2{k+2}\left(x+(k-2)\frac1x\right) &\ge 1,\\
x^2-\frac{k+2}2x+(k-2) &\ge 0.\tag3
\endalign
$$
Kvadratická funkcia na ľavej strane poslednej nerovnosti má pre každé~$k$ koreň $x=2$ a~jej koeficient pri kvadratickom člene je kladný. Takže \thetag3 platí pre každé $x\ge2$ práve vtedy, keď je vrchol paraboly tejto funkcie na číselnej osi naľavo od bodu~$2$, teda práve vtedy, keď
$$
\frac{k+2}4\le2,\qquad\text{čiže}\quad k\le6.
$$

\zaver
Hľadaná najväčšia hodnota je $k=6$.

\návody
Určte, aké hodnoty nadobúda výraz $x+1/x$ pre $x>0$. [Keďže $(\sqrt x-1/\sqrt x)^2\ge0$, máme $x+1/x\ge2$. Rovnosť nastáva pre $x=1$. Výraz nadobudne aj všetky hodnoty väčšie ako $2$, pretože rovnica $x+1/x=p$ má pre $p>2$ dva kladné korene $\frac12p\pm\frac12\sqrt{p^2-4}$.]

Určte všetky hodnoty parametra~$p$, pre ktoré nadobúda kvadratická funkcia $f(x)=x^2+px+p-1$ na obore kladných čísel len kladné hodnoty. [Keďže $f(x)=(x+1)(x+p-1)$, koreňmi funkcie sú $\m1$ a~$1-p$. Zadaná podmienka je splnená práve vtedy, keď ani jeden z~koreňov nie je kladný, teda keď $p\ge1$.]

\D
Dokážte, že pre ľubovoľné kladné reálne čísla $a$, $b$ platí
$$
\sqrt{ab}\le{2(a^2+3ab+b^2)\over5(a+b)}\le{a+b\over2},
$$
a~pre každú z~oboch nerovností zistite, kedy prechádza na rovnosť.
\vpravo{[59--C--I--5]}

Dokážte, že pre ľubovoľné rôzne kladné čísla $a$, $b$ platí
$$
\frac{a+b}{2}<\frac{2(a^2+ab+b^2)}{3(a+b)}<\sqrt{\frac{a^2+b^2}{2}}.
$$
\vpravo{[58--C--I--6]}

\endnávod
}

{%%%%%   B-I-1
Uvažované desaťciferné čísla označme
$\overline{a_9a_8a_7a_6a_5a_4a_3a_2a_1a_0}$, pričom $a_9$, $a_8$, \dots, $a_0$ sú
navzájom rôzne cifry, teda všetky cifry $0$, $1$, $2$, \dots, $9$ v~nejakom poradí.
Ďalej označme $s_2=a_0+a_2+a_4+a_6+a_8$ súčet jeho cifier
na párnych\footnote{Miesta číslujeme podľa mocnín desiatky v~dekadickom zápise; pre riešenie
samozrejme nie je podstatné, ktoré miesta označíme za párne a~ktoré za nepárne, dôležité je
len to, že sa párne a~nepárne miesta striedajú.}
miestach a~$s_1=a_1+a_3+a_5+a_7+a_9$ súčet cifier na nepárnych miestach.

Na zistenie deliteľnosti jedenástimi použijeme známe kritérium: Číslo
$\overline{a_9a_8\dots a_1a_0}$ je deliteľné jedenástimi práve vtedy,
keď je jedenástimi deliteľný príslušný rozdiel $s_2-s_1$.
Zrejme $|s_2-s_1|\le(9+8+7+6+5)-(4+3+2+1+0)=25$, čiže $\m25 \le s_2-s_1 \le 25$.
Súčet $s_2+s_1=0+1+2+\dots+9=45$ je nepárne číslo, preto musí byť nepárne aj číslo
$s_2-s_1$. Pre vyhovujúce číslo môžu teda nastať dve možnosti: $s_2-s_1=\m11$ alebo $s_2-s_1=11$.

V~prvom prípade zo sústavy rovníc $s_2+s_1=45$, $s_2-s_1=11$ dostaneme
$s_2=28$, $s_1=17$, v~druhom naopak $s_2=17$, $s_1=28$.

Číslo~$17$ rozpíšeme všetkými možnými spôsobmi na súčet piatich navzájom rôznych
cifier:
$$
\align
17=&9+5+2+1+0=9+4+3+1+0=\\
  =&8+6+2+1+0=8+5+3+1+0=8+4+3+2+0=\\
  =&7+6+3+1+0=7+5+4+1+0=7+5+3+2+0=7+4+3+2+1=\\
  =&6+5+4+2+0=6+5+3+2+1.
\endalign
$$

Medzi desaťcifernými číslami zapísanými všetkými desiatimi ciframi sú určite najväčšie
tie, ktoré začínajú ciframi $987$ alebo dokonca $9876$. Vzhľadom na nájdené
rozklady čísla~$17$ to zrejme nemožno dosiahnuť pre $s_1=17$, zato pre $s_2=17$ áno:
stačí za~$s_2$ zobrať súčet $17=8+6+2+1+0$, čo je zároveň jediná možnosť.
Ostatné cifry už na základe tejto voľby doplníme jednoznačne tak, aby sme dostali
číslo čo najväčšie. Hľadané najväčšie číslo je teda $9\,876\,524\,130$.
%% $s_2=28$, $s_1=17$ uvažujeme součty $s_1=9+5+2+1+0$ a $s_1=9+4+3+1+0$. Z
%% obou součtů je vidět, že na druhém místě zleva bude číslice $8$. Na třetím
%% místě zleva bude číslice $5$. Hledané největší číslo tedy může být
%% $9\,857\,261\,403$. V případě $s_2=17$, $s_1=28$ uvažujeme součty $s_2=8+6+2+1+0$,
%% $s_2=8+5+3+1+0$ a $s_2=8+4+3+2+0$. Z těchto tří součtů je vidět, že na
%% druhém místě zleva bude číslice $8$. Na třetím místě zleva bude číslice~$7$,
%% na čtvrtém místě zleva číslice $6$. Hledané největší číslo tedy může být
%% $9\,876\,524\,130$. Ze dvou získaných desetimístných čísel je větší číslo
%% $9\,876\,524\,130$.

Najmenšie číslo nájdeme analogickým postupom. Keďže $a_9\ne0$, sú medzi
uvažovanými číslami určite najmenšie tie, ktoré začínajú ciframi $102$. Z~nájdených
rozkladov čísla~$17$ opäť vidíme, že to možno dosiahnuť jedine voľbou $s_1=17=6+5+3+2+1$.
Tomu potom zodpovedá číslo (keďže poznáme všetky jeho cifry na nepárnych aj párnych miestach,
je ich usporiadanie určené požiadavkou, aby výsledné číslo bolo najmenšie)
$1\,024\,375\,869$.

%% Hledané nejmenší číslo bude zřejmě mít zleva postupně číslice $1, 0$. Proto
%% v případě $s_2=28$, $s_1=17$ uvažujeme součet $s_1=6+5+3+2+1$. Hledané
%% největší číslo tedy může být $1\,024\,375\,869$. V případě $s_2=17$,
%% $s_1=28$ uvažujeme součty $s_2=7+5+3+2+0$ a $s_2=6+5+4+2+0$. Z obou součtů
%% je vidět, že na třetím místě zleva bude číslice $3$. Hledané nejmenší číslo
%% tedy může být $1\,032\,748\,596$. Ze dvou získaných desetimístných čísel je
%% menší číslo $1\,024\,375\,869$.

\návody
Dokážte spomenuté kritérium deliteľnosti jedenástimi, \tj. že celé číslo je deliteľné jedenástimi
práve vtedy, keď je jedenástimi deliteľný súčet jeho cifier braných striedavo so znamienkom
plus a~mínus. [Kritérium vyplýva z~toho, že $10$ dáva po delení jedenástimi rovnaký zvyšok
ako~$\m1$, teda jednotlivé rády $10^n$ dávajú zvyšok~$(-1)^n$.]

Dokážte, že žiadne desaťciferné číslo zložené z~navzájom rôznych cifier,
v~ktorého dekadickom zápise sa striedajú párne a~nepárne cifry, nie je deliteľné
jedenástimi.

Určte počet päťciferných čísel zložených z~navzájom rôznych a) nepárnych,
b)~párnych cifier a~deliteľných jedenástimi. [a) 0, b) 16]

Bez delenia ukážte, že číslo $20\,111\,102$ je deliteľné jedenástimi. Potom k~nemu nájdite
najbližšie menšie a~najbližšie väčšie číslo deliteľné jedenástimi zložené
z~rovnakých cifier ako dané číslo. [menšie $20\,110\,211$, väčšie $20\,111\,201$]

Dokážte, že platí: Číslo
$\overline{a_9a_8a_7a_6a_5a_4a_3a_2a_1a_0}$ je deliteľné jedenástimi práve vtedy,
keď je deliteľné jedenástimi číslo
$\overline{a_9a_8}+\overline{a_7a_6}+\overline{a_5a_4}+\overline{a_3a_2}+\overline{a_1a_0}$.
\endnávod
}

{%%%%%   B-I-2
Označme úsečky (a~ich dĺžky) ako na \obr. Keďže $DAF$
a~$EBF$ sú pravouhlé rovnoramenné trojuholníky, majú uhly pri ich
preponách veľkosť~$45^{\circ}$, takže $|\uhol DFE|=90\st$ a~trojuholník $DEF$ je pravouhlý
s~odvesnami, ktoré sú zároveň preponami oboch rovnoramenných pravouhlých trojuholníkov.
Pre obsahy $P$ a~$Q$ oboch uvažovaných trojuholníkov preto platí
$$
P=\frac12(c_a+c_b)v \qquad \hbox{a} \qquad
Q=\frac12 \cdot c_a\sqrt{2}\cdot c_b\sqrt{2}.
$$
% kde $c_a=|AF|$, $c_b=|BF|$.
\insp{b61.1}%

Podľa Euklidovej vety o~výške v~danom pravouhlom trojuholníku
% $ABC$
platí  $v=\sqrt{c_ac_b}$. Na dôkaz danej nerovnosti stačí teda overiť,
že
$$
\frac12(c_a+c_b)\sqrt{c_ac_b}\ge \frac12 \cdot c_a\sqrt{2}\cdot c_b\sqrt{2}.
$$

Po jednoduchej (ekvivalentnej) úprave dostaneme
$$
c_a+c_b\geq 2\sqrt{c_ac_b}, \qquad \hbox{čiže} \qquad
(\sqrt{c_a}-\sqrt{c_b})^2\ge 0.
$$
Keďže posledná nerovnosť očividne platí, je dôkaz tvrdenia ukončený.
Rovnosť pritom nastane práve vtedy, keď $c_a=c_b$, \tj. práve vtedy, keď je daný
pravouhlý trojuholník $ABC$ rovnoramenný.

\ineriesenie
Rovnako ako v~prvom riešení vyjdeme zo zrejmého poznatku, že trojuholník $DEF$ je pravouhlý.
Odvesny oboch uvažovaných pravouhlých trojuholníkov majú rovnaké kolmé priemety na priamku~$AB$, pritom
$$
|AF|=|AC|\cos\g_1=|DF|\cos45\st,\quad
|BF|=|BC|\cos\g_2=|EF|\cos45\st,
$$
kde $\g_1$, $\g_2$ označujú zodpovedajúce časti pravého uhla pri vrchole~$C$, takže
$\g_2=90\st-\g_1$. Keďže $\cos45\st=\frac12\sqrt2$, vyplýva odtiaľ pre dvojnásobky
oboch obsahov
$$
\align
2P=&|AC|\cdot|BC|=|DF|\cdot|EF|\cdot\frac{\cos45\st}{\cos\g_1}\cdot\frac{\cos45\st}{\cos\g_2}=\\
  =&2Q\cdot\frac1{2\cos\g_1\sin\g_1}=2Q\cdot\frac1{\sin2\g_1}\ge 2Q.
\endalign
$$

Rovnosť $P=Q$ zrejme nastane práve vtedy, keď $\sin2\g_1=1$, čiže $\g_1=\g_2=45\st$, teda práve vtedy,
keď je daný trojuholník $ABC$ rovnoramenný.

\návody
Dokážte, že pre každé dve kladné reálne čísla $a$, $b$ platí nerovnosť
$\sqrt{\frac{a\vphantom b}{b}}+\sqrt{\frac{b}{a}}\ge 2$.

V~obdĺžniku $ABCD$ s~dĺžkami strán $|AB|=a$, $|BC|=b$ označme $E$ pätu
kolmice spustenej z~vrcholu~$B$ na uhlopriečku~$AC$. Určte dĺžky úsečiek $AE$,
$CE$, $BE$. [$|AE|=\frac{a^2}{\sqrt{a^2+b^2}}$,
$|CE|=\frac{b^2}{\sqrt{a^2+b^2}}$, $|BE|=\frac{ab}{\sqrt{a^2+b^2}}$]

V~pravouhlom trojuholníku $ABC$ s~preponou~$AB$ je $E$ päta výšky z~vrcholu~$C$,
$D$~päta výšky z~bodu~$E$ na stranu~$AC$ a~$F$ päta výšky z~bodu~$E$ na
stranu~$BC$. Dokážte, že obsah štvoruholníka $CDEF$ je nanajvýš rovný
polovici obsahu trojuholníka $ABC$. Kedy nastane rovnosť?
% [Rozměry pravoúhelníku jsou $x=\frac{c_av_c}{a}, y=\frac{c_bv_c}{b}$,
% máme dokázat nerovnost $\frac{(c_ac_b)^2}{ab}\leq \frac{ab}{4}$.]
[Keďže trojuholníky $AED$ a~$EBF$ sú podobné trojuholníku $ABC$ s~koeficientmi podobnosti
$\a$ a~$1-\a$, je obsah pravouholníka $CDEF$ rovný $S-(\a^2+(1-\a)^2)S=2\a(1-\a)S$,
pričom $S$ označuje obsah daného trojuholníka $ABC$. Požadovaná nerovnosť je tak ekvivalentná
s~nerovnosťou $\a(1-\a)\le\frac14$, čiže $(2\a-1)^2\ge0$.]
\endnávod
}

{%%%%%   B-I-3
Z~druhej rovnice vyplýva, že $\lfloor x \rfloor\ne0$, a~$y=\frc 8{\lfloor x \rfloor}$
je tým pádom nenulové číslo v~absolútnej hodnote nie väčšie ako~$8$.
Po dosadení do prvej rovnice dostaneme rovnicu
$$
x\cdot\Bigl\lfloor {\frac 8{\lfloor x \rfloor} \Bigr\rfloor} =7, \tag1
$$
%%
%% Rovnice (1), která je důsledkem zadané soustavy a kterou budeme vzápětí
%% řešit, je s touto soustavou ve skutečnosti ekvivalentní v následujícím
ktorá je v~skutočnosti s~danou sústavou ekvivalentná v~nasledujúcom
zmysle: ak priradíme ľubovoľnému riešeniu~$x$ rovnice~\thetag1 hodnotu
$y=8/\lfloor x \rfloor$, bude zrejme dvojica $(x,y)$ riešením pôvodnej sústavy.

Budeme preto postupne hľadať riešenia rovnice~\thetag1 pre jednotlivé hodnoty
celých čísel $a=\bigl\lfloor8/\lfloor x \rfloor\bigr\rfloor\in\{\m8,\dots,\m1,1,\dots,8\}$
tak, že vypočítame $x=7/a$, $y=8/\lfloor x \rfloor$ a~overíme, či $\lfloor y\rfloor=a$.
Navyše je vzhľadom na nerovnosť $\lfloor x \rfloor\ne0$ z~rovnice~\thetag1 zrejmé, že $a\ne8$.

{\openup 3pt
%\obeylines
Pre $a=\m8$ je $x=\m\frac78$, $\lfloor x\rfloor=\m1$ a~$y=\m8$, teda $\lfloor y\rfloor=a$.

Pre $a=\m7$ je $x=\m1=\lfloor x\rfloor$ a~$y=\m8$, teda $\lfloor y\rfloor<a$.

Pre $a=\m6$ je $x=\m\frac76$, $\lfloor x\rfloor=\m2$ a~$y=\m4$, teda $\lfloor y\rfloor>a$.

Pre $a=\m5$ je $x=\m\frac75$, $\lfloor x\rfloor=\m2$ a~$y=\m4$, teda $\lfloor y\rfloor>a$.

Pre $a=\m4$ je $x=\m\frac74$, $\lfloor x\rfloor=\m2$ a~$y=\m4$, teda $\lfloor y\rfloor=a$.

Pre $a=\m3$ je $x=\m\frac73$, $\lfloor x\rfloor=\m3$ a~$y=\m\frac83$, teda $\lfloor y\rfloor=a$.

Pre $a=\m2$ je $x=\m\frac72$, $\lfloor x\rfloor=\m4$ a~$y=\m2$, teda $\lfloor y\rfloor=a$.

Pre $a=\m1$ je $x=\m7=\lfloor x\rfloor$ a~$y=-\frac87$, teda $\lfloor y\rfloor<a$.

Pre $a=1$ je $x=7=\lfloor x\rfloor$ a~$y=\frac87$, teda $\lfloor y\rfloor=a$.

Pre $a=2$ je $x=\frac72$, $\lfloor x\rfloor=3$ a~$y=\frac83$, teda $\lfloor y\rfloor=a$.

Pre $a=3$ je $x=\frac73$, $\lfloor x\rfloor=2$ a~$y=4$, teda $\lfloor y\rfloor>a$.

Pre $a\in\{4,5,6,7\}$ je $\lfloor x\rfloor=1$ a~$y=8$, teda $\lfloor y\rfloor>a$.
}

\zaver
Sústava rovníc má 6 riešení, sú nimi usporiadané dvojice
$\left(-\frac78,-8\right)$,
$\left(-\frac74,-4\right)$,
$\left(-\frac73,-\frac83\right)$,
$\left(-\frac72,-2\right)$,
$\left(7,\frac87\right)$
a~$\left(\frac72,\frac83\right)$.

\návody
V~obore reálnych čísel riešte rovnicu: a) ${\lfloor x \rfloor}^2=4$,
b)~$\lfloor x^2 \rfloor=4$, c)~$\bigl\lfloor{\frac {\lfloor x\rfloor
+3}{2}\bigr\rfloor}=4$, d)~$\bigl\lfloor{\frac {2011}{\lfloor
x\rfloor}\bigr\rfloor}=4$. [a)~$x\in\langle-2,-1)\cup\langle2,3)$,
b)~$2\leq |x|<\sqrt{5}$, c)~$x\in\langle5,7)$, d)~$x\in\langle403,503)$]

V~obore reálnych čísel riešte sústavu rovníc $xy=2$, $x\left\lfloor y
\right\rfloor =4$. [Zrejme $x<0$, $y<0$. Dokážte ďalej, že
$\lfloor-u\rfloor=\m\lfloor u\rfloor$ pre každé celé
číslo a~$\lfloor\m u\rfloor=\m\lfloor u\rfloor-1$ inak.
Dobre je to tiež vidno z~grafu funkcie $y=\lfloor x \rfloor$.
Vyjde $x=\m4$, $y=\m\frac{1}{2}$.]

Dokážte, že pre každé reálne číslo~$x$ a~každé celé číslo~$k$ platí
$\left\lfloor x+k \right\rfloor =\left\lfloor x \right\rfloor +k$.
\endnávod
}

{%%%%%   B-I-4
Označme $S$ priesečník uhlopriečok $AC$ a~$BD$, ktorý má ležať na
priamke $b=PB$. Pritom nemôže byť $S=P$, pretože potom by na priamke~$a$ ležal aj vrchol~$C$.
Taká možnosť odporuje zadaniu.

Preto ak zvolíme na priamke~$b$ ľubovoľný bod~$S'$, $S'\ne P$, existuje práve
jedna rovnoľahlosť so stredom~$P$, ktorá zobrazí bod~$S$  na~$S'$. V~tejto
rovnoľahlosti sa pravouholník $ABCD$ zobrazí na pravouholník $A'B'C'D'$
s~priesečníkom uhlopriečok~$S'$, pritom $A'\in a$, $B',D'\in b$ a~$C'\in c$.
Keďže vrcholy $A'$, $C'$ sú súmerne združené podľa zvoleného stredu~$S'$
(\obr), zostrojíme bod~$A'$ ako priesečník priamky~$a$ s~priamkou~$c'$, ktorá je
súmerne združená s~priamkou~$c$ podľa stredu~$S'$. Potom už ľahko z~bodov
$A'$, $S'$ určíme bod~$C'$ a~napokon~-- vďaka pravým uhlom $A'B'C'$
a~$A'D'C'$~-- nájdeme body $B'$, $D'$ ako priesečníky priamky~$b$ s~Tálesovou kružnicou nad priemerom~$A'C'$.
Pritom tieto dva priesečníky môžeme označiť ako $B'$, $D'$ v~ľubovoľnom poradí
s~výnimkou prípadu, keď jeden z~priesečníkov splynie s~bodom~$P$; v~takom prípade môže
byť jedine $D'=P$, lebo z~$B\ne P$ vyplýva $B'\ne P$. Nakoniec zobrazíme pravouholník
$A'B'C'D'$ v~\uv{spätnej} rovnoľahlosti, v~ktorej $B'\mapsto B$. Tak
dostaneme štvoruholník $ABCD$, ktorý má zrejme všetky požadované vlastnosti.
\insp{b61.2}%

\diskusia
Pre zvolený bod $S'\in b$, $S'\ne P$, body $A'$ a~$C'$
existujú a~sú jediné (priamky $a$, $c'$ sú totiž rôznobežky a~žiadna z~nich
stredom súmernosti~$S'$ neprechádza). Kružnica nad priemerom~$A'C'$ má kladný
polomer, a~preto má s~priamkou~$b$ prechádzajúcou jej stredom~$S'$ vždy dva
priesečníky. Ak sú oba rôzne od bodu~$P$, má úloha dve riešenia. Jeden
z~týchto dvoch priesečníkov splynie s~bodom~$P$ práve vtedy, keď bude uhol $A'PC'$
pravý, teda práve vtedy, keď dané priamky $a$, $c$ budú navzájom kolmé. V~takom prípade bude
$D'=P$ a~úloha bude mať jediné riešenie (vrchol~$D$ splynie s~bodom~$P$).

\návody
Sú dané dve rôznobežky $a$, $c$ a~bod~$S$ neležiaci na žiadnej z~nich.
Zostrojte štvorec $ABCD$ so stredom~$S$ tak, aby bod~$A$ ležal na priamke~$a$
a~bod~$C$ na priamke~$c$. [Zostrojíme priamku~$a'$ ako obraz priamky~$a$
v~stredovej súmernosti so stredom~$S$, prienik priamok $a'$, $c$ dáva bod~$C$.]

Sú dané dve rôznobežky $a$, $c$, ktorých priesečník~$P$ je
mimo výkresu, a~bod~$B$ neležiaci na žiadnej z~nich. Zostrojte priamku~$b$
prechádzajúcu bodmi $B$, $P$. [Zostrojíme ľubovoľný trojuholník $ABC$, kde
$A\in a$ a~$C\in c$, a~potom zostrojíme trojuholník $A'B'C'$, ktorý bude jeho obrazom
v~nejakej rovnoľahlosti so stredom v~bode~$P$.]

Daná je úsečka~$AB$. Zostrojte pravouhlý trojuholník $ABC$ s~preponou~$AB$ tak, aby $|AC|=2\cdot |BC|$. [Zostrojíme trojuholník
$A'B'C'$ s požadovanými vlastnosťami a~potom pomocou rovnoľahlosti
(napr. so stredom v~jednom z~vrcholov) zostrojíme trojuholník, ktorého
prepona bude mať dĺžku~$|AB|$.]
\endnávod
}

{%%%%%   B-I-5
a) Nech $m$ je počet klebetníkov. Keďže každý klebetník je v~spojení s~tromi klebetnicami, je medzi všetkými celkom $3m$~spojení. A~keďže k~rovnakému výsledku musíme dôjsť, keď spočítame všetky spojenia jednotlivých klebetníc, z~ktorých každá
je v~spojení s~tromi klebetníkmi, je klebetníc tiež~$m$.

\smallskip
b) Predpokladajme, že po odsťahovaní jedného z~klebetníkov sa sieť rozpadne
na niekoľko súvislých častí.
To znamená, že odsťahovaný klebetník bol v~spojení s~aspoň jednou
klebetnicou v~každej zo vzniknutých častí, inak by príslušná časť
nebola prepojená so zvyškom siete už pred jeho odchodom. Odtiaľ je ďalej
zrejmé, že vzniknuté časti sú nanajvýš tri, pričom počet klebetníc, ktoré boli
v~spojení s~odsťahovaným klebetníkom, musí v~každej z~nich byť 1 alebo~2.

Uvažujme ľubovoľnú z~častí, na ktoré sa sieť rozpadla, a~označme $m$ a~$n$
prislúchajúce počty klebetníkov a~klebetníc v~tejto časti. Ak teraz
spočítame počet spojení všetkých klebetníkov v~tejto časti, dostaneme~$3m$.
Vzhľadom na to, že jedna alebo dve klebetnice o~jedno spojenie prišli,
je celkový počet ich spojení s~klebetníkmi $3n-2$ alebo $3n-1$. Ani
jedno z~týchto čísel však nie je deliteľné tromi, preto sa nemôže nikdy
rovnať celkovému počtu spojení klebetníkov vo zvolenej časti. To je spor,
ktorý dokazuje tvrdenie~b) úlohy.

\návody
V~sieti na šírenie klebiet je $m$~klebetníkov a~$n$~klebetníc. Každý
z~klebetníkov je v~spojení s~$a$~klebetnicami a~každá klebetnica je v~spojení s~$b$~klebetníkmi.
Inak sa klebety nešíria. Aký je vzťah medzi premennými $a$, $b$, $m$, $n$? [$ma=nb$]

Vytvorte model súvislej siete opísanej v~zadaní úlohy pre 3, 4, 5,~{\dots}
klebetníkov a~klebetníc. Ukážte v~tomto modeli, že po odstránení
ktoréhokoľvek klebetníka zostane sieť súvislá.

Pre aký počet klebetníkov a~klebetníc môže byť sieť opísaná v~zadaní úlohy nesúvislá? [pre 6, 7, 8,~{\dots}]

V~súvislej sieti na šírenie klebiet je každý klebetník v~spojení s~aspoň a)
jedným, b) dvoma ďalšími klebetníkmi. Zostane sieť súvislá, ak sa jeden z~nich odsťahuje?
[a) aj~b): môže, ale nemusí zostať súvislá, záleží na tvare siete]
\endnávod
}

{%%%%%   B-I-6
Opísaná hra je zrejme spravodlivá v~tom zmysle, že obaja hráči majú rovnaký
počet možností ako vyhrať. Aby sme zistili požadovaný počet, stačí zistiť,
koľkými spôsobmi môže nastať remíza, teda jeden z~výsledkov $0:0$, $1:1$ a~$2:2$.

Prípad $0:0$ nastane, ak obaja hráči vyložia v~každom kole rovnaké karty.
Takých možností je $5!=1\cdot2\cdot3\cdot4\cdot5$.

Výsledok $1:1$ znamená, že hráči vyložia rovnaké karty v~troch kolách a~v~dvoch
zvyšných kolách vyložia dve rôzne karty $(x, y)$, každý v~inom poradí.
Každý taký výsledok je teda jednoznačne určený poradím kariet jedného
z~hráčov a~výberom kôl, v~ktorých druhý hráč zahrá rovnako.
Tri kolá z~piatich možno vybrať 10~spôsobmi a~päť
kariet možno usporiadať $5!$~spôsobmi. Výsledok $1:1$ tak nastane
v~$10\cdot 5!$~prípadoch.

Ostáva vyšetriť, kedy nastane výsledok $2:2$.
Tú kartu~$x$, ktorú vyložia hráči v~jednom z~piatich
kôl obaja naraz, je možné vybrať 5~spôsobmi. Anne aj Borisovi potom zvýšia štyri karty $a<b<c<d$.
Keďže na poradí kôl nezáleží, spočítajme najprv, koľko je možností v~prípade, že
Anna vyloží karty $x$, $a$, $b$, $c$, $d$ v~tomto poradí. Aby nedošlo k~ďalšej
remíze, musí Anna získať ďalší bod
v~poslednom kole za kartu~$d$, zatiaľ čo Boris musí získať bod v~druhom
kole, keď Anna vyloží kartu~$a$. Preto stačí zistiť, aké má
Boris v~treťom a~štvrtom kole možnosti, aby tieto dve kolá skončili $1:1$.

V~týchto kolách musí Boris vyložiť jednu z~dvojíc
$(a,d)$, $(c,a)$, $(c,b)$, $(d,a)$, $(d,b)$,
ktoré možno doplniť kartami pre druhé a~piate kolo tak, aby v~nich
nenastala remíza, do celkom siedmich poradí:
$$
\gather
(x,b,a,d,c),\
(x,c,a,d,b),\
(x,d,c,a,b),\
(x,d,c,b,a),\
(x,b,d,a,c),\\
(x,c,d,a,b),\
(x,c,d,b,a).
\endgather
$$
Anna môže karty $x$, $a$, $b$, $c$, $d$ vyložiť 5!~spôsobmi. Výsledok $2:2$
tak nastane v~$5\cdot 7\cdot 5!$ prípadoch.

Celkovo môžu ako Anna, tak Boris vyložiť karty 5!~spôsobmi, to je dokopy ${5!}^2$ možností.
Keďže počet všetkých možných priebehov hry, v~ktorých nastane remíza, je rovný
$5!+10\cdot 5!+5\cdot 7\cdot 5!=5!\cdot 46$, je počet možných výhier každého z~nich
$\frac12({5!}^2-5!\cdot 46)=5!\cdot37$. Výhrou Anny teda skončí
$$
\frac{5!\cdot37}{{5!}^2}=\frac{37}{120}\approx0{,}31= 31\,\%
$$
všetkých možných hier.

\návody
Aké sú možné bodové výsledky kartovej hry v~zadanej úlohe? [$4:1$,
$1:4$, $3:2$, $2:3$, $3:1$, $1:3$, $2:2$, $2:1$, $1:2$, $1:1$, $0:0$]

Koľkými spôsobmi mohol prebehnúť "skrátený" volejbalový set medzi družstvami $A$ a~$B$, ak sa hral do 5 bodov, zvíťazilo družstvo~$A$ a~vieme, že víťaz vyhral aspoň o~dva body? (Zaujíma nás nielen výsledok, ale celý priebeh, ako body pribúdali.)
[${{4}\choose{0}}+{{5}\choose{1}}+{{6}\choose{2}}+{{7}\choose{3}}=1+5+15+35=56$,
pri takých malých hodnotách sa dá počítať aj bez kombinačných čísel.]

Aký je počet desaťciferných čísel zložených z~rôznych cifier, v~ktorých sa
striedajú párne a~nepárne cifry? Koľko je to percent zo všetkých desaťciferných
čísel zložených z~rôznych cifier? [$5!\cdot 5!+4\cdot 4!\cdot 5!=9\cdot
4!\cdot 5!$, $\frac{9\cdot 4!\cdot 5!}{9\cdot 9!}
=\frac{1}{2\cdot63}\approx 0{,}008= 0{,}8\,\%$]
\endnávod
}

{%%%%%   C-I-1
Dvojnásobným použitím algoritmu delenia dostaneme
$$\align
ax^2+bx+c&=(ax+b-a)(x+1)+c-b+a,\\
ax^2+bx+c&=(ax+b-2a)(x+2)+c-2b+4a.
\endalign
$$
Dodajme k~tomu, že nájdené zvyšky $c-b+a$ a~$c-2b+4a$ sú
zrejme rovné hodnotám $p(\m1)$, resp. $p(\m2)$, čo je v~zhode
s~poznatkom, že akýkoľvek mnohočlen $q(x)$ dáva pri delení
dvojčlenom $x-x_0$ zvyšok rovný číslu $q(x_0)$.

Podľa zadania platí $c-b+a=2$ a~$c-2b+4a=1$.
Tretia rovnica $a+b+c=61$ je vyjadrením podmienky $p(1)=61$.
Získanú sústavu troch rovníc vyriešime jedným z~mnohých možných
postupov.

Z~prvej rovnice vyjadríme $c=b-a+2$, po dosadení do tretej rovnice
dostaneme $a+b+(b-a+2)=61$, čiže $2b=59$. Odtiaľ $b=59/2$, čo po
dosadení do prvej a~druhej rovnice dáva $a+c=63/2$, resp.
$c+4a=60$. Ak odčítame posledné dve rovnice od seba, dostaneme
$3a=57/2$, odkiaľ $a=19/2$, takže $c=63/2-19/2=22$. Hľadaný
trojčlen je teda jediný a~má tvar
$$
p(x)=\frac{19}{2}\cdot
x^2+\frac{59}{2}\cdot x+22=\frac{19x^2+59x+44}{2}.
$$

\návody
Ukážte, že pre každé číslo $a$ je
mnohočlen $x^4+(1-a)x^3+x^2+a$ deliteľný mnohočlenom $x^2+x+1$
bezo zvyšku. [Podiel je rovný $x^2-a x+a$.]

Určte všetky reálne čísla $a$, pre ktoré je trojčlen
$x^2+5x+6$ deliteľný dvojčlenom $x+a$. Riešte jednak použitím
algoritmu delenia, jednak použitím pravidla (často nazývaného {\it
Bezoutova veta\/}), že mnohočlen $p(x)$
je deliteľný dvojčlenom $x-x_0$ práve vtedy, keď $p(x_0)=0$. [Vyhovujú
čísla $a=2$ a~$a=3$, lebo priamym delením dostaneme rovnosť
mnohočlenov $x^2+5x+6=({x+a})({x+5-a})+a^2-5a+6$, takže hľadané čísla
$a$ sú korene rovnice $a^2-5a+6=0$.]

Určte všetky reálne čísla~$a$, pre ktoré trojčlen
$x^2+5x+6$ dáva pri delení dvojčlenom $x+a$ zvyšok~$2$. [Vyhovujú
čísla $a=1$ a~$a=4$, ktoré dostaneme, keď pre
všeobecný zvyšok $a^2-5a+6$ (pozri úlohu~2) zostavíme a~vyriešime
rovnicu $a^2-5a+6=2$.]


Ukážte, že všetky trojčleny $p(x)=ax^2+2(a-1)x-4$, kde
$a$ je ľubovoľné číslo, sú
deliteľné jedným dvojčlenom $x+b$ s~vhodným koeficientom
$b$. Akým? [$b=2$. Číslo~$b$ má požadovanú vlastnosť práve vtedy,
keď platí $p(\m b)=0$. Pretože $p(\m b)=a({b^2-2b})+2b-4=(b-2)(ab+2)$, je
rovnosť $p(\m b)=0$ splnená pre každé~$a$ práve vtedy, keď $b=2$.]

Určte všetky dvojice reálnych čísel $a$ a~$b$, pre ktoré
je mnohočlen $x^4+ax^2+b$ deliteľný mnohočlenom $x^2+bx+a$.
\vpravo{[56--B--S--1]}
 \endnávod
}

{%%%%%   C-I-2
Využijeme všeobecný poznatok, že body dotyku vpísanej kružnice delia
hranicu trojuholníka na šesť úsečiek, a~to tak, že každé dve z~nich, ktoré
vychádzajú z~toho istého vrcholu trojuholníka, sú zhodné. (Dotyčnice z~daného
bodu k~danej kružnici sú totiž súmerne združené podľa spojnice
daného bodu so stredom danej kružnice.)

V~našej úlohe je najdlhšia strana trojuholníka rozdelená na úseky,
ktorých dĺžky označíme $3x$ a~$4x$; dĺžku úsekov
z~vrcholu oproti najdlhšej strane označíme~$y$ (\obr). Strany
trojuholníka majú teda dĺžky $7x$, $4x+y$ a~$3x+y$, kde $x$, $y$ sú neznáme
kladné čísla (dĺžky berieme bez jednotiek).
Ak má byť $7x$ dĺžka najdlhšej strany, musí platiť $7x>4x+y$,
čiže  $3x>y$. Zdôraznime, že hľadané čísla $x$, $y$ nemusia byť
nutne celé, podľa zadania to však platí o~číslach $7x$, $4x+y$ a~$3x+y$.
\insp{c61.1}%

Údaj o~obvode trojuholníka zapíšeme rovnosťou
$$
72=7x+(3x+y)+(4x+y),\quad\text{čiže}\quad 36=7x+y.
$$
Pretože $7x$ je celé číslo, je celé i~číslo
$y=36-7x$; a~pretože podľa zadania i~čísla $4x+y$
a~$3x+y$ sú celé, je celé i~číslo $x=(4x+y)-(3x+y)$.
Preto od tohto okamihu už hľadáme dvojice {\it celých\/} kladných čísel $x$, $y$,
pre ktoré platí
$$
3x>y\quad\hbox{a}\quad 7x+y=36.
$$
Odtiaľ vyplýva $7x<36<7x+3x=10x$, teda $x\le5$
a~súčasne $x\ge4$.

Pre $x=4$ je $y=8$
a~$(7x,4x+y,3x+y)=(28,24,20)$, pre $x=5$ je $y=1$
a~$(7x,4x+y,3x+y)=(35,21,16)$. Strany trojuholníka sú teda $(28,24,20)$
alebo $(35,21,16)$. (Trojuholníkové nerovnosti sú zrejme splnené.)

\návody
Pomocou dĺžok $a$, $b$, $c$ strán všeobecného trojuholníka
vyjadrite dĺžky úsečiek, na ktoré sú tieto strany rozdelené bodmi
dotyku kružnice vpísanej tomuto trojuholníku. Na
príklade potom ukážte, že tieto dĺžky nemusia byť vyjadrené celými číslami,
aj~keď strany trojuholníka takéto vyjadrenia majú. [Ide o~dve úsečky dĺžky $x=\frac12(a+b-c)$, dve úsečky dĺžky
$y=\frac12(b+c-a)$ a~dve úsečky dĺžky $z=\frac12(c+a-b)$. Tieto
dĺžky nie sú celočíselné, ak sú napríklad všetky tri dĺžky
$a$, $b$, $c$ vyjadrené nepárnymi číslami.]

Ak zostrojíme z~troch úsečiek ľubovoľných dĺžok $p$, $q$, $r$
úsečky dĺžok $a=p+q$, $b=q+r$ a~$c=r+p$, budú tieto
tri nové úsečky dĺžkami strán nejakého trojuholníka.
Vysvetlite a~potom zistite, aký význam v~takom
trojuholníku budú mať pôvodné dĺžky $p$, $q$, $r$. [Overiť algebraicky
trojuholníkové nerovnosti $a+b>c>|a-b|$ je triviálne, lebo ide o~zrejmé
nerovnosti $p+2q+r>p+r>|p-r|$. V~trojuholníku so stranami
$a$, $b$, $c$ sú dĺžky $p$, $q$, $r$ dĺžkami úsečiek, na ktoré
sú strany $a$, $b$, $c$ rozdelené bodmi dotyku vpísanej kružnice,
ako to vyplýva z~výsledku úlohy~1.]

Trojuholník $ABC$ spĺňa pri zvyčajnom
označení dĺžok strán podmienku $a\le b\le c$.
Vpísaná kružnica sa dotýka strán $AB$,
$BC$ a~$AC$ postupne v~bodoch $K$,
$L$ a~$M$. Dokážte, že z~úsečiek
$AK$, $BL$ a~$CM$ je možné zostrojiť trojuholník
práve vtedy, keď platí $b+c<3a$.
\vpravo{[57--C--II--1]}

Dokážte, že v~každom pravouhlom trojuholníku je súčet polomerov
vpísanej kružnice a~opísanej kružnice rovný aritmetickému priemeru dĺžok oboch
odvesien. [Prvé riešenie úlohy 59--A--S--2.]

Určte dĺžku prepony pravouhlého trojuholníka, ak poznáte
polomer~$r$ kružnice vpísanej a~polomer~$R$ kružnice pripísanej
k~prepone tohto trojuholníka (\tj.~kružnice, ktorá sa dotýka
zvonku prepony a~predĺženia oboch odvesien
trojuholníka).
\vpravo{[45--C--I--6]}
\endnávod
}

{%%%%%   C-I-3
Prvky danej množiny $\mm M$ rozložíme na prvočinitele:
$$
{\mm M}=\{2,\,3,\,5,\,2^2\cdot3\cdot5,\,2\cdot3^2\cdot5,\,2^2\cdot3^2\cdot5\}.
$$
Odtiaľ vyplýva, že v~rozklade hľadaných čísel $a$, $b$, $c$ vystupujú
iba prvočísla $2$, $3$ a~$5$. Každé z~nich je pritom
prvočiniteľom práve dvoch z~čísel $a$, $b$, $c$: keby bolo
prvočiniteľom len jedného z~nich, chýbalo by v~rozklade troch
najväčších spoločných deliteľov a~jedného najmenšieho spoločného
násobku, teda v~štyroch číslach z~$\mm M$; keby naopak bolo
prvočiniteľom všetkých troch čísel $a$, $b$, $c$, nechýbalo by v~rozklade
žiadneho čísla z~$\mm M$. Okrem toho vidíme, že v~rozklade každého
z~čísel $a$, $b$, $c$ je prvočíslo~$5$ najviac v~jednom exemplári.

Podľa uvedených zistení môžeme čísla $a$, $b$, $c$ usporiadať tak, že
rozklady čísel $a$,~$b$ obsahujú po jednom exemplári prvočísla $5$
(potom $(c,5)=1$) a~že $(a,2)=2$ (ako vieme,
aspoň jedno z~čísel $a$, $b$ musí byť párne).
Číslo $5$ z~množiny~$\mm M$ je potom nutne rovné $(a,b)$, takže
platí $(b,2)=1$, a~preto $(b,3)=3$ (inak by platilo $(b,c)=1$),
odtiaľ zase s~ohľadom na $(a,b)=5$ vyplýva $(a,3)=1$.
Máme teda $a=5\cdot2^s$ a~$b=5\cdot3^t$ pre vhodné prirodzené
čísla $s$ a~$t$.

Z~rovnosti $[a,b]=2^{s}\cdot3^{t}\cdot5$ vyplýva, že nastane
jeden z~troch nasledovných prípadov.
\ite(1) $2^{s}\cdot3^{t}\cdot5=60=2^{2}\cdot3^{1}\cdot5$. Vidíme,
že platí $s=2$ a~$t=1$, čiže $a=20$ a~$b=15$. Ľahko
určíme, že tretím číslom je $c=18$.
\ite(2) $2^{s}\cdot3^{t}\cdot5=90=2^{1}\cdot3^{2}\cdot5$.
V~tomto prípade $a=10$, $b=45$ a~$c=12$.
\ite(3) $2^{s}\cdot3^{t}\cdot5=180=2^{2}\cdot3^{2}\cdot5$.
Teraz $a=20$, $b=45$ a~$c=6$.

\odpoved
Hľadané čísla $a$, $b$, $c$ tvoria jednu z~množín
$\{20,15,18\}$, $\{10,45,12\}$ a~$\{20,45,6\}$.

\ineriesenie
V~danej rovnosti je množina napravo tvorená šiestimi rôznymi číslami
väčšími  ako~$1$, takže čísla $(a,b)$, $(a,c)$, $(b,c)$ musia byť
netriviálnymi deliteľmi postupne čísel $[a,b]$, $[a,c]$, $[b,c]$.
Čísla $2$, $3$, $5$ ale žiadne netriviálne delitele nemajú,
musí teda platiť
$$
\bigl\{(a,b),(a,c),(b,c)\bigr\}=\{2,3,5\}
\quad\text{a}\quad
\bigl\{[a,b],[a,c],[b,c]\bigr\}=\{60,90,180\}.
$$
Pretože poradie čísel $a$, $b$, $c$ nehrá žiadnu úlohu, môžeme predpokladať,
že platí $(a,b)=2$, $(a,c)=3$ a~$(b,c)=5$. Odtiaľ vyplývajú
vyjadrenia
$$
a=2\cdot3\cdot x=6x,\quad b=2\cdot5\cdot y=10y,
\quad c=3\cdot5\cdot z=15z
$$
pre vhodné prirodzené čísla $x$, $y$, $z$. Zo známej rovnosti
$[x,y]\cdot(x,y)=xy$ tak dostaneme vyjadrenia najmenších spoločných
násobkov v tvare
$$
[a,b]=\frac{6x\cdot10y}{2}=30xy,\quad
[a,c]=\frac{6x\cdot15z}{3}=30xz,\quad
[b,c]=\frac{10y\cdot15z}{5}=30yz.
$$
Z~rovnosti $\{30xy,30xz,30yz\}=\{60,90,180\}$ upravenej na
$\{xy,xz,yz\}=\{2,3,6\}$ potom vďaka tomu, že $2$ a~$3$ sú prvočísla,
vyplýva $\{x,y,z\}=\{1,2,3\}$. Pretože z~podmienky
$5=(b,c)=(10y,15z)$ vyplýva $y\ne3$ a~$z\ne2$, prichádzajú do~úvahy
len trojice $(x,y,z)$ rovné $(1,2,3)$, $(2,1,3)$ a~$(3,2,1)$,
ktorým postupne zodpovedajú trojice $(a,b,c)$ rovné $(6,20,45)$,
$(12,10,45)$, $(18,20,15)$. Skúškou sa presvedčíme, že všetky
tri vyhovujú množinovej rovnosti zo zadania úlohy.

\návody
Určte, pre ktoré prirodzené čísla $a$, $b$ platí
$(a,b)=10$ a~zároveň $[a,b]=150$. [$\{a,b\}=\{10,150\}$ alebo
$\{a,b\}=\{30,50\}$. Pretože $10=2\cdot5$
a~$150=2\cdot3\cdot5^2$, požadované rovnosti sú splnené práve vtedy, keď
$a=2\cdot3^{s}\cdot5^{t}$ a~$b=2\cdot3^{u}\cdot5^{v}$,
kde $\{s,u\}=\{0,1\}$ a~$\{t,v\}=\{1,2\}$.]

Dokážte, že pre ľubovoľné prirodzené čísla $a$, $b$ platí vzťah
$[a,b]\cdot(a,b)=ab$.
[Podľa úvahy o~počtoch zastúpení každého prvočísla
v~číslach $a$, $b$, $(a,b)$ a~$[a,b]$ stačí vysvetliť, prečo pre
akékoľvek čísla $\alpha$, $\beta$ platí rovnosť
$\min\{\alpha,\beta\}+\max\{\alpha,\beta\}=\alpha+\beta$. K~tomu stačí rozlíšiť
prípady $\alpha<\beta$, $\alpha=\beta$ a~$\alpha>\beta$.
Iné riešenie: Nech $d=(a,b)$, potom $a=xd$, $b=yd$
pre nesúdeliteľné $x$ a~$y$, odtiaľ vyplýva $[a,b]=xyd$, takže oba
súčiny $[a,b]\cdot(a,b)$ a~$ab$ sa rovnajú číslu $xyd^2$.]

Nájdite všetky trojice $a$, $b$, $c$ prirodzených čísel, pre
ktoré súčasne platí $(ab,c)=2^8$, $(bc,a)=2^9$ a~$(ca,b)=2^{11}$.
\vpravo{[50--C--S--1]}

Nájdite všetky dvojice prirodzených čísel $a$, $b$,
pre ktoré platí $a+b+[a,b]+(a,b)=50$.
\vpravo{[50--C--II--1]}

Pre ľubovoľné prirodzené čísla $a$, $b$ dokážte nerovnosť
$a\cdot(a,b)+b\cdot[a,b]\ge2ab$.  Zistite tiež,
kedy v~tejto nerovnosti nastane rovnosť.
\vpravo{[60--C--I--5]}
\endnávod
}

{%%%%%   C-I-4
a) Z~rovnosti $16=ab+bc+cd+da=(a+c)(b+d)$ vyplýva, že
obidva súčty $a+c$ a~$b+d$ nemôžu byť väčšie ako~$4$ súčasne,
lebo v~opačnom prípade by bol ich súčin väčší ako $16$.
Preto vždy aspoň jeden zo súčtov $a+c$ alebo $b+d$
má požadovanú vlastnosť.

\smallskip
b) Využijeme všeobecnú rovnosť
$$
a^2+b^2+c^2+d^2=\tfrac12(a-b)^2+\tfrac12(b-c)^2+
\tfrac12(c-d)^2+\tfrac12(d-a)^2+ab+bc+cd+da,
$$
o~platnosti ktorej sa ľahko presvedčíme úpravou pravej strany.
Vzhľadom na nezápornosť
druhých mocnín $(a-b)^2$, $(b-c)^2$, $(c-d)^2$ a~$(d-a)^2$
dostávame pre ľavú stranu rovnosti odhad
$$
a^2+b^2+c^2+d^2\ge ab+bc+cd+da=16.
$$

Je nájdené číslo $16$ najmenšou hodnotou uvažovaných
súčtov? Ináč povedané: nastane pre niektorú vyhovujúcu štvoricu
v~odvodenej nerovnosti rovnosť?  Z~nášho postupu je jasné, že
musíme rozhodnúť, či pre niektorú z~uvažovaných štvoríc
platí $a-b=b-c=c-d=d-a=0$, čiže
$a=b=c=d$. Pre takú štvoricu má rovnosť $ab+bc+cd+da=16$
tvar $4a^2=16$, čomu vyhovuje $a=\pm2$. Pre
vyhovujúce štvorice $a=b=c=d=2$ a~$a=b=c=d=\m2$ má súčet
$a^2+b^2+c^2+d^2$ naozaj hodnotu~$16$, preto ide o~hľadané
minimum.

\návody
Ak reálne čísla $x$, $y$, $z$ vyhovujú rovnici
$x^2+y^2=z^2$, potom aspoň jedno z~čísel $|x+z|$, $|x-z|$
neprevyšuje hodnotu $|y|$. Dokážte. [Keby
$|x+z|$, $|x-z|$ boli dve (kladné) čísla väčšie ako $|y|$,
bolo by číslo $|x+z|\cdot|x-z|$ väčšie ako $|y|^2$,
podľa zadania ale ide o~dve rovnaké čísla.]

Nech $x$, $y$, $z$ sú kladné reálne čísla. Ukážte,
že čísla $x+y+z-xyz$ a~${xy+yz+zx-3}$ nemôžu byť súčasne záporné.
\vpravo{[60--C-II--4]}

Je dané prirodzené číslo $n$ ($n\ge2$) a~reálne čísla
$x_1,x_2,\dots,x_n$, pre ktoré platí
$x_1x_2=x_2x_3=\dots=x_{n-1}x_n=x_nx_1=1$.
Dokážte nerovnosť
$x_1^2+x_2^2+\dots+x_n^2\ge n$.
\vpravo{[55--C--I--4]}

Ak reálne čísla  $a$, $b$, $c$, $d$ vyhovujú rovnosti
$a^{2}+b^{2}=b^{2}+c^{2}=c^{2}+d^{2}=1$,
platí nerovnosť
$ab+ac+ad+bc+bd+cd\le3$.
Dokážte a~zistite, kedy pritom nastane rovnosť.
\vpravo{[55--C--II--2]}

Dokážte, že nerovnosť $(a^2+1)(b^2+1) - (a-1)^2
(b-1)^2\ge4$ platí pre ľubovoľné čísla $a$, $b$ z~intervalu
$\langle1,\infty)$. Zistite, kedy nastane rovnosť.
\vpravo{[59--C--II--2]}

Dokážte, že nerovnosť $(a+bc)(b+ac)\ge ab(c+1)^2$
platí pre ľubovoľné nezáporné čísla $a$, $b$, $c$.
Zistite, kedy nastane rovnosť.
\vpravo{[58--C--S--1]}

Nerovnosti
$\dfrac{a+b}{2}<\dfrac{2(a^2+ab+b^2)}{3(a+b)}<\sqrt{\dfrac{a^2+b^2}{2}}$
dokážte pre ľubovoľné rôzne kladné čísla $a$, $b$.
\vpravo{[58--C--I--6]}

Nech $a$, $b$, $c$ sú reálne čísla, ktorých súčet je $6$.
Dokážte, že aspoň jedno z~čísel $ab+bc$, $bc+ca$ alebo $ca+ab$
nie je väčšie ako~$8$.
\vpravo{[60--B--I--3]}
\endnávod
}

{%%%%%   C-I-5
Označme $S$ stred základne~$BC$ daného rovnoramenného trojuholníka $ABC$,
$O$ stred jeho opísanej kružnice, $M$ stred vpísanej kružnice
a~$P$ pätu kolmice z~bodu~$M$ na rameno~$AC$ (\obr).
\insp{c61.2}%

Z~pravouhlého trojuholníka $BSA$ pomocou Pytagorovej vety vyjadríme
veľkosť~$v$ výšky~$AS$, pričom v~pravouhlom trojuholníku $BSO$
s~preponou dĺžky~$R$ pre odvesnu~$OS$ platí
$|OS|=\bigl||AS|-|AO|\bigr|=|v-R|$
(musíme si uvedomiť, že v~tupouhlom trojuholníku $ABC$ bude bod~$S$ ležať medzi bodmi $A$ a~$O$!).
Dostávame tak dve rovnosti
$$
\align
v^2&=b^2-\frac{a^2}{4},\\
R^2&=\frac{a^2}{4}+(v-R)^2;
\endalign
$$
ich sčítaním vyjde
$$
v^2+R^2=b^2+(v-R)^2, \quad\text{čiže}\quad b^2=2vR.
$$
Dosadením z~prvej rovnice $v=\frac12\sqrt{4b^2-a^2}$ do poslednej
rovnosti dostaneme hľadaný vzorec pre~$R$.

Dodajme, že rovnosť $b^2=2vR$, ktorú sme práve odvodili
a~z ktorej už ľahko vyplýva vzorec pre polomer~$R$,
je Euklidovou vetou o~odvesne~$AB$ pravouhlého
trojuholníka $ABA'$ s~preponou $AA'$, ktorá je priemerom kružnice opísanej trojuholníku $ABC$
(\obrr1).

Nájdený vzorec pre polomer~$R$ zapíšeme prehľadne
spolu s~druhým hľadaným vzorcom pre polomer~$r$, ktorého odvodeniu
sa ešte len budeme venovať:
$$
R=\frac{b^2}{\sqrt{4b^2-a^2}}\quad\text{a}\quad
r=\frac{a\sqrt{4b^2-a^2}}{2(a+2b)}.
\tag$*$
$$

Druhý zo vzorcov $(*)$ sa dá získať okamžite zo známeho vzťahu
$r=2S/(a+b+c)$ pre polomer~$r$  kružnice vpísanej do
trojuholníka so~stranami $a$, $b$, $c$ a~obsahom~$S$; v~našom prípade
stačí len dosadiť $b=c$ a~$2S=av$, kde
$v=\frac12\sqrt{4b^2-a^2}$ podľa úvodnej časti riešenia.

Ďalšie dva spôsoby odvodenia druhého zo vzorcov $(*)$ založíme na
úvahe o~pravouhlom trojuholníku $AMP$, ktorého strany majú dĺžky
$$
|AM|=v-r,\quad |MP|=r,\quad
|AP|=|AC|-|PC|=b-|SC|=b-\frac{a}{2}.
$$
Pre tento trojuholník môžeme napísať Pytagorovu vetu alebo využiť jeho
podobnosť s~trojuholníkom $ACS$, konkrétne zapísať rovnosť sínusov ich
spoločného uhla pri vrchole~$A$. Podľa toho dostaneme rovnice
$$
(v-r)^2=r^2+\Bigl(b-\frac{a}{2}\Bigr)^{\!2},\quad\text{resp.}\quad
\frac{r}{v-r}=\frac{\frac12a}{b},
$$
ktoré sú obidve lineárne vzhľadom na~neznámu~$r$ a~majú riešenie
$$
r=\frac{v}{2}-\frac{1}{2v}\cdot\Bigl(b-\frac{a}{2}\Bigr)^{\!2},
\quad\text{resp.}\quad r=\frac{av}{a+2b}.
$$
Po dosadení za $v$ v~oboch prípadoch dostaneme hľadaný vzorec pre~$r$.
V~druhom prípade je to zrejmé, v~prvom to ukážeme:
$$
\align
r&=\frac{v}{2}-\frac{1}{2v}\cdot\Bigl(b-\frac{a}{2}\Bigr)^{\!2}=
\frac{v^2-b^2+ab-\frac14a^2}{2v}=\frac{2ab-a^2}{4v}=\\
&=\frac{a(2b-a)}{2\sqrt{(2b-a)(2b+a)}}=
\frac{a\sqrt{2b-a}}{2\sqrt{2b+a}}=\frac{a\sqrt{4b^2-a^2}}{2(a+2b)}.
\endalign
$$

Ešte ostáva dokázať nerovnosť $R\ge2r$.
Využijeme na~to odvodené vzorce $(*)$, z~ktorých dostávame
(pripomíname, že $2b>a>0$)
$$
\frac{R}{2r}=
{R}\cdot\frac{1}{2r}=\frac{b^2}{\sqrt{4b^2-a^2}}\cdot
\frac{a+2b}{a\sqrt{4b^2-a^2}}=\frac{b^2}{a(2b-a)}.
$$
Nerovnosť $R\ge2r$ teda platí práve vtedy, keď
$b^2\ge a(2b-a)$. Posledná
nerovnosť je však ekvivalentná s~nerovnosťou $(a-b)^2\ge0$,
ktorej platnosť je už zrejmá. Tým je dôkaz nerovnosti $R\ge2r$
hotový. Navyše vidíme, že rovnosť v~nej nastane
jedine v~prípade, keď $(a-b)^2=0$, čiže $a=b$, teda práve vtedy, keď
je pôvodný trojuholník nielen rovnoramenný, ale dokonca rovnostranný.

\návody
Pre všeobecný trojuholník $ABC$ so~stranami $a$, $b$, $c$ a~obsahom~$S$
platí pre polomer~$r$ vpísanej kružnice vzorec $r=2S/(a+b+c)$.
Dokážte. [Stred~$M$ vpísanej kružnice rozdeľuje uvažovaný trojuholník
$ABC$ na tri menšie trojuholníky $BCM$, $ACM$, $ABM$ o~obsahoch
$\frac12ar$, $\frac12br$, $\frac12cr$, ktorých súčet je~$S$, odkiaľ vyplýva dokazovaný vzorec.]

Kružnice $k(S;6\cm)$ a~$l(O;4\cm)$ majú vnútorný dotyk v~bode~$B$.
Určte dĺžky strán trojuholníka $ABC$, kde bod~$A$ je priesečník priamky~$OB$
s~kružnicou~$k$ a~bod~$C$ je priesečník kružnice~$k$ s~dotyčnicou z~bodu~$A$
ku kružnici~$l$.
\vpravo{[59--C--S--2]}

Kružnica $l(T; s)$ prechádza stredom kružnice $k(S;2\cm)$.
Kružnica $m(U; t)$ sa zvonku dotýka kružníc  $k$ a~$l$, pričom
$US\perp ST$.
Polomery $s$ a~$t$ vyjadrené v~centimetroch sú celé čísla. Určte ich.
\vpravo{[59--B--II--1]}

Pravouhlému trojuholníku $ABC$ s~preponou~$AB$ je opísaná kružnica.
Päty kolmíc z~bodov $A$, $B$ na dotyčnicu k~tejto kružnici v~bode~$C$
označme $D$, $E$. Vyjadrite dĺžku úsečky~$DE$ pomocou dĺžok
odvesien trojuholníka $ABC$.
\vpravo{[58--C--I--2]}

Pravouhlému trojuholníku $ABC$ s~preponou~$AB$ a~obsahom~$S$
je opísaná kružnica. Dotyčnica k~tejto kružnici v~bode~$C$ pretína
dotyčnice vedené bodmi $A$ a~$B$ v~bodoch $D$ a~$E$.
Vyjadrite dĺžku úsečky~$DE$ pomocou dĺžky~$c$ prepony
a~obsahu~$S$.
\vpravo{[58--C--II--4]}

Rovnoramennému lichobežníku $ABCD$ so základňami $AB$, $CD$ je možné
vpísať kružnicu so stredom~$O$. Určte obsah~$S$ lichobežníka,
ak sú dané dĺžky úsečiek $OB$ a~$OC$.
\vpravo{[56--C--II--3]}

Kružnice $k$, $l$, $m$ sa po dvoch zvonku dotýkajú a~všetky tri majú spoločnú dotyčnicu.
Polomery kružníc $k$, $l$ sú $3\cm$ a~$12\cm$. Vypočítajte polomer kružnice $m$.
Nájdite všetky riešenia.
\vpravo{[55--C--I--2]}

Kružnice $k$, $l$ s~vonkajším dotykom ležia obe
v~obdĺžniku $ABCD$, ktorého obsah je $72\cm^2$. Kružnica~$k$ sa
pritom dotýka strán $CD$, $DA$ a~$AB$ a kružnica~$l$ sa dotýka
strán $AB$ a~$BC$.  Určte polomery  kružníc $k$ a~$l$,
ak je polomer kružnice~$k$ v~centimetroch vyjadrený celým číslom.
\vpravo{[55--C--II--3]}
\endnávod
}

{%%%%%   C-I-6
Ak je celkový počet políčok hracej plochy párny
(v~zadaní pre $n=4$ a~$n=6$), môže v~poradí druhá
hráčka pomýšľať na túto víťaznú stratégiu: spárovať všetky
políčka hracej dosky do dvojíc tak, aby v~každom páre boli políčka
navzájom dosiahnuteľné jedným skokom. Pokiaľ také spárovanie políčok
druhá hráčka nájde, má víťaznú stratégiu: v~každom ťahu urobí
skok na druhé políčko toho páru, na ktorého prvom políčku figúrka práve leží.

Ak je celkový počet políčok hracej plochy nepárny (v~zadaní
pre $n=5$), môže v~poradí prvá hráčka pomýšľať na túto
víťaznú stratégiu: spárovať všetky políčka hracej dosky okrem
jedného do dvojíc tak, aby v~každom páre boli políčka
navzájom dosiahnuteľné jedným skokom. Pokiaľ také spárovanie prvá
hráčka nájde, má víťaznú stratégiu: v~prvom ťahu položí figúrku
na (jediné) nespárované políčko a~v~každom ďalšom ťahu urobí skok
na druhé políčko toho páru, na ktorého prvom políčku figúrka práve leží.

Nájsť požadované spárovania políčok je pre zadané príklady ľahké
a~je to možné urobiť viacerými spôsobmi. Ukážme tie z~nich, ktoré majú
určité črty pravidelnosti. Na \obr{} zľava je vidno, ako je
možné spárovať políčka časti hracej plochy o~rozmeroch $4\times2$; celú
hraciu plochu $4\times4$ rozdelíme na dva také bloky a~urobíme
spárovanie v~každom z~nich. I~na
spárovanie políčok hracej plochy $6\times6$ môžeme využiť spárovanie v~dvoch
blokoch $4\times2$; na \obrr1{} uprostred je znázornené
možné stredovo súmerné spárovanie všetkých políčok. Nakoniec na \obrr1{} vpravo
je príklad spárovania políčok hracej plochy $5\times5$ s~nespárovaným
políčkom v~ľavom hornom rohu (nespárované políčko nemusí byť nutne rohové);
opäť je pritom využitý jeden blok
$4\times 2$.
\insp{c61.3}%

\návody
Riešte jednoduchší variant zadanej úlohy,
keď povolené {\it skoky\/} sú ťahy šachovou vežou, \tj. presuny
figúrky v~smere riadkov alebo v~smere stĺpcov hracej dosky
(o~ľubovoľný počet políčok). Dokážete objaviť víťaznú stratégiu pre
tento variant hry v~prípade hracej dosky ľubovoľných rozmerov $m\times n$?
[Ak sú obe čísla $m$ a~$n$ nepárne, má víťaznú stratégiu prvá
hráčka, ak je aspoň jedno z~čísel $m$, $n$ párne,
má víťaznú stratégiu druhá hráčka. V~oboch prípadoch si uvedená
hráčka vopred v~duchu rozdelí
všetky políčka hracej dosky do dvojíc (v~prvom prípade jedno políčko ostane,
naň potom hráčka položí figúrku v~úvodnom ťahu), a~to tak,
aby v~každom zostavenom páre boli políčka navzájom dosiahnuteľné
jedným skokom (pre ťahy vežou je to ľahké, stačí párovať len susedné
políčka riadku alebo stĺpca); v~priebehu hry potom táto hráčka môže vždy
skočiť z~jedného políčka na druhé políčko toho istého páru, takže vyhrá.]

Na tabuli sú napísané všetky prvočísla menšie ako $100$.
Gitka a~Terka sa striedajú v~ťahoch pri nasledujúcej hre. Najprv
Gitka zmaže jedno z~prvočísel. Ďalej vždy hráčka, ktorá
je na ťahu, zmaže jedno z~prvočísel, ktoré má s~predchádzajúcim
zmazaným prvočíslom jednu zhodnú číslicu (tak po prvočísle~$3$ je možné zmazať
trebárs $13$ alebo~$37$).
Hráčka, ktorá je na ťahu a~nemôže už žiadne prvočíslo zmazať,
prehráva. Ktorá z~oboch hráčok môže hrať tak, že
vyhrá nezávisle od ťahov súperky? [Pretože prvočísel
menších ako~$100$ je nepárny počet (25), ponúka sa hypotéza, že
víťaznú stratégiu bude mať prvá hráčka. Ukážme, že to tak
naozaj je. Táto hráčka si vopred v~duchu spáruje (podľa spoločnej
číslice) napísané prvočísla
(dá sa to urobiť viacerými spôsobmi, uvedieme ten, pri ktorom v~každom kroku
párujeme najmenšie doposiaľ nespárované prvočíslo s~najmenším ďalším
doposiaľ nespárovaným prvočíslom so~spoločnou číslicou):
$(2,23)$, $(3,13)$, $(5,53)$,
$(7,17)$, $(11,19)$, $(29,59)$, $(31,37)$, $(41,43)$, $(47,67)$,
$(61,71)$, $(73,79)$, $(83,89)$; jediné zostávajúce nespárované
prvočíslo~$97$ preto Gitka zmaže ako prvé a~ďalej pri hre
bude mazať vždy prvočíslo, ktoré je v páre
s~predchádzajúcim zmazaným prvočíslom. Týmto postupom musí vyhrať.]

Dve hráčky majú k~dispozícii pre hru, ktorú opíšeme,
neobmedzený počet dvadsaťcentových mincí a~stôl s~kruhovou doskou s~priemerom 1~meter.
Hra prebieha tak, že sa hráčky pravidelne striedajú v~ťahoch.
Najprv prvá hráčka položí jednu mincu kamkoľvek na prázdny stôl.
Ďalej vždy hráčka, ktorá je na ťahu,
položí na voľnú časť stola ďalšiu mincu (tak, aby nepresahovala
okraj stola a~aby sa skôr položených mincí nanajvýš dotýkala).
Ktorá z~oboch hráčok môže hrať tak, že vyhrá nezávisle
od ťahov súperky? [Víťaznú stratégiu má prvá hráčka:
prvú mincu položí doprostred stola
a~v~každom ďalšom kroku položí mincu na miesto súmerne združené
podľa stredu stola s~miestom práve položenej mince.]
%Ako ale to súmerne združené miesto nájde?%
\endnávod
}

{%%%%%   A-S-1
Sčítaním, resp. odčítaním oboch rovníc a~úpravou dostaneme
$$
\aligned
4(x + y) &= 4(x^3 + y^3),\\
(x + y) &= (x + y)(x^2 - xy + y^2),
\endaligned
\quad\text{resp.}\quad
\aligned
2(x - y) &= 4(x^3 - y^3),\\
\frac12(x - y) &= (x - y)(x^2 + xy + y^2).
\endaligned
\ifrocenka\else\tag1\fi
$$

Ak $x+y=0$, tak dosadením $y =\m x$ napríklad do prvej rovnice pôvodnej sústavy dostaneme $2x=4x^3$, teda po úprave $x(2x^2-1)=0$. Riešením tejto rovnice sú zrejme hodnoty $x=0$ a $x=\pm\frac12\sqrt2$, máme tak prvé tri riešenia sústavy: usporiadané dvojice $(0,0)$, $(\frac12\sqrt2,\m\frac12\sqrt2)$ a $(\m\frac12\sqrt2,\frac12\sqrt2)$.

Ak $x-y=0$, tak dosadením $y =x$ do prvej rovnice dostaneme $4x=4x^3$, čiže $x(x^2-1)=0$. Riešením tejto rovnice sú $x=0$ a $x=\pm1$. Pre $x=0$ dostaneme už skôr objavené riešenie $(0,0)$, pre $x=\pm1$ máme ďalšie dve riešenia sústavy $(1,1)$ a $(\m1,\m1)$.

\smallskip
Ostáva rozobrať prípad, keď sú oba výrazy $x+y$, $x-y$ nenulové. Pri tejto podmienke môžeme rovnice odvodené na začiatku uvedenými výrazmi vydeliť a~dostaneme sústavu rovníc
$$
\aligned
1 &= x^2 - xy + y^2,\\
\frac12 &= x^2 + xy + y^2.
\endaligned
\ifrocenka\else\tag2\fi
$$
Z~nej opäť sčítaním, resp. odčítaním oboch rovníc odvodíme $x^2+y^2=\frac34$ a~$xy=\m\frac14$.
Na základe toho máme
$$
(x+y)^2=x^2+y^2+2xy=\frac34-\frac12=\frac14,
$$
teda $x+y=\frac12$ alebo $x+y=\m\frac12$. Hodnoty neznámych $x$, $y$ vieme potom podľa Vi\`etových vzťahov dostať riešením kvadratických rovníc
$$
t^2-\frac12t-\frac14=0,\qquad\text{resp.}\qquad t^2+\frac12t-\frac14=0.
$$
Keďže ich koreňmi sú $t_{1,2}=\frac14\pm\frac14\sqrt5$, resp. $t_{3,4}=\m\frac14\pm\frac14\sqrt5$, dostávame štyri riešenia $(t_1,t_2)$, $(t_2,t_1)$, $(t_3,t_4)$, $(t_4,t_3)$.

\zaver
Sústava má spolu 9 rôznych riešení, sú nimi usporiadané dvojice
$$
\gathered
(0,0),\ \left(\frac12\sqrt2,\m\frac12\sqrt2\right),\ \left(\m\frac12\sqrt2,\frac12\sqrt2\right),
(1,1),\ (\m1,\m1),\\
\left(\frac14+\frac14\sqrt5,\frac14-\frac14\sqrt5\right),\
\left(\frac14-\frac14\sqrt5,\frac14+\frac14\sqrt5\right),\\
\left(-\frac14+\frac14\sqrt5,-\frac14-\frac14\sqrt5\right),\
\left(-\frac14-\frac14\sqrt5,-\frac14+\frac14\sqrt5\right).
\endgathered
\ifrocenka\else\tag3\fi
$$
Z~postupu vyplýva, že všetky spĺňajú pôvodnú sústavu, skúška teda (pri tomto postupe) nie je nutná.



\ineriesenie
Ak $|x| > 1$, tak z~prvej rovnice vyplýva $|y| = |x|(4x^2-3) > |x| > 1$. Z~druhej rovnice
následne rovnako odvodíme $|x| > |y|$, čo je spor. Preto $\m1\le x\le1$ a~existuje $t$ v~intervale $\langle0,\pi\rangle$ také, že $x = \cos t$.
Dosadením do prvej rovnice dostaneme $y = 4\cos^3t-3\cos t = \cos 3t$,\footnote{Na odvodenie rovnosti $\cos 3t=4\cos^3t-3\cos t$ stačí použiť známy vzorec $\cos(a+b)=\cos a\cos b-\sin a\sin b$.} z~druhej potom podobne $x = 4\cos^33t-3\cos 3t = \cos 9t$. Preto musí platiť
$\cos t = \cos 9t$, odkiaľ po úprave
$$
\align
\cos 9t-\cos t&=0,\\
-2\sin\frac{9t+t}2\sin\frac{9t-t}2&=0,\\
\sin5t\sin4t&=0.\ifrocenka\else\tag4\fi
\endalign
$$
Preto $5t$ alebo $4t$ musí byť násobkom~$\pi$. Spolu s~podmienkou $0\le t\le\pi$ dostávame
riešenia tvaru $(\cos t,\cos3t)$ pre
$$
t\in\left\{0,\frac{\pi}5,\frac{\pi}4,\frac{2\pi}5,\frac{\pi}2,\frac{3\pi}5,\frac{3\pi}4,\frac{4\pi}5,\pi\right\}.
$$
(Skúška ani v~tomto prípade nie je nutná.)

\ineriesenie
Vyjadrením $y=4x^3-3x$ z~prvej rovnice a~dosadením za $y$ do druhej dostaneme po úprave
$$
\align
x+3(4x^3-3x)&=4(4x^3-3x)^3,\\
256x^9 - 576x^7 + 432x^5 - 120x^3 + 8x &=0,    \ifrocenka\else\tag5\fi\\
x(32x^8 - 72x^6 + 54x^4 - 15x^2 +1) &=0.
\endalign
$$
Mnohočlen v~zátvorke po substitúcii $x^2=z$ prejde na mnohočlen štvrtého
stupňa $32z^4-72z^3+54z^2-15z+1$, ktorý môžeme rozložiť na súčin tak, že uhádneme jeho korene
$z=1$ a $z=\frac12$ a následne ho vydelíme príslušnými koreňovými
činiteľmi. Odvodená rovnica pre neznámu~$x$ tak prejde na tvar
$$
2x(x^2-1)(x^2-\tfrac12)(16x^4-12x^2+1)=0.
$$
Doriešením bikvadratickej rovnice $16x^4-12x^2+1=0$ (napríklad substitúciou $x^2=z$) už ľahko určíme všetky riešenia. Sú nimi
$$
x\in\left\{0,\ \pm1,\ \pm\tfrac12\sqrt{2},\ \pm\sqrt{\tfrac38\pm\tfrac18\sqrt5}\right\}
$$
(treba sa presvedčiť, že výraz pod odmocninou je pre každú kombináciu znamienok kladný). Ku každej z~uvedených deviatich hodnôt $x$ už ľahko dopočítame riešenie tvaru $(x,4x^3-3x)$, skúška opäť vzhľadom na postup nie je nutná.

\nobreak\medskip\petit\noindent
Za úplné vyriešenie úlohy dajte 6~bodov, a to aj v~prípade že riešenia nie sú uvedené presne v~tvare \thetag3; stačí aj tvar ako pri druhom alebo treťom uvedenom postupe alebo akýkoľvek iný podobný zápis obsahujúci všetkých 9~riešení.

Pri postupe ako v~prvom uvedenom riešení dajte po jednom bode za každý z~rozkladov v~\thetag1 a~po jednom bode za rozbor situácie $x+y=0$, resp. $x-y=0$, spolu však najviac 3~body. Štvrtý bod dajte za prepis na sústavu \thetag2, piaty bod za určenie oboch hodnôt $x+y=\pm\frac12$ a~$xy=\m\frac14$ a~šiesty bod za správne vyriešenie kvadratických rovníc.

Pri druhom postupe dajte 1~bod za dôkaz, že $\m1\le x\le 1$, ďalší bod dajte za substitúciu $x=\cos t$. Tretí bod je za odvodenie $y=\cos 3t$, štvrtý bod za rovnicu $\cos t=\cos9t$ a~posledné 2~body za jej kompletné vyriešenie v~intervale $\langle0,\pi\rangle$ (tieto 2~body možno rozdeliť, napr. za odvodenie \thetag4 bez následného nájdenia riešení možno udeliť piaty bod).

Pri treťom postupe len za vyjadrenie $y=4x^3-3x$ a~dosadenie do druhej rovnice bez ďalšej úpravy nedávajte žiadny bod -- prvý bod dajte až za bezchybnú úpravu na tvar~\thetag5. Po jednom bode dajte za vyňatie pred zátvorku každého z~činiteľov $x$, $(x^2-1)$, $(x^2-\frac12)$ (resp. za nájdenie príslušných koreňov a~zníženie stupňa mnohočlena, ktorého korene hľadáme). Posledné dva body dajte za vyriešenie bikvadratickej rovnice.

Ak žiak (pri akomkoľvek správnom postupe) nenájde všetkých 9~riešení, dajte najviac 5~bodov. Toľko dajte aj v~prípade, že žiak rieši sústavu dôsledkovými (neekvivalentnými) úpravami (\tj. z~postupu jednoducho nevyplýva, že nájdené riešenia naozaj spĺňajú pôvodnú sústavu), nájde všetky riešenia, ale neurobí skúšku. (Bod za chýbajúcu skúšku strhnite iba v~prípade, že inak je postup bezchybný.)
\endpetit
\bigbreak
}

{%%%%%   A-S-2
Úsečka~$BM$ je ťažnicou v~trojuholníku $ABC$, delí ho teda na dva trojuholníky s~rovnakým obsahom. Podľa zadania má jeden z~týchto trojuholníkov rovnaký obsah ako trojuholník $ACD$. Preto má trojuholník $ABC$ dvakrát väčší obsah ako trojuholník $ACD$. Tieto trojuholníky majú pritom zhodné výšky na strany $AB$, $CD$ (ktoré sa zhodujú s~výškou uvažovaného lichobežníka). Vzhľadom na ich obsahy teda platí $|AB|=2|CD|$.
\insp{a61.17}%

Na priamke~$CD$ uvažujme taký bod~$E$, že $D$ je stredom úsečky~$CE$ (\obr). Z~odvodenej rovnosti $|AB|=2|CD|$ vyplýva zhodnosť
úsečiek $CE$ a~$AB$. Štvoruholník $ABCE$ je preto rovnobežník a~bod~$M$ (ako stred jeho uhlopriečky~$AC$) je súčasne stredom jeho uhlopriečky~$BE$. Úsečka~$DM$ je teda strednou priečkou v~trojuholníku $BCE$, čiže je rovnobežná s~jeho stranou~$BC$, čo sme chceli dokázať.

\poznamka
Dôkaz rovnobežnosti priamok $DM$ a $BC$ možno podobne urobiť s~využitím stredu~$F$ základne~$AB$ uvažovaného lichobežníka $ABCD$ (čím vznikne rovnobežník $AFCD$). Iným možným postupom je zostrojiť priesečník~$G$ priamok $AD$ a~$BC$ a~využiť vlastnosti stredných priečok v~trojuholníkoch $ABG$ a~$CGA$.

\nobreak\medskip\petit\noindent
Za úplné vyriešenie úlohy dajte 6~bodov, z~toho 1~bod za pozorovanie, že trojuholník $ABC$ má oproti trojuholníku $ACD$ dvojnásobný obsah, ďalšie 2~body za odvodenie rovnosti $|AB|=2|CD|$ (alebo rovnosti $|AF|=|CD|$ vyplývajúcej z toho, že trojuholníky $AFM$ a $CDM$ majú
zhodné výšky z~vrcholu~$M$), 1~bod za objav rovnobežníka $ABCE$ (alebo rovnobežníka $AFCD$) a~posledné 2~body za dôkaz rovnobežnosti $DM$, $BC$.
\endpetit
\bigbreak
}

{%%%%%   A-S-3
Skúmajme pre ktoré hodnoty $n$ sú jednotlivé činitele zadaného súčinu deliteľné piatimi. Vypíšeme činitele pre malé hodnoty $n$ a~vypíšeme tiež ich zvyšky po delení piatimi.
$$
\vbox{\offinterlineskip
       \halign{\strut\vrule\ \hfil # \vrule&&\hbox to 2.5em{\hss$#$\hss}\vrule\cr
\noalign{\hrule}
$n$                & 1 & 2 & 3 &  4 &  5 &  6 &   7 &   8 & \dots\cr
\noalign{\hrule height1pt}
$2^n+1$            & 3 & 5 & 9 & 17 & 33 & 65 & 129 & 257 & \dots\cr
\noalign{\hrule}
zvyšok po delení 5 & 3 & 0 & 4 &  2 &  3 &  0 &   4 &   2 &\dots\cr
\noalign{\hrule height1pt}
$3^n+2$            & 5 & 11 & 29 & 83 & 245 & 731 & 2\,189 & 6\,563 & \dots\cr
\noalign{\hrule}
zvyšok po delení 5 & 0 &  1 &  4 &  3 &  0 &    1 &   4    &   3 &\dots\cr
\noalign{\hrule}
}}
$$

Ako možno uhádnuť z~tabuľky, postupnosť zvyškov činiteľa $2^n+1$ po delení piatimi je tvorená štvoricou $3$, $0$, $4$, $2$, ktorá sa periodicky opakuje. Dokázať to môžeme napríklad tak, že ukážeme, že čísla $2^n+1$ a~$2^{n+4}+1$ dávajú pre každé prirodzené $n$ po delení piatimi rovnaký zvyšok, teda že ich rozdiel je deliteľný piatimi:
$$
(2^{n+4}+1)-(2^n+1)=2^{n+4}-2^n = 2^n(2^4-1)=5\cdot3\cdot2^n.
$$
Podobne postupnosť zvyškov činiteľa $3^n+2$ po delení piatimi tvorí periodicky sa opakujúca štvorica $0$, $1$, $4$, $3$, lebo rozdiel
$$
(3^{n+4}+2)-(3^n+2)=3^{n+4}-3^n=3^n(3^4-1)=5\cdot16\cdot3^n
$$
je pre každé prirodzené $n$ deliteľný piatimi.

Z~uvedeného vyplýva, že $2^n+1$ je deliteľné piatimi len pre $n=2,6,10,\dots$, zatiaľ čo $3^n+2$ je deliteľné piatimi len pre $n=1,5,9,\dots$, čiže pre žiadne $n$ nie sú piatimi deliteľné oba činitele zadaného súčinu. Aby bol teda súčin deliteľný číslom $5^n$, musí ním byť deliteľný jeden z~činiteľov. Avšak pre každé $n\ge2$ je zrejme $5^n>3^n+2$ a~tiež $5^n>2^n+1$, takže $5^n$ nemôže deliť ani jedného z~činiteľov. Jedine pre $n=1$ máme $5^1=3^1+2$.
Preto jediné vyhovujúce číslo je $n=1$.

\nobreak\medskip\petit\noindent
Za úplné vyriešenie úlohy dajte 6~bodov. Po jednom bode (spolu dva body) dajte za objav postupností zvyškov po delení piatimi pre každý činiteľ (body udeľte aj v~prípade, že riešiteľ len bez dôkazu prehlási, že postupnosti sú periodické, nakoľko táto skutočnosť je dostatočne evidentná a~známa), resp. za akékoľvek správne zdôvodnenie, že $5\mid 2^n+1$ len pre $n$ tvaru $4k+2$ a~$5\mid3^n+2$ len pre $n$ tvaru $4k+1$. Ďalšie dva body dajte za úvahu, že $5^n$ musí deliť jedného činiteľa; piaty bod dajte za zdôvodnenie, že pre $n\ge2$ to nie je možné (pritom zrejmé nerovnosti
$5^n>3^n+2$ a  $5^n>2^n+1$ nie je nutné zdôvodňovať); posledný bod za uvedenie správnej odpovede $n=1$. Žiak, ktorý bez akéhokoľvek zdôvodnenia iba prehlási, že jediné vyhovujúce je $n=1$, dostane 1~bod. Ak žiak považuje za prirodzené číslo aj $n=0$ a~uvedie ho tiež v~odpovedi, body nestŕhajte.
\endpetit
\bigbreak
}

{%%%%%   A-II-1
Súčet $S_n$ nájdeme tak, že zistíme, koľkokrát sa ktorá cifra nachádza vo všetkých uvažovaných číslach na mieste jednotiek, desiatok, stoviek, atď. Potom určíme, aký je "príspevok" jednotlivých cifier do celkového súčtu.

Ak je cifra~$1$ na mieste jednotiek, tak zvyšných $n-1$ pozícií môžeme zaplniť $3^{n-1}$ spôsobmi (pre každú pozíciu máme tri možnosti). Takto sme ale bohužiaľ započítali aj čísla zložené len z~jednotiek a~dvojok, teda čísla neobsahujúce {\it aspoň jednu} cifru~$3$; tých je zrejme $2^{n-1}$. Takisto nesmieme započítať ani čísla zložené len z~jednotiek a~trojok, ktorých je tiež $2^{n-1}$. (Stále počítame s~cifrou~$1$ na mieste jednotiek.)\footnote{Čísla majúce na zvyšných $n-1$ pozíciách len dvojky a~trojky započítavať budeme, pretože požadovanú aspoň jednu (a~zároveň práve jednu) cifru~$1$ majú na mieste jednotiek.}
Keďže číslo zložené zo samých jednotiek sa nachádza v~obidvoch nesprávne započítaných skupinách (a~žiadne iné také číslo nie je), navyše sme započítali $2\cdot2^{n-1}-1$ čísel. Cifra $1$ sa teda na mieste jednotiek nachádza $k$-krát, pričom $k=3^{n-1}-(2\cdot2^{n-1}-1)=3^{n-1}-2^n+1$.

Rovnako veľa krát sa nachádza cifra~$1$ aj na každej ďalšej pozícii (za rovnakých podmienok zapĺňame vždy $n-1$ zvyšných pozícií). Pre príspevok $p$ cifry~$1$ do celkového súčtu preto platí
$$
p=k+10k+100k+\cdots+10^{n-1}k=(1+10+100+\cdots+10^{n-1})k=\frac{10^n-1}9k.
$$

Príspevok cifry $2$, resp. $3$, je zrejme 2-krát, resp. 3-krát väčší ako príspevok cifry~$1$, keďže na jednotlivých pozíciách sa tieto cifry nachádzajú rovnako veľa krát ako cifra~$1$. Spolu máme
$$
S_n=p+2p+3p=6p=6\cdot\frac{10^n-1}9k=\frac23(10^n-1)(3^{n-1}-2^n+1).
$$
Toto číslo je deliteľné {\it prvočíslom}~$7$ práve vtedy, keď je ním deliteľný aspoň jeden z~činiteľov $10^n-1$, $3^{n-1}-2^n+1$ (činiteľ $\frac23$ deliteľnosť siedmimi samozrejme neovplyvňuje). Vypíšeme činitele pre malé hodnoty $n$ a~vypíšeme tiež ich zvyšky po delení siedmimi.
$$
\vbox{\offinterlineskip
       \halign{\strut\vrule\ \hfil # \vrule&&\hbox to 3.5em{\hss$#$\hss}\vrule\cr
\noalign{\hrule}
$n$                & 1 & 2 & 3 &  4 &  5 &  6 &   7 &   8 \cr
\noalign{\hrule height1pt}
$10^n-1$            & 9 & 99 & 999 & 9\,999 & 99\,999 & 999\,999 & \dots &  \cr
\noalign{\hrule}
zvyšok po delení 7    &2 & 1 & 5 & 3 &  4 &  0 &  2 &   1 \cr
\noalign{\hrule height1pt}
$3^{n-1}-2^n+1$          & 0 & 0 & 2 & 12 & 50 & 180 & 602 & 1\,932 \cr
\noalign{\hrule}
zvyšok po delení 7 & 0 &  0 &  2 &  5 &  1 &    5 &   0    &   0 \cr
\noalign{\hrule}
}}
$$

Ako možno uhádnuť z~tabuľky, postupnosť zvyškov činiteľa $10^n-1$ po delení siedmimi je tvorená šesticou $2$, $1$, $5$, $3$, $4$, $0$, ktorá sa periodicky opakuje\footnote{Pri vypĺňaní tabuľky nemusíme prácne deliť siedmimi čísla $10^n-1$. Stačí využiť, že $10^{n+1}$ dáva po delení siedmimi rovnaký zvyšok ako 10-násobok {\it zvyšku\/} čísla~$10^n$.}. Dokázať to môžeme napríklad tak, že ukážeme, že čísla $10^n-1$ a~$10^{n+6}-1$ dávajú pre každé prirodzené $n$ po delení siedmimi rovnaký zvyšok, teda že ich rozdiel je deliteľný siedmimi:
$$
(10^{n+6}-1)-(10^n-1)=10^{n+6}-10^n = 10^n(10^6-1)=7\cdot142\,857\cdot10^n.
$$
(Pri poslednej úprave sa možno vyhnúť priamemu deleniu a~skonštatovať, že z~malej Fermatovej vety vyplýva $7\mid10^6-1$.)

Postupnosť zvyškov činiteľa $3^{n-1}-2^n+1$ po delení siedmimi je taktiež periodická a~tvorí ju opakujúca sa šestica $0$, $0$, $2$, $5$, $1$, $5$, pretože rozdiel
$$
\align
(3^{n+5}-2^{n+6}+1)-(&3^{n-1}-2^n+1)=(3^{n+5}-3^{n-1})-(2^{n+6}-2^n)=\\
&=3^{n-1}(3^6-1)-2^n(2^6-1)=7\cdot(104\cdot3^{n-1}-9\cdot2^n)
\endalign
$$
je pre každé prirodzené $n$ deliteľný siedmimi.

Z~uvedeného vyplýva, že $S_n$ je deliteľné siedmimi (\tj. dáva zvyšok~$0$ po delení siedmimi) práve pre tie prirodzené čísla~$n\ge3$, ktoré možno zapísať v~tvare $6m$, $6m+1$ alebo $6m+2$ pre nejaké prirodzené číslo~$m$, čiže pre čísla $n$ dávajúce po delení šiestimi zvyšok $0$, $1$ alebo $2$.

\nobreak\medskip\petit\noindent
Za správne riešenie úlohy dajte 6~bodov. Za odvodenie vzorca pre výpočet $S_n$ dajte 3~body; za kompletnú analýzu, pre ktoré $n$ je $10^n-1$, resp. $3^{n-1}-2^n+1$ deliteľné siedmimi, dajte 1~bod, resp. 2~body.

Pri analýze výrazu $10^n-1$ bod udeľte aj v~prípade, že riešiteľ len bez dôkazu prehlási, že postupnosť zvyškov je periodická od prvého opakujúceho sa člena, nakoľko táto skutočnosť je dostatočne evidentná a~známa. Pri výraze $3^{n-1}-2^n+1$ je periodickosť s~periódou dĺžky~$6$ nutné zdôvodniť (napr. tak, ako je uvedené v~riešení, alebo osobitnou analýzou zvyškov jednotlivých mocnín $3^{n-1}$ a~$2^n$), bez toho z~dvoch bodov jeden strhnite.

Ak žiak nesprávne odvodí vzorec pre $S_n$ tak, že započíta aj čísla zložené len z~dvoch rôznych cifier, \tj. pracuje so vzťahom $S_n=\frac23(10^n-1)3^{n-1}$, za prvú časť udeľte 1~bod (za správne určenie jednotlivých príspevkov, resp. iné odvodenie tohto vzťahu) a za druhú časť tiež 1~bod (za správnu analýzu deliteľnosti siedmimi výrazu $10^n-1$), spolu teda najviac 2~body.

Ak žiak odvodí vzorec v~tvare $S_n=\frac23(10^n-1)(3^{n-1}-2^n)$ (\tj. nesprávne aplikuje princíp inklúzie a~exklúzie a~zabudne pripočítať $1$), ktorý ďalej analyzuje správne, za prvú časť dajte 1~bod a~za druhú 2~body (po jednom za každý činiteľ), teda spolu najviac 3~body.

\endpetit
\bigbreak
}

{%%%%%   A-II-2
Aritmetická postupnosť $\Cal A$ s~prvým členom~$a$ a~kladnou diferenciou~$d$
obsahuje práve tie z~čísel $a^2$, $a^3$, $a^4$, $a^5$, pre ktoré
je príslušný rozdiel
$$\aligned
a^2-a&=a(a-1),\\
a^3-a&=a(a-1)(a+1),\\
a^4-a&=a(a-1)(a^2+a+1),\\
a^5-a&=a(a-1)(a+1)(a^2+1)\\
\endaligned
\tag1$$
celým násobkom čísla~$d$. Predpokladajme, že $\Cal A$ obsahuje práve dve z~čísel $a^2$, $a^3$, $a^4$, $a^5$.

Ak $a^2\in\Cal A$, tak prvý rozdiel $a(a-1)$ je celým násobkom~$d$. Keďže $a\in\Bbb Z$, sú potom aj ostatné tri rozdiely v~\thetag1 (ktoré sú očividne celočíselnými násobkami prvého rozdielu) celými násobkami~$d$. To však znamená, že $\Cal A$ obsahuje všetky čísla $a^2$, $a^3$, $a^4$, $a^5$, čo je v~spore s~predpokladom, že obsahuje práve dve z~nich. Preto $a^2\notin\Cal A$, teda $\Cal A$ obsahuje práve dve z~čísel $a^3$, $a^4$, $a^5$.

Ak $a^3\notin\Cal A$, tak nutne $a^4,a^5\in\Cal A$, čiže výrazy
$$
\frac{a^4-a}d,\qquad\frac{a^5-a}d
$$
sú oba celočíselné. Potom je celým číslom aj ich kombinácia
$$
\frac{a^5-a}{d}-a\cdot\frac{a^4-a}{d}=\frac{a^2-a}{d},
$$
teda $a^2\in\Cal A$, čo je spor. Preto $a^3\in\Cal A$.

Dokázali sme, že hľadaná aritmetická postupnosť~$\Cal A$ musí obsahovať číslo~$a^3$. Pre jej diferenciu~$d$ tak platí odhad $d\le a^3-a$, pričom rovnosť nastane (\tj. diferencia bude najväčšia) v~prípade, že $a^3$ je jej druhým členom. Aritmetická postupnosť s~prvým členom~$a$ a~diferenciou $d=a^3-a$ určite neobsahuje číslo~$a^2$, obsahuje $a^3$ a~obsahuje aj $a+(a^2+1)(a^3-a)=a^5$. Stačí už iba overiť, že neobsahuje $a^4$. To vyplýva z~toho\footnote{Argumentovať možno aj takto: V~predošlom odseku sme všeobecne dokázali, že ak $a^4,a^5\in\Cal A$, tak $a^2\in\Cal A$. My už o~$\Cal A$ vieme, že obsahuje $a^5$ a~neobsahuje $a^2$. Preto nemôže obsahovať $a^4$.}, že výraz
$$
\frac{a^4-a}{d}=\frac{a^4-a}{a^3-a}=\frac{a^2+a+1}{a+1}=a+\frac{1}{a+1}
$$
nie je celým číslom pre žiadne kladné celé číslo~$a$.
\zaver
Hľadaná postupnosť je aritmetická postupnosť s~prvým členom $a$ a~diferenciou $d=a^3-a$.

\poznamka
Ak uhádneme výsledok a~ukážeme, že postupnosť s~diferenciou $d=a^3-a$ obsahuje $a^3$, $a^5$ a~neobsahuje $a^2$, $a^4$, na dokončenie riešenia už stačí ukázať iba to, že diferencia nemôže byť väčšia ako $a^3-a$. Ak je však diferencia väčšia, tak $\Cal A$ zrejme neobsahuje $a^2$ ani $a^3$, musí teda obsahovať $a^4$ aj $a^5$ a~z~toho rovnako ako v~uvedenom riešení možno odvodiť $a^2\in\Cal A$ a~tým dôjsť k~sporu.

\nobreak\medskip\petit\noindent
Za správne riešenie úlohy dajte 6~bodov.

Ak žiak správne odvodí odhad $d\le a^3-a$, skonštatuje, že najväčšia možná diferencia je $a^3-a$, ale neukáže, že postupnosť s~touto diferenciou vyhovuje, dajte len 4~body. Ďalší 1~bod dajte až za overenie, že v~tomto prípade $a^5\in\Cal A$, resp. 1~bod za dôkaz, že $a^4\notin\Cal A$.

Ak naopak žiak ukáže, že postupnosť s~diferenciou $a^3-a$ obsahuje $a^3$, $a^5$ a~neobsahuje $a^2$, $a^4$, ale nevysvetlí, prečo diferencia nemôže byť väčšia, dajte len 3~body.

V~prípade, že sa riešiteľ nedostane k~úvahám o~diferencii $a^3-a$, dajte 1~bod za dôkaz, že $a^2\notin\Cal A$ (\tj. za dôkaz implikácie $a^2\in\Cal A\ \Rightarrow\ a^3,a^4,a^5\in\Cal A$), 1~bod za dôkaz implikácie $a^3\in\Cal A\ \Rightarrow\ a^5\in\Cal A$ a 3~body za dôkaz implikácie $a^4,a^5\in\Cal A\ \Rightarrow\ a^2\in\Cal A$; spolu však v~tejto vetve najviac 4~body.

\endpetit
\bigbreak
}

{%%%%%   A-II-3
Označme $k$ kružnicu opísanú zadanému šesťuholníku. Keďže $AB\perp BD$, je~$k$ Tálesovou kružnicou nad priemerom~$AD$ a~stred~$S$ uhlopriečky~$AD$ je zároveň stredom kružnice~$k$.

Dokážeme, že trojuholník $KLD$ má pravý uhol pri vrchole~$D$. Veľkosť uhla $KDL$ je súčtom veľkostí uhlov $KDS$ a~$LDS$. Trojuholníky $KDS$, $KBS$ sú zhodné podľa vety {\it sus\/}: stranu $SK$ majú spoločnú, strany $SD$, $SB$ sú obe polomerom kružnice~$k$ a~uhol pri vrchole $S$ majú trojuholníky zhodný, lebo $SK$ je osou uhla $BSP$ (\obr). Preto $|\uhol KDS|=|\uhol KBS|$. Odtiaľ vzhľadom na to, že $BK$ je osou uhla $SBP$, vyplýva $|\uhol KDS|=\frac12|\uhol CBS|$.

Analogickou úvahou (s~využitím zhodnosti trojuholníkov $LDS$, $LES$) dostaneme $|\uhol LDS|=\frac12|\uhol QES|$.
\insp{a61.18}%

Pre dokončenie riešenia stačí aplikovať zatiaľ nepoužitý predpoklad $|BC|=|EF|$. Vďaka nemu sú podľa vety {\it sss\/} trojuholníky $BCS$, $EFS$ zhodné  (všetky ich zvyšné strany sú polomermi kružnice~$k$), teda $|\uhol CBS|=|\uhol FES|$. Spojením odvodených poznatkov máme
$$
\align
|\uhol KDL|&=|\uhol KDS|+|\uhol LDS|=\frac12|\uhol CBS|+\frac12|\uhol QES|=\frac12|\uhol FES|+\frac12|\uhol QES|=\\
&=\frac12(|\uhol FES|+|\uhol QES|)=\frac12\cdot180\st=90\st.
\endalign
$$

\nobreak\medskip\petit\noindent
Za správne riešenie úlohy dajte 6~bodov. Z~toho 1~bod za pozorovanie, že $AD$ je priemerom~$k$; 3~body za odvodenie rovností $|\uhol KDS|=\frac12|\uhol CBS|$, $|\uhol LDS|=\frac12|\uhol QES|$ (ak má riešiteľ len jednu z~nich, dajte 2~body; ak ukáže len $|\uhol KDS|=|\uhol KBS|$ a/alebo $|\uhol LDS|=|\uhol LES|$, dajte 1~bod); 2~body za rovnosť $|\uhol CBS|=|\uhol FES|$ a~úspešné dokončenie riešenia.
\endpetit
\bigbreak
}

{%%%%%   A-II-4
Keďže zadané rovnosti obsahujú zmiešané súčiny premenných, výhodné je skúmať druhú mocninu súčtu $a+b+c+d$. Úpravou s~dosadením zadaných rovností dostaneme
$$
\aligned
(a+b+c+d)^2&=a^2+b^2+c^2+d^2+2(ab+cd+ac+bd+ad+bc)=\\
=&a^2+b^2+c^2+d^2+2(4+4+5)=a^2+d^2+b^2+c^2+26.
\endaligned
\tag1
$$
Využijeme známe nerovnosti $a^2+d^2\ge2ad$, $b^2+c^2\ge2bc$, v~ktorých rovnosť platí práve vtedy, keď $a=d$ a~$b=c$. Na základe toho z~\thetag1 máme
$$
(a+b+c+d)^2\ge 2ad+2bc+26=2\cdot5+26=36.
$$
Preto pre kladné čísla $a$, $b$, $c$, $d$ musí platiť $a+b+c+d\ge\sqrt{36}=6$.

Rovnosť platí práve vtedy, keď $a=d$ a~$b=c$. Dosadením do pôvodných rovností dostaneme sústavu
$$
2ab=4,\qquad a^2+b^2=5.
$$
Tú možno vyriešiť viacerými rôznymi postupmi. Napríklad môžeme vyjadriť
$$
(a+b)^2=a^2+b^2+2ab=5+4=9,
$$
teda $a+b=3$ (keďže $a,b>0$). Podľa Vi\`etových vzťahov sú $a$, $b$ koreňmi kvadratickej rovnice $x^2-3x+2=0$, teda $\{a,b\}=\{1,2\}$. Ľahko overíme, že štvorice $a=d=1$, $b=c=2$, resp. $a=d=2$, $b=c=1$ skutočne spĺňajú zadané rovnosti a~platí pre ne $a+b+c+d=6$.

\odpoved
Najmenšia možná hodnota súčtu $a+b+c+d$ je $6$, dosahujú ju štvorice $(1,2,2,1)$ a~$(2,1,1,2)$.

\ineriesenie
Z~rovnosti $ab+cd=ac+bd$ vyplýva $a(b-c)=d(b-c)$, takže platí $a=d$ alebo $b=c$. Vzhľadom na symetriu môžeme uvažovať iba vyhovujúce štvorice $(a,b,c,d)$, v~ktorých je $d=a$, a~hľadať tak najmenšiu hodnotu súčtu $S=2a+b+c$ za predpokladu, že kladné čísla $a$, $b$, $c$ spĺňajú rovnosti
$a(b+c)=4$ a~$a^2+bc=5$.

Podľa Vi\`etových vzťahov sú potom kladné čísla $b$, $c$ koreňmi kvadratickej rovnice
$$
x^2-\frac{4}{a}x+(5-a^2)=0.
$$
Tá má dva kladné (nie nutne rôzne) korene práve vtedy, keď je jej diskriminant
$$
D=\frac{16}{a^2}-4(5-a^2)=\frac{4(a^2-1)(a^2-4)}{a^2}
$$
nezáporný a~keď okrem nerovnosti $a>0$ platí aj $a^2<5$. Dokopy to znamená, že $a\in(0,1\rangle\cup\bigl\langle2,\sqrt5\bigr)$.

Všimnime si, že pre $a=1$ vychádzajú korene $b=c=2$, naopak pre $a=2$ je $b=c=1$. V~oboch týchto prípadoch má zrejme výraz $S=2a+b+c$ hodnotu~$6$. Ak ukážeme, že pre ostatné prípustné $a$ platí $S>6$, bude to znamenať, že $\min S=6$ a~že $(1,2,2,1)$ a~$(2,1,1,2)$ sú jediné štvorice dávajúce nájdené minimum (keďže v~oboch z~nich platí $b=c$, žiadne iné také štvorice~-- napriek obmedzeniu našich úvah na prvú z~možností $a=d$, $b=c$~-- neexistujú). Z~vyjadrenia rozdielu $S-6$ v~tvare
$$
S-6=2a+b+c-6=2a+\frac{4}{a}-6=\frac{2(a-1)(a-2)}{a}
$$
vidíme, že žiadaná nerovnosť $S>6$ naozaj platí pre každé $a\in(0,1)\cup\bigl(2,\sqrt5\bigr)$.

\ineriesenie
Sčítaním rovníc $ab+cd=4$ a~$ad+bc=5$ dostaneme $(a+c)(b+d)=9$, odkiaľ podľa AG-nerovnosti pre dvojicu čísel $a+c$ a~$b+d$ obdržíme
$$
(a+c)+(b+d)\ge2\sqrt{(a+c)(b+d)}=2\sqrt9=6.
$$
Rovnosť nastane práve vtedy, keď bude platiť $a+c=b+d=3$ (a~tiež $a+b=c+d=3$, čo
odvodíme rovnakou úvahou z~podobne získanej rovnosti $(a+b)(c+d)=9$). Z~toho
už ľahko zistíme (napr. postupom ako v~závere prvého riešenia), že minimálnu hodnotu $a+b+c+d=6$ dávajú len
vyhovujúce štvorice $b=c=1$ a~$a=d=2$, resp. $b=c=2$ a~$a=d=1$.

\nobreak\medskip\petit\noindent
Za správne riešenie úlohy dajte 6~bodov, z~toho 4~body za odhad $a+b+c+d\ge6$ a~2~body za určenie štvoríc, pre ktoré platí rovnosť.

Ak riešiteľ na odhad štvorcov v~\thetag1 použije AG-nerovnosti $a^2+b^2\ge2ab$, $c^2+d^2\ge2cd$ a~odvodí tak slabší odhad $a+b+c+d\ge\sqrt{34}$, dajte 2~body.

Pri postupe ako v~druhom riešení dajte 1~bod za odvodenie rovnosti $(a-d)(b-c)=0$, 1~bod za prechod k~riešeniu sústavy o~dvoch neznámych $b$, $c$ s~parametrom~$a$, 2~body za nájdenie množiny prípustných hodnôt~$a$ (alebo aspoň odvodenie nutnej podmienky $a\le 1 \vee a\ge2$), 1~bod za dôkaz nerovnosti $2a+4/a\ge6$ pre všetky prípustné~$a$ a~1~bod za určenie oboch hľadaných štvoríc.

V~prípade, že žiak len uhádne výsledok a~objaví štvorice $(1,2,2,1)$ a~$(2,1,1,2)$, dajte 1~bod (tento bod udeľte len v~prípade, že žiak v~úlohe nezíska žiadne iné body).


\endpetit
\bigbreak
}

{%%%%%   A-III-1
Zadaný výraz možno jednoduchou úpravou rozložiť na súčin:
$$
n^4-3n^2+9 = n^4+6n^2+9-9n^2 = (n^2+3)^2-(3n)^2  =(n^2+3n+3)(n^2-3n+3).
$$
Oba činitele sú celými číslami, teda deliteľmi súčinu. Aby súčin bol prvočíslom~$p$, musí byť niektorý z~činiteľov rovný $1$, resp. $\m1$ (a~druhý $p$, resp. $\m p$). Avšak diskriminant oboch kvadratických činiteľov je $(\pm3)^2-4\cdot4=\m3$, teda záporný, čiže oba nadobúdajú len kladné hodnoty. Vzhľadom na to nemôžu nadobúdať hodnotu $\m1$ a~stačí uvažovať kvadratické rovnice
$$
n^2+3n+3=1 \qquad\text{a}\qquad n^2-3n+3=1.
$$

Riešením prvej rovnice sú hodnoty $n=\m1$ a~$n=\m2$, pre ktoré druhý činiteľ nadobúda hodnoty $7$ a~$13$, čo sú prvočísla.

Riešením druhej rovnice sú $n=1$ a~$n=2$, pre ktoré prvý činiteľ nadobúda opäť prvočíselné hodnoty $7$ a~$13$.

\odpoved Zadaný výraz je prvočíslom práve vtedy, keď $n\in\{\m2,\m1,1,2\}$.
}

{%%%%%   A-III-2
Označme $T$ ťažisko trojuholníka $ABC$ a~$K$, $L$, $M$ stredy strán $BC$, $CA$, $AB$. Ťažnice delia trojuholník $ABC$ na šesť menších trojuholníkov s~rovnakým obsahom: Napr. trojuholník $AMT$ má stranu $|AM|=\frac12c$ a~jeho výška na stranu~$AM$ má zrejme veľkosť $\frac13v_c$, takže $S_{AMT}=\frac12\cdot\frac12c\cdot\frac13v_c=\frac16\cdot\frac12c\cdot v_c=\frac16S_{ABC}$ a~analogicky to platí aj pre zvyšných päť trojuholníkov.

Úloha určiť najväčší možný obsah trojuholníka $ABC$ je teda ekvivalentná s~úlohou určiť najväčší možný obsah jedného zo šiestich menších trojuholníkov -- výsledok stačí vynásobiť šiestimi.
\insp{a61.19}%

Uvažujme napríklad trojuholník $ATL$ (\obr). Pre jeho dve strany platí
$$
|AT|=\frac23t_a\le\frac43,\qquad |TL|=\frac13t_b\le1.
$$
Preto pre jeho obsah dostávame
$$
S_{ATL}=\frac12|AT|\cdot|TL|\cdot\sin|\uhol ATL|\le\frac12\cdot\frac43\cdot1\cdot\sin|\uhol ATL|=\frac23\sin|\uhol ATL|\le\frac23.
$$
Tým sme dokázali, že obsah trojuholníka $ABC$, ktorého ťažnice spĺňajú zadané nerovnosti, nemôže byť väčší ako $6\cdot\frac23=4$. Pritom rovnosť $S_{ATL}=\frac23$ (\tj. $S_{ABC}=4$) nastane len vtedy, keď $t_a=2$, $t_b=3$ a~$|\uhol ATL|=90\st$.\footnote{Rovnaký výsledok vieme ľahko dostať aj bez použitia funkcie sínus: Pre výšku~$v$ na stranu~$AT$ v~trojuholníku $ATL$ zrejme platí $v\le|TL|$, takže $S_{ATL}=\frac12v|AT|\le\frac12|TL||AT|$, pričom rovnosť platí, keď $AT\perp TL$.}

Trojuholník $ABC$ s~takýmito vlastnosťami vieme ľahko "zrekonštruovať": Najskôr narysujeme pravouhlý trojuholník $ATL$, v~ktorom poznáme dĺžky oboch odvesien $|AT|=\frac43$, $|TL|=1$ a~následne zostrojíme bod~$C$ ako obraz bodu~$A$ v~stredovej súmernosti so stredom~$L$ a~bod~$B$ ako obraz bodu~$L$ v~rovnoľahlosti so stredom~$T$ a~koeficientom $\m2$ (\obr). Stačí už len overiť, že v~takomto trojuholníku $ABC$ platí $t_c\le4$.
\insp{a61.20}%

Dĺžku $t_c$ možno vypočítať rôznymi spôsobmi. Napríklad v~pravouhlom trojuholníku $ABT$ má prepona~$AB$ podľa Pytagorovej vety dĺžku
$$
|AB|=\sqrt{|AT|^2+|TB|^2}=\sqrt{\frac{16}9+4}=\sqrt{\frac{52}9}=\frac23\sqrt{13},
$$
takže veľkosť polomeru Tálesovej kružnice nad priemerom~$AB$ je
$$
|MT|=\frac12|AB|=\frac13\sqrt{13}.
$$
Odtiaľ $t_c=3|MT|=\sqrt{13}<4$.\footnote{Ak si nevšimneme Tálesovu kružnicu nad $AB$, môžeme postupovať tak, že aj z~pravouhlých trojuholníkov $ATL$ a~$BTK$ dopočítame prepony, ktoré sú polovicami strán $b$, $a$ trojuholníka $ABC$: $\frac12b=\sqrt{1+\frac{16}9}=\frac53$, $\frac12a=\sqrt{4+\frac49}=\frac23\sqrt{10}$. Teda $a=\frac43\sqrt{10}$, $b=\frac{10}3$, $c=\frac23\sqrt{13}$ a~na výpočet ťažnice môžeme použiť známy vzorec $t_c=\frac12\sqrt{2a^2+2b^2-c^2}$.}

\odpoved
Najväčší možný obsah trojuholníka $ABC$ je $4$.

\goodbreak
\ineriesenie
Obsah trojuholníka $ABC$ možno vypočítať podľa vzorca
$$
S_{ABC}=\frac13\sqrt{2(t_a^2t_b^2+t_b^2t_c^2+t_c^2t_a^2)-(t_a^4+t_b^4+t_c^4)}.\tag1
$$
Tento vzorec možno ľahko odvodiť nasledovnou úvahou: Ak zobrazíme trojuholník $ABC$ v~stredovej súmernosti so stredom~$K$ (pri označení bodov ako v~prvom riešení) a~obrazy bodov~$M$, $A$ označíme $M'$, $A'$, tak trojuholník $BM'L$ má zrejme dĺžky strán rovné $t_a$, $t_b$, $t_c$ (\obr). Pritom jeho obsah tvorí $\frac38$ obsahu rovnobežníka $ABA'C$, čiže $\frac34$ obsahu trojuholníka $ABC$. Takže podľa Herónovho vzorca máme $S_{ABC}=\frac43S_{BM'L}=\frac43\sqrt{u(u-t_a)(u-t_b)(u-t_c)}$, pričom $u=\frac12(t_a+t_b+t_c)$, a~po úprave dostaneme \thetag1.
\insp{a61.21}%

Označme $t_a^2=x$, $t_b^2=y$, $t_c^2=z$. Výraz na pravej strane \thetag1 má najväčšiu hodnotu vtedy, keď v~ňom má najväčšiu hodnotu výraz pod odmocninou. Zabudnime teda na geometrický význam vzorca a~hľadajme najväčšiu možnú hodnotu výrazu
$$
V=2(xy+yz+zx)-(x^2+y^2+z^2)
$$
pri podmienkach $0\le x\le 4$, $0\le y\le 9$, $0\le z\le 16$.

Ak zvolíme hodnoty $x$, $y$ pevné, výraz~$V$ je kvadratický v~premennej~$z$. Úpravou na štvorec dostaneme
$$
V=-\bigl(z-(x+y)\bigr)^2+4xy\le 4xy\le4\cdot4\cdot9=144.
$$
Rovnosť pritom nastane pre $x=4$, $y=9$ a~$z=x+y=13$. Našli sme teda najväčšiu možnú hodnotu výrazu~$V$ pri stanovených podmienkach. Keď tento výsledok dosadíme do \thetag1, obdržíme
$$
S_{ABC}=\frac13\sqrt{V}\le\frac13\sqrt{144}=4.
$$
Rovnosť $S_{ABC}=4$ je splnená pre trojuholník s~ťažnicami $t_a=\sqrt{4}=2$, $t_b=\sqrt{9}=3$ a~$t_c=\sqrt{13}<4$. Trojuholník s takýmito dĺžkami ťažníc naozaj existuje -- uvedené dĺžky totiž spĺňajú trojuholníkové nerovnosti ($2+3>\sqrt{13}$), takže vieme zostrojiť trojuholník $BM'L$ z~\obrr1{} ($|BM'|=t_c=\sqrt{13}$, $|M'L|=t_a=2$, $|LB|=t_b=3$) a~následne dorysovať trojuholník $ABC$ (bod~$K$ je ťažiskom trojuholníka $BM'L$, vrchol~$C$ je obrazom $B$ v~stredovej súmernosti podľa $K$, vrchol~$A$ je obrazom $C$ v~stredovej súmernosti podľa $L$).
}

{%%%%%   A-III-3
Ekvivalentnými úpravami nerovnosti zo zadania (s~využitím toho, že výraz $1+x^2$ je kladný pre každé $x\in\Bbb R$) dostaneme
$$
\align
100\left|(u-v)(1-uv)\right|&\le (1 + u^2)(1 + v^2),\\
100\left|u-v-u^2v+uv^2\right|&\le (1 + u^2)(1 + v^2),\\
\left|u(1+v^2)-v(1+u^2)\right|&\le \frac1{100}(1 + u^2)(1 + v^2),\\
\left|\frac{u}{1 + u^2}-\frac{v}{1 + v^2}\right|&\le\frac1{100}.\tag1
\endalign
$$

Všetky hodnoty funkcie
$$
f(x)=\frac{x}{1+x^2},\quad x\in\Bbb R
$$
ležia v~intervale $\langle\m\frac12,\frac12\rangle$, pretože pre každé reálne číslo~$x$ platí
$$
\aligned
(1-x)^2&\ge0,\\
1+x^2&\ge 2x,\\
\frac12&\ge\frac x{1+x^2}
\endaligned
\qquad\text{a}\qquad
\aligned
(1+x)^2&\ge0,\\
1+x^2&\ge-2x,\\
-\frac12&\le\frac x{1+x^2}.
\endaligned
$$
Rozdeľme interval $\langle\m\frac12,\frac12\rangle$ s~dĺžkou~$1$ rovnomerne na sto malých intervalov s~dĺžkou~$\frac1{100}$. Podľa Dirichletovho princípu medzi ľubovoľnými 101 číslami nájdeme dve čísla $u$,~$v$ také, že $f(u)$, $f(v)$ ležia v~tom istom malom intervale. Pre túto dvojicu zrejme platí ${|f(u)-f(v)|\le\frac1{100}}$, čo je presne nerovnosť \thetag1, ktorá je ekvivalentná so zadanou nerovnosťou.
}

{%%%%%   A-III-4
Keďže súčet obsahov oboch častí, na ktoré priamka delí rovnobežník $ABCD$, je stále rovnaký, líšiť sa budú čo najviac práve vtedy, keď menší z~obsahov bude najmenší možný. Riešenie začneme pozorovaním, že ak bod~$X$ leží v~strede rovnobežníka $ABCD$, tak každá priamka, ktorá ním prechádza, delí rovnobežník na dve časti s~rovnakým obsahom. Obe časti sú totiž v~takom prípade zhodné -- jedna sa zobrazí na druhú v~stredovej súmernosti podľa stredu~$X$ (\obr).

Uvedený fakt využijeme aj pri všeobecnej polohe bodu~$X$. Predpokladajme, že bod $A'$, ktorý je obrazom bodu~$A$ v~stredovej súmernosti podľa~$X$, leží vnútri rovnobežníka $ABCD$. Ak označíme $K$, $L$, $M$, $N$ postupne stredy strán $AB$, $BC$, $CD$, $DA$ a~$S$ stred rovnobežníka $ABCD$, tak opísaná situácia nastane práve vtedy, keď $X$ leží vnútri rovnobežníka $AKSN$ (\obr).
\inspinsp{a61.22}{a61.23}%

Bodom~$A'$ veďme rovnobežky so stranami rovnobežníka a~ich priesečníky so stranami $AB$ a~$AD$ označme $P$ a~$Q$. Štvoruholník $APA'Q$ je zrejme rovnobežník a~bod~$X$ je jeho stredom. Preto každá priamka prechádzajúca cez $X$ rozdeľuje $APA'Q$ na dva útvary s~rovnakým obsahom. Každý z~týchto dvoch útvarov pritom patrí do inej časti rovnobežníka $ABCD$ (\obr a), takže žiadna z~dvoch častí rovnobežníka $ABCD$ nemôže mať obsah menší ako polovica obsahu rovnobežníka $APA'Q$. Najmenší obsah menšej časti dosiahneme, keď okrem útvaru pochádzajúceho z~rovnobežníka $APA'Q$ nebude časť obsahovať nič iné, čo je možné len v~prípade, že deliaca priamka je totožná s~priamkou~$PQ$ (\obrr1b).
\inspinspab{a61.24}{a61.25}%

Ak bod~$X$ leží vnútri rovnobežníka $KBLS$, $SLCM$, resp. $NSMD$ (\obrr2), deliacu priamku zostrojíme obdobným postupom: Pomocou obrazu bodu $B$, $C$ resp. $D$ v~stredovej súmernosti podľa $X$ zostrojíme menší rovnobežník, ktorý leží celý vnútri rovnobežníka $ABCD$, má stred $X$ a~dve jeho strany ležia na stranách pôvodného rovnobežníka. Rozdeľujúcou priamkou musí byť uhlopriečka menšieho rovnobežníka oddeľujúca jeho polovicu od zvyšku rovnobežníka $ABCD$.

Ostáva vyšetriť prípad, keď $X$ leží vnútri jednej z~úsečiek $KM$, $NL$ (mimo stredu rovnobežníka $ABCD$). Aj v~tejto situácii vieme zostrojiť menší rovnobežník, ktorý celý leží v~rovnobežníku $ABCD$, bod~$X$ je jeho stredom a~strany (tentoraz až tri) má na stranách pôvodného rovnobežníka.
\inspinspab{a61.26}{a61.27}%

Ak $X$ leží vnútri úsečky~$KS$, tak takým rovnobežníkom je $ABA'B'$, pričom $A'$, $B'$ sú obrazy bodov $A$, $B$ v~stredovej súmernosti podľa $X$.
Aj v~tomto prípade musíme deliacu priamku cez $X$ viesť tak, aby jedna z~častí rovnobežníka $ABCD$ neobsahovala okrem útvaru pochádzajúceho z~rovnobežníka $ABA'B'$ nič iné. Je zrejmé, že vyhovujúcou bude práve každá priamka~$UX$, kde $U$ je ľubovoľný bod úsečky~$AB'$ (\obr a, b).

Analogicky nájdeme deliace priamky v~prípade, že $X$ leží vnútri niektorej z~úsečiek $SM$, $NS$ či $SL$.

\zaver
Ak $X$ je stredom rovnobežníka $ABCD$, riešením je ľubovoľná priamka prechádzajúca cez $X$. Ak $X$ leží mimo úsečiek $KM$, $NL$, riešením je jediná priamka. Ak $X$ leží vnútri niektorej z~úsečiek $KS$, $SM$, $NS$, $SL$, riešením je nekonečne veľa priamok. V~každom z~týchto prípadov je konštrukcia zrejmá z~predošlých úvah.
}

{%%%%%   A-III-5
Rozdelenie vyhovujúce zadaniu priamo neskonštruujeme, iba dokážeme, že také rozdelenie existuje. Všetkých možných rozdelení 90 detí na tri 30-členné skupiny (pokiaľ nezáleží na poradí skupín) je dokopy
$$
V={90\choose 30}\cdot {60\choose 30}\cdot\frac1{3!},
$$
pretože každé také rozdelenie môžeme vytvoriť tak, že najskôr vyberieme zo všetkých detí jednu 30-člennú skupinu a~potom zo zvyšných 60 detí vyberieme druhú 30-člennú skupinu. Tretia skupina bude tvorená deťmi, ktoré ostali (členom $3!$ treba výsledný súčin prirodzene vydeliť, keďže každé rozdelenie sme započítali pre každé možné poradie troch skupín).

Rozdelenie nazveme {\it zlé kvôli dieťaťu~$A$}, ak pri ňom dieťa~$A$ nemá vo svojej skupine žiadneho kamaráta. Zaoberajme sa tým, koľko je všetkých zlých rozdelení (teda takých, ktoré nevyhovujú zadaniu), ich počet označme~$Z$. Stačí, ak ukážeme, že zlých rozdelení je menej ako všetkých, \tj. $Z<V$.

Skúmajme, aký je počet rozdelení, ktoré sú zlé kvôli~$A$ -- ich počet označme $Z_A$. Ak $A$ má medzi všetkými $n$ kamarátov (čiže má $89-n$ "nekamarátov"), tak existuje\footnote{V~prípade, že $n>60$, tak prirodzene neexistuje žiadne rozdelenie zlé kvôli $A$. Aby sme sa vyhli rozoberaniu osobitných prípadov, dodefinujeme, tak ako je zvykom,  ${k\choose l}=0$ v~prípade, že $k<l$.}
$$
{89-n\choose 29}
$$
30-členných skupín, v~ktorých je $A$ spoločne s~ďalšími 29 deťmi, z~ktorých ani jedno nie je jeho kamarát. Pre každú takúto skupinu vieme zvyšných 60 detí rozdeliť
$$
{60\choose 30}\cdot\frac12
$$
spôsobmi na dve 30-členné skupiny (stále neberieme ohľad na poradie skupín). Takže počet rozdelení zlých kvôli $A$ je
$$
Z_A={89-n\choose 29}\cdot{60\choose 30}\cdot\frac12\le{59\choose 29}\cdot{60\choose 30}\cdot\frac12
\tag1
$$
(v~nerovnosti sme využili zadané ohraničenie $n\ge30$, čiže $89-n\le59$ -- zrejme z~čím väčšej množiny 29 prvkov vyberáme, tým viac kombinácií dostaneme).

Celkový počet zlých rozdelení určite nie je väčší ako súčet počtov zlých rozdelení pre jednotlivé deti (každé zlé rozdelenie je totiž zlé kvôli jednému alebo viacerým deťom). Keďže detí je 90, podľa \thetag1 máme
$$
Z\le90\cdot{59\choose 29}\cdot{60\choose 30}\cdot\frac12.
$$
Na dôkaz nerovnosti $Z<V$ tak stačí dokázať nerovnosť
$$
90\cdot{59\choose 29}\cdot{60\choose 30}\cdot\frac12<{90\choose 30}\cdot {60\choose 30}\cdot\frac1{3!},
\tag2
$$
ktorú ekvivalentne upravíme:
$$
\openup 5pt
%\eqalign{
\align
45\cdot{59\choose 29} &< {90\choose 30}\cdot\frac16,\\
6\cdot 45\cdot {59!\over 29! \cdot 30!} &< {90!\over 30!\cdot 60!},\\
6\cdot 45\cdot 59\cdot 58\cdot \dots\cdot 30 &< 90\cdot 89\cdot \dots\cdot 61,\\
6\cdot 45  &< {90\over 59}\cdot {89\over 58}\cdot \dots \cdot {61\over 30}.\tag3
\endalign
%}
$$
Každý z~tridsiatich zlomkov na pravej strane poslednej nerovnosti je zrejme väčší ako $1{,}5$. Preto pravá strana je väčšia ako $1{,}5^{30}=2{,}25^{15}>2^{15}>270=6\cdot45$. Takže nerovnosť \thetag3 a~teda aj \thetag2 platí, čo znamená, že existuje rozdelenie, ktoré nie je zlé.
}

{%%%%%   A-III-6
Najskôr odhadneme ľavú stranu prvej rovnice danej sústavy pomocou nerovnosti $4x^2\le x^4+4$, ktorá je splnená pre ľubovoľné reálne číslo $x$, pretože je ekvivalentná s~nerovnosťou $0\le(x^2-2)^2$. Rovnosť v~nej nastane práve vtedy, keď $x^2=2$, \tj. práve vtedy, keď $x=\sqrt{2}$ alebo $x=\m\sqrt{2}$.

Dostaneme tak
$$
4x^2+y^2\le x^4+y^2+4=5yz.
$$
Analogicky odhadneme aj ľavé strany zvyšných dvoch rovníc sústavy. Obdržíme tak trojicu nerovníc
$$
\aligned
 4x^2+y^2 &\le 5yz,\\
 4y^2+z^2 &\le 5zx,\\
 4z^2+x^2 &\le 5xy.
\endaligned
\tag1
$$
Ich súčtom dostaneme po jednoduchej úprave nerovnicu
$$
x^2+y^2+z^2\le xy+yz+zy,
$$
ktorú ekvivalentne upravíme na tvar
$$
(x-y)^2+(y-z)^2+(z-x)^2 \le 0.
\tag2
$$
Súčet druhých mocnín nemôže byť záporný. Preto v~nerovnici \thetag2 nutne nastáva rovnosť, \tj. platí $x=y=z$. Rovnosť musí ale platiť tiež v~každej nerovnici v~\thetag1. Odtiaľ vyplýva
$$
x=y=z=\sqrt{2} \quad \text{alebo} \quad x=y=z=-\sqrt{2}.
$$

Skúškou sa ľahko presvedčíme, že obe nájdené trojice danej sústave vyhovujú.

\zaver
Daná sústava rovníc má v~obore reálnych čísel práve dve riešenia, sú to trojice $(\sqrt{2},\sqrt{2},\sqrt{2})$
a $(\m\sqrt{2},\m\sqrt{2},\m\sqrt{2})$.

\ineriesenie
Po substitúcii $x=\sqrt2a$, $y=\sqrt2b$, $z=\sqrt2c$ (ktorú prirodzene urobíme, aby sústava mala triviálne riešenie $a=b=c=\pm1$) riešime sústavu
$$
\aligned
4a^4+2b^2+4&=10bc,\\
4b^4+2c^2+4&=10ca,\\
4c^4+2a^2+4&=10ab.
\endaligned
\tag3
$$
Pritom podľa nerovnosti medzi váženým aritmetickým a~geometrickým priemerom (nezáporných) čísel $a^4$, $b^4$, $a^2$, $b^2$, $1$ máme
$$
\frac{2a^4+2b^4+a^2+b^2+4}{10}\ge \root 10\of{a^{10}b^{10}}=|ab|\ge ab,
$$
teda $2a^4+2b^4+a^2+b^2+4\ge10ab$. Sčítaním tejto nerovnosti s~dvoma nerovnosťami, ktoré z~nej získame cyklickou zámenou premenných, dostaneme, že súčet ľavých strán v~\thetag3 je väčší alebo rovný súčtu pravých strán v~\thetag3, pričom rovnosť nastane jedine ak nastane v~použitých AG-nerovnostiach, teda keď $a^2=b^2=c^2=1$. Pritom $a$, $b$, $c$ musia mať totožné znamienka, aby platila rovnosť v~nerovnosti $|ab|\ge ab$ a~v~podobných nerovnostiach s~premennou~$c$. Teda jediným riešením sústavy \thetag3 sú trojice $(1,1,1)$ a~$(\m1,\m1,\m1)$, ktorým zodpovedajú rovnaké trojice $(x,y,z)$, aké sme našli v~prvom riešení (a~urobili pre ne skúšku).
}

{%%%%%   B-S-1
Vynásobením oboch strán danej rovnice štyrmi dostaneme
$$
4x^2+4y^2+4x+4y=16.
$$
Výraz na ľavej strane takto upravenej rovnice doplníme na súčet druhých
mocnín dvoch dvojčlenov. Obdržíme tak
$$
(4x^2+4x+1)+(4y^2+4y+1)=(2x+1)^2+(2y+1)^2=18.
$$
Stačí teda vyšetriť všetky rozklady čísla~$18$
na súčet dvoch kladných nepárnych čísel, pretože čísla $2x+1$ a~$2y+1$
nie sú deliteľné dvoma pre žiadne celé~$x$ a~$y$.

Uvažujme preto nasledujúce rozklady:
$$
18=1+17=3+15=5+13=7+11=9+9.
$$
Medzi uvedenými súčtami je iba jeden ($18=9+9$) súčtom druhých
mocnín dvoch celých čísel. Môžu teda nastať nasledujúce štyri prípady:
$$
\vcenter{\openup\jot\let\\=\cr
\halign{&$#$\hss\cr
2x+1=3, \enspace &2y+1=3,  \qquad&\text{\tj. }x=1, \enspace &y=1,\\
2x+1=3, \enspace &2y+1=-3, \qquad&\text{\tj. }x=1, \enspace &y=-2,\\
2x+1=-3,\enspace &2y+1=3,  \qquad&\text{\tj. }x=-2,\enspace &y=1,\\
2x+1=-3,\enspace &2y+1=-3, \qquad&\text{\tj. }x=-2,\enspace &y=-2.\\
}}
$$

\zaver
Danej rovnici vyhovujú práve štyri dvojice celých čísel $(x,y)$,
a~to $(1,1)$, $(1,\m2)$, $(\m2,1)$ a~$(\m2,\m2)$.

\ineriesenie
Danú rovnicu možno upraviť na tvar $x(x+1)+y(y+1)=4$, z~ktorého vidno, že číslo~$4$
je nutné rozložiť na súčet dvoch celých čísel, z~ktorých každé je
súčinom dvoch po sebe idúcich celých čísel.
Keďže najmenšie hodnoty výrazu $t(t+1)$ pre kladné aj záporné celé~$t$
sú $0, 2, 6, 12,\dots$, do úvahy prichádza iba rozklad $4=2+2$, takže každá
z~neznámych $x$, $y$ sa rovná jednému z~čísel $1$ či $\m2$ -- jediných celých
čísel~$t$, pre ktoré $t(t+1)=2$. Navyše je jasné, že naopak každá zo štyroch
dvojíc $(x,y)$ zostavených z~čísel $1$, $\m2$ je riešením danej úlohy.

\nobreak\medskip\petit\noindent
Za systematické a~úplné riešenie dajte 6~bodov. Za uhádnutie len jednej dvojice $(x,y)$, ktorá je riešením,
nedávajte žiadny bod, za ďalšie jedno uhádnuté riešenie dajte 1~bod, za uhádnutie všetkých
štyroch riešení dajte 2~body. Pri správnom postupe naopak strhnite 1~bod za každé
chýbajúce riešenie. Jednoznačnosť rozkladu čísla~$18$ na súčet dvoch druhých mocnín je natoľko
zrejmá, že ju nie je nutné zdôvodňovať (ako v~tu uvedenom riešení).
\endpetit
\bigbreak
}

{%%%%%   B-S-2
Keďže $DAF$ a~$EBF$ sú pravouhlé rovnoramenné trojuholníky, majú uhly pri ich
preponách veľkosť~$45^{\circ}$, takže trojuholník $DEF$ je pravouhlý.
Označme $S$ stred úsečky~$AB$ (\obr). Keďže
stred prepony pravouhlého trojuholníka je zároveň stredom jeho opísanej
kružnice, zrejme platí $|RF|=\frac12|DE|$ a~$|CS|=\frac12|AB|$.
Pritom $AD$ a~$BE$ sú dve rovnobežné priamky, ktorých vzdialenosť je rovná~$|AB|$,
a~preto $|DE|\ge|AB|$. Platí teda
$$
\postdisplaypenalty10000
|RF|=\frac12|DE|\ge\frac12|AB|=|CS|\ge |CF|,
$$
čo sme chceli dokázať.
\insp{b61.3}%

Rovnosť nastane práve vtedy, keď $|DE|=|AB|$ a~$|CS|=|CF|$, teda práve vtedy, keď $S=F$
(potom je aj $|AD|=|AS|=|BS|=|BE|$ a~$|DE|=|AB|$), čiže práve vtedy, keď je trojuholník $ABC$ rovnoramenný.

\ineriesenie
Označme $c_a=|BF|$ a~$c_b=|AF|$. Vzhľadom na to, že $|AD|=c_b$
a~$|BE|=c_a$ (\obrr1), vidíme, že pre dĺžku prepony~$DE$ v~pravouhlom
trojuholníku $DEF$ (poz. riešenie 2.~úlohy domáceho kola) dostaneme použitím
nerovnosti medzi aritmetickým a~geometrickým priemerom pre dvojicu kladných
čísel $c_a^2$ a~$c_b^2$ a~ďalej použitím Euklidovej vety o~výške~$CF$ v~pravouhlom
trojuholníku $ABC$ odhad
$$
|DE|%=\sqrt{(c_a+c_b)^2+(c_a-c_b)^2}
    =\sqrt{2(c_a^2+c_b^2)}\geq
\sqrt{2\cdot 2c_ac_b}=2\sqrt{c_ac_b}=2|CF|.
$$
Keďže v~pravouhlom trojuholníku $DEF$ platí $|RF|=\frac12|DE|$,
dostávame využitím uvedenej nerovnosti
$$
2|RF|=|DE|\ge 2|CF| \quad \hbox{a odtiaľ} \quad |RF|\geq |CF|,
$$
čo sme chceli dokázať.

Rovnosť nastane práve vtedy, keď sa obe priemerované hodnoty $c_a^2$ a~$c_b^2$ rovnajú, \tj. keď
platí $c_a=c_b$, čo nastane práve v~prípade pravouhlého rovnoramenného trojuholníka $ABC$.


\nobreak\medskip\petit\noindent
Za úplné riešenie dajte 6~bodov.
Za vynechanie podmienky, kedy nastane rovnosť, strhnite 2~body. Len za uhádnutie, kedy nastane rovnosť, dajte 1~bod.
\endpetit
\bigbreak
}

{%%%%%   B-S-3
Označme $A$ toho klebetníka, po ktorého odsťahovaní sa sieť rozpadne, a~$M_1$, $Z_1$
počty klebetníkov a~klebetníc v~jednej oddelenej skupine, $M_2$ a~$Z_2$ v~druhej.
Keďže v~každej skupine existuje aspoň jedna klebetnica a~tá je
ešte stále v~spojení s~aspoň dvoma klebetníkmi, je
$M_1\ge2$ a~$M_2\ge2$. V~každej skupine medzi počtami klebetníc a~klebetníkov
platia teraz vzťahy $3Z_1-1=2M_1$ a~$3Z_2-1=2M_2$. Rovnica tvaru $3z-1=2m$
nemá celočíselné riešenie~$z$ ani pre $m=2$, ani pre
$m=3$, až pre $m=4$ vychádza celé $z=3$.
Najmenší možný počet členov siete tak môže byť $M_1=M_2=4$, $Z_1=Z_2=3$.

Takú sieť ľahko zostrojíme podľa \obr, v~ktorom
$z_i$ sú klebetnice a~$m_i$ klebetníci.
\insp{b61.4}%

Zároveň vidíme, že uvedená sieť sa stane nesúvislou po odsťahovaní jednej z~klebetníc $z_3$
či $z_4$. Najmenší počet členov siete s~požadovanou vlastnosťou je preto~15.

\nobreak\medskip\petit\noindent
Za úplné riešenie dajte 6~bodov.
Ak bude nájdený len minimálny odhad $Z_1=Z_2=3$, $M_1=M_2=4$
a~nebude uvedená konkrétna konfigurácia, dajte 4~body. Ak bude nájdená
iba konkrétna konfigurácia s~15 členmi a~nebude dokázané, že menší počet členov siete
byť nemôže, dajte len 2~body.
\endpetit
\bigbreak
}

{%%%%%   B-II-1
Dané čísla označme $a_1, a_2, \dots , a_{2012}$.
Zrejme existuje index $k\ge1$ taký, že
$$
a_1+ \cdots +a_k < 1 \le a_1+ \cdots +a_k+a_{k+1} < 2.
$$
Čísla $a_1, \dots, a_{k+1}$ tvoria prvú požadovanú skupinu. Ďalej
zrejme existuje index $l\ge {k+2}$ taký, že
$$
a_{k+2}+ \cdots +a_l < 1 \le a_{k+2}+ \cdots +a_l+a_{l+1} <2.
$$
Čísla $a_{k+2}, \dots, a_{l+1}$ tvoria druhú požadovanú skupinu.
Analogicky zrejme existuje index $m\ge l+2$ taký, že
$$
a_{l+2}+ \cdots +a_m < 1 \le a_{l+2}+ \cdots +a_m+a_{m+1} <2.
$$
Čísla $a_{l+2}, \dots, a_{m+1}$ tvoria tretiu požadovanú skupinu.
Keďže $a_1+ \cdots +a_{m+1} < 6$, platí $a_{m+2}+ \cdots +a_{2012} \ge 1$,
takže čísla $a_{m+2},\dots , a_{2012}$ tvoria štvrtú požadovanú skupinu.

\nobreak\medskip\petit\noindent
Za systematické a~úplné riešenie udeľte 6 bodov.
\endpetit
\bigbreak
}

{%%%%%   B-II-2
Najskôr si uvedomme, že čísla $27$, $57$ a~$87$ sú
deliteľné tromi, čísla $37$, $67$ a~$97$ dávajú po delení tromi zvyšok~$1$
a~čísla $17$, $47$ a~$77$ dávajú po delení tromi zvyšok~$2$. Označme
$\overline{0}=\{27,57,87\}$, $\overline{1}=\{37,67,97\}$
a~$\overline{2}=\{17,47,77\}$. Uvažujme teraz pravidelný deväťuholník
$ABCDEFGHI$. Pri skúmaní deliteľnosti tromi súčtu trojíc
prirodzených čísel priradených trom susedným vrcholom uvažovaného
deväťuholníka stačí uvažovať namiesto daných čísel iba ich zvyšky po
delení tromi. Pritom tri čísla môžu dať v~súčte číslo deliteľné tromi
jedine tak, že buď všetky tri dávajú po delení tromi rovnaký zvyšok
(patria do rovnakej zvyškovej triedy),
alebo sú každé z~inej zvyškovej triedy. Keby však boli tri čísla priradené
po sebe idúcim vrcholom z~rovnakej zvyškovej triedy, muselo by z~tej istej triedy
byť aj číslo priradené nasledujúcemu vrcholu (ktorýmkoľvek smerom), a~tým pádom
aj všetky ďalšie čísla. Takých deväť čísel k~dispozícii nemáme,
preto ľubovoľným trom susedným vrcholom musia
byť priradené čísla z~navzájom rôznych zvyškových tried.

Predpokladajme najskôr, že vrcholu $A$ je
priradené niektoré číslo z~množiny $\overline{0}$.
V~takom prípade možno vrcholom uvažovaného deväťuholníka vzhľadom na
podmienky úlohy priradiť zvyškové triedy $\overline{0}$, $\overline{1}$,
$\overline{2}$ iba dvoma spôsobmi podľa toho, ktorému z~dvoch susedných
vrcholov vrcholu~$A$ priradíme~$\overline{1}$ a~ktorému~$\overline{2}$.
Ďalším vrcholom sú potom už zvyškové triedy vzhľadom na podmienku deliteľnosti
priradené jednoznačne. Výsledné priradenie môžeme zapísať ako
$$
(A,B,C,D,E,F,G,H,I)\leftrightarrow
(\overline{0},\overline{1},\overline{2},\overline{0},\overline{1},\overline{2},
\overline{0},\overline{1},\overline{2}),
$$
alebo
$$
(A,B,C,D,E,F,G,H,I)\leftrightarrow
(\overline{0},\overline{2},\overline{1},\overline{0},\overline{2},\overline{1},
\overline{0},\overline{2},\overline{1}).
$$

Podobne, ak je vrcholu $A$ priradená zvyšková trieda $\overline{1}$,
platí buď
$$
(A,B,C,D,E,F,G,H,I)\leftrightarrow
(\overline{1},\overline{0},\overline{2},\overline{1},\overline{0},
\overline{2},\overline{1},\overline{0},\overline{2}),
$$
alebo
$$
(A,B,C,D,E,F,G,H,I)\leftrightarrow
(\overline{1},\overline{2},\overline{0},\overline{1},\overline{2},
\overline{0},\overline{1},\overline{2},\overline{0}).
$$

% \item
Napokon, ak je vrcholu $A$ priradená zvyšková trieda $\overline{2}$,
platí buď
$$
(A,B,C,D,E,F,G,H,I)\leftrightarrow
(\overline{2},\overline{0},\overline{1},\overline{2},\overline{0},
\overline{1},\overline{2},\overline{0},\overline{1}),
$$
alebo
$$
(A,B,C,D,E,F,G,H,I)\leftrightarrow
(\overline{2},\overline{1},\overline{0},\overline{2},\overline{1},
\overline{0},\overline{2},\overline{1},\overline{0}).
$$

Podľa počtu možností, ako postupne vyberať čísla z~jednotlivých zvyškových tried,
každému z~uvedených prípadov prislúcha podľa pravidla súčinu práve
$$
(3\cdot 2\cdot 1)\cdot(3\cdot 2\cdot 1)\cdot(3\cdot 2\cdot 1)=6^3
$$
možností.
Vzhľadom na to, že iný prípad okrem šiestich uvedených nemôže nastať, vidíme, že
hľadaný počet možností, ako priradiť vrcholom uvažovaného deväťuholníka
daných deväť čísel, je
$$
6 \cdot 6^3=6^4=1\,296.
$$

\nobreak\medskip\petit\noindent
Za úplné riešenie udeľte 6 bodov, z~toho 2 body za zdôvodnenie, že čísla
priradené trom susedným vrcholom majú navzájom rôzne zvyšky po delení tromi, 2~body za uvedenie šiestich možností, ako môžu byť zvyškové triedy rozdelené, a~2~body za správny výpočet počtu možností.

Za každé vynechanie jedného zo šiestich prípadov strhnite 1~bod. Za chybne
spočítaný počet možností strhnite 2~body; za menej závažnú numerickú chybu
strhnite 1 bod. Za správny výpočet počtu možností, ktoré riešiteľ uvažoval,
aj keď neuviedol všetky možnosti, dajte 2~body.
\endpetit
\bigbreak
}

{%%%%%   B-II-3
a) Označme $S$ stred a~$r$ polomer kružnice vpísanej trojuholníku $ABC$ a~$L$,~$M$ body dotyku tejto kružnice postupne so stranami $BC$, $CA$ (\obr).
Ak označíme $|AK|=x$, $|BK|=y$, tak $|AP|=|AM|=x$, $|KP|=x\sqrt{2}$,
$|BQ|=|BL|=y$, $|KQ|=y\sqrt{2}$. Keďže oba uhly $AKP$, $BKQ$  majú veľkosť $45\st$,
je trojuholník $PQK$ pravouhlý, takže jeho obsah je
$$
S_{PQK}=\frac{x\sqrt{2} \cdot y\sqrt{2}}{2}=xy.
$$
\insp{b61.5}%

Štvoruholník $SLCM$ je štvorec so stranou dĺžky $r$ a~$|AM|=x$, $|BL|=y$.
Obsah trojuholníka $ABC$ je rovný súčtu obsahov trojuholníkov $ABS$, $BCS$
a~$CAS$, teda
$$
S_{ABC}=\frac{(x+y)r+(y+r)r+(x+r)r}{2}=(x+y+r)r.
$$
Obsah trojuholníka $ABC$ je zároveň rovný
$$
S_{ABC}=\frac{|AC|\cdot
|BC|}{2}=\frac{(x+r)(y+r)}{2}=\frac{xy}{2}+\frac{(x+y+r)r}{2}=\frac{xy}{2}+\frac{S_{ABC}}{2}.
$$
Odtiaľ dostávame $S_{ABC}=xy$, čiže $S_{ABC}=S_{PQK}$, čo sme mali dokázať.

\smallskip
b) V~trojuholníku $ABC$ sú dĺžky strán $a=y+r$, $b=x+r$, $c=x+y$. Obvod
trojuholníka $ABC$ je $a+b+|AB|$, obvod trojuholníka $PQK$ je
$x\sqrt{2}+y\sqrt{2}+|PQ|$.

Zrejme platí $|AB|\le |PQ|$ ($|AB|$ je vzdialenosťou rovnobežiek $AP$, $BQ$, \obrr1).
Rovnosť nastane jedine v~prípade $|AP|=|BQ|$, čiže $x=y$.

Ešte dokážeme, že $a+b\le x\sqrt{2}+y\sqrt{2}$, teda že $a+b\le c\sqrt{2}$. Posledná nerovnosť je ekvivalentná s~nerovnosťou, ktorú dostaneme
jej umocnením na druhú, pretože obe jej strany sú kladné. Dostaneme tak
$a^2+b^2+2ab \le 2c^2$. Keďže v~pravouhlom
trojuholníku $ABC$ platí $a^2+b^2=c^2$, máme dokázať nerovnosť $2ab \le a^2+b^2$, ktorá je však ekvivalentná s~nerovnosťou $0 \le (a-b)^2$. Tá
platí pre všetky reálne čísla $a$, $ b$ a~rovnosť v~nej nastane jedine pre
$a=b$, \tj. $x=y$.

Celkovo vidíme, že obvod trojuholníka $ABC$ je menší alebo rovný obsahu
trojuholníka $PQK$ a~rovnosť nastane práve vtedy, keď je pravouhlý trojuholník
$ABC$ rovnoramenný.

\nobreak\medskip\petit\noindent
Za úplné riešenie dajte 6 bodov, z~toho 3~body za časť a) a~3~body za časť b).
V časti b) za vynechanie podmienky, kedy nastane rovnosť, strhnite 1~bod.
\endpetit
\bigbreak
}

{%%%%%   B-II-4
Nutne platí $xy\ne0$. Zrejme tiež $x\ne y$;
keby bolo $x=y$, dostali by sme $\lfloor \frc{x}{y}\rfloor =
\lfloor \frc{y}{x} \rfloor =1$, takže z~prvej rovnice by vyšlo
$x=5$ a~z~druhej rovnice $y=\m6$, čo je v~rozpore s~rovnosťou $x=y$.
Podobne nemôže byť ani $x =\m y$, takže dokonca $|x|\ne|y|$.

Ak by obe neznáme mali rovnaké znamienka, bolo by jedno z~čísel
$\frc{x}{y}$, $\frc{y}{x}$ z~intervalu $(0, 1)$, takže jeho celá časť by
bola~$0$, čo nevedie k~riešeniu. Jedna z~neznámych teda musí byť kladná a~druhá
záporná a~obe celé časti v~rovniciach sú záporné. Z~nich preto tiež
vyplýva, že $x<0$ a~$y>0$.

Znamienka čísel $x$, $y$ sú rôzne a~absolútne hodnoty týchto čísel sú tiež
rôzne, preto práve jedno z~čísel $\frc{x}{y}$, $\frc{y}{x}$ je z~intervalu
$(\m1, 0)$, a~jeho celá časť je teda~$\m1$.

Najskôr preskúmame prípad $\lfloor \frc{x}{y} \rfloor=\m1$.
Z~druhej zadanej rovnice dostaneme $y=6$. Prvá zadaná rovnica
má potom tvar $\lfloor \frc{6}{x}\rfloor=\frc{5}{x}$. Ak navyše využijeme
definíciu celej časti, dostaneme
$$
\frac{5}{x}=\Bigl\lfloor \frac{6}{x}\Bigr\rfloor \le \frac{6}{x},
\quad\hbox{\tj.}\quad \frac{5}{x} \le \frac{6}{x}.
$$
Posledná nerovnica však nemôže pre záporné číslo~$x$ platiť, lebo je
ekvivalentná s~nerovnosťou $5 \ge 6$. Daná sústava
rovníc nemá teda v~prípade $\lfloor \frc{x}{y}\rfloor=\m1$ riešenie.

Ostáva prípad $\lfloor\frc{y}{x}\rfloor=\m1$.
Z~prvej zadanej rovnice máme $x=\m5$. Druhá zadaná rovnica
má potom tvar $\lfloor \frc{\m5}{y}\rfloor=\m\frc{6}{y}$. Podľa
definície celej časti teda platí
$$
\frac{-5}{y}<\left\lfloor \frac{-5}{y}\right\rfloor +1
=\frac{-6}{y}+1,\quad\hbox{\tj.}\quad  \frac{-5}{y}<\frac{-6}{y}+1,
$$
odkiaľ
po vynásobení kladným číslom~$y$ dostaneme $y>1$. Keď využijeme definíciu celej
časti aj pre rovnicu $\lfloor \frc{y}{\m5}\rfloor=\m1$, dostaneme
$$
-1\le\frac{y}{-5}<0, \quad\hbox{čiže}\quad 0<y\le 5.
$$
Spojením oboch podmienok pre neznámu~$y$ tak dostávame $1<y\le5$.
Naopak, pre každé také $y$ a~$x=\m5$ je prvá rovnica sústavy splnená.
Túto podmienku postupne upravíme na ekvivalentné nerovnosti
$1>\frc{1}{y}\ge\frac{1}{5}$ a~$\m5<\frc{\m5}{y}\le\m1$.
Z~tej poslednej vyplýva, že výraz $\lfloor \frc{\m5}{y}\rfloor$ môže nadobúdať
jedine hodnoty $\m5$, $\m4$, $\m3$, $\m2$, $\m1$.

Z~druhej rovnice upravenej na tvar
$y=\m6/\lfloor \frc{\m5}{y}\rfloor$
tak vyplýva, že neznáma~$y$ môže nadobúdať jedine hodnoty
$\frac65$, $\frac32$, $2$, $3$, $6$.
Posledná z~nich však (na rozdiel od prvých štyroch) nespĺňa odvodené
kritérium $1<y\le 5$ platnosti prvej rovnice sústavy. Či je pre
prvé štyri hodnoty splnená druhá rovnica sa musíme presvedčiť
aspoň tak, že overíme hodnotu výrazu $\lfloor \frc{\m5}{y}\rfloor$,
ktorá nás k~nim doviedla. Pre $y$ rovné postupne
$\frac65$, $\frac32$, $2$, $3$ je zlomok $\frc{\m5}{y}$ postupne rovný
$\m\frac{25}6$, $\m\frac{10}3$, $\m\frac{5}2$ a~$\m\frac{5}3$ s~prislúchajúcimi
celými časťami skutočne $\m5$, $\m4$, $\m3$, $\m2$.

\zaver
Zhrnutím všetkých úvah dostávame, že riešením danej sústavy sú nasledujúce
dvojice čísel $(x, y)$: $(\m5, \frac{6}{5})$, $(\m5,\frac{3}{2})$, $(\m5, 2)$, $(\m5,3)$.

\nobreak\medskip\petit\noindent
Za úplné riešenie udeľte 6~bodov, z~toho 1~bod za zistenie, že obe celé časti
v~rovniciach sú nutne záporné, a~ďalšie dva body za zistenie, že musí platiť $x=\m5$ alebo $y=6$.
Pri postupe založenom len na experimentovaní so zvolenými číslami
za skúšaním nájdené všetky
štyri riešenia dajte 2~body, ak by niektoré riešenie chýbalo, udeľte 1~bod, za
nájdenie jediného riešenia nedávajte žiadny bod.
\endpetit
\bigbreak
}

{%%%%%   C-S-1
Z~danej rovnosti vyplýva, že číslo~$b$ je nepárne
(inak by obe čísla naľavo boli párne), a~teda číslo $a$
je párne (inak by obe čísla naľavo boli nepárne).
Rovnosť množín preto musí byť splnená nasledovne:
$$
a\cdot[a,b]=180\quad\text{a}\quad
b\cdot(a,b)=45.
\tag1
$$
Keďže číslo $a$ delí číslo $[a,b]$, je číslo
$180=2^2\cdot3^2\cdot5$ deliteľné druhou mocninou (párneho) čísla
$a$, takže musí platiť buď $a=2$, alebo $a=6$. V~prípade $a=2$
(vzhľadom na to, že $b$ je nepárne) platí
$$
a\cdot[a,b]=2\cdot[2,b]=2\cdot2b=4b,
$$
čo znamená, že prvá rovnosť v~\thetag1 je splnená jedine pre $b=45$.
Vtedy $b\cdot(a,b)=45\cdot(2,45)=45$,  takže je splnená
aj druhá rovnosť v~\thetag1, a~preto dvojica $a=2$, $b=45$ je
riešením úlohy.

V~prípade $a=6$ podobne dostaneme
$$
a\cdot[a,b]=6\cdot[6,b]=6\cdot2\cdot[3,b]=12\cdot[3,b],
$$
čo znamená, že prvá rovnosť v~\thetag1 je splnená práve vtedy, keď
$[3,b]=15$. Tomu vyhovujú jedine hodnoty $b=5$ a~$b=15$.
Z~nich však iba hodnota $b=15$ spĺňa druhú rovnosť v~\thetag1,
ktorá je teraz v~tvare $b\cdot(6,b)=45$. Druhým riešením úlohy je
teda dvojica $a=6$, $b=15$, žiadne ďalšie riešenia neexistujú.

\odpoved
Hľadané dvojice sú dve, a~to $a=2$, $b=45$
a~$a=6$, $b=15$.

\ineriesenie
Označme $d=(a,b)$. Potom $a=ud$ a~$b=vd$, pričom
$u$, $v$ sú nesúdeliteľné prirodzené čísla, takže $[a,b]=uvd$.
Z~rovností
$$
a\cdot[a,b]=ud\cdot uvd=u^2vd^2\quad\text{a}\quad
b\cdot(a,b)=vd\cdot d=vd^2
$$
vidíme, že číslo $a\cdot[a,b]$ je $u^2$-násobkom čísla
$b\cdot(a,b)$, takže zadaná rovnosť množín môže byť splnená
jedine tak, ako sme zapísali vzťahmi \thetag1 v~prvom riešení.
Tie teraz môžeme vyjadriť rovnosťami
$$
u^2vd^2=180\quad\text{a}\quad vd^2=45.
$$
Preto platí $u^2=\frac{180}{45}=4$, čiže $u=2$. Z~rovnosti
$vd^2=45=3^2\cdot5$ vyplýva, že buď $d=1$ (a~$v=45$), alebo $d=3$
(a~$v=5$). V~prvom prípade $a=ud=2\cdot1=2$ a~$b=vd=45\cdot1=45$, v~druhom $a=ud=2\cdot3=6$ a~$b=vd=5\cdot3=15$.

\poznamka
Keďže zo zadanej rovnosti okamžite vyplýva, že obe čísla $a$, $b$
sú deliteľmi čísla $180$ (takým deliteľom je dokonca aj ich
najmenší spoločný násobok $[a,b]$), je možné úlohu vyriešiť
rôznymi inými cestami, založenými na testovaní
konečného počtu dvojíc konkrétnych čísel $a$ a~$b$.
Takýto postup urýchlime, keď vopred zistíme niektoré nutné podmienky,
ktoré musia čísla $a$, $b$ spĺňať.
Napríklad spresnenie rovnosti množín na dvojicu rovností \thetag1
možno (aj bez použitia úvahy o~parite čísel $a$, $b$) vysvetliť
všeobecným postrehom: súčin $a\cdot[a,b]$ je {\it vždy\/}
deliteľný súčinom $b\cdot(a,b)$,
pretože ich podiel možno zapísať v~tvare
$$
\postdisplaypenalty 10000
\frac{a\cdot[a,b]}{b\cdot(a,b)}=\frac{a}{(a,b)}\cdot
\frac{[a,b]}{b},
$$
teda ako súčin dvoch {\it celých\/} čísel.

\nobreak\medskip\petit\noindent
Za úplné riešenie dajte 6~bodov. Ak chýba zdôvodnenie rovností \thetag1
v~inak úplnom riešení (argumenty $(a,b)<[a,b]$ či $(a,b)\mid[a,b]$
samotné nestačia; stačí však napr. argument $a\ge(a,b)$ a~$[a,b]\ge b$), strhnite 1~bod.  Za nájdenie oboch vyhovujúcich
dvojíc (napríklad uhádnutím) dajte 1~bod, ďalšie body
potom podľa kvality podaného postupu hľadania, hlavne jeho
systematickosti.
\endpetit
\bigbreak
}

{%%%%%   C-S-2
Vďaka súmernosti podľa priamky~$CS$ sa obe
vpísané kružnice dotýkajú výšky~$CS$ v~rovnakom bode, ktorý
označíme~$D$. Body dotyku týchto kružníc s~úsečkami $AS$,
$BS$, $AC$, $BC$ označíme postupne $E$, $F$, $G$, $H$
(\obr). Pre vyjadrenie všetkých potrebných dĺžok ešte zavedieme
označenie $x=|SD|$ a~$y=|CD|$.
\insp{c61.4}%

Vzhľadom na symetriu dotyčníc z~daného bodu k~danej kružnici platia
rovnosti
$$
|SD|=|SE|=|SF|=x\quad\text{a}\quad |CD|=|CG|=|CH|=y.
$$
Úsečka~$EF$ má preto dĺžku $2x$, ktorá je podľa zadania zároveň dĺžkou
úsečiek $AE$ a~$BF$, a~teda aj dĺžkou úsečiek $AG$ a~$BH$
(opäť vďaka symetrii dotyčníc). Odtiaľ už bezprostredne vyplývajú
rovnosti
$$
|AB|=6x,\quad |AC|=|BC|=2x+y\quad\text{a}\quad |CS|=x+y.
$$

Závislosť medzi dĺžkami $x$ a~$y$ zistíme použitím Pytagorovej vety
pre pravouhlý trojuholník $ACS$ (s~odvesnou~$AS$ dĺžky $3x$):
$$
(2x+y)^2=(3x)^2+(x+y)^2.
$$
Roznásobením a~ďalšími úpravami odtiaľ dostaneme ($x$ a~$y$ sú kladné hodnoty)
$$\align
4x^2+4xy+y^2&=9x^2+x^2+2xy+y^2,\\
2xy&=6x^2,\\
y&=3x.
\endalign$$
Hľadaný pomer tak má hodnotu
$$
|AB|:|CS|=6x:(x+y)=6x:4x=3:2.
$$

Poznamenajme, že prakticky rovnaký postup celého riešenia
možno zapísať aj pri štandardnom označení $c=|AB|$ a~$v=|CS|$.
Keďže podľa zadania platí $|AE|=\frac13c$, a~teda $|SE|=\frac16c$,
z~rovnosti $|SD|=|SE|$ vyplýva $|CD|=|CS|-|SD|=v-\frac16c$, odkiaľ
$$
|AC|=|AG|+|CG|=|AE|+|CD|=\tfrac13c+(v-\tfrac16c)=v+\tfrac16c,
$$
takže z~Pytagorovej vety pre trojuholník $ACS$,
$$
(v+\tfrac16 c)^2=(\tfrac12 c)^2+v^2,
$$
vychádza $3v=2c$, čiže $c:v=3:2$.


\nobreak\medskip\petit\noindent
Za úplné riešenie dajte 6~bodov. Za vyjadrenie potrebných dĺžok
pomocou dvoch parametrov (napr. $x$, $y$ alebo $c$, $v$) dajte 3~body,
ďalšie 2~body pridajte za efektívne využitie Pytagorovej vety
a~1~bod za konečné určenie hľadaného pomeru.
\endpetit
\bigbreak
}

{%%%%%   C-S-3
Druhá zo zadaných rovníc napovedá, že by sme mali
skúmať odchýlky čísel v~dvojiciach $p$, $q$ a~$r$, $s$. Pre súčet
druhých mocnín týchto odchýlok platí
$$
(p-q)^2+(r-s)^2=\bigl(p^2+q^2+r^2+s^2\bigr)-2(pq+rs)=4-2\cdot1=2.
$$
Avšak ak je súčet dvoch reálnych čísel rovný číslu~$2$, nemôžu byť
oba sčítance ani väčšie ako $1$, ani menšie ako $1$. Jedno
z~čísel $(p-q)^2$, $(r-s)^2$ je teda najviac~$1$ a~jedno je najmenej~$1$.
To isté potom platí aj o~číslach $|p-q|$ a~$|r-s|$, čo sme chceli
dokázať.

\nobreak\medskip\petit\noindent
Za úplné riešenie dajte 6~bodov. Určenie súčtu $(p-q)^2+(r-s)^2$
ohodnoťte 4~bodmi. Ak riešiteľ pri správnej úvahe urobí záver
o~číslach $p-q$ a~$r-s$ bez absolútnych hodnôt (a~nepodotkne
pritom, že možno bez ujmy na všeobecnosti predpokladať, že $p\ge q$ a~$r\ge s$), strhnite 1~bod.
\endpetit
\bigbreak
}

{%%%%%   C-II-1
Aby sme mohli použiť
vzorec $A^2-B^2=(A-B)(A+B)$, presuňme najskôr
jeden z~krajných členov ľavej strany, napríklad člen $z^2$,
na pravú stranu:
$$
\align
x^2-y^2&>(x-y+z)^2-z^2,\\
(x-y)(x+y)&>(x-y+z-z)(x-y+z+z),\\
(x-y)(x+y)&>(x-y)(x-y+2z).
\endalign
$$
Keďže spoločný činiteľ $x-y$ oboch strán poslednej nerovnosti je
podľa predpokladu úlohy číslo záporné, budeme s~dôkazom hotoví, keď
ukážeme, že zvyšné činitele spĺňajú opačnú nerovnosť
$x+y<x-y+2z$. Tá je však zrejme ekvivalentná s~nerovnosťou
$2y<2z$, čiže $y<z$, ktorá podľa zadania úlohy naozaj platí.

\ineriesenie
Podľa vzorca pre druhú mocninu trojčlena platí
$$
(x-y+z)^2=x^2+y^2+z^2-2xy+2xz-2yz.
$$
Dosaďme to do pravej strany dokazovanej nerovnosti a~urobme
niekoľko ďalších ekvivalentných úprav:
$$
\align
x^2-y^2+z^2&>x^2+y^2+z^2-2xy+2xz-2yz,\\
0&>2y^2-2xy+2xz-2yz,\\
0&>2y(y-x)+2z(x-y),\\
0&>2(y-x)(y-z).
\endalign
$$
Posledná nerovnosť už vyplýva z~predpokladov úlohy, podľa ktorých
je činiteľ $y-x$ kladný, zatiaľ čo činiteľ $y-z$ je záporný.

\nobreak\medskip\petit\noindent
Za úplné riešenie udeľte 6 bodov. Len za overovanie nerovnosti pre
konkrétne trojice čísel $x<y<z$ žiadny bod nedávajte.
\endpetit
\bigbreak
}

{%%%%%   C-II-2
\def\ov#1{\overline{#1}}%
Označme $\ov{abc}$ to trojciferné číslo, o~ktorého
trojnásobku sa píše v~texte úlohy. Platí tak rovnica
$$
3\ov{abc}+6=\ov{abc}+\ov{acb}+\ov{bac}+\ov{bca}+\ov{cab}+\ov{cba}.
$$
Keďže na pravej strane je každá z~cifier $a$, $b$, $c$ dvakrát
na mieste jednotiek, desiatok aj stoviek, môžeme rovnicu prepísať na
tvar
$$
300a+30b+3c+6=222a+222b+222c,\quad
\text{čiže}\quad 78a+6=192b+219c.
$$
Po vydelení číslom $3$ dostaneme rovnicu $26a+2=64b+73c$, z~ktorej
vidíme, že $c$ je párna cifra. Platí preto $c\ge2$, čo spolu so
zrejmou nerovnosťou $b\ge1$ (pripomíname, že všetky tri neznáme
cifry sú podľa zadania nenulové) vedie na odhad
$$
64b+73c\ge64+146=210.
$$
Musí preto platiť $26a+2\ge210$, odkiaľ $a\ge(210-2):26=8$,
takže cifra $a$ je buď $8$, alebo $9$. Pre $a=8$ však v~nerovnosti
z~predošlej vety nastane rovnosť, takže nutne $b=1$ a~$c=2$
(a~rovnica zo zadania úlohy je potom splnená). Pre $a=9$ dostávame rovnicu
$$
64b+73c=26\cdot9+2=236,
$$
z~ktorej vyplýva, že $c$ je jednak deliteľné štyrmi,
jednak je menšie ako $4$, čo nemôže nastať súčasne.

\odpoved
Cifry na kartičkách sú $8$, $2$ a~$1$.

\poznamka
Riešiť odvodenú rovnicu $26a+2=64b+73c$ pre
neznáme (nenulové a~navzájom rôzne) cifry $a$, $b$, $c$
možno viacerými úplnými a~systematickými postupmi, uviedli sme
len jeden z~nich.

\nobreak\medskip\petit\noindent
Za úplné riešenie dajte 6 bodov, z~toho najviac 3~body za správne
zostavenú rovnicu, rozvoj dekadických zápisov čísel
a~úpravu na lineárnu rovnicu s~neznámymi $a$, $b$, $c$. Ďalšími 3~bodmi potom ohodnoťte postup
pri hľadaní riešenia odvodenej rovnice, pritom len za uhádnutie hľadanej trojice dajte 1~bod.
\endpetit
\bigbreak
}

{%%%%%   C-II-3
Keďže štvoruholník $ABCE$ je podľa zadania dotyčnicový, pre dĺžky
jeho strán platí známa rovnosť\footnote{Rovnosť sa odvodí rozpísaním
dĺžok strán na ich úseky vymedzené bodmi dotyku vpísanej kružnice
a~následným využitím toho, že každé dva z~týchto úsekov, ktoré vychádzajú
z~rovnakého vrcholu štvoruholníka, sú zhodné.}
$$
|AB|+|CE|=|BC|+|AE|.
$$
V~našej situácii pri označení $a=|AB|$ platí $|BC|=|AD|=\frac23a$
a~$|CE|=|DE|=\frac12a$ (\obr), odkiaľ po dosadení do uvedenej rovnosti zistíme, že
$|AE|=\frac56a$.
\insp{c61.5}%

Teraz si všimneme, že pre dĺžky strán trojuholníka $ADE$ platí
$$
|AD|:|DE|:|AE|=\tfrac23a:\tfrac12a:\tfrac56a=4:3:5,
$$
takže podľa (obrátenej časti) Pytagorovej vety má trojuholník $ADE$ pravý
uhol pri vrchole~$D$, a~teda rovnobežník $ABCD$ je obdĺžnik.
Dotyčnica~$BC$ kružnice vpísanej štvoruholníku
$ABCE$ je teda kolmá na dve jej (navzájom rovnobežné)
dotyčnice $AB$ a~$CE$. To už zrejme znamená, že bod dotyku dotyčnice~$BC$
je stredom úsečky~$BC$ (vyplýva to zo zistenej kolmosti vyznačeného
priemeru kružnice na jej vyznačený polomer).

\ineriesenie
Ukážeme, že požadované tvrdenie možno dokázať
aj bez všimnutia si, že rovnobežník $ABCD$ je v~danej úlohe obdĺžnikom.
Namiesto toho využijeme, že úsečka~$CE$ je stredná priečka trojuholníka $ABF$, pričom
$F$ je priesečník polpriamok $BC$ a~$AE$ (\obr), lebo $CE\parallel AB$
\insp{c61.6}%
a~$|CE|=\frac12|AB|$.
Označme preto $a=|AB|=2|CE|$, $b=|BC|=|CF|$ a~$e=|AE|=|EF|$ (rovnosť
$2a=3b$ použijeme až neskôr).
Rovnako ako v~prvom riešení využijeme rovnosť
$b+e=a+\frac12a\ (=\!\frac32a)$, ktorá platí
pre dĺžky strán dotyčnicového štvoruholníka $ABCE$. Kružnica jemu
vpísaná sa dotýka strán $BC$, $CE$, $AE$ postupne v~bodoch $P$, $Q$,
$R$ tak, že platia rovnosti
$$
|CP|=|CQ|,\quad|EQ|=|ER|\quad\text{a tiež}\quad|FP|=|FR|.
$$
Pre súčet zhodných dĺžok $|FP|$ a~$|FR|$ teda platí
$$
\postdisplaypenalty 10000
\align
|FP|+|FR|&=(b+|CP|)+(e+|ER|)=(b+e)+(|CP|+|ER|)=\\
         &=\tfrac32a+(|CQ|+|EQ|)=\tfrac32a+\tfrac12a=2a,
\endalign
$$
čo znamená, že $|FP|=|FR|=a$.

Teraz už riešenie úlohy ľahko dokončíme.
Rovnosť $|BP|=\frac12b$, ktorú máme v~našej situácii dokázať,
vyplýva z~rovnosti
$$
|BP|=|BF|-|FP|=2b-a,
$$
keď do nej dosadíme zadaný vzťah $a=\frac32b$.

\nobreak\medskip\petit\noindent
Za úplné riešenie dajte 6~bodov, z~toho 2~body za použitie kritéria
dotyčnicovosti štvoruholníka $ABCE$ na vyjadrenie dĺžky strany~$AE$.
\endpetit
\bigbreak
}

{%%%%%   C-II-4
Úloha dvoch po sebe zotieraných čísel je v~zadanej hre symetrická:
ak je po čísle~$x$ možné zotrieť číslo~$y$, je (pri inom priebehu
hry) po čísle~$y$ možné zotrieť číslo~$x$. Preto si môžeme
celú hru (so zadaným číslom~$n$) "sprehľadniť" tak,
že najskôr vypíšeme všetky takéto
(nazývajme ich {\it prípustné\/}) dvojice $(x,y)$.
Keďže na poradí čísel v~prípustnej dvojici nezáleží, stačí
vypisovať len tie dvojice $(x,y)$, v~ktorých $x<y$.

\smallskip
V~prípade $n=6$ všetky prípustné dvojice sú
$$
(1,3),\ (1,4),\ (1,5),\ (1,6),\ (2,5),\  (3,5).
$$
Z~tohto zoznamu ľahko odhalíme, že víťaznú stratégiu
má (prvá) hráčka Marína. Ak totiž zotrie na začiatku hry číslo~$4$,
musí Tamara zotrieť číslo~$1$, a~keď potom Marína zotrie číslo~$6$,
nemôže už Tamara žiadne ďalšie číslo
zotrieť. Okrem tohto priebehu $4\to1\to6$ si môže Marína zaistiť
víťazstvo aj inými, pre Tamaru  "vynútenými" priebehmi,
napríklad $6\to1\to4$ alebo $4\to1\to3\to5\to2$.

\smallskip
V~prípade $n=12$ je všetkých prípustných dvojíc výrazne
väčšie množstvo. Preto si položíme otázku, či
všetky čísla od $1$ do $12$ možno rozdeliť na šesť prípustných dvojíc.
Ak totiž nájdeme takú šesticu, môžeme opísať víťaznú stratégiu druhej hráčky
(Tamary): ak zotrie Marína pri ktoromkoľvek svojom ťahu číslo~$x$,
Tamara potom vždy zotrie to číslo~$y$, ktoré s~číslom~$x$ tvorí jednu
zo šiestich nájdených dvojíc. Tak nakoniec Tamara zotrie aj posledné
(dvanáste) číslo a~vyhrá (prípadne hra skončí skôr tak, že Marína nebude môcť zotrieť žiadne číslo).

Hľadané rozdelenie všetkých 12 čísel do šiestich dvojíc naozaj
existuje, napríklad
$$
(1,4),\ (2,9),\ (3,8),\ (5,12),\ (6,11),\ (7,10).
$$
Iné vyhovujúce rozdelenie dostaneme, keď v~predošlom dvojice
$(1,4)$ a~$(6,11)$ zameníme dvojicami $(1,6)$ a~$(4,11)$. Ďalšie,
menej podobné vyhovujúce rozdelenie je napríklad
$$
(1,6),\ (2,5),\ (3,10),\ (4,9),\ (7,12),\ (8,11).
$$

\odpoved
Pre $n=6$ má víťaznú stratégiu Marína, pre $n=12$
Tamara.

\nobreak\medskip\petit\noindent
Za úplné riešenie dajte 6~bodov, z~toho 2~body za vyriešenie prípadu
$n=6$ a~4~body za vyriešenie prípadu $n=12$.
\endpetit
\bigbreak
}

{%%%%%   vyberko, den 1, priklad 1
...}

{%%%%%   vyberko, den 1, priklad 2
...}

{%%%%%   vyberko, den 1, priklad 3
...}

{%%%%%   vyberko, den 1, priklad 4
...}

{%%%%%   vyberko, den 2, priklad 1
...}

{%%%%%   vyberko, den 2, priklad 2
...}

{%%%%%   vyberko, den 2, priklad 3
...}

{%%%%%   vyberko, den 2, priklad 4
...}

{%%%%%   vyberko, den 3, priklad 1
...}

{%%%%%   vyberko, den 3, priklad 2
...}

{%%%%%   vyberko, den 3, priklad 3
...}

{%%%%%   vyberko, den 3, priklad 4
...}

{%%%%%   vyberko, den 4, priklad 1
...}

{%%%%%   vyberko, den 4, priklad 2
...}

{%%%%%   vyberko, den 4, priklad 3
...}

{%%%%%   vyberko, den 4, priklad 4
...}

{%%%%%   vyberko, den 5, priklad 1
...}

{%%%%%   vyberko, den 5, priklad 2
...}

{%%%%%   vyberko, den 5, priklad 3
...}

{%%%%%   vyberko, den 5, priklad 4
...}

{%%%%%   trojstretnutie, priklad 1
Keďže $\tau(1)=\varphi(1)=1$, hodnota $n=1$ zadaniu vyhovuje. Ďalej predpokladajme, že $n>1$. V~takom prípade zrejme $\tau(n)\le n$ a~$\varphi(n)<n$, takže $n$ nemôže byť aritmetickým priemerom čísel $\tau(n)$ a~$\varphi(n)$. Rozoberieme preto dve možnosti.

\pripad1
Nech $\tau(n)=\frac12(\varphi(n)+n)$. Potom $\tau(n)>\frac12n$. Pre každý deliteľ $d$ čísla $n$ je aj $n/d$ deliteľom $n$. Aspoň jedno z~čísel $d$, $n/d$ je menšie alebo rovné $\sqrt n$, teda množina $\{1,2,\dots,\lfloor\sqrt{n}\rfloor\}$ obsahuje aspoň polovicu\footnote{Presne polovicu obsahuje práve vtedy, keď $n$ nie je druhou mocninou celého čísla.} deliteľov čísla $n$. Z~toho vyplýva nerovnosť $\frac12\tau(n)\le\sqrt n$, odkiaľ máme
$$
2\sqrt n\ge\tau(n)>\frac12n\qquad\Rightarrow\qquad 4n>\frac14n^2 \qquad\Rightarrow\qquad 16>n.
$$
Pre $1<n<16$ spočítame hodnotu $\tau(n)$, skontrolujeme podmienku $\tau(n)>\frac12n$ a~vo zvyšných málo prípadoch dopočítame $\varphi(n)$:
$$
\vbox{\offinterlineskip
       \halign{\strut\vrule\ \hfil # \hfil\vrule width1.2pt&&\hbox to 1.8em{\hss$#$\hss}\vrule\cr
\noalign{\hrule}
$n$                 & 2 & 3 &  4 &  5 &  6 &   7 &   8 & 9 & 10 & 11 & 12 & 13 & 14 & 15\cr
\noalign{\hrule}
$\tau(n)$           & 2 & 2 &  3 &  2 &  4 &   2 &   4 & 3 & 4 & 2 & 6 & 2 & 4 & 4\cr
\noalign{\hrule}
$\tau(n)>\frac12n$~?\vphantom{\vrule height3.5mm depth 1.6mm}  & \checkmark & \checkmark & \checkmark & \times & \checkmark & \times & \times & \times & \times & \times & \times & \times & \times & \times\cr
\noalign{\hrule}
$\varphi(n)$           & 1 & 2 & 2 &  & 2 &  &  &  &  &  &  &  &  & \cr
\noalign{\hrule}
$\tau(n)=\frac12(\varphi(n)+n)$~?\vphantom{\vrule height3.5mm depth 1.6mm}           & \times & \times & \checkmark &  & \checkmark &  &  &  &  &  &  &  &  & \cr
\noalign{\hrule}
}}
$$
Zistili sme, že v~tomto prípade vyhovujú zadaniu jedine $n=4$ a~$n=6$.

\pripad2
Nech $\varphi(n)=\frac12(\tau(n)+n)$. Upravíme tento vzťah na
$$
\tau(n)=2\varphi(n)-n.
\tag1
$$
Ak $n$ je párne, tak žiadne párne číslo nie je nesúdeliteľné s~$n$, čiže $\varphi(n)\le\frac12n$. Potom však z~\thetag1 dostaneme $\tau(n)\le0$, čo je spor. Preto $n$ musí byť nepárne. Z~\thetag1 následne vyplýva, že aj $\tau(n)$ musí byť nepárne, čo znamená, že $n$ je druhou mocninou celého (nepárneho) čísla. Predpokladajme, že rozklad na súčin prvočísel čísla $n$ je
$$
n=p_1^{2\alpha_1}\cdot\dots\cdot p_k^{2\alpha_k},\qquad k\ge1,\quad p_i\ge3, \quad \alpha_i\ge1.
$$
Na základe známych vzorcov\footnote{Ak $m=p_1^{\alpha_1}\dots p_k^{\alpha_k}$, tak $\tau(m)=(\alpha_1+1)\dots(\alpha_k+1)$ a~$\varphi(m)=m(1-1/p_1)\dots(1-1/p_k)$.} pre výpočet $\tau(n)$ a~$\varphi(n)$ upravíme \thetag1 na
$$
\align
(2\alpha_1+1)\cdot\dots\cdot(2\alpha_k+1)&=2p_1^{2\alpha_1-1}(p_1-1)\cdot\dots\cdot p_k^{2\alpha_k-1}(p_k-1)-p_1^{2\alpha_1}\cdot\dots\cdot p_k^{2\alpha_k}=\\
&=p_1^{2\alpha_1-1}\cdot\dots\cdot p_k^{2\alpha_k-1}\bigl(2(p_1-1)\cdot\dots\cdot(p_k-1)-p_1\cdot\dots\cdot p_k\bigr).
\endalign
$$
Pravá strana je deliteľná súčinom $p_1^{2\alpha_1-1}\cdot\dots\cdot p_k^{2\alpha_k-1}$, takže aj ľavá ním musí byť deliteľná. Z~toho dostávame
$$
p_1^{2\alpha_1-1}\cdot\dots\cdot p_k^{2\alpha_k-1} \le(2\alpha_1+1)\cdot\dots\cdot(2\alpha_k+1).
\tag2
$$
Avšak pre každé celé čísla $p\ge3$ a~$\alpha\ge1$ platí nerovnosť $p^{2\alpha-1}\ge(2\alpha+1)$, pričom rovnosť nastáva jedine pre $p=3$ a~$\alpha=1$. To dokážeme matematickou indukciou vzhľadom na $\alpha$: Prípad $\alpha=1$ je triviálny (s~rovnosťou nastávajúcou jedine pre $p=3$) a~keď $\alpha$ zväčšíme o~$1$, pravá strana sa zväčší o~$2$, zatiaľ čo ľavá strana sa zväčší až o
$$
p^{2(\alpha+1)-1}-p^{2\alpha-1}=p^{2\alpha-1}(p^2-1)>2.
$$
Každý z~činiteľov na ľavej strane \thetag2 je teda väčší alebo rovný ako prislúchajúci činiteľ napravo a~všetky činitele sú kladné. jediný spôsob, ako splniť nerovnosť \thetag2, je položiť $k=1$, $p_1=3$, $\alpha_1=1$, čiže $n=9$. V~takom prípade naozaj dostávame $\tau(9)=3$ a~$\varphi(9)=6$, \tj. rovnosť \thetag1 je splnená.

\odpoved
Zadaniu vyhovujú hodnoty $n\in\{1,4,6,9\}$.
}

{%%%%%   trojstretnutie, priklad 2
Zadanú rovnosť prepíšeme na ekvivalentný tvar
$$
f(x+f(y))=(x+f(y))^4-x^4+f(x).\tag1
$$

Dosaďme do \thetag1 $x=-f(z)$, $y=z$:
$$
f(0)=-(f(z))^4+f(-f(z)) \quad\text{pre všetky $z\in\Bbb R$.}\tag2
$$
Keď do \thetag1 dosadíme $x=-f(z)$, s~využitím \thetag2 dostávame
$$
f(f(y)-f(z))=(f(y)-f(z))^4-(f(z))^4+f(-f(z))=(f(y)-f(z))^4+f(0)
$$
pre všetky $y,z\in\Bbb R$. To znamená, že ak sa číslo~$t$ dá vyjadriť ako rozdiel dvoch hodnôt funkcie~$f$, \tj. $t=f(y)-f(z)$ pre nejaké $y,z\in\Bbb R$, tak $f(t)=t^4+f(0)$. Ukážeme, že ak $f$ nadobúda nejakú nenulovú hodnotu, tak každé číslo je rozdielom nejakých dvoch hodnôt funkcie~$f$.

Nech $f(a)=b\ne0$. Dosadením $y=a$ do zadanej rovnosti obdržíme
$$
f(x+b)-f(x)=(x+b)^4-x^4.
$$
Keďže $b\ne 0$, výraz na pravej strane je v~premennej~$x$ mnohočlenom stupňa $3$, a~preto jeho oborom hodnôt je celá množina $\Bbb R$. Z~toho vyplýva, že aj ľavá strana, na ktorej je rozdiel dvoch hodnôt funkcie~$f$, nadobúda pre $x\in\Bbb R$ všetky reálne hodnoty. Spojením zistených vlastností dostávame $f(t)=t^4+f(0)$ pre všetky $t\in\Bbb R$.
Ľahko možno overiť, že všetky funkcie tvaru $f(x)=x^4+k$ spĺňajú zadanú rovnosť.

Konštantná nulová funkcia očividne takisto vyhovuje.

\odpoved
Riešením úlohy sú funkcie $f(x)=0$ a~$f(x)=x^4+k$ pre ľubovoľný reálny parameter~$k$.
}

{%%%%%   trojstretnutie, priklad 3
Nech $M$, $N$ sú postupne stredy oblúkov $AB$, $AD$ (neobsahujúcich žiadne iné vrcholy štvoruholníka $ABCD$).
Ako vieme, stred $I$ kružnice vpísanej trojuholníku $ABC$ leží na priamke~$CM$ (ktorá je osou uhla $BCA$, čo vyplýva zo zhodnosti veľkostí obvodových uhlov nad zhodnými tetivami $AM$ a~$BM$), podobne $J$ leží na $CN$ a~$K$ leží na $BN$. Navyše platí
$$
|MI|=|MA|=|MB| \qquad\text{a}\qquad |NJ|=|NA|=|ND|=|NK|
\tag1
$$
(tieto známe rovnosti vyplývajú z~jednoduchého vyjadrenia veľkosti uhla $AIB$ z~trojuholníka $AIB$ a~veľkosti uhla $AMB$, resp. z~analogických vyjadrení pre body $J$, $K$).

Poznamenajme, že bod $K$ nutne leží vnútri rovnoramenného trojuholníka $BDE$ (pretože ľahko možno odvodiť nerovnosti $|\uhol BDK|<|\uhol BDE|$ a~$|\uhol DBK|<|\uhol DBE|$),
a~teda priamka~$EK$ pretína úsečku~$BD$ a~bod~$F$ leží v~polrovine $BDC$ (\obr).
\insp{cps.2}%

Uvažujme oblúk~$BD$ kružnice~$\omega$ obsahujúci bod~$A$.
Bod~$E$ je jeho stredom a~body $M$ a~$N$ sú stredy kratších oblúkov $BA$ a~$AD$, ktorých zjednotením je oblúk~$BD$.
Z~toho vyplýva, že oblúky $BM$ a~$EN$ majú rovnakú dĺžku (a~podobne aj oblúky $ME$ a~$ND$). Z~obvodových uhlov potom
$$
|\angle BFM|=|\angle EFN|=|\angle KFN|.
\tag2
$$
Taktiež platí
$$
|\angle BMF|=|\angle BNF|=|\angle KNF|.
\tag3
$$
Z~rovností \thetag2 a~\thetag3 dostávame podobnosť trojuholníkov
$MBF$ a~$NKF$, preto
$$
{|MB|\over |MF|}={|NK|\over |NF|},
$$
čo s~využitím \thetag1 upravíme na
$$
{|MI|\over |MF|}={|NJ|\over |NF|}.
$$
Z~toho vzhľadom na rovnosti
$$
|\angle IMF|=|\angle CMF|=|\angle CNF|=|\angle JNF|
$$
vyplýva podobnosť trojuholníkov $MIF$ a~$NJF$,
odkiaľ
$$
|\angle IFM|=|\angle JFN| \qquad\text{a}\qquad
|\angle IFJ|=|\angle MFN|.
$$
Spolu máme
$$
|\angle IFJ|=|\angle MFN|=|\angle MCN|=|\angle ICJ|,
$$
teda body $C$, $F$, $I$, $J$ ležia na jednej kružnici.
}

{%%%%%   trojstretnutie, priklad 4
V~riešení budeme pre obsah trojuholníka $XYZ$ používať označenie $S_{XYZ}$.
\insp{cps.3}%

Nech $PR$ je priemer kružnice opísanej trojuholníku $ABC$ (\obr). Zrejme $ARBP$ je pravouholník.
Keďže strany $BR$, $PA$ sú rovnobežné, máme $S_{PBK} = S_{PRK}$, odkiaľ
$$
S_{BKL} = S_{LPR}.
$$
Z~rovnosti $|PA|=|BR|$ vyplýva $|\uhol PCA| = |\uhol BPR| = |\uhol LPR|$.

Štvoruholník $CPAR$ je tetivový, takže $|\uhol CAP| = |\uhol CRP|$ a~teda trojuholníky $LPR$ a~$PCA$ sú podobné.

Napokon dostávame
$$
S_{BKL} : S_{ACP} = S_{LPR} : S_{ACP} = |PR|^2 : |AC|^2 = |AB|^2 : |AC|^2,
$$
čo je hodnota nezávislá od polohy bodu~$P$.


\ineriesenie
Veľkosti strán a uhlov trojuholníka $ABC$ označíme zvyčajným spôsobom. Ďalej označíme $\varphi=|\uh PAC|$ (\obr).
Potom $|\uh CPB|=\alpha$, $|\uh PBC|=\varphi$, $|\uh LCB|=|\uh ACP|=|\uh ABP|=\beta-\varphi$,
$|\uh BLC|=|\uh APC|=90^{\circ}+\alpha$, $|AP|=c\cos(\alpha+\varphi)$, $|BP|=c\sin(\alpha+\varphi)$, $|\uh PKL|=\alpha$.
\insp{cps.4}%

Zo sínusových viet pre trojuholníky $BCP$ a~$BCL$ máme
$$
\align
|CP|&=\frac{|BC|\sin\varphi}{\sin\alpha}=\frac{a\sin\varphi}{\sin\alpha},\\
|BL|&=\frac{|BC|\sin(\beta-\varphi)}{\sin(90^{\circ}+\alpha)}=\frac{a\cos(\alpha+\varphi)}{\cos\alpha}.
\endalign
$$
Ďalej
$$
\align
|PL|&=|BP|-|BL|=c\sin(\alpha+\varphi)-\frac{a\cos(\alpha+\varphi)}{\cos\alpha}=\\
&=\frac{c\sin(\alpha+\varphi)\cos\alpha-c\sin\alpha\cos(\alpha+\varphi)}{\cos\alpha}=\frac{c\sin\varphi}{\cos\alpha},\\
|KL|&=\frac{|PL|}{\sin\alpha}=\frac{c\sin\varphi}{\sin\alpha\cos\alpha}.
\endalign
$$
Pomer obsahov trojuholníkov je
$$
\frac{S_{KLB}}{S_{APC}}=\frac{\frac12 |BL|\cdot |KL|\cdot\sin(90\st+\alpha)}{\frac12 |AP| \cdot |CP|\cdot\sin(90\st+\alpha)}=\frac{|BL|\cdot|KL|}{|AP|\cdot|CP|}=
\frac{\frac{a\cos(\alpha+\varphi)}{\cos\alpha}\cdot\frac{c\sin\varphi}{\sin\alpha\cos\alpha}}
{{c\cos(\alpha+\varphi)}\cdot\frac{a\sin\varphi}{\sin\alpha}}=\frac1{\cos^2\alpha},
$$
veľkosť uhla $\alpha$ je zrejme nezávislá od polohy bodu~$P$.

\poznamka
Predošlé riešenie môžeme skrátiť, keď si všimneme, že z~veľkostí uhlov vyplýva podobnosť trojuholníkov $BLC$ a~$APC$ (\obrr1). Z~toho $|BL|/|AP|=|LC|/|PC|$ a~zadaný pomer možno vyjadriť vzťahom
$$
\frac{S_{KLB}}{S_{APC}}=
\frac{|BL|\cdot |KL|}{|AP| \cdot |CP|}=
\frac{|LC|\cdot |KL|}{|PC|^2}.
$$
Z~pravouhlého trojuholníka $KLP$ s~výškou~$PC$ máme podľa Euklidovej vety o~strane $|LC|\cdot |KL|=|LP|^2$, odkiaľ
$$
\frac{S_{KLB}}{S_{APC}}=\frac{|LP|^2}{|PC|^2}=\frac1{\cos^2\alpha}.
$$
}

{%%%%%   trojstretnutie, priklad 5
Každý 100-metrový úsek ulice medzi dvoma križovatkami budeme nazývať {\it šípka}. Šípku, ktorá má rovnaký smer ako ulica $WS$ alebo $WN$ (\tj. dá sa po nej ísť juhovýchodným alebo severovýchodným smerom), budeme nazývať {\it dopredná\/}, inak ju nazveme {\it spätná\/}.

V~riešení použijeme nasledovnú zrejmú lemu: {\sl Uvažujme ľubovoľné rozdelenie množiny všetkých križovatiek na dve disjunktné množiny $\Cal A$ a~$\Cal B$ také, že $W\in\Cal A$ a~$E\in\Cal B$. Potom počet prejazdov auta pozdĺž šípok z~$\Cal A$ do $\Cal B$ je o~1 väčší ako počet prejazdov pozdĺž šípok z~$\Cal B$ do $\Cal A$.}

Rozdeľme všetky križovatky mesta Mar del Plata na dve množiny $\Cal A$, $\Cal B$ zvislou (\tj. severo-južnou) priamkou spájajúcou dva body vzdialené $100k+50$ metrov od $W$ -- jeden na ulici~$WN$ a~druhý na ulici $WS$ -- pre nejaké $k\in\{0,1,\dots,n-1\}$. Na \obr{}a a~\obrrnum0b je znázornené rozdelenie pre $k=3$, resp. $k=4$.
\inspinspab{cps.6}{cps.5}%

Ak $k$ je nepárne, tak deliaca priamka pretína $k+1$ dopredných šípok (idúcich z~$\Cal A$ do $\Cal B$) a~$k+1$ spätných šípok (idúcich z~$\Cal B$ do $\Cal A$). Aj keby auto prešlo pozdĺž všetkých $k+1$ dopredných šípok, podľa lemy by prešlo len pozdĺž $k$ spätných šípok. Preto aspoň jedna spätná šípka ostane nepoliata.

Ak $k$ je párne a~$k\ge2$, tak deliaca priamka pretína $k+2$ dopredných šípok a~$k$ spätných šípok. Dve najsevernejšie dopredné šípky preťaté priamkou začínajú v~križovatke, ktorá má len jednu vchádzajúcu šípku. Túto križovatku dokáže auto prejsť len raz, takže aspoň jedna z~dvoch najsevernejších dopredných šípok ostane nepoliata. To isté platí pre dve najjužnejšie dopredné šípky. Existuje teda nanajvýš $k$ dopredných šípok, ktorými auto prejde, a~podľa lemy nanajvýš $k-1$ spätných šípok. Spolu máme na tejto úrovni aspoň 3 nepoliate šípky.

Pre $k=0$ máme len dve dopredné šípky začínajúce vo $W$ preťaté deliacou priamkou. Očividne iba jednou z~nich môže auto prejsť, druhá ostane nepoliata.

\smallskip
Podobným spôsobom rozdelíme križovatky zvislými priamkami spájajúcimi dva body vo vzdialenostiach $100k+50$ metrov od $E$ -- jeden na ulici~$SE$, druhý na ulici~$NE$ -- pre $k\in\{0,1,\dots,n-1\}$. Situácia je znázornená na \obr{}a a~\obrrnum0b pre $k=3$, resp. $k=4$.
\inspinspab{cps.8}{cps.7}%

Ak $k$ je nepárne, deliaca priamka pretína $k+1$ dopredných a~$k+1$ spätných šípok; aspoň jedna zo spätných šípok ostane nepoliata.

Ak $k$ je párne a~$k\ge2$, tak deliaca priamka pretína $k+2$ dopredných a~$k$ spätných šípok. Dve najsevernejšie dopredné šípky končia v~križovatke, odkiaľ vychádza iba jedna šípka, takže aspoň jedna z~nich ostane nepoliata. To isté platí pre dve najjužnejšie dopredné šípky. Opäť máme nanajvýš $k$ dopredných a~nanajvýš $k-1$ spätných šípok, pozdĺž ktorých auto prejde, čiže na tejto úrovni ostanú aspoň 3 nepoliate šípky.

Pre $k=0$ máme dve dopredné šípky končiace v~$E$, jednu z~nich auto prejde, druhá ostane nepoliata.

\smallskip
Pre každé nepárne $k$ máme jednu, pre každé párne $k\ge2$ tri a~pre $k=0$ jednu nepoliatu šípku a~celý postup sa zopakoval dvakrát. Keďže $n$ je párne a~$k\in\{0,1,\dots,{n-1}\}$, dokopy máme
$$
2\left(\tfrac12n+3(\tfrac12n-1)+1\right)=4n-4
$$
nepoliatych šípok. Celkový počet šípok je $n\cdot 2(n+1)$, takže auto nedokáže poliať viac ako
$$
n\cdot 2(n+1)-(4n-4)=2n^2-2n+4
$$
šípok.

\smallskip
Na druhej strane, existuje mnoho trás auta pozostávajúcich z~$2n^2-2n+4$ šípok. Jedna je načrtnutá na \obr{} pre $n=6$. Keď použijeme rovnaký vzor vo všeobecnom prípade, trasa bude rozdelená na $\frac12n$ častí križovatkami ležiacimi na ulici~$WS$ vzdialenými $200k$ metrov od $W$ pre $k=1,2,\dots,\frac12n-1$.
\insp{cps.9}%

Prvých $\frac12n-1$ častí sa líši len posunutím. Každá z~nich pozostáva z~$n$ šípok majúcich rovnaký smer ako $WN$, z~$n$ šípok opačného smeru ako $WN$ a~$2(n-1)$ šípok kolmých na $WN$. Posledná časť pozostáva z~$n$ šípok majúcich rovnaký smer ako $WN$ a~$2(n+1)$ šípok kolmých na $WN$. Celkový počet šípok na trase je
$$
\left(\tfrac12n-1\right)\bigl(2n+2(n-1)\bigr)+n+2(n+1)=2n^2-2n+4.
$$

\odpoved
Dĺžka najdlhšej možnej trasy polievacieho auta je $\frac1{10}(2n^2-2n+4)$~km.
}

{%%%%%   trojstretnutie, priklad 6
Označme $V=ab+bc+cd+da$. Určíme najväčšiu možnú hodnotu výrazu
$$
V^2 = (a+c)^2(b+d)^2=(a^2+c^2+2ac)(b^2+d^2+2bd).
\tag1
$$

Výraz $V$ je "cyklický" -- jeho hodnota sa nezmení, ak hodnotu $a$ zmeníme na $b$, $b$ na $c$, $c$ na $d$ a~$d$ na $a$.
Keďže $ac\cdot bd = 4$, aspoň jedno z~čísel $ac$, $bd$ je aspoň $2$. Vzhľadom na cyklickosť bez ujmy na všeobecnosti môžeme predpokladať, že $bd\ge2$.

Zo zadaných rovností odvodíme $ac = 4/bd$, $a^2+c^2=10-b^2-d^2$ a~tieto vyjadrenia dosadíme do \thetag1:
$$
\align
V^2&=\left(10-b^2-d^2+\frac{8}{bd}\right)(b^2+d^2+2bd)=\\
&=10(b^2+d^2) + 20bd + {8(b^2+d^2)\over bd} + 16 - (b^2+d^2)^2-2bd(b^2+d^2).
\tag2
\endalign
$$
Označme $P = b^2+d^2$ a~$Q=bd$; potom zrejme $P\ge 2Q$ a~podľa predpokladu $Q\ge 2$.
Po dosadení do \thetag2 pokračujeme v~úpravách:
$$
\align
V^2&=10P+20Q+{8P\over Q} + 16 - P^2 - 2PQ =\\
&= -(P^2-10P+25)  + \left(41 - 2PQ + 20Q + {8P\over Q}\right)=\\
&=-(P-5)^2+ \left[P\left(\frac8Q-2Q\right) + 41 + 20Q\right].
\endalign
$$
Zrejme $\m(P-5)^2\le 0$. Z~podmienky $Q\ge2$ vyplýva $8/Q-2Q\le8/2-2\cdot2=0$, takže výraz v~hranatých zátvorkách je nerastúcou lineárnou funkciou v~premennej~$P$ a~nadobúda svoje maximum pre najmenšie možné~$P$. Vzhľadom na $P\ge 2Q$ dostávame
$$
V^2\le 2Q\left(\frac8Q-2Q\right)+41+20Q=-4Q^2+20Q+57=-(2Q-5)^2+82\le82.
$$

Na záver stačí nájsť kladné reálne čísla $a$, $b$, $c$, $d$ také, že $V=\sqrt{82}$.
Rovnosť nastáva, keď $P = 2Q = 5$, čo je splnené pre $b=d=\frac12\sqrt{10}$. Čísla $a$, $c$ spĺňajú
$a^2+c^2 = 5$, $ac = 8/5$, teda
$$
\{a,c\} = \left\{{\sqrt{41}-3\over 2\sqrt{5}}, {\sqrt{41}+3\over 2\sqrt{5}}\right\}.
$$

\odpoved
Najväčšia možná hodnota výrazu $ab+bc+cd+da$ je $\sqrt{82}$.

}

{%%%%%   IMO, priklad 1
Označme veľkosti uhlov pri vrcholoch $A$, $B$, $C$ postupne $2\alpha$, $2\beta$, $2\gamma$.
Uhly $AKJ$ a~$ALJ$ sú pravé, preto $K$ aj $L$ ležia na Tálesovej kružnici~$k$ s~priemerom~$AJ$.
Kľúčovým pozorovaním je to, že aj body $F$ a~$G$ ležia na tejto kružnici. To teraz dokážeme.

Bod~$J$ leží na osi uhla $CAB$, preto uhol $LAJ$ má veľkosť~$\alpha$. Stačí dokázať, že aj uhol $LFJ$ má veľkosť~$\alpha$ (body $A$ a~$F$ ležia v~tej istej polrovine vzhľadom na priamku~$JL$, lebo polpriamka opačná k~$ML$, na ktorej leží bod~$F$, leží celá v~polrovine $BCA$). Bod~$J$ leží na vonkajších osiach uhlov pri vrcholoch $B$ a~$C$, preto uhol $CML$ má veľkosť~$\gamma$ a~uhol $MBJ$ má veľkosť $90^\circ-\beta$. Z~trojuholníka $BMF$ dostaneme, že $|\angle LFJ|=|\angle MBJ|-|\angle BMF|=90^\circ-\beta-\gamma =\alpha$, čiže bod~$F$ leží na kružnici~$k$. Analogicky dokážeme, že aj $G$ leží na $k$.
\insp{mmo.1}%

Keďže bod~$F$ leží na kružnici~$k$, je priamka~$AF$ kolmá na~$FJ$.
Lenže $FJ$ je osou vonkajšieho uhla pri vrchole~$B$ a~$KM\perp FJ$, preto je úsečka~$SM$ osovo súmerná s~úsečkou~$AK$, a~vďaka tomu $|SM|=|AK|$. Analogicky $|TM|=|AL|$. Avšak $AK$ aj $AL$ sú dotyčnice k~pripísanej kružnici so stredom~$J$, čiže $|AK|=|AL|$.
Preto $|SM| = |MT|$ a~teda bod~$M$ je stredom úsečky~$ST$.
}

{%%%%%   IMO, priklad 2
Keby sme priamočiaro použili nerovnosť medzi aritmetickým a~geometrickým priemerom na každý súčet na ľavej strane, dostaneme jej dolný odhad, v~ktorom sa mocniny čísel $a_2$, $a_3$, \dots, $a_n$ líšia, a~teda nebudeme môcť priamo využiť väzbu. Preto pred použitím tejto nerovnosti vhodne členy "navážime". Dostaneme odhad
$$
(1+a_k)^k = \left((k-1){1\over k-1}+a_k\right)^k\ge {k^k\over (k-1)^{k-1}} a_k
$$
platný pre každé celé číslo $k\ge 2$. Rovnosť v~tomto odhade nastáva pre $a_k = 1/(k-1)$, čo je pre $k>2$ menej ako $1$.

Vynásobením takýchto odhadov pre $k\in\{2,3,\dots,n\}$ dostaneme
$$
(1+a_2)^2(1+a_3)^3\dots(1+a_n)^n\ge n^na_2a_3\dots a_n = n^n.
$$
Rovnosť však nemôže nastať: ak by nastala, tak $a_2=1$ a~$a_k$ by bolo menšie ako $1$ pre všetky $k\ge 3$, to je však pre $n\ge 3$ spor s~podmienkou $a_2a_3\dots a_n=1$.

\ineriesenie
Matematickou indukciou dokážeme nasledujúce silnejšie tvrdenie:
$$
(1+a_2)^2(1+a_3)^3\dots(1+a_n)^n\ge n^n a_2a_3\dots a_n,
$$
pričom rovnosť nastáva práve vtedy, keď zároveň platí $a_k = 1/(k-1)$ pre každé $k\in \{2,3,\dots, n\}$.

Pre $n=2$ je nerovnosť $(1+a_2)^2\ge 4a_2$ ekvivalentná s~$(a_2-1)^2\ge 0$; rovnosť nastane pre $a_2=1$.

V~druhom kroku nám po využití indukčného predpokladu stačí dokázať, že
$$
(1+a_{n+1})^{n+1}\ge {(n+1)^{n+1}\over n^n} a_{n+1}.
\tag1%{mi2}
$$
Uvažujme funkciu
$$
f(x) = (1+x)^{n+1}-{(n+1)^{n+1}\over n^n}x
$$
definovanú na množine kladných reálnych čísel. Táto funkcia je na celom definičnom obore diferencovateľná a~jej derivácia je
$$
f'(x) = (n+1)(1+x)^n-{(n+1)^{n+1}\over n^n} = {n+1\over n^n}\big((n+nx)^n-(n+1)^n\big).
$$
Funkcia~$f$ je klesajúca na intervale $(0, 1/n)$, kde má zápornú deriváciu, a~rastúca na intervale $(1/n,\infty)$, kde má kladnú deriváciu.
Preto funkcia~$f$ má v~bode $x=1/n$ globálne minimum. Z~$f(1/n)=0$ potom vyplýva platnosť odhadu \thetag1 a~tiež to, že v~ňom rovnosť nastáva len pre $a_{n+1}=1/n$.
}

{%%%%%   IMO, priklad 3
a) \podla{Eduarda Batmendijna}
Víťazná stratégia hráča~$B$ vyzerá nasledovne. Začneme s~množinou $\mm M=\{1,2,\dots,N\}$ a~budeme túto množinu postupne zmenšovať tak, aby stále obsahovala číslo~$x$.
Keď bude mať $\mm M$ nanajvýš $2^k$ prvkov, prehlásime ju za finálnu odpoveď.

Predpokladajme, že $\mm M$ obsahuje viac ako $2^k$ prvkov. Ukážeme, ako ju zmenšiť. Zvoľme ľubovoľné $y\in\mm M$. Budeme sa hráča~$A$ opakovane pýtať na množinu~$\{y\}$. Ak $A$ odpovie $k+1$ ráz "nie", vieme, že $y\ne x$, preto môžeme $y$ z~$\mm M$ odstrániť a~zopakovať celý postup s~novou menšou množinou $\mm M$.

Ak hráč~$A$ na niektorú z~našich otázok odpovedal "áno", tak vlastne povedal "nie" pre množinu $\mm P_1 = \mm M-\{y\}$. Množina $\mm P_1$ obsahuje aspoň $2^k$ prvkov, lebo $\mm M$ obsahovala viac ako $2^k$ prvkov. Pritom vieme, že pre každý prvok množiny $\mm P_1$ už $A$ raz odpovedal "nie". Množinu $\mm P_1$ rozdelíme na dve rovnako veľké časti a~spýtame sa na jednu z~nich. Či už $A$ odpovie "áno" alebo "nie", vieme to interpretovať ako "nie" pre jednu z~našich častí; označme ju $\mm P_2$. Pre každý prvok v~$\mm P_2$ už $A$ dvakrát odpovedal "nie". Množinu $\mm P_2$ opäť rozdelíme na dve rovnako veľké podmnožiny a~tento postup opakujeme, až kým nedostaneme množinu $\mm P_{k+1}$. (Množina $\mm P_{k+1}$ je neprázdna, lebo $\mm P_1$ obsahovala aspoň $2^k$ prvkov.) Pre ľubovoľný jej prvok~$z$ hráč~$A$ dal $k+1$ po sebe idúcich odpovedí "nie" a~aspoň raz musel vravieť pravdu, preto $z\ne x$ a~môžeme ho odstrániť z~$\mm M$.

\smallskip
b) Dokážeme, že ak $\lambda\in(1,2)$ a $n = \lfloor(2-\lambda)\lambda^{k+1}\rfloor -1$, hráč $B$ si nedokáže zabezpečiť víťazstvo.
Na dokončenie dôkazu potom stačí zvoliť $\lambda\in(1{,}99; 2)$ a~$k$ dostatočné veľké na to, aby platilo $n\ge 1{,}99^k$.

Predpokladajme, že hráč~$B$ sa pýta na množinu~$\mm S$. O~odpovedi hráča~$A$ budeme hovoriť, že {\it nie je v~súlade s~prvkom~$i$}, ak $i\in\mm S$ a~odpoveď je "nie", alebo $i\notin\mm S$ a~odpoveď je "áno". Odpoveď je nepravdivá práve vtedy, keď nie je v~súlade s~prvkom~$x$.

Uvažujme nasledujúcu stratégiu hráča~$A$. Najprv zvolí $N = n+1$ a~ľubovoľné $x\in \{1, 2, \dots, n+1\}$.
Po každej svojej odpovedi hráč~$A$ pre každé $i\in\{1,2,\dots,n+1\}$ určí počet $m_i$ po sebe idúcich predchádzajúcich odpovedí, ktoré nie sú v~súlade s~$i$.
Pri rozhodovaní o~svojej nasledujúcej odpovedi využije hodnotu
$$
\phi = \sum_{i=1}^{n+1} \lambda^{m_i}.
$$
Bez ohľadu na otázku hráča~$B$ odpovie $A$ tak, aby minimalizoval hodnotu~$\phi$.

Tvrdíme, že pri tejto stratégii je hodnota~$\phi$ vždy menšia ako~$\lambda^{k+1}$. Žiadny z~exponentov~$m_i$ potom nemôže presiahnuť~$k$, a~teda hráč~$A$ pre žiadne~$i$ nepovie viac ako $k$ po sebe idúcich odpovedí, ktoré nie sú v~súlade s~$i$. Preto nebude klamať viac ako $k$-krát po sebe. Takáto stratégia vyhovuje pravidlám; navyše vôbec nezávisí od voľby~$x$, čiže nedáva hráčovi~$B$ žiadnu informáciu. Ostáva dokázať, že $\phi$ je vždy menšie ako $\lambda^{k+1}$.

Na začiatku je $m_i=0$ pre každé~$i$, preto $\phi=n+1$ a~platnosť nášho tvrdenia vyplýva z~voľby $n = \lfloor(2-\lambda)\lambda^{k+1}\rfloor -1$ a~podmienky $\lambda\in(1,2)$. Predpokladajme, že v~niektorom momente je $\phi<\lambda^{k+1}$ a~hráč~$B$ sa pýta, či $x\in\mm S$ pre nejakú množinu~$\mm S$. Podľa toho, či hráč~$A$ odpovie "áno" alebo "nie", bude nová hodnota $\phi$
$$
\phi_1=\sum_{i\in\mm S}1 + \sum_{i\notin\mm S} \lambda^{m_i+1}\qquad\text{alebo}\qquad \phi_2=\sum_{i\in\mm S}\lambda^{m_i+1} + \sum_{i\notin\mm S}1.
$$
Keďže hráč~$A$ minimalizuje $\phi$, bude nová hodnota rovná $\min\{\phi_1, \phi_2\}$. Pritom
$$
\min\{\phi_1, \phi_2\}\le {1\over 2}(\phi_1+\phi_2)=
{1\over 2}\left(\sum_{i\in\mm S}(1+\lambda^{m_i+1}) + \sum_{i\notin\mm S}(\lambda^{m_i+1}+1)\right)={1\over 2}(\lambda\phi+n+1).
$$
Keďže $\phi < \lambda^{k+1}$, z~predpokladov $\lambda<2$ a $n = \lfloor(2-\lambda)\lambda^{k+1}\rfloor -1$ dostaneme
$$
\min\{\phi_1, \phi_2\} < {1\over 2}(\lambda^{k+2}+(2-\lambda)\lambda^{k+1}) = \lambda^{k+1}.
$$
Tým sme dôkaz ukončili.
}

{%%%%%   IMO, priklad 4
Po dosadení $a=b=c=0$ vidíme, že $3f(0)^2=6f(0)^2$, preto $f(0)=0$.
Dosadením $b=\m a$, $c=0$ dostaneme $(f(a)-f(\m a))^2=0$, a~teda $f(\m a)=f(a)$ pre každé celé číslo~$a$, čiže funkcia~$f$ je párna.

Zvoľme $b=a$ a $c=\m2a$; dostaneme $2f(a)^2+f(2a)^2=2f(a)^2+4f(a)f(2a)$. Preto
$$
f(2a)=0\quad\text{alebo}\quad f(2a)=4f(a)\qquad\text{pre každé $a\in \Bbb Z$}.
\tag1%{eq3}
$$

Ak $f(r)=0$ pre nejaké $r\ge 1$, zo substitúcie $b=r$ a $c=\m a-r$ dostaneme $(f(a+r)-f(a))^2=0$, čiže funkcia $f$ je periodická s~periódou~$r$.

Ak $f(1)=0$, tak $f$ je konštantná nulová funkcia. Tá je evidentne riešením zadanej rovnice. Vo zvyšku riešenia budeme predpokladať, že $f(1)=k\ne 0$.

Zo vzťahu~\thetag1 vidíme, že buď $f(2)=0$, alebo $f(2)=4k$. Ak $f(2)=0$, tak funkcia $f$ je periodická s~periódou~$2$ a~teda jej funkčné hodnoty v~párnych číslach sú nulové a~v~nepárnych číslach sú všetky rovné~$k$. Takáto funkcia je riešením; skúšku správnosti však v~tejto chvíli odložíme a~budeme vo zvyšku riešenia predpokladať, že $f(2)=4k\ne 0$.

Opätovným využitím vzťahu~\thetag1 dostaneme, že buď $f(4) = 0$, alebo $f(4)=16k$. V~prvom prípade je funkcia $f$ periodická s~periódou~$4$. Pritom $f(3)=f(\m1)=f(1)=k$, čiže periodicky sa budú postupne opakovať hodnoty $0$, $k$, $4k$, $k$. Aj takáto funkcia je riešením (overíme neskôr); vo zvyšku riešenia budeme predpokladať, že $f(4)=16k\ne 0$.

Teraz dokážeme, že $f(3)=9k$. Najprv zo substitúcie $a=1$, $b=2$, $c=\m3$ dostaneme $f(3)^2-10k f(3)+9k^2=0$, preto $f(3)\in\{k, 9k\}$.
Následne zo substitúcie $a=1$, $b=3$, $c=\m4$ dostaneme $f(3)^2-34k f(3)+ 225k^2=0$, preto $f(3)\in\{9k, 25k\}$. Niet teda inej možnosti ako $f(3)=9k$.

Nakoniec matematickou indukciou dokážeme, že $f(x)=kx^2$ pre všetky celé čísla~$x$. Doteraz sme ukázali, že je to pravda pre $x\in\{0,1,2,3,4\}$.
V~druhom kroku indukcie budeme predpokladať, že $n\ge 4$ a~že $f(x)=kx^2$ pre všetky celé čísla~$x$ z~množiny $\{0, 1,\dots, n\}$.
Dosadenia $a=n$, $b=1$, $c=\m n-1$ a $a=n-1$, $b=2$, $c=\m n-1$ vedú k~tomu, že
$$
f(n+1)\in\{k(n+1)^2, k(n-1)^2\}\qquad \text{a}\qquad f(n+1)\in\{k(n+1)^2, k(n-3)^2\}.
$$
Keďže pre $n\ne 2$ sú hodnoty $k(n-1)^2$ a~$k(n-3)^2$ rôzne, jedinou možnosťou je ${f(n+1)}=k(n+1)^2$.
Tým sme ukončili druhý krok indukcie a~dokázali, že $f(x)=kx^2$ pre všetky nezáporné celé~$x$.
Keďže funkcia~$f$ je párna, platí to aj pre záporné~$x$.
Pri overovaní tohto riešenia stačí dokázať, že $a^4+b^4+(a+b)^4=2a^2b^2+2a^2(a+b)^2+2b^2(a+b)^2$, to však vidno z~roznásobenia jednotlivých strán.

Jedinými možnými riešeniami zadanej funkcionálnej rovnice sú teda konštantná funkcia $f_1(x)=0$  a~funkcie $f_2(x)=kx^2$,
$$
f_3(x)=\left\{
\aligned
0&\quad\text{pre $x$ párne,}\\
k&\quad\text{pre $x$ nepárne,}
\endaligned
\right.\qquad
f_4(x)=\left\{
\aligned
0&\quad\text{pre $x\equiv 0\pmod 4$,}\\
k&\quad\text{pre $x\equiv 1\pmod 2$,}\\
4k&\quad\text{pre $x\equiv 2\pmod 4$}
\endaligned
\right.
$$
pre ľubovoľné nenulové celé číslo~$k$.\footnote{Funkcie $f_2$, $f_3$, $f_4$ samozrejme vyhovujú aj pre $k=0$, vtedy sú však totožné s~funkciou $f_1$.} Ostáva spraviť skúšku správnosti pre funkcie $f_3$ a~$f_4$. Začneme funkciou~$f_3$. Ak $a+b+c=0$, tak buď sú $a$, $b$, $c$ všetky párne, vtedy $f(a)=f(b)=f(c)=0$, alebo jedno je párne a~dve sú nepárne, vtedy hodnota oboch strán zadanej rovnice je $2k^2$. Pre funkciu~$f_4$ podobnou úvahou s~využitím symetrie zadanej rovnice zistíme, že stačí overiť trojice $(0, k, k)$, $(4k, k, k)$, $(0,0,0)$, $(0,4k, 4k)$. Pre každú z~nich platí rovnosť.
}

{%%%%%   IMO, priklad 5
Označme $C'$ obraz bodu~$C$ v~osovej súmernosti podľa priamky~$AB$. Kružnica~$k_1$ má stred v~bode~$A$ a~prechádza bodmi $C$, $L$ a~$C'$.
Kružnica~$k_2$ má stred v~bode~$B$ a~prechádza bodmi $C$, $K$ a~$C'$. Keďže uhol $ACB$ je pravý, v~bode~$C$ sa priamka~$AC$ dotýka $k_2$ a~priamka~$BC$ sa dotýka $k_1$.
Označme $K'$ priesečník priamky~$AX$ s~$k_2$ rôzny od $K$ a~$L'$ priesečník priamky~$BX$ s~$k_1$ rôzny od $L$.

Z~mocnosti bodu~$X$ ku kružniciam $k_1$ a~$k_2$ dostaneme
$$
|XK|\cdot |XK'| = |XC|\cdot |XC'| = |XL|\cdot |XL'|,
$$
čiže body $L'$, $K$, $L$, $K'$ ležia na jednej kružnici~$k_3$.
\insp{mmo.2}%

Z~mocnosti bodu~$A$ ku kružnici~$k_2$ máme $|AL|^2 = |AC|^2 = |AK|\cdot |AK'|$, preto $AL$ je dotyčnicou ku $k_3$ v~bode~$L$.
Analogicky $BK$ je dotyčnicou ku $k_3$ v~bode~$K$.
Takže priamky $MK$ a~$ML$ sú dotyčnice z~bodu~$M$ ku kružnici~$k_3$, a~preto $|MK|$ = $|ML|$.
}

{%%%%%   IMO, priklad 6
Vhodné čísla $a_1$, $a_2$, \dots, $a_n$ existujú pre $n$ so zvyškom $1$ alebo $2$ po delení štyrmi.

Najprv dokážeme, že táto podmienka je nutná. Predpokladajme, že $\sum_{k=1}^n k/3^{a_k} = 1$; nech $a$ je najväčšie z~čísel~$a_k$.
Po prenásobení $3^a$ dostaneme
$$
1\cdot x_1 + 2\cdot x_2 + \dots + n\cdot x_n = 3^a,
$$
pričom $x_k$ sú mocniny čísla~$3$.
Pravá strana v~získanej rovnosti je nepárna a~ľavá má takú istú paritu ako súčet $1+2+\cdots+n$. Preto $n$ dáva zvyšok $1$ alebo $2$ po delení štyrmi.
Ostáva dokázať, že uvedená podmienka je postačujúca.

Postupnosť $b_1$, $b_2$, \dots, $b_n$ budeme volať {\it prípustná}, ak existujú nezáporné celé čísla $a_1$, $a_2$, \dots, $a_n$ také, že
$$
{1\over 2^{a_1}}+{1\over 2^{a_2}}+\dots+{1\over 2^{a_n}} = {b_1\over 3^{a_1}}+{b_2\over 3^{a_2}}+\dots+{b_n\over 3^{a_n}} = 1.
$$

Pre danú postupnosť vieme robiť na jej členoch rôzne operácie. Vezmime prípustnú postupnosť $b_1$, $b_2$, \dots, $b_n$ (so zodpovedajúcimi exponentmi $a_1$, $a_2$, \dots, $a_n$ z~definície prípustnosti) a~zvoľme nezáporné celé čísla $u$, $v$ so súčtom $3b_k$. Postupnosť $b_1$, $b_2$, \dots, $b_{k-1}$, $u$, $v$, $b_{k+1}$, \dots, $b_n$ je tiež prípustná, pretože
$$
{1\over 2^{a_k+1}}+{1\over 2^{a_k+1}} = {1\over 2^{a_k}}\qquad\text{a}\qquad {u\over 3^{a_k+1}}+{v\over 3^{a_k+1}}={b_k\over 3^{a_k}}.
$$
Toto môžeme sformulovať aj naopak: ak v~postupnosti nahradíme dva členy $u$ a~$v$ jedným členom $(u+v)/3$ a~dostaneme tým prípustnú postupnosť, tak aj pôvodná postupnosť bola prípustná. (Tieto dva členy nemusia nasledovať bezprostredne po sebe.)

Predpokladajme, že $n\equiv 1$ alebo $n\equiv2\pmod 4$ a~označme $\alpha_n$ postupnosť $1, 2, \dots, n$.
Ukážeme, že postupnými úpravami typu $\{u,v\}\mapsto (u+v)/3$ možno zredukovať túto postupnosť na jednoprvkovú postupnosť~$\alpha_1$, ktorá je zjavne prípustná (s~exponentom $a_1=0$).
Špeciálnym prípadom tejto operácie je $\{m,2m\}\mapsto m$, čiže môžeme vynechať číslo~$2m$, ak postupnosť okrem neho obsahuje aj číslo~$m$.

Predpokladajme, že $n\ge 16$. Ukážeme, že $\alpha_n$ vieme zredukovať na $\alpha_{n-12}$ pomocou 12~operácií. Pre vhodné $k\ge 1$ a~$0\le r\le 11$ platí $n=12k+r$.
Ak $r\in\{0,1,\dots,5\}$, z~posledných 12~členov $\alpha_n$ vynecháme $12k-6$ a~$12k$ a~zvyšok rozdelíme na päť dvojíc
$$
\{12k-6-i, 12k-6+i\},\ i\in\{1,2,\dots,5-r\};\quad \{12k-j, 12k+j\}, \ j\in\{1,2,\dots,r\}.
$$
Päť operácií vykonaných na uvedených pároch odstráni 10~čísel a~pridá 5 nových čísel rovných $8k-4$ alebo $8k$.
Všetky pridané čísla však môžeme vynechať, lebo $4k-2$ aj $4k$ ostali v~postupnosti (nerovnosť $4k\le n-12$ je ekvivalentná nerovnosti $8k\ge {12-r}$,
ktorej platnosť ľahko overíme pre každé $r$). Takto sme v~tomto prípade úspešne zredukovali $\alpha_n$ na $\alpha_{n-12}$.

V~prípade $r\in\{6,7,\dots,11\}$ postupujeme analogicky. Čísla $12k$ a~$12k+6$ vynecháme a~zvyšok rozdelíme na dvojice
$$
\{12k+6-i, 12k+6+i\},\ i\in\{1,2,\dots,r-6\};\quad \{12k-j, 12k+j\}, \ j\in\{1,2,\dots,11-r\}.
$$
Po aplikovaní operácie na jednotlivé páry nám pribudnú čísla $8k$ a~$8k+4$, ktoré môžeme vynechať. Získame tak opäť $\alpha_{n-12}$.

Ostáva vyšetriť $n\in\{2,5,6,9,10,13,14\}$. Prípady $n\in\{2,6,10,14\}$ sa vynechaním posledného člena zredukujú na $n\in\{1,5,9,13\}$.
Pre $n=5$ použijeme $\{4,5\}\mapsto 3$, potom $\{3,3\}\mapsto 2$ a~nakoniec vynecháme dve dvojky. Pre $n=9$ najprv vynecháme $6$ a~potom použijeme
$\{5,7\}\mapsto 4$, $\{4,8\}\mapsto 4$, $\{3,9\}\mapsto 4$. Ďalej vynecháme tri štvorky a~nakoniec vynecháme dvojku. Prípad $n=13$ sa dá použitím
$\{11,13\}\mapsto 8$ a~vynechaním $8$ a~$12$ previesť na prípad $n=10$.
}

{%%%%%   MEMO, priklad 1
Skúmajme najskôr, pre aké hodnoty $x$ platí $x+f(y)=xy+1$. Po jednoduchej úprave dostaneme z~tejto rovnosti za predpokladu $y\ne1$ ekvivalentné vyjadrenie
$$
x=\frac{f(y)-1}{y-1}.
\tag1
$$
Ak by existovalo kladné $y\ne1$ také, že výraz $\thetag1$ je kladný, mohli by sme takéto $x$ a~$y$ dosadiť do zadanej rovnosti a~dostali by sme
$f(A)=yf(A)$ (pričom $A=x+f(y)=xy+1$), čo je v~spore s~tým, že $y\ne1$ a~zároveň $f(A)\ne0$. Preto výraz \thetag1 je pre každé $y\ne1$ záporný, čiže pre $y>1$ platí $f(y)<1$ a~pre $y<1$ platí $f(y)>1$.

Zvoľme ľubovoľné $y>1$ a~položme $x=1-1/y$. Dosadením týchto hodnôt do zadanej rovnosti dostaneme
$$
f\left(1-\frac1y+f(y)\right)=yf(y).
$$
Ak $f(y)>1/y$, tak pravá, a~teda aj ľavá strana predošlej rovnosti je väčšia ako $1$, z~čoho vzhľadom na vlastnosť odvodenú v~predošlom odseku vyplýva
$$
1-\frac1y+f(y)\le1,\qquad \text{čiže}\quad f(y)\le\frac1y,
$$
čo je spor. Analogicky z~predpokladu $f(y)<1/y$ odvodíme spor $f(y)\ge1/y$. Preto pre všetky $y>1$ platí $f(y)=1/y$.

Napokon uvažujme ľubovoľné $0<a\le1$ a~zvoľme ľubovoľné $y$ spĺňajúce $y>1/a\ge1$. Položením $x=a-1/y$ dostaneme (s~využitím zadanej rovnosti a~už odvodených hodnôt $f$ pre argumenty väčšie ako $1$)
$$
f(a)=f\left(x+\frac1y\right)=f(x+f(y))=yf(xy+1)=y\cdot\frac1{xy+1}=\frac1{x+\frac1y}=\frac1a.
$$
Preto jediným kandidátom na riešenie je funkcia $f(x)=1/x$ pre všetky $x\in\Bbb R^\p$. O~tom, že vyhovuje, sa presvedčíme triviálnou skúškou.
}

{%%%%%   MEMO, priklad 2
Ak trojica rôznych prirodzených čísel $a$, $b$, $c$ spĺňa $a\mid b$ a~súčasne $b\mid c$ (takúto trojicu budeme nazývať {\it zakázaná}\/), tak $b\ge2a$ a~$c\ge2b$, z~čoho vyplýva $c\ge4a$. Ak teda najväčšie číslo z~$\mm S$ je menšie ako štvornásobok najmenšieho čísla z~$\mm S$, neobsahuje množina~$\mm S$ žiadnu zakázanú trojicu, čiže je prípustná. Dajme do $\mm S$ všetky čísla väčšie ako $\frac14N$, teda položme
$$
\mm S=\left\{\lfloor\tfrac14N\rfloor+1,\lfloor\tfrac14N\rfloor+2,\dots,N\right\}.
$$
Táto prípustná množina má $N-\lfloor\frac14N\rfloor=\lceil\frac34N\rceil$ prvkov. Ukážeme, že viac prvkov žiadna prípustná množina obsahovať nemôže.

Množina $\mm M=\{1,2,\dots,N\}$ obsahuje $\lceil\frac12N\rceil$ nepárnych čísel. Pre každé také nepárne číslo $q$ uvažujme množinu
$$
\mm H_q=\left\{q,2q,4q,\dots,2^{i_q}q\right\},
$$
pričom $i_q$ je najväčšie nezáporné celé číslo~$i$ také, že $2^i\cdot q\le N$. Množiny~$\mm H_q$ tvoria rozklad množiny~$\mm M$ (každé číslo z~$\mm M$ sa dá jednoznačne napísať v~tvare $2^i\cdot q$ pre nejaké nepárne $q\le N$ a~celé $i\ge0$). Je zrejmé, že ľubovoľná trojica čísel z~tej istej množiny~$\mm H_q$ je zakázaná. V~množine~$\mm S$ teda môžu byť najviac dve čísla z~každej množiny~$\mm H_q$. Avšak pre $q>\frac12N$ je množina $\mm H_q$ jednoprvková -- obsahuje iba číslo~$q$.

Spolu dostávame, že $\mm S$ môže obsahovať z~množín $\mm H_q$ najviac po dve čísla pre $1\le q\le\frac12N$ a~po jednom čísle pre $\frac12N<q\le N$, teda
$$
|\mm S|\le 2\cdot \left\lceil\tfrac12\lfloor\tfrac12N\rfloor\right\rceil + 1\cdot\left(\lceil\tfrac12N\rceil-\left\lceil\tfrac12\lfloor\tfrac12N\rfloor\right\rceil\right)=
\left\lceil\tfrac12\lfloor\tfrac12N\rfloor\right\rceil + \lceil\tfrac12N\rceil = \lceil\tfrac34N\rceil.
$$
Poslednú úpravu je možné urobiť napríklad osobitným rozobraním dvoch prípadov podľa parity čísla~$N$.

\odpoved
Prípustná množina môže mať najviac $\lceil\frac34N\rceil$ prvkov.
}

{%%%%%   MEMO, priklad 3
Z~rovnobežnosti priamok $AD$ a~$FC$, resp. priamok $AB$ a~$CD$ vyplýva
$$
|\uhol ADE|=|\uhol FEB|,\qquad\text{resp.}\qquad |\uhol CDB|=|\uhol ABD|.
$$
Podľa zadania je však $DB$ osou uhla $ADC$, takže všetky štyri uvedené uhly majú rovnakú veľkosť (\obr).
\insp{memo.1}%
Trojuholníky $BDA$ a~$BEF$ sú teda rovnoramenné a~vzhľadom na to, že $AFCD$ je rovnobežník, platí
$$
|AB|=|AD|=|CF|\qquad\text{a}\qquad |FE|=|FB|.
\tag1
$$

Trojuholníky $EFO$ a~$FBO$ sú oba rovnoramenné a~ich ramená majú zhodné dĺžky rovné polomeru kružnice opísanej trojuholníku $BEF$. Podľa \thetag1 majú zhodné aj základne, sú teda zhodné, čiže $|\uhol EFO|=|\uhol FBO|$. Z~toho a~z~\thetag1 dostávame, že aj trojuholníky $CFO$ a~$ABO$ sú zhodné, a~to podľa vety {\it sus}. Odtiaľ $|CO|=|AO|$, preto trojuholník $ACO$ je rovnoramenný, a~vzhľadom na to, že uhol $ACO$ má veľkosť $60\st$, je dokonca rovnostranný. Potom $|\uhol AOC|=60\st$, \tj. trojuholník $CFO$ je obrazom trojuholníka $ABO$ v~otočení o~$60\st$ (keďže sú zhodné). Preto aj $|\uhol BOF|=60\st$ a~trojuholník $FBO$ je rovnostranný.

S~využitím \thetag1 teda dostávame
$$
|AF|+|FO|=|AF|+|FB|=|AB|=|CF|,
$$
čo bolo treba dokázať.
}

{%%%%%   MEMO, priklad 4
Pomocou matematickej indukcie možno ľahko nahliadnuť, že všetky členy zadanej postupnosti sú párne. Párne sú totiž prvé dva členy a~ak sú párne členy $a_{n-1}$ aj $a_n$ pre nejaké $n$, tak súčin $a_na_{n-1}$ je deliteľný štyrmi, čiže číslo $\frac12a_na_{n-1}$ je párne a~člen $a_{n+1}$ je ako súčet troch párnych čísel tiež párny. Prvočíslo $p=2$ teda nevyhovuje. Naopak, $p=3$ vyhovuje, pretože $3\mid a_1-1$. Ďalej budeme predpokladať, že $p\ge5$.

Zadané vyjadrenie člena $a_{n+1}$ možno upraviť na tvar
$$
a_{n+1}=\frac{a_na_{n-1}+2a_n+2a_{n-1}}2=\frac{(a_n+2)(a_{n-1}+2)}2-2.
$$
Ak označíme $b_n=\frac12(a_n+2)$, platí $b_{n+1}=b_nb_{n-1}$. Vzhľadom na to, že $b_0=2$ a~$b_1=3$, sú všetky $b_n$ prirodzené čísla majúce v~rozklade na súčin prvočísel len činitele $2$ a~$3$, teda $p\nmid b_n$. Označme $z_n$ zvyšok čísla $b_n$ po delení~$p$ a~skúmajme postupnosť týchto zvyškov. Keďže každý ďalší zvyšok $z_{n+1}$ je jednoznačne určený dvojicou predošlých zvyškov $(z_{n-1},z_n)$~a rôznych dvojíc zvyškov je len konečne veľa, musí sa niektorá dvojica v~postupnosti zopakovať a~postupnosť $\{z_n\}$ je periodická, dĺžku periódy označme~$d$.

Tvrdíme, že prvá dvojica $(z_{n-1},z_n)$, ktorá sa v~postupnosti zopakuje, je dvojica $(z_0,z_1)$ (\tj. postupnosť je periodická od začiatku -- nemá predperiódu). Predpokladajme sporom, že prvá sa zopakuje nejaká dvojica $(z_{k-1},z_k)=(z_{d+k-1},z_{d+k})$ pre $k>1$. Potom platí
$$
\align
z_k&\equiv z_{k-2}z_{k-1}\pmod p,\\
z_k&\equiv z_{d+k}\equiv z_{d+k-2}z_{d+k-1}\equiv z_{d+k-2}z_{k-1} \pmod p.
\endalign
$$
Odčítaním kongruencií dostaneme
$$
0\equiv (z_{k-2}-z_{d+k-2})z_{k-1}\pmod p,
$$
z~čoho vzhľadom na to, že $p\nmid b_{k-1}$, vyplýva $z_{k-2}=z_{d+k-2}$, teda aj $(z_{k-2},z_{k-1})=(z_{d+k-2},z_{d+k-1})$. To je však spor s~predpokladom, že prvá zopakovaná dvojica bola $(z_{k-1},z_k)$.

Zistili sme, že dvojica $(z_0,z_1)=(2,3)$ sa v~postupnosti zvyškov objaví aj na mieste $(z_d,z_{d+1})$. Potom máme
$$
3\equiv b_{d+1}=b_db_{d-1}\equiv 2b_{d-1}\pmod p,
\qquad\text{čiže}\qquad p\mid 2b_{d-1}-3=a_{d-1}-1.
$$

\odpoved
Zadaným podmienkam vyhovujú všetky prvočísla okrem $2$.
}

{%%%%%   MEMO, priklad t1
Predpokladajme, že čísla $x$, $y$, $z$ vyhovujú zadaniu. Z~prvej rovnosti triviálne dostávame $x\ne0$ a~podobne z~ostatných dvoch rovností máme $y,z\ne0$. Preto môžeme rovnice prepísať na tvar
$$
z=\frac{2x^3+1}{3x},\qquad
x=\frac{2y^3+1}{3y},\qquad
y=\frac{2z^3+1}{3z}.
$$
Z~týchto vyjadrení vyplývajú implikácie
$$
x>0\quad\Rightarrow\quad z>0\quad\Rightarrow\quad y>0\quad\Rightarrow\quad x>0,
$$
teda ak je ktorékoľvek z~čísel $x$, $y$, $z$ kladné, sú kladné aj zvyšné dve. Preto buď sú všetky tri čísla kladné, alebo všetky záporné. Tieto dva prípady vyšetríme osobitne, pričom v~oboch rozboroch použijeme rovnosti, ktoré dostaneme vzájomným odčítaním dvojíc zadaných rovností:
$$
\align
2(x^3-y^3)&=3x(z-y),\tag1\\
2(y^3-z^3)&=3y(x-z),\tag2\\
2(z^3-x^3)&=3z(y-x).\tag3\\
\endalign
$$

\item{$\triangleright$}
{\it Prípad $x,y,z>0$.}
Ak $x>y$, tak ľavá strana~\thetag1 je kladná, čiže aj pravá strana je kladná, \tj. $z>y$. Potom je ľavá strana \thetag2 záporná, čiže aj pravá je záporná, \tj. $z>x$. Potom je však ľavá strana \thetag3 kladná, čiže aj pravá je kladná, \tj. $y>x$, čo je v~spore s~úvodným predpokladom. Zrejme rovnako dostaneme spor z~predpokladu $x<y$ (všetky nerovnosti sa len otočia). Nutne teda $x=y$ a~napr. z~\thetag1 následne $x=y=z$.

\item{$\triangleright$}
{\it Prípad $x,y,z<0$.} Postupujeme podobne ako v~predošlom prípade, avšak berieme do úvahy zápornosť činiteľov pred zátvorkami na pravých stranách, takže zátvorky na oboch stranách v~každej rovnici majú opačné znamienka. Ak $x>y$, tak z~\thetag1 vyplýva $z<y$. Z~toho podľa \thetag2 máme $z>x$. Spolu teda $x>y>z>x$, čo je spor. Prípad $x<y$ vedie rovnako k~sporu a~znova dostávame $x=y=z$.

\smallskip\noindent
Ukázali sme, že všetky tri čísla musia byť rovnaké. Vyhovujúce trojice už nájdeme ľahko vyriešením rovnice $2x^3+1=3x^2$, ktorú možno po prevedení členov na jednu stranu rozložiť na súčinový tvar $(2x+1)(x-1)^2=0$, \tj. $x=1$ alebo $x=\m\frac12$.

\odpoved
Zadaniu vyhovujú trojice $(1,1,1)$ a~$(\m\frac12,\m\frac12,\m\frac12)$.
}

{%%%%%   MEMO, priklad t2
Pri riešení použijeme nasledovné pomocné tvrdenie: Ak pre kladné reálne čísla $x$, $y$, $z$ platí $x+y+z<3$, tak
$$
\frac{9-x^2}x\cdot\frac{9-y^2}y\cdot\frac{9-z^2}z>512.
$$

\dokaz
Podľa AG-nerovnosti pre štvoricu čísel platí
$$
3+x=1+1+1+x\ge4\root4\of x
$$
a~analogicky $3+y\ge4\root4\of y$, $3+z\ge4\root4\of z$. Vynásobením týchto troch nerovností získame
$$
(3+x)(3+y)(3+z)\ge64\root4\of{xyz}.
\tag1
$$
Podľa AG-nerovnosti pre trojicu čísel máme $3>x+y+z\ge3\root3\of{xyz}$, z~čoho vyplýva
$$
1>\root3\of{xyz},\qquad\text{takže aj}\qquad 1>xyz
\tag2
$$
a~tiež
$$
xy+yz+zx\ge3(xyz)^{\frac23}.
$$
S~využitím týchto odhadov dostávame
$$
\aligned
(3-x)(3-y)(3-z)&=9(3-x-y-z)+3(xy+yz+zx)-xyz>\\
               &>9(xyz)^{\frac23}-xyz=(xyz)^{\frac23}\left(9-\root3\of{xyz}\right)>8(xyz)^{\frac23}
\endaligned
$$
a~po vynásobení s~\thetag1 vzhľadom na \thetag2 platí
$$
(9-x^2)(9-y^2)(9-z^2)>512(xyz)^{\frac{11}{12}}>512xyz.
$$
Z~toho už triviálne vyplýva dokazovaná nerovnosť.

\smallskip
Vráťme sa k~pôvodne zadanému tvrdeniu. Označme
$$
x=\sqrt{9+16a^2}-4a,\qquad y=\sqrt{9+16b^2}-4b,\qquad z=\sqrt{9+16c^2}-4c.
$$
Zrejme $x,y,z>0$. Navyše
$$
9-x^2=9-(9+16a^2)-16a^2+8a\sqrt{9+16a^2}=8a\left(\sqrt{9+16a^2}-4a\right)=8ax
$$
a~podobne $9-y^2=8by$, $9-z^2=8cz$. Preto
$$
\frac{9-x^2}{x}\cdot\frac{9-y^2}{y}\cdot\frac{9-z^2}{z}=512abc=512,
$$
a~aby sme nedostali spor s~pomocným tvrdením, nutne musí byť $x+y+z\ge3$. To je však ekvivalentné so zadanou nerovnosťou.
}

{%%%%%   MEMO, priklad t3
Označme $\mm A$ množinu slov dĺžky~$n$ s~párnym počtom blokov $ME$ aj $MO$ a~$\mm B$ množinu slov dĺžky~$n$ s~nepárnym počtom oboch typov blokov. Uvažujme ľubovoľné slovo z~$\mm B$. Keď v~tomto slove nájdeme prvý blok $ME$ alebo $MO$ (ten, ktorý sa vyskytne skôr) a~nahradíme ho druhým blokom, čiže zmeníme $E$ na $O$ alebo naopak, v~slove sa počet blokov jedného typu o~jedna zmenší a~počet blokov druhého typu o~jedna zväčší. Keďže pred zmenou boli oba počty nepárne, po zmene budú oba párne, teda výsledné slovo bude patriť do $\mm A$. Takúto operáciu vieme vykonať s~ľubovoľným slovom z~$\mm B$, pretože každé také slovo obsahuje nepárny, \tj. nenulový počet blokov oboch typov.

Pre každé slovo z~$\mm A$, ktoré vzniklo uvedenou operáciou, vieme spätne identifikovať prvý blok $ME$ alebo $MO$ a~zmenou $E$ na $O$ alebo naopak dostaneme pôvodné slovo z~$\mm B$. To znamená, že každé slovo z~$\mm B$ sa operáciou zmení na iné slovo z~$\mm A$, z~čoho okamžite vyplýva $|\mm A|\ge|\mm B|$. Na zdôvodnenie toho, že v~skutočnosti dokonca $|\mm A|>|\mm B|$, si stačí uvedomiť, že v~$\mm A$ existujú slová, ktoré nedostaneme opísanou operáciou zo žiadneho slova z~$\mm B$. Takými slovami sú zrejme tie, ktoré neobsahujú žiadny blok $ME$ ani $MO$ (podľa zadania takéto slová patria do $\mm A$); ich počet je pre každé $n$ nenulový.
}

{%%%%%   MEMO, priklad t4
Nech $\pi_0$ je ľubovoľná permutácia. Uvažujme ďalších $p-1$ permutácií $\pi_1$, $\pi_2$, \dots, $\pi_{p-1}$, ktoré dostaneme z~$\pi_0$ postupným zväčšovaním zložiek o~$1$, pričom číslo~$p$ namiesto zväčšenia na $p+1$ nahradíme číslom $1$. Platí teda $\pi_j(i)\equiv\pi_0(i)+j\pmod p$.

Spočítame, aká je priemerná hodnota $f(\pi)$, ak uvažujeme len permutácie $\pi_0$, \dots, $\pi_{p-1}$. Na to stačí určiť, aký je počet násobkov $p$ medzi všetkými číslami v~zozname
$$
\alignedat3
   &\pi_0(1),&\quad &\pi_0(1)+\pi_0(2),&\quad \dots,\quad &\pi_0(1)+\pi_0(2)+\dots+\pi_0(p),\\
   &\pi_1(1),&\quad &\pi_1(1)+\pi_1(2),&\quad \dots,\quad &\pi_1(1)+\pi_1(2)+\dots+\pi_1(p),\\
   &&&&\quad \vdots\phantom{,}\quad&\\
   &\pi_{p-1}(1),&\quad &\pi_{p-1}(1)+\pi_{p-1}(2),&\quad \dots,\quad &\pi_{p-1}(1)+\pi_{p-1}(2)+\dots+\pi_{p-1}(p)\\
\endalignedat
\tag1
$$
a~výsledok vydeliť počtom riadkov, teda číslom~$p$. Pre čísla v~$k$-tom stĺpci zoznamu~\thetag1 platí
$$
\align
\pi_j(1)+\pi_j(2)+\cdots+\pi_j(k)&\equiv(\pi_0(1)+j)+(\pi_0(2)+j)+\cdots+(\pi_0(k)+j)\equiv\\
&\equiv\pi_0(1)+\pi_0(2)+\cdots+\pi_0(k)+kj\pmod p.
\endalign
$$
Ak teda označíme $\pi_0(1)+\pi_0(2)+\cdots+\pi_0(k)=s_k$, dávajú čísla v~$k$-tom stĺpci zoznamu~\thetag1 po delení $p$ rovnaké zvyšky ako čísla
$$
s_k,\quad s_k+k,\quad s_k+2k,\quad \dots,\quad s_k+(p-1)k.
\tag2
$$

Ak $k<p$, sú čísla $k$ a~$p$ nesúdeliteľné a~zoznam~\thetag2 neobsahuje žiadne dve čísla s~rovnakým zvyškom po delení~$p$, pretože ak $j\ne j'$, tak
$$
(s_k+jk)-(s_k+j'k)=k(j-j')\nequiv0\pmod p.
$$
V~zozname \thetag2 po delení $p$ sa potom každý zvyšok objaví práve raz, čiže práve jedno číslo je v~ňom násobkom~$p$.

Ak $k=p$, tak $s_k=s_p=\pi_0(1)+\pi_0(2)+\cdots+\pi_0(p)=1+2+\cdots+p=\frac12p(p+1)$, čo je pre $p>2$ násobkom~$p$. V~zozname \thetag2 (resp. v~poslednom stĺpci zoznamu \thetag1, ktorý obsahuje $p$ totožných čísel $s_p$) sú teda všetky čísla násobkom $p$.

Spolu je počet násobkov prvočísla~$p$ v~zozname \thetag1 rovný $(p-1)\cdot1+p={2p-1}$ a~priemerná hodnota $f(\pi)$ je rovná $(2p-1)/p$. Táto hodnota nie je závislá na zvolenej permutácii~$\pi_0$. Pritom množinu všetkých permutácií množiny~$\mm S$ vieme rozložiť na $p$\hbox{-}prv\-ko\-vé triedy tak, že v~každej triede sa budú permutácie líšiť iba o~posunutie zložiek modulo $p$ (rovnako ako permutácie $\pi_0$, \dots, $\pi_{p-1}$ opísané v~úvode). Keďže pre každú takú triedu vychádza priemerná hodnota $f(\pi)$ rovná $(2p-1)/p$, je to zároveň priemerná hodnota pre množinu všetkých permutácií množiny~$\mm S$.
}

{%%%%%   MEMO, priklad t5
Označme $P$ priesečník osi strany~$AB$ s~kolmicou na stranu~$AC$ vedenou bodom~$L$.

Uvažujme najskôr prípad, že $P$ leží vnútri polroviny $ABC$ (\obr{}a). Body $K$, $L$ ležia na Tálesovej kružnici s~priemerom~$AP$. Z~vlastností obvodových uhlov nad tetivou~$AK$ tejto kružnice potom vyplýva $|\uhol ALK|=|\uhol APK|$. Podľa zadania
$$
|\uhol ALK|=180\st-|\uhol CLK|=180\st-|\uhol KMC|=|\uhol BMK|
$$
a~zo súmernosti podľa osi~$KP$ máme $|\uhol APK|=|\uhol BPK|$. Spolu dostávame $|\uhol BMK|=|\uhol BPK|$, teda body $K$, $B$, $M$, $P$ ležia na jednej kružnici. Keďže $PK\perp KB$, je $BP$ priemerom tejto kružnice a~odtiaľ $PM\perp BM$, teda kolmice zo zadania sa pretínajú v~bode~$P$.
\inspinspab{memo.2}{memo.3}%

Prípad, keď $P$ leží v~polrovine opačnej k~polrovine $ABC$ (\obrr1b), vyšetríme analogicky. Z~Tálesovej kružnice nad priemerom $AP$ vyplýva $|\uhol ALK|=180\st-|\uhol APK|$, takže postupom z~predošlého prípadu odvodíme vzťah $|\uhol BMK|=180\st-|\uhol BPK|$, z~čoho vzhľadom na polohu bodov $P$, $M$ v~rôznych polrovinách určených priamkou~$BK$ vyplýva, že $K$, $B$, $M$, $P$ ležia na jednej kružnici. Záver je rovnaký ako v~prvom prípade.

Špeciálny prípad, keď $P=K$, vedie k~záveru triviálne -- vtedy $|\uhol BMK|=|\uhol ALK|=90\st$, teda zadané kolmice sa pretínajú v~bode~$K$.
}

{%%%%%   MEMO, priklad t6
Predpokladajme, že označenie bodov $E$, $F$, $G$, $H$ je zvolené tak ako na \obr. Priesečník priamok $AB$ a~$CD$ označme~$U$ a~priesečník priamok $AD$ a~$BC$ označme~$V$.

Bod $H$ je priesečníkom osí uhlov $DAB$ a~$ADC$, preto je -- v~závislosti od toho, či $U$ leží na polpriamke $AB$ alebo $BA$ -- buď stredom kružnice vpísanej trojuholníku $ADU$ alebo stredom kružnice pripísanej k~strane $AD$ trojuholníka $ADU$. V~oboch prípadoch leží na osi uhla $AUD$. Analogicky bod~$F$ leží na osi uhla $BUC$, ktorý je však totožný s~uhlom $AUD$. Uhlopriečka~$HF$ štvoruholníka $EFGH$ je teda osou uhla $AUD$. Rovnakou úvahou odvodíme, že $EG$ je osou uhla $AVB$. Takže $K$ leží na priesečníku osí uhlov $AUD$ a~$AVB$.
\insp{memo.4}%

Označme $|\uhol BAD|=\alpha$ a~$|\uhol ABC|=|\uhol CDA|=\beta$. Ak $\alpha+\beta<180\st$, ležia body $U$, $V$ v~polrovine $BDC$, ak $\alpha+\beta>180$, ležia v~polrovine $BDA$ (prípad $\alpha+\beta=180\st$ vzhľadom na rôznobežnosť strán $AB$, $CD$ nastať nemôže). V~oboch prípadoch ležia body $B$, $D$ v~tej istej polrovine určenej priamkou $UV$, a~keďže $|\uhol UBV|=|\uhol UDV|=180\st-\beta$, ležia body $U$, $B$, $D$, $V$ na jednej kružnici, ktorú označme~$k$. Uhly $DVB$, $DUB$ nad tetivou $DB$ kružnice~$k$ majú rovnakú veľkosť\footnote{Rovnosť veľkostí uhlov $DVB$, $DUB$ možno odvodiť aj bez kružnice $k$ z~podobných trojuholníkov $ADU$, $ABV$ -- každý z~nich má dva uhly s~veľkosťami $\alpha$, $\beta$ (resp. $180\st-\alpha$, $180\st-\beta$, ak $U$, $V$ ležia v~polrovine $BDA$), čiže sú podobné podľa vety {\it uu}.} a~preto majú rovnakú veľkosť aj príslušné polovičné uhly, \tj. $|\uhol KVB|=|\uhol KUB|$. Z~toho vyplýva, že body $U$, $B$, $K$, $V$ ležia na jednej kružnici (zrejme $K$ a~$B$ ležia v~tej istej polrovine určenej priamkou~$UV$). Táto kružnica je totožná s~kružnicou $k$, pretože s~ňou má spoločné body $U$, $B$, $V$. Na $k$ teda leží všetkých päť bodov $U$, $B$, $K$, $D$, $V$, z~čoho už triviálne dostaneme dokazované tvrdenie.
}

{%%%%%   MEMO, priklad t7
Predpokladajme, že $x$, $y$, $z$ vyhovujú zadaniu. Rozoberieme dva prípady podľa parity čísla~$x$.

\smallskip
Ak $x$ je nepárne, tak ľavá strana druhej rovnice je nepárna, takže $y$ je nepárne. Ľavá strana prvej rovnice je potom párna a~teda $z$ je párne.
Z~druhej rovnice zjavne $y\ne1$, čiže $y\ge2$ a~pravá strana prvej rovnice je deliteľná štyrmi. Ak celú sústavu prepíšeme v~zvyškoch po delení štyrmi, nakoľko aj $4\mid 2012$, dostaneme
$$
x^y+y^x\equiv 0\pmod4,\qquad x^y\equiv y^{z+1}\pmod4.
$$
Dosadením za $x^y$ z~druhej kongruencie do prvej máme $y^{z+1}+y^x\equiv0\pmod4$. Tomu však nevyhovuje žiadne nepárne $y$, pretože ak $y\equiv1\pmod4$, tak
$$
y^{z+1}+y^x\equiv1^{z+1}+1^x\equiv2\pmod4
$$
a~ak $y\equiv3\pmod4$, tak vzhľadom na nepárnosť $x$ aj $z+1$ platí
$$
y^{z+1}+y^x\equiv3^{z+1}+3^x\equiv(\m1)^{z+1}+(\m1)^x\equiv\m2\equiv2\pmod4.
$$

\smallskip
Ak $x$ je párne, je ľavá strana druhej rovnice párna, takže $y$ je párne. Z~prvej rovnice potom aj $z$ je párne, \tj. $z\ge2$. Pravá strana druhej rovnice je teda deliteľná ôsmimi. Ak by bolo $y>2$, bolo by aj $x^y$ deliteľné ôsmimi, a~keďže $8\nmid2012$, druhá rovnica by nemohla byť splnená. Nutne teda $y=2$ a~sústava sa redukuje na tvar
$$
x^2+2^x=z^2,\qquad x^2+2012=2^{z+1}.
\tag1
$$
Podľa prvej z~týchto rovníc $z^2>x^2$, teda $z>x$, takže $2^{z+1}>2^{x+1}$, z~čoho podľa druhej z~rovníc $x^2+2012>2^{x+1}$. Ľahko sa dosadením presvedčíme, že pre $x=12$ táto nerovnosť neplatí a~matematickou indukciou dokážeme, že neplatí ani pre žiadne $x>12$: Ak totiž $x$ zväčšíme o~$1$, hodnota $2^{x+1}$ sa zväčší 2-krát, zatiaľ čo hodnota $x^2+2012$ sa zväčší iba $q$-krát, pričom
$$
q=\frac{(x+1)^2+2012}{x^2+2012}=1+\frac{2x+1}{x^2+2012}=1+\frac{2+\frac1x}{x+\frac{2012}x}<1+\frac3x<2.
$$
Ostáva prípad $x<12$, \tj. $x\in\{2,4,6,8,10\}$. Pre tieto hodnoty je číslo $z=\sqrt{x^2+2^x}$ celým jedine ak $x=6$, vtedy $z=10$. Ľahko sa presvedčíme, že potom je splnená aj druhá rovnica v~\thetag1.

\odpoved
Jedinou vyhovujúcou trojicou je $(6,2,10)$.
}

{%%%%%   MEMO, priklad t8
Čísla s~vlastnosťami zo zadania existujú. Okrem príkladov takých dvojíc uvedieme aj postup, ako ich možno objaviť.

Pripomeňme známy vzorec, že ak číslo $n$ má rozklad na súčin prvočísel $n=p_1^{\alpha_1}p_2^{\alpha_2}\dots p_k^{\alpha_k}$, tak $\tau(n)=(\alpha_1+1)(\alpha_2+1)\dots(\alpha_k+1)$.\footnote{Vzorec možno odvodiť takto: každý deliteľ čísla~$n$ má tvar $p_1^{a_1}p_2^{a_2}\dots p_k^{a_k}$, pričom $0\le a_i\le\alpha_i$, každé $a_i$ teda môže nadobúdať $(\alpha_i+1)$ rôznych hodnôt, celkovo potom exponenty $a_1$, \dots, $a_k$ možno kombinovať $(\alpha_1+1)(\alpha_2+1)\dots(\alpha_k+1)$ spôsobmi.} Podľa tohto vzorca máme tiež $\tau(n^2)=(2\alpha_1+1)\dots(2\alpha_k+1)$ a~$\tau(n^3)=(3\alpha_1+1)\dots(3\alpha_k+1)$. Pre skrátenie zápisu označme $\alpha_i+1=\beta_i$. Potom
$$
\tau(n)=\beta_1\dots \beta_k,\quad
\tau(n^2)=(2\beta_1-1)\dots(2\beta_k-1),\quad
\tau(n^3)=(3\beta_1-2)\dots(3\beta_k-2).
$$

Uvažujme dvojice
$$
(2,3),\quad(3,5),\quad(4,7),\quad(8,15),\quad(18,35),\quad(32,63).
\tag1
$$
Všetky sú typu $(m,2m-1)$ pre vhodné $m$ a~rozklad na súčin prvočísel každého z~dvanástich čísel v~uvedených dvojiciach obsahuje len prvočísla z~množiny $\{2,3,5,7\}$. Čísla $a$, $b$ budeme hľadať v~takom tvare, aby exponenty z~ich rozkladov zväčšené o~$1$, \tj. príslušné hodnoty $\beta_i$, boli spomedzi čísel nachádzajúcich sa na prvých zložkách dvojíc v~\thetag1; príslušné hodnoty $2\beta_i-1$ potom budú na druhých zložkách. Rovnosti $\tau(a)=\tau(b)$ a~$\tau(a^2)=\tau(b^2)$ prepíšme na tvar $\tau(a)/\tau(b)=1$ a~$\tau(a^2)/\tau(b^2)=1$. Snažíme sa teda nájsť celé čísla $u$, $v$, $w$, $x$, $y$, $z$ také, že
$$
2^u\cdot3^v\cdot4^w\cdot8^x\cdot18^y\cdot32^z=1\qquad\text{a}\qquad 3^u\cdot5^v\cdot7^w\cdot15^x\cdot35^y\cdot63^z=1.
\tag2
$$
Kladné čísla zo šestice $(u,v,w,x,y,z)$ spĺňajúcej \thetag2 použijeme na vytvorenie čísla~$a$, záporné na vytvorenie čísla~$b$.

Rovnosti \thetag2 upravíme na tvar
$$
2^{u+2w+3x+y+5z}\cdot3^{v+2y}=1,\qquad 3^{u+x+2z}\cdot5^{v+x+y}\cdot7^{w+y+z}=1,
$$
čiže
$$
\align
u+2w+3x+y+5z&=0,\\
v+2y&=0,\\
u+x+2z&=0,\\
v+x+y&=0,\\
w+y+z&=0.\\
\endalign
$$
Táto sústava piatich lineárnych rovníc o~šiestich neznámych má okrem triviálneho riešenia $u=v=w=x=y=z=0$ (ktoré na vytvorenie vyhovujúcich $a$, $b$ použiť nedokážeme) nekonečne veľa ďalších riešení, ktoré možno ľahko nájsť postupnou elimináciou neznámych. Všeobecné riešenie sústavy je $(u,v,w,x,y,z)=(t,\m2t,0,t,t,\m t)$, kde $t$ je reálny parameter. Pre účely nášho postupu stačí zobrať nejaké celočíselné riešenie, napr.
$(u,v,w,x,y,z)=(1,\m2,0,1,1,\m1)$. Podľa \thetag1 následne na základe kladných hodnôt $u=x=y=1$  zvolíme $\beta_1=2$, $\beta_2=8$, $\beta_3=18$ a~zo záporných hodnôt $v=\m2$, $z=\m1$ určíme $\beta_1'=\beta_2'=3$, $\beta_3'=32$. Vzhľadom na postup platí
$$
\beta_1\beta_2\beta_3=\beta_1'\beta_2'\beta_3',\quad
(2\beta_1-1)(2\beta_2-1)(2\beta_3-1)=(2\beta_1'-1)(2\beta_2'-1)(2\beta_3'-1)
\tag3
$$
a~ľahko možno nahliadnuť, že $(3\beta_1-2)(3\beta_2-2)(3\beta_3-2)\ne(3\beta_1'-2)(3\beta_2'-2)(3\beta_3'-2)$. Číslami vyhovujúcimi zadaniu sú teda napr.
$$
a=2^1\cdot3^7\cdot5^{17},\qquad b=2^2\cdot3^2\cdot5^{31}
$$
(trojice prvočísel v~základoch môžeme samozrejme voliť ľubovoľne).

\ineriesenie
Podobne ako v~prvom postupe budeme hľadať $\beta_i$, $\beta_i'$ spĺňajúce \thetag3. Pokúsime sa ich nájsť v~tvare
$$
\beta_1=pq,\quad
\beta_2=r,\quad
\beta_3=s,\qquad
\beta_1'=p,\quad
\beta_2'=q,\quad
\beta_3'=rs
$$
pre vhodné prirodzené čísla $p$, $q$, $r$, $s$. Prvá rovnosť z~\thetag3 je pri takomto vyjadrení splnená triviálne, pre splnenie druhej rovnosti musí platiť
$$
\frac{(2p-1)(2q-1)}{2pq-1}=\frac{(2r-1)(2s-1)}{2rs-1}.
$$
Označme $V(m,n)=(2m-1)(2n-1)/(2mn-1)$. Naším cieľom je nájsť dve rôzne dvojice $\{m,n\}$, pre ktoré tento výraz nadobúda rovnakú hodnotu. Úpravou dostávame
$$
V(m,n)=\frac{4mn-2m-2n+1}{2mn-1}=2-\frac{2m+2n-3}{2mn-1}.
$$
Hľadajme také $m$, $n$, že $V(m,n)=2-1/k$ pre nejaké prirodzené číslo~$k$. Po úprave získame ekvivalentné vyjadrenie
$$
(m-k)(n-k)=\frac12(2k-1)(k-1).
$$
Dosadením povedzme $k=5$ (na pravej strane predošlej rovnosti vtedy bude celé číslo a~zároveň nie prvočíslo) dostaneme
$$
(m-5)(n-5)=18.
$$
Riešením tejto rovnice sú napríklad dvojice $(6,23)$, $(7,14)$, stačí teda zvoliť $p=6$, $q=23$, $r=7$, $s=14$, z~čoho dostávame (po presvedčení sa, že $\tau(a^3)\ne \tau(b^3)$) vyhovujúcu dvojicu
$$
a=2^{6\cdot23-1}\cdot 3^{7-1}\cdot5^{14-1}=2^{137}\cdot 3^6\cdot5^{13},\qquad
b=2^{6-1}\cdot 3^{23-1}\cdot5^{7\cdot14-1}=2^5\cdot 3^{22}\cdot5^{97}.
$$
} 