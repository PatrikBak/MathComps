{%%%%%   Z4-I-1
...}
\podpis{...}

{%%%%%   Z4-I-2
...}
\podpis{...}

{%%%%%   Z4-I-3
...}
\podpis{...}

{%%%%%   Z4-I-4
...}
\podpis{...}

{%%%%%   Z4-I-5
...}
\podpis{...}

{%%%%%   Z4-I-6
...}
\podpis{...}

{%%%%%   Z5-I-1
Traja kamaráti Pankrác, Servác a~Bonifác išli cez prázdniny na nočnú prechádzku
prírodným labyrintom. Pri vstupe dostal každý sviečku a~vydal sa iným smerom
ako zvyšní dvaja. Všetci traja labyrintom úspešne prešli, ale každý išiel inou cestou.
V~štvorcovej sieti na \obr{} sú vyznačené ich cesty. Vieme, že Pankrác
nikdy nešiel na juh a~že Servác nikdy nešiel na západ. Koľko metrov prešiel
v~labyrinte Bonifác, keď vieme, že Pankrác prešiel presne 500\,m?
\insp{z61.1}%
}
\podpis{Michaela Petrová}

{%%%%%   Z5-I-2
Do každého nevyplneného štvorčeka na \obr{} doplňte číslo $1$, $2$ alebo $3$ tak, aby v~každom stĺpci a~riadku bolo každé z~týchto čísel práve raz a~aby boli splnené dodatočné požiadavky v~každej vyznačenej oblasti.
\insp{z61.2}%

(Ak vo vyznačenej oblasti požadujeme určitý podiel, máme na mysli podiel, ktorý získame vydelením väčšieho čísla menším. Podobne pracujeme aj s~rozdielom.)
}
\podpis{Svetlana Bednářová}

{%%%%%   Z5-I-3
Julka pripravuje pre svoje kamarátky občerstvenie -- chlebíčky. Natrie ich
zemiakovým šalátom a~navrch chce dať ešte prísady: šunku, tvrdý syr, plátok vajíčka a~prúžok nakladanej papriky. Nechce však, aby niektoré dva chlebíčky obsahovali
úplne rovnakú kombináciu prísad. Aký najväčší počet navzájom rôznych chlebíčkov
môže nachystať, ak žiadny z~nich nemá mať všetky štyri prísady a~žiadny z~nich
nie je iba so šalátom (t. j. bez ďalších prísad)?
}
\podpis{Michaela Petrová}

{%%%%%   Z5-I-4
Na \obr{} je nakreslená stavba zlepená z~rovnako veľkých kociek. Stavba je vlastne
veľká kocka s~tromi rovnými tunelmi, ktorými sa dá pozerať skrz a~ktoré majú všade
rovnaký prierez. Z~koľkých kociek je stavba zlepená?
\insp{z61.3}%
}
\podpis{Marie Krejčová}

{%%%%%   Z5-I-5
V~rozprávke o~siedmich zhavranených bratoch bolo sedem bratov, z~ktorých každý sa
narodil presne rok a~pol po predchádzajúcom bratovi. Keď bol najstarší z~bratov práve
štyrikrát starší ako najmladší, matka všetkých bratov zakliala. Koľko rokov
mali jednotliví bratia, keď ich matka zakliala?}
\podpis{Marta Volfová}

{%%%%%   Z5-I-6
Janka a~Hanka sa rady hrajú s~modelmi zvieratiek. Hanka pre svoje kravičky
postavila z~uzáverov z~PET fliaš obdĺžnikovú ohradu ako na \obr{}.
Janka zo všetkých svojich uzáverov zložila pre ovečky ohradu tvaru rovnostranného trojuholníka. Potom ju rozobrala a~postavila pre ne štvorcovú ohradu, takisto zo všetkých svojich uzáverov. Koľko mohla mať Janka uzáverov? Nájdite aspoň dve riešenia.
\insp{z61.4}%
}
\podpis{Marta Volfová}

{%%%%%   Z6-I-1
Mirka čistila fazuľu do polievky na papieri, ktorý vytiahla zo stolíka svojej sestry.
Na papieri boli nakreslené tri rovnako veľké kruhy, v~ktorých spoločné časti boli vyfarbené sivou pastelkou. Do bielych častí jednotlivých kruhov umiestnila toľko fazuliek, koľko je napísané na \obr{}. Koľko fazuliek má umiestniť do sivých častí, aby bol v~každom kruhu rovnaký počet fazuliek?
\insp{z61.5}
}
\podpis{Libor Šimůnek}

{%%%%%   Z6-I-2
Do hračkárstva priviezli nové plyšové zvieratká: vážky, pštrosy a~kraby.
Každá vážka má 6~nôh a~4~krídla, každý pštros má 2~nohy a~2~krídla a~každý
krab má 8~nôh a~2~klepetá.
Dohromady majú tieto privezené hračky 118 nôh, 22~krídiel a~22~klepiet.
Koľko majú dohromady hláv?}
\podpis{Michaela Petrová}

{%%%%%   Z6-I-3
Na \obr{} je stavba zlepená z~rovnakých kociek. Stavba je vlastne
veľká kocka s~tromi rovnými tunelmi, ktorými sa dá pozerať skrz a~ktoré majú všade
rovnaký prierez. Túto stavbu sme celú ponorili do farby.
Koľko kociek, z ktorých je kocka zložená, má zafarbenú aspoň jednu stenu?
\insp{z61.3}%
}
\podpis{Marie Krejčová}

{%%%%%   Z6-I-4
Do každého nevyplneného štvorčeka na \obr{} doplňte číslo $1$, $2$, $3$ alebo $4$ tak, aby v~každom stĺpci a~riadku bolo každé z~týchto čísel práve raz a~aby boli splnené dodatočné požiadavky v~každej vyznačenej oblasti.
\insp{z61.6}%

(Ak vo vyznačenej oblasti požadujeme určitý podiel, máme na mysli podiel, ktorý získame vydelením väčšieho čísla menším. Podobne pracujeme aj s~rozdielom.)
}
\podpis{Svetlana Bednářová}

{%%%%%   Z6-I-5
Ondro, Maťo a~Kubo dostali na Vianoce od prarodičov každý jednu z~nasledujúcich hračiek:
veľké hasičské auto, %%(s~dvojítým húkáním a  výsuvnými rebríkmi),
vrtuľník na diaľkové ovládanie a~stavebnicu Merkur.
Bratranec Peťo doma hovoril:

"Ondro dostal to veľké hasičské auto.
Želal si ho síce Kubo, ale ten ho nedostal.
Maťo nemá v~obľube stavebnice, takže Merkur
nebol pre neho."

Ukázalo sa, že Peťo sa dvakrát mýlil v~informácii, kto
dostal či nedostal daný darček a~len raz povedal pravdu.
Ako to teda s~darčekmi bolo a~kto teda dostal aký darček?}
\podpis{Marta Volfová}

{%%%%%   Z6-I-6
Marta, Libuška a~Mária si vymysleli hru, ktorú by chceli hrať na
obdĺžnikovom ihrisku zloženom z~18~rovnakých štvorcov ako na \obr.
Na hru potrebujú ihrisko rozdeliť dvoma rovnými čiarami na tri rovnako veľké časti.
Navyše tieto čiary musia obe prechádzať tým rohom ihriska, ktorý je na obrázku vľavo dole. Poraďte dievčatám, ako majú dokresliť čiary, aby sa mohli začať hrať.
\insp{z61.7}%
}
\podpis{Erika Novotná}

{%%%%%   Z7-I-1
Trpaslíci chodia na vodu k~potoku. Džbánik každého z~trpaslíkov je inak veľký: majú objemy 3, 4, 5, 6, 7, 8 a~9 litrov. Trpaslíci si džbániky
medzi sebou nepožičiavajú a~vždy ich prinesú úplne plné vody.
\begin{itemize}
  \item Kýchal prinesie vo~svojom džbániku viac vody ako Smejko.
  \item Spachtoš by musel ísť po vodu trikrát, aby priniesol práve toľko vody, koľko Plaško v~jednom svojom džbániku.
  \item Vedkov džbánik je len o~dva litre väčší ako Smejkov.
  \item Sám Kýblik prinesie toľko vody, koľko Spachtoš a Smejko dokopy.
  \item Keď idú po vodu Vedko a~Kýblik, prinesú rovnako vody ako Dudroš, Kýchal a~Smejko dokopy.
\end{itemize}
    Koľko vody prinesú Kýchal a~Kýblik dohromady?
%Kejchal = Kýchal, Dřímal = Spachtoš, Stydlín= Plaško, Prófa = Vedko, Štístko = Smejko,
%Šmudla = Kýblik, Rejpal=Dudroš
}
\podpis{Michaela Petrová}

{%%%%%   Z7-I-2
Na \obr{} je štvorec $ABCD$, v~ktorom sú umiestnené štyri zhodné rovnoramenné trojuholníky $ABE$, $BCF$, $CDG$ a~$DAH$, všetky vyfarbené sivou. Strany štvorca $ABCD$ sú základňami týchto rovnoramenných trojuholníkov. Vieme, že sivé plochy štvorca $ABCD$ majú dokopy rovnaký obsah ako biela plocha štvorca. Ďalej vieme, že $|HF|= 12\cm$.
Určte dĺžku strany štvorca $ABCD$.
\insp{z61.8}%
}
\podpis{Libor Šimůnek}

{%%%%%   Z7-I-3
Sedem bezprostredne po sebe idúcich celých čísel stálo v~rade, zoradené od najmenšieho po najväčšie. Po chvíli sa čísla začali nudiť, a~tak sa najskôr vymenilo prvé s~posledným. Potom sa druhé najväčšie posunulo úplne na začiatok radu a nakoniec sa najväčšie z~čísel postavilo do stredu. Na svoju veľkú spokojnosť sa tak ocitlo vedľa čísla, ktoré bolo presne jeho polovicou. Ktorých sedem čísel mohlo stáť pôvodne v~rade?
%%Sedem bezprostredne po sebe idúcich celých čísel stálo v~rade, zoradené od najmenšieho po najväčšie. Po chvíli sa čísla začali nudiť, a~tak sa najskôr vymenilo prvé s~posledným. Potom sa prostredné posunulo úplne na začiatok radu a nakoniec sa najväčšie z~čísel postavilo do stredu. Na svoju veľkú spokojnosť sa tak ocitlo vedľa čísla, ktoré sa od neho líšilo iba znamienkom. Ktorých sedem čísel mohlo stáť na začiatku v~rade?
}
\podpis{Svetlana Bednářová}

{%%%%%   Z7-I-4
Učiteľka Smoliarska pripravovala previerku pre svoju triedu v~troch verziách, aby žiaci nemohli odpisovať. V~každej verzii zadala tri hrany kvádra v~centimetroch a~úlohou bolo vypočítať jeho objem. Úlohy si ale dopredu nepreriešila, a tak netušila, že výsledok je vo všetkých troch verziách rovnaký. Do zadaní žiakom napísala tieto dĺžky hrán: $12$, $18$, $20$, $24$, $30$, $33$ a~$70$. Z~deviatich celočíselných dĺžok hrán, ktoré učiteľka Smoliarska zadala, sme vám teda prezradili iba sedem a~ani sme vám neprezradili, ktoré dĺžky patria do toho istého zadania. Podarí sa vám napriek tomu určiť zostávajúce dve dĺžky hrán?}
\podpis{Libor Šimůnek}

{%%%%%   Z7-I-5
Jeden vnútorný uhol trojuholníka má $50\st$.
Aký veľký uhol zvierajú osi dvoch zostávajúcich vnútorných uhlov trojuholníka?}
\podpis{Libuše Hozová}

{%%%%%   Z7-I-6
Hľadáme šesťciferný číselný kód, o~ktorom vieme, že:
\begin{itemize}
  \item žiadna cifra v~ňom nie je viackrát,
  \item obsahuje aj~$0$, tá však nie je na predposlednom mieste,
  \item vo svojom zápise nemá nikdy vedľa seba dve párne ani dve nepárne cifry,
  \item susedné cifry sa od seba líšia aspoň o~$3$,
  \item keď číslo rozdelíme na tri dvojčíslia, tak prvé aj druhé dvojčíslie sú obe násobkom tretieho, teda posledného dvojčíslia.
\end{itemize}
Určte hľadaný kód.}
\podpis{Marta Volfová}

{%%%%%   Z8-I-1
Korešpondenčná matematická súťaž prebieha v~troch kolách, ktorých náročnosť sa stupňuje.
Do druhého kola postupujú len tí riešitelia, ktorí boli úspešní v~prvom kole, do tretieho kola postupujú len úspešní riešitelia druhého kola.
Víťazom je každý, kto je úspešným riešiteľom posledného, teda tretieho kola. V~poslednom ročníku tejto súťaže bolo presne 14\,\% riešiteľov úspešných v~prvom kole, presne 25\,\% riešiteľov druhého kola postúpilo do tretieho kola a~presne
8\,\% riešiteľov tretieho kola zvíťazilo. Aký je najmenší počet súťažiacich, ktorí sa mohli zúčastniť prvého kola?
Koľko by bolo v~takomto prípade víťazov?}
\podpis{Michaela Petrová}

{%%%%%   Z8-I-2
Je daný rovnoramenný trojuholník $ABC$ so základňou~$AB$ dlhou $10\cm$
a~ramenami dlhými $20\cm$. Bod $S$~je stred základne~$AB$.
Rozdeľte trojuholník $ABC$ štyrmi priamkami prechádzajúcimi bodom~$S$ na päť
častí s~rovnakým obsahom.
Zistite, aké dlhé úseky vytnú tieto priamky na ramenách trojuholníka $ABC$.}
\podpis{Erika Novotná}

{%%%%%   Z8-I-3
Hľadáme päťciferné číslo s~nasledujúcimi vlastnosťami:
je to palindróm (t. j. číta sa odzadu rovnako ako odpredu),
je deliteľné dvanástimi a~vo svojom zápise obsahuje cifru~$2$ bezprostredne za cifrou~$4$.
Určte všetky možné čísla, ktoré vyhovujú zadaným podmienkam.}
\podpis{Martin Mach}

{%%%%%   Z8-I-4
Na stred hrnčiarskeho kruhu sme položili kocku, ktorá mala na každej svojej stene napísané jedno prirodzené číslo. Tesne predtým, ako sme kruh roztočili, sme z~miesta, kde stojíme, videli tri steny kocky a~teda len tri čísla. Ich súčet bol $42$. Po otočení hrnčiarskeho kruhu o~$90\st$ sme z~rovnakého miesta videli tri steny s~číslami, ktoré mali súčet $34$ a~po otočení o~ďalších  $90\st$ sme videli tri čísla so súčtom $53$.
\begin{enumerate}
  \item Určte súčet troch čísel, ktoré z~nášho miesta uvidíme, keď sa kruh otočí ešte o~ďalších $90\st$.
  \item Kocka celý čas ležala na stene s~číslom~$6$. Určte maximálny možný súčet všetkých šiestich čísel na kocke.
\end{enumerate}
}
\podpis{Libor Šimůnek}

{%%%%%   Z8-I-5
Pankrác, Servác a~Bonifác sú traja bratia, ktorí majú $P$, $S$ a~$B$ rokov.
Vieme, že $P$, $S$ a~$B$ sú prirodzené čísla menšie ako $16$,
pre ktoré platí:
$$
\align
P&=\frac52(B-S),\\
S&=2(B-P),\\
B&=8(S-P).
\endalign
$$
Určte vek všetkých troch bratov.}
\podpis{Libuše Hozová}

{%%%%%   Z8-I-6
Janka narysovala obdĺžnik s~obvodom $22\cm$ a~dĺžkami strán vyjadrenými v~centimetroch celými číslami. Potom obdĺžnik rozdelila bezo zvyšku na tri obdĺžniky, z~ktorých jeden mal rozmery $2\cm\times 6\cm$. Súčet obvodov všetkých troch obdĺžnikov bol o~$18\cm$ väčší ako obvod pôvodného obdĺžnika. Aké rozmery mohol mať pôvodný obdĺžnik? Nájdite všetky riešenia.}
\podpis{Monika Dillingerová}

{%%%%%   Z9-I-1
Pokladníčka v~galérii predáva návštevníkom vstupenky s~číslom podľa toho,
koľkí v~poradí v~ten deň prišli. Prvý návštevník dostane vstupenku s~číslom~$1$,
druhý s~číslom~$2$, atď. Počas dňa sa však minul žltý papier, na ktorý sa
vstupenky tlačili, preto musela pokladníčka pokračovať tlačením na červený papier. Za celý deň predala rovnako veľa žltých vstupeniek ako červených.
Večer zistila, že súčet čísel na žltých vstupenkách bol o~$1\,681$ menší ako
súčet čísel na červených vstupenkách.
Koľko vstupeniek v~ten deň predala?}
\podpis{Martin Mach}

{%%%%%   Z9-I-2
Filoména má mobil s~rozmiestnením tlačidiel ako na \obr.
\insp{z61.9}%

Deväťciferné telefónne číslo jej najlepšej kamarátky Kláry má tieto
vlastnosti:
\begin{itemize}
  \item všetky cifry Klárinho telefónneho čísla sú rôzne,
  \item prvé štyri cifry sú zoradené podľa veľkosti od najmenšej po najväčšiu a~stredy ich tlačidiel tvoria štvorec,
  \item stredy tlačidiel posledných štyroch cifier takisto tvoria štvorec,
  \item telefónne číslo je deliteľné tromi aj piatimi.
\end{itemize}
Koľko rôznych deväťciferných čísel by mohlo byť Kláriným telefónnym číslom?}
\podpis{Karel Pazourek}

{%%%%%   Z9-I-3
Alenka pozorovala veveričky na záhradke, kde rástli tieto tri stromy:
smrek, buk a~jedľa.
Veveričky sedeli pokojne na stromoch, takže ich mohla spočítať -- bolo ich
34.
Keď preskákalo 7 veveričiek zo~smreka na~buk, bolo ich na buku rovnako veľa ako na oboch ihličnanoch dokopy. Potom ešte preskákalo 5 veveričiek z~jedle na buk, a~vtedy bolo na jedli rovnako veľa veveričiek ako na smreku. Na buku ich bolo vtedy dvakrát viac, ako na jedli na úplnom začiatku. Koľko veveričiek pôvodne sedelo na každom strome?}
\podpis{Martin Mach}

{%%%%%   Z9-I-4
V~pravidelnom dvanásťuholníku $ABCDEFGHIJKL$ vpísanom do kružnice
s~polomerom $6\cm$ určte obvod päťuholníka $ACFHK$ (\obr).
\insp{z61.10}%
}
\podpis{Karel Pazourek}

{%%%%%   Z9-I-5
Pred vianočným koncertom ponúkali žiaci na predaj 60 výrobkov z~hodín výtvarnej
výchovy. Cenu si mohol každý zákazník určiť sám a~celý výťažok išiel na
dobročinné účely. Na začiatku koncertu žiaci spočítali, koľko centov v~priemere
utŕžili za jeden predaný výrobok, a~vyšlo im celé číslo. Keďže
ale nepredali všetkých 60 výrobkov, ponúkali ich aj po koncerte. Po koncerte si ľudia kúpili ešte sedem výrobkov, za ktoré dali dokopy 2\,505 centov. Tým sa priemerná
tržba za jeden predaný výrobok zvýšila na rovných 130 centov. Koľko výrobkov potom ostalo nepredaných?}
\podpis{Libor Šimůnek}

{%%%%%   Z9-I-6
V~obdĺžnikovej záhrade rastie broskyňa. Tento strom je od dvoch susedných rohov
záhrady vzdialený 5~metrov a~12~metrov a~vzdialenosť medzi spomínanými dvoma rohmi je
13~metrov. Ďalej vieme, že broskyňa stojí na uhlopriečke záhrady.
Aká veľká môže byť plocha záhrady?}
\podpis{Martin Mach}

{%%%%%   Z4-II-1
...}
\podpis{...}

{%%%%%   Z4-II-2
...}
\podpis{...}

{%%%%%   Z4-II-3
...}
\podpis{...}

{%%%%%   Z5-II-1
Vojto si kúpil 24 rovnako veľkých štvorcových dlaždíc. Strana každej dlaždice merala $40\cm$.
Zo všetkých dlaždíc postavil pred svoju
chatku obdĺžnikovú plošinu s~najmenším možným obvodom. Koľko
meral obvod tejto obdĺžnikovej plošiny? (Vojto dlaždice nerezal ani inak nelámal.)}
\podpis{Libuše Hozová}

{%%%%%   Z5-II-2
Lúpežník Rumcajs učil Cipúšika písať čísla. Ako prvé napísal Cipúšik číslo~$1$, potom
napísal číslo~$2$ a~potom pokračoval písaním ďalších bezprostredne po sebe idúcich
prirodzených čísel.
O~chvíľu to Cipúšika prestalo baviť a~prosil, aby prestali. Rumcajs sa nechal
prehovoriť a~písanie skončili. Ktoré číslo napísal Cipúšik ako posledné, ak sa cifra
nula nachádza v~napísaných číslach 35-krát?}
\podpis{Marie Krejčová}

{%%%%%   Z5-II-3
Na drôte sedí 9~lastovičiek, pričom vzdialenosť každých dvoch susedných lastovičiek je rovnaká. Vzdialenosť
prvej a~poslednej lastovičky je $720\cm$.
\begin{itemize}
  \iitem Aká je vzdialenosť dvoch susedných lastovičiek?
  \iitem Koľko by na drôte sedelo lastovičiek, keby si medzi každé dve susedné lastovičky, ktoré práve teraz sedia na drôte, sadli ďalšie tri?
\end{itemize}}
\podpis{Marie Krejčová}

{%%%%%   Z6-II-1
Danka a~Janka dostali na narodeniny dve rovnako veľké biele kocky. Každá kocka bola zlepená zo 125 malých kociek, tak ako vidíte na \ifobrazkyvedla{}obrázku\else\obr{}\fi{}. Aby kocky rozoznali, dohodli sa, že ich omaľujú. Danka vzala štetec a~tri celé steny svojej kocky omaľovala červenou farbou. Janka vzala štetec a~tri celé steny svojej kocky ofarbila zelenou farbou. Po čase sa dievčatá rozhodli, že každú z~kociek rozkrájajú na jednotlivé malé kocky. Keď to urobili, zistili, že počet kociek, ktoré majú aspoň jednu stenu červenú, je iný, ako počet kociek, ktoré majú aspoň jednu stenu zelenú. Zistite, aký je rozdiel týchto počtov.
\ifobrazkyvedla\else\insp{z61ii.61}\fi%
}
\podpis{Erika Novotná}

{%%%%%   Z6-II-2
Jurko zbiera podpisy známych športovcov a~spevákov. Na podpisy si kúpil zvláštny zošit. Rozhodol sa, že podpisy budú vždy na každom liste zošita iba na prednej strane. Tieto strany postupne očísloval číslami $1,3,5,7,9,\dots$ aby zistil, keby sa mu nejaký list zo zošita stratil. V~celom zošite tak napísal dokopy 125 cifier. Zistite:
\begin{itemize}
  \iitem Koľko mal Jurkov zošit listov.
  \iitem Koľko jednotiek napísal do zošita.
\end{itemize}}
\podpis{Marta Volfová}

{%%%%%   Z6-II-3
Na nočný pochod dostali družiny Vlkov a~Líšok po jednej rovnakej sviečke.
Sviečky zapálili spoločne na štarte a~vyrazili. Počas pochodu každý člen družiny Vlkov niesol sviečku takú dobu,
za ktorú sa jej dĺžka skrátila na polovicu. Vlci dobehli do cieľa v~momente,
keď mal šiesty člen družiny podať sviečku siedmemu. Od toho okamihu ich sviečka celá dohorela o~tri minúty. Líšky dobehli do cieľa za 2~hodiny a~57~minút.
\begin{itemize}
  \iitem Ktorá družina bola v~cieli ako prvá?
  \iitem O~koľko minút dobehla víťazná družina do cieľa skôr ako druhá družina?
\end{itemize}
(Sviečka horí rovnomerne: za rovnaký čas vždy rovnaký kúsok.)
}
\podpis{Monika Dillingerová}

{%%%%%   Z7-II-1
Peter a~Karol spolu hrali veľa partií dámy. Dohodli sa, že za výhru si hráč pripočíta 3~body, za prehru si 2~body odčíta a~za remízu sa žiadne body nepripočítavajú ani neodčitujú. Petrova sestra chcela vedieť, koľko už Peter a~Karol odohrali partií a~kto vedie, ale dozvedela sa iba, že Peter šesťkrát vyhral, dvakrát remizoval a~niekoľkokrát prehral a~Karol má práve 9~bodov. Zistite, koľko partií chlapci hrali a~kto práve teraz vedie.
}
\podpis{Marta Volfová}

{%%%%%   Z7-II-2
V~trojuholníku $ABC$ sa osi jeho vnútorných uhlov pretínajú v~bode~$S$. Uhol
$BSC$ má veľkosť $130\st$ a~uhol $ASC$ má $120\st$. Zistite veľkosti všetkých vnútorných uhlov trojuholníka $ABC$.}
\podpis{Eva Patáková}

{%%%%%   Z7-II-3
Trojciferné prirodzené číslo budeme nazývať {\it párnomilné}, ak obsahuje dve párne cifry a~cifru~$1$. Trojciferné prirodzené číslo budeme nazývať {\it nepárnomilné},
ak obsahuje dve nepárne cifry a~cifru~$2$. Zistite, koľko je všetkých trojciferných párnomilných a~koľko nepárnomilných čísel.}
\podpis{Erika Novotná}

{%%%%%   Z8-II-1
Divadelný súbor uviedol počas sezóny tridsaťkrát "Večer plný improvizácie". Filoména, obdivovateľka hlavného hrdinu, si na začiatku sezóny vypočítala, koľko by dokopy minula peňazí, keby chodila na každé predstavenie. Po niekoľkých predstaveniach však vstupné nečakane vzrástlo o~6€. Neskôr získal súbor sponzora a~túto novú cenu znížil o~8{,}50€. Na konci sezóny Filoména mohla povedať, že nevynechala ani jedno predstavenie a~na vstupné minula presne toľko, koľko si vypočítala na začiatku sezóny. Koľkokrát išla Filoména na predstavenie za vstupné v~pôvodnej výške?}
\podpis{Libor Šimůnek}

{%%%%%   Z8-II-2
Nájdite najmenšie také prirodzené číslo, že jeho polovica je deliteľná tromi, jeho tretina je deliteľná štyrmi, jeho štvrtina je deliteľná jedenástimi a~jeho polovica dáva po delení siedmimi zvyšok~$5$.}
\podpis{Eva Patáková}

{%%%%%   Z8-II-3
Je daný trojuholník $ABC$, ktorého náčrt vidíte na \ifobrazkyvedla{}obrázku\else\obr{}\fi{}. Na strane~$AB$ leží bod~$X$ a~na strane~$BC$ leží bod~$Y$ tak, že $CX$ je ťažnica, $AY$ je výška a~$XY$ je stredná priečka trojuholníka $ABC$. Vypočítajte obsah trojuholníka vyznačeného na obrázku sivou farbou, ak obsah trojuholníka $ABC$ je $24\cm^2$.
\ifobrazkyvedla\else\insp{z61ii.81}\fi%
}
\podpis{Monika Dillingerová}

{%%%%%   Z9-II-1
Daný je kosodĺžnik $ABCD$ ako na \ifobrazkyvedla{}obrázku\else\obr{}\fi{}.
Po strane~$AB$ sa pohybuje bod~$E$ a~po strane~$DC$ sa pohybuje bod~$G$ tak, že úsečka~$EG$ je stále rovnobežná s~$AD$.
Keď bol priesečník~$F$ úsečiek $EG$ a~$DB$ v~pätine uhlopriečky~$DB$ (bližšie k~bodu~$D$),
bol obsah vyfarbenej časti kosodĺžnika o~$1\cm^2$ väčší, ako keď
bol $F$ v~dvoch pätinách $DB$ (opäť bližšie k~$D$).
Určte obsah kosodĺžnika $ABCD$.
\ifobrazkyvedla\else\insp{z61ii.9}\fi%
}
\podpis{Eva Patáková}

{%%%%%   Z9-II-2
Snehulienka má na záhrade 101 sadrových trpaslíkov zoradených podľa
hmotnosti od najťažšieho po najľahšieho, pričom rozdiel hmotností každých dvoch susedných
trpaslíkov je rovnaký.
Raz Snehulienka trpaslíkov vážila a~zistila, že prvý (teda najťažší) trpaslík váži presne 5~kg.
Snehulienku však najviac prekvapilo, že keď na váhu postavila všetkých trpaslíkov od 76. po 80.,
vážili dokopy rovnako ako všetci trpaslíci od 96. po 101.
Koľko váži najľahší trpaslík?}
\podpis{Martin Mach}

{%%%%%   Z9-II-3
Turistický oddiel usporiadal trojdňový cyklistický výlet. Prvý deň chceli prejsť $\frac13$ celej plánovanej trasy, ale prešli o~4~km menej ako chceli.
Druhý deň chceli prejsť viac: polovicu zvyšku, ale prešli o~2~km menej ako chceli.
Tretí deň však všetko dobehli a~prešli $\frac{10}{11}$ zvyšku trasy a~ešte 4~km, takže dorazili do plánovaného cieľa.
Aká dlhá bola trasa a~koľko prešli prvý, druhý a~tretí deň?}
\podpis{Marta Volfová}

{%%%%%   Z9-II-4
Organizátor výstavy "Staviam, staviaš, staviame" rozdelil výstavu na dve časti: "Vývoj stavebných techník"
a~"Budovy z~iného uhla". Keďže ho zaujímala reakcia návštevníkov, vyplnil každý návštevník pri odchode
jednoduchý dotazník. Vyplynuli z~neho tieto zaujímavé skutočnosti:
\begin{itemize}
  \iitem medzi tými, ktorým sa páčila prvá časť, bolo 96\,\% takých, ktorým sa páčila aj druhá časť;
  \iitem medzi tými, ktorým sa páčila druhá časť, bolo 60\,\% takých, ktorým sa páčila aj prvá časť;
  \iitem medzi všetkými návštevníkmi bolo 59\,\% takých, ktorým sa nepáčila ani prvá ani druhá časť.
\end{itemize}

Koľko percent všetkých návštevníkov uviedlo, že sa im páčila prvá časť výstavy?}
\podpis{Michaela Petrová}

{%%%%%   Z9-III-1
Na \ifobrazkyvedla{}obrázku\else\obr{}\fi{} je obdĺžnik $ABCD$, ktorého dĺžky strán $AB$ a~$BC$ sú v~pomere $7:5$. Vnútri obdĺžnika $ABCD$ ležia body $X$ a~$Y$ tak, že trojuholníky $ABX$ a $CDY$ sú pravouhlé rovnoramenné s~pravými uhlami pri vrcholoch $X$ a~$Y$. Spoločná sivá plocha oboch trojuholníkov tvorí štvorec s~obsahom $72\cm^2$. Určte dĺžky strán $AB$ a~$BC$ obdĺžnika $ABCD$.
\ifobrazkyvedla~\else\insp{z61iii.1}\fi%
}
\podpis{Libor Šimůnek}

{%%%%%   Z9-III-2
Marienka mala desať kartičiek, na ktoré napísala desať po sebe idúcich prirodzených
čísel, na každú kartičku práve jedno. Nešťastnou náhodou však jednu kartičku stratila.
Súčet čísel na zostávajúcich deviatich kartičkách bol $2\,012$. Zistite, aké číslo bolo
napísané na stratenej kartičke.}
\podpis{Libuše Hozová}

{%%%%%   Z9-III-3
Zistite, koľkými rôznymi spôsobmi sa dajú do jednotlivých políčok trojuholníka na \ifobrazkyvedla{}obrázku\else\obr{}\fi{}
vpísať čísla $1$, $2$, $3$, $4$, $5$, $6$, $7$, $8$ a~$9$ tak, aby súčet v~každom štvorpolíčkovom trojuholníku
bol $23$ a~aby na niektorom políčku v~smere každej šípky bolo vpísané číslo zadané pri šípke.
\ifobrazkyvedla~\else\insp{z61iii.2}\fi%
}
\podpis{Erika Novotná}

{%%%%%   Z9-III-4
Vojto chcel na kalkulačke sčítať niekoľko trojciferných prirodzených čísel. Na prvý pokus
dostal výsledok $2\,224$. Pre kontrolu sčítal tieto čísla ešte raz a~vyšlo mu $2\,198$. Preto sčítal čísla ešte raz a~teraz dostal súčet $2\,204$. Piate pripočítavané číslo bolo totiž prekliate -- Vojto pri každom pokuse nestlačil niektorú z~jeho cifier dostatočne silno a~do kalkulačky zadal
vždy namiesto trojciferného čísla len dvojciferné. Žiadne ďalšie chyby pri sčitovaní neurobil. Aký je správny súčet Vojtových čísel?}
\podpis{Libor Šimůnek}

