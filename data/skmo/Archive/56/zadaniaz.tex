{%%%%%   Z4-I-1
...}
\podpis{...}

{%%%%%   Z4-I-2
Miss Strangelandia dostala tak veľa rôznych ponúk
od rôznych modelingových agentúr, že si z nich nevedela
vybrať. Napokon sa rozhodla, že prijme ponuku tej z nich,
ktorá ako prvá uhádne číslo jej topánok. Prezradila im, že je
to dvojciferné číslo s ciferným súčtom 12 a že, keby sa v tomto čísle zmenilo poradie číslic,
vzniklo by číslo o 54 väčšie. Aké číslo topánok nosí Miss Strangelandia?}
\podpis{S. Bednářová}

{%%%%%   Z4-I-3
Jurko včera na magnetickej tabuli vytvoril vzorový príklad. Dnes, keď prišiel do školy, zistil,
že pani upratovačka ho "upratala" a všetky použité číslice a znaky zoradila na dolnom okraji
tabule takto:
$$
3\ \ 4\ \ 5\ \ 6\ \ 8\ \ 9\ \ {+}\ \ {=}
$$
Aký príklad vytvoril včera Jurko? Nájdi všetky možnosti.}
\podpis{M. Dillingerová}

{%%%%%   Z4-I-4
Rodičia prvej slovenskej superstar K. K. sa starajú o jaskyňu "Zlá diera". V jaskyni je tma,
návštevníci si svietia karbidovými lampášikmi. Pre seba a pre pani učiteľku má pán Košč
väčšiu lampičku. Lampášikov však mali len 9, preto sprievodca pán Košč zoradil našu skupinu
ako husi tak, že tesne pred tým, kto nemal lampášik, išiel niekto s lampášikom alebo
lampičkou. Prvý v zástupe bol on, posledná pani učiteľka s lampičkou. Lampášik niesli 4
chlapci, 4 dievčatá lampášik nedostali. Koľko dievčat bolo v našej skupine? Koľko tam bolo
chlapcov? Poznámka: Nik neniesol dva lampášiky, a žiadni dvaja s lampášikmi nešli tesne za
sebou.}
\podpis{S. Bednářová}

{%%%%%   Z4-I-5
Sedem Snehulienkiných trpaslíkov bolo na hríboch. Prišli s košíčkami, v ktorých mali 34, 19,
50, 44, 31, 28 a 37 hríbov. Snehulienka im povedala, aby niektoré košíčky uložili do špajze,
niektoré dali ku sporáku a ostatné položili na stôl tak, že všade malo byť rovnako veľa hríbov.
Trpaslíci sa rozhodli, že hríby z košíčkov nebudú vyberať. Podarí sa im uložiť košíčky tak, ako
to chcela Snehulienka? Nájdi aspoň jeden spôsob.}
\podpis{Š. Ptáčková}

{%%%%%   Z4-I-6
Z čísla $1\,583\,719$ vyškrtni tri číslice tak, aby vzniklo čo najväčšie číslo, ktorého každá číslica je
nepárna.}
\podpis{M. Dillingerová}

{%%%%%   Z5-I-1
...}
\podpis{...}

{%%%%%   Z5-I-2
Cyklistického preteku Krížom-krážom sa zúčastnili šesťčlenné družstvá z celej Európy.
Prvých desať etáp ešte zvládli všetci, ale v jedenástej etape po hromadnom páde odstúpilo 17
cyklistov. V každej ďalšej etape ich potom odstúpilo o troch menej ako v predošlej. Do cieľa
poslednej, 15. etapy, došlo 53 cyklistov. Koľko družstiev sa zúčastnilo preteku?}
\podpis{Š. Ptáčková}

{%%%%%   Z5-I-3
Tomáškova cvičená blcha Skákalka stála na ciferníku hodín
na bodke pri čísle 12. Hrala sa s ním takúto hru: Tomáško
hodil kockou a blcha skočila o toľko bodiek ďalej, koľko mu
padlo na kocke. Ale po prvom hode skákala v smere pohybu
hodinových ručičiek, po druhom proti smeru a po treťom opäť
v smere pohybu hodinových ručičiek. Vieme, že Tomáško
hodil dvojku, päťku a šestku, ale nevieme, v akom poradí mu
padli.
\begin{itemize}
\itemvar{a)} Na bodke pri ktorom čísle mohla skončiť
Skákalka po treťom skoku?
\itemvar{b)} Na ktoré číslo sa blcha počas hry vôbec
nemohla dostať?
\end{itemize}
}
\podpis{L. Hozová}

{%%%%%   Z5-I-4
Pomocou číslic 0 až 9 a dvoch desatinných čiarok utvor dve desatinné čísla tak, aby ich súčet
bol čo najmenší. Nájdi všetky možnosti! (Každú číslicu treba použiť práve raz!)}
\podpis{S. Bednářová}

{%%%%%   Z5-I-5
Sedem Snehulienkiných trpaslíkov bolo na hríboch. Prišli s košíčkami, v ktorých mali 34, 19,
50, 44, 31, 28 a 37 hríbov. Snehulienka im povedala, aby niektoré košíčky uložili do špajze,
niektoré dali ku sporáku a ostatné položili na stôl tak, že všade malo byť rovnako veľa hríbov.
Trpaslíci sa rozhodli, že hríby z košíčkov nebudú vyberať. Podarí sa im uložiť košíčky tak, ako
to chcela Snehulienka? Nájdi aspoň jeden spôsob.}
\podpis{Š. Ptáčková}

{%%%%%   Z5-I-6
...}
\podpis{...}

{%%%%%   Z6-I-1
Lukáš natieral latkový plot. Každých 10 minút natrel 8 latiek. Jeho mladší brat Kubko mu
chvíľku pomáhal. Za 7 minút natrel vždy 4 latky, takže Lukáš skončil o štvrť hodiny skôr, ako
predpokladal. Ako dlho mu Kubko pomáhal?}
\podpis{M. Raabová}

{%%%%%   Z6-I-2
...}
\podpis{...}

{%%%%%   Z6-I-3
...}
\podpis{...}

{%%%%%   Z6-I-4
...}
\podpis{...}

{%%%%%   Z6-I-5
Viacciferné číslo sa nazýva optimistické, ak jeho číslice zľava doprava rastú. Ak číslice čísla
zľava doprava klesajú, hovoríme, že je to číslo pesimistické. Súčet sedemciferného
pesimistického a sedemciferného optimistického čísla
zloženého z tých istých číslic je $11\,001\,000$. Ktoré
číslice sme použili na zápis týchto dvoch čísel?}
\podpis{S. Bednářová}

{%%%%%   Z6-I-6
Naša trieda plánovala turistický výlet. Niektorí žiaci sa
dohadovali o dĺžke jeho trasy a tvrdili, že je to 28, 16, 32, 37 a 15 km. Mýlili
sa však o 5, 7, 8, 9 a 14 km. Aký dlhý bol výlet v skutočnosti?}
\podpis{M. Volfová}

{%%%%%   Z7-I-1
Janka narysovala 6-uholník, ktorého dĺžky strán vyjadrené v cm sú celé čísla. Potom si
uvedomila, že každé dve jeho susedné strany sú na seba kolmé. Zisti, ako mohol vyzerať
Jankin 6-uholník, ak jeho obvod má byť $16\cm$ a jeho obsah má byť $12\cm^2$. Narysuj obe
možnosti.}
\podpis{M. Dillingerová}

{%%%%%   Z7-I-2
...}
\podpis{...}

{%%%%%   Z7-I-3
Urči počet zlomkov, ktorých hodnota je násobkom troch
a čitateľ aj menovateľ sú trojciferné prirodzené čísla.}
\podpis{L. Šimůnek}

{%%%%%   Z7-I-4
Rozprávkový deduško niesol vrece zrna do mlyna. Po ceste mu zrno začalo z vreca vypadávať.
Tri vtáčiky si všimli, že za deduškom ostáva cestička označená jednotlivými zrnkami. Prvý
išiel zobať zrnká zelený vtáčik a zozobal každé štvrté zrnko ležiace na zemi. Potom priletel
zobať červený vtáčik a zozobal každé piate na zemi ležiace zrnko. Nakoniec zlietol na cestičku
modrý vtáčik a zozobal každé tretie na zemi ležiace zrnko. Koľko zrniek stratil deduško
z vreca, ak vtáčiky zozobali spolu 79 zrniek?}
\podpis{M. Dillingerová}

{%%%%%   Z7-I-5
Troj- a viacciferné číslo s navzájom rôznymi ciframi, pre ktorého žiadne tri za sebou idúce
číslice $a$, $b$, $c$ neplatí $a<b<c$ ani $a>b>c$, sa nazýva vlnité. Napíš
\begin{itemize}
\itemvar{a)} najväčšie vlnité číslo, ktoré nie je deliteľné $3$,
\itemvar{b)} najväčšie vlnité číslo deliteľné $150$.
\end{itemize}
}
\podpis{S. Bednářová}

{%%%%%   Z7-I-6
...}
\podpis{...}

{%%%%%   Z8-I-1
Z číslic 1 až 9 sme utvorili tri zmiešané čísla $a\frac bc$. Potom sme tieto tri čísla správne sčítali. Aký
najmenší súčet sme mohli dostať?
(Každú číslicu sme použili práve raz!)}
\podpis{S. Bednářová}

{%%%%%   Z8-I-2
Pán kráľ si dal naliať plnú čašu vína. Pätinu vína z nej odpil. Potom si nechal čašu doplniť
vodou a odpil štvrtinu objemu. Opäť mu ju doliali vodou a kráľ odpil tretinu. Nakoniec čašu
ešte raz doliali vodou doplna. Koľko percent objemu čaše tvorí pôvodné víno?}
\podpis{M. Krejčová}

{%%%%%   Z8-I-3
Je daný pravidelný deväťuholník $ABCDEFGHI$. Vypočítajte veľkosť uhla, ktorý zvierajú
priamky $DG$ a $BE$.}
\podpis{M. Krejčová, M. Raabová}

{%%%%%   Z8-I-4
...}
\podpis{...}

{%%%%%   Z8-I-5
Na lúke sa pásli ovce. Tých s rohami bolo dvakrát menej ako tých bez rohov. Tých s tmavým
kožúškom bolo toľko ako tých so svetlým kožúškom. (Iné ovce, jednorohé, fľakaté a pod., sa
na lúke nepásli.) Iba tri tmavé ovečky nemali rohy a svetlé vôbec nemali rohy. Koľko sa páslo
ovečiek na lúke?}
\podpis{M. Dillingerová}

{%%%%%   Z8-I-6
Výška rovnoramenného trojuholníka $ABC$ delí tento trojuholník na dve časti, ktorých obsahy
sú v pomere $1:3$. Určte obsah a obvod trojuholníka $ABC$, ak viete, že $|AC| = |BC|$ a $|AB| = \sqrt{32}\cm$.}
\podpis{L. Hozová}

{%%%%%   Z9-I-1
Zistite, koľko je takých šesťciferných čísel, ktoré majú ciferný súčin $750$.}
\podpis{P. Tlustý}

{%%%%%   Z9-I-2
...}
\podpis{...}

{%%%%%   Z9-I-3
Do kružnice s polomerom $2\cm$ je vpísaný pravidelný šesťuholník $ABCDEF$. Priesečník
priamok $FE$ a $CD$ označme $M$. Vypočítajte dĺžku úsečky $AM$.}
\podpis{M. Volfová}

{%%%%%   Z9-I-4
Matematickej súťaže sa zúčastnilo 142 žiakov. Po skončení súťaže autor zistil, že priemerný
počet bodov udelených za tretiu úlohu pripadajúci na jedného súťažiaceho je 2,7 (zaokrúhlené
na desatiny). Nie každý súťažiaci však tretiu úlohu odovzdal, takže priemerný počet bodov
udelených za tretiu úlohu pripadajúci na jedno odovzdané riešenie bol 3,9 (zaokrúhlené na
desatiny). Koľko súťažiacich mohlo odovzdať tretiu úlohu?
Poznámka: Udeľovali sa len celé body, neodovzdaná úloha bola hodnotená 0 bodmi.}
\podpis{L. Šimůnek}

{%%%%%   Z9-I-5
Trojuholník $REZ$ s obsahom $300\cm^2$, stranou $RE$ dĺžky $25\cm$ a stranou $ZE$ dĺžky $30\cm$ sme
dvomi priamymi rezmi rozdelili na 3 časti a z týchto častí zložili (bez prekrývania) obdĺžnik.
Aké rozmery mohol mať tento obdĺžnik? Nájdite všetky možnosti.}
\podpis{S. Bednářová}

{%%%%%   Z9-I-6
Ukážte, že číslo
$$
(1 \cdot 3 \cdot 5 \cdot 7 \cdot \dots \cdot 2003 \cdot 2005 ) + (2 \cdot 4 \cdot 6 \cdot 8 \cdot \dots \cdot 2004 \cdot 2006)
$$
je deliteľné číslom $2007^4$.}
\podpis{P. Tlustý}

{%%%%%   Z4-II-1
...}
\podpis{...}

{%%%%%   Z4-II-2
V stánku s kvetmi majú ruže, tulipány a klinčeky (a žiadne iné kvety). Ruží je o 12 viac ako
tulipánov a tulipánov je o 6 menej ako klinčekov. Spolu je v stánku 60 kvetov. Koľko z tohto
počtu je ruží? Koľko je tulipánov a koľko klinčekov?}
\podpis{M. Dillingerová}

{%%%%%   Z4-II-3
Z čísel $1\,523$ a $6\,346$ vyškrtni spolu dve číslice tak, aby súčet vzniknutých čísel bolo čo
najväčšie číslo, ktorého každá číslica je nepárna.}
\podpis{M. Dillingerová}

{%%%%%   Z5-II-1
Dievčatá zbierali uzávery PET-fľaší. Šárka ich nazbierala 20, Svetlana 29, Marta 31, Maruška
49 a Monika 51. Každé z dievčat nasypalo všetky svoje nazbierané uzávery buď do modrej,
alebo do červenej škatule. Paľko pri počítaní uzáverov zistil, že v modrej škatuli je dvakrát viac
uzáverov ako v červenej. Ktoré dievčatá nasypali svoje uzávery do modrej a ktoré do červenej
škatule?}
\podpis{L. Hozová}

{%%%%%   Z5-II-2
...}
\podpis{...}

{%%%%%   Z5-II-3
Na záhradke vyrástlo štyrikrát viac kalerábov ako brokolíc a trikrát viac reďkvičiek ako
kalerábov. Celková hmotnosť brokolíc bola 5 kg. Koľko kusov zeleniny vyrástlo na záhradke, ak
každá brokolica vážila 250 g? (Iná zelenina tam nerástla.)}
\podpis{L. Černíček}

{%%%%%   Z6-II-1
...}
\podpis{...}

{%%%%%   Z6-II-2
Cyril má o štvrtinu guliek viac ako Boris a ten o štvrtinu guliek viac ako Adam. Spolu majú
122 guliek. Koľko má každý z nich?}
\podpis{M. Volfová}

{%%%%%   Z6-II-3
Viacciferné číslo sa nazýva optimistické, ak jeho číslice zľava doprava rastú. Ak číslice čísla
zľava doprava klesajú, hovoríme, že je to číslo pesimistické. Súčet pesimistického a
optimistického čísla zloženého z tých istých číslic je $109\,900$, ich rozdiel je $84\,942$. Ktoré sú to
čísla?}
\podpis{S. Bednářová}

{%%%%%   Z7-II-1
Baba Jaga pripravila vzácny elixír. Na trhu ponúkala 6 fľaštičiek -- o objeme 11 ml, 12 ml,
17 ml, 19 ml, 21 ml a 26 ml. Elixír bol však iba v piatich z nich. Jedna fľaštička obsahovala
zafarbenú vodu. Prvý kupec si odniesol 2 fľaštičky s elixírom. Druhý získal tiež iba elixír a
bolo ho dvakrát viac, než koľko si odniesol prvý kupec. Viac kupcov neprišlo a tak si baba
Jaga musela svoju fľaštičku s vodou odniesť domov. Ktorá to bola? Koľko ml elixíru kúpil
prvý, koľko druhý kupec?}
\podpis{M. Volfová}

{%%%%%   Z7-II-2
...}
\podpis{...}

{%%%%%   Z7-II-3
Hanka sa hrá s tromi prázdnymi nádobami. Najprv naplnila vodou až po okraj najmenšiu a
stredne veľkú nádobu. Túto vodu preliala z oboch nádob do najväčšej. Tým ju naplnila na
50\%. Potom znovu naplnila vodou stredne veľkú nádobu až po okraj. Touto vodou najprv
naplnila najmenšiu nádobu a celý zvyšok preliala do najväčšej, ktorú tým doplnila do $2/3$ jej
objemu. Aké sú objemy najmenšej a najväčšej nádoby, ak stredne veľká nádoba má objem
6 dl?}
\podpis{M. Raabová}

{%%%%%   Z8-II-1
...}
\podpis{...}

{%%%%%   Z8-II-2
\begin{itemize}
\itemvar{a)} Zistite súčet všetkých dvojciferných čísel, ktorých druhá mocnina končí číslicou 9.
\itemvar{b)} Nájdite všetky dvojciferné čísla, ktorých druhá mocnina končí dvojčíslím 44.
\end{itemize}
}
\podpis{M. Smitková, M. Dillingerová}

{%%%%%   Z8-II-3
Do kružnice $k$ so stredom $O$ je vpísaný lichobežník $ABCD$ ($AB\parallel CD$), pričom priamka $AC$ je
osou uhla $DAB$. Ukážte, že potom $|AD|=|CD|$ a navyše, priamka $CO$ je osou uhla $BCD$.}
\podpis{F. Kardoš}

{%%%%%   Z9-II-1
...}
\podpis{...}

{%%%%%   Z9-II-2
...}
\podpis{...}

{%%%%%   Z9-II-3
Nováková, Vašková a Sudková vyhrali štafetu a okrem diplomov dostali aj bonboniéru, ktorú hneď po
závode zjedli. Keby zjedla Petra o 3 bonbóny viac, zjedla by ich práve toľko čo Miška s Janou dokopy.
A keby si Jana pochutnala ešte na siedmich bonbónoch, tiež by ich mala toľko ako druhé dve spolu. Ešte
vieme, že počet bonbónov, ktoré zjedla Vašková, je deliteľný tromi, a že Sudková si pochutila na
siedmich bonbónoch. Ako sa volali dievčatá? Koľko bonbónov zjedla každá z nich?}
\podpis{M. Volfová}

{%%%%%   Z9-II-4
Daný je obdĺžnik $KLMN$, kde $|KL| = 6\cm$ a $|ML| = 4\cm$. Vypočítajte obvody všetkých rovnoramenných
trojuholníkov $KLX$, ktorých vrchol $X$ leží na strane $MN$.}
\podpis{M. Dillingerová}

{%%%%%   Z9-III-1
Pavol si zvolil dve prirodzené čísla a vypočítal rozdiel ich druhých mocnín. Vyšlo mu
$2\,007$.
Ktoré dvojice čísel si mohol Pavel zvoliť?}
\podpis{P. Tlustý}

{%%%%%   Z9-III-2
V laboratóriu na polici stojí uzavretá sklenená nádoba tvaru kvádra. Nachádza sa v nej
2,4 litra destilovanej vody, ale objem nádoby je väčší. Voda v nádobe siaha do výšky
$16\cm$. Keď kvádrovú nádobu preklopíme na inú jej stenu, bude hladina vo výške
$10\cm$. Keby sme ju postavili na ešte inú stenu, voda by siahala iba do výšky $9{,}6\cm$.
Určte rozmery nádoby.}
\podpis{L. Šimůnek}

{%%%%%   Z9-III-3
Prečítajte si výsledky ankety, pri ktorej bolo oslovených 1\,240 ľudí:
"V existenciu Yetiho verí 46\% opýtaných (zaokrúhlené na
celé číslo), 31\% v jeho existenciu neverí (zaokrúhlené na celé
číslo). Ostatní účastníci ankety odmietli na túto otázku
akokoľvek reagovať."
\begin{itemize}
\itemvar{a)} Koľko najmenej ľudí mohlo v ankete odpovedať, že veria v existenciu Yetiho?
\itemvar{b)} Koľko najviac ľudí mohlo odmietnuť na anketu odpovedať?
\end{itemize}
\noindent
Uveďte konkrétne počty, nie percentá.}
\podpis{L. Šimůnek}

{%%%%%   Z9-III-4
...}
\podpis{...}

