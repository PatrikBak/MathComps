{%%%%%   A-I-1
V~obore reálnych čísel vyriešte rovnicu
$$
4x^4-12x^3-7x^2+22x+14=0,
$$
keď viete, že má štyri rôzne reálne korene, pričom súčet dvoch
z~nich sa rovná číslu~1.}
\podpis{Jaromír Šimša}

{%%%%%   A-I-2
Kružnica vpísaná do daného trojuholníka $ABC$ sa dotýka strán $BC$,
$CA$, $AB$ postupne v~bodoch $K$, $L$, $M$. Označme $P$ priesečník
osi vnútorného uhla pri vrchole~$C$ s~priamkou~$MK$. Dokážte, že
priamky $AP$ a~$LK$ sú rovnobežné.}
\podpis{Peter Novotný}

{%%%%%   A-I-3
Ak $x$, $y$, $z$ sú reálne čísla z~intervalu $\langle-1,1\rangle$ spĺňajúce
podmienku $xy+yz+zx=1$, tak platí
$$
6\root3\of{(1-x^2)(1-y^2)(1-z^2)}\leq 1+(x+y+z)^2.
$$
Dokážte a~zistite, kedy nastáva rovnosť.}
\podpis{Jaroslav Švrček}

{%%%%%   A-I-4
Určte, pre ktoré prirodzené čísla~$n$ sa množina
$\mm{M}=\{1,2,\dots ,n\}$ dá rozdeliť
\ite a) na dve,
\ite b) na tri

\noindent
navzájom disjunktné podmnožiny s~rovnakým počtom prvkov tak, aby
každá z~nich obsahovala aj aritmetický priemer všetkých svojich
prvkov.}
\podpis{Peter Novotný}

{%%%%%   A-I-5
V~rovine je daná kružnica~$k$ so stredom~$S$ a~bod $A\ne S$.
Určte množinu stredov kružníc opísaných všetkým trojuholníkom $ABC$,
ktorých strana~$BC$ je priemerom kružnice~$k$.}
\podpis{Jiří Dula}

{%%%%%   A-I-6
Určte všetky funkcie $f\colon\Bbb Z \to \Bbb Z$ také, že pre všetky celé
čísla $x$, $y$ platí
$$
f\bigl(f(x)+y\bigr)=x+f(y+2\,006).
$$}
\podpis{Petr Kaňovský}

{%%%%%   B-I-1
Nájdite všetky dvojice celých čísel $(a,b)$, ktoré sú riešením rovnice
$$
a^2+7ab+6b^2+5a+4b+3=0.
$$}
\podpis{Pavel Novotný}

{%%%%%   B-I-2
Daná je kružnica~$k$ s~priemerom~$AB$. K~ľubovoľnému bodu~$Y$
kružnice~$k$, $Y\ne A$, zostrojme na polpriamke~$AY$ bod~$X$, pre
ktorý platí $|AX|=|YB|$. Určte množinu všetkých takých bodov~$X$.}
\podpis{Pavel Leischner}

{%%%%%   B-I-3
Nájdite najmenšie prirodzené číslo~$k$ také, že každá $k$-prvková
množina trojciferných po dvoch nesúdeliteľných čísel obsahuje aspoň
jedno prvočíslo.}
\podpis{Pavel Novotný}

{%%%%%   B-I-4
V~ľubovoľnom trojuholníku $ABC$ označme $T$ ťažisko, $D$ stred
strany~$AC$ a~$E$ stred strany~$BC$. Nájdite všetky pravouhlé
trojuholníky $ABC$ s~preponou~$AB$, pre ktoré je štvoruholník
$CDTE$ dotyčnicový.}
\podpis{Ján Mazák}

{%%%%%   B-I-5
Nájdite všetky dvojice reálnych čísel $(p,q)$ také,
že mnohočlen $x^2+px+q$ je deliteľom mnohočlena $x^4+px^2+q$.}
\podpis{Jozef Moravčík}

{%%%%%   B-I-6
Daná je úsečka~$AA_0$ a~priamka~$p$. Zostrojte trojuholník
s~vrcholom~$A$ a~výškou~$AA_0$, ktorého ťažisko a~stred kružnice
opísanej ležia na priamke~$p$.}
\podpis{Eva Řídká}

{%%%%%   C-I-1
Určte všetky dvojice prirodzených čísel $(a,b)$, pre ktoré platí
$$
a+5\sqrt b=b+5\sqrt a.
$$}
\podpis{Jaroslav Švrček}

{%%%%%   C-I-2
Nájdite všetky trojuholníky, ktoré sa dajú rozrezať na lichobežníky so
stranami dĺžok 1\,cm, 1\,cm, 1\,cm a~2\,cm.}
\podpis{Ján Mazák}

{%%%%%   C-I-3
Nájdite všetky prirodzené čísla, ktorých zápis neobsahuje nulu
a~má nasledujúcu vlastnosť: ak v~ňom vynecháme ľubovoľnú
číslicu, dostaneme číslo, ktoré je deliteľom pôvodného čísla.}
\podpis{Jaromír Šimša}

{%%%%%   C-I-4
Daný je lichobežník $ABCD$ so základňami $AB$ a~$CD$. Označme $E$
stred strany~$AB$, $F$ stred úsečky~$DE$ a~$G$ priesečník úsečiek
$BD$ a~$CE$. Vyjadrite obsah lichobežníka $ABCD$ pomocou jeho
výšky~$v$ a~dĺžky~$d$ úsečky~$FG$ za predpokladu, že body $A$,
$F$, $C$ ležia na jednej priamke.}
\podpis{Ján Mazák}

{%%%%%   C-I-5
Zistite, pre ktoré prirodzené číslo~$n$ je podiel
$$
33\,000\over(n-4)(n+1)
$$
a) čo najväčšie, b) čo najmenšie
prirodzené číslo.}
\podpis{Eva Řídká}

{%%%%%   C-I-6
Daný je ostrouhlý trojuholník $ABC$, v~ktorom $D$ je päta výšky
z~vrcholu~$C$ a~$V$ priesečník výšok. Dokážte, že
$|AD|\cdot|BD|=|AB|\cdot|VD|$ práve vtedy, keď $|CD|=|AB|$.}
\podpis{Jaroslav Zhouf}

{%%%%%   A-S-1
Určte všetky reálne čísla~$s$, pre ktoré má rovnica
$$
4x^4-20x^3+sx^2+22x-2=0
$$
štyri rôzne reálne korene, pričom súčin dvoch z~nich je rovný
číslu~$\m2$.}
\podpis{Jaromír Šimša}

{%%%%%   A-S-2
Uvažujme množinu $\{1,2,4,5,8,10,16,20,32,40,80,160\}$
a~všetky jej trojprvkové podmnožiny. Rozhodnite, či je viac tých,
ktoré majú súčin svojich prvkov väčší ako $2\,006$,
alebo tých, ktoré majú súčin svojich prvkov menší ako $2\,006$.}
\podpis{Peter Novotný}

{%%%%%   A-S-3
Daný je lichobežník $ABCD$ s~pravým uhlom pri vrchole~$A$ a~základňou~$AB$,
v~ktorom platí $|AB|>|CD|\ge |DA|$. Označme $S$
priesečník osí jeho vnútorných uhlov pri vrcholoch
$A$, $B$ a~$T$ priesečník osí vnútorných uhlov pri vrcholoch $C$, $D$. Podobne
označme $U$, $V$ priesečníky osí vnútorných uhlov pri vrcholoch
$A$, $D$, resp. $B$, $C$.
\ite a) Dokážte, že priamky $UV$ a~$AB$ sú rovnobežné.
\ite b) Dokážte, že priesečník~$E$ polpriamky~$DT$ s~priamkou~$AB$
        a~body $S$, $T$, $B$ ležia na jednej kružnici.}
\podpis{J. Švrček, P. Calábek}

{%%%%%   A-II-1
Zistite, aký je najmenší možný obsah trojuholníka $ABC$, ktorého
výšky spĺňajú nerovnosti $v_a\ge3\cm$, $v_b\ge4\cm$,
$v_c\ge5\cm$.}
\podpis{Pavel Novotný}

{%%%%%   A-II-2
Nech $a$, $b$ sú reálne čísla. Ak má rovnica
$$
x^4-4x^3+4x^2+ax+b=0
$$
dva rôzne reálne korene také, že ich súčet sa rovná ich
súčinu, tak platí $a+b>0$ a~pritom daná rovnica nemá žiadne iné
reálne korene. Dokážte.}
\podpis{Jaromír Šimša}

{%%%%%   A-II-3
Nech $M$ je ľubovoľný vnútorný bod prepony~$AB$ pravouhlého
trojuholníka $ABC$. Označme $S$, $S_1$, $S_2$ stredy kružníc
opísaných postupne trojuholníkom $ABC$, $AMC$, $BMC$.
\ite a) Dokážte, že body $M$, $C$, $S_1$, $S_2$ a~$S$ ležia na jednej kružnici.
\ite b) Pre ktorú polohu bodu~$M$ má táto kružnica najmenší polomer?}
\podpis{Jaroslav Švrček}

{%%%%%   A-II-4
Nech $p$, $q$ sú dané prirodzené čísla, pričom $p<q$. Určte najmenšie
prirodzené číslo~$m$ s~vlastnosťou: Súčet všetkých zlomkov v~základnom
tvare, ktoré majú menovateľa~$m$ a~ktorých hodnoty ležia
v~otvorenom intervale $(p,q)$, je aspoň $56(q^2-p^2)$.}
\podpis{Vojtech Bálint}

{%%%%%   A-III-1
Na niektoré políčko štvorcovej šachovnice $n\times n$ ($n\geq 2$)
postavíme figúrku a~potom ju posúvame striedavo "šikmo" a~"priamo".
"Šikmo" znamená na políčko, ktoré má s~predchádzajúcim spoločný práve
jeden bod. "Priamo" znamená na susedné políčko, ktoré má
s~predchádzajúcim spoločnú stranu. Určte všetky $n$, pre ktoré
existuje východiskové políčko a~taká postupnosť ťahov začínajúca "šikmo",
že figúrka prejde celú šachovnicu a~na každom políčku sa ocitne
práve raz.}
\podpis{Peter Novotný}

{%%%%%   A-III-2
V~tetivovom štvoruholníku $ABCD$ označme $L$, $M$ stredy kružníc
vpísaných postupne do trojuholníkov $BCA$, $BCD$. Ďalej označme $R$
priesečník kolmíc vedených z~bodov $L$ a~$M$ postupne na priamky $AC$
a~$BD$. Dokážte, že trojuholník $LMR$ je rovnoramenný.}
\podpis{Pavel Leischner}

{%%%%%   A-III-3
Označme $\Bbb N$ množinu všetkých prirodzených čísel a~uvažujme všetky
funkcie $f:\Bbb N\to \Bbb N$ také, že pre ľubovoľné $x,y\in \Bbb N$ platí
$$
f\bigl(xf(y)\bigr)=yf(x).
$$
Určte najmenšiu možnú hodnotu $f(2\,007)$.}
\podpis{Pavel Calábek}

{%%%%%   A-III-4
Množina~$\mm M$ obsahuje všetky prirodzené čísla od $1$ do $2\,007$
vrátane a~má nasledujúcu vlastnosť: Ak je číslo~$n$ prvkom množiny~$\mm M$,
ležia v~$\mm M$ všetky členy aritmetickej postupnosti s~prvým
členom~$n$ a~diferenciou $n+1$. Rozhodnite, či množina~$\mm M$ musí
obsahovať všetky prirodzené čísla väčšie ako určité číslo~$m$.}
\podpis{Jaromír Šimša}

{%%%%%   A-III-5
Daný je ostrouhlý trojuholník $ABC$ taký, že $|AC|\ne |BC|$.
Vnútri jeho strán $BC$ a~$AC$ uvažujme body $D$ a~$E$, pre ktoré je
$ABDE$ tetivový štvoruholník. Priesečník jeho uhlopriečok $AD$ a~$BE$
označme~$P$. Dokážte, že ak sú priamky $CP$ a~$AB$ navzájom kolmé, tak $P$
je priesečníkom výšok trojuholníka $ABC$.}
\podpis{Ján Mazák}

{%%%%%   A-III-6
Určte všetky usporiadané trojice $(x,y,z)$ navzájom rôznych reálnych
čísel, ktoré vyhovujú množinovej rovnici
$$
\{x,y,z\}=
     \left\{\frac{x-y}{y-z},\frac{y-z}{z-x},\frac{z-x}{x-y}\right\}.
$$}
\podpis{Jaromír Šimša}

{%%%%%   B-S-1
Určte všetky dvojice reálnych čísel $a$, $b$, pre ktoré je polynóm $x^4+ax^2+b$ deliteľný polynómom $x^2+bx+a$.}
\podpis{Jaromír Šimša}

{%%%%%   B-S-2
V~trojuholníku $ABC$ označme $D$ stred strany~$BC$, $E$ stred strany~$AC$ a~$T$ ťažisko. Dokážte, že ak je strana~$BC$ dlhšia ako strana~$AC$, má kružnica vpísaná trojuholníku $BDT$ menší polomer ako kružnica vpísaná trojuholníku $ATE$.}
\podpis{Pavel Novotný}

{%%%%%   B-S-3
Nájdite najmenšie prirodzené číslo~$n$, pre ktoré je podiel $\displaystyle\frac {n^2+15n}{33\,000}$ prirodzené číslo.}
\podpis{Jaromír Šimša}

{%%%%%   B-II-1
Určte reálne čísla $a$, $b$, $c$ tak, aby polynóm $x^4+ax^2+bx+c$ bol deliteľný polynómom $x^2+x+1$ a~pritom súčet $a^2+b^2+c^2$ bol čo najmenší.}
\podpis{Jaromír Šimša}

{%%%%%   B-II-2
Daný je trojuholník $ABC$ so stranou~$BC$ dĺžky $22\cm$ a~stranou~$AC$ dĺžky $19\cm$, ktorého ťažnice $t_a$, $t_b$ sú navzájom kolmé. Vypočítajte dĺžku strany~$AB$.}
\podpis{Pavel Novotný}

{%%%%%   B-II-3
Prirodzené číslo nazveme {\it vlnitým}, ak pre každé tri po sebe idúce číslice $a$, $b$, $c$ jeho dekadického zápisu platí $(a-b)(b-c)<0$. Dokážte, že z~číslic $0,1,\dots,9$ je možné zostaviť viac ako $25\,000$ desaťciferných vlnitých čísel, z~ktorých každé obsahuje všetky číslice od nuly po deviatku (číslica~$0$ nesmie byť na prvom mieste).}
\podpis{Jaromír Šimša}

{%%%%%   B-II-4
Je daný ostrouhlý trojuholník $ABC$. Pre ľubovoľný bod~$L$ jeho strany~$AB$ označme $K$, $M$ päty kolmíc z~bodu~$L$ na strany $AC$, $BC$. Zistite, pre ktorú polohu bodu~$L$ je úsečka~$KM$ najkratšia.}
\podpis{Jaroslav Švrček}

{%%%%%   C-S-1
Určte počet všetkých štvorciferných prirodzených čísel, ktoré sú deliteľné šiestimi a~v~ich zápise sa vyskytujú práve dve jednotky.}
\podpis{Pavel Leischner}

{%%%%%   C-S-2
Kružnica~$k$ so stredom~$S$ je opísaná pravidelnému šesťuholníku $ABCDEF$. Dotyčnica v~bode~$A$ ku kružnici~$k$ pretína priamku~$SB$ v~bode~$K$ a~dotyčnica v~bode~$B$ pretína priamku~$SC$ v~bode~$L$. Dokážte, že štvoruholníku $KLCB$ sa dá opísať kružnica, ktorá je zhodná s~kružnicou~$k$.}
\podpis{Jaroslav Zhouf}

{%%%%%   C-S-3
Určte všetky dvojice $(a,b)$ prirodzených čísel, ktorých rozdiel $a-b$ je piatou mocninou niektorého prvočísla a~pre ktoré platí $a-4\sqrt b=b+4\sqrt{a\vphantom b}$.}
\podpis{Jaroslav Švrček}

{%%%%%   C-II-1
V~rovine sú dané dva rôzne body $L$, $M$ a~kružnica~$k$. Zostrojte
trojuholník $ABC$ s~čo najväčším obsahom tak, aby jeho vrchol~$C$
ležal na kružnici~$k$, bod~$L$ bol stredom strany~$AC$ a~bod~$M$
stredom strany~$BC$.}
\podpis{Pavel Leischner}

{%%%%%   C-II-2
Nech $p$, $q$, $r$ sú prirodzené čísla, pre ktoré platí
$p+r\sqrt{p+q}+q=2\,007$.
\ite a) Určte, aké hodnoty môže nadobúdať súčet $p+q+r$.
\ite b) Určte počet všetkých usporiadaných trojíc $(p,q,r)$ prirodzených čísel,
        ktoré vyhovujú danej rovnici.

        }
\podpis{Jaroslav Švrček}

{%%%%%   C-II-3
Rovnoramennému lichobežníku $ABCD$ so základňami $AB$, $CD$ sa dá
vpísať kružnica so stredom~$O$. Určte obsah~$S$ lichobežníka,
ak sú dané dĺžky úsečiek $OB$ a~$OC$.}
\podpis{Pavel Leischner}

{%%%%%   C-II-4
Určte najväčšie dvojciferné číslo~$k$ s~nasledujúcou vlastnosťou:
existuje prirodzené číslo~$N$, z~ktorého po škrtnutí prvej číslice
zľava dostaneme číslo $k$-krát menšie. (Po škrtnutí číslice môže
zápis čísla začínať jednou či niekoľkými nulami.) K~určenému
číslu~$k$ potom nájdite najmenšie vyhovujúce číslo~$N$.}
\podpis{Jaromír Šimša}

{%%%%%   vyberko, den 1, priklad 1
Označme $A_n$ počet permutácií $a_1, a_2,\dots, a_n$ čísel $1,2,\dots,n$ takých, že
pre všetky $k=1,2,\dots,n$ je $|a_k-k|$ rovné 0, 1 alebo 2. Dokážte, že
pre $n\ge 6$~platí
$$
A_n=2A_{n-1}+2A_{n-3}-A_{n-5}.
$$}
\podpis{Tomáš Jurík:Mongolian national mathematics olympiad 2004, day 2, problem 1}

{%%%%%   vyberko, den 1, priklad 2
Dokážte, že ak $a$, $b$, $c$, $d$ sú kladné reálne čísla spĺňajúce nerovnosť
$$
\left(a^2+b^2+c^2+d^2\right)^2 > 3\left(a^4+b^4+c^4+d^4\right),
$$
tak ľubovoľná trojica z~čísel $a$, $b$, $c$, $d$ môže byť
dĺžkami strán trojuholníka.}
\podpis{Tomáš Jurík:Inequalities 2.2.2, manfrino, ortega, delgado}

{%%%%%   vyberko, den 1, priklad 3
Trojuholník $ABC$ je vpísaný do kružnice $\Omega$. Nech $M_1$, $M_2$, $M_3$ sú postupne
stredy strán $BC$, $CA$, $AC$ a~$T_1$, $T_2$, $T_3$ sú stredy oblúkov $BC$, $CA$, $AB$
na kružnici~$\Omega$, ktoré neobsahujú protiľahlý vrchol. Pre $i=1,2,3$ nech
$\omega_i$ je kružnica s~priemerom~$M_iT_i$ a~$p_i$ je spoločná vonkajšia dotyčnica
kružníc $\omega_j$, $\omega_k$ ($\{i,j,k\}=\{1,2,3\}$) taká, že $\omega_i$ leží
na opačnej strane priamky~$p_i$ ako $\omega_j$ a~$\omega_k$. Dokážte, že priamky
$p_1$, $p_2$, $p_3$ vytvárajú trojuholník podobný s~$ABC$ a~nájdite pomer podobnosti.
%Zadanie bude zverejnené po IMO 2007.
}
\podpis{Tomáš Jurík:Shortlist 2006, G7}

{%%%%%   vyberko, den 2, priklad 1
Vnútri strany~$BC$ ostrouhlého trojuholníka $ABC$ leží bod~$D$. Body $P$ a~$Q$ sú stredmi kružníc opísaných trojuholníkom $ABD$ a~$ACD$. Dokážte, že existuje bod~$M$ rôzny od bodu~$A$ taký, že ním prechádzajú všetky kružnice opísané všetkým možným trojuholníkom $APQ$ (trojuholník $ABC$ je pevný a~bod~$D$ sa pohybuje po strane~$BC$).}
\podpis{Ján Mazák:Ukrajina 2006, Final Round, 4/9.8}

{%%%%%   vyberko, den 2, priklad 2
Prirodzené číslo~$N$ sa dá napísať v~tvare $a^2+b^2+c^2$, kde $a$, $b$, $c$ sú celé čísla deliteľné tromi. Dokážte, že sa dá napísať aj v~tvare $x^2+y^2+z^2$, kde $x$, $y$, $z$ sú celé čísla, z~ktorých žiadne nie je deliteľné tromi.}
\podpis{Ján Mazák:Rusko 2005/6, 6/11.7}

{%%%%%   vyberko, den 2, priklad 3
V~tabuľke $300\times 300$ je $N$~políčok zafarbených čiernou farbou, ostatné sú biele. Žiadne tri čierne políčka netvoria roh (\tj. útvar \Image*{56-tst-d2-3}{1.1}, ľubovoľne otočený). Keď však zafarbíme ľubovoľné biele políčko načierno, táto podmienka prestane platiť. Nájdite minimálnu hodnotu~$N$.}
\podpis{Ján Mazák:Rusko 2005/6, 6/11.8}

{%%%%%   vyberko, den 3, priklad 1
Postupnosť reálnych čísel $a_1,a_2,a_3,\dots$ je definovaná predpisom
$$
a_{n+1}=\lfloor a_n\rfloor \cdot \langle a_n\rangle \qquad\text{pre $n\ge1$},
$$
pričom $a_1$ je ľubovoľné reálne číslo. Výraz $\lfloor a_n\rfloor$ označuje najväčšie celé číslo, ktoré neprevyšuje~$a_n$. Výraz $\langle a_n\rangle$ je určený predpisom $\langle a_n\rangle=a_n-\lfloor a_n\rfloor$. Dokážte, že počnúc od určitej hodnoty pre každé $n$ platí $a_n=a_{n+2}$.
%Zadanie bude zverejnené po IMO 2007.
}
\podpis{Peter Novotný:Shortlist 2006, A1}

{%%%%%   vyberko, den 3, priklad 2
Daný je konvexný päťuholník $ABCDE$, pričom
$$
|\uhol BAC|=|\uhol CAD|=|\uhol DAE|\qquad\text{a}\qquad |\uhol ABC|=|\uhol ACD|=|\uhol ADE|.
$$
Uhlopriečky $BD$ a~$CE$ sa pretínajú v~bode~$P$. Dokážte, že priamka~$AP$ prechádza stredom strany~$CD$.
%Zadanie bude zverejnené po IMO 2007.
}
\podpis{Peter Novotný:Shortlist 2006, G3}

{%%%%%   vyberko, den 3, priklad 3
Dané je racionálne číslo~$x$ z~intervalu $(0,1)$. Nech $y$ je také číslo z~intervalu $(0,1)$, ktorého $n$-tá cifra za desatinnou čiarkou sa rovná $2^n$-tej cifre za desatinnou čiarkou čísla~$x$. Dokážte, že aj $y$ je racionálne.
%Zadanie bude zverejnené po IMO 2007.
}
\podpis{Peter Novotný:Shortlist 2006, N2}

{%%%%%   vyberko, den 3, priklad 4
Na ulici je v~rade za sebou $n$~lámp $L_1,\dots,L_n$, pričom $n\ge2$. Každá z~nich buď svieti, alebo je zhasnutá. Každú sekundu naraz upravíme stav všetkých lámp podľa nasledujúcich pravidiel:
  \item ak lampa $L_i$ a~s~ňou susediace lampy sú v~rovnakom stave, tak $L_i$ bude po úprave zhasnutá;
  \item inak bude $L_i$ po úprave svietiť.

Na začiatku prvá lampa svieti a~všetky ostatné sú zhasnuté.
  \item{a)} Dokážte, že existuje nekonečne veľa prirodzených čísel~$n$, pre ktoré budú po istom čase všetky lampy naraz
    zhasnuté.
  \item{b)} Dokážte, že existuje nekonečne veľa prirodzených čísel~$n$, pre ktoré nebudú nikdy všetky lampy naraz zhasnuté.
%Zadanie bude zverejnené po IMO 2007.
}
\podpis{Peter Novotný:Shortlist 2006, C1}

{%%%%%   vyberko, den 4, priklad 1
Nech $ABC$ je rovnoramenný trojuholník a~$D$ je stred jeho základne~$BC$. Označme ďalej $M$ stred $AD$ a~$N$ pätu výšky z~bodu~$D$ na priamku~$BM$. Dokážte, že uhol $ANC$ je pravý.}
\podpis{Michal Takács:???}

{%%%%%   vyberko, den 4, priklad 2
Nech $d(n)$ označuje počet kladných deliteľov čísla~$n$. Nájdite všetky prirodzené $n$ také, že $n=d(n)^4$.}
\podpis{Michal Takács:Írsko 1999}

{%%%%%   vyberko, den 4, priklad 3
Nájdite všetky funkcie $f:\Bbb R\to\Bbb R$ také, že
$$
f(xf(x) + f(y)) = f(x)^2 + y
$$
pre všetky $x,y \in \Bbb R$.}
\podpis{Jakub Závodný:Baltická olympiáda, 2000}

{%%%%%   vyberko, den 4, priklad 4
Párny počet poslancov rokuje za okrúhlym stolom. Po krátkej prestávke si znova posadajú, inak ako predtým. Ukážte, že existujú dvaja poslanci, medzi ktorými sedí rovnaký počet poslancov ako pred prestávkou.}
\podpis{Jakub Závodný:Shortlist 1988}

{%%%%%   vyberko, den 5, priklad 1
Dokážte, že mnohouholník s~obsahom väčším ako $n$
sa dá vložiť do roviny tak, že zakryje aspoň $n+1$ mrežových
bodov.}
\podpis{Ondrej Budáč:???}

{%%%%%   vyberko, den 5, priklad 2
Dokážte, že súčin piatich po sebe idúcich
prirodzených čísel nemôže byť štvorec.}
\podpis{Ondrej Budáč:???}

{%%%%%   vyberko, den 5, priklad 3
Označme $c$ najväčší reálny koreň
$x^3-3x^2+1=0$. Dokážte, že $\lfloor c^{1988}\rfloor$ je deliteľné
$17$-timi.}
\podpis{Ondrej Budáč:???}

{%%%%%   vyberko, den 1, priklad 4
...}
\podpis{...}

{%%%%%   vyberko, den 2, priklad 4
...}
\podpis{...}

{%%%%%   vyberko, den 5, priklad 4
...}
\podpis{...}

{%%%%%   trojstretnutie, priklad 1
Nájdite všetky mnohočleny $P$ s~reálnymi koeficientmi, pre ktoré rovnosť
$$
P(x^2)=P(x)\cdot P(x+2)
$$
platí pre ľubovoľné reálne číslo~$x$.}
\podpis{Pavel Calábek}

{%%%%%   trojstretnutie, priklad 2
Nech $a_1=a_2=1$ a~$a_{k+2}=a_{k+1}+a_{k}$ pre každé $k\in\Bbb N$ (Fibonacciho postupnosť čísel). Dokážte, že pre
každé prirodzené číslo~$m$ existuje taký index~$k$, pre ktorý je číslo $a_k^4-a_k-2$ deliteľné číslom~$m$.}
\podpis{Ján Mazák}

{%%%%%   trojstretnutie, priklad 3
Nech $k$ je kružnica opísaná takému konvexnému štvoruholníku $ABCD$, že polpriamky $DA$ a~$CB$ sa pretínajú v~bode~$E$, pre ktorý platí $|CD|^2=|AD|\cdot |ED|$. Označme $F$ ($F\ne A$) priesečník kružnice~$k$ s~priamkou predchádzajúcou bodom~$A$ a~kolmou na~$ED$. Dokážte, že potom platí: Úsečky $AD$ a~$CF$ sú zhodné práve vtedy, keď stred kružnice~$l$ opísanej trojuholníku $ABE$ leží na priamke~$ED$.}
\podpis{Jaroslav Švrček}

{%%%%%   trojstretnutie, priklad 4
Dokážte, že pre každé reálne číslo $p\ge1$ možno z~množiny reálnych čísel~$x$ spĺňajúcich nerovnosti
$$
p<x<\biggl(2+\sqrt{p+\dfrac14}\biggr)^{\!2}
$$
vybrať štyri navzájom rôzne prirodzené čísla $a$, $b$, $c$, $d$, pre ktoré platí rovnosť $ab=cd$.}
\podpis{Jaromír Šimša}

{%%%%%   trojstretnutie, priklad 5
Zistite, pre ktoré
$$
n\in\{3\,900,3\,901,3\,902,3\,903,3\,904,3\,905,3\,906,3\,907,3\,908,3\,909\}
$$
možno množinu $\{1,2,3,\dots,n\}$ rozdeliť na disjunktné trojice tak, aby v~každej trojici sa jedno číslo rovnalo súčtu
ostatných dvoch čísel.}
\podpis{Peter Novotný}

{%%%%%   trojstretnutie, priklad 6
Nech $ABCD$ je konvexný štvoruholník. Kružnica prechádzajúca bodmi $A$ a~$D$ má vonkajší dotyk s~kružnicou predchádzajúcou bodmi $B$ a~$C$ vo vnútornom bode~$P$ uvažovaného štvoruholníka. Predpokladajme, že
$$
|\uhol PAB|+|\uhol PDC|\le 90^{\circ} \quad \hbox{a} \quad
       |\uhol PBA|+|\uhol PCD|\le 90^{\circ}.
$$
Dokážte, že potom platí $|AB|+|CD|\ge |BC|+|AD|$.}
\podpis{Waldemar Pompe}

{%%%%%   IMO, priklad 1
Dané sú reálne čísla $a_1, a_2, \dots, a_n$. Pre
každé~$i$ ($1 \le i \le n$) definujme
$$
d_i = \max \{ a_j : 1 \le j \le i\} - \min \{ a_j : i \le j \le n\}.
$$
Nech
$$
d = \max \{d_i : 1 \le i \le n\}.
$$
\ite (a) Dokážte, že pre ľubovoľné reálne čísla
$x_1 \le x_2 \le \cdots \le x_n$ platí nerovnosť
$$
\max \{ |x_i - a_i | : 1 \le i \le n \} \ge \frac d2. \tag{$\ast$}
$$
\ite (b) Ukážte, že existujú také reálne čísla $x_1 \le x_2 \le \cdots \leq
x_n$, že v~$(\ast)$ nastane rovnosť.}
\podpis{Nový Zéland}

{%%%%%   IMO, priklad 2
Uvažujme päť takých bodov $A$, $B$, $C$, $D$, $E$, že $ABCD$ je rovnobežník
a~štvoruholník $BCED$ je tetivový. Priamka~$l$ prechádza bodom $A$, pričom pretína
úsečku~$DC$ v~jej vnútornom bode~$F$ a~priamku~$BC$ v~bode~$G$. Predpokladajme, že
$|EF|=|EG|=|EC|$. Dokážte, že priamka~$l$ je osou uhla $DAB$.}
\podpis{Luxembursko}

{%%%%%   IMO, priklad 3
Niektorí účastníci matematickej súťaže sú priatelia. Priateľstvo je vzájomné. Skupinu súťažiacich nazveme {\it klika}, ak každí dvaja z~nich sú priatelia. (Špeciálne, ľubovoľná skupina pozostávajúca z~menej ako dvoch súťažiacich je klika.) Počet členov kliky nazveme jej {\it rozmerom}.

Vieme, že najväčší rozmer kliky pozostávajúcej z~účastníkov súťaže je párne číslo. Dokážte, že všetkých súťažiacich možno rozsadiť do dvoch miestností tak, aby najväčší rozmer kliky v~jednej miestnosti sa rovnal najväčšiemu rozmeru kliky v~druhej miestnosti.}
\podpis{Rusko}

{%%%%%   IMO, priklad 4
Os uhla $BCA$ trojuholníka $ABC$ pretína jeho opísanú kružnicu v~bode~$R$ rôznom od bodu~$C$,
os strany~$BC$ v~bode~$P$ a~os strany~$AC$ v~bode~$Q$. Stred strany~$BC$ označme~$K$
a~stred strany~$AC$ označme~$L$. Dokážte, že obsahy trojuholníkov
$RPK$ a~$RQL$ sa rovnajú.}
\podpis{Česká rep.}

{%%%%%   IMO, priklad 5
Kladné celé čísla $a$, $b$ sú také, že číslo $(4a^2 -1)^2$ je deliteľné $4ab-1$. Dokážte, že $a=b$.}
\podpis{Veľká Británia}

{%%%%%   IMO, priklad 6
Nech $n$ je kladné celé číslo. Uvažujme množinu
$$
S = \left\{ (x, y, z)\ :\ x, y, z \in \{0,1, \ldots, n\},\ x+y+z >0\right\}
$$
pozostávajúcu z~$(n+1)^3-1$ bodov trojrozmerného priestoru.
Určte najmenší možný počet rovín,
ktorých zjednotenie obsahuje všetky body z~$S$, ale neobsahuje bod $(0,0,0)$.}
\podpis{Holandsko}

{%%%%%   MEMO, priklad 1
Nech $a$, $b$, $c$, $d$ sú kladné reálne čísla, pričom $a+b+c+d=4$. Dokážte, že
$$
a^2bc+b^2cd+c^2da+d^2ab\le4.
$$}
\podpis{Švajčiarsko}

{%%%%%   MEMO, priklad 2
Množina loptičiek obsahuje $n$~loptičiek, ktoré sú označené číslami $1, 2, 3, \ldots, n$. Daných je $k>1$ takých množín. Chceme zafarbiť všetky loptičky dvoma farbami, čiernou a~bielou, a~to tak, aby
 \ite ($i$) loptičky označené rovnakým číslom mali rovnakú farbu,
 \ite ($ii$) každá množina $k+1$ loptičiek označených (nie nutne rôznymi) číslami $a_1, a_2, \ldots, a_{k+1}$, ktoré spĺňajú podmienku $a_1+a_2+\ldots+a_k=a_{k+1}$, obsahovala z~každej farby aspoň jednu loptičku.

\noindent
V~závislosti od $k$ nájdite najväčšie možné číslo~$n$, pre ktoré existuje takéto zafarbenie.}
\podpis{Slovinsko}

{%%%%%   MEMO, priklad 3
Nech $k$ je kružnica a~$k_1$, $k_2$, $k_3$, $k_4$ sú štyri menšie kružnice so stredmi $O_1$, $O_2$, $O_3$, $O_4$ ležiacimi na~$k$. Pre $i=1,2,3,4$ a~$k_5=k_1$ sa kružnice $k_i$ a~$k_{i+1}$ pretínajú v~bodoch $A_i$ a~$B_i$ tak, že $A_i$ leží na~$k$. Body $O_1$, $A_1$, $O_2$, $A_2$, $O_3$, $A_3$, $O_4$, $A_4$ ležia v~tomto poradí na~$k$ a~sú navzájom rôzne. Dokážte, že $B_1B_2B_3B_4$ je pravouholník.}
\podpis{Švajčiarsko}

{%%%%%   MEMO, priklad 4
Určte všetky dvojice $(x,y)$ kladných celých čísel spĺňajúcich rovnosť
$$
x!+y!=x^y.
$$}
\podpis{Česká rep.}

{%%%%%   MEMO, priklad t1
Nech $a$, $b$, $c$, $d$ sú ľubovoľné reálne čísla z~uzavretého intervalu $\left\langle\frac12,2\right\rangle$ spĺňajúce $abcd=1$. Nájdite maximálnu hodnotu výrazu
$$
\left(a+\frac1b\right)\left(b+\frac1c\right)\left(c+\frac1d\right)\left(d+\frac1a\right).
$$}
\podpis{Česká rep.}

{%%%%%   MEMO, priklad t2
Pre ľubovoľnú množinu~$P$ piatich bodov v~rovine vo všeobecnej polohe označíme $a(P)$ počet ostrouhlých trojuholníkov s~vrcholmi v~$P$ (body sú vo všeobecnej polohe, ak sú navzájom rôzne a~žiadne tri z~nich neležia na jednej priamke). Určte najväčšiu možnú hodnotu $a(P)$.}
\podpis{Švajčiarsko}

{%%%%%   MEMO, priklad t3
Nech $s(T)$ označuje súčet dĺžok hrán štvorstena~$T$. Uvažujme štvorsteny s~vlastnosťou, že dĺžky ich šiestich hrán sú navzájom rôzne kladné celé čísla, pričom jedno je $2$ a~jedno je~$3$. Nazvime ich MEMO-štvorstenmi.
    \ite a) Nájdite všetky kladné celé čísla~$n$, pre ktoré existuje MEMO-štvorsten~$T$ taký, že $s(T) = n$.
    \ite b) Koľko existuje navzájom nezhodných MEMO-štvorstenov~$T$ takých, že $s(T)=2007$?

\noindent
Dva štvorsteny sú nezhodné, ak jeden nemôže byť zobrazený na druhý pomocou súmerností podľa rovín, posunutí a~otočení.

(Nie je potrebné dokázať, že štvorsteny uvažované v~riešení sú nedegenerované, \tj. že majú nenulový objem.)}
\podpis{Rakúsko}

{%%%%%   MEMO, priklad t4
Určte všetky kladné celé čísla~$k$ s~nasledujúcou vlastnosťou: existuje celé číslo $a$ také, že $(a+k)^3-a^3$ je násobkom čísla $2007$.}
\podpis{Rakúsko}

