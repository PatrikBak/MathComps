{%%%%%   A-I-1
Označme hľadané korene $x_1$, $x_2$, $x_3$, $x_4$ tak, aby
platilo $x_1+x_2=1$. Potom
$$
4x^4-12x^3-7x^2+22x+14=4(x-x_1)(x-x_2)(x-x_3)(x-x_4).
$$
Porovnaním koeficientov pri zodpovedajúcich mocninách~$x$
dostaneme známe Vi\`etove vzťahy
$$
\align
x_1+x_2+x_3+x_4&=3,\vphantom{\frac74}\tag1\\
x_1x_2+x_1x_3+x_1x_4+x_2x_3+x_2x_4+x_3x_4&=-\frac74,\tag2\\
x_1x_2x_3+x_1x_2x_4+x_1x_3x_4+x_2x_3x_4&=-\frac{11}2,\tag3\\
x_1x_2x_3x_4&=\frac72.\tag4
\endalign
$$
Keďže $x_1+x_2=1$, z~\thetag1 vyplýva $x_3+x_4=2$. Rovnice \thetag2 a~\thetag3 prepíšeme na
tvar
$$
\eqalign{
(x_1+x_2)(x_3+x_4)+x_1x_2+x_3x_4&=-\frac74,\cr
(x_1+x_2)x_3x_4+(x_3+x_4)x_1x_2&=-\frac{11}2,}
$$
čo po dosadení hodnôt $x_1+x_2=1$ a~$x_3+x_4=2$ dáva
$$
\eqalign{
x_1x_2+x_3x_4&=-\frac{15}4,\cr
2x_1x_2+x_3x_4&=-\frac{11}2.}
$$
Z~tejto sústavy dvoch lineárnych rovníc už ľahko dostaneme
$$
x_1x_2=-\frac74,\qquad x_3x_4=-2.
$$

Všimnime si, že pre tieto hodnoty súčinov $x_1x_2$ a~$x_3x_4$ je
splnená aj rovnica~\thetag4, ktorú sme zatiaľ nevyužili. Z~podmienok $x_1+x_2=1$,
$x_1x_2=\m\frac74$ vyplýva, že $x_1$ a~$x_2$ sú korene
kvadratickej rovnice
$$
x^2-x-\frac74=0,\qquad\text{teda}\quad x_{1,2}=\frac12\pm\sqrt2.
$$
Podobne z~podmienok $x_3+x_4=2$ a~$x_3x_4=-2$ dostaneme
$$
x_{3,4}=1\pm\sqrt3.
$$

Skúšku nájdených koreňov netreba robiť, pretože je
splnená, ako sme zdôraznili, celá sústava rovníc \thetag1 až~\thetag4.

\zaver
Daná rovnica má korene $\frac12+\sqrt2$, $\frac12-\sqrt2$,
$1+\sqrt3$, $1-\sqrt3$.


\ineriesenie
Z~podmienok úlohy vyplýva, že ľavá strana rovnice je súčinom
mnohočlenov
$$
x^2-x+p\qquad\text{a}\qquad 4x^2+qx+r,
$$
pričom $p$, $q$ a~$r$ sú reálne čísla. Po ich vynásobení
a~porovnaní koeficientov pri zodpovedajúcich mocninách~$x$ dostaneme
sústavu štyroch rovníc o~troch neznámych
$$
\eqalign{
r-4&=-12,\cr
4p+q-r&=-7,\cr
pr-q&=22,\cr
pq&=14.}
$$
Prvé tri rovnice majú jediné riešenie $r=\m8$, $p=\m\frac74$
a~$q=\m8$, ktoré vyhovuje aj štvrtej rovnici. Platí teda rozklad
$$
4x^4-12x^3-7x^2+22x+14=\left(x^2-x-\tfrac74\right)(4x^2-8x-8).
$$
Rovnica $x^2-x-\frac74=0$ má korene $\frac12\pm\sqrt2$, rovnica
$4x^2-8x-8=0$ má korene $1\pm\sqrt3$.


\návody
Uvažujme rovnicu
$$
x^5-9x^4+kx^3-3x^2-\tfrac{92}3x+20=0,
$$
pričom $k$ je reálny parameter. Určte všetky jej korene a~hodnotu
parametra~$k$, ak viete, že táto rovnica má aspoň dva reálne
korene, ktoré sa líšia iba znamienkom. [Využite Vi\`etove
vzťahy, korene sú $\pm2\frac{\sqrt3}3,1,3,5$, $k=\frac{65}3$.]

Nech $P$, $Q$ sú také kvadratické mnohočleny, že čísla
$\m22$, 7, 13 sú tri z~koreňov rovnice $P(Q(x))=0$. Určte štvrtý
koreň tejto rovnice. [49--A--I--1]

Nech $P$ je kvadratický trojčlen. Určte všetky korene
rovnice
$$
P(x^2+4x-7)=0,
$$
ak viete, že medzi nimi je číslo~1 a~aspoň jeden koreň je
dvojnásobný. [49--A--II--1]

\endnávod
}

{%%%%%   A-I-2
\epsplace a56.1 \hfil\Obr

Označme $k$ kružnicu vpísanú do trojuholníka $ABC$ a~$S$ jej
stred. Veľkosti vnútorných uhlov trojuholníka $ABC$ označme
zvyčajným spôsobom $\alpha$, $\beta$, $\gamma$. Keďže body $K$,
$L$ sú súmerne združené podľa osi vnútorného uhla pri
vrchole~$C$, sú priamky $KL$ a~$CP$ na seba kolmé a~$|\uhol LPC|=|\uhol KPC|$ (\obr).

\inspicture{}

Keď vyjadríme veľkosti vnútorných uhlov pri základniach $KM$ a~$LK$
v~rovnoramenných trojuholníkoch $KMB$ a~$LKC$, dostaneme $|\uhol
MKB|=90^{\circ}-\beta/2$, $|\uhol
LKC|=90^{\circ}-\gamma/2$.  Z~priamosti uhla $BKC$ tak vyplýva
$|\uhol MKL|=90^{\circ}-\alpha/2$. Analogicky vyjde
$|\uhol KLM|=90^{\circ}-\beta/2$, $|\uhol
LMK|=90^{\circ}-\gamma/2$.

Keďže $|\uhol KPC|+\gamma/2=|\uhol BKP|=90\st-\beta/2$,
dostaneme pre veľkosť súmerne združených uhlov $LPC$ a~$KPC$ rovnosť
$$
|\uhol LPC|=|\uhol KPC|=90^{\circ}-\frac{\beta+\gamma}2=\frac{\alpha}2.
$$

Kružnica~$k$ vpísaná do trojuholníka $ABC$ je súčasne kružnicou
opísanou trojuholníku $KLM$, ktorý je -- ako sme zistili výpočtom
jeho uhlov -- ostrouhlý. Jej stred~$S$ je preto vnútorným bodom
tohto trojuholníka, a~teda aj vnútorným bodom úsečky~$CP$.
Keďže
$$
|\uhol LPC|=|\uhol LPS|=|\uhol LAS|=\frac{\alpha}2,
$$
$APSL$ je tetivový štvoruholník. Vzhľadom na to, že uhol $ALS$ je
pravý, je aj uhol $APS$ pravý (priamky $AP$ a~$CP$ sú na seba
kolmé). Preto sú priamky $KL$ a~$AP$ rovnobežné, čo bolo treba dokázať.

\poznamka
Keďže kružnica~$k$ je opísaná trojuholníku $KLM$, môžeme
jeho vnútorné uhly ľahko vyjadriť z~príslušných stredových uhlov:
$|\uh KSL|=180\st-\gamma$, takže $|\uh KML|=90\st-\gamma/2$, atď.


\návody
V~rovine je daný štvorec $ABCD$. Vnútri jeho strán $BC$, $CD$ sú
postupne zvolené body $P$, $Q$ také, že $|\uhol
PAQ|=45^{\circ}$. Označme ďalej $R$, $S$ priesečníky jeho
uhlopriečky~$BD$ postupne s~priamkami $AP$, $AQ$. Dokážte, že body
$P$, $Q$, $R$, $S$ ležia na jednej kružnici. [Ukážte, že uhly $PSQ$
a~$PRQ$ sú pravé.]

V~rovine je daný pravouhlý lichobežník $ABCD$ s~dlhšou
základňou~$AB$ a~pravým uhlom pri vrchole~$A$. Označme $k_1$
kružnicu zostrojenú nad priemerom~$AD$ a~$k_2$ kružnicu
prechádzajúcu vrcholmi $B$, $C$ a~dotýkajúcu sa priamky~$AB$.
Ak majú kružnice $k_1$, $k_2$ vonkajší dotyk v~bode~$P$, je
priamka~$BC$ dotyčnicou kružnice opísanej trojuholníku $CDP$. Dokážte.
[52--B--II--4]

Nech $L$ je ľubovoľný vnútorný bod kratšieho oblúka~$CD$ kružnice
opísanej štvorcu $ABCD$. Označme $K$ priesečník priamok $AL$
a~$CD$, $M$ priesečník priamok $AD$ a~$CL$ a~napokon $N$ priesečník priamok
$MK$ a~$BC$. Dokážte, že body $B$, $L$, $M$, $N$ ležia na jednej
kružnici. [53--A--III--5]

\endnávod
}

{%%%%%   A-I-3
Pre ľubovoľné reálne čísla $x,y,z\in \langle \m1,1 \rangle$ platí
$1-x^2\ge 0$, $1-y^2\ge 0$, $1-z^2\ge 0$. Použitím nerovnosti
medzi aritmetickým a~geometrickým priemerom pre trojicu nezáporných
reálnych čísel $1-x^2$, $1-y^2$, $1-z^2$ tak dostaneme
$$
\align
 \root 3 \of{(1-x^2)(1-y^2)(1-z^2)}\le&\frac{(1-x^2)+(1-y^2)+(1-z^2)}3=\cr
 =&\frac{3-(x^2+y^2+z^2)}3,
\endalign
$$
takže
$$
6\root 3 \of{(1-x^2)(1-y^2)(1-z^2)}\le 6-2(x^2+y^2+z^2).
\tag1
$$

Ak reálne čísla $x,y,z\in \langle -1,1 \rangle$ vyhovujú podmienke
$xy+yz+zx=1$, ukážeme, že spĺňajú aj nerovnosť
$$
6-2(x^2+y^2+z^2)\le 1+(x+y+z)^2.
\tag2
$$
Pravú stranu tejto nerovnosti upravíme na tvar
$$
1+x^2+y^2+z^2+2(xy+yz+zx)=3+(x^2+y^2+z^2),
$$
čo po dosadení do \thetag2 vedie k~ekvivalentnej nerovnosti
$$
x^2+y^2+z^2\ge 1.
$$
Jej platnosť overíme ľahko. Stačí totiž dokázať, že pre reálne čísla
$x$, $y$, $z$, ktoré vyhovujú podmienkam úlohy, platí nerovnosť
$$
x^2+y^2+z^2\ge xy+yz+zx,
$$
čo je však ekvivalentné s~nerovnosťou
$$
(x-y)^2+(y-z)^2+(z-x)^2\ge 0,
% \eqno{(7)}
$$
ktorá platí pre všetky reálne čísla $x$, $y$, $z$.

\zaver
Nerovnosť, ktorú sme mali dokázať, vyplýva z~dokázaných
nerovností \thetag1 a~\thetag2. Rovnosť v~nej pritom nastane práve vtedy, keď
nastane súčasne v~oboch spomenutých nerovnostiach. To nastane práve vtedy,
keď $x=y=z$, čo vzhľadom na podmienku $xy+yz+zx=1$ dáva iba
dve možnosti $x=y=z=\pm\frac13\sqrt{3}$, pre ktoré v~dokázanej
nerovnosti platí rovnosť.

% $(x,y,z)$ reálných čísel, a~to
% $(\frac13\sqrt{3},\frac13\sqrt{3},\frac13\sqrt{3})$
% a~$(-\frac13\sqrt{3},-\frac13\sqrt{3},-\frac13\sqrt{3})$.


\návody
Pre ľubovoľné reálne čísla $a$, $b$, $c$ platí nerovnosť
$$
3(a^2+b^2+c^2)\ge (a+b+c)^2.
$$
Dokážte a~zistite, kedy nastane rovnosť. [Upravte danú nerovnosť
tak, aby na jednej strane bol súčet troch druhých mocnín
reálnych čísel a~na druhej strane nula.]

Dokážte, že pre ľubovoľné tri nezáporné čísla $x$, $y$, $z$
platí nerovnosť
$$
x(x-\sqrt{yz})+y(y-\sqrt{zx})+z(z-\sqrt{xy})\ge 0.
$$
Zistite, v~ktorých prípadoch nastane rovnosť. [17--A--II--2]

Dokážte, že pre ľubovoľné tri nezáporné čísla $x$, $y$, $z$
platí nerovnosť
$$
(x^2+x+1)(y^2+y+1)(z^2+z+1)\ge 27xyz.
$$
[Pre každý z~činiteľov na ľavej strane nerovnosti použite
nerovnosť medzi aritmetickým a~geometrickým priemerom trojice
nezáporných čísel.]

Dokážte, že pre ľubovoľné reálne čísla $a$, $b$, $c$ z~intervalu
$\langle 0,1 \rangle$ platí
$$
1\le a+b+c+2(ab+bc+ca)+3(1-a)(1-b)(1-c)\le 9.
$$
[55--B--II--4]

\endnávod
}

{%%%%%   A-I-4
a) Označme $\mm A$ a~$\mm B$ hľadané podmnožiny. Keďže obe majú
rovnaký počet prvkov, je počet prvkov množiny~$\mm M$ nutne párny.
Teda $n=2k$, pričom $k$ je vhodné prirodzené číslo.

Pre $n=4$ neexistuje rozklad množiny $\mm M=\{1,2,3,4\}$ na dve
podmnožiny daných vlastností, pretože aritmetický priemer
ľubovoľných dvoch rôznych čísel z~množiny~$\mm M$ sa nemôže rovnať
žiadnemu z~týchto čísel. Zostrojme vyhovujúci rozklad množiny~$\mm M$
pre niekoľko prvých párnych čísel~$n$ (aritmetický priemer prvkov
podmnožín vyznačíme tučne).

\medskip
{\tabskip 0pt plus\hsize
\halign to\hsize{$n=#$:\hfil
        &$\mm A=\{#\}$\hfil &$\mm B=\{#\}$\hfil\cr
2&{\text{\bf 1}}&{\text{\bf 2}}\cr
4&\multispan{2}rozklad neexistuje\hfil\cr
6&1,{\text{\bf 2}},3&4,{\text{\bf 5}},6\cr
8&2,3,{\text{\bf 4}},7&1,{\text{\bf 5}},6,8\cr
10&1,2,{\text{\bf 3}},4,5&6,7,{\text{\bf 8}},9,10\cr
12&1,2,3,{\text{\bf 4}},6,8&5,7,{\text{\bf 9}},10,11,12\cr
}}

\medskip
Teraz ukážeme, že hľadaný rozklad množiny~$\mm M$ existuje pre
ľubovoľné $n=2k$ také, že $k\ne2$.

Pre nepárne čísla~$k$ vyhovuje napríklad rozklad
množiny~$\mm M$ na podmnožiny
$$
\mm A=\{1,2,\dots,k\},\qquad \mm B=\{k+1,k+2,\dots,2k\}.
$$
Súčet všetkých prvkov množiny~$\mm A$ je $\frac12k(k+1)$, ich
aritmetický priemer je $\frac12(k+1)$, čo je prirodzené číslo.
Keďže $1\le \frac12(k+1)\le k$, aritmetický priemer všetkých
prvkov množiny~$\mm A$ je prvkom množiny~$\mm A$. Podobne aritmetický priemer
$\frac12(3k+1)$ všetkých prvkov množiny~$\mm B$ je prvkom množiny~$\mm B$.

Pre $k=4$ sme existenciu rozkladu ukázali v~tabuľke, pre párne
čísla $k\ge 6$ vyhovuje napríklad rozklad množiny~$\mm M$ na podmnožiny
$$
\textstyle \mm A=\{1,2,\dots,k-2,k,\frac12(3k-2)\},\qquad
\mm B=\mm M\setminus\mm A.
$$
Platí $k<\frac12(3k-2)\le2k$ a~$\frac12(3k-2)$ je
prirodzené číslo. Množina~$\mm A$ teda obsahuje $k$ prirodzených čísel
z~množiny~$\mm M$. Súčet všetkých prvkov množiny~$\mm A$ je
$$
\textstyle
1+2+\dots+(k-2)+k+\frac12(3k-2)=\frac12(k-2)(k-1)+k+\frac12(3k-2)\frac12k(k+2).
$$
Ich aritmetický priemer je $\frac12(k+2)$, čo je
prirodzené číslo. Keďže $1\le\frac12(k+2)\le k-2$,
aritmetický priemer všetkých prvkov množiny~$\mm A$ je prvkom množiny~$\mm A$.
Podobne ukážeme, že aritmetický priemer $\frac32 k$ všetkých prvkov
množiny~$\mm B$ je prvkom množiny $\mm B$.

\poznamka
Pre párne~$k$ nevyhovuje napríklad rozklad množiny~$\mm M$ na podmnožiny
$$
\textstyle \mm A=\{1,2,\dots,k-1,\frac32 k\},\qquad
\mm B=\mm M\setminus\mm A,
$$
pretože priemer $\frac32 k$ všetkých prvkov množiny~$\mm B$ je prvkom množiny~$\mm A$.

\medskip
b) Označme $\mm A$, $\mm B$ a~$\mm C$ hľadané podmnožiny množiny~$\mm M$. Keďže
všetky majú rovnaký počet prvkov, je číslo~$n$ nutne deliteľné
tromi, má teda tvar $n=3k$, pričom $k$ je vhodné prirodzené
číslo. Pre súčet~$s$ všetkých prvkov množiny~$\mm M$ platí $s=\frac12
3k(3k+1)$. Súčet troch aritmetických priemerov všetkých prvkov
jednotlivých množín $\mm A$, $\mm B$ a~$\mm C$ je potom rovný $\frc sk$, teda
$\frac32 (3k+1)$. Tento súčet musí byť podľa podmienok úlohy
prirodzené číslo, preto je $k$ nutne nepárne.

Pre čísla $n=3k$, pričom $k$ je nepárne, ukážeme, že zadaniu vyhovuje
napríklad rozklad množiny~$\mm M$ na podmnožiny
$$
\mm A=\{1,2,\dots,k\},\quad  \mm B=\{k+1,k+2,\dots,2k\}\quad
\text{a}\quad   \mm C=\{2k+1,2k+2,\dots,3k\}.
$$
Súčet všetkých prvkov~$\mm A$ je $\frac12k(k+1)$, ich aritmetický priemer je
$\frac12(k+1)$, čo je prirodzené číslo. Keďže $1\le\frac12(k+1)\le
k$, aritmetický priemer všetkých prvkov množiny~$\mm A$ je prvkom množiny~$\mm A$.
Podobne ukážeme, že aritmetický priemer $\frac12(3k+1)$ všetkých
prvkov množiny~$\mm B$ je prvkom množiny~$\mm B$ a~aritmetický priemer
$\frac12(5k+1)$ všetkých prvkov množiny~$\mm C$ je prvkom množiny~$\mm C$.

\zaver
Podmienkam úlohy v~prípade~a) vyhovujú všetky párne čísla~$n$
rôzne od~4, v~prípade~b) všetky nepárne čísla~$n$ deliteľné tromi.


\návody
Na stole leží $k$~kôpok s~$1,2,3,\dots,k$ kameňmi, pričom $k\ge 3$.
V~každom kroku vyberieme tri ľubovoľné kôpky na stole,
zlúčime ich do jednej a~pridáme k~nej jeden kameň, ktorý na stole
doposiaľ neležal. Ak po niekoľkých krokoch vznikne jediná
kôpka, nie je výsledný počet kameňov deliteľný tromi. Dokážte.
[54--B--I--3]

Na stole leží 54~kôpok s~$1,2,3,\dots,54$ kameňmi. V~každom
kroku vyberieme ľubovoľnú kôpku, povedzme s~$k$~kameňmi,  
a~odoberieme ju celú zo stola spolu s~$k$~kameňmi z~každej tej kôpky,
v~ktorej je aspoň $k$~kameňov. Napríklad po prvom kroku, pri ktorom
vyberieme kôpku s~52~kameňmi, zostanú na stole kôpky s~$1,2,3,\dots,51,1$ a~2~kameňmi.
Predpokladajme, že po určitom
počte krokov zostane na stole jediná kôpka. Zdôvodnite, koľko
kameňov v~nej môže byť. [54--B--S--1]

Rozhodnite, či je možné rozložiť množinu čísel
$\{1,2,\dots,1995\}$ na dve podmnožiny tak, aby v~prvej
podmnožine bolo a)~dvakrát, b)~trikrát, c)~štyrikrát viac čísel ako
v~druhej a~aby súčty čísel v~oboch podmnožinách boli rovnaké.
[45--C--I--2]

Určte, pre ktoré prirodzené čísla~$n$ je možné rozdeliť množinu
$\{1,2,\dots,n\}$ na dve podmnožiny tak, aby v~prvej bolo trikrát
viac čísel ako v~druhej a~aby súčty všetkých čísel v~oboch
podmnožinách boli rovnaké. [45--C--II--1]

\endnávod
}

{%%%%%   A-I-5
\epsplace a56.2 \hfil\Obr \par
\epsplace a56.3 \hfil\Obr

Polomer danej kružnice~$k$ označme~$r$. Ak bod~$A$ leží
na kružnici~$k$, je bod~$S$ stredom každej kružnice opísanej niektorému
z~uvažovaných trojuholníkov $ABC$ a~hľadanou množinou je
jednobodová množina~$\{S\}$. Ďalej rozlíšime dva prípady:

\smallskip
a) Nech $|AS|>r$. Uvažujme najskôr rovnoramenný trojuholník
$ABC$ so základňou~$BC$, ktorý vyhovuje podmienkam úlohy. Stred~$O$
kružnice jemu opísanej je vnútorným bodom úsečky~$AS$ a~pritom
platí $|AO|=|BO|=|CO|$.

Teraz ukážeme, že hľadanou množinou~$\mm O$ stredov kružníc opísaných
všetkým trojuholníkom $ABC$, ktoré vyhovujú podmienkam úlohy, je
priamka~$p$, ktorá je kolmá na $AS$ a~prechádza bodom~$O$ (\obr).

\inspicture{}

Uvažujme ľubovoľný trojuholník $AB'C'$, pričom $B'C'$ je priemer
kružnice~$k$, a~označme $O'$ priesečník osi jeho strany~$B'C'$
s~priamkou~$p$, takže $|O'B'|=|O'C'|$ (bod~$O'$ leží na
osi~$B'C'$). Podľa Pytagorovej vety v~pravouhlom trojuholníku
$C'O'S$ platí
$$
|O'B'|=|O'C'|=\sqrt{|O'S|^2+r^2}=\sqrt{|OO'|^2+|OS|^2+r^2}.
$$
Pre veľkosť úsečky~$O'A$ pritom máme
$$
|O'A|=\sqrt{|AO|^2+|OO'|^2}=\sqrt{|BO|^2+|OO'|^2}\sqrt{|OS|^2+r^2+|OO'|^2}.
$$
Odtiaľ $|O'A|=|O'B'|=|O'C'|$, čiže bod~$O'$ je stredom kružnice
opísanej trojuholníku $AB'C'$ a~podľa konštrukcie leží na priamke~$p$.

Naopak, pre ľubovoľný bod~$O'$ priamky~$p$ možno zostrojiť priemer~$B'C'$
kružnice~$k$, ktorý je kolmý na priamku~$O'S$. Z~predchádzajúcich
úvah vyplýva, že $|O'A|=|O'B'|=|O'C'|$, takže sme našli
trojuholník~$AB'C'$ s~požadovanými vlastnosťami, ktorého opísaná kružnica
má stred~$O'$.

\smallskip
b) Nech $|AS|<r$. V~tomto prípade možno postupovať analogicky.
Stred~$O$ je teraz vnútorným bodom polpriamky opačnej k~polpriamke~$SA$.
Dostaneme pritom rovnaký výsledok ako v~prípade~a).

\zaver
Ak bod~$A$ nie je bodom kružnice~$k$, hľadanou množinou~$\mm O$
je priamka~$p$, ktorá je kolmá na $AS$ a~súčasne prechádza
stredom~$O$ kružnice opísanej rovnoramennému trojuholníku $ABC$ so
základňou~$BC$, ktorá je priemerom kružnice~$k$ kolmým na $AS$.
Ak $A$ je bodom kružnice~$k$, tak $\mm O=\{S\}$.

\ineriesenie
Pre daný bod~$A$, ktorý neleží na kružnici~$k$, uvažujme
trojuholník $ABC$ s~danými vlastnosťami. Označme~$l$ kružnicu
opísanú trojuholníku $ABC$ (\obr). Keďže bod~$S$ je stredom
spoločnej tetivy~$BC$ kružníc $k$ a~$l$, pretne kružnica~$l$
polpriamku opačnú k~polpriamke~$SA$ vo vnútornom bode, ktorý
označíme~$A'$. Pre mocnosť $m_l(S)$ bodu~$S$ ku kružnici~$l$
pritom platí
$$
m_l(S)=-|BS|\cdot |CS|=-r^2=-|AS|\cdot |A'S|,
\tag1
$$
pričom $r$ je polomer kružnice~$k$. Odtiaľ vyplýva, že vzdialenosť
$|A'S|$, a~teda aj poloha bodu~$A'$ na polpriamke opačnej k~$SA$,
sú jednoznačne určené polohou bodu~$A$. Pre všetky
trojuholníky~$ABC$ vyhovujúce podmienkam úlohy je teda $AA'$ pevná
úsečka. Kružnice opísané všetkým uvažovaným trojuholníkom~$ABC$ preto
majú spoločnú tetivu~$AA'$, takže ich stredy ležia na osi~$p$
úsečky~$AA'$. V~prípade, že $ABC$ je rovnoramenný trojuholník so
základňou~$BC$, je úsečka~$AA'$ priemerom kružnice~$l$ a~jej
stred~$O$ je súčasne stredom úsečky~$AA'$. Priamka~$p$ prechádza
týmto bodom~$O$ kolmo na priamku~$AS$.

\inspicture{}

Naopak, ku každému bodu~$O'$ priamky~$p$ nájdeme trojuholník $ABC$
s~požadovanými vlastnosťami, ktorý má stred opísanej kružnice v~bode~$O'$.
Stačí zostrojiť priemer~$BC$ kružnice~$k$, ktorý je kolmý
na priamku~$O'S$. Pre pevne uvažované body $A$, $A'$ a~$S$ sme tak
zostrojili body $B$, $C$, pre ktoré platí vzťah~\thetag1. To znamená,
že body $A$, $B$, $C$ a~$A'$ ležia na jednej kružnici~$l$.
Vzhľadom na to, že bod~$O'$ je priesečníkom osí tetív $AA'$ a~$BC$
tejto kružnice, ktoré nie sú rovnobežné, je bod~$O'$ stredom
kružnice~$l$, teda stredom kružnice opísanej trojuholníku $ABC$.

\návody
V~rovine je daný štvorec $ABCD$. Uvažujme štvorec
  $KLMN$, ktorého uhlopriečka je zhodná so stranou štvorca $ABCD$
  a~jeho vrcholy $K$ a~$M$ ležia na stranách štvorca $ABCD$. Určte
  množinu vrcholov~$L$ všetkých takých štvorcov $KLMN$.
  [19--B--I--5]

V~rovine je daná priamka~$q$ a~bod~$A$, ktorý na nej neleží.
  Určte v~tejto rovine množinu stredov~$S$ všetkých štvorcov $ABCD$
  takých, že bod~$B$ leží na priamke~$q$. [47--B--I--2]

V~rovine je daná úsečka~$AB$. Zostrojte množinu ťažísk všetkých
  ostrouhlých trojuholníkov~$ABC$, pre ktoré platí: Vrcholy $A$ a~$B$,
  priesečník výšok~$V$ a~stred~$S$ kružnice vpísanej do trojuholníka $ABC$
  ležia na jednej kružnici. [55--A--III--4]

\endnávod
}

{%%%%%   A-I-6
Nech $f$ je ľubovoľná funkcia s~požadovanými vlastnosťami. Najskôr
ukážeme, že je prostá. Predpokladajme, že existujú dve celé
čísla $x_1$ a~$x_2$, pre ktoré $f(x_1)=f(x_2)$. Potom pre
všetky celé čísla~$y$ platí
$$
\postdisplaypenalty 10000
x_1+f(y+2006)=f\bigl(f(x_1)+y\bigr)=f\bigl(f(x_2)+y\bigr)x_2+f(y+2006).
$$
Preto $x_1=x_2$, funkcia~$f$ je teda prostá.

Voľbou $x=0$ v~danej rovnici dostaneme
$$
f\bigl(f(0)+y\bigr)=f(y+2006),
$$
odkiaľ vzhľadom na to, že funkcia~$f$ je prostá, vyplýva
$f(0)+y=y+2006$, čiže $f(0)=2006$.

Ak daný vzťah platí pre všetky celé čísla $x$, $y$,
platí aj pre $y=0$. Takže
$$
f\bigl(f(x)\bigr)=x+f(2006).
$$

Položme v~tejto rovnosti $x=z$, pričom $z$ je ľubovoľné celé číslo,
pripočítajme potom k~obom stranám~$y$ a~aplikujme na ne funkciu~$f$.
Dostaneme
$$
f\bigl(y+z+f(2006)\bigr)=f\bigl(f(f(z))+y\bigr)=f(z)+f(y+2006),
$$
pričom sme využili rovnosť zo zadania pre $x=f(z)$.
Ak v~odvodenom vzťahu zameníme dvojicu $(y,z)$ dvojicou
$(y+1,z-1)$, dostaneme
$$
f\bigl(y+z+f(2006)\bigr)=f\bigl((y+1)+(z-1)+f(2006)\bigr)
=f(z-1)+f(y+2007).
$$
Preto pre ľubovoľné celé čísla $y$ a~$z$ platí
$$
f(z)+f(y+2006)=f(z-1)+f(y+2007),
$$
čiže
$$
f(z)-f(z-1)=f(y+2007)-f(y+2006).
$$
Položme teraz v~poslednej rovnosti $y=0$ a~označme
$d=f(2007)-f(2006)$, čo je nutne celé číslo. Z~predchádzajúceho
vzťahu vyplýva, že pre každé celé číslo~$z$ platí
$$
f(z)-f(z-1)=d.     
\tag1
$$

Vzťah~\thetag1 hovorí, že pre celé nezáporné čísla~$z$ tvoria hodnoty~$f(z)$
aritmetickú postupnosť s~diferenciou~$d$, takže $f(z)=f(0)+dz$.

Ostatný vzťah platí aj pre záporné čísla~$z$, čo možno z~\thetag1 ľahko
odvodiť matematickou indukciou. Keďže $f(0)=2006$, pre všetky
celé čísla~$z$ nutne platí
$$
f(z)=2006+dz.          
\tag2
$$

Teraz zistíme, ktoré funkcie tvaru \thetag2 vyhovujú zadaniu úlohy.
Pre všetky celé čísla $x$ a~$y$ musí platiť
$$
\align
2006+d(2006+dx+y)=&f\bigl(2006+dx+y\bigr)
    =f\bigl(f(x)+y\bigr)=x+f(y+2006)=\cr
    =&x+2006+d(y+2006).
\endalign
$$
Oba krajné výrazy sa rovnajú práve vtedy, keď pre všetky celé čísla~$x$ platí
$$
\postdisplaypenalty 10000
d^2x=x.
$$
Odtiaľ $d=1$ alebo $d=\m1$.

\zaver
Danej úlohe vyhovujú iba dve funkcie, a~to
$$
f_1(x)=2006-x\qquad\text{a}\qquad f_2(x)=2006+x.
$$


\návody
Nech $f\colon \Bbb N\to\Bbb N$ je ľubovoľná funkcia spĺňajúca nerovnosť
$$
f(n)+f(n+2)\le2f(n+1)
$$
pre každé prirodzené číslo~$n$. Ukážte, že potom v~rovine existuje
priamka, na ktorej leží nekonečne veľa bodov s~karteziánskymi súradnicami
$[k,f(k)]$, $k\in\Bbb N$. [43--A--III--1]

Nájdite všetky funkcie $f\colon\Bbb Z\to\Bbb Z$ také, že
$$
f(x)+f(y)=f(x+2xy)+f(y-2xy)
$$
platí pre každé $x$, $y$ celé a~navyše $f(\m1)=f(1)$. [42--A--3--5]

Uvažujme funkciu $f\colon\Bbb N\to\Bbb N$, ktorá je rýdzo rastúca a~pre
každé dve prirodzené čísla $m$, $n$ spĺňa rovnosť
$$
f(mn)=f(m)f(n).
$$
Určte $f(30)$, ak viete, že $f(2)=4$. [40--A--2--4]

Nech $f$ je zobrazenie množiny $\{1,2,\dots,1988\}$ do seba. Pre
ľubovoľné prirodzené číslo~$n$ položme $x_1=f(1)$,
$x_{n+1}=f(x_n)$. Zistite, či existuje také číslo~$m$, že
platí $x_m=x_{2m}$. [37--A--III--1]

\endnávod
}

{%%%%%   B-I-1
Rovnicu riešime ako kvadratickú s~neznámou~$a$ a~parametrom~$b$. Jej diskriminant je 
$$
D=(7b+5)^2-4(6b^2+4b+3)=25b^2+54b+13
$$
a~korene
$$
a_{1,2}=\frac {-7b-5\pm \sqrt D}2.
$$
Ak sú $a$ aj $b$ celé čísla, musí byť aj $\sqrt D=\pm (2a+7b+5)$ celé číslo. Môžeme teda písať
$$
D=25b^2+54b+13=c^2,
$$
pričom $c$ je celé nezáporné. Rovnicu
$$
25b^2+54b+13-c^2=0
$$
opäť riešime ako kvadratickú. Jej korene sú 
$$
b_{1,2}=\frac {-27\pm \sqrt {27^2-25\cdot 13+25c^2}}{25}.
$$
Ak sú $b$ a~$c$ celé čísla, musí byť $\sqrt {404+25c^2}$
druhou mocninou nejakého celého nezáporného čísla~$d$. Pre celé nezáporné čísla $c$, $d$ teda platí $d^2-25c^2=404$, čiže 
$$
(d+5c)(d-5c)=404.
$$
Rozdiel $(d+5c)-(d-5c)=10c$ je párny, takže čísla $d+5c$ a~$d-5c$ majú rovnakú paritu. Navyše $d+5c\ge d-5c$ a~$d+5c\ge 0$, takže z~rozkladov čísla $404$ na súčin dvoch celých čísel vyhovuje jediný, a~to
$$
d+5c=202,\quad d-5c=2.
$$
Odtiaľ $d=102$, $c=20$. Z~koreňov
$$
b_{1,2}=\frac {-27\pm d}{25}
$$
je celým číslom iba $b=3$. Potom
$$
a_{1,2}=\frac {-7b-5\pm c}2,
$$
teda $a_1=\m3$ a~$a_2=\m23$.

Danej rovnici vyhovujú dve dvojice čísel $(a,b)$, a~to $(\m3,3)$ a~$(\m23,3)$.

\ineriesenie
Trojčlen $a^2+7ab+6b^2$ sa dá rozložiť na súčin $(a+b)(a+6b)$. Pokúsme sa na súčin rozložiť aj výraz $a^2+7ab+6b^2+5a+4b+c$, pričom $c$ je vhodná konštanta. Rozklad bude mať tvar
$$
a^2+7ab+6b^2+5a+4b+c=(a+b+x)(a+6b+y).
$$
Po roznásobení pravej strany a~porovnaní koeficientov pri $a$ a~$b$ dostaneme
$$
x+y=5,\quad6x+y=4,
$$
čiže
$$x=-\frac 15,\quad y=\frac {26}5
$$
a~nakoniec
$$
c=xy=-\frac {26}{25}.
$$
Danú rovnicu teda môžeme postupne upraviť na tvar
$$
\align
  a^2+7ab+6b^2+5a+4b-\frac {26}{25}&=-3-\frac {26}{25},\\ 
  \left(a+b-\frac 15\right)\left(a+6b+\frac {26}5\right)&=-\frac {101}{25},\\
  (5a+5b-1)(5a+30b+26)&=-101.
\endalign  
$$
Keďže $5a+5b-1\equiv \m1 \pmod 5$, $5a+30b+26\equiv 1 \pmod 5$, vyhovujú zo štyroch vyjadrení čísla $\m101$ v~tvare súčinu dvoch celých čísel len dve nasledovné:

$5a+5b-1=\m1$, $5a+30b+26=101$, a~teda $a=\m3$, $b=3$;

$5a+5b-1=\m101$, $5a+30b+26=1$, a~teda $a=\m23$, $b=3$.

%{\bf Podobné úlohy:}
\návody
V~množine celých čísel vyriešte rovnicu $4x^2-12xy+9y^2+7x-6y-11=0$.
[$x=4-3n^2$, $y=3+n-2n^2$, $n\in\Bbb Z$]

Nájdite všetky celočíselné riešenia rovnice $2x^2-4xy+3y^2-x-2y-19=0$.
[$x=7$, $y=6$; $x=7$, $y=4$; $x=2$, $y=\m1$;  $x=\m1$, $y=2$; $x=\m2$, $y=\m3$; $x=\m2$, $y=1$]

V~množine celých čísel vyriešte rovnicu
$$
\frac {2x+1}y+\frac{3y-1}x=5.
$$
[$x=y=n$, $n\in\Bbb Z-\{0\}$; $x=3n-2$, $y=2n-1$, $n\in\Bbb Z$]

Nájdite všetky dvojice celých čísel $a$, $b$ takých, že súčet $a+b$ je koreňom rovnice $x^2+ax+b=0$.
[$a=b=0$; $a=0$, $b=\m1$; $a=\m6$, $b=8$; $a=\m6$, $b=9$; 55--A--II--1]

\endnávod
}

{%%%%%   B-I-2
\epsplace b56.1 \hfil\Obr \par
\epsplace b56.2 \hfil\Obr

Keď $Y=B$, potom $X=A$. 

Nech $Y\ne B$. Nech $p$ je priamka prechádzajúca bodom~$A$ kolmá na $AB$ a~$C$ ten bod priamky~$p$ ležiaci v~tej istej polrovine určenej priamkou~$AB$ ako bod~$Y$, pre ktorý platí $|AC|=|AB|$ (\obr). Podľa zadania platí $|AX|=|BY|$. Uhol $AYB$ je podľa Tálesovej vety pravý, preto $|\uhol ABY|=90^\circ -|\uhol YAB|=|\uhol CAX|$. Trojuholníky $ABY$ a~$CAX$ sú teda zhodné podľa vety {\it sus}. Odtiaľ vyplýva, že $|\uhol CXA|=|\uhol AYB|=90^\circ$. Bod~$X$ teda leží na Tálesovej polkružnici nad priemerom~$AC$.

Nech naopak $X$ je ľubovoľný vnútorný bod tejto polkružnice a~$Y$ priesečník priamky~$AX$ s~kružnicou~$k$ $(Y\ne A)$. Trojuholníky $CAX$ a~$ABY$ sú zhodné podľa vety {\it usu}, a~preto $|AX|=|BY|$. Bod~$X$ teda patrí do hľadanej množiny.

Hľadanou množinou všetkých bodov~$X$ je zjednotenie dvoch polkružníc nad priemermi $AC_1$ a~$AC_2$ ležiacich v~tej istej polrovine ako bod~$B$; $C_1$ a~$C_2$ sú body ležiace na kolmici vedenej bodom~$A$ na priamku~$AB$, pričom $|AC_1|=|AC_2|=|AB|$ (\obr). Bod~$A$ do hľadanej množiny patrí, body $C_1$ a~$C_2$ nie.

\midinsert
\centerline{\inspicture-!\hss\inspicture-!}
\endinsert

\návody
Nech $X$ je ľubovoľný vnútorný bod strany~$BC$ štvorca $ABCD$. Označíme $P$, $Q$ päty kolmíc z~bodov $B$ a~$D$ na priamku~$AX$. Dokážte, že trojuholníky $ABP$ a~$DAQ$ sú zhodné.

Je daný obdĺžnik $ABCD$. Dokážte, že priesečník~$P$ kružníc zostrojených nad priemermi $AB$ a~$AD$ ( pričom $P \ne A$) leží na úsečke~$BD$.

\endnávod
}

{%%%%%   B-I-3
Na konštrukciu množiny po dvoch nesúdeliteľných trojciferných zložených čísel s~veľkým počtom prvkov môžeme využiť to, že mocniny dvoch rôznych prvočísel sú nesúdeliteľné. Množina 
$$
\{2^7, 3^5, 5^3, 7^3, 11^2, 13^2, 17^2, 19^2, 23^2, 29^2, 31^2\}
$$ 
obsahuje $11$ po dvoch nesúdeliteľných trojciferných čísel a~nie je v~nej žiadne prvočíslo.

Dokážeme, že každá aspoň dvanásťprvková množina po dvoch nesúdeliteľných trojciferných čísel obsahuje prvočíslo. Keďže $37^2>1000$, je každé zložené trojciferné číslo deliteľné aspoň jedným prvočíslom menším ako $37$. Preto sa dá množina všetkých zložených trojciferných čísel rozdeliť na $11$ podmnožín 
$\mm A_2$, $\mm A_3$, $\mm A_5$, $\mm A_7$, $\mm A_{11}$, $\mm A_{13}$, $\mm A_{17}$, $\mm A_{19}$, $\mm A_{23}$, $\mm A_{29}$, $\mm A_{31}$,
pričom $\mm A_i$ obsahuje tie čísla, ktorých najmenším prvočiniteľom je číslo~$i$. Každé dve rôzne čísla z~tej istej množiny~$\mm A_i$ sú súdeliteľné. Nech množina~$\mm B$ trojciferných po dvoch nesúdeliteľných čísel má aspoň $12$~prvkov. Keby v~$\mm B$ boli iba zložené čísla, podľa Dirichletovho princípu by $\mm B$ obsahovala dve čísla z~tej istej množiny~$\mm A_i$; tieto čísla by ale boli súdeliteľné. Preto množina~$\mm B$ musí obsahovať aspoň jedno prvočíslo.

Hľadané najmenšie číslo~$k$ je teda $12$.

\návody
Nájdite najmenšie prirodzené číslo~$n$ s~nasledujúcou vlastnosťou: Ak zvolíme ľubovoľných $n$ rôznych čísel z~množiny $\{1,2,\dots,100\}$, sú medzi nimi dve čísla s~rozdielom a) 11; b) 13. [a) 56; b) 53]

Na večierku je 20~ľudí. Dokážte, že sú medzi nimi dvaja, ktorí majú medzi ostatnými účastníkmi večierka rovnaký počet priateľov (priateľstvo je symetrické: ak $A$ je priateľom~$B$, potom $B$ je priateľom~$A$).

Určte najmenšie prirodzené číslo~$n$ s~nasledujúcou vlastnosťou: Ak zvolíme ľubovoľných $n$ prirodzených čísel menších ako 2006, sú medzi nimi dve také, že podiel súčtu a~rozdielu ich druhých mocnín je väčší ako~3. [21; 55-B-I-6]

\endnávod
}

{%%%%%   B-I-4
\epsplace b56.3 \hfil\Obr

Konvexný štvoruholník je dotyčnicový práve vtedy, keď súčty dĺžok jeho protiľahlých strán sú rovnaké.

\midinsert
\inspicture{}
\endinsert

V~pravouhlom trojuholníku $ABC$ s~preponou~$AB$ označme $a=|BC|$, $b=|AC|$ (\obr). Podľa Pytagorovej vety platí
$$
|BD|=\sqrt {|BC|^2+|CD|^2}=\sqrt{a^2+\left(\frac b2\right)^2},\quad |AE|=\sqrt{b^2+\left(\frac a2\right)^2}.
$$
Keďže ťažisko trojuholníka delí ťažnicu v~pomere $1:2$, máme
$$
|TD|=\frac 13 |BD|=\frac 13 \sqrt{a^2+\left(\frac b2\right)^2},\quad
|TE|=\frac 13 |AE|=\frac 13 \sqrt{b^2+\left(\frac a2\right)^2}.
$$
Štvoruholník $CDTE$ je dotyčnicový práve vtedy, keď $|CD|+|TE|=|EC|+|TD|$, teda 
$$
\frac b2 + \frac 13 \sqrt{b^2+\left(\frac a2\right)^2}=\frac a2 + \frac 13 \sqrt{a^2+\left(\frac b2\right)^2}.
$$

Ak $a=b$, potom rovnosť platí.

Ak $a>b$, potom $a^2+\frac14{b^2}>b^2+\frac14{a^2}$, a~teda 
$$
\frac b2 + \frac 13 \sqrt{b^2+\left(\frac a2\right)^2}<\frac a2 + \frac 13 \sqrt{a^2+\left(\frac b2\right)^2}.
$$

Podobne, ak $a<b$, potom 
$$\frac b2 + \frac 13 \sqrt{b^2+\left(\frac a2\right)^2}>\frac a2 + \frac 13 \sqrt{a^2+\left(\frac b2\right)^2}.
$$

Štvoruholník $CDTE$ je teda dotyčnicový práve vtedy, keď je trojuholník $ABC$ rovnoramenný.

\ineriesenie
Z~rovnosti 
$$
\frac b2 + \frac 13 \sqrt{b^2+\left(\frac a2\right)^2}=\frac a2 + \frac 13 \sqrt{a^2+\left(\frac b2\right)^2}
$$
postupne vyplýva 
$$
\align
3b+\sqrt {4b^2+a^2}&=3a+\sqrt {4a^2+b^2},\\
3b-3a&=\sqrt {4a^2+b^2}-\sqrt {4b^2+a^2},\\
9b^2-18ab+9a^2&=5a^2+5b^2-2\sqrt {4a^4+17a^2b^2+4b^4},\\
2a^2-9ab+2b^2&=-\sqrt {4a^4+17a^2b^2+4b^4},\\
4a^4+81a^2b^2+4b^4-36a^3b-36ab^3+8a^2b^2&=4a^4+17a^2b^2+4b^4,\\
72a^2b^2-36a^3b-36ab^3&=0,\\
-36ab(a-b)^2&=0,\\
a&=b.
\endalign
$$
Naopak, z~rovnosti $a=b$ vyplýva 
$$
\frac b2 + \frac 13 \sqrt{b^2+\left(\frac a2\right)^2}=\frac a2 + \frac 13 \sqrt{a^2+\left(\frac b2\right)^2}.
$$
Štvoruholník $CDTE$ je teda dotyčnicový práve vtedy, keď je
trojuholník $ABC$ rovnoramenný.

\návody
Dokážte, že konvexný štvoruholník $ABCD$ je dotyčnicový práve vtedy, keď $|AB|+|CD|=|BC|+|DA|$.

Dokážte, že v~trojuholníku $ABC$ platí $v_a<v_b$ práve vtedy, keď $t_a<t_b$. [Obe nerovnosti sú ekvivalentné s~$a>b$.]

V~rovnoramennom trojuholníku $ABC$ má základňa~$AB$ dĺžku $c=4$ a~rameno~$AC$ dĺžku~$7$. Vypočítajte dĺžky ťažníc.
[$t_a=t_b=\frac 92$, $t_c=\sqrt{45}$]

\endnávod
}

{%%%%%   B-I-5
Delením polynómu $x^4+px^2+q$ polynómom $x^2+px+q$ zistíme, že platí
$$
x^4+px^2+q=(x^2+px+q)(x^2-px+p^2+p-q)+(2pq-p^3-p^2)x+q-p^2q-pq+q^2.
$$
Polynóm $x^2+px+q$ je deliteľom polynómu $x^4+px^2+q$ práve vtedy, keď je zvyšok $(2pq-p^3-p^2)x+q-p^2q-pq+q^2$ nulový polynóm, teda práve vtedy, ak súčasne platia rovnosti $2pq-p^3-p^2=0$ a~$q-p^2q-pq+q^2=0$. Tieto upravíme na tvar 
$$
p(2q-p^2-p)=0,\quad\text{a}\quad q(1-p^2-p+q)=0.
$$

Ak $p=0$, potom $q=0$ alebo $q=\m1$.

Ak $q=0$, potom $p=0$ alebo $p=\m1$.

Ak $p\ne 0$ a~$q\ne 0$, potom musí platiť $2q-p^2-p=0$ a~$1-p^2-p+q=0$. Z~druhej rovnice vyjadríme $q=p^2+p-1$. Po dosadení do prvej rovnice máme $2p^2+2p-2-p^2-p=0$ a~odtiaľ $p=1$, $q=1$ alebo $p=\m2$, $q=1$.

Vyhovuje teda päť dvojíc $(p,q)$, a~to $(0,0)$, $(0,\m1)$, $(\m1,0)$, $(1,1)$, $(\m2,1)$.

\ineriesenie
Polynóm $x^2+px+q$ je deliteľom polynómu $x^4+px^2+q$ práve vtedy, keď existujú také reálne čísla $a$, $b$, že
$$
\align
x^4+px^2+q=&(x^2+px+q)(x^2+ax+b)=\\
          =&x^4+(a+p)x^3+(b+ap+q)x^2+(bp+aq)x+bq.
\endalign
$$
Porovnaním koeficientov dostaneme podmienky
$$
\align
  a+p&=0, \tag1\\
  b+ap+q&=p,\tag2\\
  bp+aq&=0,\tag3\\
  bq&=q.\tag4
\endalign
$$

Ak $q=0$, potom podľa \thetag3 $p=0$ alebo $b=0$. Dosadením $b=0$ do \thetag2 s~využitím \thetag1 dostaneme $\m p^2=p$, a~teda okrem $p=0$ vyhovuje aj $p=\m1$.

Ak $q\ne0$, vyplýva zo \thetag4 $b=1$. Vzťahy \thetag3 a~\thetag1 potom dávajú $p-pq=0$, teda $p=0$ alebo $q=1$. V~prvom prípade musí byť podľa \thetag2 $q=\m1$, v~druhom $1-p^2+1=p$ a~odtiaľ $p=1$ alebo $p=\m2$. 

\návody
Dokážte, že pre každé $a$ je polynóm $x^4+(1-a)x^3+x^2+a$ deliteľný polynómom $x^2-ax+a$.

Zistite, pre ktorú hodnotu parametra~$a$ je polynóm $2x^3+4x^2+2ax+9$ deliteľný polynómom $x^2-x-a$.
[$a=\m\frac 32$]

Nájdite spoločné korene polynómov $x^4+2x^3+x^2-9$ a~$x^4-3x^2+4x-3$.
[$\m\frac12\pm \frac12\sqrt{13}$]

\endnávod
}

{%%%%%   B-I-6
\epsplace b56.4 \hfil\Obr

Strana~$BC$ hľadaného trojuholníka leží na priamke~$q$, ktorá prechádza bodom~$A_0$ a je kolmá na výšku~$AA_0$. Na tejto priamke leží aj stred~$D$ strany~$BC$. Ťažisko~$T$ je obrazom bodu~$D$ v~rovnoľahlosti so stredom~$A$ a~koeficientom~$\frac23$, leží preto na priamke~$q'$, ktorá je obrazom priamky~$q$ v~uvedenej rovnoľahlosti. Stred~$S$ opísanej kružnice leží na osi~$o$ strany~$BC$, čiže na priamke, ktorá prechádza bodom~$D$ a je rovnobežná s~výškou~$AA_0$ (\obr).

\inspicture{}

\konstrukcia
Bodom~$A_0$ vedieme priamku~$q$ kolmú na úsečku~$AA_0$. Zostrojíme obraz~$q'$ priamky~$q$ v~rovnoľahlosti so stredom~$A$ a~koeficientom~$\frac23$. Označíme $T$ priesečník priamky~$q'$ s~priamkou~$p$ a~$D$ priesečník priamky~$AT$ s~priamkou~$q$. Bodom~$D$ vedieme rovnobežku~$o$ s~$AA_0$ a~jej priesečník s~priamkou~$p$ označíme~$S$. Priesečníky kružnice~$k$ so stredom~$S$ a~polomerom $|SA|$ s~priamkou~$q$ sú vrcholy $B$ a~$C$ hľadaného trojuholníka.

\smallskip{\it Dôkaz správnosti}.
Úsečka~$AA_0$ je kolmá na stranu~$BC$, je to teda výška trojuholníka $ABC$. Bod~$S$ ležiaci na priamke~$p$ je stredom kružnice opísanej trojuholníku $ABC$. Zo zhodnosti trojuholníkov $BDS$ a~$CDS$ (veta {\it Ssu}) vyplýva, že $D$ je stred strany~$BC$. Preto je $AD$ ťažnica a~$T$ ťažisko trojuholníka $ABC$ (platí totiž $|AT|=\frac23 |AD|$).

\smallskip{\it Diskusia}.
Ak priamka~$p$ nie je rovnobežná s~úsečkou~$AA_0$ ani nie je na ňu kolmá, sú body $T$ a~$S$ jednoznačne určené. V~tom prípade má úloha práve jedno riešenie (až na označenie bodov $B$ a~$C$), pokiaľ kružnica~$k$ pretína priamku~$q$ v~dvoch rôznych bodoch; ak kružnica~$k$ nepretína priamku~$p$ v~dvoch rôznych bodoch, nemá úloha riešenie.

Ak je úsečka~$AA_0$ časťou priamky~$p$, nie je bod~$S$ jednoznačne určený; vyhovujú všetky rovnoramenné trojuholníky so základňou~$BC$, ktorá má stred v~bode~$A_0$. Ak je úsečka~$AA_0$ rovnobežná s~priamkou~$p$, ale neleží na nej, nemá úloha riešenie.

Ak je priamka~$p$ kolmá na úsečku~$AA_0$, má úloha riešenie len vtedy, keď sú priamky $q'$ a~$p$ totožné. To nastane vtedy, keď priamka~$p$ pretína úsečku~$AA_0$ v~bode~$V$, pre ktorý platí $|AV|=2|A_0V|$. V~takom prípade môžeme bod~$T$ zvoliť na $p$ kdekoľvek a~úloha má nekonečne veľa riešení.

\návody
Dokážte, že v~každom nerovnostrannom trojuholníku leží ortocentrum~$V$, ťažisko~$T$ a~stred~$S$ opísanej kružnice na jednej priamke, pričom $T$ leží medzi $V$ a~$S$ a~platí $|TV|=2|ST|$. (Priamka, na ktorej ležia body $S$, $T$ a~$V$, sa nazýva {\it Eulerova priamka}.)

Sú dané body $A$, $D$ a~$V$. Zostrojte trojuholník $ABC$, v~ktorom $D$ je stred strany $BC$ a~$V$ priesečník výšok.

\endnávod
}

{%%%%%   C-I-1
Substitúciou $m=\sqrt a$, $n=\sqrt b$ prevedieme rovnicu na tvar
$m^2-n^2-5(m-n)=0$, odkiaľ s~pomocou vzorca pre rozdiel štvorcov
dostaneme $(m-n)(m+n-5)=0$. Takže $m-n=0$ alebo $m+n=5$.

V~prvom prípade po spätnej substitúcii zistíme, že úlohe vyhovujú
všetky dvojice prirodzených čísel $a$, $b$, pre ktoré platí $b=a$.
V~druhom dostávame $\sqrt a+\sqrt b=5$. Teda $1\le\sqrt a,\sqrt b\le4$,
preto stačí postupne dosadzovať $a=1,2,\dots,16$
do vzťahu
$$
b=\bigl(5-\sqrt a\bigr)^2 \tag1
$$
a~zisťovať, či je prislúchajúce číslo~$b$ prirodzené.

Daná rovnica sa nemení zámenou neznámych $a$, $b$. Môžeme teda
predpokladať, že $a\le b$, čo spolu s~rovnosťou $\sqrt a+\sqrt b=5$
znamená, že $\sqrt a\le2{,}5$. Odtiaľ $a\le6{,}25$. Preto sa stačí
pri dosadzovaní zaoberať len hodnotami $a=1,2,\dots,6$ a~zvyšné
riešenia určiť zámenou čísel $a$, $b$ v~nájdených dvojiciach.

Dôvtipnejší postup spočíva v~umocnení zátvorky na pravej strane
vzťahu~\thetag1 a~následnej úprave na tvar
$$
{25+a-b\over10}=\sqrt a, \tag2
$$
z~ktorého je zrejmé, že číslo~$a$ (a~vzhľadom na symetriu danej
rovnice aj číslo~$b$) je druhou mocninou prirodzeného čísla.  
(V~opačnom prípade by na ľavej strane rovnosti~\thetag2 bolo
racionálne číslo, zatiaľ čo na pravej strane iracionálne.) Potom je aj ľavá
strana vzťahu~\thetag2 prirodzené číslo menšie ako päť. Odtiaľ vyplýva, že
rozdiel $a-b$ je nepárny násobok piatich. Pri predpoklade $a<b$ teda
buď $(a,b)=(4,9)$, alebo $(a,b)=(1,16)$. Ďalšie dve riešenia vzniknú
zámenou čísel $a$, $b$.

\zaver
Danej rovnici vyhovujú len dvojice $(a,b)=(1,16),(4,9),(9,4),(16,1)$
a~všetky dvojice $(a,a)$, pričom $a$ je ľubovoľné prirodzené číslo.


\návody
Súčet druhých odmocnín prirodzených čísel $a$, $b$ je číslo
prirodzené práve vtedy, keď sú čísla $a$, $b$ druhými mocninami
prirodzených čísel. Dokážte.

Nájdite všetky dvojice $(x,y)$ prirodzených čísel, pre ktoré platí
$$
y\sqrt x-3\sqrt x-4y+12=0.
$$
[Riešením sú všetky dvojice $(16,n)$ a~$(n,3)$, pričom $n\in\Bbb N$.]

Nájdite všetky dvojice $a$, $b$ nezáporných reálnych čísel, pre
ktoré platí
$$
\sqrt{a^2+b}+\sqrt{b^2+a}=\sqrt{a^2+b^2}+\sqrt{a+b}.
$$
[48--C--S--1]
\endnávod
}

{%%%%%   C-I-2
\fontplace
\tpoint1; \tpoint1; \rBpoint1; \lBpoint1; \bpoint1; \rBpoint1;
\rtpoint\up2\unit1;
[1] \hfil\Obr a

\fontplace
\tpoint \vphantom1a; \rBpoint a;
[2] \hfil\Obrr1b\qquad

\fontplace
\tpoint 1; \rBpoint 1;
[3]

\fontplace
\tpoint 3; \tpoint 3; \tpoint 3;
\lBpoint 3; \lBpoint 3; \lBpoint 3;
\rBpoint 3; \rBpoint 3; \rBpoint 3;
[4] \hfil\Obr

Lichobežníky so stranami dĺžok 1\,cm, 1\,cm, 1\,cm a~2\,cm
sú všetky navzájom zhodné a~skladajú sa z~troch rovnostranných  
trojuholníkov (\obr a). (Základne každého lichobežníka majú dve
rôzne dĺžky, v~našom prípade to musia byť 2\,cm a~1\,cm.) Budeme
ich nazývať {\it základné lichobežníky}. Rovnostranný trojuholník  
s~dĺžkou strany 1\,cm nazveme {\it základný trojuholník}.

\midinsert
\line{\hss\inspicture-!\hss\inspicture-!\kern-12mm\inspicture-!\hss}
\endinsert

Vidíme, že každý z~hľadaných trojuholníkov sa dá rozrezať na konečný
počet základných trojuholníkov. Preto sú veľkosti jeho
vnútorných uhlov násobkami šesťdesiatich stupňov. Vnútorné uhly každého
trojuholníka sú tri a~súčet ich veľkostí je $180\st$, má
teda zmysel hľadať len rovnostranné trojuholníky. Z~podmienky
rozrezania na konečný počet základných trojuholníkov ďalej vyplýva, že
dĺžka strany hľadaného trojuholníka vyjadrená v~centimetroch je
prirodzené číslo. Ak ju označíme~$a$, dá sa náš trojuholník rozrezať
práve na $a^2$ základných trojuholníkov. To možno odvodiť napríklad
vydelením jeho obsahu $S_a=\frac14a^2\sqrt3$ a~obsahu
$S_1=\frac14\sqrt3$ základného trojuholníka. Všeobecnejšie platí, že dva
trojuholníky, ktoré sú podobné s~koeficientom~$k$, majú obsahy
v~pomere~$k^2$.

\inspicture{}

Iné odvodenie počtu základných trojuholníkov v~rovnostrannom
trojuholníku so stranou $a$~centimetrov vyplýva z~doplnenia trojuholníka na
kosoštvorec podľa \obrr1b, kde bolo zvolené $a=3$. Kosoštvorec je
zložený z~dvoch rovnostranných trojuholníkov so stranou dĺžky
$a$~centimetrov. Dá sa teda rozrezať na $a^2$ kosoštvorcov (jeden je
zobrazený v~pravej dolnej časti \obrr1b), z~ktorých každý je zložený
z~dvoch základných trojuholníkov a~ktorým tiež budeme hovoriť
základné. Odtiaľ vyplýva, že rovnostranný trojuholník obsahuje
rovnaký počet základných trojuholníkov, ako jemu prislúchajúci
kosoštvorec obsahuje základných kosoštvorcov.

Zistili sme, že každý z~hľadaných trojuholníkov je rovnostranný
so stranou dĺžky $a$~centimetrov ($a\in\Bbb N$) a~že je zložený
z~$a^2$~základných trojuholníkov. Keďže každý základný lichobežník
obsahuje práve tri základné trojuholníky, musí byť číslo~$a^2$,  
a~teda aj číslo~$a$, deliteľné tromi. Z~\obr{} potom vyplýva, že každý
rovnostranný trojuholník so stranou dĺžky $3n$~centimetrov, pričom
$n=1,2,\dots$, sa dá rozrezať na základné lichobežníky.

\zaver
Podmienkam úlohy vyhovujú len rovnostranné
trojuholníky s~dĺžkou strany $3n$~centimetrov, pričom $n$ je prirodzené
číslo.

\návody
Daný rovnostranný trojuholník rozdeľte na:   a)~ 18,   b)~ 19,
c)~20  rovnostranných,  nie nutne zhodných trojuholníkov.
[41--Z7--II--1]

Rozdeľte štvorec so stranou dĺžky 12\,cm na tri obdĺžniky s~čo
najmenšími rovnakými obvodmi.
[49--Z6--I--2]

Určte všetky štvorce, ktoré sa dajú bez zvyšku rozrezať na
T-tetraminá (útvary~\Tm{} zložené zo štyroch jednotkových
štvorcov).                        [41--Z8--I--6]

\endnávod
}

{%%%%%   C-I-3
Hľadané číslo~$n$ obsahuje aspoň dve cifry. Zapíšme ho v~tvare
$n=10a+b$, pričom $a$ je číslo, ktoré vznikne škrtnutím poslednej
cifry~$b$ čísla~$n$. Podľa zadania $a\deli10a+b$. Odtiaľ
$a\deli b$. Nakoľko vieme, že $b\ne0$, musí byť $a$
jednociferné číslo, takže $n$ je dvojciferné s~nenulovými
ciframi $a$, $b$, pričom $b=ka$, $k\in\Bbb N$.

Ak škrtneme cifru~$a$ v~čísle~$n$, zostane číslo~$b$, ktoré
musí deliť pôvodné číslo $n=10a+b$, z~čoho postupne dostávame
$b\deli10a$, $ka\deli10a$, $k\deli10$ a~odtiaľ $k\in\{1,2,5\}$.
Dosadením do $b=ka$ dostaneme tri možné prípady $b=a$, $b=2a$  
a~$b=5a$ a~v~každom z~nich ľahko určíme vyhovujúce dvojice cifier
$a$, $b$. Tak zistíme, ako musia hľadané čísla $n=10a+b$
vyzerať.

\zaver
Riešením úlohy sú čísla 11, 12, 15, 22, 24, 33,
36, 44, 48, 55, 66, 77, 88 a~99. Skúškou sa presvedčíme, že
všetky vyhovujú podmienkam úlohy.


\návody
Nájdite všetky celé čísla od 1 do~1\,000\,000, ktoré sa škrtnutím
prvej číslice 73-krát zmenšia.
[9\,125,  91\,250,  912\,500; 45--Z7--I--3]

Pred dané trojciferné číslo napíšeme jeho osemnásobok. Vznikne
šesťciferné alebo sedemciferné číslo. (Napríklad pre číslo 103
vznikne číslo 824\,103.) Ukážte, že vzniknuté číslo je vo~všetkých
prípadoch deliteľné aspoň tromi rôznymi prvočíslami.
[41--Z8--III--3]
\endnávod
}

{%%%%%   C-I-4
\fontplace
\tpoint A; \tpoint B; \bpoint C; \bpoint D;
\tpoint E; \rpoint F; \lpoint\up\unit G; \bpoint\xy-.5,-.5 d;
[5] \hfil\Obr

Podľa zadania sú uhly $EFD$ a~$AFC$ priame, takže (\obr)
$$
\align
|\uh CDF|=&|\uh AEF| \quad\text{(striedavé uhly)}, \\
|\uh CFD|=&|\uh AFE| \quad\text{(vrcholové uhly)}.
\endalign
$$

\inspicture{}

Navyše bod~$F$ rozpoľuje úsečku~$DE$, preto $|DF|=|EF|$
a~trojuholníky $CDF$ a~$AEF$ sú zhodné podľa vety {\it usu}.
Odtiaľ vyplýva, že $|CD|=|AE|$, čo spolu s~rovnosťou $|AE|=|EB|$ vedie
k~záveru, že $EB$ a~$DC$ sú dve zhodné a~rovnobežné úsečky.
To znamená, že štvoruholník $EBCD$ je rovnobežník. Priesečník~$G$
jeho uhlopriečok preto rozpoľuje každú z~nich. Body $F$ a~$G$ sú
stredy strán $AC$, $EC$ trojuholníka $AEC$, takže úsečka~$FG$ je
jeho strednou priečkou a~$|AE|=2|FG|$. Preto
$$
|AB|=2|AE|=4d\quad\hbox{a}\quad |CD|=|AE|=2d.
$$
Obsah lichobežníka $ABCD$ je $S=\frac12(|AB|+|CD|)v=3dv$.


\návody
V~konvexnom štvoruholníku $ABCD$ označme $K$, $L$, $M$  
a~$N$ postupne stredy strán $AB$, $BC$, $CD$ a~$DA$. Dokážte, že
štvoruholník $KLMN$ je rovnobežník. Pre ktoré štvoruholníky $ABCD$
je $KLMN$ štvorec?

Zostrojte lichobežník, ak sú dané dĺžky 9\,cm a~12\,cm jeho
uhlopriečok, dĺžka 8\,cm jeho strednej priečky a~vzdialenosť 2\,cm
stredov uhlopriečok.
[50--C--I--2]
\endnávod
}

{%%%%%   C-I-5
Platí
33\,$000=1\cdot2\cdot2\cdot2\cdot3\cdot5\cdot5\cdot5\cdot11$  
a~$(n+1)-(n-4)=5$. Keďže pre každé prirodzené~$n$ je hodnota $n+1$
kladná, daný podiel je kladný len vtedy, keď je kladná aj hodnota
$n-4$, odtiaľ $n\ge5$.

a) Pre každé prirodzené $n\ge5$ platí $n-4\ge1$ a~$n+1\ge6$, preto
je najväčšia hodnota daného podielu rovná
$33\,000:(1\cdot6)=5\,500$ a~dostaneme ju pre $n=5$.

b) Pri hľadaní najmenšieho podielu označme $a$, $b$ čísla
$n+1$, $n-4$ v~poradí, ktoré ešte upresníme. Predpokladajme
najskôr, že rozklad čísla~$ab$ na súčin prvočiniteľov obsahuje
prvočísla 11 a~5. Potom sú $a$, $b$ po sebe idúce násobky piatich  
a~práve jedno z~nich, dajme tomu $a$, je násobkom čísla~55.

Uvažujme najskôr $a=55$. Z~dvoch možných hodnôt $b=50$ a~$b=60$
vyberieme tú väčšiu (aby sme dostali menšiu hodnotu skúmaného podielu).
Hodnote $b=60$ z~rovnosti $n+1=60$ (alebo z~rovnosti $n-4=55$)
prislúcha $n=59$ a~skúmaný podiel je potom rovný číslu~10.

Pre $a=110$ (resp. $a=165$) nie je číslo 33\,000 deliteľné
žiadnym zo susedných násobkov piatich, teda číslami 105 a~115 (resp.~
160 a~170).

Pre ďalšie (väčšie) násobky~$a$ čísla~55 dostávame
$ab\ge215\cdot220>33\,000$.

Ak rozklad čísla~$ab$ na súčin prvočiniteľov neobsahuje
prvočíslo~11 alebo prvočíslo~5, je skúmaný podiel (ak je
celočíselný) deliteľný číslom~ 11 resp. číslom~125, takže
je to číslo väčšie ako hodnota~10, ktorú sme našli skôr.
                 
\zaver
Najväčšia hodnota daného podielu je 5\,500 pre $n=5$
a~najmenšia je 10 pre $n=59$.

\návody
Z~(nie nutne všetkých) cifier 0, 1, 2, 3, 4, 5, 6, 8, 9 utvorte čo
najväčšie číslo s~rôznymi ciframi tak, aby bolo deliteľné číslom~72.
[98\,653\,104; 42--Z6--I--2]

V~čísle 71\,839\,664\,518 nahraďte niektoré cifry štvorkami tak, aby
vzniklo čo najmenšie číslo deliteľné číslom~18.
[41\,434\,444\,548; 48--Z7--I--1]
\endnávod
}

{%%%%%   C-I-6
\fontplace
\tpoint A; \tpoint B; \bpoint C; \tpoint D; \lBpoint E;
\rBpoint V; \cpoint\beta;
[6] \hfil\Obr

Pri označení podľa \obr{} dostaneme
$$
\gather
|\uh ADV|=|\uh CDB|=90\st,\\
|\uh V\!AD|=|\uh BAE|=90\st-\b=|\uh BCD|.
\endgather
$$

\inspicture{}

Trojuholníky $ADV$ a~$CDB$ sú teda podobné podľa vety~{\it uu}.
Z~tejto podobnosti vyplýva
$$
{|AD|\over|CD|}={|VD|\over|BD|}
$$
a~odtiaľ $|AD|\cdot|BD|=|CD|\cdot|VD|$. Zdôraznime, že táto
rovnosť platí pre každý ostrouhlý trojuholník $ABC$. Vzťah
$|AD|\cdot|BD|=|AB|\cdot|VD|$ zo zadania úlohy teda platí práve vtedy,
keď $|CD|=|AB|$.

\návody
Uhlopriečky konvexného štvoruholníka $ABCD$ sa pretínajú v~bode~$F$.
Dokážte, že strany $BC$ a~$AD$ sú rovnobežné práve vtedy, keď
$|AE|\cdot|BE|=|CE|\cdot|DE|$.

Nech $V$ je priesečník výšok trojuholníka $ABC$ a~$A'$, $B'$,
$C'$ sú päty jeho výšok z~vrcholov $A$, $B$, $C$. Dokážte, že
$|AV|\cdot|A'V|= |BV|\cdot|B'V|= |CV|\cdot|C'V|$.

Nech $AC$ je dlhšia uhlopriečka rovnobežníka $ABCD$ a~body $E$  
a~$F$ sú päty kolmíc z~vrcholu~$C$ na priamky $AB$ a~$AD$.
Dokážte, že $|AB|\cdot|AE| +|AD|\cdot|AF|=|AC|^2$. [Návod:
Označte $G$ pätu kolmice z~bodu~$B$ na úsečku~$AC$ a~dokážte
najskôr podobnosť trojuholníkov $ABG$, $ACE$ a~podobnosť
trojuholníkov $CBG$, $ACF$.]
\endnávod
}

{%%%%%   A-S-1
Predpokladajme, že číslo~$s$ vyhovuje zadaniu úlohy a~označme
korene $x_1$, $x_2$, $x_3$, $x_4$ danej rovnice tak, aby platilo
$$
x_1x_2=-2.
\tag1
$$
Z~rozkladu na koreňové činitele
$$
4x^4-20x^3+sx^2+22x-2=4(x-x_1)(x-x_2)(x-x_3)(x-x_4)
$$
po roznásobení a~porovnaní koeficientov pri rovnakých mocninách~$x$
dostaneme Vi\`etove vzťahy
$$
\align
x_1+x_2+x_3+x_4&=5, \tag2\\
x_1x_2+x_1x_3+x_1x_4+x_2x_3+x_2x_4+x_3x_4&=\frac{s}4,\tag3\\
x_1x_2x_3+x_1x_2x_4+x_1x_3x_4+x_2x_3x_4&=-\frac{11}{2},\tag4\\
x_1x_2x_3x_4&=-\frac12.\tag5
\endalign
$$
Z~rovností \thetag1  a~\thetag5 ihneď vyplýva
$$
x_3x_4=\frac14.                       %\tag5
$$
Z~rovnosti \thetag4 upravenej na tvar
$$
(x_1+x_2)x_3x_4+(x_3+x_4)x_1x_2=-\frac{11}{2}
$$
po dosadení hodnôt $x_1x_2$ a~$x_3x_4$ vychádza rovnica
$$
\frac14(x_1+x_2)-2(x_3+x_4)=-\frac{11}{2},
$$
ktorá spolu s~rovnicou \thetag2 tvorí sústavu dvoch lineárnych rovníc
pre neznáme súčty $x_1+x_2$ a~$x_3+x_4$. Jednoduchým výpočtom
zistíme, že riešením tejto sústavy je dvojica hodnôt
$$
x_1+x_2=2\quad\text{a}\quad x_3+x_4=3.
%\tag6
$$
Ak dosadíme všetko, čo sme už zistili, do rovnosti \thetag3 upravenej na tvar
$$
x_1x_2+(x_1+x_2)(x_3+x_4)+x_3x_4=\frac{s}{4},
$$
zistíme, že nutne $s=17$.

Teraz musíme urobiť skúšku: z~rovností
$$
x_1+x_2=2\quad\text{a}\quad x_1x_2=-2
$$
vyplýva, že čísla $x_{1,2}$ sú korene kvadratickej rovnice
$$
x^2-2x-2=0,\quad\text{teda}\quad
x_{1,2}=1\pm\sqrt{3};                      \tag6
$$
z~rovností
$$
x_3+x_4=3\quad\text{a}\quad x_3x_4=\frac14
$$
zase vyplýva, že čísla $x_{3,4}$ sú korene kvadratickej rovnice
$$
x^2-3x+\frac14=0,\quad\text{teda}\quad
x_{3,4}=\frac32\pm\sqrt{2}.                   \tag7
$$

Vidíme, že $x_{1,2,3,4}$ sú skutočne štyri navzájom
rôzne reálne čísla, ktoré spĺňajú sústavu rovníc \thetag2--\thetag5 pre
hodnotu $s=17$, takže to sú korene rovnice zo zadania.
Zdôraznime, že úlohou nebolo tieto
korene vypočítať. Nestačilo by však len overiť, že každá 
z~kvadratických rovníc v~\thetag6 a~\thetag7 má dva rôzne reálne korene (to
nastane práve vtedy, keď ich diskriminanty sú kladné čísla),
okrem toho by bolo nutné ešte ukázať, že tieto dve rovnice nemajú
spoločný koreň.

Hľadané číslo~$s$ je jediné a~má hodnotu $s=17$.

\ineriesenie
Označme $x_{1,2}$ tie korene danej rovnice, pre ktoré
má platiť $x_1x_2=\m2$.
Mnohočlen z~ľavej strany rovnice je deliteľný mnohočlenom
$(x-x_1)(x-x_2)$, teda mnohočlenom tvaru $x^2+px-2$ (kde $p=\m x_1-x_2$),
existuje teda rozklad
$$
4x^4-20x^3+sx^2+22x-2=(x^2+px-2)(4x^2+qx+r).
$$
Roznásobením a~porovnaním
koeficientov pri rovnakých mocninách~$x$ dostaneme sústavu
$$
-20=4p+q,\ s=-8+pq+r,\ 22=-2q+pr,\ -2=-2r.
$$
Zo štvrtej rovnice máme $r=1$, po dosadení do tretej $22=\m2q+p$,
čo spolu s~prvou rovnicou dáva $p=\m2$ a~$q=\m12$. Zo
zvyšnej (druhej) rovnice potom určíme hodnotu $s=17$. Vieme, že pre ňu
má mnohočlen zo zadanej rovnice rozklad
$$
4x^4-20x^3+17x^2+22x-2=(x^2-2x-2)(4x^2-12x+1),
$$
ostáva urobiť skúšku (rovnako ako pri prvom postupe).


\nobreak\medskip\petit\noindent
Za úplné riešenie dajte 6~bodov, z~toho 4~body
za nájdenie hodnoty $s=17$, pritom 1~bod za nájdenie
oboch trojčlenov s~koreňmi $x_{1,2}$ resp. $x_{3,4}$
a~1~bod za skúšku. Pokiaľ
riešiteľ iba správne vypíše sústavu Vi\`etových vzťahov 
(a~z~nej prípadne ešte určí hodnotu súčinu $x_3x_4$), dajte
2~body.
\endpetit
\bigbreak
}

{%%%%%   A-S-2
Uvažovaná množina je množinou práve {\it všetkých (prirodzených)
deliteľov\/} čísla~$160=2^5\cdot5$. Jej prvky môžeme združiť do
dvojíc tak, aby súčin čísel v~každej dvojici bol rovný číslu~160:
$$
1\cdot160=2\cdot80=4\cdot40=5\cdot32=8\cdot20=10\cdot16.
$$
To znamená, že ak
$A = \{a, b, c\}$ je trojica navzájom rôznych deliteľov čísla~160,
je aj $A'=\{160/a,160/b,160/c\}$ trojica navzájom rôznych deliteľov
čísla~160.

Súčin $abc$ prvkov trojice~$A$ sa dá vyjadriť v~tvare
$$
2^k5^l,\quad \text{pričom $k\in\{0,1,2,\dots,14\}$, $l\in\{0,1,2,3\}$}
\tag1
$$
(číslo 160 má len dva delitele, ktoré sú násobkom~$2^5$, preto sa
v~rozklade súčinu $abc$ nemôže objaviť $2^{15}$).
Nie je ťažké zistiť, že najväčšie prirodzené číslo tvaru~\thetag1, ktoré
je menšie ako $2\,006$, je číslo $2\,000 = 2^4\cdot5^3$ a~najmenšie
prirodzené číslo, ktoré je tvaru~\thetag1 a~je väčšie ako $2\,006$, je
číslo $2\,048=2^{11}$ (samotné číslo $2\,006$ tvaru~\thetag1 nie je).
Pritom $2\,000\cdot2\,048=160^3$.

Ak je teda súčin prvkov trojice~$A$ menší ako $2\,006$, je nutne
$abc\le2\,000$ a~súčin $160^3/(abc)$ prvkov zodpovedajúcej trojice~$A'$
je najmenej $160^3/2\,000=2\,048$. Naopak, ak je súčin prvkov
trojice~$A$ väčší ako číslo $2\,006$, je $abc\ge2\,048$ a~súčin
prvkov trojice~$A'$ je najviac $160^3/2\,048=2\,000$. Inými
slovami {\it trojprvkových podmnožín so súčinom prvkov menším ako
$2\,006$ je práve toľko ako trojprvkových podmnožín so súčinom
prvkov väčším ako $2\,006$}.

%% Proto i~všechny tříprvkové podmnožiny $\mm T$ množiny $\mm D$
%% můžeme sdružit do dvojic $(\mm T_1,\mm T_2)$, aby při vhodném
%% označení jejich prvků
% $$
% \mm T_1=\{a_1,b_1,c_1\}\quad\text{a}\quad \mm T_2=\{a_2,b_2,c_2\}     \tag1
% $$
% platilo
% $$
% a_1\cdot a_2=b_1\cdot b_2=c_1\cdot c_2=160.           \tag2
% $$
%% Zdůrazněme, že sdružené množiny $\mm T_1$ a~$\mm T_2$ nemusejí být
%% disjunktní (platí-li například $a_1b_1=160$, je  $a_2=b_1$
%% a~$b_2=a_1$), vždy však platí $\mm T_1\ne \mm T_2$. V~případě
%% $\mm T_1=\mm T_2$ bychom totiž dostali tři navzájem spárované
%% prvky, což nelze, protože námi zvolené párování tvoří disjunktní
%% rozklad množiny $\mm D$ na dvouprvkové podmnožiny. Ukážeme-li
%% proto dále, že pro každou dvojici sdružených množin~(1) je jeden
%% ze součinů $a_1b_1c_1$, $a_2b_2c_2$ menší než 2\,006 a~druhý
%% větší než 2\,006, budeme mít úlohu vyřešenu: {\it tříprvkových
%% podmnožin se součinem prvků menším než $2\,006$ je právě tolik
%% jako tříprvkových podmnožin se součinem prvků větším než
% $2\,006$}.
%
%% Podle (2) pro součiny $a_1b_1c_1$, $a_2b_2c_2$ platí
% $$
% (a_1b_1c_1)\cdot(a_2b_2c_2)=160^3=
% 2^{11}\cdot\bigl(2^{4}\cdot5^3\bigr)=2\,048\cdot2\,000,
% $$
%% odkud plyne implikace
% $$
% a_1b_1c_1\leqq2\,000\quad\Longleftrightarrow\quad
% a_2b_2c_2\geqq2\,048.
% $$
%% Proto zmíněné tvrzení o~srovnání součinů
%% $a_1b_1c_1$, $a_2b_2c_2$ s~číslem 2\,006 bude dokázáno,
%% ověříme-li, že pro žádnou tříprvkovou množinu
%% $\mm T=\{a,b,c\}\subseteq \mm D$ nepadne hodnota součinu~$abc$ do
%% intervalu čísel $\<2\,001,2\,047\>$. To je snadné: každý
%% uvažovaný součin~$abc$ má rozklad tvaru $2^{m}5^{n}$, kde
%% $n\leqq3$, pro každou možnou hodnotu exponentu $n$ určíme čísla
%% uvedeného tvaru, která jsou kritickému intervalu (z~obou stran)
%% nejblíže:
% $$
% \vbox{\openup\jot \let\\=\cr
% \halign{$#$\hss&\qquad\hss$#$,&\qquad\hss$#$\cr
% n=0\:  & 2^{10}5^{0}=1\,024 & 2^{11}5^{0}=2\,048,\\
% n=1\:  &  2^{8}5^{1}=1\,280 &  2^{9}5^{1}=2\,560,\\
% n=2\:  &  2^{6}5^{2}=1\,600 &  2^{7}5^{2}=3\,200,\\
% n=3\:  &  2^{4}5^{3}=2\,000 &  2^{5}5^{3}=4\,000.\\
% }}
% $$

\nobreak\medskip\petit\noindent
Za úplné riešenie dajte 6~bodov, z~toho 2~body za zistenie, že
uvažovaná množina je množinou deliteľov čísla~160, ďalšie 2~body
za následné párovanie trojprvkových podmnožín. Ak riešiteľ uvedie
fakt $abc\notin\<2\,001,2\,047\>$ bez dôkazu, nestrhnite žiadny bod
(celé čísla z~tohto intervalu možno na deliteľnosť prvočíslami rýchlo
jednotlivo otestovať). Riešenie, v~ktorom sa žiak iba zaoberá
otázkou, aké konkrétne hodnoty môže súčin~$abc$ nadobúdať (nie
však koľkokrát sa tak pre jednotlivé hodnoty stane), oceňte
najviac 2~bodmi.

\endpetit
\bigbreak
}

{%%%%%   A-S-3
\fontplace
\tpoint A; \tpoint B; \bpoint C; \bpoint D;
\bpoint S; \tpoint\xy-.6,0 T; \bpoint\xy.7,.5 U; \bpoint V;
[4] \hfil\Obr

\fontplace
\tpoint A; \tpoint B; \bpoint C; \bpoint D; \tpoint E;
\bpoint S; \tpoint\xy-.6,0 T; \bpoint\xy.7,.5 U; \bpoint V;
[5] \hfil\Obr

\fontplace
\tpoint A; \tpoint B; \bpoint C; \bpoint D; \tpoint\xy-1,0 E;
\bpoint\, S; \tpoint\xy-.6,0 T; \bpoint\xy.7,.5 U; \bpoint V;
[6] \hfil\Obr

Bod~$U$ ako priesečník osí vnútorných uhlov pri vrcholoch $A$ a~$D$
daného lichobežníka má rovnakú vzdialenosť od strán $AB$, $AD$ 
a~zároveň aj od strán $AD$, $DC$. To znamená, že má rovnakú
vzdialenosť od oboch základní $AB$, $CD$ lichobežníka $ABCD$.
Podobne aj bod~$V$, ktorý je priesečníkom osí uhlov pri vrcholoch $B$
a~$C$, má od oboch základní rovnakú vzdialenosť. Priamky
$UV$ a~$AB$ sú teda rovnobežné. Tým je vyriešená časť~a).

Keďže súčet vnútorných uhlov ako pri vrcholoch $A$ a~$D$, tak
pri vrcholoch $B$ a~$C$ je $180\st$, je súčet vnútorných uhlov
trojuholníka $ADU$ pri strane~$AD$ rovný $90\st$ rovnako ako súčet vnútorných uhlov
trojuholníka $BCV$ pri strane~$BC$. To znamená, že oba uvedené
trojuholníky sú pravouhlé (s~pravým uhlom pri vrchole~$U$, resp.~$V$,
\obr). Štvoruholník $UTVS$ je teda tetivový (z~predpokladu úlohy
$|AB|>|CD|\ge|DA|$ vyplýva, že polpriamky $AU$ a~$CV$ sa
nepretínajú, body $S$ a~$T$ preto ležia v~opačných polrovinách
určených priamkou~$UV$ a~body $U$, $T$, $V$, $S$ ležia na kružnici
v~uvedenom poradí).

\inspicture

Ako už vieme, priamky $UV$, $AB$ a~$CD$ sú rovnobežné, teda
$|\uh VUT|=|\uh CDT|=45\st$. Z~rovnosti obvodových uhlov
nad stranou~$TV$ tetivového štvoruholníka $UTVS$ tak vyplýva
$|\uh VST|=|\uh VUT|=45\st$. To je zároveň aj veľkosť obvodového uhla
$TSB$ prislúchajúceho tetive~$TB$ kružnice opísanej trojuholníku $STB$ (\obr).
Ostáva ukázať, že na rovnakej kružnici leží aj bod~$E$. To je zrejmé,
pokiaľ $E=T$. V~opačnom prípade stačí zistiť, že veľkosť uhla
$TEB$ je $180\st-45\st$ alebo $45\st$ podľa toho, či priamka~$BT$
body $S$, $E$ oddeľuje alebo nie, čo okamžite vyplýva z~toho, že
priamka~$DT$ zviera so základňou~$AB$ uhol~$45\st$ (\obrr1{}
a~\obrnum). Tým je vyriešená časť~b).

\twocpictures

\nobreak\medskip\petit\noindent
Za úplné riešenie dajte 6~bodov, z~toho 2~body za dôkaz
rovnobežnosti $UV\parallel AB$, ďalej 1~bod za objav pravých uhlov
$AUD$ a~$BVC$ a~1~bod za dôsledok, že štvoruholník $UTVS$ je
tetivový. Nemožno očakávať, že žiaci dokážu cyklickosť
štyroch bodov pomocou orientovaných uhlov priamok, ich riešenie by teda
malo pamätať prinajmenšom na dve možné vzájomné polohy bodov $E$
a~$T$ (zabudnutie na triviálny prípad $E=T$ stratou bodu
netrestajte). Pokiaľ bude dôkaz urobený len pre jednu z~možných
polôh, strhnite 1~bod.

\endpetit
\bigbreak
}

{%%%%%   A-II-1
Označme $a$, $b$, $c$ veľkosti strán trojuholníka $ABC$.
Pre jeho výšku~$v_b$ platí nerovnosť
$$
c\ge v_b,
$$
pretože $v_b$ je dĺžka najkratšej úsečky spájajúcej vrchol~$B$
s~bodom priamky~$AC$. Pre obsah~$S$ trojuholníka $ABC$ tak platí
$$
S=\frac{cv_c}2\ge \frac{v_bv_c}2\ge10\cm^2.
$$

%% Pro jeho výšky $v_b$ a~$v_c$ platí nerovnosti
%% $$
%% a\geq  v_b,\qquad a\geq v_c,\qquad b\geq v_c.
%% $$
%% S~ohledem na tyto nerovnosti pro obsah $S$ trojúhelníku
%% $ABC$ tak platí:
%% $$
%% S=\frac{av_a}2\geq \frac{v_bv_a}2\geq6\,{\rm cm}^2,\quad
%% S=\frac{av_a}2\geq \frac{v_cv_a}2\geq\frac{15}2\,{\rm cm}^2,\quad
%% S=\frac{bv_b}2\geq \frac{v_cv_b}2\geq10\,{\rm cm}^2
%% $$
%% neboli
%% $$
%% S\geq\max\left\{6\,{\rm cm}^2,\tfrac{15}2\,{\rm cm}^2,
%%  10\,{\rm cm}^2\right\}=10\,{\rm cm}^2.
%% $$

Ak existuje trojuholník $ABC$ vyhovujúci podmienkam úlohy,
ktorého obsah je práve $10\cm^2$, potom v~oboch nerovnostiach
$S=\frac12{cv_c}\ge\frac12{v_bv_c}\ge10\cm^2$ nastáva
rovnosť. Vychádza teda $c=v_b=4\cm$ a~súčasne $v_c=5\cm$.
Z~prvej rovnosti vyplýva, že taký trojuholník musí byť pravouhlý
s~pravým uhlom pri vrchole~$A$. Pre dĺžku jeho odvesny~$AC$ teda platí
$b=v_c=5\cm$ a~dĺžka~$a$ jeho prepony~$BC$ je rovná
$\sqrt{41}\cm$. Zo vzorca $S=\frac12{av_a}$ pre jeho
výšku~$v_a$ vyplýva
$$
v_a=\frac{2S}a=\frac{20}{\sqrt{41}}\cm>3\cm.
$$
Pravouhlý trojuholník $ABC$ s~odvesnami $b=5\cm$ a~$c=4\cm$
teda vyhovuje podmienkam úlohy.

Najmenší možný obsah trojuholníka $ABC$, ktorého výšky vyhovujú
podmienkam úlohy, je $10\cm^2$.

%% {\it Poznámka.}
%% Stejně dobrý odhad dostaneme i~z~nerovnosti $bv_b\ge v_cv_b$,
%% zatímco z~ostatních kombinací plynou odhady slabší.

\nobreak\medskip\petit\noindent
Za úplné riešenie dajte 6~bodov, pritom 4~body dajte za dôkaz
nerovnosti $S\ge 10\cm^2$ a~zostávajúce 2~body za uvedenie
príkladu vyhovujúceho trojuholníku $ABC$ s~obsahom $10\cm^2$.
Zrejmé nerovnosti typu $b\ge v_c$ (rovnako dobrý odhad
dostaneme aj z~tejto nerovnosti) nie je nutné dokazovať,
ani nie je nutné zdôvodňovať, kedy v~nich nastane rovnosť.
\endpetit
\bigbreak
}

{%%%%%   A-II-2
Predpokladajme, že rovnica
$$
x^4-4x^3+4x^2+ax+b=0  \eqno{\Num{uv_rov}}
$$
má dva rôzne reálne korene $x_1$ a~$x_2$, pre ktoré platí
$x_1+x_2=x_1x_2=p$. Potom polynóm
na jej ľavej strane je deliteľný polynómom
$(x-x_1)(x-x_2)=x^2-px+p$ a~má rozklad
$$
x^4-4x^3+4x^2+ax+b=(x^2-px+p)(x^2+rx+s),
$$
pričom $r$ a~$s$ sú reálne čísla. Roznásobením výrazu na pravej
strane poslednej rovnosti a~porovnaním koeficientov pri jednotlivých
mocninách~$x$ polynómov na oboch stranách dostaneme
$$
\eqalignno{
  -4&=-p+r,  &\Num{s1}\cr
   4&=p+s-pr,&\Num{s2}\cr
   a&=-ps+pr,&\Num{s3}\cr
   b&=ps.&\Num{s4}
   }
$$
Zo vzťahu~\ref{s1} vyplýva
$$
r=p-4.\eqno{\Num{r}}
$$
Dosadením za $r$ do vzťahu~\ref{s2} dostaneme
$$
s=4-p+p(p-4)=(p-4)(p-1).\eqno{\Num{s}}
$$
Keďže kvadratická rovnica $x^2-px+p=0$ má dva rôzne reálne
korene $x_1$ a~$x_2$, je jej diskriminant kladné číslo, takže
$$
p^2-4p>0.\eqno{\Num{nerovnost}}
$$
Keď sčítame rovnice \ref{s3} a~\ref{s4} a~dosadíme za $r$ podľa~\ref{r},
vyjde podľa predchádzajúceho vzťahu
$$
a+b=pr=p(p-4)=p^2-4p>0,
$$
čo sme chceli dokázať.

Pre diskriminant~$D$ rovnice
$$
x^2+rx+s=0
$$
podľa vzťahov \ref{r}, \ref{s} a~\ref{nerovnost} platí
$$
D=r^2-4s=(p-4)^2-4(p-4)(p-1)=-3p(p-4)=-3(p^2-4p)<0.
$$
Rovnica teda nemá reálne korene. Daná rovnice~\ref{uv_rov} preto
nemá iné reálne korene ako $x_1$ a~$x_2$.

\nobreak\medskip\petit\noindent
Za úplné riešenie dajte 6~bodov, pritom 2~body dajte za
správny rozklad polynómu na ľavej strane rovnice na súčin dvoch
polynómov stupňa~2 a~porovnanie ich koeficientov, \tj. za
správne uvedenie vzťahov \ref{s1}, \ref{s2}, \ref{s3}, \ref{s4}.
Ďalšie 2~body dajte za dôkaz nerovnosti $a+b>0$ a~zostávajúce 2~body
za dôkaz neexistencie ďalších koreňov danej rovnice.
\endpetit
\bigbreak
}

{%%%%%   A-II-3
\fontplace
\tpoint A; \tpoint B; \bpoint C;
\tpoint M; \tpoint\xy-.5,0 S; \rpoint S_1; \lpoint S_2;
\cpoint\a; \cpoint\b;
\cpoint2\a; \cpoint2\b;
[9] \hfil\Obr

\fontplace
\tpoint A; \tpoint B; \bpoint C;
\tpoint M; \tpoint S; \bpoint S_1; \bpoint S_2;
\tpoint P_1; \tpoint P_2;
[10] \hfil\Obr

\fontplace
\tpoint A; \tpoint B; \bpoint C;
\blpoint C_0; \rpoint A'; \lpoint B'; \tpoint M;
\tpoint S; \trpoint S_1; \bpoint S_2;
% \cpoint\phi; \cpoint\phi; \cpoint\phi;
[11] \hfil\Obr 

Označme postupne $\alpha$ a~$\beta$ veľkosti vnútorných uhlov pri
vrcholoch $A$ a~$B$ uvažovaného pravouhlého trojuholníka $ABC$.

\smallskip
a) Zo vzťahu medzi obvodovým a~stredovým uhlom pre spoločnú tetivu~$CM$
kružníc $k_1$ a~$k_2$ opísaných postupne trojuholníkom $AMC$ a~$BMC$
vyplýva (\obr)
$$
|\uhol MS_1C|+|\uhol MS_2C|=2\alpha+2\beta=180^{\circ}.
$$
Štvoruholník $CS_1MS_2$ je teda tetivový. Keďže body
$M$ a~$C$ sú súmerne združené podľa osi úsečky~$CM$, na ktorej
súčasne leží úsečka $S_1S_2$, platí ďalej
$$
|\uhol S_1MS_2|=|\uhol S_1CS_2|=90^{\circ}.
$$
Kružnica opísaná štvoruholníku $CS_1MS_2$ je teda Tálesovou
kružnicou zostrojenou nad priemerom~$S_1S_2$. Body $S$ a~$S_1$
však ležia súčasne na osi odvesny~$AC$, podobne body $S$
a~$S_2$ ležia na osi odvesny~$BC$ uvažovaného trojuholníka. Takže
$|\uhol S_1SS_2|=90^{\circ}$ a~bod~$S$ leží tiež
na Tálesovej kružnici opísanej štvoruholníku $CS_1MS_2$. (Ak
$M=S$, platí toto tvrdenie triviálne.) Tým je dokázaná časť~a).

\inspicture{}

\smallskip
b) Označme $P_1$ a~$P_2$ postupne stredy úsečiek $AM$ a~$BM$ (\obr).
Platí $|S_1S_2|\ge |P_1P_2|=\frac12|AB|$. Kružnica opísaná
\inspicture{}
štvoruholníku $CS_1MS_2$ má preto najmenší priemer~$\frac12|AB|$
práve vtedy, keď $S_1S_2\parallel AB$, čo vzhľadom na kolmosť úsečky~$CM$
a~jej osi~$S_1S_2$ nastane práve vtedy, keď $M$ je pätou výšky
z~vrcholu~$C$ v~trojuholníku $ABC$. (Polomer~$r$ tejto kružnice má
potom veľkosť $r=\frac14|AB|$.)

\ineriesenie
Označme $C_0$ pätu výšky z~vrcholu~$C$ a~uvažujme podobné
zobrazenie zložené z~otočenia o~uhol $\phi=|\uhol C_0CM|$
a~rovnoľahlosti so stredom~$C$, ktoré zobrazí bod~$C_0$ na bod~$M$.
Daný trojuholník $ABC$ sa tak zobrazí na trojuholník $A'B'C$ (\obr) s~výškou~$CM$.
\inspicture{}
Keďže $|\uhol AMA'|=|\uhol ACA'|=\phi$, ležia body $A$, $A'$,
$M$ a~$C$ na kružnici s~priemerom~$A'C$ opísanej trojuholníku $AMC$.
Podobne stred~$S_2$ strany~$B'C$ je stredom kružnice opísanej trojuholníku
$BMC$. Keďže $S_1S_2$ je stredná priečka pravouhlého
trojuholníka $A'CB'$, ležia body $M$ a~$C$ na Tálesovej kružnici
s~priemerom~$S_1S_2$. Na tejto kružnici leží aj stred~$S$ prepony~$AB$
pravouhlého trojuholníka $ABC$, pretože $SS_1$ je os strany~$AC$
a~$SS_2$ os strany~$BC$, takže aj trojuholník $S_1S_2S$ je pravouhlý.

Tým je dokázaná časť~a). Časť~b) vyriešime rovnako ako 
v~predchádzajúcom riešení.

\nobreak\medskip\petit\noindent
Za úplné riešenie dajte 6~bodov. Za úplné riešenie časti~a)
pritom dajte 4~body, z~toho 2~body za dôkaz, že štvoruholník
$CS_1MS_2$ je tetivový. Za úplné riešenie časti~b) dajte 2~body.
Vyjadrenie najmenšieho polomeru, napr.\ pomocou $|AB|$, sa
nevyžaduje.
\endpetit
\bigbreak
}

{%%%%%   A-II-4
\CR=0%
Ukážeme, že najmenšie~$m$ je 113 (nezávisle na hodnotách $p$, $q$).
Zrejme $m>1$. Pre ľubovoľné prirodzené čísla $c<d$ a~$m>1$
označme $S_m(c,d)$ súčet všetkých zlomkov v~základnom tvare, ktoré
ležia v~otvorenom intervale $(c,d)$ a~ktorých menovateľ je $m$.
Potom platí nerovnosť
$$
S_m(c,c+1)\le \Bigl(c+\frac1m\Bigr)+\Bigl(c+\frac2m\Bigr)+\dots
   +\Bigl(c+\frac{m-1}m\Bigr)=(m-1)c+\frac{m-1}2,
$$
v~ktorej rovnosť nastane práve vtedy, keď všetky čísla $1,2,\dots,m-1$ sú
nesúdeliteľné s~$m$, \tj. práve vtedy, keď $m$ je prvočíslo.

\smallskip
Pre dané prirodzené čísla $p$, $q$ a~$m>1$ platí
$$
\align
   S_m(p,q)=&S_m(p,p+1)+S_m(p+1,p+2)+\dots +S_m(q-1,q)\le \cr
   \le&\Bigl((m-1)p+\frac{m-1}2\Bigr)+
     \Bigl((m-1)(p+1)+\frac{m-1}2\Bigr)+\dots\cr
    &+\Bigl((m-1)(q-1)+\frac{m-1}2\Bigr)=\cr
   =&(m-1)\frac{(q-p)(p+q-1)}2+(m-1)\frac{q-p}2=\cr
   =&(m-1)\frac{q-p}2(p+q-1+1)=\frac{(m-1)(q^2-p^2)}2,
\endalign
$$
teda
$$
S_m(p,q)\le \frac{(m-1)(q^2-p^2)}2. \eqno{\Num{Sm}}
$$
Rovnosť vo vzťahu~\ref{Sm} pritom nastane práve vtedy, keď $m$ je prvočíslo.
Podľa zadania však platí
$$
S_m(p,q)\ge 56(q^2-p^2).
$$
Zo vzťahu~\ref{Sm} vidíme, že nutne platí $\frac12(m-1)\ge
56$, \tj. $m\ge 113$. Vzhľadom na to, že číslo 113 je prvočíslo,
je najmenšie hľadané číslo $m=113$.

\ineriesenie
Súčet všetkých zlomkov, ktoré majú menovateľa~$m$, nie sú celé čísla
a~ležia v~intervale $(p,q)$, môžeme tiež určiť ako rozdiel súčtu všetkých
zlomkov s~menovateľom~$m$ ležiacich v~uzavretom intervale
$\langle p,q \rangle$ a~súčtu všetkých prirodzených čísel z~tohto
intervalu. Pre uvažovaný rozdiel~$d$ potom platí
$$
d=\sum_{j=pm}^{qm} \frac{j}{m}- \sum_{j=p}^{q} j.
$$
Menšenec aj menšiteľ v~uvažovanom rozdiele sa dajú vyjadriť ako súčty
členov aritmetických postupností. Pre súčet prvých $n$~členov
aritmetickej postupnosti $(a_i)$ využijeme známy vzťah
$$
a_1+a_2+\dots +a_n=\textstyle \frac12n(a_1+a_n).
$$
%% U~menšence je přitom $a_1=p$, $n=(q-p)m+1$ a~$a_n=q$; u~menšitele
%% $a_1=p$, $n=q-p+1$ a~$a_n=q$. Takže
Pre hľadaný rozdiel~$d$ tak platí
$$
\eqalign{
     d &= \tfrac12(p+q)\big((q-p)m+1\big)-\tfrac12(p+q)(q-p+1)= \cr
       &= \tfrac12(p+q)\big[\big((q-p)m+1\big)-(q-p+1)\big]          \tfrac12(m-1)(q^2-p^2).
    }
$$
Ďalej budeme postupovať rovnako ako v~predchádzajúcom spôsobe riešenia.

\nobreak\medskip\petit\noindent
Za úplné riešenie dajte 6~bodov, z~toho 4~body za odvodenie
nerovnosti~\ref{Sm} alebo nejakého ekvivalentného odhadu, ďalej
1~bod za odvodenie podmienky $m\ge113$ a~1~bod za zdôvodnenie
skutočnosti, že najmenšie~$m$ je práve 113.
\endpetit
}

{%%%%%   A-III-1
\fontplace[13] \hfil\Obr

%\fontplace
%\cpoint\hbox{\copy\viitab};
%[14] \hfil\Obr
\epsplace A56.14
\hfil\Obr

Najskôr ukážeme, že úloha má riešenie pre ľubovoľné párne~$n$.
Ak postavíme figúrku napr. na ktorékoľvek rohové políčko šachovnice
$n\times n$, prejdeme celú šachovnicu po susedných blokoch typu
$2\times n$ spôsobom naznačeným na \obr\ pre $n=8$. Postupnosti
ťahov tu zodpovedá postupnosť na seba nadväzujúcich orientovaných
úsečiek. Celkom analogicky možno postupovať pre každé párne~$n$.

\twocpictures

Teraz ukážeme, že pre žiadne nepárne $n\ge3$ nemožno šachovnicu prejsť
požadovaným spôsobom. Dôkaz urobíme sporom. Pripusťme, že pre
určité nepárne~$n$ na šachovnici $n\times n$ existuje postupnosť
ťahov vyhovujúca podmienkam úlohy. Všetky políčka ofarbíme
podobne ako bežnú šachovnicu $8\times 8$, a~to tak, že
rohové políčka budú čierne (podobne ako na \obr\ pre $n=7$). Ďalej
všetky čierne políčka označíme písmenami A a~B tak, aby žiadne dve
čierne políčka majúce spoločný práve jeden bod (vrchol) neboli
označené rovnakým písmenom. Ak budú rohové (čierne) políčka označené
napr. písmenom~A, bude zrejme počet políčok~A o~$n$ väčší ako počet
políčok~B.

Políčka šachovnice, ktoré figúrka požadovaným spôsobom prejde,
označme postupne $1,2,3,\dots ,n^2$ a~$k$-ty ťah zápisom
$k\mapsto k+1$. Ak je políčko s~číslom~1 čierne, sú čierne práve
políčka s~číslami $1,2,5,6,9,10,\dots$; pritom každý (šikmý) ťah
$1\mapsto2$, $5\mapsto6$, $9\mapsto10$,~\dots{} spája čierne
políčka označené rôznymi písmenami, takže sa celkové počty políčok A a~B líšia
najviac o~1, čo odporuje zistenému rozdielu. K~rovnakému sporu
dôjdeme aj v~prípade, keď je políčko s~číslom~1 biele, takže čierne
sú práve políčka s~číslami $3,4,7,8,11,12,\dots$ spojená (šikmými)
ťahmi $3\mapsto4$, $7\mapsto8$, $11\mapsto12$,~\dots{}

Tým je úloha vyriešená, riešením sú všetky párne $n\ge2$.
}

{%%%%%   A-III-2
\fontplace
\rBpoint A; \rtpoint B; \ltpoint C; \lBpoint D;
\tpoint H; \tpoint\xy-.5,-.6 L; \ltpoint\xy-1.4,0 M;
\lBpoint P; \rBpoint Q; \bpoint R;
\lBpoint k;
\cpoint\epsilon; \cpoint\epsilon; \cpoint\epsilon;
\cpoint\epsilon; \cpoint\epsilon;
\cpoint\phi; \cpoint\phi;
[15] \hfil\Obr

 Priesečník osí vnútorných uhlov pri vrcholoch $A$, $D$ 
v~trojuholníkoch $BCA$, $BCD$ označme $H$ (\obr). Ako je známe,
\inspicture{}
bod~$H$ je stredom príslušného oblúka~$BC$ kružnice~$k$ opísanej
štvoruholníku $ABCD$ (oblúka, ktorý neobsahuje vrcholy $A$ a~$D$).
Označme $\epsilon= |\uhol BAH|= |\uhol CAH|= |\uhol BDH| = |\uhol CDH|= |\uhol CBH|$ a~$\phi=|\uhol ABL|=|\uhol CBL|$.
Potom platí
$$
|\uhol BLH|=|\uhol BAL|+|\uhol ABL|%=|\uhol HBC|+|\uhol CBL|
            =\varepsilon+\varphi=|\uhol LBH|.
$$
Trojuholník $HLB$ je teda rovnoramenný so základňou~$LB$, takže
$|HB|=|HL|$. Analogicky aj $|HC|=|HM|$. A~pretože $|HB|=|HC|$,
dostávame $|HL|=|HM|$. Takže trojuholník $HML$ je rovnoramenný
a~$|\uhol HLM|=|\uhol HML|$.

Označme ešte $P$ kolmý priemet bodu~$L$ na priamku~$AC$ a~$Q$
kolmý priemet bodu~$M$ na priamku~$BD$ (uvažovaný bod~$R$ je tak
priesečníkom priamok $LP$ a~$MQ$). Keďže pravouhlé
trojuholníky $APL$ a~$DQM$ sa zhodujú v~uhloch pri vrcholoch $A$
a~$D$, sú zhodné aj uhly $PLA$ a~$QMD$ pri vrcholoch $L$ a~$M$.
Odtiaľ a~z~rovnosti $|\uhol HLM|=|\uhol HML|$ tak vyplýva
rovnosť $|\uhol PLM|=|\uhol QML|$. To znamená,
%% Tím jsme dokázali,
že trojuholník $LMR$ je rovnoramenný, ako sme mali dokázať.
}

{%%%%%   A-III-3
Uvažujme ľubovoľnú funkciu~$f$ s~požadovanými vlastnosťami. Najskôr
ukážeme, že je prostá. Ak $f(y_1)=f(y_2)$, tak
pre všetky prirodzené čísla~$x$ platí
$$
y_1f(x)=f\left(xf(y_1)\right)=f\left(xf(y_2)\right)=y_2f(x),
$$
a~nakoľko $f(x)$ je prirodzené číslo, vyplýva odtiaľ $y_1=y_2$, čo
znamená, že funkcia~$f$ je prostá.

Voľbou $x=1$ v~danej rovnici dostaneme
$f\left(f(y)\right)=yf(1)$, čo pre $y=1$ dáva $f\left(f(1)\right)=f(1)$.
Keďže $f$ je prostá, vyplýva odtiaľ
$$
f(1)=1,           \tag1
$$
takže pre všetky prirodzené čísla~$y$ navyše platí
$$
f\left(f(y)\right)=y.     \tag2
$$

Z~práve odvodeného vzťahu zároveň vyplýva, že oborom hodnôt funkcie~$f$ je
celá množina~$\Bbb N$. Môžeme teda pre ľubovoľné prirodzené číslo~$z$
nájsť~$y$, pre ktoré $y=f(z)$ a~zároveň $f(y)=z$, takže podľa
vzťahu zo zadania potom platí
$$
f(xz)=f\left(x(f(y)\right)=yf(x)=f(z)f(x).
$$
Odtiaľ možno matematickou indukciou ľahko odvodiť, že pre všetky prirodzené
čísla~$n$, $x_1,x_2,\dots,x_n$ platí
$$
f(x_1x_2\dots x_n)=f(x_1)f(x_2)\dots f(x_n). \tag3
$$

%% Nechť $p$ je libovolné prvočíslo.
Ukážeme, že obraz $f(p)$ ľubovoľného prvočísla~$p$ je tiež
prvočíslo.
Predpokladajme, že $f(p)=ab$, pričom $a$, $b$ sú
prirodzené čísla rôzne od~$1$. Podľa \thetag2 a~\thetag3 platí
$$
p=f\left(f(p)\right)=f(ab)=f(a)f(b).
$$
Keďže funkcia~$f$ je prostá a~$f(1)=1$, platí $f(a)>1$, $f(b)>1$, čo je v~rozpore
s~predpokladom, že $p$ je prvočíslo.

Keďže $2\,007=3^2\cdot223$ je rozklad čísla $2\,007$
na prvočísla, dostaneme podľa~\thetag3
$$
f(2\,007)=f^2(3)f(223),
$$
pričom obe čísla $f(3)$ a~$f(223)$ sú prvočísla.
Ak $f(3)=2$, potom podľa~\thetag2 platí $f(2)=3$ a~najmenšia možná
hodnota $f(223)$ je 5, takže $f(2\,007)\ge20$. Pokiaľ $f(3)=3$,
najmenšia možná hodnota $f(223)$ je $2$ a~platí $f(2\,007)\ge18$.
Ľahko vidíme, že pre každú inú voľbu hodnôt $f(3)$ a~$f(223)$
platí $f(2\,007)\ge 18$.

Ukážeme, že existuje funkcia vyhovujúca zadaniu, pre ktorú platí
$f(2\,007)=18$. Definujme funkciu~$f$ nasledovným spôsobom: Pre
ľubovoľné prirodzené číslo~$x$, ktoré zapíšeme ako $x=2^k\cdot223^m\cdot q$, pričom $k$
a~$m$ sú celé nezáporné čísla a~$q$ je prirodzené číslo nesúdeliteľné
s~číslami $2$ a~$223$, zadáme hodnotu $f(x)$ vzťahom
$$
f(2^k\cdot223^m\cdot q)=2^m\cdot223^k\cdot q.
$$
Potom $f(2\,007)=f(223\cdot 3^2)=2\cdot 3^2=18$. Overíme,
že táto funkcia~$f$ má požadovanú vlastnosť. Nech $x=2^{k_1}\cdot223^{m_1}\cdot q_1$
a~$y=2^{k_2}\cdot223^{m_2}\cdot q_2$ sú ľubovoľné
prirodzené čísla zapísané vyššie uvedeným spôsobom. Potom
$$
\eqalign{
f\left(xf(y)\right)&=f\left(2^{k_1}\cdot223^{m_1}\cdot q_1\cdot f(2^{k_2}\cdot223^{m_2}\cdot 
q_2)\right)=f(2^{k_1+m_2}\cdot223^{m_1+k_2}\cdot q_1\cdot q_2)=\cr
&=2^{k_2+m_1}\cdot 223^{m_2+k_1}\cdot q_1\cdot q_2}
$$
a~súčasne
$$
yf(x)=2^{k_2}\cdot223^{m_2}\cdot q_2\cdot f(2^{k_1}\cdot 223^{m_1}\cdot q_1)=2^{k_2+m_1}\cdot 223^{m_2+k_1}\cdot q_1\cdot q_2.
$$

Najmenšia možná hodnota čísla $f(2\,007)$ je $18$.

\poznamka
Z~vyššie uvedeného riešenia vyplýva, že každá funkcia~$f$, ktorá
vyhovuje danej funkcionálnej rovnici, je určená nejakou
bijekciou~$\phi$ množiny prvočísel na seba, ktorá pre každé
prvočíslo~$p$ spĺňa rovnosť $\phi\left(\phi(p)\right)=p$, a~to
predpisom
$$
\eqalign{
f(1)&=1,\cr
f(p_1^{k_1}p_2^{k_2}\dots p_m^{k_m})&=\phi(p_1)^{k_1}\phi(p_2)^{k_2}\dots \phi(p_m)^{k_m},\cr
}
$$
pričom $p_i$ sú navzájom rôzne prvočísla a~$k_i$ nezáporné celé
čísla. Každá bijekcia~$\phi$ uvedenej vlastnosti rozkladá množinu
prvočísel na zjednotenie jednoprvkových a~dvojprvkových navzájom
disjunktných množín takých, že pre každú z~nich tvaru $\{p\}$
platí $\phi(p)=p$ a~pre každú z~nich tvaru $\{p_1,p_2\}$ platí
$\phi(p_1)=p_2$, $\phi(p_2)=p_1$. Naopak každý taký rozklad
určuje vyhovujúcu bijekciu~$\phi$.
}

{%%%%%   A-III-4
Ukážeme, že uvedený záver všeobecne
neplatí. Ako protipríklad zvolíme množinu
$$
\mm M=\Bbb N\setminus\left\{a;\ a+1\text{ je prvočíslo väčšie ako
2\,008}\right\},
$$
ktorá zrejme obsahuje všetky čísla od $1$ do $2\,007$.
Pritom aritmetická postupnosť
$(a_n)_{n=1}^{\infty}$ s~prvým členom $a_1=n\in\mm M$ 
a~diferenciou $d=n+1$ má všeobecný člen tvaru
$$
a_k=a_1+(k-1)d=n+(k-1)(n+1)=(n+1)k-1,
$$
odkiaľ vyplýva, že číslo $a_k+1=(n+1)k$ nie je prvočíslo
pre žiadny index $k>1$,
takže $a_k$ leží v~$\mm M$ pre každý index~$k$ (či už $a_k\le2\,007$,
alebo $a_k\ge 2\,008$). Nakoľko prvočísel je nekonečne veľa,
je nekonečne veľa aj prirodzených čísel, ktoré v~zvolenej množine~$\mm M$ neležia.

\ineriesenie
Každá vyhovujúca množina~$\mm M$ musí obsahovať
všetky členy prvých $2\,007$ aritmetických postupností
s~prvým členom $n\le2\,007$ a~diferenciou $n+1$:
$$
A_1=(1,3,5,\dots),\ A_2=(2,5,8,\dots),\ \dots,\
A_{2\,007}=(2\,007,4\,015,6\,023,\dots).
$$
Zrejme množina hodnôt $\mm A_k=\{k,2k+1,3k+2,\dots\}$
postupnosti~$A_k$ je pre každé~$k$
tvorená všetkými prirodzenými číslami tvaru $i(k+1)+k$ s~celým
nezáporným~$i$.

Vysvetlíme, prečo
$$
\mm M=\mm A_1\cup \mm A_2\cup\dots\cup \mm A_{2\,007}
$$
je najmenšia množina s~požadovanou vlastnosťou. Ukážeme totiž, že
ak $n\in \mm A_k$ pre niektoré čísla $n$ a~$k$, tak $\mm
A_n\subseteq \mm A_k$. Nech teda $n\in\mm A_k$ a~$m\in \mm A_n$.
Potom $n=i(k+1)+k$ a~$m=j(n+1)+n$ pre vhodné celé nezáporné
$i$ a~$j$, odkiaľ $m=j(i+1)(k+1)+i(k+1)+k=(ji+j+i)(k+1)+k$, čo
znamená, že $m\in\mm A_k$.

Existuje však nekonečne veľa prirodzených čísel, ktorá v~zostrojenej
"minimálnej" vyhovujúcej množine~$\mm M$ neležia; sú to napríklad
všetky násobky čísla $2\,008!$.
}

{%%%%%   A-III-5
\fontplace
\tpoint A; \tpoint B; \bpoint C;
\lBpoint D; \rBpoint E;
\tpoint C_0; \lbpoint\xy-1,0 P;
\tpoint x; \tpoint y; \lpoint p;
\cpoint\a; \cpoint\b; \cpoint\phi; \cpoint\psi;
[16] \hfil\Obr

\fontplace
\tpoint\xy-1,0 A; \tpoint\xy.5,0 B; \bpoint\xy.9,0 C;
\lBpoint D; \bpoint\xy-.4,.3 E;
\rtpoint\xy.4,-.3 M; \tpoint N; \rpoint\up.5\unit P;
\bpoint k; \lpoint l; \rpoint m;
[17] \hfil\Obr

Označme $\phi=|\uhol BAD|$ a~$\psi=|\uhol ABE|$ (\obr). Z~rovnosti
obvodových uhlov $|\uh AEB|=|\uh ADB|$ v~tetivovom štvoruholníku
\inspicture{}
$ABDE$ tak pri zvyčajnom označení uhlov v~trojuholníku $ABC$ vyplýva
$$
\a+\psi=\b+\phi.  \tag1
$$

Označme $C_0$ pätu výšky z~vrcholu $C$, $v_c$ veľkosť výšky~$CC_0$ a~$x$, $y$, $p$ veľkosti
príslušných úsekov $AC_0$, $BC_0$, $PC_0$ (\obrr1), takže
$$
\gathered
\tg\phi={p\over x},\qquad \tg\psi={p\over y},\\
\tg\a={v_c\over x},\qquad \tg\b={v_c\over y}.
\endgathered                                  \tag2
$$
Ak bod~$P$ nie je priesečník výšok (\tj. uhol $\a+\psi$ nie je
pravý), môžeme podľa~\thetag1 písať
$$
\tg(\a+\psi)=\tg(\b+\phi),
$$
čo podľa známeho vzťahu pre tangens súčtu po dosadení z~\thetag2
dáva (využívame rovnosť $\tg\a\tg\psi=\tg\b\tg\phi$, ktorá z~\thetag2 tiež vyplýva)
$$
{v_c\over x}+{p\over y}={v_c\over y}+{p\over x},
$$
čiže
$$
(p-v_c)(x-y)=0.
$$
Keďže vzhľadom na dané predpoklady $p<v_c$ a~$x\ne y$,
nemôže ostatná rovnosť platiť. Takže $\a+\psi=90\st$ a~bod~$P$
je priesečníkom výšok, čo sme chceli dokázať.
                      
\ineriesenie
Označme $k$ kružnicu opísanú tetivovému štvoruholníku $ABDE$ 
a~uvažujme ešte kružnice $l$ a~$m$ opísané trojuholníkom $BEC$ a~$ADC$
(\obr). Tetiva~$BE$ kružnice~$l$ pretína tetivu~$AD$
kružnice~$m$ v~bode~$P$, kružnice $l$, $m$ teda majú okrem bodu~$C$
ešte ďalší priesečník, ktorý označíme~$M$. Z~uvedenej konštrukcie
vplýva, že bod~$P$ leží vnútri každej z~troch uvažovaných kružníc
a~má k~nim rovnakú mocnosť (je to ich {\it potenčný\/} bod),
preto bod~$P$ leží vnútri úsečky~$CM$.
\inspicture{}

Z~rovnosti obvodových uhlov nad tetivou~$BC$ kružnice~$l$ vyplýva
$|\uhol BMC|=|\uhol BEC|=180\st-|\uhol AEB|$ a~analogicky $|\uhol
AMC|=|\uhol ADC|=180\st-|\uhol ADB|$, čo vzhľadom na rovnosť
obvodových uhlov $|\uhol AEB|=|\uhol ADB|$ nad tetivou~$AB$ kružnice~$k$ znamená, že
$$
|\uhol BMC|=|\uhol AMC|.
$$
Označme $N$ pätu výšky z~vrcholu~$C$ trojuholníka $ABC$. Ak $M\ne N$,
znamená ostatná rovnosť, že pravouhlé trojuholníky $BNM$ a~$ANM$ sú
zhodné, čo však odporuje predpokladu $|AC|\ne |BC|$. Preto
$M=N$, $|\uhol ADC|=|\uhol BMC|=|\uhol AMC|=90\st$ a~bod~$P$ je
priesečníkom výšok trojuholníka $ABC$, čo sme chceli dokázať.
}

{%%%%%   A-III-6
Ak sú $x$, $y$, $z$ tri navzájom rôzne reálne čísla, tak hodnoty
$$
u=\frac{x-y}{y-z},\quad v=\frac{y-z}{z-x},\quad
w=\frac{z-x}{x-y}
\tag1
$$
sú zrejme čísla rôzne od $0$ a~$\m1$ a~ich súčin je rovný~$1$.
Rovnakú vlastnosť teda musia mať aj hodnoty $x$, $y$, $z$ 
z~každej hľadanej trojice. Budeme preto neustále predpokladať, že
$$
x,y,z\in\Bbb R\setminus\{0,\m1\},\quad
x\ne y\ne z\ne x,\quad xyz=1.          \tag2
$$

Keďže daná množinová rovnica je pre usporiadané trojice
$(x,y,z)$, $(z,x,y)$ a~$(y,z,x)$ rovnaká, budeme okrem \thetag2
predpokladať, že $x>\max\{y,z\}$, a~rozoberieme dva prípady
podľa toho, či $y>z$, alebo $z>y$. Zaveďme ešte označenie
intervalov $I_1=(0,\infty)$, $I_2=(\m1,0)$,
$I_3=(\m\infty,\m1)$.

\prip{$x>y>z$}
Pre zlomky~\thetag1 zrejme platí $u\in I_1$,
$v\in I_2$ a~$w\in I_3$, takže $u>v>w$. Daná množinová rovnica
preto môže byť splnená jedine tak, že $u=x$, $v=y$ a~$w=z$.
Po dosadení zlomkov \thetag1 a~jednoduchej úprave dôjdeme k~rovniciam
$$
xy+y=yz+z=zx+x,\quad\text{pričom}\ x\in I_1,\ y\in I_2,\ z\in I_3.
\tag3
$$
Podľa podmienky $xyz=1$ z~\thetag2 môžeme do rovnice $xy+y=zx+x$
za člen~$zx$ dosadiť $1/y$ a~rovnicu ďalej
upraviť:
$$
xy+y=\frac{1}{y}+x\Rightarrow
x(y-1)=\frac{1-y^2}{y}\Rightarrow
x=-\frac{1+y}{y}\Rightarrow
y=-\frac{1}{1+x}.
$$
(Využili sme to, že vzhľadom na $y\in I_2$ platí $y\ne1$.)
Z~ostatného vzťahu vyplýva, že hodnota prvého
výrazu v~sústave~\thetag3 je rovná $\m1$, takže z~rovnosti druhého
výrazu číslu $\m1$ máme
$$
z=-\frac{1}{1+y}=-\frac{1}{1-\dfrac{1}{1+x}}=\m\frac{1+x}{x},
$$
potom však aj tretí výraz v~\thetag3 je rovný $\m1$.
Preto každé riešenie našej úlohy
(v~skúmanom prípade, keď $x>y>z$) má tvar
$$
(x,y,z)=\left(t,-\frac{1}{1+t},-\frac{1+t}{t}\right), \tag4
$$
pričom $t\in I_1$ je ľubovoľné (pretože platí \thetag3, skúška nie je
nutná). Z~uvedeného postupu tiež vyplýva, že
voľbou $t\in I_2$ (resp. $t\in I_3$) vo vzťahu~\thetag4
dostaneme všetky riešenia našej úlohy s~vlastnosťou $z>x>y$ (resp.~
$y>z>x$), takže pri výpise všetkých riešení v~záverečnej odpovedi nie je
nutné uvádzať cyklické permutácie trojíc zo vzťahu~\thetag4.

\prip{$x>z>y$}
Pre zlomky~\thetag1 teraz platí $u\in I_3$,
$v\in I_1$ a~$w\in I_2$, takže $v>w>u$. Daná množinová rovnica
je teda splnená práve vtedy, keď $u=y$, $v=x$ a~$w=z$. Po dosadení
zlomkov z~\thetag1 dôjdeme k~sústave
$$
x-y=y(y-z),\quad y-z=x(z-x),\quad z-x=z(x-y). \tag5
$$
Sčítaním týchto troch rovníc dostaneme
$$
0=y(y-z)+x(z-x)+z(x-y)=(y-x)(x+y-2z),
$$
odkiaľ vzhľadom na $x\ne y$ vyplýva $z=\frac12(x+y)$. Po dosadení
späť do\thetag5 ľahko zistíme (opäť vzhľadom na $x\ne y$), že
vyhovuje iba $x=1$, $y=\m2$ a~$z=\m\frac12$.
Rovnakou trojicou čísel je tvorené (jediné) riešenie úlohy
s~vlastnosťou $y>x>z$ aj (jediné) riešenie, pre ktoré $z>y>x$.

\odpoved
Riešením úlohy sú všetky usporiadané
trojice~\thetag4, pričom $t\in\Bbb R\setminus\{0,\m1\}$, a~tri trojice
$(x,y,z)$ tvaru
$$
\left(1,-2,-\tfrac12\right),\
\left(-\tfrac12,1,-2\right),\
\left(-2,-\tfrac12,1\right).
$$

\poznamka
Ak vypíšeme všetkých šesť možných sústav
prislúchajúcich danej množinovej rovnici, dostaneme okrem sústav \thetag3
a~\thetag5 ešte sústavy
$$
\displaylines{
x-y=z(y-z),  \qquad y-z=y(z-x), \qquad z-x=x(x-y);\cr
x-y=x(y-z),  \qquad y-z=z(z-x), \qquad z-x=y(x-y);\cr
x-y=y(y-z),  \qquad y-z=z(z-x), \qquad z-x=x(x-y);\cr
x-y=z(y-z),  \qquad y-z=x(z-x), \qquad z-x=y(x-y).}
$$
Prvé dve vzniknú zo sústavy~\thetag5 cyklickou zámenou premenných,
takže ich možno riešiť rovnakým postupom ako~\thetag5. Sčítaním všetkých troch rovníc
v~každej z~dvoch zostávajúcich sústav dostaneme tú istú rovnicu
$$
x^2+y^2+z^2=xy+yz+zx,\qquad\hbox{resp.}\qquad(x-y)^2+(y-z)^2+(z-x)^2=0,
$$
ktorá má jediné riešenie $x=y=z$, čo nie sú navzájom rôzne čísla.

\ineriesenie
Ak sú $x$, $y$, $z$ tri navzájom rôzne reálne čísla, tak hodnoty
$$
u=\frac{x-y}{y-z},\quad v=\frac{y-z}{z-x},\quad
w=\frac{z-x}{x-y}
\tag1
$$
sú zrejme rôzne od čísel $0$ a~$\m1$ a~platia medzi nimi vzťahy
$$
v=f(u),\quad w=f(v)\quad\text{a}\quad u=f(w),
\tag2
$$
pričom $f$ je lineárna lomená funkcia daná predpisom
$f(t)=\m\dfrac{1}{1+t}$. Presvedčíme sa o~tom priamym výpočtom:
$$
\postdisplaypenalty 10000
f(u)=-\frac{1}{1+u}=-\frac{1}{1+\dfrac{x-y}{y-z}}     -\frac{y-z}{(x-y)+(y-z)}=\frac{y-z}{z-x}=v;
$$
z~dôvodu cyklickosti platia aj zostávajúce dva vzťahy v~\thetag2.

Uvedený poznatok
znamená, že každé riešenie úlohy je pre vhodné
$t\in\Bbb R\setminus\{0,\m1\}$
buď usporiadaná trojica tvaru
$$
(x,y,z)=\left(t,f(t),f\left(f(t)\right)\right)=\left(t,-\frac{1}{1+t},-\frac{1+t}{t}\right),     \tag3
$$
alebo usporiadaná trojica tvaru
$$
(x,y,z)=\left(t,f\left(f(t)\right),f(t)\right)=\left(t,-\frac{1+t}{t},-\frac{1}{1+t}\right).
\tag4
$$
Ostáva urobiť skúšku: ľahko sa presvedčíme, že zatiaľ čo
trojica tvaru~\thetag3 je riešením pre každé
$t\in\Bbb R\setminus\{0,\m1\}$,
trojica tvaru~\thetag4 vyhovuje iba pre $t=1$,
$t=\m2$ a~$t=\m\frac12$ a~sú to cyklické
permutácie týchto troch hodnôt.
}

{%%%%%   B-S-1
Delením polynómu $x^4+ax^2+b$ polynómom $x^2+bx+a$ zistíme, že
$$
x^4+ax^2+b=(x^2+bx+a)(x^2-bx+b^2)+(ab-b^3)x+(b-ab^2).
$$
Polynóm $x^4+ax^2+b$ je deliteľný polynómom $x^2+bx+a$ práve vtedy, keď je zvyšok $(ab-b^3)x+(b-ab^2)$ nulový polynóm, teda $ab-b^3=b(a-b^2)=0$ a~súčasne $b-ab^2=b(1-ab)=0$. Ak $b=0$, sú obe podmienky splnené. Pre $b\ne0$ musí platiť $a-b^2=0$ a~$1-ab=0$. Odtiaľ $a=b^2$, $1-b^3=0$, a~teda $a=b=1$.

\zaver
Polynóm $x^4+ax^2+b$ je deliteľný polynómom $x^2+bx+a$ práve vtedy, keď $b=0$ (a~$a$ je ľubovoľné) alebo $a=b=1$.

\ineriesenie
Polynóm $x^4+ax^2+b$ je deliteľný polynómom $x^2+bx+a$ práve vtedy, keď existujú také reálne čísla $p$, $q$, že $x^4+ax^2+b=(x^2+bx+a)(x^2+px+q)$. Roznásobením a~porovnaním koeficientov dostaneme sústavu rovníc
$$
p+b=0,\quad q+bp+a=a,\quad ap+bq=0,\quad aq=b.
$$
Z~prvej rovnice vyjadríme $p=\m b$ a~z~druhej $q=\m bp=b^2$, dosadením do tretej a~štvrtej máme $\m ab+b^3=0$, $ab^2=b$. Riešenie dokončíme rovnako ako v~prvom riešení.

\nobreak\medskip\petit\noindent
Za úplné riešenie dajte 6~bodov. 
Za správne určenie zvyšku pri delení dajte 2~body. Ďalšie 2~body za sústavu rovníc $ab-b^3=0$, $b-ab^2=0$ a~2~body za jej vyriešenie. 
Podobne dajte 2~body za rovnicu $x^4+ax^2+b=(x^2+bx+a)(x^2+px+q)$, 2~body za sústavu $p+b=0$, $q+bp+a=a$, $ap+bq=0$, $aq=b$ a~2~body za jej vyriešenie. Za uhádnutie riešenia $b=0$ dajte jeden bod.
\endpetit
\bigbreak
}

{%%%%%   B-S-2
\fontplace
\tpoint A; \tpoint B; \bpoint C; \rtpoint\xy1,-1.3 T;
\lBpoint D; \rBpoint E; \lpoint p;
[6] \hfil\Obr 

Polomer kružnice vpísanej trojuholníku je podielom jeho obsahu a~polovice obvodu.

Trojuholníky $ADE$ a~$BDE$ majú zrejme rovnaký obsah, pretože majú
spoločnú stranu~$DE$ a~zhodnú výšku na ňu ($AB$ a~$DE$ sú rovnobežné).
Rovnaký obsah teda majú aj trojuholníky $ATE$ a~$BDT$, pretože
obsahy oboch trojuholníkov sa od obsahu spomenutých trojuholníkov líšia práve o~obsah
"spoločného" trojuholníka $DET$ (\obr).

\inspicture{}

Označme $p$ os úsečky~$AB$. Ak je strana~$BC$ dlhšia ako strana~$AC$, leží bod~$C$ v~tej istej polrovine s~hraničnou priamkou~$p$ ako bod~$A$. Preto v~tejto polrovine leží aj ťažisko~$T$. Jeho vzdialenosť od bodu~$A$ je teda menšia ako vzdialenosť od bodu~$B$. To znamená, že dĺžka~$t_a$ ťažnice~$AD$ je menšia ako dĺžka~$t_b$ ťažnice~$BE$. Trojuholník $ATE$ má obvod $o_1=\frac12 b+\frac13 t_b+\frac23 t_a$, trojuholník $BDT$ má obvod $o_2=\frac12 a+\frac13 t_a+\frac23 t_b$. Z~nerovností $b<a$ a $t_a<t_b$ preto vyplýva
$$
o_2-o_1=\frac12(a-b)+\frac13(t_b-t_a)>0,
$$
čiže $o_1<o_2$. 

Trojuholníky $AET$ a~$BDT$ majú rovnaký obsah a~prvý z~nich má menší obvod, preto má kružnica vpísaná trojuholníku $AET$ väčší polomer ako kružnica vpísaná trojuholníku $BDT$.

\nobreak\medskip\petit\noindent
Za úplné riešenie dajte 6~bodov.
1~bod dajte za vyjadrenie polomeru vpísanej kružnice pomocou obsahu a~obvodu trojuholníka, 2~body za dôkaz rovnosti obsahov trojuholníkov $AET$ a~$BDT$, 2~body za dôkaz nerovnosti $o_1<o_2$ a~1~bod za z~toho vyplývajúcu nerovnosť medzi polomermi.
\endpetit
\bigbreak
}

{%%%%%   B-S-3
Číslo $n^{2}+15n=n(n+15)$ má byť deliteľné číslom $33\,000=2^3\cdot3\cdot5^3\cdot11$. Keby nebolo $n$ deliteľné tromi, nebolo by tromi deliteľné ani číslo $n+15$, čiže ani súčin $n(n+15)$. Z~rovnakého dôvodu musí byť $n$ deliteľné piatimi, a~teda aj pätnástimi. Píšme preto $n=15k$. Aby bolo číslo $n(n+15)=15^2k(k+1)$ deliteľné číslom 
$8\cdot3\cdot5^3\cdot11$, musí byť súčin $k(k+1)$ dvoch po sebe idúcich prirodzených čísel deliteľný ôsmimi, piatimi a~jedenástimi. Jeden z~činiteľov $k$, $k+1$ musí byť deliteľný aspoň dvoma z~týchto troch čísel, takže musí byť deliteľný niektorým z~čísel $40$, $55$, $88$. Najmenším takým číslom je 40. Jedenástimi ale nie je deliteľné ani číslo~40 ani žiaden z~jeho susedov 39, 41. Ďalším kandidátom je číslo~55, súčin čísel 5 a~11. O~jednotku väčšie číslo~56 je zase deliteľné ôsmimi. Preto súčin $55\cdot56$ je deliteľný ôsmimi, piatimi aj~jedenástimi; máme teda $k=55$. 
Hľadané najmenšie číslo je $n=15k=825$.

\nobreak\medskip\petit\noindent
Za úplné riešenie dajte 6~bodov.
2~body dajte za dôkaz toho, že $n$ je deliteľné pätnástimi; 1~bod za zistenie, že je potrebné hľadať dve po sebe idúce prirodzené čísla, ktorých súčin je deliteľný piatimi, ôsmimi a~jedenástimi; 1~bod za poznatok, že jedno z~hľadaných čísel $k$ a~$k+1$ musí byť deliteľné dvoma z~čísel $5$, $8$ a~$11$; 2~body potom za správne určenie najmenšieho~$n$.
\endpetit
\bigbreak 
}

{%%%%%   B-II-1
Delením polynómu $x^4+ax^2+bx+c$ polynómom $x^2+x+1$ zistíme, že platí
$$
x^4+ax^2+bx+c=(x^2+x+1)(x^2-x+a)+(b-a+1)x+(c-a).
$$
Polynóm $x^4+ax^2+bx+c$ je deliteľný polynómom $x^2+x+1$ práve vtedy, keď je zvyšok pri delení nulový polynóm, teda $b-a+1=0$ a~súčasne $c-a=0$; odtiaľ $b=a-1$, $c=a$.
Potom
$$
a^2+b^2+c^2=a^2+(a-1)^2+a^2=3a^2-2a+1=3\left(a-\tfrac13\right)^2+\tfrac23.
$$
Tento výraz má najmenšiu hodnotu pre $a=\frac13$; ľahko dopočítame $b=a-1=\m\frac23$, $c=a=\frac13$.

\ineriesenie 
Polynóm $x^4+ax^2+bx+c$ je deliteľný polynómom $x^2+x+1$ práve vtedy, keď existujú reálne čísla $p$, $q$, pre ktoré 
$$
x^4+ax^2+bx+c=(x^2+x+1)(x^2+px+q).
$$
Roznásobením pravej strany a~porovnaním koeficientov dostaneme štyri rovnice $p+1=0$, $q+p+1=a$, $q+p=b$, $q=c$. Z~nich vyjadríme $p=\m1$, $q=a$, $c=a$, $b=a-1$ a~pokračujeme ako v~prvom riešení.

\nobreak\medskip\petit\noindent
Za úplné riešenie dajte 6~bodov.
Dajte dva body za rovnosť $x^4+ax^2+bx+c=(x^2+x+1)(x^2-x+a)+(b-a+1)x+(c-a)$, jeden bod za vyjadrenie dvoch z~neznámych $a$, $b$, $c$ pomocou tretej z~nich, dva body za úpravu výrazu $a^2+b^2+c^2$ na tvar $3a^2-2a+1=3\left(a-\frac13\right)^2+\frac23$ (alebo $3\left(b+\frac23\right)^2+\frac23$) a~jeden bod za správne určenie čísel $a$, $b$, $c$.
Podobne dajte jeden bod za rovnosť $x^4+ax^2+bx+c=(x^2+x+1)(x^2+px+q)$, jeden bod za sústavu rovníc $p+1=0$, $q+p+1=a$, $q+p=b$, $q=c$, jeden bod za vyjadrenie $c=a$, $b=a-1$ a~ďalej ako v~prvom riešení.
\endpetit
\bigbreak
}

{%%%%%   B-II-2
\epsplace b56.7 \hfil\Obr

\inspicture{}
Označme $D$ stred strany~$AC$, $E$ stred strany~$BC$ a~$T$ ťažisko trojuholníka $ABC$ (\obr). Ak ďalej označíme $3x$ a~$3y$ dĺžky ťažníc $t_a$ a~$t_b$, máme $|AT|=2x, |ET|=x, |BT|=2y, |DT|=y$.
Podľa Pytagorovej vety pre trojuholníky $ATD, BET, ABT$ platí
$$
\align
(2x)^2+y^2&=\left(\frac b2\right)^{\!2},\\
x^2+(2y)^2&=\left(\frac a2\right)^{\!2},\\
(2x)^2+(2y)^2&=c^2.
\endalign
$$
Sčítaním prvých dvoch rovníc dostaneme $5(x^2+y^2)=\frac14(a^2+b^2)$  a~po dosadení do tretej rovnice máme $c^2=4(x^2+y^2)=\frac15(a^2+b^2)$. Numericky potom $c^2=\frac15(22^2+19^2)=169$, a~teda $c=13\cm$.

\nobreak\medskip\petit\noindent
Za úplné riešenie dajte 6~bodov.
Jeden bod dajte za využitie toho, že ťažisko delí ťažnicu v~pomere $1:2$, po jednom bode za použitie Pytagorovej vety pre trojuholníky $ATD$, $BET$, $ABT$ a~dva body za výpočet dĺžky strany~$AB$.
\endpetit
\bigbreak
}

{%%%%%   B-II-3
Číslice 0, 1, 2, 3 a~4 nazvime malé (skrátene~$m$), číslice 5, 6, 7, 8 a~9 veľké (skrátene~$v$). Pravidelným striedaním malých a~veľkých číslic vznikne vždy vlnité číslo. 

Čísel tvaru $vmvmvmvmvm$ je $(5!)^2$, čísel tvaru $mvmvmvmvmv$ je $4\cdot4!\cdot5!$ (na prvom mieste nesmie byť~0). Tých vlnitých čísel, ktoré vzniknú pravidelným striedaním malých a~veľkých číslic, je teda 
$5!\cdot5!+4\cdot4!\cdot5!=9\cdot24\cdot120=25\,920>25\,000$.

\poznamka
Všetkých desaťciferných vlnitých čísel s~rôznymi číslicami je $93\,106$.

\nobreak\medskip\petit\noindent
Za úplné riešenie dajte 6~bodov.
Jeden bod dajte za poznatok, že pravidelným striedaním malých a veľkých číslic vznikne vlnité číslo. Po dvoch bodoch za správne určenie počtu vlnitých čísel tvaru $vmvmvmvmvm$ a tvaru $mvmvmvmvmv$ a jeden bod za dokončenie dôkazu.
\endpetit
\bigbreak
}

{%%%%%   B-II-4
\epsplace b56.8 \hfil\Obr

Keďže sú uhly $LKC$ a~$LMC$ pravé, ležia body $K$ a~$M$ na Tálesovej kružnici nad priemerom~$CL$ (\obr).
\inspicture{}
Podľa vety o~obvodovom uhle prislúcha tetive~$KM$ stredový uhol veľkosti~$2\gamma$, a~preto $|KM|=|CL|\sin\gamma$ (v~pravouhlom trojuholníku $KPS$, kde $P$ je stred úsečky~$KM$ a~$S$ stred úsečky~$CL$, je totiž $|KS|=\frac12|CL|, |\uhol KSP|=\gamma$). Úsečka~$KM$ je teda najkratšia práve vtedy, keď je najkratšia úsečka~$CL$; to nastáva práve vtedy, keď $L$ je päta kolmice z~bodu~$C$ na stranu~$AB$. 

\nobreak\medskip\petit\noindent
Za úplné riešenie dajte 6~bodov. Dva body dajte za poznatok, že body $K$ a~$M$ ležia na Tálesovej kružnici nad priemerom~$LC$, dva body za dôkaz rovnosti $|KM|=|LC|\sin\gamma$ a~dva body za z~toho vyplývajúcu polohu bodu~$L$.
\endpetit
\bigbreak
}

{%%%%%   C-S-1
Aby číslo bolo deliteľné šiestimi, musí byť párne a~mať ciferný
súčet deliteľný tromi. Označme teda $b$ číslicu na mieste
jednotiek (tá musí byť párna, $b\in\{0,2,4,6,8\}$) a~$a$
tú číslicu, ktorá je spolu s~číslicami 1, 1 ($a\ne1$) na prvých troch
miestach štvorciferného čísla, ktoré spĺňa požiadavky úlohy.

Aby bol súčet číslic $a+1+1+b$ takého čísla deliteľný tromi,
musí číslo $a+b$ dávať po delení tromi zvyšok~1. Pre
$b\in\{0,6\}$ tak máme pre~$a$ možnosti $a\in\{4,7\}$ ($a\ne1$),
pre $b\in\{2,8\}$ máme $a\in\{2,5,8\}$ a~konečne pre $b=4$ máme
$a\in\{0,3,6,9\}$. Pre každé zvolené~$b$ a~zodpovedajúce~$a\ne0$
sú zrejme tri možnosti, ako číslice 1, 1 a~$a$ na prvých troch
miestach usporiadať, to je spolu $(2\cdot2+2\cdot3+3)\cdot3=39$
možností, pre $a=0$ (keď $b=4$) potom sú len dve možnosti
(číslica nula nemôže byť prvá číslica štvorciferného čísla).

Celkom existuje 41~štvorciferných prirodzených čísel, ktoré spĺňajú
podmienky úlohy.

\nobreak\medskip\petit\noindent
Za úplné riešenie dajte 6~bodov, z~toho 3~body za využitie poznatku,
že posledná číslica musí byť párna. Pokiaľ však niekto rozoberie
všetkých 81~prípadov zostávajúcich dvoch číslic a~k~nim správne určí
počet vyhovujúcich čísel, má tiež nárok na 6~bodov. Pri popísaní
efektívneho postupu za prípadné aritmetické chyby pri určení
konečného počtu nestrhávajte viac ako 2~body.
\endpetit
\bigbreak
}

{%%%%%   C-S-2
\fontplace
\rpoint S; \tpoint A; \lBpoint B; \lpoint\down.5\unit C; \bpoint D;
\tpoint K; \bpoint L; \rBpoint k;
[7] \hfil\Obr 

Dotyčnica ku kružnici~$k$ v~bode~$A$ je kolmá na priemer~$AD$,
a~teda aj na stranu~$BC$ daného šesťuholníka (\obr). Zároveň
priamky $SB$ a~$AB$ zvierajú s~$BC$ šesťdesiatstupňový uhol, takže
sú súmerne združené podľa osi~$BC$. Bod~$K$ je preto
súmerne združený s~bodom~$A$ podľa osi~$BC$.
\inspicture{}

Podobne dotyčnica~$BL$ je kolmá na $BS$, takže zviera s~priamkou~$BC$
uhol $30\st$ rovnako ako priamka~$BD$. Priamka~$BL$ je teda
súmerne združená s~priamkou~$BD$ podľa osi~$BC$. Aj priamky
$SC$ a~$CD$ sú súmerne združené podľa osi~$BC$, takže bod~$L$
je podľa tejto osi súmerne združený s~bodom~$D$.

Dostali sme tak, že štvoruholník $KLCB$ je súmerne združený
s~lichobežníkom $ADCB$, ktorému je opísaná kružnica~$k$. Vrcholy
štvoruholníka $KLCB$ preto ležia na kružnici súmerne združenej
s~kružnicou~$k$ podľa osi~$BC$. Tým je tvrdenie úlohy dokázané.


\nobreak\medskip\petit\noindent
Za úplné riešenie dajte 6~bodov, z~toho 2~body za rozhodnutie
riešiteľa dokazovať hypotézu o~zhodnosti štvoruholníkov $ABCD$
a~$KLCB$. Pri využití osovej súmernosti dajte 2~body za dôkaz
združenosti bodov $A$, $K$ a~2~body za združenosť bodov $D$, $L$.
Namiesto súmernosti možno dokazovať zhodnosť príslušných trojuholníkov.
Možno tiež najskôr zostrojiť obrazy $A'$, $D'$ vrcholov $A$
a~$D$ v~súmernosti podľa osi~$BC$ a~potom ukázať, že $A'=K$ 
a~$D'=L$.
\endpetit
\bigbreak
}

{%%%%%   C-S-3
Z~rovnosti $a-4\sqrt b=b+4\sqrt{a\vphantom b}$ vyplýva rovnosť
$a-b=4\bigl(\sqrt{a\vphantom b}+\sqrt b\bigr)$. Keďže
$$
a-b=\bigl(\sqrt{a\vphantom b}-\sqrt b\bigr)\bigl(\sqrt{a\vphantom b}+\sqrt b\bigr),
$$
dostávame po vydelení
kladným číslom $\sqrt{a\vphantom b}+\sqrt b$ rovnosť
$$
\sqrt{a\vphantom b}-\sqrt b=4,   \tag1
$$
čiže
$$
\sqrt{a\vphantom b}=\sqrt b+4.    \tag2
$$
Jej umocnením vyjde $a=b+16+8\sqrt b$.
Keďže číslo $r=8\sqrt b$ musí byť celé,
je $\sqrt b$ racionálna odmocnina prirodzeného čísla,
%% z~rovnosti $r^2=8^2b$ úvahou o~prvočinitelích zjišťujeme,
takže $b=n^2$ pre vhodné prirodzené číslo~$n$.
Z~rovnosti~\thetag2 tak máme $a=(n+4)^2$ a~$a-b=(n+4)^2-n^2=2^3(n+2)$.
Číslo $a-b$ je teda piatou mocninou prvočísla len vtedy, keď $n+2=2^2$,
čiže $n=2$.

Jedinou vyhovujúcou dvojicou $(a,b)$ je dvojica $(36,4)$.

\poznamka
Keď si po odvodení vzťahu~\thetag1 uvedomíme, že v~zátvorke na
pravej strane rovnosti
$a-b=4(\sqrt{a\vphantom b}+\sqrt b)$ je kladné
racionálne, a~teda prirodzené číslo, vidíme, že musí platiť $a-b=2^5$.
Pre odmocniny $\sqrt{a\vphantom b}$, $\sqrt b$ tak dostaneme
sústavu dvoch rovníc
$$
\align
\sqrt{a\vphantom b}+\sqrt b&=8,\\
\sqrt{a\vphantom b}-\sqrt b&=4,
\endalign
$$
ktorých sčítaním vyjde $\sqrt{a\vphantom b}=6$
a~odčítaním $\sqrt b=2$.

%%  $r$, pro které platí
%% $\sqrt b=r-\sqrt{a\vphantom b}$, takže $b=r^2-2r\sqrt{a\vphantom
%% b}+a$, neboli
%% $$
%% \sqrt{a\vphantom b}={r^2+a-b\over2r}.
%% $$
%% Odmocnina $\sqrt{a\vphantom b}$ jako racionální odmocnina
%% přirozeného čísla je tudíž celočíselná, což vzhledem k~symetrii
%% použitého vztahu platí i~pro odmocninu $\sqrt b$.

\nobreak\medskip\petit\noindent
Za úplné riešenie dajte 6~bodov. Poznatok, že odmocnina
z~prirodzeného čísla je buď iracionálne, alebo prirodzené číslo,
nie je nutné dokazovať (poz.\ aj prvú pomocnú úlohu k~1.~úlohe
domáceho kola).
\endpetit
\bigbreak
}

{%%%%%   C-II-1
\fontplace
\rpoint S; \rpoint A; \lpoint B; \bpoint C;
\rBpoint L; \bpoint\xy.5,.4 M;
\lBpoint\down\unit d; \lBpoint\down\unit d;
\lpoint t; \lpoint p; \rBpoint k;
[8] \hfil\Obr

Pri {\it rozbore\/} uvažujme ľubovoľný trojuholník $ABC$ s~vrcholom~$C$ na
kružnici~$k$, ktorého strany $AC$, $BC$ majú stredy postupne
v~bodoch $L$, $M$ (\obr). Keďže $LM$ je strednou priečkou
takého trojuholníka, je jeho obsah rovný štvornásobku obsahu trojuholníka
$LMC$. Tento trojuholník má pevnú stranu~$LM$, takže jeho obsah je
najväčší práve vtedy, keď je najväčšia jeho výška z~vrcholu~$C$, teda
vzdialenosť~$d$ bodu~$C$ od priamky~$p$ určenej bodmi $L$, $M$.
\inspicture{}

Dodajme, že namiesto porovnania obsahov trojuholníkov $ABC$ a~$LMC$ dôjdeme k~rovnakej
podmienke aj takto: trojuholník $ABC$ má stranu~$AB$ pevnej dĺžky
$c=2|LM|$ a~výšku $v_c=2d$. Preto je jeho obsah $\frac12cv_c$
rovný $2|LM|\cdot d$, takže je najväčší možný, keď je taká
vzdialenosť~$d$.

Pre ktorý bod $C\in k$ je vzdialenosť~$d$ najväčšia? Veďme bodom~$C$
priamku~$t$ rovnobežnú s~priamkou~$p$. Ak je vzdialenosť~$d$
najväčšia možná, musí celá kružnica~$k$ ležať v~rovnakej polrovine
s~hraničnou priamkou~$t$ ako priamka~$p$ (voľbou bodu $C\in k$
vnútri opačnej polroviny by sme vzdialenosť~$d$ zväčšili).
Priamka~$t$ je preto nutne dotyčnicou kružnice~$k$ (rovnobežnou 
s~danou priamkou~$p$) a~bod~$C$ je jej dotykovým bodom.

Odtiaľ už vyplýva {\it konštrukcia}: bod~$C$ určíme ako ten z~dvoch
priesečníkov kružnice~$k$ s~kolmicou na priamku~$p$ vedenou
stredom~$S$ kružnice~$k$, ktorý má od priamky~$p$ väčšiu vzdialenosť
(ak ju majú oba priesečníky rovnakú, vyberieme ktorýkoľvek z~nich).
Body $A$, $B$ potom zostrojíme ako obrazy bodu~$C$ v~súmernosti
podľa stredu~$L$, resp.~$M$.
%% (nebo jako obrazy bodů $L$, $M$ ve
%% stejnolehlosti se středem~$C$ a~koeficientem stejnolehlosti~2).

{\it Diskusia}. Dotyčnice kružnice~$k$ rovnobežné s~priamkou~$LM$
majú od tejto priamky dve {\it rôzne\/} vzdialenosti práve vtedy, keď
stred~$S$ kružnice~$k$ na priamke~$LM$ {\it neleží}; vtedy má
úloha jediné riešenie. V~opačnom prípade, keď stred~$S$ na priamke~$LM$
leží, má úloha dve riešenia.

\nobreak\medskip\petit\noindent
Za úplné riešenie dajte 6~bodov, z~toho 3~body za nájdenie podmienky
maximálnej vzdialenosti $C$ od $LM$, 2~body za postup konštrukcie 
a~1~bod za správne určenie oboch možných počtov riešení. Intuitívne
jasné určenie najvzdialenejšieho bodu kružnice~$k$ od priamky~$LM$
nie je nutné zdôvodňovať (tretí odstavec uvedeného riešenia).
\endpetit
\bigbreak
}

{%%%%%   C-II-2
a) Ak spĺňajú prirodzené čísla $p$, $q$, $r$ danú rovnicu,
dostaneme z~nej vyjadrenie
$$
\sqrt{p+q}=\frac{2\,007-p-q}{r},
$$
takže číslo $\sqrt{p+q}$ je racionálne, a~teda celé (odmocnina
z~prirodzeného čísla je totiž buď číslo celé, alebo číslo iracionálne).
Preto z~rovností
$$
2\,007=p+r\sqrt{p+q}+q(p+q)+r\sqrt{p+q}=\sqrt{p+q}\left(\sqrt{p+q}+r\right)
$$
dostávame rozklad čísla $2\,007$ na dva celočíselné činitele
$\sqrt{p+q}$ a~$\sqrt{p+q}+r$, pre ktoré zrejme platí
$$
1<\sqrt{p+q}<\sqrt{p+q}+r.
$$
Z~rozkladu na prvočísla $2\,007=3^2\cdot223$ vidíme, že
sú možné iba dva prípady, ktoré prehľadne
zapíšeme do schémy:
% $$
% \vbox{\offinterlineskip\let\\=\cr
%       \everycr{\noalign{\hrule}}
% \halign{\strut\vrule#&&\hss\enspace$#$\enspace\hss\vrule\cr
% &\sqrt{p+q}&\sqrt{p+q}+r&p+q&r&p+q+r\\
% &3&669&9&666&675\\
% &9&223&81&214&295\cr
% }}
% $$
%% nebo Karle méně tradičně, ale výstižněji:
$$
\matrix
\sqrt{p+q}&\sqrt{p+q}+r\\
3&669\\
9&223
\endmatrix
\quad \Longleftrightarrow\quad
\matrix
p+q&r\\
9&666\\
81&214
\endmatrix
\quad \Longrightarrow\quad
\matrix
p+q+r\\
675\\
295
\endmatrix
$$
Možné hodnoty súčtu $p+q+r$ sú teda iba dve čísla: 675 
a~295. (Konkrétne trojice $(p,q,r)$, ktoré to dokazujú, nebudeme
uvádzať, pretože priamo určíme v~časti~b) ich počet.)

b) Rovnosť $p+q+r=675$ nastane práve vtedy, keď bude trojica $(p,q,r)$
spĺňať podmienky $p+q=9$  a~$r=666$; takých trojíc je práve
toľko, ako dvojíc $(p,q)$, pre ktoré $p+q=9$, teda~8.

Rovnosť $p+q+r=295$ nastane práve vtedy, keď bude trojica $(p,q,r)$
spĺňať podmienky $p+q=81$ a~$r=214$; takých trojíc je práve
toľko, ako dvojíc $(p,q)$, pre ktoré $p+q=81$, teda 80.

\nobreak\medskip\petit\noindent
Za úplné riešenie dajte 6~bodov,
z~toho 5~bodov za časť~a) a~1~bod za časť~b). Za časť~a)
riešenia dajte len 3~body, ak chýba v~inak úplnom postupe
zdôvodnenie, prečo je hodnota $\sqrt{p+q}$ celé číslo.
\endpetit
\bigbreak
}

{%%%%%   C-II-3
\fontplace
\rtpoint\xy.5,-1.2 O; \tpoint A; \tpoint B; \bpoint C; \bpoint D;
\tpoint K; \lBpoint L; \bpoint M; \rBpoint N;
\lpoint r; \tpoint r; \lpoint r; \tpoint r;
[9] \hfil\Obr

Označme postupne $K$, $L$, $M$, $N$ body dotyku vpísanej kružnice
so stranami $AB$, $BC$, $CD$, $DA$ (\obr). Keďže $ABCD$
je rovnoramenný lichobežník, jeho vnútorné uhly pri vrcholoch $A$,
$B$, $C$, $D$ majú postupne veľkosti $\al$, $\al$,
\inspicture{}
$180^{\circ}-\al$ a~$180^{\circ}-\al$. Úsečky $OA$, $OB$, $OC$,
$OD$ ležiace na osiach týchto uhlov preto spolu so štyrmi navzájom
zhodnými úsečkami $OK$, $OL$, $OM$, $ON$ rozdeľujú celý
lichobežník na osem pravouhlých trojuholníkov, ktoré sa zhodujú v~jednej
odvesne a~majú ostré vnútorné uhly $\frac12\al$ 
a~$90^{\circ}-\frac12\al$. Týchto osem trojuholníkov možno preto rozdeliť na
dve štvorice zhodných trojuholníkov: jednu z~nich tvoria trojuholníky $OAK$,
$OAN$, $OBK$, $OBL$ a~druhú trojuholníky $OCL$, $OCM$, $ODM$ a~$ODN$.
Odtiaľ vyplýva, že obsah~$S$ lichobežníka $ABCD$ je rovný
štvornásobku súčtu obsahov trojuholníkov $OBL$ a~$OCL$, teda štvornásobku
obsahu trojuholníka $OBC$. Podľa vnútorných uhlov pri vrcholoch $B$ a~$C$
vidíme, že trojuholník $OBC$ je pravouhlý s~odvesnami $OB$ a~$OC$, takže
má obsah $\frac12|OB|\cdot|OC|$ a~hľadaný celkový obsah~$S$ je $S=2|OB|\cdot|OC|$.

\poznamka
Malá obmena časti predchádzajúceho postupu: ak je $O$ stred kružnice
vpísanej dotyčnicovému štvoruholníku $ABCD$, je ľahké ukázať, že jeho
obsah je rovný dvojnásobku súčtu obsahov trojuholníkov $OAB$ a~$OCD$
rovnako ako trojuholníkov $OBC$ a~$ODA$. Ostatné dva trojuholníky
sú pri našom rovnoramennom lichobežníku $ABCD$ zhodné.

\ineriesenie %%(Mirkovo)řešení.
Pre výšku~$v$ a~strany $a$, $b$, $c$, $d$ lichobežníka $ABCD$
s~vpísanou kružnicou $k(O,r)$ platia rovnosti $v=2r$ a~$a+c=b+d$.
Z~prvej z~nich vyplýva, že stred~$O$ leží na strednej priečke
lichobežníka, ktorej dĺžka $\frac12(a+c)$ je podľa druhej rovnosti
rovná $\frac12(b+d)$. V~našom prípade však platí $b=d$,
takže stredná priečka je zhodná s~oboma ramenami a~bod~$O$ je
jej stredom, lebo rovnoramenný lichobežník je osovo súmerný.
Spolu dostávame, že bod~$O$ leží na kružnici zostrojenej
nad priemerom~$BC$, a~preto je $OBC$ pravouhlý trojuholník s~obsahom
$\frac12|OB|\cdot|OC|$. Jeho výška na preponu~$BC$ je však
polomerom~$r$ vpísanej kružnice~$k$, takže obsah trojuholníka $OBC$ je
tiež rovný $\frac12b\cdot r$. Porovnaním oboch vyjadrení
dostaneme rovnosť $|OB|\cdot|OC|=b\cdot r$. Pre hľadaný obsah~$S$
nášho lichobežníka preto platí
$$
S=\frac{a+c}{2}\cdot v=b\cdot 2r=2\cdot|OB|\cdot|OC|.
$$

\nobreak\medskip\petit\noindent
Za úplné riešenie dajte 6~bodov, z~toho
pri prvom postupe
3~body za zdôvodnenie, že daný lichobežník
je zložený z~dvoch štvoríc zhodných trojuholníkov, alebo za rovnosť
typu $S_{OAB}+S_{OCD}=S_{OBC}+S_{ODA}$. Za hlbší poznatok
$S=4\cdot S_{OBC}$ už dajte 4~body.
1~bodom oceňte zistenie, že $OBC$ je pravouhlý trojuholník,
rovnako ako postup, keď
riešiteľ iba rozdelí daný lichobežník na štyri dvojice
zhodných trojuholníkov a~ďalší pokrok v~úvahách o~ich obsahu
nedosiahne.
\endpetit
\bigbreak
}

{%%%%%   C-II-4
Ľubovoľné $m$-ciferné prirodzené číslo~$N$ s~prvou číslicou~$c$
má vyjadrenie $N=c\cdot10^m+x$, pričom $x$ je práve to číslo, ktoré
dostaneme z~čísla $N$ po škrtnutí prvej číslice~$c$.
Podľa zadania má platiť
$N=c\cdot10^m+x=kx$, čiže $c\cdot10^m=(k-1)x$. Číslo $k-1$
teda musí byť deliteľom čísla $c\cdot10^m$,
ktoré má však iba jednociferné prvočinitele: prvočísla
2, 5 a~prvočinitele z~rozkladu číslice~$c$.
Budeme preto postupne testovať na prvočinitele čísla $k-1$ pre
najväčšie dvojciferné $k$:

$k=99$: $k-1=98=2\cdot7^2$ nevyhovuje, lebo
 $7^2\nmid c\cdot10^m$.
$k=98$: $k-1=97$ nevyhovuje,
 lebo $97$ je dvojciferné prvočíslo.
$k=97$: $k-1=96=2^5\cdot3$ vyhovuje, lebo napríklad
 $2^5\cdot3\mid c\cdot10^m$ pre $c=3$ a~$m=5$; aby sme dostali
menšie~$N$, môžeme však zvoliť menšie $m=4$ a~$c=3\cdot2=6$
(iné~$c$ pre $m=4$ nevyhovuje). Pre $m\le3$ už vzťah
$2^5\cdot3\mid c\cdot10^m$ neplatí pre žiadnu nenulovú číslicu~$c$.

Hľadané najväčšie dvojciferné~$k$ je teda $97$. Podľa predchádzajúcej
diskusie určíme najmenšie vyhovujúce~$N$, ktorému prislúcha $m=4$,
$c=6$ a~$x=6\cdot10^{4}:96=625$, takže $N=6\cdot10^4+625=60\,625$.

\odpoved
Hľadané $k$ je rovné $97$ a~najmenšie vyhovujúce~$N$ je $60\,625$.

\nobreak\medskip\petit\noindent
Za úplné riešenie dajte 6~bodov, z~toho
3~body za všeobecné zistenie,
že číslo $k-1$ musí byť deliteľom súčinu nenulovej číslice
a~mocniny čísla~10. Za úplné treba považovať
aj riešenie, keď sú vylúčené hodnoty $k=99$, $k=98$ oddelenými
postupmi a~pre hodnotu $k=97$ je určené (nie však uhádnuté)
najmenšie vyhovujúce~$N$. Pri inak úplnom riešení,
keď je pre $k=97$ uvedené síce vyhovujúce,
nie však najmenšie možné~$N$ (napríklad $N=303\,125$), dajte
5~bodov.
\endpetit
\bigbreak
}

{%%%%%   vyberko, den 1, priklad 1
...}

{%%%%%   vyberko, den 1, priklad 2
...}

{%%%%%   vyberko, den 1, priklad 3
...}

{%%%%%   vyberko, den 1, priklad 4
...}

{%%%%%   vyberko, den 2, priklad 1
...}

{%%%%%   vyberko, den 2, priklad 2
...}

{%%%%%   vyberko, den 2, priklad 3
...}

{%%%%%   vyberko, den 2, priklad 4
...}

{%%%%%   vyberko, den 3, priklad 1
...}

{%%%%%   vyberko, den 3, priklad 2
...}

{%%%%%   vyberko, den 3, priklad 3
...}

{%%%%%   vyberko, den 3, priklad 4
...}

{%%%%%   vyberko, den 4, priklad 1
...}

{%%%%%   vyberko, den 4, priklad 2
...}

{%%%%%   vyberko, den 4, priklad 3
...}

{%%%%%   vyberko, den 4, priklad 4
...}

{%%%%%   vyberko, den 5, priklad 1
...}

{%%%%%   vyberko, den 5, priklad 2
...}

{%%%%%   vyberko, den 5, priklad 3
...}

{%%%%%   vyberko, den 5, priklad 4
...}

{%%%%%   trojstretnutie, priklad 1
Konštantný mnohočlen $P(x)=c$ vyhovuje práve vtedy, keď
$c=c^2$, mnohočleny $P(x)=0$ a~$P(x)=1$ sú teda riešením úlohy.

Ukážme teraz, že jediný vyhovujúci mnohočlen~$P$ kladného stupňa~$n$
je tvaru $P(x)=(x-1)^n$. Uvedený mnohočlen je vzhľadom 
na identitu $x^2-1=(x-1)(x+1)$ zrejme riešením pre každé $n\ge1$.

Ak je $ax^n$ ($a\ne0$) vedúci člen mnohočlena $P(x)$ kladného
stupňa~$n$, je $ax^{2n}$ vedúci člen mnohočlena $P(x^2)$ 
a~$a^2x^{2n}$ vedúci člen mnohočlena $P(x)P(x+2)$. Pokiaľ $P$
vyhovuje danej rovnosti, dostávame porovnaním príslušných členov
$a=a^2$, teda $a=1$. Preto možno mnohočlen~$P$ zapísať v~tvare
$P(x)=(x-1)^n+Q(x)$, kde $Q$ je buď nulový mnohočlen,
alebo je $Q$ nenulový mnohočlen stupňa~$k$, pričom $0\le k<n$.
Porovnaním mnohočlenov
$$
\align
    P(x^2)&=(x^2-1)^n+Q(x^2),\\
P(x)P(x+2)&=[(x-1)^n+Q(x)][(x+1)^n+Q(x+2)]
\endalign
$$
obdržíme (po roznásobení a~zrušení mocniny $(x^2-1)^n$ na oboch
stranách) rovnosť
$$
Q(x^2)=(x-1)^nQ(x+2)+(x+1)^nQ(x).
$$
Vidíme, že nulový mnohočlen~$Q$ vzťah spĺňa. Pre nenulový
mnohočlen~$Q$ stupňa $k<n$ je však $Q(x^2)$ mnohočlen stupňa
$2k$, zatiaľ čo na pravej strane odvodeného vzťahu
je mnohočlen stupňa $n+k$ (jeho vedúci člen je $2bx^{n+k}$,
ak $bx^k$ je vedúci člen mnohočlena $Q(x)$). Keďže $2k<n+k$,
nemôže uvedená rovnosť platiť.

\odpoved
Úlohe vyhovujú konštantné mnohočleny $P(x)=0$ 
a~$P(x)=1$ a~pre každé prirodzené $n$ mnohočlen $P(x)=(x-1)^{n}$.

\ineriesenie
Rovnako ako pri prvom postupe nájdeme riešenia $P(x)=0$ a~$P(x)=1$. Ďalej sa zaoberajme len nekonštantnými mnohočlenmi. Predpokladajme, že mnohočlen~$P$ kladného stupňa~$n$ vyhovuje zadaniu. Keďže zadaná rovnosť platí pre všetky reálne čísla~$x$, sú na oboch stranách rovnosti totožné mnohočleny, a~teda zadaná rovnosť platí aj pre všetky komplexné čísla~$x$. Nech $z$ je ľubovoľný (komplexný) koreň polynómu~$P$. 

Po dosadení $x=z$ do zadanej rovnosti dostaneme $P(z^2)=0$, teda aj $z^2$ je koreňom~$P$. Zopakovaním tejto úvahy dostávame, že koreňmi mnohočlena~$P$ sú všetky členy postupnosti
$$
z,z^2,z^4,z^8,\dots
\tag1
$$
Keďže $P$ má len konečne veľa (najviac $n$) rôznych komplexných koreňov, musia sa v~postupnosti~\thetag1 hodnoty od určitého člena začať opakovať, \tj. $z^k=z^m$ pre nejaké $k<m$. Odtiaľ buď $z=0$, alebo $z^{m-k}=1$. Každý nenulový koreň má teda absolútnu hodnotu~$1$.

Po dosadení $x=z-2$ do zadanej rovnosti dostaneme $P((z-2)^2)=0$, teda aj $(z-2)^2$ je koreňom. Ak $z=0$, dostávame, že číslo $4$ je koreňom, čo je v~rozpore s~predošlým poznatkom. Takže všetky korene sú nenulové. Nutne teda $|(z-2)^2|=1$, čiže $|z-2|=1$. Jediné komplexné číslo s~vlastnosťou $|z|=|z-2|=1$ je $z=1$. Takže $P$ môže mať jedine koreň $1$, čiže $P(x)=a(x-1)^n$. Dosadením do zadanej rovnosti ľahko odvodíme $a=1$ a~overíme, že $P(x)=(x-1)^n$ je riešením pre každé prirodzené~$n$.
}

{%%%%%   trojstretnutie, priklad 2
Všetky kongruencie a~zvyškové triedy v~riešení uvažujeme modulo~$m$.
Žiadanú kongruenciu $a_k^4-a_k-2\equiv 0$ získame ako dôsledok
jednoduchšej kongruencie $a_k\equiv {-1}$.

Postupnosť zvyškových tried čísel~$a_k$ má nasledujúcu vlastnosť:
zvyškové triedy ľubovoľných dvoch po sebe idúcich členov $a_k$,
$a_{k+1}$ jednoznačne určujú zvyškové triedy ako všetkých
nasledujúcich členov $a_i$ ($i>k+1$), tak všetkých predchádzajúcich členov~$a_i$
($i<k$). Odtiaľ zvyčajným postupom, založeným na tom, že
všetkých usporiadaných dvojíc zvyškových tried je~$m^2$, teda konečný
počet, dostávame, že postupnosť zvyškových tried čísel $a_i$ je
periodická, a~to hneď od svojho {\it prvého\/} člena. Existuje
teda číslo $p>0$ (závislé od daného~$m$) také, že
$a_i\equiv a_{i+p}$ pre každý index~$i$. Ak $m\ne1$ (pre $m=1$
je tvrdenie úlohy triviálne), zrejme $p>1$. Keďže $a_1\equiv
a_2\equiv1$, platí aj $a_{p+1}\equiv a_{p+2}\equiv1$, odkiaľ
$a_p\equiv0$ a~$a_{p-1}\equiv{-1}$, takže môžeme zobrať $k=p-1$ 
a~dôkaz je ukončený.
}

{%%%%%   trojstretnutie, priklad 3
\fontplace
\brpoint A; \tpoint B;
\tpoint C; \tpoint D\ ; \tpoint E;
\lBpoint F; \rBpoint k;
[1] \hfil\Obr

\fontplace
\tpoint A; \lbpoint\xy-.7,.2 B;
\lBpoint C; \tpoint D\ ; \tpoint E;
\lBpoint F; \rBpoint k;
[2] \hfil\Obr

Zrejme $DF$ je priemerom kružnice~$k$. Najskôr ukážeme, že za
daných podmienok nemôže bod~$C$ ležať v~polrovine $DFA$.

%% Je-li $DB$ tečna kružnice opsané \tr-u $AEB$, je $|\uh DEB|=|\uh
%% DBA|$ (rovnost obvodového a~úsekového úhlu tětivy $AB$
%% kružnice~$l$) a~$|\uh BAE|=180\st-|\uh DBE|=|\uh DBC|$
%% (z~rovnosti obvodového a~úsekového úhlu tětivy $BE$ kružnice~
%% $l$). Přitom pro vrcholy $B$, $C$ ležící na části $DA$ oblouku
%% $DAF$ (\obr) je zřejmě úhel $DEB$ ostrý, zatímco úhel $DBA$ je
%% tupý. Podobně pro vrcholy $B$, $C$ ležící na části $AF$ oblouku
%% $DAF$ (\obr) je zas úhel $BAE$ ostrý, zatímco $|\uh DBC|\ge90\st$.

Ak body $B$, $C$ ležia na časti~$DA$ oblúka $DAF$ (\obr),
sú zrejme uhly $DCB$ a~$DBA$ tupé, preto $|DC|<|DB|<|DA|<|DE|$,
takže rovnosť $|CD|^2=|AD|\cdot|ED|$ nemôže platiť.
Pre body $B$, $C$ na časti~$AF$ oblúka $DAF$ (\obr)
je uhol $BAE$ ostrý a~pre uhol $DBE$ platí $|\uhol DBE|=180\st-|\uhol
DBC|\le90\st$. Takže prípadný ďalší priesečník~$B'$ polpriamky~$DB$
s~kružnicou~$l$ nemôže ležať za bodom~$B$. Preto
$|DC|>|DB|\ge|DB'|$. Rovnosť $|CD|^2=|AD|\cdot|ED|$ teda nemôže
platiť, pretože $|AD|\cdot|ED|=|DB|\cdot|DB'|$ vyjadruje
mocnosť bodu~$D$ ku kružnici~$l$.

\twocpictures{}

Ak bod~$C$ neleží v~polrovine $DFA$, platí $|FC|=|DA|$
práve vtedy, keď $DAFC$ je pravouholník, \tj. práve vtedy, keď $CA$ je priemer
kružnice~$k$. To je ekvivalentné tomu, že uhol $CBA$ je pravý, 
a~to je ekvivalentné tomu, že trojuholník $AEB$ je pravouhlý s~pravým
uhlom pri vrchole~$B$, čiže stred kružnice opísanej trojuholníku $AEB$
je stredom úsečky~$AE$.
}

{%%%%%   trojstretnutie, priklad 4
Čísla $a=(k-1)k$, $b=(k+1)k$, $c=(k-1)(k+1)$, $d=k^2$
zrejme spĺňajú rovnosť $ab=cd$ a~nerovnosti $a<c<d<b$ pre každé
$k>1$. Nech teda $k$ je najmenšie prirodzené číslo, pre
ktoré platí $p<a$, čiže $p<(k-1)k$ (pri zadanom~$p$).
Ukážme, že pre také $k$ potom platí $b=(k+1)k\le
p+4+2\sqrt{4p+1}$, čo je zrejme číslo o~$\frac14$ menšie ako
horné ohraničenie intervalu zo zadania, takže tým bude riešenie úlohy úplné.


Podľa výberu čísla~$k$ platí $p\ge(k-2)(k-1)$. Riešením tejto
kvadratickej nerovnice dostaneme odhad
$$
k\le\frac32+\sqrt{p+\frac14},
$$
z~ktorého už vyplýva
$$\align
b&=(k+1)k\le\biggl(\frac52+\sqrt{p+\frac14}\biggr)\cdot
\biggl(\frac32+\sqrt{p+\frac14}\biggr)=\\
&=\frac{15}{4}+4\sqrt{p+\frac14}+\biggl(p+\frac14\biggr)=
p+4+2\sqrt{4p+1}.
\endalign
$$
}

{%%%%%   trojstretnutie, priklad 5
Z~možnosti rozdelenia na disjunktné trojice vyplýva $3\mid n$.
V~každej trojici $\{a,b,a+b\}$ je súčet $2(a+b)$, teda párne číslo,
preto musí byť párny aj súčet všetkých čísel od $1$ do $n$, súčin
$n(n+1)$ musí teda byť deliteľný štyrmi. Celkom máme, že
číslo~$n$ musí byť tvaru $12k$ alebo $12k+3$, čomu z~daných čísel
vyhovujú iba $n=3\,900$ a~$n=3\,903$.

V~ďalšom odstavci popíšeme konštrukciu,
ako z~vyhovujúceho rozkladu pre dané $n=k$
vytvoriť vyhovujúce rozklady pre $n=4k$ a~$n=4k+3$. To nám
zaručí, že rozklady pre $n=3\,900$ aj $n=3\,903$ existujú,
a~to vďaka zostupnej postupnosti
$$
3\,900\to975\to243\to60\to15\to3
$$
(namiesto $3\,900$ možno začať aj číslom $3\,903$) a~vďaka triviálnemu
rozkladu pre $n=3$ (z~ktorého postupne zostrojíme rozklady pre
$n=15$, $n=60$ atď. až pre $n=3\,900$ resp. $n=3\,903$).

Z~vyhovujúceho rozkladu množiny $\{1,2,\dots,k\}$ najskôr
vyrobíme vyhovujúci rozklad množiny prvých $k$ párnych
čísel $\{2,4,\dots,2k\}$ (tak, že všetky čísla vo všetkých
trojiciach pôvodného rozkladu vynásobíme dvoma). V~prípade $n=4k$
potom zvyšné čísla
$$
\{1,3,5,\dots,2k-1,2k+1,2k+2,\dots,4k-1,4k\}
$$
rozdelíme na $k$~trojíc $\{2j-1,3k-j+1,3k+j\}$, pričom
$j=1,2,\dots,k$. Vidno ich v~stĺpcoch tabuľky
$$
\pmatrix
1   &3   &5   &\dots&2k-3&2k-1\\
3k  &3k-1&3k-2&\dots&2k+2&2k+1\\
3k+1&3k+2&3k+3&\dots&4k-1&4k
\endpmatrix.
$$
V~prípade $n=4k+3$ zvyšné čísla
$$
\{1,3,5,\dots,2k-1,2k+1,2k+2,\dots,4k+2,4k+3\}
$$
rozdelíme na $k+1$ trojíc $\{2j-1,3k+3-j,3k+j+2\}$, pričom
$j=1,2,\dots,k+1$, tvorených stĺpcami tabuľky
$$
\pmatrix
1   &3   &5   &\dots&2k-1&2k+1\\
3k+2&3k+1&3k  &\dots&2k+3&2k+2\\
3k+3&3k+4&3k+5&\dots&4k+2&4k+3
\endpmatrix.
$$

Tým je dôkaz toho, že čísla $n=3\,900$ a~$n=3\,903$ vyhovujú,
ukončený.
}

{%%%%%   trojstretnutie, priklad 6
\fontplace
\tpoint A; \tpoint B; \bpoint C; \bpoint\ D;
\rpoint P;
[3] \hfil\Obr

\fontplace
\rpoint A; \lpoint B; \lBpoint C; \rpoint D;
\tpoint\xy0,-.5 P; \tpoint\xy-.5,-.5 Q;
[4] \hfil\Obr

Ak $P$ je spoločný bod spomenutých kružníc, vyplýva z~vety 
o~obvodových a~úsekových uhloch, že je zároveň aj bodom dotyku
práve vtedy, keď (\obr)
$$
|\uhol ADP|+|\uhol BCP|=|\uhol APB|.
\tag1
$$

\inspicture{}

Uvažujme teraz kružnice opísané trojuholníkom
$ABP$ a~$CDP$ a~predpokladajme, že sa pretínajú ešte
v~ďalšom bode~$Q$ ($Q\ne P$).

Keďže bod~$A$ leží zvonka kružnice opísanej trojuholníku $BCP$, platí
$|\uhol BCP|+|\uhol BAP|<180^{\circ}$. Preto aj bod~$C$
leží zvonka kružnice opísanej trojuholníku $ABP$. Analogicky leží aj bod~$D$ zvonka tejto
kružnice. Odtiaľ vyplýva, že body $P$ a~$Q$ ležia na rovnakom oblúku~$CD$ kružnice opísanej trojuholníku $CDP$.

Analogicky body $P$ a~$Q$ leží na rovnakom oblúku~$AB$ kružnice opísanej trojuholníku
$ABP$. Bod~$Q$ teda leží buď vnútri uhla $BPC$, alebo vnútri uhla
$APD$. Bez ujmy na všeobecnosti predpokladajme, že bod~$Q$ leží
vnútri uhla $BPC$ (\obr). V~tom prípade podľa predpokladu úlohy
platí
$$
|\uhol AQD|=|\uhol PQA|+|\uhol PQD|
=|\uhol PBA|+|\uhol PCD|\le 90^{\circ}.
\tag2
$$
\inspicture{}

V~tetivových štvoruholníkoch $APQB$ a~$DPQC$ sú podľa
predpokladu úlohy uhly pri vrcholoch $A$ a~$D$ ostré, takže
príslušné protiľahlé uhly pri vrchole~$Q$ sú tupé. Odtiaľ vyplýva,
že bod~$Q$ leží nielen vnútri uhla $BPC$, ale dokonca vnútri trojuholníka
$BPC$, čiže aj vnútri štvoruholníka $ABCD$.

Z~vlastností uhlov oboch spomenutých tetivových štvoruholníkov teraz
vyplýva
$$
|\uhol BQC|=|\uhol PAB|+|\uhol PDC|,
$$
takže podľa predpokladu úlohy
$$
|\uhol BQC|\le 90^{\circ}.
\tag3
$$

Keďže navyše $|\uhol PCQ|=|\uhol PDQ|$, dostávame podľa~\thetag1
$$
\align
|\uhol ADQ|+|\uhol BCQ|
    &=|\uhol ADP|+|\uhol PDQ|+|\uhol BCP|-|\uhol PCQ|=\\
    &=|\uhol ADP|+|\uhol BCP|=|\uhol APB|.
\endalign
$$
A~keďže aj $|\uhol APB|=|\uhol AQB|$, vychádza
$$
|\uhol ADQ|+|\uhol BCQ|=|\uhol AQB|.
$$
To však znamená, ako už vieme z~úvodnej úvahy, že kružnice opísané trojuholníkom
$BCQ$ a~$DAQ$ sa dotýkajú v~bode~$Q$, čo odporuje nášmu
počiatočnému predpokladu, že $Q\ne P$. Nezostáva teda iná možnosť
ako tá, že obe kružnice opísané trojuholníkom $ABP$ a~$CDP$ majú
spoločný jediný bod~$P$, pre ktorý podľa nerovností \thetag2 a~\thetag3
navyše platí, že uhly $APD$ a~$BPC$ nie sú tupé.

Uvažujme teraz polkruhy zostrojené nad stranami $BC$ a~$DA$
"dovnútra" štvoruholníka $ABCD$. Keďže uhly $APD$ a~$BPC$ nie sú
tupé, leží každý z~oboch polkruhov celý vnútri zodpovedajúceho
kruhu prislúchajúceho kružnici opísanej trojuholníku $BQC$, resp. $AQD$. A~keďže sa obe
kružnice zvonka dotýkajú, majú aj oba polkruhy zostrojené nad
stranami $BC$ a~$DA$ najviac jeden spoločný bod. Ak označíme $M$ a~$N$ stredy strán $BC$ a~$DA$, vyplýva
z~toho nerovnosť $|MN|\ge \frac12(|BC|+|DA|)$.

Na druhej strane, zrejme platí $\overrightarrow{MN}=\frac12(\overrightarrow{BA}+\overrightarrow{CD})$,
takže $|MN|\le \frac12(|AB|+|CD|)$. Odtiaľ vychádza dokazovaná
nerovnosť $|AB|+|CD|\ge |BC|+|DA|$.
}

{%%%%%   IMO, priklad 1
(a) Každé z~čísel $d_1,d_2,\dots,d_n$, a teda aj najväčšie z~nich $d$, je definované ako rozdiel niektorých dvoch členov postupnosti $a_1,a_2,\dots,a_n$. Existujú teda indexy $p$, $q$ také, že $d=a_p-a_q$, pričom navyše $p\le q$. (Tieto indexy nemusia byť určené jednoznačne.)
\insp{mmo.1}%

Nech $x_1\le x_2\le\cdots\le x_n$ sú ľubovoľné reálne čísla. Na dôkaz časti~(a) stačí pracovať s~hodnotami $x_p$, $x_q$ (\obr).
Máme totiž $x_q\ge x_p$, a~teda
$$
  (a_p-x_p)+(x_q-a_q) = (a_p-a_q)+(x_q-x_p) \ge a_p-a_q=d.
$$
Preto platí aspoň jedna z~nerovností $a_p-x_p\ge\frac d2$, $x_q-a_q\ge\frac d2$, odkiaľ dostávame
$$
\max\{|x_i-a_i| : 1\le i\le n\} \ge \max\{|x_p-a_p|,|x_q-a_q|\} \ge \max\{a_p-x_p,x_q-a_q\} \ge \frac d2.
$$

\smallskip
(b) Položme
$$
x_1=a_1-\frac d2 \qquad\text{a}\qquad x_k=\max\left\{x_{k-1},a_k-\frac d2\right\}\quad\text{pre $2\le k\le n$.}
$$
Zrejme platí $x_1\le x_2\le\cdots\le x_n$. Ukážeme, že pre takto zvolené hodnoty~$x_i$ nastane v~\thetag{$*$} vždy rovnosť. Stačí ukázať, že pre každé $i=1,2,\dots,n$ platí $|x_i-a_i|\le \frac d2$ (rovnosť potom platí vďaka výsledku z~časti~(a), alebo aj bez toho, keďže $|x_1-a_1|=\frac d2$).

Priamo z~definície hodnoty~$x_i$ máme $x_i-a_i\ge\m\frac d2$. Ešte dokážeme, že $x_i-a_i\le\frac d2$. Nech $j\le i$ je najmenší index, pre ktorý $x_i=x_j$. Ak $j=1$, tak $x_j=a_j-\frac d2$; ak $j\ge2$, tak $x_{j-1}<x_j$, čiže tiež $x_j=a_j-\frac d2$. Máme teda 
$$
x_i=x_j=a_j-\frac d2.
$$
Z~definície hodnoty~$d$ samozrejme vyplýva $a_j-a_i\le d$. Spolu dostávame
$$
x_i-a_i=a_j-\frac d2-a_i\le d-\frac d2=\frac d2.
$$
Ukázali sme, že pre pre každé $i=1,2,\dots,n$ platí $\m\frac d2\le x_i-a_i\le\frac d2$, teda naozaj $|x_i-a_i|\le \frac d2$.
}

{%%%%%   IMO, priklad 2
Označme postupne $S$, $M$, $N$ stredy úsečiek $CA$, $CF$, $CG$. Zrejme tieto tri body ležia na jednej priamke, ktorá je obrazom priamky~$l$ v~rovnoľahlosti so stredom~$C$ a~koeficientom~$\frac12$. Označme ju~$p$. Úsečky $EM$, $EN$ sú výškami rovnoramenných trojuholníkov $EFC$, $ECG$, sú teda kolmé postupne na priamky $CD$, $BC$. Takže priamka~$p$ je Simsonovou priamkou\footnote{Veta o~Simsonovej priamke hovorí, že päty kolmíc spustených z~ľubovoľného bodu~$Q$ opísanej kružnice daného trojuholníka $XYZ$ na strany tohto trojuholníka ležia na jednej priamke; uvedená priamka sa nazýva Simsonova priamka pre bod~$Q$ a~trojuholník $XYZ$.} pre bod~$E$ a~trojuholník $BCD$.
\insp{mmo.7}%

Uhlopriečky v~rovnobežníku sa rozpoľujú, preto bod~$S$ je stredom úsečky~$BD$, a~keďže leží na Simsonovej priamke~$p$, musí byť zároveň pätou kolmice spustenej z~bodu~$E$ na stranu~$BD$ (\obr). Bod~$E$ je teda nutne stredom oblúka~$BD$ kružnice opísanej tetivovému štvoruholníku $BCED$ a~platí $|ED|=|EB|$.

Z~obvodových uhlov nad tetivou~$ED$ máme $|\uhol DBE|=|\uhol DCE|$. Takže rovnoramenné trojuholníky $DBE$, $FCE$ sú podobné (ich ramená $EB$, $EC$ zvierajú so základňami $DB$, $FC$ rovnaké uhly, \obr) a~môžeme označiť $|\uhol DEB|=|\uhol FEC|=\varphi$. 
\insp{mmo.8}%
V~otočení okolo bodu~$E$ o~uhol~$\varphi$ sa $D$ sa zobrazí na $B$ a~$F$ na $C$, preto $|DF|=|BC|$. Zároveň však $|BC|=|AD|$, odkiaľ vyplýva, že trojuholník $AFD$ je rovnoramenný. Z~toho s~využitím zhodnosti striedavých uhlov dostávame
$$
|\uhol DAF|=|\uhol DFA|=|\uhol FAB|,
$$
teda priamka~$l$ je naozaj osou uhla $DAB$.
}

{%%%%%   IMO, priklad 3
Uvedieme algoritmus rozdelenia účastníkov do miestností. Dve miestnosti, do ktorých budeme účastníkov rozdeľovať, označme $A$ a~$B$. Začneme s~určitým rozdelením a~budeme ho postupne upravovať posielaním účastníkov z~jednej miestnosti do druhej. Počas algoritmu budeme označovať $A$, $B$ množiny účastníkov, ktorí sú práve v~daných miestnostiach a~$c(A)$, $c(B)$ veľkosti najväčších klík v~príslušných miestnostiach.

\krok1
Nech $2m$ je rozmer najväčšej kliky a~$M$ je jedna z~klík s~týmto rozmerom, \tj. $|M|=2m$. Všetkých členov z~$M$ dajme do miestnosti~$A$ a~všetkých ostatných do miestnosti~$B$. Zrejme platí $c(A)=|M|\ge c(B)$.

\krok2
Kým platí $c(A)>c(B)$, posielame účastníkov po jednom z~$A$ do $B$ (\obr). (Ak $c(A)>c(B)$, miestnosť~$A$ určite nie je prázdna.)
\insp{mmo.2}%
 
Po každej zmene sa takto $c(A)$ zmenší o~$1$ a~$c(B)$ zväčší nanajvýš o~$1$. Takže na konci budeme mať $c(A)\le c(B)\le c(A)+1$. Zrejme bude platiť aj $c(A)=|A|\ge m$, inak by totiž bolo v~$B$ aspoň $m+1$ členov z~$M$ a~v~$A$ nanajvýš $m-1$ členov z~$M$, teda by bolo $c(B)-c(A)\ge(m+1)-(m-1)=2$.

\krok3
Nech $k=c(A)$. Ak $c(B)=k$, ukončíme rozdeľovanie.

Ak sme dosiahli $c(A)=c(B)=k$, našli sme požadované rozdelenie. Vo všetkých ostatných prípadoch máme $c(B)=k+1$. Vieme tiež, že $k=|A|=|A\cap M|\ge m$ a~$|B\cap M|\le m$. 

\krok4
Ak existuje účastník $x\in B\cap M$ a~klika $C\subset B$ taká, že $|C|=k+1$ a~$x\notin C$, tak pošleme $x$ do miestnosti~$A$ a~ukončíme rozdeľovanie (\obr).
\insp{mmo.3}%

Po poslaní $x$ späť do $A$ budeme v~$A$ mať $k+1$ členov z~$M$, teda $c(A)=k+1$. Keďže $x\notin C$, $c(B)=|C|$ sa nezmenší, \tj. budeme mať $c(A)=c(B)=k+1$.

Ak účastník~$x$ spĺňajúci uvedené podmienky neexistuje, tak v~miestnosti~$B$ každá klika s~rozmerom $k+1$ obsahuje ako podmnožinu celý prienik $B\cap M$.

\krok5
Kým platí $c(B)=k+1$, zvolíme niektorú kliku $C\subset B$ s~rozmerom $k+1$ a~pošleme jedného člena z~$C\setminus M$ do miestnosti~$A$ (\obr). (Keďže $|C|=k+1>m\ge|B\cap M|$, množina $C\setminus M$ nemôže byť prázdna.) 
\insp{mmo.4}%

Zakaždým, keď pošleme jedného účastníka z~$B$ do $A$, zmenší sa $c(B)$ nanajvýš o~$1$. Na konci tohto kroku teda budeme mať $c(B)=k$. V~miestnosti~$A$ máme kliku $A\cap M$ veľkosti $|A\cap M|=k$, čiže $c(A)\ge k$. Dokážeme, že v~$A$ nie je klika s~väčším rozmerom. Nech $Q\subset A$ je ľubovoľná klika. Ukážeme, že $|Q|\le k$.
\insp{mmo.5}%
V~miestnosti $A$, a~špeciálne aj v~množine $Q$, môžu byť dva typy účastníkov:
\item{$\bullet$} Členovia $M$; keďže $M$ je klika, sú to priatelia so všetkými členmi z~$B\cap M$.
\item{$\bullet$} Účastníci, ktorých sme do $A$ poslali v 5.\,kroku; každý z~nich bol v~klike, ktorá obsahovala $B\cap M$, teda aj títo sú priatelia so všetkými členmi z~$B\cap M$.

\noindent
Takže všetci členovia $Q$ sú priateľmi so všetkými členmi z~$B\cap M$ (\obr). Množiny $Q$ a~$B\cap M$ sú kliky, takže aj $Q\cup(B\cap M)$ je klika. Keďže $M$ je klika s~najväčším rozmerom, máme
$$
  |M| \ge |Q\cup(B\cap M)| = |Q|+|B\cap M| = |Q|+|M|-|A\cap M|,
$$ 
odkiaľ $|Q|\le |A\cap M|=k$.

Po 5.\,kroku teda dostaneme $c(A)=c(B)=k$.
}

{%%%%%   IMO, priklad 4
Ak $|AC|=|BC|$, trojuholník $ABC$ je rovnoramenný, trojuholníky $RPK$, $RQL$ sú súmerne združené podľa osi~$CR$ a~zadané tvrdenie je triviálne. Ďalej bez ujmy na všeobecnosti predpokladajme, že $|AC|<|BC|$. Označme $O$ stred kružnice opísanej trojuholníku $ABC$ a~$\gamma$ veľkosť uhla $ACB$. Z~pravouhlých trojuholníkov $CLQ$ a~$CKP$ máme
$$
|\uhol OQP|=|\uhol LQC|=90^{\circ}-\frac{\gamma}2,\qquad|\uhol OPQ|=|\uhol KPC|=90^{\circ}-\frac{\gamma}2,
$$
teda uhly $OQP$, $OPQ$ majú rovnakú veľkosť. Takže trojuholník $OQP$ je rovnoramenný a~ľahko dopočítame, že veľkosť jeho tretieho uhla, ktorý zvierajú zhodné ramená $OQ$, $OP$, je $\gamma$ (\obr).
\insp{mmo.6}%

Z~vlastnosti stredového a~obvodového uhla dostávame
$$
|\uhol AOR|=2|\uhol ACR|=\gamma,\qquad|\uhol ROB|=2|\uhol RCB|=\gamma.
$$
Navyše samozrejme $|AO|=|RO|=|BO|$. Uvažujme otočenie okolo bodu~$O$ o~uhol~$\gamma$. Z~uvedeného vyplýva, že v~tomto otočení sa $Q$ zobrazí na $P$, $A$ na $R$ a~$R$ na $B$. Takže trojuholníky $QAR$, $PRB$ sú zhodné a~majú aj rovnaké obsahy.

Trojuholníky $RQL$, $RQA$ majú spoločnú stranu~$RQ$, pomer ich obsahov je teda rovný pomeru dĺžok ich výšok na stranu~$RQ$. Keďže $L$ je stred strany~$CA$, tento pomer je zrejme $1:2$. Podobný vzťah dostaneme pre trojuholníky $RPK$, $RPB$. Spolu dostávame
$$
S_{RQL}=\tfrac12S_{RQA}=\tfrac12S_{RPB}=S_{RPK}.
$$
}

{%%%%%   IMO, priklad 5
Dvojicu $(a,b)$ prirodzených čísel nazveme {\it zlá}, ak $4ab-1\mid(4a^2-1)^2$. Inak povedané, ak $(a,b)$ je zlá dvojica, tak existuje také prirodzené číslo~$k$, že $k(4ab-1)=(4a^2-1)^2$. Po jednoduchej úprave dostaneme
$$
4a(bk-4a^3+2a)-1=k.
$$
Označme $c=bk-4a^3+2a$. Zrejme $c$ je celé, a~vzhľadom na rovnosť $k+1=4ac$ musí byť aj kladné. A~keďže $k=4ac-1$ je deliteľom čísla $(4a^2-1)^2$, dvojica $(a,c)$ je zlá. Navyše ak $a<b$, tak $4a^2-1<4ab-1$, čiže aj $(4a^2-1)^2<(4ab-1)(4a^2-1)$, a~preto
$$
4ac-1=k=\frac{(4a^2-1)^2}{4ab-1}<4a^2-1.
$$
Odtiaľ máme $c<a$. Ku každej zlej dvojici $(a,b)$ s~vlastnosťou $a<b$ teda existuje zlá dvojica $(a,c)$ s~vlastnosťou $c<a$.

\smallskip
Ak chápeme výraz $4ab-1$ ako lineárny dvojčlen v~premennej~$a$, postupným vydelením mnohočlena $(4a^2-1)^2=16a^4-8a^2+1$ uvedeným dvojčlenom (a~prenásobením číslom $16b^4$, aby sme sa vyhli zlomkom) dostaneme
$$
16b^4(4a^2-1)^2=(4ab-1)(64a^3b^3+16a^2b^2-32ab^3+4ab-8b^2+1)+(4b^2-1)^2.
$$
Z~tejto identity priamo vyplýva, že ak $4ab-1\mid(4a^2-1)^2$, tak aj $4ab-1\mid(4b^2-1)^2$. Teda ak $(a,b)$ je zlá dvojica, tak aj $(b,a)$ je zlá dvojica.

\smallskip
Predpokladajme že existuje nejaká zlá dvojica rôznych čísel. Potom existuje aj dvojica $(a,b)$ s~vlastnosťou $a\ne b$, v~ktorej $a$ je minimálne možné.

Ak $a<b$, podľa prvého odstavca existuje $c<a$ také, že dvojica $(a,c)$ je zlá, a~podľa druhého odstavca je potom aj dvojica $(c,a)$ zlá, čo je v~spore s~minimálnosťou~$a$.

Ak $a>b$, tak podľa druhého odstavca je aj dvojica $(b,a)$ zlá a~podľa prvého odstavca existuje $c<b<a$ také, že dvojica $(b,c)$ je zlá. Potom podľa druhého odstavca je aj dvojica $(c,b)$ zlá, čo je opäť v~spore s~minimálnosťou~$a$.

\zaver
Pre každú zlú dvojicu $(a,b)$ platí $a=b$.
}

{%%%%%   IMO, priklad 6
Najmenší možný počet rovín je $3n$. Ľahko nájdeme $3n$ rovín, ktoré spĺňajú zadané podmienky. Môžeme napríklad zobrať roviny s~rovnicami $x=i$, $y=i$, $z=i$ pre $i=1,2,\dots,n$. Iným vyhovujúcim príkladom sú roviny s~rovnicami $x+y+z=k$ pre $k=1,2,\dots 3n$. Ukážeme, že menej ako $3n$ rovín nestačí. Dokážeme najskôr pomocné tvrdenie.

\Lema
Nech $P(x_1,\dots,x_k)$ je nenulový polynóm $k$~premenných. Ak $P(x_1,\dots,x_k)=0$ pre ľubovoľné $x_1,\dots,x_k\in\{0,1,\dots,n\}$ také, že $x_1+\cdots+x_k>0$ a~$P(0,\dots,0)\ne0$, tak $\deg P\ge kn$. (Pod $\deg P$ rozumieme {\it stupeň\/} polynómu~$P$, \tj. exponent najvyššej mocniny~$x$ vo výraze $P(x,\dots,x)$.)

Lemu dokážeme matematickou indukciou vzhľadom na~$k$. Pre $k=1$ jej platnosť zabezpečí známe tvrdenie. Podľa neho, ak má nenulový polynóm jednej premennej $n$~koreňov (v~našom prípade by koreňmi boli čísla $1,2,\dots,n$), tak má stupeň aspoň~$n$.

Predpokladajme, že lema platí pre $k=m$. Dokážeme, že potom platí aj pre $k=m+1$. Kvôli prehľadnosti označme $y=x_{m+1}$. Keď chápeme $P$ ako polynóm v~premennej~$y$, môžeme ho vydeliť polynómom $Q(y)=y(y-1)\cdots(y-n)$. Pri delení dostaneme ako zvyšok polynóm $R$. Presnejšie, 
$$
P(x_1,\dots,x_m,y)=Q(y)\cdot S(x_1,\dots,x_m,y)+R(x_1,\dots,x_m,y),
$$
kde $\deg_y R\le n$ (\tj. stupeň zvyšku~$R$, keď ho chápeme ako polynóm v~jednej premennej~$y$, je menší ako stupeň polynómu~$Q$). Keďže $Q(y)=0$ pre $y=0,1,\dots,n$, máme $R(x_1,\dots,x_m,y)=P(x_1,\dots,x_m,y)$ pre ľubovoľné $x_1,\dots,x_m,y\in\{0,1,\dots,n\}$, čiže $R$ spĺňa podmienky lemy. Zrejme $\deg R\le\deg P$, stačí teda dokázať, že $\deg R\ge (m+1)n$.

Rozpíšme $R$ podľa mocnín $y$:
$$
R(x_1,\dots,x_m,y)=R_n(x_1,\dots,x_m)y^n + R_{n-1}(x_1,\dots,x_m)y^{n-1}+\cdots+R_0(x_1,\dots,x_m).
$$
Ukážeme, že na polynóm $R_n(x_1,\dots,x_m)$ môžeme použiť indukčný predpoklad.

Uvažujme polynóm $T(y)=R(0,\dots,0,y)$ stupňa nanajvýš~$n$. Tento polynóm má $n$ koreňov $y=1,2,\dots,n$. Na druhej strane, $T(y)$ nie je konštantný nulový polynóm, lebo $T(0)\ne0$. Teda $\deg T=n$ a~jeho vedúci koeficient $R_n(0,\dots,0)$ je nenulový.

Ak zoberieme ľubovoľné čísla $a_1,\dots,a_m\in\{0,1,\dots,n\}$, pričom $a_1+\cdots+a_m>0$, a~dosadíme $x_i=a_i$ do $R(x_1,\dots,x_m,y)$, dostaneme polynóm v~premennej~$y$. Tento polynóm je nulový vo všetkých bodoch $y=0,1,\dots,n$ (\tj. má aspoň $n+1$ koreňov) a~má stupeň nanajvýš $n$. Preto musí byť nulový, čiže $R_i(a_1,\dots,a_m)=0$ pre všetky $i=0,1,\dots,n$. Špeciálne máme $R_n(a_1,\dots,a_m)=0$.

Polynóm $R_n(x_1,\dots,x_m)$ teda spĺňa predpoklady lemy a~podľa indukčného predpokladu dostávame $\deg R_n\ge mn$, odkiaľ
$$
\deg P\ge\deg R\ge\deg R_n+n\ge(m+1)n.
$$

\smallskip
Teraz už ľahko dokončíme riešenie. Predpokladajme, že máme $N$~rovín pokrývajúcich všetky body z~$S$, ale neobsahujúcich bod $(0,0,0)$. Nech všeobecné rovnice týchto rovín sú $a_ix+b_iy+c_iz+d_i=0$ (pre $i=1,2,\dots,N$). Uvažujme polynóm
$$
P(x,y,z)=(a_1x+b_1y+c_1z+d_1)\cdot(a_2x+b_2y+c_2z+d_2)\cdot\dots\cdot(a_Nx+b_Ny+c_Nz+d_N),
$$
ktorého stupeň je zrejme~$N$. Tento polynóm spĺňa $P(x_0,y_0,z_0)=0$ pre ľubovoľné $(x_0,y_0,z_0)\in S$ a~$P(0,0,0)\ne0$. Podľa dokázanej lemy teda $N=\deg P\ge 3n$.
}

{%%%%%   MEMO, priklad 1
Nech $p$, $q$, $r$, $s$ sú (kladné) čísla $a$, $b$, $c$, $d$ v~takom poradí, že $p\ge q\ge r\ge s$. Potom zrejme platí 
$$
pqr\ge pqs\ge prs\ge qrs.
$$
Podľa nerovnosti usporiadania\footnote{Ak sú $a_1\le a_2\le\dots\le a_n$ a~$b_1\le b_2\le\dots\le b_n$ dve $n$-tice reálnych čísel
a~$(x_1,x_2,\dots,x_n)$, resp. $(y_1,y_2,\dots,y_n)$ ich ľubovoľné permutácie, tak pre súčet $S=x_1y_1+x_2y_2+\cdots+x_ny_n$ platí $S_{\min}\le S\le S_{\max}$, kde $S_{\min}=a_1b_n+a_2b_{n-1}+\cdots+a_nb_n$
a~$S_{\max}=a_1b_1+a_2b_2+\cdots+a_nb_n$. Táto známa nerovnosť (dokázať sa dá ľahko matematickou indukciou) sa nazýva {\it nerovnosť usporiadania}, často sa pre ňu používa anglický názov {\it rearrangement inequality}.} teda máme
$$
a\cdot abc+b\cdot bcd+c\cdot cda+d\cdot dab\le p\cdot pqr+q\cdot pqs+r\cdot prs+s\cdot qrs=(pq+rs)(pr+qs).
$$
Pre ľubovoľné reálne čísla $A$, $B$ platí $AB\le\frac14(A+B)^2$ (to dostaneme okamžite úpravou zrejmej nerovnosti $0\le\frac14(A-B)^2$). Použitím tejto nerovnosti najprv pre $A=pq+rs$, $B=pr+qs$ a~neskôr pre $A=p+s$, $B=q+r$ postupne dostávame
$$
\aligned
(pq+rs)(pr+qs)&\le\frac14(pq+rs+pr+qs)^2=\frac14((p+s)(q+r))^2\le\\
&\le\frac14\left(\frac14(p+q+r+s)^2\right)^2=\frac1{64}(a+b+c+d)^4=4.
\endaligned
$$
Tým je zadaná nerovnosť dokázaná.
}

{%%%%%   MEMO, priklad 2
Uvažujme pevné~$k$ a~predpokladajme, že máme vyhovujúce zafarbenie pre maximálne možné~$n$. Bez ujmy na všeobecnosti môžeme predpokladať, že loptičky s~číslom~$1$ sú biele. Rozoberme dva možné prípady, aké môžu byť loptičky s~číslom~$2$.

\pripad1
Nech loptičky s~číslom~$2$ sú čierne. Podľa podmienky ($ii$) musia byť loptičky s~číslom $1+1+\cdots+1=k$ čierne a~loptičky s~číslom $2+2+\cdots+2=2k$ biele. Keďže
$$
(k+1)+\underbrace{1+\cdots+1}_{\text{$(k-1)$-krát}}=2k
$$
a~loptičky s~číslami $1$ a~$2k$ sú biele, loptičky s~číslom $k+1$ musia byť čierne. Z~rovností
$$
\underbrace{1+\cdots+1}_{\text{$(k-1)$-krát}}+2k=\underbrace{2+\cdots+2}_{\text{$(k-1)$-krát}}+(k+1)=3k-1
$$
vyplýva, že loptičky s~číslom $3k-1$ nemôžu byť ani biele (medzi loptičkami s~číslami $1$, $2k$ a~$3k-1$ by nebola zastúpená čierna farba), ani čierne (medzi loptičkami s~číslami $2$, $k+1$ a~$3k-1$ by nebola zastúpená biela farba). V~tomto prípade teda nutne $n\le 3k-2$.

\pripad2
Nech loptičky s~číslom~$2$ sú biele. Z~rovností
$$
\aligned
1+1+\cdots+1+1+1&=k,\\
1+1+\cdots+1+1+2&=k+1,\\
1+1+\cdots+1+2+2&=k+2,\\
&\vdots\\
1+2+\cdots+2+2+2&=2k-1,\\
2+2+\cdots+2+2+2&=2k
\endaligned
$$
vyplýva, že loptičky s~číslami $k,k+1,\dots,2k$ sú čierne. Potom loptičky s~číslom $k+k+\cdots+k=k^2$ musia byť biele. Z~rovností
$$
\underbrace{1+\cdots+1}_{\text{$(k-1)$-krát}}+k^2=k+\underbrace{(k-1)+\cdots+(k-1)}_{\text{$(k-1)$-krát}}+(k+1)=k^2+k-1
$$
vyplýva, že loptičky s~číslom $k^2+k-1$ nemôžu byť ani biele (medzi loptičkami s~číslami $1$, $k^2$ a~$k^2+k-1$ by nebola zastúpená čierna farba), ani čierne (medzi loptičkami s~číslami $k-1$, $k+1$ a~$k^2+k-1$ by nebola zastúpená biela farba). V~tomto prípade teda nutne $n\le k^2+k-2$.

\smallskip
Pre každé $k\ge2$ platí $k^2+k-2\ge 3k-2$ (túto nerovnosť možno upraviť na tvar $k(k-2)\ge0$), takže v~oboch prípadoch $n\le k^2+k-2$. Stačí už len ukázať príklad vyhovujúceho zafarbenia pre $n=k^2+k-2$.

Budeme postupovať nasledovne. Loptičky s~číslami $1,2,\dots,k-1$ zafarbíme bielou, loptičky s~číslami $k,k+1,\dots,k^2-1$ čiernou a~loptičky s~číslami $k^2,k^2+1,\dots,k^2+k-2$ opäť bielou farbou. Súčet čísel na $k$~čiernych loptičkách je aspoň $k+\cdots+k=k^2>k^2-1$, takže čierne loptičky pravidlo ($ii$) neporušia. Ak zoberieme $k$~bielych loptičiek, ktorých čísla sú nanajvýš $k-1$, tak súčet čísel na týchto loptičkách bude najviac
$$
\underbrace{(k-1)+\cdots+(k-1)}_{\text{$k$-krát}}=k^2-k<k^2
$$
a~najmenej $1+\cdots+1=k$, teda súčet bude číslo nachádzajúce sa len na čiernych loptičkách. A~ak aspoň jedna z~$k$~bielych loptičiek bude mať číslo aspoň $k^2$, tak súčet čísel na loptičkách bude aspoň
$$
k^2+\underbrace{1+\cdots+1}_{\text{$(k-1)$-krát}}=k^2+k-1>n.
$$
Ani biele loptičky preto pravidlo ($ii$) neporušia a~uvedené zafarbenie vyhovuje.
}

{%%%%%   MEMO, priklad 3
Nech $O$ je stred kružnice~$k$ a~$P$ je priesečník tetív $O_1O_3$ a~$O_2O_4$. Pre $i=1,2,3,4$ a~$A_4=A_0$ označme $\alpha_i=|\uhol O_iOA_i|=|\uhol O_iOA_{i-1}|$. Zrejme $\alpha_1+\alpha_2+\alpha_3+\alpha_4=180^\circ$ (\obr). Z~vlastností obvodových a~stredových uhlov máme
$$
|\uhol PO_3O_2|=\frac12|\uhol O_1OO_2|=\frac{\alpha_1+\alpha_2}2,\quad
|\uhol PO_2O_3|=\frac12|\uhol O_4OO_3|=\frac{\alpha_3+\alpha_4}2.
$$
Z~trojuholníka $PO_2O_3$ potom ľahko dopočítame $|\uhol O_2PO_3|=180^\circ-\frac12(\alpha_1+\alpha_2+\alpha_3+\alpha_4)=90^\circ$. Teda tetivy $O_1O_3$ a~$O_2O_4$ sú navzájom kolmé.
\insp{memo.1}%

Dokážeme, že $B_1B_2\parallel O_1O_3$. Keďže $|O_2B_1|=|O_2B_2|$, stačí ukázať, že uhly $O_4O_2B_1$, $O_4O_2B_2$ majú rovnakú veľkosť (potom sú $B_1$, $B_2$ súmerne združené podľa priamky~$O_2O_4$, teda $B_1B_2\perp O_2O_4$ a~$B_1B_2\parallel O_1O_3$). Keďže $B_2$ je obrazom bodu~$A_2$ v~osovej súmernosti podľa priamky~$O_2O_3$ a~uhol $O_3O_2A_2$ je obvodovým uhlom k~stredovému uhlu $O_3OA_2$, máme (\obr)
$$
|\uhol O_3O_2B_2|=|\uhol O_3O_2A_2|=\frac12|\uhol O_3OA_2|=\frac{\alpha_3}2.
$$
Preto
$$
|\uhol O_4O_2B_2|=|\uhol PO_2O_3|-|\uhol O_3O_2B_2|=\frac{\alpha_3+\alpha_4}2-\frac{\alpha_3}2=\frac{\alpha_4}2.
$$
Vzhľadom na symetrickosť zadania možno rovnakým spôsobom ukázať, že $|\uhol O_4O_2B_1|=\frac{\alpha_4}2$. Teda naozaj $|\uhol O_4O_2B_1|=|\uhol O_4O_2B_2|$ a~$B_1B_2\parallel O_1O_3$. Opäť vzhľadom na symetrickosť vieme rovnako ukázať, že $B_2B_3\parallel O_2O_4$, $B_3B_4\parallel O_1O_3$ a~$B_4B_1\parallel O_2O_4$. Z~toho už priamo vyplýva, že $B_1B_2B_3B_4$ je pravouholník.
\insp{memo.2}%
}

{%%%%%   MEMO, priklad 4
Predpokladajme, že prirodzené čísla $x$, $y$ spĺňajú zadanú rovnosť. Potom $x^y=x!+y!\ge2$, teda $x\ge2$.

Uvažujme najprv prípad $x=2$. Z~rovnosti $2+y!=2^y$ vyplýva, že $y!$ je párne, preto $y\ge2$. Odtiaľ $2+y!=2^y\equiv0\pmod4$, takže $y!\equiv2\pmod4$. To je zrejme možné jedine pre $y\in\{2,3\}$. Keďže $2!+2!=2^2$ a~$2!+3!=2^3$, dostávame dve riešenia $(2,2)$ a~$(2,3)$.

Predpokladajme teraz, že $x\ge3$. V~takom prípade je číslo $x-1$ deliteľom $x!$, nie je však deliteľom $x^y$, lebo $\nsd(x,x-1)=1$. Z~toho vyplýva, že $x-1$ nedelí ani $x^y-x!=y!$, čiže $y\le x-2$. Môžeme teda písať
$$
x^y=x!+y!=y!(\underbrace{(y+1)\cdot(y+1)\cdot\dots\cdot x+1}_{\phantom{k}=k}).
$$
Činiteľ~$k$ je očividne nesúdeliteľný s~$x$ (a~teda aj s~každou mocninou~$x$) a~zároveň z~uvedenej rovnosti vyplýva, že je deliteľom čísla~$x^y$. Preto nutne $k=1$, čo je samozrejme nemožné.

Zadanú rovnosť spĺňajú iba dvojice $(2,2)$ a~$(2,3)$.
}

{%%%%%   MEMO, priklad t1
Označme daný výraz~$V$. S~využitím podmienky $abcd=1$ dostaneme postupnými úpravami
$$
\aligned
V&=\left(a+\frac1b\right)\left(b+\frac1c\right)\left(c+\frac1d\right)\left(d+\frac1a\right)=\\
&=\frac{(ab+1)(bc+1)(cd+1)(da+1)}{abcd}=
\frac{(2+ab+cd)(2+ad+bc)}{abcd}=\\
&=4+2\cdot\frac{ab+cd+ad+bc}{abcd}+\frac{a^2bd+b^2ac+c^2bd+d^2ac}{abcd}=\\
&=4+2\cdot\frac{(a+c)(b+d)}{\sqrt{abcd}}+\frac{bd(a^2+c^2)+ac(b^2+d^2)}{abcd}=\\
&=4+2\cdot\left(\sqrt{\frac ac}+\sqrt{\frac ca}\right)\left(\sqrt{\frac bd}+\sqrt{\frac db}\right)+
\left(\frac ac+\frac ca\right)+\left(\frac bd+\frac db\right)=\\
&=4+2\cdot f\left(\sqrt{\frac ac}\right)f\left(\sqrt{\frac bd}\right)+f\left(\frac ac\right)+f\left(\frac bd\right),
\endaligned
$$
kde $f$ je funkcia určená predpisom $f(x)=x+1/x$. Nerovnosť $x+1/x>y+1/y$ je pre kladné $x$, $y$ ekvivalentná s~nerovnosťou
$$
\frac{(x-y)(xy-1)}{xy}>0,
$$
a~teda uvedená funkcia je na intervale $\langle1,\infty)$ rastúca a~na intervale $(0,1\rangle$ klesajúca. Keďže $\frac12\le a,b,c,d\le2$, zrejme platia nerovnosti
$$
\frac14\le\frac ac\le4,\qquad \frac14\le\frac bd\le4, \qquad
\frac12\le\sqrt{\frac ac}\le2,\qquad \frac12\le\sqrt{\frac bd}\le2.
\tag1
$$
Navyše $f(\frac14)=f(4)=\frac{17}4$ a~$f(\frac12)=f(2)=\frac52$. Z~nerovností~\thetag1 a~s~rastúcosti, resp. klesajúcosti funkcie~$f$ na spomínaných intervaloch preto vyplýva
$$
V \le 4+2\cdot\frac52\cdot\frac52+\frac{17}4+\frac{17}4=25.
$$
Pritom rovnosť platí práve vtedy, keď $\frc ac,\frc bd\in\{\frac14,4\}$, \tj. pre štvorice $(2,2,\frac12,\frac12)$, $(2,\frac12,\frac12,2)$, $(\frac12,2,2,\frac12)$ a~$(\frac12,\frac12,2,2)$, ktoré "našťastie" spĺňajú aj podmienku $abcd=1$.

\odpoved
Maximálna hodnota uvedeného výrazu je $25$.

\poznamka
Z~uvedeného postupu vyplýva, že minimálna hodnota výrazu~$V$ je $4+2\cdot f(1)\cdot f(1)+f(1)+f(1)=16$ a~nadobúda sa pre štvorice $(a,b,c,d)$ spĺňajúce okrem podmienky $abcd=1$ aj rovnosti $a/c=b/d=1$, \tj. pre štvorice $(t,1/t,t,1/t)$, pričom $\frac12\le t\le2$.
}

{%%%%%   MEMO, priklad t2
Kvôli stručnosti nazývajme {\it škaredými\/} tie trojuholníky, ktoré nie sú ostrouhlé (\tj. sú pravouhlé alebo tupouhlé). Najskôr ukážeme, že aspoň jeden zo štyroch trojuholníkov určených danými štyrmi bodmi vo všeobecnej polohe je škaredý.

Ak sú uvedené štyri body vrcholmi konvexného štvoruholníka $W\!XY\!Z$, aspoň jeden z~jeho vnútorných uhlov má veľkosť aspoň $90^\circ$ (lebo súčet štyroch vnútorných uhlov v~ľubovoľnom štvoruholníku je $360^\circ$). Ak je to napríklad uhol pri vrchole~$X$, tak trojuholník $W\!XY$ je škaredý (\obr{}a).

Ak uvedené štyri body nie sú vrcholmi konvexného štvoruholníka, tak jeden z~nich (označme ho~$W$) leží vnútri trojuholníka $XY\!Z$ tvoreného zvyšnými tromi bodmi. Potom niektorý z~uhlov $X\!WY$, $YW\!Z$, $ZW\!X$ má veľkosť aspoň $120^\circ$ (lebo ich súčet je $360^\circ$) a~trojuholník s~týmto uhlom je škaredý (\obrr1b).
\instwopab{memo.3}{memo.4}{4.8}%

\smallskip
Uvažujme teraz ľubovoľnú päticu bodov $A$, $B$, $C$, $D$, $E$. Aspoň jeden z~trojuholníkov určených bodmi $A$, $B$, $C$, $D$ je škaredý. Bez ujmy na všeobecnosti nech je to $ABC$. Potom aspoň jeden z~trojuholníkov určených bodmi $B$, $C$, $D$, $E$ je škaredý a~trojuholník $ABC$ s~ním má spoločný aspoň jeden vrchol; bez ujmy na všeobecnosti nech je to vrchol~$B$. Aspoň jeden z~trojuholníkov určených bodmi $A$, $C$, $D$, $E$ je škaredý a~navyše to nie je žiadny z~doteraz objavených škaredých trojuholníkov (lebo nemá vrchol~$B$). Teda ľubovoľná množina~$P$ piatich bodov tvorí aspoň tri škaredé trojuholníky a~$a(P)\le7$ (zrejme päť bodov vo všeobecnej polohe tvorí práve 10~trojuholníkov).
\insp{memo.5}%

Stačí už len nájsť\footnote{Pred hľadaním je užitočné uvedomiť si, ako pre dané dva body $X$, $Y$ vyzerá množina takých bodov~$Z$, že trojuholník $XYZ$ je ostrouhlý.} príklad množiny~$P$, pre ktorú $a(P)=7$. Zoberme v~karteziánskej súradnicovej sústave päticu bodov
$$
A=(3,0),\quad B=(1,3),\quad C=(\m1,3),\quad D=(\m3,0),\quad E=(0,\m1).
$$
Ostré nie sú iba uhly $ABC$, $BCD$ a~$DEA$. Aby sme dokázali, že všetky ostatné sú ostré, stačí dokázať ostrosť uhlov $ABD$, $BCE$, $CEA$ a~$EAB$ (\obr); všetky ostatné uhly sú buď menšie (lebo sú ich časťami), alebo rovnaké (vďaka symetrii). Uhol $BCE$ je ostrý, lebo je vnútorným uhlom pri základni v~rovnoramennom trojuholníku $BCE$. Pre ostatné tri uhly stačí podľa kosínusovej vety\footnote{Alebo podľa Pytagorovej vety a~jednoduchej úvahy.} dokázať nerovnosti
$$
\aligned
|AB|^2+|BD|^2 &> |AD|^2,\\
|CE|^2+|EA|^2 &> |CA|^2,\\
|EA|^2+|AB|^2 &> |EB|^2.
\endaligned
$$
Jednoduchým výpočtom dostávame
$$
\aligned
|AB|^2+|BD|^2=(2^2+3^2)+(4^2+3^2)=38 &> 36=6^2=|AD|^2,\\
|CE|^2+|EA|^2=(4^2+1^2)+(1^2+3^2)=27 &> 25=4^2+3^2=|CA|^2,\\
|EA|^2+|AB|^2=(1^2+3^2)+(2^2+3^2)=23 &> 17=4^2+1^2=|EB|^2.
\endaligned
$$
}

{%%%%%   MEMO, priklad t3
Sú dva prípady, ako môžu byť v~MEMO-štvorstene umiestnené hrany dĺžok $2$ a~$3$.

\prip1
Nech hrany dĺžok $2$ a~$3$ vychádzajú z~jedného vrcholu~$A$; označme tieto hrany postupne $AB$ a~$AC$. Aby boli splnené trojuholníkové nerovnosti v~trojuholníku $ABC$ a~súčasne podmienky zo zadania, musí mať hrana~$BC$ dĺžku~$4$. Označme $D$ štvrtý vrchol skúmaného MEMO-štvorstena. Nech hrana~$AD$ má (celočíselnú) dĺžku~$a$. Potom hrana~$BD$ môže mať iba dĺžku $a+1$ alebo $a-1$ (aby boli splnené trojuholníkové nerovnosti v~trojuholníku $ABD$). Rozoberme obe možnosti.

Ak $|BD|=a+1$, musí platiť $a-3<|CD|<a+3$ (aby boli splnené trojuholníkové nerovnosti v~trojuholníku $ADC$). Kvôli rôznosti dĺžok všetkých šiestich hrán tak máme $|CD|\in\{a-2,a-1,a+2\}$ (\obr{}a). Ľahko overíme, že pre každú z~týchto troch hodnôt (pri zrejmej podmienke $a\ge5$) sú splnené všetky trojuholníkové nerovnosti vo~všetkých štyroch stenách štvorstena. Vypočítajme, aká môže byť hodnota~$s(T)$. 
\item{$\triangleright$} $|CD|=a-2$, $s(T)=2+3+4+a+(a+1)+(a-2)=3a+8$, nutná a~postačujúca podmienka, aby mali všetky hrany rôzne dĺžky, je $a\ge7$.
\item{$\triangleright$} $|CD|=a-1$, $s(T)=2+3+4+a+(a+1)+(a-1)=3a+9$, $a\ge6$.
\item{$\triangleright$} $|CD|=a+2$, $s(T)=2+3+4+a+(a+1)+(a+2)=3a+12$, $a\ge5$.

Ak $|BD|=a-1$, rovnako musí platiť $a-3<|CD|<a+3$ a~kvôli rôznosti dĺžok všetkých hrán dostávame $|CD|\in\{a-2,a+1,a+2\}$  (\obrr1b). Opäť jednoducho overíme platnosť všetkých trojuholníkových nerovností (pri zrejmej podmienke $a\ge6$) a vypočítame možné hodnoty~$s(T)$. 
\item{$\triangleright$} $|CD|=a-2$, $s(T)=2+3+4+a+(a-1)+(a-2)=3a+6$, $a\ge7$.
\item{$\triangleright$} $|CD|=a+1$, $s(T)=2+3+4+a+(a-1)+(a+1)=3a+9$, $a\ge6$.
\item{$\triangleright$} $|CD|=a+2$, $s(T)=2+3+4+a+(a-1)+(a+2)=3a+10$, $a\ge6$.
\instwopab{memo.6}{memo.7}{5}%

V~prvom prípade teda máme šesť rôznych typov MEMO-štvorstenov. Pri prvom type nadobúda $s(T)$ hodnoty
$$
\{3a+8;a\ge7\}=\{29,32,35,38,\dots\},
$$
pri druhom, treťom, štvrtom a~piatom type hodnoty
$$
\{3a+9;a\ge6\}=\{3a+12;a\ge5\}=\{3a+6;a\ge7\}=\{27,30,33,36,\dots\}
$$
a~pri šiestom type hodnoty
$$
\{3a+10;a\ge6\}=\{28,31,34,37,\dots\}.
$$
Zjednotením uvedených množín dostávame, že MEMO-štvorsten~$T$ spĺňajúci $s(T)=n$ existuje pre každé $n\ge27$.

\prip2
Nech hrany dĺžok $2$ a~$3$ nemajú žiadny spoločný bod; označme $AB$ hranu s~dĺžkou~$2$ a~$CD$ hranu s~dĺžkou~$3$. Aby boli splnené trojuholníkové nerovnosti v~trojuholníku $ABC$, musia sa dĺžky hrán $AC$, $BC$ líšiť o~$1$ (podľa zadania nemôžu byť rovnaké). Bez ujmy na všeobecnosti nech $|AC|=a$ a~$|BC|=a+1$ (označenie vrcholov $A$, $B$ totiž môžeme "vymeniť").

Aby boli splnené trojuholníkové nerovnosti v~trojuholníku $ACD$, nutne $a-3<|AD|<a+3$ a~kvôli rôznosti dĺžok hrán máme
$$
|AD|\in\{a-2,a-1,a+2\}.
$$
Podobným spôsobom (uvažujúc trojuholník $BCD$) dostávame $a-2<|BD|<a+4$, \tj.
$$
|BD|\in\{a-1,a+2,a+3\}.
$$
Navyše kvôli trojuholníkovej nerovnosti v~trojuholníku $ABD$ sa dĺžky $|AD|$ a~$|BD|$ musia líšiť práve o~$1$. Do úvahy teda prichádzajú iba dve možnosti: buď $|AD|=a-2$ a~$|BD|=a-1$, alebo $|AD|=a+2$ a~$|BD|=a+3$.
\item{$\triangleright$}
Ak $|AD|=a-2$ a~$|BD|=a-1$, tak $s(T)=2+3+a+(a+1)+(a-2)+(a-1)=4a+3$, pričom kvôli rôznosti hrán nutne $a\ge 6$.
\item{$\triangleright$}
Ak $|AD|=a+2$ a~$|BD|=a+3$, tak $s(T)=2+3+a+(a+1)+(a+2)+(a+3)=4a+11$, pričom $a\ge 4$.

Ľahko skontrolujeme, že pri oboch možnostiach sú splnené všetky trojuholníkové nerovnosti. V~skutočnosti sú štvorsteny, ktoré dostaneme pri prvej možnosti, zhodné so štvorstenmi z~druhej možnosti (stačí vymeniť označenie vrcholov $C$, $D$ a~"posunúť" hodnotu $a$ o~$2$). V~druhom prípade preto máme iba jeden typ štvorstenov: $|AB|=2$, $|CD|=3$, $|AC|=a$, $|BC|=a+1$, $|AD|=a+2$ a~$|BD|=a+3$, pričom $a\ge4$ (\obr). Hodnota $s(T)$ je v~tomto prípade prvkom množiny
$$
\{4a+11;a\ge4\}=\{27,31,35,39,\dots\}.
$$ 
\insp{memo.8}%

\smallskip
Na základe predošlej analýzy jednoducho vyriešime obe časti úlohy. 

\smallskip
a) Keďže v~druhom prípade sme nezískali žiadne nové hodnoty pre $s(T)$, platí odpoveď získaná v~prvom prípade: MEMO-štvorsten~$T$ spĺňajúci $s(T)=n$ existuje práve vtedy, keď $n\ge27$.

\smallskip
b) Číslo 2007 vieme zapísať v~tvaroch
$$
2007=3\cdot666+9=3\cdot665+12=3\cdot667+6=4\cdot499+11.
$$
MEMO-štvorsten~$T$ spĺňajúci $s(T)=2007$ teda vieme vytvoriť podľa druhého, tretieho, štvrtého a~piateho typu v~prvom prípade aj podľa jediného typu v~druhom prípade. Spolu je to 5~rôznych MEMO-štvorstenov (z~uvedenej analýzy je zrejmé, že sú navzájom nezhodné, o~čom sa možno ľahko presvedčiť aj vypísaním ich konkrétnych dĺžok hrán).  
}

{%%%%%   MEMO, priklad t4
Číslo $(a+k)^3-a^3$ je násobkom čísla $2007=9\cdot223$ práve vtedy, keď je násobkom $9$ aj násobkom $223$. Keďže 
$$
(a+k)^3-a^3=3(a^2k+ak^2)+k^3,
$$
nutnou podmienkou na to, aby uvedené číslo bolo násobkom deviatich (a~teda aj troch) je, aby $k^3$ bolo deliteľné tromi. To platí len pre $k$, ktoré sú sami násobkom troch. Preto každé hľadané $k$ možno zapísať v~tvare $k=3m$ pre nejaké prirodzené číslo~$m$. Naopak, ak $k=3m$, tak
$$
(a+k)^3-a^3=(a+3m)^3-a^3=9(a^2m+3am^2+3m^3),
$$
čiže uvedené číslo je deliteľné deviatimi.

Takže našou úlohou je určiť prirodzené čísla~$m$, pre ktoré existuje celé číslo~$a$ také, že číslo $(a+3m)^3-a^3$
je násobkom $223$. Uvedenú podmienku spĺňa každé prirodzené~$m$. Stačí položiť napríklad $a=38m$. Potom
$$
(a+3m)^3-a^3=(41m)^3-(38m)^3=14\,049m^3=9\cdot7\cdot223m^3.
$$

\odpoved
Zadanú vlastnosť majú všetky prirodzené čísla~$k$, ktoré sú násobkom troch.

\poznamka
K~hodnote $a=38m$ sa dá dopracovať skúšaním. Existuje však aj iná možnosť. Úlohu totiž možno ekvivalentne preformulovať v~reči kongruencií: {\sl Určte tie zvyškové triedy~$m$, ku ktorým existuje zvyšková trieda~$a$ taká, že
$$
(a+3m)^3\equiv a^3\pmod{223}.
\tag1
$$}%

Ak danú podmienku spĺňa nejaký zvyšok~$m$, spĺňa ju aj každý jeho násobok $mt$, lebo vynásobením kongruencie \thetag1 zvyškom~$t^3$ dostaneme
$$
(at+3mt)^3\equiv (at)^3\pmod{223}.
$$
Keďže $223$ je prvočíslo, stačí nájsť jeden nenulový zvyšok $m$, ktorý spĺňa \thetag1 pre nejaký zvyšok~$a$. Každý iný zvyšok sa dá napísať ako jeho vhodný násobok\footnote{Inak povedané, pre nenulový zvyšok~$m$ obsahuje množina $\{m,2m,\dots,222m\}$ všetky nenulové zvyšky modulo $223$.}. Stačí teda úlohu vyriešiť napr. pre hodnotu $m=1$, \tj. nájsť také $a$, že
$$
(a+3)^3\equiv a^3\pmod{223}.
\tag2
$$
Predpokladajme, že zvyšok $a$ spĺňa \thetag2. Potom $a\ne0$ a~teda $3\equiv ra\pmod{223}$ pre vhodné $r$. Po vydelení kongruencie \thetag2 zvyškom $a^3$ (zrejme $\nsd(a^3,223)=1$) dostaneme
$$
(1+r)^3\equiv 1\pmod{223}.
\tag3
$$
Zvyšok~$r$ spĺňajúci \thetag3 už nájdeme ľahko. Podľa malej Fermatovej vety totiž $z^{222}\equiv1\pmod{223}$ pre každý nenulový zvyšok~$z$. Zároveň $z^{222}=(z^{74})^3$. Preto zvolíme $r\equiv z^{74}-1\pmod{223}$. Nenulové $r$ tak dostaneme napr. pre $z=3$, ale aj pre mnohé iné hodnoty\footnote{Treba zobrať také $z$, že množina $\{1,z,z^2,\dots,z^{222}\}$ obsahuje všetky nenulové zvyšky, \tj. každý nenulový zvyšok práve raz. Tým je zabezpečené $z^{74}\nequiv1\pmod{223}$. Také $z$ vždy existuje, čo je známy, i~keď nie triviálny fakt z~teórie čísel.}. Konkrétne $r\equiv3^{74}-1\equiv182\pmod{223}$ a~odtiaľ $a\equiv38\pmod{223}$ (lebo $182\cdot38\equiv3\pmod{223}$).

}
