{%%%%%   A-I-1
Určte všetky dvojice $(p,q)$ reálnych čísel také, že
rovnica $x^2+px+q=0$ má riešenie v~obore reálnych čísel,
pričom platí:
Ak $t$ je koreňom tejto rovnice, potom aj $|2t-15|$ je jej koreňom.}
\podpis{P. Černek}

{%%%%%   A-I-2
V~rovine daného štvorca $KLMN$ určte množinu všetkých bodov~$P$,
pre ktoré sú uhly $NPK$, $KPL$ a~$LPM$ zhodné.}
\podpis{J. Švrček}

{%%%%%   A-I-3
Pre ľubovoľné prirodzené číslo~$k$ zostavme z~písmen $A$, $B$
všetky možná "slová" dĺžky~$k$. Rozdeľme ich do dvoch skupín $P_k$
a~$N_k$ podľa toho, či je v~danom slove párny alebo nepárny počet
"slabík"~$BA$ (za párny považujeme aj počet~$0$). Napríklad slová
$\underline{BA}BBB\underline{BA}$ a~$AAAAAAB$ patria do
skupiny~$P_7$, slová $AAB\underline{BA}BB$
a~$\underline{BA}\,\underline{BA}A\underline{BA}$ patria do
skupiny~$N_7$. Zistite, pre ktoré~$k$ majú skupiny $P_k$ a~$N_k$
rovnaký počet prvkov.}
\podpis{J. Šimša}

{%%%%%   A-I-4
Určte najmenšie reálne číslo~$p$ také, že nerovnosť
$$
\sqrt{1^2+1}+\sqrt{2^2+1}+\sqrt{3^2+1}+\dots+
\sqrt{n^2+1}\leqq\frac12 n(n+p)
$$
platí pre každé prirodzené číslo~$n$.}
\podpis{S. Trávníček}

{%%%%%   A-I-5
Nech $ABCD$ je tetivový štvoruholník, ktorého vnútorný uhol pri
vrchole~$B$ má veľkosť~$60^{\circ}$.
\ite a) Ak $|BC|=|CD|$, potom platí $|CD|+|DA|=|AB|$; dokážte.
\ite b) Rozhodnite, či platí opačná implikácia.}
\podpis{E. Kováč}

{%%%%%   A-I-6
V~obore reálnych čísel vyriešte sústavu rovníc
$$
x^2=\frac{1}{y}+\frac{1}{z},\quad
y^2=\frac{1}{z}+\frac{1}{x},\quad
z^2=\frac{1}{x}+\frac{1}{y}.
$$}
\podpis{J. Šimša}

{%%%%%   B-I-1
Každú z~hviezdičiek na mieste jednotiek vo výraze
$$
\left|\frac{777\,777\,777\,77*}{777\,777\,777\,77*}-
\frac{555\,555\,555\,554}{555\,555\,555\,559}\right|
$$
nahraďte nejakou číslicou tak, aby výraz mal čo najmenšiu hodnotu.}
\podpis{J. Šimša}

{%%%%%   B-I-2
V~rovnoramennom lichobežníku $ABCD$ platí $|BC|=|CD|=|DA|$
a~$|\uhol DAB|=|\uhol ABC|=36\st$. Na základni~$AB$ je daný bod~$K$
tak, že $|AK|=|AD|$. Dokážte, že kružnice opísané trojuholníkom $AKD$
a~$KBC$ majú vonkajší dotyk.}
\podpis{J. Zhouf}

{%%%%%   B-I-3
V~obore reálnych čísel riešte rovnicu
$$
x\lfloor x\rfloor-5x+7=0,
$$
kde $\lfloor x\rfloor$ znamená dolnú celú časť čísla~$x$, teda
najväčšie celé číslo~$k$, pre ktoré platí $k\leqq x$. (Napríklad
$\lfloor\sqrt{2}\rfloor=1$ a~$\lfloor\m3{,}1\rfloor=\m4$.)}
\podpis{E. Kováč}

{%%%%%   B-I-4
Číslo~$a_n$ vznikne tak, že za seba napíšeme prvých~$n$ po sebe
idúcich prirodzených čísel, napríklad
$a_{13}=12\,345\,678\,910\,111\,213$. Zistite, koľko čísel
deliteľných~$24$ sa nachádza medzi číslami $a_1,a_2,\dots,a_{10\,000}$.}
\podpis{P. Černek}

{%%%%%   B-I-5
Je daná priamka~$p$ a~bod~$A$, ktorý na nej neleží. Zostrojte lichobežník
$ABCD$ s~minimálnym obsahom a~ramenom~$BC$ na priamke~$p$ tak, aby boli splnené rovnosti
$|BC|=|AC|$ a~$|BE|=3|DE|$, kde $E$ je priesečník uhlopriečok lichobežníka.}
\podpis{P. Leischner}

{%%%%%   B-I-6
Určte všetky prirodzené čísla~$M$ deliteľné $240$, pre ktoré má
rovnica $M=\text{NSN}(x,y)$ s~neznámymi $x$, $y$ práve
$1\,001$ riešení v~obore prirodzených čísel. (Symbol $\text{NSN}(x,y)$ značí
najmenší spoločný násobok čísel $x$ a~$y$.)}
\podpis{P. Černek}

{%%%%%   C-I-1
Dokážte, že pre každé prirodzené číslo~$n$, ktoré je väčšie ako~$3$
a~nie je deliteľné tromi, platí:
Šachovnicu $n\times n$ je možné rozrezať na jeden štvorec
$1\times1$ a~obdĺžniky $3\times1$.}
\podpis{J. Zhouf}

{%%%%%   C-I-2
Daný je obdĺžnik $ABCD$. Nech priamky $p$ a~$q$, ktoré
prechádzajú vrcholom~$A$, pretínajú polkružnice zvonku pripísané
stranám $BC$ a~$CD$ v~bodoch $K$ a~$L$
($B\ne K\ne C\ne L\ne D$) a~taktiež strany $BC$ a~$CD$
v~bodoch $P$ a~$Q$ tak, že trojuholník $ABP$ má taký istý obsah ako
trojuholník $KCP$ a~súčasne trojuholník $AQD$ má taký istý obsah
ako trojuholník $CLQ$. Dokážte, že body $K$, $L$, $C$ ležia na
jednej priamke.}
\podpis{J. Švrček}

{%%%%%   C-I-3
Žiak mal vypočítať príklad $X\cdot Y:Z$, kde $X$ je dvojciferné
číslo, $Y$~trojciferné číslo a~$Z$ trojciferné číslo s~číslicou~$2$
na mieste jednotiek. Výsledkom príkladu malo byť prirodzené číslo. Žiak
ale prehliadol bodku a~súčin $X\cdot Y$ chápal ako päťciferné
číslo. Dostal tak sedemkrát väčší výsledok ako mal vyjsť.
Aký príklad mal žiak počítať?}
\podpis{P. Černek}

{%%%%%   C-I-4
Nech $P$ je ľubovoľný vnútorný bod rovnostranného trojuholníka
$ABC$. Uvažujme
obrazy $K$, $L$ a~$M$ bodu $P$ v~osových súmernostiach s~osami
$AB$, $BC$ a~$CA$. Určte množinu všetkých bodov~$P$ takých, že
trojuholník $KLM$ je rovnoramenný.}
\podpis{J. Zhouf}

{%%%%%   C-I-5
Prirodzené číslo nazveme {\it magickým} práve vtedy, keď sa dá
rozložiť na súčet dvoch trojmiestnych čísel zapísaných rovnakými
číslicami, ale v~opačnom poradí. Napríklad číslo $1\,413$ je
magické, lebo $1\,413=756+657$; najmenšie magické číslo je
$202$.
\ite a) Určte počet všetkých magických čísel.
\ite b) Ukážte, že súčet všetkých magických čísel je $187\,000$.}
\podpis{J. Šimša}

{%%%%%   C-I-6
Zo všetkých štvoruholníkov, ktoré sa dajú vpísať do danej kružnice
s~polomerom~$r$ a~ktoré majú dve strany danej dĺžky~$m$,
určte tie, ktoré majú najväčší obsah.}
\podpis{P. Leischner}

{%%%%%   A-S-1
Nech $P(x)=ax^2+bx+c$ je kvadratický trojčlen s~nezápornými
reálnymi koeficientmi.
Dokážte, že pre ľubovoľné kladné číslo~$x$ platí
$$
P(x)\cdot P\Bigl(\frac1x\Bigr)\geq \big(P(1)\big)^2.
$$}
\podpis{E. Kováč}

{%%%%%   A-S-2
Určte, akú najväčšiu dĺžku môže mať uhlopriečka~$CE$ konvexného
päťuholníka $ABC\!D\!E$, ktorého strana~$AB$ má dĺžku $6\cm$, vnútorné
uhly pri vrcholoch $C$ a~$E$ sú pravé a~uhol $ADB$ má
veľkosť $120^{\circ}$.}
\podpis{P. Černek}

{%%%%%   A-S-3
V~obore reálnych čísel vyriešte sústavu rovníc
$$
\align
   x^2+2yz &= 6(y+z-2),\\
   y^2+2zx &= 6(z+x-2),\\
   z^2+2xy &= 6(x+y-2).
\endalign
$$}
\podpis{J. Šimša}

{%%%%%   A-II-1
Určte počet všetkých päťmiestnych palindrómov, ktoré sú deliteľné číslom~$37$. (Palindrómom nazývame číslo,
ktorého zápis v~desiatkovej sústave sa číta rovnako spredu aj zozadu.)}
\podpis{J. Šimša}

{%%%%%   A-II-2
Pre ľubovoľné kladné celé číslo~$n$ zostavme z~písmen~$A$ a~$B$
všetky možné "slová" dĺžky~$n$ a~označme $p_n$ počet tých
z~nich, ktoré neobsahujú ani trojicu $AAA$ po sebe nasledujúcich
písmen~$A$ ani dvojicu~$BB$ po sebe nasledujúcich písmen~$B$. Zistite,
pre ktoré kladné celé čísla~$n$ platí, že obe čísla $p_n$
a~$p_{n+1}$ sú párne.}
\podpis{R. Kučera}

{%%%%%   A-II-3
Označme $K$ ľubovoľný vnútorný bod strany $AB$ daného
  trojuholníka $ABC$. Priamka $CK$ pretína kružnicu opísanú trojuholníku
  $ABC$ v~bode~$L$ ($L\ne C$). Označme $k_1$ kružnicu opísanú
  trojuholníku $AK\!L$ a~$k_2$ kružnicu opísanú trojuholníku $BK\!L$.
\ite a) Dokážte, že priamka~$AC$ je dotyčnicou ku kružnici~$k_1$ práve
vtedy, keď priamka~$BC$ je dotyčnicou ku kružnici~$k_2$.
\ite b) Predpokladajme, že priamka~$AC$ je sečnicou kružnice~$k_1$.
Nech $P$ ($P\ne A$) je priesečník priamky~$AC$ s~kružnicou~$k_1$
a~$Q$ ($Q\ne B$) je priesečník priamky~$BC$ s~kružnicou~$k_2$. Dokážte, že
bod~$K$ leží na úsečke~$PQ$.

}
\podpis{J. Šimša, J. Zhouf}

{%%%%%   A-II-4
Nech $K$, $L$ a~$M$ sú postupne priesečníky osí vnútorných
uhlov $\alpha$, $\beta$ a~$\gamma$ pri vrcholoch $A$, $B$ a~$C$
daného trojuholníka $ABC$ s~protiľahlými stranami $BC$, $CA$ a~$AB$.
Dokážte, že platí nerovnosť
$$
\frac{|BC|}{|AK|}\cos\frac{\alpha}2+
     \frac{|CA|}{|BL|}\cos\frac{\beta}2+
     \frac{|AB|}{|CM|}\cos\frac{\gamma}2\geq 3.
$$}
\podpis{J. Švrček}

{%%%%%   A-III-1
Určte všetky trojice $(x,y,z)$ reálnych čísel, pre
ktoré platí
$$
x^2+y^2+z^2\leq 6+\min\Bigl\{x^2-\frac{8}{x^4},\
      y^2-\frac{8}{y^4},\ z^2-\frac{8}{z^4}\Bigr\}.
$$}
\podpis{J. Švrček}

{%%%%%   A-III-2
Pre ľubovoľné prirodzené číslo~$n$ zostavme z~písmen $A$ a~$B$
všetky možné "slová" dĺžky~$n$ a~označme $p_n$ počet tých,
ktoré neobsahujú štvoricu $AAAA$ po sebe idúcich
písmen~$A$ ani trojicu $BBB$ po sebe idúcich písmen~$B$. Určte
hodnotu výrazu
$$
\frac{p_{2004}-p_{2002}-p_{1999}}{p_{2001}+p_{2000}}.
$$}
\podpis{R. Kučera}

{%%%%%   A-III-3
V~rovine je daná kružnica~$k$ a~$121$ jej sečníc
$p_1,p_2,\dots,p_{121}$. Vnútri tejto kružnice je na každej
priamke~$p_i$ daný bod~$A_i$. Dokážte, že na kružnici~$k$ existuje
taký bod~$X$, že úsečka~$A_iX$ zviera s~priamkou~$p_i$ uhol menší
ako $21^{\circ}$ pre najmenej $29$~rôznych indexov~$i$.}
\podpis{J. Šimša}

{%%%%%   A-III-4
Zistite, pre ktoré prirodzené čísla~$n$ je súčet
$$
{n\over1!}+{n\over2!}+\dots+{n\over n!}
$$
celé číslo.}
\podpis{E. Kováč}

{%%%%%   A-III-5
Nech $L$ je ľubovoľný vnútorný bod kratšieho oblúka~$CD$ kružnice
opísanej štvorcu $ABCD$. Označme $K$ priesečník priamok $AL$ a~$CD$,
$M$ priesečník priamok $AD$ a~$CL$ a~$N$ priesečník priamok $MK$
a~$BC$. Dokážte, že body $B$, $L$, $M$, $N$ ležia na tej istej
kružnici.}
\podpis{J. Švrček}

{%%%%%   A-III-6
Nech $\Bbb R^+$ je množina všetkých kladných reálnych čísel.
Určte všetky funkcie $f:\Bbb R^{+} \to \Bbb R^{+}$, ktoré
pre všetky kladné čísla $x$, $y$ spĺňajú rovnosť
$$
x^2\bigl(f(x)+f(y)\bigr)=(x+y)f\bigl(f(x)y\bigr).
$$}
\podpis{P. Kaňovský}

{%%%%%   B-S-1
Určte, koľko riešení má v~obore reálnych čísel rovnica
$$
x=\lfloor x\rfloor+\frac{x}{2\,004},
$$
kde $\lfloor x\rfloor$ označuje dolnú celú časť čísla~$x$, \tj. najväčšie celé číslo, ktoré
je menšie alebo rovné ako~$x$.}
\podpis{J. Šimša}

{%%%%%   B-S-2
Uveďte príklad množiny~$\mm M$ pozostávajúcej z dvojciferných čísel, ktorá má
maximálny počet prvkov a~pritom spĺňa obidve nasledujúce podmienky:
\ite (i) Každé dve čísla patriace do množiny~$\mm M$ sú nesúdeliteľné.
\ite (ii) Ak zmeníme poradie číslic ľubovoľného čísla patriaceho do množiny~$\mm M$,
dostaneme opäť číslo patriace do množiny~$\mm M$.

}
\podpis{J. Földes}

{%%%%%   B-S-3
Nech $ABCD$ je lichobežník s~ostrými uhlami pri základni~$AB$. Nech $E$ je
taký bod základne~$AB$, že kružnice opísané trojuholníkom $AED$ a~$EBC$ sa dotýkajú
zvonku. Dokážte, že bod~$E$ leží na kružnici opísanej trojuholníku $CDV$,
kde $V$ je priesečník priamok $AD$ a~$BC$.}
\podpis{R. Horenský}

{%%%%%   B-II-1
Číslo~$a_n$ vznikne tak, že za seba zapíšeme prvých~$n$ druhých
mocnín po sebe idúcich prirodzených čísel. Napríklad
$a_{11}=149\,162\,536\,496\,481\,100\,121$. Zistite, koľko čísel
deliteľných dvanástimi je medzi číslami $a_1,a_2,\dots,a_{100\,000}$.}
\podpis{P. Černek}

{%%%%%   B-II-2
Nájdite všetky kvadratické trojčleny
$ax^2  + bx + c$
také, že ak ľubovoľný z~koeficientov $a$, $b$, $c$ zväčšíme
o~$1$, dostaneme nový kvadratický trojčlen, ktorý bude mať
dvojnásobný koreň.}
\podpis{E. Kováč}

{%%%%%   B-II-3
Pre dané prirodzené číslo~$n$ vyriešte
v~obore kladných reálnych čísel rovnicu
$$
\bigl\lfloor{x\sqrt{n^2-1}}\bigr\rfloor=nx-1.
$$
(Symbol $\lfloor r\rfloor$ označuje najväčšie
celé číslo, ktoré nie je väčšie ako~$r$.)}
\podpis{J. Šimša}

{%%%%%   B-II-4
Daný je ostrouhlý trojuholník $VBA$. Zostrojte dotyčnicový štvoruholník $ABCD$
s~minimálnym obsahom tak, aby jeho vrcholy $C$, $D$ ležali postupne na polpriamkach
opačných k~polpriamkam $BV$ a~$AV$.}
\podpis{P. Leischner}

{%%%%%   C-S-1
Určte počet všetkých trojciferných čísel, ktoré sú devätnásťkrát
väčšie ako ich ciferný súčet.}
\podpis{J. Šimša}

{%%%%%   C-S-2
Je daný štvorec so stranou dĺžky $5\cm$. Uvažujme štvoruholník, ktorý
leží v~danom štvorci tak, že má dve strany dlhé $2\cm$, pričom obidve tieto
strany ležia na obvode daného štvorca. Nájdite všetky štvoruholníky s~danými
vlastnosťami, ktoré majú maximálny obsah.}
\podpis{P. Leischner}

{%%%%%   C-S-3
Dlaždica~A je zložená z~troch jednotkových štvorcov a~má tvar~\Image*{53-c-s-3 l}.
Dlaždica~B je zložená zo štyroch jednotkových štvorcov a~má tvar~\Image*{53-c-s-3 t}.
Koľko dlaždíc jednotlivých typov potrebujeme na vydláždenie
štvorca so stranou 6~jednotiek? Pre každý možný počet dlaždíc
uveďte príklad takého pokrytia.}
\podpis{J. Földes}

{%%%%%   C-II-1
V~rovine je daný obdĺžnik $ABCD$, kde $|AB|=a<b=|BC|$. Na jeho
strane~$BC$ existuje bod~$K$ a~na strane~$CD$ bod~$L$ tak, že
daný obdĺžnik je úsečkami $AK$, $KL$ a~$LA$ rozdelený na štyri
navzájom podobné trojuholníky. Určte hodnotu pomeru $a:b$.}
\podpis{J. Švrček}

{%%%%%   C-II-2
Nájdite všetky trojice prvočísiel $p$, $q$, $r$, pre ktoré platí
$$
{14\over p}+{51\over q}={65\over r}.
$$}
\podpis{Pavel Novotný}

{%%%%%   C-II-3
Do kružnice s~polomerom $r=6$ vpíšte osemuholník $ABCDEFGH$,
ktorého strany $AB$, $CD$, $EF$ a~$GH$ majú postupne dĺžky $3$, $4$, $5$
a~$6$ a~strany $BC$, $DE$, $FG$ a~$HA$ sú zhodné.}
\podpis{Pavel Novotný}

{%%%%%   C-II-4
Žiaci mali vypočítať príklad $x+y\cdot z$ pre trojciferné číslo
$x$ a~dvojciferné čísla $y$, $z$. Martin vie násobiť a~sčitovať
čísla zapísané v~desiatkovej sústave, ale zabudol na pravidlo
prednosti násobenia pred sčitovaním. Preto mu vyšlo síce zaujímavé
číslo, ktoré sa číta rovnako zľava doprava ako sprava doľava,
ale správny výsledok bol o~$2004$ menší. Určte čísla $x$, $y$, $z$.}
\podpis{J. Šimša}

{%%%%%   vyberko, den 1, priklad 1
Je daný trojuholník $ABC$ a~na strane~$BC$ bod~$D$ tak, že $|AD|>|BC|$. Bod~$E$ na strane~$AC$ je určený pomerom
$$
\frac{|AE|}{|EC|}=\frac{|BD|}{|AD|-|BC|}.
$$
Dokážte, že platí $|AD|>|BE|$.}
\podpis{Mgr. František Kardoš:???}

{%%%%%   vyberko, den 1, priklad 2
Nájdite všetky ohraničené postupnosti $a_1, a_2, \dots$ prirodzených čísel také, že platí
$$
a_n = \frac{a_{n-1}+a_{n-2}}{(a_{n-1},a_{n-2})}
$$
pre každé $n>2$. ($(k,\ell)$ označuje najväčšieho spoločného deliteľa čísel $k$ a~$\ell$.)}
\podpis{Mgr. František Kardoš:???}

{%%%%%   vyberko, den 1, priklad 3
\ite a) Anička má dve škatule, z~ktorých v~každej je 6~loptičiek označených číslami $1$ až $6$. Náhodne vyberie po jednej loptičke z~každej škatule. Nech $p_n$ označuje pravdepodobnosť, že súčet vybraných čísel je rovný~$n$. Vypočítajte $p_n$ pre všetky $n\in\Bbb{N}$.
\ite b) Betka má tiež dve škatule a~v~každej 6~loptičiek označených (neznámymi) prirodzenými číslami. Čísla sa môžu opakovať, nemusia byť také isté v~oboch škatuliach. Ak Betka vyberie náhodne po jednej loptičke z~každej škatule, pravdepodobnosť, že súčet bude rovný~$n$, je opäť~$p_n$ (rovnako ako u~Aničky). Aké čísla sú na Betkiných loptičkách? Nájdite všetky možnosti.}
\podpis{Mgr. František Kardoš:???}

{%%%%%   vyberko, den 1, priklad 4
Každý štvorček veľkého štvorca $50\times 50$ je ofarbený jednou zo štyroch farieb. Ukážte, že existuje štvorček, ktorý má rovnakú farbu ako niektorý štvorček napravo, naľavo, hore i~dole od neho (nie nutne susedný).}
\podpis{Mgr. František Kardoš:???}

{%%%%%   vyberko, den 2, priklad 1
Na oblúku~$BC$ kružnice opísanej trojuholníku $ABC$, ktorý neobsahuje bod~$A$, zvolíme bod~$P$.
Na polpriamkach $AP$, $BP$ zvolíme postupne body $X$, $Y$ tak, aby $|AC|=|AX|$, $|BC|=|BY|$. Ukážte, že priamky $XY$ prechádzajú pre pohybujúci sa bod~$P$ pevným bodom.}
\podpis{Tomáš Jurík:???}

{%%%%%   vyberko, den 2, priklad 2
Nech $\Cal S$ je množina $2004^2+1$ prirodzených čísel väčších ako~$1$. Platí, že pre každé prirodzené číslo~$n$ existuje nejaké $s\in\Cal S$ také, že $NSD(s,n)=1$ alebo $NSD(s,n)=s$. Dokážte, že existujú $s,t\in\Cal S$ také, že $NSD(s,t)$ je prvočíslo.}
\podpis{Tomáš Jurík:putnam 99 B6}

{%%%%%   vyberko, den 2, priklad 3
Pre prirodzené číslo~$n$ a~reálne číslo~$c$ definujeme $x_k$ rekurzívne: $x_0=0$, $x_1=1$ a
$$
x_{k+2}=\frac{cx_{k+1}-(n-k)x_k}{k+1}\quad\text{pre}\ k\ge 0.
$$
Pre pevné~$n$ označme $c$ najväčšie číslo, pre ktoré $x_{n+1}=0$. Pre takto zvolené~$c$ nájdite predpis pre $x_k$ iba za pomoci $n$ a~$k$ pre $1\le k\le n$.}
\podpis{Tomáš Jurík:???}

{%%%%%   vyberko, den 3, priklad 1
Nech $a$, $b$, $c$ sú nezáporné reálne čísla také, že $a+b+c=1$. Nájdite maximum výrazu
$$
a^2+b^2+c^2+\sqrt{12abc}.
$$}
\podpis{Ján Mazák:???}

{%%%%%   vyberko, den 3, priklad 2
Daný je štvorsten taký, že guľa so stredom v~bode~$O$ sa dotýka
všetkých jeho šiestich hrán. Navyše štyri gule so stredmi vo vrcholoch štvorstena
sa po dvoch zvonka dotýkajú a~všetky štyri sa dotýkajú inej gule so stredom
v~bode~$O$. Dokážte, že takýto štvorsten musí byť pravidelný.}
\podpis{Ján Mazák:???}

{%%%%%   vyberko, den 3, priklad 3
Nech $ABC$ je trojuholník. Na jeho stranách $AB$, $AC$ ležia
v~tomto poradí body $D$, $E$ tak, že priamka~$DE$ je rovnobežná s~priamkou~$BC$.
Nech $P$ je ľubovoľný vnútorný bod trojuholníka $ADE$ a~nech $F$, $G$
sú postupne priesečníky priamky~$DE$ s~priamkami $BP$ a~$CP$. Nech $Q$ je druhý
priesečník (rôzny od~$P$) kružníc opísaných trojuholníkom $PDG$ a~$PFE$.
Dokážte, že body $A$, $P$, $Q$ ležia na priamke.}
\podpis{Ján Mazák:???}

{%%%%%   vyberko, den 3, priklad 4
Dokážte, že ak $n$ je prirodzené číslo také, že rovnica
$$
x^3-3xy^2+y^3=n
$$
má riešenie $(x,y)$ v~celých číslach, tak má táto rovnica aspoň tri takéto riešenia.
Nájdite všetky jej riešenia pre $n=2891$.}
\podpis{Ján Mazák:???}

{%%%%%   vyberko, den 4, priklad 1
Bod~$P$ leží vnútri trojuholníka $ABC$. $D$, $E$ a~$F$ sú päty kolmíc spustených z~$P$ postupne na strany $BC$, $CA$ a~$AB$. Predpokladajme, že platí
$$
|AP|^2 + |PD|^2 = |BP|^2 + |PE|^2 = |CP|^2 + |PF|^2.
$$
Označme $I_A$, $I_B$ a~$I_C$ stredy kružníc pripísaných ku stranám trojuholníka $ABC$. Dokážte, že $P$ je stred kružnice opísanej trojuholníku $I_AI_BI_C$.}
\podpis{Peter Novotný:Shortlist Tokyo 2003}

{%%%%%   vyberko, den 4, priklad 2
Každé prirodzené číslo~$a$ podstúpi nasledovnú procedúru pre získanie hodnoty $d=d(a)$:
\ite (i) poslednú cifru čísla~$a$ presunieme na začiatok, dostaneme tak číslo~$b$;
\ite (ii) umocníme $b$ na druhú, dostaneme tak číslo~$c$;
\ite (iii) prvú cifru čísla~$c$ presunieme na koniec, dostaneme číslo~$d$.

(Všetky čísla sú zapísané v~desiatkovej sústave.) Napríklad pre $a=203$ dostaneme $b=320$, $c=102400$ a~$d=024001=24001=d(203)$. Nájdite všetky čísla~$a$, pre ktoré $d(a)=a^2$.}
\podpis{Peter Novotný:Shortlist Tokyo 2003}

{%%%%%   vyberko, den 4, priklad 3
Nájdite všetky prirodzené čísla~$n$ také, že pravidelný šesťuholník sa dá rozdeliť na $n$~rovnobežníkov, ktoré majú všetky rovnaký obsah.}
\podpis{Peter Novotný:International Tournament of Towns, 1986}

{%%%%%   vyberko, den 5, priklad 1
Nech $n$ a~$r$ sú kladné celé čísla a~nech $A$ je podmnožina množiny mrežových bodov (body s~celočíselnými súradnicami) v~rovine, taká, že ľubovoľný kruh (bez hraničnej kružnice) s~polomerom~$r$
obsahuje bod z~množiny~$A$.
Dokážte, že ak ľubovoľne ofarbíme množinu~$A$ s~$n$ farbami, tak budú existovať štyri body rovnakej
farby, ktoré tvoria vrcholy obdĺžnika.}
\podpis{Mgr. Juraj Földes:mathematical contests 1996/1997 str. 50 priklad 8. rumunsko}

{%%%%%   vyberko, den 5, priklad 2
Označme $T$ ťažisko trojuholníka $ABC$. Dokážte, že platí
$$
\sin{|\uhol CAT|} + \sin{|\uhol CBT|} \leq \frac{2}{\sqrt{3}} \,.
$$}
\podpis{Mgr. Juraj Földes:mathematical contests 97/98 str.20 priklad 19 bulharsko}

{%%%%%   vyberko, den 5, priklad 3
Nájdite všetky funkcie $f: \Bbb N_0 \rightarrow \Bbb N_0$, ktoré spĺňajú rovnosť
$$
f(f(n)) + f(n) = 2n + 3k \qquad \text{pre všetky} \quad n \in \Bbb N_0,
$$
kde $k$ je pevné kladné celé číslo a~$\Bbb N_0$ označuje množinu všetkých nezáporných celých čísel.}
\podpis{Mgr. Juraj Földes:funcional equations str20 problem 2.9}

{%%%%%   vyberko, den 2, priklad 4
-}
\podpis{-:-}

{%%%%%   vyberko, den 4, priklad 4
-}
\podpis{-:-}

{%%%%%   vyberko, den 5, priklad 4
-}
\podpis{-:-}

{%%%%%   trojstretnutie, priklad 1
Dokážte, že reálne čísla $p$, $q$, $r$ spĺňajú podmienku
$$
p^4(q-r)^2+2p^2(q+r)+1=p^4
%  \tag1
$$
práve vtedy, keď kvadratické rovnice
$$
x^2+px+q=0,\qquad y^2-py+r=0
% \tag2
$$
majú reálne korene (nie nutne rôzne), ktoré možno označiť
$x_{1,2}$ resp.~$y_{1,2}$ v~takom poradí, že platí rovnosť
$x_1y_1-x_2y_2=1$.}
\podpis{J. Šimša}

{%%%%%   trojstretnutie, priklad 2
Dokážte, že pre každé prirodzené číslo~$k$ existuje najviac konečne
veľa takých trojíc navzájom rôznych prvočísel $p$, $q$, $r$,
pre ktoré je číslo $qr-k$ násobkom~$p$, číslo $pr-k$ násobkom~$q$
a~súčasne číslo $pq-k$ násobkom~$r$.}
\podpis{...}

{%%%%%   trojstretnutie, priklad 3
Vnútri tetivového štvoruholníka $ABCD$ je daný
bod~$P$ tak, že platí
$$
|\uhol BPC|=|\uhol BAP|+|\uhol PDC|.
$$
Označme $E$, $F$, $G$ päty kolmíc z~bodu $P$ postupne na priamky
$AB$, $AD$ a~$DC$. Dokážte, že trojuholník $FEG$ je podobný
s~trojuholníkom $PBC$.}
\podpis{J. Švrček}

{%%%%%   trojstretnutie, priklad 4
V~obore reálnych čísel riešte sústavu rovníc
$$
\frac{1}{xy}=\frac{x}{z}+1,\quad
\frac{1}{yz}=\frac{y}{x}+1,\quad
\frac{1}{zx}=\frac{z}{y}+1.
$$}
\podpis{J. Földes}

{%%%%%   trojstretnutie, priklad 5
Vnútri strán $AB$, $BC$, $CA$ daného trojuholníka $ABC$ sú
zvolené postupne body $K$, $L$, $M$ tak, že platí
$$
\frac{|AK|}{|KB|}=\frac{|BL|}{|LC|}=\frac{|CM|}{|MA|}.
$$
Dokážte, že trojuholníky $ABC$ a~$KLM$ majú spoločný priesečník
výšok práve vtedy, keď je trojuholník $ABC$ rovnostranný.}
\podpis{P. Černek}

{%%%%%   trojstretnutie, priklad 6
Na stole leží $k$~kôpok s~$1, 2, \dots, k$ kamienkami, pričom
$k\geqq3$. V~prvom kroku vyberieme 3~ľubovoľné kôpky na stole,
spojíme ich do jednej a~z~tejto novej kôpky odstránime 1~kamienok
(preč zo stola). V~druhom kroku opäť spojíme niektoré tri
kôpky do jednej a~potom z~nej odoberieme 2~kamienky. Všeobecne v~$i$-tom
kroku spojíme ľubovoľné tri kôpky, v~ktorých je spolu
viac ako $i$~kamienkov, do jednej kôpky a~potom z~nej $i$~kamienkov
odstránime. Predpokladajme, že po niekoľkých krokoch zostane na
stole jediná kôpka, v~ktorej je $p$~kamienkov. Dokážte, že číslo~$p$
je štvorec práve vtedy, keď obe čísla $2k+2$ a~$3k+1$ sú štvorce.
Ďalej potom nájdite najmenšie~$k$, pre ktoré je číslo~$p$ štvorec.}
\podpis{R. Kučera}

{%%%%%   IMO, priklad 1
Nech $ABC$ je ostrouhlý trojuholník, v~ktorom $|AB|\not =|AC|$.
Kružnica nad priemerom $BC$ pretína strany $AB$ a~$AC$ postupne
v~bodoch $M$ a~$N$. Označme~$O$ stred strany~$BC$. Osi uhlov $BAC$
a~$MON$ sa pretínajú v~bode~$R$. Dokážte, že kružnice opísané
trojuholníkom $BMR$ a~$CNR$ prechádzajú spoločným bodom ležiacim na
strane~$BC$.}
\podpis{Rumunsko}

{%%%%%   IMO, priklad 2
Nájdite všetky mnohočleny~$P(x)$ s~reálnymi
koeficientmi, ktoré spĺňajú rovnosť
$$
P(a-b)+P(b-c)+P(c-a)=2P(a+b+c)
$$
pre všetky reálne čísla $a$, $b$, $c$ také, že $ab+bc+ca=0$.}
\podpis{Južná Kórea}

{%%%%%   IMO, priklad 3
Nazvime {\it hák\/} útvar vytvorený zo
šiestich jednotkových štvorčekov ako na obrázku
alebo ľubovoľný útvar, ktorý vznikne jeho otočením či súmernosťou.
Určte všetky pravouholníky $m\times n$, ktoré sa dajú hákmi
pokryť tak, že
\item{$\bullet$}
   pravouholník je pokrytý bez medzier a~prekrytí;
\item{$\bullet$}
   žiadna časť háku nepokrýva plochu mimo pravouholníka.
\insp{53-imo-3}   
   }
\podpis{Estónsko}

{%%%%%   IMO, priklad 4
Nech ${n\ge 3}$ je celé číslo. Nech
$t_1,t_2,\dots,t_n$ sú kladné reálne čísla také, že
$$
n^2+1>(t_1+t_2+\cdots+t_n)\left(\frac{1}{t_1}+\frac{1}{t_2}+\cdots+\frac{1}{t_n}\right).
$$
Ukážte, že $t_i$, $t_j$, $t_k$ sú dĺžky strán trojuholníka
pre všetky $i$, $j$, $k$, kde $1\le i<j<k\le n$.}
\podpis{Južná Kórea}

{%%%%%   IMO, priklad 5
V~konvexnom štvoruholníku~$ABCD$
uhlopriečka~$BD$ nerozpoľuje ani jeden z~uhlov $ABC$, $CDA$.
Bod~$P$ leží vnútri $ABCD$ a~spĺňa rovnosti
$$
|\uhol PBC|=|\uhol DBA|\quad\text{a}\quad |\uhol PDC|=|\uhol BDA|.
$$
Dokážte, že $ABCD$ je tetivový práve vtedy, keď $|AP|=|CP|$.}
\podpis{Poľsko}

{%%%%%   IMO, priklad 6
Prirodzené číslo nazveme {\it striedavé}, ak každé dve
susedné číslice v~jeho desiatkovom zápise majú rôznu paritu.
Nájdite všetky prirodzené čísla~$n$ také, že $n$ má striedavý
násobok.}
\podpis{Irán}

