{%%%%%   A-I-1
\fontplace
\rtpoint O; \tpoint s; \rpoint t;
\tpoint3; \tpoint5; \tpoint\frac{15}2;
\tpoint9; \tpoint15;
\rpoint3; \rpoint5; \rpoint\frac{15}2;
\rBpoint9; \rpoint15;
\lBpoint s=|2t-15|; \lpoint t=|2s-15|;
[6] \hfil\Obr

Nech $t$, $s$ sú reálne korene danej kvadratickej rovnice.
Zaoberajme sa najskôr prípadom, keď uvažovaná kvadratická rovnica má
dvojnásobný (reálny) koreň. Vtedy platí $t=s$, pričom podľa
podmienok úlohy je $t=|2t-15|$. Pre $t\ge15/2$ dostávame rovnicu
$t=2t-15$ s~riešením $t=15$, pre $t<15/2$ rovnicu $t=\m(2t-15)$ 
s~riešením $t=5$. Im prislúchajúce kvadratické rovnice majú tvar
$(x-5)^2=x^2-10x+25=0$ a~$(x-15)^2=x^2-30x+225=0$.

Venujme sa ďalej prípadu, keď uvažovaná kvadratická rovnica má dva
rôzne reálne korene $t$, $s$. Rozoberieme tri prípady.

\smallskip
Ak $t=|2t-15|$ a~súčasne $s=|2s-15|$, tak riešenia oboch
rovníc (podľa predchádzajúceho) tvoria dvojicu $\{t,s\}=\{5,15\}$. Prislúchajúca
kvadratická rovnica má tvar $(x-5)(x-15)=x^2-20x+75=0$.

\smallskip
Ak $t=|2s-15|$ a~súčasne $s=|2t-15|$, tak riešením
štyroch sústav rovníc
$$
t=\pm(2s-15),\quad s=\pm(2t-15)
$$
(ktoré zodpovedajú rôznym voľbám znamienok) dostaneme dvojice $(s,t)$
rovné $(15,15)$, $(5,5)$, $(3,9)$ a~$(9,3)$. Z~nich len
posledné dve vyhovujú pôvodnej sústave a~podmienke $s\ne t$.
Dodajme, že sústavu rovníc $t=|2s-15|$ a~$s=|2t-15|$ možno
riešiť aj graficky v~rovine $Ost$, do ktorej zakreslíme obe lomené
čiary $t=|2s-15|$ a~$s=|2t-15|$ (\obr).
Dvojiciam $(3,9)$ a~$(9,3)$ prislúcha kvadratická rovnica
$(x-3)(x-9)=x^2-12x+27=0$.
\inspicture{}

\smallskip
Ak $t=|2t-15|=|2s-15|$, tak už vieme, že rovnica
$t=|2t-15|$ má riešenie $t=5$ a~$t=15$. Pre $t=5$ z~rovnice
$5=|2s-15|$ vyplýva $s=5$ alebo $s=10$, pre $t=15$ z~rovnice
$15=|2s-15|$ vyplýva $s=0$ alebo $s=15$. Vzhľadom na podmienku $s\ne
t$ tak dostávame dve riešenia $(t,s)=(5,10)$
a~$(t,s)=(15,0)$. Týmto riešeniam potom prislúchajú postupne dve
kvadratické rovnice $(x-5)(x-10)=x^2-15x+50=0$ 
a~$(x-15)x=x^2-15x=0$.

\zaver
Danej úlohe vyhovuje šesť dvojíc $(p,q)$ reálnych
čísel, a~to dvojice $(-10,25)$, $(-30,225)$, $(-20,75)$,
$(-12,27)$, $(-15,50)$ a~$(-15,0)$.}

{%%%%%   A-I-2
\epsplace a53.3 \hfil\Obr\par
\epsplace a53.4 \hfil\Obr\par
\epsplace a53.5 \hfil\Obr\par
% \epsplace a53.1 \hfil\Obr\par
% \epsplace a53.2 \hfil\Obr\par
Označme $\Cal P$ hľadanú množinu bodov a~$S$ stred štvorca $KLMN$.
Zrejme $S\in \Cal P$ (\obr).

Ďalej určíme všetky hľadané body $P$ $(P\ne S$), ktoré ležia
vnútri pásu ohraničeného rovnobežkami $KN$ a~$LM$. Ukážeme, že každý
taký bod~$P$ leží v~polrovine opačnej k~polrovine $MNK$ alebo vnútri trojuholníka $MNS$.
Pre každý bod~$P$ uvažovaného pásu, ktorý leží v~polrovine opačnej
k~polrovine $KLM$, totiž platí $|\uhol KPL|>|\uhol KPN|$,
lebo polpriamka~$PN$ leží v~uhle $KPL$. 
Ďalej pre body~$P$ vnútri trojuholníka $KSN$ zrejme platí $|\uhol NPK|>90\st>|\uhol LPM|$ a~pre 
body~$P$ vnútri trojuholníka $LMS$ zasa $|\uhol NPK|<90\st<|\uhol LPM|$.
A~konečne pre každý bod~$P$ v~trojuholníku $KLS$ (mimo jeho vrcholov) je uhol $KPL$ väčší ako $90\st$, zatiaľ čo
aspoň jeden z~uhlov $NPK$ a~$LPM$ je menší ako $90\st$ (vnútorné oblasti Tálesových kružníc nad priemermi $NK$
a~$LM$ majú prázdny prienik).
%Podobne zistíme, že
%žiadny bod štvorca $KLMN$ (s~výnimkou jeho stredu~$S$) nemá danú
%vlastnosť.
\inspicture{}

Ak teda hľadaný bod~$P$ leží vo vyšrafovanej oblasti na \obrr1,
sú priamky $PK$ a~$PL$ podľa zadania osami uhlov $NPL$ 
a~$KPM$. Preto v~trojuholníku $LPN$ os~$PK$ uhla $NPL$
pretína kružnicu opísanú tomuto trojuholníku (okrem bodu~$P$) 
v~bode ležiacom na osi strany~$NL$. Týmto bodom je však
vrchol~$K$ štvorca $KLMN$. Body $P$, $N$, $K$, $L$ teda ležia
na jednej kružnici, ktorou je kružnica opísaná štvorcu $KLMN$.
(Analogický výsledok dostaneme, keď uvažujeme os~$PL$ uhla $KPM$.)
Bod~$P$ preto leží na kratšom oblúku~$MN$ kružnice
opísanej štvorcu $KLMN$ (označme ho~$\ell$).
Naopak, pre každý bod $P\in\ell$ platí podľa vety
o~obvodových uhloch (pre zhodné tetivy $NK$, $KL$, $LM$)
$$
|\uhol NPK|=|\uhol KPL|=|\uhol LPM|=45^{\circ}.
$$
Tým je hľadanie bodov~$P$ v~páse medzi rovnobežkami $KM$ a~$LM$
ukončené.

Ďalej ľahko nahliadneme, že ľubovoľný vnútorný bod~$P$ každej
z~polpriamok opačných k~polpriamkam $KM$, $LN$, $MK$, $NL$ danú
vlastnosť má. Ukážeme, že žiadny ďalší bod roviny štvorca
$KLMN$ uvedenú vlastnosť nemá. Stačí sa pritom vďaka symetrii
zaoberať len jednou z~polrovín ohraničených osou~$o$ strany~$KL$ daného
štvorca. Pretože sme už vyšetrili celý pás ohraničený rovnobežkami
$KN$ a~$LM$, stačí (bez ujmy na všeobecnosti) skúmať len body
polroviny opačnej k~polrovine $LMN$. Priamky $KL$, $MN$, $LM$,
$KM$ a~$LN$ delia túto polrovinu na päť častí (\obr), pritom
žiadny bod priamok $KL$, $LM$ a~$MN$ danú vlastnosť očividne nemá.
\inspicture{}

Ukážeme, že žiadny vnútorný bod každej z~oblastí I~až~V roviny štvorca
$KLMN$ nie je prvkom množiny~$\Cal P$. Ak $P$ je vnútorným bodom
oblasti~I, evidentne platí $|\uhol KPL|>|\uhol LPM|$ (\obrr1).
Ak $P$ je vnútorným bodom ľubovoľnej z~oblastí II alebo~III, platí
naopak $|\uhol KPL|<|\uhol LPN|$. Pre ľubovoľný vnútorný bod
oblasti~IV zasa platí $|\uhol NPK|>|\uhol KPL|$ a~pre ľubovoľný
vnútorný bod~$P$ oblasti~V platí naopak $|\uhol NPK|<|\uhol
KPL|$. Vo všetkých piatich uvažovaných prípadoch sme sa tak vždy
dostali do sporu s~podmienkami úlohy.

Tým sme preskúmali všetky body roviny štvorca $KLMN$.

% \twopictures{}

\zaver
Hľadaná množina bodov~$P$ sa skladá zo všetkých
vnútorných bodov kratšieho oblúka~$MN$ kružnice opísanej danému
štvorcu $KLMN$, zo všetkých vnútorných bodov polpriamok opačných
k~polpriamkam $KM$, $LN$, $MK$ a~$NL$ a~zo stredu~$S$ daného
štvorca (\obr).
\inspicture{}}

{%%%%%   A-I-3
Skupinu~$P_k$ rozdeľme na dve časti
$(PA)_k$ a~$(PB)_k$ podľa toho, či slovo skupiny~$P_k$ končí
písmenom~$A$, alebo písmenom~$B$. Skupinu~$N_k$ rozdeľme analogicky na dve
časti $(NA)_k$ a~$(NB)_k$. Označme ďalej $p_k$, $n_k$,
$(pA)_k$, $(pB)_k$, $(nA)_k$, $(nB)_k$ postupne počty prvkov skupín
$P_k$, $N_k$, $(PA)_k$, $(PB)_k$, $(NA)_k$, $(NB)_k$. Pre každé
prirodzené číslo~$k$ potom podľa nášho rozdelenia platí
$$
\aligned
p_k=&(pA)_k+(pB)_k,\\
n_k=&(nA)_k+(nB)_k.
\endaligned          \tag1
$$
Každé slovo zo skupiny $(PA)_{k+1}$ vznikne tak, že pripíšeme
písmeno~$A$ buď na koniec slova zo skupiny $(PA)_k$, alebo na koniec
slova zo skupiny $(NB)_k$. Platí preto
$$
(pA)_{k+1}=(pA)_k+(nB)_k.
$$
Analogicky platia tiež vzťahy
$$
\aligned
(pB)_{k+1}=&(pA)_k+(pB)_k,\\
(nA)_{k+1}=&(pB)_k+(nA)_k,\\
(nB)_{k+1}=&(nA)_k+(nB)_k.
\endaligned          \tag2
$$
Pre $n=1$ majú skupiny tvar
$$
\postdisplaypenalty 10000
(PA)_1=\{A\}, \quad (PB)_1=\{B\}, \quad (NA)_1=\emptyset, \quad (NB)_1=\emptyset,
$$
a~teda $(pA)_1=(pB)_1=1$ a~$(nA)_1=(nB)_1=0$.

Predpokladajme, že pre nejaké prirodzené číslo~$m$ obsahujú
skupiny $(PA)_m$ a~$(PB)_m$ rovnaký počet prvkov, ktorý označíme~$q$,
a~zároveň skupiny $(NA)_m$ a~$(NB)_m$ majú rovnaký počet
prvkov, ktorý označíme~$r$. Naviac predpokladajme, že platí $q\ne
r$, ako to platí v~prípade $m=1$, keď $q=1$ a~$r=0$. Do nasledujúcej
tabuľky zapíšme počty prvkov v~skupinách pre čísla $k$ rovné $m$, $m+1$, $m+2$, $m+3$ a~$m+4$.
Pritom pre výpočty hodnôt využijeme vzťahy \thetag{1} a~\thetag{2}.
$$
\centerline{\box0}
$$
Z~tabuľky možno vyčítať niekoľko poznatkov. Pretože $q\ne r$, platí
aj $2q\ne2r$, $3q+r\ne q+3r$ a~$6q+10r\ne10q+6r$. Vidíme, že
$p_m\ne n_m$, $p_{m+1}\ne n_{m+1}$, $p_{m+2}=n_{m+2}$,
$p_{m+3}\ne n_{m+3}$ a~že skupiny $(PA)_{m+4}$ a~$(PB)_{m+4}$
obsahujú opäť rovnaký počet prvkov a~skupiny $(NA)_{m+4}$ 
a~$(NB)_{m+4}$ opäť rovnaký počet prvkov, pritom tieto počty sú
navzájom rôzne.

Použitím matematickej indukcie zdôvodníme, že uvedená tabuľka má
všetky spomenuté vlastnosti pre každé $m=4\ell+1$, kde $\ell$ je celé
nezáporné číslo, takže rovnosť $p_k=n_k$ platí práve vtedy, keď
$k=m+2=4\ell+3$.

\zaver
Skupiny $P_k$ a~$N_k$ majú rovnaký počet prvkov práve vtedy,
keď $k=4\ell+3$, kde $\ell$ je celé nezáporné číslo.}

{%%%%%   A-I-4
\def\cl#1{\sqrt{#1^2+1}}%
Pre $n=1$ má daná nerovnosť tvar
$$
\sqrt2\le\frac12(p+1),\quad\text{čiže}\quad
p\geq2\sqrt2-1.
$$
Označme $p_1=2\sqrt2-1$. Zistili sme, že žiadne číslo~$p$ menšie
ako~$p_1$ požadovanú vlastnosť nemá. Číslo~$p_1$ teda bude hľadaným
číslom, ak ukážeme, že pre každé $n\ge1$ platí
$$
\cl1+\cl2+\cl3+\cdots+\cl{n}\leq\frac12 n(n+p_1).\tag1
$$

Dôkaz urobíme matematickou indukciou.

\itemitem{$1^\circ$}
Pre $n=1$ je nerovnosť~\thetag{1} splnená vďaka spôsobu,
akým sme číslo~$p_1$ určili.

\itemitem{$2^\circ$}
 Predpokladajme, že nerovnosť~\thetag{1} platí pre určité
prirodzené číslo~$n$ a~ukážeme, že platí aj pre prirodzené číslo
$n+1$. Nech teda
$$
\aligned
F(n)&=\cl1+\cl2+\cl3+\cdots+\cl{n}\le\\
    &\leq\frac12 n(n+p_1).
\endaligned                             \tag2
$$
Pretože
$$
F(n+1)= F(n)+\cl{(n+1)},
$$
podľa indukčného predpokladu~\thetag{2} a~definície čísla~$p_1$ platí
$$
F(n+1)\leq \frac12 n\bigl(n+2\sqrt2-1\bigr)+\cl{(n+1)}.\tag3
$$
Teraz dokážeme nerovnosť
$$
\frac12
n\bigl(n+2\sqrt2-1\bigr)+\cl{(n+1)}\leq\frac12(n+1)\bigl(n+1+2\sqrt2-1\bigr).
\tag4
$$
Jej úpravou dostaneme s~ňou ekvivalentnú nerovnosť
$$
\cl{(n+1)}\leq n+\sqrt2,
$$
o~platnosti ktorej sa ľahko presvedčíme po umocnení oboch
strán na druhú:
$$
\bigl(n+\sqrt2\bigr)^2=n^2+2\sqrt2n+2>n^2+2n+2=(n+1)^2+1.
$$
Podľa \thetag{3} a~\thetag{4} platí
$$
F(n+1)\leq \frac12(n+1)\bigl(n+1+2\sqrt2-1\bigr)=\frac12(n+1)(n+1+p_1),
$$
čo je nerovnosť~\thetag{1} pre hodnotu $n+1$.

\zaver
Hľadaným reálnym číslom je číslo $p=2\sqrt2-1$.}

{%%%%%   A-I-5
\fontplace
\lBpoint S;
\tpoint A; \lBpoint B; \bpoint C; \rpoint D;
\tpoint\ X;
\lpoint k;
[7] \hfil\Obr

\fontplace
\tpoint A; \lBpoint B; \bpoint C; \rpoint D;
\rpoint D';
[9] \hfil\Obr

\fontplace
\medmuskip3mu
\tpoint A; \lBpoint B; \bpoint C; \rpoint D;
\cpoint\f; \cpoint\f; \cpoint\xy-1,0 \f;
\cpoint\xy3,0 120\st-\f; \tpoint60\st-\f;
[8] \hfil\Obr

Najskôr sa zamyslime, ako môže taký tetivový štvoruholník $ABCD$
s~šesťde\-siatstupňovým uhlom pri vrchole~$B$ a~so zhodnými
stranami $BC$ a~$CD$ vyzerať. Označme~$k$ kružnicu, ktorá je
štvoruholníku $ABCD$ opísaná. Pretože $|\uhol ABC|=60\st$, je už
určená veľkosť uhlopriečky~$AC$, ktorá je tetivou prislúchajúcou
obvodovému uhlu~60\st. Vrchol~$D$ potom musí byť vnútorným bodom
kratšieho oblúka~$AC$ kružnice~$k$ (v~polrovine opačnej k~$ACB$)
a~vrchol~$B$ je obrazom bodu~$D$ v~súmernosti podľa priamky~$SC$
(\obr), kde $S$ je stred kružnice~$k$.

\inspicture{}

Pretože podľa predpokladu platí $|BC|=|CD|$, sú obvodové uhly $BAC$
a~$CAD$ prislúchajúce zhodným tetivám zhodné. Vidíme teda, že
polpriamky $AD$ a~$AB$ sú súmerne združené podľa osi~$AC$.
Označme $X$ obraz bodu~$D$ v~tejto súmernosti (\obrr1). Bod~$X$
zrejme leží vnútri strany~$AB$ (obraz kratšieho oblúka~$AC$ leží
celý vo vnútornej oblasti kružnice~$k$), a~pretože
$|CX|=|CD|=|BC|$, je trojuholník $XBC$ rovnoramenný.
%
% je velikost vnitřního úhlu při vrcholu~$D$ rovna~$120^\circ$,
% takže $|\uh AXC|=|\uh ADC|=120\st$ a~ $|\uh BXC|=180\st-|\uh
% AXC|=60\st$, což znamená, že
%
Trojuholník $XBC$ je dokonca rovnostranný, pretože veľkosť jeho
uhla pri vrchole~$B$ je $60^{\circ}$. Preto $|BX|=|BC|=|CD|$.
Zo súmernosti naviac vyplýva $|DA|=|XA|$, takže
$|CD|+|DA|=|BX|+|XA|=|AB|$, čo je požadovaná rovnosť v~časti~a).

\smallskip
Ľahko nahliadneme, že opačná implikácia neplatí. Stačí zobrať
taký štvoruholník $ABCD$, ktorý spĺňa predpoklady úlohy, 
a~zároveň v~ňom platí $|CD|\ne|DA|$ (taký určite existuje, ako
sme naznačili hneď v~úvode riešenia). Keď vymeníme strany
$CD$ a~$DA$, \tj. nahradíme vrchol~$D$ vrcholom~$D'$
súmerne združeným s~vrcholom~$D$ podľa osi uhlopriečky~$AC$
% \inspicture{}
(\obr), dostaneme tetivový štvoruholník $ABCD'$ 
s~šesťdesiatstupňovým uhlom pri vrchole~$B$, ktorý bude aj naďalej
spĺňať rovnosť $|CD'|+|D'A|=|DA|+|CD|=|AB|$, ale bude v~ňom
platiť $|BC|=|CD|=|D'A|\ne|D'C|$.

\midinsert
\inspicture{}
\endinsert

\ineriesenie
Pripomenieme si sínusovú vetu v~nasledujúcom tvare, ktorý
vyplýva z~vety o~obvodových uhloch: Ak $R$ je polomer
kružnice opísanej trojuholníku $ABC$, tak $\sin\a=a/(2R)$, kde
$a=|BC|$. (Keď doplníme cyklicky ďalšie dve rovnosti, dostaneme
odtiaľ jednoducho bežné znenie sínusovej vety.)

Ak teraz označíme $\phi$ obvodový uhol prislúchajúci zhodným tetivám
$BC$ a~$CD$ ($0\st<\phi<60\st$), zistíme, že tetive~$DA$
prislúcha obvodový uhol $60\st-\phi$ a~tetive~$AB$ obvodový uhol
$120\st-\phi$ (\obr). Dokazovaná rovnosť je potom podľa sínusovej vety
ekvivalentná s~rovnosťou
$$
\sin\phi+\sin(60\st-\phi)=\sin(120\st-\phi).
$$
Pretože $\sin(120\st-\phi)=\sin(60\st+\phi)$, je uvedená rovnosť
(po jednoduchej úprave) ekvivalentná s~rovnosťou
$$
\sin\phi=2\cos60\st\sin\phi,
$$
ktorá triviálne platí.
\inspicture{}

Rovnako ako v~predchádzajúcom riešení si uvedomíme, že rovnosť
$|CD|+|DA|=|AB|$ ostane zachovaná, ak v~danom štvoruholníku
vymeníme strany $CD$ a~$DA$. Nový štvoruholník ostane
tetivový, veľkosť jeho vnútorného uhla pri vrchole~$B$ sa
nezmení, ale namiesto rovnosti $|BC|=|CD|$ bude splnená rovnosť
$|BC|=|DA|$.

\ineriesenie
Označme dĺžky strán štvoruholníka $ABCD$, ktorý spĺňa podmienky
úlohy, zvyčajným spôsobom $a$, $b$, $c$, $d$. Pretože vnútorné uhly
pri vrcholoch $B$ a~$D$ majú veľkosť 60\st, resp\. 120\st,
z~kosínusovej vety pre trojuholníky $ABC$ a~$CDA$ vyplýva (po porovnaní dvoch
vyjadrení hodnoty $|AC|^2$) rovnosť
$$
  a^2+b^2-ab=c^2+d^2+cd.      \tag6
$$

a) Ak $b=c$, možno z~rovnosti~\thetag{6} postupne odvodiť
$$
\postdisplaypenalty 10000
\align
  a^2+c^2-ac=&c^2+d^2+cd,\\
     a^2-d^2=&ac+cd,\\
  (a-d)(a+d)=&c(a+d),\\
         a-d=&c.
\endalign
$$
Rovnosť $a=c+d$, ktorú sme mali dokázať, teda platí.

\smallskip
b) Ak platí $a=c+d$, po dosadení za $a$
do rovnosti~\thetag{6} dostaneme
$$
(c+d)^2+b^2-(c+d)b=c^2+d^2+cd.
$$
Odtiaľ po úprave máme vzťah $(b-c)(b-d)=0$, z~ktorého vyplýva, že
platí $b=c$ alebo $b=d$. Opačná implikácia teda všeobecne neplatí.}

{%%%%%   A-I-6
Ak sú čísla $x$, $y$, $z$ riešením danej sústavy, zrejme platí
$xyz\ne0$. Vynásobme preto jednotlivé rovnice postupne činiteľmi $yz$, $zx$,
$xy$ a~v~obore nenulových reálnych čísel riešme
ekvivalentnú sústavu rovníc
$$
x^2yz=y+z,\quad xy^2z=x+z,\quad xyz^2=x+y. \tag1
$$
Súčtom ľavých a~pravých strán tejto sústavy rovníc získame po
úprave rovnicu
$$
(xyz-2)(x+y+z)=0.
$$
Odtiaľ vidíme, že platí $xyz=2$ alebo $x+y+z=0$. Rozoberme tieto dva prípady osobitne.

\smallskip
Nech $xyz=2$. Po dosadení za súčin $xyz$ v~sústave~\thetag{1}
dostaneme
$$
2x=y+z,\quad 2y=x+z,\quad 2z=x+y,
$$
čo je ekvivalentné so sústavou
$$
3x=x+y+z,\quad 3y=x+y+z,\quad 3z=x+y+z.
$$
Odtiaľ vyplýva $x=y=z$. Vzhľadom na podmienku $xyz=2$ dostaneme
$x=y=z=\root3\of{2}$. Skúškou overíme, že trojica
$(\root3\of{2},\root3\of{2}, \root3\of{2})$ je skutočne
riešením sústavy~\thetag{1}, a~teda aj pôvodnej sústavy rovníc.

\smallskip
Nech $x+y+z=0$. Z~prvej rovnice sústavy~\thetag{1} vyplýva $x^2yz=-x$,
odkiaľ vzhľadom na podmienku $x\ne0$ dostaneme $xyz=-1$. Overme,
že každá trojica nenulových reálnych čísel $(x,y,z)$ spĺňajúca
sústavu dvoch rovníc
$$
x+y+z=0,\quad xyz=-1   \tag2
$$
je riešením pôvodnej sústavy. Z~rovností~\thetag{2} totiž vyplýva
$$
\postdisplaypenalty 10000
\frac{1}{y}+\frac{1}{z}=\frac{y+z}{yz}=\frac{-x}{-1/x}=x^2
$$
(vzhľadom na symetriu zadanej sústavy stačilo overiť jednu rovnicu).

Sústava rovníc~\thetag{2} má v~obore nenulových reálnych čísel
nekonečne veľa riešení, ktoré získame napríklad tak, že
jednu premennú (napr.~$z$) zvolíme ako parameter. Tým dostaneme
sústavu
$$
x+y=-z,\quad xy=-\frac{1}{z}.
$$
Po dosadení za $x$ z~prvej rovnice do druhej dostaneme
$$
(y+z)y=\frac{1}{z},
$$
teda
$$
y^2+yz-\frac{1}{z}=0.  \tag3
$$
Jedná sa o~kvadratickú rovnicu s~neznámou~$y$ a~parametrom~$z$.
Jej diskriminant je rovný $D=z^2+4/z$. Nutnou a~postačujúcou
podmienkou pre to, aby táto rovnica mala reálne korene,
je nerovnosť $D\geq0$. Vyriešením nerovnice $(z^3+4)/z\ge0$
dostaneme pre parameter~$z$ podmienku
$$
z\in\bigl(-\infty,-\root3\of{4}\bigr\rangle\cup(0,\infty).\tag4
$$
Za podmienky~\thetag{4} má kvadratická rovnica~\thetag{3} korene
$$
y_1=\frac{-z+\sqrt{z^2+4/z}}{2}
\quad\text{a}\quad
y_2=\frac{-z-\sqrt{z^2+4/z}}{2},
$$
ktorým podľa vzťahu $x=-y-z$ zodpovedajú hodnoty
$$
x_1=\frac{-z-\sqrt{z^2+4/z}}{2}
\quad\text{a}\quad
x_2=\frac{-z+\sqrt{z^2+4/z}}{2}.
$$
Pritom $(x_1,y_1)=(x_2,y_2)$ platí iba v~prípade $z=\m\root3\of4$.

\zaver
Daná sústava má riešenie $x=y=z=\root3\of2$.
Všetky ostatné riešenia sú trojice $(x,y,z)$ tvaru
$$
(x,y,z)=\biggl(\frac{-z\pm\sqrt{z^2+4/z}}{2},
              \frac{-z\mp\sqrt{z^2+4/z}}{2},z\biggr),
$$
kde $z$ je ľubovoľné číslo spĺňajúce podmienku~\thetag{4}.}

{%%%%%   B-I-1
Označme $x$ a~$y$ číslice, ktoré doplníme do čitateľa, resp\.
menovateľa prvého zlomku. Pretože celý výraz v~absolútnej
hodnote budeme algebraicky upravovať, kvôli prehľadnejším zápisom
zavedieme označenie $N=111\,111\,111\,110$. Jednotlivé čísla
z~daného výrazu potom majú vyjadrenia
$$
\align
777\,777\,777\,77x&=7N+x,\\
777\,777\,777\,77y&=7N+y,\\
555\,555\,555\,554&=5N+4,\\
555\,555\,555\,559&=5N+9.\\
\endalign
$$
Skúmaný výraz teda možno zapísať a~upraviť nasledujúcim
spôsobom:
$$
\align
&\left|\frac{7N+x}{7N+y}-\frac{5N+4}{5N+9}\right|=
\frac{|(7N+x)(5N+9)-(5N+4)(7N+y)|}{(7N+y)(5N+9)}=\\
&=\frac{|(35N^2+5xN+63N+9x)-(35N^2+5yN+28N+4y)|}{(7N+y)(5N+9)}=\\
&=\frac{|5\cdot(7-y+x)\cdot N+9x-4y|}{(7N+y)(5N+9)}.
\endalign
$$
Označme ešte čitateľa a~menovateľa získaného zlomku
$$
C=|5\cdot(7-y+x)\cdot N+9x-4y|\qquad\text{a}\qquad
J=(7N+y)(5N+9).
$$
Ak budeme za $x$, $y$ dosadzovať rôzne dvojice číslic,
menovateľ~$J$ bude nadobúdať iba desať rôznych hodnôt v~rozmedzí
$$
(7N+0)(5N+9)\leqq J\leqq (7N+9)(5N+9).
$$
Pozrime sa teraz, aké hodnoty bude
nadobúdať
čitateľ~$C$. Pretože číslo $9x-4y$ je najviac
dvojciferné, zatiaľ čo číslo~$N$ dvanásťciferné,
rád čitateľa~$C$
bude závisieť na tom, či bude činiteľ $(7-y+x)$ rovný nule alebo
nie. Preto tieto dve možnosti rozoberieme osobitne.

\smallskip
{\it Prípad $7-y+x=0$}. Vtedy platí $y=x+7$ a~skúmaný
čitateľ~$C$
má tvar
$$
C=|5\cdot0\cdot N+(9x-4y)|=|9x-4(x+7)|=|5x-28|.
$$
Keďže číslica~$y$ (rovná $x+7$) je najviac~9,
je číslica~$x$ rovná 0, 1 alebo 2,
takže výraz $|5x-28|$ sa rovná 28, 23 alebo 18.
{\it Najmenšia\/}
hodnota čitateľa~$C$ je preto rovná~18 a~dostaneme
ju jedine pre $x=2$ a~$y=9$. Šťastnou "zhodou okolností"
má práve pre $y=9$ menovateľ~$J$ {\it najväčšiu\/}  hodnotu,
takže
$$
\min\left\{\frac{C}{J}\right\}=\frac{18}{(7N+9)(5N+9)}.
$$

\smallskip
{\it Prípad $7-y+x\ne0$}. Ukážme, že hodnoty čitateľa~$C$
(teda aj hodnoty zlomku $C/J$) sú v~tomto prípade
"obrovské"
v~porovnaní s~prvým prípadom. Z~nerovnosti $7-y+x\ne0$ vyplýva
odhad $|7-y+x|\geqq1$ (číslo $7-y+x$ je celé),
teda máme
$$
C=|5\cdot(7-y+x)\cdot N+9x-4y|\geqq
5\cdot|7-y+x|\cdot N-|9x-4y|\geqq
5N-|9x-4y|.
$$
Pretože $x$ a~$y$ sú číslice, platí zrejme
$|9x-4y|\leqq81$.
Z~ostatného odhadu~$C$ a~maximálnej hodnoty~$J$ preto vyplýva
nerovnosť
$$
\frac{C}{J}\geqq\frac{5N-81}{(7N+9)(5N+9)}.
$$
Ostatný zlomok je "mnohonásobne" väčší ako zlomok v~závere
prvého prípadu, lebo oba zlomky majú rovnaký menovateľ, zatiaľ čo pre
čitatele zrejme platí $5N-81\gggtr18$ (nerovnosť
$5N-81>18$ platí už od hodnoty $N=20$).

\zaver
Do čitateľa doplníme číslicu $x=2$, do menovateľa
číslicu $y=9$.}

{%%%%%   B-I-2
\fontplace
\rpoint A; \lpoint B; \bpoint C; \bpoint\xy1,0 D;
\ltpoint K; \tpoint L;
\everymath{\ssize}%
\cpoint\xy.3,0 \ssize36\st; \cpoint\xy.2,0 \ssize36\st; \cpoint\xy-1,0 36\st;
\cpoint 72\st; \cpoint 72\st;
[1] \hfil\Obr

\fontplace
\rpoint A; \lpoint B; \bpoint C; \bpoint\xy1,0 D;
\ltpoint\xy-1,0 K;
\brpoint\xy1,0 X;
\everymath{\ssize}%
\cpoint 36\st; \cpoint 36\st; \cpoint 36\st;
\cpoint 36\st; \cpoint\xy.5,0 36\st;
\cpoint 72\st;
[2] \hfil\Obr

V~rovnoramennom trojuholníku $AKD$ poznáme uhol $DAK$ oproti
základni~$KD$. Môžeme dopočítať zvyšné dva uhly pri základni
(\obr): $|\uhol ADK|=|\uhol AKD|=(180^\circ-|\uhol
DAK|)/2=72^\circ$. Štvoruholník $AKCD$ má protiľahlé strany $AK$
a~$CD$ zhodné a~rovnobežné, takže je to rovnobežník, preto
priamky $KC$ a~$AD$ sú rovnobežné. Uhly $DAK$ a~$CKB$ sú teda
súhlasné a~uhly $CKB$ a~$KCD$ striedavé. Preto $|\uhol CKB|=|\uhol
KCD|=36^\circ$. Uhol $DKC$ je doplnkom
uhlov $AKD$ a~$CKB$ do priameho uhla, platí teda $|\uhol
DKC|=180^\circ-36^\circ-72^\circ=72^\circ$.
\inspicture{}

Na polpriamke opačnej k~polpriamke~$KD$ zvoľme bod~$L$ tak, že
$|KL|=|AD|$. Potom $|\uhol LKB|=|\uhol AKD|=72^\circ$
a~$|\uhol CKL|=|\uhol LKB|+|\uhol CKB|=108^\circ$.
Dopočítaním uhlov v~lichobežníku $ABCD$ dostávame $|\uhol
BCD|=(360^\circ-2\cdot 36^\circ)/2=144^\circ$ a~môžeme
vyjadriť veľkosť uhla $BCK$: $|\uhol BCK|=|\uhol BCD|-|\uhol
KCD|=144^\circ-36^\circ=108^\circ$. Teraz už vieme, že $|KL|=|CB|$
a~$|\uhol LKC|=|\uhol KCB|$, čo znamená, že $LBCK$ je rovnoramenný
lichobežník, možno mu teda opísať kružnicu (zhodnú s~kružnicou
opísanou trojuholníku $KBC$). Ďalej môžeme z~lichobežníka $LBCK$
dopočítať $|\uhol KLB|=(360^\circ - {2\cdot
108^\circ})/2=72^\circ=|\uhol KDA|$. Z~tejto rovnosti vyplýva, že
$AD\parallel BL$, takže trojuholníky $ADK$ a~$BLK$ sú
navzájom rovnoľahlé podľa stredu~$K$. Rovnoľahlé sú potom
aj kružnice im opísané. Pretože obe prechádzajú stredom~$K$
spomenutej rovnoľahlosti, majú v~tomto bode vonkajší dotyk.

\ineriesenie
Rovnako ako v~prvom riešení zistíme, že $|AKD|=72^\circ$.
Štvoruholník $AKCD$ je rovnobežník (\obr),
\inspicture{}
takže $|CK|=|AD|$. Z~rovnosti $|CK|=|BC|$ v~trojuholníku $KBC$ vyplýva,
že $|\uhol CKB|=|\uhol KBC|=36^\circ$. Preto na základni~$CD$
existuje bod~$X$ taký, že $|\uhol AKX|=108^\circ$ (a~$|\uhol
BKX|=72^\circ$). Potom $|\uhol DKX|=|\uhol AKX|-|\uhol
AKD|=108^\circ-72^\circ=36^\circ$, a~teda $|\uhol DKX|=|\uhol DAK|$.
Takže uhol~$DKX$ je úsekovým uhlom prislúchajúcim oblúku~$DAK$
v~kružnici opísanej trojuholníku $AKD$, čo znamená, že priamka~$KX$
je jej dotyčnicou. Podobne $|\uhol CKX|=|\uhol BKX|- |\uhol
BKC|=72^\circ-36^\circ=36^\circ=|\uhol KBC|$, takže $KX$ je
dotyčnicou aj ku kružnici opísanej trojuholníku $KBC$. Kružnice opísané
trojuholníkom $AKD$ a~$KBC$ majú teda spoločnú dotyčnicu~$KX$
prechádzajúcu spoločným bodom~$K$. Obe kružnice sa preto v~tomto
bode dotýkajú.}

{%%%%%   B-I-3
Označme $k=\lfloor x\rfloor$, teda $x=k+\al$, $0\le\al<1$. Daná
rovnica má potom tvar $(k+\al)k-5(k+\al)+7=0$. Odtiaľ
$\al=(k^2-5k+7)/(5-k)$. Hľadáme teda celé čísla~$k$, pre
ktoré platí
$$
0\le\frac{k^2-5k+7}{5-k}< 1.       \tag{1}
$$
Každú z~týchto nerovností vyšetríme osobitne. Pretože kvadratický trojčlen
$k^2-5k+7$ má záporný diskriminant, platí $k^2-5k+7\ge0$ pre každé reálne číslo~$k$.
Takže ľavá nerovnosť v~\thetag{1} platí práve vtedy, keď
$5-k>0$, čiže $k<5$. Vyriešme pravú nerovnicu:
$$
\eqalign{
\frac{k^2-5k+7}{5-k}      &<1,\cr
\frac{k^2-5k+7-(5-k)}{5-k}&<0,\cr
\frac{k^2-4k+2}{5-k}      &<0,\cr
\frac{(k-2-\sqrt{2})(k-2+\sqrt{2})}{5-k}&<0.\cr
}
$$
Podľa polohy čísel $2-\sqrt{2}$, $2+\sqrt{2}$ a~$5$ na číselnej osi
zistíme, že ostatná nerovnosť platí práve vtedy, keď
$k$ patrí do množiny $(2-\sqrt{2},2+\sqrt{2})\cup(5,\infty)$.
Nerovnosti~\thetag{1} teda platia súčasne práve vtedy, keď $k$ leží v~intervale
$(2-\sqrt{2},2+\sqrt{2})$. Tejto podmienke vyhovujú iba
tri celé čísla 1, 2 a~3. Pre $k=1$ dopočítame
$\al=3/4$, pre $k=2$ vyjde $\al=1/3$ a~pre $k=3$ je
$\al=1/2$. Celkom dostávame tri riešenia $x_1=7/4$,
$x_2=7/3$ a~$x_3=7/2$.

\ineriesenie
Označíme $k=\lfloor x\rfloor$ ako v~prvom riešení 
a~z~rovnice $kx-5x+7=0$ vyjadríme $x$ v~tvare $x=7/(5-k)$. Teraz
hľadáme celé čísla~$k$, pre ktoré platí $k\le7/(5-k)<k+1$.
Obe nerovnosti sú splnené jedine pre celé čísla 1, 2 a~3, ktorým
zodpovedajú riešenia $7/4$, $7/3$ a~$7/2$.}

{%%%%%   B-I-4
Prirodzené číslo je deliteľné číslom~$24$ práve vtedy, keď je deliteľné
súčasne (navzájom nesúdeliteľnými) číslami $3$ a~$8$.
Pre ciferný súčet prirodzeného čísla~$k$ zaveďme označenie~$S(k)$.
Číslo~$a_n$ je deliteľné tromi práve vtedy, keď je tromi
deliteľný jeho ciferný súčet, teda číslo
$S(1)+S(2)+\cdots+S(n)$. Zvyšok po delení tromi tohto súčtu
závisí iba na zvyškoch (po delení tromi) jednotlivých sčítancov~$S(k)$.
Pretože po delení tromi dáva číslo~$S(k)$ rovnaký zvyšok
ako číslo~$k$, dávajú čísla $S(1),
S(4),S(7),\dots$ zvyšok~$1$, čísla $S(2),S(5),
S(8),\dots$ zvyšok~$2$ a~čísla $S(3),S(6),S(9),\dots$
zvyšok~$0$. Preto napríklad číslo $S(a_{14})$, teda súčet
$S(1)+S(2)+\cdots+S(14)$, dáva po delení tromi rovnaký zvyšok
ako súčet
$$
(1+2+0)+(1+2+0)+(1+2+0)+(1+2+0)+1+2.
$$
Podľa uzátvorkovaných trojíc ľahko vidíme, že tento súčet je
deliteľný tromi. Pretože všeobecne súčet $S(3i-2)+S(3i-1)+S(3i)$ je
deliteľný tromi pre každé prirodzené~$i$, môžeme podobným spôsobom
uzátvorkovať každý súčet
$$
S(1)+S(2)+\cdots+S(n)
$$
a~zistiť, že jeho zvyšok po delení tromi
je rovný~1, ak $n=3k-2$ a~je rovný~0, ak $n=3k-1$ alebo $n=3k$.

Čísla~$a_n$ teda budú deliteľné tromi práve vtedy, keď $n$ bude tvaru
$3k$ alebo $3k-1$ ($k=1,2,\dots$).

Teraz rozoberme, kedy budú čísla~$a_n$ naviac deliteľné ôsmimi.
Prirodzené číslo je deliteľné ôsmimi práve vtedy, keď je deliteľné ôsmimi
posledné trojčíslie jeho zápisu v~desiatkovej sústave. Naše úvahy
budú závisieť na počte číslic čísla~$n$.

\smallskip
Pre aspoň trojciferné čísla~$n$ je $a_n$
deliteľné ôsmimi práve vtedy, keď je deliteľné ôsmimi číslo~$n$.
Pretože sa zvyšky čísel~$a_n$ po delení tromi opakujú po troch
a~zvyšky po delení ôsmimi po ôsmich číslach~$a_n$, budú sa zvyšky po
delení číslom~24 opakovať po najmenšom spoločnom násobku týchto
periód, teda po dvadsiatich štyroch. Pre trojciferné~$n$ ľahko
zistíme, že podmienke v~úlohe vyhovujú čísla tvaru $104+24k$ 
a~$120+24k$ ($n$~musí byť deliteľné ôsmimi a~dávať zvyšok dva alebo
nula po delení tromi). Do 10\,000 máme 413~čísel tvaru $104+24k$
($413=\lfloor(10\,000-104)/24\rfloor +1$) a~412~čísel
tvaru $120+24k$.

\smallskip
Aby pri dvojcifernom~$n$ bolo číslo~$a_n$ deliteľné ôsmimi, musí byť
deliteľné aj štyrmi. O~deliteľnosti štyrmi rozhoduje posledné
dvojčíslie, takže štyrmi budú deliteľné práve všetky tie~$a_n$,
pre ktoré je $n$ deliteľné štyrmi. Číslo $n-1$ je potom nepárne, teda
aj $a_{n-1}$ je číslo nepárne a~číslo $100a_{n-1}$ dáva zvyšok
štyri po delení ôsmimi. Potom číslo $a_n=100a_{n-1}+n$ bude
deliteľné ôsmimi práve vtedy, keď $n$ bude tiež dávať zvyšok štyri po
delení ôsmimi, bude teda tvaru $8k+4$. Spolu s~podmienkou na
deliteľnosť tromi dostávame, že vyhovujúce dvojciferné čísla~$n$
majú (rovnako ako vyššie) periódu~$24$ a~sú tvaru $n=12+24k$
a~$n=20+24k$, $k\in\{0,1,2,3\}$. Po sto tak máme spolu $4+4=8$~čísel.

\smallskip
Pri jednocifernom~$n$ ľahko zistíme, že zo všetkých
párnych čísel~$a_n$ vyhovuje iba $a_6=123\,456$.

\smallskip
Celkom vyhovuje 834 čísel.}

{%%%%%   B-I-5
\fontplace
\rBpoint A; \lBpoint B; \bpoint\xy1.4,0 C; \bpoint D;
\rpoint\xy-.5,-.5 E; \rBpoint\xy1,0 K; \lpoint L;
\rpoint p;
[3] \hfil\Obr

\fontplace
\lbpoint A; \tpoint B_1; \tpoint B_2;
\tpoint\down.5mm C; \lpoint D_1; \rpoint D_2;
\rpoint m_1; \lpoint m_2;
\lBpoint n_1; \rBpoint n_2;
\rpoint p;
[4] \hfil\Obr

Predpokladajme, že $ABCD$ je hľadaný lichobežník a~$K$, $L$ sú
päty kolmíc z~vrcholov~$B$, $D$ na priamku~$AC$ (\obr).
Z~podobnosti pravouhlých trojuholníkov $BKE$ a~$DLE$ vyplýva, že
dĺžky strán $BK$ a~$DL$, \tj. odvesien v~spomenutých
trojuholníkoch, sú v~rovnakom pomere ako dĺžky ich
prepôn $BE$ a~$DE$, teda $3:1$. $BK$ a~$DL$ sú však aj výškami
v~trojuholníkoch $ABC$ a~$ACD$, a~to na spoločnú stranu~$AC$.
Obsahy týchto trojuholníkov sú teda tiež v~pomere $3:1$, takže
obsah lichobežníka $ABCD$ je rovný~$4P/3$, kde $P$ je obsah
rovnoramenného trojuholníka $ABC$. Výška tohto trojuholníka
z~bodu~$A$ na stranu~$BC$ je daná (vzdialenosť bodu~$A$ od
priamky~$p$). Obsah trojuholníka $ABC$ bude teda minimálny, keď bude
minimálna dĺžka strany~$BC$, a~teda aj $AC$, \tj. keď úsečka~$AC$
bude kolmá na~$p$.
\inspicture{}

\smallskip
{\it Konštrukcia\/}. Najskôr zostrojíme bod~$C$ (päta kolmice z~$A$
na~$p$). Vrchol~$B$ nájdeme ako priesečník priamky~$p$
s~kružnicou $k(C,|AC|)$ (dve možnosti). Vrchol~$D$ je priesečníkom
priamky~$m$ vedenej bodom~$C$ rovnobežne s~$AB$ a~priamky~$n$
rovnobežnej s~$AC$ vo vzdialenosti $4|BC|/3$ od vrcholu~$B$
vnútri polroviny opačnej k~$ACB$.

\zaver
Úloha má dve riešenia súmerne združené podľa
priamky~$AC\perp p$ (\obr).
\inspicture{}}

{%%%%%   B-I-6
Najskôr ukážme, že pre číslo~$M$ s~prvočíselným rozkladom
$M=\prod_{i=1}^np_i^{c_i}$ ($p_i$~sú rôzne prvočísla) je počet
riešení rovnice $\NSN(x,y)=M$ rovný
${\prod_{i=1}^n (2c_i+1)}$. Naozaj, každé riešenie $(x,y)$
danej rovnice má tú vlastnosť, že ľubovoľné prvočíslo~$p_i$
($i=1,\dots,n$) delí aspoň jedno z~čísel $x$ a~$y$ (a~to
najviac s~takým exponentom, s~akým delí~$M$) a~žiadne iné prvočísla
už ani $x$, ani $y$ nedelia. Čísla $x$ a~$y$ teda majú tvar
$$
x=\prod_{i=1}^np_i^{a_i}, \quad y=\prod_{i=1}^np_i^{b_i}, \quad a_i,b_i\in \Bbb N_0,
\quad \text{a~naviac} \quad \max(a_i,b_i)=c_i,\quad i=1,\dots,n.
$$
Čísla $x$ a~$y$ tak jednoznačne určujú $n$-tice
čísel $a_i$ a~$b_i$ a~naopak, sú nimi jednoznačne určené.
Všetky riešenia danej rovnice sú teda popísané dvojicami $n$-tíc
prirodzených čísel takých, že  na $i$-tej pozícii je v~oboch
$n$-ticiach číslo z~množiny $\{0,\dots, c_i\}$ a~aspoň v~jednej
z~nich sa priamo rovná~$c_i$. Takých $n$-tíc je $\prod_{i=1}^n
(2c_i+1)$, nakoľko dve \hbox{$n$-tice} čísel $(a_1,a_2,\dots,a_n)$
a~$(b_1,b_2,\dots,b_n)$ môžeme považovať za $n$~dvojíc čísel
$(a_1,b_1)$, $(a_2,b_2)$,~\dots, $(a_n,b_n)$. Ľubovoľná dvojica
$(a_i,b_i)$ môže nezávisle nadobúdať $(2c_i+1)$ rôznych hodnôt
$(0,c_i)$, $(1,c_i)$,~\dots, $(c_i-1,c_i)$, $(c_i,c_i)$,
$(c_i,c_i-1)$,~\dots, $(c_i,1)$, $(c_i,0)$. Podľa
kombinatorického pravidla súčinu dostávame vyššie uvedený počet.

Prvočíselný rozklad čísla $1\,001$ je $7\cdot11\cdot13$. Aby mala
daná rovnica práve $1\,001$ riešení, musia exponenty~$c_i$
z~prvočíselného rozkladu čísla~$M$ (obsahujúceho podľa zadania
najmenej tri prvočísla, a~to 2, 3 a~5) spĺňať rovnosť
$\prod_{i=1}^n (2c_i+1)=7\cdot11\cdot13$. V~prvočíselnom rozklade
čísla~$M$ teda musia byť zastúpené práve tri prvočísla, a~to
s~exponentmi $(7-1)/2=3$, $(11-1)/2=5$
a~$(13-1)/2=6$. Pretože $M$ má byť deliteľné číslom
$240=2^4\cdot 3\cdot 5$, teda prvočíslami $2$, $3$ a~$5$
so~zodpovedajúcimi exponentmi, sú jediné možné voľby pre $M$ čísla
$2^5\cdot 3^3\cdot5^6$, $2^5\cdot 3^6\cdot5^3$, $2^6\cdot
3^5\cdot5^3$, $2^6\cdot 3^3\cdot5^5$.}

{%%%%%   C-I-1
Keď budeme premýšľať nad postupmi rezania šachovníc
veľkých rozmerov, určite si uvedomíme, že na obdĺžniky
$3\times1$ možno rozrezať každý "pás" šachovnice tvorený tromi
susednými riadkami alebo stĺpcami. Také pásy sa preto oplatí od
šachovnice opakovane odrezávať (pokiaľ je to možné), a~tak
zmenšovať jej rozmery o~násobky troch. Preto bude pre našu úlohu
o~šachovnici $n\times n$ výhodné rozlíšiť, či dané číslo $n>3$ dáva
po delení tromi zvyšok~1, alebo zvyšok~2 (zvyšok~0 je zadaním
vylúčený). Každý z~týchto prípadov preskúmame osobitne.

\smallskip
{\it Prípad $n=3k+1$}. Najskôr zo šachovnice $(3k+1)\times(3k+1)$
odrežeme pás prvých $3k$~stĺpcov, teda obdĺžnik $(3k+1)\times3k$,
ktorý potom rozrežeme (po trojiciach stĺpcov) na $k$~pásov
$(3k+1)\times3$ a~každý z~nich nakoniec rozrežeme na $3k+1$
obdĺžnikov $1\times3$. Z~pôvodnej šachovnice nám tak zostane nerozrezaný
posledný stĺpec. Pretože má $3k+1$~políčok, ľahko ho rozrežeme na
jeden štvorec $1\times1$ a~$k$~obdĺžnikov $3\times1$. Na \obr{} je
znázornené výsledné rozrezanie šachovnice $7\times 7$ (počiatočné
odrezanie pásu $7\times6$ je vyznačené šípkami, zvyšný stĺpec je
sivý). Na tom istom obrázku vidíme aj spôsob rozrezania pre $n=4$.
\insp{C53.1}%
%\inspicture{}

\smallskip
{\it Prípad $n=3k+2$}. Keby sme šachovnicu $(3k+2)\times(3k+2)$
dôsledne "orezávali" postupom z~úvodu riešenia, dostali by sme (po
oddelení dvoch pásov $(3k+2)\times3k$ a~$3k\times2$) ako zvyšok
šachovnicu $2\times2$, ktorú však nie je možné rozrezať požadovaným
spôsobom (na diely $1\times1$ a~$3\times1$). To je možné urobiť až
s~"nasledujúcou" šachovnicou $5\times5$, ako vidíme na \obr.
%\twocpictures

$$
\vbox{\halign{# \qquad\qquad&\qquad\qquad # \cr
\hfil\lower-.173cm\hbox{\epsfbox{C53.2}}\hfil  & \hfil\hbox{\epsfbox{C53.3}}\hfil \cr 
\hfil\Obr\hfil  & \hfil\Obr\hfil \cr}}
$$

Ostáva popísať, ako každú väčšiu šachovnicu $(3k+2)\times(3k+2)$
rezaním zredukovať na štvorec $5\times5$.
Najskôr oddelíme pás $(3k+2)\times(3k-3)$ tvorený prvými $(k-1)$
trojicami stĺpcov šachovnice. Zo zvyšnej šachovnice $(3k+2)\times5$
potom oddelíme pás $(3k-3)\times5$ tvorený jej poslednými $(k-1)$
trojicami riadkov, z~pôvodnej šachovnice tak zostane žiadaný
štvorec $5\times5$ v~pravom hornom rohu
(sivý na \obr{} pre šachovnicu $8\times8$).

\smallskip
Dodajme, že pri riešení danej úlohy sme nebrali do úvahy ofarbenie
políčok šachovnice. Farby políčok sa používajú v~iných situáciách,
hlavne vtedy, keď potrebujeme dokázať, že rozrezanie šachovnice
na diely predpísaného tvaru nie je možné.}

{%%%%%   C-I-2
\fontplace
\tpoint A; \tpoint B; \lbpoint C; \rpoint D;
\lpoint K; \lBpoint L;
\rbpoint P; \bpoint\xy-1,-.8 Q;
[5] \hfil\Obr

Trojuholníky $ABP$ a~$KCP$ majú podľa zadania rovnaké obsahy.
Keď ku každému z~nich pripojíme trojuholník $ACP$ (\obr), nahliadneme, že
rovnaké obsahy majú aj trojuholníky $ABC$ a~$AKC$. Pretože oba tieto
trojuholníky majú spoločnú stranu~$AC$, obe k~nej prislúchajúce výšky musia byť
zhodné. Body $B$ a~$K$ teda majú rovnakú vzdialenosť od priamky~$AC$
(a~ležia v~rovnakej polrovine touto priamkou určenou). To
znamená, že $BK\parallel AC$. Podľa Tálesovej vety však platí
$BK\perp CK$, takže platí aj $AC\perp CK$.
\inspicture{}

Podobne z~rovnosti obsahov trojuholníkov $AQD$, $CLQ$ a~kolmosti priamok
$CL$ a~$DL$ odvodíme, že $AC\perp CL$. Spolu to znamená, že
uhol $KCL$ je zložený z~dvoch pravých uhlov $ACK$ 
a~$ACL$. Body $K$ a~$L$ teda ležia na priamke, ktorá prechádza
bodom~$C$ kolmo na uhlopriečku~$AC$.}

{%%%%%   C-I-3
\input graphicx
Pretože $Y$ je trojciferné číslo, päťciferné číslo so zápisom~$XY$
je číslo $1\,000X+Y$. Žiak teda počítal príklad $(1\,000X+Y):Z$ 
a~podľa zadania mu v~porovnaní s~pôvodným príkladom vyšiel
sedemkrát väčší výsledok, teda
$$
\frac{1\,000X+Y}{Z}=7\cdot\frac{X\cdot Y}{Z}.
$$
Odtiaľ po násobení číslom~$Z$ dostaneme rovnicu $1\,000X+Y=7XY$,
ktorú vyriešime vzhľadom na neznámu $Y$:
$$
Y=\frac{1\,000X}{7X-1}.
$$
Pre ktoré $X$ je ostatný zlomok celočíselný? Inak povedané,
kedy je číslo $1\,000X$ deliteľné číslom $7X-1$? Pretože čísla $X$ 
a~$7X-1$ sú nesúdeliteľné (nesúdeliteľné sú totiž dve po sebe idúce
čísla $7X-1$ a~$7X$), hľadáme tie~$X$, pre ktoré číslo $7X-1$
delí číslo 1\,000. Aby sme nemuseli vypisovať všetky
delitele čísla 1\,000, uvedomíme si, že $X$ je dvojciferné, teda
$69\leqq7X-1\leqq692$. Rozložme preto číslo 1\,000 všetkými spôsobmi na súčin
dvoch činiteľov tak, aby jeden (povedzme prvý) z~činiteľov
bol z~intervalu $\langle69,692\rangle$:
$$
1\,000=500\cdot2=250\cdot4=200\cdot5=125\cdot8=100\cdot10.
$$
Z~rovníc
$$
7X-1=500,\ 7X-1=250,\ 7X-1=200,\ 7X-1=125,\ 7X-1=100
$$
má jedine rovnica $7X-1=125$ celočíselné riešenie
$X=18$, pre ktoré vychádza $Y=1\,000X/(7X-1)=1\,000\cdot18/125=144$.

Teraz určíme neznáme číslo~$Z$. Využijeme na to podmienku zo zadania,
že hodnota výrazu $X\cdot Y:Z$ je prirodzené číslo. Pretože $X=18$
a~$Y=144$, jedná sa o~číslo $18\cdot144:Z$, teda číslo
$2^5\cdot3^4:Z$. Také číslo je celé práve vtedy, keď má číslo~$Z$
rozklad na prvočinitele tvaru $2^a3^b$, kde $0\le a\leqq5$
a~$0\le b\leqq4$. Exponenty $a$, $b$ nájdeme podľa podmienky zo zadania,
že číslo $Z=2^a3^b$ je trojciferné a~na mieste jednotiek má
číslicu~2. Pretože $3^4=81$ a~$2^5\cdot3=96$, musí byť $a\geqq1$
a~$b\geqq2$. Všetky čísla $2^a3^b$, kde $a\in\{1,2,3,4,5\}$
a~$b\in\{2,3,4\}$ teraz vypíšeme do tabuľky.
\def\clap#1{\hbox to 0pt{\hss#1\hss}}
\def\sikma{\smash{\clap{\kern-0.3mm\rotatebox{-27}{\rule[19.13mm]{.92cm}{.4pt}}}}}
$$
\sikma
\vbox{\offinterlineskip \let\\=\cr
\def\ab{\raise\ht\strutbox\hbox{}%\special{em: point 1}}%
        \,\lower2pt\hbox{$b$}\quad
        \raise.5ex\hbox{$a$}\,\lower\dp\strutbox\hbox{}}%\special{em: point 2}}}
\halign{\vrule\strut\hss#\vrule
        &\enspace\hss$#$\enspace&\hss$#$\enspace&\hss$#$\enspace&\hss$#$\enspace&\hss$#$\enspace\vrule\\
\noalign{\hrule}
\ab   &1   &2    &3    &4     &5\\
\noalign{\hrule}
2  &18  &36   &72   &144    &288\\
3  &54  &108  &216  &432    &864\\
4  &162 &324  &648  &1\,296 &2\,592\cr
\noalign{\hrule}
}}%\special{em: line 1,2}
$$
Z~vypočítaných čísel majú požadovanú vlastnosť iba čísla
$Z=432=2^43^3$ a~$Z=162=2^13^4$.

\odpoved
Úloha má dve riešenia. Žiak mal počítať buď príklad
$18\cdot144:432$, alebo príklad $18\cdot144:162$.

\ineriesenie
Tak, ako v~prvom riešení, odvodíme vyjadrenie
$$
Y=\frac{1\,000X}{7X-1}.
$$
Teraz však získaný zlomok upravíme čiastočným vydelením čísla
1\,000 číslom~7. Na základe rovnosti
$1\,000=7\cdot143-1$ dostávame
$$
Y=\frac{1\,000X}{7X-1}=\frac{143(7X-1)+143-X}{7X-1}=143+
\frac{143-X}{7X-1}.
$$
Aby bolo $Y$ celé, musí byť ostatný zlomok $(143-X)/(7X-1)$
celočíselný. Pretože číslo~$X$ je dvojciferné, náš zlomok
spĺňa odhady
$$
\frac{143-99}{7\cdot99-1}<\frac{143-X}{7X-1}<
\frac{143-10}{7\cdot10-1}.
$$
Ľavý zlomok je rovný $44/692$, pravý je rovný $133/69$, takže
jediná možná celočíselná hodnota prostredného zlomku je rovná~1.
Musí teda platiť $Y=144$. Rovnica
$$
\frac{143-X}{7X-1}=1
$$
má potom jediné riešenie $X=18$. Ďalej už postupujeme ako v~prvom
riešení.

\ineriesenie
Skôr získanú rovnicu $1\,000X+Y=7XY$
upravíme na súčinový tvar $Y=X\cdot(7Y-1\,000)$. Musí preto
platiť $7Y-1\,000>0$, odkiaľ
$$
Y>\frac{1\,000}{7}>142,\quad\text{čiže}\quad
Y\geqq143.
$$
Číslo~$X$ je dvojciferné, preto z~rovnosti $Y=X\cdot(7Y-1\,000)$
vychádza odhad
$$
Y\geqq10\cdot(7Y-1\,000),\quad\text{čiže}\quad
Y\leqq\frac{10\,000}{69}<145.
$$
Spolu dostávame, že číslo~$Y$ je rovné jednému z~čísel $143$
alebo $144$. Rovnica $143=X\cdot(7\cdot143-1\,000)$ má riešenie
$X=143$, čo však nie je dvojciferné číslo. Rovnica
$144=X\cdot(7\cdot144-1\,000)$ má riešenie $X=18$. Tak sme znovu
ukázali, že $X=18$ a~$Y=144$. Číslo~$Z$ určíme ako v~prvom
riešení.}

{%%%%%   C-I-4
\fontplace
\tpoint A; \tpoint B; \bpoint C;
\tpoint K; \lpoint L; \rpoint M;
\lbpoint\xy-1,.1 P; \cpoint\a;
[6] \hfil\Obr

Označme $\a=|\uhol BAP|$, $0\st<\a<60\st$ (\obr).
\inspicture{}
Pretože uhly $BAP$ a~$BAK$ sú súmerne združené podľa osi~$AB$,
platí tiež $|\uhol BAK|=\a$. Pretože $|\uhol CAP|=|\uhol
CAB|-|\uhol BAP|=60\st-\a$, zo súmernosti podľa osi~$CA$ vyplýva
rovnosť $|\uhol CAM|=60\st-\a$. Pre veľkosť uhla $KAM$ teda
platí
$$
|\uhol KAM|=|\uhol BAK|+|\uhol BAC|+|\uhol CAM|=
\a+60\st+(60\st-\a)=120\st.
$$
Zo súmerností podľa osí $AB$ a~$CA$ vyplývajú tiež rovnosti
$|AK|=|AP|=|AM|$. Preto je trojuholník $KAM$ rovnoramenný a~jeho uhol
pri hlavnom vrchole~$A$ má veľkosť $120\st$. Podobne sa
zdôvodní, že aj trojuholníky $LBK$ a~$MCL$ sú rovnoramenné a~ich
vnútorné uhly pri hlavných vrcholoch $B$ a~$C$ majú veľkosť
$120\st$.

Pri posudzovaní podmienky, či trojuholník $KLM$ je rovnoramenný, musíme
rozlíšiť, ktoré z~jeho strán $KL$, $LM$, $MK$ sú zhodné.
Vzhľadom na symetriu rozoberieme podrobne iba prípad, keď
$|KL|=|MK|$. Z~podobných rovnoramenných trojuholníkov $KAM$ a~$LBK$
vyplýva, že ich základne $MK$ a~$KL$ sú zhodné práve vtedy, keď
sú zhodné ich ramená $AK$ a~$BK$. Zapíšme to pomocou dĺžok
úsečiek: rovnosť $|KL|=|MK|$ platí práve vtedy, keď platí rovnosť
$|AK|=|BK|$, čiže rovnosť $|AP|=|BP|$. Ostatná rovnosť však
nastane práve vtedy, keď bod~$P$ leží na osi strany~$AB$. Podobne sa
zistia podmienky ekvivalentné rovnostiam $|MK|=|LM|$ a~$|KL|=|LM|$.

\odpoved
Trojuholník $KLM$ je rovnoramenný práve vtedy, keď
bod~$P$ leží na aspoň jednej z~osí strán daného rovnostranného
trojuholníka $ABC$. Hľadaná množina je preto zjednotením troch úsečiek --
výšok trojuholníka $ABC$ (bez ich krajných bodov).}

{%%%%%   C-I-5
Na príklade čísla $1\,413$ vidíme, že niekedy nie je ľahké spoznať,
či dané trojmiestne alebo štvormiestne číslo je magické.
Pozrime sa preto najskôr, ako sa magické číslo~$x$ vyjadrí
pomocou číslic tých trojmiestnych čísel $\overline{abc}$ a~$\overline{cba}$,
ktorých je súčtom:
$$
x=\overline{abc}+\overline{cba}=(100a+10b+c)+(100c+10b+a)=101(a+c)+20b.
$$
Vidíme, že číslo~$x$ je určené číslicami $a$, $b$, $c$ tak, že
závisí len na $b$ a~na súčte $a+c$. Znamená to, že rôzne trojice
číslic $a$, $b$, $c$ môžu určovať to isté magické číslo~$x$
(nemyslíme tým iba trojice líšiace sa vzájomnou výmenou
číslic $a$ a~$c$).
Ak napr\. $a+c=14$ a~$b=9$, nájdeme tri rôzne vyjadrenia magického
čísla $1\,594$:
$$
1\,594=599+995=698+896=797+797.
$$
Existujú ešte iné "magické" vyjadrenia čísla $1\,594$? Všetko
závisí od toho, či sú rovnicou
$
1\,594=101s+20b
$
hodnoty súčtu číslic $s=a+c$ a~číslice $b$ jednoznačne určené.
Z~rovnice ihneď vidíme, že číslo~$s$ končí číslicou~$4$, takže $s=4$
alebo $s=14$ (iné hodnoty súčtu $s=a+c$ nie sú číslicami $a$, $c$
dosiahnuteľné). Kým hodnote $s=14$ zodpovedá (ako dobre vieme)
hodnota $b=9$, pre $s=4$ dostaneme rovnicu $1\,594=404+20b$, ktorá
nemá celočíselné riešenie.

Poučení uvedeným príkladom sa pokúsime stanoviť počet magických
čísel ako počet čísel tvaru
$
x=101s+20b,
$
kde číslo~$s$ (rovné súčtu číslic $a$ a~$c$, ktoré sú
{\it nenulové\/}) prebieha množinu $\{2,3,4,\dots,18\}$, zatiaľ čo
číslica~$b$ prebieha (nezávisle od súčtu~$s$) množinu
$\{0,1,2,\dots,9\}$. Pretože číslo~$s$ nadobúda celkom
17~rôznych hodnôt a~číslo~$b$ celkom 10~rôznych hodnôt, je počet
všetkých dvojíc $(s,b)$, ktoré môžeme do vzťahu $x=101s+20b$
dosadiť, rovný číslu $17\cdot10=170$. Ak teraz ukážeme, že po
dosadení ľubovoľných dvoch rôznych dvojíc $(s_1,b_1)$ a~$(s_2,b_2)$
dostaneme dve rôzne magické čísla
$$
x_1=101s_1+20b_1\quad\text{a}\quad x_2=101s_2+20b_2,
$$
bude to znamenať, že počet všetkých hodnôt~$x$ (teda {\it počet\/}
všetkých magických čísel) je tiež rovný číslu~170.

Pripusťme, že pre niektoré dvojice $(s_1,b_1)$ a~$(s_2,b_2)$ platí
$x_1=x_2$. Rovnosť
$
101s_1+20b_1=101s_2+20b_2
$
upravíme na tvar
$101(s_1-s_2)=20(b_2 -b_1)$, z~ktorého vzhľadom na nesúdeliteľnosť
čísel 20 a~101 vyplýva, že číslo $b_2 -b_1$ je násobkom
čísla~101. Musí sa pritom jednať o~nulový násobok, lebo
$|b_2-b_1|\leqq 9$ ($b_1$ a~$b_2$ sú číslice). Platí teda $b_2
-b_1=0$, takže tiež $s_1-s_2=0$, čo spolu znamená, že
dvojice $(s_1,b_1)$ a~$(s_2,b_2)$ sú rovnaké. Len v~tomto
prípade je teda rovnosť $x_1=x_2$ možná.

\smallskip
{\it Súčet\/} všetkých magických čísel (teda čísel tvaru $x=101s+20b$)
výhodne určíme, keď čísla najskôr usporiadame do obdĺžnikovej schémy
(podľa rovnakých hodnôt~$s$ do riadkov a~podľa rovnakých hodnôt~$b$
do stĺpcov)
$$
\medmuskip 3mu
\matrix
101\cdot2 +20\cdot0&101\cdot2 +20\cdot1&101\cdot2 +20\cdot2&\dots&101\cdot2 +20\cdot9\\
101\cdot3 +20\cdot0&101\cdot3 +20\cdot1&101\cdot3 +20\cdot2&\dots&101\cdot3 +20\cdot9\\
101\cdot4 +20\cdot0&101\cdot4 +20\cdot1&101\cdot4 +20\cdot2&\dots&101\cdot4 +20\cdot9\\
\vdots&\vdots&\vdots&\ddots&\vdots\\
101\cdot17+20\cdot0&101\cdot17+20\cdot1&101\cdot17+20\cdot2&\dots&101\cdot17+20\cdot9\\
101\cdot18+20\cdot0&101\cdot18+20\cdot1&101\cdot18+20\cdot2&\dots&101\cdot18+20\cdot9
\endmatrix
$$
a~potom čísla sčítame buď po stĺpcoch, alebo po riadkoch. Rozhodnime
sa pre sčítanie po stĺpcoch, pričom budeme brať do úvahy, o~koľko
sa čísla uvažovaného stĺpca líšia od príslušných čísel prvého
stĺpca. Súčet čísel v~prvom stĺpci je
$$
101\cdot(2+3+\cdots+18)=101\cdot170,
$$
v~druhom stĺpci je súčet $101\cdot170+17\cdot20\cdot1$,
v~treťom
$101\cdot170+17\cdot20\cdot2$, atď., až v~poslednom (desiatom)
stĺpci je súčet čísel rovný $101\cdot170+17\cdot20\cdot9$.
Súčet všetkých magických čísel je teda rovný
$$
10\cdot101\cdot170+17\cdot20\cdot(1+2+\cdots+9)=187\,000.
$$}

{%%%%%   C-I-6
\font\zapf=uzdr scaled 2000

\fontplace
\rpoint A; \tpoint B; \lpoint C; \bpoint D;
\rBpoint D';
\lBpoint m; \rBpoint m;
\lpoint v_D; \lbpoint v_{D'};
\lBpoint k; \bpoint t; \lpoint v_{D'}<v_D;
[7] \hfil\Obr

\fontplace
\tpoint A; \tpoint B; \bpoint C; \bpoint D;
\tpoint S;
\bpoint m; \tpoint m; \bpoint o;
\rBpoint k;
[8]

\fontplace
\tpoint A; \tpoint B; \lpoint C'; \bpoint D;
\tpoint S;
\bpoint m; \rpoint m; \bpoint o;
\rBpoint k;
[9]

\fontplace
\lpoint S;
\tpoint m; \tpoint m;
[10]

\fontplace
\tpoint S;
\lpoint m; \rpoint m;
[11]

\fontplace
\lpoint S;
\bpoint m; \rpoint m;
\tpoint m; \lpoint m;
[12]

V~celom riešení budeme predpokladať, že dané dĺžky $m$ a~$r$
spĺňajú nerovnosť $m<2r$, inak žiadny štvoruholník požadovaných
vlastností neexistuje. Strany dĺžky~$m$ každého takého
štvoruholníka sú totiž tetivami kružnice s~polomerom~$r$
a~najviac jedna z~nich môže byť jej priemerom.

Skúmané štvoruholníky rozdelíme do dvoch skupín podľa toho, či
sú ich strany danej dĺžky~$m$ susedné, alebo protiľahlé.

\smallskip
Ľubovoľný štvoruholník z~prvej skupiny označíme $ABCD$ tak, aby
platilo $|AB|=|BC|=m$. Uhlopriečka rozdelí tento tetivový
štvoruholník na dva trojuholníky $ABC$ a~$ACD$ (\obr), pritom je
\inspicture{}
jasné, že prvý z~nich, trojuholník $ABC$, je polo\-merom~$r$
opísanej kružnice~$k$ a~dĺžkou~$m$ dvoch jeho strán určený (až na
zhodnosť) jednoznačne, takže má pevne určený obsah. Preto bude
obsah takého štvoruholníka $ABCD$ maximálny práve vtedy, keď bude
maximálny obsah trojuholníka $ACD$. Tento trojuholník má určenú
dĺžku strany~$AC$, takže jeho obsah bude maximálny práve vtedy, keď
bude maximálna jeho výška~$v_D$ z~vrcholu~$D$. Pri pevnej polohe
trojuholníka $ABC$ bod~$D$ prebieha ten oblúk~$AC$ kružnice~$k$,
ktorý neobsahuje bod~$B$, takže výška~$v_D$ je zrejme najväčšia
práve vtedy, keď bod~$D$ je stredom tohto oblúka, leží teda (rovnako
ako bod~$B$) na osi úsečky~$AC$. (Tvrdenie zdôvodníme pomocou
dotyčnice~$t$ ku kružnici~$k$, ktorá prechádza nájdeným bodom~$D$
rovnobežne s~priamkou~$AC$, \obrr1). Tak prichádzame k~záveru, že
v~prvej skupine má maximálny obsah ten štvoruholník, ktorý je
deltoid (ak $m\ne r\sqrt2$), respektíve štvorec (ak $m=r\sqrt2$).

\smallskip
Prejdime teraz k~štvoruholníkom druhej skupiny. Ľubovoľný z~nich
označme $ABCD$ tak, aby platilo $|AB|=|CD|=m$ (\obr).

\bigskip
% \midinsert
\centerline{$\vcenter{\hbox{\inspicture-!}}$\hss
            $\vcenter{\hbox{\zapf\char222 }}$\hss
            $\vcenter{\hbox{\inspicture-!}}$\qquad}
\centerline\Obr
% \endinsert
\bigskip

Obrázok ukazuje, ako k~takému štvoruholníku $ABCD$ zostrojiť
pomocný štvoruholník $ABC'D$, ktorý má rovnaký obsah ako $ABCD$,
je vpísaný do tej istej kružnice~$k$ a~má susedné strany $AB$ a~$BC'$
danej dĺžky~$m$. Konštrukciu teraz popíšeme a~spomenuté vlastnosti
štvoruholníka $ABC'D$ podrobne zdôvodníme. Bod~$C'$ zostrojíme
ako obraz bodu~$C$ v~súmernosti podľa osi~$o$ úsečky~$BD$.
Pretože kružnica~$k$ je súmerná podľa osi každej svojej tetivy,
platí $C'\in k$. Trojuholníky $BCD$ a~$DC'B$ sú súmerne
združené podľa osi~$o$, takže majú rovnaký obsah, preto rovnaký
obsah majú aj štvoruholníky $ABCD$ a~$ABC'D$. Zo spomenutej
súmernosti rovnako vyplývajú rovnosti $|CD|=|BC'|$ a~$|BC|=|DC'|$,
takže štvoruholníky $ABCD$ a~$ABC'D$ sa líšia iba "vymenením"
dvoch susedných strán. Tým sú potrebné vlastnosti štvoruholníka
$ABC'D$ zdôvodnené. Ako už vieme z~predchádzajúceho odstavca,
štvoruholník $ABC'D$ má najväčší možný obsah práve vtedy, keď platí
rovnosť $|C'D|=|AD|$, ktorú môžeme prepísať ako rovnosť
$|BC|=|AD|$. Tá nastane práve vtedy, keď je štvoruholník $ABCD$
rovnobežník (lebo od začiatku predpokladáme, že $|AB|=|CD|$).
Každý rovnobežník vpísaný do kružnice je ale pravouholník (súčet
protiľahlých vnútorných uhlov tetivového štvoruholníka je $180\st$,
také uhly sú ale v~prípade rovnobežníka zhodné, a~teda
pravé). Zhrňme výsledok tohto odstavca. V~druhej skupine
štvoruholníkov má maximálny obsah ten štvoruholník, ktorý je
obdĺžnik (ak $m\ne r\sqrt2$), respektíve štvorec (ak $m=r\sqrt2$.)

\zaver
Hľadané štvoruholníky s~maximálnym obsahom
tvoria v~prípade $m<2r$, $m\ne r\sqrt2$, dve skupiny: skupinu
zhodných deltoidov a~skupinu zhodných obdĺžnikov. V~prípade
$m=r\sqrt2$ sú všetky hľadané štvoruholníky zhodné štvorce
(\obr). (V~prípade $m\geqq2r$ je množina uvažovaných štvoruholníkov
prázdna).

\midinsert
\centerline{\inspicture-!\hss\inspicture-!\hss\inspicture-!}
\centerline\Obr
\endinsert}

{%%%%%   A-S-1
\def\PP{P(x)\cdot P\Bigl(\frac1x\Bigr)}%
Pretože $P(x)$ je kvadratický trojčlen s~nezápornými
koeficientmi, je nutne $a>0$.

Nech $x$ je ľubovoľné kladné reálne číslo a~$n$ je číslo prirodzené.
Pretože
$$
0\leq\left(\sqrt{x^n}-\frac1{\sqrt{x^n}}\right)^{\!\!2}=
x^n+\frac1{x^n}-2,
$$
platí
$$
{x^n}+\frac1{x^n}\geq2.\tag1
$$
Pritom rovnosť nastáva práve vtedy, keď
$\sqrt{x^n}=1/{\sqrt{x^n}}$, \tj. keď $x=1$.

Pretože čísla $ab$, $bc$ a~$ca$ sú podľa
predpokladov úlohy nezáporné, použitím nerovnosti~\thetag{1}
ďalej dostaneme
$$
\eqalign{
P(x)&\cdot P\Bigl(\frac1x\Bigr)
 =(ax^2+bx+c)\Bigl(a\frac1{x^2}+b\frac1x+c\Bigr)=\cr
 &=a^2+b^2+c^2+ab\Bigl(x+\frac1x\Bigr)+bc\Bigl(x+\frac1x\Bigr)
        +ca\Bigl(x^2+\frac1{x^2}\Bigr)\geq\cr
 &\geq a^2+b^2+c^2+2ab+2bc+2ca=(a+b+c)^2=\bigl(P(1)\bigr)^2.\vphantom{\PP}\cr}
$$
Rovnosť nastáva práve vtedy, keď $x=1$, alebo $ab=bc=ca=0$, čo
vzhľadom na podmienku $a>0$ dáva $b=c=0$.

Pre ľubovoľné kladné reálne číslo~$x$ teda platí
$$
\PP\geq\bigl(P(1)\bigr)^2,
$$
pričom rovnosť nastáva práve vtedy, keď $x=1$ alebo $b=c=0$.

\poznamka
Úlohu možno vyriešiť aj použitím Cauchyho nerovnosti:
$$
\align
P(x)&\cdot P\Bigl(\frac1x\Bigr)
=(ax^2+bx+c)\Bigl(a\frac1{x^2}+b\frac1x+c\Bigr)=\\
   =&\left(\bigl(\sqrt ax\bigr)^2+
           \bigl(\sqrt{bx}\bigr)^2+
           \bigl(\sqrt c\bigr)^2\right)
     \left(\left(\frac{\sqrt a}x\right)^{\!\!2}+
           \biggl(\sqrt{\frac bx}\biggr)^{\!\!2}+
           \bigl(\sqrt{c}\bigr)^2\right)\geq\\
    \ge&\left(\sqrt ax\cdot\frac{\sqrt a}x+\sqrt{bx}\cdot\sqrt{\frac bx}
              +\sqrt c\cdot\sqrt c\right)^{\!\!2}=
   (a+b+c)^2=\bigl(P(1)\bigr)^2.
\endalign
$$}

{%%%%%   A-S-2
\epsplace a53.10 \hfil\Obr

\fontplace
\trpoint A; \tlpoint B;
\bpoint\xy1,0 D; \bpoint D^*;
\lBpoint K; \lBpoint K^*;
\cpoint\omega; \cpoint\omega; \cpoint\frac12\omega;
[17] \hfil\Obr

Nech $ABCDE$ je ľubovoľný konvexný päťuholník s~uvažovanými
vlastnosťami. Označme $P$, $R$ postupne stredy strán $AD$, $BD$
trojuholníka $ABD$ (\obr). Potom platí
$$
|PR|=\frac12|AB|,\qquad |CR|=\frac12|BD|,\qquad |PE|=\frac12|AD|,
\tag1
$$
pretože $PR$ je stredná priečka trojuholníka $ABD$ a~v~pravouhlom
trojuholníku je stred prepony zároveň stredom jeho opísanej kružnice
(Tálesova veta).
\inspicture{}

Z~trojuholníkovej nerovnosti je zrejmé, že pre dĺžku uhlopriečky~$CE$ platí
$$
|CE|\leq |CR|+|RP|+|PE|=s,
$$
pričom dĺžka~$s$ lomenej čiary $CRPE$ je podľa~\thetag{1} zároveň rovná
polovici obvodu trojuholníka $ABD$.

Ďalej skúmajme, kedy bude mať trojuholník $ABD$ daných vlastností
($|AB|=6\cm$, $|\uhol ADB|=120^\circ$) najväčší obvod.
Ak označíme $\alpha$ a~$\beta$ (\obrr1) veľkosti vnútorných uhlov
pri vrcholoch $A$ a~$B$ trojuholníka $ABD$ ($\alpha+\beta=60^{\circ}$),
dostaneme zo sínusovej vety v~trojuholníku~$ABD$
$$
|BD|=|AB|\frac{\sin\alpha}{\sin 120^{\circ}},\quad
|AD|=|AB|\frac{\sin\beta}{\sin 120^{\circ}}.
$$
Sčítaním oboch predchádzajúcich rovností vyjde
$$
\aligned
|AD|+|BD|=&|AB|\frac{\sin\alpha+\sin\beta}{\sin 120^{\circ}}=\\
   =&2|AB|\frac{\sin 30^{\circ}}{\sin 120^{\circ}}
   \cos\frac{\alpha-\beta}2\leq 2|AB|\frac{\sqrt{3}}3,
\endaligned
$$
pričom rovnosť v~ostatnej nerovnosti nastáva práve vtedy, keď
$\cos(\alpha/2-\beta/2)=1$, \tj. pre $\alpha=\beta=30^{\circ}$.
Trojuholník $ABD$ má teda najväčší obvod práve vtedy, keď je
rovnoramenný a~jeho uhly pri základni~$AB$ majú veľkosť
$30^{\circ}$. Vzhľadom na to, že $|AB|=6$\,cm, platí pre
ľubovoľný päťuholník $ABCDE$ požadovaných vlastností
$$
\aligned
|CE|\leq s=&\frac12(|AB|+|AD|+|BD|)\le
    \frac12|AB|\left(1+2\frac{\sqrt{3}}3\right)=\\
          =&\bigl(3+2\sqrt{3}\bigr)\cm.
\endaligned
$$

Pritom pre uvažovaný päťuholník $ABCDE$ v~situácii, keď
trojuholník $ABD$ je rovnoramenný a~vrcholy $C$, $E$ ležia na
priamke~$RP$, platí $|CE|=(3+2\sqrt{3})$\,cm.

Najväčšia dĺžka uhlopriečky~$CE$ päťuholníka $ABCDE$ vyhovujúceho
podmienkam úlohy je teda $(3+2\sqrt{3})$\,cm.

\poznamka
V~druhej časti riešenia sme (pre konkrétnu hodnotu $\omega=120\st$)
ukázali, že trojuholník $ABD$ s~danou stranou~$AB$ a~daným uhlom~$\omega$
pri vrchole~$D$ má najväčší obvod práve vtedy, keď je
rovnoramenný so základňou~$AB$. To vyplýva aj z~nasledujúcej úvahy.

Bod~$D$ prebieha oblúk, z~ktorého je úsečku~$AB$ vidno pod uhlom~$\omega$.
Na polpriamke opačnej k~$DA$ (\obr) zostrojme bod~$K$
tak, aby $|DB|=|DK|$. Z~rovnoramenného trojuholníka $BDK$ vyplýva, že
$|\uhol AKB|=\omega/2$. Bod~$K$ preto leží na oblúku, z~ktorého
je úsečku~$AB$ vidno pod uhlom $\omega/2$. Dĺžka
$|AK|=|AD|+|BD|$ bude teda najväčšia práve vtedy, keď bude úsečka~$AK$
priemerom~$AK^*$ spomenutého oblúka. Vtedy je bod~$D$ stredom~$D^*$
príslušnej kružnice, takže platí $|AD^*|=|BD^*|=|D^*K^*|$.
\inspicture{}}

{%%%%%   A-S-3
Odčítaním prvej rovnice danej sústavy od druhej dostaneme rovnicu
$$
y^2-x^2+2zx-2yz=6(z+x-2)-6(y+z-2),
$$
ktorú upravíme na tvar
$$
(x-y)(x+y-2z+6)=0.
$$
Podobne odčítaním prvej rovnice sústavy od tretej dostaneme
$$
(x-z)(x+z-2y+6)=0.
$$
Daná sústava je preto ekvivalentná so sústavou rovníc
$$
\eqalign{
 x^2+2yz-6(y+z-2)=0,\cr
 (x-y)(x+y-2z+6)=0,\cr
 (x-z)(x+z-2y+6)=0.\cr}
 \tag2
$$

Vzhľadom na druhú a~tretiu rovnicu tejto sústavy stačí rozobrať
štyri prípady.

\smallskip
Nech $x-y=0$ a~súčasne $x-z=0$. Potom $x=y=z$ a~dosadením
za $y$ a~$z$ do prvej rovnice sústavy~\thetag{2} dostaneme
rovnicu
$$
3x^2-12x+12=0,
$$
ktorá má dvojnásobný reálny koreň $x=2$. Preto trojica
$(x,y,z)=(2,2,2)$ je v~tomto prípade jediným riešením danej sústavy.

\smallskip
Nech $x-y=0$ a~súčasne $x+z-2y+6=0$.
Potom $y=x$ a~$z=x-6$. Dosadením do prvej rovnice sústavy~\thetag{2} dostaneme po
úprave rovnicu
$$
3x^2-24x+48=0,
$$
ktorá má dvojnásobný reálny koreň $x=4$. Preto trojica
$(x,y,z)=(4,4,-2)$ je v~tomto prípade jediným riešením danej sústavy.

\smallskip
Nech $x+y-2z+6=0$ a~súčasne $x-z=0$. Podobne ako
v~predchádzajúcom prípade dostaneme jediné riešenie $(x,y,z)=(4,-2,4)$.

\smallskip
Nech $x+y-2z+6=0$ a~súčasne $x+z-2y+6=0$.
Odčítaním druhej rovnice od prvej dostaneme, že $3y-3z=0$, teda
$y=z$. Z~prvého predpokladu tak máme $y=x+6$. Dosadením do prvej
rovnice sústavy~\thetag{2} dostaneme po úprave rovnicu
$$
3x^2+12x+12=0,
$$
ktorá má dvojnásobný reálny koreň $x={-2}$. Preto trojica
$(x,y,z)=(-2,4,4)$ je v~tomto prípade jediným riešením danej sústavy.

\smallskip
Daná sústava má v~obore reálnych čísel štyri riešenia $(x,y,z)$.
Sú nimi trojice $(2,2,2)$, $(4,4,{-2})$, $(4,{-2},4)$
a~$({-2},4,4)$.

\poznamka
Keď si všimneme, že sčítaním všetkých troch rovníc
danej sústavy dostaneme po úprave
$$
(x+y+z-6)^2=0,
$$
tak napríklad z~podmienky $z+x-2y+6=0$ priamo vyplýva $y=4$, čo
predchádzajúce úvahy zjednoduší.}

{%%%%%   A-II-1
Každý päťmiestny palindróm~$p$ sa dá zapísať v~tvare
$p=\overline{abcba}$, kde $a$, $b$, $c$ sú číslice
v~desiatkovej sústave, $a\ne0$. Z~vyjadrenia
$$
p=10\,001a+1\,010b+100c=37(270a+27b+3c)+11(a+b-c)
$$
vyplýva, že $p$ je deliteľné číslom 37 práve vtedy, keď je číslom 37
deliteľné číslo $a+b-c$. Vzhľadom na to, že $a$, $b$, $c$ sú
číslice ($a\ne 0$), platí $-8\leq a+b-c\leq 18$. Preto je číslo
$a+b-c$ deliteľné 37 práve vtedy, keď $a+b-c=0$, čiže $c=a+b$.
Číslice $a$, $b$ teda musia spĺňať podmienku $a+b\le9$.

Ku každému $a\in\{1,2,3,\dots, 9\}$ možno číslicu~$b$ zvoliť $10-a$
spôsobmi tak, aby platilo $a+b\leq 9$ ($b\in\{0,1,2,\dots,9-a\}$).
Číslica~$c$ je potom určená jednoznačne ako súčet $a+b$.
Palindrómov s~číslicou $a=1$ je preto~9, palindrómov s~číslicou $a=2$
je~8, atď., až napokon pre číslicu $a=9$ existuje práve jeden
palindróm.

\smallskip
Počet všetkých päťmiestnych palindrómov, ktoré sú deliteľné
číslom~37, je teda
$$
9+8+7+\cdots+1=45.
$$}

{%%%%%   A-II-2
\input colordvi
Počet vyhovujúcich slov dĺžky $n\geq2$, ktoré končia dvojicami písmen
$AA$, $AB$, $BA$ označme postupne $(aa)_n$, $(ab)_n$, $(ba)_n$;
počet vyhovujúcich slov dĺžky $n\geq1$, ktoré končia písmenom~$A$,
resp.~$B$, označme $a_n$, resp.~$b_n$. Pre všetky prirodzené čísla
$n\geq2$ platí
$$
\eqalign{
a_n&=(aa)_n+(ba)_n,\cr
b_n&=(ab)_n,\cr
p_n&=a_n+b_n=(aa)_n+(ba)_n+(ab)_n.\cr}
$$

Existujú práve dve vyhovujúce slová dĺžky jedna, a~to slová~$A$
a~$B$, a~práve tri vyhovujúce slová dĺžky dva, a~to slová~$AA$,
$AB$, $BA$. Preto $a_1=b_1=1$, $p_1=2$, $(aa)_2=(ab)_2=(ba)_2=1$,
$a_2=2$, $b_2=1$, $p_2=3$.

Každé vyhovujúce slovo dĺžky $n\geq3$, ktoré končí dvojicou
písmen~$AA$, dostaneme tak, že pripíšeme písmeno~$A$ na koniec
slova dĺžky $n-1$ končiaceho dvojicou~$BA$. Preto platí
$$
\align
(aa)_n&=(ba)_{n-1}.\cr
\noalign{\vskip\belowdisplayskip}
\intext{Analogicky zistíme, že pre každé $n\ge3$ platia tiež
vzťahy}
\noalign{\vskip\abovedisplayskip}
 (ba)_n&=(ab)_{n-1},\cr
 (ab)_n&=(aa)_{n-1}+(ba)_{n-1}.
\endalign
$$
Pretože nás zaujíma iba parita prirodzeného čísla~$p_n$ a~výrazov,
pomocou ktorých ho počítame, môžeme na základe uvedených rovností
zostaviť tabuľku zo symbolov $P$ a~$N$, ktorým zodpovedajú párne
resp\. nepárne čísla.
\def\ramik{\smash{\clap{\Color{.25 .19 .2 0}{\kern3mm\rule[-.97cm]{5mm}{1.263cm}}}}}
\def\ramikk{\smash{\clap{\Color{.25 .19 .2 0}{\kern9.5mm\rule[-.125cm]{10.5mm}{.421cm}}}}}
$$
\vbox{\offinterlineskip
      \def\d{\omit\hfill\quad\dots}
      \dimen1=-2\ht\strutbox\advance\dimen1 by-3\dp\strutbox
  %    \def\sq{\psframe*[linecolor=lightgray](-.05,-\dp\strutbox)(1,\ht\strutbox)}
  %    \def\sqq{\psframe*[linecolor=lightgray](-.1,\ht\strutbox)(.4,\dimen1)}
    \halign{\vrule\strut\hfil$#$\space\vrule&&\hfil\quad$#$\cr
       n&1\,&2\,&3\,&4\,&5\,&6\,&7\,&8\,&9\,&\d\cr
       \noalign{\hrule}
       (aa)_n& &\ramik N&N&N&P&P&N&P&\ramik N&\d\cr
       (ba)_n& &N&N&P&P&N&P&N&N&\d\cr
       (ab)_n& &N&P&P&N&P&N&N&N&\d\cr
       \noalign{\hrule}
       a_n   &N&P&P&N&P&N&N&N&P&\d\cr
       b_n   &N&N&P&P&N&P&N&N&N&\d\cr
       \noalign{\hrule}
       p_n   &P&N&P&N&N&N&\ramikk P&P&N&\d\cr}
   }
$$
Táto tabuľka je nutne periodická, pretože existuje iba osem
rôznych usporiadaných trojíc písmen~$P$ a~$N$, takže najviac po
ôsmich stĺpcoch sa vzhľadom na dokázanú rekurenciu začnú hodnoty
postupností $((aa)_n)$, $((ba)_n)$, $((ab)_n)$ opakovať. Hodnoty
postupností $(a_n)$, $(b_n)$, $(p_n)$ sú z~nich odvodené,
takže sa začnú opakovať tiež. Z~tabuľky vidíme, že jej
perióda je~7 (prvé dva zhodné stĺpce sú pre $n=2$ a~$n=9$).
%
% Matematickou indukcí snadno ukážeme, že taková tabulka je
% periodická s~periodou~7, proto jsou
%
% Zároveň tak i~vidíme, že
A~pretože v~príslušnom úseku tabuľky je dvojica susedných párnych
čísel~$p_7$, $p_8$ jediná, sú obe čísla $p_n$ a~$p_{n+1}$ párne
práve vtedy, keď je číslo~$n$ deliteľné siedmimi.

\poznamka
Z~vyššie uvedených vzťahov môžeme odvodiť
rekurentné rovnice pre čísla~$a_n$ a~$b_n$. Pre všetky prirodzené
čísla $n\geq4$ platí
$$
\aligned
a_n&=(aa)_n+(ba)_n=(ba)_{n-1}+(ab)_{n-1}=\\
   &=(ab)_{n-2}+(ab)_{n-1}=b_{n-2}+b_{n-1},\cr
b_n&=(ab)_n=(aa)_{n-1}+(ba)_{n-1}=a_{n-1}.
\endaligned
\tag1
$$
Tieto rovnice môžeme odvodiť aj nasledujúcou úvahou. Vyhovujúce
slovo končiace písmenom~$A$ má koncovku $BA$ alebo $BAA$, počet
slov prvého typu je $b_{n-1}$, slov druhého typu je $b_{n-2}$.
Vyhovujúce slovo končiace písmenom~$B$ má nutne koncovku~$AB$
a~týchto slov je $a_{n-1}$.

Zo vzťahov uvedených v~\thetag{1} možno odvodiť
rekurentnú rovnicu priamo pre čísla~$p_n$. Pre každé $n\geq4$ totiž platí
$$
\eqalign{
a_n&=b_{n-1}+b_{n-2}=a_{n-2}+a_{n-3},\cr
 b_n&=a_{n-1}=b_{n-2}+b_{n-3}.\cr}
$$
Vzhľadom na to, že $p_n=a_n+b_n$, dostaneme sčítaním týchto
vzťahov rovnicu
$$
p_n=p_{n-2}+p_{n-3},
$$
ktorú môžeme odvodiť aj takto: Každé vyhovujúce slovo dĺžky~$n$
má práve jednu z~koncoviek $ABAA$, $ABA$, $BAB$, $BAAB$, pritom
koncovky $ABA$ a~$BAB$ má práve $p_{n-2}$~slov, zatiaľ čo koncovky
$ABAA$ a~$BAAB$ má práve $p_{n-3}$~slov.}

{%%%%%   A-II-3
\epsplace a53.11 \hfil\Obr

a) V~tetivovom štvoruholníku $ALBC$ platí $\alpha=|\uhol
BAC|=|\uhol BLC|$ a~$\beta=|\uhol ABC|=|\uhol ALC|$ (\obr).
Z~rovnosti obvodového a~príslušného úsekového uhla pre tetivu~$AK$
v~kružnici~$k_1$ vyplýva, že priamka~$AC$ je dotyčnicou ku
kružnici~$k_1$ práve vtedy, keď platí $|\uhol CAK|=|\uhol ALK|$, \tj.
práve vtedy, keď $\alpha=\beta$. Z~analogických dôvodov je priamka~$BC$
dotyčnicou ku kružnici~$k_2$ práve vtedy, keď $\beta=\alpha$. Priamka~$AC$
je preto dotyčnicou ku kružnici~$k_1$ práve vtedy, keď priamka~$BC$ je
dotyčnicou ku kružnici~$k_2$, čo sme chceli dokázať.

% (Ke stejnému výsledku lze dospět také užitím mocnosti bodu~$C$
% ke kružnicím~$k_1$ a~$k_2$, a~to s~ohledem na rovnost
% $\alpha=\beta$, tj.~$|CA|=|CB|$, a~rovnost $|CA|^2=|CK|\cdot
% |CL|=|CB|^2$.)

\inspicture

\smallskip
b) Podľa časti~a) vieme, že platí $\alpha=|\uhol BAC|=|\uhol
BLK|$ a~$\beta=|\uhol ABC|=|\uhol ALK|$. Bez ujmy na všeobecnosti
môžeme predpokladať, že platí $\alpha<\beta$. Dotyčnica v~bode~$A$
ku kružnici~$k_1$ zviera s~tetivou~$AK$ úsekový uhol $\b>\a$,
preto leží bod~$P$ na polpriamke~$AC$, zatiaľ čo bod~$Q$ leží
analogicky na polpriamke opačnej k~$BC$. Z~tetivových štvoruholníkov
$ALKP$ a~$BQLK$ vyplývajú rovnosti $|\uhol KPC|=\b$ a~$|\uhol
BQK|=\a$ (\obrr1). Trojuholníky $APK$ a~$QBK$ sa preto zhodujú v~dvoch
uhloch (pri vrcholoch~$A$, $Q$ a~$P$, $B$). Zhodujú sa teda
aj v~uhle pri spoločnom vrchole~$K$, takže
$$
|\uhol AKP|=|\uhol BKQ|\ (=\beta-\alpha).
$$
Odtiaľ vyplýva, že body $P$, $K$, $Q$ ležia na jednej priamke. Tým je
tvrdenie časti~b) dokázané.

\poznamka
Dokázali sme vlastne nasledujúce tvrdenie:
Ak je trojuholník $ABC$ rovnoramenný s~ramenami $AC$, $BC$, dotýkajú sa
obe ramená zodpovedajúcich kružníc $k_1$ a~$k_2$ vo vrcholoch $A$ 
a~$B$. A~tiež naopak, ak trojuholník $ABC$ nie je rovnoramenný,
pretínajú jeho strany $AC$ a~$BC$ zodpovedajúce kružnice $k_1$ a~$k_2$
v~ďalších bodoch~$P$ a~$Q$ ($P\ne A$, $Q\ne B$), pričom ich
spojnica~$PQ$ prechádza daným bodom~$K$.}

{%%%%%   A-II-4
Použitím sínusovej vety v~trojuholníkoch $BKA$ a~$CKA$ dostaneme
$$
\frac{|BK|}{|AK|}=\frac{\sin \frac{\alpha}2}{\sin\beta}
    \qquad \hbox{a} \qquad
    \frac{|CK|}{|AK|}=\frac{\sin \frac{\alpha}2}{\sin\gamma}.
$$
Sčítaním oboch predošlých rovností vyjde
$$
\frac{|BC|}{|AK|}=\frac{|BK|}{|AK|}+\frac{|CK|}{|AK|}=
    \sin\frac{\alpha}2\left(\frac1{\sin\beta}+\frac1{\sin\gamma}\right).
$$
Ak obe strany ostatnej nerovnosti vynásobíme výrazom $2\cos(\alpha/2)$,
dostaneme po úprave
$$
2\frac{|BC|}{|AK|}\cos\frac{\alpha}2=
     \sin\alpha\left(\frac1{\sin\beta}+\frac1{\sin\gamma}\right).
\tag1
$$
Cyklickou zámenou získame ďalšie dve analogické rovnosti
$$
\align
2\frac{|CA|}{|BL|}\cos\frac{\beta}2=&
     \sin\beta\left(\frac1{\sin\gamma}+\frac1{\sin\alpha}\right),
     \tag2                       \\
2\frac{|AB|}{|CM|}\cos\frac{\gamma}2=&
     \sin\gamma\left(\frac1{\sin\alpha}+\frac1{\sin\beta}\right).
     \tag3
\endalign
$$
Sčítaním rovností \thetag{1}, \thetag{2} a~\thetag{3} dostaneme po vydelení dvoma
rovnosť
$$
\displaylines{
\frac{|BC|}{|AK|}\cos\frac{\alpha}2+
          \frac{|CA|}{|BL|}\cos\frac{\beta}2+
          \frac{|AB|}{|CM|}\cos\frac{\gamma}2=\cr
\frac12\left(\frac{\sin\alpha}{\sin\beta}+
             \frac{\sin\beta}{\sin\alpha}\right)+
\frac12\left(\frac{\sin\beta}{\sin\gamma}+
             \frac{\sin\gamma}{\sin\beta}\right)+
\frac12\left(\frac{\sin\gamma}{\sin\alpha}+
             \frac{\sin\alpha}{\sin\gamma}\right).}
$$
Vzhľadom na to, že $\sin\alpha$, $\sin\beta$, $\sin\gamma$ sú
kladné čísla, môžeme každý z~troch výrazov v~zátvorkách na pravej
strane ostatnej rovnosti odhadnúť zdola číslom dva (využívame
známu nerovnosť $a/b+b/a\ge2$, ktorá je pre ľubovoľné kladné
čísla~$a$, $b$ ekvivalentná so zrejmou nerovnosťou $(a-b)^2\ge0$).
Odtiaľ vyplýva požadovaná nerovnosť. Tým je dôkaz hotový.}

{%%%%%   A-III-1
Ak nejaká trojica $(x,y,z)\in\Bbb R^3$ ($xyz\ne 0$) vyhovuje
podmienkam úlohy, je riešením sústavy nerovníc
$$
\eqalign{
  x^2+y^2+z^2&\leq 6+x^2-\frac{8}{x^4},\cr
  x^2+y^2+z^2&\leq 6+y^2-\frac{8}{y^4},\cr
  x^2+y^2+z^2&\leq 6+z^2-\frac{8}{z^4},\cr
}\qquad
\hbox{\tj.}
\qquad
\eqalign{
  \frac{8}{x^4}+y^2+z^2 &\leq 6,\cr
  x^2+\frac{8}{y^4}+z^2 &\leq 6,\cr
  x^2+y^2+\frac{8}{x^4} &\leq 6.
}
$$
Sčítaním všetkých troch nerovníc tejto sústavy dostaneme nerovnicu
$$
\left(\frac{8}{x^4}+x^2+x^2\right)+
   \left(\frac{8}{y^4}+y^2+y^2\right)+
   \left(\frac{8}{z^4}+z^2+z^2\right)\leq 18.
$$
Výrazy v~každej z~troch zátvoriek na ľavej strane možno odhadnúť použitím
nerovnosti medzi aritmetickým a~geometrickým priemerom
trojice kladných čísel. Obdržíme tak postupne
$$
\eqalign{
18&\geq \left(\frac{8}{x^4}+x^2+x^2\right)+
          \left(\frac{8}{y^4}+y^2+y^2\right)+
          \left(\frac{8}{z^4}+z^2+z^2\right) \geq\cr
  &\geq 3\root 3\of{\frac{8}{x^4}\cdot x^2 \cdot x^2}+
        3\root 3\of{\frac{8}{y^4}\cdot y^2 \cdot y^2}+
        3\root 3\of{\frac{8}{z^4}\cdot z^2 \cdot z^2}=18.}
$$
Odtiaľ vyplýva, že v~každej z~troch použitých nerovností medzi
aritmetickým a~geometrickým priemerom nastane rovnosť, takže
príslušná trojica čísel má vždy tri rovnaké zložky. Musí teda
súčasne platiť
$$
\frac{8}{x^4}=x^2,\qquad \frac{8}{y^4}=y^2,\qquad
   \frac{8}{z^4}=z^2,
$$
\tj.
$$
x^6=y^6=z^6=8.
$$
Z~ostatnej podmienky bezprostredne vyplýva
$$
(x,y,z)=
   \bigl(\varepsilon_1\sqrt{2},\varepsilon_2\sqrt{2},\varepsilon_3\sqrt{2}\bigr),
   \quad \hbox{kde }\varepsilon_i\in\{-1;1\}
   \hbox{ pre }  i=1,2,3.\tag{1}
$$

\smallskip
Vzhľadom na použité dôsledkové úpravy je nutné urobiť skúšku,
pomocou ktorej zistíme, že všetkých 8~trojíc reálnych čísel určených
vzťahom~\thetag{1} vyhovuje podmienkam úlohy.

\ineriesenie
Nech trojica $(x,y,z)\in\Bbb R^3$ ($xyz\ne 0$)
je riešením danej úlohy. Označme
$$
A=\min\{x^2,y^2,z^2\}>0.
$$
Potom platí
$$
\min\left\{x^2-\frac{8}{x^4},y^2-\frac{8}{y^4},
   z^2-\frac{8}{z^4}\right\}=A-\frac{8}{A^2}.
$$
Preto tiež
$$
\eqalign{
A+A+A&\leq x^2+y^2+z^2\leq 6+\min\left\{x^2-\frac{8}{x^4},y^2-\frac{8}{y^4},
   z^2-\frac{8}{z^4}\right\}=\cr
 &=6+A-\frac{8}{A^2}.\cr}
$$
Po úprave dostaneme nerovnosť, ktorej pravú stranu odhadneme
použitím nerovnosti medzi aritmetickým a~geometrickým priemerom:
$$
6\geq A+A+\frac{8}{A^2}\geq3\root 3\of{A\cdot A\cdot\frac{8}{A^2}}
=6.
$$
To znamená, že vo všetkých použitých nerovnostiach musí nastať rovnosť, a~teda
$$
2=A=x^2=y^2=z^2.
$$

Skúškou opäť overíme, že všetky trojice určené vzťahom~\thetag{1}
sú riešením zadanej nerovnice.}

{%%%%%   A-III-2
Počet vyhovujúcich slov dĺžky~$n$, ktoré končia písmenom~$A$, resp.~$B$,
označme~$a_n$, resp.~$b_n$.  Platí
$$
p_n=a_n+b_n.\tag{1}
$$
Nech $n\geq4$. Vyhovujúce slovo končiace písmenom~$A$ má jednu
z~koncoviek $BA$, $BAA$, alebo $BAAA$. Počet slov prvého typu je
$b_{n-1}$, druhého typu $b_{n-2}$, tretieho typu $b_{n-3}$. Preto
$$
a_n=b_{n-1}+b_{n-2}+b_{n-3}.\tag{2}
$$
Podobne pre $n\geq3$ má vyhovujúce slovo končiace písmenom~$B$
jednu z~koncoviek $AB$, $ABB$, teda
$$
b_n=a_{n-1}+a_{n-2}.\tag{3}
$$

Nech ďalej $n\geq6$. Každé z~čísel~$b_i$ vo vzťahu~\thetag{2}
vyjadrime pomocou~\thetag{3}, dostaneme tak
$$
\eqalign{
a_n&=b_{n-1}+b_{n-2}+b_{n-3}=\cr
   &=(a_{n-2}+a_{n-3})+(a_{n-3}+a_{n-4})+(a_{n-4}+a_{n-5})=\cr
   &=a_{n-2}+2a_{n-3}+2a_{n-4}+a_{n-5}.}\tag{4}
$$
Podobne dostaneme
$$
\eqalign{
b_n&=a_{n-1}+a_{n-2}=\cr
   &=(b_{n-2}+b_{n-3}+b_{n-4})+(b_{n-3}+b_{n-4}+b_{n-5})=\cr
   &=b_{n-2}+2b_{n-3}+2b_{n-4}+b_{n-5}.}\tag{5}
$$
Sčítaním vzťahov \thetag{4} a~\thetag{5} dostaneme podľa~\thetag{1}
$$
p_n=p_{n-2}+2p_{n-3}+2p_{n-4}+p_{n-5}.
$$
Preto pre ľubovoľné prirodzené číslo $n\geq6$ platí
$$
\frac{p_n-p_{n-2}-p_{n-5}}{p_{n-3}+p_{n-4}}=2,
$$
takže zadaný zlomok má hodnotu~2 aj pre $n=2004$.}

{%%%%%   A-III-3
\fontplace
\bpoint A_i; \rBpoint X_i; \rpoint Y_i;
\lpoint U_i; \lBpoint V_i; \tpoint S;
\tpoint p_i; \cpoint\epsilon; \cpoint\epsilon;
[12] \hfil\Obr

\fontplace
\rBpoint X_i; \rpoint Y_i;
\lpoint U_i; \lBpoint V_i; \rtpoint U_i';
\cpoint2\epsilon;
[13] \hfil\Obr

\inspicture{}
Pre ľubovoľné~$i$, $1\le i\le121$, označme $\Cal M_i$ množinu
všetkých bodov~$X$ kružnice~$k$, pre ktoré úsečka~$A_iX$ zviera
s~príslušnou priamkou~$p_i$ uhol veľkosti menšej ako
$\varepsilon=21^{\circ}$. Množina $\Cal M_i$ je zrejme tvorená
dvoma oblúkmi $X_iY_i$ a~$U_iV_i$ (\obr). Obom uvažovaným
oblúkom kružnice~$k$ zodpovedá dvojica stredových uhlov $X_iSY_i$
a~$U_iSV_i$, kde $S$ je stred danej kružnice~$k$. Ukážeme, že pre
každé $i\in\{1,2,\dots,121\}$ platí $|\uhol X_iSY_i|+|\uhol
U_iSV_i|=4\varepsilon=84^{\circ}$.

\inspicture{}
V~trojuholníku $A_iY_iU_i$ je súčet veľkostí vnútorných uhlov pri
vrcholoch $Y_i$ a~$U_i$ rovný veľkosti vedľajšieho uhla pri
vrchole $A_i$, \tj.~$2\varepsilon$. Na druhej strane, súčet oboch uvažovaných
uhlov v~tomto trojuholníku je rovný súčtu
obvodových uhlov prislúchajúcich oblúkom $X_iY_i$ a~$U_iV_i$. Zo
vzťahu medzi obvodovým a~stredovým uhlom dostávame
$$
|\uhol X_iSY_i|+|\uhol U_iSV_i|=2\cdot 2\varepsilon
=4\varepsilon=84^{\circ}.
$$
Celkovo tak 121 uvažovaným tetivám~$p_i$ a~ich bodom~$A_i$
zodpovedá 121~dvojíc oblúkov $X_iY_i$ a~$U_iV_i$ kružnice~$k$ 
s~celkovou oblúkovou dĺžkou $121\cdot 84^{\circ}=10\,164^{\circ}$.
Pokiaľ každý bod~$X$ kružnice~$k$ leží najviac v~28~množinách~$\Cal
M_i$, musí byť uvedený súčet všetkých oblúkových dĺžok rovný najviac
$28\cdot 360^{\circ}=10\,080^{\circ}$, čo neplatí. Preto
existuje aspoň jeden bod kružnice~$k$, ktorý leží súčasne
aspoň v~29~množinách~$\Cal M_i$, čo sme mali dokázať.

\poznamka
Že obom oblúkom $X_iY_i$ a~$U_iV_i$ zodpovedá spolu stredový
uhol~$4\epsilon$, nahliadneme ľahko aj z~\obr, lebo oblúky
$U'_iY_i$ a~$U_iV_i$ sú zhodné.}

{%%%%%   A-III-4
Pre $n$ rovné 1, 2 a~3 je daný súčet postupne rovný celým číslam 1, 3, 5.
Predpokladajme preto ďalej, že $n>3$. Jednoduchou úpravou
dostaneme
$$
\align
\frac{n}{1!}+&\frac{n}{2!}+\cdots+\frac{n}{(n-2)!}+\frac{n}{(n-1)!}+
     \frac{n}{n!}=\cr
&=\frac{n(n-1)\cdot\dots\cdot2+n(n-1)\cdot\dots\cdot3+\cdots
     +n(n-1)+n+1}{(n-1)!}.
\endalign
$$
Ak je ostatný zlomok celé číslo, je nutne číslo $n-1$ deliteľom
jeho čitateľa. Preto je číslo~$n-1$ deliteľom čísla $n+1$.
Pretože najväčší spoločný deliteľ dvoch čísel je deliteľom 
aj ich rozdielu, je najväčší spoločný deliteľ čísel $n-1$ a~$n+1$
deliteľom čísla~$2$, takže $n-1\in\{1,2\}$, čo je v~spore
s~predpokladom $n>3$.

Daný súčet je celé číslo pre prirodzené čísla~$n$ z~množiny
$\{1,2,3\}$.}

{%%%%%   A-III-5
\fontplace
\tlpoint A; \blpoint B; \brpoint C; \trpoint D;
\ltpoint\xy-.9,-1 K; \rBpoint L; \tpoint M; \bpoint N;
[14] \hfil\Obr

\fontplace
\tlpoint A; \blpoint B; \brpoint C; \trpoint D;
\rBpoint L; \tpoint M; \bpoint N;
[15] \hfil\Obr

\fontplace
\tlpoint A; \blpoint B; \brpoint C; \trpoint D;
\ltpoint\xy-.9,-1 K; \rBpoint L; \tpoint M; \bpoint N;
\bpoint\xy.5,.5 P;
[16] \hfil\Obr

Uhlopriečka~$AC$ je priemerom kružnice opísanej štvorcu $ABCD$, takže
podľa Tálesovej vety je uhol $ALC$ pravý (\obr). Bod~$K$ je tak
priesečníkom výšok $CD$ a~$AL$ v~trojuholníku $ACM$, takže aj priamka~$MK$
je kolmá na~$AC$ a~pretína stranu~$BC$ daného štvorca
v~jej vnútornom bode~$N$, lebo $MK\parallel DB$.
\inspicture{}

Teraz možno tvrdenie úlohy dokázať niekoľkými spôsobmi.

\smallskip
{\it Prvý spôsob\/}.
Štvoruholníky $BCLD$ a~$KLMD$ sú tetivové, preto podľa vety
o~obvodových uhloch postupne platí
$$
|\uhol NBL|=|\uhol CBL|=|\uhol CDL|=|\uhol KDL|
            =|\uhol KML|=|\uhol NML|.
$$
Pretože body~$B$ a~$M$ ležia v~rovnakej polrovine určenej
priamkou~$NL$, ležia body $B$, $L$, $M$, $N$ na jednej kružnici.

\smallskip
{\it Druhý spôsob\/}.
Pretože $MN\parallel DB$, platí $|\uhol MNC|=45\st$, rovnako uhol
$BLC$ nad tetivou~$BC$ kružnice~$k$ má veľkosť 45\st{} (\obr),
\inspicture{}
takže $|\uhol BLM|=|\uhol BNM|=135\st$. Body $L$ a~$N$ zrejme
ležia v~rovnakej polrovine určenej priamkou~$MB$, preto ležia body $B$,
$L$, $M$, $N$ na jednej kružnici.

% \twocpictures

\smallskip
{\it Tretí spôsob\/}.
Označme $P$ pätu výšky z~vrcholu~$M$ na stranu~$AC$
a~uvažujme štvoruholníky $ABNP$, $APKD$ a~$DKLM$ (\obr). Podľa
\inspicture{}
Tálesovej vety sú všetky tri štvoruholníky tetivové. Vrchol~$C$
daného štvorca $ABCD$ leží mimo každej z~troch kružníc opísaných
uvažovaným tetivovým štvoruholníkom, takže použitím vety o~mocnosti
bodu~$C$ ku kružniciam opísaným postupne štvoruholníkom $ABNP$,
$APKD$, $DKLM$ dostaneme tri rovnosti
$$
\eqalign{
  |CN|\cdot |CB| &= |CP|\cdot |CA|, \cr
  |CP|\cdot |CA| &= |CK|\cdot |CD|, \cr
  |CK|\cdot |CD| &= |CL|\cdot |CM|, \cr
}
$$
z~ktorých bezprostredne vyplýva rovnosť
$$
|CN|\cdot |CB|=|CL|\cdot |CM|.
$$
Odtiaľ už vyplýva, že body $B$, $L$, $M$, $N$ leží na jednej kružnici.}

{%%%%%   A-III-6
Nech $f$ je ľubovoľná z~hľadaných funkcií. Označme $f(1)=p$.
Vzhľadom na podmienky úlohy platí $p>0$.

V~danom vzťahu položme $x=1$, $y=1$. Po úprave dostaneme
$$
p=f(p).\tag{1}
$$
V~danom vzťahu ďalej položme $x=p$, $y=1$. Potom
$$
p^2\bigl(f(p)+p\bigr)=(p+1)f\bigl(f(p)\bigr)
$$
a~podľa~\thetag{1} vyjde
$$
2p^3=(p+1)p.
$$
Táto algebraická rovnica má tri reálne korene $\m1/2$, $0$, $1$.
Jediný koreň vyhovujúci podmienke $p>0$ je $p=1$, teda
$$
f(1)=1.              \tag{2}
$$
Nech $t$ je ľubovoľné kladné reálne číslo. V~danom vzťahu
položme $x=1$, $y=t$, takže vzhľadom na \thetag{2} dostaneme
$$
1+f(t)=(1+t)f(t).
$$
Odtiaľ po úprave
$$
f(t)=\frac1t.       \tag{3}
$$

Dosadením ľahko overíme, že funkcia $f(t)=1/t$ vyhovuje
rovnici zo zadania.
Funkcia určená vzťahom~\thetag{3} je jediné riešenie danej úlohy.

\ineriesenie
Predpokladajme, že existuje funkcia daných vlastností
a~ľubovoľnú z~takých funkcií označme~$f$.

Nech $t$ je ľubovoľné kladné reálne číslo. V~danom vzťahu
položme $x=t$, $y=t$. Po úprave dostaneme
$$
t f(t)=f\big(t f(t)\big).
$$
Odtiaľ vyplýva, že množina $\mm P=\{p\in\Bbb R^{+};\ p=f(p)\}$ je neprázdna,
pretože pre každé kladné reálne číslo~$t$ je $tf(t)$ prvkom~$\mm P$.

Predpokladajme, že množina~$\mm P$ obsahuje aspoň dve
rôzne čísla, označme ich $a$ a~$b$. V~danom vzťahu položme
$x=a$, $y=b$. Dostaneme
$$
a^2\big( f(a)+f(b)\big)=(a+b)f\big( f(a)b\big).
$$
Vzhľadom na to, že $a=f(a)$, $f(a)+f(b)=a+b\ne 0$,
dostaneme odtiaľ po úprave
$$
a^2=f(ab).\tag4
$$
Ak položíme v~danom vzťahu naopak $x=b$, $y=a$, dostaneme
po podobnej úprave
$$
b^2=f(a b).\tag5
$$
Vzhľadom na to, že $a$ a~$b$ sú kladné čísla, vyplýva zo vzťahov
\thetag{4} a~\thetag{5} $a=b$, čo je spor s~predpokladom, že množina~$\mm P$
obsahuje aspoň dve rôzne čísla.

Množina~$\mm P$ teda obsahuje práve jedno číslo, označme ho
$p$ ($p\in\Bbb R^{+}$). Z~predchádzajúcich vzťahov vyplýva, že pre každé kladné
reálne číslo~$t$ platí $tf(t)=p$, preto
funkcia~$f$ má nutne tvar
$$
f(t)=\frac{p}t.
$$

Teraz dosaďme tento predpis do pôvodného vzťahu.
Pre všetky $x,y\in\Bbb R^{+}$ tak dostaneme
$$
x^2\left(\frac{p}x+\frac{p}y\right)=(x+y)
  \frac{p}{\frac{p}{x} y}.
$$
Úpravou získame $p=1$.

Teda funkcia~$f$ daná pre všetky kladné reálne čísla~$t$ predpisom
$$
f(t)=\frac1t
$$
je jediná funkcia, ktorá vyhovuje danému vzťahu.}

{%%%%%   B-S-1
Predpokladajme najskôr, že celé číslo $k=\lfloor x\rfloor$ poznáme,
dosaďme ho do rovnice ako "parameter" a~získanú rovnicu
vyriešme:
$$
\align
x&=k+\frac{x}{2\,004},\\
2\,004x&=2\,004k+x,\\
x&=\frac{2\,004k}{2\,003}.
\endalign
$$
Keď budeme do ostatného vzťahu dosadzovať jednotlivé celé čísla~$k$,
bude príslušné~$x$ naozaj riešením
skúmanej rovnice vtedy, keď sa jeho celá časť bude rovnať práve
číslu~$k$, teda keď budú platiť nerovnosti
$$
k\leqq \frac{2\,004k}{2\,003}<k+1.
$$
Zistíme, ktoré celé~$k$ vyhovujú obom nerovnostiam. Ľavá
nerovnosť je ekvivalentná s~nerovnosťou $k\geqq0$, pravá nerovnosť
s~nerovnosťou $k<2\,003$. Hľadané~$k$ sú teda práve hodnoty
$k\in\{0,1,\dots,2\,002\}$. Každá z~nich určuje jediné riešenie~$x$,
takže všetkých riešení~$x$ zadanej rovnice je práve 2\,003.
Dodajme, že vyhovujúce~$k$ možno určiť aj úpravou odvodeného
vzťahu na tvar
$$
x=\frac{2\,004k}{2\,003}=k+\frac{k}{2\,003},
$$
z~ktorého vidno, že číslo~$k$ je celou časťou čísla~$x$
práve vtedy, keď platia nerovnosti
$$
0\leqq\frac{k}{2\,003}<1,\quad\text{čiže}\quad 0\le k<2\,003.
$$

\ineriesenie
Pretože pre každé reálne~$x$ platí
$\lfloor x\rfloor\leqq x\leqq\lfloor x\rfloor+1$, porovnaním so
zadanou rovnicou zistíme, že každé riešenie~$x$ musí
spĺňať nerovnosti
$$
0\leqq\frac{x}{2\,004}<1,\quad\text{čiže}\quad
0\leqq x<2\,004.
$$
Číslo~$x$ spĺňajúce ostatné nerovnosti bude riešením skúmanej
rovnice práve vtedy, keď hodnota $x-x/2\,004$ bude celočíselná.
Pretože platí
$$
x-\frac{x}{2\,004}=\frac{2\,003x}{2\,004},
$$
dá sa ostatná podmienka vysloviť takto: číslo $2\,003x$ je
celočíselným násobkom čísla $2\,004$. To vzhľadom na nerovnosti
$0\leqq 2\,003x<2\,003\cdot2\,004$ znamená, že číslo $2\,003x$ sa rovná
niektorému z~čísel
$$
0\cdot2\,004,\,1\cdot2\,004,\,2\cdot2\,004,\,\dots,\,2\,002\cdot2\,004,
$$
takže
skúmaná rovnica má práve 2\,003 riešení
$$
\frac{0\cdot2\,004}{2\,003},\,\frac{1\cdot2\,004}{2\,003},\,
\frac{2\cdot2\,004}{2\,003},\,\dots,\,\frac{2\,002\cdot2\,004}{2\,003}.
$$}

{%%%%%   B-S-2
Kvôli podmienke~(i) môže byť v~množine~$\mm M$ najviac jedno
z~čísel $11,22,33,\dots,99$ zapísaných dvoma rovnakými
číslicami, ktoré sú všetky deliteľné jedenástimi. Kvôli
podmienke~(ii) a~deliteľnosti dvoma tam zasa nemôže byť žiadne číslo
zapísané dvoma rôznymi párnymi číslicami. S~jednou párnou číslicou
môže byť v~$\mm M$ najviac jedna dvojica čísel $\overline{ab}$,
$\overline{ba}$.

Ostáva zistiť, koľko môže množina~$\mm M$ obsahovať dvojíc čísel
$\overline{ab}$, $\overline{ba}$ zapísaných dvoma rôznymi nepárnymi
číslicami~$a$ a~$b$. Žiadne z~týchto čísel nesmie byť deliteľné
tromi (ak je číslo~$\overline{ab}$ deliteľné tromi, je také 
aj číslo~$\overline{ba}$), preto do úvahy prichádza iba sedem dvojíc
takých čísel: $(13,31)$, $(17,71)$, $(19,91)$, $(35,53)$,
$(37,73)$, $(59,95)$ a~$(79,97)$. Kvôli deliteľnosti piatimi, siedmimi
a~devätnástimi však môže byť v~$\mm M$ iba jedna z~dvojíc
$(19,91)$, $(35,53)$ a~$(59,95)$, teda najviac päť zo všetkých siedmich
vypísaných dvojíc.

Celkove zisťujeme, že množina~$\mm M$ obsahuje najviac
$1+2+2\cdot5=13$ čísel. Príkladom trinásťprvkovej množiny
je
$$
\mm M=\{11,23,32,13,31,17,71,35,53,37,73,79,97\}.
$$
(Existujú aj iné príklady, naše úvahy však ukazujú, že každá
trinásťprvková množina~$\mm M$ musí obsahovať čísla 13,
31, 17, 71, 37, 73, 79, 97 a~jednu z~dvojíc $(35,53)$ alebo
$(59,95)$; dvojica $(19,91)$ je vylúčená, lebo číslo 91 je
násobkom čísla~13.)}

{%%%%%   B-S-3
\fontplace
\rpoint A; \lpoint B; \bpoint\ C; \bpoint\xy.6,0 D;
\ltpoint\xy-1,0 E; \bpoint V;
\cpoint\a; \cpoint\b;
[5] \hfil\Obr

Označme $\alpha$ a~$\beta$ postupne vnútorné uhly pri vrcholoch $A$
a~$B$ (\obr). Bodom~$E$ prechádza spoločná dotyčnica oboch uvažovaných
\inspicture{}
kružníc, uhol $DEC$ je teda súčtom úsekových uhlov prislúchajúcich
tetive~$DE$ v~jednej kružnici (s~obvodovým uhlom~$\alpha$) 
a~tetive~$EC$ v~druhej kružnici (s~obvodovým uhlom~$\beta$). Jeho
veľkosť je preto $\alpha+\beta$. A~pretože veľkosť uhla $CV\!D$ je
$180\st-(\a+\b)$, zisťujeme, že v~štvoruholníku $CV\!DE$ sa uhly
pri protiľahlých vrcholoch~$E$ a~$V$ dopĺňajú do $180\st$. To, ako vieme,
znamená, že $CV\!DE$ je tetivový štvoruholník, \tj. bod~$E$
leží na kružnici opísanej trojuholníku~$CDV$.}

{%%%%%   B-II-1
Ako vieme, každé prirodzené číslo~$k$ dáva po delení
tromi rovnaký zvyšok ako číslo~$S(k)$ rovné súčtu číslic
pôvodného čísla~$k$.
Číslo~$a_n$ preto dáva po delení tromi rovnaký zvyšok ako
súčet $S(1^2)+S(2^2)+\cdots+S(n^2)$, teda aj ako súčet
$1^2+2^2+\cdots+n^2$. Dvoma spôsobmi ukážeme, že ostatný súčet
je deliteľný tromi práve vtedy, keď číslo~$n$ je tvaru $9k-5$, $9k-1$
alebo~$9k$, pričom $k$ je prirodzené číslo.

\smallskip
Pri {\it prvom spôsobe\/} využijeme známy vzťah
$$
1^2+2^2+\cdots+n^2=\dfrac{n(n+1)(2n+1)}{6},
\tag1
$$
z~ktorého vyplýva, že skúmaný súčet je deliteľný tromi práve vtedy, keď
je súčin $n(n+1)(2n+1)$ deliteľný deviatimi. Pretože čísla $n$,
$n+1$ a~$2n+1$ sú navzájom nesúdeliteľné, hľadáme práve tie~$n$,
pre ktoré je deliteľné deviatimi jedno z~čísel $n$, $n+1$ alebo
$2n+1$, a~to sú postupne čísla tvaru $9k$, $9k-1$, $9k-5$.

 \smallskip
{\it Druhý spôsob\/} je založený na pozorovaní, že zvyšky čísel
$1^2, 2^2, 3^2, 4^2, 5^2,6^2,\dots$ po delení tromi sú
$1, 1, 0, 1, 1, 0, \dots$, opakujú sa teda s~periódou dĺžky~3.
Skutočne, čísla $(k+3)^2$ a~$k^2$ dávajú rovnaký zvyšok po delení
tromi, lebo ich rozdiel je číslo $3(2k+3)$, čo je násobok
troch. Sčítaním uvedených zvyškov dostaneme postupne zvyšky prvých
deviatich súčtov~\thetag{1}: 1, 2, 2, 0, 1, 1, 2, 0,~0. Zvyšky ďalších
súčtov~\thetag{1} sa začnú periodicky opakovať. (Vyplýva to z~toho,
% že pro $n=10$ vychází stejný zbytek součtu (1) jako pro $n=1$ a
% že rozdíl $10-1$ je
že predchádzajúci súčet deviatich čísel dáva nulový zvyšok a~súčasne je
počet sčítancov násobkom periódy~3 sčítaných zvyškov.)

Vieme už, ktoré čísla~$a_n$ sú deliteľné tromi. Zaoberajme sa teraz
jednoduchšou otázkou, ktoré~$a_n$ sú deliteľné štyrmi. Ukážeme, že sú to
všetky~$a_n$ s~párnym $n>2$ (a~žiadne iné). Číslo~$a_n$ 
s~nepárnym~$n$ je totiž nepárne, číslo~$a_2$ sa rovná~14 a~číslo~$a_n$
s~párnym $n>2$ končí rovnakým dvojčíslím ako číslo~$n^2$, takže je
také~$a_n$ (rovnako ako spomenuté dvojčíslie) deliteľné štyrmi.

Keď spojíme výsledky o~deliteľnosti tromi a~štyrmi,
zistíme, že číslo~$a_n$ je deliteľné dvanástimi práve vtedy,
keď číslo~$n$ je tvaru $18k-14$, $18k-10$ alebo $18k$,
pričom $k$ je ľubovoľné prirodzené číslo. Pretože
$100\,000=5\,556\times18-8$, je medzi prirodzenými číslami od~1
do~100\,000 práve 5\,556 čísel tvaru $18k-14$, 5\,556 čísel tvaru $18k-10$
a~5\,555 čísel tvaru $18k$. Spolu je to 16\,667 čísel.}

{%%%%%   B-II-2
Pre koeficient~$a$ musí platiť $a\ne0$ a~$a\ne\m1$,
aby všetky uvažované trojčleny boli naozaj kvadratické
trojčleny. Ako vieme, kvadratický trojčlen má dvojnásobný koreň
práve vtedy, keď je jeho diskriminant nulový. Zostavme preto
diskriminanty všetkých troch trojčlenov so zväčšenými koeficientmi:
$$
\align
(a+1)x^2+bc+c\quad&\text{má diskriminant}\quad D_1=b^2-4(a+1)c,\\
ax^2+(b+1)x+c\quad&\text{má diskriminant}\quad D_2=(b+1)^2-4ac,\\
ax^2+bx+(c+1)\quad&\text{má diskriminant}\quad D_3=b^2-4a(c+1).
\endalign
$$
Hľadáme teda reálne čísla $a$, $b$, $c$, pre ktoré platí $a\ne0$,
$a\ne\m1$ a~$D_1=D_2=D_3=0$.

Z~rovnosti $D_1=D_3$ vyplýva $c=a$, takže
$D_2=(b+1)^2-4a^2=(b+1-2a)(b+1+2a)$. Rovnosť $D_2=0$ potom znamená,
že platí $b=\pm2a-1$, a~preto $D_1=(\pm2a-1)^2-4(a+1)a=
4a^2\mp4a+1-4a^2-4a=1\mp4a-4a$, teda $D_1=\m8a+1$ alebo $D_1=1$.
Preto z~rovnosti $D_1=0$ vyplýva $a=1/8$, $b=2a-1={\m3/4}$
a~$c=a=1/8$. (Skúška je jednoduchá, nie je však nutná, pretože naším postupom
máme zaručené rovnosti $D_1=D_3$, $D_2=0$ a~$D_1=0$.)

\odpoved
Úlohe vyhovuje jediný trojčlen
$x^2/8-3x/4+1/8$.}

{%%%%%   B-II-3
Kladné číslo~$x$ je riešením rovnice s~daným~$n$ práve vtedy,
keď je číslo~$nx$ prirodzené a~platia nerovnosti
$$
nx-1\leqq x\sqrt{n^2-1}<nx.
$$
Pravá nerovnosť platí pre každé $x>0$, lebo
zrejme platí $\sqrt{n^2-1}<\sqrt{n^2}=n$.
Zostáva teda vyriešiť ľavú nerovnicu
(vzhľadom na neznámu~$x$). Po jednoduchej úprave dostávame
$$
\gather
x\bigl(n-\sqrt{n^2-1}\bigr)\leqq 1,\\
x\leqq\frac{1}{n-\sqrt{n^2-1}}=n+\sqrt{n^2-1}.
\endgather
$$
Využili sme to, že výraz $n-\sqrt{n^2-1}$ je kladný
a~v~súčine so združeným výrazom $n+\sqrt{n^2-1}$ dáva číslo~1.
Po vynásobení oboch strán odvodenej nerovnosti
číslom~$n$ dostaneme pre prirodzené číslo $k=nx$
ekvivalentnú podmienku
$$
k\leqq n^2+n\sqrt{n^2-1},
$$
ktorá je splnená práve pre $k\in\{1,2,\dots,2n^2-1\}$, lebo
pre druhý sčítanec z~pravej strany ostatnej
nerovnosti zrejme platia celočíselné odhady
$$
n^2-1\leqq n\sqrt{n^2-1}<n^2
$$
(znovu využívame iba nerovnosť $\sqrt{n^2-1}<n$).
Všetky riešenia danej rovnice majú tvar $x=k/n$ a~tvoria tak množinu
zlomkov
$$
\left\{\frac{1}{n},\ \frac{2}{n},\ \dots,\ \frac{2n^2-1}{n}\right\}.
$$}

{%%%%%   B-II-4
\fontplace
\rpoint V;
\tpoint A; \bpoint B; \bpoint C; \lbpoint D;
\bpoint C'; \tpoint D';
\lpoint C''; \rpoint D'';
\rtpoint T; \lBpoint M;
\rBpoint k; \lpoint t;
[6]
\hfil\Obr

\fontplace
\lbpoint S_1; \lbpoint S_2;
\bpoint C; \tpoint D;
\bpoint C'; \tpoint D';
\bpoint C''; \tpoint D'';
\bpoint T_1; \bpoint T_2;
\tpoint U_1; \tpoint U_2;
\bpoint\xy-1.3,-.3 T_2'; \tpoint\xy.5,-.5 T_1';
[8]
\hfil\Obr

Kružnica vpísaná do hľadaného štvoruholníka je kružnicou~$k$ pripísanou
strane~$AB$ trojuholníka $BAV$. Označme $T$ ten z~dvoch priesečníkov osi uhla
$AV\!B$ s~kružnicou~$k$, ktorý je ďalej od vrcholu~$V$ (\obr).
Hľadané body $C$ a~$D$ nájdeme ako priesečníky
dotyčnice~$t$ v~bode~$T$ ku kružnici~$k$ postupne s~priamkami $V\!B$ a~$V\!A$.
Dokážme, že takto zostrojený štvoruholník má zo všetkých
štvoruholníkov vyhovujúcich podmienkam úlohy najmenší obsah.
\inspicture{}

Označme $C'$, $D'$ vrcholy iného dotyčnicového štvoruholníka
s~vpísanou kružnicou~$k$ (priamka~$C'D'$ je dotyčnicou kružnice~$k$).
Bez ujmy na všeobecnosti môžeme predpokladať, že priesečník~$M$
dotyčníc $t$ a~$C'D'$ leží vnútri úsečky~$TC$. To znamená, že
platí $|MD|>|MC|$ (\obrr1).
% Ukážeme, že je také $|MD'|>|MC'|$:
Označme $C''$ a~$D''$ zodpovedajúce päty kolmíc spustených
z~bodov $C'$ a~$D'$ na priamku~$t$. Bod~$C''$ leží vnútri
úsečky~$MC$ a~$D''$ na polpriamke~$MD$ mimo úsečky~$MD$, takže
$|MC''|<|MC|<|MD|<|MD''|$ a~z~podobnosti pravouhlých trojuholníkov
$MC'C''$ a~$MD'D''$ vyplýva $|C'C''|<|D'D''|$. Trojuholník $DMD'$
má teda väčší obsah ako trojuholník $CMC'$. Rozdiel ich obsahov
je však rovný rozdielu obsahov štvoruholníkov $ABC'D'$ a~$ABCD$, teda
obsah štvoruholníka $ABC'D'$ je väčší ako obsah štvoruholníka
$ABCD$.

\ineriesenie
Rovnako ako v~prvom riešení označme $C$, $D$ priesečníky dotyčnice~$t$
pripísanej kružnice~$k$ s~ramenami $V\!B$, $V\!A$.
Ak $C'$, $D'$ sú vrcholy iného dotyčnicového štvoruholníka
s~vpísanou kružnicou~$k$, pre obsahy dotyčnicových
štvoruholníkov $ABCD$ a~$ABC'D'$ platí
$$
\gather
S_{ABCD}=S_{VCD}-S_{V\!AB}, \\
S_{ABC'D'}=S_{VC'D'}-S_{V\!AB}.
\endgather
$$
Stačí teda ukázať, že pre ľubovoľnú takú dotyčnicu~$C'D'$, ktorá
nie je kolmá na os uhla~$AVB$, platí $S_{VC'D'}>S_{VCD}$. To je však
zrejmé z~\obr{} (oba sivé trojuholníky majú vďaka stredovej súmernosti
rovnaký obsah a~pritom $S_{VC'D'}>S_{VC_1D_1}>S_{VCD}$).
\insp{B53.7}% 

\ineriesenie
Obsah dotyčnicového štvoruholníka $ABCD$, ktorého vpísaná kružnica má
polomer~$r$, je $S=r(|AB|+|BC|+|CD|+|DA|)/2=
r(2|AB|+2|CD|)/2=r(|AB|+|CD|)$. Obsah dotyčnicového štvoruholníka
$ABCD$ spĺňajúceho podmienky úlohy bude teda najmenší práve vtedy, keď
bude najkratšia úsečka~$CD$.

Uvažujme kružnicu pripísanú strane~$C'D'$ trojuholníka $VC'D'$ (\obr). 
\inspicture{}
Z~vlastností dotyčníc postupne nahliadneme, že $|T_1C|=|CD|/2$,
$|T_2C''|=|C''D''|/2$ a~tiež $|C'D'|=|T_1T_2|$. Ostatná
rovnosť vyplýva zo známych vlastností vpísanej a~pripísanej
kružnice, totiž že ich body dotyku na spoločnú stranu sú
súmerne združené podľa stredu strany. Dôkaz tohto tvrdenia
vyžaduje trochu počítania:
$$
\align
|T_1T_2|=&|T_1C'|+|C'T_2|=|T_1'C'|+|C'T_2'|=|T_1'T_2'|+2|T_2'C'|,\\
|U_1U_2|=&|U_1D'|+|D'U_2|=|T_1'D'|+|D'T_2'|=|T_1'T_2'|+2|T_1'D'|.\\
\endalign
$$
Zo súmernosti podľa osi uhla $AV\!B$ vyplýva $|T_1T_2|=|U_1U_2|$,
takže $|T_2'C'|=|T_1'D'|$. Teda
$|C'D'|=|T_1'T_2'|+2|T_2'C'|=|T_1T_2|$.

Pretože obe kružnice sú oddelené spoločnou dotyčnicou~$C'D'$,
nemôžu sa dotýkať. Takže $|CD|<|C''D''|$, čiže
$|CT_1|<|C''T_2|$. To znamená, že
$$
|CD|=2|T_1C|<|T_1C|+|C''T_2|<|T_1T_2|=|C'D'|,
$$
čo sme chceli dokázať.}

{%%%%%   C-S-1
Trojciferné číslo so zápisom $\overline{abc}$ má požadovanú vlastnosť
práve vtedy, keď jeho číslice $a$, $b$, $c$ spĺňajú rovnosť
$$
100a+10b+c=19(a+b+c),\quad\text{čiže}\quad   9a=b+2c.
$$
Pretože $b\leqq9$ a~$c\leqq9$, platí nerovnosť $b+2c\leqq27$. 
Z~rovnosti $9a=b+2c$ preto vyplýva odhad $a\leqq3$, takže platí
$a\in\{1,2,3\}$ (číslica $a=0$ nie je na začiatku zápisu dovolená).
Pre $a=1$ dostávame rovnicu $9=b+2c$, z~ktorej vyplýva $c\leqq4$;
pre každé také $c\in\{0,1,2,3,4\}$ je  číslica~$b$ určená
rovnosťou $b=9-2c$. Preto s~číslicou $a=1$ existuje práve
5~vyhovujúcich čísel. Práve toľko je aj vyhovujúcich čísel s~číslicou
$a=2$, z~rovnice $18=b+2c$ totiž vyplýva $c\in\{5,6,7,8,9\}$ 
a~$b=18-2c$. Nakoniec pre $a=3$ z~rovnice $27=b+2c$ vyplýva $b=c=9$.
Hľadaný počet čísel je teda $5+5+1=11$.

\ineriesenie
Súčet číslic ľubovoľného trojciferného čísla neprevyšuje
číslo~27, ktorého devätnásťnásobok je 513. Preto každé vyhovujúce
číslo neprevyšuje 513, takže súčet jeho číslic je najviac
$4+9+9=22$. Pretože najmenší trojciferný násobok čísla~19 je číslo
$114=19\cdot6$, bude úloha vyriešená, keď zistíme, koľko čísel
tvaru $19s$, pričom $s\in\{6,7,8,\dots,22\}$, má súčet číslic rovný
práve číslu~$s$. Triviálnym preverením zistíme, že z~uvedených 17~čísel
vyhovujú práve čísla 114, 133, 152, 171, 190, 209, 228,
247, 266, 285 a~399. Týchto čísel je~11.}

{%%%%%   C-S-2
\fontplace
\tpoint A; \tpoint B; \bpoint C; \bpoint D;
\tpoint E; \tpoint F; \bpoint G; \bpoint H;
\tpoint 2\,{\text{cm}};
[13] \hfil\Obr

\fontplace
\tpoint A; \tpoint B; \bpoint C; \bpoint D;
\tpoint E; \tpoint F; \rpoint G; \rpoint H;
\tpoint 2\,{\text{cm}};
[14] \hfil\Obr

\fontplace
\tpoint A=E; \tpoint B; \bpoint C; \bpoint D;
\tpoint F; \lpoint G; \rpoint H;
\bpoint 2\,{\text{cm}};
\lpoint x; \tpoint y;
[15] \hfil\Obr

Štvoruholník $EFGH$ môžeme do daného štvorca $ABCD$ umiestniť tromi
spôsobmi.

\smallskip
{\it Prvý spôsob\/}. Dve strany dĺžky 2\,cm ležia na protiľahlých stranách daného
štvorca (\obr). Obsah každého takého štvoruholníka
(rovnobežníka) je $S=5\cdot2\,{\text{cm}}^2=10\,{\text{cm}}^2$.

\midinsert
\centerline{\inspicture-!\hss\inspicture-!}
\endinsert 

\smallskip
{\it Druhý spôsob\/}. Obe strany dĺžky 2\,cm ležia na susedných stranách daného štvorca
a~pritom sú protiľahlými stranami štvoruholníka $EFGH$ (\obr).
Obsah takého štvoruholníka je
$$
\align
S=&\frac12|EF|\cdot|AG|+\frac12|GH|\cdot|AE|=
% \\=&
\frac12\cdot2\,{\text{cm}}\cdot|AG|+\frac12\cdot2\,{\text{cm}}\cdot|AE|\le
\\ \le&
\bigl(5+(5-2)\bigr)\,{\text{cm}}^2=8\,{\text{cm}}^2<10\,{\text{cm}}^2.
\endalign
$$

\smallskip
{\it Tretí spôsob\/}. Obe strany dĺžky 2\,cm ležia na susedných stranách daného
štvorca a~pritom sú susednými stranami štvoruholníka $EFGH$
\inspicture{}
(\obr). Ak označíme postupne $x$ a~$y$ vzdialenosti bodu~$G$ od
strán $AB$ a~$AD$ (teda výšku trojuholníka $EFG$ na stranu~$EF$ a~výšku
trojuholníka $EHG$ na stranu~$EH$), je obsah takého štvoruholníka
$$
S=\frac12|EF|\cdot x+\frac12|AH|\cdot y\le
  2\cdot\frac12\cdot2\cdot5=10\,{\text{cm}}^2.
$$
Rovnosť nastane práve vtedy, keď $x=y=5$\,cm, \tj. práve vtedy, keď
$G=C$.

\zaver
Najväčší možný obsah (10\,cm$^2$) majú všetky
rovnobežníky, ktorých dve strany dĺžky 2\,cm ležia na protiľahlých
stranách daného štvorca a~štyri deltoidy, ktorých jedna
uhlopriečka je zároveň uhlopriečkou daného štvorca.}

{%%%%%   C-S-3
Predpokladajme, že štvorec so stranou 6~jednotiek je vydláždený
$a$~dlaždicami typu~A a~$b$~dlaždicami typu~$B$ (nevylučujeme prípad, že
$a=0$ alebo $b=0$). Pre obsah vydláždenej plochy potom platí
rovnosť $36=3a+4b$, z~ktorej vyplýva, že číslo~$a$ je násobkom štyroch
(a~číslo~$b$ násobkom troch). Preto má rovnica $36=3a+4b$ v~obore
celých nezáporných čísel ako riešenia iba tieto dvojice $(a,b)$:
$(0,9)$, $(4,6)$, $(8,3)$ a~$(12,0)$. Zistíme ďalej,
či pre jednotlivé dvojice $(a,b)$ je príslušné
vydláždenie daného štvorca možné. %\looseness-1

\ite (i) 9~dlaždíc~B. Vysvetlíme, prečo také vydláždenie
neexistuje. Ofarbime jednotkové štvorčeky celého štvorca ako
zvyčajnú šachovnicu. Získame 18~čiernych a~18~bielych "políčok".
Každá dlaždica~B pokrýva tri políčka jednej farby a~jedno políčko druhej
farby. Pripusťme, že celý štvorec pokrýva 9~dlaždíc~B, pritom
práve $x$ z~nich má tú vlastnosť, že pokrývajú po 3~čierne
políčka, takže $9-x$ z~nich má tú vlastnosť, že pokrývajú po
1~čiernom políčku. Pre celkový počet čiernych políčok potom platí rovnosť
$18=3x+(9-x)$, odkiaľ $x=9/2$, čo je spor.

\ite (ii) 4~dlaždice~A a~6~dlaždíc~B. Možné riešenie vidno na
\obr.

\ite (iii) 8~dlaždíc~A a~3~dlaždice~B. Možné riešenie vidno na
\obr.

\ite (iv) 12~dlaždíc~A. Možné riešenie vidno na \obr.

$$
\vbox{\halign{# \qquad&\qquad # \qquad&\qquad # \cr
\hfil\hbox{\epsfbox{C53.16}}\hfil & \hfil\hbox{\epsfbox{C53.17}}\hfil & \hfil\hbox{\epsfbox{C53.18}}\hfil \cr
 &&\cr
 \hfil\Obr\hfil & \hfil\Obr\hfil & \hfil\Obr\hfil  \cr}}
$$

\poznamka
Uveďme ešte iný argument, prečo nemožno deviatimi
dlaždicami~B vyplniť uvažovaný štvorec. Dlaždica, ktorá pokrýva
rohové políčko, môže byť umiestnená (až na súmernosť podľa uhlopriečky
štvorca) jediným spôsobom, napr\. tak ako dlaždica~B v~ľavom
dolnom rohu štvorca na \obrr2. Potom ale dlaždica~B, ktorá
v~takom prípade pokrýva druhé políčko zľava v~dolnom riadku, musí byť
v~polohe ako na obrázku. Posledné dve políčka dolného riadku potom už
jednou ani dvoma dlaždicami~B pokryť nemožno.}

{%%%%%   C-II-1
\fontplace
\tpoint A; \tpoint B; \bpoint C; \bpoint D;
\lpoint K; \bpoint L; \tpoint a; \rpoint b;
\cpoint\a; \cpoint\b;
[19] \hfil\Obr

V~pravouhlom trojuholníku $ABK$ označme
$\al=|\uhol BAK|$, $\be=|\uhol AKB|=90\st-\al$ (\obr).
\inspicture{}
Rovnaké vnútorné uhly $90\st$, $\al$, $\be$ majú aj trojuholníky $AKL$
a~$ADL$, lebo sú podľa zadania s~trojuholníkom $ABK$ podobné. Všimnime si
ich (ostré) uhly pri spoločnom vrchole~$A$. Pretože $|\uhol
KAD|=90\st-\al=\be$, sú oba uhly $KAL$ a~$LAD$ menšie ako $\be$,
takže sa rovnajú uhlu~$\al$. Pravý uhol $BAD$ je teda
polpriamkami $AK$, $AL$ rozdelený na tri zhodné uhly veľkosti~$\al$,
odkiaľ $\al=30\st$ (a~$\be=60\st$). Z~pravouhlých trojuholníkov
$ADL$ a~$ABK$ potom vyplýva, že $|AK|=|AB|/\cos30\st=2a/\sqrt3$
a~$|AL|=|AD|/\cos30\st=2b/\sqrt3$. Odtiaľ vzhľadom na podmienku
$a<b$ vyplýva nerovnosť $|AK|<|AL|$, teda preponou v~trojuholníku $AKL$
je $AL$ (dlhšia z~oboch strán $AK$, $AL$). Pre pomer dĺžok odvesny
$AK$ a~prepony~$AL$ potom platí $\cos30\st=|AK|:|AL|=a:b$, takže
$a:b=\sqrt{3}:2$.

\smallskip
Úlohu možno riešiť viacerými obmenenými postupmi, napríklad rozlíšiť dva
prípady, keď trojuholník $KAL$ má pravý uhol pri vrchole~$K$ respektíve
$L$, a~v~každom z~nich vyjadriť vnútorné uhly všetkých štyroch podobných
trojuholníkov (v~druhom prípade vtedy ale vyjde $a:b=2:\sqrt3>1$, čo
odporuje zadaniu úlohy).}

{%%%%%   C-II-2
Všimnime si najskôr, že pre čitatele zlomkov z~danej rovnice platí
vzťah $14+51=65$. Preto je riešením každá trojica rovnakých
prvočísiel $p=q=r$ a~navyše pre ľubovoľné riešenie platí: ak sú
niektoré dve z~čísel $p$, $q$, $r$ rovnaké, je rovnaké aj tretie
číslo. Budeme teda ďalej predpokladať, že prvočísla $p$, $q$, $r$
spĺňajúce danú rovnicu sú navzájom rôzne (a~teda navzájom
nesúdeliteľné).

Po vynásobení rovnice súčinom $pqr$ dostaneme
$$
14qr+51pr=65pq,
$$
odkiaľ vzhľadom na spomenutú nesúdeliteľnosť vyplýva
$$
p\mid 14=2\cdot7,\quad q\mid51=3\cdot17\quad\text{a}\quad
r\mid65=5\cdot13.
$$
To znamená, že $p\in\{2,7\}$, $q\in\{3,17\}$ a~$r\in\{5,13\}$.
Teraz môžeme utvoriť a~do rovnice dosadiť všetkých osem možných
trojíc $(p,q,r)$. Zistíme tak, že vyhovuje jedine trojica
$(7,17,13)$.

\smallskip
Overenie dosadzovaním môžeme skrátiť tak, že vylúčime niektorú
z~hodnôt $p=2$, $q=3$, resp. $r=5$. Napríklad po dosadení $r=5$
dostaneme po vydelení piatimi rovnicu $14q+51p=13pq$, ktorá nemá
celočíselné riešenie~$p$ ani pre $q=3$ ($14+17p=13p$), ani pre
$q=17$ ($14+3p=13p$). Iná možnosť: z~rovnice $14qr+51pr=65pq$
vyplýva $2p(q-r)=7(2qr+7pr-9pq)$, takže súčin $p(q-r)$ je
deliteľný siedmimi. Pretože však $q\in\{3,17\}$ a~$r\in\{5,13\}$,
nie je rozdiel $q-r$ deliteľný siedmimi, preto je siedmimi
deliteľné číslo~$p$. Podobne možno zdôvodniť, prečo $17\mid q$
a~$13\mid r$.

\ineriesenie
Z~danej rovnice vyjadríme $r$ pomocou $p$ a~$q$:
$$
r=\frac{65pq}{51p+14q}=\frac{5\cdot13\cdot p\cdot q}{51p+14q}.
$$
V~ostatnom zlomku sme zvýraznili rozklad čitateľa na (štyri)
prvočinitele. Taký zlomok bude rovný niektorému prvočíslu~$r$ práve
vtedy, keď jeho menovateľ bude súčinom troch prvočiniteľov z~čitateľa
(iné krátenie zlomku nie je možné).
Hľadáme teda situácie, keď platí niektorý z~prípadov
$$
\vcenter{\let\\=\cr \openup\jot
\halign{\hss$#$&${}#$&#&$#$&$#$&${}#$\hss\cr
51p+14q&= 5\cdot13\cdot p&\quad&\text{a}&\quad r&=q,\\
51p+14q&= 5\cdot13\cdot q&\quad&\text{a}&\quad r&=p,\\
51p+14q&= 5\cdot p\cdot q&\quad&\text{a}&\quad r&=13,\\
51p+14q&=13\cdot p\cdot q&\quad&\text{a}&\quad r&=5.\cr
}}
$$
Jednoduchou úpravou rovníc zistíme, že prvé dva prípady nastanú
iba v~situácii, keď $p=q$ (vtedy tiež $p=r$).
Posledné dva prípady vedú k~vyjadreniam
$$
q=\frac{3\cdot17\cdot p}{5p-14},\quad\text{resp.}\quad
q=\frac{3\cdot17\cdot p}{13p-14},
$$
z~ktorých analogickou úvahou o~krátení zlomkov (prípad $p=q$ už
môžeme vynechať) s~prihliadnutím k~zrejmým nerovnostiam $5p-14<17p$
a~$13p-14<17p$ dostaneme rovnice
$$
5p-14=3p,\quad\text{resp.}\quad 13p-14=3p.
$$
Prvá rovnica má riešenie $p=7$ (ktorému zodpovedá $q=17$ a~$r=13$),
druhá rovnica celočíselné riešenie nemá.

\odpoved
Všetky riešenia $(p,q,r)$ sú trojice $(p,p,p)$,
kde $p$ je ľubovoľné prvočíslo a~trojica $(7,17,13)$.}

{%%%%%   C-II-3
\fontplace
\tpoint A; \tlpoint B; \lpoint C; \lBpoint D;
\bpoint E; \rBpoint F; \rpoint G; \rtpoint H;
\bpoint\xy2.6,.5 S;
\bpoint3; \rBpoint x; \rpoint4; \trpoint x;
\tpoint5; \ltpoint x; \lpoint6; \lBpoint x;
[21]

\fontplace
\tpoint K; \tlpoint L; \lpoint M; \lpoint N;
\bpoint O; \rBpoint P; \rpoint Q; \trpoint R;
\lbpoint\xy-.4,.4 T;
\bpoint3; \rBpoint 4; \rpoint5; \trpoint6;
\tlpoint x; \lpoint x; \lpoint x; \bpoint x;
[20]

{\it Rozbor\/}. Okrem hľadaného osemuholníka $ABCDEFGH$ uvažujme
ešte pomocný osemuhol\-ník $KLMNOPQR$, ktorý je tiež vpísaný do
kružnice s~polomerom $r=6$ a~ktorého strany spĺňajú podmienky
$|KL|=3$, $|LM|=4$, $|MN|=5$, $|NO|=6$, $|OP|=|PQ|=|QR|=|RK|$ (\obr).
\vadjust{\bigskip
         \centerline{\inspicture-!\hss\inspicture-!}
         \centerline\Obr
         \bigskip}%
Označme $S$, resp.~$T$ stred kružnice s~vpísaným osemuholníkom
$ABCDEFGH$, resp.~$KLMNOPQR$.
Podľa vety $sss$ platia zhodnosti 
$$\Delta ABS\cong\Delta KLT,\quad
  \Delta CDS\cong\Delta LMT,\quad
  \Delta EFS\cong\Delta MNT,\quad
  \Delta GHS\cong\Delta NOT,
$$
a~preto sú zhodné stredové uhly
$ASB$ a~$KTL$, $CSD$ a~$LTM$, $ESF$ a~$MTN$, $GSH$ a~$NTO$. Ďalej
podľa vety $sss$ sú zhodné trojuholníky $BCS$, $DES$, $FGS$ a~$HAS$,
rovnako ako trojuholníky $OPT$, $PQT$, $QRT$ a~$RKT$. Zo zhodnosti
ich uhlov pri hlavnom vrchole~$S$, resp.~ $T$ preto vyplýva
$$
\align
|\uhol BSC|&=\frac1{4}({360\st-|\uhol ASB|-|\uhol CSD|-|\uhol
             ESF|-|\uhol GSH|})=\\
&=\frac1{4}({360\st-|\uhol KTL|-|\uhol LTM|-|\uhol MTN|-|\uhol
        NTO|})=\\
&=|\uhol OTP|.
\endalign
$$

Využili sme to, že stredy $S$ a~$T$ sú {\it vnútorné\/}
body oboch osemuholníkov (teda súčet všetkých ôsmich stredových uhlov
je v~oboch prípadoch $360\st$), lebo v~opačnom prípade by jeden 
z~ôsmich stredových uhlov bol rovný súčtu siedmich ostatných; musel by
to byť uhol prislúchajúci tetive dĺžky~6, ten je však zrejme menší
ako súčet uhlov prislúchajúcich tetivám dĺžok~3, 4 a~5. Trojuholníky
$BCS$ a~$OPT$ sú preto zhodné podľa vety $sus$, takže štvorice
zhodných strán oboch osemuholníkov majú jednu spoločnú dĺžku.
Ak teda dokážeme zostrojiť pomocný osemuholník $KLMNOPQR$, bude už
konštrukcia osemuholníka $ABCDEFGH$ jednoduchá.

\smallskip
{\it Konštrukcia\/}. Na ľubovoľnej kružnici $t(T;6)$
zostrojíme v~jednom smere body $K$, $L$, $M$, $N$ a~$O$ tak, aby
$|KL|=3$, $|LM|=4$, $|MN|=5$ a~$|NO|=6$. Uhol $KTO$ (ten, ktorý
neobsahuje body~$L$, $M$, $N$)
potom rozdelíme na štyri zhodné diely: najskôr zostrojíme
priesečník~$Q$ kružnice~$t$ s~osou uhla $KTO$, potom priesečníky~$P$,
$R$ kružnice~$t$ s~osami uhlov $OTQ$ resp. $QTK$. Potom pristúpime
ku~konštrukcii hľadaného osemuholníka $ABCDEFGH$: na kružnici
$k(S,6)$ zvolíme bod $A$ a~na nej v~jednom smere zostrojíme
postupne body $B$, $C$, $D$, $E$, $F$, $G$, $H$ tak, aby $|AB|=3$,
$|BC|=|OP|$, $|CD|=4$, $|DE|=|OP|$, $|EF|=5$, $|FG|=|OP|$,
$|GH|=6$.

\smallskip
{\it Dôkaz správnosti\/}. Zo zhodnosti siedmich dvojíc
$$
\gather
\Delta ABS\cong\Delta KLT,\quad 
\Delta BCS\cong\Delta OPT,\quad
\Delta CDS\cong\Delta LMT,\quad
\Delta DES\cong\Delta QRT,\\
\Delta EFS\cong\Delta MNT,\quad
\Delta FGS\cong\Delta QRT,\quad
\Delta GHS\cong\Delta NOT
\endgather
$$ 
vyplýva zhodnosť uhlov $HSA$ a~$RTK$,
a~teda aj zhodnosť ôsmej dvojice trojuholníkov $HAS$ a~$RKT$.
Preto majú dĺžky strán zostrojeného osemuholníka $ABCDEFGH$
(zhodné so stranami $KLMNOPQR$) všetky potrebné vlastnosti.

\poznamka
O~osemuholníku $KLMNOPQR$ sme nemuseli v~celom
riešení vôbec hovoriť a~mohli sme úvahy robiť nasledovne. Uhly zhodné so stredovými
uhlami $ASB$, $CSD$, $ESF$, $GSH$ dokážeme zostrojiť. Pre spoločnú
veľkosť~$\om$ zhodných stredových uhlov $BSC$, $DSE$, $FSG$
a~$HSA$ potom platí rovnica
$$
4\om+|\uhol ASB|+|\uhol CSD|+|\uhol ESF|+|\uhol GSH|=360\st,
\tag1
$$
ktorú možno ľahko konštrukčne vyriešiť. Osemuholník $KLMNOPQR$ je
však pre tento účel ideálnou pomôckou.}

{%%%%%   C-II-4
Martin vypočítal hodnotu $(x+y)z$ namiesto $x+yz$, takže podľa
zadania platí
$$
(x+y)z-(x+yz)=2\,004,\quad\text{čiže}\quad x\cdot(z-1)=
2\,004=12\cdot167,
$$
pričom $167$ je prvočíslo. Činitele $x$ a~$z-1$ určíme, keď si
uvedomíme, že $z$ je dvojciferné číslo, takže $9\leqq z-1\leqq
98$. Vidíme, že nutne $z-1=12$ a~$x=167$, odkiaľ $z=13$. Martin
teda vypočítal číslo $V=(167+y)\cdot13$. Číslo~$V$ je preto
štvorciferné, a~pretože sa číta spredu rovnako ako zozadu, má tvar
$\overline{abba}=1\,001a+110b$. Pretože $1\,001=13\cdot77$, musí platiť
rovnosť $(167+y)\cdot13=13\cdot77a+110b$, z~ktorej vyplýva, že
číslica~$b$ je deliteľná trinástimi, takže $b=0$. Po dosadení dostaneme (po
delení trinástimi) rovnosť $167+y=77a$, ktorá vzhľadom na
nerovnosti $10\leqq y\leqq 99$ znamená, že číslica~$a$ sa
rovná~3, takže $y=64$.

V~druhej časti riešenia sme mohli postupovať aj nasledovne.
Pre číslo $V=(167+y)\cdot13$ vychádzajú z~nerovností
$10\leqq y\leqq 99$ odhady $2\,301\leqq V\leqq3\,458$.
Zistíme preto, ktoré
z~čísel $\overline{2bb2}$, kde $b\in\{3,4,5,6,7,8,9\}$ a~čísel $\overline{3bb3}$,
kde $b\in\{0,1,2,3,4\}$, sú deliteľné trinástimi. Aj keď sa
týchto dvanásť čísel dá rýchlo otestovať, urobme to
všeobecne ich čiastočným vydelením trinástimi:
$$
\align
\overline{2bb2}&=2\,002+110b=13\cdot(154+8b)+6b,\\
\overline{3bb3}&=3\,003+110b=13\cdot(231+8b)+6b.
\endalign
$$
Vidíme, že vyhovuje jedine číslo $\overline{3bb3}$ pre $b=0$.
Vtedy $167+y=231$, takže $y=64$.

\odpoved
Žiaci mali počítať príklad $167+64\cdot13$, teda
$x=167$, $y=64$ a~$z=13$.}

{%%%%%   vyberko, den 1, priklad 1
...}

{%%%%%   vyberko, den 1, priklad 2
...}

{%%%%%   vyberko, den 1, priklad 3
...}

{%%%%%   vyberko, den 1, priklad 4
...}

{%%%%%   vyberko, den 2, priklad 1
...}

{%%%%%   vyberko, den 2, priklad 2
...}

{%%%%%   vyberko, den 2, priklad 3
...}

{%%%%%   vyberko, den 2, priklad 4
...}

{%%%%%   vyberko, den 3, priklad 1
...}

{%%%%%   vyberko, den 3, priklad 2
...}

{%%%%%   vyberko, den 3, priklad 3
...}

{%%%%%   vyberko, den 3, priklad 4
...}

{%%%%%   vyberko, den 4, priklad 1
...}

{%%%%%   vyberko, den 4, priklad 2
...}

{%%%%%   vyberko, den 4, priklad 3
...}

{%%%%%   vyberko, den 4, priklad 4
...}

{%%%%%   vyberko, den 5, priklad 1
...}

{%%%%%   vyberko, den 5, priklad 2
...}

{%%%%%   vyberko, den 5, priklad 3
...}

{%%%%%   vyberko, den 5, priklad 4
...}

{%%%%%   trojstretnutie, priklad 1
Tvrdenie úlohy v~sebe zahŕňa dve implikácie. Dokážeme najskôr jednu a~potom druhú.

\smallskip
{\it Prvá časť\/}. Nech kvadratické rovnice zo zadania majú reálne
korene spĺňajúce $x_1y_1-x_2y_2=1$. Podľa známeho vzťahu majú tieto
korene vyjadrenia
$$
x_{1,2}=\frac{\m p\pm K}{2}\qquad\text{a}\qquad
y_{1,2}=\frac{p\pm L}{2},                        \tag1
$$
pričom reálne
čísla $K$, $L$ spĺňajú rovnosti $K^2=p^2-4q$ a~$L^2=p^2-4r$
(číslam $K$, $L$ priradíme znamienka podľa očíslovania koreňov). Potom
$$
1=x_1y_1-x_2y_2=\frac{(\m p+K)(p+L)-(\m p-K)(p-L)}{4}=
\frac{p(K-L)}{2},
$$
odkiaľ $p\ne0$ a~$K-L=2/p$. Dosadením do
rovnosti
$$
(K+L)(K-L)=K^2-L^2=(p^2-4q)-(p^2-4r)=4(r-q)
$$
vyjde $K+L=2p(r-q)$. Zo získaných vyjadrení
čísel $K+L$ a~$K-L$ dostaneme $K=1/p-p(q-r)$, po umocnení
$K^2=1/p^2-2(q-r)+p^2(q-r)^2$. Keď to porovnáme s~rovnosťou
$K^2=p^2-4q$, obdržíme po jednoduchej úprave žiadanú rovnosť zo zadania.

\smallskip
{\it Druhá časť\/}. Nech reálne čísla $p$, $q$, $r$ spĺňajú prvú rovnosť zo zadania. Potom zrejme $p\ne0$.
Danú rovnosť upravíme dvoma podobnými spôsobmi na tvary
$$
p^4(r-q)^2+2p^2(r-q)+1=p^4-4p^2q
\quad\text{a}\quad
p^4(q-r)^2+2p^2(q-r)+1=p^4-4p^2r.
$$
Odtiaľ po vydelení číslom~$p^2$ zisťujeme, že diskriminanty
kvadratických rovníc zo zadania majú vyjadrenia
$$
p^2-4q=\left(\frac{p^2(r-q)+1}{p}\right)^{\!\!2}
\quad\text{a}\quad
p^2-4r=\left(\frac{p^2(q-r)+1}{p}\right)^{\!\!2},
$$
takže to sú nezáporné čísla a~príslušné (reálne) korene majú tvar~\thetag{1}, pričom
$$
K=\frac{p^2(r-q)+1}{p}\quad\text{a}\quad
L=\m\frac{p^2(q-r)+1}{p}.
$$
Znamienka čísel $K$ a~$L$ sme zvolili tak, aby vyšlo (pozri prvú časť)
$$
x_1y_1-x_2y_2=\frac{p(K-L)}{2}=\frac{p}{2}\cdot
\left(\frac{p^2(r-q)+1}{p}
+\frac{p^2(q-r)+1}{p}\right)=1.
$$}

{%%%%%   trojstretnutie, priklad 2
Navzájom rôzne prvočísla $p$, $q$, $r$ vyhovujú podmienkam
úlohy práve vtedy, keď číslo $pq+pr+qr-k$ je deliteľné každým z~čísel
$p$, $q$, $r$, čiže ich súčinom $pqr$. Rovnosť
$pq+pr+qr-k=n\cdot pqr$ pre vhodné celé~$n$ prepíšme na
$k=pq+pr+qr-n\cdot pqr$. Ak $n\leqq0$, vyplýva z~ostatnej
rovnosti, že $\max\{pq,pr,qr\}\leqq k$. Potom však každé
z~prvočísel $p$, $q$, $r$ je najviac $k/2$ a~takých trojíc
je konečný počet. Ak $n\geqq1$, dostávame odhad $k\leqq
pq+pr+qr-pqr$. Ukážme, že s~výnimkou trojice
$\{p,q,r\}=\{2,3,5\}$ je ostatný výraz vždy záporný (čo bude v~spore
s~tým, že $k>0$). V~takom prípade môžeme určite
predpokladať, že $2\leqq p<q<r$ a~$r\geqq7$. Potom
$pq\geqq2\cdot3=6$ a~z~nerovnosti $(p-2)(q-2)\geqq0$ vyplýva
$p+q\leqq pq/2+2$, takže
$$
\gather
pq+pr+qr-pqr=(p+q)r+pq-pqr\leqq(\tfrac12 pq+2)r+pq-pqr=\\
=2r-pq(\tfrac12r-1)\leqq 2r-6(\tfrac12r-1)=6-r<0.
\endgather
$$}

{%%%%%   trojstretnutie, priklad 3
\fontplace
\bpoint\xy.6,.5 P;
\rtpoint A; \tpoint\xy.5,0 B; \lBpoint C; \rBpoint D;
\tpoint E; \rpoint F; \bpoint G;
\lpoint T; \rpoint Q;
\tlpoint k; \tpoint k_1; \rBpoint k_2;
[1] \hfil\Obr

Označme $k$ kružnicu opísanú štvoruholníku $ABCD$
a~$k_1$, $k_2$ kružnice opísané trojuholníkom $PAB$, $PCD$.
\inspicture{}
Vnútri uhla $BPC$ uvažujme takú polpriamku~$PT$, pre ktorú platí
$|\uhol BPT|=|\uhol BAP|$. Podľa zadania potom platí (\obr)
$$
|\uhol CPT|=|\uhol BPC|-|\uhol BPT|
=|\uhol BPC|-|\uhol BAP|=|\uhol PDC|.
$$
Priamka $PT$ je teda spoločnou vnútornou dotyčnicou oboch kružníc $k_1$
a~$k_2$.

Uvažujme najskôr prípad, keď strany $AB$ a~$CD$ uvažovaného
tetivového štvoruholníka nie sú rovnobežné. Vzhľadom na to, že
úsečky $AB$ a~$CD$ sú spoločnými tetivami prislúchajúcich dvojíc
kružníc $k_1$, $k$ a~$k_2$, $k$, existuje jediný bod~$Q$, ktorý
má rovnakú mocnosť ku všetkým trom kružniciam $k$, $k_1$ a~$k_2$.
Týmto bodom~$Q$ je spoločný bod všetkých troch priamok (chordál) $AB$,
$CD$ a~$PT$. Bez ujmy na všeobecnosti predpokladajme, že bod~$Q$
leží na polpriamke~$BA$ za bodom~$A$ (\obrr1).

Podľa Tálesovej vety sú zrejme štvoruholníky $AEPF$, $FPGD$ 
a~$QEPG$ tetivové. Z~rovností príslušných obvodových uhlov tak vyplýva
$$
\align
|\uhol EFG|=&|\uhol EFP|+|\uhol GFP|=|\uhol BAP|+|\uhol PDC|=\\
            =&|\uhol BPT|+|\uhol CPT|=|\uhol BPC|.
\endalign
$$
Podobne zistíme, že
$$
\align
|\uhol FEG|=&|\uhol FEP|-|\uhol GEP|=|\uhol FAP|-|\uhol GQP|=\\
            =&|\uhol DAP|-|\uhol DQP|=|\uhol QDA|-|\uhol QPA|,
\endalign
$$
lebo $|\uhol DAP|+|\uhol QPA|=|\uhol QDA|+|\uhol DQP|$.
Pre úsekový uhol $QPA$ navyše platí $|\uhol QPA|=|\uhol PBA|$,
takže
$$
\align
|\uhol FEG|=&|\uhol QDA|-|\uhol QPA|=|\uhol QDA|-|\uhol PBA|=\\
            =&|\uhol QBC|-|\uhol PBA|=|\uhol PBC|.
\endalign
$$
Tým sme dokázali, že trojuholníky $FEG$ a~$PBC$ sa zhodujú v~dvoch
vnútorných uhloch, a~sú teda podobné ($uu$).

Ak sú priamky $AB$ a~$CD$ rovnobežné, je $ABCD$ rovnoramenný
lichobežník so základňami $AB$ a~$CD$. Odtiaľ vyplýva, že body~$E$,
$P$, $G$ ležia na osi súmernosti lichobežníka $ABCD$
a~trojuholníky $APD$ a~$BPC$ sú zhodné. Z~vlastností obvodových uhlov
tetivových štvoruholníkov $AEPF$ a~$FPGD$ ľahko vyplýva, že
trojuholníky $EFG$ a~$APD$ sú podobné ($|\uhol
FEG|=|\uhol PAD|$ a~$|\uhol EGF|=|\uhol ADP|$). Odtiaľ už vyplýva
podobnosť trojuholníkov $FEG$ a~$PBC$. Tým je dôkaz hotový.}

{%%%%%   trojstretnutie, priklad 4
Z~tvaru rovníc vyplýva podmienka $xyz\ne0$. Aspoň dve z~čísel $x$, $y$,
$z$ musia mať rovnaké znamienko. Potom je kladná pravá strana rovnice,
v~ktorej sú tieto dve čísla v~podiele, preto je kladná aj príslušná
ľavá strana, takže zostávajúce z~čísel $x$, $y$, $z$ má rovnaké znamienko
ako prvé dve. Platí teda buď $x,y,z>0$, alebo $x,y,z<0$.

Zaoberajme sa iba prvým prípadom, druhý sa totiž prevedie na
prvý zmenou riešenia $(x,y,z)$ na riešenie $(\m x,\m y,\m z)$. Prvé
dve rovnice sústavy vynásobme výrazom $xyz$ a~potom ich odčítajme.
Po úprave dostaneme $z-x=y(x^2-yz)$. Ak je trojica $(x,y,z)$
riešením, sú riešeniami aj trojice $(y,z,x)$ a~$(z,x,y)$, ktoré
dostaneme cyklickou zámenou. Preto môžeme predpokladať, že
$x=\max\{x,y,z\}$. Potom $z-x\leqq0$ a~$x^2-yz\geqq0$
(nezabúdajme, že $x,y,z>0$), takže z~rovnosti $z-x=y(x^2-yz)$
a~podmienky $y>0$ vyplýva $z-x=x^2-yz=0$, čo znamená $x=y=z$. Máme
teda jedinú rovnicu $1/x^2=1+1$, ktorá má (jediný) kladný koreň
$x=\sqrt2/2$.

\odpoved
Sústava má práve dve riešenia $x=y=z=\pm\sqrt2/2$.}

{%%%%%   trojstretnutie, priklad 5
\def\ve#1#2{\vv{#1#2}}%
Bod~$V$ roviny trojuholníka $ABC$ je priesečníkom jeho výšok práve vtedy,
keď platí zároveň $AV\perp BC$ a~$BV\perp AC$, čiže
$$
\ve{A}{V}\cdot\ve{B}{C}=0\qquad\text{a}\qquad\ve{B}{V}\cdot\ve{A}{C}=0.
$$
Po dosadení $\ve{B}{C}=\ve{B}{V}-\ve{C}{V}$, $\ve{A}{C}=\ve{A}{V}-\ve{C}{V}$ a~jednoduchej úprave
dostaneme ekvivalentnú podmienku vo forme rovnosti skalárnych
súčinov
$$
\ve{A}{V}\cdot\ve{B}{V}=\ve{A}{V}\cdot\ve{C}{V}=
\ve{B}{V}\cdot\ve{C}{V}.
\tag1
$$
Našou úlohou je teda zistiť, kedy platí sústava~\thetag{1} zároveň
s~podobnou sústavou
$$
\ve{K}{V}\cdot\ve{L}{V}=\ve{K}{V}\cdot\ve{M}{V}=
\ve{L}{V}\cdot\ve{M}{V},
\tag2
$$
ktorá vyjadruje, že bod~$V$ je priesečníkom výšok trojuholníka $KLM$.
Vyjadríme vektory z~\thetag{2} ako lineárne kombinácie vektorov z~\thetag{1}.
Podľa zadania existuje číslo~$p$, $0<p<1$, pre ktoré platí
$$
\ve{A}{K}=p\,\ve{A}{B},\quad
\ve{B}{L}=p\,\ve{B}{C},\quad
\ve{C}{M}=p\,\ve{C}{A}.
$$
Keď do prvej rovnosti dosadíme $\ve{A}{K}=\ve{A}{V}-\ve{K}{V}$
a~$\ve{A}{B}=\ve{A}{V}-\ve{B}{V}$, dostaneme po úprave
prvú z~rovností
$$
\ve{K}{V}=(1-p)\ve{A}{V}+p\,\ve{B}{V},\
\ve{L}{V}=(1-p)\ve{B}{V}+p\,\ve{C}{V},\
\ve{M}{V}=(1-p)\ve{C}{V}+p\,\ve{A}{V}.
$$
Druhé dve rovnosti odvodíme analogicky. Odtiaľ vynásobením
dostaneme
$$
\align
\ve{K}{V}\cdot\ve{L}{V}&=(1-p)^2\ve{A}{V}\cdot\ve{B}{V}+
p(1-p)\ve{A}{V}\cdot\ve{C}{V}+p(1-p)\ve{B}{V}^2=\\
&=(1-p)s+p(1-p)\ve{B}{V}^2,
\endalign
$$
kde písmeno~$s$ označuje spoločnú hodnotu súčinov z~\thetag{1}.
Analogicky platí
$$
\ve{K}{V}\cdot\ve{M}{V}=(1-p)s+p(1-p)\ve{A}{V}^2\quad\text{a}
\quad
\ve{L}{V}\cdot\ve{M}{V}=(1-p)s+p(1-p)\ve{B}{V}^2.
$$
Vidíme, že sústava~\thetag{2} je ekvivalentná so sústavou rovností
$$
p(1-p)\ve{A}{V}^2=p(1-p)\ve{B}{V}^2=p(1-p)\ve{C}{V}^2,
$$
ktorá je vzhľadom na $p(1-p)\ne0$ splnená práve vtedy, keď
$|AV|=|BV|=|CV|$. Táto podmienka znamená, že priesečník výšok~$V$
trojuholníka $ABC$ splýva so stredom kružnice opísanej. To nastane práve vtedy,
keď je trojuholník $ABC$ rovnostranný.

\ineriesenie
V~prvej časti riešenia predpokladajme, že $ABC$ je rovnostranný
trojuholník, a~označme $O$ stred kružnice opísanej tomuto trojuholníku. Pri jednej
z~rotácií o~$120{\st}$ okolo bodu~$O$ platí $A\mapsto B\mapsto
C\mapsto A$ a~tiež $K\mapsto L\mapsto M\mapsto K$, lebo
napríklad body $K$, resp.~$L$ delia v~rovnakom pomere úsečku~$AB$,
resp.~úsečku~$BC$, ktorá je obrazom prvej úsečky v~spomenutej
rotácii. To znamená, že aj trojuholník $KLM$ je rovnostranný a~bod~$O$
je ortocentrom oboch trojuholníkov $ABC$ a~$KLM$.

V~druhej časti riešenia predpokladajme, že $ABC$ nie je rovnostranný
trojuholník. Potom stred~$O$ kružnice opísanej tomuto trojuholníku nesplýva
s~jeho ťažiskom~$T$. Ľahko ukážeme, že bod $T=(A+B+C)/3$ je
ťažiskom aj trojuholníka $KLM$. Podľa zadania totiž existuje číslo
$p\in(0,1)$ také, že
$$
K=pA+(1-p)B,\quad L=pB+(1-p)C,\quad M=pC+(1-p)A,
$$
odkiaľ okamžite vyplýva rovnosť $(K+L+M)/3=(A+B+C)/3$.

Pripusťme, že trojuholníky $ABC$ a~$KLM$ majú okrem ťažiska~$T$ spoločné
aj ortocentrum, ktoré označíme~$V$. Podľa známej vety ležia body
$V$, $T$, $O$ v~uvedenom poradí na jednej priamke, nazývanej
Eulerova priamka trojuholníka $ABC$, pričom platí $|VT|:|TO|=2:1$. Stred
kružnice opísanej je teda ťažiskom a~ortocentrom jednoznačne určený.
Úvahou o~Eulerovej priamke trojuholníka $KLM$ tak zisťujeme,
že bod~$O$ je nielen stredom kružnice opísanej trojuholníku $ABC$,
ale aj stredom kružnice opísanej trojuholníku $KLM$. Body $K$, $L$, $M$
majú preto rovnakú vzdialenosť od bodu~$O$, takže majú aj rovnakú
mocnosť ku kružnici opísanej trojuholníku $ABC$. Tieto mocnosti
sa rovnajú hodnotám
$$
\align
\m|AK|\cdot|BK|&=\m p(1-p)|AB|^2,\\
\m|BL|\cdot|CL|&=\m p(1-p)|BC|^2,\\
\m|CM|\cdot|AM|&=\m p(1-p)|AC|^2,\\
\endalign
$$
ktorých porovnaním dostaneme rovnosti $|AB|=|BC|=|CA|$ (lebo
$p\notin\{0,1\}$). To je v~spore s~predpokladom, že trojuholník
$ABC$ nie je rovnostranný.}

{%%%%%   trojstretnutie, priklad 6
Po $i$~spravených krokoch bude na stole $k-2i$ kôpok.
Ak teda zostane nakoniec na stole jediná kôpka, bolo číslo~$k$
nepárne a~celkový počet krokov bol $(k-1)/2$. Rozoberieme, či číslo~$k$
dáva po delení štyrmi zvyšok~1, alebo zvyšok~3.

{\it Prípad $k=4c+1$}. Na začiatku leží na stole
$1+\cdots+k=k(k+1)/2=(4c+1)(2c+1)$ kamienkov, vo všetkých $2c$~krokoch
odstránime celkom $1+\cdots+2c=c(2c+1)$ kamienkov, takže
počet kamienkov v~poslednej kôpke bude
$$
p=(4c+1)(2c+1)-c(2c+1)=(2c+1)(3c+1).
$$
Čísla $2c+1$ a~$3c+1$ sú však nesúdeliteľné, takže $p$ je štvorec
práve vtedy, keď sú štvorce obe čísla $2c+1$ 
a~$3c+1$, teda práve vtedy, keď sú štvorce ich štvornásobky
$4(2c+1)=2k+2$ a~$4(3c+1)=3k+1$.

{\it Prípad $k=4c+3$}. Na začiatku leží na stole
$1+\cdots+k=k(k+1)/2=2(c+1)(4c+3)$ kamienkov, vo všetkých $2c+1$
krokoch odstránime celkom $1+\cdots+(2c+1)=(c+1)(2c+1)$ kamienkov, takže
počet kamienkov v~poslednej kôpke bude
$$
p=2(c+1)(4c+3)-(c+1)(2c+1)=(c+1)(6c+5).
$$
Keby bolo číslo~$p$ štvorec, museli by byť štvorcami
obe nesúdeliteľné čísla $c+1$ a~$6c+5$. Ukážme, že to nie je
možné. Pripusťme existenciu prirodzených čísel $x$, $y$ takých, že
$c+1=x^2$ a~$6c+5=y^2$. Z~rovnosti $6x^2-y^2=1$ vyplýva, že číslo~$y$
je nepárne, takže číslo~$y^2$ dáva po delení ôsmimi zvyšok~1.
Číslo~$6x^2$ potom po delení ôsmimi dáva zvyšok~2, odkiaľ vyplýva, že
číslo~$3x^2$ po delení štyrmi dáva zvyšok~1, a~to je spor.
V~prípade $k=4c+3$ teda $p$ nikdy nie je štvorec, rovnako ako
nie je štvorec ani číslo $3k+1=12c+10$ (párne číslo, ktoré nie je
deliteľné štyrmi).

Teraz nájdeme najmenšie číslo $k=4c+1$, $c\geqq1$, pre ktoré sú
obe čísla $2c+1$ a~$3c+1$ štvorce. Z~rovností $2c+1=x^2$ 
a~$3c+1=y^2$ pre vhodné celé $x,y>1$ vyplýva $3x^2-2y^2=1$, takže $x$
je nepárne. Potom číslo~$2y^2$ po delení štyrmi dáva zvyšok~2, takže
aj $y$ je nepárne. Položme $x=2a+1$, $y=2b+1$ ($a,b>0$ celé) 
a~dosaďme do rovnosti $3x^2-2y^2=1$. Po úprave dostaneme vzťah
$3a(a+1)=2b(b+1)$, kam postupne dosadzujeme $a=1,2,\dots$
Nájdeme tak rýchlo najmenšie vyhovujúce $a=4$ a~$b=5$, ktorým
zodpovedá $x=9$, $y=11$, $c=40$ a~$k=161$.}

{%%%%%   IMO, priklad 1
\fontplace
\trpoint B; \tlpoint C; \bpoint A;
\rBpoint M; \lBpoint N;
\tpoint\xy1,0 P; \rpoint\xy-.7,1 R;
[1] \hfil\Obr

\fontplace
\tpoint B; \tpoint C; \bpoint A;
\rBpoint M; \lBpoint N;
\tpoint O; \tpoint P; \lpoint R;
\lBpoint S;
\cpoint\beta; \cpoint\gamma;
\cpoint\xy0,.7 \beta; \cpoint\gamma;
[2] \hfil\Obr

\podla{Františka Konopeckého}
Pretože body $M$, $N$ ležia na kružnici s~priemerom~$BC$, platí
$|OM|=|ON|=|OB|$. Trojuholník $MNO$ je teda rovnoramenný a~os
jeho uhla $MON$ je zároveň osou úsečky~$MN$. Bod~$R$, ktorý je
priesečníkom osi uhla $M\!AN$ s~osou protiľahlej strany~$MN$ trojuholníka
$AMN$, leží preto na kružnici opísanej trojuholníku $AMN$. Pritom obe osi
sú totožné len vtedy, keď $|AM|=|AN|$, čo vzhľadom na predpoklad
$|AB|\ne|AC|$ nie je možné, lebo z~vlastností tetivového
štvoruholníka $BCNM$ ľahko vyplýva, že trojuholníky $AMN$ a~$ACB$ sú
podobné (zhodujú sa v~dvoch uhloch).

Z~mocnosti bodu~$A$ ku kružnici s~priemerom~$BC$ vyplýva, že bod~$A$
má rovnakú mocnosť aj k~obom kružniciam opísaným trojuholníkom $BMR$ 
a~$CNR$ (\obr).
\inspicture
Ak označíme $P$ druhý spoločný bod týchto dvoch kružníc
(jedným je bod~$R$), musí bod~$A$ ležať na ich spoločnej sečnici~$PR$
(to je práve množina všetkých bodov, ktoré majú k~obom kružniciam
rovnakú mocnosť). Z~tetivových štvoruholníkov $BPRM$ a~$CPRN$ teraz
spočítame, že
$$
|\uhol BPC|=|\uhol BPR|+|\uhol CPR|=|\uhol AMR|+|\uhol ANR|=180\st,
$$
lebo $AMR$ a~$ANR$ sú protiľahlé uhly tetivového štvoruholníka
$AMRN$. Uhol $BPC$ je teda priamy, takže spoločný bod~$P$ oboch
kružníc leží na strane~$BC$.

\ineriesenie
Označme $S$ stred úsečky~$MN$ a~$P$ priesečník osi uhla $BAC$ so
stranou~$BC$. Pretože trojuholníky $AMN$ a~$ACB$ sú podobné, pričom
ťažnici~$AS$ zodpovedá ťažnica~$AO$, platí $|\uhol BAO|=|\uhol CAS|$ (\obr),
\inspicture
takže os uhla $BAC$ je zároveň aj osou uhla $OAS$. Preto
$$
{|RS|\over|RO|}={|AS|\over|AO|}.
$$
Z~uvedenej podobnosti ďalej vyplýva
$$
{|AS|\over|AO|}={|MN|\over|BC|}=
{|MS|\over|BO|}={|MS|\over|MO|},
$$
čo spolu s~predchádzajúcou rovnosťou znamená, že $MR$ je osou vnútorného
uhla $OMS$.

Označme vnútorné uhly trojuholníka $ABC$ zvyčajným spôsobom.
Pretože $|OM|=|OB|$, teda $|\uhol BMO|=\beta$, a~pretože $|\uhol
AMN|=\gamma$, veľkosť uhla $OMN$ je~$\alpha$. Takže $|\uhol
BMR|=\beta+\alpha/2=|\uhol CPA|$. Dostali sme, že štvoruholník $BPRM$
je tetivový. Analogicky je tetivový aj štvoruholník $CPRN$. Teda
bod $P\in BC$ je spoločným bodom oboch kružníc opísaných trojuholníkom $BMR$
a~$CNR$.}

{%%%%%   IMO, priklad 2
Ukážeme, že riešením sú iba mnohočleny $P(x)=sx^2+tx^4$ pre ľubovoľné reálne $s$, $t$.

Nech $P(x)$ spĺňa podmienky zadania. Ak $a=b=0$, tak $ab+bc+ca=0$ pre každé reálne~$c$.
Preto dostávame
$$
P(0-0)+P(0-c)+P(c-0)=2P(0+0+c),\quad\text{čiže}\quad P(0)+P(\m c)=P(c)
$$
pre každé reálne~$c$. Dosadením $c=0$ dostaneme $P(0)=0$, takže $P(c)=P(\m c)$ pre všetky~$c\in\Bbb R$.
Teda $P$ je párna funkcia a~musí byť tvaru
$$
P(x)=a_nx^{2n}+a_{n-1}x^{2n-2}+\cdots a_1x^2,\qquad a_1,\dots,a_n\in\Bbb R.
$$
Teraz dokážeme, že stupeň mnohočlena~$P$ môže byť najviac~4.

Pre ľubovoľné reálne čísla $u$ a~$v$ trojica $a=uv$, $b=(1-u)v$, $c=(u^2-u)v$ spĺňa
$$
ab+bc+ca=(a+b)c+ab=v(u^2-u)v+uv(1-u)v=0.
$$ 
Dosadením tejto trojice do rovnosti zo zadania dostaneme
$$
P\left((2u-1)v\right)+P\left((1-u^2)v\right)+P\left((u^2-2u)v\right)=2P\left((u^2-u+1)v\right)
$$
pre všetky reálne $u$, $v$. Pri pevnom~$u$ môžeme ostatnú rovnosť považovať za identitu mnohočlenov v~premennej~$v$.
Porovnaním vedúcich koeficientov na oboch stranách máme pre všetky $u\in\Bbb R$ rovnosť 
$$
(2u-1)^{2n}+(1-u^2)^{2n}+(u^2-2u)^{2n}=2(u^2-u+1)^{2n}.
$$ 
Zvolením $u=\m2$ dostaneme $5^{2n}+3^{2n}+8^{2n}=2\cdot7^{2n}$, takže $8^{2n}<2\cdot7^{2n}$. Avšak už pre $n=3$ platí
$8^{2n}>2\cdot7^{2n}$ ($8^{2\cdot3}=262\,144>235\,298=2\cdot7^{2\cdot3}$), a~teda tým skôr to platí aj pre $n>3$.
Takže $n\le2$, z~čoho $P(x)=sx^2+tx^4$ pre nejaké reálne $s$ a~$t$.

Na druhej strane, každý mnohočlen uvedeného tvaru spĺňa podmienky zadania. Aby sme to overili, uvedomme si najskôr,
že ľubovoľná lineárna kombinácia dvoch mnohočlenov, ktoré spĺňajú podmienky zadania, tiež spĺňa tieto podmienky.
Takže stačí overiť mnohočleny $x^2$ a~$x^4$. To, že vyhovuje $x^2$, vyplýva z~identity
$$
(a-b)^2+(b-c)^2+(c-a)^2-2(a+b+c)^2=\m6(ab+bc+ca).
$$ 
Overme aj $x^4$. Nech $ab+bc+ca=0$. Označme $p=a-b$, $q=b-c$ a $r=c-a$. Pri overení $x^2$ sme vlastne ukázali, že
$p^2+q^2+r^2=2(a+b+c)^2$. Potom, keďže $p+q+r=0$, želanú rovnosť dostaneme nasledovne:
$$
\align
pq+qr+rp&=\m\tfrac12(p^2+q^2+r^2)=\m(a+b+c)^2,\\
(pq)^2+(qr)^2+(rp)^2&=(pq+qr+rp)^2-2pqr(p+q+r)=(a+b+c)^4,\\
p^4+q^4+r^4&=(p^2+q^2+r^2)^2-2\left((pq)^2+(qr)^2+(rp)^2\right)=2(a+b+c)^4.
\endalign
$$  

\ineriesenie
Pre každé $z\in\Bbb R$ trojica $(a,b,c)=(6z,3z,\m2z)$ spĺňa podmienku $ab+bc+ca=0$. Dosadením do zadanej rovnosti dostávame
$$
P(3z)+P(5z)+P(\m8z)=2P(7z).
$$
Takže ak $P(x)=a_nx^n+\cdots+a_1x+a_0$, platí nutne pre každé $i=0,1,2,\dots$ rovnosť
$$
\left(3^i+5^i+(\m8)^i-2\cdot7^i\right)a_i=0.
$$
Výraz v~zátvorkách je záporný pre nepárne~$i$ a~kladný pre $i=0$ a~pre všetky párne $i\ge6$. Iba pre $i=2$ a~$i=4$ je výraz nulový. Preto $P(x)=sx^2+tx^4$ pre nejaké $s,t\in\Bbb R$. Ostáva len overiť, že všetky mnohočleny tohto tvaru vyhovujú, čo urobíme rovnako ako v~prvom riešení.}

{%%%%%   IMO, priklad 3
\epsplace mmo45.6 \hfil\Obr

\epsplace mmo45.7 \hfil\Obr

\epsplace mmo45.8 \hfil\Obr

Predpokladajme, že pravouholník $m\times n$ je pokrytý hákmi tak, ako sa spomína v~zadaní. Pre každý hák~$A$ máme
jediný hák~$B$, ktorý pokrýva {\it vnútorný\/} štvorček háku~$A$ jedným zo svojich {\it koncových\/} štvorčekov. Pritom
vnútorný štvorček háku~$B$ je nutne pokrytý koncovým štvorčekom háku~$A$. Takže v~pokrytí sú všetky háky popárované do
\vadjust{\bigskip\centerline{\inspicture-!\hss\inspicture-!}\bigskip}
dvojíc. Sú len dve možnosti, ako umiestniť $B$ ku~$A$, aby nevznikli medzery a~prekrytia. V~jednom prípade tvoria
$A$ a~$B$ obdĺžnik $4\times3$ (\obr), v~druhom prípade osemuholníkový útvar (\obr).
Pravouholník $m\times n$ teda vieme pokryť hákmi práve vtedy, keď ho vieme pokryť týmito {\it dvojútvarmi\/} zloženými
z~12~štvorčekov. Preto $mn$ musí byť nutne deliteľné dvanástimi. Ukážeme, že niektorý z~rozmerov $m$ a~$n$ musí byť deliteľný štyrmi.

Predpokladajme, že to neplatí. Potom $m$ aj $n$ sú párne, nakoľko $4\mid mn$. Rozdeľme pravouholník na jednotkové
štvorčeky a~do istých štvorčekov vpíšme čísla 1 alebo 2 ako na \obr{} (jednotky sú vpísané v~každom štvrtom riadku a~v~každom štvrtom stĺpci, dvojky sú na priesečníkoch "ojednotkovaných" riadkov a~stĺpcov).
\inspicture 
Keďže počet štvorčekov v~každom riadku aj stĺpci je párny, súčet všetkých čísel vpísaných do pravouholníka je párny.
Ľahko možno overiť, že každý obdĺžnik $4\times3$ vždy pokryje čísla, ktorých súčet je 3 alebo 7. Útvar z~\obrr2{}
zasa vždy pokryje čísla, ktorých súčet je 5 alebo 7. Teda počet všetkých dvojútvarov musí byť párny
(súčet nepárneho počtu nepárnych čísel by bol nepárny). Potom ale $24\mid mn$, čiže $mn$ je deliteľné aj ôsmimi, čo je v~spore s~predpokladom, že $m$ ani $n$ nie je deliteľné štyrmi.

Uvedomme si ešte, že žiadny z~rozmerov $m$, $n$ sa nemôže rovnať 1, 2 ani 5 (nech ukladáme háky akokoľvek, nevieme nimi pokryť riadok či stĺpec pozdĺž strany pravouholníka dĺžky 1, 2 či 5). Zdôvodnili sme teda, že ak je pokrytie možné, tak aspoň jeden rozmer je deliteľný tromi, aspoň jeden štyrmi a~$m,n\notin\{1,2,5\}$.

Naopak, ak sú tieto podmienky splnené, potom sa pravouholník hákmi pokryť dá (iba použitím obdĺžnikov $4\times3$). Totiž ak jeden rozmer je deliteľný tromi a~druhý štyrmi, je existencia pokrytia zrejmá. A~ak je jeden z~rozmerov, povedzme~$m$, deliteľný dvanástimi a~$n\notin\{1,2,5\}$, potom $n$ vieme rozložiť na súčet niekoľkých trojok a~niekoľkých štvoriek. Celý pravouholník preto môžeme rozdeliť na pásy $m\times3$ a~$m\times4$. Pritom každý z~takých pásov zrejme pokryť vieme.}

{%%%%%   IMO, priklad 4
Tvrdenie dokážeme matematickou indukciou. 
\itemitem{$1^\circ$}
Aby sme tvrdenie dokázali pre $n=3$, potrebujeme ukázať, že každé tri kladné reálne čísla $t_1$, $t_2$, $t_3$ spĺňajúce nerovnosť
$$
10>(t_1+t_2+t_3)\left(\frac1{t_1}+\frac1{t_2}+\frac1{t_3}\right)
\tag1
$$
spĺňajú aj trojuholníkové nerovnosti. Predpokladajme sporom, že to neplatí. Bez ujmy na všeobecnosti (vďaka symetrickosti nerovnosti~\thetag{1}) môžeme predpokladať, že $t_1\ge t_2+t_3$. Označme $V$ výraz na ľavej
strane~\thetag{1}. Úpravou dostávame
$$
\aligned
V&=(t_1+t_2+t_3)\left(\frac1{t_1}+\frac1{t_2}+\frac1{t_3}\right)=
3+\frac{t_1}{t_2}+\frac{t_1}{t_3}+\frac{t_2}{t_1}+\frac{t_3}{t_1}+\underbrace{\frac{t_2}{t_3}+\frac{t_3}{t_2}}_{\ge2}\ge\\
&\ge 5+t_1\left(\frac1{t_2}+\frac1{t_3}\right)+\frac{t_2+t_3}{t_1}\ge 
5+2\frac{t_1}{\sqrt{t_2t_3}}+2\frac{\sqrt{t_2t_3}}{t_1}.
\endaligned
\tag2
$$
Pri ostatnom odhade sme použili nerovnosť medzi aritmetickým a~geometrickým priemerom čísel $1/t_2$ a~$1/t_3$, resp\. čísel $t_2$ a~$t_3$. Keď rovnakú nerovnosť použijeme na predpoklad $t_1\ge t_2+t_3$, dostaneme $t_1\ge 2\sqrt{t_2t_3}$.
Označme pre zjednodušenie úprav $t_1/\sqrt{t_2t_3}=a$. Máme teda $a\ge2$. Pokračovaním v~úpravách~\thetag{2} získame
$$
V\ge 5+2a+\frac{2}{a}=10+\frac{2a^2-5a+2}{a}=10+\frac{(2a-1)(a-2)}{a}\ge10
\tag3
$$
(využili sme odvodenú nerovnosť $a\ge2$). Spojením \thetag{1}, \thetag{2} a~\thetag{3} dostávame zjavný spor $10>V\ge10$.
\itemitem{$2^\circ$}
Predpokladajme, že tvrdenie platí pre hodnotu $n-1\ge3$. Tvrdenie dokazujme opäť sporom. Nech teda kladné reálne čísla $t_1,t_2,\dots,t_n$ spĺňajú
$$
n^2+1>(t_1+t_2+\cdots+t_n)\left(\frac{1}{t_1}+\frac{1}{t_2}+\cdots+\frac{1}{t_n}\right)
\tag4
$$
a~niektoré tri z~nich nespĺňajú trojuholníkové nerovnosti. Vďaka symetrickosti nerovnosti~\thetag{4} môžeme opäť bez ujmy na všeobecnosti predpokladať, že $t_1\ge t_2+t_3$. Výraz na pravej strane~\thetag{4} označme~$W$. Úpravami dostávame
$$
\aligned
W&=(t_1+t_2+\cdots+t_n)\left(\frac{1}{t_1}+\frac{1}{t_2}+\cdots+\frac{1}{t_n}\right)=\\
&=(t_1+t_2+\cdots+t_{n-1})\left(\frac{1}{t_1}+\frac{1}{t_2}+\cdots+\frac{1}{t_{n-1}}\right)+\\
&+t_n\left(\frac{1}{t_1}+\frac{1}{t_2}+\cdots+\frac{1}{t_{n-1}}\right)+
\frac1{t_n}(t_1+t_2+\cdots+t_{n-1})+1\ge\\
&\ge(n-1)^2+1+\Bigl(\underbrace{\frac{t_n}{t_1}+\frac{t_1}{t_n}}_{\ge2}\Bigr)+\cdots+
\Bigl(\underbrace{\frac{t_n}{t_{n-1}}+\frac{t_{n-1}}{t_n}}_{\ge2}\Bigr)+1\ge\\
&\ge(n-1)^2+1+2(n-1)+1=n^2+1.
\endaligned
\tag5
$$
Všimnime si, ako sme využili indukčný predpoklad. Keďže $t_1\ge t_2+t_3$, nemôže platiť
$$
(n-1)^2+1>(t_1+t_2+\cdots+t_{n-1})\left(\frac{1}{t_1}+\frac{1}{t_2}+\cdots+\frac{1}{t_{n-1}}\right)
$$
(bolo by to v~spore s~tým, že tvrdenie platí pre kladné reálne čísla $t_1,t_2,\dots,t_{n-1}$), platí teda opačná nerovnosť, ktorú sme pri úpravách~\thetag{5} použili. Spojením \thetag{4} a~\thetag{5} máme $n^2+1>W\ge n^2+1$, čo je spor.

\poznamka
Riešenie je zaujímavé tým, že druhý krok indukcie je oveľa jednoduchší ako prvý krok. Pri matematickej indukcii to zvyčajne býva naopak.}

{%%%%%   IMO, priklad 5
\fontplace
\rpoint A; \tpoint B; \lpoint C; \bpoint D;
\rpoint P; \tpoint B'; \bpoint D';
\lBpoint o;
[3] \hfil\Obr

\fontplace
\rpoint A; \tpoint B; \lpoint C; \bpoint D;
\rpoint P; \tpoint B'; \bpoint D';
\lpoint k;
[4] \hfil\Obr

\podla{Františka Konopeckého}
\inspicture
Vzhľadom na to, že uhlopriečka~$BD$ nie je osou ani jedného
z~vnútorných uhlov uvažovaného štvoruholníka $ABCD$ (pri vrcholoch
$B$ a~$D$) a~bod~$P$ leží vnútri $ABCD$, nemôže ležať na
uhlopriečke~$BD$.

Predpokladajme najskôr, že $ABCD$ je tetivový. Označme $B'$ 
a~$D'$ priesečníky jemu opísanej kružnice s~polpriamkami opačnými k~$PD$,
resp\. k~$PB$. Rovnosť uhlov $ABD$ a~$PBC$ (\obr) tak znamená
rovnosť príslušných oblúkov $AD$ a~$CD'$. Podobne sa zhodujú 
aj oblúky $AB$ a~$CB'$. To ale znamená, že bod~$B'$ je obrazom bodu~$B$
%\inspicture
a~bod~$D'$ obrazom bodu~$D$ v~osovej súmernosti podľa osi~$o$
úsečky~$AC$. V~tejto osovej súmernosti je tak úsečka~$BD'$ obrazom
úsečky~$B'D$ a~ich priesečník~$P$ preto leží na osi uhlopriečky~$AC$.
Teda $|AP|=|CP|$.

Obrátene, nech $|AP|=|CP|$. Uvažujme kružnicu~$k$ opísanú trojuholníku
$ABD$ a~označme $B'$ a~$D'$ jej priesečníky s~polpriamkami opačnými
\inspicture
k~$PD$, resp\. k~$PB$ (\obr). Z~mocnosti bodu~$P$ ku kružnici~$k$
vyplýva
$$
{|PB|\over|PD|}={|PB'|\over|PD'|},
$$
takže trojuholníky $BPD$ a~$B'PD'$ sú podobné. To ale naviac znamená,
že trojuholníky $BDC$ a~$B'D'A$ sa zhodujú v~dvojiciach uhlov pri
stranách $BD$ a~$B'D'$, takže sú podobné s~rovnakým koeficientom
podobnosti ako trojuholníky $BPD$ a~$B'PD'$. A~pretože v~tejto
podobnosti si zodpovedajú dve zhodné úsečky $CP$ a~$AP$, jedná sa
o~zhodnosť (ľahko nahliadneme, že sa jedná o~osovú súmernosť).
Táto zhodnosť zobrazí kružnicu~$k$ opísanú trojuholníku $ABD$ na seba,
proti bod~$C$, ktorý je obrazom bodu~$A$, leží tiež na tejto
kružnici a~štvoruholník $ABCD$ je teda tetivový.}

{%%%%%   IMO, priklad 6
Striedavých čísel vieme vytvoriť veľa, no vo všeobecnosti ich ťažko možno popísať v~tvare, z~ktorého by bolo vidieť, čím sú deliteľné. Zamerajme sa preto na nejakú úzku skupinu striedavých čísel, o~ktorých vieme povedať viac. Najjednoduchšie striedavé čísla vytvoríme z~jednotiek a~núl. Zistime, či už medzi takýmito číslami nenájdeme násobky nejakej väčšej skupiny čísel. Presnejšie povedané, snažme sa pre číslo~$n$ vytvoriť striedavé číslo tvaru $1010\dots101$ (nazývajme ďalej takéto číslo {\it superstriedavé} a~označujme ho $s_k$, kde $k$ je počet jednotiek v~jeho zápise), ktoré by bolo násobkom~$n$. Medzi superstriedavými číslami nájdeme vďaka Dirichletovmu princípu určite dve rôzne, ktoré dávajú po delení číslom~$n$ rovnaký zvyšok. Teda ich rozdiel je násobkom~$n$. Na druhej strane, rozdiel dvoch superstriedavých čísel $s_k$ a~$s_\ell$ pre $k>\ell$ je zrejme tvaru
$$
\underbrace{1010\dots101}_{\text{$k$ jednotiek}}-\underbrace{1010\dots101}_{\text{$\ell$ jednotiek}}=
\underbrace{1010\dots101}_{\text{$k-\ell$ jednotiek}}\!\underbrace{00\dots00}_{\text{$2\ell$ núl}}=
s_{k-\ell}\cdot10^{2\ell}.
$$
Takže pre každé~$n$ vieme nájsť $k$ a~$\ell$ také, že $n\mid s_{k-\ell}\cdot10^{2\ell}$. Ak $n$ je nesúdeliteľné s~číslom~10, tak dokonca $n\mid s_{k-\ell}$, teda $n$ má striedavý násobok. 

Vidíme, že problém je s~číslami~$n$, ktoré sú párne, resp\. deliteľné piatimi. Zrejme žiadny ich násobok nie je superstriedavý. Striedavé násobky k~nim teda musíme hľadať v~inom tvare. Skúsme ich nájsť najskôr pre čísla~$n$, ktoré sú mocninou čísla~5, \tj. pre čísla $n=5^\alpha$. Pre malé hodnoty $\alpha$ sa nám to naozaj darí, keďže
$$
5,\quad 25=5^2,\quad 125=5^3,\quad 8\,125=13\cdot5^4,\quad 78\,125=25\cdot5^5
\tag1
$$
sú striedavé. Pre väčšie~$\alpha$ môžeme striedavý násobok čísla $5^\alpha$ vytvoriť indukciou. Totiž keď máme vytvorený striedavý násobok $A_k$ čísla~$5^k$, ktorý má $k$~číslic (pripúšťame, aby sa desiatkový zápis začínal nulou), môžeme vytvoriť striedavý násobok čísla $5^{k+1}$ tak, že pred~$A_k$ napíšeme vhodne jednu číslicu. Prvý krok indukcie je v~\thetag{1}. Pre druhý krok predpokladajme, že už máme vytvorené striedavé číslo
$$
A_k=\overline{a_ka_{k-1}\dots a_1}=5^k\cdot d,\qquad d\in\Bbb N,\quad
a_1,a_2,\dots a_k\in\{0,1,2,\dots,9\}.
$$
Potom predpísaním nejakej číslice~$a_{k+1}$ pred~$A_k$ dostaneme
$$
A_{k+1}=\overline{a_{k+1}A_k}=a_{k+1}\cdot10^k+A_k=10^ka_{k+1}+5^kd=5^k(2^ka_{k+1}+d).
$$
Aby $A_{k+1}$ bolo striedavé a~súčasne násobkom~$5^{k+1}$, stačí $a_{k+1}$ zvoliť tak, aby malo opačnú paritu ako $a_k$ a~aby $2^ka_{k+1}+d$ bolo deliteľné piatimi. To zrejme vždy ide, nakoľko prvú podmienku spĺňa 5~rôznych číslic (buď číslice z~$\{1,3,5,7,9\}$, alebo z~$\{0,2,4,6,8\}$) a~pre každú z~nich dáva číslica $a_{k+1}$, a~teda aj číslo $2^ka_{k+1}+d$ iný zvyšok po delení piatimi (keďže $2^k$ a~5 sú nesúdeliteľné). Jeden z~tých zvyškov teda musí byť nulový a~vtedy $5\mid2^ka_{k+1}+d$. Ukázali sme teda, že všetky mocniny piatich majú striedavé násobky. Pritom z~uvedeného postupu vyplýva, že pre dané $n=5^\alpha$ vieme striedavý násobok vytvoriť tak, aby mal párny počet číslic a~končil sa nepárnou číslicou.

Venujme sa teraz mocninám čísla dva. Opäť ukážeme, že pre každé $n=2^\beta$ existuje striedavý násobok~$n$. Postup bude podobný, ako pri mocninách piatich, avšak pridávať budeme až dve číslice a~na vytvorené čísla budeme mať prísnejšie predpoklady. Presnejšie, dokážeme, že pre každé prirodzené~$k$ existuje $(2k-1)$-ciferné striedavé číslo~$B_k$, ktoré je deliteľné číslom $2^{2k-1}$ ale nie je deliteľné číslom $2^{2k}$. Prvý krok indukcie nám zabezpečia striedavé čísla
$$
B_1=2,\quad B_2=232=2^3\cdot29,\quad B_3=27\,232=2^5\cdot851,\quad B_4=2\,127\,232=2^7\cdot16\,619.
$$
V~druhom kroku predpokladajme, že máme striedavé číslo
$$
B_k=\overline{b_{2k-1}b_{2k-2}\dots b_1}=2^{2k-1}\cdot d,\qquad d\in\Bbb N,\ 2\nmid d,
\quad b_1,\dots b_{2k-1}\in\{0,1,2,\dots,9\}
$$
(zo~striedavosti a~párnosti $B_k$ nutne číslica $b_{2k-1}$ je párna). Chceme zvoliť číslo $b=\overline{b_{2k+1}b_{2k}}$, (pričom $b_{2k}\in\{1,3,5,7,9\}$ a~$b_{2k+1}\in\{0,2,4,6,8\}$) tak, aby číslo
$$
B_{k+1}=\overline{bB_k}=b\cdot10^{2k-1}+B_k=10^{2k-1}b+2^{2k-1}d=2^{2k-1}(5^{2k-1}b+d)
$$
bolo deliteľné číslom $2^{2k+1}$, ale nebolo deliteľné číslom $2^{2k+2}$. Potrebujeme teda, aby $5^{2k-1}b+d$ bolo deliteľné štyrmi, ale nebolo deliteľné ôsmimi. Avšak podľa predpokladu $d$ je nepárne, teda dáva po delení ôsmimi jeden zo zvyškov 1, 3, 5 alebo 7. Za $b$ teda stačí zvoliť jedno z~čísel 21, 23, 25 alebo 27 tak, aby $5^{2k-1}b$ dávalo po delení ôsmimi taký zvyšok, že keď ho sčítame s~$d$, výsledok bude dávať po delení ôsmimi zvyšok 4 (\tj. ak $d$ dáva zvyšok~1, $5^{2k-1}b$ chceme so zvyškom 3, k~zvyšku~3 chceme zvyšok~1, k~zvyšku~5 zvyšok~7 a~k~zvyšku~7 zvyšok~5). Keďže $5^{2k-1}$ je nesúdeliteľné s~8 a~čísla 21, 23, 25 a~27 dávajú rôzne nepárne zvyšky po delení ôsmimi, pre jednu hodnotu $b\in\{21,23,25,27\}$ bude zvyšok $5^{2k-1}b$ po delení ôsmimi naozaj taký, ako chceme. Aj pre mocniny dvoch sme teda našli striedavé násobky. Pritom ak pred $B_k$ pripíšeme ľubovoľnú nepárnu číslicu, dostaneme striedavé číslo, ktoré je tiež násobkom $2^{2k-1}$, pretože $B_k$ má $2k-1$ číslic. Takže ku každému $n=2^\beta$ vieme nájsť striedavé číslo, ktoré má párne veľa číslic.

Takto vyzbrojení môžeme prejsť ku všeobecnému prípadu, keď $n=5^\alpha\cdot2^\beta\cdot m$, pričom $m$ nie je deliteľné dvoma ani piatimi. Pre $\alpha=\beta=0$ sme už úlohu vyriešili (číslo~$n$ má v~takom prípade superstriedavý násobok). 

Uvažujme ďalej prípad, keď $\beta=0$. Číslo $5^\alpha$ má striedavý násobok s~párnym počtom číslic, označme ho~$M$. Zrejme potom aj číslo $S_k=\overline{MM\dots M}$ ($k$-krát napíšeme za sebou číslo~$M$) je striedavým násobkom~$5^\alpha$. Rovnakou úvahou ako pri superstriedavých číslach nájdeme $k$ a~$\ell$ také, že $m\mid S_{k-\ell}$.
Potom $S_{k-\ell}$ je striedavým násobkom~$n$.

Úplne rovnako nájdeme striedavý násobok~$n$ v~prípade, keď $\alpha=0$. 

Keď $\beta=1$ a $\alpha\ge1$, stačí k~číslu $S_{k-\ell}$, ktoré sme našli pre $n=5^\alpha\cdot m$, pripísať sprava nulu. Dostaneme číslo, ktoré bude striedavé (keďže $M$ a~teda aj $S_{k-\ell}$ končili nepárnou číslicou) a~bude násobkom dvoch aj násobkom $5^\alpha\cdot m$, \tj. bude násobkom $5^\alpha\cdot2^1\cdot m$.

Ostal prípad, keď $\beta\ge2$ a~$\alpha\ge1$. V~takom prípade je ale $n=5^\alpha\cdot2^\beta\cdot m$ násobkom čísla~20, teda ľubovoľný jeho násobok je tiež násobkom 20, čiže končí jedným z~dvojčíslí 00, 20, 40, 60, 80. Také číslo zrejme nie je nikdy striedavé.

\odpoved
Hľadanými číslami sú všetky čísla, ktoré nie sú násobkom čísla~20.}

