{%%%%%   Z4-I-1
...}
\podpis{...}

{%%%%%   Z4-I-2
...}
\podpis{...}

{%%%%%   Z4-I-3
...}
\podpis{...}

{%%%%%   Z4-I-4
...}
\podpis{...}

{%%%%%   Z4-I-5
...}
\podpis{...}

{%%%%%   Z4-I-6
...}
\podpis{...}

{%%%%%   Z5-I-1
...}
\podpis{...}

{%%%%%   Z5-I-2
...}
\podpis{...}

{%%%%%   Z5-I-3
...}
\podpis{...}

{%%%%%   Z5-I-4
...}
\podpis{...}

{%%%%%   Z5-I-5
...}
\podpis{...}

{%%%%%   Z5-I-6
...}
\podpis{...}

{%%%%%   Z6-I-1
...}
\podpis{...}

{%%%%%   Z6-I-2
...}
\podpis{...}

{%%%%%   Z6-I-3
...}
\podpis{...}

{%%%%%   Z6-I-4
...}
\podpis{...}

{%%%%%   Z6-I-5
...}
\podpis{...}

{%%%%%   Z6-I-6
...}
\podpis{...}

{%%%%%   Z7-I-1
...}
\podpis{...}

{%%%%%   Z7-I-2
...}
\podpis{...}

{%%%%%   Z7-I-3
...}
\podpis{...}

{%%%%%   Z7-I-4
...}
\podpis{...}

{%%%%%   Z7-I-5
...}
\podpis{...}

{%%%%%   Z7-I-6
...}
\podpis{...}

{%%%%%   Z8-I-1
Ocko oprel náš nový rebrík o stenu domu. Spodná (prvá) priečka je $24\cm$ nad zemou.
Posledná -- štrnásta priečka je päťkrát tak vysoko, ako tretia priečka. Vzdialenosť medzi
každými dvoma priečkami je rovnaká.
\ite a) Ako vysoko nad zemou je tretia priečka?
\ite b) O koľko vyššie sa dostane ocko, ak z tretej priečky vystúpi na piatu?
}
\podpis{M. Dillingerová}

{%%%%%   Z8-I-2
Myslím si 4 dvojciferné čísla. Dve sú prvočísla a dve sú zložené čísla. Súčet prvočísel je 100
a súčet zložených čísel je tiež 100. Aj prvočísla aj zložené čísla sú tvorené tou istou štvoricou
rôznych cifier. Ktoré čísla si myslím?}
\podpis{Š. Ptáčková}

{%%%%%   Z8-I-3
V Jurkovej stavebnici je 64 rovnako veľkých kociek. Steny týchto kociek sú jednofarebné --
biele alebo čierne. Zo všetkých kociek sa dá zložiť jedna veľká kocka, ktorej každá stena je
z polovice biela a z polovice čierna. Koľko najviac celkom bielych kociek môže byť
v Jurkovej stavebnici? Načrtnite obrázok, ako by táto veľká kocka vyzerala pri pohľade zdola,
zozadu, spredu, zhora, zľava a sprava.}
\podpis{S. Bednářová}

{%%%%%   Z8-I-4
Sedem trpaslíkov našlo košík jabĺk. Bez toho, aby jabĺčka krájali, podelili sa o ne. Prvý dostal
jedno jablko a $1/9$ zvyšku. Potom prišiel druhý, vzal si dve jablká a $1/9$ zvyšku, tretí tri jablká
a $1/9$ zvyšku a tak ďalej, až siedmy si vzal 7 jabĺk a $1/9$ zvyšku. Jablká, ktoré po tomto delení
zostali v košíku, priniesli Snehulienke. Koľko najmenej jabĺk našli trpaslíci? Koľko jabĺk
dostal každý z nich a koľko Snehulienka?}
\podpis{L. Hozová}

{%%%%%   Z8-I-5
Martin má z matematiky rôzne známky, ktorých aritmetický priemer je 2,1. Pätorku nemá ani
jednu. Zato jednotky tvoria 35\% a dvojky 30\% všetkých jeho známok z matematiky.
Koľko percent všetkých Martinových známok z matematiky tvoria trojky a koľko štvorky?
Koľko má známok z matematiky, ak trojok má 5?}
\podpis{M. Dillingerová}

{%%%%%   Z8-I-6
Stredná priečka delí lichobežník na dve časti, z ktorých menšia má obsah $18\cm^2$. Aký obsah
bude mať väčšia z častí, na ktoré delí tento lichobežník jeho uhlopriečka, ak menšia má obsah
$16\cm^2$?}
\podpis{S. Bednářová}

{%%%%%   Z9-I-1
V Jurkovej stavebnici je 64 rovnako veľkých kociek. Steny týchto kociek sú jednofarebné --
biele alebo čierne. Zo všetkých kociek sa dá zložiť jedna veľká kocka, ktorej každá stena je
z polovice biela a z polovice čierna. Koľko najviac celkom bielych kociek môže byť
v Jurkovej stavebnici? Načrtnite obrázok, ako by táto veľká kocka vyzerala pri pohľade zdola,
zozadu, spredu, zhora, zľava a sprava.}
\podpis{S. Bednářová}

{%%%%%   Z9-I-2
...}
\podpis{...}

{%%%%%   Z9-I-3
Tromi spôsobmi napíšte číslo $2\,004$ ako súčet niekoľkých po sebe idúcich prirodzených čísel.}
\podpis{L. Hozová}

{%%%%%   Z9-I-4
Ak v príklade na násobenie dvoch prirodzených čísel zaokrúhlime 1. činiteľa na desiatky,
súčin sa zväčší o $48$. Ak v pôvodnom príklade zaokrúhlime 2. činiteľa na desiatky, pôvodný
súčin sa zmenší o $1\,512$. Nájdite pôvodný príklad (všetky možnosti).}
\podpis{M. Dillingerová}

{%%%%%   Z9-I-5
Danka, Milka, Jožko a Peťo sa vážili. Najľahší a najťažší z nich vážia dohromady 113\,kg.
Danka, Milka a Jožko vážia spolu 169\,kg. Milka, Jožko a Peťo majú spolu 166\,kg a Danka
s Peťom 99\,kg. Koľko váži každý z nich, keď Jožko nie je najťažší?}
\podpis{Š. Ptáčková}

{%%%%%   Z9-I-6
Nájdite prirodzené čísla $a$, $b$, ktoré spĺňajú podmienky:
\begin{itemize}
\item $3a+b=165$,
\item najmenší spoločný násobok čísel $a$, $b$ je desaťkrát väčší než najväčší spoločný deliteľ
čísel $a$, $b$.
\end{itemize}
}
\podpis{P. Tlustý}

{%%%%%   Z4-II-1
...}
\podpis{...}

{%%%%%   Z4-II-2
...}
\podpis{...}

{%%%%%   Z4-II-3
...}
\podpis{...}

{%%%%%   Z5-II-1
...}
\podpis{...}

{%%%%%   Z5-II-2
...}
\podpis{...}

{%%%%%   Z5-II-3
...}
\podpis{...}

{%%%%%   Z6-II-1
...}
\podpis{...}

{%%%%%   Z6-II-2
...}
\podpis{...}

{%%%%%   Z6-II-3
...}
\podpis{...}

{%%%%%   Z7-II-1
...}
\podpis{...}

{%%%%%   Z7-II-2
...}
\podpis{...}

{%%%%%   Z7-II-3
...}
\podpis{...}

{%%%%%   Z8-II-1
Myslím si dve dvojciferné čísla. Ak prvé z nich vydelíme druhým, dostaneme zvyšok $45$. Ak
druhé vydelíme prvým, dostaneme zvyšok $34$. Aké čísla si myslím?}
\podpis{S. Bednářová}

{%%%%%   Z8-II-2
Uhlopriečka delí lichobežník na dve časti, ktorých obsahy sú v pomere $2:3$. V akom pomere sú
obsahy dvoch častí, na ktoré delí tento lichobežník jeho stredná priečka?}
\podpis{S. Bednářová}

{%%%%%   Z8-II-3
Po sezóne ostali v predajni letné tričká s cenou 70\,Sk. Majiteľ predajne znížil ich cenu o viac
ako 25\%, ale o menej ako 50\%. Nová cena vyjadrená v korunách bola celé číslo. Všetky takto
zlacnené tričká predal a získal za ne 2\,430\,Sk. Koľko tričiek predal po zlacnení?}
\podpis{L. Hozová}

{%%%%%   Z9-II-1
Telesová uhlopriečka kvádra má dĺžku $17\cm$. Aké môžu byť rozmery tohto kvádra, ak dĺžky jeho
hrán vyjadrené v cm sú navzájom rôzne celé čísla?}
\podpis{Š. Ptáčková}

{%%%%%   Z9-II-2
V delfináriu mali 4 € jednotné vstupné pre všetkých návštevníkov. Poslednú nedeľu znížili
vstupné, čím sa zvýšil počet návštevníkov o dve tretiny a príjem v pokladni stúpol o 25\%. O koľko
eur bolo znížené vstupné, ak bolo opäť pre všetkých návštevníkov jednotné?}
\podpis{M. Krejčová}

{%%%%%   Z9-II-3
Číslo, v ktorom niektoré 4 za sebou idúce číslice sú $2$, $0$, $0$, $3$ (v uvedenom poradí) sa nazýva "moderné".
Nájdite najmenšie moderné číslo, ktoré je deliteľné každou svojou nenulovou číslicou.}
\podpis{S. Bednářová}

{%%%%%   Z9-II-4
...}
\podpis{...}

{%%%%%   Z9-III-1
Nájdite také dve prirodzené čísla, pre ktoré súčasne platí:
\begin{itemize}
\item jedno je dvojciferné, druhé je trojciferné,
\item ich druhé mocniny končia rovnakým trojčíslím,
\item ich druhé odmocniny sú prirodzené čísla a končia rovnakou cifrou.
\end{itemize}
}
\podpis{M. Dillingerová}

{%%%%%   Z9-III-2
Máme červenú guľôčku, ktorá má rovnakú hmotnosť ako zelená a biela spolu. Modrá guľôčka spolu so žltou
má rovnakú hmotnosť ako tri biele. Dve zelené a dve biele majú rovnakú hmotnosť ako modrá s bielou.
A nakoniec červená a modrá guľôčka presne vyvážia dve zelené, žltú a dve biele. Ktoré dve guľôčky rôznych
farieb majú rovnakú hmotnosť?}
\podpis{Š. Ptáčková}

{%%%%%   Z9-III-3
...}
\podpis{...}

{%%%%%   Z9-III-4
40\% detí Jankinej triedy tvoria chlapci. Ich priemerná výška je $145\cm$. Priemerná výška všetkých detí Jankinej
triedy je $142\cm$. Medzi dievčatami svojej triedy je Janka nadpriemerne vysoká, ale je menšia, než je priemerná
výška všetkých detí v jej triede. Zistite, koľko meria Janka, ak jej výška v centimetroch je celé číslo.}
\podpis{S. Bednářová}

