{%%%%%   Z4-I-1
Z čísla $6\,574\,839\,201$ vyškrtni niektoré číslice tak, aby vzniknuté číslo bolo čo najmenšie a aby
súčet cifier na mieste desiatok a mieste tisícok bol $7$ a rozdiel cifier na mieste stoviek a mieste
jednotiek bol $2$.}
\podpis{S. Bodláková, M. Dillingerová}

{%%%%%   Z4-I-2
Obdĺžnik $JURO$ má obvod $36\cm$, štvorec $LENA$ má obvod $24\cm$. $JURO$ a $LENA$ sú rodičmi
obdĺžnikov a štvorcov, ktoré majú polovičný obvod ako $LENA$ alebo polovičný obvod ako
$JURO$. Pritom všetky dcéry sú štvorce a žiadny syn nie je štvorec. Koľko najviac môžu mať
$JURO$ a $LENA$ rôznych synov a koľko rôznych dcér, ak rozmery detí sú v centimetroch
zapísané celými číslami?}
\podpis{S. Bodláková}

{%%%%%   Z4-I-3
...}
\podpis{...}

{%%%%%   Z4-I-4
Od pondelka do piatku býva náš psík Bobi doma sám, lebo rodičia sú v práci a deti v škole.
Jeden mesiac bol však sám iba 11 dní. Najprv bol tri dni od nedele do utorka doma Ivan, lebo
ho bolelo brucho. Len čo vyzdravel, ostala doma Majka s chrípkou, a to od stredy do štvrtka
ďalšieho týždňa. Koľko dní mohol mať ten mesiac?}
\podpis{M. Dillingerová}

{%%%%%   Z4-I-5
Na Vianočných trhoch boli vystavené 5, 7 a 9 ramenné svietniky, pričom na každom ramene
bola osadená jedna sviečka. Spolu tam bolo 9 svietnikov a 67 sviečok. Zisti, koľko ktorých
svietnikov tam bolo.}
\podpis{M. Dillingerová}

{%%%%%   Z4-I-6
Na telocviku bolo 9 chlapcov a 8 dievčat v dvoch zástupoch. V zástupe chlapcov bol každý
chlapec o $1\cm$ vyšší než ten, ktorý stál tesne pred ním. V dievčenskom zástupe bolo každé
dievča o $2\cm$ vyššie než to, ktoré stálo tesne pred ním. Najvyššie dievča bolo o $15\cm$ vyššie
ako najnižší chlapec. Tretí najvyšší chlapec meria $134\cm$. Koľko cm meria najnižšie dievča?}
\podpis{M. Dillingerová}

{%%%%%   Z5-I-1
Viktor chcel napísať zoznam všetkých dvojciferných čísel, ktoré po delení piatimi dávajú
zvyšok 3. Keď napísané čísla sčítal, dostal súčet 911. Ktoré číslo zabudol na zoznam napísať,
ak tam žiadne nesprávne napísané nebolo?}
\podpis{L. Hozová}

{%%%%%   Z5-I-2
Anička a Marienka mali mať zraz presne o 17:30 pred kinom. Anička si myslela, že jej idú
hodinky o 4 minúty napred, ale v skutočnosti jej meškali 8 minút. Marienka si myslela, že jej
hodinky 8 minút meškajú, ale išli jej o 4 minúty napred. Kedy ktoré z dievčat prišlo na zraz,
ak si obe mysleli, že prišli presne o 17:30?}
\podpis{Š. Ptáčková}

{%%%%%   Z5-I-3
Traja princovia išli bojovať s mnohohlavým drakom. Najprv mu prvý princ ľavou rukou
odsekol polovicu hláv a pravou rukou ešte ďalšie dve hlavy. Potom mu druhý princ tiež ľavou
rukou odsekol polovicu zvyšných hláv a pravou rukou ešte ďalšie dve a napokon tretí princ
mu ľavou rukou odsekol polovicu zvyšných hláv a pravou rukou ešte ďalšie dve. Potom drak
padol bezhlavý na zem. Koľkohlavý drak to bol?}
\podpis{Š. Ptáčková}

{%%%%%   Z5-I-4
...}
\podpis{...}

{%%%%%   Z5-I-5
...}
\podpis{...}

{%%%%%   Z5-I-6
...}
\podpis{...}

{%%%%%   Z6-I-1
Desatinné číslo nazveme {\it vyvážené}, ak je súčet cifier nachádzajúcich sa pred desatinnou
čiarkou rovný súčtu cifier za desatinnou čiarkou. Napr. číslo $25{,}133$ je vyvážené. Napíšte
\begin{itemize}
\ite a) najväčšie,
\ite b) najmenšie
\end{itemize}
vyvážené číslo, ktorého žiadne dve číslice nie sú rovnaké.
}
\podpis{...}

{%%%%%   Z6-I-2
...}
\podpis{...}

{%%%%%   Z6-I-3
V krajine Čísielkovo žijú iba prirodzené čísla. Muži a chlapci sú párne čísla, ženy a dievčatá
sú nepárne čísla. Manželia majú deti hneď po svadbe, a to všetky čísla, ktoré delia ich súčin
bezo zvyšku. Ktorého nápadníka z čísel 2, 8, 14 si má vybrať slečna Sedmička, ak chce mať
čo najviac rôznych detí? Ktorého nápadníka z čísel 2, 8, 14 si má vybrať slečna Sedmička, ak
chce mať rovnako veľa synov ako dcér, ale žiadne dve deti rovnaké?}
\podpis{M. Dillingerová}

{%%%%%   Z6-I-4
Na kartičke mám napísané párne štvorciferné číslo. Rozstrihnem ju tak, že získam dve
dvojciferné čísla, ktorých súčin je $2\,562$. Ktoré štvorciferné číslo som mala napísané na
kartičke?}
\podpis{M. Raabová}

{%%%%%   Z6-I-5
...}
\podpis{...}

{%%%%%   Z6-I-6
V Petroviciach, Bodríkovciach a Micinkove žije spolu 6\,000 obyvateľov. V každej z týchto
troch dediniek pripadá v priemere na 20 obyvateľov 1 pes a na 30 obyvateľov 1 mačka.
V Petroviciach a Bodríkovciach žije celkom 234 psov, v Bodríkovciach a Micinkove žije
celkom 92 mačiek. Koľko obyvateľov majú jednotlivé dedinky?}
\podpis{Š. Ptáčková}

{%%%%%   Z7-I-1
Dlhý, Široký a Bystrozraký si merali svoju výšku. Zistili, že Dlhý je dvakrát taký vysoký ako
Široký. Bystrozrakého výška tvorí len dve tretiny výšky Dlhého, ale je o $44\cm$ vyšší ako
Široký. Zistite, aký vysoký je Dlhý, Široký aj Bystrozraký.}
\podpis{M. Dillingerová}

{%%%%%   Z7-I-2
Je dané päťciferné číslo deliteľné tromi. Ak z neho vyškrtneme číslice na nepárnych miestach,
dostaneme dvojciferné číslo, ktoré je 67-krát menšie, než číslo získané z pôvodného
päťciferného čísla vyškrtnutím číslic na párnych miestach. Zistite, aké mohlo byť pôvodné
päťciferné číslo.}
\podpis{M. Raabová}

{%%%%%   Z7-I-3
V krajine Čísielkovo žijú iba prirodzené čísla. Muži a chlapci sú párne čísla, ženy a dievčatá
sú nepárne čísla. Manželia majú deti hneď po svadbe, a to všetky čísla, ktoré delia ich súčin
bezo zvyšku.
\begin{itemize}
\ite a) Ktorého nápadníka z čísel 2, 16, 28, 46 si má vybrať slečna Deviatka ak chce mať čo
najviac rôznych detí?
\ite b) Ktorého nápadníka z čísel 2, 16, 28, 46 si má vybrať slečna Deviatka ak chce mať
rovnako veľa synov ako dcér, ale žiadne dve deti rovnaké?
\end{itemize}
}
\podpis{M. Dillingerová}

{%%%%%   Z7-I-4
...}
\podpis{...}

{%%%%%   Z7-I-5
Myška Hryzka našla tehlu syra. Prvý deň zjedla $\frac18$ syra, druhý deň $\frac17$ zvyšku, tretí deň $\frac16$ zvyšku a štvrtý deň
$\frac15$ zvyšku syra. Potom už zo syra zostala iba kocka s povrchom $150\cm^2$.
Aký objem mala pôvodná tehla syra?}
\podpis{M. Dillingerová}

{%%%%%   Z7-I-6
...}
\podpis{...}

{%%%%%   Z8-I-1
Poradové číslo dňa v mesiaci je smutné, lebo v istom roku ani raz naň nepripadla nedeľa. Aké
číslo to bolo a na ktorý deň v týždni pripadol v tom roku Nový rok?}
\podpis{M. Volfová}

{%%%%%   Z8-I-2
...}
\podpis{...}

{%%%%%   Z8-I-3
O lichobežníku $LICH$ so základňou $LI$ vieme, že $LC\perp HI$ , $|\uhol ILC| = |\uhol IHC|$ a aritmetický
priemer dĺžok jeho základní je $8\cm$. Vypočítajte obsah tohto lichobežníka.}
\podpis{S. Bednářová}

{%%%%%   Z8-I-4
Koľko je medzi číslami $1,2,3,\dots,999,1\,000$ takých, ktoré nie sú deliteľné žiadnym z čísel
$2$, $3$, $4$, $5$?}
\podpis{M. Volfová}

{%%%%%   Z8-I-5
...}
\podpis{...}

{%%%%%   Z8-I-6
...}
\podpis{...}

{%%%%%   Z9-I-1
Dvojciferné číslo nazveme {\it exkluzívne\/} práve vtedy, keď má nasledujúcu vlastnosť: ak cifry
exkluzívneho čísla navzájom vynásobíme a k tomuto súčinu pripočítame ciferný súčet daného
exkluzívneho čísla, dostaneme práve toto exkluzívne číslo. Napríklad číslo $79$ je exkluzívne,
lebo $79 = 7\cdot9 + (7 + 9)$. Nájdite všetky exkluzívne čísla.}
\podpis{P. Tlustý}

{%%%%%   Z9-I-2
Vnútri pravidelného šesťuholníka so stranou dĺžky $2\sqrt3\cm$ sa
pohybuje kruh s priemerom dĺžky $1\cm$ tak, že sa stále dotýka
obvodu pravidelného šesťuholníka. Vypočítajte obsah tej časti
šesťuholníka, ktorá nemôže byť nikdy prekrytá pohybujúcim sa
kruhom.}
\podpis{M. Dillingerová}

{%%%%%   Z9-I-3
Koľko existuje spôsobov ako vybrať sedem čísel z množiny $\{1, 2, \dots, 8, 9\}$ tak, aby ich
súčet bol deliteľný tromi?}
\podpis{P. Tlustý}

{%%%%%   Z9-I-4
Nech sú dané kruh a štvorec s rovnakým obsahom. Do daného kruhu vpíšeme štvorec, do
daného štvorca vpíšeme kruh. Ktorý z vpísaných obrazcov má väčší obsah?}
\podpis{M. Volfová}

{%%%%%   Z9-I-5
Pán Párny mal párny počet ovečiek, pán Nepárny mal nepárny počet ovečiek. Počet všetkých
ovečiek týchto dvoch pánov tvoril trojciferné číslo, ktoré malo všetky číslice rovnaké. Každej
ovečke pána Párneho sa narodili tri ovečky, každej ovečke pána Nepárneho dve ovečky.
Jedného dňa však vlk roztrhal tri ovečky pánovi Párnemu. Teraz má pán Párny rovnako veľa
ovečiek, ako pán Nepárny. Koľko ovečiek mal pôvodne každý z chovateľov?}
\podpis{L. Hozová}

{%%%%%   Z9-I-6
Päť detí postupne povedalo:
\begin{itemize}
\item "Včera bol pondelok."
\item "Dnes je štvrtok."
\item "Pozajtra bude piatok."
\item "Zajtra bude sobota."
\item "Predvčerom bol utorok."
\end{itemize}
Keby ste vedeli, koľko detí klamalo, hneď by bolo jasné, ktorý je dnes deň. Ktorý je dnes
deň?}
\podpis{Š. Ptáčková}

{%%%%%   Z4-II-1
Obdĺžnik $VILO$ je dvakrát tak vysoký ako široký a jeho obvod je $24\cm$. Jeho deti sú
obdĺžniky (a štvorce) a majú polovičný obvod ako $VILO$. Jediný jeho syn zdedil po otcovi tú
istú vlastnosť, že je dvakrát tak vysoký ako široký. Aké rozmery má syn? Aké rozmery môžu
mať Vilove dcéry, ak sú ich rozmery v centimetroch zapísané celými číslami a každá je
iná?}
\podpis{S. Bodláková}

{%%%%%   Z4-II-2
Z čísla $8\,163\,452\,709$ vyškrtni niektoré číslice tak, aby vzniknuté číslo bolo čo najmenšie a
malo súčin cifier na mieste desiatok a mieste tisícok $24$ a súčet cifier na mieste stoviek
a mieste jednotiek $8$.}
\podpis{S. Bodláková}

{%%%%%   Z4-II-3
Monika si v škole v prírode od pondelka až do soboty každý deň kupuje balíček cukríkov.
Polovicu cukríkov z balíčka rozdá kamarátkam a dva zje sama. Ostatné odkladá sestričke
Ivanke. V pondelok bolo v kúpenom balíčku 24, v utorok 16, v stredu 14, vo štvrtok 20
a v piatok 18 cukríkov. V sobotu ešte jeden balíček kúpila a ako vždy dala kamarátkam aj
sebe. V nedeľu už cukríky nekupovala, ale počas cestovania domov polovicu ušetrených
cukríkov s kamarátkami pojedli. Ivanke priniesla presne 20 cukríkov. Koľko bolo cukríkov
v balíčku, ktorý kúpila v sobotu?}
\podpis{M. Dillingerová}

{%%%%%   Z5-II-1
Karol mal sčítať všetky dvojciferné čísla, ktoré po delení $10$ dávajú zvyšok, ktorý sa dá bezo
zvyšku deliť $5$. Jedno z čísel však ale omylom zarátal trikrát, takže mu správnym sčitovaním
vyšiel súčet $1\,035$. Ktoré číslo zarátal tri razy?}
\podpis{S. Bednářová}

{%%%%%   Z5-II-2
Krtko si začal raziť nový tunel. Najprv vykopal 5 metrov na sever, potom 23\,dm na západ,
150\,cm na juh, 37\,dm na západ, 620\,cm na juh, 53\,cm na východ a 27\,dm na sever. Koľko
centimetrov mu ešte ostáva vykopať, aby sa dostal na začiatok tunelu?}
\podpis{M. Dillingerová}

{%%%%%   Z5-II-3
Z čísla $9\,876\,543\,210$ vyškrtni čo najmenší počet číslic tak, aby číslica na mieste desiatok bola
číslo trikrát menšie ako číslo na mieste tisícok a číslica na mieste jednotiek bola číslo o tri
menšie ako na mieste stoviek. Nájdi všetky riešenia.}
\podpis{S. Bodláková}

{%%%%%   Z6-II-1
Desatinné číslo nazveme {\it vyvážené}, ak súčet jeho číslic ležiacich pred desatinnou čiarkou je
rovný súčtu číslic za desatinnou čiarkou. Napr. číslo $25{,}133$ je vyvážené.
V číslach $497\,365{,}198\,043$ a $197\,352{,}598\,062$ škrtnite niekoľko číslic tak, aby vzniklo
\begin{itemize}
\ite a) čo najväčšie vyvážené číslo,
\ite b) vyvážené číslo s čo najväčším počtom číslic.
\end{itemize}
}
\podpis{S. Bednářová}

{%%%%%   Z6-II-2
...}
\podpis{...}

{%%%%%   Z6-II-3
Keď v pekárni napečú buchty, rozdelia ich do balíčkov po 6 a po 12 kusoch. Z predaja 6-
kusového balíčka majú zisk 4\,Sk a z predaja 12-kusového balíčka 9\,Sk. Koľko najviac a koľko
najmenej buchiet môže byť na jednom pekáči, ak zisk z ich predaja je 219\,Sk?}
\podpis{M. Dillingerová}

{%%%%%   Z7-II-1
V krajine Čísielkovo žijú iba prirodzené čísla. Muži a chlapci sú párne čísla a ženy a dievčatá sú
nepárne čísla. Manželia majú deti hneď po svadbe a to všetky čísla, ktoré delia ich súčin bezo
zvyšku. Pritom žiadne dve deti nemajú rovnakú hodnotu. Súčet hodnôt všetkých detí manželov
Kvádrikových je 28. Otec Kvádrik má nižšiu hodnotu než aspoň jeden z jeho synov. Určte
hodnoty pána a pani Kvádrikových.}
\podpis{M. Dillingerová}

{%%%%%   Z7-II-2
Koľko malých kociek, z ktorých každá má povrch $54\cm^2$, potrebujeme na postavenie plnej
veľkej kocky s povrchom $864\cm^2$?}
\podpis{M. Krejčová}

{%%%%%   Z7-II-3
Máme štyri nádoby. V prvých troch je voda, štvrtá je prázdna. V druhej je dvakrát viac vody ako
v prvej a v tretej je dvakrát viac vody ako v druhej. Do štvrtej nádoby prelejeme polovicu vody
z prvej nádoby, tretinu vody z druhej nádoby a štvrtinu vody z tretej nádoby. V štvrtej nádobe
sme takto získali 26\,l vody. Koľko litrov vody je vo všetkých nádobách spolu?}
\podpis{M. Raabová}

{%%%%%   Z8-II-1
...}
\podpis{...}

{%%%%%   Z8-II-2
V domácej úlohe na výpočet hodnoty výrazu
$$
2 - 3 + 4 - 5 + 6 - 7 + 8 - 9 + 10 =
$$
si Rado zabudol zapísať dvoje zátvorky. Takže mu pri správnom počítaní vyšiel výsledok o $18$
väčší, ako by získal, keby na zátvorky nezabudol. Doplňte dvomi rôznymi spôsobmi zátvorky
a napíšte, aké číslo Radovi vyšlo a aké mu malo vyjsť.}
\podpis{M. Dillingerová}

{%%%%%   Z8-II-3
V priebehu prvých jedenásť dní odpovedalo na anketnú otázku 700 ľudí. Každý z nich si vybral
práve jednu z troch ponúkaných možností. Pomer početnosti jednotlivých odpovedí bol $4:7:14$.
Dvanásty deň sa ankety zúčastnilo ešte niekoľko ľudí, čím sa pomer početnosti odpovedí zmenil
na $6:9:16$. Koľko najmenej ľudí muselo odpovedať na anketu dvanásty deň?}
\podpis{L. Šimůnek}

{%%%%%   Z9-II-1
Princ Zrýchlený pozval princeznú Spomalenú na svoj hrad. Keď dlho nechodila, vydal sa jej
naproti. Po dvoch dňoch putovania ju stretol v jednej pätine jej cesty. Spolu pokračovali
v ceste už 2-krát tak rýchlo, ako keď cestovala princezná sama. Na princov hrad dorazili
druhú sobotu od vzájomného stretnutia. Ktorý deň sa stretli, ak princezná zo svojho hradu
vyrazila v piatok?}
\podpis{M. Dillingerová}

{%%%%%   Z9-II-2
Vnútri štvorca so stranou dĺžky $4\cm$ sa pohybuje kruh s priemerom dĺžky $1\cm$
tak, že sa stále dotýka obvodu štvorca. Vypočítaj obsah tej časti štvorca, ktorá
nemôže byť nikdy prekrytá pohybujúcim sa kruhom.}
\podpis{M. Dillingerová}

{%%%%%   Z9-II-3
Do okresného kola súťaže postúpili Peter, Mojmír, Karol a Eva. V škole potom povedali:
\begin{itemize}
\item Eva: "Z našej štvorice som nebola ani prvá ani posledná."
\item Mojmír: "Nebol som z našej štvorice posledný."
\item Karol: "Ja som bol z nás štyroch prvý."
\item Peter: "Ja som bol z našej štvorice posledný."
\end{itemize}
Vieme, že tri deti hovorili pravdu a jedno klamalo. Kto z nich bol v okresnom kole najlepší?}
\podpis{M. Volfová}

{%%%%%   Z9-II-4
Upratovačka umývala schodisko v mrakodrape. Aby jej práca lepšie ubiehala, počítala umyté
schody. V dobe, keď mala umytých presne polovicu všetkých schodov, si urobila prestávku.
Za chvíľu sa pustila znova do práce a chcela pokračovať v počítaní schodov. Keď si ale
spomínala na počet už umytých schodov, dopustila sa omylu. Správne trojciferné číslo
"prečítala" odzadu, čím vzniklo číslo menšie. Ďalej teda počítala schody od tohto menšieho
čísla. Po umytí všetkých schodov došla k číslu $746$. Koľko schodov mohla umyť
v skutočnosti ak sa už viackrát nepomýlila?}
\podpis{L. Šimůnek}

{%%%%%   Z9-III-1
Prirodzené dvojciferné číslo $N$ strašne závidelo svojmu priateľovi, dvojcifernému desatinnému
číslu $x$, jeho desatinnú čiarku. Číslo $x$ malo dobré srdce, a tak mu tú svoju darovalo. Celé šťastné $N$ si ju
vložilo medzi svoje dve číslice a vôbec mu nevadilo, že je teraz o $567$ desatín menšie, ako bolo
predtým. Aj $x$ bolo spokojné, lebo teraz bolo na číselnej osi k svojmu priateľovi $N$ dva razy
bližšie, ako predtým. Zistite, o ktorých dvoch číslach $N$ a $x$ je táto príhoda.}
\podpis{S. Bednářová}

{%%%%%   Z9-III-2
Vnútri rovnostranného trojuholníka so stranou dĺžky $4\cdot\sqrt3\cm$ sa pohybuje kruh s priemerom
dĺžky $1\cm$ tak, že sa stále dotýka obvodu trojuholníka. Vypočítaj obsah tej časti trojuholníka,
ktorá nemôže byť nikdy prekrytá pohybujúcim sa kruhom.}
\podpis{M. Dillingerová}

{%%%%%   Z9-III-3
Mamička pripravila na oslavu Jurkových narodenín pomarančový džús do džbánu tak, že
zmiešala 1 liter 100\%-ného džúsu s $\frac23$ litra 30\%-ného džúsu. Jurko si odlial do pohára
a ochutnal. Pretože má radšej slabšiu koncentráciu džúsu, dolial do pripraveného džbánu vodu do
pôvodného množstva. Výsledný džús mal koncentráciu 61{,}2\%. To už Jurkovi vyhovovalo. Aké
množstvo džúsu si Jurko odlial do pohára?}
\podpis{M. Raabová}

{%%%%%   Z9-III-4
Je daný trojuholník. Ak jeho najdlhšiu stranu skrátime o tretinu jej dĺžky, najkratšiu
zdvojnásobíme a poslednú zmenšíme o $2\cm$, dostaneme trojuholník zhodný s pôvodným
trojuholníkom. Zistite rozmery pôvodného trojuholníka.}
\podpis{M. Raabová}

