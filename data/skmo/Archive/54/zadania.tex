{%%%%%   A-I-1
Neprázdnu podmnožinu prirodzených čísel
nazveme {\it malou\/}, keď má menej prvkov, ako je jej najmenší
prvok. Určte počet všetkých tých malých množín $\mm M$, ktoré sú
podmnožinami množiny $\{1,2,3,\dots,100\}$ a~majú nasledovnú vlastnosť:
ak do $\mm M$ patria dve rôzne čísla $x$ a~$y$, potom do
$\mm M$ patrí aj číslo $|x-y|$.}
\podpis{J. Földes}

{%%%%%   A-I-2
Nech $M$ je ľubovoľný vnútorný bod kratšieho oblúka~$CD$
kružnice opísanej štvorcu $ABCD$. Označme $P$, $R$ priesečníky
priamky~$AM$ postupne s~úsečkami $BD$, $CD$ a~podobne $Q$, $S$
priesečníky priamky~$BM$ s~úsečkami $AC$, $DC$.
Dokážte, že priamky $PS$ a~$QR$ sú navzájom kolmé.}
\podpis{J. Švrček}

{%%%%%   A-I-3
Nech $k$ je ľubovoľné prirodzené číslo. Uvažujme
dvojice $(a,b)$ celých čísel, pre ktoré majú kvadratické rovnice
$$
x^2-2ax+b=0,\qquad y^2+2ay+b=0
$$
reálne korene (nie nutne rôzne), ktoré možno označiť
$x_{1,2}$ resp.~$y_{1,2}$ v~takom poradí, že platí rovnosť
$x_1y_1-x_2y_2=4k$.
\ite a) Pre dané $k$ určte najväčšiu možnú hodnotu $b$ zo všetkých
takých dvojíc $(a,b)$.
\ite b) Pre $k=2\,004$ určte počet všetkých takých dvojíc $(a,b)$.
\ite c) Pre dané $k$ vypočítajte súčet čísel~$b$ zo všetkých takých
dvojíc $(a,b)$, pričom každé číslo~$b$ sa pripočíta toľkokrát, v~koľkých
dvojiciach $(a,b)$ vystupuje.

}
\podpis{E. Kováč}

{%%%%%   A-I-4
Dané aritmetické postupnosti $(x_i)_{i=1}^{\infty}$
a~$(y_i)_{i=1}^{\infty}$ majú rovnaký prvý člen a~nasledovnú
vlastnosť: existuje index $k$ ($k>1$), pre ktorý platia rovnosti
$$
x_{k}^2-y_{k}^2=53,\quad
x_{k-1}^2-y_{k-1}^2=78,\quad
x_{k+1}^2-y_{k+1}^2=27.
$$
Nájdite všetky také indexy $k$.}
\podpis{V. Bálint}

{%%%%%   A-I-5
V~lichobežníku $ABCD$, kde $AB\parallel CD$,
platí $|AB|=2|CD|$. Označme $E$ stred ramena~$BC$. Dokážte, že
rovnosť $|AB|=|BC|$ platí práve vtedy, keď štvoruholník $AECD$ je
dotyčnicový.}
\podpis{R. Horenský}

{%%%%%   A-I-6
Nájdite všetky funkcie
$f:\langle0,+\infty)\to\langle0,+\infty)$, ktoré splňujú
zároveň tri nasledovné podmienky:
\ite a) Pre ľubovoľné nezáporné čísla $x$, $y$ také, že
$x+y>0$, platí rovnosť
$$
f\bigl(x f(y)\bigr)f(y)=f\Bigl(\frac{xy}{x+y}\Bigr);
$$
\ite b) $f(1)=0$;
\ite c) $f(x)>0$ pre ľubovoľné $x>1$.}
\podpis{P. Calábek}

{%%%%%   B-I-1
Určte všetky dvojice $(a,b)$ reálnych čísel, pre ktoré
má každá z~rovníc
$$
x^2+ax+b=0,\qquad
x^2+(2a+1)x+2b+1=0
$$
dva rôzne reálne korene, pričom korene druhej rovnice sú
prevrátenými hodnotami koreňov prvej rovnice.}
\podpis{E. Kováč}

{%%%%%   B-I-2
Daný je rovnobežník $ABCD$. Priamka vedená bodom~$D$ pretína úsečku~$AC$
v~bode~$G$, úsečku~$BC$ v~bode~$F$ a~polpriamku~$AB$ v~bode~$E$
tak, že trojuholníky $BEF$ a~$CGF$ majú rovnaký obsah. Určte pomer
$|AG|:|GC|$.}
\podpis{T. Jurík}

{%%%%%   B-I-3
Na stole leží $k$~hromádok o~$1, 2, 3, \dots, k$ kameňoch, kde
$k\ge3$. V~každom kroku vyberieme tri ľubovoľné hromádky na stole,
zlúčime ich do jednej a~pridáme k~nej jeden kameň, ktorý dovtedy na stole
nebol. Dokážte, že ak po niekoľkých krokoch vznikne jediná
hromádka, potom výsledný počet kameňov nie je deliteľný tromi.}
\podpis{J. Zhouf}

{%%%%%   B-I-4
Označme $V$ priesečník výšok a~$S$ stred kružnice opísanej  trojuholníku
$ABC$, ktorý nie je rovnostranný. Dokážte, že ak uhol pri vrchole~$C$
má $60\st$, potom os uhla $ACB$ je osou úsečky~$V\!S$.}
\podpis{J. Zhouf}

{%%%%%   B-I-5
V~obore reálnych čísel vyriešte rovnicu
$$
{x\over x+4}={5\lfloor x \rfloor-7\over7\lfloor x\rfloor-5},
$$
kde $\lfloor x \rfloor$ označuje najväčšie celé číslo, ktoré nie je väčšie ako~$x$
(tzv.~{\it dolná celá časť\/} reálneho čísla~$x$).}
\podpis{J. Šimša}

{%%%%%   B-I-6
Do kružnice~$k$ s~polomerom~$r$ sú vpísané dve kružnice $k_1$,
$k_2$ s~polomerom~$r/2$, ktoré sa vzájomne dotýkajú. Kružnica~$\ell$
sa zvonka dotýka kružníc $k_1$, $k_2$ a~s~kružnicou~$k$ má
vnútorný dotyk. Kružnica~$m$ má vonkajší dotyk s~kružnicami $k_2$
a~$\ell$ a~vnútorný dotyk s~kružnicou~$k$. Vypočítajte polomery kružníc
$\ell$ a~$m$.}
\podpis{L. Boček}

{%%%%%   C-I-1
Nech $a$, $b$, $c$, $d$ sú také reálne čísla, že $a+d=b+c$.
Dokážte nerovnosť
$$
(a-b)(c-d)+(a-c)(b-d)+(d-a)(b-c)\ge0.
$$}
\podpis{E. Kováč}

{%%%%%   C-I-2
Zistite, pre ktoré prirodzené čísla $n\ge2$ je možné z~množiny
$\{1,2,\dots,n-1\}$ vybrať navzájom rôzne párne čísla
tak, aby ich súčet bol deliteľný číslom~$n$.}
\podpis{J. Zhouf}

{%%%%%   C-I-3
V~ľubovoľnom konvexnom štvoruholníku $ABCD$ označme $E$ stred
strany~$BC$ a~$F$ stred strany~$AD$. Dokážte, že trojuholníky $AED$
a~$BFC$ majú rovnaký obsah práve vtedy, keď sú strany $AB$ a~$CD$
rovnobežné.}
\podpis{J. Šimša}

{%%%%%   C-I-4
Tri štvormiestne čísla $k$, $\ell$, $m$ majú rovnaký tvar $ABAB$,
\tj.~číslica na mieste jednotiek je rovnaká ako číslica na mieste
stoviek a~číslica na mieste desiatok je rovnaká ako číslica na mieste
tisícok. Číslo $\ell$ má číslicu na mieste jednotiek o~$2$ väčšiu
a~číslicu na mieste desiatok o~$1$ menšiu ako číslo~$k$. Číslo~$m$ je
súčtom čísel $k$ a~$\ell$ a~je deliteľné deviatimi. Určte všetky
také čísla~$k$.}
\podpis{T. Joska}

{%%%%%   C-I-5
Určte počet všetkých trojíc dvojmiestnych prirodzených čísel $a$,
$b$, $c$, ktorých súčin $abc$ má zápis, v~ktorom sú všetky
číslice rovnaké. Trojice líšiace sa len poradím čísel považujeme
za rovnaké, \tj. započítavame ich len raz.}
\podpis{J. Šimša}

{%%%%%   C-I-6
V~trojuholníku $ABC$ so stranou~$BC$ dĺžky $2\cm$ je bod~$K$ stredom
strany~$AB$. Body~$L$ a~$M$ rozdeľujú
stranu~$AC$ na tri zhodné úsečky. Trojuholník $KLM$ je
rovnoramenný a~pravouhlý. Určte dĺžky strán $AB$, $AC$
všetkých takých trojuholníkov $ABC$.}
\podpis{P. Leischner}

{%%%%%   A-S-1
Určte počet všetkých nekonečných aritmetických postupností celých
čísel, ktoré majú medzi svojimi prvými desiatimi členmi obe čísla~$1$
a~$2\,005$.}
\podpis{V. Bálint, J. Šimša}

{%%%%%   A-S-2
V~rovnobežníku~$ABCD$ platí $|AB|>|BC|$. Označme $K$, $L$, $M$
a~$N$ postupne body dotyku kružníc vpísaných trojuholníkom $ACD$, $BCD$,
$ABC$ a~$ABD$ s~príslušnou uhlopriečkou~$AC$, resp.~$BD$. Dokážte,
že $KLMN$ je obdĺžnik.}
\podpis{R. Horenský}

{%%%%%   A-S-3
Zistite, pre ktoré prirodzené čísla~$k$ má sústava nerovníc
$$
k(k-2)\le\Bigl(k+\frac{1}{k}\Bigr)x\le k^2(k+3)
$$
s~neznámou~$x$ práve $(k+1)^2$ riešení v~obore celých čísel.}
\podpis{J. Šimša}

{%%%%%   A-II-1
Ak je súčin kladných reálnych čísel $a$, $b$, $c$ rovný~$1$, platí
nerovnosť
$$
\frac{a}{(a+1)(b+1)}+\frac{b}{(b+1)(c+1)}+
\frac{c}{(c+1)(a+1)}\ge\frac34.
$$
Dokážte a~zistite, kedy nastáva rovnosť.}
\podpis{J. Šimša}

{%%%%%   A-II-2
V~obore celých čísel riešte sústavu rovníc
$$
\align
x(y+z+1)&=y^2+z^2-5,\\
y(z+x+1)&=z^2+x^2-5,\\
z(x+y+1)&=x^2+y^2-5.
\endalign
$$}
\podpis{J. Šimša}

{%%%%%   A-II-3
V~rovine je daný rovnoramenný trojuholník~$KLM$ so základňou~$KL$.
Uvažujme ľubovoľné dve kružnice $k$ a~$\ell$, ktoré majú vonkajší
dotyk a~ktoré sa dotýkajú priamok $KM$ a~$LM$ postupne v~bodoch
$K$ a~$L$. Určte množinu dotykových bodov~$T$ všetkých
takých kružníc $k$ a~$\ell$.}
\podpis{J. Švrček}

{%%%%%   A-II-4
Nájdite všetky dvojice prirodzených čísel, ktorých
súčet má poslednú číslicu~$3$, rozdiel je prvočíslo a~súčin
je druhou mocninou prirodzeného čísla.}
\podpis{J. Földes}

{%%%%%   A-III-1
Uvažujme ľubovoľné aritmetické postupnosti reálnych
čísel $(x_i)_{i=1}^{\infty}$ a~$(y_i)_{i=1}^{\infty}$, ktoré
majú rovnaký prvý člen a~spĺňajú pre niektoré $k>1$ rovnosti
$$
x_{k-1}y_{k-1}=42,\quad
x_{k}y_{k}=30\quad\text{a}\quad
x_{k+1}y_{k+1}=16.
$$
Nájdite všetky také postupnosti, pre ktoré je index~$k$
najväčší možný.}
\podpis{J. Šimša}

{%%%%%   A-III-2
Zistite, pre ktoré~$m$ existuje práve $2^{15}$ podmnožín~$\mm X$
množiny $\{1,2,3,\dots,47\}$ s~vlastnosťou: Číslo~$m$ je najmenší
prvok množiny~$\mm X$ a~pre každé $x\in\mm X$ platí buď $x+m\in\mm X$,
alebo $x+m>47$.}
\podpis{R. Kučera}

{%%%%%   A-III-3
V~lichobežníku~$ABCD$ ($AB\parallel CD$) označme $E$ stred ramena~$BC$.
Ak sú oba štvoruholníky $ABED$ a~$AECD$ dotyčnicové, spĺňajú
dĺžky strán lichobežníka $ABCD$ označené zvyčajným spôsobom rovnosti
$$
a+c=\frac{b}{3}+d\qquad\text{a}\qquad \frac{1}{a}+\frac{1}{c}=
\frac{3}{b}.
$$
Dokážte.}
\podpis{R. Horenský}

{%%%%%   A-III-4
V~rovine je daný ostrouhlý trojuholník~$AKL$. Uvažujme ľubovoľný
pravouholník $ABCD$, ktorý je trojuholníku~$AKL$ opísaný tak, že bod~$K$
leží na strane~$BC$ a~bod~$L$ leží na strane~$CD$. Určte množinu
priesečníkov~$S$ uhlopriečok $AC$, $BD$ všetkých takých pravouholníkov~$ABCD$.}
\podpis{J. Šimša}

{%%%%%   A-III-5
Dokážte, že pre ľubovoľné reálne čísla $p$, $q$, $r$, $s$ za
podmienok $q\ne\m1$ a~$s\ne\m1$ platí: Kvadratické rovnice
$$
x^2+px+q=0,\qquad x^2+rx+s=0
$$
majú v~obore reálnych čísel spoločný koreň a~ich ďalšie korene
sú navzájom prevrátené čísla práve vtedy, keď koeficienty
$p$, $q$, $r$, $s$ spĺňajú rovnosti
$$
pr=(q+1)(s+1)        \quad\text{a}\quad
p(q+1)s=r(s+1)q.
$$
(Dvojnásobný koreň kvadratickej rovnice počítame dvakrát.)}
\podpis{J. Šimša}

{%%%%%   A-III-6
Rozhodnite, či pre každé poradie čísel $1, 2, 3, \dots, 15$
možno tieto čísla zapísať najviac štyrmi rôznymi farbami tak, aby
všetky čísla rovnakej farby tvorili v~danom poradí monotónnu
(\tj.~ rastúcu alebo klesajúcu) postupnosť.
(Jednočlenná postupnosť je monotónna.)}
\podpis{J. Šimša}

{%%%%%   B-S-1
Na stole leží $54$~kôpok s~$1, 2, 3, \dots, 54$ kameňmi.
V~každom kroku vyberieme ľubovoľnú kôpku, povedzme s~$k$~kameňmi,
a~odoberieme ju celú zo stola spolu
s~$k$~kameňmi z~každej tej kôpky, v~ktorej je aspoň $k$~kameňov.
Napríklad po prvom kroku, v~ktorom vyberieme  kôpku s~$52$~kameňmi,
zostanú na stole kôpky s~$1, 2, 3, \dots, 51, 1$ a~$2$ kameňmi.
Predpokladajme, že po určitom počte krokov zostane na
stole jediná kôpka. Zdôvodnite, koľko kameňov v~nej môže byť.}
\podpis{J. Šimša}

{%%%%%   B-S-2
Nech $ABC$ je pravouhlý trojuholník so stranami $a<b<c$. Označme
$Q$ stred odvesny~$BC$ a~$S$ stred prepony~$AB$. Priesečník osi
úsečky~$AB$ s~odvesnou~$CA$ označme~$R$. Dokážte, že $|RQ|=|RS|$ práve vtedy, keď
$$
a^2:b^2:c^2=1:2:3.
$$}
\podpis{J. Švrček}

{%%%%%   B-S-3
V~obore reálnych čísel riešte rovnicu
$$
\lfloor{\frac{x}{1-x}}\rfloor=\frac{\lfloor{x}\rfloor}{1-\lfloor{x}\rfloor},
$$
kde $\lfloor{a}\rfloor$ označuje najväčšie celé číslo, ktoré neprevyšuje
číslo~$a$.}
\podpis{J. Šimša}

{%%%%%   B-II-1
Kružnica~$k_1$ s~polomerom~$1$ má vonkajší dotyk s~kružnicou~$k_2$
s~polomerom~$2$. Každá z~kružníc $k_1$, $k_2$ má vnútorný dotyk
s~kružnicou~$k_3$ s~polomerom~$3$. Vypočítajte polomer kružnice~$k$,
ktorá má s~kružnicami $k_1$, $k_2$ vonkajší dotyk a~s~kružnicou~$k_3$
vnútorný dotyk.}
\podpis{Pavel Novotný}

{%%%%%   B-II-2
Na jednej internetovej stránke prebieha hlasovanie o~najlepšieho
hokejistu sveta posledného desaťročia. Počet hlasov pre jednotlivých
hráčov sa uvádza po zaokrúhlení v~celých percentách. Po Jožkovom
hlasovaní pre Miroslava Šatana sa jeho zisk 7\,\% nezmenil. Najmenej
koľko ľudí vrátane Jožka hlasovalo? Predpokladáme, že každý
účastník ankety hlasoval práve raz, a~to pre jediného hráča.}
\podpis{M. Panák}

{%%%%%   B-II-3
Nech $ABC$ je ostrouhlý trojuholník. Označme $K$ a~$L$ päty
výšok z~vrcholov $A$ a~$B$, $M$~stred strany~$AB$ a~$V$ priesečník
výšok trojuholníka~$ABC$. Dokážte, že os uhla~$KML$ prechádza
stredom úsečky~$V\!C$.}
\podpis{J. Švrček}

{%%%%%   B-II-4
Nájdite všetky trojice reálnych čísel $x$, $y$, $z$, pre
ktoré platí
$$
\lfloor{x}\rfloor-y=2\cdot\lfloor{y}\rfloor-z=3\cdot\lfloor{z}\rfloor-x=\frac{2004}{2005},
$$
kde $\lfloor{a}\rfloor$ označuje najväčšie celé číslo, ktoré neprevyšuje
číslo~$a$.}
\podpis{J. Šimša}

{%%%%%   C-S-1
Nájdite všetky trojice celých čísel $x$, $y$, $z$, pre ktoré platí
$$
\align
x + yz &= 2005, \\
y + xz &= 2006.\\
\endalign
$$}
\podpis{J. Šimša}

{%%%%%   C-S-2
Pre ktoré prirodzené čísla~$n$ možno z~množiny
$\{n, n + 1, n + 2,\dots, n^2\}$ vybrať štyri navzájom rôzne
čísla $a$, $b$, $c$, $d$ tak, aby platilo  $ab = cd$\,?}
\podpis{J. Šimša}

{%%%%%   C-S-3
Je daná úsečka~$AB$. Zostrojte bod~$C$ tak, aby sa obsah
trojuholníka~$ABC$ rovnal $1/8$ obsahu~$S$ štvorca so stranou~$AB$
a~súčet obsahov štvorcov so stranami $AC$ a~$BC$ sa rovnal~$S$.}
\podpis{A. Jančařík}

{%%%%%   C-II-1
Určte číslice $x$, $y$, $z$ tak, aby platila rovnosť
$$
{x+y\over z}=\overline{z{,}yx},
$$
kde $\overline{z{,}yx}$ označuje číslo zložené zo $z$~jednotiek,
$y$~desatín a~$x$~stotín.}
\podpis{J. Zhouf}

{%%%%%   C-II-2
Ku každému prirodzenému číslu $n>2$ nájdite aspoň jednu dvojicu
rôznych prirodzených čísel $p$, $q$ tak, aby číslo $1/n$ bolo
aritmetickým priemerom čísel $1/p$ a~$1/q$.}
\podpis{L. Boček}

{%%%%%   C-II-3
Ľubovoľným vnútorným bodom~$P$ uhlopriečky~$AC$ daného obdĺžnika~$ABCD$
sú vedené rovnobežky s~jeho stranami tak, že pretínajú
úsečky $AB$, $BC$, $CD$ a~$DA$ postupne v~bodoch $K$, $L$, $M$
a~$N$. Dokážte, že
\ite a) priamky $LM$ a~$KN$ sú rovnobežky,
\ite b) vzdialenosť rovnobežiek $LM$ a~$KN$ je konštantná
        (nezávisí na voľbe bodu~$P$),
\ite c) pre obvod~$o$ štvoruholníka~$KLMN$ platí nerovnosť $o\ge2|AC|$.

}
\podpis{J. Švrček}

{%%%%%   C-II-4
Popíšte konštrukciu lichobežníka~$ABCD$ so základňami $AB$ a~$CD$,
ktorému sa dá opísať kružnica s~polomerom $r=5\cm$, keď je daná
vzdialenosť $d=2\cm$ jej stredu od priesečníka uhlopriečok
a~$|\uhol BAC|=70^\circ$.}
\podpis{E. Kováč}

{%%%%%   vyberko, den 1, priklad 1
Daná je priamka $p$ a kružnica $k$, ktorá s ňou nemá
spoločný bod. Nech $AB$ je jej priemer kolmý na $p$, pričom $B$ je
bližšie k $p$ ako $A$. Vyberieme ľubovoľný bod $C \ne A,B$ na
$k$. Priamka $AC$ pretína priamku $p$ v bode $D$. Priamka $DE$ je
dotyčnicou $k$ v bode $E$, pričom $B$ a $E$ sú na tej istej strane
$AC$. Priamka $BE$ pretína $p$ v bode $F$ a priamka $AF$ pretína
$k$ v $G \ne A$. Nech $H$ je obraz bodu $G$ v súmernosti podľa
priamky $AB$. Dokážte, že $H$ leží na priamke $CF$.}
\podpis{Martin Potočný:???}

{%%%%%   vyberko, den 1, priklad 2
Nech $a$, $b$, $c$ sú kladné reálne čísla, pre ktoré platí $ab+bc+ca=1$. Dokážte nerovnosť
$$
\root 3 \of {{1\over a}+6b} + \root 3 \of {{1\over b}+6c} + \root 3 \of {{1\over c}+6a} \leq {1\over abc}.
$$
Kedy nastáva rovnosť?}
\podpis{Martin Potočný:???}

{%%%%%   vyberko, den 1, priklad 3
Nech $A$ je matica typu $n \times n$ a nech $X_i$ je
množina koeficientov v $i$-tom riadku a $Y_j$ je množina
koeficientov v~$j$-tom stĺpci (pre všetky $1 \leq i,j \leq n$).
Maticu $A$ nazveme {\it zlatá}, ak
$X_1,X_2,\dots,X_n,Y_1,Y_2,\dots,Y_n$ sú rôzne množiny. Nájdite
najmenšie $n$ také, že existuje zlatá matica typu $2005 \times
2005$ s množinou koeficientov $\{1,2, \dots,n\}$.}
\podpis{Martin Potočný:???}

{%%%%%   vyberko, den 2, priklad 1
Reálne čísla $x$, $y$, $z$ spĺňajú vzťahy
$$
\eqalign{
x+y+z = 4,\cr
x^2+y^2+z^2 = 6.\cr
}
$$
Aké hodnoty môže nadobúdať $x$?}
\podpis{Ján Mazák:???}

{%%%%%   vyberko, den 2, priklad 2
Daná je tabuľka $25\times 100$. Allan a~Bob hrajú takúto hru: Hráč, ktorý je na ťahu, nakreslí
trojuholník s~vrcholmi v stredoch políčok tabuľky. Žiadne dva nakreslené trojuholníky nesmú mať spoločný
bod. V ťahoch sa hráči pravidelne striedajú. Prehráva ten, čo je na ťahu, ale už nemôže nakresliť žiaden
ďalší trojuholník. Allan začína.
Má niektorý z hráčov víťaznú stratégiu?}
\podpis{Ján Mazák:???}

{%%%%%   vyberko, den 2, priklad 3
Daný je ostrouhlý trojuholník $ABC$. Nech $P$, $N$ sú päty jeho výšok z vrcholov $A$, $B$.
Nech $K$, $L$ sú priesečníky osí uhlov $BAC$, $ABC$ s protiľahlými stranami. Nech $O$ je
stred opísanej kružnice a $I$ stred vpísanej kružnice trojuholníka $ABC$.
Dokážte, že body $N$, $P$, $I$ sú kolineárne práve vtedy, keď body $L$, $K$, $O$ sú kolineárne.}
\podpis{Ján Mazák:???}

{%%%%%   vyberko, den 2, priklad 4
Nájdite všetky prirodzené čísla $n$ také, že $n\mid 3^n-2^n$.}
\podpis{Ján Mazák:???}

{%%%%%   vyberko, den 3, priklad 1
Nech $P$ je konvexný mnohouholník. Dokážte,
že existuje konvexný šesťuholník, ktorý je vpísaný
do~$P$ a~obsahuje aspoň 75$\%$ jeho obsahu.}
\podpis{Tomáš Jurík:IMO 2004 shortlist G6}

{%%%%%   vyberko, den 3, priklad 2
Majme tri postupnosti $(x_1,x_2,\dots,x_n)$,
$(y_1,y_2,\dots,y_n)$, $(z_2,z_3,\dots,z_{2n})$
kladných reálnych čísel, pričom platí
$$
z_{i+j}\ge x_iy_j,\quad\text{pre všetky }1\le i, j\le n.
$$
Označme $M=\max\{z_2,z_3,\dots,z_{2n}\}$. Dokážte nerovnosť
$$
\left({M+z_2+z_3+\cdots+z_{2n}}\over{2n}\right)^{\!2}\ge
{x_1+x_2+\cdots+x_n \over n}
\cdot{y_1+y_2+\cdots+y_n \over n}.
$$}
\podpis{Tomáš Jurík:IMO 2003 shortlist A6}

{%%%%%   vyberko, den 3, priklad 3
Nech $f(k)$ označuje počet prirodzených čísel $n$ s~vlastnosťami
\ite (i) $0\le n<10^k$, \tj. číslo $n$ má v~desiatkovom zápise práve $k$~číslic (nuly na začiatku sú povolené);
\ite (ii) číslice čísla $n$ môžu byť poprehadzované tak, že výsledné číslo bude deliteľné číslom 11 bezo zvyšku.

Dokážte, že pre každé prirodzené číslo $m$ platí
$f(2m)=10f(2m-1)$.}
\podpis{Tomáš Jurík:IMO 2003 shortlist C6}

{%%%%%   vyberko, den 4, priklad 1
Nájdite všetky funkcie $f:\Bbb R\to\Bbb R$, ktoré spĺňajú rovnosť
$$
f(xf(x)+f(y)) = f(x)^2 + y
$$
pre každú dvojicu reálnych čísel $x$, $y$.}
\podpis{Peter Novotný:Functional Equations, A Problem Solving Approach (Problem 3.8, str. 67)}

{%%%%%   vyberko, den 4, priklad 2
Pre dané prirodzené číslo $n>1$ označme $s_n$ súčin všetkých takých kladných celých čísel~$x$ menších ako~$n$, že $n$ je deliteľom čísla $x^2-1$. (Máme teda $s_2=1$, $s_3=1\cdot2$, $s_4=1\cdot3$,...) Pre každé $n>1$ určte zvyšok $s_n$ po delení číslom~$n$.}
\podpis{Peter Novotný:Shortlist Atény 2004}

{%%%%%   vyberko, den 4, priklad 3
Daný je trojuholník $ABC$. Označme postupne $P$, $Q$, $R$ päty kolmíc spustených z~vrcholov $A$, $B$, $C$ na osi vonkajších uhlov trojuholníka pri vrcholoch $C$, $A$, $B$. Označme~$d$ priemer kružnice opísanej trojuholníku~$PQR$. Dokážte, že $d^2=\varrho^2+s^2$, pričom $\varrho$ je polomer kružnice vpísanej do trojuholníka~$ABC$ a~$s$ je polovica obvodu trojuholníka~$ABC$.}
\podpis{Peter Novotný:Bulgarian Mathematical Competitions 1997-2002, Final Round 2002 (str. 201)}

{%%%%%   vyberko, den 5, priklad 1
Označme $p(k)$ najväčší nepárny deliteľ prirodzeného čísla $k \geq 1$. Dokážte, že pre každé prirodzené
číslo $n$ platí
$$
{2\over 3} n < \sum_{k=1}^{n} {{p(k)\over k} } < {2 \over 3} (n+1).
$$}
\podpis{Tomáš Váňa:???}

{%%%%%   vyberko, den 5, priklad 2
Políčka šachovnice $ n \times n $, kde $ n \geq 3 $, sú zafarbené na čierno a bielo klasickým spôsobom.
V jednom ťahu môžeme vybrať štvorec $ 2 \times 2 $ a zmeniť farbu všetkých jeho políčok na opačnú.
Nájdite všetky $n$ také, že po konečnom počte popísaných ťahov vieme zafarbiť šachovnicu tak, že
všetky políčka majú rovnakú farbu.}
\podpis{Tomáš Váňa:???}

{%%%%%   vyberko, den 5, priklad 3
Nech $O$ je stred kružnice opísanej ostrouhlému trojuholníku $ABC$, v ktorom platí $\beta < \gamma$.
Priamka $AO$ pretína stranu $BC$ v bode $D$. Stredy kružníc opísaných trojuholníkom $ABD$ a $ACD$
označme postupne $E$ a $F$. Bod $G$ leží na priamke $AB$ tak, že bod $A$ je vnútorným bodom úsečky $BG$
a platí $|AG| = |AC|$. Bod $H$ leží na priamke $AC$ tak, že bod $A$ je vnútorným bodom úsečky $CH$
a platí $|AH| = |AB|$. Dokážte, že štvoruholník $EFGH$ je
obdĺžnik práve vtedy, keď $\gamma - \beta = 60 ^\circ $.}
\podpis{Tomáš Váňa:???}

{%%%%%   vyberko, den 5, priklad 4
Nájdite všetky funkcie $f : \Bbb N_0 \rightarrow \Bbb N_0$ spĺňajúce $f(1) > 0$ a rovnosť $ f( m^2 + n^2 ) = f(m)^2 + f(n)^2 $
pre všetky $m$, $n$ z $\Bbb N_0$.}
\podpis{Tomáš Váňa:???}

{%%%%%   vyberko, den 1, priklad 4
-}
\podpis{-:-}

{%%%%%   vyberko, den 3, priklad 4
-}
\podpis{-:-}

{%%%%%   vyberko, den 4, priklad 4
-}
\podpis{-:-}

{%%%%%   trojstretnutie, priklad 1
Nech $n$ je dané prirodzené číslo.
V~obore nezáporných reálnych čísel vyriešte sústavu rovníc
$$
\eqalign{x_1+x_2^2+x_3^3+\cdots+x_n^n&=n,\cr
x_1+2x_2+3x_3+\cdots+nx_n&={n(n+1)\over 2}\cr}
$$
s~neznámymi $x_1,x_2,\dots,x_n$.}
\podpis{...}

{%%%%%   trojstretnutie, priklad 2
Konvexný štvoruholník $ABCD$ je vpísaný do kružnice so stredom~$O$
a~opísaný kružnici so stredom~$I$.
Uhlopriečky $AC$ a~$BD$ sa pretínajú v~bode~$P$.
Dokážte, že body $O$, $I$ a~$P$ ležia na jednej priamke.}
\podpis{...}

{%%%%%   trojstretnutie, priklad 3
Určte všetky prirodzené čísla $n\geq 3$, pre ktoré sa
polynóm
$$P(x)=x^n-3x^{n-1}+2x^{n-2}+6$$
dá vyjadriť ako súčin dvoch polynómov, ktoré majú kladné stupne
a~celočíselné koeficienty.}
\podpis{...}

{%%%%%   trojstretnutie, priklad 4
Rozdeľme $n\geq1$ označených guliek medzi deväť osôb
$A$, $B$, $C$, $D$, $E$, $F$, $G$, $H$, $I$.
Určte, koľkými spôsobmi ich môžeme rozdeliť za podmienky,
že osoba~$A$ dostane rovnaký počet guliek ako osoby $B$, $C$, $D$, $E$ spolu.}
\podpis{...}

{%%%%%   trojstretnutie, priklad 5
Daný je konvexný štvoruholník $ABCD$.
Určte množinu všetkých bodov~$P$ ležiacich vnútri
štvoruholníka $ABCD$, pre ktoré platí
$$S_{PAB}\cdot S_{PCD}=S_{PBC}\cdot S_{PDA},$$
pričom $S_{XYZ}$ označuje obsah trojuholníka $XYZ$.}
\podpis{...}

{%%%%%   trojstretnutie, priklad 6
Nájdite všetky dvojice celých čísel $(x,y)$, ktoré spĺňajú rovnosť
$$y(x+y)=x^3-7x^2+11x-3.$$}
\podpis{...}

{%%%%%   IMO, priklad 1
Na stranách rovnostranného trojuholníka $ABC$ je zvolených šesť bodov: body $A_1$, $A_2$ na strane $BC$, body $B_1$, $B_2$ na strane $CA$ a~body $C_1$, $C_2$ na strane $AB$. Tieto body sú vrcholmi konvexného šesťuholníka $A_1A_2B_1B_2C_1C_2$ s~rovnako dlhými stranami. Dokážte, že priamky $A_1B_2$, $B_1C_2$ a~$C_1A_2$ sa pretínajú v~jednom bode.}
\podpis{Rumunsko}

{%%%%%   IMO, priklad 2
Nech $a_1, a_2, \ldots$ je postupnosť celých čísel s~nekonečným počtom kladných členov a~s~nekonečným počtom záporných členov. Predpokladajme, že pre každé prirodzené číslo~$n$ čísla $a_1, a_2, \ldots, a_n$ po delení číslom~$n$ dávajú $n$~rôznych zvyškov. Dokážte, že každé celé číslo sa v~postupnosti vyskytuje práve raz.}
\podpis{Holandsko}

{%%%%%   IMO, priklad 3
Nech $x$, $y$ a~$z$ sú kladné reálne čísla také, že $xyz \geq 1$. Dokážte, že
$$
\frac{x^5 - x^2}
{x^5 + y^2 + z^2}
+
\frac{y^5 - y^2}
{y^5 + z^2 + x^2}
+
\frac{z^5 - z^2}
{z^5 + x^2 + y^2}
\geq 0.
$$}
\podpis{Južná Kórea}

{%%%%%   IMO, priklad 4
Uvažujme postupnosť $a_1, a_2, \ldots$ definovanú vzťahom
$$a_n = 2^n + 3^n + 6^n -1\ \ (n=1,2,\ldots).$$
Určte všetky kladné celé čísla, ktoré sú nesúdeliteľné s~každým členom postupnosti.}
\podpis{Poľsko}

{%%%%%   IMO, priklad 5
Nech $ABCD$ je daný konvexný štvoruholník s~rovnako dlhými stranami $BC$ a~$AD$, ktoré nie sú rovnobežné. Nech body $E$ a~$F$ ležia postupne vnútri strán $BC$ a~$AD$ tak, že $|BE|=|DF|$. Priamky $AC$ a~$BD$ sa pretínajú v~bode~$P$, priamky $BD$ a~$EF$ v~bode~$Q$, priamky $EF$ a~$AC$ v~bode~$R$. Uvažujme všetky trojuholníky $PQR$ určené meniacou sa polohou bodov $E$ a~$F$. Ukážte, že kružnice opísané týmto trojuholníkom majú spoločný bod rôzny od~$P$.}
\podpis{Poľsko}

{%%%%%   IMO, priklad 6
V~matematickej súťaži bolo súťažiacim zadaných 6~úloh. Každú dvojicu úloh vyriešili viac ako $2/5$~súťažiacich. Nikto nevyriešil všetkých 6~úloh. Dokážte, že práve 5~úloh vyriešili aspoň dvaja súťažiaci.}
\podpis{Rumunsko}

