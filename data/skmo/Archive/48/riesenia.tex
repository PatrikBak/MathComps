{%%%%%   A-I-1
...}

{%%%%%   A-I-2
...}

{%%%%%   A-I-3
...}

{%%%%%   A-I-4
Ak rozklad čísla~ $n$ na prvočinitele je
$n=\prod\limits_{i=1}^{k}p_{i}^{s_{i}}$,
kde $p_{1},\dots,p_{k}$ sú rôzne prvočísla
a~$s_{1},\dots,s_{k}$ nezáporné celé čísla, platí pre počet jeho
kladných deliteľov vzorec
$\tau(n)=\prod\limits_{i=1}^{k}(s_i+1)$.

Aby daná rovnica mala vôbec zmysel, musí byť jej riešenie~ $n$
také prirodzené číslo, pre ktoré je aj~$1{,}6n=\frac85n$
prirodzené, takže $n=2^\a5^\b n'$, kde $\b\ge1$ a~$n'$ je
nesúdeliteľné s~$2\cdot5$. Danú rovnicu potom môžeme prepísať ako
$$
(\a+4)\b\tau(n')=\frac85(\a+1)(\b+1)\tau(n'),
$$
čo po krátení kladným číslom $\tau(n')$ a~ďalšej úprave dá rovnicu
% $$
% 3\b(\a-4)+8(\a+1)=3\b(\a-4)+8(\a-4)+40=(3\b+8)(\a-4)+40=0.
% $$
% Odtiaľ vychádza, že
$$
(3\b+8)(4-\a)=40.
$$
Vzhľadom k~tomu, že $3\b+8\ge11$ a~číslo $3\b+8$ dáva pri
delení tromi zvyšok~2, vyhovuje zo všetkých rozkladov čísla~ 40 na súčin
jedine
$$
3\b+8=20,\qquad 4-\a=2,
$$
teda $\a=2$, $\b=4$, $n=2^2\cdot5^4n'$.

Pre podiel $\tau(0{,}16n):\tau(n)$ tak vychádza
$$
{\tau\bigl(\frac4{25}n\bigr)\over\tau(n)}
={\tau(2^4\cdot5^2)\tau(n')\over\tau(2^2\cdot5^4)\tau(n')}
={5\cdot3\over3\cdot5}=1.
$$
}

{%%%%%   A-I-5
...}

{%%%%%   A-I-6
...}

{%%%%%   B-I-1
...}

{%%%%%   B-I-2
...}

{%%%%%   B-I-3
...}

{%%%%%   B-I-4
...}

{%%%%%   B-I-5
...}

{%%%%%   B-I-6
 Hľadáme vlastne všetky tie hodnoty parametra~ $s$, pre
ktoré má sústava rovníc
$$
\align
 x^{3} + y^{3} =& 3 xy ,\\
        x +  y =&s
\endalign
$$
 riešenie v~obore reálnych čísel. Z~druhej rovnice vyjadríme $y =
s~-x$ a~dosadíme do prvej rovnice, ktorú budeme riešiť
vzhľadom na~neznámu $x$:
$$
\gather
 x^{3} + ( s~- x)^{3} = 3 x( s~- x),  \\
 x^{3} + s^{3} - 3 s^{2} x + 3 sx^{2} - x^{3} = 3 sx - 3x^{2},\\
 3( s~+ 1) x^{2} - 3 s( s~+ 1) x + s^{3} = 0.
\endgather
$$
 Táto rovnica zrejme nemá riešenie pre $s = -1$. Pre $s\ne-1$ dostávame
kvadratickú rovnicu, ktorá má v~obore reálnych čísel
riešenie, práve keď je jej diskriminant~ $D$ nezáporný.
Výpočtom
$$
 D = 9 s^{2}(s + 1)^{2} - 12( s~+ 1) s^{3} = 3 s^{2}( s~+ 1)(3 - s)
$$
 zisťujeme, že $D\ge 0$, práve keď $-1\le s\le3$, čo spolu
s~podmienku $s\ne -1$ dáva hľadanú množinu možných hodnôt
súčtov~ $s$: je to polouzavretý interval $\(-1,3\>$.

 Spätne je vidieť, že pre každé $s$ z~intervalu $\(-1,3\>$
existuje číslo $x$, ktoré je koreňom vyššie uvedenej kvadratickej
rovnice, a~že toto číslo $x$ a~zodpovedajúca hodnota $y= s~- x$
spĺňajú rovnicu $x^{3} + y^{3} = 3 xy$.

 Pre úplnosť vypočítame tie čísla $x$, $y$, ktoré sú
pre ľubovoľné $s\in \(-1,3\>$ riešením uvažovanej sústavy:
$$
x_{1,2}={3 s( s~+ 1)\pm s\sqrt{3( s~+ 1)(3-s)}\over6(s+1)}=
    \frac s2\pm s\sqrt{3-s\over12(s+1)} .
$$
 Po dosadení do $y = s~- x$ zisťujeme, že danému $s$
prináležia dve dvojice
$$
\align
 [x,y] =&\left\{\frac s2+s\sqrt{3-s\over12(s+1)},
              \frac s2-s\sqrt{3-s\over12(s+1)} \right\}\cr
\intext{či}
[x,y] =&\left\{\frac s2-s\sqrt{3-s\over12(s+1)},
                 \frac s2+s\sqrt{3-s\over12(s+1)} \right\}.
\endalign
$$
 Rovnosť $x^{3} + y^{3} = 3 xy$ možno pre nájdené $x$ a~$y$ buď
overiť dosadením a~priamym výpočtom, alebo pomocou Vi\`etových
vzorcov pre korene uvedenej kvadratickej rovnice
$$
 x + y = s,\quad  xy ={s^3\over3(s+1)} ,
$$
 podľa ktorých
$$
 x^{3} + y^{3} = ( x + y)^{3} - 3 xy( x + y) =
    s^{3}-3s\,{s^3\over3(s+1)}={s^3\over s+1} =3 xy.
$$
}

{%%%%%   C-I-1
Najprv zistíme, koľko štvorciferných čísel je možné
zostaviť z~pevne zvolenej štvorice číslic. V~ďalších úvahách
budeme nenulové cifry označovať malými písmenami. Rôzne písmená nikdy
neoznačujú rovnakú číslicu, symbol~ 0 označuje nulu.

 Štvorice číslic $(a,a,a,a)$ a~$(a,0,0,0)$ určujú každá
len jedno štvorciferné číslo, cifry $a$, $a$, $a$, 0 tri:
$a\,aa0$, $a\,a0a$, $a\,0aa$. Z~číslic $a$,$a$,$a$,$b$ ($a\ne
b$) je možné zostaviť len štyri štvorciferné čísla: $a\,aab$,
$a\,aba$, $a\,baa$, $b\,aaa$.

 Z~cifier $a$, $a$, 0, 0 zostavíme len tri čísla $a\,a00$, $a\,0a0$,
$a\,00a$. Pre cifry $a$, $a$, $b$, $b$ ($a\ne b$) máme spolu
šesť možností: $a\,abb$, $a\,bab$, $a\,bba$, $b\,aab$, $b\,aba$,
$b\,baa$. To je stále málo.

 Číslice $a$, $a$, $b$, 0 určujú práve deväť čísel: $a\,ab0$,
$a\,a0b$, $a\,ba0$, $a\,b0a$, $a\,0ab$, $a\,0ba$, $b\,aa0$,
$b\,a0a$, $b\,0aa$. Analogickým postupom zistíme, že vyšetrovanie
ďalších možností už k~počtu~ 9 nevedie. Prehľad všetkých možných
výsledkov je uvedený v~nasledujúcej tabuľke.
$$
\vbox{  \offinterlineskip
\spaceskip .29em
\halign{\strut\vrule#&\ #\hss\ \vrule&&\hss\ $#$ \hss\vrule\cr
\noalign{\hrule}%
 & Typ  &a\,000&a\,aaa&a\,aa0&a\,aab&a\,a00&a\,abb&a\,ab0&a\,b00&a\,abc&a\,bc0&a\,bcd \cr
\noalign{\hrule}%
 & Počet & 1  & 1  & 3  & 4  & 3  & 6  & 9  & 6  & 12  & 18  & 24 \cr
\noalign{\hrule}%
}}
$$

 Dané číslo má teda cifry $a$, $a$, $b$, 0, kde $a$, $b$
sú rôzne nenulové číslice.

  Rozlíšme dve situácie a~zapíšme v~každej z~nich písomné
sčítanie troch najmenších čísel:
$$
\vbox{\let\par\cr
\def\AAAAA{\omit\kern-.5em\vrule width 4em
height.4pt\relax\kern-.5em}
\halign{&\hss#\ &\hss$#$&\qquad\qquad# \cr
           I.& a~< b            &&       II.&  a~> b \cr
\noalign{\smallskip}%
             & a\,0ab           &&          &   b\,0aa \cr
             & a\,0ba           &&          &   b\,a0a \cr
             & a\,a0b           &&          &   b\,aa0 \cr
\noalign{\nointerlineskip\vskip2pt}%
             &\AAAAA            &&          &\AAAAA \cr
             &12\,528           &&          &  12\,528 \cr
}}
$$

  V~prípade I~je z~ľavých dvoch stĺpcov zrejmé, že $a = 4$,
a~z~pravého stĺpca dostaneme $2b + 4 = 8$ alebo $2b + 4 = 18$.
Podmienke $a < b$ vyhovuje len $b = 7$. Ľahko sa presvedčíme, že
cifry $a = 4$ a~$b = 7$ sú riešením úlohy.

 V II. prípade je z~pravých dvoch stĺpcov vidieť, že číslo $2a$ má
poslednú číslicu~ 8 a~zároveň~ 2 alebo~ 1. To však nie je možné.

 Iná možnosť vyšetrenia situácie~ I: Naznačený súčet môžeme
prepísať v tvare $3\,000a + 100a + 10(a+b) + a~+ 2b = 12\,528$.
Úpravou ľahko zistíme, že $a = 4 + (28-4b)/1\,037$.
Z~podmienky, že posledný zlomok je celé číslo $k$, vyjde $b = 7 -
1\,037k/4$, $ a~= 4 + k$. Je zrejmé, že $a$, $b$ budú
číslice len pre $k = 0$. Analogicky možno postupovať aj~v~prípade~II.

  {\it Záver\/}: Číslice hľadaného čísla sú~ 4, 4, 7 a~0.
}

{%%%%%   C-I-2
...}

{%%%%%   C-I-3
...}

{%%%%%   C-I-4
...}

{%%%%%   C-I-5
...}

{%%%%%   C-I-6
Najprv je nutné overiť, či sú dané výrazy definované
pre všetky reálne čísla $a$, $b$. Stačí dokázať, že pre
každé reálne $u$ je výraz $U= u^{2} + u~+ 1$ nezáporný.

  {\it 1. spôsob\/}:
$$
 U~= u^{2} + 2\cdot\frac12 u~+\Bigl(\frac12\Bigr)^{\!2} -
      \Bigl(\frac12\Bigr)^{\!2} + 1
   = \Bigl(u~+ \frac12\Bigr)^{\!2} + \frac34.
$$
 Odtiaľ vidíme, že je dokonca
$$
 U~= u^{2} + u~+ 1 \ge \frac34,      \tag3
$$
 pretože druhá mocnina reálneho výrazu je vždy nezáporná.

 {\it 2. spôsob\/}:
  Pre  $u~\ge 0$  je  zrejme  výraz  $U$ kladný.  Ak je
$u< 0$, je
$$
 U~> u^{2} + u~+ 1 + u~= (u~+ 1)^{2} \ge 0.
$$


{\it 3. spôsob\/}:
  Predstavme  si rovnosť  $U~= u^{2} + u~+ 1$ ako
kvadratickú rovnicu $u^{2} + u~+ (1-U) = 0$ s~parametrom~ $U$.
Tento vzťah je splnený pre nejaké reálne $u$, len keď je
príslušný diskriminant nezáporný, t\.j\.~ $1 - 4(1-U) \ge 0$, a~odtiaľ
$U~\ge \frac34$.

  Ďalej je možné skúšať výraz
$$
V~= \sqrt{a^{2} + a~+ 1} + \sqrt{b^{2} + b + 1}   \tag4
$$
 upravovať, aby sme ho mohli odhadnúť. Ako asi budú
postupovať? Uvedieme niekoľko možností:

 I. Dosadením $b = 1 - a$ do (3) dostaneme
$$
V~= \sqrt{a^{2} + a~+ 1} + \sqrt{a^{2} - 3a + 3}.  \tag5
$$

 Tým sme sa však k~cieľu veľmi nepriblížili. Skúsme ešte obidve strany
 rovnosti~ (5) umocniť:
 $$
\align
 V^{2} =& a^{2} + a^{2} - 2a + 1 + 3 + 2\sqrt{a^{2} + a~+ 1}
          \sqrt{a^{2} - 3a + 3} =\\
       =& a^{2} +  (a~- 1)^{2} + 3 + 2\sqrt{a^{4} - 2a^{3} +a^{2}+3}.
\endalign
$$

 Výraz pod odmocninou sa dá ešte po vyňatí $a$ z~prvých troch členov
 upraviť, takže dostaneme
$$
V^{2} = 3 + a^{2} +(a~- 1)^{2} + 2\sqrt{3 + a^{2} (a- 1)^{2}}
\tag6
$$

 II. Rovnosť  (4) umocníme priamo a~pri ďalších úpravách
opakovane nahrádzame súčty $a~+ b$ jednotkami:
$$
\align
 V^{2} =& a^{2} + b^{2} + 3 + 2\sqrt{a^{2}b^{2} + ab (a~+ b +
         1) + a^{2}+ b^{2} + a~+ b + 1}= \\
       =&3 + a^{2} + 2\sqrt{3 + a^{2}b^{2}}+ b^{2}. \tag7
\endalign
$$

 Dôkaz nerovnosti (1).

  {\smc 1\. riešenie} (bez umocňovania výrazu $V$):
  Ak sú $a$, $b$ nezáporné, je $V~> \sqrt{1}+\sqrt{1}= 2$.
Ak je $b < 0$, potom musí byť $a~> 1$. Položme teda na pravej
strane vzťahu~ (4) $a~= 1$ a~druhú odmocninu odhadneme pomocou~
(3). Dostaneme tak silnejší odhad, ako sa požaduje: $V~>
\sqrt3 + \sqrt{\frac34}= \frac32\sqrt3 > \frac52$.

  {\smc 2\. riešenie}.
  Keď uvážime, že druhá mocnina každého reálneho čísla je
nezáporná, odhadneme z~(6), že $V^{2} \ge 3 + 2\sqrt3 > 4$, a~po
odmocnení dostaneme $V~> 2$.

  {\smc 3\. riešenie}.
  Zo vzťahu (7) vidíme, že
$$
\postdisplaypenalty 10000
\align
V^{2} >& 3 +\bigl(a^{2} +2\sqrt{a^{2}}\sqrt{b^{2}} + b^{2}\bigr)=\\
      =& 3 +\bigl(\sqrt{a^{2}}+\sqrt{b^{2}}\bigr)^{2}=3+(|a|+|b|)^2 \ge 4,
\endalign
$$
 a~teda $V~> 2$.

\medskip
 Dôkaz nerovnosti (2).

  {\smc 1\. riešenie}.
  Pretože $a$, $b$ sú nezáporné a~nemôže platiť $a=b=0$, platí
$$
 V~< \sqrt{a^{2} + 2a + 1}+ \sqrt{b^{2} + 2b + 1}=
   (a~+ 1) + (b + 1) = 3.
$$

  {\smc 2\. riešenie}.
  Z~podmienky $a~+ b = 1$ pre nezáporné  čísla $a$, $b$ máme $0
\le a~\le 1$, $ 0 \le b \le 1$ a~hodnotu výrazu $V$ môžeme
odhadnúť dosadením $a~= b = 1$ do~ (7): $V^{2} < 3 + 1 +2\sqrt 4
+ 1 = 9$, takže $V~< 3$.
}

{%%%%%   A-S-1
...}

{%%%%%   A-S-2
...}

{%%%%%   A-S-3
...}

{%%%%%   A-II-1
Označme $\mm P$ skúmanú množinu prvočísel a~ukážme
najprv, že $2\notin\mm P$. Číslo 2 je jediné prvočíslo, ktoré nie je
nepárne. Keby teda platilo $2\in\mm P$, bol by
súčet {\it nepárneho\/} počtu prvočísel z~$\mm P$ {\it párny},
a~súčet {\it párneho\/} počtu naopak {\it nepárny}, takže uvažovaný
aritmetický priemer
by nemohol byť rovný nepárnemu číslu 27.
Preto $2\notin\mm P$.

Pretože číslo 27 nie je prvočíslo, platí pre najväčší prvok $p^\ast$
množiny $\mm P$ odhad $p^\ast>27$. Teraz využijeme tento zrejmý poznatok:
{\it Aritmetický priemer $A$ skupiny reálnych čísel sa zmenší,
kedykoľvek k~tejto skupine pridáme číslo menšie ako $A$ alebo
z~nej odstránime číslo väčšie ako $A$}. Doplňme preto do danej množiny
$\mm P$ všetky chýbajúce prvočísla $p$, $2<p<27$, a~odstráňme z~nej
všetky prvočísla $p$, $27<p<p^\ast$. Dostaneme tak množinu deviatich
prvočísel $\{3,5,7,11,13,17,19,23,p^\ast\}$, pre ktorých aritmetický priemer
(ktorý už nemusí byť celým číslom!) platí odhad
$$
\frac{3+5+7+11+13+17+19+23+p^*}{9}\leqq27.
$$
(Rovnosť nastane, pokiaľ sme ani žiadne prvočíslo nepridali, ani
žiadne neodstránili.)  Odtiaľ vychádza $p^\ast\leqq145$.
Najväčšie prvočíslo, ktoré spĺňa poslednú nerovnosť, je číslo~ 139.

Hodnota $p^\ast=139$ je možná, ako ukazuje príklad
$$
\mm P=\{3,5,7,11,13,17,19,29,139\},
$$
ktorý objavíme, keď v~súčte $3+5+7+11+13+17+19+23$ zameníme
prvočíslo 23 prvočíslom o~$145-139=6$ väčším. (Keby sme si vopred
neuvedomili, že $2\notin\mm P$, dostali by sme z~nerovnosti
$$
\frac{2+3+5+7+11+13+17+19+23+p^*}{10}\leqq27
$$
slabší odhad $p^*\leqq170$. Potom by bolo nutné postupne
vylúčiť hodnoty $p^\ast=167$, 163, 157, 151, 149. Pritom si asi
uvedomíme, prečo $2\notin\mm P$.)
}

{%%%%%   A-II-2
...}

{%%%%%   A-II-3
...}

{%%%%%   A-II-4
Danú nerovnicu ekvivalentne upravíme na
$$
{2x^{3}+(a-2d)x^2+(b-2a)x+c\over x^2-dx-a}\ge0.   \tag1
$$

Nerovnicu tvaru $\dfrac{A(x)}{B(x)}\ge0$ vieme vyriešiť, ak poznáme
reálne korene oboch mnohočlenov $A(x)$, $B(x)$. Odpovedajúce
množiny ich reálnych koreňov označme $\mm A$, $\mm B$.
Ak označíme $\mm R$ množinu riešení nerovnice
$\dfrac{A(x)}{B(x)}>0$, ktorá je zrejme ekvivalentná nerovnici
${A(x)}{B(x)}>0$, bude množinou riešení pôvodnej nerovnice
$\dfrac{A(x)}{B(x)}\ge0$ množina
$$(\mm R\cup\mm A)\setminus\mm B\,.$$

Na riešenie nerovnice ${A(x)}{B(x)}>0$ nemajú zrejme vplyv prípadné
kvadratické trojčleny so  záporným diskriminantom, ktoré nemajú
reálne korene. A~pretože nás zaujíma riešenie nerovnice
${A(x)}{B(x)}>0$ najmä pre $x\notin\mm A\cup\mm B$, budeme
miesto nej riešiť nerovnicu, ktorú dostaneme vydelením všetkými
možnými párnymi mocninami koreňových činiteľov. Miesto
nerovnice ${A(x)}{B(x)}>0$ tak budeme riešiť nerovnicu
$$
(x-\a_1)(x-\a_2)\dots(x-\a_k)>0 ,
$$
kde $\a_1<\a_2<\dots<\a_k$ sú zvyšné (všetky jednoduché)
reálne korene mnohočlena ${A(x)}{B(x)}$. Riešením poslednej
nerovnice je pre $k=1$ interval $(\a_1,+\infty)$, pre $k=2$
zjednotenie $(-\infty,\a_1)\cup(\a_2,+\infty)$, pre $k\ge3$ nepárne
zjednotenie
$(\a_1,\a_2)\cup\dots\cup(\a_{k-2},\a_{k-1})\cup(\a_k,+\infty)$
a~pre $k>2$ párne zjednotenie
$(-\infty,\a_1)\cup\dots\cup(\a_{k-2},\a_{k-1})\cup(\a_k,+\infty)$.

Vráťme sa teraz k~nerovnici~(1).
Pretože $x=0$ je jej riešením, musí byť nula koreňom čitateľa, nie však
koreňom menovateľa, preto $c=0$ a~$a\ne0$. Naviac z~toho, že nula
je "izolovaným" bodom riešenia, vyplýva podľa našich predchádzajúcich
úvah, že
nula je koreňom párnej násobnosti, teda dvojnásobným. Preto je tiež
$b-2a=0$.

Pretože naopak krajný bod druhého intervalu $x = 4$ do množiny
riešení nepatrí, je koreňom menovateľa, takže $a+4d=16$.
Po dosadení $a=16-4d$ a~rozklade menovateľa dostaneme
ekvivalentnú nerovnicu
$$
{x^2(x+8-3d) \over (x-4)(x-d+4)}\ge 0 .
$$
Odtiaľ však vyplýva, že riešením nerovnice
$$
(x-4)(x+8-3d)(x-d+4)>0
$$
musí byť interval $(4,\infty)$, preto
$3d-8=d-4$, alebo $d=2$, $a=8$, $b=16$, $c=0$.
Pre tieto hodnoty tak dostávame nerovnicu
$$
{x^2(x +2) \over (x-4)(x+2)}\ge  0,
$$
ktorej množinou riešení je skutočne $\{0\}\cup(4,+\infty)$.
}

{%%%%%   A-III-1
...}

{%%%%%   A-III-2
\fontplace
\rpoint A; \tpoint B; \lpoint C; \bpoint D;
\lpoint K; \rBpoint L;
\rBpoint E; \rBpoint F; \rBpoint\toright.5mm G;
\tpoint\toleft.5mm D_0; \lbpoint\down1mm A_0;
[12] \hfil\Obr

Označme $K$ a~$L$ stredy hrán $BC$ a~$AD$ a~$A_0$, $D_0$
príslušné ťažiská stien oproti vrcholom $A$, $D$ (\obr). Obe
ťažnice $AA_0$, $DD_0$ ležia v~rovine $AKD$, pričom ich
priesečník $T$ (ťažisko štvorstena) ich delí v~pomere $3:1$ a~zároveň
je stredom spojnice~ $KL$ (to je zrejmé z~vlastností ťažiska:
$T=\frac14(A+B+C+D)=\frac12\bigl(\frac12(A+D)
+\frac12(B+C)\bigr)=\frac12(K+L)$). Odtiaľ vyplýva, že je
$|ET|:|AT|=|FT|:|DT|=1:3$, takže $|EF|=\frac13|AD|$.

\inspicture c

Rovina $BCL$ rozpoľuje obidve úsečky
$AD$ aj~$EF$, a~preto tiež rozdeľuje obidva uvažované štvorsteny
$ABC\!D$ aj~$BCEF$ na časti rovnakého objemu. Označme $G$
stred úsečky $EF$, pre príslušné objemy potom platí
$$
\align
{V(BCEF)\over V(ABC\!D)}=&{V(BCGF)\over V(BCLD)}=
{|GF|\over |LD|}\cdot{S(BCG)\over S(BCL)}=\\
=&\frac13{|KG|\over |KL|}=\frac13\cdot\frac23=\frac29.
\endalign
$$
}

{%%%%%   A-III-3
...}

{%%%%%   A-III-4
...}

{%%%%%   A-III-5
...}

{%%%%%   A-III-6
...}

{%%%%%   B-S-1
...}

{%%%%%   B-S-2
...}

{%%%%%   B-S-3
...}

{%%%%%   B-II-1
Po odčítaní druhej rovnice od prvej a~tretej rovnice od druhej
dostaneme dve rovnice, ktoré majú po jednoduchej úprave tvar
$$
(a~- d)(b - c) = 0, \qquad
(a~- b)(c - d) = 0.
$$
Odtiaľ vyplýva, že zo štyroch čísel $a$, $b$, $c$, $d$ sú aspoň tri
rovnaké. Nech je napr\. $a = b = c$. Po dosadení do prvej rovnice
pôvodnej sústavy dostaneme
$$
a^{2}+ad =a(a+d) = 1\,999.
$$
Nakoľko 1\,999 je prvočíslo, vyhovuje jedine $a = 1$ a~$a+d =
1\,999$. Odtiaľ vyplýva, že
$$
a~= b = c = 1,\qquad d = 1\,998.
$$
Zámenou dostaneme ďalšie tri riešenia
$$
a~= b = d = 1,\ c = 1\,998;\quad
a~= c = d = 1,\ b = 1\,998;\quad
b = c = d = 1,\ a~= 1\,998.
$$
Dosadením sa ľahko presvedčíme, že všetky štyri štvorice vyhovujú
zadaniu.
}

{%%%%%   B-II-2
...}

{%%%%%   B-II-3
...}

{%%%%%   B-II-4
...}

{%%%%%   C-S-1
...}

{%%%%%   C-S-2
...}

{%%%%%   C-S-3
...}

{%%%%%   C-II-1
...}

{%%%%%   C-II-2
...}

{%%%%%   C-II-3
...}

{%%%%%   C-II-4
...}

{%%%%%   vyberko, den 1, priklad 1
...}

{%%%%%   vyberko, den 1, priklad 2
...}

{%%%%%   vyberko, den 1, priklad 3
...}

{%%%%%   vyberko, den 1, priklad 4
...}

{%%%%%   vyberko, den 2, priklad 1
...}

{%%%%%   vyberko, den 2, priklad 2
...}

{%%%%%   vyberko, den 2, priklad 3
...}

{%%%%%   vyberko, den 2, priklad 4
...}

{%%%%%   vyberko, den 3, priklad 1
...}

{%%%%%   vyberko, den 3, priklad 2
...}

{%%%%%   vyberko, den 3, priklad 3
...}

{%%%%%   vyberko, den 3, priklad 4
...}

{%%%%%   vyberko, den 4, priklad 1
...}

{%%%%%   vyberko, den 4, priklad 2
...}

{%%%%%   vyberko, den 4, priklad 3
...}

{%%%%%   vyberko, den 4, priklad 4
...}

{%%%%%   vyberko, den 5, priklad 1
...}

{%%%%%   vyberko, den 5, priklad 2
...}

{%%%%%   vyberko, den 5, priklad 3
...}

{%%%%%   vyberko, den 5, priklad 4
...}

{%%%%%   trojstretnutie, priklad 1
Položme $x=b+2c$, $y=c+2a$, $z=a+2b$. Potom
$a=\frac19(4y+z-2x)$, $b=\frac19(4z+x-2y)$, $c=\frac19(4x+y-2z)$,
takže dokazovaná nerovnosť získava tvar
$$
\frac{4y+z-2x}{9x}+\frac{4z+x-2y}{9y}+\frac{4x+y-2z}{9z}\geq1.
$$
Ekvivalentnými úpravami dostaneme nerovnosť
$$
\Bigl(\frac xy+\frac yx\Bigr)+
\Bigl(\frac yz+\frac zy\Bigr)+
\Bigl(\frac zx+\frac xz\Bigr)+
3\cdot\Bigl(\frac yx+\frac xz+\frac zy\Bigr)\geqq 15,
$$
ktorá platí, lebo každý z~výrazov v~prvých troch
zátvorkách je aspoň~2, zatiaľčo výraz vo štvrtej zátvorke je aspoň~3
(podľa nerovností medzi aritmetickým a~geometrickým
priemerom pre príslušné dvojice či trojice kladných čísel). Dodajme,
že upravená nerovnosť rovnako vyplýva z~jedinej AG-nerovnosti pre
skupinu 15~ čísel, ktorej výber je podľa tvaru ľavej strany zrejmý.

\medskip
{\bf Iné riešenie}. Z~Cauchyho nerovnosti
$$
(u_1v_1+u_2v_2+u_3v_3)^2\leqq(u_1^2+u_2^2+u_3^2)(v_1^2+v_2^2+v_3^2)
$$
vyplýva odhad
$$
(a+b+c)^2\leqq
\Bigl(\frac{a}{b+2c}+\frac{b}{c+2a}+\frac{c}{a+2b}\Bigr)
[a(b+2c)+b(c+2a)+c(a+2b)];
\tag1$$
z~nerovnosti $(a-b)^2+(b-c)^2+(c-a)^2\geqq0$ zase vyplýva odhad
$$
a(b+2c)+b(c+2a)+c(a+2b)\leqq(a+b+c)^2.
$$
Dôsledkom oboch odhadov je nerovnosť
$$
\align
a(b+2c)+&b(c+2a)+c(a+2b)\leqq\\
\leqq&\Bigl(\frac{a}{b+2c}+\frac{b}{c+2a}+\frac{c}{a+2b}\Bigr)
[a(b+2c)+b(c+2a)+c(a+2b)],
\endalign
$$
z~ktorej po delení (kladným) výrazom $a(b+2c)+b(c+2a)+c(a+2b)$ vyjde
dokazovaná nerovnosť.
}

{%%%%%   trojstretnutie, priklad 2
...}

{%%%%%   trojstretnutie, priklad 3
...}

{%%%%%   trojstretnutie, priklad 4
...}

{%%%%%   trojstretnutie, priklad 5
...}

{%%%%%   trojstretnutie, priklad 6
...}

{%%%%%   IMO, priklad 1
...}

{%%%%%   IMO, priklad 2
...}

{%%%%%   IMO, priklad 3
...}

{%%%%%   IMO, priklad 4
...}

{%%%%%   IMO, priklad 5
...}

{%%%%%   IMO, priklad 6
Označme  $ A = \{ f(x) \mid x\in\Bbb R \} $ a~$c=f(0)$.
Dosadením $x=y=0$ do rovnice zo zadania dostávame
$ f(-c) = f(c) + c - 1 $, čo znamená, že $c\ne0$.
Ak položíme $x=f(y)$ dostávame, že pre každé $x\in A$ platí
$$
  f(x) = \frac{c+1}2 - \frac{x^2}2 \,.  \tag1
$$
Ukážeme, že $A-A=\Bbb R$ (kde $X-Y=\{x-y \mid x\in X,\,y\in Y\}$ znamená
tzv.\ {\it algebraický rozdiel\/} množín $X$ a~$Y$). Dosadením
$y=0$ a~využitím $c\ne0$ dostávame
$$
  \{ f(x-c) - f(x) \mid x\in\Bbb R \} =
  \{ cx + f(c) - 1 \mid x\in\Bbb R \} = \Bbb R \,.
$$
Teraz už môžeme vypočítať hodnotu $f(x)$ pre ľubovoľné $x\in\Bbb R$.
Ak zvolíme $y_1,y_2\in A$ také, že $x=y_1-y_2$, využitím (1) dostávame
$$
  \align
    f(x) &= f(y_1-y_2)
          = f(y_2) + y_1y_2 + f(y_1) - 1 = \\
         &= \frac{c+1}2 - \frac{y_2^2}2 + y_1y_2 +
            \frac{c+1}2 - \frac{y_1^2}2 - 1 = \\
         &= c - \frac{(y_1-y_2)^2}2
          = c - \frac{x^2}2 \,.  \tag2
  \endalign
$$
Porovnaním (1) a~(2) tiež dostávame $c=1$, takže
$$
  f(x) = 1 - \frac{x^2}2 \,.
$$
Ľahko sa overí, že táto funkcia je naozaj riešením úlohy.
}

{%%%%%   MEMO, priklad 1
...}

{%%%%%   MEMO, priklad 2
...}

{%%%%%   MEMO, priklad 3
...}

{%%%%%   MEMO, priklad 4
...}

{%%%%%   MEMO, priklad t1
...}

{%%%%%   MEMO, priklad t2
...}

{%%%%%   MEMO, priklad t3
...}

{%%%%%   MEMO, priklad t4
...}

{%%%%%   MEMO, priklad t5
...}

{%%%%%   MEMO, priklad t6
...}

{%%%%%   MEMO, priklad t7
...}

{%%%%%   MEMO, priklad t8
...} 