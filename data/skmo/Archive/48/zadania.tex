{%%%%%   A-I-1
Nájdite najmenšie prirodzené číslo, ktoré možno dostať doplnením zátvoriek
do výrazu
$$
15:14:13:12:11:10:9:8:7:6:5:4:3:2.
$$}
\podpis{P. Černek}

{%%%%%   A-I-2
Nájdite všetky kladné čísla~$k$, pre ktoré platí: zo všetkých
trojuholníkov $ABC$,
v~ktorých $|AB|=5\cm$ a~$|AC|:|BC|=k$, má najväčší obsah
rovnoramenný trojuholník.}
\podpis{P. Černek}

{%%%%%   A-I-3
Pre ktoré  celé čísla $a$
je maximum aj minimum funkcie
$$
y=\frac{12x^2-12ax}{x^2+36}
$$
celé číslo?}
\podpis{P. Černek}

{%%%%%   A-I-4
Označme $\tau(k)$ počet všetkých kladných deliteľov prirodzeného
čísla~$k$ a~nech číslo~$n$ je riešením rovnice
$\tau(1{,}6n)=1{,}6\tau(n)$.
Určte hodnotu podielu $\tau(0{,}16n):\tau(n)$.}
\podpis{P. Černek}

{%%%%%   A-I-5
Dokážte, že existuje trojuholník $ABC$, v~ktorom pri zvyčajnom označení
platia obidve {\it pytagorejské\/} rovnosti $t_a^2+t_b^2=t_c^2$
a~$v_a^2+v_b^2=v_c^2$.
Ďalej ukážte, že pre vnútorné uhly takého
trojuholníka platí $|\alpha-\beta|=90\st$ a~$\cos\gamma=\frac25\sqrt5$.}
\podpis{J. Šimša}

{%%%%%   A-I-6
Z~papiera bol zlepený model štvorstena, ktorého každé dve protiľahlé
hrany sú zhodné.
Rozhodnite, či môžeme model rozrezať pozdĺž troch úsečiek
tak, aby ho potom bolo možné rozvinúť do roviny a~vznikol pritom
obdĺžnik. Existujú pre pravidelný štvorsten dva takéto spôsoby
rozrezania, pri ktorých vzniknú nezhodné obdĺžniky?}
\podpis{M. Hejný, P. Leischner}

{%%%%%   B-I-1
Na lúke sú deti aj~dospelí. Počet percent chlapcov zo
všetkých detí je rovný počtu percent dievčat zo všetkých prítomných osôb
a~tiež počtu všetkých dospelých. Koľko chlapcov, dievčat a~dospelých
je na lúke?}
\podpis{Ľ. Fabšo, P. Černek}

{%%%%%   B-I-2
Uvažujme zhodné polkružnice, ktoré ležia v~danom
pravom uhle a~ktorých koncové body ležia každý na inom ramene pravého
uhla.
Určte množinu, ktorú vyplnia body všetkých týchto polkružníc.}
\podpis{J. Zhouf}

{%%%%%   B-I-3
Nájdite všetky trojmiestne čísla v~desiatkovej sústave,
ktoré sú rovné tretine čísla s~tým istým zápisom v~inej číselnej sústave.}
\podpis{P. Černek}

{%%%%%   B-I-4
Daný je rovnostranný trojuholník $ABC$.
Na strane $BC$ nájdite bod $P$ tak, aby kružnica vpísaná trojuholníku
$ABP$ a~kružnica pripísaná k strane $PC$ trojuholníka $APC$
boli zhodné.}
\podpis{J. Švrček}

{%%%%%   B-I-5
Z~gule s~polomerom $R$ je oddelený guľový odsek s~výškou $v$
($v<R$). Do tohto odseku je vpísaná guľa $K$ s~polomerom $v\over2$.
Ďalej je do odseku vpísaných osem zhodných menších gulí,
z~ktorých každá sa dotýka gule $K$. Žiadne dve z~nich nemajú spoločný
vnútorný bod a~každá z~nich sa dotýka práve dvoch ostatných.
Určte pomer $v:R$.}
\podpis{J. Zhouf}

{%%%%%   B-I-6
Nájdite všetky možné hodnoty súčtu $x+y$,
kde reálne čísla $x$, $y$ spĺňajú rovnosť $x^3+y^3=3xy$.}
\podpis{J. Šimša}

{%%%%%   C-I-1
Dané je štvorciferné číslo. Zmenou poradia jeho cifier možno zostaviť práve osem
ďalších štvorciferných čísel. Súčet najmenších troch zo
všetkých týchto deviatich čísel je $12\,528$.
Určte cifry daného čísla.}
\podpis{P. Černek}

{%%%%%   C-I-2
V~obdĺžniku $ABC\!D$ platí $|AB|>|BC|$.
Oblúk $AC$ kružnice, ktorej stred leží na strane $AB$,
pretína stranu $C\!D$ v~bode $M$. Dokážte, že priamky $AM$ a~$BD$
sú navzájom kolmé.}
\podpis{L. Boček, J. Švrček}

{%%%%%   C-I-3
Zistite, či je číslo $19^{1998}+98^{1999}$ deliteľné deviatimi.}
\podpis{P.~Leischner}

{%%%%%   C-I-4
Adam a~Braňo sa zúčastnili na turnaji hranom systémom každý
s~každým jeden zápas, na~ktorom mal každý hráč odohrať denne práve
jeden zápas. Adam a~Braňo však ochoreli a~ako jediní
nedokončili turnaj. Braňo odstúpil o~päť dní prv ako Adam.
Spolu sa odohralo 350 zápasov. Koľko zápasov odohral Adam?
Hral s~Braňom?}
\podpis{P. Černek}

{%%%%%   C-I-5
Daný je trojuholník $ABC$, v~ktorom $|\angle BAC|=150\st$,
$|AB|=4$\,cm a~$|AC|=6$\,cm.
Zostrojte trojuholník s dvojnásobným obsahom,
ktorého dve strany sú zhodné s~niektorými dvomi stranami
trojuholníka $ABC$. Nájdite všetky riešenia.}
\podpis{P. Černek}

{%%%%%   C-I-6
Pre ľubovoľnú dvojicu reálnych čísel $a$, $b$ spĺňajúcu vzťah $a+b=1$
platí
$$
\sqrt{a^2+a+1}+\sqrt{b^2+b+1}>2.
$$
Ak sú navyše čísla $a$, $b$ nezáporné, platí tiež
$$
\sqrt{a^2+a+1}+\sqrt{b^2+b+1}<3.
$$
Obidve tvrdenia dokážte.}
\podpis{P. Leischner, J. Švrček}

{%%%%%   A-S-1
Dokážte, že existuje ostrouhlý trojuholník $ABC$, ktorého
ťažnice z~vrcholov $A$ a~$B$ sú po rade zhodné so stranami
$AC$ a~$AB$.}
\podpis{J. Šimša}

{%%%%%   A-S-2
V~rovine sú dané dva rôzne body $A$ a~$B$.
Nájdite všetky reálne čísla $k>1$, pre ktoré platí:
Zo všetkých trojuholníkov $ABC$, v~ktorých $|AC|:|BC| = k$,
má najväčší možný vnútorný uhol pri vrchole~ $A$  rovnoramenný trojuholník.}
\podpis{J. Šimša, L. Boček}

{%%%%%   A-S-3
Ukážte, že pre každé prirodzené číslo $n$ je súčin
$$
\Bigl(4-\frac21\Bigr)
\Bigl(4-\frac22\Bigr)
\Bigl(4-\frac23\Bigr)\dots
\Bigl(4-\frac2n\Bigr)
$$
celé číslo.}
\podpis{R. Horenský}

{%%%%%   A-II-1
Aritmetický priemer niekoľkých navzájom rôznych prvočísel sa rovná~$27$. Určte, aké najväčšie prvočíslo medzi nimi môže byť.}
\podpis{S. Bednářová, P. Černek}

{%%%%%   A-II-2
Daný je štvorec $ABCD$. Dokážte, že pre všetky body~ $P$ toho
oblúka~$AB$ kružnice štvorcu opísanej, ktorý neobsahuje body~ $C$,
$D$, má výraz
$$
|AP|+|BP|\over|CP|+|DP|
$$
rovnakú hodnotu. Určte ju.}
\podpis{J. Švrček}

{%%%%%   A-II-3
V~ľubovoľnom trojuholníku $ABC$ označme $M$ a~$N$ po rade
stredy strán $BC$ a~$AC$. Dokážte, že ťažisko trojuholníka $ABC$ leží
na kružnici opísanej trojuholníku $CMN$ práve vtedy, keď platí rovnosť
$$
4\cdot|AM|\cdot|BN|=3\cdot|AC|\cdot|BC|.
$$}
\podpis{J. Šimša}

{%%%%%   A-II-4
Nájdite reálne čísla $a$, $b$, $c$, $d$, pre
ktoré všetky riešenia $x$ nerovnice
$$
\frac{ax^2+bx+c}{a+dx-x^2}\le 2x
$$
tvoria množinu $\{0\}\cup(4,+\infty)$.}
\podpis{P. Černek}

{%%%%%   A-III-1
Do čitateľa aj~menovateľa zlomku
$$
\frac{29:28:27:26:25:24:23:22:21:20:19:18:17:16}
{15:14:13:12:11:10:\phantom{2}9:\phantom{2}8:\phantom{2}7:\phantom{2}6:\phantom{1}5
:\phantom{1}4:\phantom{1}3:\phantom{1}2}
$$
smieme opakovane vpisovať zátvorky, a~to vždy na rovnaké miesta pod
seba.
\itemitem{a)} Určte najmenšiu možnú celočíselnú hodnotu výsledného výrazu.
\itemitem{b)} Nájdite všetky možné celočíselné hodnoty výsledného výrazu.
}
\podpis{J. Šimša}

{%%%%%   A-III-2
Vo~všeobecnom štvorstene $ABCD$ označme $E$ a~$F$
stredy ťažníc z~vrcholov $A$ a~$D$. Určte
pomer objemov štvorstenov $BCEF$ a~$ABCD$. (Ťažnicou v~štvorstene je
spojnica vrcholu s~ťažiskom protiľahlej steny.)}
\podpis{P. Leischner}

{%%%%%   A-III-3
Ukážte, že existuje trojuholník $ABC$, v~ktorom pri zvyčajnom označení
strán a~ťažníc platí $a\ne b$ a~zároveň $a+t_a=b+t_b$. Ďalej
dokážte existenciu takého čísla~ $k$, že pre každý spomínaný trojuholník
platí $a+t_a=b+t_b=k(a+b)$. Nakoniec nájdite všetky pomery
$a:b$ strán $a$, $b$ takých trojuholníkov.}
\podpis{J. Šimša}

{%%%%%   A-III-4
V~istom jazyku sú len dve písmena $A$ a~$B$.
Pre slová tohto jazyka platí:
\itemitem{1)} Jednopísmenové slová neexistujú, dvojpísmenové slová sú
len $AB$ a~$BB$.
\itemitem{2)} Postupnosť písmen dĺžky $n>2$ je slovo
práve vtedy, keď vznikne z~nejakého slova tohto jazyka s počtom písmen
menším ako $n$,
a~to tak, že v~tomto slove písmená $A$ ponecháme na mieste
a každé písmeno $B$ súčasne nahradíme nejakým
(nie nutne rovnakým) slovom tohto jazyka.

Dokážte, že počet $n$-písmenových slov tohto jazyka je pre každé $n$
rovný číslu
$$
\frac{2^n + 2\cdot(\m1)^n}{3}.
$$
}
\podpis{P. Hliněný, P. Kaňovský}

{%%%%%   A-III-5
V~rovine je daný ostrý uhol $APX$. Zostrojte
štvorec $ABCD$ tak, aby bod~$P$ ležal na strane~$BC$ a~aby
polpriamka~$PX$ preťala stranu $CD$ v~takom bode~$Q$, že bod~$P$ leží na osi uhla $BAQ$.}
\podpis{J. Šimša}

{%%%%%   A-III-6
Nájdite všetky dvojice reálnych čísel $a$
a~$b$, pre ktoré má sústava rovníc
$$
\frac{x+y}{x^2+y^2}=a,\qquad \frac{x^3+y^3}{x^2+y^2}=b
$$
s~neznámymi $x$ a~$y$ riešenie v~obore reálnych čísel.}
\podpis{J.~Šimša}

{%%%%%   B-S-1
Na ihrisku je menej ako 500~detí. Pritom počet percent chlapcov zo
všetkých detí sa rovná počtu všetkých dievčat. Koľko chlapcov a~koľko
dievčat je na ihrisku? Nájdite všetky možnosti.}
\podpis{P. Černek}

{%%%%%   B-S-2
V~trojuholníku $ABC$ poznáme $a=|BC|$, polomer~$\rho$ kružnice vpísanej
a~polomer~$\rho_a$ kružnice pripísanej k~strane~$BC$. Dokážte, že
vzdialenosť stredov oboch kružníc sa rovná
$\sqrt{a^2+(\rho_a-\rho)^2}$.}
\podpis{P. Leischner}

{%%%%%   B-S-3
Kvadratická rovnica $x^2-35x+334=0$, ktorej koeficienty sú
zapísané v~číselnej sústave so~základom~ $z$ ($z\ge6$), má dva rôzne
reálne korene. Určte $z$ a~obidva korene.}
\podpis{J. Šimša}

{%%%%%   B-II-1
Nájdite všetky štvorice prirodzených čísel $a$, $b$, $c$, $d$,
pre ktoré platí
$$
\align
ab+cd&=1\,999,\\
ac+bd&=1\,999,\\
ad+bc&=1\,999.
\endalign
$$}
\podpis{J. Bábeľa}

{%%%%%   B-II-2
Daný je  pravouhlý trojuholník $ABC$, nad ktorého odvesnami $AB$ a~$BC$ (ako nad
priemermi) sú zvonku trojuholníka zostrojené po rade polkružnice $k$
a~$l$. Vrcholom~ $B$ veďte priamku~ $p$, ktorá pretína
polkružnice~ $k$ a~$l$ po rade v~bodoch~ $X$ a~$Y$ tak, aby
štvoruholník~ $AXYC$ mal čo najväčší obvod.}
\podpis{J. Šimša, J. Švrček}

{%%%%%   B-II-3
Nájdite všetky možné hodnoty výrazu
$$
{x+y\over x^2+y^2},
$$
kde $x$ a~$y$ sú ľubovoľné reálne čísla spĺňajúce podmienku
$x+y\ge1$.}
\podpis{J. Šimša}

{%%%%%   B-II-4
Nech $A$ a~$B$ sú rôzne body roviny. Ďalej je daný
orientovaný uhol $\omega$ ($0\st<\omega<90\st$). Pre ľubovoľný
bod~ $X$ označme po rade $X_A$, $X_B$ obrazy bodu~ $X$
v~otočeniach okolo stredov~ $A$ a~$B$ o~uhol~ $\omega$. Nájdite
všetky také body~$X$, pre ktoré je trojuholník $XX_AX_B$ rovnostranný.}
\podpis{E. Kováč}

{%%%%%   C-S-1
Nájdite všetky dvojice $a$, $b$ nezáporných reálnych čísel, pre
ktoré platí
$$
\sqrt{a^2+b}+\sqrt{b^2+a}=
\sqrt{a^2+b^2}+\sqrt{a+b}.
$$}
\podpis{J. Šimša}

{%%%%%   C-S-2
Určte najväčšie štvorciferné číslo $n$, pre ktoré je súčet
$n^{19}+99^n$ deliteľný desiatimi.}
\podpis{J. Švrček}

{%%%%%   C-S-3
V~rovine je daný obdĺžnik $ABCD$, nad ktorého stranami $AB$ a~$BC$
(ako nad priemermi) sú zvonku obdĺžnika zostrojené po rade
polkružnice $k$ a~$l$. Nájdite úsečku $XY$ čo najväčšej dĺžky $d$
tak, aby platilo $X\in k$ a~$Y\in l$. Dĺžku $d$ potom vyjadrite
pomocou dĺžok $a=|AB|$ a~$b=|BC|$.}
\podpis{J. Švrček}

{%%%%%   C-II-1
Zistite, ktoré dvojice pravidelných mnohouholníkov majú veľkosti
vnútorných uhlov v~pomere~ $2:3$.}
\podpis{S. Bednářová}

{%%%%%   C-II-2
Nájdite najväčšie trojciferné číslo $n$, pre ktoré je súčet
$$
1^2+2^3+3^4+4^5+\dots+n^{n+1}
$$
deliteľný tromi.}
\podpis{J. Šimša}

{%%%%%   C-II-3
Zostrojte lichobežník $ABCD$, pre ktorý platí
$$
|AC|=8\cm, \ |BD|=6\cm,\ |AB|+|CD|=10\cm
$$
a~stred kružnice opísanej trojuholníku
$ACD$ leží na základni~$AB$.}
\podpis{P. Leischner}

{%%%%%   C-II-4
Dokážte, že pre každé tri reálne čísla $x$, $y$, $z$, ktoré
spĺňajú nerovnosti
$$
0<x<y<z<1,
$$
platí tiež nerovnosť
$$
x^2+y^2+z^2<xy+yz+zx+z-x.
$$}
\podpis{J.~Bábeľa}

{%%%%%   vyberko, den 1, priklad 1
...}
\podpis{...}

{%%%%%   vyberko, den 1, priklad 2
...}
\podpis{...}

{%%%%%   vyberko, den 1, priklad 3
...}
\podpis{...}

{%%%%%   vyberko, den 1, priklad 4
...}
\podpis{...}

{%%%%%   vyberko, den 2, priklad 1
...}
\podpis{...}

{%%%%%   vyberko, den 2, priklad 2
...}
\podpis{...}

{%%%%%   vyberko, den 2, priklad 3
...}
\podpis{...}

{%%%%%   vyberko, den 2, priklad 4
...}
\podpis{...}

{%%%%%   vyberko, den 3, priklad 1
...}
\podpis{...}

{%%%%%   vyberko, den 3, priklad 2
...}
\podpis{...}

{%%%%%   vyberko, den 3, priklad 3
...}
\podpis{...}

{%%%%%   vyberko, den 3, priklad 4
...}
\podpis{...}

{%%%%%   vyberko, den 4, priklad 1
...}
\podpis{...}

{%%%%%   vyberko, den 4, priklad 2
...}
\podpis{...}

{%%%%%   vyberko, den 4, priklad 3
...}
\podpis{...}

{%%%%%   vyberko, den 4, priklad 4
...}
\podpis{...}

{%%%%%   vyberko, den 5, priklad 1
...}
\podpis{...}

{%%%%%   vyberko, den 5, priklad 2
...}
\podpis{...}

{%%%%%   vyberko, den 5, priklad 3
...}
\podpis{...}

{%%%%%   vyberko, den 5, priklad 4
...}
\podpis{...}

{%%%%%   trojstretnutie, priklad 1
Pre ľubovoľné kladné čísla $a$, $b$, $c$ dokážte
nerovnosť
$$
\frac{a}{b+2c}+\frac{b}{c+2a}+\frac{c}{a+2b}\geqq1.
$$
}
\podpis{J. Bábeľa}

{%%%%%   trojstretnutie, priklad 2
...}
\podpis{...}

{%%%%%   trojstretnutie, priklad 3
...}
\podpis{...}

{%%%%%   trojstretnutie, priklad 4
...}
\podpis{...}

{%%%%%   trojstretnutie, priklad 5
...}
\podpis{...}

{%%%%%   trojstretnutie, priklad 6
...}
\podpis{...}

{%%%%%   IMO, priklad 1
...}
\podpis{...}

{%%%%%   IMO, priklad 2
...}
\podpis{...}

{%%%%%   IMO, priklad 3
...}
\podpis{...}

{%%%%%   IMO, priklad 4
...}
\podpis{...}

{%%%%%   IMO, priklad 5
...}
\podpis{...}

{%%%%%   IMO, priklad 6
Určte všetky také funkcie  $ f : \Bbb R \to \Bbb R $, že rovnosť
$$
  f(x-f(y)) = f(f(y)) + xf(y) + f(x) -  1
$$
platí pre všetky $x,y\in\Bbb R$.}
\podpis{Japonsko}

{%%%%%   MEMO, priklad 1
}
\podpis{}

{%%%%%   MEMO, priklad 2
}
\podpis{}

{%%%%%   MEMO, priklad 3
}
\podpis{}

{%%%%%   MEMO, priklad 4
}
\podpis{}

{%%%%%   MEMO, priklad t1
}
\podpis{}

{%%%%%   MEMO, priklad t2
}
\podpis{}

{%%%%%   MEMO, priklad t3
}
\podpis{}

{%%%%%   MEMO, priklad t4
}
\podpis{}

{%%%%%   MEMO, priklad t5
}
\podpis{}

{%%%%%   MEMO, priklad t6
}
\podpis{}

{%%%%%   MEMO, priklad t7
}
\podpis{}

{%%%%%   MEMO, priklad t8
}
\podpis{}
