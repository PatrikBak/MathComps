{%%%%%   A-I-1
V~obore reálnych čísel riešte rovnicu
$$
\sqrt{2}(\sin t+\cos t)=\tg^3 t+\cotg^3 t.
$$}
\podpis{J. Švrček}

{%%%%%   A-I-2
Nech $ABCD$ je tetivový štvoruholník s~navzájom kolmými uhlopriečkami.
Označme postupne $p$, $q$ kolmice z~bodov $D$,
$C$ na priamku~$AB$. Ďalej označme $X$ priesečník priamok $AC$ a~$p$
a~$Y$ priesečník priamok $BD$ a~$q$. Dokážte, že $XYCD$ je
kosoštvorec alebo štvorec.}
\podpis{E. Kováč}

{%%%%%   A-I-3
Postupnosť $(a_n)_{n=0}^{\infty}$ {\it nenulových\/}
celých čísel má tú vlastnosť, že pre každé $n\ge0$ platí $a_{n+1}=a_n-b_n$,
kde $b_n$ je číslo, ktoré má rovnaké znamienko ako číslo~$a_n$, ale
opačné poradie číslic (zápis čísla~$b_n$ môže na rozdiel od zápisu čísla~$a_n$ začínať
jednou alebo viacerými nulami).
Napríklad pre $a_0=1\,210$ je $a_1=1\,089$, $a_2={-}8\,712$,
$a_3={-}6\,534,\dots$
\ite a) Dokážte, že postupnosť $(a_n)$ je periodická.
\ite b) Zistite, aké najmenšie prirodzené číslo môže byť~$a_0$.}
\podpis{T. Jurík}

{%%%%%   A-I-4
Nájdite všetky kubické mnohočleny~$P(x)$, ktoré majú aspoň dva rôzne reálne korene,
z~ktorých jeden je číslo~7, a~ktoré pre každé reálne číslo~$t$
spĺňajú podmienku: Ak $P(t)=0$, tak $P(t+1)=1$.}
\podpis{Pavel Novotný}

{%%%%%   A-I-5
Dané sú úsečky dĺžok $a$, $b$, $c$, $d$. Dokážte, že rovnosť $a^2+c^2=b^2+d^2$
platí práve vtedy, keď existujú konvexné
štvoruholníky so stranami dĺžok $a$, $b$, $c$, $d$ (pri
zvyčajnom označení), pričom uhlopriečky každého takého štvoruholníka
zvierajú jeden a~ten istý uhol.}
\podpis{J. Šimša}

{%%%%%   A-I-6
Nájdite všetky usporiadané dvojice $(x,y)$ prirodzených čísel, pre
ktoré platí
$$
x^2+y^2=2005(x-y).
$$}
\podpis{J. Moravčík}

{%%%%%   B-I-1
Určte všetky hodnoty celočíselného parametra~$a$, pre ktoré má rovnica
$$
(x+a)(x+2a)=3a
$$
aspoň jeden celočíselný koreň.}
\podpis{J. Zhouf}

{%%%%%   B-I-2
V~danom trojuholníku~$ABC$ označme $D$ ten bod polpriamky~$CA$, pre ktorý platí $|CD|=|CB|$.
Ďalej označme postupne $E$, $F$ stredy úsečiek $AD$ a~$BC$. Dokážte, že $|\uh BAC|=2|\uh CEF|$
práve vtedy, keď $|AB|=|BC|$.}
\podpis{P. Leischner}

{%%%%%   B-I-3
Rozhodnite, či nerovnosť
$$
a(b+1)+b(c+1)+c(d+1)+d(a+1)\ge\tfrac12(a+1)(b+1)(c+1)(d+1)
$$
platí pre všetky také kladné čísla $a$, $b$, $c$, $d$, ktoré spĺňajú podmienku
\ite a) $ab=cd=1$;
\ite b) $ac=bd=1$.}
\podpis{J. Šimša}

{%%%%%   B-I-4
Každú z~hviezdičiek v~zápisoch dvanásťmiestnych čísel
$A=*88\,888\,888\,888$, $B=*11\,111\,111\,111$ nahraďte
nejakou číslicou tak, aby výraz $|14A-13B|$ mal čo najmenšiu hodnotu.}
\podpis{J. Šimša}

{%%%%%   B-I-5
Kruh so stredom~$S$ a~polomerom~$r$ je rozdelený na štyri časti dvoma tetivami, z~ktorých
jedna má dĺžku~$r$ a~druhá má od stredu~$S$
vzdialenosť~$r/2$. Dokážte, že absolútna hodnota rozdielu obsahov tých dvoch častí,
ktoré majú spoločný práve jeden bod a~pritom žiadna z~nich neobsahuje stred~$S$,
je rovná jednej šestine obsahu kruhu.}
\podpis{P. Leischner}

{%%%%%   B-I-6
Určte najmenšie prirodzené číslo~$n$ s~nasledujúcou vlastnosťou: Keď zvolíme $n$~rôznych
prirodzených čísel menších ako 2\,005, sú medzi nimi dve také, že podiel súčtu
a~rozdielu ich druhých mocnín je väčší ako tri.}
\podpis{J. Zhouf}

{%%%%%   C-I-1
\ite a) Dokážte, že pre každé prirodzené číslo~$m$ je rozdiel $m^6-m^2$
deliteľný číslom~60.
\ite b) Určte všetky prirodzené čísla~$m$, pre ktoré je rozdiel $m^6-m^2$
deliteľný číslom~120.}
\podpis{J. Moravčík}

{%%%%%   C-I-2
Kružnice $k$, $\ell$, $m$ sa po dvoch zvonka dotýkajú a~všetky tri majú spoločnú dotyčnicu.
Polomery kružníc $k$, $\ell$ sú 3\,cm a~12\,cm. Vypočítajte polomer kružnice~$m$.
Nájdite všetky riešenia.}
\podpis{L. Boček}

{%%%%%   C-I-3
Určte počet všetkých trojíc navzájom rôznych trojmiestnych prirodzených čísel, ktorých súčet
je deliteľný každým z~troch sčítaných čísel.}
\podpis{J. Šimša}

{%%%%%   C-I-4
Je dané prirodzené číslo~$n$ ($n\ge2$) a~reálne čísla $x_1,x_2,\dots,x_n$, pre ktoré platí
$$
x_1x_2=x_2x_3=\dots=x_{n-1}x_n=x_nx_1=1.
$$
Dokážte, že
$$
x_1^2+x_2^2+\dots+x_n^2\ge n.
$$}
\podpis{J. Švrček}

{%%%%%   C-I-5
V~ostrouhlom trojuholníku~$ABC$ označme $D$ pätu výšky z~vrcholu $C$ a~$P$, $Q$
zodpovedajúce päty kolmíc vedených bodom~$D$ na strany $AC$ a~$BC$. Obsahy
trojuholníkov $ADP$, $DCP$, $DBQ$, $CDQ$ označme postupne $S_1$, $S_2$, $S_3$, $S_4$.
Vypočítajte $S_1:S_3$, ak $S_1:S_2=2:3$ a~$S_3:S_4=3:8$.}
\podpis{Pavel Novotný}

{%%%%%   C-I-6
Rozhodnite, ktoré z~čísel
$$
\sqrt{p+\sqrt q}+\sqrt{q+\sqrt p}, \quad
\sqrt{p+\sqrt p}+\sqrt{q+\sqrt q}
$$
je väčšie, ak $p$ a~$q$ sú rôzne kladné čísla.}
\podpis{J. Moravčík}

{%%%%%   A-S-1
Nájdite všetky dvojice celých čísel $x$ a~$y$, pre ktoré platí
$$
\sqrt{x\sqrt5}-\sqrt{y\sqrt5}=\sqrt{6\sqrt5-10}.
$$}
\podpis{J. Moravčík}

{%%%%%   A-S-2
Daný je rovnostranný trojuholník $ABC$ s~obsahom~$S$ a~jeho vnútorný
bod~$M$. Označme postupne $A_1$, $B_1$, $C_1$ tie body strán $BC$, $CA$
a~$AB$, pre ktoré platí $MA_1\parallel AB$, $MB_1\parallel BC$ 
a~$MC_1\parallel CA$. Priesečníky osí úsečiek $MA_1$, $MB_1$ a~$MC_1$
tvoria vrcholy trojuholníka s~obsahom~$T$. Dokážte, že platí $S=3T$.}
\podpis{J. Švrček}

{%%%%%   A-S-3
V~obore reálnych čísel vyriešte rovnicu
$$
1+\sin\frac{x+\pi}{5}\cdot\sin\frac{x-\pi}{11}=0.
$$}
\podpis{J. Šimša}

{%%%%%   A-II-1
Nájdite všetky dvojice takých celých čísel~$a$, $b$, že súčet $a+b$ je
koreňom rovnice $x^2+ax+b=0$.}
\podpis{E. Kováč}

{%%%%%   A-II-2
Postupnosť reálnych čísel $(a_n)_{n=1}^{\infty}$
spĺňa pre každé $n\geqq1$ rovnosť
$$
\frac{a_{n+3}-a_{n+2}}{a_{n}-a_{n+1}}=
\frac{a_{n+3}+a_{n+2}}{a_{n}+a_{n+1}}
$$
a~naviac platí $a_{11}=4$, $a_{22}=2$, $a_{33}=1$. Dokážte, že pre
každé prirodzené číslo $k$ je súčet
$$
a_1^k+a_2^k+\cdots+a_{100}^k
$$
druhou mocninou prirodzeného čísla.}
\podpis{J. Zhouf}

{%%%%%   A-II-3
Daný je trojuholník $ABC$ a~vnútri neho bod~$P$. Označme $X$ priesečník
priamky~$AP$ so stranou~$BC$ a~$Y$ priesečník priamky~$BP$ so stranou~$AC$.
Dokážte, že štvoruholník $ABXY$ je tetivový práve vtedy, keď druhý
priesečník (rôzny od bodu~$C$) kružníc opísaných trojuholníkom $ACX$ a~$BCY$ leží na
priamke~$CP$.}
\podpis{E. Kováč}

{%%%%%   A-II-4
V~obore reálnych čísel riešte sústavu rovníc
$$
\align
  \sin^2 x+\cos^2 y &= y^2,\\
  \sin^2 y+\cos^2 x &= x^2.
\endalign
$$}
\podpis{J. Švrček}

{%%%%%   A-III-1
Postupnosť $(a_n)_{n=1}^{\infty}$ prirodzených čísel má tú
vlastnosť, že pre každé $n\ge1$ platí $a_{n+1}=a_n+b_n$, pričom
$b_n$ je číslo, ktoré má opačné poradie číslic ako číslo~$a_n$
(zápis čísla~$b_n$ môže na rozdiel od zápisu čísla~$a_n$ začínať
jednou alebo viacerými nulami). Napríklad pre $a_1=170$ platí
$a_2=241, a_3=383, a_4=766,\dots$
Rozhodnite, či $a_7$ môže byť prvočíslo.}
\podpis{Peter Novotný}

{%%%%%   A-III-2
Nech $m$ a~$n$ sú také prirodzené čísla, že rovnica
$$
(x+m)(x+n)=x+m+n
$$
má aspoň jedno celočíselné riešenie. Dokážte, že platí 
$$
\postdisplaypenalty 10000
\frac12<\frac mn<2.
$$}
\podpis{J. Šimša}

{%%%%%   A-III-3
V~trojuholníku $ABC$, ktorý nie je rovnostranný, označme $K$ priesečník osi
vnútorného uhla $BAC$ so stranou~$BC$ a~$L$ priesečník osi vnútorného uhla
$ABC$ so stranou~$AC$. Ďalej označme $S$ stred kružnice vpísanej,
$O$ stred kružnice opísanej a~$V$ priesečník výšok v~trojuholníku $ABC$. 
Dokážte, že nasledujúce dve tvrdenia sú ekvivalentné:
\ite a) Priamka~$KL$ sa dotýka kružníc opísaných trojuholníkom
        $ALS$, $BVS$ a~$BKS$.
\ite b) Body $A$, $B$, $K$, $L$ a~$O$ ležia na jednej kružnici.}
\podpis{T. Jurík}

{%%%%%   A-III-4
V~rovine je daná úsečka~$AB$. Zostrojte množinu ťažísk
všetkých ostrouhlých trojuholníkov $ABC$, pre ktoré platí:
Vrcholy $A$ a~$B$, priesečník výšok~$V$ a~stred~$S$ kružnice vpísanej
do trojuholníka $ABC$ ležia na jednej kružnici.}
\podpis{J. Švrček}

{%%%%%   A-III-5
Nájdite všetky trojice navzájom rôznych prvočísel $p$, $q$,
$r$ spĺňajúce nasledujúce podmienky:
$$
\align
p&\mid q+r,\\
q&\mid r+2p,\\
r&\mid p+3q.
\endalign
$$}
\podpis{M. Panák}

{%%%%%   A-III-6
V~obore reálnych čísel riešte sústavu rovníc
$$
\align
  \tg^2 x+2\cotg^2 2y &= 1,\\
  \tg^2 y+2\cotg^2 2z &= 1,\\
  \tg^2 z+2\cotg^2 2x &= 1.
\endalign
$$}
\podpis{J. Švrček, P. Calábek}

{%%%%%   B-S-1
Dokážte, že pre ľubovoľné kladné čísla $a$, $b$, $c$ platí nerovnosť
$$
\Bigl(a+{1\over b}\Bigr)
\Bigl(b+{1\over c}\Bigr)
\Bigl(c+{1\over a}\Bigr)\ge8.
$$
Zistite, kedy nastáva rovnosť.}
\podpis{J. Šimša}

{%%%%%   B-S-2
Na prepone~$AB$ pravouhlého trojuholníka $ABC$ uvažujme také body $P$ a~$Q$,
že $|AP|=|AC|$ a~$|BQ|=|BC|$. Označme $M$ priesečník kolmice z~vrcholu~$A$
na priamku~$CP$ a~kolmice z~vrcholu~$B$ na priamku~$CQ$. Dokážte,
že priamky $PM$ a~$QM$ sú navzájom kolmé.}
\podpis{J. Švrček}

{%%%%%   B-S-3
Nájdite všetky dvojice celých čísel $a$, $b$, pre ktoré žiadna 
z~rovníc
$$
x^2+ax+b=0, \quad y^2+by+a=0
$$
nemá dva rôzne reálne korene.}
\podpis{E. Kováč}

{%%%%%   B-II-1
Určte všetky dvojice prvočísel $p$, $q$, pre ktoré platí
$$
p+q^2=q+p^3.
$$}
\podpis{J. Švrček}

{%%%%%   B-II-2
Obdĺžnik $ABCD$ so stranami dĺžok $|AB|=2008$ a~$|BC|=2006$ je 
rozdelený na $2008\times2006$ jednotkových štvorcov a~tie sú striedavo
ofarbené čiernou, sivou a~bielou farbou podobne ako obdĺžnik na \obr{}:
štvorce pri vrcholoch $A$ a~$B$ sú čierne, štvorce pri vrcholoch 
$C$ a~$D$ sú biele. Určte obsah tej časti trojuholníka $ABC$, ktorá je sivá.
\insp{b55.1}
}
\podpis{Pavel Novotný}

{%%%%%   B-II-3
V~lichobežníku $ABCD$, ktorého základňa~$AB$ má dvakrát väčšiu dĺžku ako
základňa~$CD$, označme $E$ stred ramena~$BC$. Dokážte, že kružnica
opísaná trojuholníku $CDE$ prechádza stredom uhlopriečky~$AC$ práve vtedy, keď strany $AB$
a~$BC$ sú navzájom kolmé.}
\podpis{P. Leischner}

{%%%%%   B-II-4
Dokážte, že pre ľubovoľné reálne čísla $a$, $b$, $c$ z~intervalu
$\langle0,1\rangle$ platí 
$$
1\le a+b+c+2(ab+bc+ca)+3(1-a)(1-b)(1-c)\le9.
$$}
\podpis{J. Šimša}

{%%%%%   C-S-1
Na hokejovom turnaji sa zúčastnili štyri družstvá, pričom každé zohralo
s~každým práve jeden zápas. V~žiadnych dvoch zápasoch nepadlo rovnako veľa gólov, ale počet gólov
strelených v~každom zápase delí celkový počet gólov strelených na turnaji.
Koľko najmenej gólov mohlo na turnaji padnúť?}
\podpis{M. Panák}

{%%%%%   C-S-2
Vrchol~$C$ štvorcov $ABCD$ a~$CJKL$ je vnútorným bodom
úsečky~$AK$ aj úsečky~$DJ$. Body $E$, $F$, $G$ a~$H$ sú postupne stredy
úsečiek $BC$, $BK$, $DK$ a~$DC$. Vyjadrite obsah štvoruholníka $EFGH$ pomocou obsahov~$S$ a~$T$ štvorcov $ABCD$ a~$CJKL$.}
\podpis{P. Leischner}

{%%%%%   C-S-3
Kružnice $k$, $\ell$, $m$ sa dotýkajú spoločnej dotyčnice v~troch rôznych bodoch a~ich stredy ležia na jednej priamke. Kružnice $k$ a~$\ell$, a~tiež kružnice $\ell$ a~$m$, majú vonkajší dotyk. Určte polomer kružnice~$\ell$,
ak polomery kružníc $k$ a~$m$ sú $3\cm$ a~$12\cm$.}
\podpis{L. Boček}

{%%%%%   C-II-1
Základňa~$AB$ lichobežníka $ABCD$ je trikrát dlhšia ako
základňa~$CD$. Označme $M$ stred strany~$AB$ a~$P$ priesečník
úsečky~$DM$ s~uhlopriečkou~$AC$. Vypočítajte pomer obsahov
trojuholníka $CDP$ a~štvoruholníka $MBCP$.}
\podpis{Pavel Novotný}

{%%%%%   C-II-2
Ak reálne čísla $a$, $b$, $c$, $d$ spĺňajú rovnosti
$$
a^{2}+b^{2}=b^{2}+c^{2}=c^{2}+d^{2}=1,
$$
platí nerovnosť
$$
ab + ac + ad + bc + bd + cd\le3.
$$
Dokážte a~zistite, kedy za daných podmienok nastane rovnosť.}
\podpis{J. Švrček}

{%%%%%   C-II-3
Kružnice $k$, $\ell$ s~vonkajším dotykom ležia obe 
v~obdĺžniku $ABCD$, ktorého obsah je $72\cm^{2}$. Kružnica~$k$ sa
pritom dotýka strán $CD$, $DA$ a~$AB$, zatiaľ čo kružnica~$\ell$ sa dotýka
strán $AB$ a~$BC$. Určte polomery kružníc $k$ a~$\ell$,
ak viete, že polomer kružnice~$k$ je v~centimetroch vyjadrený celým číslom.}
\podpis{J. Švrček}

{%%%%%   C-II-4
Nájdite všetky dvojice prvočísel $p$, $q$, pre ktoré platí
$$
p+q^{2}=q+145p^{2}.
$$}
\podpis{J. Moravčík}

{%%%%%   vyberko, den 1, priklad 1
Ciferný súčet čísla $N$ je $100$, ciferný súčet čísla $5N$ je $50$. Dokážte, že $N$ je párne.}
\podpis{Ján Mazák:Rusko 2004/5, District Round, 8th form, 8.5}

{%%%%%   vyberko, den 1, priklad 2
Nech $AA'$ a~$BB'$ sú výšky ostrouhlého trojuholníka $ABC$. Bod~$D$ leží na oblúku $ACB$ kružnice opísanej trojuholníku $ABC$.
Nech $P$ je priesečník priamok $AA'$ a~$BD$, nech $Q$ je priesečník priamok $BB'$ a~$AD$. Dokážte, že priamka~$A'B'$ prechádza stredom úsečky~$PQ$.}
\podpis{Ján Mazák:Rusko 2004/5, Final Round, 9.7, tiez 10.6}

{%%%%%   vyberko, den 1, priklad 3
Niekoľko jabĺk a~pomarančov je rozmiestnených v~$99$ škatuliach. Dokážte, že vieme vybrať $50$~škatúľ tak, aby obsahovali aspoň polovicu všetkých pomarančov a~aspoň polovicu všetkých jabĺk.}
\podpis{Ján Mazák:Rusko 2004/5, District Round, 8th form, 8.8}

{%%%%%   vyberko, den 1, priklad 4
Vnútri štvorca so stranou dĺžky~$6$ sa nachádzajú body $A$, $B$, $C$, $D$ tak, že vzdialenosť ľubovoľných dvoch z~týchto bodov je aspoň~$5$.
Dokážte, že body $A$, $B$, $C$, $D$ vytvárajú konvexný štvoruholník s~obsahom aspoň $21$.}
\podpis{Ján Mazák:Rumunsko 2003, 4th selection test for the IMO, 53/11}

{%%%%%   vyberko, den 2, priklad 1
V~trojuholníku $ABC$, pre ktorý platí $|AB|+|BC|=3|AC|$, sa jemu vpísaná kružnica so stredom v~bode~$I$ dotýka strán $AB$ a~$BC$ v~bodoch $D$ a~$E$. Obrazy bodov $D$, $E$ v~stredovej súmernosti podľa bodu~$I$ označme $K$, $L$. Dokážte, že štvoruholník $ACKL$ je tetivový.
%{\it Zadanie bude zverejnené až po IMO 2006 v~Slovinsku.}
}
\podpis{Tomáš Jurík:IMO 2005 shortlist G1}

{%%%%%   vyberko, den 2, priklad 2
Ak pre prirodzené čísla $a$, $b$ platí, že pre každé prirodzené číslo~$n$ je číslo $b^n+n$ deliteľné číslom $a^n+n$, tak $a=b$. Dokážte.
%{\it Zadanie bude zverejnené až po IMO 2006 v~Slovinsku.}
}
\podpis{Tomáš Jurík:IMO 2005 shortlist N6}

{%%%%%   vyberko, den 2, priklad 3
Pre ľubovoľné reálne čísla $a_i\ge 1$ ($i=1,2,\dots,2006$) nájdite najmenšiu možnú hodnotu výrazu
$$
\frac{a_1^2+a_2}{a_1+a_2^2}+\frac{a_2^2+a_3}{a_2+a_3^2}+\cdots+
\frac{a_{2005}^2+a_{2006}}{a_{2005}+a_{2006}^2}+\frac{a_{2006}^2+a_1}{a_{2006}+a_1^2}\cdot
$$}
\podpis{Tomáš Jurík:vsorosivskaja MO 10.10}

{%%%%%   vyberko, den 3, priklad 1
Nech $\Bbb R^+$ je množina všetkých kladných reálnych čísel. Dokážte, že jediná funkcia $f:\Bbb R^+\to\Bbb R^+$, ktorá spĺňa rovnosť
$$
f(x)f(y) = 2f(x+yf(x))
$$
pre každú dvojicu kladných reálnych čísel $x$, $y$, je konštantná funkcia $f(x)\equiv2$.
%{\it Zadanie bude zverejnené až po IMO 2006 v~Slovinsku.}
}
\podpis{Peter Novotný:Shortlist 2005, A2}

{%%%%%   vyberko, den 3, priklad 2
Daný je rovnobežník $ABCD$. Priamka~$\ell$ prechádzajúca bodom~$A$ pretína polpriamky $BC$, $DC$ postupne v~bodoch $X$, $Y$. Označme $K$, $L$ stredy kružníc pripísaných postupne k~stranám $BX$, $DY$ trojuholníkov $ABX$, $ADY$. Dokážte, veľkosť uhla $KCL$ nezávisí na polohe priamky~$\ell$.
%{\it Zadanie bude zverejnené až po IMO 2006 v~Slovinsku.}
}
\podpis{Peter Novotný:Shortlist 2005, G3}

{%%%%%   vyberko, den 3, priklad 3
Na stole je $n$~mincí ($n\ge2$). Každá je z~jednej strany biela a~z~druhej strany čierna. Mince sú na začiatku poukladané v~jednom rade bielou stranou nahor. V~každom kroku, pokiaľ je to možné, zvolíme jednu mincu otočenú bielou stranou nahor (nie však jednu z~dvoch krajných mincí), odložíme ju zo stola preč a~obrátime naopak mince, ktoré s~odloženou susedili, t.\,j. najbližšiu mincu napravo a~najbližšiu naľavo. Zistite, pre ktoré $n$ môžu (ak robíme správne kroky) na stole ostať len dve krajné mince.
%{\it Zadanie bude zverejnené až po IMO 2006 v~Slovinsku.}
}
\podpis{Peter Novotný:Shortlist 2005, C5}

{%%%%%   vyberko, den 4, priklad 1
Dokážte, že každé prirodzené číslo $k>1$ má kladný násobok menší ako $k^5$, ktorého dekadický zápis obsahuje nanajvýš štyri rôzne číslice.}
\podpis{Erika Trojáková:priprava na jarnika 2006}

{%%%%%   vyberko, den 4, priklad 2
Nech $ABCD$ je štvoruholník, pre ktorý platí $|\uhol CBD|= 2|\uhol ADB|$, $|\uhol ABD|= 2|\uhol CDB|$ a~$|AB|=|CB|$. Dokážte, že $|AD|=|CD|$.}
\podpis{Erika Trojáková:Kanadska MO 2000}

{%%%%%   vyberko, den 4, priklad 3
Zistite, či existuje množina $18$ po sebe idúcich prirodzených čísel, ktorá sa dá rozdeliť na dve disjunktné množiny 
$A$, $B$ tak, že súčin prvkov množiny $A$ sa rovná súčinu prvkov množiny~$B$.}
\podpis{Erika Trojáková:Estonska MO 2000}

{%%%%%   vyberko, den 4, priklad 4
Nájdite všetky kvadratické polynómy $p(x)$ tvaru $x^2+ax+b$ s~celočíselnými koeficientmi, pre ktoré existuje polynóm $q(x)$ s~celočíselnými koeficientmi taký, že $p(x)q(x)$ je polynóm, ktorého všetky koeficienty sú $\pm 1$.
%{\it Zadanie bude zverejnené až po IMO 2006 v~Slovinsku.}
}
\podpis{Erika Trojáková:Shortlist 2005, A1}

{%%%%%   vyberko, den 5, priklad 1
Najnovší výskum v~oblasti červích dier konečne umožnil
medzigalaktické cestovanie. Do vybudovania GIGANT (Global
Intergalactic Network for Travel) sa zapojilo $n$~galaxií
$G_1,G_2,\dots,G_n$. Každá dvojica z~nich má byť prepojená práve
jednou priamou diaľnicou, ktorej dĺžka vyjadrená v~jednotkách yt
(years of travel) bude prirodzené číslo a~nebude presahovať $r$ tak,
že
\ite (1)
pre každé prirodzené číslo od 1 po $r$ (vrátane) bude aspoň jedna diaľnica
s danou dĺžkou;
\ite (2)
pre každú okružnú cestu $G_iG_jG_k$ budú dve z~týchto troch diaľnic
rovnako dlhé a~každá z~nich bude dlhšia ako tretia.

Úlohy:
\ite (a)
Určte najväčšiu hodnotu prirodzeného čísla~$r$, pre ktorú sa dá
vybudovať GIGANT.
\ite (b)
Koľko rôznych diaľničných sietí sa dá vybudovať pre túto hodnotu?
%{\it Zadanie bude zverejnené až po IMO 2006 v~Slovinsku.}
}
\podpis{Martin Potočný:Shortlist 2005, C4}

{%%%%%   vyberko, den 5, priklad 2
Nech $ABC$ je ostrouhlý trojuholník $|AB|\ne |AC|$, nech $H$ je jeho ortocentrum a~$M$
je stred strany~$BC$. Body $D$ na $AB$ a~$E$ na $AC$ sú také, že $|AE|=|AD|$
a $D$, $H$, $E$ sú kolineárne.
Dokážte, že priamka~$HM$ je kolmá na chordálu kružníc opísaných trojuholníkom
$ABC$ a~$ADE$.
%{\it Zadanie bude zverejnené až po IMO 2006 v~Slovinsku.}
}
\podpis{Martin Potočný:Shortlist 2005, G5}

{%%%%%   vyberko, den 5, priklad 3
Nájdite všetky prirodzené čísla $n>1$ také, pre ktoré
existuje jediné prirodzené číslo $a\le n!$ také, že $a^n+1$ je
deliteľné číslom~$n!$.
%{\it Zadanie bude zverejnené až po IMO 2006 v~Slovinsku.}
}
\podpis{Martin Potočný:Shortlist 2005, N4}

{%%%%%   vyberko, den 2, priklad 4
...}
\podpis{...}

{%%%%%   vyberko, den 3, priklad 4
...}
\podpis{...}

{%%%%%   vyberko, den 5, priklad 4
...}
\podpis{...}

{%%%%%   trojstretnutie, priklad 1
Na kružnici s~polomerom~$r$ leží 5~rôznych bodov $A$, $B$, $C$, $D$, $E$ v~tomto poradí, pričom platí $|AC|=|BD|=|CE|=r$. Dokážte, že trojuholník, ktorého vrcholy sú ortocentrá trojuholníkov $ACD$, $BCD$ a~$BCE$, je pravouhlý.}
\podpis{Tomáš Jurík}

{%%%%%   trojstretnutie, priklad 2
Okolo okrúhleho stola sedí $n$~detí. Erika je z~nich najstaršia a~má $n$~cukríkov. Ostatné deti nemajú žiadne cukríky. Erika sa rozhodla, že cukríky rozdelí a~stanovila nasledovné pravidlá. V~každom kole zdvihnú ruky všetky deti, ktoré majú pri sebe aspoň dva cukríky. Erika jedného z~prihlásených vyberie a~ten dá každému svojmu susedovi jeden cukrík. (V~prvom kole sa teda prihlási iba Erika a~dá svojim dvom susedom po cukríku.) Zistite, pre ktoré $n\ge3$ môže delenie po konečnom počte kôl skončiť tak, že každé dieťa bude mať práve jeden cukrík.}
\podpis{Peter Novotný}

{%%%%%   trojstretnutie, priklad 3
Súčet štyroch reálnych čísel sa rovná 9, súčet ich druhých mocnín sa rovná 21. Dokážte, že dané čísla možno označiť $a$, $b$, $c$ a~$d$ tak, aby platila nerovnosť $ab-cd\ge2$.}
\podpis{Jaromír Šimša}

{%%%%%   trojstretnutie, priklad 4
Dokážte, že pre každé prirodzené číslo $k\ge1$ existuje také prirodzené číslo~$n$, že v~zápise čísla $2^n$ v~desiatkovej sústave sa nachádza blok práve $k$ za sebou idúcich núl, \tj.
$$
  2^n=\dots a\underbrace{00\dots0}_{\text{$k$~núl}}b\dots,
$$
pričom cifry $a$, $b$ sú nenulové.}
\podpis{Peter Novotný}

{%%%%%   trojstretnutie, priklad 5
Zistite, koľko existuje postupností celých čísel $(a_n)_{n=1}^\infty$ takých, že pre každé prirodzené číslo~$n$ platí
$$
a_n\ne-1
\qquad
\text{a}
\qquad
a_{n+2}=\frac{a_n+2\,006}{a_{n+1}+1}.
$$}
\podpis{Peter Novotný}

{%%%%%   trojstretnutie, priklad 6
Zistite, či existuje taký konvexný päťuholník $A_1A_2A_3A_4A_5$, že pre každé $i=1,2,3,4,5$  sú priamky
$A_iA_{i+3}$, $A_{i+1}A_{i+2}$ rôznobežné a~pretínajú sa v~bode~$B_i$, pričom body $B_1$, $B_2$, $B_3$, $B_4$, $B_5$
ležia na jednej priamke. (Uvažujeme $A_6=A_1$, $A_7=A_2$ a~$A_8=A_3$.)}
\podpis{Waldemar Pompe}

{%%%%%   IMO, priklad 1
Nech $I$ je stred kružnice vpísanej do trojuholníka $ABC$. Bod~$P$
z~vnútra trojuholníka spĺňa
$$
|\uhol PBA|+|\uhol PCA|=|\uhol PBC|+|\uhol PCB|.
$$
Dokážte, že ${|AP|\ge |AI|}$, pričom rovnosť nastane práve vtedy, keď $P=I$.}
\podpis{Južná Kórea}

{%%%%%   IMO, priklad 2
Nech $P$ je pravidelný 2006-uholník. Jeho uhlopriečka sa nazýva {\it dobrá}, ak jej koncové
body rozdeľujú hranicu mnohouholníka~$P$ na dve časti, z~ktorých každá pozostáva z~nepárneho
počtu strán. Strany mnohouholníka~$P$ sa tiež považujú za {\it dobré}.
Predpokladajme, že $P$ je rozdelený na trojuholníky 2003 uhlopriečkami,
z~ktorých žiadne dve nemajú spoločný bod vo vnútri~$P$. Nájdite
maximálny možný počet rovnoramenných trojuholníkov, ktoré majú dve dobré strany.}
\podpis{Srbsko a~Čierna Hora}

{%%%%%   IMO, priklad 3
Určte najmenšie reálne číslo $M$ tak, aby nerovnosť
$$
|ab(a^2-b^2)+bc(b^2-c^2)+ca(c^2-a^2)| \le M(a^2+b^2+c^2)^2
$$
platila pre všetky reálne čísla $a$, $b$, $c$.}
\podpis{Írsko}

{%%%%%   IMO, priklad 4
Určte všetky dvojice $(x,y)$ celých čísel takých, že
$$
1+2^{x}+2^{2x+1}=y^2.
$$}
\podpis{USA}

{%%%%%   IMO, priklad 5
Nech $P(x)$ je polynóm stupňa ${n>1}$ s~celočíselnými koeficientmi 
a~nech $k$ je kladné celé číslo. Uvažujme polynóm $Q(x)=P(P(\dots P(P(x))\dots))$,
kde $P$ sa vyskytuje $k$-krát. Dokážte, že existuje najviac $n$ celých čísel~$t$ takých,
že $Q(t) = t$.}
\podpis{Rumunsko}

{%%%%%   IMO, priklad 6
Každej strane~$b$ konvexného mnohouholníka~$P$ priradíme maximálny obsah 
trojuholníka, ktorého jedna strana je $b$ a~ktorý je obsiahnutý v~$P$.
Dokážte, že súčet obsahov priradených všetkým stranám mnohouholníka~$P$ je aspoň dvojnásobkom
obsahu mnohouholníka~$P$.}
\podpis{Srbsko a Čierna Hora}

