{%%%%%   Z4-I-1
Jurko rád kreslí autíčka. V pondelok namaľoval niekoľko autíčok, v utorok ich namaľoval
trikrát toľko ako v pondelok, ale potom 12 pondelkových vygumoval. V stredu mal zlú
náladu, nič nenamaľoval a ešte roztrhal polovicu utorkových, čiže 24. Koľko autíčok
namaľoval v utorok? Koľko autíčok namaľoval v pondelok, v utorok a v stredu spolu?}
\podpis{M. Dillingerová}

{%%%%%   Z4-I-2
Z čísla $9\,635\,347$ vyškrtni niekoľko číslic tak, aby vzniklo čo najväčšie číslo a súčet všetkých
jeho číslic bol menší ako $20$.}
\podpis{M. Dillingerová}

{%%%%%   Z4-I-3
Rozprávkový nafukovací štvorec, ktorý vie rozprávať, mal pred 5 minútami dĺžku strany $8\cm$.
Pri každom klamstve zväčší svoj obvod dvojnásobne. Pri každej vyslovenej pravde sa zmenší
dĺžka každej jeho strany o $2\cm$. Za posledných 5 minút 2-krát klamal a 2-krát hovoril pravdu.
\begin{itemize}
\itemvar{a)} Aký najväčší obvod môže mať teraz?
\itemvar{b)} Aký najmenší obvod môže mať teraz?
\end{itemize}
}
\podpis{S. Bodláková}

{%%%%%   Z4-I-4
Peter má na papier napísať z čísel od 1 do 200 všetky také, ktoré sa dajú deliť piatimi bezo
zvyšku, ale nedajú sa deliť bezo zvyšku siedmimi. Pavol má vypísať z čísel od 1 do 200
všetky také, ktoré po delení siedmimi dávajú zvyšok 5. Koľko čísel má napísať Peter a koľko
Pavol?}
\podpis{M. Smitková}

{%%%%%   Z4-I-5
Žabka Rosnička stála na rebríku, ktorý mal 5 priečok, na tretej priečke. Urobila šesť skokov
a zostala stáť na piatej priečke. Vypíš všetky možnosti, ako mohla Rosnička skákať, ak vždy
skočila len o jednu priečku hore alebo o jednu priečku dole.}
\podpis{S. Bodláková}

{%%%%%   Z4-I-6
Majka má v stavebnici len rovnako veľké kocky s hranou dĺžky $3\cm$. Keď z nich postaví
vežu, ktorá má na každom podlaží 4 kocky, bude mať veža výšku $54\cm$. Aká vysoká by bola
iná veža z takého istého počtu rovnakých kociek, ktorá by mala v každom podlaží deväť
kociek?}
\podpis{M. Dillingerová}

{%%%%%   Z5-I-1
...}
\podpis{...}

{%%%%%   Z5-I-2
Blcha Skákalka skáče po vodorovnej číselnej osi. Vie robiť iba skoky dvoch dĺžok. Jedným
preskočí o 14 doprava alebo doľava, druhým preskočí o 18 doprava alebo doľava. Práve stojí
na čísle 2.
\begin{itemize}
\itemvar{a)} Nájdi spôsob, ako má skákať, aby sa štyrmi skokmi dostala na číslo 10.
\itemvar{b)} Tvrdí, že včera bola na čísle 13. Hovorí pravdu alebo klame? Zdôvodni.
\end{itemize}
}
\podpis{M. Dillingerová}

{%%%%%   Z5-I-3
Rozprávkový nafukovací štvorec, ktorý vie rozprávať, mal pred 5 minútami dĺžku strany $8\cm$.
Pri každom klamstve zväčší svoj obvod dvojnásobne. Pri každej vyslovenej pravde sa zmenší
dĺžka každej jeho strany o $2\cm$. Za posledných 5 minút 2-krát klamal a 2-krát hovoril pravdu.
\begin{itemize}
\itemvar{a)} Aký najväčší obvod môže mať teraz?
\itemvar{b)} Aký najmenší obvod môže mať teraz?
\end{itemize}
}
\podpis{S. Bodláková}

{%%%%%   Z5-I-4
Peťo si kúpil na jarmoku 4 autíčka, biele, zelené, červené a modré. Biele autíčko stálo dvakrát
toľko ako červené, zelené trikrát toľko ako biele a za modré zaplatil toľko ako za červené a
biele spolu. Pritom červené stálo o 70 Sk menej ako zelené. Koľko Sk stáli jednotlivé autíčka?}
\podpis{Š. Ptáčková}

{%%%%%   Z5-I-5
Mama stonožka má dve deti a manžela. Každý z nich má sto nôh a každý člen rodiny si berie
každý deň čisté ponožky. V sobotu ráno o 6:00 začala mamička stonožka prať špinavé
ponožky. Naraz sa jej ich do práčky zmestí 357 a jedna várka sa operie za dve a pol hodiny.
Zisti, kedy skončí s praním, ak vieš, že ponožky perie raz za týždeň, uloženie ponožiek do
práčky jej trvá 2 minúty a ich vybratie z práčky 3 minúty.}
\podpis{S. Bednářová}

{%%%%%   Z5-I-6
...}
\podpis{...}

{%%%%%   Z6-I-1
...}
\podpis{...}

{%%%%%   Z6-I-2
Snehulienka so siedmimi trpaslíkmi zbierali lieskové oriešky. Snehulienka ich mala toľko ako všetci trpaslíci spolu. Keď sa vracali domov, stretli veveričku Finku. Snehulienka aj každý trpaslík jej dali rovnaký počet orieškov. Doma trpaslíci a Snehulienka vysypali oriešky na kôpky na stole, každý na inú kôpku a Vedko zapísal počty orieškov v kôpkach: 120, 316, 202, 185, 333, 297, 111 a 1672. Koľko orieškov dostala veverička Finka?}
\podpis{L. Hozová}

{%%%%%   Z6-I-3
Keď sme čísla 80 a 139 vydelili tým istým prirodzeným číslom, získali sme zvyšky 8 a 13. Ktorým číslom sme delili?}
\podpis{M. Volfová}

{%%%%%   Z6-I-4
Obvod trojuholníka je $16\cm$. Aké môžu byť dĺžky jeho strán, keď sú to prirodzené čísla a súčet dĺžok dvoch strán je o $6\cm$ väčší ako dĺžka tretej strany?}
\podpis{L. Hozová}

{%%%%%   Z6-I-5
Maruška dostala päť rôzne ťažkých koláčov. Priemerná hmotnosť koláča bola 200 gramov. Maruška jeden koláč zjedla a priemerná hmotnosť zvyšných koláčov potom bola 160 gramov. Koľko gramov vážil koláč, ktorý Maruška zjedla?}
\podpis{B. Šťastná}

{%%%%%   Z6-I-6
...}
\podpis{...}

{%%%%%   Z7-I-1
Pat a Mat upravovali nový asfalt na ceste. Najprv išli s valcom 10\,m dopredu, potom 7\,m cúvli, opäť prešli 10\,m dopredu a 7\,m cúvli...Takto pokračovali, až kým po prvý raz nezišli z nového asfaltu.
\begin{itemize}
\itemvar{a)} Koľko metrov najazdil valec na novom 540 metrovom úseku cesty?
\itemvar{b)} Koľkokrát prešli po devätnástom metri nového asfaltu?
\end{itemize}
}
\podpis{M. Dillingerová}

{%%%%%   Z7-I-2
...}
\podpis{...}

{%%%%%   Z7-I-3
...}
\podpis{...}

{%%%%%   Z7-I-4
Nájdite všetky päťciferné čísla, ktoré sa škrtnutím prvej a poslednej cifry zmenšia 250-krát.}
\podpis{L. Šimůnek}

{%%%%%   Z7-I-5
Pavol dostal na domácu úlohu vyjadriť desatinným číslom zlomky $\frac37$ a $\frac7{13}$. Aby urobil pani učiteľke radosť, písal úlohu miesto do zošita na latky školského plota. Najprv vyjadroval $\frac37$, teda na prvú latku navrch napísal $0$, na druhú desatinnú čiarku, na tretiu $4$ atď. Keď skončil, napísal pod tieto čísla vyjadrenie zlomku $\frac7{13}$. Na prvú latku dole napísal $0$, na druhú desatinnú čiarku, na tretiu $5$ atď. Koľko bolo latiek na plote, ak viete, že číslicu $5$ napísal presne 667-krát a že na 668 latkách bola dvojica rovnakých čísel?}
\podpis{M. Dillingerová, P. Tlustý}

{%%%%%   Z7-I-6
...}
\podpis{...}

{%%%%%   Z8-I-1
Určte všetky dvojciferného čísla, pre ktoré súčin ciferného súčtu a ciferného súčinu je 126.}
\podpis{M. Raabová}

{%%%%%   Z8-I-2
Pani Zručná sa uchádzala o miesto vo výrobe vianočných perníkov. Pri pohovore s vedúcim chcela povedať, koľko perníkov ozdobí za koľko minút. Bola nervózna, a preto omylom prehodila počet minút s počtom perníkov. Vedúci podľa jej údajov vypočítal, koľko perníkov by mala pani Zručná stihnúť ozdobiť za päťhodinovú pracovnú dobu, a presne tento počet jej dal za úlohu ozdobiť. Pani Zručnej trvala práca o 2 hodiny a 12 minút dlhšie. Koľko perníkov ozdobila?}
\podpis{L. Šimůnek}

{%%%%%   Z8-I-3
...}
\podpis{...}

{%%%%%   Z8-I-4
Roman písal na papier za sebou celé čísla tak, že nasledujúce získal z predchádzajúceho striedavo násobením dvoma a odčítaním troch. Napr. postupnosť čísel $1$; $2$; $\m1$; $\m2$; $\m5$; $\m10$ vyhovuje jeho pravidlu, ale postupnosť $10$; $7$; $4$; $8$; $16$; $32$ jeho pravidlo nesplňuje. Po chvíli sčítal posledných 5 čísel, ktoré napísal a vyšlo mu $114$. Ktorých 5 čísel sčitoval?}
\podpis{M. Raabová}

{%%%%%   Z8-I-5
...}
\podpis{...}

{%%%%%   Z8-I-6
Jakub má tento školský rok priemer všetkých svojich známok 1,85. Za celý školský rok dostal iba štyri pätorky a práve tretina jeho známok boli jednotky. Najmenej koľko známok musel tento školský rok dostať?}
\podpis{L. Šimůnek}

{%%%%%   Z9-I-1
Určte počet trojciferných prirodzených čísel, ktoré majú práve dve rovnaké cifry.}
\podpis{P. Tlustý}

{%%%%%   Z9-I-2
...}
\podpis{...}

{%%%%%   Z9-I-3
V súradnicovej sústave sme znázornení body $A[3,2]$, $B[\m1,1]$, $C[\m2,4]$ a ich obrazy $A'$, $B'$, $C'$ v~stredovej súmernosti so stredom v začiatku súradnicovej sústavy. Vypočítajte obsah šesťuholníka $ABCA'B'C'$.}
\podpis{S. Bednářová}

{%%%%%   Z9-I-4
Starý podnikateľ zomrel a zanechal po sebe dva účty, jeden dlh a testament. V~testamente sa
písalo, že peniaze z~prvého účtu si majú rozdeliť 1. a 2. syn v pomere $1:2$, peniaze z druhého
účtu 1. a 3. syn v pomere $1:3$ a dlh majú zaplatiť 2. a 3. syn v pomere $2:3$. Zistite, koľko Sk
bolo na jednotlivých účtoch a aký dlh museli synovia splatiť, ak viete, že v konečnom
dôsledku každý z nich získal 123\,456\,Sk.}
\podpis{S. Bednářová}

{%%%%%   Z9-I-5
Dva rovnostranné papierové trojuholníky, z ktorých menší má obsah $60\cm^2$, sme položili cez
seba tak, že ich prienikom bol pravouhlý trojuholník s obsahom $30\cm^2$. Aký najmenší obsah
mohol mať väčší z rovnostranných trojuholníkov?}
\podpis{S. Bednářová}

{%%%%%   Z9-I-6
Zadanie písomnej práce obsahovalo 26 otázok, ktoré boli rozdelené podľa obtiažnosti do
troch skupín. V prvej skupine bola správna odpoveď hodnotená tromi bodmi, v druhej piatimi
bodmi a v tretej ôsmimi bodmi. Maximálny možný počet získaných bodov bol 111. Koľko
otázok mohlo byť v každej skupine?}
\podpis{L. Šimůnek}

{%%%%%   Z4-II-1
Z čísla $9\,635\,347$ vyškrtni niekoľko číslic tak, aby vzniklo čo najmenšie číslo a aby súčet všetkých
jeho číslic bol väčší ako $23$.}
\podpis{M. Dillingerová}

{%%%%%   Z4-II-2
Rozprávkový nafukovací obdĺžnik, ktorý vie rozprávať, mal pred 5 minútami dĺžky strán $8\cm$ a $10\cm$. Pri každom klamstve zväčší jednu dvojicu svojich rovnako dlhých strán dvojnásobne. Pri
každej vyslovenej pravde sa zmenší dĺžka každej jeho strany o $3\cm$. Za posledných 5 minút 1-krát
klamal a 2-krát hovoril pravdu.
\begin{itemize}
\itemvar{a)} Aký najväčší obvod môže mať teraz?
\itemvar{b)} Aký najmenší obvod môže mať teraz?
\end{itemize}
}
\podpis{S. Bodláková, M. Dillingerová}

{%%%%%   Z4-II-3
Kubkovi sa páčia čísla, ktoré dávajú po delení piatimi zvyšok 1. Lukáškovi sa páčia čísla, ktoré
nemajú žiadne dve číslice rovnaké. Napíš najmenšie trojciferné číslo, ktoré sa bude páčiť aj
Kubkovi aj Lukáškovi.}
\podpis{S. Bednářová}

{%%%%%   Z5-II-1
...}
\podpis{...}

{%%%%%   Z5-II-2
...}
\podpis{...}

{%%%%%   Z5-II-3
Myslím si trojciferné prirodzené číslo menšie ako 200. Ak jeho trojnásobok zaokrúhlim na stovky,
zväčší sa o 36. Ktoré číslo si myslím?}
\podpis{M. Dillingerová}

{%%%%%   Z6-II-1
...}
\podpis{...}

{%%%%%   Z6-II-2
V jaskyni žije päť obrov. Ich priemerná výška je $15\,504{,}63\cm$. Obor Drobček meria $174{,}53\,\text{m}$, obor Lomikameň $173\,530{,}5\,\text{mm}$, obor Zlobor $1\,745{,}23\,\text{dm}$, obor Hrompác $0{,}017\,34\,\text{km}$.
\begin{itemize}
\itemvar{a)} Zisti koľko meria piaty obor -- Kolodej.
\itemvar{b)} Zoraď obrov podľa veľkosti od najmenšieho po najväčšieho.
\end{itemize}
}
\podpis{S. Bednářová}

{%%%%%   Z6-II-3
Snehulienka dala každému trpaslíkovi rovnako veľa guľôčok. Trpaslíci potom hádzali kockou. Koľko im padlo bodiek, toľko guľôčok museli vrátiť Snehulienke. Každému z prvých šiestich trpaslíkov padol iný počet bodiek, takže každý z nich vrátil Snehulienke iný počet guľôčok. Koľko bodiek padlo poslednému trpaslíkovi, ak všetkým siedmim trpaslíkom po tomto hode ostalo spolu 46 guľôčok? Koľko guľôčok dala Snehulienka každému trpaslíkovi na začiatku?}
\podpis{S. Bednářová}

{%%%%%   Z7-II-1
...}
\podpis{...}

{%%%%%   Z7-II-2
Nájdi všetky štvorciferné čísla, ktoré sa škrtnutím prostredných dvoch cifier zmenšia 120-krát.}
\podpis{S. Bednářová}

{%%%%%   Z7-II-3
...}
\podpis{...}

{%%%%%   Z8-II-1
Určte všetky dvojciferného čísla, pre ktoré súčin ciferného súčtu a ciferného súčinu je 126.}
\podpis{M. Raabová}

{%%%%%   Z8-II-2
Pani Zručná sa uchádzala o miesto vo výrobe vianočných perníkov. Pri pohovore s vedúcim chcela povedať, koľko perníkov ozdobí za koľko minút. Bola nervózna, a preto omylom prehodila počet minút s počtom perníkov. Vedúci podľa jej údajov vypočítal, koľko perníkov by mala pani Zručná stihnúť ozdobiť za päťhodinovú pracovnú dobu, a presne tento počet jej dal za úlohu ozdobiť. Pani Zručnej trvala práca o 2 hodiny a 12 minút dlhšie. Koľko perníkov ozdobila?}
\podpis{L. Šimůnek}

{%%%%%   Z8-II-3
...}
\podpis{...}

{%%%%%   Z9-II-1
Peter a Jano išli plávať. Vzdialenosti, ktoré odplávali, boli v pomere $4:5$. (Jano odplával viac.)
Ďalší deň šli opäť plávať, teraz preplával Peter o 200 metrov menej a Jano o 100 metrov viac
ako predchádzajúci deň. Pomer vzdialeností za druhý deň bol $5:8$. Koľko metrov preplával prvý
deň Peter a koľko Jano?}
\podpis{B. Šťastná}

{%%%%%   Z9-II-2
...}
\podpis{...}

{%%%%%   Z9-II-3
...}
\podpis{...}

{%%%%%   Z9-II-4
V rovnostrannom trojuholníku vyznačte každý bod, ktorého vzdialenosť od najbližšieho
vrcholu je menšia ako vzdialenosť od ťažiska. Koľko percent plochy rovnostranného
trojuholníka tvoria všetky vyznačené body?}
\podpis{L. Šimůnek}

{%%%%%   Z9-III-1
...}
\podpis{...}

{%%%%%   Z9-III-2
...}
\podpis{...}

{%%%%%   Z9-III-3
Jakub mal v pivnici tri krabice tvaru kvádra so štvorcovou podstavou. Prvá bola
dvojnásobne vyššia ako druhá. Druhá bola jeden a pol násobne širšia ako prvá. Tretia
bola trojnásobne vyššia ako prvá a dvojnásobne užšia ako prvá. V akom pomere sú
objemy nádob?}
\podpis{Š. Ptáčková}

{%%%%%   Z9-III-4
Pri prijímacích skúškach na univerzitu sa každému záujemcovi o štúdium prideľuje
krycí kód zložený z piatich číslic. Skúšky organizoval dôkladný, ale poverčivý docent,
ktorý sa pred prideľovaním kódov rozhodol vyradiť zo všetkých možných kódov (\tj.
$00000$ až $99999$) tie, ktoré v sebe obsahovali číslo $13$, teda číslicu $3$ bezprostredne
nasledujúcu po číslici $1$. Koľko kódov musel docent vyradiť?}
\podpis{L. Šimůnek}

