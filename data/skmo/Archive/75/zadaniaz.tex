{%%%%%   Z4-I-1
...}
\podpis{...}

{%%%%%   Z4-I-2
...}
\podpis{...}

{%%%%%   Z4-I-3
...}
\podpis{...}

{%%%%%   Z4-I-4
...}
\podpis{...}

{%%%%%   Z4-I-5
...}
\podpis{...}

{%%%%%   Z4-I-6
...}
\podpis{...}

{%%%%%   Z5-I-1
Zvláštna kalkulačka má iba dve funkčné tlačidlá. Po stlačení prvého tlačidla sa k~číslu na displeji pripočíta číslo jedna, po stlačení druhého tlačidla sa číslo na displeji vynásobí dvoma.
Na displeji po každom stlačení tlačidla svieti správny výsledok.

Nájdite dva rôzne spôsoby, ako pomocou šiestich stlačení tlačidiel dostať na displeji z~čísla 1 číslo 15.
}
\podpis{Iveta Jančigová}

{%%%%%   Z5-I-2
K~zámku patrí park v tvare štvorca so stranou dĺžky 240 metrov. Náčrt parku je na obrázku.
Po stranách štvorca vedú cesty, v jeho vrcholoch stojí agát, buk, céder a~dub.
Park križujú dva ďalšie chodníky rovnobežné so stranami štvorcového parku -- jeden vedie od javora k studničke, druhý vedie od lipy k orechu.
Princezná pri svojich prechádzkach parkom chodila len po cestách a chodníkoch, nikdy sa nevracala, ani zbytočne neodbočovala. Zistila, že:
\begin{itemize}
  \item prechádzka od buka k studničke okolo javora, agáta, orecha a duba je dvakrát dlhšia ako okolo lipy a cédra.
  \item prechádzka od buka k studničke okolo lipy a cédra je rovnako dlhá ako prechádzka od duba k lipe okolo studničky a cédra.
\end{itemize}
Aká dlhá je priama cesta od agáta k orechu?
\Image{75-z5-i-2}%
}
\podpis{Marián Macko}

{%%%%%   Z5-I-3
Danka a Janka každá pre seba nazbierali jahody. Keby mala Janka o polovicu viac jahôd, než nazbierala, mala by ich rovnako ako Danka.
Keby mala Janka dvakrát viac jahôd, než nazbierala, mala by ich o~48 viac ako Danka.

Koľko jahôd nazbierala Janka a~koľko Danka?
}
\podpis{Monika Dillingerová}

{%%%%%   Z5-I-4
Anežka správne vynásobila určité prirodzené číslo siedmimi a výsledné päťciferné číslo napísala na papier.
Papagáj kus papiera vyďobal a tak sa prvá cifra výsledku nečitateľnou.
Na zvyšku papiera zostalo napísané 2887.

Aká mohla byť prvá cifra Anežkinho výsledku?
Nájdite všetky možnosti.
}
\podpis{Michaela Petrová}

{%%%%%   Z5-I-5
Veky siedmich kamarátov sú 8, 9, 10, 11, 11, 13 a~14 rokov.
Traja kamaráti sú práve v kine, dvaja sú na futbale a dvaja doma.
Súčet vekov tých v kine je 30 rokov, súčet vekov tých na futbale je 24 rokov.
Každý z kamarátov na futbale má viac rokov ako Ondrej, ktorý zostal doma.

Koľko rokov môže mať Ondrej?
Nájdite všetky možnosti.
}
\podpis{Marián Macko}

{%%%%%   Z5-I-6
Štyri dievčatá dláždili štvorcovými dlaždicami terasu v rohu dvora.
Viola používala dlaždice so stranou 1\,dm, Ruženka so stranou 2\,dm, Bianka so stranou 3\,dm a~Karmen so stranou 4\,dm.
Prvé z dievčat položilo jednu zo svojich dlaždíc do rohu.
Druhé položilo svoje dlaždice pozdĺž voľných strán predchádzajúcej dlaždice a pridalo jednu navyše tak, aby vznikol štvorec (poz. obrázok).
Podobným spôsobom položilo svoje dlaždice tretie a nakoniec aj štvrté dievča.
Takto vznikla štvorcová terasa bez medzier a prekryvov.

V akom poradí mohli dievčatá pokladať dlaždice a~koľko dlaždíc celkom použili?
Nájdite všetky možnosti.
\Image{75-z5-i-6}%
}
\podpis{Marián Macko}

{%%%%%   Z6-I-1
V~nasledujúcom sčítacom algebrograme zodpovedajú rôzne písmená rôznym cifrám, rovnaké rovnakým:
\Image{75-z6-i-1}
Nahraďte písmená ciframi tak, aby bol výpočet správny.
Nájdite všetky možnosti.
}
\podpis{Iveta Jančigová}

{%%%%%   Z6-I-2
Od 1. januára bol pán Novák zamestnaný v ~novej firme, pričom nástupná výška jeho platu bola 1\,550 eur mesačne.
Pretože sa osvedčil, od istého mesiaca v prvom polroku mu bol zvýšený plat o niekoľko celých eur. Takýto plat dostával mesačne už vo všetkých ďalších mesiacoch.
Za celý rok si zarobil dokopy 20\,000 eur.

Za ktorý mesiac mohol dostať prvýkrát zvýšený plat a~o~koľko eur?
Nájdite všetky možnosti.
}
\podpis{Marián Macko}

{%%%%%   Z6-I-3
Obdĺžnik $ABCD$ je pomocou vodorovných a zvislých čiar rozdelený na 48 rovnakých štvorčekov, tak ako vidíme na obrázku.
\Image{75-z6-i-3}%

Bod $N$ leži v štvrtine jeho strany $AB$. Rozdeľte obdĺžnik $ABCD$ tromi úsečkami vychádzajúcimi z bodu $N$
na štyri časti s~rovnakými obsahmi.
}
\podpis{Daniela Kovalčíková}

{%%%%%   Z6-I-4
Majme jedno trojciferné a~jedno dvojciferné číslo.
Trojciferné číslo zaokrúhlila na stovky a~dvojciferné zaokrúhlila na desiatky.
Rozdiel zaokrúhlených čísel bol 500.

Aký najmenší a~aký najväčší mohol byť rozdiel pôvodných nezaokrúhlených čísel?
}
\podpis{Marián Macko}

{%%%%%   Z6-I-5
Peťo má ku každému dňu v~týždni priradenú farbu:
pondelku modrú, utorku zelenú, strede bielu, štvrtku červenú, piatku oranžovú, sobote sivú a nedeli hnedú.
V týchto farbách nosí aj ponožky, a to tak, že na pravej nohe má ponožku farby dňa, avšak na ľavej nohe nemá ponožku farby tohto dňa, ani dňa nasledujúceho.
(Napr. v~sobotu má na pravej nohe sivú ponožku a~na ľavej nemá sivú, ani hnedú.)

Určte, ktorý je deň, ak mal Peťo na ľavej nohe predvčerom modrú ponožku, včera červenú a dnes má na nej hnedú.
}
\podpis{Monika Dillingerová}

{%%%%%   Z6-I-6
Obdĺžnikový park má obvod 228 metrov.
Vo vrcholoch obdĺžnika a na jeho stranách rástlo 38 okrasných kríkov tak, že vzdialenosti medzi každými dvoma susednými kríkmi boli rovnaké.
Na dvoch protiľahlých stranách obdĺžnika zasadil záhradník medzi každé dva kríky jeden ďalší.
Tým zvýšil počet kríkov po obvode parku na 60.

Určte rozmery parku.
}
\podpis{Marián Macko}

{%%%%%   Z7-I-1
Koľko štvorciferných čísel má tú vlastnosť, že ich tretina, polovica, dvojnásobok a~trojnásobok sú tiež štvorciferné čísla?
}
\podpis{Marián Macko}

{%%%%%   Z7-I-2
V~pravidelnom šesťuholníku $ABCDEF$ je bod $G$ stredom uhlopriečky $AE$.

Určte pomer obsahov trojuholníka $ADG$ a~šesťuholníka $ABCDEF$.
}
\podpis{Eva Semerádová}

{%%%%%   Z7-I-3
Pán Komický, Elegantný a Vážny sa poznajú z golfu.
Jeden sa volá Karol, jeden Erik a jeden Viktor.
Jeden nosí kravatu krémovej farby, jeden ebenovej farby a~jeden vínovej farby.
\begin{itemize}
  \item Výherca posledného zápasu nosí kravatu ebenovej farby.
  \item Pán Elegantný nebol nikdy na návšteve u pána Komického.
  \item Viktor nosí krémovú kravatu.
  \item Pánovi Komickému pripadá vtipné, že nikdy nevyhral.
  \item Karol bol po poslednom stretnutí na návšteve u pána Komického.
  \item Pán Vážny nosí kravatu vínovej farby.
\end{itemize}
Zistite, aké je vlastné meno každého z pánov a kto nosí akú kravatu.
}
\podpis{Erika Novotná}

{%%%%%   Z7-I-4
Adela, Beáta, Šárka a Julka si natrhali čerešne.
Beáta ich mala päťkrát viac ako Adela, Šárka mala o~15 čerešní viac ako Beáta, Julka mala o~200 viac ako Adela.

O~koľko najmenej sa mohli líšiť počty čerešní Šárky a~Julky?
A~koľko čerešní by v~takom prípade malo každé z~dievčat?
}
\podpis{Karel Pazourek}

{%%%%%   Z7-I-5
Václav mal niekoľko bielych kociek.
Na každej kocke zafarbil tri rôzne steny troma rôznymi farbami, a to červenou, zelenou a modrou.
Potom roztriedil kocky do skupín podľa typu zafarbenia tak, že všetky kocky v~jednej skupine vyzerali po vhodnom otočení rovnako.

Koľko najviac skupín mohol Václav vytvoriť?
}
\podpis{Iveta Jančigová}

{%%%%%   Z7-I-6
Pre štvoruholník $ABCD$ platí:
\begin{itemize}
  \item strana $AD$ a~uhlopriečka $BD$ sú rovnako dlhé,
  \item uhlopriečka $BD$ a~strana $DC$ sú kolmé,
  \item strany $AB$ a~$BC$ sú kolmé,
  \item os uhla $BDC$ a~strana $AD$ sú kolmé.
\end{itemize}

Určte veľkosť uhla $BCD$.
}
\podpis{Marián Macko}

{%%%%%   Z8-I-1
Nájdite všetky trojice navzájom rôznych prvočísel $p$, $q$, $r$, pre ktoré platí
$$
(p-q)\cdot(r-q)=195.
$$
}
\podpis{Erika Novotná}

{%%%%%   Z8-I-2
Pre rovnobežníky $ABCD$ a~$KLMN$ platí:
\begin{itemize}
  \item bod $K$ je stredom úsečky $CD$,
  \item bod $K$ je priesečníkom priamky $CD$ s~osou úsečky $BC$,
  \item bod $L$ je priesečníkom priamky $AB$ s~osou úsečky $CD$,
  \item bod $N$ je priesečníkom priamky $AB$ s~osou úsečky $BC$,
  \item uhol $BAD$ má veľkosť $60\st$.
\end{itemize}

Určte pomer obsahov rovnobežníkov $ABCD$ a~$KLMN$.
}
\podpis{Marián Macko}

{%%%%%   Z8-I-3
Tomáš zbiera pohľadnice z Islandu, Anglicka a Nórska.
Z každej krajiny má aspoň jednu pohľadnicu, celkom ich má 40.
Pohľadníc z~Anglicka má viac ako pohľadníc z~Nórska.
Pohľadníc z~Islandu má viac ako päťnásobok a menej ako šesťnásobok počtu pohľadníc z~Anglicka.

Z ktorých krajín sú pohľadnice, ktorých počet v Tomášovej zbierke je možné určiť jednoznačne?
}
\podpis{Erika Novotná}

{%%%%%   Z8-I-4
Žiaci získali v prvej písomke priemerne 84 bodov.
Tí istí žiaci napísali druhú písomku s priemerným ziskom 70 bodov.
Štyria z týchto žiakov mali v oboch písomkách po 63 bodov.
Priemerný zisk ostatných žiakov v druhej písomke bol 72 bodov.

Určte priemerné hodnotenie ostatných žiakov v~prvej písomke.
}
\podpis{Iveta Jančigová}

{%%%%%   Z8-I-5
Je daná kružnica $k$ so stredom $S$ a~polomerom 6\,cm
a~priamka $p$ prechádzajúca bodom~$S$.
Zostrojte obdĺžnik $ABCD$ tak, aby platilo:
\begin{itemize}
  \item vrcholy $A$ a~$B$ ležia na priamke $p$,
  \item kružnica $k$ sa dotýka strany $CD$,
  \item kružnica $k$ pretína stranu $AD$ v~bode $K$ a~stranu $BC$ v~bode $L$,
  \item $|AK|=|CL|=1{,}5$\,cm.
\end{itemize}
}
\podpis{Michaela Petrová}

{%%%%%   Z8-I-6
Jonáš a Michal zostavili každý svoj osemboký ihlan s deviatimi rôznymi číslami na jeho rôznych stenách.
Všetky čísla boli väčšie ako 10 a menšie ako 30.
Pre každý vrchol ihlana platilo, že súčet čísel na všetkých stenách obsahujúcich tento vrchol bol násobkom štyroch, pritom žiadne číslo násobkom štyroch nebolo.
Jonáš tvrdil, že na dvoch stenách má čísla 14 a~15.
Michal tvrdil, že na dvoch stenách má čísla 14 a~17.

Mohli obidvaja chlapci hovoriť pravdu?
}
\podpis{Karel Pazourek}

{%%%%%   Z9-I-1
V rohových políčkach tabuľky $3\times3$ sú napísané čísla tak, ako na obrázku:
\Image{75-z9-i-1}
Do prázdnych políčok doplňte kladné prirodzené čísla tak, aby súčin čísel vo všetkých riadkoch a~stĺpcoch bol rovnaký.
Nájdite všetky možnosti.
}
\podpis{Jaroslav Zhouf}

{%%%%%   Z9-I-2
Útvar na obrázku je vytvorený z piatich zhodných azúrových kosoštvorcov a jedného žltého pravidelného päťuholníka, ktorý kosoštvorce čiastočne prekrýva.
Kosoštvorce susedia celými stranami a na týchto stranách ležia vrcholy päťuholníka.
Pomer veľkostí polomeru kružnice opísanej päťuholníku a~strany kosoštvorca je $4:7$.

Rozhodnite, či neprekrytá časť každého kosoštvorca má väčší, rovnaký alebo menší obsah ako päťuholník.
\Image{75-z9-i-2}%
}
\podpis{Mária Dományová}

{%%%%%   Z9-I-3
Na tabuli je napísaných niekoľko po sebe nasledujúcich prirodzených čísel počnúc jednotkou.
Každé z týchto čísel má buď azúrovú, alebo žltú farbu.
Súčet každých dvoch rôznofarebných čísel je prvočíslo,
súčet každých dvoch čísel rovnakej farby je zložené číslo.

Koľko najviac čísel môže byť napísaných na tabuli?
}
\podpis{Patrik Bak}

{%%%%%   Z9-I-4
Prirodzené kladné číslo sa nazýva {\it kvadrátové\/}, ak obsahuje cifru alebo skupinu po sebe idúcich cifier, ktorá je druhou mocninou prirodzeného kladného čísla.
(Napr. trojciferné čísla 257 a~725 sú kvadrátové, čísla 275 a~572 nie.)

Určte počet všetkých dvojciferných kvadrátových čísel.
}
\podpis{Monika Dillingerová}

{%%%%%   Z9-I-5
V~lyžiarskom oddiele sa počet všetkých detí znížil o~10\,\%, pričom pomer dievčat voči všetkým deťom vzrástol z~50\,\% na 55\,\%.

O~koľko percent sa zmenil počet dievčat?
}
\podpis{Iveta Jančigová}

{%%%%%   Z9-I-6
Kružnice $k$ a~$l$ sa dotýkajú zvonku, pričom polomer kružnice $k$ je rovnaký ako priemer kružnice $l$.
Bod $S$ je stredom kružnice $k$,
bod $T$ je bodom dotyku kružníc,
bod $A$ leží na kružnici $l$ mimo spojnice stredov kružníc
a~bod $M$ je stredom úsečky $AS$.

Dokážte, že uhol $ATM$ je pravý.
}
\podpis{Lukáš Komín}


{%%%%%   Z4-II-1
...}
\podpis{...}

{%%%%%   Z4-II-2
...}
\podpis{...}

{%%%%%   Z4-II-3
...}
\podpis{...}

{%%%%%   Z5-II-1
...}
\podpis{...}

{%%%%%   Z5-II-2
...}
\podpis{...}

{%%%%%   Z5-II-3
...}
\podpis{...}

{%%%%%   Z6-II-1
...}
\podpis{...}

{%%%%%   Z6-II-2
...}
\podpis{...}

{%%%%%   Z6-II-3
...}
\podpis{...}

{%%%%%   Z7-II-1
...}
\podpis{...}

{%%%%%   Z7-II-2
...}
\podpis{...}

{%%%%%   Z7-II-3
...}
\podpis{...}

{%%%%%   Z8-II-1
...}
\podpis{...}

{%%%%%   Z8-II-2
...}
\podpis{...}

{%%%%%   Z8-II-3
...}
\podpis{...}

{%%%%%   Z9-II-1
...}
\podpis{...}

{%%%%%   Z9-II-2
...}
\podpis{...}

{%%%%%   Z9-II-3
...}
\podpis{...}

{%%%%%   Z9-II-4
...}
\podpis{...}

{%%%%%   Z9-III-1
...}
\podpis{...}

{%%%%%   Z9-III-2
...}
\podpis{...}

{%%%%%   Z9-III-3
...}
\podpis{...}

{%%%%%   Z9-III-4
...}
\podpis{...}

