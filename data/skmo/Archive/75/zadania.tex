{%%%%% A-I-1
Ostrov je rozdelený na niekoľko kráľovstiev. Územie každého kráľovstva je konvexný mnohouholník, ktorý má práve jeden najsevernejší, najjužnejší, najvýchodnejší a~najzápadnejší bod. Každý kráľ dostal 4 vlajky s písmenami S, J, V, Z, ktoré umiestnil do týchto významných bodov svojho kráľovstva. (Napríklad v mieste stretnutia troch štátov na obrázku sú takto zapichnuté 2~vlajky.) Uprostred ostrova je krtinec, kde sa stretáva 7 kráľovstiev. Určte všetky možné počty vlajok zapichnutých do krtinca. \inspdf{a75i_10.pdf}%
}
\podpis{Josef Tkadlec}

{%%%%% A-I-2
Nech $p$, $q$ sú reálne čísla také, že rovnici
$$
|x^2-1|=px+q
$$
s neznámou $x$ vyhovujú práve 4~navzájom rôzne reálne čísla.
\itemitem{a)} Určte všetky možné hodnoty súčtu týchto 4 čísel.
\itemitem{b)} Dokážte, že súčin týchto 4 čísel leží v intervale $(-3,1)$.
}
\podpis{Patrik Bak}

{%%%%% A-I-3
Jozef má hlavolam, ktorý sa skladá z troch vodorovných tyčí a z~$n\ge2$ rôzne veľkých kotúčov zoradených na prvej tyči podľa veľkosti, poz. obrázok. V jednom ťahu Jozef vysunie z~ľubovoľnej tyče krajný kotúč (zľava alebo sprava) a nasunie ho na inú tyč z tej istej strany. Hlavolam je vyriešený, keď sa všetky kotúče nachádzajú na druhej tyči zoradené rovnako ako na začiatku. V závislosti od $n$ určte najmenší možný počet ťahov, ktorý je potrebný na vyriešenie hlavolamu.
\inspdf{a75i_30.pdf}%
}
\podpis{ Jozef Rajník}

{%%%%% A-I-4
Sú dané nesúdeliteľné prirodzené čísla $a$ a $b$ také, že aj čísla $a^3-1$ a $b^3-1$ sú nesúdeliteľné. Dokážte, že čísla $a^2-b$ a $b^2-a$ sú tiež nesúdeliteľné.
}
\podpis{Patrik Bak}

{%%%%% A-I-5
Vnútri ostrouhlého trojuholníka $ABC$ je daný bod $R$. Na stranách $AB$ a $BC$ ležia postupne body $P$ a $Q$ tak, že obvod trojuholníka $PQR$ je najmenší možný. Podobne na stranách $BC$ a $AC$ ležia postupne body $S$ a $T$ tak, že obvod trojuholníka $RST$ je najmenší možný. Priamky $PQ$ a $ST$ sa pretínajú v~bode~$K$. Dokážte, že ak platí $|\angle BAK| = |\angle CAR|$, tak trojuholníky $PQR$ a $RST$ majú rovnaký obvod.
}
\podpis{Michal Pecho}

{%%%%% A-I-6
Na tabuli sú napísané 4 prirodzené čísla. Žirafa vykonáva nasledujúce kroky: zakaždým si vyberie jedno z čísel na tabuli, zotrie ho a namiesto neho napíše jeho druhú mocninu. Môže vždy žirafa po konečne veľa krokoch dosiahnuť to, aby rozdiel niektorých dvoch čísel na tabuli bol násobkom $97$?
}
\podpis{Josef Tkadlec}

{%%%%% B-I-1
Každej hrane štvorstena priradíme jedno reálne číslo tak, aby každá stena mala rovnaký súčet čísel svojich troch hrán. Koľko najviac zo šiestich čísel priradených hranám môže byť navzájom rôznych?
}
\podpis{Mária Dományová}

{%%%%% B-I-2
Zápis prirodzeného čísla v desiatkovej sústave končí dvojčíslím $90$. Dokážte, že súčin všetkých jeho kladných deliteľov je druhou mocninou prirodzeného čísla.
}
\podpis{ Ján Mazák}

{%%%%% B-I-3
Na tabuli je nakreslená kružnica (bez stredu) a na nej tri rôzne body $A$, $B$, $C$. Máme k dispozícii kriedu a trojuholník s ryskou bez mierky. Ten nám umožňuje len viesť priamku ľubovoľnými dvoma bodmi a~na danú priamku~$p$ viesť kolmicu daným bodom (nie nutne ležiacim na $p$). Zostrojte stred kružnice vpísanej trojuholníku $ABC$.
}
\podpis{Ema Čudaiová}

{%%%%% B-I-4
Reálne čísla $x$, $y$ spĺňajú nerovnosti $xy \ge x+y > 0$. Akú najmenšiu hodnotu môže nadobúdať súčet $x+y$?
}
\podpis{Patrik Bak}

{%%%%% B-I-5
Je daný štvoruholník $ABCD$ taký, že $|\angle ABC|=|\angle ADC|$ a $|\angle BCD|=3|\angle BAD|$. Body~$P$ a~$Q$ ležia postupne na úsečkách $AB$ a~$AD$ tak, že $APCQ$ je rovnobežník. Nech~$O$ je stred kružnice opísanej trojuholníku~$CPQ$. Dokážte, že $AO \perp BD$.
}
\podpis{Patrik Bak}

{%%%%% B-I-6
Hracia plocha na obrázku vľavo sa skladá zo 61 pravidelných šesťuholníkov so stranou dĺžky~$1$. Na každom políčku môže stáť najviac jeden jazdec. Dvaja jazdci sa ohrozujú práve vtedy, keď stoja na políčkach so stredmi vzdialenými presne $3$ (políčka ohrozené jazdcom sú na obrázku vpravo vyfarbené). Koľko najviac jazdcov môžeme umiestniť na túto plochu tak, aby sa žiadni dvaja neohrozovali?
\inspdf{b75i_603.pdf}%
}
\podpis{Jozef Rajník}

{%%%%% C-I-1
Prirodzené číslo zapísané navzájom rôznymi ciframi nazveme {\it pitoreskné}, keď každá jeho vnútorná cifra delí dvojciferné číslo tvorené zľava doprava jej susedmi. Napríklad 1324 je pitoreskné, pretože 3 delí 12 a 2 delí 34. Rozhodnite, či existuje pitoreskné číslo tvorené všetkými ciframi a) $1$ až $8$, b) $1$ až $9$.
}
\podpis{Patrik Bak}

{%%%%% C-I-2
Uvažujme prirodzené čísla $a$, $b$, $c$ také, že čísla $a$, $b$, $c$, $a+b$, $b+c$, $c+a$ sú navzájom rôzne a súčet troch najväčších z nich je 75. Určte najväčšiu možnú hodnotu súčtu $a+b+c$.
}
\podpis{Patrik Bak}

{%%%%% C-I-3
V sieti tvorenej štvorcami so stranou~1 je daná kružnica so stredom v~mrežovom bode a s~polomerom~2. Táto kružnica pretína priamky siete dokopy v 12 bodoch. Dokážte, že týchto 12 priesečníkov tvorí vrcholy pravidelného 12-uholníka.
\insp{c75i.30}%
}
\podpis{Josef Tkadlec}

{%%%%% C-I-4
Nájdite všetky štvorciferné čísla $n=\overline{abcd}$, pre ktoré platí
$$
\overline{ab}+\overline{cd}=\sqrt{n}, \qquad \overline{cd}-\overline{ab}=5.
$$
}
\podpis{Mária Dományová}

{%%%%% C-I-5
Uhlopriečky lichobežníka $ABCD$ sa pretínajú v~bode $P$ a ich osi sa pretínajú v~bode~$M$. Predpokladajme, že $M$ leží na základni $AB$. Dokážte, že $P$ je stredom kružnice vpísanej trojuholníku $CDM$.
\inspsc{c75i.50}{.8333}%
}
\podpis{Jaroslav Švrček}

{%%%%% C-I-6
Na pekáči je 21 buchiet, 10 z nich je plnených lekvárom, zvyšných 11 tvarohom. Môžeme položiť 10 otázok. V každej otázke ukážeme na dve buchty a kuchár nám povie, či majú rovnakú náplň, alebo každá inú. Je možné pýtať sa tak, aby sme o aspoň jednej buchte s~istotou zistili, čím je plnená?
}
\podpis{Josef Tkadlec, Felix Schröder}

{%%%%%   A-S-1
...}
\podpis{...}

{%%%%%   A-S-2
...}
\podpis{...}

{%%%%%   A-S-3
...}
\podpis{...}

{%%%%%   A-II-1
...}
\podpis{...}

{%%%%%   A-II-2
...}
\podpis{...}

{%%%%%   A-II-3
...}
\podpis{...}

{%%%%%   A-II-4
...}
\podpis{...}

{%%%%%   A-III-1
...}
\podpis{...}

{%%%%%   A-III-2
...}
\podpis{...}

{%%%%%   A-III-3
...}
\podpis{...}

{%%%%%   A-III-4
...}
\podpis{...}

{%%%%%   A-III-5
...}
\podpis{...}

{%%%%%   A-III-6
...}
\podpis{...}

{%%%%%   B-S-1
...}
\podpis{...}

{%%%%%   B-S-2
...}
\podpis{...}

{%%%%%   B-S-3
...}
\podpis{...}

{%%%%%   B-II-1
...}
\podpis{...}

{%%%%%   B-II-2
...}
\podpis{...}

{%%%%%   B-II-3
...}
\podpis{...}

{%%%%%   B-II-4
...}
\podpis{...}

{%%%%%   C-S-1
...}
\podpis{...}

{%%%%%   C-S-2
...}
\podpis{...}

{%%%%%   C-S-3
...}
\podpis{...}

{%%%%%   C-II-1
...}
\podpis{...}

{%%%%%   C-II-2
...}
\podpis{...}

{%%%%%   C-II-3
...}
\podpis{...}

{%%%%%   C-II-4
...}
\podpis{...}

{%%%%%   vyberko, den 1, priklad 1
...}
\podpis{...}

{%%%%%   vyberko, den 1, priklad 2
...}
\podpis{...}

{%%%%%   vyberko, den 1, priklad 3
...}
\podpis{...}

{%%%%%   vyberko, den 1, priklad 4
...}
\podpis{...}

{%%%%%   vyberko, den 2, priklad 1
...}
\podpis{...}

{%%%%%   vyberko, den 2, priklad 2
...}
\podpis{...}

{%%%%%   vyberko, den 2, priklad 3
...}
\podpis{...}

{%%%%%   vyberko, den 2, priklad 4
...}
\podpis{...}

{%%%%%   vyberko, den 3, priklad 1
...}
\podpis{...}

{%%%%%   vyberko, den 3, priklad 2
...}
\podpis{...}

{%%%%%   vyberko, den 3, priklad 3
...}
\podpis{...}

{%%%%%   vyberko, den 3, priklad 4
...}
\podpis{...}

{%%%%%   vyberko, den 4, priklad 1
...}
\podpis{...}

{%%%%%   vyberko, den 4, priklad 2
...}
\podpis{...}

{%%%%%   vyberko, den 4, priklad 3
...}
\podpis{...}

{%%%%%   vyberko, den 4, priklad 4
...}
\podpis{...}

{%%%%%   vyberko, den 5, priklad 1
...}
\podpis{...}

{%%%%%   vyberko, den 5, priklad 2
...}
\podpis{...}

{%%%%%   vyberko, den 5, priklad 3
...}
\podpis{...}

{%%%%%   vyberko, den 5, priklad 4
...}
\podpis{...}

{%%%%%   trojstretnutie, priklad 1
...}
\podpis{...}

{%%%%%   trojstretnutie, priklad 2
...}
\podpis{...}

{%%%%%   trojstretnutie, priklad 3
...}
\podpis{...}

{%%%%%   trojstretnutie, priklad 4
...}
\podpis{...}

{%%%%%   trojstretnutie, priklad 5
...}
\podpis{...}

{%%%%%   trojstretnutie, priklad 6
...}
\podpis{...}

{%%%%%   IMO, priklad 1
...}
\podpis{...}

{%%%%%   IMO, priklad 2
...}
\podpis{...}

{%%%%%   IMO, priklad 3
...}
\podpis{...}

{%%%%%   IMO, priklad 4
...}
\podpis{...}

{%%%%%   IMO, priklad 5
...}
\podpis{...}

{%%%%%   IMO, priklad 6
...}
\podpis{...}

{%%%%%   MEMO, priklad 1
}
\podpis{}

{%%%%%   MEMO, priklad 2
}
\podpis{}

{%%%%%   MEMO, priklad 3
}
\podpis{}

{%%%%%   MEMO, priklad 4
}
\podpis{}

{%%%%%   MEMO, priklad t1
}
\podpis{}

{%%%%%   MEMO, priklad t2
}
\podpis{}

{%%%%%   MEMO, priklad t3
}
\podpis{}

{%%%%%   MEMO, priklad t4
}
\podpis{}

{%%%%%   MEMO, priklad t5
}
\podpis{}

{%%%%%   MEMO, priklad t6
}
\podpis{}

{%%%%%   MEMO, priklad t7
}
\podpis{}

{%%%%%   MEMO, priklad t8
}
\podpis{}

{%%%%%   CPSJ, priklad 1
...}
\podpis{...}

{%%%%%   CPSJ, priklad 2
...}
\podpis{...}

{%%%%%   CPSJ, priklad 3
...}
\podpis{...}

{%%%%%   CPSJ, priklad 4
...}
\podpis{...}

{%%%%%   CPSJ, priklad 5
...}
\podpis{...}

{%%%%%   CPSJ, priklad t1
...}
\podpis{...}

{%%%%%   CPSJ, priklad t2
...}
\podpis{...}

{%%%%%   CPSJ, priklad t3
...}
\podpis{...}

{%%%%%   CPSJ, priklad t4
...}
\podpis{...}

{%%%%%   CPSJ, priklad t5
...}
\podpis{...}

{%%%%%   CPSJ, priklad t6
...}
\podpis{...}

{%%%%%   EGMO, priklad 1
...}
\podpis{...}

{%%%%%   EGMO, priklad 2
...}
\podpis{...}

{%%%%%   EGMO, priklad 3
...}
\podpis{...}

{%%%%%   EGMO, priklad 4
...}
\podpis{...}

{%%%%%   EGMO, priklad 5
...}
\podpis{...}

{%%%%%   EGMO, priklad 6
...}
\podpis{...}

{%%%%%   vyberko C, den 1, priklad 1
...}
\podpis{...}

{%%%%%   vyberko C, den 1, priklad 2
...}
\podpis{...}

{%%%%%   vyberko C, den 1, priklad 3
...}
\podpis{...}

{%%%%%   vyberko C, den 1, priklad 4
...}
\podpis{...}

{%%%%%   vyberko C, den 1, priklad 5
...}
\podpis{...} 