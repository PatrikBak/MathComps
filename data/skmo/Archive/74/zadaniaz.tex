{%%%%%   Z4-I-1
...}
\podpis{...}

{%%%%%   Z4-I-2
...}
\podpis{...}

{%%%%%   Z4-I-3
...}
\podpis{...}

{%%%%%   Z4-I-4
...}
\podpis{...}

{%%%%%   Z4-I-5
...}
\podpis{...}

{%%%%%   Z4-I-6
...}
\podpis{...}

{%%%%%   Z5-I-1
Na našej ulici býva rodina Čapkovcov a Nemcovcov.
Čapkovci majú dvoch synov, Karola a~o~dva roky staršieho Petra.
Nemcovci majú dcéru Boženu.
Narodeniny všetkých troch detí oslavujú obe rodiny spoločne, a to v~deň Karolových narodenín.
Pri tohtoročnej oslave bola Božena trikrát staršia ako Karol.
Za tri roky budú mať Karol a Peter spolu toľko rokov ako Božena.

Koľko rokov mali deti pri tohtoročnej oslave?
}
\podpis{Michaela Petrová}

{%%%%%   Z5-I-2
Na obrázku je sivý štvorec so stranou dĺžky 10\,cm.
Štvorec dopĺňajú štyri rovnaké pravouhlé trojuholníky do tvaru hviezdy.
Súčet obsahov týchto štyroch trojuholníkov je štvornásobkom obsahu štvorca.

Určite dĺžku strany $XY$.
\insp{z5-i-2.eps}%

\poznamka
Obrázok je iba ilustračný.
}
\podpis{Eva Semerádová}

{%%%%%   Z5-I-3
V~nasledujúcom príklade je päťkrát použité znamienko $+$ a~výsledok je násobkom troch:
$$
9+8+7+6+5+4 =39.
$$
Zmeňte dve zo znamienok $+$ na znamienko $-$ tak, aby výsledok nového príkladu bol opäť násobkom troch.
Nájdite všetky možnosti.
}
\podpis{Eva Semerádová}

{%%%%%   Z5-I-4
Pinocchio tvrdí, že číslo dňa v~dátume jeho narodenia možno bezo zvyšku deliť tromi, štyrmi, piatimi a~šiestimi.
Tri z týchto štyroch informácií sú pravdivé, jedna je nepravdivá.

Koľký deň v mesiaci môže mať Pinocchio narodeniny?
Nájdite všetky možnosti.
}
\podpis{Erika Novotná}

{%%%%%   Z5-I-5
V sieti chodníkov vyznačených na obrázku má každý chodník medzi susednými križovatkami dĺžku 1\,km.

Koľko ciest dlhých nanajvýš 3\,km vedie po chodníkoch z~miesta $A$ do miesta $B$?
\insp{z5-i-5.eps}%
}
\podpis{Eva Semerádová}

{%%%%%   Z5-I-6
Anička navlieka na niť bezprostredne za seba koráliky troch rôznych tvarov $A$, $B$, $C$.
Postupuje tak, že tvary strieda stále v rovnakom poradí a postupne zvyšuje počty tvarov v skupinách:
$$
ABCAABBCCAAABBBCCCAAAABBBBCCCC\dots
$$
Korálik tvaru $A$ zaberá 5\,mm nite, korálik tvaru $B$ zaberá 4\,mm, korálik tvaru $C$ zaberá 3\,mm.

Koľko korálikov potrebuje Anička na výrobu náhrdelníka dlhého aspoň 50\,cm?
}
\podpis{Lenka Dedková}

{%%%%%   Z6-I-1
Pán Vaflička vysmáža a predáva šišky, pán Šiška pečie a predáva vafle.
Obaja cukrári majú každý týždeň otvorené od pondelka do piatka.
Lenka u~nich kupuje každý pondelok dve vafle a~jeden šišku, každý utorok tri šišky a~jednu vafľu, každú stredu štyri šišky, každý štvrtok tri vafle a~každý piatok dve šišky a~dve vafle.
Pán Šiška si raz všimol, že od prvého pondelka tohto mesiaca predal Lenke dokopy 30 vaflí.

Koľko šišiek predal Lenke za rovnaké obdobie pán Vaflička?
}
\podpis{Michaela Petrová}

{%%%%%   Z6-I-2
V~obdĺžniku so stranami dĺžok 4\,cm a~8\,cm sú dané polkružnice $k$ a~$l$, ktorých krajné body ležia vo vrcholoch obdĺžnika.

Zostrojte štvorec $ABCD$ tak, aby vrcholy $A$ a~$B$ ležali na polkružnici $k$, vrcholy $C$ a~$D$ ležali na polkružnici $l$ a~strany štvorca boli rovnobežné so stranami obdĺžnika.
\insp{z6-i-2.eps}%
}
\podpis{Karel Pazourek}

{%%%%%   Z6-I-3
Päťciferný palindrómom je také päťciferné číslo, ktoré má na mieste jednotiek rovnakú cifru ako na mieste desaťtisícok a na mieste desiatok rovnakú cifru ako na mieste tisícok.

Nájdite najmenší päťciferný palindróm deliteľný číslom 36.
}
\podpis{Iveta Jančigová}

{%%%%%   Z6-I-4
Šárka s Ľubošom spoločne zasadili 70 tulipánov rôznych farieb.
Šárka nesadila žlté tulipány a päť devätín tulipánov, ktoré zasadila, boli červené.
Ľuboš nesadil červené tulipány a dve sedemnástiny tulipánov, ktoré zasadil, boli žlté.

Koľko zasadených tulipánov malo inú farbu ako červenú alebo žltú?
}
\podpis{Libuše Hozová}

{%%%%%   Z6-I-5
Tri kamarátky sa po rokoch zišli a rozprávali sa o tom, kde ktorá z nich býva:

\smallskip
\item{$\bullet$} Prvá: "Ja bývam v~Hruštíne."
\item{$\bullet$} Druhá: "Ja nebývam v~Očovej."
\item{$\bullet$} Tretia druhej: "Ty nebývaš ani v~Jasenove."

\smallskip
Kamarátky naozaj bývajú v spomínaných dedinách, každá v inej.
Jedna z kamarátok nepovedala pravdu a nebola to tá z Očovej.

Rozhodnite, kde ktorá z kamarátok býva.
}
\podpis{Michaela Petrová}

{%%%%%   Z6-I-6
V štvorcovej sieti bývajú tri kruhy a tri trojuholníky, každý v inom políčku.
Každý tvar má aspoň jedného suseda, pričom susedia obývajú políčka so spoločnou stranou.
Obývané políčka tvoria súvislú oblasť, teda od každého ku každému sa dá dostať cez susedov.
Každú noc sa každý tvar môže zmeniť podľa toho, ako cez deň vyzerali jeho susedia:
\begin{itemize}
  \item pokiaľ je tvar kruhom a~medzi jeho susedmi bolo viac trojuholníkov ako kruhov, tak sa tvar zmení na trojuholník,
  \item pokiaľ je tvar trojuholníkom a~medzi jeho susedmi bolo viac kruhov ako trojuholníkov, tak sa tvar zmení na kruh,
  \item v~ostatných prípadoch sa tvar nezmení.
\end{itemize}
\noindent
Príklad obývanej štvorcovej siete a~premeny po jednej noci je na obrázku nižšie.
Navrhnite, ako rozmiestniť tri kruhy a tri trojuholníky do siete tak,
\begin{enumerate}\alphatrue
  \item aby sa v~noci nezmenili;
  \item aby sa každý tvar každú noc zmenil;
  \item aby po niekoľkých nociach boli všetky tvary rovnaké.
\end{enumerate}
~\insp{z6-i-6.eps}%
}
\podpis{Iveta Jančigová}

{%%%%%   Z7-I-1
%\footnote[]{\today, ver. 3}
Alenka a Zuzka jedli slivky.
Prvý deň zjedla Alenka tri štvrtiny toho, čo v ten istý deň zjedla Zuzka.
Druhý deň zjedla Zuzka tri polovice toho, čo v ten istý deň zjedla Alenka.
Dokopy za oba dni zjedli 31 sliviek a každé dievča každý deň zjedlo celočíselný počet sliviek.

Koľko sliviek zjedla za oba dni Alenka?
}
\podpis{Libuše Hozová}

{%%%%%   Z7-I-2
Mikuláš postavil pyramídu zo šiestich rovnakých kociek s hranami dĺžky 7\,cm.
Spodné poschodie tvorili tri kocky, prostredné poschodie dve kocky a~horné poschodie jedna kocka.
Susedné kocky v~každom poschodí mali spoločnú stenu, poschodia navzájom neprečnievali.
Víťazoslav posunul kocku tak, že každá kocka v horných dvoch poschodiach stála na dvoch spodných kockách a medzi susednými kockami v spodných dvoch poschodiach boli medzery široké tretinu hrany kocky.
Až na tieto medzery poschodia navzájom neprečnievali.

O~koľko cm$^2$ sa líšia povrchy pôvodnej a~upravenej pyramídy?
\insp{z7-i-2.eps}%
}
\podpis{Vladimír Dedek}

{%%%%%   Z7-I-3
Pankrác, Servác a Bonifác sa ubytovali v hoteli.
Čísla izieb boli trojciferné a~cifra na mieste stoviek určovala poschodie, na ktorom sa izba nachádzala.
Na raňajkách si podľa príveskov na kľúčoch od izieb všimli, že:
\begin{itemize}
  \item v~číslach ich izieb sú použité všetky cifry od 1 do 9,
  \item Pankrácovo číslo je deliteľné deviatimi, Servácovo číslo je deliteľné ôsmimi, Bonifácovo číslo je deliteľné siedmimi,
  \item Bonifácovo číslo je štyrikrát väčšie ako Pankrácovo číslo,
  \item Servác býva na poschodí medzi Pankrácom a~Bonifácom.
\end{itemize}
Určte čísla izieb Pankráca, Serváca a~Bonifáca.
}
\podpis{Libuše Hozová, Erika Novotná}

{%%%%%   Z7-I-4
V~jednej z piatich nádob očíslovaných 1, 2, 3, 4, 5 je minca.
Sprievodné nápisy oznamujú:
\smallskip
\item{$\bullet$} "Minca je v~nádobe s~nepárnym číslom."
\item{$\bullet$} "Minca je v~nádobe s~číslom väčším ako 3."
\item{$\bullet$} "Minca je v~nádobe s~číslom menším ako 4."

\smallskip
Čestný strážca s bezchybným úsudkom dodáva:
\smallskip
"Jeden z nápisov nie je pravdivý, zvyšné dva pravdivé sú.
Hoci viem, ktorý nápis pravdivý nie je, neviem určiť, v ktorej nádobe je minca."

\nopagebreak\smallskip
Rozhodnite, ktorý z~nápisov nie je pravdivý.
}
\podpis{Karel Pazourek}

{%%%%%   Z7-I-5
Je daný trojuholník $ABC$ s~dĺžkami strán $|AB|=6\cm$, $|BC|=8\cm$ a~$|AC|=12\cm$.

Zostrojte polkružnicu, ktorej krajné body ležia na strane $AC$ a~ktorá sa dotýka strán $AB$ a~$BC$.
}
\podpis{Karel Pazourek}

{%%%%%   Z7-I-6
Katka a~Števo pečú každý na svojej panvici bez prestávok jednu palacinku za druhou a hotové palacinky dávajú na spoločný tanier.
Obaja začali piecť súčasne. Katke trvá každá palacinka tri minúty, Števovi trvá každá palacinka štyri minúty.
Každých päť minút od začiatku pečenia sa objaví maškrtný kocúr Lucifer.
Ak sa Katka i Števo venujú pečeniu, tak im Lucifer jednu hotovú palacinku ukradne. Pokiaľ niekto z nich akurát dáva palacinku z panvice na tanier, tak sa schová a palacinku neukradne.

Koľko palaciniek musia Katka so Števom upiecť, aby im ich ostalo 150?
Ako dlho im to bude trvať?
}
\podpis{Michaela Petrová}

{%%%%%   Z8-I-1
%\footnote[]{\today, ver. 3}
Ivan, Jaro, Karol a Ľuboš majú dokopy 90 známok.
Keby mal Ivan o dve známky menej, Jaro o dve viac, Karol dvojnásobok a Ľuboš polovicu toho, čo teraz, mali by všetci rovnako.

Koľko známok má každý z chlapcov?
}
\podpis{Libuše Hozová}

{%%%%%   Z8-I-2
Zostrojte rovnoramenný trojuholník so základňou dĺžky 12\,cm a výškou na základňu dĺžky 18\,cm.
Rozdeľte tento trojuholník na tri lichobežníky s rovnakým obsahom.
}
\podpis{Lenka Dedková}

{%%%%%   Z8-I-3
Pre čísla $a$, $b$, $c$, $d$ platí:
\begin{itemize}
  \item číslo $a$ dáva po delení tromi zvyšok 1,
  \item číslo $b$ dáva po delení šiestimi zvyšok 2,
  \item $a - b = d - c$,
  \item číslo $d$ je deliteľné tromi.
\end{itemize}
Aký zvyšok po delení deviatimi môže dávať číslo $c$?
Nájdite všetky možnosti.
}
\podpis{Eva Semerádová}

{%%%%%   Z8-I-4
Je daný pravidelný päťuholník $ABCDE$.
Rovnobežka s~priamkou $AB$ prechádzajúca bodom $C$ pretína priamku $BD$ v~bode $F$.
Kolmica na priamku $CF$ prechádzajúca bodom $C$ pretína priamku $BD$ v~bode $G$.

Určte veľkosť uhla $AGF$.
\insp{z8-i-4.eps}%
}
\podpis{Patrik Bak}

{%%%%%   Z8-I-5
Podiel najmenšieho spoločného násobku a~najväčšieho spoločného deliteľa čísel $a$ a~$b$ je 75.
Súčet čísel $a$ a~$b$ je väčší ako 100 a~menší ako 200.

Určte všetky možné dvojice čísel $a$ a~$b$ s uvedenými vlastnosťami.
}
\podpis{Eva Semerádová}

{%%%%%   Z8-I-6
Rybár Šťuka chytil niekoľko rýb.
Keď predal tri najťažšie ryby majiteľovi miestnej reštaurácie, znížil celkovú hmotnosť svojho úlovku o~35\,\%.
Keď dal tri najľahšie ryby svojmu psovi, znížil hmotnosť zostávajúcich ulovených rýb o~päť trinástin.

Koľko rýb chytil pán Šťuka?
}
\podpis{Libuše Hozová}

{%%%%%   Z9-I-1
Nájdite všetky dvojice celých čísel $x$ a~$y$ takých, že $x+y$ je prvočíslo a~$3x+5y$ je~16.
}
\podpis{Patrik Bak}

{%%%%%   Z9-I-2
Pravidelný štvorboký hranol má objem $864\cm^3$ a obsah jeho plášťa je dvojnásobkom obsahu podstavy.

Určite veľkosť telesovej uhlopriečky hranola.
}
\podpis{Vladimír Dedek}

{%%%%%   Z9-I-3
Množinu $\{1, 2, 3, 4, \dots, n\}$ pozostávajúcu z~prvých $n$ prirodzených čísel máme za úlohu rozdeliť do piatich neprázdnych podmnožín tak, aby čísla v~každej podmnožine boli po dvoch nesúdeliteľné.

Nájdite najväčšie možné $n$, pre ktoré to je možné.
}
\podpis{Tomáš Bárta}

{%%%%%   Z9-I-4
Rozhodnite, či je možné k~číslu s~ciferným súčtom 2024 pripočítať jednociferné číslo tak, aby výsledné číslo malo ciferný súčet 74.
}
\podpis{Tomáš Bárta}

{%%%%%   Z9-I-5
V~trojuholníku $ABC$ je strana $AB$ dvakrát dlhšia ako strana $AC$.
Os uhla $BAC$ pretína stranu $BC$ v~bode $D$.
Rovnobežka so stranou $AB$ prechádzajúca bodom $D$ pretína stranu $AC$ v~bode $E$.
Rovnobežka s~úsečkou $AD$ prechádzajúca bodom $E$ pretína stranu $BC$ v~bode $F$.

Určte pomer dĺžok úsečiek $AD$ a~$EF$.
\insp{z9-i-5.eps}%

\poznamka
Obrázok je iba ilustračný.
}
\podpis{Mária Dományová}

{%%%%%   Z9-I-6
Plavci Pstruh a~Kapor si chceli zmerať svoje sily.
Z protiľahlých strán bazéna skočili súčasne do susedných dráh a plávali proti sebe, každý svojou konštantnou rýchlosťou.
Prvýkrát sa plavci minuli vo vzdialenosti osem metrov od Pstruhovej štartovacej strany, na konci dráhy sa rýchlo otočili a plávali naspäť.
Druhýkrát sa plavci minuli vo vzdialenosti päť metrov od Kaprovej štartovacej strany, doplávali na koniec dráhy, a tým preteky skončili.

Určite, kto vyhral a~aká bola dĺžka bazéna.
}
\podpis{Libuše Hozová}

{%%%%%   Z4-II-1
...}
\podpis{...}

{%%%%%   Z4-II-2
...}
\podpis{...}

{%%%%%   Z4-II-3
...}
\podpis{...}

{%%%%%   Z5-II-1
Janka a~Danka jedli počas týždňa ovocie. Janka jedla len hrušky alebo jablká, Danka jedla iba čerešne.
Každý deň zjedla Janka najviac jeden kus ovocia a Danka jedla v ten istý deň čerešne podľa nasledujúceho rozpisu:

\smallskip
\item{$\bullet$} Keď Janka zjedla hrušku, zjedla Danka dve čerešne.
\item{$\bullet$} Keď Janka zjedla jablko, zjedla Danka tri čerešne.
\item{$\bullet$} Keď Janka nezjedla žiadne ovocie, zjedla Danka šesť čerešní.

\smallskip\noindent
Od pondelka do nedele zjedla Danka dokopy 19 čerešní.

Koľko ktorého ovocia mohla zjesť za ten istý týždeň Janka? Nájdite obe možnosti.
}
\podpis{Erika Novotná}

{%%%%%   Z5-II-2
Okružná cesta spája tri dediny tak ako na obrázku.
Vo vyznačenom smere to je z~Pávoviec do Krtkovian 10\,km, z~Mňaukova do Pávoviec 15\,km a~z Krtkovian do Mňaukova 16\,km.

Aká dlhá je celá okružná cesta?
\insp{z5-ii-2.eps}%
}
\podpis{Eva Semerádová}

{%%%%%   Z5-II-3
Z~2025 rovnakých štvorcov je zložený útvar podľa pravidla naznačeného na obrázku.
Strana štvorca je 1\,cm.

Určte obvod útvaru.
\insp{z5-ii-3.eps}%
}
\podpis{Karel Pazourek}

{%%%%%   Z6-II-1
Prirodzené číslo nazveme \emph{pekné}, ak je väčšie ako $7000$ a súčin jeho číslic je $252$.

Nájdite dve najmenšie pekné čísla.
}
\podpis{Iveta Jančigová}

{%%%%%   Z6-II-2
Cestičky medzi úkrytmi cvrčkov Adama, Borisa, Cyrila, Daniela a~Erika tvoria sieť ako na obrázku
(kde úkryty sú označené prvými písmenami mien cvrčkov), pričom~platí:
\begin{itemize}
 \item Cestičky medzi úkrytmi Adama, Borisa a~Erika tvoria rovnostranný trojuholník.
 \item Cestičky medzi úkrytmi Borisa, Cyrila, Daniela a~Erika tvoria obdĺžnik s~obsahom 360\,dm$^2$.
 \item Prechádzka po cestičkách od Adama k~Borisovi, Cyrilovi, Danielovi, Erikovi a ~Adamovi je o~24\,dm dlhšia ako prechádzka od Adama k~Borisovi, Erikovi a~Adamovi.
\end{itemize}

\noindent
Aká dlhá je cestička medzi úkrytmi Adama a Erika?
\insp{z6-ii-2.eps}%

Poznámka: Obrázok je len ilustračný.}
\podpis{E. Novotná}

{%%%%%   Z6-II-3
Jakub prečítal knihu počas troch dní. V utorok prečítal tretinu všetkých strán, v stredu prečítal tri sedminy zvyšných strán a posledných 32 strán prečítal vo štvrtok.

Koľko strán mala kniha?
}
\podpis{Erika Novotná}

{%%%%%   Z7-II-1
Mamička si nachystala perníčky na zdobenie.
Každý perníček zdobí rovnako dlho.
Keby pri zdobení každého perníka bola o~jednu minútu rýchlejšia, potom by mohla skončiť o~48 minút skôr, alebo by v~takto ušetrenom čase mohla ozdobiť (v~novom zrýchlenom tempe) presne 12 ďalších perníčkov.

Koľko perníčkov si mamička nachystala a ako dlho jej bude trvať ich zdobenie
(v~pôvodnom nezrýchlenom tempe)?
}
\podpis{Michaela Petrová}

{%%%%%   Z7-II-2
V útulku je 60 zvierat, a to výhradne mačky a psy. Tretina mačiek a tri osminy psov nie sú ani rok staré, 39 zvierat má rok alebo viac.

Koľko je v~útulku mačiek a~koľko psov?
}
\podpis{Lenka Dedková}

{%%%%%   Z7-II-3
Kosoštvorec $ABCD$ je zložený z~rovnobežníkov s~navzájom rovnakými obsahmi.
Vyznačená spoločná strana $TU$ dvoch rovnobežníkov má dĺžku 2\,cm.

Určte obvod kosoštvorca $ABCD$.
\insp{z7-ii-3.eps}%
}
\podpis{Karel Pazourek}

{%%%%%   Z8-II-1
Kúzelníkov povraz je dlhší ako 10\,m. Nech by rozdelil povraz na dve časti v~hociktorom z~troch pomerov $3:5$, $7:11$, $13:17$, bola by dĺžka oboch častí vyjadrená v~centimetroch zakaždým celým číslom.

Aká je najmenšia možná dĺžka kúzelníkovho povrazu?
}
\podpis{V. Dedek}

{%%%%%   Z8-II-2
Monika si vybrala dve čísla, aby preskúšala schopnosti robota Popletu. Najprv mu dala sčítať obe čísla. Popletov výsledok bol o~4,1 menší ako výsledok pri správnom sčítaní.
Potom mu dala sčítať trojnásobok prvého čísla s druhým číslom. Teraz bol Popletov výsledok o 8,4 menší ako výsledok pri správnom ščítaní.
Čoskoro zistila, že Popleta nepočíta súčet dvoch zadaných čísel, ale správne vypočíta ich aritmetický priemer.

Ktoré čísla si Monika vybrala?
}
\podpis{Karel Pazourek}

{%%%%%   Z8-II-3
V~trojuholníku $ABC$ leží bod $D$ na strane $BC$, bod $E$ na strane $AC$ tak, že~platí
$$
|AB| = |BE| = |EC| = |CD|, \quad |BD| = |DE|.
$$

Určite veľkosti uhlov $ACB$ a~$BAD$.
}
\podpis{Patrik Bak}

{%%%%%   Z9-II-1
Nájdite všetky dvojciferné prirodzené čísla, ktoré majú nasledujúcu vlastnosť:
Keď pred číslo pripíšeme súčin jeho prvej cifry a~jeho prvej cifry zväčšenej o~$1$, dostaneme druhú mocninu pôvodného čísla.
}
\podpis{Karel Pazourek}

{%%%%%   Z9-II-2
Nájdite všetky trojice navzájom rôznych čísel takých, že:
\begin{itemize}
\item jedno z nich je aritmetickým priemerom zvyšných dvoch,
\item súčet aritmetického priemeru najväčšieho a stredného čísla a aritmetického priemeru stredného a najmenšieho čísla je $628$ a rozdiel týchto dvoch aritmetických priemerov je $83$.
\end{itemize}
Na poradí čísel v trojici nezáleží.
}
\podpis{Karel Pazourek}

{%%%%%   Z9-II-3
Včera vydojili na farme Doj dvakrát viac mlieka ako na farme Hoj a na farme Loj dvakrát viac mlieka ako na farme Doj.
Každá farma poslala časť vydojeného mlieka na výrobu masla.
Farma Doj poslala na výrobu masla $7/8$ svojho mlieka  a farma Hoj $3/4$ svojho mlieka.
Z mlieka vydojeného na všetkých troch farmách išlo na výrobu masla dokopy 90\,\%.

Akú časť svojho mlieka poslala na výrobu masla farma Loj?
}
\podpis{Michaela Petrová}

{%%%%%   Z9-II-4
Obdĺžnik $ABCD$ má obsah $82$\,cm$^2$.
Bod $E$ je stredom strany $CD$ a~bod $P$ je priesečníkom úsečiek $AC$ a~$BE$.

Určte obsah trojuholníka $ABP$.
}
\podpis{Erika Novotná}

{%%%%%   Z9-III-1
Do divadla dorazili diváci buď peši, autami alebo autobusmi.
Divákov, ktorí dorazili autobusmi, bolo viac ako $150$. Autobusov bolo šesť a~v~každom prišlo rovnaké množstvo divákov.
Divákov, ktorí dorazili peši alebo autami, bolo o~$35$\,\% menej ako tých, ktorí dorazili autobusmi.
Všetkých divákov dokopy bolo najviac $400$.

Koľko presne divákov mohlo byť v~divadle?
Nájdite všetky možnosti.
}
\podpis{Erika Novotná}

{%%%%%   Z9-III-2
Štvoruholník $DRAK$ má nasledujúce vlastnosti:
\begin{itemize}
 \item je vpísaný do kružnice,
 \item je osovo súmerný podľa priamky $AD$,
 \item trojuholník $RAK$ je rovnostranný.
\end{itemize}
V~závislosti od dĺžky strany $AK$ vyjadrite dĺžky uhlopriečok štvoruholníka a~obsah štvoruholníka $DRAK$.
}
\podpis{Lenka Dedková}

{%%%%%   Z9-III-3
Nájdite všetky dvojice prirodzených čísel $(a, b)$, pre ktoré platí
$$
7a + 4b + 74 = a\cdot b.
$$
}
\podpis{Erika Novotná}

{%%%%%   Z9-III-4
Zostrojte trojuholník $XYZ$ a~obdĺžnik $ABCD$ tak, aby platili nasledujúce podmienky:
\begin{itemize}
  \item dĺžky strán trojuholníka $XYZ$ sú $|XY|=8$\,cm, $|YZ|=6$\,cm, $|XZ|=7$\,cm,
  \item body $X$ a~$Y$ ležia na priamke $AC$,
  \item úsečky $AC$ a~$XY$ sú rovnako dlhé,
  \item bod $Z$ leží na priamke $BD$,
  \item obsah trojúholníka $ACZ$ je dvakrát väčší ako obsah trojuholníka $ABC$.
\end{itemize}
Konštrukciu oboch útvarov popíšte a~zdôvodnite.
}
\podpis{Michaela Petrová}

