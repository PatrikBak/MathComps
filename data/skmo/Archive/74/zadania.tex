{%%%%% A-I-1
Predpokladajme, že pre reálne čísla $a$, $b$ majú výrazy $a^2+b$ a $a+b^2$ rovnakú hodnotu. Aká najmenšia môže táto hodnota byť?
}
\podpis{Patrik Bak}

{%%%%% A-I-2
Martin k sebe prikladá hracie kocky (rovnaké veľkosťou aj rozmiestnením čísel) tak, aby boli uložené do tvaru štvorca ľubovoľnej veľkosti a aby vždy na dvoch priliehajúcich bočných stenách boli rovnaké čísla. Koľko najviac rôznych čísel sa môže vyskytnúť na horných stenách kociek?
\inspsc{a74i.20}{.8333}%
}
\podpis{Martin Panák, Josef Tkadlec}

{%%%%% A-I-3
Na tabuli sú napísané navzájom rôzne prirodzené čísla so súčtom 2024. Každé z~nich okrem najmenšieho je násobkom súčtu všetkých menších napísaných čísel. Koľko najviac čísel môže na tabuli byť?
}
\podpis{Patrik Bak}

{%%%%% A-I-4
Pre trojuholník $ABC$ platí $|AB|=13$, $|BC|=14$, $|CA|=15$. Jeho posunutím o~vektor dĺžky 1 vznikne trojuholník $A'B'C'$. Určite najmenší možný obsah prieniku trojuholníkov $ABC$ a~$A'B'C'$.
\inspsc{a74i.40}{.8333}%
}
\podpis{Tomáš Bárta}

{%%%%% A-I-5
Saba sa snaží z prízemia nekonečne vysokého mrakodrapu dostať na $n$-té poschodie pomocou zvláštneho výťahu.
Vo výťahu sú tlačidlá $0,1,2,\dots$ Po prvom stlačení tlačidla pôjde výťah nahor a po každom ďalšom pôjde vždy opačným smerom, ako naposledy, pričom po stlačení tlačidla $k$ sa posunie vždy o $2^k$ poschodí. Navyše každé ďalšie stlačené tlačidlo musí mať menšie číslo ako to predošlé.
Dokážte, že Saba sa na každé poschodie $n\ge 1$ môže dostať práve dvoma rôznymi postupmi.
}
\podpis{Morteza Saghafian}

{%%%%% A-I-6
Označme $I_A$, $I_B$, $I_C$ postupne stredy kružníc pripísaných stranám $BC$, $CA$, $AB$ trojuholníka $ABC$. Priesečníky výšok trojuholníkov $I_ABC$, $AI_BC$, $ABI_C$ označme postupne $X$, $Y$, $Z$. Dokážte, že trojuholníky $ABC$ a $XYZ$ sú zhodné.
}
\podpis{Michal Janík}

{%%%%% B-I-1
Z cifier $1$ až $9$ vytvoríme deväťciferné číslo s navzájom rôznymi ciframi. Potom každú jeho dvojicu po sebe idúcich cifier interpretujeme ako dvojciferné číslo a na tabuľu napíšeme jeho najmenší prvočíselný deliteľ. Môžeme tak na tabuli získať práve dve rôzne prvočísla? Ak áno, určte všetky také dvojice prvočísel.
}
\podpis{Patrik Bak}

{%%%%% B-I-2
V trojuholníku $ABC$ platí $|\angle BAC|=45^\circ$. Stranám~$AB$ a~$AC$ sú zvonku pripísané pravouhlé rovnoramenné trojuholníky $ABP$ a~$ACN$ s preponami $AB$ a $AC$. Označme~$M$ stred úsečky~$PN$. Dokážte, že úsečka~$AM$ má dĺžku rovnú polovici polomeru kružnice opísanej trojuholníku $ABC$.
\inspsc{b74i.20}{.8333}%
}
\podpis{Patrik Bak, Anastasia Bredichina}

{%%%%% B-I-3
Pre ktoré prirodzené čísla $n$ sa dá rovnostranný trojuholník so stranou dĺžky $n$ rozrezať na zhodné dieliky tvaru: 
\ite a) \Image*{b74i.302},
\ite b) \Image*{b74i.303}

Dieliky sú tvorené rovnostrannými trojuholníkmi so stranou dĺžky~1.
}
\podpis{Pavel Calábek, Jaroslav Švrček}

{%%%%% B-I-4
\ite a) Nájdite príklad dvojciferného prirodzeného čísla $n$ takého, že číslo $1/n$ má vo svojom desatinnom zápise za desatinnou čiarkou práve dve cifry.
\ite b) Dokážte, že pre každé dve prirodzené čísla $k$, $l$ existujú práve dve kladné racionálne čísla, ktoré majú v desatinnom zápise za desatinnou čiarkou práve $k$ cifier a ich prevrátené hodnoty práve $l$ cifier.
\endgraf
(Desatinný zápis uvažujeme najkratší možný.)
}
\podpis{Josef Tkadlec}

{%%%%% B-I-5
Označme $k$ kružnicu opísanú ostrouhlému trojuholníku $ABC$. Jej obraz v~súmernosti podľa priamky~$BC$ pretína polpriamky opačné k~$BA$ a~$CA$ postupne v bodoch $D\ne B$ a~$E\ne C$. Predpokladajme, že úsečky $CD$ a~$BE$ sa pretínajú na kružnici~$k$. Určte všetky možné veľkosti uhla $BAC$.
}
\podpis{Patrik Bak}

{%%%%% B-I-6
Kladné reálne čísla $x$, $y$, $z$ spĺňajú nerovnosti $xy \ge 2$, $xz \ge 3$, $yz \ge 6$. Akú najmenšiu hodnotu môže nadobúdať výraz $13x^2+10y^2+5z^2$?
}
\podpis{Patrik Bak}

{%%%%% C-I-1
List papiera v tvare obdĺžnika $ABCD$ s rozmermi $a\times b$, pričom $a>b$,
preložíme ako na \obr{} tak, že vrchol $A$ splynie s bodom $C$.
Zdôvodnite, prečo pre obsah $S$ výsledného päťuholníka $PBCQR$ platí $\frac12 ab < S < \frac34 ab$.
\inspdf{c74i_10.pdf}%
}
\podpis{Josef Tkadlec}

{%%%%% C-I-2
Prirodzené čísla $a$, $b$ sú také, že $a>b$, $a+b$ je deliteľné $9$ a $a-b$ je deliteľné $11$.
\ite a) Určte najmenšiu možnú hodnotu čísla $a+b$.
\ite b) Dokážte, že čísla $a+10b$ aj $b+10a$ musia byť deliteľné $99$.
}
\podpis{Jaromír Šimša}

{%%%%% C-I-3
Rámček $2\times 3$ sa dá rozdeliť na štvorce $1\times1$ umiestnením $7$ zápaliek ako na \obr{}.
Ktoré rámčeky $a\times b$, pričom $a\le b$, sa dajú takto rozdeliť pomocou práve 110 zápaliek? Určte všetky možnosti.
\insp{c74i_30.pdf}%
}
\podpis{Josef Tkadlec}

{%%%%% C-I-4
Šachovnicovo ofarbenú tabuľku $4\times 4$ s čiernym ľavým horným políčkom vypĺňame jednotkami a nulami. V~každom štvorci $2\times 2$, ktorý má čierne ľavé horné políčko, je rovnaký počet núl ako jednotiek. Koľkými rôznymi spôsobmi je možné tabuľku vyplniť?
}
\podpis{Ján Mazák}

{%%%%% C-I-5
Nech $P$, $Q$ sú postupne stredy strán $BC$, $AC$ trojuholníka $ABC$. Rovnobežka s~$AC$ prechádzajúca stredom $K$ úsečky $PQ$ pretína priamku $BQ$ v~bode $L$ a priamka $PL$ pretína úsečku $AC$ v~bode $R$. Dokážte, že $R$ je stredom úsečky $AQ$.
}
\podpis{Jaroslav Švrček}

{%%%%% C-I-6
Štvorciferné číslo $\overline{abcd}$ s nenulovými ciframi nazveme \emph{zrkadliteľné} práve vtedy, keď pripočítaním deväťnásobku nejakého trojciferného čísla zapísaného pomocou troch rovnakých cifier vznikne číslo $\overline{dcba}$. Koľko zrkadliteľných čísel existuje?
}
\podpis{Mária Dományová, Patrik Bak}

{%%%%% A-S-1
Rozhodnite, či existujú navzájom rôzne reálne čísla $a$, $b$, $c$ také, že čísla $a^2+b$, $b^2+c$, $c^2+a$ sa v~nejakom poradí rovnajú číslam $a+b^2$, $b+c^2$, $c+a^2$.
}
\podpis{Patrik Bak}

{%%%%% A-S-2
Daný je konvexný päťuholník $ABCDE$ taký, že $|AB|=|BC|$, $|AE|=|DE|$, $AC \perp AD$ a ~$CD \parallel BE$.
Dokážte, že trojuholníky $ABC$ a~$AED$ majú rovnaké obsahy.
}
\podpis{Patrik Bak}

{%%%%% A-S-3
Povieme, že prirodzené číslo je \emph{ploché}, keď sú všetky jeho cifry rovnaké (aj jednociferné čísla považujeme za ploché).
Rozhodnite, či sa dá každé prirodzené číslo, ktoré nie je ploché, vyjadriť ako súčet niekoľkých navzájom rôznych plochých čísel.
}
\podpis{Jozef Rajník}

{%%%%% A-II-1
Dané sú dve navzájom rôzne reálne čísla $a$, $b$ také, že výrazy $a^3 + b$ a $a + b^3$ majú rovnakú hodnotu. Dokážte, že pre ich súčin platí $-1\le ab<\frac13$.
}
\podpis{Jana Kopfová, Jaromír Šimša}

{%%%%% A-II-2
Prebieha online hlasovanie medzi variantmi $A$ a~$B$. Predtým, ako Pavol hlasoval, bol počet percent hlasov pre variant $A$ rovný kladnému celému číslu.
Pavlovým hlasom sa toto číslo zväčšilo presne o jedna. Dokážte, že Pavlov hlas bol devätnástym hlasom pre variant~$A$.
}
\podpis{Josef Tkadlec}

{%%%%% A-II-3
V~tíme je sedem hráčov. V~každom kole turnaja ich päť hrá a~dvaja sedia na tribúne. Dokážte, že nezávisle od (kladného) počtu kôl aj výberu pätíc možno na konci turnaja nájsť dvoch hráčov, ktorí boli spolu (či už na ihrisku alebo na tribúne) vo viac ako polovici kôl.
}
\podpis{David Hruška}

{%%%%% A-II-4
V~rovine je daná úsečka $BC$. Uvažujeme všetky ostrouhlé trojuholníky $ABC$, v~ktorých $|\angle BAC|=45^\circ$.
V~každom takomto trojuholníku označíme $D$ a~$E$ postupne tie body strán $AB$ a~$AC$, pre ktoré je priamka $BC$ spoločnou dotyčnicou kružníc opísaných trojuholníkom $ACD$ a~$ABE$. Päty kolmíc z bodov $D$ a~$E$ na priamku $BC$ označíme postupne $P$ a~$Q$. Dokážte, že v~rovine existuje bod $X$ neležiaci na priamke $BC$, pre ktorý veľkosť uhla $PXQ$ nezávisí od polohy bodu $A$.
}
\podpis{Zdeněk Pezlar}

{%%%%% A-III-1
Pre reálne čísla $a$, $b$, $c$, $d$ platí
$$a+b+c+d=0 \qquad\hbox{a}\qquad
\frac1a+\frac1b+\frac1c+\frac1d=0.$$
Koľko z~rovností
$$ab=cd,\qquad ac=bd,\qquad ad=bc$$
môže súčasne platiť? Určte všetky také počty.
}
\podpis{Michal Janík}

{%%%%% A-III-2
Nájdite najväčšie celé číslo $n$ s nasledujúcou vlastnosťou: Kedykoľvek je v rovine daných päť navzájom rôznych bodov tak, že niektoré dva z nich ležia vo vnútri trojuholníka tvoreného zvyšnými tromi bodmi, je možné niektoré tri z týchto piatich bodov označiť $X$, $Y$, $Z$ tak, že platí $n^\circ < |\angle XYZ|\le 180^\circ$.
}
\podpis{Josef Tkadlec}

{%%%%% A-III-3
Nech $p$ je najväčšie prvočíslo deliace prirodzené číslo $n>1$.
Pre každú neprázdnu podmnožinu deliteľov čísla $n$ napíšeme na tabuľu súčet jej prvkov. Predpokladajme, že sme takto napísali viac ako $p$ čísel z~množiny $\{1,2,\ldots,p+2\}$ a žiadne číslo z tejto množiny sme nenapísali viackrát. Dokážte, že potom sme žiadne číslo nenapísali viackrát.
}
\podpis{Zdeněk Pezlar}

{%%%%% A-III-4
Pozdĺž kružnice je napísaných niekoľko (aspoň tri) navzájom rôznych prvočísel. Pre každé dve susedné prvočísla určíme najväčšie prvočíslo deliace ich súčet. Takto získame až na poradie opäť rovnaké prvočísla, ako boli tie napísané.
Nájdite všetky možné počiatočné množiny prvočísel.

(Napríklad prvočísla $2,7,3,11,17$ v tomto poradí nevyhovujú, pretože zodpovedajúce súčty $9,10,14,28,19$ majú najväčšie prvočíselné delitele $3,5,7,7,19$.)
}
\podpis{Michal Janík}

{%%%%% A-III-5
Nájdite všetky kladné celé čísla $n$ s~nasledujúcou vlastnosťou: Vo štvorcovej tabuľke $n\times n$ sa dá vyfarbiť $2n$ políčok tak, že žiadne dve z~ich nesusedia stranou ani vrcholom a~v~každom riadku aj každom stĺpci sú vyfarbené práve dve políčka.
}
\podpis{Jakub Štepo}

{%%%%% A-III-6
V danom ostrouhlom trojuholníku $ABC$ označme $H$ priesečník výšok, $\omega$ kružnicu opísanú a $O$ jej stred. Ďalej označme $M$ stred strany $BC$ a~$D \ne A$ priesečník priamky $AH$ s~kružnicou $\omega$. Priamka $DM$ pretína kružnicu $\omega$ v~bode $E\ne D$. Nech $F\ne E$ je priesečník priamky $AE$ s~kružnicou opísanou trojuholníku $OME$.
Dokážte, že platí $|FH| = |FA|$.
}
\podpis{Michal Pecho}

{%%%%% B-S-1
Z~cifier $1$ až $9$ vytvoríme deväťciferné číslo s navzájom rôznymi ciframi. Potom každú jeho dvojicu po sebe idúcich cifier interpretujeme ako dvojciferné číslo a~týchto osem čísel napíšeme na tabuľu.
\ite a) Koľko najviac mocnín prvočísel medzi nimi môže byť?
\ite b) Koľko rôznych deväťciferných čísel nás k~tomuto počtu dovedie?
\endgraf
(Uvažujeme len mocniny prvočísel s celočíselným exponentom väčším ako 1.)
}
\podpis{Dominik Rigasz}

{%%%%% B-S-2
Vnútri pravouhlého rovnoramenného trojuholníka $ABC$ s preponou $BC$ leží bod~$D$ taký, že $AD \perp BD$. V~polrovine určenej priamkou~$AD$ neobsahujúcej bod~$B$ leží štvorec $ADEF$. Dokážte, že priamka $EF$ prechádza bodom~$C$.
}
\podpis{Patrik Bak}

{%%%%% B-S-3
Pre prirodzené čísla $r$, $s$ platí, že zlomok $r/s$ leží v~intervale $\langle23/45;46/89\rangle$. Akú najmenšiu hodnotu môže mať menovateľ $s$?
}
\podpis{Pavel Calábek}

{%%%%% B-II-1
V~obore reálnych čísel riešte sústavu rovníc
$$
\aligned
 x^2+4y^2+z^2-4xy-2z+1&=0,\\
 y^2-xy-2y+2x&=0.
\endaligned
$$
}
\podpis{Mária Dományová}

{%%%%% B-II-2
Určte všetky prirodzené čísla~$n$ s~nasledujúcou vlastnosťou: Pravidelný šesťuholník so stranou dĺžky $n$ sa dá rozrezať na útvary ako na obrázku zložené zo štyroch rovnostranných trojuholníkov so stranou dĺžky~1.
\inspsc{b74iii.20}{.8333}%
}
\podpis{Anastasia Bredichina}

{%%%%% B-II-3
Nech $I$ je stredom kružnice vpísanej trojuholníku $ABC$. Obraz kružnice~$k$ opísanej trojuholníku $BIC$ v~osovej súmernosti podľa priamky~$BC$ pretína úsečky $AB$ a~$AC$ postupne v~bodoch $D\ne B$ a~$E\ne C$. Predpokladajme, že sa úsečky $BE$ a~$CD$ pretínajú na kružnici~$k$. Určte všetky možné veľkosti uhla $BAC$.
}
\podpis{Anastasia Bredichina, Patrik Bak}

{%%%%% B-II-4
Nájdite všetky prirodzené čísla $n$ také, že čísla
$$
\frac1n\qquad \hbox{a}\qquad \frac1{n+23^2}
$$
majú nekonečné desatinné rozvoje, ktoré sa zhodujú od niektorého miesta rovnakého pre obe čísla.
}
\podpis{Ján Mazák, Josef Tkadlec}

{%%%%% C-S-1
Štvorcovú tabuľku $4\times 4$ vyfarbujeme štyrmi rôznymi farbami tak, aby každé políčko tabuľky bolo vyfarbené práve jednou farbou. Rozhodnite, či je možné nájsť vyfarbenie, v~ktorom bude každá farba v~každej z~deviatich menších tabuliek $2\times2$ a tiež
\ite a) v~každom riadku a~v~každom stĺpci,
\ite b) v~každom riadku.
}
\podpis{Jaroslav Švrček}

{%%%%% C-S-2
Patrik napísal na tabuľu dve prirodzené čísla. Všimol si, že ich súčet je prvočíslo~$313$, ktoré je navyše deliteľom súčtu ich najväčšieho spoločného deliteľa a~najmenšieho spoločného násobku. Ktoré čísla Patrik na tabuľu napísal?
}
\podpis{Patrik Bak}

{%%%%% C-S-3
Hárok papiera s~rozmermi $2\times 1$ preložíme ako na \obr{} tak, že vrchol $A$ splynie so stredom $A'$ úsečky $BC$. Určite obsah preloženej časti, t.\,j. štvoruholníka $A'D'FE$.
\inspsc{c74ii.30}{.8333}%
}
\podpis{Tomáš Bárta}

{%%%%% C-II-1
Dvojciferné číslo $\overline{ab}$ nazveme \emph{nafúknuteľné}, ak z neho po pripočítaní 990-násobku vhodného jednociferného čísla získame štvorciferné číslo tvaru $\overline{axxb}$ s~nenulovou cifrou~$x$. Koľko nafúknuteľných čísel existuje?
}
\podpis{Mária Dományová}

{%%%%% C-II-2
V štvorcovej sieti leží päťuholník $ABCDE$, ktorého vrcholy sú v~mrežových bodoch rovnako ako na \obr. Dokážte, že tento päťuholník sa dá rozdeliť na dva zhodné štvoruholníky.
\inspdf{a_m_fig-5gon.pdf}%
}
\podpis{Jaroslav Zhouf, Josef Tkadlec}

{%%%%% C-II-3
Koľkými spôsobmi je možné vyfarbiť štvorcovú tabuľku $4\times 4$ štyrmi rôznymi farbami tak, aby každé jej políčko bolo vyfarbené práve jednou farbou a aby v~každej menšej štvorcovej tabuľke $2\times 2$ bola každá farba práve raz?
}
\podpis{Jana Kopfová}

{%%%%% C-II-4
Lenka stavia konštrukciu tvaru kvádra z~magnetických tyčiniek dĺžky $1$ a kovových guľôčok. Na konštrukciu $2 \times 3\times1$ (pozri \obr) spotrebovala $46$ tyčiniek a~$24$ guľôčok. Na inú konštrukciu $a\times b\times1$ spotrebovala $679$ tyčiniek. Koľko na ňu spotrebovala guľôčok? Určte všetky možnosti.
\inspdf{a_m_fig-ramecek-3d.pdf}%
}
\podpis{Jana Kopfová, Lenka Kopfová, Josef Tkadlec}

{%%%%%   vyberko, den 1, priklad 1
...}
\podpis{...}

{%%%%%   vyberko, den 1, priklad 2
...}
\podpis{...}

{%%%%%   vyberko, den 1, priklad 3
...}
\podpis{...}

{%%%%%   vyberko, den 1, priklad 4
...}
\podpis{...}

{%%%%%   vyberko, den 2, priklad 1
...}
\podpis{...}

{%%%%%   vyberko, den 2, priklad 2
...}
\podpis{...}

{%%%%%   vyberko, den 2, priklad 3
...}
\podpis{...}

{%%%%%   vyberko, den 2, priklad 4
...}
\podpis{...}

{%%%%%   vyberko, den 3, priklad 1
...}
\podpis{...}

{%%%%%   vyberko, den 3, priklad 2
...}
\podpis{...}

{%%%%%   vyberko, den 3, priklad 3
...}
\podpis{...}

{%%%%%   vyberko, den 3, priklad 4
...}
\podpis{...}

{%%%%%   vyberko, den 4, priklad 1
...}
\podpis{...}

{%%%%%   vyberko, den 4, priklad 2
...}
\podpis{...}

{%%%%%   vyberko, den 4, priklad 3
...}
\podpis{...}

{%%%%%   vyberko, den 4, priklad 4
...}
\podpis{...}

{%%%%%   vyberko, den 5, priklad 1
...}
\podpis{...}

{%%%%%   vyberko, den 5, priklad 2
...}
\podpis{...}

{%%%%%   vyberko, den 5, priklad 3
...}
\podpis{...}

{%%%%%   vyberko, den 5, priklad 4
...}
\podpis{...}

{%%%%%   trojstretnutie, priklad 1
...}
\podpis{...}

{%%%%%   trojstretnutie, priklad 2
...}
\podpis{...}

{%%%%%   trojstretnutie, priklad 3
...}
\podpis{...}

{%%%%%   trojstretnutie, priklad 4
...}
\podpis{...}

{%%%%%   trojstretnutie, priklad 5
...}
\podpis{...}

{%%%%%   trojstretnutie, priklad 6
...}
\podpis{...}

{%%%%%   IMO, priklad 1
...}
\podpis{...}

{%%%%%   IMO, priklad 2
...}
\podpis{...}

{%%%%%   IMO, priklad 3
...}
\podpis{...}

{%%%%%   IMO, priklad 4
...}
\podpis{...}

{%%%%%   IMO, priklad 5
...}
\podpis{...}

{%%%%%   IMO, priklad 6
...}
\podpis{...}

{%%%%%   MEMO, priklad 1
}
\podpis{}

{%%%%%   MEMO, priklad 2
}
\podpis{}

{%%%%%   MEMO, priklad 3
}
\podpis{}

{%%%%%   MEMO, priklad 4
}
\podpis{}

{%%%%%   MEMO, priklad t1
}
\podpis{}

{%%%%%   MEMO, priklad t2
}
\podpis{}

{%%%%%   MEMO, priklad t3
}
\podpis{}

{%%%%%   MEMO, priklad t4
}
\podpis{}

{%%%%%   MEMO, priklad t5
}
\podpis{}

{%%%%%   MEMO, priklad t6
}
\podpis{}

{%%%%%   MEMO, priklad t7
}
\podpis{}

{%%%%%   MEMO, priklad t8
}
\podpis{}

{%%%%%   CPSJ, priklad 1
...}
\podpis{...}

{%%%%%   CPSJ, priklad 2
...}
\podpis{...}

{%%%%%   CPSJ, priklad 3
...}
\podpis{...}

{%%%%%   CPSJ, priklad 4
...}
\podpis{...}

{%%%%%   CPSJ, priklad 5
...}
\podpis{...}

{%%%%%   CPSJ, priklad t1
...}
\podpis{...}

{%%%%%   CPSJ, priklad t2
...}
\podpis{...}

{%%%%%   CPSJ, priklad t3
...}
\podpis{...}

{%%%%%   CPSJ, priklad t4
...}
\podpis{...}

{%%%%%   CPSJ, priklad t5
...}
\podpis{...}

{%%%%%   CPSJ, priklad t6
...}
\podpis{...}

{%%%%%   EGMO, priklad 1
...}
\podpis{...}

{%%%%%   EGMO, priklad 2
...}
\podpis{...}

{%%%%%   EGMO, priklad 3
...}
\podpis{...}

{%%%%%   EGMO, priklad 4
...}
\podpis{...}

{%%%%%   EGMO, priklad 5
...}
\podpis{...}

{%%%%%   EGMO, priklad 6
...}
\podpis{...}

{%%%%%   vyberko C, den 1, priklad 1
...}
\podpis{...}

{%%%%%   vyberko C, den 1, priklad 2
...}
\podpis{...}

{%%%%%   vyberko C, den 1, priklad 3
...}
\podpis{...}

{%%%%%   vyberko C, den 1, priklad 4
...}
\podpis{...}

{%%%%%   vyberko C, den 1, priklad 5
...}
\podpis{...} 