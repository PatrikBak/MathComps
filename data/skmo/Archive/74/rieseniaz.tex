{%%%%%   Z4-I-1
...}

{%%%%%   Z4-I-2
...}

{%%%%%   Z4-I-3
...}

{%%%%%   Z4-I-4
...}

{%%%%%   Z4-I-5
...}

{%%%%%   Z4-I-6
...}

{%%%%%   Z5-I-1
Za tri roky bude súčet vekov Karola a Petra o šesť rokov väčší ako teraz, zatiaľ čo vek Boženy sa zväčší o tri roky.
Pretože tieto hodnoty byť rovnaké, má práve teraz Božena o tri roky viac ako súčet vekov oboch chlapcov.

Peter je o dva roky starší ako Karol, teda súčet ich vekov je rovnaký ako dvojnásobok veku Karola plus dva roky.
Celkom sa teda vek Boženy dá vyjadriť ako dvojnásobok veku Karola plus päť rokov.

Zo zadania tiež vieme, že Božena je trikrát staršia ako Karol.
Porovnaním týchto dvoch vyjadrení dostávame, že Karol má päť rokov, a teda Peter má sedem a Božena pätnásť.
\poznamka
Myšlenky prvních dvou odstavcov sa dajú zapísať takto:
$$
B=K+P+3, \quad P=K+2 \ \Longrightarrow\  B=2K+5.
$$
Porovnáním týchto dvoch vyjadrení veku Boženy dostávame:
$$
2K+5=3K \ \Longrightarrow\  K=5.
$$
Odtiaľ potom vyplýva, že $P=7$ a~$B=15$.

\ineriesenie
S rovnakým značením ako v predchádzajúcej poznámke je možné postupne vzhľadom na vek Karola vyjadrovať veky ostatných a kontrolovať podmienku rovnosti vekov po troch rokoch:
$$
\begintable
$K$\|1|2|3|4|5|\dots\cr
$P=K+2$\|3|4|5|6|7|\dots\cr
$K+3+P+3$\|10|12|14|16|\bf 18|\dots\crthick
$B=3K$\|3|6|9|12|15|\dots\cr
$B+3$\|6|9|12|15|\bf 18|\dots%
\endtable
$$
Vyhovujúce riešenie dostávame pre $K=5$, $P=7$ a ~$B=15$.
Karol má päť rokov, Peter sedem a ~ Božena pätnásť.
}

{%%%%%   Z5-I-2
Z~každých dvoch bielych trojuholníkov je možné zložiť obdĺžnik, ktorého jedna strana sa zhoduje so stranou šedého štvorca.
Zo štyroch trojuholníkov je možné zložiť dva obdĺžniky.

Súčet obsahov dvoch obdĺžnikov je štvornásobkom obsahu štvorca, práve keď druhá strana obdĺžnika je dvojnásobkom strany štvorca.
Teda strana $XY$ meria 20\,cm.
\insp{z5-i-2a.eps}%


\poznamka
Biele trojuholníky sú všetky rovnaké, teda každý z nich má rovnaký obsah ako šedý štvorec.
Z~každého trojuholníka je možné vytvoriť štvorec odstrihnutím a presunutím jeho časti (v súlade s predchádzajúcim obrázkom) takto:
\insp{z5-i-2b.eps}%
}

{%%%%%   Z5-I-3
Ako pôvodný, tak nový výsledok je násobkom troch.
Teda súčet odčítaných čísel tiež musí byť násobkom troch.
Také súčty pre dvojice čísel od 4 do 8 sú:
$$
4+5=9,\quad 4+8=12,\quad 5+7=12,\quad 7+8=15.
$$
Všetky možné zámeny znamienok (a~zodpovedajúce výsledky) sú:
$$
\begin{aligned}
	9+8+7+6-5-4 &=21, \\
	9-8+7+6+5-4 &=15, \\
	9+8-7+6-5+4 &=15, \\
	9-8-7+6+5+4 &=9.
\end{aligned}
$$

\poznamka
Počet možných zámien dvoch znamienok z piatich je desať.
Bez úvodného postrehu je možné postupovať tak, že sa vypočíta desať zodpovedajúcich príkladov a vyberú sa tie, ktorých výsledok je násobkom troch.
}

{%%%%%   Z5-I-4
Postupne vylučujeme jedného zo štyroch deliteľov a ~hľadáme najmenšie čísla, ktoré sú bezo zvyšku deliteľné zvyšnými tromi číslami.
Medzi nimi vyberáme tie, ktoré nie sú deliteľné vylúčeným číslom (nevyhovujúce možnosti sú v~nasledujúcej tabuľke preškrtnuté).
Výsledok má predstavovať deň v mesiaci, teda nás zaujímajú hodnoty neprevyšujúce 31 (vyhovujúce možnosti sú zvýraznené tučným písmom).
Závery sú zhrnuté v tabuľke:
$$
\begintable
ne|ano\|možnosti \hfill\crthick
3|4, 5, 6\|$\text{\sout{60}}$, \dots \hfill\cr
4|3, 5, 6\|{\bf 30}, $\text{\sout{60}}$, \dots \hfill\cr
5|3, 4, 6\|{\bf 12}, {\bf 24}, 36, \dots \hfill\cr
6|3, 4, 5\|$\text{\sout{60}}$, \dots \hfill
\endtable
$$

Pinocchio sa mohol narodiť 12., 24., alebo 30. deň v~mesiaci.

\ineriesenie
Po riadkoch vypíšeme všetky čísla neprevyšujúce 31, ktoré sú bezo zvyšku deliteľné tromi, štyrmi, piatimi a šiestimi:
\bgroup
\thicksize=0pt
\thinsize=0pt
\def\ctr#1{\hfil\,#1\,\hfil}
$$
\begintable
\ 3|||\ 6|\ \ ||\ 9||\ \ |\bf 12|\ \ |\ \ |15||\ \ |18|\ \ ||21|\ \ |\ \ |\bf 24||\ \ |27||\ \ |\bf 30|\nr
|\ 4||||\ 8||||\bf 12||||16||||20||||\bf 24||||28|||\nr
||\ 5|||||10|||||15|||||20|||||25|||||\bf 30|\nr
|||\ 6||||||\bf 12||||||18||||||\bf 24||||||\bf 30|%
\endtable
$$
\egroup
Čísla, ktoré sú deliteľné tromi zo štyroch uvedených deliteľov, sú zvýraznené tučným písmom.

Pinocchio sa mohol narodiť 12., 24., alebo 30. deň v~mesiaci.

\poznamky
Ak sa Pinocchio narodil vo februári, potom posledná možnosť (30.) odpadá.

Pri prvom uvedenom riešení je možné možné hodnoty v treťom stĺpci tabuľky určovať
napr. tak, že postupne pre násobky najväčšieho z~daných čísel overujeme, či sú bezo zvyšku deliteľné zvyšnými dvoma číslami.
Najmenšia z~uvedených hodnôt je tzv. {\it najmenší spoločný násobok\/} čísel.


lohu možno riešiť aj tak, že sa postupne pre čísla od 1 do 31 určia všetci delitele, ktorými možno dané číslo bezo zvyšku deliť, a medzi nimi sa hľadajú traja zo štyroch daných deliteľov..
}

{%%%%%   Z5-I-5
Cesta dĺžky 1\,km nie je žiadna.

Cesty dĺžky 2\,km sú dve:
\insp{z5-i-5a.eps}%

Ciest dĺžky 3\,km je päť:
\insp{z5-i-5b.eps}%
\insp{z5-i-5c.eps}%
\insp{z5-i-5d.eps}%

Medzi miestami $A$ a~$B$ je celkom sedem ciest dlhých nanajvýš 3\,km.

\poznamka
Cesty dĺžky 3\km sú z predchádzajúcich ciest odvodené tak, že 1\km úseky boli systematicky nahradzované dvoma 1\km úsekmi začínajúcimi a končiacimi v rovnakých križovatkách.
}

{%%%%%   Z5-I-6
Skupina troch korálikov $ABC$ zaberá 12\,mm, skupina šiestich korálikov $AABBCC$ zaberá 24\,mm, skupina deviatich korálikov $AAABBBCCC$ zaberá 36\,mm atď.
\insp{z5-i-6a.eps}%

Prvá skupina zaberá 12\,mm nite, prvé dve skupiny zaberajú 36\,mm, prvé tri skupiny zaberajú 72\,mm atď.
Vzhľadom k~počtu korálok každého tvaru v skupine, vyjadríme dĺžku skupiny, celkový počet korálok a~celkovú dĺžku doposiaľ zabranej nite:
\bgroup
\def\ctr#1{\hfil\ \ #1\ \ \hfil}
$$
\begintable
počet trojíc v skupine\|1|2|3|4|5|6|7|8|9\cr
dĺžka skupiny (mm)\hfill\|12|24|36|48|60|72|84|96|108\crthick
celkový počet korálikov\hfill\|3|9|18|30|45|63|84|108|135\cr
celková dĺžka (mm)\hfill\|12|36|72|120|180|252|336|432|540%
\endtable
$$
\egroup

Dĺžka 500\,mm je dosiahnutá v~rámci deviatej skupiny.
Posledná úplná skupina pozostáva z ôsmich korálikov každého tvaru.
Na konci tejto skupiny je celkom zabraných 432\,mm nite, zostáva 68\,mm, doteraz použitých korálikov je 108.

Deväť korálikov tvaru $A$ zaberá 45\,mm.
S týmito korálkami je celkom zabraných 477 mm nite, zostáva 23 mm, doteraz použitých korálok je 117.

Šesť korálikov tvaru $B$ zaberá 24\,mm.
S týmito korálkami je celkom zabraných 501\,mm nite, doteraz použitých korálok je 123

Anička potrebuje aspoň 123 korálikov.

\poznamka
Postupné sčítanie v~predchádzajúcej tabuľke je možné nahradiť delením so zvyškom:
$$
500:12\ \text{dává 41 a zbytek 8}.
$$
Teda do 500\,mm sa môže vliezť 41 trojíc $ABC$ a~zostane 8\,mm.
Je však nutné zohľadniť výskyt v skupinách.
V~celých skupinách vieme dostať $1+2+3+4+5+6+7+8 =36$ trojíc $ABC$ (s~deviatou skupinou by sme mali $36+9=45$, čo je viac ako 41).
Týchto 36 trojíc dokopy zaberá 432\,mm nite ($36\cdot12=432$) a ~obsahuje 108 korálikov ($36\cdot3=108$).
}

{%%%%%   Z6-I-1
Za jeden týždeň Lenka kúpi 8 vaflí ($2+1+0+3+2=8$).
Tridsať predaných vaflí zodpovedá nákupu za tri celé týždne a ~šiestim vafliam predaným vo štvrtom týždni ($30=3\cdot8+6$).
Šesť vaflí vo štvrtom týždni má po nákupe vo štvrtok ($2+1+0+3=6$).

Za jeden týžden Libuška koupí 10 šišiek ($1+3+4+0+2=10$).
Počet šišiek, které jí pan Vaflička predal za tri celé týždne a~prvé štyri dni štvrtého týždňa, bol
$$
3\cdot10+1+3+4+0 =38.
$$

\poznamka
K rovnakému výsledku je možné dospieť postupným vypisovaním a sčítaním Lenkiných nákupov.
}

{%%%%%   Z6-I-2
Všetky súmernosti obdĺžnika sú tiež súmernosťami doplnených polkružníc.
Ide o~dve osové súmernosti (ktorých osi $o_1$, $o_2$ sú osami dvojíc protiľahlých strán obdĺžnika) a~stredovú súmernosť (ktorej stred $S$ je priesečníkom osí súmerností alebo priesečníkom uhlopriečok obdĺžnika).
\insp{z6-i-2a.eps}%

Strany štvorca $ABCD$ majú byť rovnobežné so stranami obdĺžnika, preto vyššie popísané súmernosti sú tiež súmernosťami štvorca.
Najmä stred štvorca splýva so stredom obdĺžnika.
Navyše vo štvorci sú uhlopriečky ďalšími osami súmerností (ktoré označíme $p_1$, $p_2$).
Tieto uhlopriečky sú navzájom kolmé a polia vnútorné uhly štvorca (rovnako ako uhly vymedzené osami $o_1$, $o_2$).
\insp{z6-i-2b.eps}%

Vrcholy štvorca je možné zostrojiť napr. takto:
\begin{itemize}
\parindent=2\parindent
  \item $S =$ priesečník uhlopriečok obdĺžnika,
  \item $o_1$, $o_2 =$ rovnobežky so stranami obdĺžnika idúce bodom $S$,
  \item $p_1$, $p_2 =$ osi uhlov daných priamkami $o_1$, $o_2$,
  \item $A$, $B$, $C$, $D =$ priesečníky priamok $p_1$, $p_2$ s~polkružnicami $k$, $l$.
\end{itemize}

\poznamky
Dlhšia strana daného obdĺžnika je dvakrát dlhšia ako kratšia strana, teda polkružnice $k$, $l$ sa dotýkajú jeho dlhších strán.
Jedna z~osí $o_1$, $o_2$ prechádza týmito bodmi dotyku, zatiaľ čo druhá prechádza priesečníkmi polkružníc $k$, $l$.
Úloha vrátane uvedeného riešenia dáva zmysel aj pre všeobecný pomer strán.
}

{%%%%%   Z6-I-3
Číslo 36 má veľa deliteľov a~každým z~nich je hľadaný palindróm tiež deliteľný.
Vzhľadom na to, že $36=4\cdot9$ a~čísla 4 a~9 sú nesúdeliteľné, stačí kontrolovať deliteľnosť práve týmito číslami:
\begin{itemize}
\item Deliteľnosť štyrmi znamená, že posledné dvojčíslie je deliteľné štyrmi.
  Najmenšie dvojčíslie deliteľné štyrmi s najmenšou nenulovou číslicou na mieste jednotiek je 12, teda palindróm je tvaru:
  $$
  2\ 1\ *\ 1\ 2
  $$
  \item Deliteľnosť deviatich znamená, že ciferný súčet je deliteľný deviatimi.
  Najmenšie možné doplnenie predchádzajúceho tvaru je:
	$$
	2\ 1\ 3\ 1\ 2
	$$
\end{itemize}
Najmenší päťciferný palindróm deliteľný 36 je 21312.

\ineriesenie
Každý deliteľ čísla 36 je tiež deliteľom hľadaného palindrómu.
Teda palindróm je deliteľný dvomi
a~párne päťciferné palindrómy zoradené vzostupne sú:
$$
\begin{aligned}
	& 20002,\ 20102,\ 20202,\ 20302,\ 20402,\ 20502,\ 20602,\ 20702,\ 20802,\ 20902,\\
	& 21012,\ 21112,\ 21212,\ 21312,\ \dots
\end{aligned}
$$

Práve posledné uvedené číslo je prvé, ktoré je deliteľné 36.
Najmenší päťciferný palindróm deliteľný 36 je 21312.
}

{%%%%%   Z6-I-4
Počet tulipánov, ktoré zasadila Šárka, bol násobkom 9,
počet tých, ktoré zasadil Ľuboš, bol násobkom 17 a spoločne ich zasadili 70.
Pre násobky 17 neprevyšujúce 70 vyjadríme rozdiel od 70 a overíme deliteľnosť deviatimi:
$$
\begintable
Ľuboš\|17|34|51|68\cr
Šárka\|53|36|19|2\crthick
děl. 9\|nie|áno|nie|nie%
\endtable
$$
Ľuboš zasadil 34 tulipánov, Šárka ich zasadila 36.

Červené tulipány sadila Šárka a ~bolo ich $\frac59\cdot36 =20$.
Žlté tulipány sadil Ľuboš a ~bolo ich $\frac2{17}\cdot34 =4$.
Inú farbu ako červenú či žltú malo $70-20-4=46$ tulipánov.
}

{%%%%%   Z6-I-5
Postupne preveríme tri prípady podľa toho, ktorá z kamarátok nehovorila pravdu.

\smallskip
\noindent
Predpokladajme, že klamala prvá (a~druhá a~tretia hovorili pravdu:
\begin{itemize}
  \item Prvý nemôže bývať v~Očovej (pretože tá z~Očovej neklame), ani v~Hruštíne (pretože potom by hovorila pravdu).
  Teda prvá býva v Jasenove.
  \item Druhá nebýva v~Očovej (pretože hovorila pravdu), ani v~Jasenove (pretože tam býva prvá).
  Teda druhá býva v~Hruštíne.
  \item Na tretiu ostáva Očová, čo nie je s~ničím v~rozpore (ako pravdomluvná tam bývať môže a~jej výrok súhlasí s predchádzajúcim záverom o druhej kamarátke).
\end{itemize}

\noindent
Predpokladajme, že klamala druhá:
\begin{itemize}
  \item Druhá má bývať v~Očovej (pretože jej výrok nie je pravdivý) a~súčasne tam bývať nemôže (pretože tá z~Očovej neklame).
  Máme opäť rozpor.
\end{itemize}

\noindent
Predpokladajme, že klamala tretia (a~prvá a~druhá hovorili pravdu):
\begin{itemize}
  \item Prvý býva v~Hruštíne (pretože hovorila pravdu).
  \item Druhá nebýva v~Očovej (pretože hovorila pravdu), ani v~Hruštíne (pretože tam býva prvá).
  Teda druhá býva v ~Jasenove.
  \item Na tretiu ostáva Očová, čo je v~rozpore s~tým, že tá z~Očovej neklame.
\end{itemize}

\smallskip
Iba prvý prípad neviedol k žiadnemu rozporu.
Teda prvá kamarátka býva v~Jasenove, druhá v~Hruštíne a~tretia v~Očovej.
}

{%%%%%   Z6-I-6
a)
Aby sa žiadne tvary nemenili, nemôže mať žiadny kruh viac susedov trojuholníkov a naopak.
Rozmiestnenie tvarov môže vyzerať takto:
\insp{z6-i-6a.eps}%

b)
Aby sa všetky tvary menili, musí mať každý kruh viac susedov trojuholníkov a naopak.
Rozmiestnenie tvarov a~následné zmeny môžu vyzerať takto:
\insp{z6-i-6b.eps}%

c)
Aby sa zmeny po čase ustálili, potrebujeme viac meniacich sa kruhov ako trojuholníkov (či naopak).
Rozmiestnenie tvarov a~následné zmeny môžu vyzerať takto:
\insp{z6-i-6c.eps}%

\poznamka
Systémy založené na podobných pravidlách majú zaujímavé súvislosti a ~aplikácie, viď napr.
\ulink[https://cs.wikipedia.org/wiki/Hra_\%C5\%BEivota]{Hru života}.
}

{%%%%%   Z7-I-1
Prvý deň Alenka zjedla tri sedminy toho, čo v ten istý deň zjedli spoločne,
na druhý deň zjedla dve pätiny toho, čo v ten istý deň zjedli spoločne.
Počet sliviek spoločne zjedených za prvý deň bol násobkom 7,
za druhý deň násobkom 5.
Celkom za oba dni zjedli 31 sliviek.
Pre násobky 7 neprevyšujúce 31 vyjadríme rozdiel od 31 a overíme deliteľnosť piatimi:
$$
\begintable
1. den\|7|14|21|28\cr
2. den\|24|17|10|3\crthick
děl. 5\|nie|nie|áno|nie%
\endtable
$$
Prvý deň spoločne zjedli 21 sliviek, druhý deň ich zjedli 10.

Prvý deň Alenka zjedla 9 sliviek ($\frac37\cdot21=9$), druhý deň zjedla 4 slivky ($\frac25\cdot10=4$).
Alenka za oba dni zjedla 13 sliviek.

\poznamka
Prehľadný záznam konzumácie sliviek vyzerá takto:
$$
\begintable
\|Alenka|Zuzka\|dokopy\crthick
1. deň\|9|12\|21\cr
2. deň\|4|6\|10\crthick
dokopy\|13|18\|31%
\endtable
$$
}

{%%%%%   Z7-I-2
Susedné kocky na spodnom a prostrednom poschodí pôvodnej pyramídy mali 3 spoločné zvislé steny.
Teda posunutím kociek vzniklo 6 zvislých stien navyše.

K zmene povrchu prispievajú aj rozdiely na úrovni vodorovných rozhraní medzi poschodiami.
Tie nezávisia na miere posunutia poschodí voči sebe.
Na vyjadrenie prírastku povrchu je výhodné poschodia vhodne posunúť (pri zachovaní všetkých podmienok zo zadania).
Pyramídy môžu vyzerať napr. takto:
\insp{z7-i-2a.eps}%

V~novej pyramíde pribudli
2/3 steny medzi horným a~prostredným poschodím (1/3 na spodnej stene hornej kocky a~1/3 na hornej stene pravej prostrednej kocky)
a~4/3 steny medzi prostredným a~spodným poschodím (2/3 na spodných stenách prostredných kociek a~2/3 na horných stenách dvoch spodných kociek).

Dokopy je v povrchu novej pyramídy o osem stien viac ako v pôvodnej pyramíde ($6+2/3+4/3 =8$).
Jedna stena má obsah 49\,cm$^2$ ($7\cdot7=49$).
Povrchy pôvodnej a~novej pyramídy sa líšia o~$392\,\cm^2$ ($8\cdot49=392$).
}

{%%%%%   Z7-I-3
Bonifácove číslo je štvornásobkom Pankrácovho čísla
a~Pankrácove číslo je deliteľné deviatimi, teda Bonifácove číslo je násobkom 36.
Súčasne je Bonifácove číslo deliteľné siedmich, teda je deliteľné 252 (čísla 36 a~7 sú nesúdeliteľné a~$36\cdot7=252$).

Trojciferné násobky čísla 252 sú
$$
252,\quad 504,\quad 756,
$$
a~to jsou možné Bonifácove čísla.
V~prvom z~týchto čísel sa opakuje číslica 2, druhé obsahuje číslicu 0.
Oba tieto prípady nevyhovujú zadaniu.
Teda Bonifácove číslo môže byť jedine 756 a ~Pankrácove číslo môže byť jedine $756:4=189$.

Doteraz
boli použité navzájom rôzne číslice od 1 do 9.
Neboli použité číslice 2, 3, 4, teda Servácove číslo môže byť jedno z nasledujúcich čísel:
$$
234,\quad 324,\quad 342,\quad 432,\quad 423,\quad 243.
$$
Každé z týchto čísel je medzi Pankrácovým a Bonifácovým číslom, deliteľné ôsmimi je však iba 432.
Servácove číslo môže byť jedine 432.
To spolu Pankrácovým a ~ Bonifácovým číslom vyhovuje všetkým podmienkam zo zadania.

Pankrác mal izbu číslo 189, Servác 432 a Bonifác 756.

\poznamka
Podmienkami deliteľnosti je možné nakladať rôzne.
Napr. Pankrácove číslo musí byť deliteľné siedmimi, súčasne je deliteľné deviatimi, takže je deliteľné 63.
}

{%%%%%   Z7-I-4
Postupne preveríme tri prípady podľa toho, ktorý nápis nebol pravdivý.
\begin{itemize}
  \item Predpokladajme, že pravdivý nebol prvý nápis (a zvyšné dva boli pravdivé).
  Potom by mince mala byť v nádobe s párnym číslom väčším ako 3 a súčasne menším ako 4.
  To nie je možné.
  \item Predpokladajme, že pravdivý nebol druhý nápis (a zvyšné dva boli pravdivé).
  Potom by mince mala byť v nádobe s nepárnym číslom menším ako 4.
  To je možné a~mince by v~takom prípade bola buď v~nádobe 1, alebo 3.
  \item Predpokladajme, že pravdivý nebol tretí nápis (a zvyšné dva boli pravdivé).
  Potom by mince mala byť v nádobe s nepárnym číslom väčším ako 3 a nie menším ako 4.
  To je možné a~mince by v~takom prípade bola v~nádobe 5.
\end{itemize}

Možné prípady sú posledné dva.
Pritom nádoba, v ktorej sa mince nachádza, je jednoznačne určená v treťom prípade, zatiaľ čo v druhom prípade nie.
Nepravdivý nápis je ten druhý.

\poznamka
Druhý a tretí nápis si odporujú, takže nemohli byť oba pravdivé.
Nepravdivý nápis bol len jeden, teda to bol buď druhý, alebo tretí; prvý nápis bol nutne pravdivý.
}

{%%%%%   Z7-I-5
Aby sa polkružnica dotýkala strán $AB$ a~$BC$, musí mať stred na osi uhla $ABC$.
Aby krajné body polkružnice ležali na strane $AC$, musí mať stred na tejto strane.

Stred hľadanej polkružnice je teda priesečníkom osi uhla $ABC$ a~strany $AC$.
Polomer je potom určený pätou kolmice zo stredu k~priamke $AB$ alebo k~priamke $BC$.
(Tieto úsečky sú zhodné práve preto, že stred leží na osi uhla $ABC$, viď zhodnosť trojuholníkov $BSP$ a~$BSP'$ na obrázku.)
\insp{z7-i-5a.eps}%

Konštrukcia:
\begin{itemize}
%\parindent=2\parindent
  \item $o =$ os uhla $ABC$,
  \item $S =$ prienik priamok $o$ a~$AC$,
  \item $P =$ päta kolmice z~bodu $S$ na~priamku $AB$ (či $BC$),
  \item kružnica so stredom $S$ a~polomerom $SP$,
  \item polkružnica vymedzená priamkou $AC$ a~bodom $P$.
\end{itemize}

\poznamka
Zadaný trojuholník je tupouhlý s~tupým uhlom pri~vrchole $B$ a~dotykové body polkružnice vskutku vychádzajú na stranách trojuholníka.
V~obecnom trojuholníku úloha riešenie mať nemusí.
Obmedzujúce sú práve zúženie na strany (necelé priamky) a polkružnice (necelé kružnice).
}

{%%%%%   Z7-I-6
Prehľad prvých 20 minút vyprážania vyzerá takto:
\bgroup
\thicksize=0pt
\thinsize=0pt
\def\ctr#1{\hfil\ #1\ \hfil}
$$
\begintable
|1|2|3|4|5|6|7|8|9|10|11|12|13|14|15|16|17|18|19|20\cr
\hfill Katka\ |||$+$|||$+$|||$+$|||$+$|||$+$|||$+$||\nr
Števo\ ||||$+$||||$+$||||$+$||||$+$||||$+$\nr
\hfill Lucifer\ |||||$-$|||||$-$|||||$=$|||||$=$%
\endtable
$$
\egroup
167 / 5,000
Tu $+$ označuje jednu palacinku na tanieri navyše, $-$ značí o~jednu palacinku menej pri úspešnom Luciferovom pokuse a ~$=$ značí žiadnu zmenu pri neúspešnom Luciferovom pokuse.

Najmenší časový úsek, na konci ktorého sa stretávajú všetci traja, trvá 60 minút (najmenší spoločný násobok časových intervalov Katky, Števa a Lucifera).
Počas tejto doby sa predchádzajúca schéma zopakuje trikrát. Teda Katka upečie 20 palaciniek, Števo upečie 15 palaciniek a Lucifer sa objaví 12-krát.
Lucifer sa stretáva s Katkou v minútach 15, 30, 45, 60, so Števom v minútach 20, 40, 60 (pričom v ~60 minúte sa stretáva s obomi) a ~ v týchto minútach nekradne.
Vo svojich zlodejských pokusoch je teda úspešný iba šesťkrát.
Celkom za 60 minút Katka so Števom upečú 35 palaciniek, ale na tanieri ich zostane 29.

Za päť hodín Katka so Števom upečú 175 palaciniek, na tanieri ich zostane 145.
Z úvodného prehľadu je zrejmé, že k chýbajúcim piatim palacinkám sa dopracujú po ďalších 12 minútach --- celkom ich upečúem, ale dve im Lucifer ukradne.

Katka so Števom musia upiecť 182 palaciniek a bude im to trvať 5 hodín a 12 minút.
}

{%%%%%   Z8-I-1
Počty známok jednotlivých chlapcov označíme počiatočnými písmenami ich mien.
Podľa zadania platí
$$
\begin{gathered}
	I+J+K+L=90, \\
	I-2 =J+2 =2K =\frac{L}2 .
\end{gathered}
$$

Hodnotu na druhom riadku označíme $Z$, pomocou nej vyjadríme ostatné neznáme,
$$
I=Z+2,\quad J=Z-2,\quad K=\frac{Z}2,\quad L=2Z,
$$
dosadíme do prvej rovnice a~dostávame $\frac92 Z=90$.
Teda $Z=20$ a~z~predchádzajúceho vyjadrenia máme
$$
I=22,\quad J=18,\quad K=10,\quad L=40.
$$

Ivan má 22 známek, Jaro 18, Karol 10 a~Ľuboš 40.
}

{%%%%%   Z8-I-2
Základne lichobežníka sú rovnobežné.
Strany hľadaných lichobežníkov buď patria stranám trojuholníka, alebo sú s nimi rovnobežné.
Rozborom možností zistíme, že žiadna strana žiadneho lichobežníka nemôže zaberať celú stranu trojuholníka a~že žiadna strana trojuholníka nemôže byť tvorená dvoma ramenami, ani dvoma základňami lichobežníkov.
Delenie trojuholníka na lichobežníky je dané jedným bodom vo vnútri trojuholníka a tromi rovnobežkami so stranami trojuholníka (pozri prvé z nižšie uvedených obrázkov).

Potrebujeme nájsť deliaci bod (spoločný bod lichobežníkov) tak, aby obsahy lichobežníkov boli rovnaké.
Na tento účel je vhodné využiť obvyklé delenie trojuholníka na menšie zhodné trojuholníky.
V~našom prípade pomáha rozdelenie na trojuholníky s~tretinovými stranami vzhľadom k~danému trojuholníku.
Takých trojuholníkov je deväť a~každý z~hľadaných lichobežníkov je zložený z troch (pozri druhý obrázok).
\inspinsp{z8-i-2a.eps}{z8-i-2b.eps}%

Konštrukcia rovnoramenného trojuholníka $ABC$ pre danú základňu a~výšku:
\begin{itemize}
  \item úsečka $AB$ dĺžky 12\,cm,
  \item $o =$ os úsečky $AB$ a~$S =$ stred úsečky $AB$,
  \item $C =$ bod na priamke $o$ vo vzdialenosti $|SC|=18$\,cm,
  \item trojuholník $ABC$.
\end{itemize}

Úsečka $SC$ je výškou rovnoramenného trojuholníka a ~ vyššie vyznačené priečky trojuholníka ju pretínajú v tretinách
(najmä $|SD|=\frac13|SC|=6$\,cm).
Konštrukcia deliaceho bodu a ~ lichobežníkov vyzerá takto:
\begin{itemize}
  \item $D =$ bod na úsečke $SC$ vo vzdialenosti $|SD|=6$\,cm,
  \item rovnobežky so stranami trojuholníka $ABC$ idúce bodom $D$,
  \item $E$, $F$, $G =$ priesečníky týchto rovnobežiek so stranami trojuholníka,
  \item lichobežníky $AGDF$, $BEDG$, $CFDE$.
\end{itemize}
~\insp{z8-i-2c.eps}%

\poznamky
Predstavené delenie je platné aj vo~všeobecnom trojuholníku a~deliaci bod $D$ je jeho ťažiskom.
Rovnoramennosť trojuholníka má za následok len to, že jedna z ťažníc je výškou a dva z troch lichobežníkov sú zhodné.

Inú konštrukciu bodu $D$ a~lichobežníkov je možné založiť na tretinovom delení strán trojuholníka $ABC$ a~vhodnom spájaní takto vzniknutých bodov (viď druhý z vyššie uvedených obrázkov).
}

{%%%%%   Z8-I-3
Možné čísla, ktoré po delení tromi dávajú zvyšok 1, resp. po delení šiestich zvyšok 2, sú
$$
\begin{aligned}
	a &= 1,\ 4,\ 7,\ 10,\ 13,\ \dots \\
	b &= 2,\ 8,\ 14,\ 20,\ 26,\ \dots
\end{aligned}
$$
Nezáporné rozdiely $a-b$ zoradené vzostupne sú
$$
a-b = 2,\ 5,\ 8,\ 11,\ \dots
$$
Susedné čísla v~tomto výpise sa líšia o~3 a~najmenšie číslo je 2, teda zvyšok po delení $a-b$ tromi je 2.
Opačný rozdiel $b-a$ potom po delení tromi dáva zvyšok 1.

Z~podmienky $a-b=d-c$ máme $c=d+b-a$, kde číslo $d$ je deliteľné tromi a~$b-a$ dáva po delení tromi zvyšok 1.
Preto tiež číslo $c$ dáva po delení troma zvyšok 1:
$$
c = 1,\ 4,\ 7, 10,\ 13,\ 16,\ \dots \tag{$*$}
$$
Možné zvyšky po delení čísla $c$ deviatimi sú 1, 4, 7.

\poznamky
Štvoríc čísel $a$, $b$, $c$, $d$ spĺňajúcich podmienky zo zadania je nekonečné množstvo a~všetky uvedené zvyšky po delení čísla $c$ deviatich naozaj môžu nastať.
Pre príklad stačí uvážiť čísla $a=4$ a~$b=2$ a~ postupnosť čísel $c$ ako v~($*$), pre ktorú zodpovedajúca postupnosť čísel $d$ je
$$
d = 3,\ 6,\ 9,\ 12,\ 15,\ 18,\ \dots
$$

Delenie so zvyškom je platné v~obore všetkých celých čísel (nie len tých nezáporných).
Teda napr. čísla, ktoré po delení tromi dávajú zvyšok 1, sú tiež
$$
a = -2, -5, -8, -11, -14, \dots
$$
To, že sme sa v~riešení úlohy obmedzili na nezáporné čísla, ničomu nevadí, lebo nás zaujímali hlavne zvyšky.

Predchádzajúce výpočty a~úvahy vedúce k~($*$) je možné nahradiť nasledujúcimi všeobecnými vyjadreniami.
Čísla $a$, $b$, $d$ sú podľa zadania tvaru
$$
a =3k+1,\quad b=6l+2,\quad d=3m,
$$
kde $k$, $l$, $m$ sú celé čísla.
Číslo $c$ je potom vyjadrené ako
$$
c =d+b-a=3(m+2l-k)+1.
$$
Vskutku číslo $c$ po delení tromi dáva zvyšok 1.
}

{%%%%%   Z8-I-4
V nasledujúcich úvahách budeme používať niekoľko vlastností pravidelného päťuholníka:
\begin{itemize}
  \item V~pravidelnom päťuholníku sú všetky strany navzájom zhodné a~rovnako tak všetky uhlopriečky.
  Teda akýkoľvek trojuholník tvorený tromi vrcholmi pravidelného päťuholníka je rovnoramenný.
  \item Pravidelný päťuholník je osovo súmerný podľa piatich rôznych osí.
  Pri každej z týchto súmerností sa vždy jedna strana a~nepriliehajúca uhlopriečka zobrazujú samé na seba, teda sú navzájom rovnobežné.
  \item Všeobecný päťuholník pozostáva z troch trojuholníkov, teda súčet veľkostí jeho vnútorných uhlov je $3\cdot180\st=540\st$.
  Pravidelný päťuholník má všetky vnútorné uhly zhodné, teda veľkosť vnútorného uhla pravidelného päťuholníka je $540\st:5=108\st$.
\end{itemize}

V~našom príklade ukážeme, že trojuholník $ABG$ je zhodný s~trojuholníkom $ABC$, teda hľadaný uhol nájdeme medzi uhlami vymedzenými stranami a~uhlopriečkami päťuholníka.
Začneme tým, že si uvedomíme niekoľko vzťahov medzi uhlami vyznačenými na prvom z nižšie uvedených obrázkov.

Trojuholník $ABD$ je rovnoramenný, teda uhly pri jeho~základni sú zhodné, $\alpha=\beta$.
Priamky $AD$ a~$BC$ sú rovnobežné, teda súhlasné uhly pri~vrcholoch $A$ a~$B$ sú zhodné, $\alpha=\gamma$.
Uhly $\beta$ a~$\delta$ sú vrcholové uhly s~vrcholom $B$, teda sú tiež zhodné, $\beta=\delta$.
Celkom tak platí, že všetky vyznačené uhly sú navzájom zhodné.

Priamka $CG$ je kolmá na~priamku $CF$, a~tá je rovnobežná s~priamkou $AB$.
Priamky $CG$ a~$AB$ sú teda kolmé.
To spolu s predchádzajúcim poznatkom ($\gamma=\delta$) znamená, že body $C$ a~$G$ sú osovo súmerné podľa priamky $AB$.
Preto aj trojuholníky $ABC$ a~$ABG$ sú osovo súmerné a~hľadaný uhol pri~vrchole $G$ je zhodný s~vnútorným uhlom trojuholníka $ABC$ pri~vrchole $C$.
To so značením ako na druhom obrázkov znamená, že $\varepsilon=\zeta$.
\inspinspmedzera{z8-i-4b.eps}{z8-i-4c.eps}{}%

Trojuholník $ABC$ je rovnoramenný, teda uhly pri~základni sú zhodné, $\zeta=\eta$.
Vnútorný uhol pri vrchole $B$ je vnútorným uhlom päťuholníka, teda jeho veľkosť je $108\st$.
Súčet vnútorných uhlov tohto trojuholníka je $2\zeta+108\st =180\st$, odkiaľ dostávame $\zeta=36\st$.

Veľkosť uhla $AGF$ je $36\st$.

\poznamky
Zhodnosť uhlov $\gamma$ a~$\delta$ je ekvivalentná zhodnosti susedných uhlov $ABC$ a~$ABG$, a~tú je možné zdôvodniť takto:
Uhol $ABG$ sa zhoduje s~uhlom $BAE$ (striedavé uhly vzhľadom k~rovnobežkám $AE\|BD$),
uhol $BAE$ sa zhoduje s~uhlom $ABC$ (vnútorné uhly pravidelného päťuholníka), teda
uhol $ABG$ sa zhoduje s~uhlom $ABC$.

Veľkosti spomínaných uhlov (a mnoho ďalších) je možné ľahko vyjadriť, aj keď k doriešeniu úlohy to nutné nie je.
Možnosti ich odvodenia sú rozmanité,
pozri tiež úlohu \Ulink{https://www.matematickaolympiada.sk/media/3643847/z73i-komentare.pdf}{Z8-I-3} v minuloročnom ročníku MO.
Pre predstavu niektoré hodnoty zhŕňame v~prvom z~nižšie uvedených obrázkov.

Bod $F$ nebol nijako podstatný, potrebovali sme iba rovnobežnosť priamok $CF$ a~$AB$.
Bod $F$ je vlastne priesečníkom uhlopriečok $BD$ a~$CE$.

So znalosťou Thalesovej vety, resp. vety opačnej možno zhodnosť úsečiek $BC$ a~$BG$ odvodiť takto:
Pretože $AB\|CE$, $AE\|BD$ a~strany $AB$, $AE$ sú zhodné, je štvoruholník $ABFE$ kosoštvorcom.
Všetky strany kosoštvorca sú zhodné a~zhodujú sa tiež so stranou $BC$,
najmä úsečky $BF$ a~$BC$ sú zhodné.
Pretože trojuholník $FCG$ je pravouhlý s~pravým uhlom pri~vrchole $C$, leží tento bod na (Thalesovej) kružnici s~priemerom $FG$.
Pretože $BF$ sa zhoduje s~$BC$, je bod $B$ stredom tejto kružnice a~úsečky $BF$, $BC$, $BG$ sú jej polomery, viď druhý obrázok.
\inspinsp{z8-i-4a.eps}{z8-i-4d.eps}%
}

{%%%%%   Z8-I-5
Najväčší spoločný deliteľ čísel $a$ a~$b$ označíme $D$.
Čísla $a$ a~$b$ sú potom tvaru
$$
a=A\cdot D,\quad b=B\cdot D,
$$
kde $A$ a~$B$ sú nesúdeliteľné čísla.
Pretože poradie čísel nie je dôležité, budeme v~ďalšom pre zjednodušenie predpokladať, že $a<b$ alebo $A<B$.

S~týmto značením je najmenší spoločný násobok čísel $a$ a~$b$ vyjadrený ako $A\cdot B\cdot D$.
Podiel najmenšieho spoločného násobku a~najväčšieho spoločného deliteľa čísel $a$ a~$b$ je 75, teda
$$
A\cdot B =75.
$$
Pretože $75=3\cdot5\cdot5$ a~čísla $A$ a~$B$ sú nesúdeliteľné, platia buď $A=1$ a~$B=75$, alebo $A=3$ a~$B=25 $.
V~každom z~týchto prípadov zistíme, pre ktoré hodnoty $D$ platí podmienka o~súčte $a+b$.

\smallskip
\hskip-1em$\bullet$\,\,\, %% itemize --> preteka radek (?)
Pre $A=1$ a~$B=75$ dostávame:
$$
\begintable
$D$\|$a$|$b$\|$100<a+b<200$\crthick
1\| 1| 75\|nie\cr
2\| 2|150\|áno\cr
3\| 3|225\|nie%
\endtable
$$

\hskip-1em$\bullet$\,\,\,
Pre $A=3$ a~$B=25$ dostávame:
$$
\begintable
$D$\|$a$|$b$\|$100<a+b<200$\crthick
1\| 3| 25\|nie\cr
2\| 6| 50\|nie\cr
3\| 9| 75\|nie\cr
4\|12|100\|áno\cr
5\|15|125\|áno\cr
6\|18|150\|áno\cr
7\|21|175\|áno\cr
8\|24|200\|nie%
\endtable
$$

S~rastúcim $D$ sa zväčšuje aj~súčet $a+b$, teda nie je potrebné ďalšie skúšanie.
Všetky možné dvojice čísel $a$, $b$ s uvedenými vlastnosťami (až na poradie) sú:
$$
2, 150; \quad 12, 100; \quad 15, 125; \quad 18, 150; \quad 21, 175.
$$

\poznamka
Ak $N$ označíme najmenší spoločný násobok čísel $a$ a~$b$, potom z~úvodného rozboru máme $N=A\cdot B\cdot D$, a~teda
$$
N\cdot D =(A\cdot D)\cdot (B\cdot D) =a\cdot b.
$$
Vyjadrené slovami: súčin najmenšieho spoločného násobku a~najväčšieho spoločného deliteľa dvoch čísel je rovný súčinu týchto čísel.
S týmto postrehom je možné pracovať od začiatku a zjednodušiť niektoré značenia.
}

{%%%%%   Z8-I-6
Tri najťažšie ryby zodpovedajú 35\,\% váhy celého úlovku, zvyšné ryby zodpovedajú 65\,\%.
Päť trinástin tohto zvyšku tvorí štvrtinu ($\frac5{13}\cdot\frac{65}{100} =\frac14$), teda tri najľahšie ryby zodpovedajú 25\,\% váhy celého úlovku.
Zvyšný neznámy počet rýb tak zodpovedá 40\,\% váhy celého úlovku ($100-35-25=40$).

Pretože zvyšné ryby vážia viac ako tri najťažšie, boli aspoň štyri.
Pretože zvyšné ryby vážia menej ako dvojnásobok váhy troch najľahších, bolo ich nanajvýš päť.
Ďalej ryba zo skupiny troch najťažších váži priemerne $11{,}\overline{6}\,\%$ váhy celého úlovku ($35:3=11{,}\overline{6}$) a~ryba zo skupiny troch najľahších váži priemerne $8{,}\overline{3}\,\%$ váhy celého úlovku ($25:3=8{,}\overline{3}$).
S týmito poznatkami rozhodneme o ~počte zvyšných rýb:
\begin{itemize}
  \item Keby zvyšných rýb bolo päť, potom by jedna vážila priemerne $8\,\%$ váhy celého úlovku ($40:5=8$), čo je menej ako priemerná váha najľahších rýb.
  To nie je možné.
  \item Keby zvyšné ryby boli štyri, potom by jedna vážila priemerne $10\,\%$ váhy celého úlovku ($40:4=10$), čo je medzi priemernými váhami najťažsích a~najľahších rýb.
  To je možné.
\end{itemize}

Zvyšné ryby boli štyri.
Pán Šťuka chytil desať rýb.

\poznamka
Ak označíme $n$ celkový počet rýb, potom vzťahy medzi priemernými váhami troch najťažších, $n-6$ zvyšných a~troch najľahších rýb sú:
$$
\frac{35}{3} \ge \frac{40}{n-6} \ge \frac{25}{3} .
$$
To po úpravách vedie k~nerovnostiam $n \ge \frac{330}{35}$ a~$n \le \frac{270}{25}$.
Jediné vyhovujúce prirodzené číslo je $n=10$.
}

{%%%%%   Z9-I-1
Hľadáme celé čísla $x$, $y$ vyhovujúce podmienkam
$$
3x+5y=16,\quad x+y=p ,
$$
kde $p$ je neznáme prvočíslo.
Z~druhej rovnice vyjadríme $y=p-x$ a~dosadíme do prvej rovnice.
Po úpravách dostávame:
$$
\begin{aligned}
	3x+5(p-x) &=16, \\
	5p-2x &=16, \\
	5p &=2(8+x).
\end{aligned}
$$
Na pravej strane poslednej upravenej rovnice je párne číslo, teda $p$ musí byť párne prvočíslo alebo $p=2$.
Odtiaľ po dosadení dostávame
$$
5=8+x,\quad y=2-x.
$$
Jediná vyhovujúca dvojica čísel je $x=-3$ a~$y=5$.

\ineriesenie
Číslo $3x+5y=16$ je párne, teda $x$ a~$y$ musí mať rovnakú paritu (inak by uvedená kombinácia bola nepárna).
Preto súčet $x+y$ je párny a ~ jediné párne prvočíslo je 2.
Teda $p=2$ a~dostávame sústavu rovníc:
$$
3x+5y=16,\quad x+y=2 .
$$
Obvyklými úpravami (ako napr. pri~predchádzajúcom riešení) zistíme, že jediným riešením tejto sústavy je dvojica čísel $x=-3$ a~$y=5$.

\ineriesenie
Začneme tým, že popíšeme všetky celočíselné riešenia rovnice $3x+5y=16$ a potom overíme druhú podmienku.
Postupne pre násobky 5 vyjadríme rozdiel od 16 a overíme deliteľnosť tromi:
$$
\begintable
$5y$\|\dots|$-5$|0|5|10|\dots\cr
$16-5y$\|\dots|21|16|11|6|\dots\crthick
děl. 3\|\dots|áno|nie|nie|áno|\dots%
\endtable
$$
Prvému vyhovujúcemu prípadu v~tabuľke zodpovedá dvojica $x_1=7$ a~$y_1=-1$, druhému dvojica $x_2=2$ a~$y_2=2$.
Rozdiely medzi týmito dvoma riešeniami
(rovnako ako medzi akýmikoľvek dvoma susednými riešeniami)
sú $x_1-x_2=5$ a~$y_1-y_2=-3$.
Všetky celočíselné riešenia rovnice $3x+5y=16$ sú tvaru
$$
x =2+5k,\quad y =2-3k, \tag{$*$}
$$
kde $k$ je celé číslo.
Pomocou tohto medzivýsledku vyjadríme súčet,
$$
x+y =4+2k =2(2+k),
$$
čo znamená, že $x+y$ je párne číslo.
Súčasne $x+y$ má byť prvočíslo a~jediné párne prvočíslo je 2.
Teda $k=-1$ a~po dosadení do ($*$) dostávame jedinú vyhovujúcu dvojicu čísel $x=-3$ a~$y=5$.
}

{%%%%%   Z9-I-2
Označíme veľkosť hrany podstavy $a$, veľkosť výšky hranola $v$ a~veľkosť jeho telesovej uhlopriečky $u$ (všetko v~cm).
Obsah podstavy je potom vyjadrený ako $a^2$, objem hranola ako $a^2v$ a~obsah plášťa ako $4av$.
Zo zadania máme vzťahy
$$
a^2v =864,\quad 4av =2a^2,
$$
z~ktorých vyjadríme $a$, $v$ a~následne dopočítame $u$.

Po krátení z~druhého vzťahu (veľkosť $a$ je nenulová) plynie $a=2v$ a~dosadením do prvého vzťahu dostávame $4v^3=864$.
Teda
$$
v=\root 3 \of {216} =6,\quad
a=2\cdot6=12.
$$

Telesová uhlopriečka hranola je preponou pravouhlého trojuholníka, ktorého jedna odvesna je uhlopriečkou podstavy a druhá výškou hranola.
Dvojitým použitím Pytagorovej vety dostávame
$$
u =\sqrt{2a^2+v^2} =\sqrt{324} =18.
$$

Telesová uhlopriečka hranola meria 18\,cm.
}

{%%%%%   Z9-I-3
Aby sme mohli rozdeľovať do piatich množín, musí byť $n$ aspoň päť.

Pre $n=5$ je rozdelenie jediné možné:
$$
\{1\},\ \{2\},\ \{3\},\ \{4\},\ \{5\}.
$$

Pre $n=6$ je jedno z~možných rozdelení toto:
$$
\{1,2\},\ \{3\},\ \{4\},\ \{5\},\ \{6\}.
$$

Pre rastúce $n$ môžeme skúšať hľadať rozdelenie s požadovanými vlastnosťami.
Budeme úspešní až po $n=11$, kde jedno z~možných rozdelení je toto:
$$
\{1,2\},\ \{3,4\},\ \{5,6\},\ \{7,8\},\ \{9,10,11\}.
$$

Pre $n\ge 12$ je v~danej množine aspoň šesť párnych čísel 2, 4, 6, 8, 10, 12, ktoré máme rozdeliť do piatich podmnožín.
Teda aspoň v jednej podmnožine sa stretnú aspoň dve párne čísla.
Každé dve párne čísla však majú spoločného deliteľa, preto rozdelenie s~požadovanými vlastnosťami pre žiadne $n\ge 12$ nie je možné.

Najväčšie možné $n$ je 11.
}

{%%%%%   Z9-I-4
Po pripočítaní jednociferného čísla sa má ciferný súčet podstatne zmenšiť.
To sa môže stať v~čísle s~veľa deviatkami za sebou, ktoré sa zmenia na nuly.
Napr. číslo 9994 s ciferným súčtom 31 sa po pripočítaní 6 zmení na číslo 10000 s ciferným súčtom 1.

V~našom prípade sa má ciferný súčet zmenšiť o~$2024-74 =1950$.
Pritom platí $1950=216\cdot9 +6$, teda budeme potrebovať 216 deviatok.
Rozdiel 1950 v~ciferných súčtoch možno dosiahnuť tak, že k~číslu pozostávajúceho z~216 deviatok a~sedmičky (ktoré má ciferný súčet $216\cdot9+7 =1951$) pripočítame 3, čím dostaneme číslo pozostávajúce z~jedničky a~217 nul (ktoré má ciferný súčet 1).

Teraz stačí predchádzajúci nápad upraviť tak, aby pôvodné číslo malo ciferný súčet 2024.
Pretože $2024 =1951+73$, stačí zľava pridať ľubovoľnú postupnosť číslic so súčtom 73 a~jednu nulu.
Vyhovujúcou odpoveďou môže byť napr. číslo pozostávajúce zo 73 jednotiek, jednej nuly, 216 deviatok a~sedmičky:
$$
\begin{aligned}
&1111111111111111111111111111111111111111111111111111111111111111111111111\text{-}\\
&0999999999999999999999999999999999999999999999999999999999999999999999999\text{-}\\
&9999999999999999999999999999999999999999999999999999999999999999999999999\text{-}\\
&999999999999999999999999999999999999999999999999999999999999999999999997
\end{aligned}
$$
}

{%%%%%   Z9-I-5
Vďaka rovnobežnosti $AD\|EF$ sú trojuholníky $CAD$ a~$CEF$ podobné a~hľadaný pomer úsečiek zodpovedá koeficientu tejto podobnosti.
Vďaka rovnobežnosti $AB\|ED$ sú tiež trojuholníky $CAB$ a~$CED$ podobné.
Koeficient podobnosti pre prvú aj pre druhú dvojicu trojuholníkov je rovnaký, pretože je určený tým istým bodom $E$ na strane $AC$.
Tento pomer odvodíme z~daného pomeru strán trojuholníka $ABC$.

Začneme tým, že si uvedomíme niekoľko vzťahov medzi uhlami vyznačenými na nasledujúcom obrázku:
\insp{z9-i-5a.eps}%

Priamka $AD$ je osou uhla $BAC$, teda susedné uhly $\alpha$ a~$\beta$ sú zhodné.
Priamky $AB$ a~$ED$ sú rovnobežné, teda striedavé uhly $\alpha$ a~$\gamma$ pri~vrcholoch $A$ a~$D$ sú zhodné.
Celkovo platí, že uhly $\beta$ a~$\gamma$ sú zhodné.
Teda trojuholník $ADE$ je rovnoramenný so zhodnými ramenami $AE$ a~$ED$.

Trojuholníky $CAB$ a~$CED$ sú podobné, teda zodpovedajúce si pomery strán sú rovnaké.
Najmä $|ED|:|EC| =|AB|:|AC|$, a~tento pomer je podľa zadania $2:1$.
Dohromady s~predchádzajúcim poznatkom ($|AE|=|ED|$) dostávame $|AE|:|EC| =2:1$ alebo $|AC|:|EC| =3:1$.
To je hľadaný pomer podobnosti trojuholníkov $CAD$ a~$CEF$.

Úsečky $AD$ a~$EF$ sú v~pomere $3:1$.

\poznamky
Všeobecne platí, že os vnútorného uhla trojuholníka delí protiľahlú stranu v rovnakom pomere, v akom sú priľahlé strany.
V~našom prípade to znamená $|BD|:|DC| =|BA|:|AC|$ a podobne (zo vzájomnej podobnosti) $|DF|:|FC| =|DE|:|EC|$.
Tento pomer však poznáme zo zadania, tj $|DF|:|FC|=2:1$ alebo $|DC|:|FC| =3:1$.
To je hľadaný pomer podobnosti trojuholníkov $CAD$ a~$CEF$, a teda i~úseček $AD$ a~$EF$.

Všetky spomínané dvojice podobných trojuholníkov sú rovnoľahlé so stredom v~bode $C$.
Hľadaný pomer veľkosti úsečiek $AD$ a~$EF$ je koeficientom tejto rovnoľahlosti, najmä platí $|AD|:|EF|=|BC|:|DC|$.
To, že tento pomer je $3:1$, plynie z interpretácie bodu $D$ ako ťažiska vo vhodnom trojuholníku:
\insp{z9-i-5b.eps}%

Tu je bod $G$ doplnený ako bod súmerný s~bodom $A$ podľa stredu $C$.
Bod $C$ je stredom úsečky $AG$, teda priamka $BC$ je ťažnicou trojuholníka $ABG$.
Trojuholník $ABG$ je rovnoramenný a~priamka $AD$ je osou uhla vymedzeného ramenami $AB$ a~$AG$, teda to je tiež ťažnica.
Preto je bod $D$ ťažiskom trojuholníka $ABG$.
}

{%%%%%   Z9-I-6
Druhé míňanie prebehlo bližšie ku Kaprovmu brehu ako prvé míňanie k Pstruhovmu brehu.
Teda Kapor plával rýchlejšie ako Pstruh, a preto zasúžene vyhral.

Pri prvom míňaní mali Pstruh a Kapor v súčte zaplávanú jednu dĺžku bazéna, pri druhom míňaní mali v súčte tri dĺžky bazéna.
Ak označíme $t$ čas prvého míňania, potom v~čase $2t$ mal Kapor otočku za sebou, zatiaľ čo Pstruh pred sebou, a~v~čase $3t$ sa míňali druhýkrát.

Pstruh za čas $t$ zaplával 8 metrov, teda za čas $3t$ zaplával 24 metrov.
Súčasne v čase $3t$ preplával celý bazén a ďalších 5 metrov po otočku.
Bazén mal na dĺžku 19 metrov ($24-5=19$).

\poznamky
Pre kontrolu môžeme vyjadriť vzdialenosť medzi miestami míňania ako $19-8-5 =6$\,(m).
Kapor za časový interval dĺžky $t$ plával $5+6=11$\,(m).
Medzi časmi $t$ a~$2t$ doplával na koniec bazéna, otočil sa a~pridal $11-8=3$\,(m).
Medzi časmi $2t$ a~$3t$ plával ďalších 11\,m a~na druhý koniec bazéna mu chýbalo $19-3-11=5$\,(m).
To súhlasí s~údajom zo zadania.

Prikladáme pokus o ~ znázornenie celej situácie:
\insp{z9-i-6a.eps}%

\ineriesenie
K~dĺžke bazéna sa dá dopočítať pomocou neznámej $x$, ktorá značí vzdialenosť v~metroch medzi miestami prvého a~druhého míňania sa:

Z predchádzajúceho riešenia vieme, že vzdialenosť, ktorú plával každý z plavcov od prvého míňania po druhé míňanie, je dvojnásobkom vzdialenosti, ktorú plával od štartu po prvé míňanie.
Ak vyjadríme tieto vzdialenosti pre Kapra, dostávame
$$
x+10 =2\cdot8,
$$
a~teda $x=6$.
Bazén mal na dĺžku 19 metrov ($8+6+5 =19$).

\poznamky
Pre kontrolu môžeme vyjadriť uvedené vzdialenosti pre Kapra:
dostávame rovnicu
$$
16+x =2(5+x)
$$
s tým istým riešením $x=6$.

Predchádzajúce vzťahy pre Pstruha a ~Kapra možno súhrnne zapísať takto:
$$
\frac{x+10}8 =\frac{16+x}{5+x} =2 .
$$
Bez úvodného postrehu o ~pomeroch uplávaných vzdialeností medzi štartom a~miestami míňania sa máme len rovnicu
$$
\frac{x+10}8 =\frac{16+x}{5+x} .
$$
Tá po úpravách vedie ku kvadratickej rovnici
$$
x^2 +7x -78 =0,
$$
ktorá má korene $x=6$ a~$x=-13$.
Riešeniu našej úlohy zodpovedá kladný koreň.
}

{%%%%%   Z4-II-1
...}

{%%%%%   Z4-II-2
...}

{%%%%%   Z4-II-3
...}

{%%%%%   Z5-II-1
Danka každý deň zjedla dve, tri alebo šesť čerešní.
Za sedem dní ich zjedla dokopy 19. Pomocou siedmich sčítancov 2, 3 alebo 6 možno číslo 19 (až na poradie sčítancov) vyjadriť len dvomi spôsobmi:

\smallskip
  \item{$\bullet$} $6+3+2+2+2+2+2=19$,
  \item{$\bullet$} $3+3+3+3+3+2+2=19$.

\smallskip\noindent
V prvom prípade by Danka zjedla jedenkrát šesť čerešní, jedenkrát tri čerešne a päťkrát dve čerešne.
To znamená, že v prvom prípade by Janka zjedla jedno jablko, päťkrát hrušku a jeden deň by nezjedla žiadne ovocie.

V druhom prípade by Danka zjedla päťkrát tri čerešne a ~ dvakrát dve čerešne.
To znamená, že druhom prípade Janka by zjedla päťkrát jablko a dvakrát hrušku.

Janka teda za daný týždeň zjedla buď 1 jablko a~5 hrušiek, alebo 5 jabĺk a~2 hrušky.

\hodnotenie
3~body za vyjádrenie súčtu 19 pomocou siedmich sčítancov 2, 3 a~6;
3~body za vyvodenie záverov o počte zjedených jabĺk a hrušiek za daný týždeň.

\endhodnotenie
}

{%%%%%   Z5-II-2
Súčet uvedených vzdialeností medzi dedinami zodpovedá dvojnásobku dĺžky okružnej cesty (ako vidno na obrázku):
\insp{z5-ii-2a.eps}%

Součet uvedených vzdialeností je $10+15+16 =41$\,(km).

Dĺžka okružnej cesty je polovica tohto súčtu, tedy 20\,km a 500\,m.

\hodnotenie
3~body za súčet vzdialeností (s komentárom alebo schematickým obrázkom);
3~body za výsledok.
\endhodnotenie
}

{%%%%%   Z5-II-3
Počítajme obvod útvaru ako príspevky jednotlivých štvorcov. Krajné štvorce útvaru prispievajú do obvodu tromi stranami, všetky ostatné štvorce prispievajú dvoma stranami.
Krajné štvorce sú 2, ostatných je 2023.
Obvod útvaru je
$$
2\cdot3 + 2023\cdot2 = 4052\,(\Cm).
$$

\ineriesenie
Predstavme si postupné doplňovanie útvaru zľava doprava.
Útvar pozostávajúci z~jedného štvorca má obvod 4\,cm,
útvar pozostávajúci z dvoch štvorcov má obvod 6\,cm,
útvar pozostávajúci z troch štvorcov má obvod 8\,cm atď.

Priložením každého štvorca sa obvod útvaru zväčší o ~2\,cm (o tri strany nového štvorca navyše, o jednu spoločnú stranu menej).
Obvod útvaru je
$$
1\cdot4 + 2024\cdot2 = 4052\,(\Cm).
$$

\poznamka
Útvar si možno predstaviť aj tak, že k prvému štvorcu je priložených 506 dielov nasledujúceho tvaru ($1+506\cdot4=2025$):
\insp{z5-ii-3a.eps}%

Priložením každého takého dielu sa obvod útvaru zväčší o ~8\,cm (o deväť strán nového dielu navyše, o jednu spoločnú stranu menej).
Vyjadrenie obvodu útvaru potom vyzerá takto:
$$
1\cdot4 + 506\cdot8 = 4052\,(\Cm).
$$

\hodnotenie
3~body za čiastkové pozorovania; % o~přírůstcích obvodů;
3~body za zovšeobecnenie a výsledok.
\endhodnotenie
}

{%%%%%   Z6-II-1
Jednociferné delitele čísla 252 sú 1, 2, 3, 4, 6, 7 a ~9. Teda len tieto číslice sa môžu nachádzať ako cifry v zápise akýchkoľvek pekných čísel. 

Skúsme nájsť dve najmenšie pekné čísla v tvare $71{*}*$, kde
cifry na miestach hviezdičiek môžu byť len niektoré z~vyššie uvedených deliteľov.
Vieme, že $252:7=36$, teda súčin zostávajúcich dvoch cifier musí byť 36.
Cifry na miestach hviezdičiek môžu byť len 4 a~9, alebo 6 a~6.


Dve najmenšie pekné čísla sú teda 7149 a 7166.

\hodnotenie
2~body za jednociferné delitele čísla 252;
2~body za rozbor možností, príp. úplnosť skúšania;
po 1~bode za každé z~hľadaných čísel.

Pri vynechaní deliteľa 1 vychádzajú najmenšie dve čísla 7229 a ~7236.
Takéto riešenie hodnoťte najviac 3~bodmi podľa kvality komentára.
\endhodnotenie
}

{%%%%%   Z6-II-2
Štvoruholník $BCDE$ je obdĺžnik, teda jeho protiľahlé strany sú zhodné,
$|BE|=|CD|$ a~$|BC|=|DE|$.
Rozdiel dĺžok dvoch popísaných prechádzok zodpovedá dvojnásobku strany $BC$:
$$
\begin{gathered}
  (|AB|+|BC|+|CD|+|DE|+|EA|) - (|AB|+|BE|+|EA|) =\\
  = |BC|+|DE| = 2|BC| .
\end{gathered}
$$
To je podľa zadania 24\,dm, teda $|BC|=12$\,dm.

Obsah obdĺžnika $BCDE$ je
$|BC|\cdot|BE| = 360\,\text{dm}^2$.
Teda $|BE| =360/12 =30\,(\text{dm})$.

Trojuholník $ABE$ je rovnostranný, teda $|AE|=|BE|$.
Cestička medzi úkrytmi Adama a ~Erika je dlhá 30\,dm.

\hodnotenie
2~body za dĺžku strany $BC$ alebo $DE$;
2~body za dĺžku druhej strany obdĺžnika;
1~bod za dĺžku strany $AE$;
1~bod za kvalitu komentára.
\endhodnotenie
}

{%%%%%   Z6-II-3
V~utorok Jakub prečítal $\frac13$ všetkých strán.
Zostávalo mu $1-\frac13 =\frac{2}{3}$ všetkých strán.

V stredu prečítal $\frac37$ zvyšných strán, tj. $\frac37\cdot\frac23 =\frac27$ všetkých strán.
Zostávalo mu $\frac23-\frac27 =\frac{8}{21}$ všetkých strán.

Vo štvrtok prečítal všetky zostávajúce strany, a ~tých bolo 32.
Kniha mala $\frac{21}8\cdot32 = 84$ strán.

\ineriesenie
Skúmajma počty prečítaných strán od posledného dňa. Vieme, že 32 strán, ktoré Jakub prečítal vo štvrtok, je počet strán, ktoré mu zostali zo stredy.
To zodpovedá $1-\frac37=\frac47$ strán, ktoré mal na čítanie v stredu. Týchto 32 strán sú teda $\frac47$ všetkých strán zo stredy, teda $\frac17$ z tohto počtu je 8 strán. Teda v~stredu mu zostalo na čítanie $7*8=56$ strán.

Ak sa teraz pozrieme na stredu, tak 56 strán, ktoré mal Jakub na čítanie v stredu, je počet strán, ktoré mu zostali z utorka.
To zodpovedá $1-\frac13=\frac23$ strán, ktoré mal na čítanie v~utorok. Týchto 56 strán sú teda $\frac23$ všetkých strán, teda $\frac13$ všetkých strán knihy je 28 strán.
Kniha, mala teda dokopy $3*26=84$ strán.

\poznamka
Počet strán prečítaných v~utorok, resp. v stredu je vyjadrený pomocou tretín, resp. sedmín a~obdobne je vyjadrený počet zvyšných strán v~jednotlivých dňoch.
Na štvrtok pripadlo 32 strán.
Najmenšie číslo, ktoré je násobkom troch a siedmich, je 21.
Pre násobky 21 väčšie ako 32 je možné postupne dopĺňať, koľko strán Jakub prečítal, koľko strán zostalo, a~overiť, či na štvrtok vychádza 32 strán.
Pre knihu so~42, resp. s 63 stranami, na štvrtok vychádza 16, resp. 24 strán, čo je málo.
Pre knihu s~84 stranami, na štvrtok vychádza 32 strán:
$$
\begintable
\|na prečítanie|prečítal|zostalo\crthick
utorok\|84|28|56\cr
streda\|56|24|32\cr
štvrtok\|32|32|0%
\endtable
$$
Väčšie čísla ako 84 nemá zmysel skúšať, lebo potom by počet strán zostávajúcich na prečítanie vo štvrtok bol väčší.

\hodnotenie
V prípade prvého riešenia: 1~bod za vyjadrenie počtu zostávajúcich strán vo štvrtok,  2~body za vyjadrenie počtu zostávajúcich strán v stredu, 1 bod za sformulovanie rovnice, 2 body za výsledok
V prípade druhého riešenia: 3 body za dopočítanie strán zo stredy, 2 body za dopočítanie strán z utorka, 1 bod za kvalitu komentára. 
Vyššie uvedená tabuľka bez ďalšieho komentára má hodnotu 3 bodov.
\endhodnotenie
}

{%%%%%   Z7-II-1
O jednu minútu rýchlejšie zdobenie každého perníčka by viedlo k celkovej úspore 48 minút.
Mamička si teda na zdobenie nachystala 48 perníčkov.

V zrýchlenom tempe by za 48 minút ozdobila 12 perníčkov, teda zdobenie jedného perníčka by jej trvalo $48/12=4$ minúty.
V~pôvodnom tempe jej zdobenie jedného perníčka trvá $4+1=5$ minút.
Zdobenie 48 perníčkov bude mamičke trvať $48\cdot5=240$ minút.

\hodnotenie
2~body za celkový počet perníčkov;
1~bod za čas zdobenia v zrýchlenom tempe;
1~bod za čas zdobenia v pôvodnom tempe;
2~body za celkový čas zdobenia.
\endhodnotenie
}

{%%%%%   Z7-II-2
Označme počet psov $p$ a počet mačiek $m$. Keďže počet psov je násobkom ôsmich, tak pre násobky ôsmich neprevyšujúce 60 zistime, aký by bol počet mačiek a overme deliteľnosť tromi. V~kladnom prípade zistime počet zvierat starých rok alebo viac:
$$
\begintable
$p$\|8|16|24|32|40|48|56\cr
$m=60-p$\|52|44|36|28|20|12|4\cr
$m$ del. 3\|nie|nie|áno|nie|nie|áno|nie\crthick
$\frac23m+\frac58p$\|||39|||38|%
\endtable
$$
Ako vidíme v tabuľke, 39 zvierat starých rok alebo viac vychádza pre hodnoty v treťom stĺpci tabuľky.
V~útulku je teda 36 mačiek a~24 psov.

\poznamky
Počty mačiek a~psov môžeme vyjadriť ako $m=3l$ a~$p=8q$, kde $l$ a~$q$ sú prirodzené čísla.
Podmienky zo zadania je možné vyjadriť pomocou rovníc:
$$
3l+8q=60,\quad 2l+5q=39. \tag{$*$}
$$
Podobným spôsobom ako vyššie je možné určiť riešenie každej rovnice.
Pre $q=1,2,3,\ldots$ vyjadríme rozdiel $60-8q$, resp. $39-5q$, overíme deliteľnosť tromi, resp. dvomi a~v~kladnom prípade určíme $l$:
$$
l =\frac{60-8q}{3},\quad\text{resp.}\quad l=\frac{39-5q}{2} . \tag{$**$}
$$
Jediná dvojica vyhovujúca obom podmienkam je $q=3$ a~$l=12$.
To zodpovedá $p=24$ a~$m=36$.

K tomu istému výsledku je možné dospieť riešením sústavy rovníc ($*$).
Zo~vzťahov ($**$) pre neznámu $l$ dostávame:
$$
\begin{aligned}
  \frac{60-8q}{3} &= \frac{39-5q}{2}, \\
  120-16q &= 117-15q, \\
  3 &= q.
\end{aligned}
$$
Dosadením $q=3$ do ($**$) máme $l=12$.

\hodnotenie
2~body za postrehy s deliteľnosťou troma a ôsmimi, príp. formuláciu pomocou rovníc;
2~body za dôsledné skúšanie možností, príp. riešenie rovníc;
2~body za výsledok.
\endhodnotenie
}

{%%%%%   Z7-II-3
Všetky rovnobežníky majú navzájom zhodné vnútorné uhly.
Porovnávanie obsahov preto vedie k porovnávaniu dĺžok ich strán.
Kvôli ľahšiemu vyjadrovaniu si vrcholy rovnobežníkov označíme:
\insp{z7-ii-3b.eps}%

Rovnobežník $HIRS$ je rozdelený na dva rovnakoploché, a preto zhodné rovnobežníky.
Každý z nich má rovnaký obsah ako rovnobežník $AHSD$, s ktorým má rovnobežník $HIRS$ spoločnú stranu.
Preto je obsah $HIRS$ dvojnásobný vzhľadom k~obsahu $AHSD$ a~platí $|HI|=|TU|=2|AH|$ alebo
$$
|AH|=\frac12|TU|=1\,\cm.
$$

Rovnobežníky $AHSD$, $IJQR$, $KLOP$ a~$MBCN$ majú rovnaké obsahy a~dlhšiu stranu rovnakej dĺžky.
Preto sú rovnobežníky navzájom  zhodné ~platí
$$
|AH|=|IJ|=|KL|=|MB|=1\,\cm.
$$

Rovnobežník $JKPQ$ je rozdelený na tri rovnakoploché, a preto zhodné rovnobežníky.
Každý z nich má rovnaký obsah ako rovnobežník $IJQR$, s ktorým má rovnobežník $JKPQ$ spoločnú stranu.
Preto je obsah $JKPQ$ trojnásobný vzhľadom k~obsahu $IJQR$ a~platí
$$
|JK|=3|IJ|=3\,\cm.
$$
Podobná úvaha ako v~prvom odstavci pod obrázkom dáva
$$
|LM|=|HI|=2\,\cm.
$$

Dĺžka strany kosoštvorca je
$$
|AB| = 4|AH|+2|HI|+|JK| = 11\,\cm.
$$
Obvod kosoštvorca je štvornásobkom dĺžky strany, tj. 44\,cm.

\poznamka
Znázornenie pomerov medzi stranami menších rovnobežníkov môže vyzerať takto
(najmenšie úsečky na stranách $AB$, resp. $CD$ sú navzájom zhodné):
\insp{z7-ii-3a.eps}%

\hodnotenie
2~body za určenie dĺžky $1\,\cm$ ktoréhokoľvek z úsekov zhodného s AH;
1~bod za určenie dĺžky  $2\,\cm$ ktoréhokoľvek z úsekov zhodného s HI;
2~body za určenie dĺžky $3\,\cm$ úseku JK alebo QP;
1~bod za prenesenie všetkých dĺžok na stranu AB, resp. DC a obvod kosoštvorca.
\endhodnotenie
}

{%%%%%   Z8-II-1
Celkový počet dielov pri jednotlivých pomeroch je 8, 18 a ~30.
Dĺžka povrazu v~centimetroch musí byť násobkom každého z týchto čísel.

Najmenší spoločný násobok čísel 8, 18 a ~ 30 je 360.
Ďalšie násobky 360 sú 720, 1080 atď.
Najmenšia možná dĺžka povrazu je 1080\,cm.

\poznamka
Namiesto najmenšieho spoločného násobku daných čísel možno postupne hľadať násobky napr. čísla 30 väčšie ako 1000 (1020, 1050, 1080, ...) a overovať ich deliteľnosť zvyšnými dvomi číslami 8 a~18.
Najmenšie také číslo je 1080. 
\hodnotenie
2~body za možnú dĺžku povrazu ako násobok čísel 8, 18 a~30;
2~body za najmenší spoločný násobok týchto čísel alebo analogický postup;
2~body za výsledok.
\endhodnotenie
}

{%%%%%   Z8-II-2
Rozdiel medzi súčtom dvoch čísel a ich aritmetickým priemerom je rovný práve tomuto priemeru.
Pokiaľ označíme prvé Monikine číslo $a$ a~druhé $b$, potom podľa zadania platí
$$
\frac{a+b}{2}=4{,}1, \quad \frac{3a+b}{2}=8{,}4
$$
čiže
$$
a+b=8{,}2, \quad 3a+b=16{,}8.
$$

Posledné dve vyjadrenia sa líšia o~dvojnásobok prvého čísla, teda $2a =8{,}6$.
Z~toho vyplýva, že $a=4{,}3$, čo po dosadení do ktorejkoľvek z predchádzajúcich rovníc dáva $b=3{,}9$.
Monika si vybrala čísla 4,3 a ~3,9.

\hodnotenie
2~body za vyjadrenie pomocou neznámych;
2~body za úpravy;
2~body za výsledok.
\endhodnotenie
}

{%%%%%   Z8-II-3
Vďaka zhodnosti úsečiek je v~útvare niekoľko rovnoramenných trojuholníkov.
V~každom rovnoramennom trojuholníku sú vnútorné uhly pri~základni zhodné.
V~nasledujúcom obrázku sú zhodné úsečky zvýraznené rovnakými farbami a~zhodné uhly označené rovnakými symbolmi:
\insp{z8-ii-3a.eps}%

Napríklad, rovnosť uhlov $ECB$ a $EBC$ vyplýva z rovnoramennosti trojuholníka $EBC$. 

Uhol $EDC$ je vonkajším uhlom trojuholníka $EDB$, a~ten je rovný súčtu nepriliehajúcich vnútorných uhlov, tzn. $\varepsilon=2\gamma$.
Súčet vnútorných uhlov v~trojuholníku $EDC$ je $180\st$, tzn.
$2\varepsilon+\gamma = 180\st$.
Teda dostávame $5\gamma=180\st$, čiže $\gamma=36\st$.
Veľkosť uhla $ACB$ je $36\st$.

% Z předchozího víme, že $\gamma=36\st$ a $\varepsilon=72\st$.
% tedy $\alpha=180\st-\gamma-\varepsilon=72\st$.
Uhol $CEA$ je priamy a~jeho veľkosť je vyjadrená ako $\varepsilon+\gamma+\alpha$.
Z predchádzajúceho vieme, že $180\st=5\gamma$ a~$\varepsilon=2\gamma$, teda $\alpha=2\gamma$.
Uhly $CED$ a~$CAB$ sú zhodné, teda úsečky $ED$ a~$AB$ sú rovnobežné.
To znamená, že trojuholník $ABC$ je rovnoramenný so základňou $AB$.
Zo súmernosti tohto trojuholníka vyplýva, že aj uhly $BAD$ a~$ABE$ sú zhodné a majú veľkosť $\alpha-\gamma=\gamma=36\st$.
Veľkosť uhla $BAD$ je $36\st$.
\insp{z8-ii-3b.eps}%

\hodnotenie
1~bod za úvodný rozbor a~rozpoznanie zhodných uhlov;
2~body za veľkosť uhla $ACB$;
1~bod za zdôvodnenie rovnoramennosti trojuholníka $ABC$;
1~bod za veľkosť uhla $BAD$;
1~bod za kvalitu komentára.
\endhodnotenie
}

{%%%%%   Z9-II-1
Cifru na mieste desiatok označíme $a$, cifru na mieste jednotiek označíme $b$.
Hľadané čísla sú v tvare $10a+b$, kde $a=1,\ldots,9$, $b=0,\ldots,9$ a~platí
$$
100a(a+1)+10a+b =(10a+b)^2.
$$
Po úpravách dostávame:
$$
\eqalignno{
  100a^2+100a+10a+b &=100a^2+20ab+b^2, & \cr
  110a+b &= 20ab+b^2. & (\ast) \cr
}
$$

Číslo v~rovnosti $(\ast)$ je viacciferné.
Aby súhlasili cifry na mieste jednotiek, musia byť cifra na mieste jednotiek v~$b^2$ rovnaká ako $b$.
Tejto podmienke vyhovujú iba cifry 0, 1, 5 a ~6.
Postupne dosadíme všetky možnosti do $(*)$ a~doriešime:

\smallskip
\item{$\bullet$} Pre $b=0$ dostávame $$110a=0.$$ Jediným riešením tejto rovnice je $a=0$, čo je nevyhovujúca možnosť.
\item{$\bullet$} Pre $b=1$ dostávame $$110a+1=20a+1.$$ Jediným riešením tejto rovnice je $a=0$, čo je nevyhovujúca možnosť.
\item{$\bullet$} Pre $b=5$ dostávame $$110a+5=100a+25.$$ Jediným riešením tejto rovnice je $a=2$, čo je vyhovujúca možnost.
\item{$\bullet$} Pre $b=6$ dostávame $$110a+6=120a+36.$$ Jediným riešením tejto rovnice je $a=-3$, čo nie je vyhovujúca možnosť.

\smallskip\noindent
Jediné číslo s~vlastnosťou zo zadania je 25.

\hodnotenie
2~body za formuláciu rovnice a~jej úpravy;
2~body za určenie možností číslice $b$;
2~body za vyriešenie úlohy a~záver.
Bez neznámych $a$ a~$b$ sa úloha dá vyriešiť systematickým vyskúšaním všetkých možností.
V~takom prípade hodnoťte podľa úplnosti postupu a~komentára.
\endhodnotenie
}

{%%%%%   Z9-II-2
Ak by sme schematicky zakreslili trojicu hľadaných čísel na číselnú os, tak z~prvej podmienky vieme, že jedno z čísel trojice (označme toto číslo $b$) je presne v strede medzi zvyšnými dvoma číslami (označme najväčšie z nich $a$ a najmenšie z nich $c$).

Aritmetické priemery $q$ a $p$, o ktorých sa hovorí v~druhej podmienke, sú taktiež presne v strede medzi uvedenými dvojicami čísel. Preto sa vzťahy medzi jednotlivými číslami zo zadania dajú schematicky znázorniť tak, ako na obrázku:
\insp{z9-ii-2a.eps}%

Čísla znázornené na osi tvoria postupnosť piatich čísel, medzi ktorými sú rovnaké rozostupy $x$.

Z uvedených vzťahov vyplýva, že číslo $b$ je v strede medzi číslami $p$ a $q$, čiže je ich aritmetickým priemerom. Zo zadania vieme, že súčet týchto dvoch čísel je 628, a teda $b=628/2=314$.
Takisto vieme, že $2x=83$. Teda:
$$
a=314+2x=314+83=397,\quad c=314+2x=314-83=231.
$$

Hľadaná trojica čísel je 397, 314 a~231.

\ineriesenie
Hľadanú trojicu čísel od najväčšieho čísla po najmenšie označíme $a$, $b$, $c$.
Prostredné číslo je aritmetickým priemerom zvyšných dvoch, teda $b=\frac{a+c}2$.
Aritmetický priemer prvého a~druhého čísla je
$$
\frac{a+b}{2}
=\frac{a+\frac{a+c}{2}}{2}
=\frac{3a+c}{4},
$$
aritmetický priemer druhého a~tretieho čísla je
$$
\frac{b+c}{2}
=\frac{\frac{a+c}{2}+c}{2}
=\frac{a+3c}{4}.
$$

Z~informacií o~súčte a~rozdiele týchto dvoch priemerov dostávame
$$
a+c=628,\quad a-c=166.
$$
Sčítaním a odčítaním týchto dvoch rovníc dostávame
$$
2a=628+166,\quad 2c=628-166.
$$
A teda:
$$
a=314+83=397,\quad c=314-83=231,\quad
b=\frac{397+231}{2}=314.
$$

Hľadaná trojica čísel je 397, 314 a~231.

\hodnotenie
2~body za formuláciu pomocou neznámych;
2~body za čiastočné postrehy a~úpravy;
2~body za výsledok a~kvalitu komentára.
\endhodnotenie
}

{%%%%%   Z9-II-3
Množstvo mlieka vydojeného na farme Hoj označíme $h$.
Množstvo mlieka vydojeného na farme Doj bolo $2h$, na farme Loj $4h$. Na všetkých troch farmách dokopy bolo vydojené $7h$ mlieka.

Množstvo mlieka, ktoré poslala na výrobu masla farma Doj, bolo $\frac78\cdot2h=\frac74h$. Farma Hoj poslala $\frac34h$. Zo všetkých troch fariem išlo dokopy $\frac9{10}\cdot7h= \frac{63}{10}h$.
Množstvo mlieka, ktoré išlo na výrobu masla z~farmy Loj, bolo
$$
\frac{63}{10}h-\frac74h-\frac34h =\frac{126-50}{20}h =\frac{76}{20}h =\frac{19}{5}h.
$$

Celkovo se na farme Loj vydojilo $4h=\frac{20}{5}h$ mlieka, teda na výrobu masla poslala farma Loj $\frac{19}{20}=95\,\%$ svojho mlieka.

\hodnotenie
2~body za formuláciu pomocou neznámych;
2~body za čiastočné postrehy a~úpravy;
2~body za výsledok a~kvalitu komentára.
\endhodnotenie
}

{%%%%%   Z9-II-4
Úsečky $AB$ a~$CE$ sú rovnobežné a platí, že $|AB|=2|CE|$.
Preto je trojuholník $CEP$ zhodný s~priečkovými trojuholníkmi trojuholníka $ABP$, tzn. s~trojuholníkmi $KLP$, $AMK$, $MBL$ a~$LKM$ určenými strednými priečkami $ABP$ ako na obrázku:
\insp{z9-ii-4b.eps}%

Teda $|AK|=|KP|=|PC|$ alebo $|AP|=\frac23|AC|$.
Trojuholníky $ABP$ a~$ABC$ majú spoločný vrchol $B$ a~k~nemu protiľahlé strany ležia na tej istej priamke, pre ich obsahy teda platí
$$
S_{ABP} = \frac23 S_{ABC}.
$$

Trojuholník $ABC$ je polovicou obdĺžnika $ABCD$, teda platí
$$
S_{ABC}=\frac12S_{ABCD}.
$$

Pre obsah trojuholníka $ABP$ teda máme
$$
S_{ABP} = \frac23\cdot\frac12 S_{ABCD} = \frac13\cdot82 = 27{,}\overline{3}\,(\Cm^2).
$$

\ineriesenie
Uhlopriečky obdĺžnika $ABCD$ sa pretínajú v bode $S$, ktorý leží v strede každej z nich. Bod $E$ je stred úsečky $CD$.
Preto sú úsečky $SC$ a~$BE$ ťažnicami trojuholníka $BCD$, teda ich priesečník $P$ je ťažiskom tohto trojuholníka.
\insp{z9-ii-4c.eps}%
Keďže je bod $P$ ťažiskom, tak platí $|SC|=3|SP|$. Navyše $|AS|=|SC|$, teda $|AP|=4|SP|$ a~$|AC|=6|SP|$.
Odtiaľ dostávame $|AP|=\frac23|AC|$ a~ďalej postupujeme rovnako ako v~predchádzajúcom riešení.

\poznamky
So znalosťou pomeru $|AP|:|AC|=2:3$ alebo $|AP|:|PC|=2:1$ je možné obdĺžnik $ABCD$ rozdeliť na trojuholníky so známymi pomermi obsahov (viď obrázok). Odtiaľ sa dá vyjadriť pomer obsahu trojuholníka $ABP$ a~obsahu obdĺžnika $ABCD$ ako $4:12=1:3$.
\insp{z9-ii-4a.eps}%

Pomer $|AP|:|PC|=2:1$ je možné odvodiť aj z~podobnosti (rovnoľahlosti) trojuholníkov $ABP$ a~$CEP$, tzn. z~faktu, že úsečky $AB$ a~$CE$ sú rovnobežné a~pre ich veľkosti platí $|AB|=2|CE|$.

\hodnotenie
2~body za rozbor a~čiastkové postrehy;
2~body za pomocné vzťahy;
2~body za výsledok a~kvalitu komentára.
\endhodnotenie
}

{%%%%%   Z9-III-1
Zjednodušene budeme o~divákoch, ktorí dorazili autobusmi, hovoriť ako o~prvej skupine a~o~divákoch, ktorí dorazili pešo alebo autami, ako o~druhej skupine.

Prvá skupina dorazila v šiestich rovnako obsadených autobusoch.
Teda počet ľudí v prvej skupine bol násobkom šiestich.
V druhej skupine bolo o~35\,\% menej ľudí ako v~prvej, teda  pomer veľkostí druhej a~prvej skupiny bol $65\,\% = \frac{65}{100}$.
Tento zlomok v~základnom tvare je $\frac{13}{20}$, preto počet ľudí v~prvej skupine bol násobkom čísla 20.

Teda počet divákov v prvej skupine musí byť súčasne násobkom čísla 6 aj násobkom čísla 20.
Najmenší spoločný násobok týchto dvoch čísel je 60.
Pre násobky 60 väčšie ako 150 vyjadríme veľkosť druhej skupiny a overíme, či súčet veľkostí oboch skupín je menší ako 400:
$$
\begintable
1. skupina\|180|240|300|$\ldots$\crthick
2. skupina\|117|156|195|$\ldots$\crthick
    součet\|297|396|495|$\ldots$%
\endtable
$$

So~zväčšujúcim sa počtom divákov v~prvej skupine sa zväčšuje aj celkový počet divákov, teda vyhovujúce možnosti sú len v~prvých dvoch stĺpcoch tabuľky.
V divadle bolo buď 297, alebo 396 divákov.

\hodnotenie
Po 1~bode za každé vyhovujúce riešenie;
2~body za postrehy o~deliteľnosti počtu ľudí v~prvej skupine;
2~body za úplnosť rozboru možností v~rámci daných obmedzení.

Pri iných spôsoboch skúšania možností hodnoťte dôslednosť pri overovaní celočíselnosti počtov v oboch skupinách.
\endhodnotenie
}

{%%%%%   Z9-III-2
Uhlopriečka $KR$ je zároveeň strana rovnostranného trojuholníka $RAK$.
Platí teda
$$|KR|=|AK|=x.
$$
Uhlopriečka $AD$ je priemerom kružnice, do ktorej je štvoruholník $DRAK$ vpísaný, tj. priemerom kružnice opísanej trojuholníku $RAK$.
V~rovnostrannom trojuholníku splýva stred opísanej kružnice s~ťažiskom, priesečníkom výšok atď.
Polomer opísanej kružnice zodpovedá 2/3 výšky. Výška sa rovná $\sqrt3/2$ veľkosti strany trojuholníka
(možno odvodiť pomocou Pytagorovej vety).
Platí teda
$$
|AD| = 2\cdot\frac23\cdot\frac{\sqrt3}{2}\cdot x
= \frac{2\sqrt3}3\cdot x.
$$

Štvoruholník $DRAK$ je osovo súmerný podľa priamky $AD$, teda uhlopriečky $KR$ a~$AD$ sú kolmé.
Obsah štvoruholníka $DRAK$ je
$$
S_{DRAK} = \frac12\cdot|AD|\cdot|KR|
= \frac{\sqrt3}3\cdot x^2 .
$$
\insp{z9-iii-2a.eps}%

\poznamky
Predchádzajúci výpočet obsahu štvoruholníka $DRAK$ je založený na vyjadrení obsahu štvoruholníka s navzájom kolmými uhlopriečkami $AD$ a~$KR$.
K rovnakému výsledku je možné dospieť aj takto:

Obsah rovnostranného trojuholníka $RAK$ je $\sqrt3/4\cdot|AK|^2$=$\sqrt3/4\cdot x^2$.

Obsah trojuholníka $DRK$ je tretinový vzhľadom na obsah trojuholníka $RAK$  (trojuholník $DRK$ sa zhoduje s trojuholníkmi $ORK$, $OKA$, $OAR$),
celkom
$$
S_{DRAK} = \left(\frac{\sqrt3}{4}+\frac{\sqrt3}{12}\right)\cdot x^2 = \frac{\sqrt3}3\cdot x^2 .
$$

\hodnotenie
Po 1~bode za každý z~výsledkov (veľkosť $KR$, veľkosť $AD$, obsah $DRAK$);
2~body za pomocné vzťahy a~postrehy (výška rovnostranného trojuholníka, polomer opísanej kružnice, kolmosť uhlopriečok a pod.);
1~bod za kvalitu komentára.
\endhodnotenie

}

{%%%%%   Z9-III-3
Pomocou neznámej $b$ vyjadríme $a$:
$$
\eqalignno{
  7a + 4b + 74 &= a\cdot b, & \cr
  7a - a\cdot b &= -4b-74, & \cr
  a &= \frac{4b+74}{b-7}. & (*) \cr
}
$$

Čísla $a$ a~$b$ majú byť prirodzené, preto musí platiť, že $b>7$ a~$b-7$ musí deliť $4b+74$.
Ďalšou úpravou dostávame:
$$
\eqalignno{
   a &= \frac{4b+74}{b-7}=\frac{4(b-7)+102}{b-7}. & (**) \cr
}
$$
Preto prirodzené číslo $b-7$ musí deliť 102.
Prvočíselný rozklad čísla 102 je $2\cdot3\cdot17$.
Teda číslo 102 má osem (kladných) deliteľov.
Pre každý z týchto deliteľov vyjadríme $b$ a~$a$ podľa predchádzajúcich vzťahov:
$$
\begintable
$b-7$\|1|2|3|6|17|34|51|102\crthick
$b$\|8|9|10|13|24|41|58|109\cr
$a$\|106|55|38|21|10|7|6|5%
\endtable
$$

Hodnoty $a$ a~$b$ uvedené v~tabuľke tvoria všetky vyhovujúce dvojice čísel (až na poradie).

\poznamka
Úvodná úprava a~následné úvahy môžu byť skrátené takto:
$$
\eqalignno{
  7a + 4b + 74 &= a\cdot b, & \cr
  74 &= a\cdot b -7a -4b, & \cr
  102 &= (a-4)\cdot(b-7). & (***) \cr
}
$$
Čísla $a$ a~$b$ sú určené možnými rozkladmi čísla 102 na dva (kladné) činitele.
To vedie práve k riešeniam popísaným vyššie..

\hodnotenie
2 bod za vyjadrenie ($*$) a 1~bod za vyjadrenie ($**$) a~úvahy vedúce k~deliteľom čísla 102; resp. 3 body za vyjadrenie ($***$);
1~bod za delitele čísla 102;
2~body za všetky riešenia.

Pri iných spôsoboch skúšania možností hodnoťte podľa úplnosti postupu a~komentára.
\endhodnotenie

}

{%%%%%   Z9-III-4
Podľa zadania platí
$$
|BD|=|AC|=|XY|=8\,\cm.
$$
Trojuholníky $ACZ$ a~$ABC$ majú spoločnú stranu $AC$ a~obsah $ACZ$ má byť dvakrát väčší ako obsah $ABC$.
Preto vzdialenosť bodu $Z$ od priamky $AC$ je dvakrát väčšia ako vzdialenosť bodu $B$, respektíve $D$  od priamky $AC$.

Uhlopriečky obdĺžnika $ABCD$ sú zhodné a~pretínajú sa vo svojich stredoch; tento bod označíme $S$.
Ďalej body $Z$, $B$, $D$, $S$ ležia na jednej priamke a~trojuholníky $ABC$ a~$CDA$ tvoria zhodné časti obdĺžnika $ABCD$.
Celkom teda platí:
$$
|SZ|=2\cdot|SB|=2\cdot|SD|=|BD|=8\,\cm.
$$
Keďže $ABCD$ je obdĺžnik, ktorého uhlopriečky sa rozpoľujú, ležia body $B$ a $D$ na kružnici nad priemerom $AC$.
\insp{z9-iii-4c.eps}%

Konštrukcia obdĺžnika $ABCD$:
\begin{enumerate}
 \item priamka $XY$,
 \item kružnica so stredom $Z$ a~polomerom $|XY|=8$\,cm,
 \item bod $S$ je priesečníkom priamky 1) a~kružnice 2),
 \item kružnica so stredom $S$ a~polomerom $\frac12|XY|=4$\,cm,
 \item body $A$ a~$C$ sú priesečníky priamky 1) a~kružnice 4),
 \item priamka $SZ$,
 \item body $B$ a~$D$ sú priesečníky priamky 6) s~kružnicou 4).
\end{enumerate}

Možné stredy $S$ (priesečníky z tretieho kroku konštrukcie) sú dva.
Označenie dvojíc bodov $A$ a~$C$, resp. $B$ a~$D$ (v~piatom, resp. siedmom kroku konštrukcie) je zameniteľné.
Úloha má (až na značenie vrcholov) dve riešenia.

\poznamky
Z~podmienky o~obsahoch trojuholníkov $ACZ$ a~$ABC$ vyplýva, že body $B$ a~$D$ ležia na rovnobežkách s~priamkou $XY$, ktoré sú v~polovičnej vzdialenosti od $XY$ ako bod $Z$.
Preto sa bod $D$ (resp. $B$) dá zostrojiť ako priesečník priamky $ZS$ s~týmito priamkami.

\hodnotenie
1~bod za odvodenie $|BD|=8\,\cm$;
2~body za odvodenie $|SZ|=8\,\cm$;
1~bod za odvodenie polohy bodu $D$, resp. $B$ (rovnobežky v polovičnej vzdialenosti, resp. kružnica),
2~body za konštrukciu oboch obdĺžnikov a~popis konštrukcie.
Diskusia o~počte riešení (zameniteľnosť označenia vrcholov) nie je nutná k~zisku plného počtu bodov.
\endhodnotenie
}


