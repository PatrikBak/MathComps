{%%%%%   Z4-I-1
...}

{%%%%%   Z4-I-2
...}

{%%%%%   Z4-I-3
...}

{%%%%%   Z4-I-4
...}

{%%%%%   Z4-I-5
...}

{%%%%%   Z4-I-6
...}

{%%%%%   Z5-I-1
\napad
Začnite počítať s~ovocím, ktorého cenu poznáte.

\riesenie
Okrem banánu stará mama kúpila jablká a~hrušky, ktorých ceny za kus poznáme.
Jabĺk a~hrušiek boli celkom štyri kusy.
Rozoberieme jednotlivé možnosti a~určíme útratu za jablká a~hrušky.
Odčítame od celkovej útraty a~z~toho zistíme cenu banánu, ktorú porovnáme s~cenou hrušky:
\begin{itemize}
\item Štyri jablká a~žiadna hruška stoja $4\cdot50=200$ centov; jeden banán by stál ${275-200}=75$ centov, čo je viac ako cena hrušky.
\item Tri jablká a~jedna hruška stoja $3\cdot50+1\cdot60=210$ centov; jeden banán by stál $275-210=65$ centov, čo je viac ako cena hrušky.
\item Dve jablká a~dve hrušky stoja $2\cdot50+2\cdot60=220$ centov; jeden banán by stál $275-220=55$ centov, čo je menej ako cena hrušky.
\item Jedno jablko a~tri hrušky stoja $1\cdot50+3\cdot60=230$ centov; jeden banán by stál $275-230=45$ centov, čo je menej ako cena hrušky.
\item Žiadne jablko a~štyri hrušky stoja $4\cdot60=240$ centov; jeden banán by stál ${275-240}=35$ centov, čo je menej ako cena hrušky.
\end{itemize}
Banán mohol stať 35, 45, alebo 55 centov.

\poznamka
Hruška je o~10~centov drahšia ako jablko, a~práve o~toľko sa musí znížiť cena banánu, ak vymeníme v~uvažovanom nákupe jablko za hrušku.
S~týmto postrehom možno predchádzajúcu diskusiu značne zjednodušiť.
}

{%%%%%   Z5-I-2
\napad
Potrebujete poznať, kde presne sa cesty krížia?

\riesenie
Presné umiestnenie kríženia ciest nie je zo zadania zrejmé, to však nemá na výsledok žiadny vplyv.
Pri počítaní musíme dbať na to, aby sme dlaždice v~kríženiach ciest rovnako ako v~rohoch započítavali iba raz.
Preto budeme rozlišovať cesty, ktoré sú zakreslené vodorovne, od tých zakreslených zvislo.
Jednotlivé časti môžeme navyše pre lepšiu názornosť presúvať.
Napr. nasledujúca sústava ciest má rovnaký počet dlaždíc ako tá pôvodná:
\insp{z5-I-2a.eps}%


Najskôr sčítame dlaždice na vodorovných cestách (zhora nadol, zľava doprava):
$$
30+50+30+20+20+50 =2\cdot(30+50+20) =200.
$$

Teraz sčítame dlaždice na zvislých cestách, ktoré nie sú započítané v~kríženiach a~rohoch (zľava doprava, zhora nadol):
$$
(20-2)+(50-3)+(20-3)+(50-4)+(50-4) =190-16 =174.
$$

Celkom sa má vymeniť $200+174=374$ dlaždíc.
}

{%%%%%   Z5-I-3
\napad
Koľko dukátov by dostal najstarší syn, ak by kráľ rovnakým spôsobom rozdával napr. štyrom synom?

\riesenie
Pre konkrétny počet synov si možno kráľov spôsob rozdávania dukátov názorne vyskúšať.
Stačí postupovať odzadu: najmladší v~druhom kole dostal jeden dukát, druhý najmladší dva dukáty atď.
Napr. pre dvoch, troch, resp. štyroch synov by počty dukátov v~jednotlivých kolách vyzerali nasledovne (zoradené zhora nadol podľa kôl, zľava doprava podľa veku):
$$
\alggg{4&3}{2&1}
\qquad
\alggg{6&5&4}{3&2&1}
\qquad
\alggg{8&7&6&5}{4&3&2&1}
$$

Najstarší syn by v~prvom prípade dostal~6, v~druhom prípade~9, resp. v~treťom prípade 12~dukátov.
Týmto spôsobom možno postupne nájsť situáciu, keď najstarší syn dostal 21~dukátov:
$$
\vbox{\let\\=\cr
\halign{&\hbox to1.4em{\hss$#$\hss}\cr
14&13&12&11&10&9&8\\
\noalign{\vskip4pt\hrule\vskip4pt}
\ 7&\ 6&\ 5&\ 4&\ 3&2&1\\
}}
$$
Teda kráľ mal 7~synov a~celkom im rozdal 105 dukátov.

\poznamky
Namiesto skúšania si možno všimnúť, že zo zadania vyplýva nasledujúce:
najstarší syn v~druhom kole dostane práve toľko dukátov, koľko je synov, a~v~prvom kole dvojnásobok, celkom teda trojnásobok počtu synov.
Aby tento počet bol rovný 21, musí byť 7~synov a~celkový počet dukátov $1+2+\cdots+14=105$.

Súčet rozdaných dukátov možno určiť rôzne, napr. nasledujúcou skratkou:
$$
(1+14)+(2+13)+\cdots+(7+8) =7\cdot15 =105.
$$
}

{%%%%%   Z5-I-4
\napad
Koľko dvojok by Vojto napísal, keby vypisoval iba 20 cifier?

\riesenie
Číslo školského roku 20192020 je osemciferné a~obsahuje tri dvojky.
Keďže $2020=8\cdot252+4$, Vojto napísal 252-krát celé číslo 20192020
a~na zvyšné štyri miesta číslo~2019.

Celkom teda Vojto napísal $252\cdot3+1=757$ dvojok.
}

{%%%%%   Z5-I-5
\napad
Koľko jabĺk bolo na každej z~jabloní?

\riesenie
Na každej jabloni je rovnaký počet jabĺk a~celkom ich je~39.
Na každej jabloni teda rastie 13~jabĺk.
Na rôznych konároch sú rôzne počty jabĺk a~počty konárov na jednotlivých stromoch sú 2, 3 a~3.

Teda hľadáme osmice navzájom rôznych čísel, z~ktorých vyberáme dvojicu a~dve trojice s~rovnakým súčtom~13.
Také osmice sú:
\bgroup
\thinsize=0pt
\thicksize=0pt
\def\ctr#1{#1\hfil}
$$
\begintable
2,\,11,|\quad| 1,\,5,\,7,|\quad| 3,\,4,\,6, \cr
3,\,10,|\quad| 1,\,4,\,8,|\quad| 2,\,5,\,6, \cr
4,\,9,|\quad| 1,\,5,\,7,|\quad| 2,\,3,\,8, \cr
5,\,8,|\quad| 1,\,2,\,10,|\quad| 3,\,4,\,6, \cr
5,\,8,|\quad| 1,\,3,\,9,|\quad| 2,\,4,\,7.
\endtable
$$
\egroup
}

{%%%%%   Z5-I-6
\napad
Ktoré štvorce sa môžu vyskytovať v~jednom riadku, príp. stĺpci?

\riesenie
V~žiadnom riadku a~stĺpci nemôžu byť dva štvorce rovnakej farby vedľa seba; budeme uvažovať možnosti po riadkoch:
\begin{itemize}
\item
Ak by v~riadku susedil biely a~čierny štvorec, tak zodpovedajúce štvorce v~ďalšom riadku nemožno vyfarbiť žiadnou dvojicou farieb tak, aby sme vyhoveli všetkým uvedeným požiadavkám.
\item Ak by v~jednom riadku susedil biely a~sivý štvorec, tak zodpovedajúce štvorce v~druhom riadku musia byť sivý a~čierny (aby sa striedali farby a~biele štvorce nesusedili vrcholom).
Potom ale ďalší štvorec v~prvom riadku musí byť opäť biely a~zodpovedajúci štvorec v~druhom riadku sivý (aby sa striedali farby a~čierne štvorce nesusedili vrcholom).
\insp{z5-I-6a.eps}%
\item
Diskusia pre susediace čierne a~sivé štvorce je obdobná.
\end{itemize}

V~každom riadku, príp. stĺpci, sa teda striedajú buď sivé a~biele, alebo sivé a~čierne štvorce.
Tieto podmienky presne určujú vzor na obruse.
Teraz si stačí predstaviť dostatočne veľkú oblasť s~opísaným vzorom a~vybrať z~nej možné obrusy vyhovujúce posledným dvom podmienkam zo zadania.
Také možnosti sú nasledujúce:
\insp{z5-I-6.eps}%


Obrus j) je štvorcový, všetky ostatné sú obdĺžnikové, teda vyhovujúce všetkým podmienkam úlohy.
}

{%%%%%   Z6-I-1
\napad
Koľko celých rokov je 50 mesiacov?

\riesenie
50 mesiacov sú štyri roky a~2 mesiace.
Teda stará mama mala aspoň 64~rokov.
Zvyšné 2~mesiace, 40~týždňov a~30~dní predstavujú zhruba jeden ďalší rok.
Potrebujeme presne zistiť, či je to viac alebo menej, preto uvedené údaje prepočítame na dni.

2 mesiace môžu mať
59 dní (január a~február, príp. február a~marec v~nepriestupnom roku),
60 dní (január a~február, príp. február a~marec v~priestupnom roku),
61 dní (skoro všetky ostatné dvojice mesiacov), alebo
62 dní (júl a~august, príp. december a~január).
40~týždňov a~30~dní je $40\cdot7+30=310$ dní.

Celkom teda 2 mesiace, 40 týždňov a~30 dní predstavujú najmenej 369 a~najviac 372 dní.
To je vždy viac ako rok (a~menej ako dva).
Stará mama mala naposledy 65. narodeniny.
}

{%%%%%   Z6-I-2
\napad
Aká je dĺžka jednej vodorovnej úsečky trojuholníkovej siete?

\riesenie
Obvody mnohouholníkov $A$ a~$B$ sa líšia o~dve vodorovné úsečky, čo zodpovedá $56-34=22$\,(cm).
Jedna vodorovná úsečka je teda 11\,cm dlhá.

Obvod mnohouholníka $D$ je tvorený dvoma vodorovnými úsečkami (---) a~štyrmi spätne šikmými úsečkami ($\backslash$), čo dáva dokopy 42\,cm.
Tieto štyri spätne šikmé úsečky majú v~súčte $42-22=20$\,(cm), teda jedna je 5\,cm dlhá.

Obvod mnohouholníka $B$ je tvorený jednou vodorovnou (---), tromi spätne šikmými~($\backslash$) a~jednou šikmou (/) úsečkou, čo je 34\,cm.
Jedna šikmá úsečka je preto dlhá $34-11-3\cdot5=8$\,(cm).

Obvod trojuholníka $C$ je rovný $11+5+8=24$\,(cm).
}

{%%%%%   Z6-I-3
\napad
Mohli byť v~teste práve štyri ťažké úlohy?

\riesenie
Zo zadania vyplýva, že v~písomke bol párny počet stredne ťažkých úloh a~počet ťažkých úloh bol násobkom troch.
Budeme postupne uvažovať možné počty ťažkých úloh a~diskutovať dôsledky:
\begin{itemize}
\item
Ak by ťažké úlohy boli 3, tak by stredne ťažkých a~ľahkých úloh bolo dokopy ${25-3}=22$ a~ich najlepšie možné hodnotenie by bolo $84-3\cdot 5=69$ bodov.
To však nie je možné, pretože stredne ťažkých úloh bol párny počet a~ľahké úlohy boli hodnotené po 2~bodoch, avšak 69 nie je párne číslo.
(Iným dôvodom môže byť, že ani 22 stredne ťažkých úloh by nedalo 69~bodov.)
\item
Ak by ťažkých úloh bolo 6, tak by stredne ťažkých a~ľahkých úloh bolo dokopy $25-6=19$ a~ich najlepšie možné hodnotenie by bolo $84-6\cdot 5=54$ bodov.
Keby stredne ťažké aj ľahké úlohy boli hodnotené rovnako po 2~bodoch, tak by týchto 19~úloh bolo hodnotených $2\cdot 19=38$ bodmi, čo je o~16 menej ako 54.
Keďže rozdiel v~hodnotení stredne ťažkých a~ľahkých úloh je práve jeden bod, zodpovedá predchádzajúci rozdiel 16 práve počtu stredne ťažkých úloh.
Ľahkých úloh by tak bolo $19-16=3$.
V~takom prípade by Peter správne vyriešil 3~ľahké úlohy, 8 stredne ťažkých a~2~ťažké.
Za takú písomku by získal
$$
3\cdot2+8\cdot3+2\cdot5=40\ \text{bodov}.
$$
\item
Ak by ťažkých úloh bolo 9, tak by stredne ťažkých a~ľahkých úloh bolo dokopy $25-9=16$ a~ich najlepšie možné hodnotenie by bolo $84-9\cdot 5=39$ bodov.
To však nie je možné z~rovnakého dôvodu ako v~prvom diskutovanom prípade.
\item
Ak by ťažkých úloh bolo 12, tak by stredne ťažkých a~ľahkých úloh bolo dokopy $25-12=13$ a~ich najlepšie možné hodnotenie by bolo $84-12\cdot 5=24$ bodov.
To však nie je možné, lebo už 13 ľahkých úloh zodpovedá 26~bodom, čo je viac ako~24.
\item
Ďalej nie je nutné pokračovať, pretože práve pozorovaný rozdiel v~hodnotení by sa len zväčšoval.
\end{itemize}

Vychádza jediná možnosť, a~to, že Peter získal 40~bodov.

\poznamka
Ktorýkoľvek argument v~predchádzajúcej diskusii možno nahradiť systematickým skúšaním a~overovaním podmienok zo zadania.
Úplná diskusia v~tomto duchu je veľmi prácna, nemôže byť v~tejto kategórii vyžadovaná, ale mala by byť uspokojivo naznačená.
Pri hodnotení úlohy túto požiadavku zohľadnite.
}

{%%%%%   Z6-I-4
\napad
Koľkokrát ktoré páža dostávalo dukáty?

\riesenie
Posledné páža dostávalo dukáty iba raz, naproti tomu ostatné pážatá dvakrát.

Pritom predposledné páža dostalo dvojnásobok toho, čo posledné: najskôr o~2~dukáty viac ako posledné, potom o~2 dukáty menej.

Všetky okrem posledného pážaťa dostali rovnako: najskôr dostal každý o~2 dukáty viac ako nasledujúci sused v~rade, potom o~2 dukáty menej ako ten istý sused.

Navyše počet dukátov posledného pážaťa bol dvojnásobkom počtu pážat: kráľ postupne znižoval príspevok o~2~dukáty a~v~druhom kole dostalo prvé páža 2~dukáty.

Sú teda dve možnosti:
\begin{itemize}
\item
32 dukátov malo posledné páža.
Teda pážat bolo 16 a~každé z~15 ostatných pážat malo 64~dukátov.
Kráľ celkom rozdal $15\cdot64+32=992$ dukátov.
\item
32 dukátov mali všetky pážatá okrem posledného.
Teda posledné malo 16 dukátov a~pážat bolo~8.
Kráľ celkom rozdal $7\cdot32+16=240$ dukátov.
\end{itemize}

Kráľ mohol mať 16 pážat a~rozdať 992 dukátov, alebo 8 pážat a~240 dukátov.
}

{%%%%%   Z6-I-5
\napad
Koľkokrát sa vojde malý štvorec do veľkého?

\riesenie
Dĺžka strany veľkého štvorca je dvojnásobkom dĺžky strany malého štvorca, preto je obsah veľkého štvorca štvornásobkom obsahu malého štvorca, pozri obrázok.
Keď z~veľkého štvorca odstránime malý štvorec, zvyšný útvar má obsah $\frac34$ veľkého štvorca.
Teda veľký štvorec má obsah $\frac43$ zvyšného útvaru.
\insp{z6-I-5a.eps}%


Útvar v~zadaní úlohy vznikol odobratím piatich malých štvorcov od piatich veľkých štvorcov (pričom presné umiestnenie štvorcov nie je pre určovanie obsahov podstatné).
Vzťahy medzi ich obsahmi sú teda rovnaké ako vo vyššie diskutovanom prípade.
Obsah veľkého kríža je rovný
$$
\frac43\cdot 45=60\,(\Cm^2).
$$

\poznamka
Ak $S$ označuje obsah malého štvorca, tak obsah malého kríža je $5S$, obsah veľkého štvorca je $4S$ a~veľkého kríža je $20S$.
Obsah sivého útvaru je $(20-5)S=15S=45\cm^2$, teda $S=3\cm^2$ a~obsah veľkého kríža je $20\cdot3=60\,(\Cm^2)$.
}

{%%%%%   Z6-I-6
\napad
Koľko je trojciferných komických čísel začínajúcich 12? A~koľko ich začína jednotkou?

\riesenie
Ako nepárnych, tak párnych cifier je päť.
Pri komických číslach môže na prvom mieste byť ktorákoľvek z~nepárnych cifier 1, 3, 5, 7, 9.
Pre každú z~týchto piatich možností môže na druhom mieste byť ktorákoľvek z párnych cifier 0, 2, 4, 6, 8, čo dáva $5\cdot5=25$ možností.
Pre každú z~týchto 25 možností môže na treťom mieste byť ktorákoľvek z~nepárnych cifier, čo dáva $25\cdot5=125$ možností.
Trojciferných komických čísel je 125.

Počítanie trojciferných veselých čísel je obdobné, len na prvom mieste nemôže byť~0.
Trojciferných veselých čísel teda je $4\cdot5\cdot5=100$.

Trojciferných komických čísel je o~25 viac ako veselých.
}

{%%%%%   Z7-I-1
\napad
Ktoré čísla sú deliteľné štyrmi a~pritom nie sú deliteľné dvoma?

\riesenie
Počet šišiek je číslo menšie ako 350, ktoré nie je deliteľné práve jednou dvojicou po sebe idúcich čísel z~2, 3, 4, 5, 6, 7, 8, 9 a~všetkými ostatnými áno.
Žiadna z~dvojíc $(2,3)$, $(3,4)$, $(4,5)$, $(5,6)$ a~$(6,7)$ to byť nemôže, lebo by sa medzi zvyšnými číslami vždy našlo nejaké ďalšie, ktorým by počet šišiek tiež nemohol byť deliteľný:
\begin{itemize}
\item
nedeliteľnosť 2 vynucuje nedeliteľnosť 4, 6 a~8,
\item
nedeliteľnosť 3 vynucuje nedeliteľnosť 6 a~9,
\item
nedeliteľnosť 4 vynucuje nedeliteľnosť 8,
\item
nedeliteľnosť 6 vynucuje nedeliteľnosť 2 alebo 3.
\end{itemize}

Počet šišiek nie je deliteľný buď dvojicou čísel $(7,8)$, alebo $(8,9)$, a~všetkými ostatnými číslami áno.
Tento počet potom musí byť deliteľný aj najmenším spoločným násobkom ostatných čísel:
\begin{itemize}
\item
V~prvom prípade je najmenší spoločný násobok čísel 2, 3, 4, 5, 6, 9 rovný $4\cdot5\cdot9=180$.
Jediný násobok 180 menší ako 350 je práve 180.
\item
V~druhom prípade je najmenší spoločný násobok čísel 2, 3, 4, 5, 6, 7 rovný $3\cdot4\cdot5\cdot7=420$, čo je viac ako 350.
\end{itemize}

Snehulienka s~trpaslíkmi nazbierali 180 šišiek.
}

{%%%%%   Z7-I-2
\napad
Znázornite si situáciu zo zadania a~hľadajte zhodné uhly.

\riesenie
V~danom trojuholníku označíme $Y$, $Z$ päty ostatných výšok a~$N$, $O$ priesečníky osi uhla $XVL$ so stranami $KL$, $KM$, pozri obrázok.
V~obrázku tiež vyznačujeme navzájom zhodné uhly:
\insp{z7-I-2a.eps}%


V~pravouhlom trojuholníku $KZL$ poznáme uhly pri vrcholoch $K$ a~$Z$, teda uhol pri vrchole~$L$ má veľkosť $180\st-90\st-70\st=20\st$.
Podobne veľkosť uhla pri vrchole $M$ v~pravouhlom trojuholníku $KXM$ je 20\st.
Uhly $KLZ$ a~$KMX$ sú preto zhodné.

Priamky $NO$ a~$LM$ sú rovnobežné, preto sú uhly $MLZ$, $LVN$ a~$OVZ$ navzájom zhodné (striedavé a~súhlasné uhly).
Z~obdobného dôvodu sú aj uhly $LMX$, $MVO$ a~$NVX$ navzájom zhodné.
Priamka $NO$ je osou uhla $XVL$, preto sú uhly $LVN$ a~$NVX$~-- a~teda aj všetky ostatné pred chvíľou menované uhly -- navzájom zhodné.

Celkom teda uhly $KLM$ a~$KML$ sú zhodné.
To sú vnútorné uhly v~trojuholníku $KLM$, ktorého tretí uhol poznáme.
Veľkosti uhlov $KLM$ a~$KML$ sú $(180\st-70\st):2=55\st$.

\poznamky
Zhodnosť uhlov $KLM$ a~$KML$ znamená, že trojuholník $KLM$ je rovnoramenný so základňou $LM$.
Tento fakt možno priamo odvodiť nasledovne:

Priamka $NO$ je rovnobežná so stranou $LM$, teda je kolmá na výšku na túto stranu.
Keďže priamka $NO$ je osou uhla $XVL$, sú uhly $NVX$ a~$OVZ$ zhodné, a~keďže je táto priamka kolmá na výšku $KY$, sú aj uhly $KVX$ a~$KVZ$ zhodné.
Trojuholníky $KVX$ a~$KVZ$ sú pravouhlé a~majú zhodné uhly pri vrchole $V$, preto majú zhodné uhly aj pri vrchole $K$.
Teda výška $KY$ je osou uhla $LKM$, čo znamená, že trojuholník $KLM$ je rovnoramenný so základňou $LM$.

\smallskip
K~veľkostiam konkrétnych uhlov možno dôjsť rôznymi spôsobmi, napr. nasledujúcou úvahou:

Trojuholníky $KZL$ a~$VXL$ sú pravouhlé a~majú spoločný uhol pri vrchole $L$.
Vnútorné uhly týchto trojuholníkov pri vrcholoch $K$ a~$V$ sú preto zhodné a~majú veľkosť 70\st\ (zo zadania).
Priamka $VN$ je osou uhla $XVL$, teda uhol $NVX$ má veľkosť 35\st\, ($70:2=35$).
Veľkosť uhla pri vrchole $N$ v~trojuholníku $NVX$ je 55\st\, ($90-35=55$).
Priamky $VN$ a~$LM$ sú rovnobežné, teda uhly $XNV$ a~$KLM$ sú zhodné (súhlasné uhly).
Veľkosť uhla $KLM$ je 55\st.\looseness-1
% \filbreak
}

{%%%%%   Z7-I-3
\napad
Akú úlohu hrajú nuly v~Romanových číslach?

\riesenie
Po rozdelení pôvodného čísla mal Roman dve nanajvýš dvojciferné čísla.
Ich súčet teda môže byť jednociferné, dvojciferné alebo trojciferné číslo.
Ich rozdiel môže byť 0, jednociferné alebo dvojciferné číslo.
Pritom súčet je vždy väčší ako rozdiel.

Číslo 171 mohol Roman dostať jedine tak, že súčet čísel po rozdelení pôvodného čísla bol 17 a~rozdiel~1.
Teda sčítal a~odčítal čísla 9 a~8.
Číslo 171 mohol Roman vyčarovať buď z~čísla 908, alebo z~čísla 809.

Číslo 1513 mohol Roman dostať dvoma spôsobmi:
\begin{itemize}
\item Súčet čísel po rozdelení pôvodného čísla bol 15 a~rozdiel 13.
Teda sčítal a~odčítal čísla 14 a~1.
\item Súčet čísel po rozdelení pôvodného čísla bol 151 a~rozdiel 3.
Teda sčítal a~odčítal čísla 77 a~74.
\end{itemize}
Číslo 1513 mohol Roman vyčarovať z~niektorého z~čísel 1401, 114, 7774, alebo 7477.

Výsledné číslo je aspoň dvojciferné a~nanajvýš päťciferné.
Päťciferné číslo vznikne zložením trojciferného súčtu a~dvojciferného rozdielu dvojice čísel po rozdelení pôvodného čísla.
Aby výsledné číslo bolo najväčšie možné, musí byť pomocný súčet najväčší možný, teda rozdiel najmenší možný (ale dvojciferný).
Tieto požiadavky určujú dvojicu 99 a~89: $99+89=188$ a~$99-89=10$, výsledné číslo je 18810, a~to je možné vyčarovať buď z~čísla 9989, alebo z~čísla 8999.
}

{%%%%%   Z7-I-4
\napad
Koľko vody pretekalo všetkými miestami v~druhom rade a~koľko v~treťom rade?

\riesenie
Keďže sa voda v~korytách len delí a~zasa spája (nikam sa nestráca a~nič nepribúda), je súčet prietokov všetkými miestami v~každom rade rovnaký ako na začiatku.
Teda Marienka v~2019.~rade pozoruje rovnaký prietok, ktorý by pozorovala na začiatku.

Pre určenie prietokov v~Jankových vyznačených miestach budeme postupne prebiehať korytami a~označovať, aké časti pôvodného prietoku sú v~jednotlivých uzloch.
Podľa zadania každé koryto privádza do vybraného uzla polovicu prietoku z~uzla predchádzajúceho.
Zhora postupne zisťujeme nasledujúce výsledky (ktoré kvôli lepšej prehľadnosti neupravujeme na základný tvar):
\bgroup
\thinsize=0pt
\thicksize=0pt
\def\ctr#1{\hfil#1\hfil}
$$
\begintable
||||1||||\cr
|||$\frac12$||$\frac12$|||\cr
||$\frac{\bold1}{\bold4}$||$\frac24$||$\frac{\bold1}{\bold4}$||\cr
|$\frac18$||$\frac38$||$\frac38$||$\frac18$|\cr
$\frac1{16}$||$\frac{\bold4}{\bold{16}}$||$\frac6{16}$||$\frac{\bold4}{\bold{16}}$||$\frac1{16}$\endtable
$$
\egroup

Prietoky Jankovými miestami v~treťom, resp. piatom rade tvoria $\frac12\cdot\frac12=\frac14$, resp. $\frac12\cdot\frac18+\frac12\cdot\frac38=\frac4{16}=\frac14$ prietoku na začiatku.
Celkom teda súčet prietokov všetkými štyrmi Jankovými miestami je rovnaký ako na začiatku.
Celkové prietoky Jankovými a~Marienkinými miestami sú rovnaké.

\odst{Poznámka pre riešiteľa}
Spomeňte si na túto úlohu, až časom budete počuť o~tzv. Pascalovom trojuholníku.
}

{%%%%%   Z7-I-5
\napad
Aké sú súčty zvyšných čísel v~riadkoch a~stĺpcoch?

\riesenie
Ak súčet čísel v~jednotlivých riadkoch a~stĺpcoch označíme $s$, tak chýbajúce čísla v~hviezdnom štvorci sú
$$\begintable
1|2|3|$s-6$\cr
4|5|6|$s-15$\cr
7|8|9|$s-24$\cr
$s-12$|$s-15$|$s-18$|$-2s+45$\endtable
$$

Číslo $s-6$ je záporné pre $s<6$; podobne pre ďalšie doplnené čísla okrem $\m2s+45$, ktoré je záporné pre $s>\frac{45}2$.
Práve štyri záporné celé čísla dostávame pre $s=$ 12, 13 alebo~14:
$$
\hbox{\begintable
1|2|3|6\cr
4|5|6|$-3$\cr
7|8|9|$-12$\cr
0|$-3$|$-6$|21\endtable}
$$
$$
\hbox{\begintable
1|2|3|7\cr
4|5|6|$-2$\cr
7|8|9|$-11$\cr
1|$-2$|$-5$|19\endtable}
$$
$$
\hbox{\begintable
1|2|3|8\cr
4|5|6|$-1$\cr
7|8|9|$-10$\cr
2|$-1$|$-4$|17\endtable}
$$

\poznamka
Namiesto všeobecných výrazov s~neznámou~$s$ možno začať doplnením konkrétnych hodnôt tak, aby čísla v~tabuľke tvorili hviezdny štvorec.
Postupným zväčšovaním, príp. zmenšovaním čiastočných súčtov možno odvodiť vyššie uvedené doplnenia, v~ktorých sú práve štyri záporné čísla.
}

{%%%%%   Z7-I-6
\napad
Rozdeľte útvar na vhodné menšie časti.

\riesenie
Šesťuholník je tvorený šiestimi zhodnými rovnostrannými trojuholníkmi.
Vystrihnuté sivé útvary v~týchto trojuholníkoch sú dvojakého typu:
\insp{z7-I-6a.eps}%


V~prvom prípade je $K$ stredom úsečky $AB$ a~$L$ je stredom úsečky~$SA$.
Trojuholníky $SKL$ a~$SKA$ majú rovnakú výšku zo spoločného vrcholu~$K$ a~zodpovedajúca strana~$SL$ je polovičná vzhľadom na stranu~$SA$.
Preto má trojuholník $SKL$ polovičný obsah vzhľadom na trojuholník $SKA$.
Z~podobného dôvodu má trojuholník $SKA$ polovičný obsah vzhľadom na trojuholník $SBA$.
Celkom má sivá časť štvrtinový obsah vzhľadom na trojuholník $SBA$.

V~druhom prípade je $N$ stredom úsečky~$BC$, $M$ je stredom úsečky~$BN$, $O$ je stredom úsečky~$NC$ a~$P$ je stredom úsečky~$NS$.
Rovnako ako v~predchádzajúcom prípade zdôvodníme, že
trojuholník $SMP$ má polovičný obsah vzhľadom na trojuholník $SMN$, ten má polovičný obsah vzhľadom na trojuholník $SBN$, a~ten má polovičný obsah vzhľadom na trojuholník $SBC$.
Celkom trojuholník $SMP$ má osminový obsah vzhľadom na trojuholník $SBC$.
Trojuholník $SOP$ je zhodný s~trojuholníkom $SMP$.
Celkom má sivá časť štvrtinový obsah vzhľadom na trojuholník $SBC$.

Pre každý zo šiestich pomocných trojuholníkov platí, že pomer jeho obsahu a~obsahu sivého útvaru je $4:1$.
Pomer obsahov pôvodného šesťuholníka a~vystrihnutého útvaru je preto taký istý.
}

{%%%%%   Z8-I-1
\napad
Aké vlastnosti majú uhlopriečky kosoštvorca?

\riesenie
Vzdialenosť bodu $B$ od priamky $AD$ je rovnaká ako vzdialenosť bodu~$C$ od $AD$, lebo $BC$ a~$AD$ sú rovnobežné.
Uhlopriečky v~každom rovnobežníku sa navzájom rozpoľujú, v~kosoštvorci sú navyše kolmé.
Tieto postrehy stačia na zostrojenie kosoštvorca $ABCD$:
\begin{itemize}
\item zostrojíme dve rovnobežné priamky vo vzdialenosti 5\,cm (napr. ako kolmice v~koncových bodoch úsečky dĺžky 5\,cm),
\item na jednej priamke zvolíme bod $D$ a~zostrojíme kružnicu so stredom $D$ a~polomerom 8\,cm,
\item prienikom tejto kružnice s~druhou priamkou je bod $B$, resp. $B'$,
\item zostrojíme os úsečky $DB$, resp. $DB'$ (napr. pomocou priesečníkov dvoch zhodných kružníc so stredmi v~koncových bodoch),
\item prieniky tejto priamky s~rovnobežkami sú body $A$ a~$C$, resp. $A'$ a~$C'$,
\item štvoruholník $ABCD$, resp. $A'B'C'D$, je kosoštvorec s~požadovanými vlastnosťami.\insp{z8-I-1.eps}%
\end{itemize}



Všetky zostrojené kosoštvorca sú zhodné, úloha má až na zhodnosť jediné riešenie.

\poznamka
Konštrukciu je možné začať zostrojením úsečky $BD$ a~jej osi.
Ak dokážeme zostrojiť priamky idúce bodom $D$ vo vzdialenosti 5\,cm od $B$, tak ich priesečníky s~osou úsečky $BD$ budú zvyšné vrcholy $A$ a~$C$ kosoštvorca
(vzdialenosti bodu $B$ od priamok $AD$ a~$CD$ sú rovnaké, lebo to sú veľkosti výšok na prislúchajúce strany kosoštvorca).
Také priamky možno zostrojiť takto:
\begin{itemize}
\item zostrojíme kružnicu nad priemerom $BD$,
\item zostrojíme kružnicu so stredom $B$ a~polomerom 5\,cm,
\item spojnice priesečníkov $Q$ a~$R$ týchto kružníc s~bodom $D$ sú hľadané priamky.
\end{itemize}
Zdôvodnenie vyplýva z~Tálesovej vety: uhly $BQD$ a~$BRD$ sú pravé, teda úsečky $DQ$ a~$DR$ sú výšky na prislúchajúce strany.
\insp{z8-I-1a.eps}%


}

{%%%%%   Z8-I-2
\napad
Koľkociferné môžu byť súčty a~rozdiely päťciferných čísel a~ako by ste ich ručne počítali?

\riesenie
Súčet dvoch päťciferných čísel môže byť päťciferný alebo šesťciferný.
Päťciferné súčty však nemôžu začínať cifrou 1, teda súčet je šesťciferný.
Rozdiel dvoch päťciferných čísel môže byť nanajvýš päťciferný.
Kvôli jednoduchšiemu vyjadrovaniu si Richardove čísla označíme:
$$
\alggg{&A&B&C&D&E\\+&a&b&c&d&e}{1&1&*&*&*&1}
\hskip1cm
\alggg{&A&B&C&D&E\\-&a&b&c&d&e}{&*&*&*&1&1}
$$
Rôzne písmená predstavujú rôzne cifry, veľké zodpovedajú párnym, malé nepárnym, alebo naopak.
Keďže cifier je 10, budú v~Richardových číslach použité všetky.

Aby súčet začínal dvojčíslím 11, musí byť $A+a$ buď 11, alebo 10 (s~prechodom cez desiatku).
Avšak súčet párneho a~nepárneho čísla je nepárne číslo, teda musí byť
$A+a=11$.

Ak by rozdiel bol nanajvýš štvorciferný, tak by $A-a$ muselo byť nanajvýš 1.
Avšak rozdiel párneho a~nepárneho čísla je nepárne číslo, teda by muselo byť $A-a=1$.
Z~toho a~z~predchádzajúceho $A+a=11$ vychádza
$$
A=6,\quad a=5.
$$
Ak by rozdiel bol päťciferný a~začínal 2, tak by $A-a$ muselo byť buď 2, alebo 3 (prechod cez desiatku).
Avšak rozdiel párneho a~nepárneho čísla je nepárne číslo, teda by muselo byť $A-a=3$.
Z~toho a~z~predchádzajúceho $A+a=11$ vychádza
$$
A=7,\quad a=4.
$$

Aby súčet končil cifrou 1, musí byť buď $E+e=1$, alebo $E+e=11$.
Aby rozdiel končil cifrou 1, musí byť buď $E-e=1$, alebo $E=0$ a~$e=9$.
Predchádzajúce dve podmienky sú splnené súčasne buď pre
$$E=6,\quad e=5,
$$
alebo pre
$$E=1,\quad e=0.
$$

Predchádzajúca diskusia dáva štyri možnosti priradenia prvých a~posledných cifier Richardových čísel.
Aby boli splnené základné požiadavky
(\tj. aby sa žiadna cifra neopakovala a~aby jedno číslo bolo tvorené nepárnymi a~druhé párnymi ciframi),
musí byť $A=7$, $a=4$, $E=1$ a~$e=0$:
$$
\alggg{&7&B&C&D&1\\+&4&b&c&d&0}{1&1&*&*&*&1}
\hskip1cm
\alggg{&7&B&C&D&1\\-&4&b&c&d&0}{&2&*&*&1&1}
$$

Aby v~súčte bola na druhom mieste 1, musí byť $B+b<10$, a~aby rozdiel začínal 2, musí byť $B<b$.
Zo zvyšných cifier týmto podmienkam vyhovuje jedine $B=3$ a~$b=6$.
Aby v~rozdiele bola na predposlednom mieste 1, musí byť $D-d=1$.
Zo zvyšných cifier tejto podmienke vyhovuje jedine $D=9$ a~$d=8$.
Na posledné dve miesta tak ostáva $C=5$ a~$c=2$.

Richardove čísla boli 73591 a~46280, predchádzajúce výpočty vychádzajú nasledovne:
$$
\alggg{&7&3&5&9&1\\+&4&6&2&8&0}{1&1&9&8&7&1}
\hskip1cm
\alggg{&7&3&5&9&1\\-&4&6&2&8&0}{&2&7&3&1&1}
$$
}

{%%%%%   Z8-I-3
\napad
Znázornite situáciu na priamke a~uvažujte oba prípady zvlášť.

\riesenie
Označme prvú zastávku na trase autobusu $A$, druhú $B$, Vendelínov dom~$D$.
Keby Vendelín bežal do $A$, tak kým tam autobus dôjde, zabehne práve vzdialenosť~$DA$.
Keby Vendelín bežal do $B$, tak kým autobus dôjde do $A$, zabehne rovnakú vzdialenosť~$DA$ smerom k~$B$.
Najneskôr tu si uvedomujeme, že $D$ musí byť bližšie k~$A$, a~to v~troch osminách vzdialenosti~$AB$:
\insp{z8-I-3.eps}%


Teda pri behu do $B$ bude v~momente, keď autobus bude v~$A$, Vendelínovi do cieľa chýbať $\frac28=\frac14$ vzdialenosti $AB$.
Aby do $B$ dorazil súčasne s~autobusom, musí mať štvrtinovú rýchlosť vzhľadom na autobus.
Vendelín beží priemernou rýchlosťou ${60:4}=15$\,(km/h).
}

{%%%%%   Z8-I-4
\napad
Aké vlastnosti má číslo, ktoré je výsledkom opísaných operácií?

\riesenie
Výsledok opísaných operácií s~každým z~piatich čísel je stále rovnaký, a~ten si označíme~$n$.
Potom prvé až piate číslo je postupne
$$
n-1,\quad \sqrt n,\quad n+3,\quad \frac{n}4,\quad 5n.
$$
Z~druhého a~štvrtého čísla zisťujeme dôležité obmedzenia, konkrétne že $n$ je druhou mocninou prirodzeného čísla a~že je deliteľné štyrmi.

Najmenšie číslo s~týmito vlastnosťami je $n=4$; v~takom prípade by súčet čísel bol
$$
3+2+7+1+20=33,
$$
čo je málo.
Ďalšie číslo s~uvedenými vlastnosťami je $n=16$; v~takom prípade by súčet čísel bol
$$
15+4+19+4+80=122,
$$
čo je požadovaný výsledok.
Ďalšie možnosti nie je nutné skúšať, lebo s~rastúcim $n$ rastie aj celkový súčet.
Jediným riešením úlohy je pätica 15, 4, 19, 4, 80.

\ineriesenie
Päticu hľadaných čísel označíme $a$, $b$, $c$, $d$, $e$.
Zo zadania vyplýva, že
$$
a+1=b^2=c-3=4d=\frac{e}5.
$$

Aby $d=\frac{b^2}4$ bolo celé, musí byť $b^2$ násobkom 4, teda $b$ musí byť násobkom 2.
To vyjadríme tak, že $b=2k$ pre nejaké celé $k$.
Z~uvedených rovností odvodzujeme, že
$$
a=4k^2-1,\quad b=2k,\quad c=4k^2+3,\quad d=k^2,\quad e=20k^2.
$$

Súčet všetkých týchto čísel je
$s=29k^2+2k+2$.
Pre dostatočne veľké absolútne hodnoty~$k$ (\tj. bez ohľadu na znamienko) bude súčet väčší ako 122 a~ďalej sa bude strmo zväčšovať.
Postupným skúšaním zisťujeme nasledujúce:
$$
\begintable
$k$\|$\le-3$|$-2$|$-1$|0|1|2|$\ge3$\cr
$s$\|$\ge257$|114|29|2|33|122|$\ge269$\endtable
$$
Súčet je rovný 122 práve pre $k=2$, čo určuje päticu čísel
$$
a=15,\quad b=4,\quad c=19,\quad d=4,\quad e=80.
$$
}

{%%%%%   Z8-I-5
\napad
Viete porovnať obsahy niektorých útvarov priamo, \tj. bez konkrétnych hodnôt?

\riesenie
Trojuholníky $ABC$, $ABD$ a~$ABE$ majú spoločnú stranu $AB$ a~rovnakú výšku na túto stranu, teda majú rovnaký obsah.
Body $F$ a~$G$ sú stredmi úsečiek $AD$ a~$AC$, teda priamka $FG$ je rovnobežná s~$AB$.
Trojuholníky $ABF$ a~$ABG$ majú spoločnú stranu~$AB$ a~rovnakú výšku na túto stranu, ktorá je polovičná vzhľadom na výšku predchádzajúcich troch trojuholníkov.
Teda trojuholníky $ABF$ a~$ABG$ majú rovnaký obsah, ktorý je polovičný v~porovnaní s~obsahmi trojuholníkov $ABC$, $ABD$ a~$ABE$.
Celkom z~toho dostávame, že trojuholníky $AFE$, $ABF$, $BDF$, $ADG$, $ABG$ a~$BCG$ majú rovnaký obsah, a~to 12\,cm$^2$.

Trojuholník $ADG$ je zložený z~trojuholníka $AHF$ a~štvoruholníka $DFHG$, ktorý má obsah 8\,cm$^2$.
Obsah trojuholníka $AHF$ je teda rovný $12-8=4\,(\cm^2)$.
Podobne, obsah trojuholníka $BGH$ je taký istý.

Obsahy trojuholníkov $AFE$, $AHF$, $ABG$ a~$BGH$ sú postupne 12, 4, 12 a~4\,(cm$^2$).

\poznamka
Medzi zadanými objektmi sú vzťahy, ktoré sme nepoužili, ale ktoré dovoľujú alternatívne postupy pri riešení úlohy.
Napr. možno priamo ukázať, že štvoruholníky $ABCD$ a~$ABDE$ sú rovnobežníky, že trojuholníky $AHF$ a~$BGH$ majú rovnaký obsah, že ich súčet je rovný
obsahu štvoruholníka $DFHG$ a pod.

Pre zvedavých: v~úlohe neurčený obsah trojuholníka $ABH$ je rovný 8\,cm$^2$.
}

{%%%%%   Z8-I-6
\napad
Môžu byť jedny z~kocúrkovských mincí jednokoruny, alebo dvojkoruny,~\dots?

\riesenie
Budeme postupne uvažovať hodnotu jedných kocúrkovských mincí a~uvažovať o~hodnote druhých, aby boli splnené požiadavky zo zadania.

\smallskip
1) Ak by jedny z~mincí mali hodnotu 1, tak by bolo možné zaplatiť akúkoľvek sumu.

\smallskip
2) Predpokladajme, že jedny z~mincí majú hodnotu 2.
Druhé mince nemôžu mať párnu hodnotu, lebo s~takýmito mincami by bolo možné platiť iba párne sumy.

Druhé mince nemôžu mať nepárnu hodnotu menšiu ako 55, lebo s~takýmito mincami by bolo možné zaplatiť sumu 53 (napr. $2\cdot2+49$, $2+51$).
Keby druhé mince mali hodnotu 55, tak
akékoľvek párne čiastky by bolo možné zaplatiť pomocou dostatočného množstva 2
a~akákoľvek nepárna čiastka väčšia ako 53 by sa dala zaplatiť pomocou 55 a dostatočného množstva 2:
\bgroup
\thinsize=0pt
\thicksize=0pt
\def\ctr#1{\quad#1\hfil\quad}
$$
\begintable
$54=27\cdot2$|$56=28\cdot2$|$58=29\cdot2$|\dots\cr
$55=1\cdot55$|$57=2+1\cdot55$|$59=2\cdot2+1\cdot55$|\dots\endtable
$$

\smallskip
3) Predpokladajme, že jedny z~mincí majú hodnotu 3.
Druhé mince nemôžu mať hodnotu deliteľnú tromi, lebo s~takýmito mincami by sa nedali zaplatiť iné sumy ako tie deliteľné tromi.

Druhé mince nemôžu mať hodnotu 5, 8, 11, \dots, 53, lebo s~takýmito mincami by bolo možné zaplatiť sumu 53 (napr. $16\cdot3+5$, $15\cdot3+8$, \dots, $3+50$).
V~tejto skupine sme uvažovali hodnoty, ktoré po delení 3 dávajú zvyšok 2.
Ďalšie hodnoty s rovnakými zvyškami sú 56, 59 atď.
Ani tieto možnosti nevyhovujú, lebo s~takýmito mincami by sa nedala zaplatiť napr. čiastka 55 (ktorá je menšia a~po delení 3 dáva zvyšok 1).

Druhé mince nemôžu mať hodnotu 4, 7, 10, \dots, 25, lebo s~takýmito mincami by tiež bolo možné zaplatiť sumu 53 (napr. $15\cdot3+2\cdot4$, $13\cdot3+2\cdot7$, \dots, $3+2\cdot25$).
V~tejto skupine sme uvažovali hodnoty, ktoré po delení 3 dávajú zvyšok 1.
Keby druhé mince mali hodnotu 28, tak akékoľvek sumy väčšie ako 53 by sa dali platiť napr. takto:
$$
\begintable
$54=18\cdot3$|$57=19\cdot3$|$60=20\cdot3$|\dots\cr
$55=9\cdot3+1\cdot28$|$58=10\cdot3+1\cdot28$|$61=11\cdot3+1\cdot28$|\dots\cr
$56=2\cdot28$|$59=1\cdot3+2\cdot28$|$62=2\cdot3+2\cdot28$|\dots\endtable
$$
\egroup

\smallskip
Zatiaľ máme dve riešenia: na kocúrkovských minciach mohli byť hodnoty 2, 55 alebo 3, 28.
Obdobným (avšak o~niečo namáhavejším) spôsobom možno ešte odhaliť možné hodnoty 4, 19 a~7, 10.

\poznamka
Problém, ktorý riešime v~tejto úlohe, je známy ako tzv. Frobeniov problém:
vyhovujúce dvojice hodnôt mincí $a$, $b$ musia spĺňať $ab-a-b=53$, čiže
$$
(a-1)(b-1)=54.
$$
S~týmto (netriviálnym) poznatkom vieme nepriamo overiť, že uvedené štyri prípady sú jediné možné.
}

{%%%%%   Z9-I-1
\napad
Aké bolo rozdelenie orechov pred tým, ako mali všetci rovnako?

\riesenie
Nakoniec mali všetci rovnako, \tj. po 40 orechoch ($120:3$).
Predtým prisypával Kubo Ondrovi, a~to tak, že mu počet orechov zdvojnásobil.
Pred touto výmenou mal Ondro polovicu konečného stavu, takže sa presúvalo 20 orechov:
Ondro mal 20 a~Kubo 60 orechov (Maťo bezo zmeny).

Podobnými úvahami postupne odzadu určíme všetky výmeny orechov, a~tak zistíme pôvodné počty orechov:
$$\begintable
\|Ondro|Maťo|Kubo\crthick
pôvodne\hfill\|55|35|30\cr
Ondro Maťovi\hfill\|20|70|30\cr
Maťo Kubovi\hfill\|20|40|60\cr
Kubo Ondrovi\hfill\|40|40|40\endtable
$$

Pôvodne mal Ondro 55, Maťo 35 a~Kubo 30 orechov.

\ineriesenie
Ak pôvodné počty orechov označíme $x$, $y$ a~$z$, tak priebeh celej transakcie vyzeral takto:
$$\begintable
\|Ondro|Maťo|Kubo\crthick
pôvodne\hfill\|$x$|$y$|$z$\cr
Ondro Maťovi\hfill\|$x-y$|$2y$|$z$\cr
Maťo Kubovi\hfill\|$x-y$|$2y-z$|$2z$\cr
Kubo Ondrovi\hfill\|$2(x-y)$|$2y-z$|$2z-(x-y)$\endtable
$$

Podľa zadania platia nasledujúce rovnosti:
$$
2(x-y)=2y-z=2z-(x-y)=40.
$$

Rovnosť $2(x-y)=40$ je ekvivalentná $x-y=20$.

Dosadením do rovnosti $2z-(x-y)=40$ dostávame $2z=60$, teda $z=30$.

Dosadením do rovnosti $2y-z=40$ dostávame $2y=70$, teda $y=35$.

Nakoniec z~rovnosti $x-y=20$ dostávame $x=55$.

Pôvodne mal Ondro 55, Maťo 35 a~Kubo 30 orechov.
}

{%%%%%   Z9-I-2
\napad
V~opísanej spleti bodov možno nájsť viac trojuholníkov, ktorých obsahy možno porovnávať.

\riesenie
Body $C$, $Q$ a~$B$ ležia na jednej priamke a~uhly $PCB$ a~$RQB$ sú zhodné.
Teda priamky $PC$ a~$RQ$ sú rovnobežné, trojuholníky $PCQ$ a~$PCR$ majú rovnakú výšku na stranu $PC$, a~tak aj rovnaký obsah.
Pomer obsahov trojuholníkov $ABC$ a~$PCQ$ je preto rovnaký ako pomer obsahov trojuholníkov $ABC$ a~$PCR$.
\insp{z9-I-2b.eps}%


Trojuholníky $ABC$ a~$PRC$ majú rovnakú výšku zo spoločného vrcholu $C$, teda pomer ich obsahov je rovnaký ako pomer dĺžok strán $AB$ a~$PR$.
Pomer dĺžok úsečiek $AB$ a~$PB$ je $3:2$, pomer dĺžok úsečiek $PB$ a~$PR$ je $3:1$, teda pomer dĺžok úsečiek $AB$ a~$PR$ je $(3:2)\cdot(3:1)=9:2$.

Pomer obsahov trojuholníkov $ABC$ a~$PQC$ je $9:2$.

\ineriesenie
Rovnako ako v~predchádzajúcom riešení si všimnime, že priamky $PC$ a~$RQ$ sú rovnobežné.
Z toho vyplýva, že bod $Q$ na úsečke $CB$ je v rovnakom pomere ako bod $R$ na úsečke $PB$, \tj. v~jednej tretine bližšie bodu~$C$.
V rovnakom pomere sú preto aj~obsahy trojuholníkov $PQC$ a~$PQB$, lebo majú rovnakú výšku z~vrcholu $P$.
\insp{z9-I-2a.eps}%


Bod $Q$ na úsečke $CB$ je však tiež v rovnakom pomere ako bod $P$ na úsečke $AB$, teda trojuholníky $PQB$ a~$ACB$ sú podobné a~koeficient podobnosti je $2:3$.
Ich obsahy sú teda v~pomere $4:9$.

Dokopy,
pomer obsahov trojuholníkov $ABC$ a~$PQB$ je $9:4$,
pomer obsahov trojuholníkov $PQB$ a~$PQC$ je $2:1$,
teda pomer obsahov trojuholníkov $ABC$ a~$PQC$ je $(9:4)\cdot(2:1)=9:2$.
}

{%%%%%   Z9-I-3
\napad
Viete zadaný výraz nejako upraviť?

\riesenie
Zadaný výraz možno upraviť nasledovne (\uv{celá časť plus zvyšok}):
$$
\frac{x+11}{x+7} =\frac{(x+7)+4}{x+7} =\frac{x+7}{x+7}+\frac{4}{x+7} =1+\frac{4}{x+7}.
$$

Teda $\frac{x+11}{x+7}$ je celé číslo práve vtedy, keď $\frac{4}{x+7}$ je celé číslo, \tj. práve keď $x+7$ je deliteľom čísla 4.
Číslo 4 má šesť deliteľov: $-4$, $-2$, $-1$, 1, 2 a~4.
Zodpovedajúce hodnoty $x$ sú o~7 menšie.
Číslo $\frac{x+11}{x+7}$ je celé pre $x$ rovné $-11$, $-9$, $-8$, $-6$, $-5$ alebo $-3$.

\ineriesenie
Aby menovateľ nebol nulový, $x$ nemôže byť $-7$.
Dosadzovaním celých čísel v blízkosti tejto hodnoty dostávame:
$$
\begintable
$x$\|$-6$|$-8$\|$-5$|$-9$\|$-4$|$-10$\|$-3$|$-11$\|\dots\cr
$\frac{x+11}{x+7}$\|5|$-3$\|3|$-1$\|$\frac73$|$-\frac13$\|2|0\|\dots\endtable
$$

So zväčšujúcou sa vzdialenosťou $x$ od $-7$ dostávame v druhom riadku necelé čísla medzi 2 a~0, ktoré sa blížia k~1, ale nikdy ju nedosiahnu:
keby $\frac{x+11}{x+7}=1$, tak by $x+11=x+7$, čo nie je možné.
Všetky vyhovujúce možnosti sú teda obsiahnuté v~uvedenej tabuľke.
Číslo $\frac{x+11}{x+7}$ je celé pre $x$ rovné $-11$, $-9$, $-8$, $-6$, $-5$ alebo $-3$.

\poznamka
Predchádzajúcu argumentáciu považujte za vyhovujúcu, i~keď nie je dokonalá:
ukázať správne, že pre $x>-3$ alebo $x<-11$ sú hodnoty $\frac{x+11}{x+7}$ medzi 2 a~0, asi prekračuje možnosti riešiteľov v~tejto kategórii.

Táto diskusia súvisí s~monotónnym správaním funkcie $x\mapsto\frac{x+11}{x+7}$.
Zdanlivý zmätok v~hodnotách funkcie v~okolí $x=-7$ súvisí s~tým, že v~tomto bode nie je funkcia definovaná a~jej hodnoty nie sú nijako obmedzené.
Zvedaví riešitelia sa môžu zamyslieť nad správaním tejto funkcie, načrtnúť jej graf a~porovnať s~predchádzajúcimi závermi.
}

{%%%%%   Z9-I-4
\napad
Začnite pri druhom domorodcovi.

\riesenie
Či Poctivec, alebo Klamár, nikto o~sebe nemôže povedať, že je Klamár (Poctivec by klamal a~Klamár by hovoril pravdu).

Keď prvý domorodec oznamoval Matúšovi, že druhý domorodec je Klamár, určite nehovoril pravdu.
Je to teda Klamár, Matúš by mu veriť nemal a~uvedená cesta do prístavu nevedie.

\poznamka
Pre prehľadnosť uvádzame možné výroky domorodcov v~jednotlivých prípadoch:
\bgroup
\def\ctr#1{\quad#1\hfil\quad}
$$
\begintable
prvý|druhý\|druhý o sebe|prvý o druhom\crthick
Poctivec|Poctivec\|\uv{Poctivec}|\uv{Poctivec}\cr
Poctivec|Klamár\|\uv{Poctivec}|\uv{Poctivec}\cr
Klamár|Poctivec\|\uv{Poctivec}|\uv{Klamár}\cr
Klamár|Klamár\|\uv{Poctivec}|\uv{Klamár}\endtable
$$
\egroup
}

{%%%%%   Z9-I-5
\napad
Koľkociferné môžu byť súčiny trojciferných čísel a~ako by ste ich ručne počítali?

\riesenie
Súčin dvoch trojciferných čísel je aspoň päťciferný a~nanajvýš šesťciferný.
Pre ďalšie uvažovanie si Majkine čísla označíme a~naznačíme výpočet súčinu:
$$
\alggg{&a&B&c\\+&D&e&F}{1&6&1&7\\ \\ \\ \noalign{\vskip3pt}}
\hskip1cm
\algggg{&&&a&B&c\\&&\cdot&D&e&F}{&&*&*&*&0\\&*&*&*&*&\\*&*&*&*&&}{*&*&*&*&4&0}
$$
Rôzne písmená predstavujú rôzne cifry, veľké zodpovedajú párnym a~malé nepárnym.

Aby súčin končil 0, musí byť $c=5$ (medzi použitými ciframi nie je 0).
Aby súčet končil 7, musí byť $F=2$.

Aby v~súčte na mieste desiatok bola~1, musí byť $B+e=11$.
Párne~$B$ rôzne od~2 (a~od~0) a~nepárne $e$ rôzne od~5, ktoré spĺňajú túto podmienku, sú buď $B=4$, $e=7$, alebo $B=8$, $e=3$.
Rozoberieme obe možnosti:
\begin{itemize}
\item Predpokladajme, že $B=4$ a~$e=7$.
$$
\alggg{&a&4&5\\+&D&7&2}{1&6&1&7\\ \\ \\ \noalign{\vskip3pt}}
\hskip1cm
\algggg{&&&a&4&5\\&&\cdot&D&7&2}{&&*&*&9&0\\&*&*&1&5&\\*&*&*&*&&}{*&*&*&*&4&0}
$$
V~súčine vychádza posledné dvojčíslie 40 v~súlade so zadaním.
Aby súčet začínal dvojčíslím 16, musí byť $a+D=15$ (z~predchádzajúceho prechod cez desiatku).
Nepárne $a$ rôzne od 5 a~7 a~párne $D$ rôzne od 2 a~4 (a~0), ktoré spĺňajú túto podmienku, sú jedine $a=9$ a~$D=6$.
V~takom prípade súčin vychádza 635040.
\item Predpokladajme, že $B=8$ a~$e=3$.
$$
\alggg{&a&8&5\\+&D&3&2}{1&6&1&7\\ \\ \\ \noalign{\vskip3pt}}
\hskip1cm
\algggg{&&&a&8&5\\&&\cdot&D&3&2}{&&*&*&7&0\\&*&*&5&5&\\*&*&*&*&&}{*&*&*&*&4&0}
$$
V~tomto prípade by v~súčine vychádzalo posledné dvojčíslie 20 a~nie 40.
Táto možnosť teda vedie ku sporu s~požiadavkami zo zadania.
\end{itemize}

\noindent
Celkom vychádza jediná možnosť:
Majkine čísla sú 945 a~672, ich súčin je 635040.

\poznamka
Ak by sme po úvodnom odvodení $c=5$ a~$F=2$ pracovali so súčinom zapísaným opačne, tak si nemožno nevšimnúť, že bez ohľadu na hodnotu $e$ vychádza:
$$
\algggg{&&&D&e&2\\&&\cdot&a&B&5}{&&*&*&6&0\\&*&*&*&x&\\*&*&*&*&&}{*&*&*&*&4&0}
$$
(Cifra 6 v~prvom riadku v~medzivýsledku vyplýva z~nepárnosti $e$ a~prechodu cez desiatku.)
Aby posledné dvojčíslie súčinu vychádzalo 40, musí byť $x=8$, a~teda $B=4$.
Táto úvaha redukuje predchádzajúce skúšanie.
}

{%%%%%   Z9-I-6
\napad
Viete vzhľadom na opísanú vlastnosť Kristíninho čísla vyjadriť Jakubovo a~Dávidovo číslo?

\riesenie
Kristínino číslo, ktoré označíme $k$, je nepárne a~deliteľné tromi.
To znamená, že $k$ je nepárnym násobkom troch, teda
$$
k=3(2n+1)=6n+3
$$
pre nejaké prirodzené číslo $n$.
Veľkosti strán (v~mm) Jakubovho a~Dávidovho trojuholníka zodpovedajú trojiciam prirodzených čísel, ktoré v~súčte dávajú $k$ a~pre ktoré platia trojuholníkové nerovnosti (súčet každých dvoch čísel je väčší ako tretie).

Najväčšia možná dĺžka najdlhšej strany v~trojuholníku s~obvodom $k$ zodpovedá najväčšiemu prirodzenému číslu, ktoré je menšie ako polovica $k$ (kvôli trojuholníkovej nerovnosti).
Jakubovo číslo teda bolo $3n+1$.
Zvyšné dve strany mohli zodpovedať napr. $3n$ a~2, ale to nás pre doriešenie úlohy nezaujíma.

Najväčšia možná dĺžka najkratšej strany v~trojuholníku s~obvodom $k$ zodpovedá najväčšiemu prirodzenému číslu, ktoré je menšie ako tretina $k$ (príslušný trojuholník má byť čo najviac {rovnostranný}, avšak strany majú byť navzájom rôzne).
Dávidovo číslo teda bolo $2n$; zvyšné dve strany zodpovedali $2n+1$ a~$2n+2$.

Súčet Jakubovho a~Dávidovho čísla je $5n+1$, čo má podľa zadania zodpovedať 1681.
Z toho $5n=1680$, a~teda $n=336$.
Kristínino číslo bolo
$$
k=6\cdot336+3=2019.
$$
}

{%%%%%   Z4-II-1
...}

{%%%%%   Z4-II-2
...}

{%%%%%   Z4-II-3
...}

{%%%%%   Z5-II-1
Súčet všetkých sadeníc bol 606.
Teda každý z~troch zákazníkov mal dostať 202 sadeníc.

Počty sadeníc v~jednotlivých debničkách možno rozdeliť na tri skupiny so súčtom 202 niekoľkými spôsobmi:
\bgroup
\thinsize=0pt
\thicksize=0pt
\def\ctr#1{#1\quad\hfil}
$$\begintable
a) | 135+67, | 123+79, | 56+51+38+29+28, \cr
b) | 135+38+29, | 123+51+28, | 79+67+56, \cr
c) | 135+38+29, | 123+79, | 67+56+51+28, \cr
d) | 135+67, | 123+51+28, | 79+56+38+29.
\endtable
$$
\egroup

Tieto rozdelenia zodpovedajú možnostiam, ako mohol záhradník rozpredať svoje debničky.

\hodnotenie
2~body za celkový súčet a~zistenie, že každý mal dostať 202 sadeníc;
po 2~bodoch za každé z~dvoch správnych rozdelení.

\poznamky
Pri ručnom skúšaní možností je vhodné začínať s~väčšími číslami.

Uvedené čísla je možné rozdeliť na tri skupiny s~rovnakým súčtom bez toho, aby bolo nutné vopred určovať celkový súčet.
Tento postreh zohľadnite pri hodnotení.

V~každom z~uvedených prípadov je možné uvažovať šesť možných priradení troch skupín debničiek trom zákazníkom.
Riešenie založené na tejto myšlienke považujte tiež za správne.
\endhodnotenie
}

{%%%%%   Z5-II-2
Pre dvojicu štvorcov, z~ktorých jeden má vrcholy v~stredoch strán druhého, platí, že väčší štvorec má dvojnásobný obsah vzhľadom k~menšiemu štvorcu:
\insp{z5-II-2a.eps}%

Menší štvorec je totiž svojimi uhlopriečkami rozdelený na štyri zhodné trojuholníky a~tieto trojuholníky sú zhodné so štyrmi trojuholníkmi, ktoré patria do väčšieho, ale nie do menšieho štvorca.

Štvorce v~zadaní majú postupne (od najmenšieho) obsahy 1, 2, 4, 8 a~16\,cm$^2$.
Teda strana najväčšieho štvorca meria 4\,cm
a~jeho obvod je 16\,cm.

\ineriesenie
S~odkazom na ten istý poznatok ako v~predchádzajúcom riešení je možné štvorce v~zadaní rozdeliť na navzájom zhodné časti takto:
\insp{z5-II-2b.eps}%

Z~toho vyplýva, že strana najväčšieho štvorca má štvornásobnú dĺžku vzhľadom k~najmenšiemu, \tj. 4\,cm.
Teda obvod najväčšieho štvorca je 16\,cm.

\hodnotenie
3~body za určenie vzťahov medzi štvorcami;
3~body za určenie obvodu najväčšieho štvorca.
\endhodnotenie
}

{%%%%%   Z5-II-3
Medzi číslami 10 a~30 je 10 dielikov.
Teda jeden dielik má dĺžku 2 a~bodkami vyznačené čísla sú postupne 6, 8, 16, 22 a~32.
\insp{z5-II-3a.eps}%

Zo zadania vyplýva, že medzi týmito piatimi číslami sú dve menšie a~dve väčšie ako Jankovo obľúbené číslo.
Teda Jankovo číslo musí byť 16:
\begin{itemize}
\item 8 je polovicou 16,
\item 22 je o~6 väčšie ako 16,
\item 6 je o~10 menšie ako 16,
\item 32 je dvakrát väčšie ako 16.
\end{itemize}

\hodnotenie
3~body za určenie čísel vyznačených bodkami;
3~body za určenie Jankovho obľúbeného čísla
(z~toho 1~bod za kontrolu všetkých podmienok zo zadania).
\endhodnotenie
}

{%%%%%   Z6-II-1
...}

{%%%%%   Z6-II-2
...}

{%%%%%   Z6-II-3
...}

{%%%%%   Z7-II-1
...}

{%%%%%   Z7-II-2
...}

{%%%%%   Z7-II-3
...}

{%%%%%   Z8-II-1
...}

{%%%%%   Z8-II-2
...}

{%%%%%   Z8-II-3
...}

{%%%%%   Z9-II-1
Ak väčšie z~oboch čísel označíme $x$ a~menšie $y$, tak podmienka zo zadania znie
$$
(x+y)+(x-y)+xy+\frac{x}y
=98.
$$
Keďže súčet, rozdiel aj súčin čísel $x$ a~$y$ sú prirodzené čísla a~výsledkom je tiež prirodzené číslo, musí byť $x$ násobkom $y$, \tj. $x=ky$ pre nejaké prirodzené číslo~$k$.
Dosadíme do predchádzajúcej rovnosti, ktorú ďalej upravíme:
$$\align
2ky+ky^2+k &= 98, \\
k(y^2+2y+1) &=98, \\
k\cdot(y+1)^2 &= 2\cdot7^2.
\endalign
$$
Z toho vyplýva, že $k=2$ a~$y=6$, teda $x=12$.
(Prípadný rozklad $98\cdot 1^2$ sme vylúčili, lebo $y$ je prirodzené číslo, teda $y+1>1$.)
Patove a~Matove obľúbené čísla boli 6 a~12.

\hodnotenie
1~bod za zostavenie úvodnej rovnice;
3~body za zdôvodnenie $x=ky$, dosadenie do rovnice a~následné úpravy;
2~body za výsledok.

\poznamka
Záverečná úprava (doplnenie ľavej strany na štvorec a~rozklad pravej strany na súčin prvočísel) podstatne uľahčuje argumentáciu.
Riešenia bez tohto nápadu vyžadujú skúšania možností odvodené z~deliteľov, resp. rozkladov čísla~98.
Také postupy hodnoťte podľa kvality sprievodného komentára.
\endhodnotenie
}

{%%%%%   Z9-II-2
Číslo prislúchajúce najvrchnejšej kocke je určené štyrmi číslami z~druhej vrstvy, a~tie sú úplne určené číslami vo vrstve prvej.
Pritom každá rohová kocka prvej vrstvy susedí s~jednou kockou z~druhej vrstvy, každá nerohová kocka na hrane prvej vrstvy susedí s~dvoma kockami z~druhej vrstvy a~stredová kocka prvej vrstvy susedí so všetkými štyrmi kockami druhej vrstvy.
Teda do celkového súčtu v~najvrchnejšej kocke prispievajú štyri kocky jedenkrát, štyri dvakrát a~jedna štyrikrát.

Pre navzájom rôzne čísla v~prvej vrstve je najmenšie možné číslo v~najvrchnejšej kocke rovné
$$
1\cdot(9+8+7+6)+2\cdot(5+4+3+2)+4\cdot1=62.
$$
Najbližšie väčšie číslo deliteľné štyrmi je 64, čo je najmenšie číslo prislúchajúce najvrchnejšej kocke, ktoré vyhovuje všetkým požiadavkám.
Tento výsledok možno dosiahnuť napr. tak, že namiesto čísla 9 v~predchádzajúcom súčte vezmeme číslo~11.

\hodnoceni
2~body za určenie najmenšieho možného čísla;
2~body za určenie najmenšieho čísla deliteľného štyrmi a~konkrétnej vyhovujúcej realizácie;
2~body podľa kvality komentára.

\poznamka
Označme čísla v~prvej vrstve nasledovne:
$$
\aligned
a\quad b\quad c\\
d\quad e\quad f\\
g\quad h\quad i
\endaligned
$$
Potom čísla v druhej vrstve sú:
$$\aligned
a+b+d+e\quad b+c+e+f\\
d+e+g+h\quad e+f+h+i\\
\endaligned
$$
A~číslo v~najvrchnejšej kocke je:
$$
a+c+g+i+2(b+d+f+h)+4e
$$
\eres
}

{%%%%%   Z9-II-3
Označme pôvodné štvorciferné číslo $\overline{abcd}=1000a+100b+10c+d$.
Podľa zadania platí $\overline{abcd}=\overline{cdab}+99$, pričom $b$ a~$d$ sú celé čísla od 0 do 9, $a$ a~$c$ sú celé čísla od 1 do 9 a~navyše $a\ge c$.
Rozpísaním a~úpravou predchádzajúcej rovnosti dostávame:
$$\align
1000a+100b+10c+d &=1000c+100d+10a+b +99, \\
990a-990c &=99d-99b+99, \\
10(a-c) &= d-b+1.
\endalign
$$

Z~predchádzajúcich obmedzení vyplýva, že číslo $d-b+1$ na pravej strane môže nadobúdať hodnoty od ${-8}$ do 10.
Z~rovnosti navyše vyplýva, že toto číslo má byť násobkom 10, teda buď 10, alebo 0.
Obe možnosti rozoberieme zvlášť:
\begin{enumerate}\alphatrue
\item $10(a-c)=d-b+1=10$, tzn. $a=c+1$, $d=9$ a~$b=0$.
Čísel tohto tvaru je celkom osem (pre $c$ od 1 do 8):
$$
2019,\ 3029,\ 4039,\ 5049,\ 6059,\ 7069,\ 8079,\ 9089.
$$
Medzi nimi je iba číslo 5049 deliteľné deviatimi (ciferný súčet $2c+1+9$ je deliteľný deviatimi pre $c=4$).
\item $10(a-c)=d-b+1=0$, tzn. $a=c$ a~$b=d+1$.
V~tomto prípade môžeme dosadzovať $c$ od 1 do 9 a~$d$ od 0 do 8, \tj. celkom $9\cdot9=81$ možností:
$$\align
11&10,\ \ldots,\ 1918, \\
&\vdots \hskip1.6cm \vdots \\
91&90,\ \ldots,\ 9998.
\endalign
$$
Ciferný súčet čísel tohto tvaru je $2(c+d)+1$, a~ten je v~rámci našich obmedzení deliteľný deviatimi buď pre $c+d=4$, alebo pre $c+d=13$.
V~prvom prípade dostávame štyri čísla:
$$
1413,\ 2322,\ 3231,\ 4140.
$$
V druhom prípade dostávame päť čísel:
$$
5958,\ 6867,\ 7776,\ 8685,\ 9594.
$$
\end{enumerate}

Všetkých čísel vyhovujúcich prvej časti zadania je $8+81=89$.
Medzi týmito číslami je $1+4+5=10$ deliteľných deviatimi.

\hodnoceni
2~body za prípravné úpravy a~postrehy;
2~body za výsledok;
2~body za úplnosť a~kvalitu komentára.

\poznamka
Na zadanie úlohy je možné pozerať sa ako na algebrogram:
$$
\alggg{&c&d&a&b\\+&&&9&9}{&a&b&c&d}
$$
Niektoré z~predchádzajúcich úvah môžu byť v~tomto prevedení názornejšie.
\eres
}

{%%%%%   Z9-II-4
Rozbor:

Trojuholníky $ABC$ a~$ABD$ majú spoločnú stranu~$AB$, teda obsahy týchto trojuholníkov sú v~rovnakom pomere ako veľkosti ich výšok na stranu~$AB$.
Výška druhého trojuholníka preto musí byť polovičná vzhľadom k~výške prvého.
To znamená, že vrchol~$D$ leží na strednej priečke trojuholníka $ABC$ rovnobežnej so stranou~$AB$.
Táto priamka je určená bodmi $E$ a~$F$, čo sú postupne stredy strán $AC$ a~$BC$.

Trojuholníky $ABC$ a~$BCD$ majú spoločnú stranu~$BC$, teda obsahy týchto trojuholníkov sú v~rovnakom pomere ako veľkosti ich výšok na stranu~$BC$.
Výška druhého trojuholníka preto musí byť šestinová vzhľadom k~výške prvého.
To znamená, že vrchol~$D$ leží na priamke, ktorá je rovnobežná so stranou~$BC$ a~ktorej vzdialenosť od $BC$ je šestinová vzhľadom na vzdialenosť vrcholu $A$ od $BC$.
Táto priamka je určená bodmi $I$ a~$J$, ktoré ležia postupne v~šestinách úsečiek $BA$ a~$CA$, pozri obrázok.

Keďže stredná priečka~$EF$ je zhodná s~polovicou strany~$AB$, leží bod~$D$ v~tretine úsečky~$FE$, bližšie bodu~$F$.
\insp{z9-II-4.eps}%

Konštrukcia:
\begin{itemize}
\item Bod $E$ ako stred úsečky $AC$.
\item Bod $F$ ako stred úsečky $BC$.
\item Bod $D$ v~tretine úsečky $FE$, bližšie bodu~$F$.
\end{itemize}

Stred úsečky je možné zostrojiť pomocou priesečníkov zhodných kružníc so stredmi v~koncových bodoch alebo pomocou podobných trojuholníkov.
Tretinu úsečky je možné zostrojiť pomocou podobných trojuholníkov:
\insp{z9-II-4a.eps}%

\begin{itemize}
\item Body $K$, $L$, $M$ na ľubovoľnej priamke prechádzajúcej $E$ s~rovnakými vzdialenosťami $|EK|=|KL|=|LM|$.
\item Bod $D$ ako priesečník priamky $EF$ a~rovnobežky s~priamkou~$FM$ idúcej bodom~$L$.
\end{itemize}

\hodnoceni
3~body za rozbor úlohy s~jednoznačným vymedzením bodu~$D$;
3~body za konštrukciu bodu~$D$ (z~toho 1~bod za korektnú konštrukciu tretiny, resp. šestiny úsečky).

\poznamky
Riešenie možno založiť na správnom rozdelení výšok a~rovnobežkách s~príslušnými stranami.
Pomocné body $E$, $F$, $I$ a~$J$ na stranách trojuholníka nie sú na zostrojenie bodu~$D$ nevyhnutné.

Obsahy trojuholníkov $ABD$ a~$BCD$ sú postupne rovné polovici a~šestine obsahu trojuholníka $ABC$.
Obsah doplnkového trojuholníka $ACD$ je preto rovný tretine trojuholníka $ABC$ ($1-\frac12-\frac16=\frac13$).
Obsahy trojuholníkov $BCD$ a~$ACD$ sú teda v~pomere $1:2$ a~v~rovnakom pomere musia byť aj veľkosti úsečiek $DF$ a~$DE$.
Takto možno alternatívne zdôvodniť, že bod~$D$ leží v~tretine úsečky~$FE$.

Bod~$F$ je stredom úsečky~$BC$, teda priamka~$EF$ je ťažnicou trojuholníka $BCE$.
Bod~$D$ je v~tretine úsečky~$FE$ bližšie bodu~$F$, teda $D$ je ťažiskom trojuholníka $BCE$.
To znamená, že priamka~$BD$ je jeho ťažnicou, a~teda pretína úsečku~$CE$ v~jej strede.
Bod~$D$ možno alternatívne zostrojiť (bez delenia úsečiek na tretiny, resp. šestiny) ako priesečník priamok $EF$ a~$BG$, pričom $G$ je stredom úsečky~$CE$:
\insp{z9-II-4b.eps}%

\eres
}

{%%%%%   Z9-III-1
...}

{%%%%%   Z9-III-2
...}

{%%%%%   Z9-III-3
...}

{%%%%%   Z9-III-4
...}

