{%%%%% A-I-1
Pre kladné reálne čísla $a$, $b$, $c$, $d$ spĺňajúce nerovnosti $a~> b$, $c >d$ platí
$$
a~+ b > c + d, \qquad ab < cd.
$$
Dokážte, že potom nutne platí $a~> c > d > b$.
}
\podpis{Michal Rolínek}

{%%%%% A-I-2
Dokážte, že počet možností, ako sa dá útvar na \obr{} vydláždiť
dominovými kockami, je druhou mocninou celého čísla. (Dominová kocka
pokrýva vždy dve políčka susediace stranou.)
\inspdf{domino1d.pdf}%
}
\podpis{Josef Tkadlec}

{%%%%% A-I-3
Vnútri strán $AB$ a~$AC$ daného trojuholníka $ABC$
sú zvolené postupne body $P$ a~$Q$. Označme~$R$ priesečník priamok $BQ$ a~$CP$
a~$p$, $q$, $r$ postupne vzdialenosti bodov $P$, $Q$, $R$
od priamky~$BC$. Dokážte, že platí
$$
\belowdisplayskip 0pt
\frac{1}{p}+\frac{1}{q}>\frac{1}{r}.
$$
}
\podpis{Patrik Bak}

{%%%%% A-I-4
Hovoríme, že podmnožina $\mm P$ množiny $\mm M=\{1,2,3,\dots,42\}$ je
{\it polovičatá}, ak obsahuje $21$~prvkov a~každé zo~42~čísel
v~množinách $\mm P$ a~$\mm Q=\{7 x\; x\in\mm P\}$ dáva po delení
číslom~43 iný zvyšok. Určte počet polovičatých podmnožín množiny~$\mm M$.}
\podpis{Josef Tkadlec}

{%%%%% A-I-5
V~rovine sú dané dva rôzne body $O$ a~$A$. Určte množinu ortocentier
všetkých trojuholníkov $ABC$, pre ktoré je bod~$O$ stredom kružnice opísanej.}
\podpis{Pavel Šalom}

{%%%%% A-I-6
Nájdite všetky trojice $a$, $b$, $c$ kladných celých čísel
takých, že súčin
$$
(a+2b)(b+2c)(c+2a)
$$
je rovný mocnine niektorého prvočísla.}
\podpis{Jaromír Šimša}

{%%%%% B-I-1
V~reálnom obore uvažujme sústavu rovníc
$$
\eqalign{
x^4+y^2 &= \left(a+{1\over a}\right)^{\!3},\cr
x^4-y^2 &= \left(a-{1\over a}\right)^{\!3}
}$$
s~nenulovým reálnym parametrom~$a$.
\ite{a)} Nájdite všetky hodnoty $a$, pre ktoré má uvedená sústava riešenie.
\ite{b)} Dokážte, že pre ľubovoľné riešenie $(x, y)$ tejto sústavy platí $x^2+|y|\ge 4$.
}
\podpis{Ján Mazák}

{%%%%% B-I-2
Prirodzené číslo~$n$ má aspoň 73~dvojciferných deliteľov. Dokážte, že jedným
z~nich je číslo~60.
Uveďte tiež príklad čísla~$n$, ktoré má práve 73~dvojciferných deliteľov,
vrátane náležitého zdôvodnenia.}
\podpis{Josef Tkadlec}

{%%%%% B-I-3
Nech $AC$ je priemer kružnice opísanej tetivovému štvoruholníku $ABCD$. Predpokladajme,
že na polpriamkach opačných
k~polpriamkam $AD$ a~$DC$ existujú postupne body $A'\ne A$ a~$C'\ne D$ také, že
platí $|AB|=|A'B|$ a~$|BC|=|BC'|$. Dokážte tvrdenia:
\ite{a)} Body $A'$, $B$, $C'$ a~$D$ ležia na jednej kružnici~$k$.
\ite{b)} Ak je $O$ stred kružnice $k$ a~$O_A$, $O_C$ sú postupne stredy
kružníc opísaných trojuholníkom $AA'B$, $CC'B$, tak platí $OO_A\perp OO_C$.\endgraf}
\podpis{Jaroslav Švrček}

{%%%%% B-I-4
Nech $p$, $q$ sú dané nesúdeliteľné prirodzené čísla. Dokážte, že ak má rovnica
$$
px^2-(p+q)x+p=0
$$
celočíselný koreň, tak má celočíselný koreň aj rovnica
$$
\belowdisplayskip 0pt
px^2+qx+p^2-q=0.
$$
}
\podpis{Patrik Bak}

{%%%%% B-I-5
Dané sú kružnice $a(A; r_a)$, $b(B; r_b)$, ktoré sa zvonka dotýkajú v~bode~$T$.
Ich spoločná vonkajšia dotyčnica sa dotýka kružnice~$a$ v~bode~$T_a$ a~kružnice~$b$
v~bode~$T_b$. Pomocou $r_a$,~$r_b$ vyjadrite pomer polomerov kružníc $k_a$, $k_b$
opísaných postupne trojuholníkom $T_aAT$, $T_bBT$.}
\podpis{Šárka Gergelitsová}

{%%%%% B-I-6
Figúrka strelca ohrozuje na šachovnici ľubovoľné políčko diagonály, na ktorej strelec stojí.
Ak však na niektorom políčku diagonály stojí veža, strelec už políčka za ňou neohrozuje.
Určte najväčší možný počet strelcov, ktorých môžeme spolu so štyrmi vežami umiestniť
na šachovnicu $8\times8$ tak, aby sa strelci navzájom neohrozovali.}
\podpis{Ján Mazák}

{%%%%% C-I-1
Nájdite všetky štvorciferné čísla $\overline{abcd}$ s~ciferným
súčtom~12 také, že $\overline{ab}-\overline{cd}=1$.}
\podpis{Patrik Bak}

{%%%%% C-I-2
Daný je konvexný šesťuholník $ABCDEF$, ktorého všetky strany sú zhodné
a~protiľahlé strany rovnobežné. Bod~$P$ je taký, že štvoruholník
$CDEP$ je rovnobežník. Dokážte, že bod~$P$ je stredom kružnice opísanej
trojuholníku $ACE$ a~súčasne aj priesečníkom výšok trojuholníka $BDF$.}
\podpis{Jakub Löwit}

{%%%%% C-I-3
Určte všetky dvojice prirodzených čísel $a$ a~$b$, pre ktoré platí
$$
2[a,b]+3(a,b)=ab,
$$
pričom $[a,b]$ označuje najmenší spoločný násobok a~$(a,b)$
najväčší spoločný deliteľ prirodzených čísel $a$ a~$b$.}
\podpis{Jaroslav Švrček}

{%%%%% C-I-4
Vnútri strany~$BC$ trojuholníka $ABC$ je daný bod~$K$. Označme $M$ stred
strany~$BC$ a~predpokladajme, že rovnobežka s~priamkou~$AK$ vedená bodom~$M$
pretína stranu~$AC$ vo vnútornom bode~$L$. Dokážte, že priamka~$KL$ delí
trojuholník $ABC$ na dve časti s~rovnakým obsahom.}
\podpis{Josef Tkadlec}

{%%%%% C-I-5
Tabuľku $3\times 3$ máme vyplniť deviatimi danými číslami tak, aby
v~každom riadku aj stĺpci bolo najväčšie číslo súčtom ostatných dvoch.
Rozhodnite, či je možné takú úlohu splniť s~číslami
\ite a) 1, 2, 3, 4, 5, 6, 7, 8, 9;
\ite b) 2, 3, 4, 5, 6, 7, 8, 9, 10.
\endgraf\noindent
Ak áno, zistite, koľkými spôsobmi možno úlohu splniť tak, aby
najväčšie číslo bolo uprostred tabuľky.}
\podpis{Jaromír Šimša}

{%%%%% C-I-6
Pre kladné reálne čísla $a$, $b$, $c$ platí $a^2+b^2+c^2+ab+bc+ca\le1$.
Nájdite najväčšiu možnú hodnotu súčtu $a+b+c$.}
\podpis{Ján Mazák}

{%%%%% A-S-1
Predpokladajme, že navzájom rôzne reálne čísla $a$, $b$, $c$, $d$ spĺňajú nerovnosti
$$
ab+cd > bc+ad > ac+bd.
$$
Ak $a$ je z~týchto štyroch čísel najväčšie, ktoré z~nich je najmenšie?
}
\podpis{Josef Tkadlec}

{%%%%% A-S-2
Pre trojuholníky $ABC$ a~$A'B'C'$ platí
$$
|AB| = |A'B'|, \quad |AC| = |A'C'|, \quad |\uhel BAC| + |\uhel B'A'C'| = 180^\circ.
$$
Ukážte, že veľkosť uhla zovretého stranou~$BC$ a~ťažnicou
z~vrcholu~$A$ je rovnaká ako veľkosť uhla zovretého
stranou~$B'C'$ a~ťažnicou z~vrcholu~$A'$.}
\podpis{Patrik Bak}

{%%%%% A-S-3
Ukážte, že počet spôsobov, ktorými možno vydláždiť útvar
\insp{a69.0}
dominovými kockami, sa dá vyjadriť ako súčet dvoch druhých mocnín
prirodzených čísel.}
\podpis{Josef Tkadlec}

{%%%%% A-II-1
Nájdite všetky reálne riešenia sústavy rovníc
$$
{1\over x + y}+ z~= 1,\quad
{1\over y + z}+ x = 1,\quad
{1\over z~+ x}+ y = 1.
$$
}
\podpis{Tomáš Bárta}

{%%%%% A-II-2
V~ostrouhlom trojuholníku $ABC$ označme $O$ stred kružnice
opísanej. Obraz bodu~$O$ v~osovej súmernosti podľa priamky~$AC$ označme~$P$.
Dokážte, že stredy úsečiek $AO$ a~$BP$ ležia na tej istej kolmici na priamku~$BC$.}
\podpis{Patrik Bak}

{%%%%% A-II-3
Pre ľubovoľnú štvorprvkovú podmnožinu $\mm P$
množiny $\{1,2,3,\dots,12\}$ označme
$$
\mm Q=\{3x\colon x\in\mm P\}\quad\text{a}\quad \mm R=\{4x\colon x\in \mm P\}.
$$
Určte počet takých množín~$\mm P$, pre ktoré čísla z~$\mm P$, $\mm Q$, $\mm R$
dávajú po delení číslom~13 všetky možné nenulové zvyšky.}
\podpis{Jaromír Šimša}

{%%%%% A-II-4
Určte, koľkými spôsobmi možno útvar %na obrázku
\insp{a69.100}
vydláždiť dominovými kockami.}
\podpis{Josef Tkadlec}

{%%%%% A-III-1
Na tabuli sú napísané dve kladné celé čísla $m$ a~$n$. V~každom kroku jedno z~dvoch čísel
na tabuli nahradíme buď ich súčtom, alebo súčinom, alebo podielom (ak je celočíselný).
V~závislosti od čísel $m$ a~$n$
určte všetky dvojice, ktoré sa môžu na tabuli po niekoľkých krokoch objaviť.
}
\podpis{Radovan Švarc}

{%%%%% A-III-2
Daný je trojuholník $ABC$. Vnútri jeho strán $AB$ a~$AC$
sú postupne zvolené body $X$ a~$Y$. Označme~$Z$ priesečník úsečiek $BY$ a~$CX$.
Dokážte nerovnosť $$[BZX]+[CZY]>2[XYZ],$$ pričom $[DEF]$ označuje obsah trojuholníka $DEF$.}
\podpis{David Hruška, Josef Tkadlec}

{%%%%% A-III-3
Uvažujme sústavu rovníc
$$
\align
x^2-3y+p =& z,\\
y^2-3z+p =& x,\\
z^2-3x+p =& y
\endalign
$$
s~reálnym parametrom~$p$.
\ite a) Pre $p\ge4$ vyriešte uvažovanú sústavu v~obore reálnych čísel.
\ite b) Dokážte, že pre $p\in\langle1, 4)$ každé reálne riešenie sústavy
spĺňa ${x=y=z}$.\endgraf}
\podpis{Jaroslav Švrček}

{%%%%% A-III-4
Kladné celé čísla $a$, $b$ spĺňajú rovnosť $b^2 = a^2 + ab + b$.
Dokážte, že $b$ je druhou mocninou kladného celého čísla.}
\podpis{Patrik Bak}

{%%%%% A-III-5
Daný je rovnoramenný trojuholník $ABC$ so základňou~$BC$, vnútri ktorej
je zvolený bod~$D$. Nech $E$, $F$ sú postupne také body na
stranách $AB$, $AC$, že platí $|\uhel BED| = |\uhel DFC| > 90^\circ$.
Dokážte, že kružnice opísané trojuholníkom $ABF$ a~$AEC$ sa pretínajú na
priamke~$AD$ v~bode rôznom od bodu~$A$.}
\podpis{Patrik Bak, Michal Rolínek}

{%%%%% A-III-6
Pre každé kladné celé číslo~$k$ označme $P(k)$ počet všetkých kladných celých $4k$-ciferných čísel, ktoré možno zostaviť z~cifier $2$, $0$ a~ktoré
sú deliteľné číslom~$2\,020$. Dokážte nerovnosť
$$
P(k)\ge \binom{2k-1}{k}^{\!2}
$$
a~určte všetky $k$, pre ktoré nastane rovnosť.

({\it Poznámka.} Zápis kladného celého čísla nemôže začínať cifrou 0.)}
\podpis{Jaromír Šimša}

{%%%%% B-S-1
Nájdite všetky celé čísla $a$, pre ktoré má rovnica
$$
a(x^2+x)=3(x^2+1)
$$
aspoň jeden celočíselný koreň.}
\podpis{Jaromír Šimša}

{%%%%% B-S-2
Dokážte, že stredy kružníc zvonka pripísaných jednotlivým stranám
ľubovoľného konvexného štvoruholníka ležia na jednej kružnici.

(Kružnicou pripísanou napríklad strane~$AB$ konvexného štvoruholníka $ABCD$ rozumieme
kružnicu, ktorá sa dotýka strany~$AB$ a~polpriamok opačných
k~polpriamkam $AD$ a~$BC$.)}
\podpis{Jaroslav Švrček, Pavel Calábek}

{%%%%% B-S-3
Určte najväčšie prirodzené číslo~$k$, pre ktoré možno na šachovnicu
$8\times 8$ rozmiestniť $k$~veží a~$k$~strelcov tak, aby žiadna
figúrka neohrozovala inú. (Strelec ohrozuje ľubovoľné políčko diagonály a~veža
ľubovoľné políčko riadka aj stĺpca, na ktorých stoja.)}
\podpis{Josef Tkadlec}

{%%%%% B-II-1
Navzájom rôzne nenulové reálne čísla $a$, $b$, $c$ sa dajú šiestimi spôsobmi
doplniť ako koeficienty kvadratickej rovnice
$$
\boxed{\phantom{0}}\ x^2 + \boxed{\phantom{0}}\ x + \boxed{\phantom{0}} = 0.
$$
\ite a) Rozhodnite, či existuje trojica ($a$, $b$, $c$) taká, že
všetky zostavené rovnice majú aspoň jeden reálny koreň.
\ite b) Rozhodnite, či existuje trojica ($a$, $b$, $c$) taká, že
práve päť zo šiestich zostavených rovníc má aspoň jeden reálny koreň.\endgraf}
\podpis{Michal Rolínek}

{%%%%% B-II-2
Daný je ostrouhlý trojuholník s~nezhodnými stranami $AC$, $BC$
a~priesečníkom výšok~$V$. Na priamke~$AB$ zostrojme body $A'$, $B'$
($A'\ne A$, $B'\ne B$) také, že $|CA'|=|CA|$ a~$|CB'|=|CB|$. Dokážte,
že kružnice opísané trojuholníkom $ACB'$ a~$VCA'$ sú zhodné.}
\podpis{Jaroslav Švrček}

{%%%%% B-II-3
Určte najmenšie prirodzené číslo~$n$, pre ktoré platí: Ak niektoré
prirodzené číslo má aspoň $n$~trojciferných násobkov, tak 840 je jedným
z~nich.}
\podpis{Michal Rolínek}

{%%%%% B-II-4
Netradičná figúrka, ktorú nazveme "chorá dáma", ohrozuje
ľubovoľné políčko riadka aj stĺpca, na ktorých stojí, zatiaľ čo na diagonále
ohrozuje iba políčko susedné.
Koľko najmenej "chorých dam" potrebujeme rozmiestniť na šachovnicu $8\times8$
tak, aby ohrozovali všetky neobsadené políčka?
\inspdf{nemocna-dama2.pdf}%
}
\podpis{Tomáš Bárta, Josef Tkadlec}

{%%%%% C-S-1
Určte všetky prirodzené čísla $n$, pre ktoré platí
$$
n+s(n)=2\,019,
$$
pričom $s(n)$ označuje ciferný súčet čísla~$n$.
}
\podpis{Jaroslav Švrček}

{%%%%% C-S-2
Tabuľka $3\times 3$ je vyplnená navzájom rôznymi prirodzenými číslami
tak, že v~každom riadku aj stĺpci je súčet krajných čísel rovný číslu
napísanému medzi nimi. Zistite, aké najmenšie číslo môže byť napísané
uprostred tabuľky.}
\podpis{Tomáš Jurík}

{%%%%% C-S-3
Nájdite všetky pravouhlé trojuholníky s~celočíselnými dĺžkami strán,
ktorých kružnica vpísaná má polomer~2.}
\podpis{Jaroslav Zhouf}

{%%%%% C-II-1
Nájdite všetky dvojice prirodzených čísel $a$ a~$b$, ktorých najväčší
spoločný deliteľ je rovný obom číslam $30-a$ aj~$42-b$.}
\podpis{Patrik Bak}

{%%%%% C-II-2
Konvexný osemuholník $ABCDEFGH$ má všetky strany rovnako dlhé
a~protiľahlé dvojice strán rovnobežné. Uvažujme body $X$, $Y$, $Z$ také, že
štvoruholníky $ABCX$, $DEFY$, $GHAZ$ sú rovnobežníky.
Dokážte, že $XZ\perp AY$.}
\podpis{Josef Tkadlec}

{%%%%% C-II-3
Koľko trojciferných čísel má tú vlastnosť, že vyškrtnutím niektorej
cifry dostaneme dvojciferné číslo, ktoré je druhou mocninou nejakého
celého čísla?}
\podpis{Tomáš Bárta, Tomáš Jurík}

{%%%%% C-II-4
Pre nezáporné reálne čísla $a$, $b$, $c$ platí $a+b+c=1$. Nájdite
najväčšiu a~najmenšiu možnú hodnotu výrazu
$$
(a+b)^2+(b+c)^2+(c+a)^2.
$$
}
\podpis{Ján Mazák}

{%%%%%   vyberko, den 1, priklad 1
[C1] Alexandra a~Benjamín stavajú stenu na ploche $1 \times m$ políčok. Striedavo umiestňujú tehly, pričom Alexandra umiestňuje zelené tehly a~Benjamín červené. Začína Alexandra. Výška steny nikdy nesmie presiahnuť~$n$ a~v~momente, keď je umiestnených $m \times n$ tehiel, hra končí. Alexandra vyhrá, ak sa jej podarilo vytvoriť riadok pozostávajúca iba zo zelených tehiel; inak vyhrá Benjamín. Nájdite všetky dvojice $(m, n)$, pre ktoré vie Alexandra zaručene vyhrať, nech hrá Benjamín ľubovoľne.}
\podpis{...}

{%%%%%   vyberko, den 1, priklad 2
[G2] Je daný ostrouhlý trojuholník $ABC$ s~pätami výšok $D$, $E$, $F$ po rade z~vrcholov $A$, $B$, $C$. Označme $\omega_B$ a~$\omega_C$ kružnice vpísané trojuholníkom $BDF$ a~$CDE$. Ďalej označme $M$ a~$N$ po rade dotykové body $\omega_B$ a~$\omega_C$ s~úsečkami~$DF$ a~$DE$. Priamka $MN$ pretína kružnice $\omega_B$ a~$\omega_C$ po rade v~bodoch $P \ne M$ a~$Q \ne N$. Dokážte, že $|MP|=|NQ|$.}
\podpis{...}

{%%%%%   vyberko, den 1, priklad 3
[A3] Nech $n \ge 2$ je kladné celé číslo a~$a_1,\ldots,a_n$ sú reálne čísla so súčtom~0. Definujme množinu~$A$ nasledovne:
$$
A=\{(i,j)\;|\;1 \leq i < j \leq n,\,|a_i-a_j|\ge 1\}.
$$
Dokážte, že ak je množina $A$ neprázdna, tak
$$
\sum_{(i,j)\in A} a_i a_j < 0.
\belowdisplayskip0pt
$$}
\podpis{...}

{%%%%%   vyberko, den 2, priklad 1
[A1] Pre všetky celé čísla $n>1$ nájdite všetky nekonštantné polynómy~$P(x)$ s~reálnymi koeficientami, ktoré pre všetky reálne čísla~$x$ spĺňajú
$$
P(x)\cdot P(x^2) \cdots P(x^n) = P\!\left(x^{\frac12n(n+1)}\right).
\belowdisplayskip0pt
$$}
\podpis{...}

{%%%%%   vyberko, den 2, priklad 2
[N2] Nájdite všetky trojice $(a,b,c)$ kladných celých čísel spĺňajúcich $a^3+b^3+c^3=(abc)^2$.}
\podpis{...}

{%%%%%   vyberko, den 2, priklad 3
[G3] Body $X$ a~$Y$ ležia vnútri strán $AB$ a~$AC$ ostrouhlého trojuholníka $ABC$ tak, že obraz priamky $BC$ v~osovej súmernosti podľa priamky $XY$ je dotyčnica ku kružnici~$\omega$ opísanej trojuholníku~$AXY$. Označme~$O$ stred kružnice opísanej trojuholníku~$ABC$. Dokážte, že kružnica opísaná trojuholníku~$BCO$ sa dotýka~$\omega$.}
\podpis{...}

{%%%%%   vyberko, den 3, priklad 1
[A0] Nech $f : {\Bbb Z} \to {\Bbb Z}$ je funkcia taká, že pre všetky celé čísla~$x$ a~$y$ platí
$$
f(f(x)-y)=f(y)-f(f(x)).
$$
Dokážte, že funkcia~$f$ je ohraničená.}
\podpis{...}

{%%%%%   vyberko, den 3, priklad 2
[N0] Pre kladné celé číslo~$N$ označme~$f(N)$ počet usporiadaných dvojíc kladných celých čísel~$(a,b)$ takých, že číslo
$$\frac{ab}{a+b}$$
je celé a~je deliteľom~$N$. Dokážte, že pre každé kladné celé číslo~$n$ platí, že $f(n)$ je druhá mocnina celého čísla.}
\podpis{...}

{%%%%%   vyberko, den 3, priklad 3
[G0] V~ostrouhlom trojuholníku $ABC$ pretína os vnútorného uhla $BAC$ stranu~$BC$ v~bode~$D$. Os úsečky $AD$ pretína kružnicu opísanú trojuholníku $ABC$ v~bodoch~$E$ a~$F$. Dokážte, že kružnica opísaná trojuholníku $DEF$ sa dotýka $BC$.}
\podpis{...}

{%%%%%   vyberko, den 3, priklad 4
[C0] V~mestečku Výberkovo existujú tri školy $A$, $B$, $C$, pričom každú navštevuje aspoň jeden študent a~každý študent navštevuje práve jednu školu. Pre ľubovoľnú trojicu študentov takých, že jeden navštevuje~$A$, jeden navštevuje~$B$ a~jeden navštevuje~$C$, platí, že nejakí dvaja z~nich sa poznajú a~nejakí dvaja z~nich sa nepoznajú. Dokážte, že aspoň jedno z~nasledovných tvrdení je pravdivé:
\item{(i)} Nejaký študent z~$A$ pozná všetkých študentov z~$B$.
\item{(ii)} Nejaký študent z~$B$ pozná všetkých študentov z~$C$.
\item{(iii)} Nejaký študent z~$C$ pozná všetkých študentov z~$A$.}
\podpis{...}

{%%%%%   vyberko, den 4, priklad 1
[N1] Nájdite všetky možné hodnoty výrazu $a^3+b^3+c^3-3abc$, kde $a,b,c$ sú nezáporné celé čísla.}
\podpis{...}

{%%%%%   vyberko, den 4, priklad 2
[A2] Nech $u_1,\ldots,u_{2020}$ sú reálne čísla spĺňajúce
$$
\eqalign{
u_1+\ldots+u_{2020}&=0, \cr
u_1^2+\ldots+u_{2020}^2&=1.
}
$$
Označme $a=\min\{u_1,\ldots,u_{2020}\}$ a $b=\max\{u_1,\ldots,u_{2020}\}$. Dokážte, že
$$
ab \leq -\frac1{2020}.
\belowdisplayskip0pt
$$}
\podpis{...}

{%%%%%   vyberko, den 4, priklad 3
[C3] Na stole leží~69 prázdnych krabíc $K_1,\ldots,K_{69}$. Máme k~dispozícií neobme\-dze\-né množstvo guľôčok. Je dané celé číslo~$n$. Anastázia a~Bartolomej hrajú nasledovnú hru: V~prvom kole Anastázia vezme~$n$ guľôčok a~rozdelí ich do 69~krabíc podľa svojho priania. Každé ďalšie kolo pozostáva z~nasledovných krokov:
	\ite a) Bartolomej si zvolí celé číslo~$i$ spĺňajúce $1 \leq i \leq 68$ a rozdelí krabice na dve skupiny $K_1,\ldots,K_i$ a~$K_{i+1},\ldots,K_{69}$.
	\ite b) Anastázia si vyberie jednu z~týchto skupín, pridá po jednej guľôčke do každej krabice z~vybratej skupiny a~odstráni po jednej guľôčke z~každej krabice z~druhej skupiny.
	
Bartolomej vyhráva, ak na konci nejakého kola zostane nejaká krabica prázdna. Nájdite najmenšie~$n$ také, že Anastázia vie zabrániť Bartolomejovi vo víťazstve.}
\podpis{...}

{%%%%%   vyberko, den 5, priklad 1
[G1] Je daný trojuholník~$ABC$. Kružnica~$\Gamma$ prechádzajúca bodom~$A$ pretína úsečky $AB$ a~$AC$ po rade v~jej vnútorných bodoch~$D$ a~$E$, a~tiež pretína úsečku $BC$ v~jej vnútorných bodoch~$F$ a~$G$ tak, že~$F$ leží medzi~$B$ a~$G$. Dotyčnica ku kružnici opísanej $BDF$ v~$F$ a~dotyčnica ku kružnici opísanej $CEG$ v~$G$ sa pretínajú v~bode~$T$. Predpokladajme, že $A \ne T$. Dokážte, že $AT \parallel BC$.}
\podpis{...}

{%%%%%   vyberko, den 5, priklad 2
[C2] Je daná množina~$n$ cukríkov, pričom každý váži aspoň~1~gram, a ich celková váha je~$2n$ gramov. Dokážte, že pre každé reálne číslo~$r$ spĺňajúce $0 \leq r \leq 2n-2$ vieme vybrať cukríky, ktorých celková váha je aspoň~$r$ gramov, avšak najviac $r+2$ gramov.}
\podpis{...}

{%%%%%   vyberko, den 5, priklad 3
[N3] Nech ${\Bbb Z}_{>0}$ označuje množinu kladných celých čísel. Je daná kladná celočíselná konštanta~$K$. Nájdite všetky funkcie $f\colon {\Bbb Z}_{>0} \rightarrow {\Bbb Z}_{>0}$ také, že pre všetky kladné celé čísla $a$, $b$ spĺňajúce $a+b>K$ platí
$$
\abovedisplayskip0pt
a+f(b) \mid a^2+bf(a).
\belowdisplayskip0pt
$$}
\podpis{...}

{%%%%%   vyberko, den 1, priklad 4
...}
\podpis{...}

{%%%%%   vyberko, den 2, priklad 4
...}
\podpis{...}

{%%%%%   vyberko, den 4, priklad 4
...}
\podpis{...}

{%%%%%   vyberko, den 5, priklad 4
...}
\podpis{...}

{%%%%%   trojstretnutie, priklad 1
Daný je rovnobežník $ABCD$, ktorého uhlopriečky sa pretínajú v~bode~$P$. Označme $M$ stred strany~$AB$. Nech $Q$ je taký bod, že priamka~$QA$ je dotyčnicou kružnice opísanej trojuholníku $MAD$ a~$QB$ je dotyčnicou kružnice opísanej trojuholníku $MBC$. Dokážte, že body $Q$, $M$, $P$ ležia na jednej priamke.}
\podpis{Patrik Bak, Slovensko}

{%%%%%   trojstretnutie, priklad 2
Pre dané kladné celé číslo $n$ hovoríme, že reálne číslo $x$ je $n$-dobré, ak existuje $n$~kladných celých čísel $a_1,\dots,a_n$ takých, že
$$
x= \frac1{a_1}+\dots+\frac1{a_n}.
$$
Nájdite všetky kladné celé čísla~$k$, pre ktoré je pravdivé nasledovné tvrdenie: {\it Ak $a$, $b$ sú reálne čísla také, že uzavretý interval $\langle a,b\rangle$ obsahuje nekonečne veľa 2020-dobrých čísel, tak interval $\langle a,b\rangle$ obsahuje aspoň jedno $k$-dobré číslo.}}
\podpis{Josef Tkadlec, Česká rep.}

{%%%%%   trojstretnutie, priklad 3
Na tabuli sú napísané čísla $1,2,\dots,2020$. Venuša and Serena hrajú nasledujúcu hru. Najskôr Venuša spojí úsečkou dve rôzne čísla také, že jedno z~nich delí druhé. Potom Serena spojí úsečkou dve rôzne čísla také, ktoré ešte nie sú spojené a jedno z~nich delí druhé. Potom rovnako pokračuje Venuša a následne sa striedajú až do momentu, keď vznikne trojuholník s~vrcholom v~čísle 2020, t.\,j. 2020 je spojené s~dvoma číslami, ktoré sú spojené navzájom. Hráčka, ktorá nakreslí poslednú úsečku (dokončí trojuholník), vyhráva. Ktorá z~nich má vyhrávajúcu stratégiu?}
\podpis{Tomáš Bárta, Česká rep.}

{%%%%%   trojstretnutie, priklad 4
Dané je reálne číslo $\alpha$.
Nájdite všetky funkcie $f\colon\Bbb R \rightarrow \Bbb R$ také, že
$$
(x+y)(f(x)-f(y))=\alpha (x-y)f(x+y)
$$
pre všetky $x$, $y \in \Bbb R$.}
\podpis{Walther Janous, Rakúsko}

{%%%%%   trojstretnutie, priklad 5
Nech $n$ je kladné celé číslo a~nech $d(n)$ označuje počet usporiadaných dvojíc kladných celých čísel $(x,y)$ takých, že
$$(x+1)^2 - xy(2x-xy+2y) + (y+1)^2 = n.$$
Nájdite najmenšie kladné celé číslo~$n$ spĺňajúce $d(n)=61$.}
\podpis{Patrik Bak, Slovensko}

{%%%%%   trojstretnutie, priklad 6
Daný je ostrouhlý trojuholník $ABC$. Nech $P$ je taký bod, že priamky $PB$ a~$PC$ sú dotyčnicami kružnice opísanej trojuholníku $ABC$. Nech $X$ a~$Y$ sú zvolené postupne na priamkach $AB$ a~$AC$ tak, že $|\uhol XPY| = 2|\uhol BAC|$ a~$P$ leží vnútri trojuholníka $AXY$. Označme $Z$ obraz bodu $A$ v~osovej súmernosti podľa osi~$XY$. Dokážte, že kružnica opísaná trojuholníku $XYZ$ prechádza pevným bodom nezávislým od polohy bodov $X$ a~$Y$.}
\podpis{Dominik Burek, Poľsko}

{%%%%%   IMO, priklad 1
Uvažujme konvexný štvoruholník $ABCD$. Bod $P$ leží vo vnútri $ABCD$. Platia nasledujúce rovnosti pomerov:
$$
|\angle PAD|:|\angle PBA|:|\angle DPA|=1:2:3=|\angle CBP|:|\angle BAP|:|\angle BPC|.
$$
Dokážte, že nasledujúce tri priamky sa pretínajú v jednom bode: osi vnútorných uhlov $\angle ADP$ a $\angle PCB$ a os úsečky~$AB$.
}
\podpis{Poľsko}

{%%%%%   IMO, priklad 2
Reálne čísla $a$, $b$, $c$, $d$ spĺňajú $a\ge b\ge c\ge d>0$ a $a+b+c+d=1$. Dokážte, že
$$
(a+2b+3c+4d)a^a b^b c^c d^d < 1.
$$
}
\podpis{Belgicko}

{%%%%%   IMO, priklad 3
Máme $4n$ kamienkov s hmotnosťami $1, 2, 3, \dots, 4n$. Každý kamienok je zafarbený jednou z $n$ farieb a každou farbou sú zafarbené štyri kamienky. Ukážte, že vieme rozdeliť kamienky na dve kôpky tak, aby boli splnené nasledujúce dve podmienky:
\item{$\bullet$} Celkové hmotnosti oboch kôpok sú rovnaké.
\item{$\bullet$} Každá kôpka obsahuje dva kamienky z každej farby.}
\podpis{Maďarsko}

{%%%%%   IMO, priklad 4
Dané je celé číslo $n > 1$. Na zjazdovke je $n^2$ staníc, všetky v~rôznych výškach. Každá
z~dvoch firiem $A$ a~$B$ prevádzkuje $k$ lanoviek. Každá lanovka umožňuje
presun z~jednej stanice do vyššej stanice (bez medzizastávok). Všetkých $k$ lanoviek firmy $A$ má $k$ rôznych
začiatočných staníc a $k$ rôznych konečných staníc; navyše lanovka, ktorá má začiatočnú stanicu
vyššie, má aj konečnú stanicu vyššie. Rovnaké podmienky spĺňa aj $B$. Povieme, že dve stanice sú
spojené firmou, ak sa dá dostať z nižšej stanice do vyššej použitím jednej alebo viacerých lanoviek
tejto firmy (žiadne iné pohyby medzi stanicami nie sú povolené).
Určte najmenšie kladné číslo $k$, pre ktoré sa dá zaručiť, že nejaké dve stanice sú spojené oboma
firmami.}
\podpis{India}

{%%%%%   IMO, priklad 5
Majme balíček $n > 1$ kariet. Na každej karte je napísané kladné celé číslo. Balíček má
vlastnosť, že aritmetický priemer čísel na každej dvojici kariet je zároveň geometrickým priemerom
čísel nejakej skupiny jednej alebo viacerých kariet.
Pre ktoré $n$ to musí znamenať, že všetky čísla na kartách sú rovnaké?}
\podpis{Estónsko}

{%%%%%   IMO, priklad 6
Dokážte, že existuje kladná konštanta $c$ taká, že platí nasledujúce tvrdenie:

{Uvažujme celé číslo $n > 1$ a množinu $\mathsfit S$ obsahujúcu $n$ bodov v~rovine takú, že vzdialenosť každých
dvoch rôznych bodov z~$\mathsfit S$ je aspoň 1. Potom existuje priamka $l$ rozdeľujúca $\mathsfit S$ taká, že vzdialenosť
ľubovoľného bodu z~$\mathsfit S$ a~priamky $l$ je aspoň $cn^{-1/3}$.}

(Priamka $l$ rozdeľuje množinu bodov $\mathsfit S$, ak nejaká úsečka spájajúca body z $\mathsfit S$ pretína~$l$.)

\poznamka
Riešenia, ktoré nahradia výraz $cn^{-1/3}$ slabším odhadom $cn^{-\alpha}$, môžu byť ocenené bodmi v závislosti od konštanty $\alpha > 1/3$.
}
\podpis{Taiwan}

{%%%%%   MEMO, priklad 1
Nech $\Bbb N$ je množina všetkých kladných celých čísel. Určte všetky kladné celé čísla $k$, pre ktoré
existujú funkcie $f\colon\Bbb N\to\Bbb N$ a $g\colon\Bbb N\to\Bbb N$ také, že $g$ nadobúda nekonečne veľa hodnôt a
$$
f^{g(n)}(n) = f(n) + k
$$
platí pre každé kladné celé číslo $n$.

\poznamka
Zápis $f^i$ označuje funkciu $f$ aplikovanú $i$-krát, \tj. $$f^i(j) = \underbrace{f(f(\dots f(f}_{\text{$i$-krát}}(j))\dots)).$$}
\podpis{Chorvátsko}

{%%%%%   MEMO, priklad 2
Kladné celé číslo $N$ nazývame {\it nákazlivé}, ak existuje 1000 po sebe idúcich nezáporných celých
čísel takých, že súčet všetkých ich cifier je rovný $N$. Nájdite všetky nákazlivé kladné celé čísla.}
\podpis{Rakúsko}

{%%%%%   MEMO, priklad 3
Daný je ostrouhlý trojuholník $ABC$ s tromi navzájom rôzne veľkými stranami, ktorého opísanú
kružnicu označme $\omega$ a stred jemu vpísanej kružnice označme $I$. Predpokladajme, že ortocentrum $H$ trojuholníka $BIC$ leží vnútri $\omega$. Nech $M$ je stred dlhšieho oblúka $BC$ kružnice $\omega$. Nech $N$ je stred kratšieho oblúka $AM$ kružnice $\omega$. Dokážte, že existuje kružnica, ktorá sa dotýka kružnice $\omega$ v bode $N$ a dotýka sa aj kružníc opísaných trojuholníkom $BHI$ a $CHI$.
}
\podpis{Poľsko}

{%%%%%   MEMO, priklad 4
Nájdite všetky kladné celé čísla $n$, pre ktoré existujú kladné celé čísla $x_1,x_2,\dots,x_n$ také, že
$$
\frac1{x_1^2} + \frac2{x_2^2} + \frac4{x_3^2} + \dots + \frac{2^{n-1}}{x_n^2} = 1.
$$}
\podpis{Chorvátsko}

{%%%%%   MEMO, priklad t1
}
\podpis{}

{%%%%%   MEMO, priklad t2
}
\podpis{}

{%%%%%   MEMO, priklad t3
}
\podpis{}

{%%%%%   MEMO, priklad t4
}
\podpis{}

{%%%%%   MEMO, priklad t5
}
\podpis{}

{%%%%%   MEMO, priklad t6
}
\podpis{}

{%%%%%   MEMO, priklad t7
}
\podpis{}

{%%%%%   MEMO, priklad t8
}
\podpis{}

{%%%%%   CPSJ, priklad 1
...}
\podpis{...}

{%%%%%   CPSJ, priklad 2
...}
\podpis{...}

{%%%%%   CPSJ, priklad 3
...}
\podpis{...}

{%%%%%   CPSJ, priklad 4
...}
\podpis{...}

{%%%%%   CPSJ, priklad 5
...}
\podpis{...}

{%%%%%   CPSJ, priklad t1
...}
\podpis{...}

{%%%%%   CPSJ, priklad t2
...}
\podpis{...}

{%%%%%   CPSJ, priklad t3
...}
\podpis{...}

{%%%%%   CPSJ, priklad t4
...}
\podpis{...}

{%%%%%   CPSJ, priklad t5
...}
\podpis{...}

{%%%%%   CPSJ, priklad t6
...}
\podpis{...}

{%%%%%   EGMO, priklad 1
Kladné celé čísla $a_0,a_1,a_2,\dots,a_{3030}$ spĺňajú
$$
2a_{n+2} = a_{n+1} + 4a_n \text{\ \ pre $n=0,1,2,\dots,3028$.}
$$
Dokážte, že aspoň jedno z~čísel $a_0,a_1,a_2,\dots,a_{3030}$ je deliteľné $2^{2020}$.}
\podpis{Austrália}

{%%%%%   EGMO, priklad 2
Nájdite všetky postupnosti $(x_1,x_2,\dots,x_{2020})$ nezáporných reálnych čísel spĺňajúce nasledovné tri podmienky:
\item{(i)} $x_1 \leq x_2 \leq \dots \leq x_{2020}$;
\item{(ii)} $x_{2020} \le x_1 + 1$;
\item{(iii)} existuje permutácia $(y_1,y_2,\dots,y_{2020})$ postupnosti $(x_1,x_2,\dots,x_{2020})$ taká, že
$$
\sum_{i=1}^{2020} \big((x_i+1)(y_i+1)\big)^2 = 8 \sum_{i=1}^{2020} x_i^3.
$$
\endgraf\noindent
\emph{Permutácia postupnosti} je postupnosť rovnakej dĺžky s~rovnakými členmi, avšak členy môžu byť v~ľubovoľnom poradí. Napríklad, $(2,1,2)$ je permutáciou $(1,2,2)$, a~tiež obe sú permutáciou $(2,2,1)$. Každá postupnosť je permutáciou samej seba.}
\podpis{Patrik Bak, Slovensko}

{%%%%%   EGMO, priklad 3
Nech $ABCDEF$ je konvexný šesťuholník taký, že $|\angle A| = |\angle C| = |\angle E|$ a~$|\angle B| = |\angle D| = |\angle F|$ a~osi (vnútorných) uhlov $\angle A$, $\angle C$ a~$\angle E$ sa pretínajú v~jednom bode. Dokážte, že osi (vnútorných) uhlov $\angle B$, $\angle D$ a~$\angle F$ sa taktiež pretínajú v~jednom bode.

\poznamka
Platí $\angle A = \angle FAB$. Zvyšné vnútorné uhly šesťuholníka sú popísané analogicky.}
\podpis{Ukrajina}

{%%%%%   EGMO, priklad 4
Permutácia celých čísel $1,2,\dots,,m$ sa nazýva \emph{svieža}, ak neexistuje žiadne kladné celé číslo $k < m$ také, že prvých~$k$ čísel v~permutácii sú čísla $1,2,\dots,k$ v~nejakom poradí. Nech $f_m$ je počet sviežich permutácií čísel $1,2,\dots,m$.
Dokážte, že pre všetky $n \ge 3$ platí nerovnosť $f_{n} \ge n \cdot f_{n-1}$.
\poznamka
Napríklad, ak $m=4$, tak permutácia $(3,1,4,2)$ je svieža, naproti tomu permutácia $(2,3,1,4)$ nie je.}
\podpis{Patrik Bak, Slovensko}

{%%%%%   EGMO, priklad 5
Nech $ABC$ je trojuholník taký, že $|\angle BCA|>90^\circ$. Kružnica~$\Gamma$ jemu opísaná má polomer~$R$. Vnútri úsečky $AB$ leží bod~$P$ taký, že $|PB|=|PC|$ a~dĺžka úsečky $PA$ je rovná~$R$. Os úsečky $PB$ pretína $\Gamma$ v~bodoch $D$ a~$E$.
Dokážte, že $P$ je stredom kružnice vpísanej trojuholníku $CDE$. }
\podpis{Spojené kráľovstvo}

{%%%%%   EGMO, priklad 6
Nech $m>1$ je celé číslo. Postupnosť $a_1, a_2, a_3, \dots$ je definovaná vzťahmi $a_1=a_2=1$, $a_3=4$, a~pre všetky $n \ge 4$
$$
a_n = m(a_{n-1} + a_{n-2}) - a_{n-3}.
$$
Nájdite všetky celé čísla~$m$ také, že každý člen tejto postupnosti je druhá mocnina celého čísla.
}
\podpis{Dánsko}
