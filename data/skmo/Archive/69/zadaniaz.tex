{%%%%%   Z4-I-1
...}
\podpis{...}

{%%%%%   Z4-I-2
...}
\podpis{...}

{%%%%%   Z4-I-3
...}
\podpis{...}

{%%%%%   Z4-I-4
...}
\podpis{...}

{%%%%%   Z4-I-5
...}
\podpis{...}

{%%%%%   Z4-I-6
...}
\podpis{...}

{%%%%% Z5-I-1
Naša stará mama nakupovala v~obchode, v~ktorom mali iba jablká, banány a~hrušky.
Jablká boli po 50 centoch za kus, hrušky po 60 centoch a~banány boli lacnejšie ako hrušky.
Stará mama kúpila päť kusov ovocia, medzi ktorými bol práve jeden banán,
a~zaplatila 2~eurá a~75~centov.
Koľko centov mohol stáť jeden banán?
Určte všetky možnosti.
}
\podpis{Katarína Jasenčáková}

{%%%%% Z5-I-2
Všetky cesty v~parku sú meter široké a~sú tvorené celými štvorcovými dlaždicami s~rozmermi meter krát meter, ktoré k~sebe natesno priliehajú.
Cesty, pri ktorých sa majú vymeniť všetky dlaždice, sú schematicky znázornené na obrázku.
Koľko dlaždíc sa má vymeniť?
\insp{z5-I-2.eps}%
}
\podpis{Eva Semerádová}

{%%%%% Z5-I-3
Pán kráľ rozdával svojim synom dukáty.
Najstaršiemu synovi dal určitý počet dukátov, mladšiemu dal o~jeden dukát menej, ďalšiemu dal opäť o~jeden dukát menej a~takto postupoval až k~najmladšiemu.
Potom sa vrátil k~najstaršiemu synovi, dal mu o~jeden dukát menej ako pred chvíľou najmladšiemu a~rovnakým spôsobom ako v~prvom kole rozdával ďalej.
V~tomto kole vyšiel na najmladšieho syna jeden dukát.
Najstarší syn dostal celkom 21~dukátov.
Určte, koľko mal kráľ synov a~koľko im celkom rozdal dukátov.
}
\podpis{Karel Pazourek}

{%%%%% Z5-I-4
Vojto začal vypisovať do zošita číslo terajšieho školského roku 2019202020192020...a~tak pokračoval stále ďalej.
Keď napísal 2020 cifier, prestalo ho to baviť.
Koľko tak napísal dvojok?
}
\podpis{Lucie Růžičková}

{%%%%% Z5-I-5
Dedko má v~záhrade tri jablone a~na nich celkom 39~jabĺk.
Jablká rastú iba na ôsmich konároch: na jednej jabloni plodia dva konáre, na dvoch jabloniach plodia po tri konáre.
Na rôznych konároch sú rôzne počty jabĺk, ale na každej jabloni je rovnaký počet jabĺk.
Koľko jabĺk mohlo byť na jednotlivých konároch?
Určte aspoň jednu možnosť.}
\podpis{Alžbeta Bohiniková}

{%%%%% Z5-I-6
Obdĺžnikový obrus je poskladaný z~rovnako veľkých štvorcov bielej, sivej a~čiernej farby, a~to tak, že
\begin{itemize}
\item štvorce so spoločnou stranou majú rôzne farby,
\item biele štvorce nemajú spoločný vrchol,
\item čierne štvorce nemajú spoločný vrchol,
\item čiernych štvorcov je šesť,
\item na každej strane obrusu sú najmenej tri štvorce.
\end{itemize}
\noindent
Ako mohol obrus vyzerať?
Nájdite a~nakreslite aspoň tri možnosti.}
\podpis{Katarína Jasenčáková}

{%%%%% Z6-I-1
Stará mama povedala vnúčatám: \uv{Dnes mám 60~rokov a~50~mesiacov a~40~týždňov a~30~dní.}
Koľké narodeniny mala stará mama naposledy?
}
\podpis{Libuše Hozová}

{%%%%% Z6-I-2
Na obrázku je trojuholníková sieť a~v~nej štyri mnohouholníky.
Obvody mnohouholníkov $A$, $B$ a~$D$ sú postupne 56\,cm, 34\,cm a~42\,cm.
Určte obvod trojuholníka~$C$.
\insp{z6-I-2.eps}%
}
\podpis{Karel Pazourek}

{%%%%% Z6-I-3
Na písomke bolo 25~úloh trojakého druhu: ľahké po 2~bodoch, stredne ťažké po 3~bodoch a~ťažké po 5~bodoch.
Správne vyriešené úlohy boli hodnotené uvedeným počtom bodov podľa stupňa obťažnosti, inak 0.
Najlepšie možné celkové hodnotenie písomky bolo 84~bodov.
Peter správne vyriešil všetky ľahké úlohy, polovicu stredne ťažkých a~tretinu ťažkých.
Koľko bodov získal Peter za svoju písomku?
}
\podpis{Alžbeta Bohiniková}

{%%%%% Z6-I-4
Raz si kráľ zavolal všetky svoje pážatá a~postavil ich do radu.
Prvému pážaťu dal určitý počet dukátov, druhému dal o~dva dukáty menej, tretiemu opäť o~dva dukáty menej a~tak ďalej.
Keď došiel k~poslednému pážaťu, dal mu príslušný počet dukátov, otočil sa a~obdobným spôsobom postupoval na začiatok radu
(\tj. predposlednému pážaťu dal o~dva dukáty menej ako pred chvíľou poslednému atď.).
Na prvé páža v~tomto kole vyšli dva dukáty.
Potom jedno z~pážat zistilo, že má 32~dukátov.
Koľko mohol mať kráľ pážat a~koľko celkom im mohol rozdať dukátov?
Určte všetky možnosti.
}
\podpis{Karel Pazourek}

{%%%%% Z6-I-5
Útvar na obrázku vznikol tak, že z~veľkého kríža bol vystrihnutý malý kríž.
Každý z~týchto krížov môže byť zložený z~piatich zhodných štvorcov, pričom strany malých štvorcov sú polovičné vzhľadom na strany veľkých štvorcov.
Obsah sivého útvaru je 45\,cm$^2$.
Aký je obsah veľkého kríža?
\insp{z6-I-5.eps}%
}
\podpis{Lucie Růžičková}

{%%%%% Z6-I-6
Majka skúmala viacciferné čísla, v~ktorých sa po jednej striedajú nepárne a~párne cifry.
Tie, ktoré začínajú nepárnou cifrou, nazvala komické a~tie, ktoré začínajú párnou cifrou, nazvala veselé.
(Napr. číslo 32387 je komické, číslo 4529 je veselé.)
Medzi trojcifernými číslami určte, či je viac komických alebo veselých, a~o~koľko.
}
\podpis{Monika Dillingerová}

{%%%%% Z7-I-1
Snehulienka so siedmimi trpaslíkmi nazbierali šišky na táborák.
Snehulienka povedala, že počet všetkých šišiek je číslo deliteľné dvoma.
Prvý trpaslík prehlásil, že je to číslo deliteľné tromi,
druhý trpaslík povedal, že je to číslo deliteľné štyrmi,
tretí trpaslík povedal, že je to číslo deliteľné piatimi,
štvrtý trpaslík povedal, že je to číslo deliteľné šiestimi,
piaty trpaslík povedal, že je to číslo deliteľné siedmimi,
šiesty trpaslík povedal, že je to číslo deliteľné ôsmimi,
siedmy trpaslík povedal, že je to číslo deliteľné deviatimi.
Dvaja z~ôsmich zberačov, ktorí sa k~počtu šišiek vyjadrovali bezprostredne po sebe, nemali pravdu, ostatní áno.
Koľko šišiek bolo na hromade, ak ich určite bolo menej ako 350?
}
\podpis{Libuše Hozová}

{%%%%% Z7-I-2
V~ostrouhlom trojuholníku $KLM$ je $V$ priesečník jeho výšok a~$X$ je päta výšky na stranu~$KL$.
Os uhla $XVL$ je rovnobežná so stranou~$LM$ a~uhol $MKL$ má veľkosť~$70\st$.
Akú veľkosť majú uhly $KLM$ a~$KML$?
}
\podpis{Libuše Hozová}

{%%%%% Z7-I-3
Roman má rád kúzla a~matematiku.
Naposledy čaroval s~trojcifernými alebo štvorcifernými číslami takto:
\begin{itemize}
\item z~daného čísla vytvoril dve {\it pomocné\/} čísla tak, že ho rozdelil medzi ciframi na mieste stoviek a~desiatok (napr. z~čísla 581 by dostal 5 a~81),
\item pomocné čísla sčítal a~zapísal výsledok (v~uvedenom príklade by dostal 86),
\item od väčšieho z~pomocných čísel odčítal menšie a~výsledok zapísal za predchádzajúci súčet, čím vyčaroval výsledné číslo (v~uvedenom príklade by dostal 8676).
\end{itemize}
\noindent
Z~ktorých čísel mohol Roman vyčarovať a) 171, b) 1513?
Určte všetky možnosti. Aké najväčšie číslo možno takto vyčarovať a~z~ktorých čísel môže vzniknúť?
Určte všetky možnosti.
}
\podpis{Monika Dillingerová}

{%%%%% Z7-I-4
Janka a~Marienku zaujalo vodné dielo, ktorého časť je znázornená na obrázku.
Korytá sa postupne rozdeľujú a~zasa spájajú v~načrtnutých bodoch, v~každom rade je o~jeden taký bod viac ako v~rade predchádzajúcom.
Voda prúdi v~naznačených smeroch a~pri každom vetvení sa vodný tok rozdelí do dvoch korýt s~rovnakým prietokom.
Janka zaujímalo, koľko vody preteká v~súčte štyrmi miestami zvýraznenými čierno.
Marienku zaujímalo, koľko vody preteká v~súčte všetkými miestami, ktoré sú v~2019.~rade.
Porovnajte celkové prietoky Jankovými a~Marienkinými miestami.
\insp{z7-I-4.eps}%
}
\podpis{Katarína Jasenčáková}

{%%%%% Z7-I-5
Hviezdny štvorec je taká štvorcová tabuľka čísel, pre ktorú platí, že súčty čísel v~jednotlivých riadkoch a~stĺpcoch sú stále rovnaké.
Na obrázku je pozostatok hviezdneho štvorca, v~ktorom boli čísla v~jednom riadku a~jednom stĺpci zotreté.
$$
\begin{array}{cccc}
1 & 2 & 3 & \phantom{0} \\
4 & 5 & 6 & \\
7 & 8 & 9 & \\
 & & & \\
\end{array}
$$
Doplňte chýbajúce čísla tak, aby všetky boli celé a~práve štyri záporné.
Určte všetky možnosti.
}
\podpis{Eva Semerádová}

{%%%%% Z7-I-6
Z~pravidelného šesťuholníka bol vystrihnutý útvar ako na obrázku.
Pritom vyznačené body ako na obvode, tak vnútri šesťuholníka delia prislúchajúce úsečky na štvrtiny.
Aký je pomer obsahov pôvodného šesťuholníka a~vystrihnutého útvaru?
\insp{z7-I-6.eps}%
}
\podpis{Alžbeta Bohiniková}

{%%%%% Z8-I-1
Zostrojte kosoštvorec $ABCD$ tak, aby jeho uhlopriečka~$BD$ mala veľkosť 8\,cm a~vzdialenosť vrcholu~$B$ od priamky~$AD$ bola 5\,cm.
Určte všetky možnosti.
}
\podpis{Karel Pazourek}

{%%%%% Z8-I-2
Richard sa pohrával s~dvoma päťcifernými číslami.
Každé pozostávalo z~navzájom rôznych cifier, ktoré pri jednom boli všetky nepárne a~pri druhom všetky párne.
Po chvíli zistil, že súčet týchto dvoch čísel začína dvojčíslím~11 a~končí číslom~1 a~že ich rozdiel začína číslom~2 a~končí dvojčíslím~11.
Určte Richardove čísla.
}
\podpis{Monika Dillingerová}

{%%%%% Z8-I-3
Vendelín býva medzi dvoma zastávkami autobusu, a~to v~troch osminách ich vzdialenosti.
Dnes vyrazil z~domu a~zistil, že či by utekal k~jednej, alebo druhej zastávke, dorazil by na zastávku súčasne s~autobusom.
Priemerná rýchlosť autobusu je 60~km/h. Akou priemernou rýchlosťou dnes beží Vendelín?
}
\podpis{Libuše Hozová}

{%%%%% Z8-I-4
Pre päticu celých čísel platí, že keď k~prvému pripočítame jednotku, druhé umocníme na druhú, od tretieho odčítame trojku, štvrté vynásobíme štyrmi a~piate vydelíme piatimi, dostaneme zakaždým ten istý výsledok.
Nájdite všetky také pätice čísel, ktorých súčet je 122.
}
\podpis{Lenka Dedková}

{%%%%% Z8-I-5
Pre osem navzájom rôznych bodov ako na obrázku platí, že body $C$, $D$, $E$ ležia na priamke rovnobežnej s~priamkou~$AB$, $F$ je stredom úsečky~$AD$, $G$ je stredom úsečky~$AC$ a~$H$ je priesečníkom priamok $AC$ a~$BE$.
Obsah trojuholníka $BCG$ je 12\,cm$^2$ a~obsah štvoruholníka $DFHG$ je 8\,cm$^2$.
Určte obsahy trojuholníkov $AFE$, $AHF$, $ABG$ a~$BGH$.
\insp{z8-I-5.eps}%
}
\podpis{Eva Semerádová}

{%%%%% Z8-I-6
V~Kocúrkove používajú mince iba s~dvoma hodnotami, ktoré sú vyjadrené v~kocúrkovských korunách kladnými celými číslami.
Pomocou dostatočného množstva takých mincí je možné zaplatiť akúkoľvek celočíselnú sumu väčšiu ako 53 kocúrkovských korún, a~to presne a~bez vydávania. Sumu 53 kocúrkovských korún však bez vydávania zaplatiť nemožno.
Zistite, ktoré hodnoty mohli byť na kocúrkovských minciach.
Určte aspoň dve riešenia.
}
\podpis{Alžbeta Bohiniková}

{%%%%% Z9-I-1
Ondro, Maťo a~Kubo sa vracajú zo zbierania orechov, dokopy ich majú 120.
Maťo sa sťažuje, že Ondro má ako vždy najviac.
Otec prikáže Ondrovi, aby prisypal zo svojho Maťovi tak, aby mu počet orechov zdvojnásobil.
Teraz sa sťažuje Kubo, že najviac má Maťo.
Na otcov príkaz prisype Maťo Kubovi tak, že mu počet orechov zdvojnásobí.
Na to sa hnevá Ondro, že najmenej zo všetkých má teraz on.
Kubo teda prisype Ondrovi tak, že mu počet orechov zdvojnásobí.
Teraz majú všetci rovnako a~konečne je kľud.
Koľko orechov mal pôvodne každý z~chlapcov?
}
\podpis{Marta Volfová}

{%%%%% Z9-I-2
V~trojuholníku $ABC$ leží bod~$P$ v~tretine úsečky~$AB$ bližšie bodu~$A$, bod~$R$ je v~tretine úsečky~$PB$ bližšie bodu~$P$ a~bod~$Q$ leží na úsečke~$BC$ tak, že uhly $PCB$ a~$RQB$ sú zhodné.
Určte pomer obsahov trojuholníkov $ABC$ a~$PQC$.
}
\podpis{Lucie Růžičková}

{%%%%% Z9-I-3
Pre ktoré celé čísla $x$ je podiel $\displaystyle\frac{x+11}{x+7}$ celým číslom?
Nájdite všetky riešenia.
}
\podpis{Libuše Hozová}

{%%%%% Z9-I-4
Matúš dopadol padákom na ostrov obývaný dvoma druhmi domorodcov: Poctivcami, ktorí vždy hovoria pravdu, a~Klamármi, ktorí vždy klamú.
Pred dopadom zahliadol v~diaľke prístav, ku ktorému sa hodlal dostať.
Na prvom rázcestí stretol Matúš jedného domorodca a~obďaleč videl druhého.
Požiadal prvého, aby sa spýtal toho druhého, či je Klamár, alebo Poctivec.
Prvý domorodec Matúšovi vyhovel, išiel sa spýtať a~keď sa vrátil, oznámil Matúšovi, že druhý domorodec tvrdí, že je Klamár.
Potom sa Matúš prvého domorodca spýtal, ktorá cesta vedie k~prístavu.
Ten mu jednu cestu ukázal a~ďalej si Matúša nevšímal.
Má, alebo nemá Matúš domorodcovi veriť?
Vedie, alebo nevedie táto cesta k~prístavu?
}
\podpis{Marta Volfová}

{%%%%% Z9-I-5
Majka skúmala viacciferné čísla, v~ktorých sa po jednej striedajú nepárne a~párne cifry.
Tie, ktoré začínajú nepárnou cifrou, nazvala komické a~tie, ktoré začínajú párnou cifrou, nazvala veselé.
(Napr. číslo 32387 je komické, číslo 4529 je veselé.)
Majka vytvorila jedno trojciferné komické a~jedno trojciferné veselé číslo, pričom šesť použitých cifier bolo navzájom rôznych a~nebola medzi nimi 0.
Súčet týchto dvoch čísel bol 1617.
Súčin týchto dvoch čísel končil dvojčíslím 40.
Určte Majkine čísla a~dopočítajte ich súčin.
}
\podpis{Monika Dillingerová}

{%%%%% Z9-I-6
Kristína zvolila isté nepárne prirodzené číslo deliteľné tromi.
Jakub s~Dávidom potom skúmali trojuholníky, ktoré majú obvod v~milimetroch rovný Kristínou zvolenému číslu a~ktorých strany majú dĺžky v~milimetroch vyjadrené navzájom rôznymi celými číslami.
Jakub našiel taký trojuholník, v~ktorom najdlhšia zo strán má najväčšiu možnú dĺžku, a~túto hodnotu zapísal na tabuľu.
Dávid našiel taký trojuholník, v~ktorom najkratšia zo strán má najväčšiu možnú dĺžku, a~túto hodnotu tiež zapísal na tabuľu.
Kristína obe dĺžky na tabuli správne sčítala a~vyšlo jej 1\,681\,mm.
Určte, ktoré číslo Kristína zvolila.
}
\podpis{Lucie Růžičková}

{%%%%%   Z4-II-1
...}
\podpis{...}

{%%%%%   Z4-II-2
...}
\podpis{...}

{%%%%%   Z4-II-3
...}
\podpis{...}

{%%%%% Z5-II-1
Záhradník pán Malina predával jahody.
V~posledných deviatich debničkách mal postupne 28, 51, 135, 67, 123, 29, 56, 38 a~79 sadeníc jahôd.
Predával celé debničky, žiadne sadenice z~debničiek nevyťahoval.
Záhradník chcel rozpredať debničky trom zákazníkom tak, aby mu nič neostalo a~aby všetci títo zákazníci mali rovnaký počet sadeníc.
Ako to mohol urobiť?
Uveďte dve možnosti.
}
\podpis{Libuše Hozová}

{%%%%% Z5-II-2
Do štvorca sú vpísané menšie štvorce, a~to vždy tak, že vrcholy menšieho štvorca sú v~stredoch strán väčšieho štvorca, pozri obrázok.
Sivo vyfarbený štvorec má obsah 1\,cm$^2$.
Určte obvod najväčšieho štvorca.
\ifobrazkyvedla\else\insp{z5-II-2.eps}\fi%
}
\podpis{Eva Semerádová}

{%%%%% Z5-II-3
Na obrázku je číselná os s~vyznačenými číslami 10 a~30 a~ďalšími bezmennými bodmi predstavujúcimi celé čísla.
Janko si na tejto osi bodkami vyznačil svoje obľúbené číslo a~ďalšie štyri čísla, o~ktorých vieme, že
\itemitem{$\bullet$} jedno je polovicou Jankovho čísla,
\itemitem{$\bullet$} jedno je o~6 väčšie ako Jankovo číslo,
\itemitem{$\bullet$} jedno je o~10 menšie ako Jankovo číslo,
\itemitem{$\bullet$} jedno je dvakrát väčšie ako Jankovo číslo.

Zistite, ktoré číslo je Jankovo obľúbené.
%\ifobrazkyvedla\else
\insp{z5-II-3.eps}%\fi%
}
\podpis{Svetlana Bednářová}

{%%%%%   Z6-II-1
...}
\podpis{...}

{%%%%%   Z6-II-2
...}
\podpis{...}

{%%%%%   Z6-II-3
...}
\podpis{...}

{%%%%%   Z7-II-1
...}
\podpis{...}

{%%%%%   Z7-II-2
...}
\podpis{...}

{%%%%%   Z7-II-3
...}
\podpis{...}

{%%%%%   Z8-II-1
...}
\podpis{...}

{%%%%%   Z8-II-2
...}
\podpis{...}

{%%%%%   Z8-II-3
...}
\podpis{...}

{%%%%% Z9-II-1
Pat a~Mat mali každý svoje obľúbené prirodzené číslo, ale každý iné.
Obe čísla postupne sčítali, odčítali (menšie od väčšieho), vynásobili a~vydelili (väčšie menším).
Keď takto získané výsledky sčítali, vyšlo im 98.
Ktoré obľúbené čísla mali Pat a~Mat?
}
\podpis{Libuše Hozová}

{%%%%% Z9-II-2
Obrázok predstavuje pohľad zhora na trojvrstvovú pyramídu tvorenú 14 zhodnými kockami.
Každej kocke prislúcha jedno prirodzené číslo, a~to tak, že čísla zodpovedajúce kockám v~spodnej vrstve sú navzájom rôzne a~číslo na každej ďalšej kocke je súčtom čísel zo štyroch susediacich kociek z~nižšej vrstvy.
Určte najmenšie číslo deliteľné štyrmi, ktoré môže prislúchať najvrchnejšej kocke.
\ifobrazkyvedla\else\insp{z9-II-2.eps}\fi%
}
\podpis{Alžbeta Bohiniková}

{%%%%% Z9-II-3
Uvažujme štvorciferné prirodzené číslo s~nasledujúcou vlastnosťou: ak prehodíme jeho prvé dvojčíslie s~druhým, dostaneme štvorciferné číslo o~99 menšie.
Koľko je takých čísel celkom a~koľko z~nich je deliteľných~9?
}
\podpis{Karel Pazourek}

{%%%%% Z9-II-4
Do všeobecného trojuholníka $ABC$ narysujte bod~$D$ tak, aby obsah trojuholníka $ABD$ bol rovný polovici obsahu trojuholníka $ABC$ a~obsah trojuholníka $BCD$ bol rovný šestine obsahu trojuholníka $ABC$.

(Riešenie má byť všeobecne platné, teda nezávislé na zvolenom trojuholníku, jeho špeciálnych vlastnostiach či rozmeroch. Konštrukcia nemôže byť založená na meraní a počítaní. Zvoľte si trojuholník, ktorý nie je rovnoramenný ani pravouhlý.)
}
\podpis{Libuše Hozová}

{%%%%% Z9-III-1
...}
\podpis{...}

{%%%%% Z9-III-2
...}
\podpis{...}

{%%%%% Z9-III-3
...}
\podpis{...}

{%%%%% Z9-III-4
...}
\podpis{...}

