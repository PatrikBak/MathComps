{%%%%%   A-I-1
Najskôr si uvedomme, že stačí dokázať $a>c$, čo v~spojení s~predpokladom
$ab<cd$ povedie k~nerovnostiam $cd>ab>bc$, odkiaľ po vydelení krajných súčinov
číslom~$c$ vyjde $d>b$, a~teda celkovo $a>c>d>b$.

Za týmto účelom nerovnosť pre súčty prepíšeme na tvar $b > c + d - a$
a~s~využitím nerovnosti pre súčiny tak získame
$$
cd > ab > a(c+d-a) = ac +ad - a^2.
$$
Získanú nerovnosť medzi krajnými výrazmi ľahko upravíme na súčinový
tvar
$$
(a-c)(a-d) > 0.
$$
Na dokončenie dôkazu nerovnosti $a>c$ nám tak ostáva vylúčiť súčasnú
platnosť nerovností $a<c$ a~$a<d$. V~takom prípade by však vďaka nerovnosti $b<a$
platilo $b<c$ a~$b < d$, čím by sme dostali spor s~nerovnosťou $a+ b > c + d$.

\ineres
Nerovnosť $a>c$ možno ukázať aj nasledovne.
K~dvojici $(a, b)$ určíme dvojicu $(x, \delta)$ kladných
čísel takú, že $a= x + \delta$, $b = x - \delta$, teda
$x=\frac12(a+b)$ a~$\delta=\frac12(a-b)$. Podobne
určíme kladné $y$, $\varepsilon$ tak, že $c = y + \varepsilon$,
$d =y - \varepsilon$. Vzťahy zo zadania potom prepíšeme ako
$$
x > y, \qquad x^2 - \delta^2 < y^2 - \varepsilon^2
$$
a~ich kombináciou získame
$$
\delta^2 - \varepsilon^2 > x^2 - y^2 > 0,
$$
teda $\delta > \varepsilon$. Napokon tak máme
$$
a= x + \delta > y + \varepsilon = c.
$$
Sme hotoví.

\poznamka
Ukážeme, že rozhodujúcu nerovnosť $a>c$ môžeme práve opísaným
postupom odvodiť bez toho, aby sme zavádzali nejaké substitúcie. Zo zadania
úlohy vyplývajú nerovnosti
$$
(a+b)^2 > (c+d)^2 \quad\text{a}\quad -4ab > - 4cd,
$$
ktorých sčítaním dostaneme $(a-b)^2>(c-d)^2$. Keďže základy oboch
posledných mocnín sú kladné, vyplýva z~toho nerovnosť $a-b>c-d$. Jej
sčítaním s~nerovnosťou $a+b>c+d$ a~následným vydelením dvoma už
dostaneme potrebný záver $a>c$.


\ineres
Zadané nerovnosti využijeme na porovnanie dvoch kvadratických
trojčlenov. Označme
$P(x) = (x-a)(x-b)$ a~$Q(x) = (x-c)(x-d)$.
Potom pre ľubovoľné $x > 0$ platí
$$
P(x) = x^2 - (a+b)x + ab < x^2 - (c+d)x + cd = Q(x).
$$
Keďže pre kladné $x$, ktoré neležia v~intervale $(b, a)$, platí $P(x) \ge 0$, platí tiež
$Q(x) > P(x) \ge 0$, žiadne také~$x$ preto nemôže byť koreňom trojčlena~$Q(x)$.
Kladné korene~$c$,~$d$ tohto trojčlena tak musia ležať vnútri
intervalu $(b, a)$. Z~toho už vyplývajú dokazované nerovnosti $a> c > d> b$.

Pre názornosť uvádzame aj grafické znázornenie situácie (\obr).
\insp{a69.1}%


\návody
Súčet kladných čísel $a$, $b$ je nanajvýš~16. Aká je
najväčšia možná hodnota ich súčinu?
[64. Do súčinu $ab$ dosaďte vyjadrenie $a=p+\epsilon$
a~$b=p-\epsilon$, pričom $p\le 8$ je aritmetický priemer čísel $a$ a~$b$.]

Súčin kladných čísel $a$, $b$ je aspoň~16. Aká je
najmenšia možná hodnota ich súčtu?
[8. Použite rovnaké vyjadrenie čísel $a$, $b$ ako
v~N1 a~dokážte, že z~$ab\ge16$ vyplýva $p\ge4$.]

\D
Reálne čísla $a_1$, $a_2$, $b_1$, $b_2$ spĺňajú $a_1 > a_2$
a~$b_1 > b_2$. Dokážte, že $a_1b_1 + a_2b_2 > a_1b_2 + a_2b_1$.
[Dokazovanú nerovnosť ekvivalentne upravte na $(a_1-a_2)(b_1-b_2)>0$.]

[Permutačná nerovnosť] Nech $x_1 > x_2 > \dots > x_n$ sú
reálne čísla a~nech $y_1,\allowbreak y_2, \dots ,y_n$ je nejaké poradie pevne
daných navzájom rôznych reálnych
čísel $z_1, z_2, \dots , z_n$. Potom je výraz
$$
x_1y_1 + x_2y_2 +\dots + x_ny_n
$$
maximálny práve vtedy,
keď je $y_1 > y_2 > \dots > y_n$. Dokážte. [Stačí ukázať, že
usporiadania, ktoré nespĺňajú $y_1 > y_2 > \dots > y_n$, možno vylepšiť
\uv{prehodením} jednej dvojice. To rieši úloha~D1, ktorá je vlastne úlohou~D2 pre $n=2$.]

\endnávod}

{%%%%%   A-I-2
Políčka daného útvaru šachovnicovo ofarbíme. O~útvare~$\mm U$ pozostávajúcom
z~ľavého štvorca $5\times 5$ spolu s~jedným políčkom navyše (ako na
\obr) ukážeme, že je nutne tiež bezo zvyšku (a~bez prekrývania)
vydláždený.
\insp{a69.2}%

Označme $\mm V$ útvar, ktorý pokrývajú tie dominové kocky, ktoré zasahujú do
útvaru~$\mm U$. Keďže dominová kocka zakryje jedno biele a~jedno čierne
políčko, má útvar~$\mm V$ rovnaký počet čiernych a~bielych políčok. To isté však
platí aj pre útvar~$\mm U$, ako ľahko overíme.
Keďže $\mm V$ však môže oproti~$\mm U$ obsahovať navyše iba tie tri políčka susediace
s~$\mm U$ (na \obrr1{} označené $A$, $B$,~$C$), ktoré sú všetky čierne, nie je iná
možnosť, ako že $\mm U=\mm V$.
Naozaj tak platí, že každé dláždenie celého útvaru obsahuje ako svoju
časť dláždenie útvaru~$\mm U$.

Podobne môžeme argumentovať pre súmerne združený útvar~$\mm U'$ na pravej strane.
Nakoniec si stačí uvedomiť, že každé dláždenie celého útvaru je
jednoznačne určené tým, ako sú vydláždené útvary $\mm U$ a~$\mm U'$.
(Vzhľadom na úplné vydláždenie útvaru $\mm U\cup\mm U'$ je už jednoznačne určená
poloha dvoch dominových kociek, ktoré pokrývajú zvyšok daného útvaru s~políčkami $A$~a~$C$.)

Ak označíme $n$ počet dláždení útvaru~$\mm U$, bude celkový počet možností,
ako vydláždiť zadaný útvar, rovný~$n^2$.
Tým sme ukázali, že hľadaný počet spôsobov je rovný druhej mocnine
prirodzeného čísla.

\poznamka
Pre zaujímavosť uvádzame, že presný počet dláždení sa dá
spočítať s~pomocou počítača. Výsledok je $36\,864 = 192^2$.



\návody
Dokážte, že počet spôsobov, ako možno na šachovnici $8\times 8$
umiestniť maximálny počet strelcov tak, aby sa žiadni dvaja navzájom
neohrozovali, je druhá mocnina prirodzeného čísla. [Nepočítajte,
iba rozdeľte na dve podúlohy~-- rozmiestnenia strelcov na čierne
a~na biele políčka sú navzájom nezávislé.]

Koľkými spôsobmi možno vydláždiť dominovými kockami šachovnicu
$8\times 8$, z~ktorej sú odrezané dve protiľahlé rohové políčka?
[Nedá sa žiadnym spôsobom: ak sú odrezané políčka biele, je nutné pokryť 30
bielych a~32 čiernych políčok, pritom ľubovoľne umiestnená kocka domina
pokryje vždy 1 biele a~1 čierne políčko.]

\D
Koľkými spôsobmi možno pokryť dominovými kockami obdĺžnik
$2\times 10$? [Postupujte indukciou pre obdĺžniky $2\times n$. Nájdete tak súvislosť
s~Fibonacciho číslami v~podobe rovností $p(n)=p(n-1)+p(n-2)$, pričom $p(n)$
označuje počet možných vydláždení obdĺžnika $2\times n$. Výsledok je 89.]

Majme šachovnicu $8\times8$ a~ku každej "hrane", ktorá
oddeľuje dve jej políčka, napíšme prirodzené číslo, ktoré udáva počet
spôsobov, ako možno celú šachovnicu
rozrezať na obdĺžničky
$2\times1$ tak, aby dotyčná hrana bola
súčasťou rezu.
Určte poslednú cifru súčtu všetkých takto napísaných čísel.
[\pdfklink{63-A-III-3}{https://skmo.sk/dokument.php?id=1071}. Aký je príspevok jedného vydláždenia do
celkového súčtu?]

Ukážte, že pre každé prirodzené číslo~$n$ existuje útvar,
ktorý možno kockami domina vydláždiť presne $n$~spôsobmi. [Postupujte indukciou.
Stačí vhodne \uv{prilepovať} stále rovnaký kus. (\obr)]
\insp{domina.eps}%
\endnávod
}

{%%%%%   A-I-3
Dokazovanú nerovnosť budeme v~priebehu riešenia ekvivalentne upravovať.
Začneme úpravou na tvar
$$
\frac{r}{p}+\frac{r}{q} > 1. \tag1
$$
\insp{a69.3}%

Vďaka zrejmej podobnosti pravouhlých trojuholníkov (\obr) platí
$$
\frac{r}{p}=\frac{|CR|}{|CP|}=\frac{|CR|}{|CR|+|PR|}
$$
a~podobne aj
$$
\frac{r}{q}=\frac{|BR|}{|BQ|}=\frac{|BR|}{|BR|+|RQ|}.
$$
Nerovnosť \thetag{1} tak prepíšeme ako
$$
\frac{|CR|}{|CR|+|PR|}+\frac{|BR|}{|BR|+|RQ|} > 1,
$$
čo možno ďalej (ekvivalentne) upraviť roznásobením a~odčítaním zhodných
členov na tvar
$$
|BR|\cdot |CR| > |RQ| \cdot |PR|.\tag2
$$
Súčiny na oboch stranách nerovnosti~(2) nám pripomenú známy
vzorec na výpočet obsahu trojuholníka $S= \frac12 ab \sin \gamma$.
Preto vynásobíme obe strany nerovnosti~(2) kladným číslom $\frac12 \sin\varphi$,
pričom $\varphi$ je spoločná veľkosť uhlov $BRC$ a~$QRP$. Výsledná
nerovnosť
$$
\postdisplaypenalty 10000
\frac12 \sin \varphi \cdot|BR|\cdot |CR| > \frac12 \sin \varphi\cdot|RQ| \cdot |PR|
$$
potom platí práve vtedy, keď $S_{BRC} > S_{PRQ}$.

Poslednú nerovnosť už dokážeme ľahko. Keďže bod~$C$ má väčšiu
vzdialenosť od priamky~$BP$ ako bod~$Q$, platí $S_{BPC} > S_{BPQ}$. Po
odčítaní obsahu trojuholníka $BPR$ od oboch strán už získame želané
$S_{BRC} > S_{PRQ}$, a~dôkaz je tak na konci.

\ineres
V~tomto riešení budeme pracovať s~pomermi
obsahov. Ak označíme $S_1 = S_{BRC}$, $S_2 = S_{CRQ}$, $S_3 = S_{PRQ}$,
$S_4 = S_{BRP}$, môžeme dĺžky $p$, $q$, $r$ vyjadriť ako
$$
r=\frac{2S_1}{|BC|}, \quad
q=\frac{2(S_1+S_2)}{|BC|}, \quad
p=\frac{2(S_1+S_4)}{|BC|}
$$
a~následne prepísať dokazovanú nerovnosť na tvar
$$
\frac{1}{S_1+S_4}+\frac{1}{S_1+S_2} > \frac{1}{S_1},
$$
z~ktorého po prevedení ekvivalentných úprav dostaneme
$$
S_1^2 > S_2S_4. \tag3
$$
Keďže zároveň pre obsahy $S_1$, $S_2$, $S_3$, $S_4$ platí známy vzťah
$S_1S_3 = S_2S_4$ (pozri úlohu~N1), môžeme pravú stranu~(3) nahradiť výrazom $S_1S_3$,
a~po krátení nenulovým obsahom~$S_1$ tak získame ekvivalentnú nerovnosť $S_1>S_3$,
čiže $S_{BRC} > S_{PRQ}$. Tú dokážeme rovnako ako v~prvom riešení.

\ineres
Bodmi $R$ a~$Q$ vedieme rovnobežky so
stranou~$AB$ a~ich priesečníky s~$BC$ označíme postupne $R'$ a~$Q'$.
Vďaka rovnosti uhlov $|\uhel CQ'Q| = |\uhel CR'R| =|\uhel CBA| = \beta$
a~vzniknutým pravouhlým trojuholníkom (\obr) môžeme písať
$$
p = |BP| \sin \beta, \qquad r = |RR'| \sin \beta, \qquad q = |QQ'|\sin \beta.
$$
\insp{a69.4}%

Po dosadení do dokazovanej nerovnosti a~vynásobení oboch strán
kladnou hodnotou $|RR'|\sin\beta$ získame ekvivalentnú nerovnosť
$$
\frac{|RR'|}{|BP|}+\frac{|RR'|}{|QQ'|} > 1.
$$
Z~podobností $\triangle CBP \sim \triangle CR'R$ ($uu$) a~$\triangle BQ'Q
\sim \triangle BR'R$ ($uu$) dostaneme
$$
\frac{|RR'|}{|BP|}= \frac{|CR'|}{|BC|}\quad\hbox{a} \quad \frac{|RR'|}{|QQ'|} =
\frac{|BR'|}{|BQ'|}.
$$
A~keďže $Q'$ leží vnútri úsečky $BC$, platí $|BQ'| < |BC|$, takže
$$
\frac{|RR'|}{|BP|}+\frac{|RR'|}{|QQ'|}
= \frac{|CR'|}{|BC|} +\frac{|BR'|}{|BQ'|}
> \frac{|CR'|}{|BC|} + \frac{|BR'|}{|BC|} = 1,
$$
čo sme chceli dokázať.

\poznamka
Postup z~posledného riešenia so zavedenými bodmi $R'$ a~$Q'$ možno
obmeniť tak, že namiesto dokazovanej nerovnosti odvodíme jej spresnenie
v~podobe rovnosti
$$
\frac{1}{p}+\frac{1}{q}=\frac{1}{r}+\frac{1}{v},
$$
pričom $v$ označuje vzdialenosť bodu~$A$ od priamky~$BC$. Za tým účelom rovnosť po
vynásobení hodnotou~$r$ prepíšeme na tvar
$$
\frac{r}{p}+\frac{r}{q}\left(1-\frac{q}{v}\right)=1
$$
a~výraz na ľavej strane upravíme použitím výšok podobných trojuholníkov takto:
$$
\align
\frac{r}{p}+\frac{r}{q}\left(1-\frac{q}{v}\right)=&
\frac{|R'C|}{|BC|}+\frac{|BR'|}{|BQ'|}\left(1-\frac{|Q'C|}{|BC|}\right)=\\
=&\frac{|R'C|}{|BC|}+\frac{|BR'|}{|BQ'|}\cdot\frac{|BC|-|Q'C|}{|BC|}=\\
=&\frac{|R'C|}{|BC|}+\frac{|BR'|}{|BQ'|}\cdot\frac{|BQ'|}{|BC|}=
\frac{|R'C|+|BR'|}{|BC|}=\frac{|BC|}{|BC|}=1.
\endalign
$$

\ineres
Ešte iným spôsobom dokážeme rovnosť
$$
\frac1p + \frac1q = \frac1r + \frac1v
$$
z~poznámky k~predošlému riešeniu.
Bez ujmy na všeobecnosti predpokladajme, že $S_{ABC}=1$.
Označme $S_{BRC}=x$, $S_{ARC}=y$ a~$S_{ABR}=z$, čiže $x+y+z=1$. Potom platí
$$
\frac{v}{p} = \frac{|AB|}{|PB|} = 1+\frac{|AP|}{|PB|}
% = 1+\frac{S_{APR}}{S_{PBR}}
= 1 + \frac{S_{ARC}}{S_{BRC}}
= 1 + \frac{y}{x}
$$
(použili sme výsledok dopĺňajúcej úlohy D2).
Analogicky odvodíme rovnosť
$$
\frac{v}{q} = 1 + \frac{z}{x}.
$$
Navyše zrejme platí
$$
\frac{v}{r} = \frac{S_{ABC}}{S_{BRC}} = \frac1x.
$$
Spolu teda máme
$$
\frac{v}{p} + \frac{v}{q} - \frac{v}{r} = \left(1+\frac{y}{x}\right) + \left(1+\frac{z}{x}\right) - \frac1x = \frac{x+y+x+z-1}{x} = 1,
$$
čo je ekvivalentné dokazovanej rovnosti.

Dodajme, že čísla $x$, $y$, $z$ tvoria trojicu takzvaných barycentrických
súradníc bodu~$R$ v~rovine~$ABC$.
Riešenie je elementárnym prepisom úvah, ktoré sú pri ich používaní bežné.


\návody
Daný je konvexný štvoruholník $ABCD$ s~priesečníkom uhlopriečok~$P$. Dokážte, že platí
$$
S_{APB}\cdot S_{CPD} = S_{BPC}\cdot S_{DPA}.
$$
[Vyjadrite obsahy všetkých štyroch trojuholníkov pomocou takých
základní, ktoré ležia na jednej uhlopriečke štvoruholníka $ABCD$.]

Dokážte, že v~konfigurácii zo zadania úlohy platí $S_{BRC} >S_{PRQ}$.
[Porovnajte obsahy trojuholníkov $BCP$ a~$BQP$.]

\D
Body $P$ a~$Q$ ležia v~tej istej polrovine určenej priamkou~$l$. Ich kolmé priemety na priamku~$l$ označme ako $P'$ a~$Q'$
a~priesečník priamok $PQ'$ a~$P'Q$ ako $R$. Dokážte, že pre vzdialenosti $p$,
$q$, $r$ bodov $P$, $Q$, $R$ od priamky $l$ platí
$$
\frac{1}{r} = \frac{1}{p} + \frac{1}{q}.
$$
[Uvažujte o~vhodných dvojiciach podobných trojuholníkov, ktoré
sú určené zadanými bodmi a~ich kolmými priemetmi na priamku~$l$.]

Daný je bod $X$ vnútri trojuholníka $ABC$. Dokážte, že
ak označíme $D$ priesečník priamok $AX$ a~$BC$, platí
$$
\frac{S_{AXB}}{S_{AXC}} = \frac{|BD|}{|DC|}.
$$
[Nakreslite priamku $AX$ vodorovne. Potom vynikne, že oba zlomky sa rovnajú
podielu vzdialeností bodov $B$ a~$C$ od priamky~$AX$.]

[Cevova veta (časť)] Na stranách $BC$, $CA$ a~$AB$
trojuholníka $ABC$ sú postupne zvolené body $D$, $E$ a~$F$ tak, že sa
priamky $AD$, $BE$ a~$CF$ pretínajú v~jednom bode. Dokážte, že potom platí
$$
\frac{|BD|}{|DC|} \cdot\frac{|CE|}{|EA|} \cdot\frac{|AF|}{|FB|} = 1.
$$
[Využite trikrát výsledok úlohy~D2.]
\endnávod}

{%%%%%   A-I-4
So všetkými číslami budeme počítať ako so zvyškovými triedami po delení
prvočíslom~43.

Najskôr si uvedomíme, že množiny $\mm P$ a~$\mm Q$ tvoria disjunktný rozklad
množiny~$\mm M$.
Nech $\mm P$ je polovičatá množina a~nech $x\in \mm P$, pre ktoré potom $7x\in\mm Q$.

Ukážeme, že $7^2x \in \mm P$.
Ak naopak predpokladáme, že $7^2x \in \mm Q$, tak existuje $y \in \mm P$
také, že $7^2x = 7y$. Podľa návodnej úlohy~N2 je také~$y$ určené
jednoznačne, a~teda je ním~$7x$. Keďže $7x \not\in \mm P$, dostávame
potrebný spor.

Potom samozrejme $7^3x \in \mm Q$ a~zopakovaním predošlého argumentu ukážeme,
že $7^4x \in \mm P$, a~následne aj $7^5x \in \mm Q$. V~princípe by sme takto
mohli pokračovať aj ďalej, no už $7^6x \equiv 49^3x \equiv 6^3x
\equiv x \pmod{43}$, čím sa \uv{cyklus} uzavrie.
Cyklická šestica (rôznych) čísel $(x, 7x, 7^2x, 7^3x, 7^4x, 7^5x)$ má tak
zaradenie $(\mm P, \mm Q, \mm P, \mm Q, \mm P, \mm Q)$.

Množinu $\mm M$ teraz rozdelíme do siedmich takých cyklických šestíc ako
$$
\gather
(1, 7, 6, 42, 36, 37), \quad (2, 14, 12, 41, 29, 31), \quad (3, 21, 18, 40, 22, 25), \\
(4, 28, 24, 39, 15, 19), \quad (5, 35, 30, 38, 8, 13), \quad (9, 20, 11, 34, 23, 32), \\
(10, 27, 17, 33, 16, 26).
\endgather
$$

Každá z~týchto šestíc bude mať zaradenie buď $(\mm P,\mm Q,\mm P,\mm Q, \mm P, \mm Q)$,
alebo $(\mm Q, \mm P,\mm Q,\mm P,\allowbreak \mm Q, \mm P)$,
pričom každá taká voľba určí množiny $\mm P$ a~$\mm Q$
vyhovujúce podmienkam úlohy. Počet možností je preto $2^7 = 128$.

\poznamka
V skutočnosti nie je nutné sedem cyklických šestíc nájsť
priamo. Stačí iba dokázať {\it existenciu\/} takého rozkladu množiny~$\mm M$.
Tá vyplýva z~dvoch jednoduchších tvrdení:
\ite (i) Každý prvok $\mm M$ je prvkom nejakej šestice.
\ite (ii) Ak majú dve šestice spoločný jeden prvok, potom už majú
spoločné všetky prvky.

Platnosť prvého tvrdenia je okamžitá a~na dôkaz druhého si stačí
uvedomiť, že podľa konštrukcie sú šestice uzavreté na násobenie číslom~7,
a~jeden spoločný prvok tak \uv{vygeneruje} postupným násobením siedmimi
päť ďalších spoločných prvkov.



\návody
Dané je prvočíslo~$p$, množina $\mm U~= \{0, 1, \dots, p-1\}$
a~prirodzené číslo~$a$ nesúdeliteľné s~$p$. Dokážte, že žiadne dve čísla
z~množiny
$\mm V~= \{0a, 1 a, \dots, (p-1)a\}$ nedávajú rovnaký
zvyšok po delení prvočíslom~$p$.
[Predpokladajte opak a~ku sporu doveďte fakt, že $p$ delí rozdiel
nejakých dvoch čísel z~množiny~$\mm V$.]

[\uv{Sedemnásobok je jednoznačný}] Pri označení z~úlohy
dokážte, že pre každé $x \in \mm M$ existuje práve jedno $y \in \mm M$ také,
že $x \equiv 7y \pmod{43}$ (čítajte ako \uv{$x$ dáva rovnaký zvyšok ako~$7y$ po delení 43}).
[Podľa N1 pre $p=43$ a~$a=7$ sa medzi násobkami 7 všetkých nenulových zvyškov objaví
každý nenulový zvyšok práve raz, teda aj vopred zvolený zvyšok~$x$.]

Aké zvyšky dávajú mocniny siedmich po delení~43? Ak je $1 \in\mm P$,
čo možno povedať o~príslušnosti týchto zvyškov do množín $\mm P$ a~$\mm Q$ zo
zadania úlohy?
[1, 7, 6, 42, 36, 37. Ďalej si rozmyslite, prečo v~prípade
$1\in\mm P$ je už príslušnosť ostatných piatich určených zvyškov do
množín $\mm P$ a~$\mm Q$ určená jednoznačne: $6, 36\in\mm P$
a~$7, 42, 37\in\mm Q$.]

\D
Dané je prvočíslo $p$, množina $\mm U~= \{1, \dots, p-1\}$
a~jej prvok $a$. Dokážte, že potom existuje práve jeden prvok
$b\in \mm U$ taký, že $ab$ dáva zvyšok $1$ po delení $p$.
[Podľa N1 je medzi \uv{násobkami} čísla~$a$ zastúpený každý možný zvyšok
po delení $p$ práve raz. Teda aj zvyšok~$1$.]

[Malá Fermatova veta] Dané je prvočíslo $p$ a~prirodzené číslo~$a$ nesúdeliteľné s~$p$.
Dokážte, že potom $p \mid a^{p-1} - 1$.
[Podľa N1 sú množiny $\mm U$ a~$\mm V$ (po redukcii na zvyšky po
delení $p$) zhodné. Rovnať sa tak musia aj súčiny ich nenulových
prvkov. Čo vyjde?]

[Wilsonova veta] Dokážte, že pre každé prvočíslo $p$ platí $p\mid (p-1)! + 1$.
[Tvrdenie je triviálne pre $p\in\{2, 3\}$. V~prípade $p\ge5$
zo súčinu vyškrtneme všetky dvojice činiteľov, ktorých súčin dáva
po delení $p$ zvyšok 1. Ktoré dve čísla zvýšia (lebo tvoria
takú dvojicu sami so sebou)?]
\endnávod}

{%%%%%   A-I-5
Najskôr si uvedomme, že všetky uvažované trojuholníky $ABC$ majú spoločnú
kružnicu opísanú $k(O,|OA|)$. Jej priemer s~krajným bodom~$A$ má, ako
je známe, za druhý krajný bod obraz~$V_a$ ortocentra~$V$ trojuholníka $ABC$
v~súmernosti podľa stredu~$M$ strany~$BC$ (pozri návodné úlohy N1-N3
a~\obr{} vľavo). Tento bod $V_a\in k$ je teda takisto všetkým uvažovaným
trojuholníkom $ABC$ spoločný.
\insp{a69.5}%

Z~predchádzajúceho vyplýva, že hľadaná množina všetkých ortocentier~$V$ je obrazom
množiny stredov~$M$ všetkých tetív~$BC$ kružnice~$k$ ($B\ne A$, $C\ne A$) v~rovnoľahlosti~$\Cal{H}(V_a, 2)$, lebo tri rôzne body jednej kružnice vždy určujú trojuholník.
Stačí teda určiť množinu všetkých opísaných stredov~$M$, o~ktorých vopred vieme,
že všetky ležia vo vnútri kružnice~$k$.

\mppic a69.6 \hfil\Obr \par
\inspicture r(2.25)
Je jasné, že stred~$O$ do tejto množiny patrí. Pre ostatné body~$M$
vnútra danej kružnice~$k$ platí, že sú stredom jej vhodnej tetivy,
ktorá je bodom~$M$ určená
jednoznačne (podľa klasickej konštrukcie~-- stačí pretnúť kolmicu
na $OM$ vedenú bodom~$M$ s~kružnicou~$k$, tri
príklady sú vykreslené na \obrr1{} vpravo). Táto konštrukcia pritom zlyhá
(\tj.~nedostaneme {\it trojuholník}~$ABC$)
práve vtedy, keď je jedným z~krajných bodov zostrojenej tetivy bod~$A$. To sa
stane práve pre body~$M$ ležiace na obraze~$k'$ kružnice~$k$ v~rovnoľahlosti $\Cal{H}(A, \frac12)$ s~výnimkou bodu~$O$, ktorý bol
vyšetrený zvlášť (a~ku ktorému naopak existuje nekonečne veľa vhodných tetív).

Určili sme tak množinu všetkých bodov~$M$, ktorou je vnútro kružnice~$k$,
z~ktorého je odobraná kružnica~$k'$ s~výnimkou bodu~$O$. Hľadaná množina ortocentier
je, ako vieme, obrazom tejto množiny v~rovnoľahlosti~$\Cal H(V_a, 2)$.
Keďže obrazom kružnice~$k$ je kružnica so stredom~$A$
a~polomerom $|AV_a|$, zatiaľ čo obrazom kružnice~$k'$ je kružnica~$k''$,
ktorá je súčasne obrazom kružnice~$k$ v~súmernosti podľa
stredu~$A$, a~napokon obrazom bodu~$O$ v~danej rovnoľahlosti je bod~$A$, je
hľadaná množina (vrátane konštrukcie) určená a~znázornená na \obr.
Je ňou vnútro kružnice so stredom~$A$ a~polomerom~$|AV_a|$,
z~ktorého je odobraná kružnica~$k''$ a~do ktorého je naopak vrátený bod~$A$.

\návody
Daný je trojuholník $ABC$ s~priesečníkom výšok~$V$. Dokážte, že
$|\uhel BVC| = 180^\circ - |\uhel BAC|$. [Nezabudnite na prípad, keď je
trojuholník $ABC$ tupouhlý.]

Daný je trojuholník $ABC$ s~priesečníkom výšok~$V$. Dokážte, že
obraz~$V'$ bodu~$V$ v~súmernosti podľa priamky~$BC$ padne na kružnicu opísanú
trojuholníku $ABC$. Dokážte, že to isté platí aj pre obraz~$V''$ bodu~$V$ v~súmernosti podľa stredu úsečky~$BC$.
[V~oboch dôkazoch využite výsledok úlohy~N1.]

Daný je trojuholník $ABC$ s~priesečníkom výšok $V$. Z~úlohy~N2 vieme,
že obraz $V''$ bodu~$V$ v stredovej súmernosti podľa stredu
strany $BC$ leží na kružnici opísanej. Dokážte navyše, že $AV''$ je
priemerom tejto kružnice.
[Vďaka stredovej súmernosti platí $CV\parallel
BV''$, a~preto z~$CV\perp AB$ vyplýva $BV''\perp AB$.]

Daná je kružnica $k$ so stredom $S$ a~vnútri nej bod $X$. Určte množinu
stredov všetkých tetív kružnice $k$, ktoré prechádzajú bodom $X$.
[V~prípade $X=S$ sa jedná o~množinu~$\{S\}$, v~prípade $X\ne S$
o~Tálesovu kružnicu nad priemerom $XS$.]

Daná je kružnica $k$ so stredom $S$ a~vnútri nej bod $X$. Určte množinu
bodov, ktoré sú stredom nejakej jej tetivy, ktorá neobsahuje bod~$X$.
[Na konštrukciu tetivy s~daným stredom využite fakt, že os každej
tetivy prechádza stredom kružnice. Kedy však táto konštrukcia vedie
k~tetive, ktorá bodom~$X$ prechádza? Hľadanou množinou je vnútro kružnice~$k$, a~to v~prípade $X=S$
s~výnimkou jediného bodu $S$, v~prípade $X\ne S$ sú vylúčené všetky
body Tálesovej kružnice nad priemerom $XS$ okrem bodu $S$ (ktorý tentoraz
do výslednej množiny naopak patrí).]

\D
[Feuerbachova kružnica] Daný je trojuholník $ABC$
s~priesečníkom výšok $V$ a~stredom~$O$ kružnice opísanej. Dokážte, že stredy
jeho strán, stredy spojníc vrcholov s~jeho ortocentrom a~päty jeho výšok
ležia na jednej kružnici, ktorej stred je navyše stredom úsečky $VO$.
[Využite výsledok úlohy~N2 a~použite rovnoľahlosť
$\Cal{H}(V, \frac12)$.]


V~rovine~$\omega$ sú dané dva rôzne body $O$ a~$T$. Nájdite množinu vrcholov všetkých trojuholníkov, ktoré ležia v~rovine~$\omega$ a~majú ťažisko v~bode~$T$ a~stred opísanej kružnice v~bode~$O$.
[\pdfklink{58-A-III-6}{https://skmo.sk/dokument.php?id=33}]
\endnávod}

{%%%%%   A-I-6
Hľadané trojice majú spĺňať rovnosť
$$
(a+2b)(b+2c)(c+2a)=p^n, \tag1
$$
pričom $p$ je vhodné prvočíslo a~$n$ je celé číslo, ktoré je
aspoň~3, lebo každý z~činiteľov na ľavej strane \thetag{1} je väčší
ako~1.

Keby všetky tri čísla $a$, $b$, $c$ boli násobkami prvočísla~$p$,
spĺňala by trojica prirodzených čísel
$(\frc{a}{p},\frc{b}{p},\frc{c}{p})$
vďaka~(1) podobnú rovnosť
$$
\Big(\frac{a}{p}+2\frac{b}{p}\Big)
\Big(\frac{b}{p}+2\frac{c}{p}\Big)
\Big(\frac{c}{p}+2\frac{a}{p}\Big)=p^{n-3},
$$
takže by bola tiež riešením úlohy. Keby aj v~tejto trojici boli
všetky tri čísla násobkami prvočísla~$p$, mohli by sme prejsť
k~riešeniu $(\frc{a}{p^2},\frc{b}{p^2},\frc{c}{p^2})$ atď., až po
konečnom počte krokov dôjdeme k~záveru, že každá
vyhovujúca trojica $(a,b,c)$ je tvaru
$$
a=p^k~a_1,\
b=p^k~b_1,\
c=p^k~c_1,
$$
pričom $k\ge0$ je celé číslo, aspoň jedno z~prirodzených čísel
$a_1$, $b_1$, $c_1$ nie je deliteľné prvočíslom~$p$ a~pritom súčin
troch činiteľov $a_1+2b_1$, $b_1+2c_1$, $c_1+2a_1$ je mocninou
prvočísla~$p$, takže je mocninou $p$ aj každý z~nich:
$$
a_1+2b_1=p^{\al},\ b_1+2c_1=p^{\be},\ c_1+2a_1=p^{\ga}. \tag2
$$
Celočíselné exponenty $\al$, $\be$, $\ga$ sú pritom kladné, lebo na
ľavých stranách \thetag{2} sú opäť čísla väčšie ako~1.

Ak sa pozrieme na \thetag{2} ako na sústavu troch lineárnych rovníc,
ľahko určíme jej jediné riešenie
$$
a_1=\frac{p^{\al}-2p^{\be}+4p^{\ga}}{9},\quad
b_1=\frac{p^{\be}-2p^{\ga}+4p^{\al}}{9},\quad
c_1=\frac{p^{\ga}-2p^{\al}+4p^{\be}}{9}. \tag3
$$

Vďaka tomu, že $\min(\al,\be,\ga)\ge1$, sú čitatele zlomkov
v~\thetag{3} násobkami prvočísla~$p$. Keby teda neplatilo $p=3$, boli
by
všetky tri čísla $a_1$, $b_1$, $c_1$ v~rozpore s~ich výberom
deliteľné prvočíslom~$p$, lebo zlomky v~\thetag{3} majú menovateľ
$3^2$, takže ich krátenie prvočíslom~$p$ nie je možné. Nutne
preto platí $p=3$.

Vzhľadom na cyklickosť sústavy \thetag{2} môžeme predpokladať, že
číslo $\ga$ je z~čísel $\al$, $\be$, $\ga$ maximálne a~že ak
je v~tejto trojici maximálnych čísel viac, je také okrem
čísla $\ga$ aj číslo $\be$. Platí tak
$$
\text{buď}\quad\max(\al,\be)<\ga,\quad\text{alebo}\quad \al\le\be=\ga. \tag4
$$
Z~nerovnosti $b_1>0$ potom podľa~(3) s~dosadeným $p=3$ máme
$4\cdot3^{\al}>2\cdot3^{\ga}-3^{\be}\ge3^{\ga}$, odkiaľ
$3^{\ga-\al}<4$, čiže $\ga-\al\in\{0, 1\}$. Obe možné hodnoty
teraz analyzujme osobitne.

(i) V~prípade $\ga-\al=0$ máme podľa \thetag{4} rovnosti $\al=\be=\ga$.
Dosadením do (3) dostávame $a_1=b_1=c_1=3^{\ga-1}$, teda $\ga=1$
a~$(a,b,c)=(3^k, 3^k, 3^k)$.

(ii) V~prípade $\ga-\al=1$ musí byť $\be=\ga$,
inak by sme totiž podľa \thetag{4} zo vzťahov $\be\le\ga-1$
a~$\al=\ga-1$ pre hodnotu $b_1$ z~\thetag{3} dostali
$$
9b_1=3^{\be}-2\cdot3^{\ga}+4\cdot 3^{\al}\le
3^{\ga-1}-2\cdot3^{\ga}+4\cdot 3^{\ga-1}=-3^{\ga-1}<0,
$$
a~to je spor. Platí preto $\be=\ga$, čo spolu s~$\al=\ga-1$ po
dosadení do~\thetag{3} dáva
$$
\align
a_1&=\frac{3^{\ga-1}-2\cdot3^{\ga}+4\cdot3^{\ga}}{9}=7\cdot3^{\ga-3},\\
b_1&=\frac{3^{\ga}-2\cdot3^{\ga}+4\cdot3^{\ga-1}}{9}=3^{\ga-3},\\
c_1&=\frac{3^{\ga}-2\cdot3^{\ga-1}+4\cdot3^{\ga}}{9}=13\cdot3^{\ga-3},
\endalign
$$
teda $\ga=3$, $(a_1,b_1,c_1)=(7, 1, 13)$,
$(a,b,c)=(7\cdot3^{k}, 3^{k}, 13\cdot3^{k})$.

\odpoved
Hľadané trojice sú $(3^k, 3^k, 3^k)$
a~$(7\cdot3^{k}, 3^{k}, 13\cdot3^{k})$
(až na cyklickú permutáciu), pritom $k$ je ľubovoľné nezáporné celé
číslo.

\poznamka
Keby sme v~zadaní úlohy nepožadovali, aby celé
čísla $a$, $b$, $c$ boli {\it kladné}, existovali by mnohé ďalšie
vyhovujúce trojice, napríklad $(2^{k}, 0, 2^{k+1})$
alebo $({11\cdot5^k}, \m5^{k+1}, 3\cdot5^{k})$.



\návody
Určte všetky dvojice celých kladných čísel $a$ a~$b$, pre ktoré
je súčin $(a+2b)(b+2a)$ mocninou niektorého prvočísla~$p$. [$a=b=3^k$,
pričom $k\ge0$. Zo sústavy rovníc $a+2b=p^u$ a~$b+2a=p^v$ vyjadrite
neznáme $a$, $b$ a~dokážte, že ich hodnoty sú obe kladné, len keď
sa prirodzené čísla $u$ a~$v$ rovnajú -- potom ale aj $a=b$.]

Nájdite všetky trojice $a$, $b$, $c$ kladných celých čísel
takých, že súčin
$(a+b)(b+c)\*(c+a)$
je rovný mocnine niektorého prvočísla.
[Dokážte, že aspoň jedna zátvorka je párna, a~preto $p=2$.
Pri usporiadaní $a\ge b\ge c$, ktoré môžeme predpokladať, bude pre tri
mocniny $a+b$, $a+c$, $b+c$ čísla 2 platiť $a+b\ge a+c\ge b+c$. Keby
$a+b$, $a+c$ boli rôzne mocniny čísla 2, bola by prvá aspoň
dvojnásobkom druhej, \tj. $a+b\ge 2(a+c)$, odkiaľ by vyplývalo $b\ge a+2c >
a$, a~to je spor s~$a\ge b$. Preto musí platiť $a+b=a+c$, čiže
$b=c$. To už vedie k~záveru, že úlohe okrem ľahko uhádnuteľných trojíc
$[2^k, 2^k, 2^k]$ vyhovujú v~ľubovoľnom poradí aj trojice čísel
$[2^n-2^k, 2^k, 2^k]$, pričom $0\le k<n-1$, a~že žiadne iné riešenia
neexistujú.]

\D
Pre navzájom rôzne prirodzené čísla $a$, $b$, $c$ platí, že
$$(a+b+c) \mid (a+2b)(b+2c)(c+2a).$$
Dokážte, že číslo $a+b+c$ je zložené.
[Ak by $a+b+c$ bolo prvočíslom, delilo by jednu zo zátvoriek, povedzme $a+2b$,
a~teda by delilo aj rozdiel $a+2b - (a+b+c) = b-c$, takže by
platilo $a+b+c< |b-c|$, čo je spor.]

\endnávod}

{%%%%%   B-I-1
Sústavu riešime ako lineárnu sústavu rovníc s~neznámymi $x^4$ a~$y^2$.
Sčítaním oboch rovníc a~vydelením dvoma dostaneme
$$
x^4=\frac12\left(\Big(a+{1\over a}\Big)^{\!3}+\Big(a-{1\over a}\Big)^{\!3}\right)
=a^3+\frac3a.
\tag1
$$
Podobne odčítaním druhej rovnice od prvej a~vydelením dvoma dostaneme
$$
y^2=\frac12\left(\Big(a+{1\over a}\Big)^{\!3}-\Big(a-{1\over a}\Big)^{\!3}\right)
=3a+{1\over a^{\!3}}.
\tag2
$$

% \ite
a)
Ak daná sústava rovníc má riešenie v~obore reálnych čísel, nutne $y^2\ge0$.
Zo vzťahu~(2) vidíme, že musí byť $a>0$. Naopak, ak je $a>0$,
sú obe pravé strany rovníc (1) a~(2) kladné, a~ich odmocnením
tak nájdeme reálne čísla $x$ a~$y$, ktoré sú riešením ako sústavy
$(1)\land(2)$, tak aj s~ňou ekvivalentnej pôvodnej sústavy rovníc.

% \ite
b)
Pre kladné reálne číslo $a$ podľa nerovnosti medzi aritmetickým
a~geometrickým priemerom dvoch kladných čísel $x^2$ a~$|y|$ a~vďaka rovnostiam (1)
a~(2) platí
$$
x^2+|y|\ge2\sqrt{x^2|y|}=2\root4\of{x^4y^2}=2\root4\of{\Big(a^3+\frac3a\Big)
\Big(3a+\frac1{a^3}\Big)}=2\root4\of{3\Big(a^4+{1\over a^4}\Big)+10}.
$$

Podľa rovnakej nerovnosti medzi priemermi kladných reálnych čísel $a^4$ a~$\frc1{a^4}$
ďalej platí
$$
a^4+\frac1{a^4}\ge2\sqrt{a^4\cdot\frac1{a^4}}=2.
$$
Spolu dostávame
$$
\postdisplaypenalty 10000
x^2+|y|\ge2\root4\of{3\Big(a^4+{1\over a^4}\Big)+10}\ge2\root4\of{16}=4,
$$
čo sme mali dokázať.

Dva priemery (aritmetický a~geometrický), na ktoré sme sa odvolali
(dokonca dvakrát), sa vo všeobecnosti rovnajú iba v~situácii, keď sa rovnajú obe
priemerované hodnoty (pozri návodnú úlohu~N3).

Rovnosť v~dokázanej nerovnosti nastane práve vtedy, keď $x^2=|y|$ a~súčasne
$a^4=\frc1{a^4}$. Z~druhej rovnosti vzhľadom na $a>0$ dostávame $a=1$, čo dosadené
do (1) a~(2) dáva $x^4=4$ a~$y^2=4$, je teda splnená aj prvá rovnosť $x^2=|y|=2$.
Rovnosť v~dokázanej nerovnosti teda nastane práve vtedy, keď $a=1$.

\poznamka
Ak riešitelia poznajú aj nerovnosť medzi aritmetickým a~geometrickým priemerom
štyroch nezáporných čísel, môžu časť~b) dokázať nasledovne. Platí
$$
x^4=a^3+\frac3a=a^3+\frac1a+\frac1a+\frac1a
\ge4\root4\of{a^3\cdot\frac1a\cdot\frac1a\cdot\frac1a}=4,
$$
pričom rovnosť tu nastane práve vtedy, keď $a=1$. Podobne sa dokáže
aj nerovnosť
$$
y^2=3a+{1\over a^3}\ge4
$$
opäť s~dodatkom, že rovnosť nastane práve vtedy, keď $a=1$. Potom už ľahko
dostaneme
$$
x^2+|y|=\sqrt{x^4}+\sqrt{y^2}\ge\sqrt{4}+\sqrt{4}=4.
$$

\návody
Pre ktoré hodnoty reálneho parametra $a$ má sústava rovníc
$$
\eqalign{
x^2+y^2 &= a,\cr
2x^2+y^2 &= a^2
}$$
riešenie v~obore reálnych čísel?
[Riešime ako lineárnu sústavu rovníc s~neznámymi $x^2$, $y^2$,
dostaneme $x^2=a^2-a=a(a-1)$, $y^2=2a-a^2=a(2-a)$. Z~$x^2\ge0$
dostaneme $a\in\Bbb R\setminus(0;1)$, z~$y^2\ge0$ máme
$a\in\langle0;2\rangle$, obe podmienky spĺňajú $a\in\{0\}\cup\langle1;2\rangle$.]

V~obore reálnych čísel riešte sústavu rovníc
$$
\eqalign{
x+y-z&= 2a,\cr
x-y+z&= 2b,\cr
-x+y+z&= 2c.\cr
}$$
s~reálnymi parametrami $a$, $b$, $c$ [$x=a+b$, $y=a+c$, $z=b+c$.]

Pre nezáporné reálne čísla $a$, $b$ platí tzv. nerovnosť medzi
aritmetickým a~geometrickým priemerom (AG-nerovnosť)
$$
\sqrt{ab}\le \frac{a+b}2.
$$
Dokážte. Kedy nastane rovnosť?
[Nerovnosť upravíme na $\big(\sqrt{\vphantom{b}a}-\sqrt b\big)^2\ge0$, ktorá
zrejme platí. Rovnosť nastane iba v~prípade $a=b$.]

Dokážte, že pre ľubovoľné kladné čísla $a$, $b$, $c$, $d$ platí
$$(ab + cd)\Bigl(\frac1{ac}+\frac1{bd}\Bigr)\ge 4.$$
[Roznásobíme výraz na ľavej strane a~využijeme nerovnosť $x+1/x\ge2$
(platnú $\forall x>0$) pre $x=a/d$ a~pre $x=b/c$.]

\D
Dokážte, že pre ľubovoľné čísla $a$, $b$ z~intervalu $\langle 1,\infty)$
platí nerovnosť
$$
(a^2+1)(b^2+1) - (a-1)^2 (b-1)^2 \ge 4
$$
a~zistite, kedy nastane rovnosť. [\pdfklink{59-C-II-2}{https://skmo.sk/dokument.php?id=320}]

Nájdite všetky reálne čísla $x$ a~$y$, pre ktoré výraz $2x^2 + y^2- 2xy + 2x + 4$ nadobúda svoju
najmenšiu hodnotu. [\pdfklink{65-C-I-3}{https://skmo.sk/dokument.php?id=1746}, část a)]

Dokážte, že pre ľubovoľné kladné reálne čísla $a$, $b$ platí
$$
\sqrt{ab}\le{2(a^2+3ab+b^2)\over5(a+b)}\le{a+b\over2},
$$
a~pre každú z~oboch nerovností zistite, kedy prechádza na rovnosť.
[\pdfklink{59-C-I-5}{https://skmo.sk/dokument.php?id=10}]

Nájdite najmenšiu možnú hodnotu výrazu
$$
3x^2-12xy+y^4,
$$
v ktorom $x$ a~$y$ sú ľubovoľné celé nezáporné čísla.
[\pdfklink{65-C-II-1}{https://skmo.sk/dokument.php?id=1904}]

Určte najmenšiu hodnotu výrazu
$$
V=x^2+\frac2{1+2x^2},
$$
pričom $x$ je ľubovoľné reálne číslo. Pre ktoré $x$ výraz~$V$ túto hodnotu nadobúda? [\pdfklink{64-B-II-2}{https://skmo.sk/dokument.php?id=1371}]

Určte všetky dvojice $(x, y)$ reálnych čísel, pre ktoré platí nerovnosť
$$
\belowdisplayskip 0pt
(x + y)\Bigl(\frac1x+\frac1y\Bigr)\ge\Bigl(\frac xy+\frac yx\Bigr)^{\!2}.
$$
[\pdfklink{63-B-I-2}{https://skmo.sk/dokument.php?id=1008}]

Určte všetky reálne čísla~$p$ také, že pre ľubovoľné kladné
čísla $x$, $y$ platí nerovnosť
$$
\belowdisplayskip 0pt
\frac{x^3+py^3}{x+y}\geq xy.
$$
[\pdfklink{50-B-II-1}{http://www.matematickaolympiada.cz/media/440635/B50ii.pdf}]

Nájdite všetky možné hodnoty súčtu $x+y$,
kde reálne čísla $x$, $y$ spĺňajú rovnosť $x^3+y^3=3xy$. [\pdfklink{48-B-I-6}{http://www.matematickaolympiada.cz/media/440612/B48i.pdf}]
\endnávod}

{%%%%%   B-I-2
Keďže $60=2^2\cdot3\cdot5$, stačí v~prvej časti dokázať, že prirodzené
číslo~$n$ je súčasne deliteľné navzájom nesúdeliteľnými číslami 3, 4 a~5.
Na to pre každé z~týchto čísel stačí ukázať, že je ním deliteľný
aspoň jeden dvojciferný deliteľ čísla~$n$.

Dvojciferných čísel je celkom~90. Medzi nimi sú
deliteľné piatimi čísla $10=2\cdot5$, $15=3\cdot5,\dots$, $95=19\cdot5$,
ktorých je práve~18. Medzi dvojcifernými číslami je tak $90-18=72$ čísel,
ktoré 5 deliteľné nie sú. Podľa Dirichletovho princípu tak medzi
ľubovoľnými 73~dvojcifernými číslami je aspoň jedno z~nich deliteľné~5,
teda číslo 5 delí aj číslo~$n$ zo zadania
úlohy. Podobne dvojciferné čísla deliteľné štyrmi
sú $12=3\cdot4$ až $96=24\cdot4$, ktorých je práve~22, teda
zvyšných $90-22=68$ čísel štyrmi deliteľných nie je. Medzi ľubovoľnými
73~dvojcifernými číslami je tak aspoň jedno, ktoré je deliteľné~4. A~napokon
číslom~3 sú deliteľné dvojciferné čísla $12=4\cdot3$ až $99=33\cdot3$, ktorých je~30,
a~teda podľa Dirichletovho princípu medzi 73~dvojcifernými číslami je aspoň
jedno deliteľné tromi. Tým je prvá časť riešenia hotová.

Teraz nájdeme príklad čísla~$n$, ktoré má práve 73~dvojciferných deliteľov.
Stačí sa určite obmedziť na delitele najmenšieho spoločného násobku
všetkých dvojciferných čísel, teda čísla
$$
2^6\cdot3^4\cdot5^2\cdot7^2\cdot
\underbrace{11\cdot13\cdot17\cdot\dots\cdot97}_{\setbox0=
\hbox{\sevenrm súčin všetkých dvojciferných prvočísel}\wd0=0pt\box0}.
$$
Vyberajme v~poradí podľa veľkosti prvočíselné delitele pre číslo~$n$
v~mocnine, v~akej sa vyskytujú v~uvedenom súčine, a~postupne počítajme,
koľko nových dvojciferných deliteľov čísla~$n$ každé nové prvočíslo
prinesie.

Nech teda najvyššia mocnina~2 deliaca číslo~$n$ je $2^6$.
Číslo~$n$ tak bude deliteľné tromi dvojcifernými číslami
$$
2^4=16,\quad 2^5=32,\quad 2^6=64.
$$

Podobne ak najvyššia mocnina čísla~3, ktorá delí $n$, je $3^4=81$, bude číslo~$n$
ďalej deliteľné dvojcifernými číslami
$$
\displaylines{
3\cdot4=12,\quad 3\cdot 8=24,\quad 3\cdot 16=48,\quad 3\cdot32=96,\cr
3^2\cdot2=18,\quad 3^2\cdot 4=36,\quad 3^2\cdot 8=72,\quad3^3=27,\quad 3^3\cdot2=54,\quad 3^4=81,}
$$
ktorých je 10.

Podobne ak najvyššia mocnina čísla~5, ktorá delí delí~$n$, je~$5^2$, bude
číslo~$n$ ďalej deliteľné dvojcifernými číslami
$$
\let\ \quad
\displaylines{
5\cdot2=10,\ 5\cdot3=15,\ 5\cdot 4=20,\ 5\cdot5=25,\ 5\cdot6=30,\ 5\cdot 8=40,\cr
5\cdot9=45,\ 5\cdot10=50,\ 5\cdot 12=60,\ 5\cdot15=75,\ 5\cdot16=80,\ 5\cdot 18=90,
}$$
ktorých je 12.

A~ak najvyššia mocnina čísla~7, ktorá delí~$n$, je~$7^2$, bude
číslo~$n$ zatiaľ deliteľné číslami od~2 do~14 s~výnimkou čísel $11$ a~$13$.
Týchto čísel je~11. Číslo~$n$ tak bude mať ďalej dvojciferné delitele,
ktoré získame vynásobením čísla~7 týmito 11~číslami. Vidíme, že teraz
má takto vybrané číslo~$n$ už $3+10+12+11=36$ deliteľov.

Všetky dvojciferné prvočísla $11, 13, \dots$ sa v~úvodnom súčine
vyskytujú v~prvej mocnine. Pokračujme preto
v~rozširovaní doposiaľ vytvorenej hodnoty $n={2^6\cdot3^4}\cdot{5^2\cdot7^2}$
s~36~dvojcifernými deliteľmi o~dvojciferné prvočíselné delitele s~jediným
zastúpením, a~to opäť postupne podľa ich veľkosti. Pridaním
prvočísla~11 získa číslo~$n$ ďalších 9~dvojciferných deliteľov
od $11\cdot1=11$ do $11\cdot9=99$. Potom bude mať číslo~$n$ už $36+9=45$ deliteľov.

Pridaním prvočísla $13$ budú ďalšími siedmimi dvojcifernými deliteľmi čísla~$n$
čísla $13\cdot1=13$ až $13\cdot7=91$, celkom už teda máme $45+7=52$
dvojciferných deliteľov čísla~$n$. S~každým pridaným prvočíslom
$p\in\{17, 19\}$, ktorým bude číslo~$n$ deliteľné, mu pribudne päť
dvojciferných deliteľov od~$p$ do~$5p$, a~bude ich mať $52+10=62$.
S~prvočíslom~23 dostaneme ďalšie štyri dvojciferné delitele $23\cdot 1=23$
až~$23\cdot 4=92$, celkom ich už máme $62+4=66$. S~každým z~prvočísel
$p\in\{29, 31\}$ dostaneme ďalšie tri dvojciferné delitele $p$, $2p$, $3p$,
takže dvojciferných deliteľov už je $66+6=72$.
Chýba už iba jeden dvojciferný deliteľ do požadovaného počtu~73, za neho
musíme vziať nejaké prvočíslo väčšie ako~50, napríklad~53.

Práve 73 dvojciferných deliteľov tak má napríklad číslo
$$
n=2^6\cdot3^4\cdot5^2\cdot7^2\cdot11\cdot13\cdot17\cdot19\cdot23\cdot29\cdot31\cdot53.
$$

\medskip
Pri konštrukcii vyhovujúceho čísla~$n$ môžeme postupovať aj inak:
ak napríklad najvyššia mocnina čísla~3, ktorá delí~$n$, bude iba~$3^3$
a~v~závere namiesto prvočísla~53 vezmeme prvočíslo~37, bude
nové~$n$ mať navyše dvojciferné delitele 37 a~74 a~z~pôvodných príde
o~delitele 81 a~53. Dostaneme tak iné vyhovujúce číslo
$$
n_1=2^6\cdot3^3\cdot5^2\cdot7^2\cdot11\cdot13\cdot17\cdot19\cdot23\cdot29\cdot31\cdot37.
$$

\ineres
Najmenší spoločný násobok všetkých 90~dvojciferných čísel
$$
2^6\cdot3^4\cdot5^2\cdot7^2\cdot
\underbrace{11\cdot13\cdot17\cdot\dots\cdot97}_{\setbox0=\hbox{\sevenrm súčin všetkých dvojciferných prvočísel}\wd0=0pt\box0}$$
má práve 90~dvojciferných deliteľov. Ak ho vydelíme všetkými prvočíslami
od~53 do~97, ktorých je práve~10, príde výsledné číslo o desať
dvojciferných deliteľov, a bude tak mať iba 80~dvojciferných deliteľov.

Ak tento podiel ďalej vydelíme prvočíslami $p\in\{37, 41, 43, 47\}$, príde
vzniknuté číslo s~každým prvočíslom~$p$ o dvojciferné delitele $p$
a~$2p$. Po týchto štyroch deleniach tak bude mať výsledné číslo ešte
$80-8=72$ dvojciferných deliteľov. Ak
by sme ďalej vydelili toto číslo nasledujúcim prvočíslom~$31$,
stratili by sme už tri dvojciferné delitele, 31, 62 a~93, čo je priveľa.
Aby ubudol jeden dvojciferný deliteľ, stačí naše číslo deliť~3 (stratíme
iba deliteľa $3^4=81$) alebo~2 (ubudne deliteľ $2^6=64$) a~dostaneme tak
vyhovujúce čísla
$$
n_1=2^6\cdot3^3\cdot5^2\cdot7^2\cdot11\cdot13\cdot17\cdot19\cdot23\cdot29\cdot31\cdot37
$$
(ktoré sme obdržali už predchádzajúcou konštrukciou)~a\relax
$$
n_2=2^5\cdot3^4\cdot5^2\cdot7^2\cdot11\cdot13\cdot17\cdot19\cdot23\cdot29\cdot31\cdot37.
$$

Uvedomme si, že ak by sme v~poslednom kroku vydelili doposiaľ zostrojené číslo
namiesto 2 alebo 3 prvočíslom~5, ubudli by nám tri delitele 25, 50, 75.

Dodajme ešte pre zaujímavosť, že najmenší spoločný násobok všetkých
dvojciferných čísel (vypísaný v~prvej vete tohto riešenia) má
248\,446\,976~deliteľov, ktoré majú po 73~dvojciferných deliteľoch.
Najmenší z~nich je číslo~$n_1$ nájdené postupom z~pôvodného riešenia,
najväčší potom číslo~$n_3$, ktoré dostaneme, keď východiskový najmenší spoločný
násobok vydelíme súčinom $2\cdot11\cdot13$:
$$
n_3=2^5\cdot3^4\cdot5^2\cdot7^2\cdot\underbrace{17\cdot19\cdot23\cdot\dots\cdot97}_{\setbox0=\hbox{\sevenrm súčin všetkých dvojciferných prvočísel od 17 do 97}\wd0=0pt\box0}.
$$

\ineres
Ukážeme ešte jednu konštrukciu čísla, ktoré má práve 73~dvojciferných deliteľov.
Vezmime súčin všetkých dvojciferných čísel, ktoré nie sú deliteľné siedmimi,
\tj.~vynecháme čísla $2\cdot 7, \dots, 14\cdot 7$, ktorých je~13.
Vytvoríme tak súčin $90-13=77$ čísel, ktorý má práve týchto
77~dvojciferných deliteľov, lebo žiadne z~nich nie je deliteľné siedmimi, ako je
naopak každé z~13~vynechaných čísel. Je teda nutné ešte štyri
delitele zrušiť, a~to napríklad tak, že zo súčinu odstránime štyri
činitele, za ktoré vyberieme ľubovoľné štyri prvočísla väčšie ako~50,
napríklad štyri najväčšie 97, 89, 83 a~79, a~budeme potom hotoví.
Dostaneme tak vyhovujúce číslo
$$
n_4=2^6\cdot3^4\cdot5^2\cdot\underbrace{11\cdot13\cdot17\cdot\dots\cdot73}_{\setbox0=\hbox{\sevenrm súčin všetkých dvojciferných prvočísel od 11 do 73}\wd0=0pt\box0}.
$$


\návody
Prirodzené číslo $n$ nie je deliteľné siedmimi. Dokážte, že má nanajvýš 85
deliteľov menších ako~100. [Číslo $n$ určite nie je deliteľné 14~číslami
z~množiny $\{7,14,21,28,\dots,98\}$, teda počet jeho deliteľov menších
ako 100 je nanajvýš $99-14=85$.]

Nájdite prirodzené číslo $n$, ktoré nie je deliteľné siedmimi a~má práve
85~deliteľov menších ako~100. [Za $n$ stačí vziať súčin všetkých čísel menších
ako~100, ktoré nie sú násobkami siedmich.]

Koľko dvojciferných deliteľov má číslo $2020^{2019}$? [Keďže
$2020=2^2\cdot5\cdot 101$, dvojciferné delitele čísla $2020^{2019}$ budú
práve tie, ktoré majú v~prvočíselnom rozklade iba dvojky a~päťky. Sú
to čísla (usporiadané najskôr podľa mocnín čísla~5 a~potom podľa mocnín
čísla~2, ktoré ho delia)
$$
\displaylines{
2^4=16,\quad 2^5=32,\quad 2^6=64,\quad 5\cdot2=10,\quad 5\cdot2^2=20,\cr
\quad 5\cdot2^3=40,\quad 5\cdot2^4=80,\quad 5^2=25,\quad 5^2\cdot2=50,
}
$$
ktorých je práve 9.]

Koľko dvojciferných deliteľov má číslo
$2^2\cdot3^3\cdot4^4\cdot5^5\cdot6^6\cdot7^7$?
[Tých je 36. To zistíme buď priamo ich vypísaním (10, 12, 14, 15,
16, 18, 20, 21, 24, 25, 27, 28, 30, 32, 35, 36, 40, 42, 45, 48, 49,
50, 54, 56, 60, 63, 64, 70, 72, 75, 80, 81, 84, 90, 96, 98), alebo
zistením o~zozname všetkých 21 dvojciferných prvočísel 11, 13,~\dots, 97,
že majú spolu 54 dvojciferných násobkov, a to jednotlivo postupne
9, 7, 5 (dvakrát), 4, 3 (dvakrát), 2 (štyrikrát) a~1 (desaťkrát)~--
hľadaný počet je teda $90-54=36$.
]

Koľko dvojciferných deliteľov má číslo
$50!=50\cdot49\cdot48\cdot\dots\cdot2\cdot1$? [Na rozdiel od
predchádzajúcej úlohy je jednoduchšie spočítať čísla, ktoré deliteľmi nie sú.
Sú to všetky prvočísla väčšie ako~50, tých je 10. Počet všetkých
dvojciferných deliteľov tak je 80. Treba si rozmyslieť, že prvočísla
menšie ako~50 sú obsiahnuté v~čísle $50!$ v~dostatočnej mocnine.]

\D
Koľko dvojciferných deliteľov má číslo $20!$? [Stačí vylúčiť prvočísla
väčšie ako 20 a~ich násobky, tých je 28. Počet všetkých dvojciferných
deliteľov tak je 62. Treba si rozmyslieť, že prvočísla menšie ako 20
sú obsiahnuté v~čísle $20!$ v~dostatočnej mocnine.]

Nájdite všetky prirodzené čísla $n$, pre ktoré má $n!$ viac dvojciferných
deliteľov ako $(n-1)!$. [Určite sú to všetky prvočísla do sto, ktoré
sú väčšie ako~3, a~potom
čísla, keď $n!$ obsahuje vyššiu mocninu nejakého prvočísla ako $(n-1)!$
takú, že táto mocnina je menšia ako sto, \tj. pre $p=7$ je to
$n=14$, pre $p=5$ $n=10$, pre $p=3$ $n=6,9$ a~pre $p=2$ $n=4, 6, 8$, \tj.
sú to všetky prvočísla od 5 do 97 a~navyše $4, 6, 8, 9, 10, 14$.]

Existuje prirodzené číslo $n$, že $n!$ je deliteľný práve polovicou zo
všetkých dvojciferných čísel? [Treba brať prvočísla menšie ako sto od
najväčšieho a~pozerať sa, koľko jeho násobkov je menších ako 100. Ukáže
sa, že $12!$ má 43 dvojciferných deliteľov
a~$13!$ má $50$ dvojciferných deliteľov, \tj. 45 dvojciferných deliteľov
nemá žiadne $n!$.]

Nájdite najmenšie prirodzené číslo~$k$ také, že každá $k$-prvková
množina trojciferných po dvoch nesúdeliteľných čísel obsahuje aspoň
jedno prvočíslo. [\pdfklink{56-B-I-3}{https://skmo.sk/dokument.php?id=226}]

Z~množiny $\{1,2,3,\dots,99\}$ vyberte čo najväčší počet
čísel tak, aby súčet žiadnych dvoch vybraných čísel nebol násobkom
jedenástich. Vysvetlite, prečo
zvolený výber má požadovanú vlastnosť a~prečo žiadny
výber väčšieho počtu čísel nevyhovuje. [\pdfklink{58-C-I-5}{https://skmo.sk/dokument.php?id=30}]

Nájdite najmenšie prirodzené číslo~$k$, pre ktoré platí:
Ak vyberieme ľubovoľných $k$~rôznych čísel z~množiny
$\{1, 4, 7, 10, 13,\dots, 1\,999\}$, potom medzi vybranými existujú
dve rôzne čísla, ktorých súčet sa rovná $2\,000$. [\pdfklink{49-C-S-1}{http://www.matematickaolympiada.cz/media/440629/C49s.pdf}]

Určte najmenšie prirodzené číslo~$k$, pre ktoré platí: Ak vyberieme
ľubovoľných $k$~rôznych čísel z~množiny $\{1,2,3,\dots,2\,000\}$,
tak medzi vybranými číslami existujú dve, ktorých súčet alebo rozdiel
je~$667$. [\pdfklink{49-A-S-3}{http://www.matematickaolympiada.cz/media/440623/A49s.pdf}]

Nájdite všetky prirodzené čísla, ktoré majú rovnaký počet párnych
a~nepárnych deliteľov. [Sú to čísla tvaru $2l$, pričom $l$ je nepárne číslo.
Každé hľadané číslo musí byť párne~-- vtedy však predpis $d\mapsto 2d$ určuje injektívne zobrazenie
množiny všetkých jeho nepárnych deliteľov do množiny všetkých jeho párnych
deliteľov, teda toto zobrazenie musí byť podľa zadania aj surjektívne,
a~preto je hľadané číslo tvaru $2l$, pričom $l$ je jeho najväčší nepárny
deliteľ.]

Súčin všetkých kladných deliteľov prirodzeného čísla $n$ je $20^{15}$. Určte $n$. [\pdfklink{64-B-II-1}{https://skmo.sk/dokument.php?id=1371}]
\endnávod
}

{%%%%%   B-I-3
a)
Keďže štvoruholník $ABCD$ je tetivový, je súčet jeho vnútorných uhlov
pri vrcholoch $A$ a~$C$ rovný~$180^\circ$, teda uhly $BCD$ a~$BAA'$ sú zhodné
(druhý z~nich je vonkajší k~vnútornému uhlu štvoruholníka pri vrchole $A$, \obr).
Trojuholník $ABA'$ je z~rovnosti $|AB|=|A'B|$ rovnoramenný, jeho vnútorné
uhly pri vrcholoch $A$ a~$A'$ sú preto zhodné. Podobne sú zhodné vnútorné
uhly pri vrcholoch $C$ a~$C'$ v~trojuholníku $BCC'$. Teda aj uhly $DC'B$ a~$DA'B$
sú zhodné.
Keďže body $A'$ a~$C'$ ležia v~jednej polrovine $BDA$,
vyplýva z~toho podľa vety o~obvodovom uhle, že body $A'$, $B$, $D$ a~$C'$
ležia na jednej kružnici~$k$, čo sme mali dokázať.
\insp{b69.1}%

b)
Keďže $AC$ je priemerom kružnice opísanej štvoruholníku $ABCD$, je podľa
Tálesovej vety uhol $ADC$, teda aj uhol $A'DC$ (a~k~nemu vedľajší uhol
$A'DC'$) pravý. Úsečka~$A'C'$ je preto z~Tálesovej vety priemerom kružnice~$k$, teda aj uhol $A'BC'$ je pravý. Body $O$ a~$O_A$ ležia na osi úsečky~$BA'$, podobne aj body $O$ a~$O_C$ ležia na osi úsečky~$BC'$. Avšak, ako
sme už dokázali, úsečky $BA'$ a~$BC'$ sú na seba kolmé, takže aj ich osi
$OO_A$ a~$OO_C$ sú na seba kolmé, a~teda platí $OO_A\perp OO_C$, čo sme mali
dokázať.

\poznamka
Dodajme pre zaujímavosť, že obe kružnice na \obrr1{} sú
zhodné, lebo sú opísané trojuholníkom $ABC$ a~$A'BC'$, a~tie sú zhodné
podľa vety~$sus$ (ako vyplýva z~dvoch rovností zo zadania a~z~nášho
zistenia v~časti~b) riešenia, že pravý je nielen uhol $ABC$, ale aj uhol
$A'BC'$).


\návody
Pripomeňte si Tálesovu vetu a~všeobecnejší poznatok o~obvodových
a~stredových uhloch v~danej kružnici.

Štyri rôzne body $A$, $B$, $C$, $D$ ležia na jednej kružnici. Dokážte, že
osi úsečiek $AB$, $AC$, $AD$, $BC$, $BD$, $CD$ prechádzajú tým istým bodom. [Je
to stred kružnice prechádzajúcej bodmi $A$, $B$, $C$, $D$.]

Nech $M$ je vnútorný bod základne~$BC$ rovnoramenného trojuholníka $ABC$.
Na jeho ramene~$AB$ leží bod~$D$ tak, že $|MB|=|MD|$. Dokážte, že body
$A$, $C$, $M$, $D$ ležia na jednej kružnici. [Z~rovnoramenných trojuholníkov $ABC$
a~$MBD$ vyplýva postupne zhodnosť uhlov $ACM$, $ACB$, $CBA$, $MBD$
a~$MDB$, posledný z~nich je však vedľajší uhol k~uhlu $ADM$, takže súčet
uhlov pri protiľahlých vrcholoch $C$ a~$D$ štvoruholníka $ADMC$ je rovný
180\st.]

\D
Nech $ABCD$ je konvexný štvoruholník, v~ktorom $AD\perp BD$. Označme $M$
priesečník jeho uhlopriečok a~zostrojme kolmý priemet~$P$ bodu~$M$ na
priamku~$AB$ a~kolmý priemet~$Q$ bodu~$B$ na priamku~$AC$.
Dokážte, že bod~$M$ je stredom kružnice vpísanej trojuholníku~$PQD$. [\pdfklink{68-B-I-5}{https://skmo.sk/dokument.php?id=3042}]

Daná je kružnica~$k$ a~jej priemer~$AB$. Vnútri úsečky~$AB$
zvolíme ľubovoľný bod~$C$ a~potom na kružnici~$k$ vyberieme bod~$D$
tak, aby platilo $|BC|=|BD|$. Os uhla $ABD$ pretína
kružnicu~$k$ v~bode~$E$ (rôznom od bodu~$B$).
Dokážte, že trojuholníky $AEC$ a~$CBD$ sú podobné.
[\pdfklink{68-B-S-3}{https://skmo.sk/dokument.php?id=3046}]

%Je dána kružnice $k$ se středem $S$ a~tětivou $AB$, která není jejím průměrem. Na polopřímce opačné k~polopřímce $BA$ je vybrán libovolný bod $K$ různý od $B$. Dokažte, že kružnice opsaná trojúhelníku $AKS$ protne kružnici $k$ v~takovém bodě $C$, který je souměrně sdružený s~bodem $B$ podle přímky $SK$.
\MOarchiv68-B-II-3 [\pdfklink{68-B-II-3}{https://skmo.sk/dokument.php?id=3125}]


%Je dán ostroúhlý trojúhelník $ABC$. Označme $D$ patu výšky z~vrcholu $A$ a~$D_1$, $D_2$ obrazy bodu $D$ v~osových souměrnostech po řadě podle přímek $AB$, $AC$. Dále označme $E_1$ a~$E_2$ body na přímce $BC$ takové, že $D_1E_1\parallel AB$ a~$D_2E_2\parallel AC$. Dokažte, že body $D_1$, $D_2$, $E_1$, $E_2$ leží na téže kružnici, jejíž střed leží na kružnici opsané trojúhelníku $ABC$.
\MOarchiv68-A-I-2
[\pdfklink{68-A-I-2}{https://skmo.sk/dokument.php?id=3037}]

%Nechť $V$ je průsečík výšek ostroúhlého trojúhelníku $ABC$. Přímka $CV$ je společnou tečnou kružnic $k$ a~$l$, které se vně dotýkají v~bodě $V$ a~přitom každá z~nich prochází jedním z~vrcholů $A$ a~$B$. Jejich průsečíky s~vnitřky stran $AC$ a~$BC$ označme $P$ a~$Q$. % ($P\ne A$, $Q\ne B$). Dokažte, že polopřímka $VC$ je osou úhlu $PVQ$ a~že body $A$, $B$, $P$,~$Q$ leží na jedné kružnici.
\MOarchiv62-B-I-3
[\pdfklink{62-B-I-3}{https://skmo.sk/dokument.php?id=676}]

%V~rovině je dán pravoúhlý lichoběžník $ABCD$ s~delší stranou $AB$ a~pravým úhlem při vrcholu $A$. Označme $k_1$ kružnici sestrojenou nad stranou $AD$ jako průměrem a~$k_2$ kružnici procházející vrcholy $B$, $C$ a~dotýkající se přímky $AB$. Mají-li kružnice $k_1, k_2$ vnější dotyk v~bodě $P$, je přímka $BC$ tečnou kružnice opsané trojúhelníku $CDP$. Dokažte.
\MOarchiv52-B-II-4
[\pdfklink{52-B-II-4}{https://skmo.sk/dokument.php?id=268}]

%V~rovině je dán rovnoběžník $ABCD$, jehož úhlopříčka $BD$ je kolmá ke straně $AD$. Označme $M$ ($M\ne A$) průsečík přímky $AC$ s~kružnicí o~průměru $AD$. Dokažte, že osa úsečky $BM$ prochází středem strany $CD$.
\MOarchiv57-B-II-3
[\pdfklink{57-B-II-3}{https://skmo.sk/dokument.php?id=218}]

%Nechť $K$ je libovolný vnitřní bod strany $AB$ daného trojúhelníku $ABC$. Přímka $CK$ protíná kružnici opsanou trojúhelníku $ABC$ v~bodě $L$ ($L\ne C$). Označme $k_1$ kružnici opsanou trojúhelníku $AKL$ a~$k_2$ kružnici opsanou trojúhelníku $BKL$.
%\item{a)} Dokažte, že přímka $AC$ je tečna kružnice $k_1$, právě když přímka $BC$ je tečna kružnice $k_2$.
%\item{b)} Předpokládejme, že přímka $AC$ je sečna kružnice $k_1$. Nechť $P$ ($P\ne A$) je průsečík přímky $AC$ s~kružnicí $k_1$ a~$Q$ ($Q \ne B$) průsečík přímky $BC$ s~kružnicí $k_2$. Dokažte, že bod $K$ leží na úsečce $PQ$.
%\MOarchiv53-A-II-3
Označme $K$ ľubovoľný vnútorný bod strany $AB$ daného
  trojuholníka $ABC$. Priamka $CK$ pretína kružnicu opísanú trojuholníku
  $ABC$ v~bode~$L$ ($L\ne C$). Označme $k_1$ kružnicu opísanú
  trojuholníku $AK\!L$ a~$k_2$ kružnicu opísanú trojuholníku $BK\!L$.
\item{a)} Dokážte, že priamka~$AC$ je dotyčnicou ku kružnici~$k_1$ práve
vtedy, keď priamka~$BC$ je dotyčnicou ku kružnici~$k_2$.
\item{b)} Predpokladajme, že priamka~$AC$ je sečnicou kružnice~$k_1$.
Nech $P$ ($P\ne A$) je priesečník priamky~$AC$ s~kružnicou~$k_1$
a~$Q$ ($Q\ne B$) je priesečník priamky~$BC$ s~kružnicou~$k_2$. Dokážte, že
bod~$K$ leží na úsečke~$PQ$.
[\pdfklink{53-A-II-3}{https://skmo.sk/dokument.php?id=254}]

%Jsou dány kružnice $k$, $l$, které se protínají v~bodech $A$, $B$. Označme $K$, $L$ po řadě dotykové body jejich společné tečny zvolené tak, že bod $B$ je vnitřním bodem trojúhelníku $AKL$. Na kružnicích $k$ a~$l$ zvolme po řadě body $N$ a~$M$ tak, aby bod $A$ byl vnitřním bodem úsečky $MN$. Dokažte, že čtyřúhelník $KLMN$ je tětivový, právě když přímka~$MN$ je tečnou kružnice opsané trojúhelníku $AKL$.
\MOarchiv60-A-I-3
[\pdfklink{60-A-I-3}{https://skmo.sk/dokument.php?id=355}]
\endnávod}

{%%%%%   B-I-4
Nech $x_0$ je celočíselný koreň prvej rovnice. Z~jej prepisu na
tvar
$$
p(x_0^2+1)=(p+q)x_0 \tag1
$$
vidíme, že $x_0>0$, pretože
zvyšné činitele $p$, $x_0^2+1$ a~$p+q$ v~(1) sú podľa
predpokladu prirodzené čísla. Najväčší spoločný deliteľ čísel $x_0$ a~$x_0^2+1$
delí aj číslo $(x_0^2+1)-{x_0\cdot x_0}=1$, takže čísla
$x_0$ a~$x_0^2+1$ sú nesúdeliteľné, a~z~rovnice~(1) tak vyplýva, že
$x_0$ delí číslo~$p$. Naopak, z~nesúdeliteľnosti čísel $p$ a~$q$ vyplýva
aj nesúdeliteľnosť čísel $p$ a~$p+q$, a~tak z~vyššie uvedenej rovnice vidíme, že
aj číslo $p$ delí~$x_0$. Keďže obe tieto čísla sú prirodzené, vyplýva
z~toho $x_0=p$.

Keď dosadíme $p$ za $x_0$ do~(1) a~potom obe strany vydelíme nenulovou
hodnotou~$p$, dostaneme
$$
p^2+1=p+q,\quad\text{čiže}\quad q=p^2-p+1.
$$
Teraz toto vyjadrenie čísla~$q$ dosadíme do druhej zadanej rovnice
a~postupne upravíme:
$$
\align
0&=px^2+(p^2-p+1)x+p^2-(p^2-p+1)=\\
&=px^2+(p^2-p+1)x+p-1=(px+1)(x+p-1).
\endalign
$$
Vidíme, že upravená, a~teda aj druhá pôvodne zadaná rovnica
má celočíselný koreň $1-p$. Tým je tvrdenie úlohy dokázané.

\poznamka
Kľúčový vzťah $q=p^2-p+1$ môžeme tiež odvodiť tak, že
rovnosť~(1) prepíšeme na tvar
$$
\frac{p}{p+q}=\frac{x_0}{x_0^2+1},
$$
v~ktorom na oboch stranách stoja zlomky v~základnom tvare, takže sa musia
rovnať ako ich čitatele, tak aj menovatele: $p=x_0$
a~$p+q=x_0^2+1=p^2+1$, odkiaľ už máme $q=p^2-p+1$.

\poznamka
Ukážeme na príkladoch, že pre súdeliteľné prirodzené čísla $p$,
$q$ tvrdenie úlohy vo všeobecnosti neplatí. Pre $p=q=2$ má prvá rovnica
$2x^2-4x+2=0$ celočíselný koreň~1, zatiaľ čo druhá rovnica $2x^2+2x+2=0$
nemá žiadny koreň.
Iná situácia nastane pre $p=14$ a~$q=86$: prvá rovnica
$14x^2-100x+14=0$ má koreň 7, avšak druhá rovnica $14x^2+86x+110=0$ má
iba iracionálne korene.



\návody
Dokážte, že ak $a$, $b$ a~$c$ sú kladné reálne čísla, tak
je kladný aj každý koreň kvadratickej rovnice $ax^2-bx+c=0$. [Ľavá strana
rovnice je kladná pre každé $x\le0$.]

Dokážte, že ak má rovnica $ax^2+bx+c=0$ celočíselné
koeficienty $a$, $b$ a~$c$, tak každý jej koreň, ktorý je aj celým
číslom, musí byť deliteľom čísla $c$. [Vyplýva to z~úpravy rovnice na
tvar $c=\m x(ax+b)$.]

Nech prirodzené číslo $a$ je deliteľom prirodzeného čísla $b$ a~súčasne
číslo $b$ je deliteľom~$a$. Potom $a=b$. Dokážte. [Platí $a\le b$ aj~$b\le a$.]

Prirodzené čísla $a$ a~$b$ sú nesúdeliteľné, rovnako ako
prirodzené čísla $c$ a~$d$. Dokážte, že z~rovnosti $ac=bd$ potom vyplýva
$a=d$ a~$b=c$. [Uvedomte si, kedy z~$x\mid yz$ vyplýva $x\mid z$.]

Dokážte, že ak sú čísla $a$, $b$ nesúdeliteľné, platí to isté aj o~číslach
$a$, $a+b$. [Každý spoločný deliteľ čísel $a$, $a+b$ delí aj číslo
$(a+b)-a=b$.]

Nech $a$ je prirodzené číslo, určte všetky možné najväčšie spoločné
delitele čísel $a$ a~$a^2+4$. [Nech $d$ je najväčší spoločný deliteľ
oboch čísel, potom $d$ delí číslo $(a^2+4)-a\cdot a=4$. Najväčší spoločný
deliteľ oboch čísel teda môže byť $1$, $2$ alebo $4$. Prvá možnosť nastane
pre $a$ nepárne, druhá pre $a$ deliteľné dvoma a~nie štyrmi, tretia pre $a$
deliteľné štyrmi.]

Nech $p$ je prirodzené číslo. Nájdite korene rovnice
$px^2+(p^2-p+1)x+p-1=0$. [$x_1=\m1/p$, $x_2=1-p$]

\D
\MOarchiv50-B-S-1
[\pdfklink{50-B-S-1}{http://www.matematickaolympiada.cz/media/440636/B50s.pdf}]

Nájdite všetky dvojice celých čísel $(a,b)$, ktoré sú riešením rovnice
$a^2+7ab+6b^2+5a+4b+3=0$.
%\MOarchiv56-B-I-1
[\pdfklink{56-B-I-1}{https://skmo.sk/dokument.php?id=226}]

Nájdite všetky riešenia rovnice $xyz=3(x+y+z)$ v~obore celých
kladných čísel. Riešenia, ktoré sa líšia iba poradím, nepovažujeme
za rôzne. [36--B--II--3b]

Koľko existuje celých kladných čísel $x\leq 2\,002\,000$
takých, že číslo $2\,002\,000$ delí číslo $x^3-x$? [41--B--I--6]

\MOarchiv45-B-I-4
[45--B--I--4]
\endnávod
}

{%%%%%   B-I-5
Označme $Q$ priesečník spoločnej vonkajšej dotyčnice~$T_aT_b$ so spoločnou vnútornou
dotyčnicou kružníc $a$ a~$b$ v~bode~$T$. Dotyčnica ku kružnici zviera s~priemerom
tejto kružnice prechádzajúcim bodom dotyku pravý uhol, a~tak uhly $AT_aQ$
a~$ATQ$ sú pravé. Body $T_a$ a~$T$ preto ležia podľa Tálesovej vety na
kružnici s~priemerom~$AQ$, táto kružnica je potom aj kružnicou~$k_a$
opísanou trojuholníku~$T_aAT$ (\obr). Podobne ukážeme, že $QB$ je priemerom
kružnice~$k_b$.
\insp{b69.2}%

Priamky $QT$ a~$QT_a$ sú dotyčnice kružnice~$a$, priamka~$QA$ prechádzajúca
jej stredom je tak osou uhla $TQT_a$. Z~podobného dôvodu je priamka~$QB$ osou uhla $TQT_b$. Uhol $T_aQT_b$ je priamy, priamky $AQ$ a~$BQ$ tak
zvierajú pravý uhol a~trojuholník $ABQ$ je pravouhlý. Veľkosť jeho
odvesny $AQ$ je podľa Euklidovej vety $|AQ|=\sqrt{r_a(r_a+r_b)}$
a~pre veľkosť druhej odvesny platí $|BQ|=\sqrt{r_b(r_a+r_b)}$.

Hľadaný pomer polomerov kružníc opísaných trojuholníkom $T_aAT$, $T_bBT$
je rovný pomeru veľkostí ich priemerov $AQ$ a~$BQ$, teda
$$
\frac{|AQ|}{|BQ|}=\frac{\sqrt{r_a(r_a+r_b)}}{\sqrt{r_b(r_a+r_b)}}
=\sqrt{\frac{r_a}{r_b}}.
$$

\ineres
Rovnako ako v~predchádzajúcom riešení uvažujme priesečník~$Q$ spoločnej
vnútornej a~vonkajšej dotyčnice kružníc $a$, $b$. Zo symetrie dotyčníc ku kružnici~$a$ vyplýva $|TQ|=|T_aQ|$ a~podobne zo symetrie dotyčníc ku kružnici~$b$ vyplýva $|TQ|=|T_bQ|$. Bod~$Q$ je preto stredom kružnice opísanej trojuholníku
$TT_aT_b$ a~jeho strana~$T_aT_b$ je zároveň jej priemerom.
Podľa Tálesovej vety je teda uhol $T_aTT_b$ pravý. Súčet
veľkostí uhlov $ATT_a$ a~$BTT_b$ je tak~$90^\circ$.
Z~kolmosti $QT\perp AT$ potom dostávame, že uhol $T_aTQ$ má rovnakú veľkosť
ako uhol $T_bTB$, a~preto sú rovnoramenné trojuholníky $T_aQT$ a~$T_bBT$ podobné.
Bod~$Q$ ale leží na kružnici~$k_a$ opísanej trojuholníku $ATT_a$,
lebo štvoruholník $ATQT_a$ je podľa Tálesovej vety tetivový.
V~podobnosti trojuholníkov $T_aQT\sim T_bBT$ tak kružnici~$k_a$ zodpovedá kružnica~$k_b$,
ktorá je opísaná trojuholníku $T_bBT$.
Z~toho vyplýva, že pomer
polomerov kružníc $k_a$ a~$k_b$ je rovný pomeru dĺžok ich tetív $QT$ a~$BT$.
Analogickou úvahou
o~podobnosti trojuholníkov
$T_aAT\sim T_bQT$ s~prihliadnutím na to, že bod~$Q$ leží na kružnici~$b$,
dostaneme, že ten istý pomer je
rovnaký ako pomer dĺžok iných ich tetív $AT$ a~$QT$. Preto hodnotu~$p$ hľadaného pomeru môžeme získať ako odmocninu zo súčinu jej rovných
hodnôt:
$$
p=\sqrt{\frac{|QT|}{|BT|}\cdot\frac{|AT|}{|QT|}}=\sqrt{\frac{|AT|}{|BT|}}=
\sqrt{\frac{r_a}{r_b}}.
$$




\návody
Nech $V_a$ a~$V_b$ sú päty výšok trojuholníka $ABC$ postupne z~vrcholov
$A$ a~$B$ a~$V$ je priesečník jeho výšok. a) Dokážte, že body $A$, $B$,
$V_a$, $V_b$ ležia na jednej kružnici. b)~Dokážte, že body $V$, $V_a$, $C$,
$V_b$ ležia na jednej kružnici. [a) Podľa Tálesovej vety je to kružnica
s~priemerom $AB$. b) Podľa Tálesovej vety je to kružnica s~priemerom
$CV$.]

Pripomeňte si znenie Euklidových viet o~výške a~odvesne pravouhlého
trojuholníka.

Kružnica $k_b$ leží zvonka kružnice $k_a$ a~je s~ňou disjunktná. Nech
ich vonkajšie spoločné dotyčnice $T_aT_b$ a~$T_AT_B$ ($T_a, T_A\in k_a$,
$T_b, T_B\in k_b$, $T_a\ne T_A$ a~$T_b\ne T_B$) pretínajú ich
spoločnú vnútornú dotyčnicu $V_aV_b$ ($V_a\in k_a$, $V_b\in k_b$) postupne
v~bodoch $A$ a~$B$. Dokážte, že $|T_aT_b|= |T_AT_B|=|AB|$. [Prvá
rovnosť vyplýva zo súmernosti podľa priamky prechádzajúcej stredmi oboch
kružníc. Ďalej zo súmernosti platí $|T_aA|=|V_aA|$, $|T_bA|=|V_bA|$,
$|T_AB|=|V_aB|$, $|T_BB|=|V_bB|$. Sčítaním týchto rovníc dostaneme
$|T_aA|+|T_bA|+|T_AB|+|T_BB|=|V_aA|+|V_bA|+|V_aB|+|V_bB|$. Na ľavej
strane rovnice je súčet (rovnakých) dĺžok $|T_aT_b|$ a~$|T_AT_B|$, na
pravej dvojnásobok $|AB|$, odtiaľ tak vyplýva druhá dokazovaná rovnosť.]

\D
%Pravoúhlému trojúhelníku $ABC$ s~přeponou $AB$ je opsána kružnice. Paty
%kolmic z~bodů $A$, $B$ na tečnu k~této kružnici v~bodě $C$ označme $D$,
%$E$. Vyjádřete délku úsečky $DE$ pomocí délek odvěsen trojúhelníku
%$ABC$.
\MOarchiv58-C-I-2
[\pdfklink{58-C-I-2}{https://skmo.sk/dokument.php?id=30}]

%Pravoúhlému trojúhelníku $ABC$ s~přeponou $AB$ a~obsahem $S$ je opsána
%kružnice. Tečna k~této kružnici v~bodě $C$ protíná tečny vedené body $A$
%a~$B$ v~bodech $D$ a~$E$.Vyjádřete délku úsečky $DE$ pomocí délky $c$
%přepony a~obsahu $S$.
\MOarchiv58-C-II-4
[\pdfklink{58-C-II-4}{https://skmo.sk/dokument.php?id=37}]

%Nechť $k$ je polokružnice sestrojená nad průměrem $AB$, která leží ve
%čtverci $ABCD$. Uvažujme její tečnu $t_1$ z~bodu $C$ (různou od $BC$)
%a~označme $P$ její průsečík se stranou~$AD$. Nechť $t_2$ je společná
%vnější tečna polokružnice $k$ a~kružnice vepsané trojúhelníku $CDP$
%(různá od $AD$). Dokažte, že přímky $t_1$ a~$t_2$ jsou navzájem kolmé.
\MOarchiv51-B-I-3
[\pdfklink{51-B-I-3}{https://skmo.sk/dokument.php?id=276}]

%Kružnice $k(S;r)$ a~$l(O;R)$ se vně dotýkají v~bodě~$T$. Jejich společná
%tečna v~bodě~$T$ protíná jejich vnější společnou tečnu v~bodě $M$.
%Dokažte, že trojúhelník $SOM$ je pravoúhlý, a~vyjádřete jeho obsah
%pomocí poloměrů $r$ a~$R$ daných kružnic.
\MOarchiv50-C-II-2
[\pdfklink{50-C-II-2}{http://www.matematickaolympiada.cz/media/440638/C50ii.pdf}]

%V~rovině jsou dány kružnice $k$ a~$l$, které se protínají v~bodech $E$
%a~$F$. Tečna ke kružnici $l$ sestrojená v~bodě $E$ protíná kružnici $k$
%v~bodě $H$ ($H\ne E$). Na oblouku $EH$ kružnice $k$, který neobsahuje
%bod $F$, zvolme bod $C$ ($E\ne C \ne H$) a~průsečík přímky $CE$
%s~kružnicí~$l$ označme $D$ ($D\ne E$). Dokažte, že trojúhelníky $DEF$
%a~$CHF$ jsou podobné.
\MOarchiv66-B-II-3
[\pdfklink{66-B-II-3}{https://skmo.sk/dokument.php?id=2387}]

%Kružnice $k$ se středem $S$ je opsána pravidelnému šestiúhelníku
%$ABCDEF$. Tečna v~bodě~$A$ ke kružnici $k$ protne přímku $SB$ v~bodě~$K$
%a~tečna v~bodě $B$ protne přímku~$SC$ v~bodě $L$. Dokažte, že
%čtyřúhelníku $KLCB$ lze opsat kružnici, která je shodná s~kružnicí~$k$.
\MOarchiv56-C-S-2
[\pdfklink{56-C-S-2}{https://skmo.sk/dokument.php?id=230}]
\endnávod
}

{%%%%%   B-I-6
Každé biele políčko leží na niektorej zo siedmich diagonál rovnobežných s~bielou
hlavnou diagonálou a~vyznačených na \obr{} šípkami. Podobne leží čierne políčko na
niektorej zo siedmich diagonál rovnobežných s~čiernou hlavnou diagonálou. Na
každej z~týchto diagonál bez veží môže stáť nanajvýš jeden strelec.
Na prázdnu šachovnicu možno teda umiestniť nanajvýš 14~strelcov.

Vo~všeobecnej situácii, keď je na šachovnici rozmiestnených povedzme $k$~veží,
pre každú diagonálu~$D$ zo 14 nami uvažovaných označíme $k_D$ počet
veží, ktoré na nej stoja. Celé čísla~$k_D$ sú teda nezáporné a~ich
súčet je rovný~$k$. Prítomných $k_D$~veží vymedzuje na diagonále~$D$
zrejme nanajvýš ${k_D+1}$~úsekov políčok bez veže a~na každom z~nich môže byť
nanajvýš jeden strelec, teda na celej diagonále~$D$ je dokopy nanajvýš
${k_D+1}$~strelcov. Počet neohrozujúcich sa strelcov na celej šachovnici tak
neprevyšuje súčet 14~čísel $k_D+1$, ktorý je rovný $k+14$, teda~18
v~prípade $k=4$.
\inspinsp{b69.3}{b69.4}%

Obr.\,\obrnum{} ukazuje príklad šachovnice so štyrmi vežami, na ktorú sme
umiestnili 18~strelcov, aby sa navzájom neohrozovali.

\odpoved
Hľadaný najväčší
počet strelcov, ktorých možno umiestniť spolu so štyrmi vežami na šachovnicu
tak, aby sa strelci navzájom neohrozovali, je práve~18.

\mppic b69.5 \par



\návody
Aký najväčší počet strelcov možno umiestniť na biele políčka šachovnice
$8\times8$ tak, aby sa navzájom neohrozovali?
[Sedem. Na šachovnici uvažujme diagonály rovnobežné s~bielou hlavnou
diagonálou. Vrátane nej je ich práve~7. Na každú môžeme umiestniť nanajvýš
jedného strelca. Keďže každé biele políčko leží na niektorej z~týchto
diagonál, môžeme na šachovnici umiestniť nanajvýš 7 strelcov.
Vyhovujúce umiestnenie 7~strelcov môžeme vybrať napríklad tak, že budú
v~počtoch 4 a~3 v~krajných stĺpcoch šachovnice.]

Určte najväčší počet strelcov, ktorých môžeme umiestniť na bielu hlavnú
diagonálu šachovnice $8\times8$ spolu s~dvoma vežami tak, aby sa
navzájom neohrozovali v~zmysle zadania súťažnej úlohy. [Dve veže rozdelia
diagonálu na nanajvýš tri časti. Na každú môžeme umiestniť nanajvýš jedného
strelca. Preto je hľadaný počet strelcov nanajvýš tri. A~týchto troch strelcov
už vieme umiestniť na šachovnicu spolu s~dvoma vežami požadovaným
spôsobom, napríklad strelcov umiestnime na prvé, tretie a~piate políčko a~veže
na druhé a~štvrté políčko diagonály.]

Určte najväčší počet kráľov, ktorých môžeme umiestniť na šachovnicu
$8\times8$ tak, aby sa navzájom neohrozovali. [Šachovnicu rozdelíme na
16 štvorcov $2\times2$, na každý z~nich môžeme umiestniť nanajvýš jedného
kráľa, preto je kráľov nanajvýš 16. Šestnásť kráľov už na šachovnicu
umiestniť vieme, napríklad na tie políčka, ktorých obe súradnice sú nepárne.]

Určte najväčší počet kráľov, ktorých môžeme umiestniť na šachovnicu
$9\times9$ tak, aby sa navzájom neohrozovali. [Keď pridáme jeden riadok
a~jeden stĺpec šachovnice, dostaneme šachovnicu $10\times10$, na ktorú
môžeme podľa podobnej úvahy ako v~riešení~N3 umiestniť
nanajvýš 25~kráľov. A~tých vieme
umiestniť, napríklad na políčka, ktorých obe súradnice sú nepárne.]

\D
\inspicture r(2)
Nájdite najväčšie prirodzené číslo $k$, pre ktoré možno na šachovnicu $8\times8$
rozmiestniť $k$ veží a~$k+14$ navzájom sa neohrozujúcich strelcov.
[$k=18$. Celú šachovnicu možno rozložiť na sedem bielych diagonál dĺžok 2,
4, 6, 8, 6, 4,~2 a~sedem čiernych diagonál takých istých dĺžok. Ak sa na
ľubovoľnej z~týchto 14 diagonál $D$ nachádza $k_D$ veží, je na nej nanajvýš
$k_D+1$ navzájom sa neohrozujúcich strelcov. Podľa zadania je však
celkový počet strelcov o~14~väčší ako celkový počet veží, preto na každej
uvažovanej diagonále $D$ musí byť práve $k_D+1$~strelcov. To je pre
diagonály $D$ dĺžok 2, 4, 6, 8 možné, iba ak zodpovedajúce~$k_D$
neprevyšuje postupne hodnoty 0, 1, 2, 3. Preto počet $k$ veží na celej
šachovnici neprevyšuje hodnotu $2(0+1+2+3+2+1+0)=18$. Hodnota $k=18$ je
pritom možná, ako ukazuje príklad rozmiestnenia 9 veží a~$9+7=16$ strelcov
na bielych políčkach podľa obrázka; rozmiestnenie rovnakých počtov veží
a~strelcov na čiernych políčkach urobíme analogicky.]

\everypar={\llap{\ifD D\else N\fi \the\nom.\enspace}\advance\nom by1 }
%Každý vrchol pravidelného devatenáctiúhelníku je obarven jednou ze šesti
%barev. Vysvětlete, proč stejnou barvu mají všechny vrcholy některého
%tupoúhlého trojúhelníku.
\MOarchiv62-C-S-3
[\pdfklink{62-C-S-3}{https://skmo.sk/dokument.php?id=687}]

%Na desce $7\times7$ hrajeme hru lodě. Nachází se na ní jedna loď $2\times3$.
%Můžeme se zeptat na libovolné políčko desky, a~pokud loď zasáhneme, hra
%končí. Pokud ne, ptáme se znovu. Určete nejmenší počet otázek, které
%potřebujeme, abychom jistě loď zasáhli.
\MOarchiv58-B-I-4
[\pdfklink{58-B-I-4}{https://skmo.sk/dokument.php?id=29}]

Na pláne $5\times 5$ hráme hru lode. Zo štyroch políčok plánu je vytvorená jedna loď
tvaru L-tetromina. Môžeme sa spýtať na ľubovoľné políčko plánu, a~ak loď zasiahneme, hra končí.
\item{a)} Navrhnite osem políčok, na ktoré sa stačí spýtať, aby sme mali
        istotu zásahu lode.
\item{b)} Zdôvodnite, prečo žiadnych sedem otázok takú istotu nedáva.
        [\pdfklink{58-B-II-2}{https://skmo.sk/dokument.php?id=35}]

%Na některé políčko šachovnice $6\times 6$ postavíme figurku kralevice. Ta
%může v~jednom tahu poskočit buďto ve svislém, nebo ve vodorovném směru.
%Délka tohoto skoku je střídavě jedno či dvě políčka, přičemž skokem na
%sousední pole figurka začíná. Rozhodněte, zda lze zvolit výchozí pozici
%figurky tak, aby po vhodné posloupnosti 35~skoků navštívila každé pole
%šachovnice právě jednou.
\MOarchiv65-A-III-6
[\pdfklink{65-A-III-6}{https://skmo.sk/dokument.php?id=1872}]

Políčka tabuľky $n\times n$, kde $n\ge3$, sú
striedavo čierne a~biele ako na obyčajnej šachovnici,
pričom políčko v~ľavom hornom rohu je čierne.
Biele políčka budeme farbiť načierno nasledujúcim postupom.
V~jednom kroku vyberieme ľubovoľný obdĺžnik $2\times 3$ alebo $3\times
2$, v~ktorom sú ešte tri biele políčka, a~tieto tri políčka
začiernime. Pre ktoré~$n$ môžeme po určitom počte krokov
začierniť celú tabuľku?
[\pdfklink{ČR-57-A-II-3}{http://www.matematickaolympiada.cz/media/440726/A57ii.pdf}]

Zistite najmenšie prirodzené číslo~$k$, pre ktoré
platia jednotlivé tvrdenia a), b) a~c):
Ak obsadíme figúrkami ľubovoľných $k$~polí šachovnice $8\times8$,
budú obsadené niektoré
a) tri susedné polia niektorého riadku,
b) tri susedné polia niektorého šikmého radu,
c) štyri susedné polia niektorého riadku alebo stĺpca.
[\pdfklink{49-C-I-3}{http://www.matematickaolympiada.cz/media/440627/C49i.pdf}]

\endnávod}

{%%%%%   C-I-1
Podľa zadania je číslo $\overline {ab}$ o~jednotku väčšie
ako číslo~$\overline {cd}$, čiže
$$
\overline {ab} = \overline {cd}+1.
$$
Podľa hodnoty cifry~$d$ tak máme dve možnosti.

Ak $d<9$, pričítaním jednotky
"neprejdeme cez desiatku", takže pre jednotlivé cifry platí $a~= c$
a~zároveň $b = d+1$. V~tom prípade je ciferný súčet hľadaného čísla rovný
$$
a+b+c+d = c+(d+1)+c+d = 2 (c+d)+1,
$$
čo je nepárne číslo, a~teda sa nemôže rovnať číslu~12.

Pre $d = 9$ naopak pričítaním jednotky "prejdeme cez desiatku",
takže musí byť $b = 0$ a~$a~= c+1\le9$. Pre ciferný súčet hľadaného
čísla potom platí $a+b+c+d = ({c+1})+0+c+9=2c+10$, čo je rovné~12 práve vtedy, keď $c=1$.
Dostávame tak jediné riešenie $c = 1$, ku~ktorému dopočítame $a~= c+1 = 2$.

Jediné vyhovujúce štvorciferné číslo je číslo~2\,019.

\návody
Nájdite všetky štvorciferné čísla $\overline {abcd}$ s~ciferným súčtom
13 také, že $\overline {ab} = \overline {cd}$. [Také číslo $\overline{abab}$
neexistuje, lebo jeho ciferný súčet je párny.]

Nájdite všetky štvorciferné čísla $\overline {abcd}$ s~ciferným súčtom
12 také, že $\overline {ab}-\overline {cd} = 10$. [Také číslo $\overline{ab(a-1)b}$
neexistuje, lebo jeho ciferný súčet je nepárny.]

Nájdite všetky štvorciferné čísla $\overline {abcd}$ s~ciferným súčtom 8
také, že $\overline {ab} = \overline {cd}$. [1313, 2222, 3131, 4040]

\D
Nájdite najväčšie a~najmenšie štvorciferné čísla $\overline {abcd}$
s~ciferným súčtom 13 také, že $\overline {ab}-\overline {cd} = 10$. [7060,
1606~-- ak pripustíme zápis $\overline{06}$, inak 2515.]

Nájdite všetky osemciferné čísla $\overline {abcdefgh}$ s~ciferným
súčtom 16 také, že $\overline {efgh}-\overline {abcd} = 1$. [Vysvetlite najskôr,
prečo pri sčítaní $\overline{abcd}+1$ musí nastať {\it práve jeden\/} prenos jednotky
do vyššieho rádu, takže každé hľadané číslo je tvaru $\overline{abc9\,ab(c+1)0}$.
Vyhovujú čísla 1029\,1030,
1119\,1120, 1209\,1210, {\bf 2019\,2020}, 2109\,2110, 3009\,3010.]
\endnávod}

{%%%%%   C-I-2
Označme $d$ dĺžku strán daného šesťuholníka.
Rovnobežník $CDEP$ je zrejme kosoštvorec, pretože obe úsečky
$CD$ a~$DE$ majú rovnakú dĺžku~$d$. Rovnakú dĺžku~$d$ majú teda
aj úsečky $PC$ a~$PE$. Zo zadania
vieme, že úsečky $AB$ a~$DE$ sú rovnobežné a~zhodné, preto aj úsečky
$PC$ a~$AB$ sú rovnobežné a~zhodné, z~čoho vyplýva, že aj štvoruholník
$ABCP$ je kosoštvorec so stranou dĺžky~$d$ (\obr). Vďaka tomu
dostávame $|PA| = |PC| = |PE| = d$, a~bod $P$ je tak naozaj stredom
kružnice opísanej trojuholníku $ACE$.
\inspinspblizko{c69.1}{c69.2}%

Štvoruholník $BCEF$ je rovnobežník, pretože jeho protiľahlé strany $BC$
a~$EF$ sú rovnobežné a~zhodné. Potom aj jeho protiľahlé strany $CE$
a~$BF$ sú rovnobežné. Aby sme ukázali, že priamka~$DP$ je
kolmá na priamku~$BF$ (\obr), stačí si uvedomiť, že
uhlopriečky kosoštvorca $CDEP$ sú
navzájom kolmé. Naozaj, z~kolmosti ${CE \perp DP}$
a~rovnobežnosti $CE \parallel BF$ vyplýva $DP \perp BF$. To vlastne
znamená, že $DP$ je priamkou výšky z~vrcholu~$D$ trojuholníka~$BDF$.
Podobne~-- využitím navzájom kolmých uhlopriečok kosoštvorca $ABCP$
a~rovnobežných protiľahlých strán $AC$ a~$DF$ rovnobežníka $ACDF$~--
dokážeme, že na priamke~$BP$ leží výška trojuholníka $BDF$ z~vrcholu~$B$. Bod~$P$
je teda priesečníkom výšok trojuholníka $BDF$, ako sme mali tiež dokázať.


\návody
Dokážte, že uhlopriečky kosoštvorca sú na seba kolmé a~že sa navzájom
rozpoľujú. [Uhlopriečky sú zároveň výškami rovnoramenných trojuholníkov
s~vrcholmi vo vrcholoch kosoštvorca.]

Dokážte, že ak sú dve zhodné úsečky neležiace v~jednej priamke rovnobežné,
tvoria ich krajné body rovnobežník. [Dokreslite ďalšie dve úsečky, ktoré
s~danými úsečkami vytvoria konvexný štvoruholník. Jeho
uhlopriečka rozdeľuje štvoruholník na dva trojuholníky, ktoré sú zhodné podľa
vety $sus$.]

\D
Konvexný šesťuholník $ABCDEF$ má všetky strany zhodné a~protiľahlé
dvojice strán rovnobežné. Dokážte, že stred úsečky~$CF$ je totožný s~priesečníkom
priamok $AD$ a~$BE$. [$ACDF$ aj~$ABDE$ sú rovnobežníky.]

Konvexný šesťuholník $ABCDEF$ má všetky strany zhodné a~protiľahlé
dvojice strán rovnobežné. Označme $X$ a~$Y$ priesečníky výšok trojuholníkov
$ACE$ a~$BDF$. Dokážte, že v~prípade $X\ne Y$ stred úsečky~$XY$ rozpoľuje úsečku~$AD$.
[Z~rovnobežníkov $ABDE$ a~$ACDF$ vidíme, že úsečky $AD$, $BE$ a~$CF$ prechádzajú
jedným bodom. Ten určuje stredovú súmernosť, v~ktorej si zodpovedajú ako body $A$, $D$,
tak body $C$, $F$ i~body $E$,~$B$, a~teda aj trojuholníky $ACE$ a~$DFB$. Preto sú
podľa stredu úsečky~$AD$ súmerne združené nielen priesečníky výšok spomenutých trojuholníkov
(ako sme mali ukázať), ale aj ich ťažiská, stredy im opísaných aj stredy im
vpísaných kružníc. Rovnaké závery platia aj pre tri ďalšie dvojice súmerne združených
trojuholníkov $(ABC,DEF)$, $(ABF,DEC)$ a~$(AEF,DBC)$.]
\endnávod}

{%%%%%   C-I-3
Využitím známej rovnosti $m \cdot n = [ m, n ] \cdot (m, n)$, ktorá platí
pre ľubovoľné prirodzené čísla $m$ a~$n$, môžeme zadanú rovnicu
upraviť na tvar
$$
2[ a, b ]+3(a, b) = [ a, b ] \cdot (a, b).
$$
Pre zjednodušenie zápisu označme $u= [ a, b ]$ a~$v=(a, b)$, pričom $u$,
$v$ sú prirodzené čísla, pre ktoré zrejme platí $u\ge v$, lebo najmenší
spoločný násobok čísel $a$ a~$b$ určite nie je menší ako ktorýkoľvek ich
spoločný deliteľ.
Máme tak riešiť rovnicu $2u+3v=uv$, ktorú kvôli tomu upravíme na tvar
$$
uv-2u-3v+6 = 6, \quad \mbox {čiže} \quad (u-3) (v-2) = 6, \tag1
$$
pričom $u$, $v$ sú prirodzené čísla spĺňajúce podmienku $u\ge v$ a~samozrejme
aj podmienku $v\mid u$ (spoločný deliteľ delí spoločný násobok).

Pre $v=1$ nevyjde $u$ prirodzené, musí teda byť $v-2 \ge 0$, a~teda aj~$u-3\ge 0$.

Číslo~6 možno napísať ako súčin dvoch nezáporných čísel štyrmi spôsobmi:

\bulet ${6=6 \cdot 1}$, takže $u=9$, $v=3$. Keďže $(a, b)=3$,
môžeme písať $a=3\a$, $b=3\b$, pričom $(\a,\b)=1$, teda $9=[a, b]=3[\a,\b]$, čo
dáva $\{\a,\b\}=\{1, 3\}$, čiže $\{a,b\}=\{3, 9\}$.

\bulet ${6=3 \cdot 2}$, takže $u= 6$, $v=4$, potom ale nie je splnená podmienka $v\mid u$.

\bulet ${6=2 \cdot 3}$, takže $u=5$, $v=5$ a~$ab=[a, b](a, b)=uv=5\cdot5$.
Ak sa najväčší spoločný deliteľ čísel $a$ a~$b$ rovná ich najmenšiemu
spoločnému násobku, je $a= b =u= v$, teda $a=b=5$.

\bulet ${6=1 \cdot 6}$, takže $u= 4$, $v=8$, potom ale nie je splnená podmienka
$u\ge v$.

Riešeniami sú usporiadané dvojice $(a, b)\in \{(5, 5), (3, 9), (9, 3)\}$.

\ineres
Označme $(a, b) = D$. Hľadané čísla potom môžeme napísať v~tvare
$a= \alpha D$, $b = \beta D$, pričom $\alpha$ a~$\beta$ sú navzájom
nesúdeliteľné prirodzené čísla, takže najmenší spoločný násobok vyjde
ako $[a, b] = \alpha \beta D$. Dosadením do zadanej rovnice a~jej
úpravou dostaneme
$$
\aligned
2 \alpha \beta D+3D=&\alpha D \beta D, \\
2 \alpha \beta+3=&\alpha \beta D, \\
1 \cdot1 \cdot3=&\alpha \beta (D-2).
\endaligned
$$
Keďže $\alpha$ aj $\beta$ sú prirodzené čísla a~ľavá strana je
kladná, musí byť aj $D-2 \ge 1$. Z~troch možných poradí rozkladu
čísla~3 na súčin troch činiteľov
($3 = 3 \cdot1 \cdot1 = 1 \cdot3 \cdot1= 1 \cdot1 \cdot3$)
dostávame tri riešenia
$(\alpha, \beta, D) \in \{(3, 1, 3), (1, 3, 3), (1, 1, 5)\}$, ku ktorým ľahko
dopočítame usporiadané dvojice $(a, b) = (\alpha D, \beta D) \in \{(9, 3), (3, 9), (5, 5)\}$.

\ineres
Pravá strana zadanej rovnice je deliteľná
číslom~$a$, a~preto je ním deliteľná aj ľavá strana. Aj najmenší
spoločný násobok čísel $a$ a~$b$, ktorý sa vyskytuje na ľavej strane rovnice,
je číslom~$a$ deliteľný, a~preto ním musí byť deliteľné aj číslo $3(a, b)$.
Rovnaká úvaha vedie k~záveru, že číslo $3(a,b)$ je deliteľné aj číslom $b$,
takže je spolu deliteľné číslom $[a,b]$, ktoré je však samo vždy násobkom
čísla~$(a,b)$. Vzhľadom na to, že číslo~3 je prvočíslo, je číslo~$[a,b]$
rovné buď samotnému číslu~$(a,b)$, alebo jeho trojnásobku.

Prvý prípad $[a,b]=(a,b)$ nastane práve vtedy, keď $a=b$~-- z~rovnice $2a+3a=a^2$
potom vychádza $a=5$. Druhý prípad $[a,b]=3(a,b)$ vzhľadom na to, že 3~je prvočíslo,
nastane práve vtedy, keď $\{a,b\}=\{d,3d\}$ pre vhodné prirodzené $d$ -- z~rovnice
$2\cdot3d+3d=3d^2$ potom vychádza $d=3$. Je teda buď $a=b=5$, alebo
$\{a,b\}=\{3,9\}$.

Úlohe tak vyhovujú práve tri usporiadané dvojice $(3, 9)$, $(9, 3)$, $(5, 5)$.


\návody
Určte všetky dvojice $a$, $b$ celých kladných čísel, pre ktoré platí
$a\cdot [a,b]=4\cdot (a,b)$,
pričom symbol $[a,b]$ označuje najmenší spoločný násobok a~$(a,b)$
najväčší spoločný deliteľ celých kladných čísel $a$, $b$.
[\pdfklink{62-C-S-2}{https://skmo.sk/dokument.php?id=687}]

Dokážte, že najmenší spoločný násobok $[a, b]$ a~najväčší spoločný deliteľ $(a, b)$ ľubovoľných
dvoch kladných celých čísel $a$, $b$ spĺňajú nerovnosť
$a~\cdot (a, b) + b \cdot [a, b] \ge 2ab$.
Zistite, kedy v~tejto nerovnosti nastane rovnosť. [\pdfklink{60-C-I-5}{https://skmo.sk/dokument.php?id=368}]

\D
Nájdite všetky dvojice prirodzených čísel $a$, $b$, pre ktoré platí
rovnosť množín
$\{a\cdot[a,b],\allowbreak b\cdot(a,b)\}=\{45,180\}$,
pričom $(x,y)$ označuje najväčší spoločný deliteľ a~$[x,y]$ najmenší spoločný násobok čísel $x$ a~$y$. [\pdfklink{61-C-S-1}{https://skmo.sk/dokument.php?id=457}]

Nájdite všetky trojice prirodzených čísel $a$, $b$, $c$, pre ktoré platí
množinová rovnosť
$\bigl\{(a,b),(a,c),(b,c),[a,b],[a,c],[b,c]\bigr\}=
\{2,3,5,60,90,180\}$,
pričom $(x,y)$ a~$[x,y]$ označuje postupne najväčší spoločný deliteľ
a~najmenší spoločný násobok čísel $x$ a~$y$. [\pdfklink{61-C-I-3}{https://skmo.sk/dokument.php?id=453}]
\endnávod}

{%%%%%   C-I-4
Obsah trojuholníka $XYZ$ budeme označovať $S(XYZ)$.

Keďže podľa zadania sú body $A$ a~$L$ rôzne, sú rôzne aj body $K$ a~$M$.
Platí $S(KML) = S(AML)$, pretože trojuholníky $KML$ a~$AML$ majú zhodnú
základňu $ML$ aj výšku na ňu, lebo priamky $AK$ a~$LM$ sú podľa predpokladu
rovnobežné (\obr).

Vyjdeme z~toho, že ťažnica $AM$ delí trojuholník $ABC$ na dva
trojuholníky $BMA$ a~$CMA$ s~rovnakým obsahom, pretože oba
trojuholníky majú rovnakú výšku na zhodné strany $BM$ a~$CM$,
a~využijeme aj vyššie dokázanú rovnosť $S(KML) = S(AML)$. To už stačí na to,
aby sme dokázali, že $S(KCL) = \frac 12 S(ABC)$, čo dáva požadovanú
rovnosť obsahov štvoruholníka $BKLA$ a~trojuholníka $KCL$, ako ukazuje výpočet
$$
\align
S(KCL)
=&S(KML)+S(LMC)= \\
=&S(AML)+S(LMC)= \\
=&S(ACM)=\frac 12 S(ABC).
\endalign
$$
\inspinsp{c69.3}{c69.4}%

\ineres
Uvažujme číslo~$k$ určené rovnosťou $|CK|=k|BC|$ a~označme
$v$ vzdialenosť bodu~$A$ od priamky~$BC$ (\obr). Z~podobnosti trojuholníkov
$AKC$ a~$LMC$ vyplýva, že pre vzdialenosť~$w$ bodu~$L$ od priamky~$BC$
platí
$$
w=\frac{|CM|}{|CK|}\cdot v=\frac{\frac12 |BC|}{k~|BC|}\cdot v=\frac{v}{2k}.
$$
Obsah trojuholníka $KCL$ je teda rovný
$$
S(KCL)=\frac 12 |CK|w=\frac 12 k|BC|\cdot\frac{v}{2k}=
\frac 12\cdot\frac12|BC|v=\frac12 S(ABC),
$$
ako sme mali dokázať.



\návody
Presvedčte sa, že stredné priečky (\tj.~spojnice stredov strán) delia
ľubovoľný trojuholník na štyri zhodné trojuholníky. [Stredné priečky sú
rovnobežné s~protiľahlými stranami, a~preto sú všetky štyri trojuholníky
podobné. Zhodnosť vyplýva z~toho, že stredné priečky rozpoľujú
jednotlivé strany.]

Uvedomte si, že dva trojuholníky majú zhodné obsahy, ak sa zhodujú ako v~dvoch
stranách, tak aj~v~k~nim prislúchajúcich výškach. Použitím tohto pravidla
vysvetlite, prečo jedna ťažnica ľubovoľného trojuholníka rozpoľuje jeho obsah,
zatiaľ čo všetky tri ťažnice delia jeho obsah na šesť rovnako veľkých
dielov. [V~trojuholníku $ABC$ s~ťažnicami $AA_1$, $BB_1$, $CC_1$ a~ťažiskom~$T$ pre prvé
tvrdenie o~ťažnici~$AA_1$ použite pravidlo na dvojicu trojuholníkov
$(ABA_1, ACA_1)$, pre druhé tvrdenie o~šiestich dieloch navyše aj na dvojice
$(ATC_1, BTC_1)$, $(BTA_1, CTA_1)$ a~$(CTB_1, ATB_1)$.]

\D
Uvedomte si, že dva trojuholníky, ktoré sa zhodujú v~jednej strane, majú svoje
obsahy v~pomere prislúchajúcich výšok na túto stranu. Použitím tohto
pravidla vysvetlite, prečo tri úsečky, ktoré spájajú vrcholy
trojuholníka s~jeho ťažiskom, delia jeho obsah na tri rovnako veľké diely, bez toho, že pritom
použijete výsledok úlohy~N2. [Ťažisko má od každej strany trojuholníka trikrát
menšiu vzdialenosť ako protiľahlý vrchol.]

Nad preponou $AB$ pravouhlého trojuholníka $ABC$ zostrojme štvorec
$ABDE$. Zistite (v~závislosti od dĺžok strán trojuholníka), v~akom pomere
rozdeľuje priamka výšky z~vrcholu~$C$ na preponu $AB$ obsah štvorca $ABDE$.
[Podľa Euklidovej vety o~odvesne to je $a^2: b^2$, pretože priamka výšky z~vrcholu~$C$
delí štvorec na dva obdĺžniky so spoločnou stranou.]

Uvedomte si, že dva trojuholníky, ktoré sa zhodujú v~jednej výške, majú svoje
obsahy v~pomere dĺžok strán, ktorým táto výška prislúcha. Použitím tohto
pravidla potom dokážte tvrdenie: Ľubovoľný konvexný štvoruholník je svojimi
uhlopriečkami rozdelený na štyri trojuholníky s~obsahmi, ktoré možno označiť
$S_1$, $S_2$, $S_3$ a~$S_4$ tak, že platí $S_1:S_2=S_4:S_3$ a~že
rovnosť $S_2=S_4$ nastane práve vtedy, keď sú rovnobežné tie strany
štvoruholníka, ktoré prislúchajú trojuholníkom s~obsahmi $S_1$ a~$S_3$. [Ak označíme
dotyčné obsahy $S_i$ štyroch trojuholníkov v~kruhovom poradí okolo ich
spoločného vrcholu, oba pomery
$S_1:S_2$ a~$S_4:S_3$ budú zhodné s~pomerom, v~akom jedna z~uhlopriečok
štvoruholníka delí jeho druhú uhlopriečku. Rovnosť $S_2=S_4$ je
ekvivalentná s~rovnosťou $S_1+S_2=S_1+S_4$, ktorá je rovnosťou obsahov dvoch
trojuholníkov so spoločnou stranou.]
\endnávod}

{%%%%%   C-I-5
Súčet čísel v~každom riadku (aj v~každom stĺpci) je párny, rovná sa totiž
dvojnásobku najväčšieho z~troch zapísaných čísel. Preto musí byť
aj súčet všetkých čísel zapísaných v~tabuľke párny. Keďže
v~prípade~a) je súčet čísel rovný~45, úlohu nemožno splniť.

Keďže súčet čísel v~prípade~b) je párny ($10 \cdot 11 / 2-1 = 54$),
pokúsime sa tabuľku vyplniť požadovaným spôsobom. Medzi danými
deviatimi číslami je päť čísel párnych (budeme ich označovať symbolom~$P$)
a~štyri nepárne čísla (tie budeme označovať symbolom~$N$). Symbolicky
teda musí v~každom riadku aj stĺpci platiť jedna z~rovností $P~= N+N$,
$P~= P+P$ alebo $N = P+N$. Ak sa pozrieme na počty použitých párnych
a~nepárnych čísel v~týchto rovnostiach a~zohľadníme aj to, že musíme použiť
viac čísel párnych ako nepárnych, usúdime, že aspoň jeden riadok
a~jeden stĺpec musí byť typu $P~= P+P$.

Riadky našej tabuľky
môžeme prehadzovať medzi sebou, bez toho, aby sme porušili podmienku
zo zadania. Podobne to platí pre stĺpce. Môžeme preto predpokladať, že ako prvý
riadok, tak aj prvý stĺpec obsahujú iba párne čísla (sú teda typu
$P~= P+P$). Pozrime sa na ich spoločné políčko (ľavý horný roh tabuľky),
v~ktorom musí byť napísané jedno z~čísel 2, 4, 6, 8 alebo~10.

Vypíšeme pre každého z~piatich kandidátov všetky príslušné rovnosti pre splnenie
podmienky úlohy pre prvý stĺpec a~prvý riadok:
$$
\gather
2 = 6-4 = 8-6 = 10-8,\quad 4 = 6-2 = 10-6, \quad 6 = 2+4 = 8-2 = 10-4,\\
8 = 2+6 = 10-2, \quad 10 = 8+2 = 6+4.
\endgather
$$
Vidíme, že iba štvorku a~osmičku dvoma "disjunktnými" spôsobmi zapísať nemožno,
preto v~spoločnom políčku musí byť napísané jedno z~čísel 2, 6
alebo~10 a~tabuľka tak musí byť (až na poradie riadkov, stĺpcov a~prípadné
preklopenie podľa uhlopriečky) vyplnená párnymi číslami jedným z~troch
spôsobov (hviezdičky tu označujú štyri nepárne čísla 3, 5, 7, 9
zapísané v~niektorom poradí), ktoré budeme ďalej skúmať:\strut
$$
\tabulkaciLXIX

&2&8&10

&4&*&*

&6&*&*

\multispan2&\omit\hbox to \sqwciLXIX{\hss\strut Typ A\hss}\global\rulehtciLXIX0pt&~


\qquad
\tabulkaciLXIX

&6&2&8

&4&*&*

&10&*&*

\multispan2&\omit\hbox to \sqwciLXIX{\hss\strut Typ B\hss}\global\rulehtciLXIX0pt&~


\qquad
\tabulkaciLXIX

&10&4&6

&2&*&*

&8&*&*

\multispan2&\omit\hbox to \sqwciLXIX{\hss\strut Typ C\hss}\global\rulehtciLXIX0pt&~


$$
V~tabuľke typu~A by sme v~2. a~3. stĺpci museli mať $8 = 3+5$ a~$10 = 3+7$,
číslo 3 však nemôže byť v~oboch stĺpcoch, preto požadované vyplnenie
tabuľky typu~A neexistuje.

V~tabuľke typu~B máme v 3.~riadku a~3.~stĺpci opäť čísla 8 a~10
a~z~už uvedených rovností naopak vidíme, že tabuľku možno doplniť jediným
vyhovujúcim spôsobom
$$
\tabulkaciLXIX

& 6&2&8

& 4&9&5

&10&7&3


\ \to\
\tabulkaciLXIX

&2& 6&8

&7&10&3

&9& 4&5


$$
Tabuľku sme rovno premenili (vzájomnou výmenou najskôr prvého stĺpca
s~druhým a~potom druhého riadku s~tretím) na jeden z~možných tvarov, keď číslo~10
stojí uprostred tabuľky, ako vyžaduje zadanie úlohy.

Napokon v~tabuľke typu~C máme v~3.~riadku jedinú možnosť vyjadrenia
$8 = 5+3$, zatiaľ čo v~3.~stĺpci je to jedine $6 = 9-3$.
Tým je určená pozícia nielen čísel 3, 5 a~9, ale aj zvyšného
čísla~7. Výsledná tabuľka zjavne úlohe nevyhovuje.

Ostáva zistiť počet rôznych tabuliek, ktoré možno z~vyplnenej
tabuľky typu~B vytvoriť tak, aby číslo~10 bolo uprostred.
V~prostrednom riadku či stĺpci musia spolu s~10 byť čísla 6 a~4, resp. 7 a~3,
pre ich umiestnenie tak máme celkom $4\cdot2=8$ možností (šestku dáme
na jedno zo štyroch políčok susediacich s~prostrednou desiatkou, štvorka
potom bude na políčku protiľahlom, sedmičku dáme
na jedno z~dvoch zvyšných políčok susediacich s~desiatkou; čísla
v~rohových políčkach sú potom už určené jednoznačne).


\ineres
Ešte jedným spôsobom ukážeme, že požadované vyplnenie tabuľky číslami 2 až~10
je (až na možné zmeny poradia riadkov, poradia stĺpcov
a~preklopenia tabuľky pozdĺž jej uhlopriečok) jediné.

Vyberieme po najväčšom čísle z~každého riadku (respektíve každého
stĺpca) správne vyplnenej tabuľky, dostaneme v~oboch prípadoch tri čísla,
ktorých súčet sa musí podľa zadania rovnať polovici súčtu všetkých
deviatich zapísaných čísel, teda číslu $54:2=27$, čo je $10+9+8$.
Preto tri vybrané čísla musia byť 10, 9 a~8, lebo každá iná
trojica zapísaných čísel má súčet menší ako~27. Tak sme
zistili, že čísla 10, 9, 8 musia byť zapísané každé v~inom riadku
aj v~inom stĺpci. Vzhľadom na vyššie opísané zmeny tak môžeme
predpokladať, že čísla 10, 9, 8 sú zapísané na jednej uhlopriečke
tabuľky ako na poslednom obrázku predchádzajúceho riešenia:
$$
\tabulkaciLXIX

& *&*&8

& *&9&*

&10&*&*


$$

Podmienku úlohy pre riadky a~stĺpce s~číslami 8 a~10 možno splniť
jedine tak, ako vyjadrujú rovnosti $8=2+6=3+5$ a~$10=3+7=4+6$.
Z~toho vyplýva, že neurčené čísla v~dvoch rohových políčkach tabuľky
musia byť 3 a~6~-- jediné dve čísla vystupujúce v~uvedených vyjadreniach
oboch čísel 8 a~10. Umiestnenie čísel 3 a~6 je teda (až na možné
preklopenie pozdĺž uhlopriečky s~číslami 10, 9, 8) dané:
$$
\tabulkaciLXIX

& 6&*&8

& *&9&*

&10&*&3


$$
Umiestnenie zvyšných čísel 2, 4, 5, 7 je jednoduché~--
pozri poslednú tabuľku z~predchádzajúceho riešenia pred konečnou zmenou poradia riadkov a~stĺpcov.

\návody
Je možné vyplniť tabuľku $3 \times 3$ číslami $1, 2, \dots, 9$ tak, aby
súčet čísel v~každom riadku bolo párne číslo? [Nie, keďže potom by musel
byť aj súčet všetkých čísel párny a~$1+2+\dots+9 = 45$.]

Je možné vyplniť tabuľku $3\times3$ číslami $2, 3,\dots, 10$ tak, aby
v~každom riadku bolo najväčšie číslo súčtom ostatných dvoch? [Áno, vyhovuje
napríklad vyplnenie riadkov trojicami čísel $(2,6,8)$, $(4,5,9)$
a~$(3,7,10)$ v~akomkoľvek poradí.]

Je možné vyplniť tabuľku $3\times3$ číslami $4,5,\dots,12$ tak, aby
v~každom stĺpci bolo najväčšie číslo súčtom ostatných dvoch? [Nie. Súčet
všetkých 9 čísel je 72, takže súčet troch stĺpcových maxím by musel byť
36, čo je viac ako $10+11+12$.]

\D
Je možné vyplniť tabuľku $3 \times 3$ číslami $1, 2, \dots, 9$ tak, aby
súčin čísel v~každom riadku bol druhou mocninou nejakého prirodzeného
čísla? [Nie, keďže súčin zadaných čísel nie je druhou mocninou.]

Je možné vyplniť tabuľku $3 \times 3$ číslami $1, 2, \dots, 9$ tak, aby
súčet čísel v~každom riadku a~v~každom stĺpci bol nejakým prvočíslom?
[Áno, jedno z~riešení je zľava doprava a~zhora nadol s~číslami postupne
1, 3, 7, 6, 9, 2, 4, 5, 8.]
\endnávod
}

{%%%%%   C-I-6
Pri riešení tejto úlohy je užitočné hľadať vzťah medzi dvoma zadanými
výrazmi. Prvý z~nich $a^2+b^2+c^2+ab+bc+ca$ obsahuje druhé mocniny
premenných, pričom druhý $a+b+c$ iba prvé mocniny. Jeho umocnením
však dostaneme
$$
(a+b+c)^2 = a^2+b^2+c^2+2 (ab+bc+ac). \tag1
$$

Výrazy $a^2+b^2+c^2$ a~$ab+bc+ac$ sú navyše vo vzájomnom vzťahu, konkrétne
$$
\postdisplaypenalty10000
ab+bc+ac \le a^2+b^2+c^2, \tag2
$$
ako ľahko odvodíme zo zrejmej nerovnosti $(a-b)^2+(b-c)^2+(c-a)^2\ge0$.

Získaná nerovnosť (2) nám spolu s~danou nerovnosťou umožňuje písať
$$
2(ab+bc+ac) \le a^2+b^2+c^2+ab+bc+ac\le1,
$$
takže
$$
ab+bc+ac\le\tfrac12.
$$

Keď teraz šikovne dosadíme do rovnosti~(1) tak, aby sme zároveň využili
aj danú nerovnosť, dostaneme
$$
\align
(a+b+c)^2 =& a^2+b^2+c^2+2(ab+bc+ac) =\\
=& a^2+b^2+c^2+(ab+bc+ac)+(ab+bc+ac) \le 1+\tfrac12=\tfrac32 .
\endalign
$$
A~keďže čísla $a$, $b$, $c$ sú kladné, dostávame odhad
$$
a+b+c\le\sqrt{\tfrac32}.
$$


Ostáva ukázať, že túto hodnotu možno aj dosiahnuť. Najjednoduchšou
možnosťou je vyskúšať $a= b = c$ (lebo potom nerovnosť~(2)
prejde na rovnosť). V~tomto prípade z~danej nerovnosti vyjde $6a^2\le1$
s~rovnosťou pre $a= b = c=1/\sqrt6$; pritom pre tieto hodnoty naozaj
vychádza $a+b+c=\sqrt{3/2}$.


\návody
Pre kladné čísla $a$, $b$, $c$ platí $a^2+b^2+ab \le 2$. Nájdite
najväčšiu možnú hodnotu súčinu~$ab$. [Využite odhad $2ab \le a^2+b^2$,
najväčšia hodnota $ab=2/3$ sa dosiahne pre $a=b=\sqrt{2/3}$.]

Dokážte nerovnosť $a^2+b^2+c^2 \ge ab+bc+ca$ a~zistite, kedy
v~nej nastáva rovnosť. [Prenásobíme dvoma
a~upravíme na tvar $(a-b)^2+(b-c)^2+(c-a)^2 \ge 0$, rovnosť nastane
práve vtedy, keď $a=b=c$.]

Reálne čísla $a$, $b$, $c$ majú súčet 3. Dokážte, že $3\ge ab+bc+ca$.
Kedy nastane rovnosť?
[Vyplýva z~rovnosti $9 = (a+b+c)^2 = a^2+b^2+c^2+2 (ab+bc+ca)$
a~z~predošlej úlohy. Rovnosť nastane jedine v~prípade $a=b=c=1$.]

\D
Nech $a$, $b$, $c$ sú kladné reálne čísla, ktorých súčet je~3, a~každé z~nich
je nanajvýš~2. Dokážte, že platí nerovnosť $a^2+b^2+c^2+3abc<9$.
[\pdfklink{68-C-I-3}{https://skmo.sk/dokument.php?id=3043}]

Pre nezáporné reálne čísla $a$, $b$ platí $a^2+b^2=1$. Určte
najmenšiu aj najväčšiu možnú hodnotu výrazu
$V= (a^4+b^4+ab+1) / (a+ b)$.
[\pdfklink{68-B-I-4}{https://skmo.sk/dokument.php?id=3042}]
\endnávod
}

{%%%%%   A-S-1
Nerovnosť medzi prvými dvoma výrazmi upravíme odčítaním pravej strany
a~následným postupným vynímaním:
$$
\align
ab+cd - bc - ad &> 0 ,\\
a(b-d) - c(b-d) &> 0 ,\\
(a-c)(b-d) &>0.
\endalign
$$
Keďže platí $a~> c$, musí byť aj druhá zátvorka kladná, a~platí tak $b> d$.

Podobnú úpravu urobíme pre nerovnosť $bc+ad > ac+bd$, čím získame
$$
\align
bc + ad -ac-bd &> 0, \\
b(c-d) -a(c-d) &> 0, \\
(b-a)(c-d) &>0.
\endalign
$$
Prvá zátvorka je vďaka $a~> b$ záporná, a~preto musí byť záporná aj tá druhá.
Z~toho usúdime, že $d > c$.

Odvodili sme tak reťazec nerovností $a~> b > d > c$, z~ktorého vidíme, že
najmenším zo štvorice čísel $a$, $b$, $c$, $d$ môže byť jedine~$c$.

\poznamkac 1. Keďže prevedené úpravy boli ekvivalentné, môžeme konštatovať, že ak pre reálne
čísla $a$, $b$, $c$, $d$ platí $a>b>d>c$ sú obe nerovnosti zo zadania úlohy splnené.

\poznamkac 2.
Porovnaním prvého výrazu s~posledným možno
analogickým spôsobom dokázať nerovnosť $b > c$. Ak je táto nerovnosť
dokázaná spolu s~$d > c$, stačí to na úplné riešenie. Ak však dokážeme
iba $b >c$ a~$b > d$, je stále možné, že $c > d$.

\nobreak\medskip\petit\noindent
\item{$\triangleright$} Za úplné riešenie dajte 6~bodov, z~ktorých tri prislúchajú dôkazu každej
z~nerovností $b > d$ a~$d > c$ (prípadne $b > c$ a~$d > c$). Ak je
dokázaná dvojica nerovností $b >c$ a~$b > d$, dajte 4~body.

\item{$\triangleright$} Za nájdenie jedného zo súčinových tvarov $(a-c)(b-d)>0$,
$(b-a)(c-d) >0$, $(a-d)(b-c) >0$ dajte dva body. V~prípade nájdenia
viacerých súčinových tvarov dajte štyri body, ak prislúchajúce nerovnosti
vedú na úplné riešenie (pozri poznámku na konci riešenia), a~tri body ak
nie.

\item{$\triangleright$} Existencia vyhovujúcej štvorice čísel je zaručená už v~zadaní, a~netreba tak
uvádzať príklad. Z~rovnakého dôvodu nie je nutné pri akýchkoľvek
úpravách postupovať ekvivalentne a~ani prípadnú ekvivalenciu spomínať.

\endpetit}

{%%%%%   A-S-2
Vo všetkých riešeniach budeme označovať $M$, $M'$ zodpovedajúce stredy strán $BC$,
$B'C'$. Potom stačí dokázať rovnosť $|\uhel AMC|=|\uhel A'M'C'|$.

\nres
Trojuholníky $ABC$ a~$A'B'C'$ umiestnime do roviny tak, aby bolo $A'=A$,
$B'=B$ a~$C'$ bol obraz bodu~$C$ v~stredovej súmernosti podľa bodu~$A$,
čo možno práve vďaka predpokladanej rovnosti
$|\uhel BAC| + |\uhel B'A'C'| =180^\circ$.
Úsečky $AM'$, $AM$ sú tak stredné priečky
trojuholníka $BCC'$ (\obr), takže štvoruholník $BMAM'$ je rovnobežník.
Zo zhodnosti jeho vnútorných uhlov pri protiľahlých vrcholoch $M$ a~$M'$ už vyplýva
zhodnosť vyznačených uhlov $AMC$ a~$A'M'C'$, ktorú sme chceli dokázať.
\insp{a69.8}%

\nres
Budeme navyše predpokladať, že $|\uhel BAC| \ne |\uhel B'A'C'|$, lebo
inak by tvrdenie úlohy vyplývalo priamo zo zhodnosti pravouhlých trojuholníkov
$ABC$ a~$A'B'C'$. Vzhľadom na symetriu potom stačí uvažovať iba
ten prípad, keď uhol $BAC$ je ostrý.

Umiestnime oba trojuholníky tak, aby bolo $A'=A$, $C'=C$ a~$B'$ bol taký bod
polroviny $ACB$, že $|\uhel B'AC|=180^\circ-|\uhel BAC|$, pričom
$|AB'|=|AB|$ (\obr).
\insp{a69.9}%

Teraz stačí dokázať, že štvoruholník $AM'MC$ je tetivový, lebo potom platí
želané $|\uhel AMC|=|\uhel AM'C|=|\uhel A'M'C'|$ podľa vety o~obvodovom uhle.

Ak označíme $\alpha=|\uhel BAC|$, má uhol $B'AB$ veľkosť
$|\uhel B'AC| - \alpha = 180^\circ-2\alpha$. Z~rovnoramennosti
trojuholníka $ABB'$ potom vyplýva, že $|\uhel ABB'|=\alpha$, čiže $BB'
\parallel AC$. Keďže $MM'$ je strednou priečkou trojuholníka $CBB'$,
platí aj $MM' \parallel BB'$, a~teda aj~$MM' \parallel AC$. Pre stred~$N$ strany~$AC$ navyše platí $|MN|=\frac12 |AB| = \frac12 |AB'|= |M'N|$,
takže os úsečky~$MM'$ prechádza bodom~$N$. To už ale znamená, že
lichobežník $AM'MC$ je rovnoramenný, a~teda aj tetivový, čo sme
chceli dokázať.

\nres
Dokážeme zhodnosť trojuholníkov $MAC$, $M'C'A'$ podľa vety {\it sss} výpočtom.
Zo zhodnosti potom vyplynie aj požadovaná rovnosť uhlov.
Pri štandardnom označení strán oboch daných trojuholníkov
podľa kosínusovej vety pre trojuholník $A'B'C'$ platí
$$
a'{}^2=b'{}^2+c'{}^2-2b'c' \cos(180^\circ-\alpha)=b^2+c^2+2bc \cos \alpha.
$$
Pre dĺžku ťažnice $t_a$ pritom platí známy vzťah $4t_a^2=2b^2+2c^2-a^2$.
Dosadením za~$a^2$ z~kosínusovej vety pre trojuholník $ABC$ získame
$$
t_a^2=\frac{2b^2+2c^2-a^2}{4}
=\frac{b^2+c^2+2bc\cos \alpha}{4} =\frac{a'{}^2}{4},
$$
a~teda $a'/2=t_a$, čiže $|M'C'|=|MA|$. Rovnosť $|M'A'|=|MC|$ dokážeme
analogicky, a~keďže podľa zadania platí aj $|AC|=|C'A'|$, sú
trojuholníky $MAC$, $M'C'A'$ naozaj zhodné.

\nres
Iný výpočet založíme na vyjadrení $\cos |\uhel AMC|$ pomocou $b$, $c$
a~$\cos \alpha$. Ako v~predchádzajúcom riešení použijeme známy vzťah
$4t_a^2=2b^2+2c^2-a^2$ a~tiež kosínusovú vetu
$a^2=b^2+c^2-2bc\cos\alpha$. Z~kosínusovej vety pre trojuholník $AMC$
a~uvedených vzťahov postupne dostávame:
$$
\align
\cos |\uhel AMC| &= \frac{\frac1{4}{a^2}+t_a^2-b^2}{a~t_a}
= \frac{a^2+4t_a^2-4b^2}{4a t_a}
= \frac{2c^2-2b^2}{4at_a}
= \frac{c^2-b^2}{2a t_a}=\\
&= \frac{c^2-b^2}{\sqrt{(b^2+c^2-2bc \cos
\alpha)(2b^2+2c^2-a^2)}}=\\
&= \frac{c^2-b^2}{\sqrt{(b^2+c^2-2bc \cos
\alpha)(b^2+c^2+2bc\cos\alpha)}}= \\
&= \frac{c^2-b^2}{\sqrt{(b^2+c^2)^2 - 4b^2c^2 \cos^2
\alpha}}.
\endalign
$$
Použitím tohto vzorca pre trojuholník $A'B'C'$ s prvkami $b'=b$, $c'=c$
a $\alpha'=180\st-\alpha$ dostávame vďaka
$\cos^2\alpha=\cos^2(180^\circ-\alpha)$ rovnosť
$\cos |\uhel AMC|=\cos|\uhel A'M'C'|$, a~teda $|\uhel AMC|=|\uhel A'M'C'|$,
čo sme chceli dokázať.


\nobreak\medskip\petit\noindent
Za úplné riešenie dajte 6~bodov. V~prípade neúplných riešení postupujte
nasledovne:
\item{$\triangleright$} Za ľubovoľnú konštrukciu, v~ktorej riešiteľ využije rovnosť $|\uhel
BAC| + |\uhel B'A'C'| = 180^\circ$ na~netriviálne zistenie (napr. tri
body ležia na jednej priamke, štyri body ležia na kružnici), dajte 2~body. Za
konštrukciu, vďaka ktorej riešiteľ netriviálne preformuluje dokazované
tvrdenie (napríklad~-- ako v~druhom riešení~-- na to, že body $A$, $M'$,
$M$, $C$ ležia na jednej kružnici), dajte tiež 2~body. Preto ak je splnené
oboje, dajte body štyri.

\item{$\triangleright$} Za správny výpočet vedúci k~$|AM|=|B'C'|/2$ dajte tri body.
Rovnako tak za vyjadrenie $\cos |\uhel AMC|$ iba pomocou $b$, $c$
a~$\alpha$. Vzťah pre dĺžku ťažnice stačí uviesť ako (známy) fakt.
Za samotné jeho uvedenie ale body neudeľujte.

\item{$\triangleright$} Body udelené za geometrickú konštrukciu nemožno sčítať s~bodmi
udelenými za výpočty.

\item{$\triangleright$} Ak riešiteľ úlohu vyrieši pomocou konštrukcie, ktorá predpokladá
$|\uhel BAC| < 90^\circ$ alebo podobné tvrdenie, ktoré možno bez ujmy na
všeobecnosti predpokladať zo symetrie, body nestrhávajte. Plný počet bodov
dajte aj pri zabudnutí prípadu $|\uhel BAC| = 90^\circ$.


\endpetit
\bigbreak
}

{%%%%%   A-S-3
Ak by niektorá z~oboch červených úsečiek na \obr{} pretínala práve jednu
dominovú kocku, zvýšilo by nám v~štvorci $6\times 6$,
ktorý táto úsečka z~daného útvaru vyčleňuje,
vydláždiť 35~políčok. To je ale samozrejme nemožné, keďže 35~je nepárne
číslo.

Platí teda, že ktorákoľvek z~oboch červených úsečiek buď nepretína žiadnu
dominovú kocku (prípad~A ako pri ľavej úsečke na \obrr1), alebo pretína práve dve
dominové kocky (prípad~B ako pri pravej úsečke na \obrr1).
\insp{a69.7}%

Ak pri niektorej červenej úsečke nastane prípad~A, zvýši v~príslušnom
štvorci $6\times 6$ na~vydláždenie všetkých 36~políčok. Označme~$a$ počet
spôsobov, ktorými to možno spraviť. Podobne ak nastane prípad~B, ostáva
v~príslušnom štvorci vydláždiť oblasť majúcu 34~políčok. Príslušný
počet spôsobov označme~$b$. (Zrejme platí $a>b$.) Teraz rozlíšime tri prípady.

\ite(a)
Dláždení, v~ktorých nastane pri oboch červených úsečkách prípad~B, je
presne~$b^2$.

\ite(b)
Ak nastane raz prípad~A a~raz prípad~B, je dláždenie
oblasti medzi úsečkami určené jednoznačne podľa toho, či prípad~A nastal
vľavo alebo vpravo. Dláždení tohto typu je teda $2ab$.

\ite(c)
V~poslednom prípade \uv{A-A} je možné zvyšný štvorec $2\times2$
medzi červenými úsečkami
vydláždiť dvoma spôsobmi, a~výsledných dláždení je teda~$2a^2$.

Celkový počet dláždení je $b^2 + 2ab + 2a^2$, čo možno upraviť na
želaný súčet dvoch štvorcov ako $(a+b)^2 + a^2$. Tým je úloha
vyriešená.

\nobreak\medskip\petit\noindent
Za úplné riešenie dajte 6~bodov.
Tie rozdeľte medzi jednotlivé časti úlohy nasledovne:
\item{$\triangleright$}[2~body] Dôkaz, že hľadané dláždenia sú štyroch druhov A-A, B-B,
A-B, B-A~alebo ekvivalentného tvrdenia.

\item{$\triangleright$}[1~bod] Označenie počtov čiastočných dláždení $a$, $b$ (alebo
iných dvoch počtov, napríklad $a-b$ a~$b$).

\item{$\triangleright$}[2~body] Určenie počtu dláždení ako $b^2 + 2ab + 2a^2$
(alebo ekvivalentný výraz v~prípade iného označenia).

\item{$\triangleright$}[1~bod] Úprava výrazu na súčet štvorcov.

Za snahy o~priamy výpočet počtu dláždení dajte body iba v~prípade
správneho výsledku aj správnej argumentácie. Vzhľadom na to, že už
hodnoty $a= 6\,728$ a~$b = 2\,900$ sa nedajú ľahko nájsť bez pomoci počítača,
také riešenia neočakávame. Ak sa to predsa len podarí, dajte 2~body
za správny výpočet každej z~hodnôt $a$, $b$.
\endpetit
\bigbreak}

{%%%%%   A-II-1
Keď odčítame od seba prvé dve rovnice a~prevedieme oba zlomky na spoločného
menovateľa, dostaneme
$$
\frac{(y+z)-(x+y)}{(x+y)(y+z)}+z-x=0,
$$
odkiaľ po vyňatí člena $z-x$ získame
$$
(z-x)\left(\frac1{(x+y)(y+z)}+1\right) = 0. \tag1
% \label{eq:ii-1}
$$
Analogickými úpravami pre rozdiel iných dvojíc rovníc potom dostaneme
aj~$$
(y-z)\left(\frac1{(z+x)(x+y)}+1\right)=0,\quad
(x-y)\left(\frac1{(y+z)(z+x)}+1\right)=0.
$$

Predpokladajme najskôr, že sú čísla $x$, $y$, $z$ navzájom rôzne. Potom môžeme
ich nenulovými rozdielmi odvodené rovnice vydeliť a~po zjednodušení riešiť sústavu
$$
\align
(x+y)(y+z) &=-1,\\
(y+z)(z+x) &=-1,\\
(z+x)(x+y) &=-1,
\endalign
$$
ktorú možno po roznásobení prepísať ako
$$
\align
y^2 &=-1 -xy-yz-zx,\\
z^2 &=-1 -xy-yz-zx,\\
x^2 &=-1 -xy-yz-zx.
\endalign
$$
Rovnosť $x^2 = y^2 = z^2$ však znamená, že sa aspoň dve z~čísel $x$,
$y$, $z$ rovnajú, čo je v~spore s~predpokladom o~ich rôznosti.

V~ďalšom prípade, keď platí $x = y = z$, ostáva podľa pôvodnej sústavy vyriešiť rovnicu
$1/(2x) + x = 1$. Kvadratická rovnica $2x^2 - 2x + 1=0$, ktorá je jej dôsledkom,
má ale záporný diskriminant, a~tak ani rovnica $1/(2x) + x= 1$
žiadne reálne riešenie nemá.

Ostáva prípad, keď sú práve dve z~čísel $x$, $y$, $z$ rovnaké.
Vzhľadom na symetriu sa stačí zaoberať prípadom $x=y\ne z$. Z~rovnosti~(1)
po vydelení nenulovým číslom $z-x$ získame
$$
-1 = (x+y)(y+z)=2x(x+z),
$$
zatiaľ čo z~prvej zadanej rovnice vyjde
$$
\frac1{x+y}=\frac1{2x}=1-z.
$$
Potom stačí dvoma spôsobmi vyjadriť hodnotu $1/(2x)$ ako
$$
-x-z~= \frac{1}{2x} = 1 - z,
$$
odkiaľ už máme $x={-1}$ a~následne $z~= 3/2$. Ľahko overíme, že
trojica $({-1}, {-1}, \frac32)$ vyhovuje aj pôvodnej sústave rovníc.

\odpoved
Sústava rovníc má práve tri riešenia $({-1},{-1},\frac32)$,
$(\frac32, {-1}, {-1})$, $({-1}, \frac32, {-1})$.

\ineres
Nech $x$, $y$, $z$ sú riešenia sústavy. Potom ako každú z~rovníc
vynásobíme menovateľom príslušného zlomku, získame sústavu rovníc
$$
\align
1 + xz+yz &= x+y, \\
1 + yx+zx &= y+z, \\
1 + zy+xy &= z+x.
\endalign
$$
Ak odčítame od prvej rovnice rovnicu druhú, dostaneme
$$
yz-yx=x-z,\quad\text{čiže}\quad (y+1)(z-x)=0.
$$
Podobným odčítaním (alebo vzhľadom na symetriu)
v~súhrne dostaneme sústavu rovníc
$$
0=(y+1)(z-x)=(z+1)(x-y)=(x+1)(y-z). \tag2
$$

Teraz rozlíšime, či je niektoré z~čísel $x$, $y$, $z$
rovné číslu~${-1}$, alebo nie. Ak nie, podľa~(2) by sme mali $x=y=z$
a~po dosadení do pôvodnej sústavy by sme dostali kvadratickú rovnicu so záporným
diskriminantom (pozri prvé riešenie). Musí teda nastať prípad, keď sa
číslo~${-1}$ v~trojici $(x,y,z)$ nachádza.

Vzhľadom na symetriu stačí rozobrať iba prípad, keď $x={-1}$.
Vtedy sa sústava~(2) redukuje na rovnicu $(y+1)(z+1)=0$,
a~teda aspoň jedno z~čísel $y$, $z$ je tiež rovné~${-1}$. Po dosadení
napríklad $x=y={-1}$ do prvej z~pôvodných rovníc dopočítame $z=3/2$
a~ľahko sa uistíme, že potom sú splnené aj zvyšné dve rovnice. Vďaka symetrii
tak nachádzame jediné tri trojice
$({-1}, {-1}, \frac32)$, $(\frac32, {-1}, {-1})$,
$({-1}, \frac32, {-1})$, ktoré zadanej sústave vyhovujú.


\ineres
So sústavou rovníc zbavenou zlomkov, ktorú sme zapísali na úvod
druhého riešenia, teraz naložíme inak.
Po pričítaní výrazu $xy+1$ k~obom stranám prvej rovnice dostaneme
$$
2 + xy + xz + yz = xy + x +y + 1 = (x+1)(y+1).
$$
Všimnime si, že výraz na ľavej strane je symetrický v~premenných $x$,
$y$ a~$z$. Rovnakú ľavú stranu teda získame aj po podobných úpravách druhej
a~tretej rovnice:
$$
\align
2 + xy + xz + yz &= (y+1)(z+1), \\
2 + xy + xz + yz &= (z+1)(x+1).
\endalign
$$
Porovnaním pravých strán tak dostaneme, že
$$
(x+1)(y+1) = (y+1)(z+1) = (z+1)(x+1).
$$
Ak je každé z~čísel $x$, $y$, $z$ rôzne od ${-1}$, môžeme v~každej
z~rovností krátiť a~vyjde, že $x=y=z$.
Podobne ako v~prvom riešení zistíme, že tento prípad žiadne riešenie neposkytuje.

Ak je ale napríklad $x={-1}$, potom musí platiť aj $(y+1)(z+1) = 0$,
a~teda aspoň jedno z~čísel $y$, $z$ je tiež rovné~${-1}$.
Zvyšok dokončíme rovnako ako v~druhom riešení.


\nobreak\medskip\petit
\noindent
Za úplné riešenie dajte 6 bodov. Neúplné riešenia ohodnoťte podobne, ako
teraz opíšeme pri troch vyššie podaných riešeniach:
\item{$\triangleright$}
Pri postupe z~prvého riešenia dajte 4~body za vylúčenie prípadu
$x\ne y\ne z\ne x$, z~toho 1~bod za úpravu rozdielov {\it
pôvodných\/} rovníc na súčinové tvary s~činiteľmi $x-y$, $y-z$
a~$z-x$, podobné ako má rovnica (1). Ďalej dajte 1~bod za vylúčenie prípadu
$x=y=z$ a~1~bod za vyriešenie zvyšného prípadu, keď sa práve dve
z~čísel $x$, $y$, $z$ rovnajú.
\item{$\triangleright$} Pri postupe z~druhého riešenia dajte 4~body za odvodenie sústavy~
(2) a~po jednom bode za následné doriešenie dvoch opísaných prípadov
(možno ich rozlíšiť aj podľa toho, či platí $x=y=z$, alebo nie.)
\item{$\triangleright$} Pri postupe z~tretieho riešenia dajte 4~body za dôkaz, že súčiny dvoch z~troch činiteľov $x+1$, $y+1$, $z+1$ majú rovnakú hodnotu a~po 1~bode za následné doriešenie dvoch situácií, keď sa buď žiadny z~týchto činiteľov
nerovná nule, alebo aspoň jeden sa nule rovná.
\item{$\triangleright$} Pokiaľ v~inak úplnom riešení, pri ktorom sa hľadané trojice určia pomocou
inej (odvodenej) sústavy, ako je tá pôvodná, chýba v~závere aspoň
zmienka o~jednoduchej skúške dosadením do všetkých troch pôvodných rovníc alebo zdôvodnenie,
prečo prípadne skúška nutná nie je, strhnite 1~bod, \tj. dajte 5~bodov.
\item{$\triangleright$} Len za nájdenie všetkých troch riešení (bez vysvetlenia, prečo iné
neexistujú) dajte 1~bod.

\noindent
Ak skúša riešiteľ viac prístupov, čiastkové bodové zisky (ako tie uvedené
vyššie) nesčítavame, ale berieme ich maximum.
\endpetit
}

{%%%%%   A-II-2
Označme postupne $M$, $N$, $X$, $Y$ stredy úsečiek $AC$, $AB$, $BP$,
$AO$. (Bod~$M$ je samozrejme stredom aj úsečky~$OP$.)
Úsečky $NY$, $YM$, $MX$, $XN$ sú potom postupne strednými
priečkami trojuholníkov $ABO$, $OPA$, $PBO$, $BPA$ (\obr).
\insp{a69.10}%

Platí tak
$$
|NX| = \tfrac12 |AP| = |YM|
$$
a~zároveň
$$
|NY| = \tfrac12 |BO| = |XM|.
$$
Keďže bod $O$ je stredom kružnice opísanej trojuholníku $ABC$, platí
$|BO| = |AO|$ a~z~osovej súmernosti navyše aj $|AO|=|AP|$. To dokopy
znamená, že
$$
|NX| = |YM| = |NY| = |XM|.
$$
Ak neplatí $X=Y$ (keď je tvrdenie úlohy zrejmé), posledné rovnosti vedú
ako je známe k~záveru, že $NXYM$ je kosoštvorec alebo štvorec,
a~jeho uhlopriečka~$XY$ je teda kolmá na
druhú uhlopriečku~$MN$, čo je zároveň stredná priečka trojuholníka $ABC$
rovnobežná s~jeho stranou~$BC$.\footnote{Kolmosť úsečiek $XY$ a~$MN$
možno v~danej situácii zdôvodniť aj bez úvah
o~druhu štvoruholníka $NXMY$. Z~rovností $|XM|=|XN|$ a~$|YM|=|YN|$ totiž
vyplýva, že oba body $X$ a~$Y$ ležia na osi úsečky $MN$.}
Preto je $XY$ kolmé na~$BC$, ako sme chceli dokázať.

\ineres
Budeme počítať vzdialenosti bodov od kolmice na stranu~$BC$ vedenej
bodom~$B$. Pre ľubovoľný bod~$Z$ označme túto vzdialenosť~$d_Z$.
Pri rovnakom označení bodov $X$, $Y$, $M$ ako v~prvom riešení
nám stačí dokázať, že $d_Y = d_X$ (\obr).
\insp{a69.11}


Keďže $Y$ je stredom $AO$ a~$O$ leží na osi strany~$BC$, platí
$$
d_Y = \tfrac12(d_A + d_O) = \tfrac12 d_A + \tfrac14 d_C.
$$
Podobne z~určenia bodu~$X$ vyplýva rovnosť $d_X = \frac12 d_P$, a~tak stačí vyjadriť
aj vzdialenosť~$d_P$ pomocou $d_A$ a~$d_C$. To je ale jednoduché, lebo bod~$M$ je
spoločným stredom $AC$ a~$OP$, a~preto platí
$$
d_M = \tfrac12 (d_O + d_P),
$$
pričom $d_M = \frac12 (d_A+d_C)$. Z~toho už nachádzame
$$
\let\frac\tfrac
d_X = \frac12 d_P = d_M - \frac12 d_O = \frac12(d_A+d_C) - \frac14
d_C =\frac12 d_A + \frac14 d_C.
$$
Máme tak $d_Y = d_X$ a~sme hotoví.



\nobreak\medskip\petit\noindent
Za úplné riešenie dajte šesť bodov.
Neúplné postupy vedené ako v~prvom vzorovom riešení hodnoťte
nasledovne:
\item{$\triangleright$} Za dokreslenie bodu~$N$ dajte 1~bod. Ak riešiteľ navyše ukáže,
že $NXMY$ je rovnobežník (resp. kosoštvorec), dajte ďalšie dva
(resp. štyri) body. Štyrmi bodmi oceňte aj dôkaz kolmosti úsečiek $XY$
a~$MN$ bez použitia kosoštvorca ako v~poznámke~1 pod čiarou.
\item{$\triangleright$} Zabudnutie na prípad $X=Y$ (ktorý pre ostrouhlý trojuholník $ABC$, ako možno
ukázať, nastať ani nemôže) nepenalizujte.

Pri riešeniach, ktoré počítajú so vzdialenosťami od kolmice vedenej bodom~$B$
(či už priamo, alebo vo forme súradníc kolmých priemetov bodov na priamku
$BC$ v~úlohe číselnej osi), je bodovacia schéma nasledujúca:

\item{$\triangleright$} Za uvedenie, že stačí dokázať $d_X = d_Y$, dajte 1~bod.
\item{$\triangleright$} Za vhodné vyjadrenie jednej zo vzdialeností $d_X$, $d_Y$ dajte 2~body,
za vyjadrenie druhej z~nich tým istým výrazom potom dajte 3~body. Ak je
ale druhá vzdialenosť vyjadrená iným výrazom,
záverečné 3~body dajte až za dôkaz, že výrazy pre obe vzdialenosti
majú rovnakú hodnotu (napr. ekvivalentnými úpravami ich rovnosti).
\endpetit}

{%%%%%   A-II-3
So všetkými číslami budeme počítať modulo~13, teda ako so zvyškami po delení~13.
Ďalej je zrejmé, že nielen množina $\mm P$,
ale aj obe odvodené množiny $\mm Q$ a~$\mm R$ sú
štvorprvkové. Preto ak majú tri uvedené množiny obsahovať všetky
nenulové zvyšky modulo~13, ktorých je dvanásť, musia byť navzájom disjunktné.

Nech $\mm P$ je ľubovoľná množina spĺňajúca požiadavky úlohy a~nech $x\in\mm P$.
Potom z~incidencií $3x\in\mm Q$ a~$4x\in\mm R$ vyplýva, že číslo $12x=3\cdot 4x=4\cdot 3x$
nepatrí do $\mm Q$ ani do $\mm R$, lebo $4x$ ani $3x$ nepatrí do~$\mm P$
a~každé dve rôzne čísla majú rôzne trojnásobky aj rôzne štvornásobky.
Preto $12x\in\mm P$ a aj
$$
3\cdot 12 x = 10x \in\mm Q \quad \text{a} \quad 4\cdot 12 x = 9x \in\mm R.
$$
Šestica čísel $(x,4x,3x,12x,9x,10x)$ so zaradením
$(\mm P,\mm R,\mm Q,\mm P,\mm R,\mm Q)$ má navyše, ako sa ľahko overí,
tú vlastnosť, že každý jej nasledujúci člen je (modulo~13)
štvornásobkom predchádzajúceho čísla, čo platí aj cyklicky, teda aj
pre posledný a~prvý člen.

\def\black{\pdfliteral{0 g}}
\def\red {\pdfliteral{1 0 0 rg}\aftergroup\black}
\def\green{\pdfliteral{0 1 0 rg}\aftergroup\black}
\def\blue {\pdfliteral{0 0 1 rg}\aftergroup\black}
Všetky čísla od 1 do 12 vytvárajú dve cyklické šestice opísaného druhu
$$
({\red1},{\green4},{\blue3},{\red12},{\green9},{\blue10})\quad\text{a}\quad
({\red2},{\green8},{\blue6},{\red11},{\green5},{\blue7})
$$
a~na určenie vyhovujúcej množiny~$\mm P$ stačí len zadať, ktorú z~troch použitých
farieb čísla z~$\mm P$ majú~-- v~každej z~oboch šestíc nezávisle.
Hľadaný počet je preto rovný $3\times 3=9$.


\ineres
Počítajme opäť s~číslami ako s~ich zvyškami modulo~13. Vďaka tomu, že $27x = x$
pre každý zvyšok~$x$,
sa množina všetkých 12~nenulových zvyškov modulo~13 rozpadá na štyri cyklické trojice
$(x,3x,9x)$, konkrétne
$$
A=(1,3,9),\ B=(2,6,5),\ C=(4,12,10),\ D=(7,8,11).
$$
Z~ľubovoľných dvoch zvyškov v~každej trojici je jeden trojnásobkom
druhého, takže do~$\mm P$ musí patriť po jednom prvku z~každej
z~trojíc $A$, $B$, $C$, $D$. Pre taký výber jedného prvku~$a$ z~trojice~$A$
a~jedného prvku~$b$ z~trojice~$B$ máme $3\times3=9$ možností. Ukážeme,
že každé dva také prvky $a$, $b$ možno práve jedným spôsobom doplniť
niektorými prvkami~$c$ z~trojice~$C$ a~$d$ z~trojice~$D$ na
množinu $\mm P=\{a,b,c,d\}$, ktorá bude vyhovovať požiadavkám úlohy.

Všimnime si, že štvornásobok každého prvku z~trojice~$A$, resp. $C$ je
prvok patriaci naopak do trojice~$C$, resp. $A$. Túto vlastnosť má nielen
pár trojíc $A$ a~$C$, ale tiež pár trojíc $B$ a~$D$. Ak má
preto daný prvok~$a$ z~trojice~$A$ (zloženej z~$a$, $3a$, $9a$)
patriť do~$\mm P$, nesmie tam z~trojice~$C$ (zloženej
zo~zvyškov $4a$, $4\cdot3a$, $4\cdot9a$)
patriť ani jej prvok~$4a$ (ten musí ležať v~$\mm R$),
ani jej prvok~$4\cdot9a$ rovný $10a$ (lebo jeho
štvornásobok $40a=a$ by potom ležal v~$\mm P$
aj $\mm R$), takže do~$\mm P$ nutne patrí zvyšný prvok
z~$C$, čiže $c=4\cdot3a=12a$. Podobne s~daným prvkom~$b$ z~trojice~$B$
musí do~$\mm P$ patriť aj prvok~$12b$ z~trojice~$D$.

Ostáva ukázať, že každá takto zostavená množina $\mm P=\{a,b,12a,12b\}$
má požadovanú vlastnosť. Vyplýva to z~našej konštrukcie vyjadrenej
tabuľkou
$$
\postdisplaypenalty10000
\matrix
&A&B&C&D\\
\mm P&a&b&12a&12b\\
\mm Q&3a&3b&10a&10b\\
\mm R&9a&9b&4a&4b
\endmatrix
$$

Hľadaný počet vyhovujúcich množín $\mm P$ je rovný 9.


\medskip\petit\noindent
Za úplné riešenie je 6 bodov. Postupy, ktoré sa namiesto uplatnenia
kombinatorického pravidla súčinu zaoberajú konštrukciou konkrétnych
príkladov vyhovujúcich množín~$\mm P$, hodnoťte podľa tejto schémy:
\item{$\triangleright$} Za zostrojenie aspoň jednej množiny~$\mm P$ dajte 1~bod.
\item{$\triangleright$} Za konštrukciu všetkých deviatich množín~$\mm P$ dajte 3~body.
\item{$\triangleright$} Zvyšné tri body dajte za zdôvodnenie, že iné množiny~$\mm P$ nevyhovujú.
Pritom ziskom 1~bodu oceňte napríklad odvodenie čiastočného poznatku, že
$12x\in\mm P$ pre každé $x\in\mm P$ (alebo aj viac podobných, s~ním
ekvivalentných poznatkov) bez ďalšieho podstatného pokroku.
\endpetit}

{%%%%%   A-II-4
\obrplus%
Políčka daného útvaru ofarbíme šachovnicovo a~útvar rozdelíme na tri
zhodné útvary $\mm U_1$, $\mm U_2$, $\mm U_3$, ako je naznačené na \obr.
\insp{a69.101}%

%\mppic a69.101 \hfil\Obr \par
%\inspicture r(-1)
Teraz si všimneme, že útvar~$\mm U_1$ pozostáva z~rovnakého počtu čiernych
a~bielych políčok a~navyše susedí iba s~políčkami jednej farby. Pritom ľubovoľná
dominová kocka, ktorá by mohla zvonka presiahnuť do~$\mm U_1$, pokrýva jedno
z~bielych políčok diagonály~$\mm U_2$ susediacej s~$\mm U_1$, kde teda môže
zakryť jedine jedno zo susedných čiernych políčok útvaru~$\mm U_1$. Každé také
presahovanie by teda narušilo zistenú rovnováhu čiernych a~bielych políčok útvaru~$\mm U_1$
v~neprospech (voľných) čiernych políčok, preto útvar~$\mm U_1$ je
vydláždený bezo zvyšku a~bez presahovania. Analogickú úvahu možno spraviť pre
útvar~$\mm U_3$, a~tým pádom aj útvar~$\mm U_2$ je nutne vydláždený
bezo zvyšku. Ak označíme~$k$ počet možností, ako
vydláždiť dominovými kockami útvar~$\mm U_1$, je hľadaný počet spôsobov rovný~$k^3$.

Hodnotu $k$ určíme tak, že pomocou matematickej indukcie vyriešime všeobecnejšiu úlohu.
Budeme dokazovať, že útvar $\mm L_n$ vzniknutý "prilepením"
$n$~kusov tvaru obráteného~$\mm L$ k~obdĺžničku $1 \times 2$ (\obr)
možno vydláždiť práve $2n+1$ spôsobmi.
\insp{a69.102}%

Útvar $\mm L_1$ sa dá naozaj vydláždiť tromi spôsobmi. Na~prevedenie indukčného
kroku od~$m$ k~$m+1$ uvažujme útvar~$\mm L_{m+1}$ a~rozlíšme dva prípady. Ak je pravé horné
obrátené~$\mm L$ vydláždené bezo zvyšku, ostáva vydláždiť útvar~$\mm L_m$, čo
sa dá $2m+1$ spôsobmi. V~opačnom prípade je spodné políčko~$x$ spomenutého~$\mm L$ pokryté
vodorovnou kockou, čo postupne vynúti položenie ďalších vodorovných kociek (ako
na \obr{} vľavo) a~následne aj zvislých kociek (ako na \obrr1{} vpravo),
až ostane vydláždiť štvorec $2\times 2$ vpravo hore, čo možno dvoma
spôsobmi. Celkom tak máme $(2m+1)+2 = 2(m+1)+1$ spôsobov vydláždenia~$\mm L_{m+1}$, čím sme
indukčný krok dokončili.
\insp{a69.103}%

Keďže $\mm L_5 = \mm U_1$, je hľadané~$k$
rovné~11 a~celkový počet možností, ako vydláždiť pôvodný
útvar, je tak $k^3 = 11^3 = 1331$.

\ineres
Zamerajme sa na päť políčok, ktoré tvoria ľavý horný okraj pokrývaného
útvaru. Každé z~týchto políčok je pokryté vodorovnou alebo zvislou
kockou. Vidno, že ak je niektoré z~týchto políčok pokryté vodorovnou
kockou, tak aj všetky políčka šikmo nahor doprava od neho sú pokryté
vodorovnými kockami (postupne políčka 1, 2, 3 na \obr A), čo ďalej
implikuje pokrytie ďalších políčok zvislými kockami (postupne políčka 4, 5, 6
na \obrr1B). Podobne, ak je niektoré z~oných piatich políčok pokryté zvislou
kockou, sú aj všetky políčka šikmo nadol doľava od neho pokryté zvislými
kockami, čo implikuje pokrytie ďalších políčok vodorovnými kockami ako
v~dolnej časti \obrr1C.
\insp{a69.104}%

Teraz rozlíšme tri prípady.

1. Všetkých päť políčok je pokrytých zvislými kockami. Potom vynútená časť
dláždenia vyzerá ako na \obrr1D.

2. Všetkých päť políčok je pokrytých vodorovnými kockami. Potom vynútená časť
dláždenia vyzerá ako na \obrr1E, pričom štvorec~$E$ môže byť vydláždený
dvoma rôznymi spôsobmi (a~musí byť vydláždený bezo zvyšku).

3. V~jednom zo štyroch miest označených na \obrr1C šípkami dochádza
k~tomu, že políčka napravo nahor od tohto miesta sú pokryté vodorovnými
kockami a~políčka naľavo nadol od tohto miesta sú pokryté zvislými
kockami. Potom vynútená časť dláždenia vyzerá ako na \obrr1C, pričom
štvorec~$C$ môže byť vydláždený dvoma rôznymi spôsobmi (a~musí byť
vydláždený bezo zvyšku).

Ukázali sme, že útvar~$\mm U_1$ z~\obrr4 prvého riešenia musí byť
vydláždený bezo zvyšku, a~to jedenástimi možnými spôsobmi ($1+2+4\cdot
2=11$). Zo symetrie vyplýva, že aj útvar~$\mm U_3$ musí byť vydláždený bezo
zvyšku, a~to jedenástimi možnými spôsobmi. Ostáva časť~$\mm U_2$, ktorá je
rovnaká ako $\mm U_1$ a~$\mm U_3$, možno ju teda vydláždiť jedenástimi možnými
spôsobmi. Celkom možno útvar vydláždiť $11^3=1331$ spôsobmi.


\nobreak\medskip\petit\noindent
Za úplné riešenie dajte 6 bodov.
\item{$\triangleright$} Za zdôvodnenie, že hľadaný počet možností je $k^3$, pričom $k$ je
počet možností ako vydláždiť menší útvar, dajte 3~body.
\item{$\triangleright$} Za úspešné určenie $k=11$ dajte 3~body, a~to aj v~prípade, že ich
riešiteľ nájde rozborom všetkých možností.

Pri postupe bližšom druhému riešeniu dajte
\item{$\triangleright$} 1 bod za úvahy z~\obrr1A a~\obrr1B o~vynútených ťahoch,
\item{$\triangleright$} 1 bod za symetrickú úvahu zodpovedajúcu vynúteným ťahom v~dolnej časti obrázka~\obrrnum1C,
\item{$\triangleright$} 1 bod za zdôvodnenie, že útvar~$\mm U_1$ musí byť vydláždený bezo zvyšku
(pomocou vynútených ťahov ako na \obrr1C).
\item{$\triangleright$} Ak ale chýbajú prípady D, E z~\obrr1, alebo je chybne určený počet možných
vydláždení útvaru~$\mm U_1$, dajte celkom iba 4~body.
\endpetit}

{%%%%%   A-III-1
Dokážeme, že pre $m=n=1$ sa na tabuli môže objaviť ľubovoľná dvojica
kladných celých čísel a~že pre ostatné voľby $m$ a~$n$ existuje presne
jedna dvojica, ktorá sa nedá dosiahnuť, a~to $(1, 1)$.

\medskip
Je jasné, že obe čísla napísané na tabuli budú vždy kladné
a~celé.
Ďalej si ujasníme, z~akých dvojíc je možné jedným krokom získať dvojicu
$(1, 1)$. Keďže môžeme jednou operáciou zmeniť iba jedno z~čísel, musí byť
taká dvojica bez ujmy na všeobecnosti tvaru $(k, 1)$. Jej
nasledovníkom môžu byť ale iba dvojice $(k+1, 1)$, $(k, k+1)$
(nahradenie súčtom) a~$(k, k)$ (nahradenie súčinom či podielom). Jediné
vyhovujúce~$k$ je tak $k=1$. Inými slovami, dvojica $(1, 1)$ sa dá
dosiahnuť, iba ak je na tabuli napísaná od začiatku.

Teraz učiníme zásadné pozorovanie. Ak máme dvojicu $(a,b)$, môžeme
postupne napísať dvojice
$$
(a,b)\to(a,ab)\to(a,ab+a)\to(a,b+1),
$$
a~pripočítať tak k~$b$ jednotku. Analogicky možno jednotku pripočítať aj k~$a$.

Na vyriešenie úlohy stačí dokázať, že z~ľubovoľnej dvojice $(a,b)$ možno
dosiahnuť dvojice $(1, 2)$ a~$(2, 1)$, lebo potom možno pričítaním
jednotiek dosiahnuť všetky dvojice prirodzených čísel
s~výnimkou už diskutovanej dvojice $(1, 1)$.

Uvažujme dvojicu $(a, b)$, v~ktorej bez ujmy na všeobecnosti platí $a~\ge b$.
Pričítaním jednotiek najskôr získame $(a, 2a)$, odkiaľ po delení dostaneme
$(a, 2)$. Teraz nájdeme najmenšie celé číslo~$k$, pre ktoré $a<2^k$
a~pričítaním jednotiek prejdeme k~$(2^k,2)$. Ďalej postupným delením
vychádza
$$
(2^k, 2) \to (2^{k-1},2)\to(2^{k-2},2)\to \dots\to(4,2)\to(2,2).
$$
Teraz možno vydelením vytvoriť obe dvojice $(2, 1)$ a~$(1, 2)$, čím je
úloha vyriešená.
}

{%%%%%   A-III-2
Do ľavej strany ekvivalentne upravenej nerovnosti
$$
\frac{[BZX]}{[XYZ]}+\frac{[CZY]}{[XYZ]}>2
$$
dosadíme vyjadrenia
$$
\frac{[BZX]}{[XYZ]}=\frac{|BZ|}{|YZ|} \quad\text{a}\quad
\frac{[CZY]}{[XYZ]}=\frac{|CZ|}{|XZ|},
$$
ktoré sú obe porovnaním obsahov a~základní
dvoch trojuholníkov so spoločnou výškou (\obr). Našou úlohou tak je dokázať nerovnosť
$$
\frac{|BZ|}{|YZ|} + \frac{|CZ|}{|XZ|} > 2.
$$
\insp{a69.12}%

Kolmé priemety bodov $B$, $C$, $X$, $Y$ na priamku~$AZ$ označíme
$B'$, $C'$, $X'$, $Y'$ ako na \obrr1.
Tak môžeme dokazovanú nerovnosť prepísať ako
$$
\frac{|BB'|}{|YY'|} + \frac{|CC'|}{|XX'|} > 2.
$$
Teraz stačí použiť AG nerovnosť a~nerovnosti $|BB'|/|XX'|>1$
a~$|CC'|/|YY'|>1$, ktoré zrejme vyplývajú z~voľby bodov $X$, $Y$ vnútri strán $AB$,
$AC$. Dostávame tak
$$
\frac{|BB'|}{|YY'|} + \frac{|CC'|}{|XX'|} \ge
2\sqrt{\frac{|BB'|}{|YY'|}\cdot\frac{|CC'|}{|XX'|}}=
2\sqrt{\frac{|BB'|}{|XX'|}\cdot\frac{|CC'|}{|YY'|}}>2.
$$


\ineres
Využijeme pomerne známu rovnosť
$$
[XYZ]\cdot[BCZ]=[BZX]\cdot[CZY],
$$
ktorá platí pre {\it ľubovoľný\/} konvexný štvoruholník $BCYX$
(s~bodom $Z$ v~priesečníku uhlopriečok, \obrr1) a~ktorú vďaka spoločnej
hodnote sínusov štyroch uhlov $XZY$, $BZC$, $BZX$, $CZY$ okamžite dostaneme po
štvornásobnom použití školského vzorca $[ABC]=\frac12ab\sin\gamma$.

V spojení s~AG nerovnosťou potom máme
$$
[BZX]+[CZY]\geqq2\sqrt{[BZX]\cdot[CZY]}=2\sqrt{[XYZ]\cdot[BCZ]},
$$
preto nám v~našej situácii ostáva overiť iba nerovnosť
$[BCZ]>[XYZ]$. Po pripočítaní $[BXZ]$ k~obom jej stranám
dostaneme ale nerovnosť $[BXC]>[BXY]$, ktorá zrejme platí,
lebo bod~$C$ má od priamky~$BX$ väčšiu vzdialenosť ako bod~$Y$
(vďaka tomu, že $Y$ leží medzi bodmi $A$ a~$C$).


\ineres
Zaveďme tri kladné čísla
$$
\alpha = \frac{[BZC]}{[ABC]},\quad
\beta = \frac{[CZA]}{[ABC]}, \quad
\gamma = \frac{[AZB]}{[ABC]},
$$
pre ktoré zjavne platí $\alpha + \beta + \gamma = 1$.
Navyše je
$$
\frac{[BZX]}{[XYZ]} = \frac{|BZ|}{|ZY|} = \frac{|BY|}{|ZY|} - 1
=\frac{[ABC]}{[CZA]}-1=\frac{1}{\beta} - 1,
$$
pričom v~prvej rovnosti sme využili spoločné výšky, zatiaľ čo v~tretej
rovnosti spoločné základne oboch dotyčných trojuholníkov.
Podobne možno odvodiť, že
$$
\frac{[CZY]}{[XYZ]} = \frac{1}{\gamma} - 1,
$$
a~dokazovanú nerovnosť možno tak po vydelení číslom $[XYZ]$ (a~pripočítaní
dvojky) ekvivalentne prepísať ako
$$
\frac{1}{\beta} + \frac{1}{\gamma} > 4.
$$
Tá však vzhľadom na $\b+\g<1$ už vyplýva zo známej (Cauchyho)
nerovnosti:
$$
\frac{1}{\beta} + \frac{1}{\gamma} > \Big(\frac{1}{\beta} +
\frac{1}{\gamma}\Big)(\beta + \gamma) \ge 4.
$$
}

{%%%%%   A-III-3
Keď sčítame zadané rovnice, získame
$$
x^2+y^2+z^2 - 4x - 4y - 4z + 3p = 0,
$$
čo možno doplnením na štvorce upraviť na tvar
$$
(x-2)^2+(y-2)^2+(z-2)^2 + 3p-12 = 0.
$$

Ak je $p>4$, je ľavá strana kladná a~rovnosť nastať nemôže. Pre $p=4$
je nutne $x=y=z=2$, čo je aj riešenie pôvodnej sústavy. Tým sme vyriešili
časť~a).

Pre prípad $p\in\<1,4\)$ najskôr ukážeme, že $x$, $y$ a~$z$ sú nezáporné
reálne čísla. Predpokladajme (bez ujmy na všeobecnosti), že $y<0$. Potom
z~prvej a~tretej rovnice odvodíme
$$
\align
x^2 - z + p &< 0, \tag1\\
z^2 - 3x + p &< 0, \tag2
\endalign
$$
odkiaľ vzhľadom na $p>0$ vyplýva, že $x> 0$, $z> 0$.
Zároveň ale platia nerovnosti
$$
\align
x^2 - z + p &< 0 \le (x-\sqrt{p})^2,\\
z^2 - 3x + p &< 0 \le (z-\sqrt{p})^2.
\endalign
$$
Porovnaním ľavých a~pravých strán vzhľadom na $p\ge 1$ získame
$$
\align
z &> 2x\sqrt{p} \ge 2x,\\
x &> \frac23 z\sqrt{p} \ge \frac23 z.
\endalign
$$
Oboje pre kladné $x$, $z$ nie je možné, a~tým sme doviedli predpoklad
$y<0$ k~sporu.

\smallskip
Ďalej môžeme bez ujmy na všeobecnosti predpokladať, že $x \ge y \ge 0$
a~$x\ge z\ge 0$. Potom platí $x^2\ge z^2$, ${-3y}\ge {-3x}$, a~preto
$$
z=x^2-3y+p\ge z^2-3x+p=y.
$$
Je teda $x\ge z\ge y \ge 0$. To však znamená, že ${-3y}\ge {-3z}$
a~súčasne $x^2\ge y^2$. Potom
$$
z=x^2-3y+p\ge y^2-3z+p=x,
$$
teda $z\ge x$, a~preto $x=z$. Odčítaním tretej rovnice od prvej
(v~pôvodnej sústave) dostávame vzhľadom na rovnosť $x=z$ vzťah
$3(x-y) =x-y$, z~ktorého ale vyplýva $x=y$. Naozaj tak platí $x=y=z$.


\ineres
V~tomto riešení ukážeme, že sústava nemá iné riešenia ako
tie, pre ktoré $x=y=z$, dokonca pri ľubovoľnom $p > {-10}$. Keďže v~prípade
rovnosti dvoch neznámych možno ako v~samotnom závere prvého riešenia dokázať
rovnosť všetkých troch, stačí nám pre spor predpokladať, že uvažovaná sústava
má riešenie $(x,y,z)$ s~navzájom rôznymi číslami $x$, $y$~a~$z$.

Rozdiel prvých dvoch rovníc upravíme na tvar
$$
(x-y)(x+y) = 3y - 2z - x,
$$
ktorého pravú stranu možno upraviť dvoma spôsobmi ako
$$
3y - 2z - x =
\cases
(y-x) + 2(y-z), \\
3(y-x) + 2(x-z).
\endcases
$$
Rozdiel prvých dvoch rovníc možno tak prepísať dvoma spôsobmi
$$
\align
(x-y)(x+y+1) &= 2(y-z), \\
(x-y)(x+y+3) &= 2(x-z).
\endalign
$$
Po zostavení ďalších dvoch analogických dvojíc rovností
(prislúchajúcich rozdielom iných dvoch rovníc)
medzi sebou vynásobíme tri rovnosti jedného typu a~ďalšie tri rovnosti
druhého typu, čím po vykrátení nenulových rozdielov neznámych
(vďaka predpokladu $x\ne y \ne z\ne x$) získame
$$
\aligned
(x+y+1)(y+z+1)(z+x+1) &= 8, \\
(x+y+3)(y+z+3)(z+x+3) &= -8.
\endaligned
$$
Ľavé strany oboch týchto rovností sú symetrické funkcie premenných $x$, $y$, $z$.
Tretiu takú rovnosť dostaneme sčítaním všetkých troch pôvodných rovníc:
$$
x^2+y^2+z^2 - 4(x+y+z) + 3p = 0.
$$
Technickú náročnosť situácie teraz zmiernime prechodom k~premenným
$$
\align
a &= x+y+1, \\
b &= y+z+1, \\
c &= z+x+1,
\endalign
$$
pričom naším cieľom bude prepísať trojicu symetrických rovníc iba
pomocou elementárnych symetrických polynómov
$$
\align
\alpha &= a+b+c ,\\
\beta &= ab+bc+ca, \\
\gamma &= abc.
\endalign
$$
Po chvíli úprav naozaj získame sústavu rovníc
$$
\align
\gamma &= 8, \\
\gamma + 2\beta + 4\alpha &= -16, \\
3\alpha^2 - 8\beta - 10 \alpha + 12p &= -27.
\endalign
$$
Z~prvých dvoch rovníc možno vyjadriť $\beta = {-12}-2\alpha$, čo po
dosadení do tretej rovnice vedie na jej zjednodušenie na tvar
$$
\alpha^2 + 2\alpha + 41 + 4p = 0.
$$
Diskriminant tejto kvadratickej rovnice je
$$
4 - 4(41 + 4p) = -16p - 160,
$$
čo je pre $p > {-10}$ záporný výraz, a~máme tak želaný spor
s~predpokladanou existenciou riešenia s~navzájom rôznymi číslami $x$, $y$ a~$z$.


\poznamka
Pre $p={-10}$ má sústava riešenie $(x,y,z)$ s~približnými
hodnotami $({-0{,}21}; {-4{,}08}; 2{,}30)$ (a~jeho cyklické obmeny), čo sú
korene istého konkrétneho kubického polynómu (príslušné
čísla $a$, $b$, $c$ sú korene polynómu $P(t)=t^3+t^2-10t-8$).
Pre $p={-11}$ má sústava dokonca celočíselné riešenie $({-1}, {-4}, 2)$.
}

{%%%%%   A-III-4
Označme $d$ najväčší spoločný deliteľ čísel $a$, $b$. Potom platí, že $a =
da_1$, $b= db_1$ pre nejaké nesúdeliteľné prirodzené čísla $a_1$, $b_1$.
Rovnosť zo zadania preto možno (po vydelení číslom~$d$) písať ako
$$
db_1^2 = da_1^2 + da_1b_1 + b_1.
$$
Keďže $d$ delí ľavú stranu tejto rovnosti, musí deliť
aj tú pravú, čo nastane práve vtedy, keď $d \mid b_1$. Podobne platí, že
$b_1$ delí ľavú stranu, a~teda aj tú pravú, čiže
$b_1 \mid da_1^2$. Keďže čísla $a_1$ a~$b_1$ sú nesúdeliteľné,
musí platiť $b_1 \mid d$. Zo vzájomnej deliteľnosti $d \mid b_1$ a~$b_1
\mid d$ teda vyplýva, že $d = b_1$. Preto $b = db_1 = b_1^2$, čo
je druhá mocnina prirodzeného čísla, a~dôkaz je tak ukončený.

\ineres
Keďže prirodzené číslo~$a$ je riešením kvadratickej rovnice
$x^2 + xb + {b-b^2}=0$, je diskriminant tejto rovnice $b^2 - 4(b-b^2) = b(5b-4)$
rovný druhej mocnine nejakého prirodzeného čísla~$k$.

Najväčší spoločný deliteľ~$d$ čísel $b$ a~$5b-4$ je zrejme deliteľom
aj čísla~4. Preto prichádzajú do úvahy iba hodnoty $d \in \{1, 2, 4\}$.

Ak je $d=1$, sú čísla $b$ a~$5b-4$ nesúdeliteľné, a~rovnosť $b(5b-4) =k^2$
tak znamená, že obe tieto čísla (a~špeciálne číslo~$b$) sú druhé mocniny
nejakého prirodzeného čísla.

Ak je $d = 4$, možno písať $b=4r$ a~potom
$$
k^2 = b(5b-4)=4r(20r-4)=16r(5r-1).
$$
Druhými mocninami sú teda nielen čísla $k^2$ a~$16=4^2$, ale aj hodnota
súčinu $r(5r-1)$ dvoch zjavne nesúdeliteľných čísel, a~teda druhými mocninami
sú aj oba činitele. Pre isté prirodzené číslo~$m$ tak platí $r=m^2$,
odkiaľ $b=4r=4m^2$ čiže $b=(2m)^2$.

Nakoniec ukážeme, že zvyšný prípad $d=2$ nastať nemôže. Podobne
ako v~predchádzajúcom prípade odvodíme, že $\frac1{2}{b} = p^2$
a~$\frac1{2}(5b-4) = q^2$ pre nejaké prirodzené čísla $p$ a~$q$.
Elimináciou~$b$ získame rovnosť $q^2 = 5p^2-2$, v~ktorej pravá strana dáva
zvyšok~3 po delení piatimi. Ale druhá mocnina prirodzeného čísla dáva
vždy jeden zo zvyškov 0, 1, 4, a~prípad $d=2$ tak naozaj nemôže
nastať.

Tým sme dokázali, že $b$ je druhou mocninou prirodzeného čísla.

\poznamka
Príklady $(2, 4)$ a $(15, 25)$ dvojíc $(a,b)$ ukazujú, že čísla
zo zadania naozaj existujú.
}

{%%%%%   A-III-5
Dokážeme, že bod~$D$ má rovnakú mocnosť k~obom
kružniciam. Podľa známeho tvrdenia o~{\it chordále\/} je množinou takých
bodov priamka, na ktorej v~prípade pretínajúcich sa kružníc ležia oba
priesečníky. Bod~$D$ tak bude ležať na tejto spojnici, čo je iba iná
formulácia dokazovaného tvrdenia.

%\goodbreak
Označme $P$ priesečník kružnice opísanej trojuholníku $AEC$ s~úsečkou~$BC$ (\obr).
Potom z~tetivového štvoruholníka $AEPC$ získame
$$
|\uhel BEP|=180^\circ - |\uhel PEA| = |\uhel ACP|
$$
a~z~rovnoramennosti trojuholníka $ABC$ ďalej tiež $|\uhel ACP|=|\uhel PBE|$.
Celkom tak máme $|\uhel BEP|=|\uhel PBE|$, a~platí preto $|PB|=|PE|$.
Podobne pre priesečník~$Q$ kružnice opísanej trojuholníku $AFB$ s~úsečkou~$BC$
platí $|QC|=|QF|$. Keďže podľa zadania sú oba uhly $BED$ a~$CFD$ tupé, ležia
body $P$ a~$Q$ vnútri prislúchajúcich úsečiek $DB$ a~$DC$, čo zároveň znamená,
že bod~$D$ leží vnútri oboch prislúchajúcich kruhov.
\inspdf{u5.pdf}%

Všimnime si, že trojuholníky $BED$ a~$CFD$ sú podobné podľa vety~$uu$
a~bodu~$P$ pritom zodpovedá bod~$Q$. Preto
ak označíme $k$ koeficient tejto podobnosti, platí
$$
|DB|\cdot|DQ| = \frac{1}{k}|DC| \cdot k|DP| = |DC|\cdot|DP|.
$$
Bod~$D$ tak má rovnakú mocnosť ku kružniciam opísaným trojuholníkom~$ABF$
a~$AEC$, ako sme chceli dokázať.
}

{%%%%%   A-III-6
Všetky vyhovujúce $4k$-ciferné čísla sú tvaru $2\cdot J$, pričom
$J$ je ľubovoľné $4k$\spojovnik{}ciferné číslo zostavené z~cifier $1$ a~$0$, ktoré
je deliteľné číslom $1\,010=10\cdot101$. Hodnota $P(k)$ tak udáva počet
takých $4k$-ciferných čísel~$J$ zostavených z~cifier $1$ a~$0$,
ktoré končia cifrou nula a~ktoré sú deliteľné číslom~101.

$4k$-ciferné číslo
$J=\overline{c_{4k-1}c_{4k-2}\dots c_{1}c_{0}}$ dáva po delení číslom~101
rovnaký zvyšok ako číslo
$$
S(J)=s_0+10s_1-s_2-10s_3=(s_0-s_2)+10(s_1-s_3), \tag{1}
$$
pričom
$$
\align
s_0&=c_{0}+c_{4}+\dots+c_{4k-4},\quad
s_1=c_{1}+c_{5}+\dots+c_{4k-3},\\
s_2&=c_{2}+c_{6}+\dots+c_{4k-2},\quad
s_3=c_{3}+c_{7}+\dots+c_{4k-1}.
\endalign
$$
Vyplýva to z~toho, že mocniny $10^{4i}$, $10^{4i+1}$, $10^{4i+2}$,
$10^{4i+3}$ dávajú po delení číslom~101 postupne rovnaké zvyšky
ako čísla 1, 10, ${-1}$, ${-10}$, čo je jednoduché overiť indukciou vzhľadom
na celé číslo $i\ge0$.

Pre splnenie podmienky $101\mid S(J)$ podľa~(1) určite stačí, aby
platila silnejšia podmienka
$$
s_0-s_2=0\quad\text{a}\quad s_1-s_3=0. \tag{2}
$$
Pri danom $k$ určíme počet takých čísel $J$ s~ciframi
$c_i\in\{1, 0\}$, pre ktoré okrem~(2) platí $c_{4k-1}=1$ a~$c_0=0$
(aby naozaj išlo o~$4k$-ciferné číslo deliteľné desiatimi).
Ak upravíme rovnosť~(2) s~dosadenými hodnotami $c_{4k-1}$, $c_0$
na tvar
$$
\align
0+c_4+c_8+\dots+c_{4k-4}+(1-c_2)+(1-c_6)+\dots+(1-c_{4k-2}) &=k,\\
c_1+c_5+\dots+c_{4k-3}+(1-c_3)+(1-c_7)+\dots+(1-c_{4k-5})+0&=k,
\endalign
$$
dôjdeme podľa kombinatorického pravidla súčinu k~záveru, že
hľadaný počet je rovný číslu
$$
\binom{2k-1}{k}^{\!2}.
$$
To platí, lebo na ľavých stranách upravených rovností tvoria sčítance
ľubovoľnú permutáciu $k$~jednotiek a~$k$~núl,
pri ktorej je však jedna pozícia nuly (na úplnom začiatku alebo konci) zadaná.
Tým je nerovnosť zo zadania úlohy dokázaná.

V~druhej časti riešenia ukážeme, že rovnosť v~dokázanej nerovnosti
nastane práve vtedy, keď $k\le9$. Pre každé~$k$ totiž vďaka podmienkam
$c_{4k-1}=1$ a~$c_0=0$ súčty $s_i$ spĺňajú nerovnosti
$$
0\le s_0\le k-1,\quad
0\le s_1\le k, \quad
0\le s_2\le k, \quad
1\le s_3\le k.
$$
Podľa nich pre oba rozdiely $d_0=s_0-s_2$ a~$d_1=s_1-s_3$ platí
$-k\le d_i\le k-1$, odkiaľ pre súčet $S(J)$ z~(1)
vyplývajú odhady $-11 k\le S(J)\le11(k-1)$. V~prípade $k\le9$
to zrejme znamená, že $101\mid S(J)$ práve vtedy, keď $S(J)=0$, čiže
$d_0={-10d_1}$. To potom ale nastane jedine tak, že $d_0=d_1=0$,
lebo všeobecné odhady čísla $d_0$ pre $k\le9$ vedú k~nerovnostiam
${-9}\le d_0\le8$, takže $10\mid d_0$ iba pre $d_0=0$.

Ostáva pre každé $k\ge10$ uviesť príklad vyhovujúceho čísla $J$,
pre ktoré je $S(J)$ nenulový násobok čísla $101$, napr. $S(J)={-101}$.
Posledné platí, ak $s_0=s_1=0$, $s_2=1$ a~$s_3=10$. Že možno
také cifry $c_i\in\{1,0\}$ pre každé $k\ge10$ vybrať
(aby pritom platilo $c_{4k-1}=1$ a~$c_0=0$), je zrejmé: za
jednotky zvolíme iba jeden sčítanec v~súčte~$s_2$
a~sčítanec $c_{4k-1}$ a~ľubovoľných deväť ďalších v~súčte~$s_3$.
}

{%%%%%   B-S-1
Ak má rovnica celočíselný koreň~$m$, je číslo $3(m^2+1)$
rovné $am(m+1)$, takže je deliteľné číslom~$m$, ktoré je zrejme
rôzne od nuly. Vzhľadom na nesúdeliteľnosť
čísel $m$ a~$m^2+1$ z~toho vyplýva, že číslo~$m$ je deliteľom čísla~$3$,
takže $m\in\{-3, -1, 1, 3\}$. Do pôvodnej rovnice postupne dosadíme
za $x$ všetky štyri hodnoty~$m$ a~zistíme tak, že celočíselné $a$ vyjde
pre práve dve z~nich. Je to jednak $x={-3}$, ktorému zodpovedá
$a=5$, jednak $x=1$, pre ktoré $a=3$. (Pre $x={-1}$ vychádza $0\cdot a=6$,
pre $x=3$ rovnica $12a=30$ s~neceločíselným koreňom~$a$.)

\odpoved
Rovnica má celočíselný koreň pre $a=3$ alebo $a=5$.

\poznamka
Rovnosť pre celočíselný koreň~$m$ by sme mohli tiež upraviť na tvar
$$am(m+1)=3(m^2+1)=3(m-1)(m+1)+6,$$
odkiaľ vidíme, že číslo $m+1$ delí číslo 6. Táto úvaha vedie k~ôsmim
možnostiam pre číslo~$m$ s~podobnou diskusiou.


\ineriesenie
Danú rovnicu upravíme na štandardný tvar
$$
(a-3)x^2+ax-3=0. \tag1
$$
Pre $a=3$ sa jedná o~lineárnu rovnicu s~celočíselným koreňom $x=1$,
takže $a=3$ vyhovuje. Pre celé $a\ne3$ je rovnica kvadratická a~jej
celočíselný diskriminant
$$
D=a^2+12(a-3)=(a+6)^2-72
$$
musí byť druhou mocninou celého čísla (už aj v~prípade racionálneho koreňa).

Nech $D=n^2$ pre celé nezáporné číslo~$n$. Platí tak
$$
n^2=(a+6)^2-72.
$$
Túto rovnicu upravíme na tvar
$$
(a+n+6)(a-n+6)= (a+6)^2-n^2=72=2^3\cdot3^2.
$$
Oba celočíselné činitele $a-n+6$ a~$a+n+6$ majú zrejme rovnakú paritu
(ich súčet je párny), takže podľa ich súčinu~72 to musia byť dve
párne čísla, pre ktoré navyše platí $a-n+6\le a+n+6$. Nastanú tak pre ne iba možnosti
uvedené v~nasledujúcej tabuľke, v~ktorej súčasne uvádzame aj zodpovedajúce
hodnoty čísel~$a$, $n$ a~zodpovedajúce korene kvadratickej rovnice~(1):
$$
%%
\setbox1=\vbox{\hbox{\strut rovnica nie je}
\hbox{\strut kvadratická}}
%%
\vbox{\offinterlineskip \def\mez{\hskip 7pt plus 1 fil}
\def\gez{\hskip 8pt plus 1 fil} \def\m{\hphantom{-}}
\halign{\vrule\strut\mez$#$\mez\vrule\hskip1pt\vrule&&\mez$#$\gez\vrule\cr
\noalign{\hrule}
a-n+6&-36&-18&-12&2 &4 & 6\cr
a+n+6&-2 &-4&-6 &36&18&12\cr
\noalign{\hrule}
a &-25&-17&-15&13&5 &3\cr
n & \m17& -7&-3 &17&7 &3\cr
\noalign{\hrule}
\hbox{korene (1)}\vrule height11pt depth 5pt width 0pt
& -\dfrac34,-\dfrac17 &-\dfrac35,-\dfrac14
& -\dfrac12,-\dfrac13 &-\dfrac32,\dfrac15
& -3,\dfrac12&\vcenter{\box1}\cr
\noalign{\hrule}
}}¨
$$
Vidíme, že okrem už diskutovaného prípadu $a=3$ má daná rovnica
celočíselný koreň iba v~prípade $a=5$.


\nobreak\medskip\petit\noindent
Za úplné riešenie dajte 6 bodov, z~toho záverečný 1~bod za zhŕňajúcu
odpoveď $a\in\{3, 5\}$. Predchádzajúcich 5~bodov v~závislosti
od zvoleného postupu prideľujte nasledovne.

Pri postupe podľa prvého riešenia dajte 2~body za dôkaz, že
výraz len v~premennej~$m$ delí určené celé číslo (napr. $m\mid3$, $m+1\mid6$,
$m(m+1)\mid 6$,~\dots),
1~bod za uvedenie všetkých možností pre~$m$
a~ďalej nanajvýš 2~body za správne dopočítanie prislúchajúcich hodnôt~$a$.

Pri postupe podľa druhého riešenia dajte 1~bod za diskusiu lineárnej
rovnice v~prípade $a=3$, 1~bod za tvrdenie, že diskriminant kvadratickej
rovnice musí byť kvadrátom celého čísla, 1~bod za nájdenie všetkých
možností pre $D$ ($D\in\{3^2, 7^2, 17^2\}$), 1~bod za zdôvodnenie, prečo
iné hodnoty $D$ nevyhovujú a~1~bod za vylúčenie všetkých možností pre~$a$
okrem $a=5$.

Pozor! Riešiteľ môže získať čiastkové body iba za jeden
postup, čiastkové body podľa rôznych postupov sa nesčítajú.
Len za uhádnutie oboch hodnôt $a\in\{3, 5\}$ dajte 1~bod,
ak je uhádnutá iba jedna hodnota, žiadny bod neprideľujte.

\endpetit
}

{%%%%%   B-S-2
Posudzované stredy kružníc zvonka
\insp{b69.6}%
pripísaných stranám daného štvoruholníka
$ABCD$ označíme $K$, $L$, $M$, $N$ podľa \obr. Našou úlohou je
dokázať, že štvoruholník $KLMN$ je tetivový. Využijeme na to zvyčajným spôsobom
označené vnútorné uhly $\alpha$, $\beta$, $\gamma$, $\delta$ počiatočného
štvoruholníka $ABCD$.
Bod~$K$ je priesečníkom osí vonkajších uhlov tohto štvoruholníka pri
vrcholoch $A$, $B$. Preto
$$
|\angle BAK|=\tfrac12(180^\circ-\alpha)=90^\circ-\tfrac12\alpha,\qquad
|\angle ABK|=\tfrac12(180\st-\beta)=90^\circ-\tfrac12\beta.
$$
Z toho pre veľkosť tretieho vnútorného uhla trojuholníka $ABK$ vyplýva
$$
|\angle AKB|=180^\circ-(90^\circ-\tfrac12\alpha)-
(90^\circ-\tfrac12\beta)=\tfrac12(\alpha+\beta).
$$
Podobne pre stred $M$ kružnice zvonka pripísanej strane~$CD$ štvoruholníka
platí
$$
|\angle CMD|=\tfrac12(\gamma+\delta).
$$

Súčet veľkostí vnútorných uhlov pri protiľahlých vrcholoch $K$ a~$M$
štvoruholníka $KLMN$ je tak
$$
|\angle AKB|+|\angle CMD|=\tfrac12(\alpha+\beta+\gamma+\delta)
=\tfrac12\cdot360\st=180^\circ,
$$
čo je nutná a~postačujúca podmienka na to, aby štvoruholník $KLMN$ bol
tetivový.
Tým je dôkaz ukončený.


\nobreak\medskip\petit\noindent
Za úplné riešenie dajte 6~bodov, z~toho 5~bodov za dôkaz, že súčet dvoch protiľahlých
uhlov štvoruholníka $KLMN$ je 180\st, a~1~bod za následné konštatovanie, že
(práve) také štvoruholníky sú tetivové (namiesto toho je možné uviesť
aj potrebné poznatky o~oblúkoch ako ekvigonálach).

Hodnotenie dôkazu 5~bodmi rozdeľte takto: 1~bod za poznatok, že stredy
kružníc zvonka pripísaných ležia na osiach vonkajších uhlov štvoruholníka $ABCD$,
2~body za potrebné vyjadrenia štyroch uhlov pri obvode $ABCD$ ako
napr. uhla $BAK$ pomocou uhla $\alpha$, 1~bod za vyjadrenie dvoch protiľahlých
uhlov v~$KLMN$ a~1~bod za ich sčítanie na hodnotu~180\st.

Ak riešiteľ ukáže, že tvrdenie platí pre štvoruholníky $ABCD$
konkrétneho tvaru, tak v~prípade štvorcov body nedávajte, v~prípade
kosoštvorcov a~obdĺžnikov nanajvýš 1~bod, vo~všeobecnejších prípadoch
(rovnobežníky a~lichobežníky) nanajvýš 2~body; tieto body sa nesčítajú.

\endpetit
}

{%%%%%   B-S-3
Uvažujme ľubovoľné vyhovujúce rozmiestnenie $k$~veží a~$k$~strelcov.
Dokážme, že platí nerovnosť $k\le5$.

Ak je v~niektorom riadku (resp. stĺpci) šachovnice umiestnená veža,
musí byť jedinou figúrkou tohto riadka (resp. stĺpca). Pre počet~$k$
rozmiestnených veží tak nutne platí $k\le8$ a~navyše je nimi neobsadených celkom
$8-k$ riadkov a~$8-k$ stĺpcov šachovnice. Strelci v~počte~$k$ tak môžu stáť
iba na niektorom z~$(8-k)^2$ políčok, v~ktorých sa tieto riadky a~stĺpce šachovnice
pretínajú. Preto musí platiť nerovnosť
$$
k\le (8-k)^2.
$$
Z toho už vyplýva $k\le 5$, lebo odvodená nerovnosť neplatí pre žiadne
$k\in\{6, 7, 8\}$, ako sa ľahko presvedčíme dosadením.

Ako vidíme z~nasledujúceho obrázka, 5~veží a~5~strelcov sa dá rozmiestniť
na šachovnicu tak, aby sa navzájom neohrozovali, preto najväčšie
možné~$k$ danej vlastnosti je $k=5$.
\insp{b69.7}%

\poznamka
Odvodenie nerovnosti $k\le5$ možno podať stručnejšie takto: najskôr vylúčime
rovnako ako v~podanom riešení hodnotu $k=6$ (6~strelcov by muselo stáť na 4~políčkach,
ktoré ležia v~dvoch riadkoch a~dvoch stĺpcoch neobsadených 6~vežami) a~potom konštatujeme,
že ak by existovalo vyhovujúce rozmiestnenie pre $k>6$, odobratím ľubovoľných
$k-6$ veží a~$k-6$ strelcov by sme dostali vyhovujúce rozmiestnenie pre $k=6$.


\nobreak\medskip\petit\noindent
Za úplné riešenie dajte 6~bodov. Z~toho 1~bod za vylúčenie
možností $k>6$ a~2~body za vylúčenie možnosti $k=6$. Napokon 3~bodmi
ohodnoťte konštrukciu príkladu správneho rozostavenia pre $k=5$.
Za konštatovanie, že k~nájdenému príkladu pre $k=5$ už sa nedá žiadna
ďalšia veža pridať, ale ďalšie body nedávajte~-- na úplné vyriešenie úlohy
by bolo potrebné takto otestovať všetky vyhovujúce rozmiestnenia pre $k=5$.
Ak je uvedený správny príklad rozostavenia iba pre $k=4$, dajte 1~bod.
\endpetit
}

{%%%%%   B-II-1
...}

{%%%%%   B-II-2
...}

{%%%%%   B-II-3
...}

{%%%%%   B-II-4
...}

{%%%%%   C-S-1
Pokúsme sa najskôr vydedukovať nejaký odhad hodnoty čísla~$n$,
ktoré by mohlo spĺňať zadanú rovnicu. Na jej ľavej strane máme
súčet dvoch prirodzených čísel $n$ a~$s(n)$, a~preto obe musia
byť menšie ako 2\,019. Presnejšie, hodnota hľadaného čísla
$n = 2\,019-s(n)$ bude nanajvýš 2\,018, keďže $s(n)$ je prirodzené
číslo. Potom vieme, že ciferný súčet čísla $n \le 2\,018$ je
nanajvýš $1+9+9+9 = 28$, a~preto samotné číslo~$n$ musí byť niekde
medzi 2\,018 a~$2\,019-28 = 1\,991$.

Teraz stačí vyskúšať, pre ktoré z~hodnôt $n \in\{2\,018, 2\,017,
\dots, 1\,991\}$ vyjde rovnosť $n+s(n) = 2\,019$:
$$
\let\\=\cr \let\enspace\quad \let\bf\tenb
\centerline{%
\vtop
{\kern-10pt\halign{\strut\hss#\unskip\hss&&\enspace\hss#\unskip\hss\cr
\noalign{\hrule}
$n$&$s(n)$&$n+s(n)$ \\
\noalign{\hrule}
2\,018&11&2\,029 \\
2\,017&10&2\,027 \\
2\,016&9&2\,025 \\
2\,015&8&2\,023 \\
2\,014&7&2\,021 \\
{\bf 2\,013}&{\bf 6}&{\bf 2\,019} \\
2\,012&5&2\,017 \\
2\,011&4&2\,015 \\
2\,010&3&2\,013 \\
\noalign{\hrule}
}}\hfil\vtop
{\kern-10pt\halign{\strut\hss#\unskip\hss&&\enspace\hss#\unskip\hss\cr
\noalign{\hrule}
$n$&$s(n)$&$n+s(n)$ \\
\noalign{\hrule}
2\,009&11&2\,020\\
2\,008&10&2\,018\\
2\,007&9&2\,016 \\
2\,006&8&2\,014 \\
2\,005&7&2\,012 \\
2\,004&6&2\,010 \\
2\,003&5&2\,008 \\
2\,002&4&2\,006 \\
2\,001&3&2\,004 \\
2\,000&2&2\,002 \\
\noalign{\hrule}
}}\hfil\vtop
{\kern-10pt\halign{\strut\hss#\unskip\hss&&\enspace\hss#\unskip\hss\cr
\noalign{\hrule}
$n$&$s(n)$&$n+s(n)$ \\
\noalign{\hrule}
1\,999&28&2\,027 \\
1\,998&27&2\,025 \\
1\,997&26&2\,023 \\
1\,996&25&2\,021 \\
{\bf 1\,995}&{\bf 24}&{\bf 2\,019} \\
1\,994&23&2\,017 \\
1\,993&22&2\,015 \\
1\,992&21&2\,013 \\
1\,991&20&2\,011 \\
\noalign{\hrule}
}}}
$$
Rozoberanie všetkých 28 možností od 1\,991 po 2\,018 (prípadne v~inom
podobnom intervale) možno rôznymi spôsobmi skrátiť.

Prvým je napríklad pozorovanie, že ak zmenšíme číslo~$n$
o~jednotku a~neprejdeme pritom cez desiatku, zmenší sa hodnota súčtu
$n+s(n)$ o~2 (napríklad pre $n = 2\,005$ je $n+s(n) = 2\,012$, pre
$n = 2\,004$ vyjde $2\,010$ a pod.).
Preto má skúmaný súčet predpísanú hodnotu 2\,019 pre nanajvýš jedno~$n$
v~každej desiatke prirodzených čísel, ktoré sa líšia iba na mieste
jednotiek, a~my ho dokonca môžeme určiť z~hodnoty súčtu pre jediné číslo
tejto desiatky, napr. pre to, ktoré končí cifrou~9. Podľa predchádzajúcich odhadov
tak budeme potrebovať hodnoty $n+s(n)$ iba pre~$n$ rovné 2\,019,
2\,009 a~1\,999.

Pre $n = 2\,019$ dostávame
$n+s(n) = 2\,031 = 2\,019+2 \cdot 6$, a~preto medzi číslami od 2\,010 po 2\,019
vyhovuje iba číslo $2\,019-6 = 2\,013$, ktoré je naozaj
jedným z~riešení danej úlohy (skúška vďaka vypozorovanému klesaniu súčtu o~hodnotu~2
nie je nutná).

Pre $n = 2\,009$ dostávame
$n+s(n) = 2\,020$, ktoré je na rozdiel od 2\,019 párne, a~tak medzi
číslami od 2\,000 po 2\,009 by sme riešenia hľadali márne.

Napokon pre $n = 1\,999$ dostávame $n+s(n) = 2\,027 = 2\,019+2 \cdot 4$, a~preto
vyhovuje iba číslo $1\,999-4 = 1\,995$, ktoré je druhým riešením.

\medskip
Iný spôsob, ako možno eliminovať počet rozoberaných možností, je
uvedomiť si, že číslo~$n$ dáva po delení tromi rovnaký zvyšok
ako jeho ciferný súčet~$s(n)$. Ak tento zvyšok označíme ako~$d$,
bude číslo $n+s(n) = 2\,019$ po delení tromi dávať zvyšok $d+d = 2d$.
Ciferný súčet čísla $2\,019$ je~$12$, preto aj číslo $2\,019$ je
deliteľné tromi, a~preto zvyšok čísla~$2d$ po delení tromi je nula,
a~teda $d = 0$. Inak povedané, hľadané číslo~$n$ je deliteľné tromi.
V~tomto prípade by sme potrebovali preveriť iba 9~čísel medzi 1\,991 a~2\,018,
ktoré sú deliteľné tromi, \tj. 9~možností $n \in\{2\,016, 2\,013, \dots, 1\,992\}$.

\medskip
Predchádzajúci spôsob eliminácie možností (využitím zvyšku po
delení tromi) môžeme ešte vylepšiť, ak si uvedomíme, že
podobne možno použiť aj deliteľnosť deviatimi. Platí, že číslo~$n$
a~jeho ciferný súčet~$s(n)$ dávajú rovnaký zvyšok po delení
deviatimi. Označme tento zvyšok~$r$. Číslo 2\,019 dáva
po delení deviatimi zvyšok~$3$ (\tj.~zvyšok po delení čísla
$2+0+1+9 = 12$ po delení deviatimi). Z~rovnice $2\,019 = n+s(n)$ potom
podobne ako v~predchádzajúcom odseku vyplýva, že $n$~dáva po
delení deviatimi zvyšok~6, keďže jeho dvojnásobok má dávať
zvyšok~3. Čísla od 1\,991 po 2\,018, ktoré dávajú zvyšok~3 po
delení deviatimi, sú iba tri, a~to $n \in\{2\,013, 2\,004, 1\,995\}$.

\odpoved
Úloha má dve riešenia, a~to $n = 2\,013$ a~$n = 1\,995$.

\ineriesenie
Ak by hľadané číslo~$n$ bolo nanajvýš trojciferné, bol by súčet
$n+s(n)$ nanajvýš $999+(9+9+9) <2\,019$. Z~druhej strany číslo~$n$
nemôže byť päť- a~viacciferné, keďže potom by bolo
$n+s(n)> 10\,000$. Označme cifry hľadaného štvorciferného čísla~$n$
ako $a$, $b$, $c$, $d$, \tj. $n = 1\,000a+100b+10c+d$, pričom $a\ne0$.
Danú rovnicu potom môžeme upraviť nasledovne:
$$
\align
(1\,000a+100b+10c+d)+(a+b+c+d)=&2\,019 ,\\
1\,001a+101b+11c+2d=&2\,019. \tag1
\endalign
$$
Keďže $0 \le a, b, c, d \le 9$, zrejme $a\le 2$, inak by bola
ľavá strana rovnice~(1) aspoň~3\,003.
Keďže $a\ne 0$, rozoberieme dve možnosti pre $a\in \{1,2\}$.

\item{$\triangleright$}$a= 1$:\hfil\break
Potom $101b+11c+2d = 2\,019-1\,001 = 1\,018$. Pre $b \le 8$ by bola ľavá strana
nanajvýš $101 \cdot 8+11 \cdot 9+2 \cdot 9 = 925 <1\,018$, preto $b = 9$, čo po
dosadení dáva $11c+2d = 1\,018-909 = 109$. Ďalej ak by bolo $c \le 8$, bola by
ľavá strana nanajvýš $88+18 = 106 <109$, preto môže byť jedine $c = 9$, pre ktoré
dopočítame $d = {(109-99) / 2} = 5$. Pre $a= 1$ dostávame jediné
riešenie $n = 1\,995$.

\item{$\triangleright$}$a= 2$:\hfil\break
Potom $101b+11c+2d = 2\,019-2\,002 = 17$. Zrejme $b = 0$, teda $11c+2d = 17$
a~$c \le 1$. Keďže pravá strana je nepárna, musí byť aj $c$
nepárne (tým sme vylúčili $c = 0$), teda $c = 1$, pre ktoré
dopočítame $d = (17-11) / 2 = 3$. Pre $a= 2$ dostávame tiež jediné
riešenie $n = 2\,013$.

\odpoved
Úloha má dve riešenia $n = 2\,013$ a~$n = 1\,995$.


\nobreak\medskip\petit\noindent
Ak riešiteľ postupuje podľa prvého
riešenia a~ohraničí množinu hodnôt $n$ zdola aj zhora tak, že
ostane len vyskúšať nanajvýš 30 možností, ale nenájde obe
správne riešenia, dajte 4~body (po 2~bodoch za odhad čísla $n$
zhora aj zdola). Zvyšné 2~body rozdeľte po 1~bode za každé
nájdené správne číslo~$n$.

Ak riešiteľ postupuje podľa druhého riešenia, dajte 1~bod za
zostavenie rovnice $1\,001a+101b+11c+2d = 2\,019$ a~ďalší bod, ak
sa riešiteľ dostane ku skúmaniu $a\in \{1,2\}$. Ďalšie dva body
dajte, ak riešiteľ ďalej systematicky vylučuje rôzne hodnoty $b$,
$c$, $d$ a~posledné 2~body dajte iba v~prípade, že úlohu správne
dokončí a~nájde obe čísla 1\,995 aj~2\,013.

Akokoľvek riešiteľ postupuje, ak nájde iba jedno z~riešení bez
toho, že by vylúčil všetky ostatné možnosti, dajte 1~bod, a~ak
uhádne obe riešenia 1\,995 aj 2\,013 bez ďalšej analýzy, dajte 2~body.
\endpetit
}

{%%%%%   C-S-2
Označme čísla v~rohoch tabuľky $a$, $b$, $c$, $d$ (zľava doprava, zhora
nadol). Týmito štyrmi číslami sú jednoznačne určené všetky
ostatné čísla tabuľky, pretože postupne možno dopočítať čísla
medzi nimi a~nakoniec aj číslo $a+b+c+d$ uprostred tabuľky.
$$
\tabulkacsLXIX

&a& &b

& & &{}

&c& &d


\quad
\Longrightarrow
\quad
\sqwcsLXIX 2\sqwcsLXIX
\tabulkacsLXIX

&a&a+b&b

&a+c&& b+d

&c&c+d&d


\quad
\Longrightarrow
\quad
\sqwcsLXIX 0\sqwcsLXIX
\tabulkacsLXIX

&a&a+b&b

&a+c&a+b+c+d&b+d

&c&c+d&d


$$

Pre hodnoty $a~= 1$, $b = 3$, $c = 6$ a~$d = 2$ dostaneme tabuľku rôznych čísel
$$
\tabulkacsLXIX

&1&4&3

&7&12&5

&6&8&2


$$
s~číslom $a+b+c+d = 12$ uprostred.

Teraz ukážeme, že uprostred tabuľky nemôže byť menšie číslo ako~12.
Menšie číslo ako~12 sa dá ako súčet štyroch rôznych prirodzených
čísel dostať iba dvoma spôsobmi: ako $1+2+3+4$ alebo $1+2+3+5$.

V~oboch prípadoch sa
medzi číslami vpísanými do rohov tabuľky nachádzajú čísla 1, 2
aj~3. Vyskúšajme, či ich môžeme vpísať do rohov tabuľky tak, aby sme
ju vedeli celú vyplniť požadovaným spôsobom.

Čísla 1 a~2 nesmú byť napísané
v~tom istom riadku či stĺpci, pretože by potom medzi nimi bolo napísané
číslo~3 ako ich súčet, a~my potrebujeme mať číslo~3
v~niektorom rohu tabuľky. Preto musia byť čísla 1 a~2
v protiľahlých rohoch tabuľky.
Pre číslo~3 máme už
iba dve možné rohové políčka tabuľky, a~nech už ho vpíšeme
do ktoréhokoľvek z~nich, budeme musieť medzi čísla 3 a~1 vpísať číslo~4
a~medzi čísla 3 a~2 číslo~5, teda nebudeme môcť mať v~poslednom
rohu ani číslo~4, ani číslo~5. Do rohov tabuľky preto nedokážeme
napísať ani čísla 1, 2, 3, 4, ani 1, 2, 3, 5, a~preto číslo
$a+b+c+d$ uprostred tabuľky spĺňajúce podmienky zadania má vždy
hodnotu aspoň~12.

\ineriesenie
Označme čísla v~rohoch tabuľky rovnako ako v~predošlom
riešení a~doplnením ostatných čísel v~tabuľke vidíme, že
súčet~$S$ všetkých čísel v~tabuľke je
$$
\align
S=& a+(a+b)+b+(a+c)+(a+b+c+d)+(b+d)+c+(c+d)+d =\\
=&4 (a+b+c+d),
\endalign
$$
čo je číslo deliteľné štyrmi. Presnejšie, je to štvornásobok
čísla napísaného uprostred tabuľky. Nájsť najmenšie možné
číslo napísané uprostred tabuľky je teda to isté ako nájsť
štvrtinu najmenšieho možného súčtu~$S$ všetkých čísel napísaných
v~tabuľke.

Každé z~deviatich čísel v~tabuľke je prirodzené
a~napísané nanajvýš raz, preto súčet~$S$ všetkých čísel v~tabuľke
bude aspoň $S\ge 1+2+\cdots+9 = 45$, čo je súčet deviatich najmenších
prirodzených čísel. Už vieme, že tento súčet~$S$ musí byť
deliteľný štyrmi, a~preto najmenší súčet všetkých čísel
v~tabuľke musí byť aspoň~$48$ (najmenšie číslo deliteľné
štyrmi, ktoré spĺňa podmienku $S\ge45$). Potom ale najmenšia možná
hodnota čísla uprostred tabuľky je $S/ 4 = 48/4 = 12$.\footnote{Tento odhad čísla
$a+b+c+d$ uprostred tabuľky možno získať aj bez úvahy o~celkovom súčte: Súčet ôsmich
ostatných čísel, ktorý je aspoň $1+2+\dots+8=36$, je rovný $3(a+b+c+d)$,
odkiaľ $a+b+c+d\ge12$.}
Vyhovujúcu tabuľku nájdeme skúšaním a~môže to byť napríklad tá z~prvého riešenia.

\poznamka
Počet všetkých vyhovujúcich tabuliek s~číslom~12 uprostred je~8. Číslo~12 sa dá
napísať ako súčet štyroch rôznych prirodzených čísel iba
dvoma spôsobmi ako
$6+3+2+1$ a~$5+4+2+1$. Dá sa pritom overiť, že druhá možnosť
nevedie k~tabuľke s~rôznymi číslami a~že prvá možnosť vedie
k~správnemu vyplneniu tabuľky, len keď sú čísla 1 a~2 v~protiľahlých
rohoch. Tým pádom dokážeme spočítať počet vyhovujúcich tabuliek
takto: Pre číslo~1 máme štyri možnosti, kam ho umiestniť, a~potom už je
poloha čísla~2 jednoznačne určená. Následne pre číslo~3 máme
2~možnosti, číslo~6 je v~poslednom voľnom rohu tabuľky, a~zvyšné
čísla sú samozrejme určené tiež jednoznačne. Dokopy tak existuje
8~možných príkladov pre tabuľku s~číslom~12 uprostred (každý sa
pritom dá dostať z~ktoréhokoľvek iného postupnými výmenami
krajných riadkov resp. stĺpcov).


\nobreak\medskip\petit\noindent
Za úplné riešenie úlohy dajte 6 bodov, z~toho 2~body za príklad tabuľky
s~číslom~12 uprostred a~4~body za dôkaz, že číslo uprostred tabuľky
musí byť aspoň~12. Slabšie výsledky oceňte takto: 1~bod za príklad tabuľky
s~číslom~13 uprostred; za zdôvodnenie, že číslo uprostred musí byť aspoň 10,
resp. 11, dajte~2, resp. 3~body.

Ak riešiteľ postupuje pri odhade čísla uprostred podľa druhého riešenia,
dajte 2~body za tvrdenie, že súčet všetkých čísel v~tabuľke je
štvornásobkom čísla uprostred. Ďalšie 2~body za ukázanie, že
súčet musí byť aspoň 48, a~teda najmenšie číslo uprostred je 12
(inak možno získať 4 body aj za postup z~poznámky pod čiarou).
\endpetit

}

{%%%%%   C-S-3
Hľadaný pravouhlý trojuholník označme $ABC$ tak, aby pravý uhol bol pri vrchole~$C$,
\insp{c69.5}%
dĺžky jeho strán označme štandardne ako $|BC| = a$, $|AC| = b$ a~$|AB| = c$. Navyše
označme $I$ stred jeho vpísanej kružnice a~$T$, $U$ a~$V$ postupne jej body dotyku
so stranami $BC$, $AC$ a~$AB$ (\obr).

Štvoruholník $CUIT$ má vnútorné uhly pri vrcholoch $C$, $T$ aj~$U$
pravé a~navyše zo zadania vyplýva, že $|IT| = |IU| = 2$, preto je to štvorec.
Z~rovnosti $|CT| = 2$ potom máme $|BT| = a-2$ a~z~rovnosti úsekov dotyčníc
z~vrcholu~$B$ ku vpísanej kružnici máme $|BV| = |BT| = a-2$. Podobne
získame rovnosti $|AV| = |AU| = b-2$. Veľkosť prepony~$AB$ tak môžeme
vyjadriť jednak ako $|BV|+|AV| = (a-2)+(b-2) = {a+b-4}$, jednak z~Pytagorovej
vety ako $|AB| = \sqrt {a^2+b^2}$. Z~toho dostávame rovnicu, ktorú
umocníme a~postupne upravíme:
$$
\abovedisplayskip\abovedisplayshortskip
\align
|AB| = a+b-4=&\sqrt {a^2+b^2}, \quad\big |{}^2 \tag1 \\
(a+b-4)^2=&a^2+b^2 ,\\
a^2+b^2+2ab-8a-8b+16=&a^2+b^2 ,\\
ab-4a-4b+8=&0 ,\\
(a-4) (b-4)=&8. \tag2
\endalign
$$

Číslo 8 sa dá rozložiť na súčin dvoch celých čísel ako
$$
8 = 8 \cdot 1 = 4 \cdot 2 = (-1) \cdot (-8) = (-2) \cdot (-4).
$$
Bez ujmy na všeobecnosti môžeme predpokladať, že $a\ge b$, čiže
$a-4 \ge b-4$. Potom máme nasledujúce štyri možnosti (hľadáme ale
iba kladné riešenia, pretože sa jedná o~dĺžky strán trojuholníka):\footnote{Keďže celá
vpísaná kružnica leží vnútri pravouhlého trojuholníka $ABC$
a~jej priemer je~4, sú obe jeho odvesny väčšie ako~4, a~tak sme si
mohli rozoberanie dvoch prípadov ušetriť.}

\item{$\triangleright$}$a-4 = 8$ a~$b-4 = 1$, odkiaľ $a~= 12$, $b = 5$, $c = a+b-4 = 13$;
\item{$\triangleright$}$a-4 = 4$ a~$b-4 = 2$, odkiaľ $a~= 8$, $b = 6$, $c = a+b-4 = 10$;
\item{$\triangleright$}$a-1 =-1$ a~$b-4 =-8$, odkiaľ $a~= 0$, $b = 4$, čo úlohe nevyhovuje;
\item{$\triangleright$}$a-1 =-2$ a~$b-4 =-4$, odkiaľ $a~=-1$, $b = 0$, čo úlohe nevyhovuje.

Nakoniec ostáva preveriť, že vpísaná kružnica oboch pravouhlých trojuholníkov
s~dĺžkami strán $5$, $12$, $13$ a~$6$, $8$, $10$ má naozaj polomer
dĺžky~2. Ak napíšeme jej polomer~$r$ do \obrr1{} všade namiesto čísla~2,
dostaneme rovnosť $c=a+b-2r$, z~ktorej vyplýva vzorec $r=\frac12(a+b-c)$,
podľa ktorého pre obe trojice $(5,12,13)$ a $(6,8,10)$ naozaj vyjde $r=2$.

\odpoved
Hľadaný trojuholník má strany dĺžok 5, 12, 13 alebo 6, 8, 10.

\ineriesenie
Použijeme rovnaké označenie strán trojuholníka ako v~predošlom
riešení, \tj. $a$, $b$ budú odvesny a~$c$ prepona hľadaného
trojuholníka $ABC$ s~obsahom~$S$. Obsah každého trojuholníka možno
vyjadriť vzorcom $S= s\cdot r$, pričom $s$~označuje polovicu jeho
obvodu a~$r$ polomer jeho vpísanej kružnice. Navyše obsah
pravouhlého trojuholníka možno vyjadriť aj ako polovicu súčinu
jeho odvesien. Spolu tak dostaneme rovnicu, ktorú s~využitím
daného polomeru $r = 2$, pytagorejskej rovnosti
$c^2 = a^2+b^2$ a~nerovnosti $ab> 0$ upravíme na tvar~(2):
$$
\align
S~= \frac {(a+b+c) \cdot r} 2=&\frac {ab} 2 ,\\
(a+b+c) \cdot 2=&ab ,\\
2c=&ab-2 (a+b) , \qquad\big |{}^2 \\
4c^2=&\left (ab-2 (a+b) \right)^2 ,\\
4 (a^2+b^2)=&a^2b^2-4ab (a+b)+4 (a^2+2ab+b^2) ,\\
0=&ab (ab-4 (a+b)+8) ,\\
0=&ab-4a-4b+8 ,\\
8=&(a-4) (b-4). \tag3
\endalign
$$
A~ďalej postupujeme ako v~prvom riešení.

\ineriesenie
Ľahko spočítame, že rovnoramenný pravouhlý trojuholník s~polomerom kružnice
vpísanej $r=2$ má odvesny dĺžky $(r+r\sqrt 2)\cdot \sqrt 2 = 4+2\sqrt2 < 7$
(lebo $(2\sqrt 2)^2=8<9=(7-4)^2$).
\insp{c69.6}%

Akýkoľvek pravouhlý trojuholník s~$r=2$ musí mať teda jednu odvesnu kratšiu ako~7.
Na druhej strane obe odvesny musia byť určite dlhšie ako priemer kružnice
vpísanej, teda~4. Pri označení ako vyššie tak pre dĺžku kratšej odvesny máme dve
možnosti $a=5$, $a=6$, pričom pre každú z~nich existuje jednoznačne určená
(reálna) dĺžka druhej odvesny~$b$, pre ktorú vyjde $r=2$. Tipneme si
celočíselné trojuholníky $(a,b,c)\in\{(5,12,13),(6,8,10)\}$ a~dosadením do
vzorca $r=ab/(a+b+c)$ alebo $r=\frac12(a+b-c)$ sa presvedčíme, že oba spĺňajú $r=2$,
a~sú teda riešením.


\nobreak\medskip\petit\noindent
Za úplné riešenie dajte 6 bodov.

Ak riešiteľ postupuje geometrickým spôsobom opísaným v~prvom
riešení, dajte 1~bod za napísanie rovnosti $|CT| = |CU| = 2$.
Druhý bod dajte za vyjadrenie dĺžok strán $|BT| = |BV| = a-2$
a~$|AU| = |AV| = b-2$. Tretí bod za rovnosť~(1) a~štvrtý bod
za jej úpravu na tvar~(2). Zvyšné 2~body
pripadajú na jej vyriešenie (po bode za každé riešenie).
Absenciu skúšky polomeru opísanú v~závere prvého riešenia nepenalizujte.

Ak riešiteľ postupuje algebraickým spôsobom opísaným v~druhom
riešení, dajte po 1~bode za uvedenie každého z~dvoch vzorcov pre výpočet
obsahu trojuholníka a~za vytvorenie rovnosti medzi
nimi s~dosadením $r = 2$ dajte tretí bod. Štvrtý bod dajte, ak
sa riešiteľ dostane až k~rovnici~(3). Zvyšné 2~body
pripadajú na jej vyriešenie podobne ako v~prvom riešení.

Ak sa riešiteľovi podarilo uhádnuť strany hľadaného
pravouhlého trojuholníka (bez zdôvodnenia, že kratšia odvesna je kratšia ako 7 a~dlhšia ako 4) a~dopočítať k~nim nejakým spôsobom
veľkosť polomeru vpísanej kružnice (napríklad zo vzťahu
$r = \frac12(a+b-c)$), za každé z~dvoch riešení $(6, 8, 10)$, $(5, 12, 13)$
dajte po 1~bode.
\endpetit}

{%%%%%   C-II-1
...}

{%%%%%   C-II-2
...}

{%%%%%   C-II-3
...}

{%%%%%   C-II-4
...}

{%%%%%   vyberko, den 1, priklad 1
...}

{%%%%%   vyberko, den 1, priklad 2
...}

{%%%%%   vyberko, den 1, priklad 3
...}

{%%%%%   vyberko, den 1, priklad 4
...}

{%%%%%   vyberko, den 2, priklad 1
...}

{%%%%%   vyberko, den 2, priklad 2
...}

{%%%%%   vyberko, den 2, priklad 3
...}

{%%%%%   vyberko, den 2, priklad 4
...}

{%%%%%   vyberko, den 3, priklad 1
...}

{%%%%%   vyberko, den 3, priklad 2
...}

{%%%%%   vyberko, den 3, priklad 3
...}

{%%%%%   vyberko, den 3, priklad 4
...}

{%%%%%   vyberko, den 4, priklad 1
...}

{%%%%%   vyberko, den 4, priklad 2
...}

{%%%%%   vyberko, den 4, priklad 3
...}

{%%%%%   vyberko, den 4, priklad 4
...}

{%%%%%   vyberko, den 5, priklad 1
...}

{%%%%%   vyberko, den 5, priklad 2
...}

{%%%%%   vyberko, den 5, priklad 3
...}

{%%%%%   vyberko, den 5, priklad 4
...}

{%%%%%   trojstretnutie, priklad 1
...}

{%%%%%   trojstretnutie, priklad 2
...}

{%%%%%   trojstretnutie, priklad 3
...}

{%%%%%   trojstretnutie, priklad 4
...}

{%%%%%   trojstretnutie, priklad 5
...}

{%%%%%   trojstretnutie, priklad 6
...}

{%%%%%   IMO, priklad 1
...}

{%%%%%   IMO, priklad 2
...}

{%%%%%   IMO, priklad 3
...}

{%%%%%   IMO, priklad 4
...}

{%%%%%   IMO, priklad 5
...}

{%%%%%   IMO, priklad 6
...}

{%%%%%   MEMO, priklad 1
...}

{%%%%%   MEMO, priklad 2
...}

{%%%%%   MEMO, priklad 3
...}

{%%%%%   MEMO, priklad 4
...}

{%%%%%   MEMO, priklad t1
...}

{%%%%%   MEMO, priklad t2
...}

{%%%%%   MEMO, priklad t3
...}

{%%%%%   MEMO, priklad t4
...}

{%%%%%   MEMO, priklad t5
...}

{%%%%%   MEMO, priklad t6
...}

{%%%%%   MEMO, priklad t7
...}

{%%%%%   MEMO, priklad t8
...} 