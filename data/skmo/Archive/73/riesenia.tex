{%%%%%   A-I-1
a) Ukážeme, že pre $k=5$ môžu sympatie na párty dopadnúť tak,
že hľadaný pár neexistuje. Predpokladajme, že chlapcov možno
rozdeliť na dve päťčlenné skupiny $A$ a $B$ a dievčatá na dve
päťčlenné skupiny $C$ a $D$ tak, že platí: všetkým chlapcom z~$A$
sa páčia práve dievčatá z~$C$, všetkým chlapcom z~$B$ práve dievčatá z~$D$,
všetkým dievčatám z~$C$ sa páčia práve chlapci z~$B$ a všetkým dievčatám
z~$D$ práve chlapci z~$A$ (poz. obrázok).
Vtedy sa naozaj každému páči práve $k=5$ osôb opačného pohlavia,
avšak neexistuje jediný pár so vzájomnými sympatiami.
\medskip
\centerline{\setbox9=\hbox{\epsfboxsc{a73i.11}{.8333}}%
             \rlap{\raise .7\ht9\hbox{Chlapci}}%
             \raise .24\ht9\hbox{Dievčatá}\hskip 12pt
             \box9}
\medskip

b) Dokážeme, že pre $k=6$ dvojica požadovanej vlastnosti
existuje.

V~prípade $k=6$ je počet dvojíc opačného pohlavia,
v ktorých sa chlapcovi páči dievča, rovný číslu
$10k = 60$. Taktiež je rovný 60 počet dvojíc, v~ktorých
sa dievčaťu páči chlapec. Tieto dve skupiny dvojíc
nemôžu mať prázdny prienik, pretože $60+60=120$ a všetkých dvojíc
opačného pohlavia je iba $10\cdot10=100$. Nutne
tak existuje pár, v ktorom sa obom páči ten
druhý.\fnote{Z našej úvahy vyplýva,
že takýchto dvojíc existuje aspoň $120 - 100 = 20$.
V~úlohe D1 ukážeme, že ich môže byť práve 20.}

\poznamka
V~riešení časti a) sme uvažovali množinu všetkých dvojíc
opačného pohlavia a dve jej podmnožiny: množinu $A$
tých dvojíc, v~ktorých sa dievča páči chlapcovi, a~množinu~$B$ tých dvojíc, v~ktorých sa chlapec páči dievčaťu.
Podľa známeho Vennovho diagramu platí
$$
|A\cap B|=|A|+|B|-|A\cup B|,
\tag1
$$
kde $|X|$ označuje počet prvkov množiny $X$. V~prípade $k=6$ je
$|A|=|B|=60$, takže z~odhadu $|A\cup B|\le100$
podľa \thetag1 vidíme, že $A\cap B$ je nielen neprázdna množina
(ako sme mali dokázať), ale že má dokonca aspoň 20 prvkov
(ako je uvedené v~poznámke pod čiarou).

\ineriesenie
Časť b) možno vyriešiť aj nasledovne.
Ak v~prípade $k=6$ rozdá každý z 10
chlapcov po jednej ruži tým dievčatám, ktoré sa mu páčia,
bude všetkým 10~dievčatám rozdaných celkom $60$ ruží.
Jedno dievča tak dostane priemerne $60 : 10 = 6$~ruží,
niektorá z nich preto dostane aspoň 6 ruží.
Keďže sa jej páči 6 z 10 chlapcov,
musí aspoň 2~ruže dostať od chlapcov, ktorí sa jej páčia.

\návody
{\everypar{}
\smallskip
V~nasledujúcich úlohách predpokladáme, že na danej párty sú
aspoň dve osoby a~známosti jej účastníkov sú vzájomné. To
sa však netýka sympatií z~návodnej úlohy~N1 a dopĺňajúcej
úlohy D1.
\smallskip
}

Pri stole sedia traja chlapci a štyri dievčatá. Každému chlapcovi sa páčia
tri dievčatá, každému dievčaťu len jeden chlapec. Existuje medzi nimi vždy dvojica
opačného pohlavia, v~ktorej sa obom páči ten druhý?
[Áno. Všetkých dvojíc je 12. Zvážte, v koľkých z nich sa a)~chlapcovi páči dievča, b)~dievčaťu
páči chlapec. Alebo podľa celkového počtu chlapčenských sympatií
dokážte, že niektoré dievča sa páči všetkým trom chlapcom
(použitím Dirichletovho princípu alebo sporom).]

Na párty sa každého z~návštevníkov spýtame, koľko
ostatných návštevníkov pozná. Ukážte, že ak ich odpovede
sčítame, vyjde vždy párne číslo.
[Návod: Koľkokrát sme počítali
každú známosť?]

Ukážte, že na každej párty možno nájsť aspoň dvoch účastníkov,
ktorí tam majú rovnaký počet známych.
[Na párty s~$n$ účastníkmi
je počet známych každého jedno z~$n$ čísel $0,1,\dots,n-1$.
Aspoň dva z týchto $n$ počtov musia byť rovnaké, pretože je vylúčené,
aby sa jeden počet rovnal číslu 0 a iný číslu $n-1$,
teda rôznych počtov je najviac $n-1$.]

\D
Môže sa na párty zo súťažnej úlohy v~prípade $k=6$ stať, že
bude existovať práve 20 párov, v ktorých sa obom páči ten druhý?
[Áno. Popíšeme jeden z mnohých možných príkladov.
Päť štvorčlenných skupín po 2~chlapcoch a 2~dievčatách
rozostavme po obvode kruhu a~vyberme \uv{kladný} smer jeho
prechádzania. Predpokladajme, že každému chlapcovi sa
páčia práve 2~dievčatá z~jeho skupiny a ďalšie 4~dievčatá z dvoch skupín,
ktoré sú od jeho skupiny najbližšie v~kladnom smere, a že podobne
každému dievčaťu sa páčia práve 2~chlapci z~jej skupiny a~ďalší
4~chlapci z dvoch skupín, ktoré sú od jej skupiny najbližšie
v~kladnom smere. Potom všetky páry so vzájomnými sympatiami
sú časťami vytvorených piatich štvoríc a ich počet tak je
$5\cdot2\cdot2=20$.]

Na párty sa každý účastník pozná práve s troma ďalšími. Ukážte, že
počet účastníkov párty je párny. Ďalej uveďte príklady známostí na takých
párty s~6, 8 a~2024 účastníkmi.
[Označte $n$ počet
účastníkov a uvážte, že dvojnásobok počtu všetkých známostí
na párty je párne číslo, ktoré sa rovná $3n$.
Pre $n=6$ si predstavte, že
účastníci sú rozmiestnení na obvode kruhu a že sa poznajú práve tí,
ktorí spolu nesusedia (možné sú aj iné príklady).
Pre $n$ rovné 8, 2024 alebo všeobecne $n=4k$
uvážte napríklad situáciu, keď účastníci
sú rozdelení do $k$ štvoríc,
pričom navzájom sa poznajú práve ľudia z rovnakej štvorice.]

V~istom meste majú vybudovanú sieť na šírenie klebiet, v~ktorej si
každý klebetník vymieňa informácie s~tromi klebetnicami a~každá
klebetnica si vymieňa informácie s~tromi klebetníkmi.
Inak sa klebety nešíria.
 a) Dokážte, že klebetníkov a~klebetníc je rovnako veľa.
 b) Predpokladajme, že sieť na šírenie klebiet je súvislá (klebety od
ľubovoľného klebetníka a~ľubovoľnej klebetnice sa môžu dostať ku všetkým
ostatným). Dokážte, že aj keď sa jeden klebetník z~mesta odsťahuje, zostane sieť
súvislá. [\pdfklink{61-B-I-5}{https://skmo.sk/dokument.php?id=452\#page=5}]

Lukáš a Marek, ktorí sa poznajú, sa zišli na párty, na ktorej platilo:
Ak majú niektorí dvaja účastníci rovnaký počet známych,
potom nemajú žiadneho spoločného známeho. Dokážte, že na párty je niekto,
kto tam má práve jedného známeho. Návod: Najprv vyberte jedného
účastníka, ktorý pozná na párty najväčší počet osôb.
[Označme $X$ jedného z~tých účastníkov, ktorí na párty poznajú
najviac, povedzme $k$ osôb. Podľa zadania je $k\geqq1$, v~prípade
$k=1$ sme hotoví. Ak je $k>1$, žiadni dvaja z~$k$ známych vybraného
$X$ nemajú rovnaký počet známych, takže týchto $k$ počtov
tvorí celú množinu $\{1,2,\dots,k\}$.]

V~spoločnosti ľudí sú niektoré dvojice spriatelené. Pre
kladné celé číslo $k\ge 3$ hovoríme, že spoločnosť je $k$-dobrá, ak
možno každú $k$-ticu ľudí zo spoločnosti rozsadiť okolo okrúhleho stola
tak, že sa každí dvaja susedia priatelia. Dokážte, že ak je spoločnosť
6-dobrá, tak je aj~7-dobrá.
[\pdfklink{67-A-III-1}{https://skmo.sk/dokument.php?id=2726\#page=2}]

V~skupine 90~detí má každé aspoň 30~kamarátov (kamarátstvo je vzájomné). Dokážte, že ich možno rozdeliť do troch 30-členných skupín tak, aby každé dieťa malo vo svojej skupine aspoň jedného kamaráta.
[\pdfklink{61-A-III-5}{https://skmo.sk/dokument.php?id=492\#page=6}]

\endnávod

}

{%%%%%   A-I-2
a) Áno, vyhovuje napríklad číslo 137\,658\,942. Súčty trojíc
jeho susedných cifier totiž zľava doprava sú 11, 16, 18, 19,
22, 21, 15.
\smallskip

b) Ukážeme, že pre každé deväťciferné číslo
$\overline{a_1a_2\ldots a_8a_9}$ zostavené z~cifier 1 až~9 platí, že
súčet siedmich čísel určených zadaním úlohy je najviac 122.
Keďže pre čísla zo zadania b) platí $11+15+16+18+19+21+23=123$,
ukážeme tým, že požadované deväťciferné číslo neexistuje.

Uvažovaný súčet $S$ siedmich čísel je možné zapísať a následne
odhadnúť nasledovne:
$$\eqalign{
S&=(a_1+a_2+a_3)+(a_2+a_3+a_4)+
\dots+(a_6+a_7+a_8)+(a_7+a_8+a_9)=\cr
&= a_1+2a_2+3a_3+ 3a_4+ 3a_5+3a_6
+3a_7+2a_8+a_9=\cr
&=3(\underbrace{a_1+a_2+\dots+a_8+a_9}_{=1+2+\dots+9=45})-
(\underbrace{a_1+a_2+a_8+a_9}_{\geq1+2+3+4=10})-
(\underbrace{a_1+a_9}_{\geq1+2=3})\leqq\cr
&\leqq3\cdot45-10-3=122.
}$$
Tým je tvrdenie z prvej vety nášho riešenia časti b) dokázané.

\poznamka
Hoci je vyššie uvedené riešenie úplné, ukážeme teraz, ako je možné nájsť
príklad vyhovujúceho čísla z~riešenia časti~a). Nájdeme ich
dokonca všetky, a to na základe pozorovania, ktoré nás
doviedlo aj k~riešeniu časti~b).

Prvým dobrým krokom je povšimnúť si, že obe sedmice čísel zo zadania
úlohy obsahujú \uv{pomerne veľké} čísla na to, aby vznikli zo
všetkých zastúpených cifier 1 až~9. To nás môže motivovať
na posúdenie celkového súčtu $S$ všetkých siedmich zadaných čísel.
Ako môže byť tento súčet veľký? Odpoveď na túto otázku
určite súvisí s~tým, koľkokrát je ktorá cifra v~súčte $S$
zastúpená. Preto už je vcelku ľahké prísť na horný odhad ~ $S$
číslom 122 spôsobom, akým sme to vyššie urobili.

Dodajme teraz, čo sme v~riešení časti~b) nepotrebovali.
Z~nášho odvodenia vyplýva, že rovnosť $S=122$ nastane
práve vtedy, keď číslo $\overline{a_1\ldots a_9}$ spĺňa rovnosti
$$
\{a_1,a_2,a_8,a_9\}=\{1,2,3,4\}\quad\hbox{a}\quad
\{a_1,a_9\}=\{1,2\},
$$
čiže $\{a_1,a_9\}=\{1,2\}$ a $\{a_2,a_8\}=\{3,4\}$.\fnote{Je ľahké príklad takého čísla uviesť, preto 122 je
\emph{najväčšia možná} hodnota súčtu $S$.}

Keďže súčet $11+15+16+18+19+21+22$ čísel zadaných
v~časti~a) má práve uvažovanú hodnotu~122, sú hľadané
vyhovujúce čísla $\overline{a_1a_2\ldots a_8a_9}$ s ciframi 1
až~9 práve tie, pre ktoré platí $\{a_1,a_9\}=\{1,2\}$
a $\{a_2,a_8\}=\{3,4\}$. Podľa možných pozícií cifier
1, 2, 3, 4 tak prichádzajú do úvahy iba 4 typy čísel.

Skúmajme najskôr čísla $\overline{13a_3\ldots a_742}$. Aby
$1+3+a_3$ bol jeden z~predpísaných súčtov, musí byť $a_3=7$;
podobnou úvahou o~súčte $a_7+4+2$ potom nutne $a_7=9$.
Máme teda čísla $\overline{137a_4a_5a_6942}$. Zostáva tak
posúdiť šesť spôsobov, ako trojici $a_4$, $a_5$ a $a_6$
priradiť zvyšné cifry 5, 6 a~8. Jednotlivé spôsoby už je ľahké
otestovať, je pritom možné niektoré z~nich vopred
vylúčiť (napríklad využiť to, že $a_4\ne5$). Bez týchto
podrobností uveďme, že vo výsledku dostaneme práve dve
vyhovujúce čísla 137\,658\,942 a ~137\,685\,942.

\smallskip
Skúmanie čísel $\overline{14a_3\ldots a_732}$ je kratšie:
Tentoraz zistíme, že obe cifry $a_3$ a $a_7$ by museli
byť rovné 6, teda žiadne číslo tohto typu nevyhovuje.

\smallskip
Posudzovanie zvyšných čísel $\overline{23a_3\ldots a_741}$ a
$\overline{24a_3\ldots a_731}$ už nie je potrebné. Stačí ich previesť na
predchádzajúce dva typy, a to vďaka všeobecnému postrehu:
číslo $\overline{a_1a_2\ldots a_8a_9}$ vyhovuje zadaniu
práve vtedy, keď mu vyhovuje \uv{zrkadlovo prevrátené} číslo
$\overline{a_9a_8\ldots a_2a_1}$.

\smallskip
Zhrňme, čo sme zistili: Čísla, ktoré vyhovujú zadaniu a),
sú práve štyri. Sú to čísla
137\,658\,942, 137\,685\,942 a ich \uv{zrkadlové obrazy}
249\,856\,731 a ~249\,586\,731.

\návody

Nájdite päťciferné čísla, z ktorých každé má päť rôznych
nepárnych cifier, pritom súčet prvých troch cifier je 11 a súčet
posledných troch cifier je 15.
[Prostredná cifra musí byť 1, pretože $11+15$ je o jedna viac
ako $1+3+5+7+9$. Na prvých dvoch miestach potom musí byť 3 a 7,
na posledných dvoch 9 a 5. Všetky takéto čísla vyhovujú:
37159, 37195, 73159, 73195.]

Určte najväčšie možné hodnoty nasledujúcich súčtov, v ktorých
$a_1,a_2,\dots,a_8,a_9$ je ľubovoľné poradie
cifier $1,2,\dots,8,9$:\hfil\break
a) $a_1+a_2+a_3+a_4+a_5+a_6+a_7+a_8$,\hfil\break
b) $a_1+a_2+a_3+a_4+a_5+a_6+a_7+a_8+2a_9$,\hfil\break
c) $a_1+2a_2+2a_3+2a_4+2a_5+2a_6+2a_7+2a_8+a_9$.\hfil\penalty-100
[a) $45-1=44$, b) $45+9=54$, c) $2\cdot45-(1+2)=87$.]

\D
Nájdite najväčšie možné šesťciferné číslo, ktorého každá cifra
(počnúc treťou cifrou zľava) je súčtom predchádzajúcich
dvoch.
[303\,369. Cifry zľava doprava sú $a$, $b$, $a+b$,
$a+2b$, $2a+3b$ a $3a+5b$. Z~nerovnosti $3a+5b\leq9$ vyplýva
$a\leq3$, pritom pre $a=3$ je nutne $b=0$.]

Dané prirodzené číslo $n$ má cifry, ktorých hodnoty
sa zľava doprava zväčšujú. Ukážte, že ciferný súčet
čísla $9n$ je vždy rovný deviatim.
Návod: $9n=10n-n$.
[Ak má dané~$n$ cifry
$c_1<c_2<\dots<c_k$, sú cifry rozdielu
$10n-n$ podľa písomného algoritmu pre odčítanie zľava doprava rovné
$c_1$, $c_2-c_1$,\dots, $c_{k-1}-c_{k-2}$, $c_{k}-(c_{k-1}+1)$,
$10-c_k$. Ich súčet je naozaj 9.]

Nájdite najväčšie možné prirodzené číslo, ktorého každá
cifra (okrem oboch krajných) je menšia ako aritmetický priemer
cifier susedných.
[96\,433\,469.
Číslo so zápisom $\overline{c_1c_2\ldots c_n}$, kde $n\geqq3$,
vyhovuje zadaniu práve vtedy, keď platí $c_{i+1}-c_i>c_i-c_{i-1}$
pre každé prípustné $i$. Hľadáme tak najväčšie číslo, pre ktoré
je zodpovedajúca postupnosť rozdielov $c_2-c_1$, ${c_3-c_2,
\dots}$, $c_n-c_{n-1}$ rastúca. Nech je prvých $k$ rozdielov záporných a
posledných $n-1-k$ rozdielov nezáporných. Ich súčty sú postupne
$c_{k+1}-c_1\geqq{-9}$ a~$c_{n}-c_{k+1}\leqq9$. Odtiaľ s~ohľadom na
$1+2+3+4>9$ vyplýva, že $k\leqq3$ a $n-1-k\leqq4$
(je možný aj rozdiel 0), čiže
$n-k\leqq5$. Preto $n=k+(n-k)\leqq8$. Ukážme, že pre $n=8$ (keď
nutne $k=3$) vyhovujúce číslo existuje. Keďže hľadáme čo
najväčšie také, vyberieme $c_1=9$ ($c_1$ je možné vždy
zväčšiť). Potom však z~$c_1=9$ a $k=3$
vyplýva, že $c_2-c_1\leqq{-3}$, čiže $c_2\leqq6$, pritom pre $c_2=6$
je nutne $c_3-c_2={-2}$ a $c_4-c_1={-1}$, takže potom štvorčíslie
$c_1c_2c_3c_4$ je rovné $9643$. Z~podmienky nezápornosti čísla $c_5-c_4$
pri $c_4=3$ však ľahko vyplýva, že jediné vyhovujúce štvorčíslie
$c_5c_6c_7c_8$ potom je $3469$.]

\endnávod
}

{%%%%%   A-I-3
Dokážeme, že pomer $|AT|:|DE|$ má jedinú možnú
hodnotu $2\sqrt{2}$.

Najprv odvodíme, že pre dĺžky úsečiek $TD$ a $TE$ platí
$|TD|/|DE|=\sqrt2$.\fnote{Je
možné dokonca ukázať, že $TDE$ je rovnoramenný trojuholník
s~pravým uhlom pri vrchole~$E$, ale to k~riešeniu úlohy potrebovať
nebudeme.} Zameriame sa pri tom na trojuholníky $BTC$ a $KTL$, ktoré sú
vyfarbené na \obr{}. Ich uhly pri spoločnom vrchole $T$
sú zhodné, pretože platí
$$
|\angle BTC|=|\angle BTK|\!+\!|\angle KTC|=
45^\circ\!+\!|\angle KTC|=|\angle KTC|\!+\!|\angle CTL|=|\angle KTL|.
$$
Navyše platí $|BT|/|TC|=|KT|/|TL|$ vďaka podobnosti trojuholníkov
$BKT$ a $CLT$. Preto sú podľa vety $sus$ naše trojuholníky $BTC$ a
$KTL$ podobné, a to v~pomere $|BT|:|KT|$ s hodnotou $\sqrt2$
(vďaka pravouhlému rovnoramennému trojuholníku $BKT$).
Pre dĺžky príslušných ťažníc $TD$ a $TE$ vyznačených trojuholníkov teda
naozaj platí $|TD|/|TE|=\sqrt{2}$.
\inspsc{a73i.31}{.8333}%

Odvodený výsledok zostáva spojiť s~poznatkom, že pre ťažisko $T$
trojuholníka $ABC$ platí $|AT|=2|TD|$. Dostaneme tak
$$
\frac{|AT|}{|DE|}=\frac{2|TD|}{|DE|}=2\sqrt{2}.
$$
Tým je dôkaz tvrdenia z~prvej vety riešenia hotový.

\poznamka
Úlohu sme riešili v~situácii ako na zadanom obrázku, podľa ktorého
je trojuholník $BTK$ prikreslený do polroviny $BTC$ a trojuholník $CTL$ do
polroviny $CTA$. Potrebnú zhodnosť uhlov $BTC$ a $KTL$ sme
v~riešení odvodili za predpokladu, že bod~$K$ leží v~uhle $BTC$
ako na obrázku zo zadania. Tento predpoklad je splnený
len pre tie trojuholníky $ABC$,
v ktorých spomínaný uhol~$BTC$ má veľkosť aspoň $45^\circ$.
Tak to však nie je napríklad na \obr.
Prečo sú uhly $BTC$ a $KTL$ zhodné {\it vždy}, je možné zdôvodniť nasledovne:
Vďaka zadaniu sú orientované uhly
$BTK$ a $CTL$ zhodné (majú totiž oba veľkosti $45^\circ$ a
rovnakú orientáciu ako uhol $BTC$), teda v~nimi určenom
otočení so stredom~$T$ bude obrazom uhla $BTC$ práve uhol $KTL$.
\inspsc{a73i.32}{.8333}%

\ineriesenie
Poznatky odvodené v hlavnej časti prvého riešenia získame
teraz postupom, pri ktorom využijeme rovinné zobrazenia.
Obmedzíme sa pritom iba na dôkaz toho,
že pre dĺžky úsečiek $TD$ a $TE$
platí rovnosť $|TD|/|DE|=\sqrt2$. Zistíme navyše aj to,
čo sme napísali v poznámke pod čiarou k úvodnej vete prvého
riešenie, že totiž $TDE$ je rovnoramenný trojuholník s~pravým uhlom
pri vrchole~$E$.

Pri novom postupe využijeme otočenie so stredom $T$,
o~ktorom sme už písali v~predchádzajúcej poznámke. Zložíme
ho ešte s~vhodnou rovnoľahlosťou, ktorá bude mať
ten istý stred~$T$. Dostaneme tak podobné zobrazenie nazývané
{\it špirálová podobnosť so stredom\/}~$T$.
Zmienime sa o ňom aj v~úvodnom odseku k~návodným úlohám. Tam
taktiež nájdete odkaz na vhodný študijný text.

Uvažujme teda špirálovú podobnosť $\Cal S$ so stredom $T$,
koeficientom $\sqrt{2}/2$ a orientovaným uhlom $BTK$,
ktorý je zhodný s orientovaným uhlom $CTL$ tej istej veľkosti~$45^{\circ}$.
\inspsc{a73i.33}{.8333}%

V~našom zobrazení $\Cal S$ prejde bod $B$ do bodu $K$
a $C$ do bodu $L$ (\obr). Z~toho vyplýva, že obrazom
úsečky $BC$ je úsečka $KL$.\fnote{Vďaka tomu
obrazom trojuholníka $BTC$ je trojuholník $KTL$. Podobnosť týchto dvoch trojuholníkov
bola základom nášho prvého riešenia, v ktorom sme sa použitiu
špirálovej podobnosti vyhli.} Znamená to, že aj stred~$D$
úsečky~$BC$ prejde do stredu $E$ úsečky $KL$.
Z $B\to K$, $C\to L$ a $D\to E$ ako je známe vyplýva, že
trojuholník $TDE$ je podobný obom trojuholníkom $TBK$ a $TCL$,\fnote{Toto tvrdenie,
ktoré nie je nutné v~úplnom riešení zdôvodňovať, vyplýva z~toho, že pri špirálovej
podobnosti so stredom $T$ sú všetky trojuholníky $TXX'$, kde $X'$ je
obraz $X$, navzájom podobné vďaka vete $sus$.}
takže aj $TDE$ je pravouhlý rovnoramenný trojuholník, ktorý pritom
má pravý uhol pri vrchole~$E$. Tým je sľúbené tvrdenie dokázané.

\návody
{\everypar{}
\smallskip
Pri riešení návodných a dopĺňajúcich úloh je možné vhodne využiť
\emph{\it špirálovú podobnosť}.\fnote{S týmto pojmom
a jeho použitím pri riešení mnohých úloh sa dá prístupným spôsobom zoznámiť
v~bakalárskej práci \pdfklink{Tomáša Hrdličku \emph{Spirální podobnost v planimetrii}}{https://theses.cz/id/5tj389/Spirln_podobnost.pdf}
so zoznamom použitej literatúry, ktorý zahŕňa
odkazy na ďalšie (internetové) zdroje.}
Tak bežne nazývame podobné zobrazenia, ktoré sú výsledkom
zloženia rovnoľahlosti a otočenia so spoločným stredom, ktorý potom
nazývame \emph{stredom} danej špirálovej podobnosti.\smallskip
}

Navzájom rôzne body $S$, $A$, $A'$, $B$, $B'$ sú zvolené tak,
že oba trojuholníky $SAA'$ a $SB'B'$ sú rovnostranné, pričom
pri pohľade z bodu $S$ sú body $A$, $A'$, $B$, $B'$ práve takto
usporiadané v kladnom smere (ako na \obr{}). Dokážte tvrdenie:
(i) Trojuholníky $SAB$ a $SA'B'$ sú zhodné.
(ii) Ak označíme $S_{AB}$ a $S_{A'B'}$ postupne stredy úsečiek
$AB$ a $A'B'$, je trojuholník $SS_{AB}S_{A'B'}$ rovnostranný.
\inspdf{a73i_34.pdf}%
[Uvažujte otočenie so stredom $S$ a~uhlom $+60^\circ$.
Čo je obrazom úsečky $AB$? Čo je obrazom bodu $S_{AB}$?]

Dokážte obmenu oboch tvrdení z N1 pre všeobecnejšiu situáciu,
keď trojuholníky $SAA'$ a $SBB'$ sú (priamo) podobné (ako na
\obrr1{}):\hfil\break
(i) $\triangle SAB\sim\triangle SA'B'$,\qquad
(ii) $\triangle SS_{AB}S_{A'B'}\sim\triangle SAA'\sim\triangle SBB'$.
[Uvažujte zloženie rovnoľahlosti a otočenia so spoločným stredom
v~bode $S$ (čiže špirálovú podobnosť so stredom $S$),
ktoré zobrazí $A$ na $A'$, a teda aj $B$ na $B'$.
Potom vykonajte analogické úvahy ako pri riešení úlohy N1.]

\D
Vo vnútri trojuholníka $ABC$ ležia body $D$, $E$,~$F$ také, že
trojuholníky $BCD$, $CAE$ a~$ABF$ sú rovnostranné. Ukážte, že ťažisko
týchto trojuholníkov tvorí rovnostranný trojuholník.
[Označte spomínané ťažiská postupne $A_1$, $B_1$,~$C_1$ a~uvažujte
špirálovú podobnosť so stredom v~$C$, ktorá zobrazuje $B_1$
na~$A$, a teda aj $A_1$ na~$D$. Pre úsečku~$B_1A_1$ a
jej obraz $AD$ potom platí $|AD|=\sqrt{3}|B_1A_1|$.
Analogickými úvahami odvodíme nielen $|BE|=\sqrt{3}|C_1B_1|$
a $|CF|=\sqrt{3}|A_1C_1|$, ale aj
$|BE|=\sqrt{3}|A_1B_1|$, $|CF|=\sqrt{3}|B_1C_1|$ a
$|AD|=\sqrt{3}|C_1A_1|$. Odtiaľ už vyplýva
$|A_1B_1|=|A_1C_1|=|B_1C_1|$.]

Vnútri pravouhlého trojuholníka $ABC$ s~preponou $AB$ a
vnútorným uhlom pri vrchole $A$ o~veľkosti $60^\circ$
existuje bod $P$, pre ktorý platí $|\uhol APB| = 120^\circ$,
$|BP| = 4$ a $|CP| = 1$. Určite dĺžku úsečky $AP$.
[$|AP|=2$. Uvažujte špirálovú podobnosť so stredom v~bode $A$,
ktorá zobrazí $C$ na $B$. Obraz bodu $P$ označte $P'$.
Koeficient tejto podobnosti je
$|BA|/|CA|=1/\cos60^{\circ}=2$, teda
pre obraz $BP'$ úsečky $CP$ s dĺžkou~1 platí $|BP'| = 2$.
Ďalej z $\triangle ABC\sim\triangle AP'P$ (podľa vety
$sus$) vyplýva $|\uhol AP'P|=30^\circ$
a $|\uhol APP'|=90^\circ$, teda
$|\uhol P'PB|=|\uhol APB|-|\uhol APP'|=30^\circ$,
čo spolu s $|BP|=4$ a $|BP'|=2$
dáva $|\uhol BP'P|=90^\circ$. Priečka~$PP'$ tak zviera zhodné
striedavé uhly ako s~priamkami~$AP$ a~$P'B$, tak s~priamkami~$AP'$
a~$BP$. Štvoruholník $APBP'$ je preto rovnobežník, odkiaľ
$|AP|=|BP'|=2$.]

\endnávod
}

{%%%%%   A-I-4
Dokážeme sporom, že dve po sebe idúce špeciálne prvočísla neexistujú.

Nech $p_1=2<p_2=3<p_3<\dots$ je rastúca postupnosť
všetkých prvočísel. Pripusťme, že pre niektoré $n\geqq2$ sú obe prvočísla
$p_n$ a $p_{n+1}$ špeciálne. Potom existujú prirodzené čísla $a$, $b$
také, že platia rovnosti
$$
p_1+p_2+\ldots+p_{n-1}=ap_n\quad\hbox{a}\quad
p_1+p_2+\ldots+p_{n-1}+p_{n}=bp_{n+1}.
\tag1
$$
Po dosadení súčtu z~prvého vzťahu do druhého dostaneme
$ap_n+p_{n}=bp_{n+1}$, čiže ${(a+1)}p_{n}=bp_{n+1}$.
Prvočíslo $p_{n}$ tak delí súčin
$bp_{n+1}$ a je pritom nesúdeliteľné s~prvočíslom $p_{n+1}$,
takže delí číslo $b$. Preto platí $b=cp_n$
pre vhodné prirodzené číslo~$c$. Po dosadení $b=cp_n$
do druhej rovnosti v~\thetag1 dostaneme
$$
p_1+p_2+\ldots+p_{n-1}+p_{n}=cp_np_{n+1}.
$$
Ukážeme však, že namiesto tejto rovnosti platí ostrá nerovnosť
$$
p_1+p_2+\ldots+p_{n-1}+p_{n}<cp_np_{n+1}.
$$
Ak totiž spočítame nerovnosti $p_i<p_n$ pre $i=1,\dots,n-1$ s
rovnosťou $p_n=p_n$ a ak potom prihliadneme postupne
na zrejmé nerovnosti $p_{n+1}>n$ a $c\geqq1$, dostaneme
$$
p_1+p_2+\ldots+p_{n-1}+p_{n}<np_n<p_{n+1}p_n\leq cp_np_{n+1}.
$$
Tým je dôkaz sporom ukončený.\fnote{Dokázali sme vlastne
všeobecnejšie tvrdenie: V žiadnej rastúcej postupnosti navzájom
nesúdeliteľných prirodzených čísel sa nenájdu dva susedné
členy, z ktorých každý je deliteľom súčtu všetkých jemu
predchádzajúcich členov.}

\poznamka
Špeciálne prvočísla naozaj existujú. Všetky doposiaľ známe sú
$$
5,\ 71,\ 369\,119,\ 415\,074\,643,\ 55\,691\,042\,365\,834\,801.
$$
Prípadný ďalší vývoj je možné sledovať na \pdfklink{https://oeis.org/A007506}{https://oeis.org/A007506}.


\návody

Nájdite všetky dvojice prirodzených čísel $u$ a $v$,
v ktorých $u$ je deliteľom~$2v$ a~$v$ je deliteľom~$3u$.
[Existujú prirodzené $a$, $b$ tak, že $2v=au$ a $3u=bv$.
Vynásobením dostaneme $6vu=abuv$, čiže $ab=6$.
Preto $(a,b)$ je nutne jedna z dvojíc $(1,6)$, $(2,3)$, $(3,2)$,
$(6,1)$. Požadované rovnosti $2v=au$ a $3u=bv$ sa potom postupne
redukujú na $u=2v$, $u=v$, $u=\frac23v$, $u=\frac13v$. Vyhovujú
teda práve dvojice $(u,v)$ tvarov $(2n,n)$, $(n,n)$, $(2n,3n)$ a
$(n,3n)$, kde $n$ je prirodzené číslo.]

Ukážte, že pre žiadne nepárne prvočíslo $p$ nie je súčet
$1+2+3+\ldots+p$ deliteľný nejakým prvočíslom
väčším ako $p$.
[Daný súčet je rovný $\frac12 p(p+1)$, teda
každý jeho prvočiniteľ~$q$ delí niektoré z~čísel $p$ alebo $p+1$,
a preto $q\leqq p$, pretože prvočíslo $p$ je nepárne, a tak
párne číslo~$p+1$ je aspoň 4, a je teda zložené.]

Pre dané nepárne prvočíslo $p$ označme $S$ súčet všetkých
prirodzených čísel menších ako~$p$, ktoré majú
vo svojich dekadických zápisoch aspoň jednu cifru z dekadického
zápisu čísla $p$. Ukážte, že ak $p \mid S$, tak číslo $S$
nemá okrem $p$ žiadneho prvočiniteľa
väčšieho ako $\frac12(p-1)$.
[Zrejme $S\leq1+2+\ldots+(p-1)=\frac12p(p-1)$,
odkiaľ $S/p\leqq\frac12(p-1)$, kde $S/p$ je prirodzené
číslo vďaka predpokladu $p\mid S$. Ak je preto $q$ nejaký
prvočiniteľ čísla $S$ rôzny od $p$, je aj prvočiniteľom
čísla $S/p$, ktoré samo ako vieme neprevyšuje $\frac12(p-1)$.]

\D
Ukážte, že súčet dvoch po sebe idúcich prvočísel nemôže byť
dvojnásobok iného prvočísla.
[Pre dôkaz sporom pri zrejmom
označení rovnosť $p_n+p_{n+1}=2q$ prepíšme do tvaru
$\frac12(p_n+p_{n+1})=q$. Číslo $q$ teda leží vo vnútri
otvoreného intervalu $(p_n,p_{n+1})$, v ktorom však
nie sú žiadne prvočísla.]

Dokážte, že pokiaľ pre prirodzené čísla $a$, $b$, $c$ platí,
$a+b+c\mid abc$, potom je $a+b+c$
zložené číslo.
[Sporom: Ak by číslo $s=a+b+c$
bolo prvočíslo, bolo by deliteľom (vďaka zadanej podmienke $s\mid abc$)
aspoň jedného z~čísel $a$, $b$, $c$. Tie sú však všetky tri menšie ako
ich súčet $s$, čo je spor.]

Pre každé $n>1$ označme $S_n$ súčet $n$ prvých prvočísel.
Ukážte, že v~intervale $\langle S_n, S_{n+1} \rangle$ leží
vždy druhá mocnina niektorého prirodzeného čísla.
[Ukážme najprv, že na to, aby vo všeobecnejšom intervale
$\langle S,S+(2k+1)\rangle$, kde $S$ a~$k$ sú prirodzené
čísla, ležala druhá mocnina, stačí, aby platilo
$S\leqq(k+1)^2$ (pokiaľ totiž v~$\langle S,S+(2k+1)\rangle$
neleží žiadne z~čísel $1^2,2^2,\dots,k^2$, je potom
$k^2<S\leqq(k+1)^2$, teda v~danom intervale
leží číslo~$(k+1)^2$).
Pri zrejmom označení tak v~našej úlohe stačí pre každé $n>1$ dokázať
nerovnosť $p_1+\ldots+p_n\leqq\frac14(p_{n+1}+1)^2$
(podľa predchádzajúceho tvrdenia pre $k=\frac12(p_{n+1}-1)$).
Keďže
$p_1-1$, $p_2$, $p_3,\dots,p_n$ sú rôzne čísla
z množiny nepárnych čísel $\{1,3,5,\dots,p_{n+1}-2\}$ so súčtom prvkov
$\frac14(p_{n+1}-1)^2$, stačí dokázať, že platí
$1+\frac14(p_{n+1}-1)^2\leqq\frac14(p_{n+1}+1)^2$.
To je ale zrejmé, pretože celé číslo $\frac14(p_{n+1}-1)^2$ je menšie
ako celé číslo $\frac14(p_{n+1}+1)^2$.]

Na tabuli sú napísané (nie nutne rôzne) prvočísla, ktorých súčin je 105-krát väčší ako ich súčet. Určte všetky napísané prvočísla, ak ich je
a) päť; b) sedem.[\pdfklink{70-A-I-1}{https://skmo.sk/dokument.php?id=3467}]

\endnávod
}

{%%%%%   A-I-5
Taká podmnožina políčok existuje, nájdeme totiž jej konkrétny
príklad.

Konštrukciu príkladu a overenie jeho správnosti si zjednodušíme,
keď od piatich vyplňovaní tetraminami prejdeme k dvom vyplňovaniam
{\it oktaminami}, ktoré vidíte na \obr{} - prvé oktamino v~troch kópiách, druhé oktamino v dvoch kópiách.\fnote{Zdôraznime, že
nie je vopred jasné, či takéto zjednodušenie úlohy dopadne
\uv{dobre}. Chýba nám dôkaz tvrdenia, že z existencie piatich vyplnení
tetraminami vyplýva existencia dvoch vyplnení oktaminami.}
Takýto prechod je možný, pretože každé z~piatich oktamín
je vyplnené dvoma kópiami iného z~piatich tetramín zo zadania.
\inspdf{a73i_51.pdf}%

Neprázdnu podmnožinu políčok, ktorú je možné vyplniť kópiami
každého z dvoch navrhnutých oktamín, možno nájsť ľahšie.
Jej príklad aj s oboma vyplneniami je na \obr{}.
\inspdf{a73i_52.pdf}%

Tým je riešenie úlohy hotové.

\návody

Nájdite neprázdnu podmnožinu políčok tabuľky $20\times20$,
ktorú je možné vyplniť (bezo zvyšku a prekrývania)
kópiami ľavého útvaru aj kópiami pravého útvaru z \obr{}.
\inspdfsirka{a73i_53.pdf}{11cm}%
[Použite každý útvar štyrikrát \uv{dookola}.]

Určte, koľko políčok obsahuje najmenšia plocha, ktorú je možné vyplniť
a) ako tetraminami typu I, tak aj tetraminami typu O, a tiež aj
tetraminami typu L,
b) ako tetraminami typu S, tak aj tetraminami typu T.
[8 políčok pre obe úlohy. Každá vyhovujúca plocha musí zložená
z~aspoň dvoch kópií každého zo spomínaných tetramín
a mať tak aspoň 8~políčok.
Možné príklady plôch s~8~políčkami aj s~požadovanými vyplneniami
\inspdfsirka{a73i_51.pdf}{6cm}%
sú na \obr{}.]

\D
Rozhodnite, či je možné tabuľku na \obr{} vyplniť
tetraminami typu L.
\inspdfsirka{a73i_54.pdf}{3cm}%
[Dá sa to. Dokonca je možné bez presahu vyplniť \uv{polovicu} tabuľky,
rozdelenej jej zvislou (alebo vodorovnou) osou súmernosti.]

Rozhodnite, či je možné tabuľku $10\times 10$ vyplniť
tetraminami typu T.
[Nejde to. Ofarbite tabuľku bielymi a čiernymi políčkami ako
šachovnicu. Potom každé tetramino T pokrýva nepárny počet čiernych políčok.
Pri vyplnení 25 tetraminami typu T by celkový počet
pokrytých čiernych políčok bol nepárny.]

Rozhodnite, či je možné tabuľku $10\times 10$ vyplniť tetraminami
typu I.
[Nejde to. Uvažujte \uv{hrubšie} šachovnicové zafarbenie,
keď jednofarebné štvorce majú veľkosť $2 \times 2$. Koľko
čiernych políčok potom pokryje jedno tetramino typu I, koľko
by ich bolo pokrytých pri vyplnení jeho 25 kópiami?]

Na niektoré políčko štvorcovej šachovnice $n\times n$ ($n\geq 2$)
postavíme figúrku a~potom ju posúvame striedavo "šikmo" a~"priamo".
"Šikmo" znamená na políčko, ktoré má s~predchádzajúcim spoločný práve
jeden bod. "Priamo" znamená na susedné políčko, ktoré má
s~predchádzajúcim spoločnú stranu. Určte všetky $n$, pre ktoré
existuje východiskové políčko a~taká postupnosť ťahov začínajúca "šikmo",
že figúrka prejde celú šachovnicu a~na každom políčku sa ocitne
práve raz.
[\pdfklink{56-A-III-1}{https://skmo.sk/dokument.php?id=225}]

Nech $n \geq 3$ je celé číslo. Uvažujme štvorčekový papier s~rozmermi $n \times n$, ktorého jednotlivé štvorčeky môžu mať buď bielu, alebo čiernu farbu. V každom kroku zmeníme farby piatich štvorčekov, ktoré tvoria útvar
\lower1\baselineskip\hbox{\epsfbox{a72iii.60}}
v~ľubovoľnom natočení. Na začiatku sú všetky štvorčeky biele. Rozhodnite, pre ktoré $n$ možno po konečnom počte krokov dosiahnuť to, že všetky štvorčeky budú čierne. [\pdfklink{72-a-iii-6}{https://skmo.sk/dokument.php?id=4440\#page=5}]


\endnávod
}

{%%%%%   A-I-6
Vo všetkých častiach textu budeme predpokladať, že čísla
$a$, $b$, $c$, $d$ spĺňajú podmienky uvedené v~zadaní.

\smallskip
V~prvej časti riešenia dokážeme, že zadaná nerovnosť platí.

Najprv si všimneme, že menovatele zastúpených zlomkov
sú kladné čísla, pretože čísla $a$, $b$, $c$, $d$ sú aspoň 1.
Keďže navyše ležia v intervale dĺžky 1, pre každé dve z~nich, napríklad
pre čísla $a$ a $b$, platí $(a-b)^2 \leq 1$, čiže $a^2+b^2-1 \leq
2ab$. Z~nerovností $0<a^2+b^2-1 \leq 2ab$ vyplýva,
že pre prvý zlomok zo zadania platí
$$
\frac{1}{a^2+b^2-1}\geqq\frac{1}{2ab}.
\tag1
$$
Ak spočítame túto nerovnosť s jej obdobami pre ďalšie tri
zadané zlomky, dostaneme
$$
\frac{1}{a^2+b^2-1}+\frac{1}{b^2+c^2-1}+\frac{1}{c^2+d^2-1}
+\frac{1}{d^2+a^2-1}\geqq
\frac{1}{2ab}+\frac{1}{2bc}+\frac{1}{2cd}+\frac{1}{2da}.
\tag2
$$
Nerovnosť zo zadania tak bude dokázaná, ak overíme jednoduchšiu
nerovnosť
$$
\frac{1}{ab}+\frac{1}{bc}+\frac{1}{cd}+\frac{1}{da} \geq 2.
\tag3
$$
Využijeme na to známu nerovnosť pre {\it súčin dvoch súčtov},
konkrétne súčet niekoľkých kladných čísel
a súčet ich prevrátených hodnôt.\fnote{Táto
nerovnosť, uvedená nižšie v~úlohe N3, je ekvivalentná s~nerovnosťou
medzi {\it aritmetickým\/} a~{\it harmonickým\/} priemerom
niekoľkých kladných čísel, o~ktorej pojednáva úloha N4.}
V~prípade 4 čísel má táto nerovnosť tvar, ktorý rovno zapíšeme pre
čísla
$ab$, $bc$, $cd$, $da$:
$$
(ab+bc+cd+da)\left(\frac{1}{ab} + \frac{1}{bc}
+ \frac{1}{cd} + \frac{1}{da} \right)\geqq 4^2.
\tag4
$$
Želaná nerovnosť \thetag3 je už ľahkým dôsledkom~\thetag4 a
predpokladu $(a+c)(b+d)=8$ zo zadania:
$$
\frac{1}{ab}+\frac{1}{bc}+\frac{1}{ca}+\frac{1}{da} \geq
\frac{4^2}{ab+bc+cd+da} =
\frac{16}{(a+c)(b+d)}=\frac{16}{8}=2.
$$

V~druhej časti riešenia určíme, kedy v~dokázanej nerovnosti
nastane rovnosť. Vtedy je nutné, aby nastala rovnosť v~\thetag2, teda
aj rovnosti v~nerovnosti $(a-b)^2\leqq1$ a
troch ďalších jej obdobách. Hľadané štvorice $(a,b,c,d)$
tak nutne spĺňajú rovnosti
$$
|a-b|=|b-c|=|c-d|=|d-a|=1.
\tag5
$$
Pokiaľ platí $a>b$, ľahko už takú štvoricu čísel z~intervalu
$\langle 1, 2 \rangle$ jednoznačne určíme ako $(2, 1, 2, 1)$.
V~opačnom prípade $b\geqq a$ obdobne dospejeme k~jedinej
štvorici $(1, 2, 1, 2)$. Dosadením každej z dvoch určených štvoríc
do pôvodnej nerovnosti zistíme, že rovnosť naozaj spĺňajú, pretože
každý zo štyroch zlomkov na ľavej strane je po dosadení rovný $1/4$.

\zaver
Rovnosť nastane iba pre štvorice $(a,b,c,d)$ rovné
$(2, 1, 2, 1)$ a $(1, 2, 1, 2)$.

\poznamka
Vysvetlime trochu umelý obrat z~úvodu podaného
riešenia. Dokazovanú nerovnosť možno len ťažko nejako
výhodne algebraicky upraviť. Preto sme sa rozhodli odhadnúť
zdola jednotlivé zlomky, ktoré sa sčítajú na ľavej strane.
Najjednoduchšie by bolo, keby platili štyri odhady typu
$$
\frac{1}{a^2+b^2-1}\geqq\frac{1}{4}.
$$
Potrebovali by sme tak nerovnosť $a^2+b^2\leqq5$,
ktorej platnosť pre naše čísla $a$, $b$ (z~intervalu $\langle1,2\rangle$)
však zaručenú nemáme.\fnote{Neovplyvní to ani podmienka $(a+c)(b+d)=8$.}
Preto sme na úvod odvodili dolný odhad~\thetag1 s~nekonštantnou pravou stranou,
závislou od súčinu čísel $a$ a $b$. To sa ukázalo výhodné
aj preto, že
súčin $ab$ je jedným zo štyroch sčítancov, ktoré dostaneme
roznásobením súčinu $(a+c)(b+d)$ so zadanou hodnotou 8.
Výhoda sa potom prejavila po využití užitočnej
nerovnosti \thetag4 pre
{\it súčin dvoch súčtov} (pracovný termín upresnený
v texte riešenia), ktorá vo všeobecnej podobe z úlohy N3 stojí za
zapamätanie. Dodajme, že túto nerovnosť
je možné s úspechom uplatniť tiež rovno na súčet
na ľavej strane pôvodnej nerovnosti (iné riešenie nižšie).

Poznamenajme ešte, že v~druhej časti riešenia sme sa zaoberali iba
podmienkou rovnosti v~\thetag2. Tá je v skutočnosti ekvivalentná
s~rovnosťami \thetag5. Záverečnú skúšku
dosadením sme aj napriek tomu museli vykonať, lebo sme neposúdili
rovnosť v druhej použitej nerovnosti \thetag4. Tá podľa úlohy N3 nastane
práve vtedy, keď platí $ab=bc=cd=da$, čiže $a=c$ a $b=d$. Tieto dve
rovnosti však obe dvojice nájdené podľa \thetag5 spĺňajú, teda
nutnosť skúšky pri odvodení podmienok $a=c$ a $b=d$ odpadá.

\ineriesenie
Označme $L$ ľavú stranu dokazovanej
nerovnosti. Rovnako ako v~prvom riešení zdôvodníme, že $L$ je
súčtom prevrátených hodnôt štyroch kladných čísel, ktorých
súčet je pritom zrejme rovný $2(a^2+b^2+c^2+d^2-2)$.
Podľa nerovnosti pre {\it súčin dvoch súčtov},
o~ktorej sme sa zmienili v~prvom
riešení, tak platí
$$
2(a^2+b^2+c^2+d^2-2)\cdot L\geqq 4^2=16,\quad\hbox{odkiaľ}
\quad L\geqq\frac{8}{a^2+b^2+c^2+d^2-2}.
$$
Želaná nerovnosť $L\geqq1$ tak bude dokázaná,
ak overíme, že platí
$$
a^2+b^2+c^2+d^2-2\leqq8.
\tag6
$$
Za tým účelom každé z~čísel ${-2}$ a $8$ prevedieme na opačnú stranu
nerovnosti a druhé z~nich zameníme súčinom $(a+c)(b+d)$. Zostane nám tak
overiť nerovnosť
$$
a^2+b^2+c^2+d^2-(a+c)(b+d)\leqq2.
$$
Jej ľavá strana je však zrejme rovná súčtu
$$
\frac12(a-b)^2+\frac12(b-c)^2+\frac12(c-d)^2+\frac12(d-a)^2,
\tag7
$$
ktorý naozaj neprevyšuje 2, pretože každá zo štyroch
zastúpených druhých mocnín neprevyšuje 1 (vďaka tomu, že čísla $a$, $b$,
$c$, $d$ ležia v~intervale dĺžky 1). Tým je sľúbené overenie hotové.

Ak v~dokázanej nerovnosti nastane rovnosť, musí sa každá
z~druhých mocnín zastúpených v~\thetag7 rovnať 1, t.\,j. musia platiť rovnosti
\thetag5 z~prvého riešenia. Preto rovnako ako tam určíme
dve štvorice $(2, 1, 2, 1)$
a $(1, 2, 1, 2)$ a overíme, že pre ne rovnosť naozaj nastáva.

\poznamka
Naznačme malú obmenu práve podaného riešenia,
pri ktorej dokazovanú nerovnosť pre súčet štyroch zlomkov
dostaneme sčítaním dvoch odhadov
$$\eqalign{
\frac{1}{a^2+b^2-1}+\frac{1}{c^2+d^2-1}&\geqq\frac12,\cr
\frac{1}{b^2+c^2-1}+\frac{1}{d^2+a^2-1}&\geqq\frac12.
}$$
Tieto dva odhady dokážeme súčasne. Ich ľavé strany, ktoré
označíme $L_{1,2}$, sú totiž súčty prevrátených hodnôt
dvoch kladných čísel, pritom súčet týchto dvoch čísel
je v~oboch prípadoch rovný $a^2+b^2+c^2+d^2-2$.
Podľa nerovnosti z úlohy N3, tentoraz použitej pre $n=2$,
teda platí
$$
\bigl(a^2+b^2+c^2+d^2-2\bigr)\cdot L_{1,2}\geqq 2^2=4.
$$
Oba želané odhady $L_{1,2}\geqq\frac12$ tak opäť vyplývajú z~nerovnosti
\thetag6, ktorou sme sa zaoberali v~predchádzajúcom riešení.

\návody

Pre ľubovoľné reálne čísla $x$, $y$, $z$ dokážte\hfil\break
a) $x^2+y^2+z^2\geqq2(x+y)z-2xy$,\quad b) $2+x^2\bigl(1+y^2\bigr)\geqq2x(1 +y)$.\hfil\penalty-100
[Nerovnosť z a) upravte na $(x+y-z)^2\geqq0$, z b) na
$(xy-1)^2+(x-1)^2\geq0$.]

Reálne čísla $a$, $b$ ležia v~intervale $\langle 1, 2 \rangle$.
Ukážte, že platia nasledujúce nerovnosti:\hfil\break
a) $a^2 + b^2 \leq 1 + 2ab$,\qquad b) $a^2+ b^2 \leq \frac52 ab$,\hfil\break
c) $2\leq a/b+b/a\leq\frac52$,\qquad d) $a^2+2b^2\leqq(2a+1)b+3$. \hskip1cm\hfil\penalty-100
[Nerovnosť z~a) upravte na $(a-b)^2 \leq 1$, nerovnosť z~b)
na $(2a-b)\cdot(2b-a)\geqq0$. Ľavá nerovnosť z~c) platí
pre ľubovoľné $a,b>0$ a možno ju dokázať napríklad úpravou na
$(a-b)^2 \geq 0$ (alebo použitím A-G nerovnosti pre dve čísla $a/b$ a
$b/a$). Pravá nerovnosť z~c) vyplýva z~b). Nerovnosť z d) upravte
na $(a-b)^2+(b-\frac12)^2\leqq\frac{13}{4}$ a~využite to, že
$|a-b|\leqq1$ a $0<b-\frac12\leqq\frac32$.]

Dokážte, že pre ľubovoľnú $n$-ticu kladných reálnych čísel
$a_1,a_2,\dots,a_n$ platí
$$
(a_1+a_2+\ldots+a_n)\cdot
\left(1/a_1+ 1/a_2+\ldots+1/a_n\right)\geq n^2.
$$
Kedy nastane rovnosť?
[Po roznásobení dostanete $n$ jednotiek a $n(n-1)/2$ dvojíc zlomkov
$a_i/a_j$ a $a_j/a_i$ (kde $1\leqq i<j\leqq n$),
pritom súčet každých dvoch
takých zlomkov je aspoň 2 podľa riešenia časti c) z úlohy N2.
Rovnosť nastane práve vtedy, keď platí $a_i/a_j=a_j/a_i$ kedykoľvek $i\ne j$,
teda práve vtedy, keď všetky čísla $a_i$ sú rovnaké.]

Pre ľubovoľnú $n$-ticu kladných reálnych čísla $a_1, a_2, \dots, a_n$
definujeme ich aritmetický priemer ${\Cal A}_n$ a harmonický priemer
${\Cal H}_n$ vzorcami
$$
{\Cal A}_n=\frac{a_1+a_2+\ldots + a_n}{n}\quad\hbox{a}\quad
{\Cal H}_n =\frac{n}{1/a_1+ 1/a_2+\ldots+1/a_n}.
$$
Dokážte, že vždy platí ${\Cal A}_n\geq{\Cal H}_n$.
[Upravte na nerovnosť z~N3.]

\D
Dokážte, že pre ľubovoľné kladné reálne čísla $a$, $b$, $c$ platí
$$
\frac{a}{b+c} + \frac{b}{c+a} + \frac{c}{a+b}\geq\frac32.
$$
[Pričítajte ku každému zlomku 1 a použite N3 pre trojicu
$b+c$, $c+a$, $a+b$.]

Pre ľubovoľné kladné reálne čísla $x$, $y$, $z$ dokážte
nerovnosť
$$
(x+y+z)\Bigl(\frac{1}{x}+\frac{1}{y}+\frac{1}{z}\Bigr)\leqq m^2,\quad
\text{pričom}\ m=\min\Bigl(\frac{x}{y}+\frac{y}{z}+\frac{z}{x},
\frac{y}{x}+\frac{z}{y}+\frac{x}{z}\Bigr).
$$
Zistite tiež, kedy v~dokázanej nerovnosti nastane rovnosť.
[\pdfklink{63-A-I-2}{https://skmo.sk/dokument.php?id=992\#page=2}]

Dokážte, že pre ľubovoľné reálne čísla $a$, $b$, $c$
z~intervalu $\langle 0, 1 \rangle$ platí
$$
\frac{a}{1+bc} + \frac{b}{1+ca} + \frac{c}{1+ab} \leq 2.
$$
[Vzhľadom na symetriu určite môžeme predpokladať,
že $a\geq\max(b,c)$.
Prvý zlomok je zrejme najviac 1. Druhý zlomok je
najviac $b/(a+c)$, pretože z~nerovnosti $(1-a)(1-c)\geq 0$
vyplýva $1+ca\geq a +c$.
Podobne tretí zlomok je najviac $c/(a+b)$.
Stačí tak dokázať, že súčet $b/(a+c)+c/(a+b)$ je najviac 1. To
však vďaka predpokladu $a\geq\max(b,c)$ platí, lebo potom
$b/(a+c)\leq b/(b+c)$ a $c/(a+b)\leqq c/(b+c)$.]

Pre reálne čísla $a$, $b$ platí $9a^2+8ab+7b^2\leq6$. Dokážte, že
$7a+5b+12ab \leq 9$.
[Všimnime si, že v~oboch uvedených nerovnostiach nastáva rovnosť pre
$a=b=\frac12$. Uplatnime preto odhad $\bigl(a-\frac12\bigr)^2 \geq 0$ v~tvare
$a\leq a^2+\frac14$ a jeho obdobu $b\leq b^2+\frac14$.
Dostaneme $7a+5b+12ab\leqq 7\bigl(a^2+\frac14\bigr)+5\bigl(b^2+\frac14\bigr)+12ab$,
pritom výraz napravo je rovný
$\bigl(9a^2+8ab+7b^2\bigr)-2(a-b)^2+3$.]

Pre kladné reálne čísla $a$, $b$, $c$, $d$ platí $abcd=4$ a
$a^2+b^2+c^2+d^2=10$. Určite najväčšiu možnú hodnotu výrazu
$ab+bc+cd+da$.\hfil\break
[\pdfklink{https://skmo.sk/dokument.php?id=994\#page=9}{https://skmo.sk/dokument.php?id=994\#page=9}]

Nájdite najmenšie kladné reálne číslo $t$ s nasledujúcou
vlastnosťou: Kedykoľvek reálne čísla $a$, $b$, $c$, $d$ spĺňajú
rovnosti $a+b+c+d=6$ a $a^2+b^2+c^2+d^2=10$, možno z týchto čísel
vybrať dve, ktorých rozdiel má absolútnu hodnotu najviac 4. [\pdfklink{https://iksko.org/files/1/vzorak1.pdf\#page=1}{https://iksko.org/files/1/vzorak1.pdf\#page=1}]

\endnávod
}

{%%%%%   B-I-1
Vyjdeme z toho, že pri delení tromi je zvyšok súčtu niekoľkých celých čísel
rovnaký ako zvyšok čísla, ktoré dostaneme sčítaním zvyškov
jednotlivých sčítancov.\fnote{Táto zrejmá poučka platí pri delení
akýmkoľvek prirodzeným číslom, nielen číslom 3.}
Preto 10 zadaných čísel rozdelíme podľa ich zvyškov
(po delení tromi~-- to ďalej už nebudeme písať) do troch množín
$$
A_0=\{0,3,6,9\},\quad A_1=\{1,4,7\},\quad A_2=\{2,5,8\}.
$$
Sčítance z~$A_0$ nijako neovplyvňujú zvyšky skúmaných súčtov.
Každý z týchto výsledných zvyškov je jednoznačne určený dvoma počtami:
počtom sčítancov z~$A_1$, ktorý označíme $x$,
a~počtom sčítancov z~$A_2$, ktorý označíme $y$. Zrejme
$x,y\in\{0,1,2,3\}$, teda prebratím všetkých možných dvojíc $(x,y)$
môžeme otestovať, pre ktoré z nich je súčet $x$~čísel z~$A_1$
a $y$~čísel z~$A_2$ deliteľný tromi.\fnote{Rýchlejší postup je možné
založiť na tom, že spomínaný súčet je deliteľný tromi
práve vtedy, keď je deliteľný tromi súčet $x+2y$, ktorý je možné
výhodne zameniť číslom o~$3y$ menším, teda rozdielom $x-y$.
Pre čísla $x,y\in\{0,1,2,3\}$ to znamená, že buď $x=y$, alebo
$|x-y|=3$.}

V dvoch skupinách teraz popíšeme (už bez vysvetlenia)
všetky typy vyhovujúcich výberov čísel z~oboch množín $A_1$,
$A_2$ a uvedieme ich počty.

\smallskip

\item{$\bullet$}
V~prípade, že z~$A_1$ nevyberieme žiadne číslo, z~$A_2$
musíme vybrať 0 alebo 3~čísla. Rovnaký záver o~výbere
z~$A_2$ platí aj v~prípade, keď z~$A_1$ vyberieme 3 čísla.
V~oboch prípadoch dokopy tak máme pre výber čísel
z~$A_1\cup A_2$ štyri možnosti (vrátane jedného \uv{prázdneho} výberu):
$\{\,\}$, $\{2,5,8\}$, $\{1,4,7\}$,
$\{1,4,7,2,5,8\}$.\fnote{Pre prázdnu množinu sme dali
prednosť druhému z dvoch bežných označení $\emptyset$ a $\{\,\}$.}

\item{$\bullet$}
Zostávajú prípady, keď z~$A_1$ vyberieme 1 alebo 2 čísla. Potom rovnaký počet
čísel musíme vybrať aj z~$A_2$. Keďže každá trojprvková množina má
3~jednoprvkové a~3~dvojprvkové podmnožiny, máme dokopy ďalších
$3\cdot3+3\cdot3=18$ možností výberu čísel z~$A_1\cup A_2$
(nebudeme ich tu samozrejme vypisovať).

\smallskip\noindent
Existuje teda celkom $4+18=22$ vyhovujúcich výberov čísel
z~$A_1\cup A_2$. Každý z~nich je možné doplniť o~niektoré čísla zo 4-prvkovej
množiny~$A_0$ práve $2^4=16$ spôsobmi (pozri úlohu N1),
lebo pritom musíme počítať
aj s~\uv{prázdnym} doplnením. Počet všetkých vyhovujúcich výberov
z~$A_0\cup A_1\cup A_2$ je teda rovný $22\cdot16=352$, kde je
avšak započítaný aj \uv{prázdny} výber. Preto je hľadaný počet 351.

\zaver
Požadovanú vlastnosť má 351 neprázdnych podmnožín.

\ineriesenie
Budeme riešiť všeobecnejšiu úlohu: Pre každé celé $k\geqq0$ nájsť
počet $p(k)$ tých nie nutne neprázdnych podmnožín množiny
$\{0,1,\ldots,3k\}$, ktoré majú súčet prvkov deliteľný
tromi.\fnote{Za súčet prvkov prázdnej množiny považujeme číslo~0.}
Odpoveď na otázku z pôvodnej úlohy potom bude $p(3)-1$.

V~prípade $k=0$ je daná množina $\{0\}$, takže zrejme platí
$p(0)=2$ (vyhovujú obe podmnožiny $\{\,\}$ a $\{0\}$).

Ďalej odvodíme vzorec, podľa ktorého je možné hodnotu $p(k+1)$
vypočítať z~hodnôt $k$ a~$p(k)$, a to pre každé
celé $k\geqq0$.\fnote{V takejto situácii hovoríme, že
hodnoty $p(0), p(1), p(2),\dots$ sú určené \emph{rekurentne}.}

Pri danom $k\geqq0$ môžeme každú podmnožinu množiny
$\{0,1,\ldots,3k+3\}$ zostrojiť tak, že najskôr vyberieme
podmnožinu množiny $\{0,1,\ldots,3k\}$ a tú potom
zjednotíme s~podmnožinou množiny~$\{3k+1,3k+2,3k+3\}$.
Posúďme všetky možnosti, kedy takto vznikne
množina so súčtom prvkov, ktorý je deliteľný tromi.

\smallskip

\item{$\bullet$}
Ak vyberieme podmnožinu~$M$ množiny $\{0,1,\ldots,3k\}$ so súčtom
prvkov deliteľným tromi, musíme ju potom zjednotiť s~jednou zo štyroch
množín
$$
\{\,\},\quad\{3k+3\},\quad \{3k+1,3k+2\},\quad \{3k+1,3k+2,3k+3\}.
$$
Keďže takýchto podmnožín $M$ je práve $p(k)$, dostaneme z nich
prvých $4p(k)$ podmnožín, ktoré patria do hľadaného počtu $p(k+1)$.

\item{$\bullet$}
Podmnožín množiny $\{0,1,\ldots,3k\}$, ktoré sme v~predchádzajúcom
odseku neuvažovali, je práve $2^{3k+1}-p(k)$.
Ľubovoľná z nich má súčet prvkov, ktorý pri delení tromi dáva
buď zvyšok~1, alebo zvyšok~2. V~prípade zvyšku 1 potom musíme
takúto podmnožinu zjednotiť z~jednou z dvoch množín
$$
\{3k+2\}, \{3k+2,3k+3\},
$$
v~prípade zvyšku 2 s~jednou z dvoch množín
$$
\{3k+1\}, \{3k+1,3k+3\}.
$$
Bez ohľadu na to, koľko je prvých prípadov a koľko je tých
druhých,\fnote{Počty oboch prípadov sú v skutočnosti rovnaké.
Všetky podmnožiny $M_1$ so \uv{súčtovým} zvyškom 1 možno totiž
spárovať so všetkými podmnožinami $M_2$ so \uv{súčtovým} zvyškom 2
nasledovne: V~ľubovoľnom páre $(M_1,M_2)$ jednu
z~množín dostaneme z druhej množiny tak, že v~nej každé
zastúpené číslo~$c$ zameníme číslom $3k-c$ (premyslite si).}
je jasné, že dokopy
dostaneme ďalších $2\bigl(2^{3k+1}-p(k)\bigr)$ podmnožín,
ktoré patria do hľadaného počtu $p(k+1)$.

\smallskip\noindent
Sčítaním oboch čiastkových počtov už získame sľúbený vzorec:
$$
p(k+1)=4p(k)+2\bigl(2^{3k+1}-p(k)\bigr)=
2\bigl(2^{3k+1}+p(k)\bigr).
$$
Jeho opakovaným použitím zo známej hodnoty $p(0)=2$ postupne určíme
$p(1)=8$, $p(2)=48$, $p(3)=352$, \dots (ďalšie hodnoty $p(k)$
už na riešenie pôvodnej úlohy nepotrebujeme).

\poznamka
Použitím princípu matematickej indukcie možno ľahko overiť, že hodnota
$p(k)$ z~práve podaného riešenia je určená explicitným vzorcom
$$
p(k)=\frac{\bigl(2^{2k}+2\bigr)\cdot2^{k+1}}{3}.
$$
V~dôsledku toho platí nerovnosť $p(k)>\frac13\cdot2^{3k+1}$,
ktorú možno zaujímavo interpretovať:
Viac ako tretina zo všetkých podmnožín zadanej $(3k+1)$-prvkovej množiny
$\{0,1,\ldots,3k\}$ má súčet prvkov deliteľný tromi.

Dodajme, že explicitný vzorec pre $p(k)$ možno tiež
odvodiť priamo bez použitia rekurentnej metódy, a to pozoruhodným
postupom, ktorý uvedieme v~nasledujúcom riešení. Prekvapivo pri
ňom využijeme poznatky o~dvojkovej sústave pri zapisovaní
prirodzených čísel.\fnote{O~takom zapisovaní čísel sa
možno dočítať v~kapitole 4 brožúrky \pdfklink{\emph{O~dělitelnosti čísel celých}}{https://www.dml.cz/handle/10338.dmlcz/403960} od Antonína Vrby.}

\ineriesenie
Znovu sa budeme venovať zovšeobecneniu súťažnej úlohy z predchádzajúceho
riešenia: Nájsť pre každé celé $k\geqq0$ počet $p(k)$
tých nie nutne neprázdnych podmnožín množiny $\{0,1,\ldots,3k\}$,
ktoré majú súčet prvkov deliteľný tromi.

Úvodná časť bude rovnaká ako v~prvom riešení.
Zadaných $3k+1$ čísel rozdelíme podľa ich zvyškov (po delení tromi
ako aj všade ďalej) do troch množín
$$
A_0=\{0,3,\dots 3k\},\quad A_1=\{1,4,\dots,3k-2\},\quad
A_2=\{2,5,\dots,3k-1\}
$$
a každú vyhovujúcu podmnožinu budeme konštruovať v tvare
$M_0\cup M$, kde $M_0\subseteq A_0$ a~$M\subseteq A_1\cup A_2$.
Keďže čísla z~$M_0$ sú deliteľné tromi, môžeme za $M_0$ vybrať
ktorúkoľvek z $2^{k+1}$ podmnožín $(k+1)$-prvkovej množiny $A_0$. Preto
budeme hľadať hodnotu $p(k)$ v~tvare
$$
p(k)=2^{k+1}\cdot q(k),
\tag1
$$
kde $q(k)$ je počet tých podmnožín $M$ množiny $A_1\cup A_2$,
ktoré majú súčet prvkov deliteľný tromi.

Ako vieme z~prvého riešenia, hodnota $p(k)$ je úplne určená tým,
že množina $A_1$ je zložená z~$k$ rôznych čísel so zvyškom 1
a množina $A_2$ z~$k$ rôznych čísel so zvyškom~2. Vezmeme preto
iné dve množiny týchto vlastností, konkrétne
$$
A'_1=\{2^0,2^2,\dots,2^{2k-2}\}\quad\hbox{a}\quad
A'_2=\{2^1,2^3,\dots,2^{2k-1}\}
$$
(využili sme to, že 1 je zvyšok čísla $2^{2j}$ a 2 je zvyšok
čísla $2^{2j+1}$ pre každé celé $j\geq0$) a~hľadajme hodnotu $q(k)$
ako počet tých podmnožín v~$A'_1\cup A'_2$, ktoré majú
súčet prvkov deliteľný tromi. Súčet prvkov {\it akejkoľvek\/}
množiny $M\subseteq A'_1\cup A'_2$ je číslo z~množiny
$B=\{0,1,2,\dots, 2^{2k}-1\}$, pretože $2^{2k}-1$ je súčet {\it
všetkých\/}
čísel z~množiny $A'_1\cup A'_2$.\fnote{Číslo $2^{2k}-1$ je totiž v~dvojkovej sústave zapísané $2k$ jednotkami.}
Naopak, z~jednoznačnosti zápisu čísel
v dvojkovej sústave vyplýva, že pre každé $s\in B$ existuje práve jedna
množina $M\subseteq A'_1\cup A'_2$ so súčtom prvkov rovným
číslu $s$. Vďaka tejto bijekcii\fnote{Týmto termínom
sa označujú zobrazenia, ktoré sú vzájomne jednoznačné.}
medzi množinou $B$ a množinou všetkých
podmnožín~$M$ množiny $A'_1\cup A'_2$ dochádzame k~záveru, že
hľadaný počet $q(k)$ je rovný počtu tých čísel z~$B$, ktoré sú
deliteľné tromi. Keďže najmenšie číslo 0 aj najväčšie číslo $2^{2k}-1$
z~množiny~$B$ sú deliteľné tromi, je počet násobkov čísla 3
v~množine $B$ rovný $1+\bigl(2^{2k}-1\bigr)/3$, čiže
${\bigl(2^{2k}+2\bigr)}/3$. Po dosadení tejto hodnoty za $q(k)$ do
vzťahu~\thetag1 už získame výsledný vzorec
$$
p(k)=2^{k+1}\cdot\frac{2^{2k}+2}{3}=
\frac{\bigl(2^{2k}+2\bigr)\cdot2^{k+1}}{3}
$$
ktorý sme predtým uviedli v~poznámke 1 bez dôkazu.

\poznamka
Podobne ako v predchádzajúcich dvoch riešeniach je možné tiež
dokázať nasledujúce tvrdenia.

Pre každé celé $n\geqq0$ označme $P(n)$ počet tých nie nevyhnutne
neprázdnych podmnožín množiny $\{0,1,\ldots,n\}$,
ktoré majú súčet prvkov deliteľný tromi. Potom
$P(1)=2$, $P(2)=4$ a pre každé celé
číslo $k\geqq0$ platí
$$
P(3k+4)=2\cdot\bigl(2^{3k+2}+P(3k+1)\bigr),\quad\hbox{explicitne}\quad
P(3k+1) = \frac{\bigl(2^{2k+1}+1\bigr)\cdot 2^{k+1}}{3},
$$
a
$$
P(3k+5)=2\cdot\bigl(2^{3k+3}+P(3k+2)\bigr),\quad\hbox{explicitne}\quad
P(3k+2)=\frac{\bigl(2^{2k+1}+1\bigr)\cdot 2^{k+2}}{3}.
$$
(Hodnoty $P(3k)$ sme predtým mali pod označením $p(k)$.)

\návody
Pre každé prirodzené číslo $n$ dokážte: Počet všetkých
podmnožín $n$-prvkovej množiny je rovný~$2^n$.
[Predstavme si, že ľubovoľnú podmnožinu postupne
zostrojujeme: Pre každý prvok sa rozhodujeme, či ho vyberieme alebo nie.
Keďže týchto výberov je~$n$ a~sú navzájom nezávislé,
je rôznych výsledkov konštrukcie práve~$2^n$.]

Koľko neprázdnych podmnožín množiny $\{0,1,2,3,4,5\}$
má párny súčet prvkov?
[31. V~danej množine sú tri párne a tri nepárne čísla.
Uvedomme si, že súčet prvkov jej podmnožiny je
párny práve vtedy, keď je v~nej párny počet nepárnych čísel -- teda buď žiadne
nepárne číslo (1 možnosť), alebo 2 nepárne čísla (tie je možné vybrať 3~spôsobmi). Počet všetkých vyhovujúcich výberov nepárnych čísel je tak
rovný 4. Každý z nich je potom možné doplniť o niektoré (prípadne aj žiadne)
z troch párnych čísel práve $2^3=8$ spôsobmi. Od výsledku $4\cdot 8=32$
treba odčítať 1, pretože sme započítali aj podmnožinu zloženú
z~0 párnych a 0 nepárnych čísel, teda prázdnu podmnožinu.]

Koľkými spôsobmi je možné z~množiny~$\{1,2,\ldots,9\}$ vybrať dve čísla
so súčtom deliteľným tromi?
[12 spôsobov. Podľa zvyškov po delení tromi
rozdelíme 9 zadaných čísel do troch množín $A_0=\{1,4,7\}$,
$A_1=\{2,5,8\}$ a $A_2=\{3,6,9\}$. Dve z týchto čísel majú
súčet deliteľný tromi práve vtedy, keď nastane jeden z dvoch prípadov:
buď jedno číslo je z~$A_1$ a druhé z~$A_2$, alebo sú obe čísla z~$A_0$.
Pre výber dvoch vyhovujúcich čísel máme v~prvom prípade $3\cdot3=9$
možností, v~druhom prípade 3~možnosti.]

\D
Označme $M$ počet všetkých možných vyplnení tabuľky $3\times 3$ navzájom rôznymi prirodzenými číslami od $1$ do $9$.
Ďalej označme $N$ počet tých vyplnení, kde sú navyše súčty všetkých čísel v~každom riadku aj stĺpci nepárne čísla. Určte pomer $N:M$.
[\pdfklink{72-B-I-2}{https://skmo.sk/dokument.php?id=4362}]

Označme $M$ počet všetkých možných vyplnení tabuľky $3 \times 3$
navzájom rôznymi prirodzenými číslami od~$1$ do~$9$.
Ďalej označme $D$ počet tých vyplnení, keď je navyše \emph{súčin} čísel v~niektorom riadku alebo stĺpci násobkom desiatich.
Určte pomer $D : M$.
[\pdfklink{72-B-S-1}{https://skmo.sk/dokument.php?id=4368}]

Koľko $33$-ciferných čísel deliteľných~$3$ neobsahuje vo svojom zápise cifru~3? Výsledok zapíšte v tvare súčinu mocnín prvočísel.
[\pdfklink{72-B-II-4}{https://skmo.sk/dokument.php?id=4449\#page=5}]

Určte počet neprázdnych podmnožín množiny
$\{0,1,\ldots,2n-1\}$ s párnym súčtom prvkov, kde $n$ je
dané prirodzené číslo (zovšeobecnenie úlohy N2).
[$2^{2n-1}-1$. V~danej množine je $n$ párnych a $n$ nepárnych čísel.
Hľadáme počet podmnožín s párnym počtom nepárnych čísel. Ukážeme
najprv, že počet spôsobov, akými je možné z~$n$~nepárnych čísel vybrať
párny počet zástupcov, je rovný $2^{n-1}$. Na to stačí
dokázať, že medzi všetkými
$2^n$ podmnožinami danej $n$-prvkovej množiny je tých s párnym
počtom prvkov rovnako ako tých s~nepárnym počtom prvkov, lebo oba
počty sa potom rovnajú číslu $2^n:2=2^{n-1}$. Pre dôkaz zvolíme pevne
jeden prvok $a$ z~danej $n$-prvkovej množiny a všetky jej
podmnožiny rozdelíme na dve skupiny podľa toho, či prvok $a$
obsahujú alebo nie. Zástupcov týchto dvoch skupín možno spárovať:
každú množinu $M$ bez prvku $a$ dáme do páru s~množinou $M \cup \{a\}$.
Keďže v~každom páre je zrejme jedna množina s párnym a jedna s~nepárnym
počtom prvkov, je dôkaz hotový. Na dokončenie riešenia zvážime, že
každý z~$2^{n-1}$ vyhovujúcich výberov nepárnych čísel možno doplniť
o~niektoré (prípadne žiadne) z~$n$ párnych čísel práve $2^n$ spôsobmi.
Od výsledku ${2^{n-1}\cdot2^n}$ je potrebné odčítať 1, pretože sme
započítali aj podmnožinu zloženú z 0 párnych a 0 nepárnych čísel,
teda prázdnu podmnožinu.]

Riešte toto zovšeobecnenie súťažnej úlohy:
Pre každé celé $k$ označme~$p(k)$ počet tých podmnožín množiny
$\{0,1,\ldots,3k\}$, ktoré majú súčet prvkov deliteľný
tromi (započítame medzi ne aj prázdnu množinu). Dokážte vzorec
$p(k)=\frac13\bigl(2^{2k}+2\bigr)\cdot2^{k+1}$.
{\it Návod}: Odvoďte najprv, že pre každé $k$ platí
$p(k+1)=2\bigl(2^{3k+1}+p(k)\bigr)$, a potom využite
matematickú indukciu.
[Úplné riešenie nájdete v~celom dokumente k~domácemu kolu, keď bude po jeho skončení zverejnený
na internetových \pdfklink{stránkach MO}{https://skmo.sk}.]


{\everypar{}
\smallskip
\emph{Doplňujúca literatúra}:

Na pripomenutie pravidiel o deliteľnosti odporúčame
brožúrku Antonína Vrby \pdfklink{\emph{O~dělitelnosti čísel celých}}{https://www.dml.cz/handle/10338.dmlcz/403960} edície \pdfklink{\emph{Škola mladých matematikov}}{https: //www.dml.cz/handle/10338.dmlcz/403423}. Prácu so zvyškami po delení v obore celých celých
uľahčujú zápisy, ktoré majú rovnaký názov ako ďalšia brožúrka Aloise Apfelbecka
\pdfklink{\emph{Kongruence}}{https://www.dml.cz/handle/10338.dmlcz/403650}.
Nájdete v~nej nielen ich zavedenie, ale aj zaujímavé príklady použitia.
Pravidlá pre určovanie rôznych \uv{veľkých} počtov výberov možností
alebo počtov určitých typov skupín prvkov nájdete v~brožúrke Antonína Vrby
\pdfklink{\emph{Kombinatorika}}{https://www.dml.cz/handle/10338.dmlcz/403960}.
Napokon v~brožúrke Rudolfa Výborného
\pdfklink{\emph{Matematická indukce}}{https://www.dml.cz/handle/10338.dmlcz/404198}
nájdete úvodné poučenie o~tomto významnom matematickom princípe
a~následné rozmanité príklady jeho využitia.
\smallskip
}

\endnávod
}

{%%%%%   B-I-2
Zo zadania vyplýva, že čísla $a$, $b$, $c$ spĺňajú podmienky
$$
b+c\ne 0,\quad c+a\ne 0,\quad a+b\ne 0.
\tag{P}
$$
Za predpokladov \thetag{P} ekvivalentne upravíme prvú zo zadaných rovností:
$$
\displaylines{
\frac{a}{b+c} = \frac{b}{c+a}, \cr
a(c+a) = b(b+c),\cr
ac+a^2 = b^2+bc,\cr
(a-b)(a+b) + c(a-b) = 0,\cr
(a-b)(a+b+c) = 0.
}
$$
Analogickými úpravami druhej rovnosti a tiež rovnosti tretieho
zlomku s~prvým zlomkom dostaneme
$$
(b-c)(a+b+c)=0 \qquad\hbox{a}\qquad (c-a)(a+b+c)=0.
$$
Vidíme, že za predpokladov \thetag{P} reálne čísla $a$, $b$, $c$ spĺňajú
zadanie práve vtedy, keď platí $a+b+c=0$ alebo $a=b=c$.

V~prípade, keď $a+b+c=0$, pre zadaný výraz máme
$$
\frac{a^3+b^3+c^3}{(b+c)^3 + (c+a)^3 + (a+b)^3} =
\frac{a^3+b^3+c^3}{(-a)^3 + (-b)^3 + (-c)^3} = -1,
$$
ak platí $a^3+b^3+c^3\ne0$, inak výraz nemá zmysel.

V druhom prípade, keď $a=b=c$, máme
$$
\frac{a^3+b^3+c^3}{(b+c)^3 + (c+a)^3 + (a+b)^3}=
\frac{a^3+a^3+a^3}{(2a)^3 + (2a)^3 + (2a)^3} = \frac18,
$$
tentoraz za podmienky, že $a^3\ne0$.

Zostáva ukázať, že každá z oboch nájdených hodnôt je dosiahnuteľná.

Podmienky \thetag{P} a rovnosť $a+b+c=0$ sú splnené napríklad
pre trojicu $(a,b,c)=(1,1,{-2})$, pre ktorú má výraz
$a^3+b^3+c^3$ (nenulovú) hodnotu 6.

Podmienky \thetag{P} a rovnosť $a=b=c$ sú splnené napríklad pre trojicu
$(a,b,c)=(1,1,1)$, pre ktorú má výraz $a^3$ (nenulovú)
hodnotu 1.

\zaver
Zadaný výraz má jediné dve možné hodnoty: ${-1}$ a $1/8$.

\poznamka
Bez overenia podmienky $a^3+b^3+c^3\ne0$ pre vybranú trojicu $(1,1,{-2})$
by podané riešenie nebolo úplné. Síce prácnejšie, ale vo výsledku zaujímavejšie
je ukázať, že za podmienok \thetag{P} {\it nikdy\/}
neplatia obe rovnosti $a+b+c=0$ a $a^3+b^3+c^3=0$.
Pokiaľ totiž platí prvá z~ich, môžeme do druhej rovnosti dosadiť
$c={-a}-b$ a ďalej jej ľavú stranu upraviť:
$$
a^3+b^3+c^3=a^3+b^3+(-a-b)^3=-3a^2b-3ab^2=-3ab(a+b)=
-3(b+c)(c+a)(a+b),
$$
kde sme v~poslednom kroku zamenili činitele $a$, $b$ za
${-b}-c$, resp. ${-c}-a$. Zostáva dodať, že vďaka podmienkam \thetag{P}
nie je súčin $(b+c)(c+a)(a+b)$ rovný nule.


\ineriesenie
Označme $k$ spoločnú hodnotu zadaných troch zlomkov. Potom zrejme
platia rovnosti
$$
a = k (b + c), \quad b = k (c + a), \quad c = k (a + b).
\tag1
$$
Akonáhle určíme možné hodnoty $k$, budeme s~riešením hotoví.
Pokiaľ totiž má zadaný výraz zmysel, je jeho hodnota rovná $k^3$,
a to vďaka rovnostiam \thetag{1} umocneným na tretiu:
$$
\frac{a^3+b^3+c^3}{(b+c)^3+(c+a)^3+(a+b)^3}=
\frac{k^3(b+c)^3+k^3(c+a)^3+k^3(a+b)^3}{(b+c)^3+(c+a) ^3+(a+b)^3}=k^3.
$$

Pre určenie možných hodnôt $k$ rovnosti \thetag{1} sčítame. Dostaneme
$$
(a+b+c)=2k(a+b+c),\quad\hbox{odkiaľ}\quad (a+b+c)(1-2k)=0.
$$
Platí teda $a+b+c=0$ alebo $1-2k=0$. Prvá rovnosť znamená
$k={-1}$, druhá $k=1/2$. Že sú obe tieto hodnoty $k$ dosiahnuteľné,
ukazujú rovnaké príklady trojíc $(a,b,c)$ ako v~prvom riešení.
Keďže pre ne má zadaný výraz zmysel, jediné jeho možné hodnoty
sú $({-1})^3={-1}$ a $(1/2)^3=1/8$.

\poznamka
Ukážme, že na určenie možných hodnôt $k$ môžeme so sústavou rovností~\thetag{1}
narábať aj inak. Odčítaním druhej rovnosti od prvej dostaneme $a-b=k(b-a)$
čiže $(a-b)(k+1)=0$. Analogicky získame $(b-c)(k+1)=0$.
Platí teda $k={-1}$ alebo $a-b=b-c=0$. Posledné však znamená
$a=b=c$, čomu zodpovedá $k=1/2$.

\ineriesenie
Upravíme zadané rovnosti troch zlomkov, a to tak, že ku každému
z~ich pripočítame 1. Dostaneme
$$
\frac{a+b+c}{b+c}=\frac{a+b+c}{c+a}=\frac{a+b+c}{a+b}.
$$
Platí teda $a+b+c=0$ alebo
$$
\frac{1}{b+c}=\frac{1}{c+a}=\frac{1}{a+b},\quad\hbox{odkiaľ}
\quad b+c=c+a=a+b.
$$
Z~posledných dvoch rovností však vyplýva $a=b=c$. Ďalej už môžeme
pokračovať ako v~prvom riešení.

\návody
Pre reálne čísla platí $a/(b+1)=b/(a+1)$.
Dokážte, že čísla $a$, $b$ sú rovnaké
alebo sa ich súčet rovná ${-1}$.
[Zadanú rovnosť upravíme na $a^2+a=b^2+b$, ďalej na
$a^2-b^2=b-a$ a napokon použitím rozkladu $a^2-b^2=(a-b)(a+b)$ na
$(a-b)(a+b+1)=0$. Odtiaľ už vyplýva tvrdenie.]

Pre reálne čísla $a$, $b$, $c$ platí $a/b=b/c=c/a$.
Určte všetky možné hodnoty súčtu $a/(b+c)+b/(c+a)+c/(a+b)$.
[3/2. Čísla $a$, $b$, $c$ sú nenulové vďaka existencii podielov
$a/b$, $b/c$, $c/a$, ktorých spoločnú hodnotu označíme $k$.
Vynásobením rovností $k=a/b$, $k=b/c$, $k=c/a$ dostaneme
$k^3=1$, odkiaľ vyplýva $k=1$ (je možné odvolať sa na graf funkcie
$y=x^3$, alebo využiť rozklad $k^3-1=(k-1)(k^2+k+1)$, kde
$k^2+k+1>0$, nech je reálne číslo~$k$ akékoľvek). Rovnosť $k=1$
podľa určenia čísla~$k$ však znamená, že $a=b=c\ne0$, teda
podiely $a/(b+c)$, $b/(c+a)$, $c/(a+b)$ majú zmysel a rovnakú
hodnotu $1/2$.]

\D
Určte, aké hodnoty môže nadobúdať výraz
$$\frac{a+bc}{a+b}+\frac{b+ca}{b+c}+\frac{c+ab}{c+a},$$
ak sú $a$, $b$, $c$ kladné reálne čísla so súčtom~1.
[\pdfklink{70-C-I-4}{https://skmo.sk/dokument.php?id=3473\#page=4}]

Nech $a$, $b$, $c$ sú kladné reálne čísla, pre ktoré platí $ab+bc+ca=1$.
Určte, aké hodnoty nadobúda výraz
{\postdisplaypenalty=50
$$
\frac{a(b^2+1)}{a+b}+\frac{b(c^2+1)}{b+c}+\frac{c(a^2+1)}{c+a}\vadjust{\nobreak}.
$$}
[\pdfklink{70-C-S-3}{https://skmo.sk/dokument.php?id=3477\#page=2}]

Nech $x$, $y$, $z$ sú kladné reálne čísla, ktorých súčin je
rovný 1. Dokážte, že platí rovnosť
$$
\frac{1}{1+x+xy}+\frac{1}{1+y+yz}+\frac{1}{1+z+zx}=1.
$$
[Prvý zlomok rozšírte $z$, druhý $xz$ a trikrát využite
podmienku $xyz=1$.]

Pre reálne čísla $x$, $y$, $z$ platí
$$
|x+y|=1-z,\quad|y+z|=1-x,\quad |z+x|=1-y.
$$
Zistite, aké všetky hodnoty môže nadobúdať súčet $x+y+z$. Pre
každý vyhovujúci súčet uveďte príklad prislúchajúcich čísel
$x$, $y$, $z$.
[\pdfklink{70-B-S-1}{https://skmo.sk/dokument.php?id=3476}]

Pre reálne čísla $a$, $b$, $c$ sú oba súčty $a+b+c$ a
$a^3+b^3+c^3$ rovné nule. Nájdite všetky možné hodnoty súčinu
$abc$.
[0. Zapíšme, že nula sa rovná súčtu $a^3+b^3+c^3$, kam priamo
dosadíme $c={-a}-b$:
$$
0=a^3+b^3+(-a-b)^3=a^3+b^3-\bigl(a^3+3a^2b+3ab^2+b^3\bigr)=-3ab(a+b).
$$
Vidíme, že nulové je aspoň jedno z~čísel $a$, $b$, $a+b$,
pritom tretie z~ich je rovné~${-c}$. Odtiaľ už vyplýva $abc=0$.]

Nájdite všetky reálne riešenia sústavy rovníc
$$
{1\over x + y}+ z~= 1,\quad
{1\over y + z}+ x = 1,\quad
{1\over z~+ x}+ y = 1.
$$
[\pdfklink{69-A-II-1}{https://skmo.sk/dokument.php?id=3386}]

Pre nenulové reálne čísla $a$, $b$, $c$ platí
$a^2-b^2=bc$ a $b^2-c^2=ca$. Ukážte, že potom $a^2-c^2=ab$.
[Sčítaním daných rovností obdržíme $a^2-c^2=bc+ca$, takže stačí
overiť rovnosť $bc+ca=ab$ alebo $a(b-c)=bc$. Za tým účelom dané rovnosti
upravíme na tvar $a^2=b(b+c)$, $(b-c)(b+c)=ca$ a potom medzi sebou vynásobíme,
čím dostaneme $a^2(b-c)(b+c)=abc(b+c)$. Odtiaľ už
po vydelení oboch strán súčinom $a(b+c)$ získame overovanú rovnosť.
Spomínané vydelenie je korektné, lebo podľa zadania platí $a\ne0$ a~prípadná rovnosť $b+c=0$ by spolu s~$b^2-c^2=ca$ viedla k~rovnosti
$0=ca$, ktorá je v spore s~nenulovosťou čísel $a$, $c$.]
\endnávod
}

{%%%%%   B-I-3
Uvedieme tri pozorovania, z ktorých vyplynie zhodnosť štyroch uhlov
vyznačených na \obr{} dvoma oblúčikmi. Vďaka dvom z~nich -- uhlom $AFE$ a $GFE$ -- priamka~$CE$ naozaj rozpoľuje uhol $AFG$.
\inspsc{b73i.31}{.8333}%

\item{$\bullet$}
Keďže oba uhly $BAE$ a $BFE$ sú pravé (a ich vrcholy
$A$, resp. $F$ ležia v~opačných polrovinách s~hraničnou priamkou $BE$),
podľa Tálesovej vety je štvoruholník $ABFE$ tetivový.
V~kružnici jemu opísanej tak (podľa vety o~obvodových uhloch)
platí $|\angle AFE|=|\angle ABE|$.

\item{$\bullet$}
Trojuholníky $ABE$ a~$DCE$ sú zhodné podľa vety $sus$, pretože
$|AB|=|DC|$, $|AE|=|DE|$ a oba uhly $BAE$ a~$CDE$ sú pravé.
Preto platí $|\angle ABE|=|\angle DCE|$.\fnote{Namiesto použitia vety
$sus$ stačilo konštatovať známu súmernosť pravouholníka $ABCD$
podľa spoločnej osi protiľahlých strán $BC$ a $AD$
(ktorá prechádza bodom $E$).}

\item{$\bullet$}
Úsečky $CD$ a $FG$ sú kolmé na stranu $AD$, a teda navzájom
rovnobežné. Podľa vety o~súhlasných uhloch tak platí
$|\angle DCE|=|\angle GFE|$.

\smallskip\noindent
Dokopy už dostávame
$$
|\angle AFE|=|\angle ABE|=|\angle DCE|=|\angle GFE|,
$$
ako sme chceli dokázať.

\ineriesenie
Tentoraz potrebnú zhodnosť uhlov $AFE$ a $GFE$ dokážeme použitím pomocného
bodu $H$, ktorý zavedieme ako priesečník priamok $CE$ a~$AB$ (\obr).

Trojuholníky $AEH$ a~$DEC$ sú zhodné podľa vety $usu$,
lebo $|AE|=|DE|$, pri vrcholoch $A$, $D$ majú pravé uhly a
pri vrchole $E$ zhodné vrcholové uhly. Vďaka tomu platí
$|AH|=|DC|$, teda aj $|AH|=|AB|$.\fnote{Táto rovnosť vyplýva aj z~toho, že $AE$ je stredná priečka trojuholníka
$BCH$. Je totiž rovnobežná so stranou $BC$ a má oproti nej
polovičnú dĺžku.} Bod~$A$ je tak stredom úsečky~$HB$,
ktorá je preponou
pravouhlého trojuholníka $HBF$. Stred kružnice jemu opísanej je
podľa Tálesovej vety teda práve bod $A$. Platí preto
$|AF|=|AH|$, teda v~trojuholníku $AFH$ sú zhodné vnútorné uhly
$AHF$ a~$AFH$, čo je uhol $AFE$. K~jeho zhodnosti s~uhlom
$GFE$ tak už len zostáva dokázať zhodnosť uhlov
$AHF$ a $GFE$. To sú však striedavé uhly
medzi dvoma kolmicami $AH$ a~$FG$ na stranu $AD$,
teda medzi dvoma rovnobežkami.
\inspsc{b73i.32}{.8333}%

\návody
{\everypar{}
\smallskip
Na úvod pripomenieme, že konvexný štvoruholník $ABCD$ je tetivový práve vtedy, keď platí
ktorákoľvek z~podmienok:

\noindent\llap{$\triangleright$ }%
Súčet niektorých dvoch jeho protiľahlých vnútorných uhlov je $180^\circ$,
napríklad tých pri vrcholoch $B$ a $D$:
$|\angle ABC|+|\angle CDA|=180^\circ$. (Vtedy na tetivovosť
akéhokoľvek štvoruholníka $ABCD$ stačí, aby body $B$ a $D$ ležali
vo vnútri opačných polrovín s~hraničnou priamkou~$AC$.)

\noindent\llap{$\triangleright$ }%
Uhly \uv{nad} niektorou jeho stranou sú zhodné, napríklad
nad stranou $AB$ sa jedná o~uhly s~vrcholmi $C$ a $D$:
$|\angle ACB|=|\angle ADB|$. (Vtedy na tetivovosť
akéhokoľvek štvoruholníka $ABCD$ či $ABDC$ stačí,
aby body $C$ a $D$ ležali vo vnútri rovnakej polroviny
s~hraničnou priamkou~$AB$.)

Pri riešení úloh často najprv použitím jednej z~týchto podmienok
tetivovosť niektorého konvexného štvoruholníka dokážeme a potom
vhodne využijeme platnosť druhej podmienky.
\smallskip
}

V~konvexnom štvoruholníku $ABCD$ platí rovnosť
$|\angle BAD|=42^\circ$, $|\angle ABC|=79^\circ$,
$|\angle DCB|=138^\circ$ a $|\angle BDC|=25^\circ$.
Určte $|\angle ACB|$.
[$76^\circ$. Vďaka súčtu $42^\circ+138^\circ=180^\circ$
je štvoruholník $ABCD$ tetivový, takže v~ňom
platí $|\angle ACB|=|\angle ADB|$ a~$|\angle ADC|=180^\circ-|\angle ABC|=101^\circ$,
teda $|\angle ADB|=|\angle ADC|-|\angle BDC|=101^\circ-25^\circ=76^\circ$.]

Je daný ostrouhlý trojuholník $ABC$ s~výškami $AD$
a~$BE$.\fnote{Ako zvyčajne \emph{výškou trojuholníka\/} rozumieme
\emph{úsečku}, ktorú popisujeme jej krajnými bodmi.}
Dokážte, že stred kružnice opísanej trojuholníku $CDE$
leží na výške trojuholníka $ABC$ z~vrcholu~$C$.
[Označme~$H$ priesečník výšok $AD$ a~$BE$. Vďaka pravým uhlom $CEH$ a~$CDH$
je podľa Tálesovej vety štvoruholník $CEHD$ tetivový
a stred kružnice jemu opísanej je stredom úsečky~$CH$, teda
naozaj leží na tretej výške trojuholníka $ABC$.]

Je daný ostrouhlý trojuholník $ABC$ s~výškami $AD$, $BE$ a $CF$.
Dokážte, že tieto výšky rozpoľujú vnútorné uhly trojuholníka $DEF$.
[Vzhľadom na symetriu stačí dokázať iba rovnosť
$|\angle DFC|=|\angle EFC|$.
Vďaka pravým uhlom nad stranou~$AC$
je štvoruholník $AFDC$ tetivový, odkiaľ $|\angle DFC|=|\angle DAC|$.
Podobne je tetivový aj štvoruholník $ABDE$, takže
$|\angle DAC|=|\angle DAE|=|\angle DBE|=|\angle CBE|$.
Napokon aj štvoruholník $CEFB$ je tetivový,
takže $|\angle CBE|=|\angle CFE|$. Dokopy už máme
$|\angle DFC|=|\angle CFE|$, ako sme sľúbili dokázať.
Kratší dôkaz zhodnosti uhlov $DFC$ a $EFC$: V~tetivovom
štvoruholníku $ACDF$ sú zhodné uhly $DFC$ a~$DAC$, pritom
druhý z~ich má z~$\triangle DAC$ veľkosť
$90^{\circ}-\gamma$, kde $\gamma=|\angle BCA|$. Uhol $DFC$ tak má
veľkosť $90^{\circ}-\gamma$, ktorá sa nezmení, ak vymeníme
navzájom označenie vrcholov $A$ a $B$. Túto veľkosť preto má aj uhol
$EFC$.]

\D
Sú dané dva tetivové štvoruholníky $ABXY$ a $CDYX$, pritom
ich spoločné vrcholy $X$ a $Y$ ležia postupne na úsečkách
$AC$ a $BD$. Dokážte, že platí $AB\parallel CD$.
[Stačí dokázať zhodnosť striedavých uhlov $BAC$ a $DCA$, teda uhlov
$BAX$ a $DCX$ (lebo $X$ leží medzi $A$ a $C$).
Využijeme na to vlastnosti oboch tetivových
štvoruholníkov a to, že uhly $BYX$ a $XYD$ sú vedľajšie
(lebo $Y$ leží medzi $B$ a $D$):
$|\angle BAX|=|\angle BYX|=180^{\circ}-|\angle
XYD|=180^{\circ}-(180^{\circ}-|\angle DCX|)=|\angle DCX|.$]

Nech $D$ je vnútorný bod prepony~$AB$ pravouhlého trojuholníka $ABC$.
Označme $X$ a $Y$ stredy kružníc opísaných postupne
trojuholníkom $ADC$ a~$CDB$. Ukážte, že body $C$, $D$, $X$ a $Y$
ležia na jednej kružnici.
[Keďže uhly $CAD$ a $DBC$ sú ostré,
stredy $X$ a $Y$ ležia v~opačných polrovinách s~hraničnou priamkou
$DC$ a podľa vety o~obvodovom a~stredovom uhle pre veľkosti
konvexných stredových uhlov $CXD$ a~$DYC$
platí $|\angle CXD|=2|\angle CAD|$ a $|\angle DYC|=2|\angle DBC|$.
Sčítaním oboch rovností dostaneme
$|\angle CXD|+|\angle DYC|=2(|\angle CAD|+|\angle
DBC|)=2\cdot90^\circ=180^\circ$, teda $CXDY$ je tetivový
štvoruholník. Dokonca platí, že kružnica jemu opísaná má
priemer $XY$, pretože trojuholníky $CXY$ a $DXY$ sú zhodné podľa vety
$sss$, takže v~tetivovom štvoruholníku $CXDY$ sú uhly
pri protiľahlých vrcholoch $C$ a~$D$ zhodné, a teda pravé.]

Je daný ostrouhlý trojuholník $ABC$. Dotyčnice v~bodoch $A$, $B$ ku
kružnici tomuto trojuholníku opísanej sa pretínajú v~bode $T$.
Predpokladajme, že priamka rovnobežná so stranou $AC$, ktorá
prechádza bodom $T$, pretína stranu $BC$ v~bode $D$.
Ukážte, že $|AD| = |CD|$.
[Kľúčom k~riešeniu je odhalenie
tetivovosti štvoruholníka $ATBD$. Na dôkaz tohto poznatku
stačí overiť, že oba uhly $BAT$ a $BDT$ majú rovnakú veľkosť
$\gamma=|\uhol BCA|$ -- podľa zadania totiž oba body $A$, $D$ ležia
s~celou opísanou kružnicou na jednej strane od jej dotyčnice $BT$.
Pre prvý uhol $TAB$ to priamo vyplýva z~vety o~obvodovom a
úsekovom uhle, pre druhý uhol $BDT$ je to dôsledok súhlasnej
rovnobežnosti úsečiek $AC$ a $TD$. Štvoruholník $ATBD$
je teda naozaj tetivový.
Teraz už dokazovaná rovnosť $|AD|=|CD|$ ľahko vyplynie z~toho,
že v~trojuholníku $CAD$ má veľkosť $\gamma$ nielen uhol $DCA$, ale
aj uhol $CAD$. Ten je totiž zhodný so striedavým uhlom $ADT$,
ten vďaka nášmu odhaleniu s~uhlom $ABT$, ktorý je napokon podľa
$|TA|=|TB|$ zhodný s~uhlom $TAB$, o~ktorého veľkosti~$\gamma$ už vieme.]

Nech $AC$ je priemer kružnice opísanej tetivovému štvoruholníku $ABCD$. Predpokladajme,
že na polpriamkach opačných
k~polpriamkam $AD$ a~$DC$ existujú postupne body $A'\ne A$ a~$C'\ne D$ také, že
platí $|AB|=|A'B|$ a~$|BC|=|BC'|$. Dokážte tvrdenia:
\hfil\break
a) Body $A'$, $B$, $C'$ a~$D$ ležia na jednej kružnici~$k$.\hfil\break
b) Ak je $O$ stred kružnice $k$ a~$O_A$, $O_C$ sú postupne stredy
kružníc opísaných trojuholníkom $AA'B$, $CC'B$, tak platí $OO_A\perp OO_C$.
[\pdfklink{69-B-I-3}{https://skmo.sk/dokument.php?id=3387\#page=7}]

Na stranách $AB$ a~$BC$ daného trojuholníka $ABC$ ležia postupne také body~$D$ a~$E$, že $|BD|=|DC|=|CA|$ a~$|EC|=|ED|$. Dokážte, že $|AE|=|BE|$.
[\pdfklink{72-B-II-3}{https://skmo.sk/dokument.php?id=4449\#page=3}]

Daný je pravouhlý trojuholník $ABC$ s pravým uhlom pri vrchole $C$.
Nech $D$ je ľubovoľný vnútorný bod odvesny $AC$ a $p$ kolmica z bodu $D$
na preponu $AB$. Označme $E\ne D$ bod priamky $p$ taký, že
body $A$, $B$, $D$, $E$ ležia na kružnici. Označme ešte~$F$
priesečník priamok $p$ a $BC$. Dokážte, že $|AE|=|AF|$.
[\pdfklink{70-B-II-3}{https://skmo.sk/dokument.php?id=3604\#page=2}]

Daný je rovnoramenný trojuholník $ABC$ so základňou $AB$ a bod~$P$
vnútri jeho výšky z~vrcholu $C$.
Priamka $AP$ pretína kružnicu opísanú trojuholníku $ABC$ v~bode~$Q\ne A$.
Rovnobežka so základňou $AB$ vedená bodom $P$ pretína rameno~$BC$
v~bode~$R$. Dokážte, že polpriamka $QR$ je osou uhla $AQB$.
[\pdfklink{71-A-II-3}{https://skmo.sk/dokument.php?id=3921\#page=3}]

{\everypar{}
\smallskip
K~téme súťažnej úlohy (uhly, ktoré spájame s~kružnicami,
tetivové štvoruholníky) odporúčame brožúrku Stanislava Horáka \pdfklink{\emph{Kružnice}}{https://www.dml.cz/handle/10338.dmlcz/403589}.
Nájdete v nej aj dôkazy všetkých poznatkov, ktoré sme pripomenuli pred
uvedením úlohy N1.
\smallskip
}

\endnávod
}

{%%%%%   B-I-4
{\it Časť\/} (i).
Najskôr určíme, ktoré deliteľnosti majú platiť.

V~zadanej pätici $a,a,a,a,b$ sú dve rôzne čísla $a$ a $b$. Číslo $a$
má deliť jednak súčet $a+a+a$, čo platí pre každé $a$, jednak
súčet $a+a+b$, čo zrejme nastane práve vtedy, keď $a\mid b$.
Na deliteľnosť číslom $a$ tak máme jedinú podmienku $a\mid b$.

Číslo $b$ má deliť jediný súčet $a+a+a=3a$, má teda platiť $b\mid 3a$.

Máme teda rozhodnúť, či môže zároveň platiť $a\mid b$ a
$b\mid 3a$. Odpoveď je áno, ako potvrdzuje príklad $a=1$ a $b=3$,
kedy zadaná pätica je $1,1,1,1,3$.

Aj keď to zadanie úlohy nevyžaduje, ukážeme, že všetky vyhovujúce pätice
sú $a,a,a,a,3a$. Naozaj, prvá podmienka $a\mid b$ znamená, že
$b=ka$ pre vhodné prirodzené číslo $k$. Zvyšnú druhú podmienku
$b\mid 3a$ potom môžeme prepísať ako $ka\mid 3a$ alebo $k\mid
3$, a tak $k=1$ alebo $k=3$. Podľa zadania však platí $a\ne b$,
teda $a\ne ka$, odkiaľ $k\ne1$. Jediné vyhovujúce $k$ je tak 3.

\smallskip
{\it Časť\/} (ii). Aj teraz najskôr určíme deliteľnosti, ktoré majú
čísla z~pätice $a,a,b,b,c$ spĺňať. (Zistíme, že deliteľnosti
číslom $c$ nebude nutné posudzovať.)

Číslo $a$ má deliť tri súčty $a+b+b$, $a+b+c$, $b+b+c$.
Rozdiel posledných dvoch súčtov je $a-b$ a jeho deliteľnosť číslom $a$
je ekvivalentná s~podmienkou $a\mid b$. Ak je splnená, potom
z troch výrazov $a+2b$, $a+b+c$, $2b+c$ je číslom $a$ zaručene
deliteľný len ten prvý, zatiaľ čo pre zvyšné dva výrazy to potom
platí práve vtedy, keď $a\mid c$. Preto na deliteľnosť číslom $a$
máme celkom dve podmienky
$$
a\mid b\quad\hbox{a}\quad a\mid c.
\tag1
$$
Číslo $b$ má deliť tri súčty $a+a+b$, $a+a+c$, $a+b+c$.
To nastane práve vtedy, keď $b$ je deliteľom postupne výrazov
$2a$, $2a+c$, $a+c$. Rozdiel posledných
dvoch súčtov je $a$ a jeho deliteľnosť číslom $b$, teda podmienka
$b\mid a$, redukuje naše tri deliteľnosti číslom~$b$
opäť na dve:\fnote{Keďže zastúpenie čísel $a$ a $b$ je
v~pätici $a,a,b,b,c$ rovnocenné, je možné podmienky \thetag2 rovno získať
z~podmienok \thetag1 tak, že v~nich čísla $a$ a $b$ navzájom vymeníme.}
$$
b\mid a\quad\hbox{a}\quad b\mid c.
\tag2
$$
Podmienky na deliteľnosť tretím číslom $c$ už nie je nutné
rozoberať, pokiaľ si všimneme, že podľa \thetag1 a \thetag2 má súčasne
platiť $a\mid b$ a $b\mid a$, čo zrejme (pozri úlohu N3) znamená,
že $a=b$. Podľa zadania má však platiť $a\ne b$, teda žiadna
vyhovujúca pätica $a,a,b,b,c$ neexistuje.

\poznamkac1.
Z~nášho postupu v~časti (ii) vyplýva, že
pätica $a,a,b,b,c$ neexistuje, ani
keď zo zadania úlohy vynecháme podmienky deliteľnosti
číslom $c$.

\poznamkac2.
Podané riešenia oboch častí sme začali
systematickým výpisom {\it všetkých\/} (aj tých triviálne splnených)
podmienok deliteľnosti, ktoré sme pre každého deliteľa rovno čo
najviac zjednodušili. Úplné riešenie však možno zapísať stručnejšie.
Vysvetlíme teraz ako.

V~časti (i) môžeme nejaký vyhovujúci príklad rovno vypísať
(ako keby sme ho uhádli) a potom vykonať skúšku (tá by však nemala chýbať).
Alebo sa na úvod rozhodneme výhodne zvoliť $a=1$
(číslo 1 je totiž samozrejmý deliteľ)
a potom si všimneme, že stačí,
aby zodpovedajúca pätica $1,1,1,1,b$ spĺňala
jedinú podmienku $b\mid 1+1+1$.

V~časti (ii) sa stačilo obmedziť na odvodenie podmienky $a\mid b$,
ďalej k~nej~-- podľa symetrie spomínanej v~poznámke pod čiarou~--
pripojiť podmienku $b\mid a$, a dôjsť tak k spornej rovnosti $a=b$.
Uveďme tiež jednu z dlhších alternatív (bez
odvodenia $a\mid b$ a $b\mid a$),
keď do výkladu zapojíme deliteľnosti číslom~$c$:
Najprv z~podmienok $a \mid a+b+b=a+2b$, $b \mid b+a+a=b+2a$,
čiže $a\mid2b$, $b\mid2a$ odvodíme,
že platí buď $a=2b$, alebo $b=2a$: Keby totiž ani
jedna z~týchto rovností neplatila, z~$a\mid2b$ by sme podľa
tvrdenia z~N4 mali
$2a\leqq 2b$, a teda $a<b$ (vďaka predpokladu $a\ne b$),
zatiaľ čo z~$b\mid2a$ by podobne vyplynulo $b<a$, dokopy spor.
Vzhľadom na symetriu
sa ďalej stačí zaoberať prípadom $b=2a$,
keď z~podmienok $c\mid a+2b$ a $c\mid b+2a$ máme
$c\mid(a+2b)-(b+2a)=b-a=a$, t.\,j. $c\mid a$. Zároveň však
z $a\mid 2b+c$ vzhľadom na $b=2a$ máme $a\mid c$. Dokopy
tak (opäť vďaka~N3) dochádzame k spornej rovnosti $a=c$.

\návody

Pre celé čísla $n$, $a$, $b$ platí $n\mid a$ a $n\mid b$.
Dokážte, že potom pre ľubovoľné celé čísla $k$,~$l$ platí
$n\mid ka+lb$ (špeciálne napríklad $n\mid a+b$ a $n\mid a-b$).
[Podľa podmienok $n\mid a$ a $n\mid b$ existujú celé čísla
$a'$ a $b'$ tak, že $a=a'n$ a $b=b'n$. Potom $ka+lb=ka'n+lb'n=
(ka'+lb')n$, kde $ka'+lb'$ je celé číslo, a preto $n\mid ka+lb$.]

Pre ktoré prirodzené čísla $n$ je zaručené, že celé čísla $u$, $v$
spĺňajúce obe podmienky $n\mid u+v$ a $n\mid u-v$ sú samy
deliteľné číslom $n$?
[Práve pre nepárne $n$. Z~$n\mid u+v$ a~$n\mid u-v$ vyplýva,
že $n\mid 2u$ a $n\mid 2v$, pretože $2u=(u+v)+(u-v)$ a
$2v=(u+v)-(u-v)$. Ak je $n$ nepárne, je $n$ s~číslom~2 nesúdeliteľné,
a preto z~$n\mid 2u$ a $n\mid 2v$ už vyplýva $n\mid u$, resp. $n\mid v$.
Pre párne $n$ uvážte protipríklad $u=v=n/2$.]

Ukážte, že ak pre prirodzené čísla $a$ a~$b$
platí $a \mid b$ a $b \mid a$, potom $a=b$.
[Keďže pre každý kladný deliteľ $d$ daného prirodzeného čísla $u$
zrejme platí $d\leqq u$, z~$a \mid b$ a $b \mid a$ vyplýva postupne
$a\leqq b$ a $b\leqq a$, spolu $a=b$.
Inak môžeme zapísať rovnosti $a=kb$ a~$b=la$ pre
vhodné prirodzené čísla $k$ a~$l$, odkiaľ po vynásobení dostaneme
$ab=kbla$ a~ďalej po vydelení $ab$ obdržíme $kl=1$, čo znamená,
že nutne $k=l=1.$]

Ukážte, že ak pre rôzne prirodzené čísla $u$ a~$v$
platí $u \mid v$, tak $2u\leqq v$.
[Z~$u \mid v$ vyplýva $v=ku$ pre vhodné prirodzené číslo, pritom
z~$u\ne v$ vyplýva $k\ne1$, teda $k\geqq2$. Preto $v=ku\geqq2u$.]

\D
Pre prirodzené čísla $a$, $b$ platí $a\mid9b$ a $b\mid9a$.
Určte všetky možné hodnoty podielu $a/b$.
[$9,3,1,1/3,1/9$. Vzťah $a\mid 9b$ znamená $9b=ka$
pre vhodné prirodzené $k$. Preto teraz $b \mid 9a$ prepíšeme
ako $9b\mid 81a$, odkiaľ po dosadení za $9b$ dostaneme
vzťah $ka\mid 81a$, čiže $k\mid 81$. Číslo $k$ je teda
jeden z~deliteľov $1,3,9,27,81$ čísla~81.
Keďže z~$9b=ka$ vyplýva $a/b=9/k$, každá hodnota $a/b$ sa musí
rovnať jednému z čísel $9,3,1,1/3,1/9$. Všetky tieto hodnoty
sú dosiahnuteľné, napríklad dvojicami $(a,b)$ z~množiny
$\{(9,1), (3,1), (1,1), (1,3), (1,9)\}$. Iné riešenie: Hodnota podielu
$a/b$ ani platnosť podmienok $a\mid9b$, $b\mid9a$ sa nezmení, keď
čísla $a$, $b$ vydelíme ich najväčším spoločným deliteľom.
Preto stačí uvažovať len dvojice nesúdeliteľných čísel $a$ a $b$.
Pre ne sú podmienky $a\mid9b$ a $b\mid9a$ postupne
ekvivalentné s~podmienkami $a\mid9$ a $b\mid9$, takže stačí
vypočítať hodnoty $a/b$ pre všetky dvojice $(a,b)$ nesúdeliteľných
deliteľov čísla 9, teda pre dvojice z~množiny
$\{(9,1), (3,1), (1,1), (1,3), (1,9)\}$.]

V školskej záhrade hrá skupina žiakov hru zvanú molekuly. Učiteľ im najprv prikázal, aby sa rozdelili do trojíc.
Jeden žiak zvýšil, a tak z ďalšej hry vypadol. Zvyšní žiaci sa potom mali rozdeliť do štvoríc. Opäť jeden žiak zvýšil
a vypadol. Potom sa zvyšní žiaci mali rozdeliť do piatich, zase jeden žiak zvýšil a vypadol. Učiteľ teraz káže, aby
sa zvyšní žiaci rozdelili do šesť. Dokážte, že opäť jeden žiak zvýši.
[\pdfklink{71-C-I-1}{https://skmo.sk/dokument.php?id=3925}]

Určte všetky dvojice $(m,n)$ kladných celých čísel, pre ktoré je číslo $4(mn+1)$ deliteľné číslom $(m+n)^2$.
[\pdfklink{60-A-II-3}{https://skmo.sk/dokument.php?id=362\#page=4}]

Určte všetky celé kladné čísla $m$, $n$ také, že $n$ delí $2m-1$
a~$m$ delí $2n-1$. [\pdfklink{59-A-II-3}{https://skmo.sk/dokument.php?id=7\#page=4}]


Nájdite všetky trojice navzájom rôznych prvočísel $p$, $q$,
$r$ spĺňajúce tri podmienky $p\mid q+r$, $q\mid r+2p$ a $r\mid p+3q$.
[\pdfklink{55-A-III-5}{https://skmo.sk/dokument.php?id=235\#page=5}]


{\everypar{}
\smallskip
\emph{Doplňujúca literatúra}:

K súťažnej úlohe 4 odporúčame
brožúrky Františka Veselého \pdfklink{\emph{O~dělitelnosti čísel celých}}{https://www.dml.cz/handle/10338.dmlcz/403560}
a Aloisa Apfelbecka
\pdfklink{\emph{Kongruence}}{https://www.dml.cz/handle/10338.dmlcz/403650}
ako k~súťažnej úlohe 1.

\smallskip
}

\endnávod
}

{%%%%%   B-I-5
Označme $a$, $b$ dĺžky odvesien a $c$ dĺžku prepony ľubovoľného
pravouhlého trojuholníka. Polomer $R$ kružnice jemu opísanej je daný
vzorcom $R=\frac12 c$ vďaka Tálesovej vete, zatiaľ čo pre
polomer $r$ kružnice vpísanej je známy vzorec
$r=\frac12({a+b-c})$, ktorý dokážeme v~riešení úlohy N2.

Po tomto všeobecnom úvode prejdeme k samotnému riešeniu.
Zo zadaného vzťahu $r:R=2:5$ po dosadení uvedených vzorcov
dostaneme
$$
\frac{a+b-c}{c}=\frac25,
$$
čo ešte po rozpísaní ľavej strany na tri zlomky upravíme na tvar
$$
\frac ac + \frac bc = \frac 75.
\tag1
$$
Rovnosť $a^2+b^2=c^2$ z~Pytagorovej vety vzhľadom na
\thetag1 vydelíme $c^2$. Dostaneme
$$
\left(\frac ac\right)^2 + \left(\frac bc\right)^2=1.
\tag2
$$
Teraz je výhodné zaviesť označenie
$$
x=\frac ac \quad\hbox{a}\quad y=\frac bc
\tag3
$$
a rovnosti \thetag1, \thetag2 považovať za sústavu dvoch rovníc
$$
\eqalign{
x+y&={\textstyle\frac 75},\cr
x^2+y^2&=1
}$$
s~neznámymi $x$ a $y$. Tú vyriešime bežnou metódou:
Vyjadrenie ~$y=\frac75-x$ z~\thetag1 dosadíme do
\thetag2 a získame tak pre $x$ kvadratickú rovnicu
$x^2+(\frac 75-x)^2=1$ s~koreňmi $x_1=\frac35$ a~$x_2=\frac45$.
Tým podľa \thetag1 zodpovedajú hodnoty $y_1=\frac45$ a $y_2=\frac35$.
Naša sústava rovníc tak má práve dve riešenia
$$
(x_1,y_1)=\left(\frac 35, \frac 45\right)
\quad\hbox{a}\quad
(x_2,y_2)=\left(\frac 45,\frac 35\right).
$$
To podľa \thetag3 znamená, že trojica strán $(a,b,c)$ má pre každý
vyhovujúci trojuholník tvar $(3d,4d,5d)$ alebo
$(4d,3d,5d)$, kde $d$ označuje vhodnú dĺžku.\fnote{Možno
si všimnúť, že vďaka vzorcu $r=\frac12(a+b-c)$ je
dĺžka $d$ rovná práve polomeru $r$.}
Keďže dĺžka $4d$ je aritmetickým priemerom zvyšných dĺžok $3d$ a
$5d$, dôkaz je hotový.

\poznamka
Naznačme, ako je možné po odvodení vzťahu $(a+b-c)/c=2/5$
postupovať inak. Vyplýva z~neho vyjadrenie $c=5(a+b)/7$, ktoré
dosadíme za $c$ do rovnosti $c^2=a^2+b^2$ z~Pytagorovej vety.
Po jednoduchých úpravách dostaneme rovnosť
$$
12a^2 -25ab+12b^2 = 0.
$$
Vyriešením tejto kvadratickej rovnice s~neznámou $a$ a parametrom $b$
získame korene $a_1=3b/4$ a $a_2=4b/3$. Dosadením do $c=5(a+b)/7$
potom určíme $c_1=5b/4$ a $c_2=5b/3$. Dĺžky strán nášho trojuholníka
tak tvoria jednu z~trojíc $(3b/4,b,5b/4)$ alebo $(4b/3,b,5b/3)$,
z ktorých po preznačení $b=4d$, resp. $b=3d$ dostaneme rovnaké
trojice $(3d,4d,5d)$ a $(4d,3d,5d)$ ako v~pôvodnom riešení.

\ineriesenie
Tentoraz sa vyhneme priamemu použitiu vzorca $r=\frac12(a+b-c)$.
Pre potrebné geometrické úvahy využijeme \obr{}.
Na ňom je nakreslený pravouhlý trojuholník, kružnica jemu vpísaná
s~vyznačenými bodmi dotyku a tri im prislúchajúce polomery veľkosti $r$.
Tie rozdeľujú celý trojuholník na štvorec so stranou dĺžky $r$ a dva
štvoruholníky, ktoré sú deltoidmi vďaka ich súmernostiam
podľa vyznačených uhlopriečok ležiacich na osiach vnútorných uhlov trojuholníka.
Jeden deltoid tak má strany
dĺžok $r$, $r$, $x$, $x$ a druhý strany dĺžok $r$, $r$, $y$, $y$.
Odvesny trojuholníka majú teda dĺžky $x+r$, $y+r$ a jeho prepona
má dĺžku $x+y$. Preto podľa Pytagorovej vety platí
$$
(x+r)^2+(y+r)^2=(x+y)^2
\tag4
$$
a polomer $R$ kružnice opísanej má vďaka Tálesovej vete veľkosť
$R=\frac12(x+y)$. Po jej dosadení do zadanej podmienky
$r:R=2:5$ dostaneme $r=\frac15(x+y)$.
Z~toho vyplýva $x+y=5r$, a teda $y=5r-x$. Po dosadení
za $x+y$ a $y$ do \thetag4 tak dostaneme
$$
(x+r)^2+((5r-x)+r)^2=(5r)^2,\quad\hbox{po úprave}\quad
x^2-5rx+6r^2=0.
$$
To je kvadratická rovnica s~neznámou $x$ a~parametrom $r$,
ktorú môžeme vyriešiť aj rozkladom na $(x-2r)(x-3r)=0$.
Táto rovnica má teda dva korene $x_1=2r$ a $x_2=3r$, ktorým podľa
vzorca $y=5r-x$ zodpovedajú hodnoty $y_1=3r$ a $y_2=2r$. Trojica
$(x+r,y+r,5r)$ dĺžok strán nášho trojuholníka je preto jedna z~trojíc
$(3r,4r,5r)$ alebo $(4r,3r,5r)$, takže tvrdenie zo zadania úlohy
platí.
\inspsc{b73i.51}{.8333}%

\návody

Zdôvodnite, že polomer kružnice opísanej
pravouhlému trojuholníku je polovicou dĺžky jeho prepony.
[Podľa Tálesovej vety je prepona pravouhlého trojuholníka priemerom
kružnice jemu opísanej.]

Ukážte, že v~pravouhlom trojuholníku s~odvesnami dĺžok $a$, $b$
a preponou dĺžky~$c$ platí pre polomer $r$ kružnice jemu vpísanej
vzorec $r=\frac12(a+b-c)$.
[Body dotyku kružnice vpísanej rozdeľujú strany trojuholníka na šesť
úsekov. Dva z nich (tie pri vrchole pravého uhla) sú susednými stranami
štvorca so stranou dĺžky $r$. Celé odvesny tak
majú dĺžky $a=r+x$ a $b=r+y$, kde $x$ a $y$ sú ich dĺžky
úsekov pri vrcholoch ostrých uhlov. Tie sú (vďaka súmernostiam podľa osí
týchto uhlov) zhodné s dvoma úsekmi, na ktoré je rozdelená
prepona, ktorá má preto dĺžku $c=x+y$. Teraz je jasné,
že zrejmú rovnosť $2r=(r+x)+(r+y)-(x+y)$ môžeme prepísať
ako $2r=a+b-c$. Po vydelení číslom~2 sme s~dôkazom hotoví.]

Pravouhlý trojuholník o~odvesnami dĺžok~$a$, $b$ a preponou dĺžky~$c$
spĺňa podmienku $3a+4b=5c$.
Určte všetky možné hodnoty pomeru $a:c$.
[Zadanú rovnosť vydelíme~$c$ a dostaneme $3a/c+4b/c=5$.
Z~Pytagorovej vety máme $a^2+b^2=c^2$, takže $(a/c)^2+(b/c)^2=1$.
To nám dáva pre podiely $x=a/c$ a $y=b/c$ sústavu rovníc $3x+4y=5$ a
$x^2+y^2=1$. Ak dosadíme vyjadrenie~$y=(5-3x)/4$ z~prvej rovnice
do druhej rovnice, dostaneme po vynásobení číslom 16 rovnicu
$16x^2+(5-3x)^2=16$, po úprave $(5x-3)^2=0$. Odtiaľ $x=3/5$.
Nájdená hodnota $a/c=x=3/5$ pomeru $a:c$ je možná -- napríklad
$a=3$, $b=4$ a $c=5$ sú dĺžky strán pravouhlého trojuholníka a platí
pre ne $3a+4b=5c$.]

\D
Pre polomer $r$ kružnice vpísanej všeobecnému trojuholníku dokážte
vzorec $r=S/s$, kde $S$ je obsah tohto trojuholníka a $s$ je
polovica jeho obvodu.
[Označme~$I$ stred kružnice vpísanej trojuholníku $ABC$ so zvyčajne
označenými dĺžkami strán. Obsah~$S$ celého trojuholníka je súčtom
obsahov trojuholníkov $BCI$, $CAI$ a~$ABI$, ktoré sa postupne
rovnajú $\frac12ar$, $\frac12br$ a~$\frac12cr$. Z~rovnosti
$S=\frac12ar+\frac12br+\frac12cr$ už jednoduchou úpravou
dostaneme dokazovaný vzorec.]

Odvoďte vzorec~$r=\frac12(a+b-c)$ z~úlohy N1
použitím výsledku úlohy D1.
[Trojuholník zo zadania úlohy N1 má obsah $S=\frac12ab$.
Podľa vzorca z~D1 tak s~prihliadnutím na $c^2=a^2+b^2$ platí
$$\displaylines{
r=\frac{S}{s}=\frac{\frac{ab}2}{\frac{a+b+c}2}=
\frac{ab}{a+b+c}=
\frac{ab(a+b-c)}{(a+b+c)(a+b-c)}=
\frac{ab(a+b-c)}{(a+b)^2-c^2}=\cr=
\frac{ab(a+b-c)}{(a+b)^2-(a^2+b^2) }=
\frac{ab(a+b-c)}{2ab}=
\frac{a+b-c}{2}.]}
$$

Daný je pravouhlý trojuholník $ABC$ s~preponou $AB$. Dokážte, že veľkosť
jeho výšky $CD$ sa rovná súčtu polomerov kružníc vpísaných trojuholníkom
$ABC$, $CAD$ a $CBD$.
[Pri bežnom označení $a=|BC|$, $b=|CA|$, $c=|AB|$, $v=|CD|$,
$c_a=|AD|$ a $c_b=|BD|$ platia pre polomery $r$, $r_a$, $r_b$
kružníc vpísaných postupne pravouhlým trojuholníkom $ABC$, $CAD$ a $CBD$
podľa úlohy N2 vzorce
$$
r=\frac{a+b-c}{2},\quad r_a=\frac{c_a+b-c}{2}\quad\hbox{a}\quad
r_b=\frac{c_b+v-a}{2}.
$$
Ich sčítaním už dostaneme $r+r_a+r_b=v+\frac12(c_a+c_b-c)=v$,
lebo $c_a+c_b=c$.]

Pravouhlý trojuholník má celočíselné dĺžky strán a obvod $11\,990$.
Navyše vieme, že jedna jeho odvesna má prvočíselnú dĺžku. Určte
ju. [\pdfklink{71-B-I-1}{https://skmo.sk/dokument.php?id=3924}]

Pravouhlý trojuholník má celočíselné dĺžky strán. Jeho obvod je druhá mocnina
prirodzeného čísla. Tiež vieme, že jedna jeho odvesna má dĺžku
rovnú druhej mocnine prvočísla. Určte všetky možné hodnoty tejto
dĺžky. [\pdfklink{71-B-S-3}{https://skmo.sk/dokument.php?id=3928\#page=3}]

V~pravouhlom trojuholníku $ABC$ s~preponou~$AB$ a~odvesnami dĺžok
$|AC|=4$ a~$|BC|=3$ ležia navzájom sa dotýkajúce kružnice
$k_1(S_1,r_1)$ a~$k_2(S_2,r_2)$ tak, že $k_1$ sa dotýka strán $AB$ a~$AC$
a~$k_2$ sa dotýka strán $AB$ a~$BC$. Určte polomery $r_1$ a~$r_2$, ak platí
$4r_1=9r_2$. [\pdfklink{62-A-II-3}{https://skmo.sk/dokument.php?id=675\#page=2}]

\endnávod
}

{%%%%%   B-I-6
Dokážeme sporom, že to nie je možné.

Predpokladajme, že taká vyhovujúca tabuľka existuje.
Potom v~jej prvom riadku je číslo $a$
a tri čísla so súčtom $7a$,
v druhom číslo $b$ a tri čísla so súčtom $7b$, v treťom číslo $c$
a tri čísla so súčtom $7c$, vo štvrtom číslo $d$ a tri čísla so súčtom $7d$.
Súčty čísel v~riadkoch sú potom postupne $8a$, $8b$, $8c$ a $8d$, takže pre
súčet všetkých~16~zapísaných čísel platí
$$
8(a+b+c+d)=1+2+\ldots+16=(1+16)+(2+15)+\ldots+(8+9)=8\cdot17,
$$
odkiaľ vyplýva $a+b+c+d=17$. Keďže súčet akýchkoľvek troch čísel
z~tabuľky je menší ako $3\cdot16=48$, sú menšie ako 48 všetky
štyri čísla $7a$, $7b$, $7c$ a $7d$. Preto každé z~navzájom
rôznych čísel $a$, $b$, $c$, $d$ je menšie ako 7. Ich súčet
je však 17 a pritom $5+4+3+2=14<17$, preto jedno z~čísel
$a$, $b$, $c$, $d$ je rovné 6.

Ak zopakujeme predchádzajúcu úvahu pre stĺpce uvažovanej tabuľky,
zistíme dokopy, že číslo 6 má v~tabuľke takúto pozíciu:
číslu $7\cdot6=42$ sa rovná ako súčet troch ďalších čísel z~riadku
čísla~6, tak súčet troch ďalších čísel z~jeho stĺpca. Súčet šiestich
rôznych čísel tabuľky sa tak rovná číslu $2\cdot42=84$, pritom
však súčet šiestich najväčších čísel z~tabuľky je iba
$$
16+15+14+13+12+11=(16+11)+(15+12)+(14+13)=3\cdot27=81.
$$
Získaný spor existenciu vyhovujúcej tabuľky $4\times4$ vylučuje.

\poznamkac1.
V~tabuľke vyplnenej číslami od 1 do 16
$$
\vbox
{\offinterlineskip\everycr{\noalign{\hrule}}\let\par\cr\obeylines %
\halign{\vrule#&&\hbox to 16pt{\hss#\unskip\hss}\vrule height 11.2pt depth 4.8pt\cr
&3&1&10&2\cr
&8&4&11&9\cr
&7&15&5&13\cr
&6&12&14&16\cr
}}
$$
sa nájde -- s~výnimkou posledného stĺpca -- v~každom riadku aj stĺpci
číslo, ktorého sedemnásobok je rovný súčtu ostatných troch čísel
(v~riadkoch ide o~čísla $2$, $4$, $5$ a $6$, v stĺpcoch o~čísla
$3$, $4$ a $5$). Vidíme, že z ôsmich požiadaviek úlohy ich možno splniť
sedem.

\poznamkac2.
Ukážme, že náš dôkaz sporom je možné dokončiť inak. Zistili sme, že
číslo 6 musí zdieľať rovnaký riadok aj rovnaký stĺpec v oboch
prípadoch s tromi číslami, ktorých súčet sa rovná $7\cdot 6=42$.
Ľahko zistíme, že do úvahy prichádzajú iba súčty
$16+15+11$, $16+14+12$ a~$15+14+13$. Každé
dva z nich však majú spoločný sčítanec, teda v riadku a stĺpci
čísla 6 nemôže byť šesť ďalších navzájom rôznych čísel.

\ineriesenie
Dôkaz sporom začneme rovnako ako v~prvom riešení až do odvodenia
rovnosti $a+b+c+d=17$ so sčítancami menšími ako 7. Ďalej už budeme
pokračovať inak.

Čísla $a$, $b$, $c$, $d$ so súčtom 17 sú štyri sčítance
zo súčtu $1+2+3+4+5+6$ rovného~21. Zvyšné dva sčítance tak majú
súčet $21-17=4$, jedná sa preto o~čísla 1 a~3, teda platí
$\{a,b,c,d\}=\{2,4,5,6\}$. Zdôraznime, že je to štvorica čísel
z~rôznych riadkov a~analogicky aj z~rôznych stĺpcov.

Uvažujme teraz pozíciu čísla 2. Ako vieme,
v~jeho riadku aj v~jeho stĺpci sú po tri čísla so súčtom
rovným $7\cdot2=14$. Dokopy sa jedná o~šesť rôznych čísel so
súčtom $2\cdot14=28$, ktoré sú navyše rôzne od 2, 4, 5 a 6.
Súčet takých šiestich čísel je však aspoň
$1+3+7+8+9+10=38$, čo je spor.\fnote{Iné dokončenie: Obe
trojice čísel, ktoré zdieľajú s~číslom 2 rovnaký riadok alebo rovnaký
stĺpec, musia byť zložené z~čísel 1, 3 a 10, lebo iné tri
do úvahy prichádzajúce čísla s~požadovaným súčtom
$7\cdot2=14$ zrejme neexistujú.}

\návody

Rozhodnite, či je možné štvorcovú tabuľku $3 \times 3$ vyplniť
navzájom rôznymi prirodzenými číslami od 1 do 9 tak,
aby v~každom riadku existovalo číslo, ktoré sa rovná súčtu
zvyšných dvoch čísel.
[Nejde to. Dôkaz sporom: Majme vyhovujúcu tabuľku.
Ak je v~jej prvom riadku číslo $a$ a ďalšie dve čísla so súčtom $a$,
je súčet všetkých troch čísel rovný $2a$, čo je párne číslo.
Podobne sú párne súčty čísel v druhom aj treťom riadku, a teda aj
súčet všetkých~9 čísel v~tabuľke. Ten je však rovný $1+\ldots+9=45$, spor.]

Rozhodnite, či je možné štvorcovú tabuľku $3 \times 3$ vyplniť
navzájom rôznymi prirodzenými číslami od 1 do 9 tak,
aby v~každom riadku existovalo číslo, ktorého štvornásobok
sa rovná súčtu zvyšných dvoch čísel.
[Nejde to. Dôkaz sporom: Majme vyhovujúcu tabuľku. V~prvom riadku
je číslo $a$ a zvyšné dve čísla so súčtom $4a$, súčet všetkých troch čísel
tak je $5a$. Analogický význam ako $a$ bude mať číslo
$b$ z~druhého riadka a číslo $c$ z~tretieho riadka.
Súčet 45 všetkých 9 čísel v~tabuľke
je tak rovný $5(a+b+c)$, odkiaľ $a+b+c=9$. Keďže súčet akýchkoľvek
dvoch čísel z~tabuľky neprevyšuje $9+8=17$, platí to aj pre čísla
$4a$, $4b$ a $4c$, takže $a,b,c\in\{1,2,3,4\}$. Spolu s $a+b+c=9$
to znamená, že $\{a,b,c\}=\{2,3,4\}$. Číslo~4 tak zdieľa riadok
s~dvoma číslami so súčtom $4\cdot4=16$, ide teda o~čísla 7 a 9. Odtiaľ
už vyplýva, že číslo 3 nemôže zdieľať riadok s dvoma číslami so súčtom
$4\cdot3=12$, pretože každá z možných dvojíc $(3,9)$, $(4,8)$ a~$(5,7)$ je vylúčená kvôli riadku s~číslami 4, 7, 9.]

Tabuľka $4 \times 4$ je vyplnená rôznymi celými číslami od 1 do 16.
Isté číslo $s$ v~tejto tabuľke má tú vlastnosť, že jeho
štvornásobok je rovný ako súčtu ostatných troch čísel z~jeho riadka,
tak súčtu ostatných troch čísel z~jeho stĺpca.
Určite najväčšie možné takéto $s$.
[9. Nech vyhovujúce číslo $s$ zdieľa riadok s~trojicou čísel $(a,b,c)$,
a stĺpec s~trojicou $(d,e,f)$. Sčítaním rovností $4s=a+b+c$ a
$4s=d+e+f$ dostaneme, že číslo $8s$ je rovné súčtu šiestich rôznych
čísel z~tabuľky, ktorý neprevyšuje súčet čísel
od 11 do 16 rovný 81. Platí teda $8s\leqq81$, odkiaľ $s\leqq10$.
Hodnota $s=10$ však možná nie je: súčty $a+b+c$ a $d+e+f$
by museli byť (až na poradie sčítancov)
medzi súčtami
$16+15+9$, $16+13+11$, $15+14+11$ a~$15+13+12$~-- každé dva
z nich však majú spoločný sčítanec. Hodnota $s=9$ už možná je:
uvažujme tabuľku, v ktorej číslo 9
zdieľa riadok s~trojicou $(16,15,5)$
a~stĺpec s~trojicou $(14,12,10)$, ostatné čísla sú rozmiestnené
akokoľvek.]

\D
Tabuľka $10\times10$ je vyplnená číslami $1$ a ${-1}$ tak, že súčet
čísel v~každom riadku aj stĺpci je deliteľný tromi. Určte
najväčší možný súčet čísel v~tabuľke a dokážte, že väčší byť
nemôže. Uveďte tiež príklad tabuľky s~určeným najväčším
súčtom. [\pdfklink{71-C-S-1}{https://skmo.sk/dokument.php?id=3929}]

V školskej záhrade hrá skupina žiakov hru zvanú molekuly. Učiteľ im najprv prikázal, aby sa rozdelili do trojíc.
Jeden žiak zvýšil, a tak z ďalšej hry vypadol. Zvyšní žiaci sa potom mali rozdeliť do štvoríc. Opäť jeden žiak zvýšil
a vypadol. Potom sa zvyšní žiaci mali rozdeliť do piatich, zase jeden žiak zvýšil a vypadol. Učiteľ teraz káže, aby
sa zvyšní žiaci rozdelili do šesť. Dokážte, že opäť jeden žiak zvýši.
[\pdfklink{71-C-I-1}{https://skmo.sk/dokument.php?id=3925}]

Tabuľka $10\times 10$ je vyplnená číslami $1$ a~${-1}$ tak, že súčet čísel
v~každom riadku až na jeden je rovný~0 a súčet čísel v~každom stĺpci až
na jeden je rovný rovnakému číslu~$s$. Určte najväčšiu možnú
hodnotu $s$ a dokážte, že väčšia byť nemôže. Uveďte tiež príklad tabuľky
s~určenou najväčšou hodnotou~$s$. [\pdfklink{71-C-II-4}{https://skmo.sk/dokument.php?id=4055\#page=4}]

Tabuľka $10\times 10$ je vyplnená číslami ${-4}$, $3$ a $10$ tak, že súčet čísel
v~každom riadku až na jeden je nanajvýš~0 a súčet čísel v~každom stĺpci až
na jeden je nanajvýš~0. Určte najväčší možný súčet čísel v~tabuľke.
[\pdfklink{71-B-II-4}{https://skmo.sk/dokument.php?id=4057\#page=5}]

V~tabuľke $n\times n$, pričom $n\ge2$, sú po riadkoch napísané všetky čísla $1,2,\dots,n^2$ v~tomto poradí (v~prvom riadku sú za sebou napísané čísla $1,2,\dots,n$, v~druhom riadku ${n+1}, {n+2},\dots, 2n$, atď.). V~jednom kroku môžeme zvoliť ľubovoľné dve čísla na susedných políčkach (\tj. na takých, ktoré majú spoločnú stranu), a~ak je ich aritmetický priemer celé číslo, obe nahradíme týmto priemerom. Pre ktoré $n$ možno po konečnom počte krokov dostať tabuľku, v~ktorej sú všetky čísla rovnaké?
[\pdfklink{57-A-II-2}{https://skmo.sk/dokument.php?id=214\#page=2}]

Pre ktoré prirodzené čísla~$n$ možno do tabuľky $n \times n$ vpísať
všetky celé čísla od $1$ po $n^2$
tak, aby aritmetický priemer čísel v~každom riadku
aj stĺpci tabuľky bol celým číslom?
[\pdfklink{68-A-III-6}{https://skmo.sk/dokument.php?id=3121\#page=9}]

\endnávod
}

{%%%%%   C-I-1
Áno, takých desať čísel existuje. Príkladom je týchto
desať po sebe
idúcich čísel, pod ktoré rovno napíšeme ich požadované
delitele:
$$
\underset9\to{153}, \underset7\to{154}, \underset5\to{155},
\underset3\to{156}, \underset1\to{157}, \underset1\to{158},
\underset3\to{159}, \underset5\to{160},
\underset7\to{161}, \underset9\to{162}.
$$
Naozaj, deliteľnosť číslom 1 platí triviálne, číslami 3 a 9 podľa
ciferných súčtov, číslom 5 podľa posledných cifier a deliteľnosť
číslom 7 vyplýva z~rovností $154=7\cdot22$ a $161=154+7$.

\poznamkac1.
Uvedené riešenie je úplné, napriek tomu uvedieme úvahy, ktoré k nemu viedli.

Označme najmenšie z~hľadaných desiatich čísel písmenom $n$. Rovnako
ako v~riešení zapíšme prehľadne, čím má byť ktoré číslo
deliteľné:
$$
\underset9\to{n}, \underset7\to{n+1}, \underset5\to{n+2},
\underset3\to{n+3}, \underset1\to{n+4}, \underset1\to{n+5},
\underset3\to{n+6}, \underset5\to{n+7},
\underset7\to{n+8}, \underset9\to{n+9}.
$$

\smallskip
\item{$\bullet$} Ak zvolíme $n$ deliteľné $9$, bude aj číslo $n+9$ deliteľné $9$. Zároveň budú čísla $n+3$ aj $n+6$ deliteľné $3$.
\item{$\bullet$} Ak bude číslo $n+1$ deliteľné $7$, bude aj číslo $n+8$ deliteľné~$7$.
\item{$\bullet$} Ak bude číslo $n+2$ deliteľné $5$, bude aj číslo $n+7$ deliteľné~$5$.

\smallskip\noindent
Keďže je deliteľnosť číslom $1$ splnená vždy,
stačí nájsť číslo $n$ spĺňajúce tri podmienky:

\smallskip
\item{(i)} $n$ je deliteľné $9$,
\item{(ii)} $n+1$ je deliteľné $7$,
\item{(iii)} $n+2$ je deliteľné $5$.

\smallskip\noindent
Zameriame sa najskôr na podmienky (i) a (ii). Zo všetkých čísel $9$, $18$,
$27$, $36,\ldots$ deliteľných~$9$ vyberieme najmenšie, ktoré
spĺňa (ii). Tým je $27$, pretože $28$ je deliteľné~$7$.

Zamyslime sa, ako vyzerajú všetky ďalšie čísla $n$ spĺňajúce (i) a (ii).
Zapíšeme ich ako $27+k$ s~celým číslom $k>0$. Aby sme splnili (i),
zrejme musí byť $k$ násobok~$9$ (a~to je zároveň postačujúce). Podmienka (ii)
znamená, že číslo $n+1=28+k$ má byť deliteľné~$7$, takže $k$ musí byť
aj násobok $7$. Vďaka nesúdeliteľnosti čísel $7$ a $9$ to znamená,
že $k$ je násobok čísla $9\cdot7=63$. Takže všetky $n$
spĺňajúce súčasne (i) a (ii) sú tvaru $n=27+63l$, kde $l$ je
nezáporné celé číslo. Sú to čísla $27$, $90$, $153$, $216,\dots$

Zostáva vyhovieť podmienke (iii). Z~čísel
$27+2$, $90+2$, $153+2$, $216+2$, $\ldots$ máme vybrať také,
ktoré je deliteľné~$5$. Prvým z nich je zrejme číslo
$155$, takže najmenším~$n$ spĺňajúcim trojicu podmienok (i), (ii),
(iii) je číslo $n=153$ (ktoré sme uviedli v~riešení).
Keby sme zopakovali úvahu z~predchádzajúceho odseku,
zistili by sme, že (vzhľadom na nesúdeliteľnosť čísel $63$, $5$
a rovnosť $63\cdot5=315$) všetky vyhovujúce čísla $n$ sú tvaru
$153+315l$. Patrí medzi ne napríklad číslo so zaujímavým
dekadickým zápisom $888\,888\,888$.

\poznamkac2.
Zadaním úlohy bolo iba rozhodnúť o~existencii vyhovujúcej
skupiny desiatich po sebe idúcich prirodzených čísel, nebolo preto
nutné takú skupinu čísel hľadať. Ukážeme, že jej existencia
je dôsledkom \emph{čínskej zvyškovej vety}.\fnote{Nájdete ju aj
s~dôkazom v študijnom texte spomínanom v~závere k návodným úlohám.}
za tým účelom sa na podmienky (i), (ii) a (iii), ktoré sme odvodili v~prvej poznámke, pozrieme tak, že to isté číslo $n$ má dávať
požadované zvyšky po deleniach niekoľkými rôznymi číslami. Konkrétne

\smallskip
\item{(i)} $n$ dáva zvyšok $0$ po delení číslom $9$,
\item{(ii)} $n$ dáva zvyšok $6$ po delení číslom $7$,
\item{(iii)} $n$ dáva zvyšok $3$ po delení číslom $5$.

\smallskip\noindent
Keďže čísla 9, 7 a 5 sú po dvoch nesúdeliteľné, podľa čínskej
zvyškovej vety naozaj existuje číslo $n$, ktoré
tri uvedené podmienky spĺňa, ako sme mali ukázať.

\návody

Peter napísal na tabuľu 7 po sebe idúcich prirodzených čísel.
Pavol ich nevidel, ale tvrdí, že jedno z nich je deliteľné siedmimi.
Má pravdu? Prečo?
[Pavol má pravdu. Označme $n$ najmenšie zo siedmich čísel a $z$
jeho zvyšok po delení siedmimi. Ak neplatí $z=0$, je
$1\le7-z\le6$,
takže číslo $n+(7-z)$ je napísané na tabuli a je deliteľné siedmimi,
lebo je súčtom čísel $n-z$ a 7, ktoré sú obe deliteľné siedmimi.]

Myslím si prirodzené číslo. Ak ho zmenším o~$1$, dostanem
číslo deliteľné $3$. Ak myslené číslo zmenším o~$2$, dostanem
číslo deliteľné $4$.
a) Aké najmenšie číslo si môžem myslieť?
b) Nájdite všetky čísla, ktoré si môžem myslieť.
[a) Číslo $10$. Z~čísel $4$, $7$, $10$, $13,\ldots$
vyberieme najmenšie, ktoré spĺňa aj druhú podmienku. b)~Číslo
$10+12k$ pre ľubovoľné nezáporné celé $k$. Každé číslo sa v porovnaní s~číslom 10
líši o~násobok $3$ (z~prvej podmienky) a zároveň o~násobok $4$
(z~druhej podmienky), teda o~násobok $12$.]

Blchy Adam a Baša skáču po očíslovaných schodoch stále hore.
Adam začína na 1.~schode a skáče o~$a$ schodov. Baša
začína na 3. schode a skáče o~$b$~schodov. Schody, na ktoré obe
blchy doskočia, nazveme \uv{dvakrát navštívené}. Určte najmenší
kladný rozdiel poradových čísel dvakrát navštívených schodov,
a~to v~prípadoch a)~$a=4$ a~$b=5$,
b)~$a=4$ a~$b=6$, c)~$a=6$ a~$b=9$.
[a) 20, b) 12, c)~také schody neexistujú.
Hľadaný rozdiel je vo všeobecnosti rovný
najmenšiemu spoločnému násobku čísel $a$ a~$b$, v~prípade, že dvakrát navštívené schody vôbec existujú.
To je v~prípade c) vylúčené, pretože celé čísla $k$, $l$
pre rovnicu $1+ka=3+lb$ vtedy neexistujú -- z~čísel
$1+6k$ a~$3+9l$ je vždy iba to druhé deliteľné tromi.]

\D
Rozmyslite si, prečo z~$n$ po sebe idúcich prirodzených
čísel je vždy práve jedno deliteľné číslom $n$.
[Zvyšky $n$ po sebe idúcich prirodzených čísel tvoria -- podľa
zvyšku $k$ najmenšieho z~týchto čísel -- v prípade $k=0$
$n$-ticu $(0,1,2,\ldots,n-1)$, v~prípade $k=1,2,\dots,n-1$
$n$-ticu $(k,k+1,\dots,n-1,0,1,\dots,k-1)$.]

V školskej záhrade hrá skupina žiakov hru zvanú molekuly. Učiteľ im najprv prikázal, aby sa rozdelili do trojíc.
Jeden žiak zvýšil, a tak z ďalšej hry vypadol. Zvyšní žiaci sa potom mali rozdeliť do štvoríc. Opäť jeden žiak zvýšil
a vypadol. Potom sa zvyšní žiaci mali rozdeliť do piatich, zase jeden žiak zvýšil a vypadol. Učiteľ teraz káže, aby
sa zvyšní žiaci rozdelili do šesť. Dokážte, že opäť jeden žiak zvýši.
[Jedná sa o~úlohu \pdfklink{71-C-I-1}{https://skmo.sk/dokument.php?id=3925}. Keď jej znenie prevedieme
do jazyka čísel, z~riešenia úlohy N2 nám vyplynie, že počet žiakov je tvaru
$10 + 12k$. Vďaka rovnosti $10+12k=6(1+2k)+4$ má zvyšok
takého čísla po delení~$6$ potrebnú hodnotu 4.]

Štyri dni po sebe som zdolával rovnaké schodisko majúce menej ako 100
schodov. Bral som ho prvý deň po 2 schodoch, druhý deň po 3,
tretí deň po 4 a štvrtý deň po 5~schodoch, na posledný krok mi ostali
postupne 1, 2, 3 a 4 schody. Koľko schodov celé schodisko malo?
[59. Keby malo schodisko o~1 schod viac, bol by ich počet
deliteľný každým z~čísel 2, 3, 4,~5.]

Myslím si prirodzené číslo, ktoré je väčšie ako 2000, menšie
ako 3000 a je deliteľné~$17$.
Ak myslené číslo zväčším o~$1$,
dostanem číslo deliteľné $11$. Ak svoje číslo naopak zmenším o~$1$,
dostanem číslo deliteľné $6$. Aké číslo si myslím?
[Hľadajme najskôr najmenšie prirodzené číslo $n$
s~vlastnosťami $17\mid n$, $11\mid n+1$ a $6\mid n-1$.
Prácnejšiemu postupu sa vyhneme, keď
$6\mid n-1$ zapíšeme ako $6\mid n+17$ a~prejdeme tak len k dvom
podmienkam $(17\cdot6)\mid n+17$ a $11\mid n+1$. Keďže
$17\cdot6=102$, z~prvej podmienky máme $n=102k-17$, teda hľadáme
najmenšie prirodzené $k$, pre ktoré $11\mid 102k-16$.
To sa dá zjednodušiť na $11\mid 3k+6$
a ďalej ešte na $11\mid k+2$, odkiaľ $k=9$ a $n=102\cdot9-17=901$.
Keďže $17\cdot6\cdot11=1122$, všetky vyhovujúce~$n$ sú tvaru
$901+1122l$. Preto je myslené číslo $901+1122=2023$.]

{\everypar{}
\smallskip
\emph{Doplňujúca literatúra}:

Svoje poznatky a schopnosti k~téme úlohy si môžete zopakovať a
doplniť podľa brožúrky Františka Veselého \pdfklink{\emph{O~dělitelnosti čísel celých}}{https://www.dml.cz/handle/10338.dmlcz/403560}.
Záujemcom o~hlbšie poučenie odporúčame študijný text \pdfklink{\emph{Seriál -- Teorie čísel}}{https://prasa.cz/archive/28/9.pdf}.

\smallskip
}

\endnávod
}

{%%%%%   C-I-2
Podľa Tálesovej vety je bod $M$ stredom kružnice
opísanej trojuholníku $ABC$, teda platí $|MA|=|MB|=|MC|$. Zo
zadania tiež platí $|BC|=|CM|$, takže všetky úsečky
$MA$, $MB$, $MC$, $BC$ sú zhodné a trojuholník $BCM$ je rovnostranný (\obr).
\inspsc{c73i.21}{.8333}%

Z~kópií rovnostranného trojuholníka $BCM$ vytvorme pravidelnú trojuholníkovú
sieť podľa \obr{}. V~nej všetky tri vrcholy $A$, $B$, $C$
pôvodného trojuholníka $ABC$ budú uzlovými bodmi.
\inspsc{c73i.22}{.8333}%

Pre ďalší vyznačený uzlový bod $N$ potom platí $|NA|=|NB|=|NM|$, takže
dĺžka jednej úsečky siete je polomerom ako kružnice opísanej
trojuholníku $ABM$, tak aj polomerom kružnice opísanej trojuholníku
$ABC$.


\poznamka
Aj keď je konštrukcia trojuholníkovej siete, ktorú sme
v~riešení využili, celkom jasná, ukážme, že sme sa mohli
bez zmienky o~nej zaobísť.

Po nájdení štyroch zhodných úsečiek z~\obrr2{} zostrojíme
v~polrovine~$ACB$ rovnostranný trojuholník $AMN$, ktorý je zhodný
s~trojuholníkom $BCM$. Tretí trojuholník $BNM$ je potom rovnoramenný
so základňou $BN$ a pri hlavnom vrchole $M$ má uhol
veľkosti $180^{\circ}-2\cdot60^{\circ}=60^{\circ}$.
Je to teda tiež rovnostranný trojuholník, navyše zhodný
s trojuholníkmi $BCM$ a $AMN$. Z~rovností $|NA|=|NM|=|NB|=|AM|$ už vyplýva,
čo sme mali dokázať.

Mohli sme tiež začať tak, že k~rovnostrannému trojuholníku $BCM$
prikreslíme (zvonku) rovnostranný trojuholník $BNM$.
Dostaneme tak kosoštvorec $BCMN$, ktorého strana $MN$ je rovnako
ako protiľahlá strana~$BC$ kolmá na úsečku $AB$.
Z~$MN\perp AB$ a $|MA|=|MB|$ vyplýva, že priamka $MN$ je
osou úsečky~$AB$. Preto platí aj $|NA|=|NB|$,
čo spolu s~$|NB|=|NM|$ znamená, že $N$ je stred
kružnice opísanej trojuholníku $ABM$. Jej polomer $NB$
má pritom rovnakú dĺžku ako polomer $MC$ kružnice opísanej trojuholníku $ABC$.

\ineriesenie
Podľa Tálesovej vety je bod $M$ stredom kružnice
opísanej trojuholníku $ABC$ a jej polomer $r$ je spoločnou dĺžkou
úsečiek $MA$, $MB$, $MC$. Označme~$K$ stred odvesny $AB$ a
strednú priečku $MK$ o dĺžke $\frac12|BC|$ predĺžme za bod $K$
do úsečky~$MN$ dvojnásobnej dĺžky. Vznikne nám tak rovnobežník
$AMBN$, ktorého uhlopriečka $MN$ je zhodná s~odvesnou $BC$ (\obr). Vďaka rovnosti
$|MA|=|MB|=r$ sa jedná o~kosoštvorec so (zhodnými) stranami dĺžky $r$.
\inspsc{c73i.23}{.8333}%

Až teraz využijeme podmienku zo zadania, podľa ktorej platí $|BC|=r$.
Dĺžku~$r$ majú teda nielen strany kosoštvorca $AMBN$, ale
aj jeho uhlopriečka~$MN$. Z~rovností $r=|NA|=|NB|=|NM|$ už vyplýva,
že $N$ je stred kružnice s~polomerom~$r$, ktorá je opísaná
trojuholníku $ABM$.

\ineriesenie
Na úvod rovnako ako v~prvom riešení zistíme, že trojuholník $BCM$ je
rovnostranný. Stred kružnice opísanej trojuholníku $ABM$, ktorý
označíme $O$, určite leží na osi jeho strany $AB$ (\obr).
Táto os je vďaka rovnosti $|AM|=|BM|$ súčasne osou uhla $AMB$,
ktorý má veľkosť $180^{\circ}-|\uhol CMB|=180^{\circ}-60^{\circ}=120^{\circ}$.
Preto má uhol $AMO$ polovičnú veľkosť $60^{\circ}$.
Je to však uhol pri základni $AM$ rovnoramenného trojuholníka $AMO$
(lebo $|OA|=|OM|$ podľa zavedenia bodu $O$),
ktorý je teda rovnostranný. Platí teda $|MA|=|OM|$ a požadovaný
dôkaz rovnosti polomerov dvoch kružníc je hotový.
\inspsc{c73i.24}{.8333}%

\návody

Nájdite všetky rovnoramenné trojuholníky, ktoré majú aspoň jeden
vnútorný uhol veľkosti a) $30^{\circ}$, b) $60^{\circ}$.
[a) Sú to trojuholníky s trojicami vnútorných uhlov
$(30^{\circ},30^{\circ},120^{\circ})$ a~$(30^{\circ},75^{\circ},75^{\circ})$ . b) Vyhovujú len
rovnostranné trojuholníky.]

Bez použitia Tálesovej vety dokážte, že stred kružnice opísanej
pravouhlému trojuholníku splýva so stredom jeho prepony. Využite
na to vhodne dve stredné priečky.
[Stred kružnice opísanej všeobecnému trojuholníku leží v priesečníku osí
jeho troch strán, na jeho určenie stačí využiť dve z týchto osí.
Stredné priečky pravouhlého trojuholníka, ktoré sú rovnobežné s~jeho
odvesnami, ležia na osiach týchto dvoch strán. Preto tieto dve osi
prechádzajú spoločným bodom spomínaných dvoch priečok, teda stredom
prepony.]


Uvažujme šesť bodov: vrcholy daného rovnostranného trojuholníka
a stredy jeho strán. Zistite, koľko pravouhlých trojuholníkov má
ako vrcholy tri zo šiestich uvažovaných bodov.
[Šesť. Jednou z~odvesien každého trojuholníka musí byť
niektorá výška pôvodného rovnostranného trojuholníka.]

\D
V danom pravouhlom trojuholníku $ABC$ označme $K$ stred prepony $AB$ a $L$ stred kratšej odvesny $AC$. Kružnica s priemerom $BC$ pretína úsečku $KL$ v bode $P$. Dokážte, že uhly $PAC$ a $PBC$ sú zhodné. [\pdfklink{72-C-S-2}{https://skmo.sk/dokument.php?id=4367\#page=2}]

Daný je trojuholník $ABC$, v~ktorom $D$, $E$ sú postupne stredy
strán $BC$,~$AB$. Nech $F$ je stred úsečky $BE$
a $G$ vnútorný bod strany $AC$, pre ktorý platí $|AG|=3\,|CG|$.
Dokážte, že priesečník priamok $DF$ a $GE$ leží na tej rovnobežke
s~priamkou~$BC$, ktorá prechádza bodom $A$. [\pdfklink{70-C-II-3}{https://skmo.sk/dokument.php?id=3605\#page=3}]

Šesťuholník, ktorého všetky vnútorné uhly majú rovnakú
veľkosť, má štyri po sebe idúce strany s~dĺžkami $1$, $7$, $4$
a $2$. Zistite dĺžku zvyšných dvoch strán.
[5 a~6. Keďže vnútorné uhly daného šesťuholníka majú
veľkosť $(6-2)\cdot180^{\circ}/6=120^{\circ}$, možno jeho vrcholy
umiestniť do uzlov rovinnej siete tvorenej rovnostrannými trojuholníkmi so stranou
dĺžky~$1$.]

\endnávod
}

{%%%%%   C-I-3
a) Áno, je to možné. V prípade, keď štyria súrodenci sú dvaja bratia
a dve sestry, môžu pravdivo vysloviť štyri navzájom rôzne
výroky. Jeden brat povie: \uv{Mám práve jedného brata} a druhý:
\uv{Mám práve dve sestry,} jedna sestra povie: \uv{Mám práve jednu sestru} a
druhá: \uv{Mám práve dvoch bratov.}

\smallskip
b) Pre $n=4$ sme zistili, že každý zo štyroch súrodencov
môže vysloviť iný pravdivý výrok.

V~prípade $n\ge 5$ sú aspoň traja súrodenci rovnakého pohlavia.
Predpokladajme, že každý z nich povie pravdu o počte
svojich sestier alebo bratov. Potom však aspoň dva
z týchto vyslovených výrokov budú rovnaké (tie o~počte sestier,
alebo tie o~počte bratov). Žiadne celé $n\geqq5$ teda
nemá požadovanú vlastnosť.

\zaver
Najväčšie vyhovujúce $n$ je rovné 4.

\poznamkac1.
V~časti b) podaného riešenia sme
využili poznatok, ktorý sa nazýva \emph{Dirichletov princíp}
alebo tiež \emph{priehradkový princíp}.\fnote{Pozri poznámku pod čiarou
k~úlohe N2.}
Tvrdí napríklad to, že keď rozmiestnime $13$ predmetov
do $2$ priehradiek, bude v~niektorej priehradke aspoň $7$ predmetov.
Alebo keď rozmiestnime $13$ predmetov do $3$ priehradiek, bude
v~niektorej priehradke aspoň $5$ predmetov. Alebo pri rozmiestnení
$2n+1$ predmetov do $2$ priehradiek bude v~niektorej priehradke aspoň
$n+1$ predmetov. Všeobecne vyjadrené: pri rozmiestnení aspoň ${kn+1}$
predmetov do $k$ priehradiek bude v~niektorej priehradke aspoň $n+1$
predmetov (zapísali sme to vyššie ako pre $k=2$ a $n=6$, tak pre
$k=3$ a $n=4$, aj pre $k=2$ a všeobecné $n$).
Pritom \uv{predmety} môžu byť čísla,
geometrické útvary, ľudia, výroky, v~podstate čokoľvek.
Priehradky potom môžu vyjadrovať ľubovoľné vlastnosti jednotlivých
predmetov. Napríklad do $2$ priehradiek často
rozdeľujeme celé čísla podľa toho, či sú párne alebo nepárne
(pozri návodnú úlohu N2).

V našom riešení sme použili Dirichletov princíp dvakrát,
vždy pre $2$ priehradky. V prvom prípade súrodenci hrali úlohu
\uv{predmetov} a pohlavie (mužské/ženské) hralo úlohu
\uv{priehradiek}. V druhom prípade boli \uv{predmety} výroky a
\uv{priehradkami} boli výroky o~sestrách a
výroky o~bratoch.

\poznamkac2.
Ukážme, že výklad časti b) riešenia možno podať aj inak.
Ak je v~danej skupine $n$ súrodencov práve $B$ bratov a práve $S$ sestier,
môže každý z nich o sebe pravdivo vysloviť iba jeden zo štyroch výrokov
(prvé dva sú vyhlásenia bratov, zvyšné dva vyhlásenia sestier):

\smallskip
\item{1.} Mám práve $S$ sestier.
\item{2.} Mám práve $B-1$ bratov.
\item{3.} Mám práve $S-1$ sestier.
\item{4.} Mám práve $B$ bratov.

\smallskip\noindent
Ak preto majú byť vyslovené výroky v~počte $n=B+S$ navzájom rôzne,
musí platiť nerovnosť $n\leqq4$.

\návody

V jednej rodine žije a) 7 bratov bez sestry,
b) 7 bratov a 1 sestra. Každý z nich vysloví jedno pravdivé
vyhlásenie z ponuky v súťažnej úlohe. Určte maximálny
počet rôznych výrokov, ktoré môžu zaznieť.
[a) 1, b) 3. V~prípade b) prichádzajú do úvahy iba 3~výroky:
o~práve 1 sestre, o~práve 7~bratoch a~o~práve 6~bratoch.]

Majme $n$ navzájom rôznych prirodzených čísel. Každé
z nich zafarbíme buď namodro, alebo načerveno. Zistite, pre
aké najmenšie $n$ už zaručene nájdeme dve čísla rovnakej farby,
ktorých rozdiel je párny.
\fnote{Ako na riešenie tejto, tak aj
mnohých podobných úloh je možné využiť \emph{Dirichletov princíp},
nazývaný tiež \emph{priehradkový princíp}. Viac sa môžete
dozvedieť v~brožúrke Leva Bukovského \pdfklink{\emph{Dirichletov princíp}}{https://www.dml.cz/handle/10338.dmlcz/403696}.}
[$n=5$. Rozdiel dvoch čísel je párny práve vtedy, keď čísla majú rovnakú
paritu (párne/nepárne). V~prípade $n\geqq5$ nájdeme podľa
Dirichletovho princípu aspoň 3 zafarbené čísla rovnakej parity.
Niektoré dve z~nich budú mať rovnakú farbu.
Pre $n=4$ (a~tým vlastne aj pre $n<4$)
uvedieme protipríklad: za predpokladu, že máme zadané čísla od 1 do 4, zafarbíme namodro čísla 1, 2 a~načerveno čísla 3,~4.]

\D
Do vreca je nahádzaných $5$ párov čiernych, $6$ párov modrých
a $7$ párov sivých papúč, pritom páry papúč rovnakej farby
sú nerozlíšiteľné. Koľko papúč musíme vytiahnuť, aby sme
určite medzi nimi mali a) dve papuče rovnakej farby,
b) dve papuče rovnakej farby, ktoré tvoria pár?
[a) 4, pretože papuče sú troch farieb. b) 19. V~vreci je celkom
$5+6+7=18$ párov papúč. Ak vytiahneme 18 papúč a
ak ešte nebude medzi nimi pár tej istej farby, bude to znamenať, že sme
z každej farby vytiahli buď iba ľavé, alebo iba pravé papuče,
a teda buď všetky ľavé, alebo všetky pravé papuče z každej farby. Preto
po vytiahnutí 19. papuče sa zaručene jeden pár tej istej farby skompletizuje.]

Na Ostrove Logiky patrí každý jeho obyvateľ buď medzi
poctivcov, ktorí hovoria vždy pravdu, alebo medzi klamárov, ktorí
vždy klamú. Pri stole sa stretnú traja obyvatelia tohto ostrova --
Oliver, Pavol a Romana. Oliver povie:
\uv{Medzi nami nie je ani jeden poctivec.} Pavol dodá:
\uv{Medzi nami je práve jeden poctivec.}
Je Romana poctivec, alebo klamár?
[Klamár. Zvážte najskôr, do ktorej skupiny patrí Oliver, potom do ktorej
Pavol.]

Na Ostrov Logiky z~úlohy D2 zavíta turista, ktorý si najme
miestneho sprievodcu. Pri ceste uvidia v~diaľke ďalšieho domorodca.
Turista za ním vyšle svojho sprievodcu, aby sa ho spýtal, či je
poctivcom, alebo klamárom. \uv{Tvrdí, že je poctivec,} prinesie
správu sprievodca. Je sprievodca poctivec, alebo klamár?
[Poctivec. Zvážte, že každý obyvateľ ostrova o~sebe tvrdí, že je
poctivec.]

{\everypar{}

\smallskip
Výroky poctivcov a klamárov preslávila dodnes
populárna kniha s~názvom \emph{What is the Name of this Book?},
ktorú napísal Raymond Smullyan. Vyšla aj v českom preklade s názvom \emph{Jak se jmenuje tahle knížka?} Nájdete v~nej množstvo logických
hračiek, paradoxov a hádaniek, nielen z~ostrova poctivcov a klamárov.
Rovnakej téme sa venuje aj text
\pdfklink{\emph{Poctivci, lháři a matematici}}{https://olympiada.karlin.mff.cuni.cz/prednasky/olsak.pdf} od Mirka Olšáka.
}

\endnávod
}

{%%%%%   C-I-4
Najskôr ekvivalentne upravíme prvú rovnicu zo zadania:
$$\eqalign{
(a+b)(c+d) &=(b+c)(a+d),\cr
ac+ad+bc+bd&= ba+bd+ca+cd,\cr
ad+bc-ab-cd&=0,\cr
a(d-b)-c(d-b)&=0,\cr
(a-c)(d-b)&=0.}
$$
Podobne zistíme, že druhá rovnica je ekvivalentná s~rovnicou,
ktorú kvôli prehľadnosti zapíšeme s~upravenou prvou rovnicou vedľa seba:
$$
(a-c) (d-b) = 0, \quad (a-b) (d-c) = 0.
$$
Podľa toho, ktoré zo štyroch činiteľov sa v~odvodených rovniciach
rovnajú nule, sú všetky riešenia $(a,b,c,d)$ zadaných rovníc
štvorice jedného zo štyroch typov:

\smallskip
\item{$\triangleright$} $a-c=0$ a $a-b=0$, čiže $a=b=c$ a $d$ je ľubovoľné,
\item{$\triangleright$} $a-c=0$ a $d-c=0$, čiže $a=c=d$ a $b$ je ľubovoľné,
\item{$\triangleright$} $d-b=0$ a $a-b=0$, čiže $a=b=d$ a $c$ je ľubovoľné,
\item{$\triangleright$} $d-b=0$ a $d-c=0$, čiže $b=c=d$ a $a$ je ľubovoľné.

\smallskip\noindent
Došli sme k~záveru, že rovnice zo zadania platia práve vtedy, keď
aspoň tri z~čísel $a$, $b$, $c$, $d$ sú si rovné. Rozlíšime
teraz, či rovnaké čísla vo štvorici $(a,b,c,d)$ sú štyri, alebo
tri.

\smallskip
\item{$\bullet$} Ak sú rovnaké všetky štyri čísla, podmienke $a+b+c+d=100$ vyhovuje jediná štvorica $a=b=c=d=25$.
\item{$\bullet$} Ak majú tri čísla rovnakú hodnotu $x$ a štvrté má
inú hodnotu $y$, podmienka $a+b+c+d=100$ prejde na rovnicu
$3x+y=100$, ktorú uvažujeme v~obore prirodzených čísel
spĺňajúcich podmienku $x\ne y$. Z~ekvivalentnej rovnice
$y=100-3x$ vyplýva, že číslo $y$ je kladné
len pre $x=1,2,\ldots,33$, pritom podmienka $y \ne x$ je splnená,
ak $x \ne 25$. Máme teda najskôr $33-1=32$ možností pre výber
čísla $x$ a potom 4~možnosti, ktorému zo štyroch čísel $a$, $b$,
$c$, $d$ priradíme hodnotu $y=100-3x$ (a~trom ostatným hodnotu $x$). Počet vyhovujúcich štvoríc $(a,b,c,d)$
s~práve troma rovnakými číslami je teda rovný $32\cdot4=128$.

\smallskip\noindent
Celkový počet vyhovujúcich štvoríc je teda $1+128$, čiže 129.

\poznamkac1
Nad rámec zadania sme zistili, že $128$
vyhovujúcich štvoríc je tvaru $(x,x,x,100-3x)$, $(x,x,100-3x,x)$,
$(x,100-3x,x,x)$, resp. $(100-3x,x,x,x)$, pričom
$x=1,2,\ldots,33$ okrem $x=25$, a zvyšná 129.
vyhovujúca štvorica je $(25,25,25,25)$.

\poznamkac2.
Prácu s dvojicou rovníc
$$
(a+b)(c+d)=(b+c)(a+d) \quad\hbox{a}\quad (b+c)(a+d)=(a+c)(b+d)
$$
so zhodnými stranami $(b+c)(a+d)$ si možno uľahčiť pozorovaním,
že druhú rovnicu dostaneme z~prvej, keď v~nej písmená $b$, $c$
navzájom vymeníme.\fnote{Spomínaný súčin $(b+c)(a+d)$ sa
nezmení ani pri výmene $b \leftrightarrow c$, ani pri výmene
$a \leftrightarrow d$. Nášmu účelu môže poslúžiť ktorákoľvek
z~oboch výmen.}
Akonáhle teda prvú rovnicu upravíme do tvaru
$(a-c)(d-b)=0$ a vykonáme v~nej výmenu $b\leftrightarrow c$,
dostaneme bez výpočtov ekvivalentný tvar $(a-b)(d-c)=0$ druhej
rovnice.

\návody

Pre prirodzené čísla $a$, $b$, $c$, $d$ platí
$ab+bc+cd+da=77$. Určte všetky možné hodnoty ich súčtu.
[Jediná hodnota $18$. Platí $77=ab+bc+cd+da={{(a+c)}{(b+d)}}$ a pritom
$77=7\cdot11$. Oba činitele $a+c$ a $b+d$
sú väčšie ako $1$, takže sú v~nejakom poradí rovné
prvočíslam $7$ a $11$. Tak či onak platí
$(a+c)+(b+d)=18$. Zostáva uviesť nejakú vyhovujúcu štvoricu.
Je ňou napríklad $(a,b,c,d)=(1,1,6,10)$.]

Predpokladajme, že navzájom rôzne reálne čísla $a$,
$b$, $c$, $d$ spĺňajú nerovnosti
$$
ab + cd > bc + ad > ac +bd.
$$
Ak $a$ je z týchto štyroch čísel najväčšie, ktoré z nich je
najmenšie?
[Číslo $c$. Prvú zadanú nerovnosť upravte na tvar
$(a-c)(b-d)>0$, druhú na tvar $(b-a)(c-d)>0$. Ak je
$a$ z~daných štyroch čísel najväčšie, platí $a-c>0$ a~$b-a<0$,
takže z~odvodených nerovností vyplýva $b-d>0$ a $c-d<0$, čiže
$b>d$ a $c<d$, celkovo $a>b>d>c$.]

\D
Nájdite všetky štvorice $a>b>c>d$ celých čísel so súčtom 71,
ktoré spĺňajú rovnicu
$$
(a-b)(c-d)+(a-d)(b-c)=26.
$$
[Ľavú stranu upravte na súčin dvoch mnohočlenov.
Kompletné riešenie: \pdfklink{71-C-S-3}{https://skmo.sk/dokument.php?id=3929\#page=2}.]

Určite všetky možné hodnoty súčtu $a+b+c+d$, pričom
$a$, $b$, $c$, $d$ sú kladné celé čísla spĺňajúce rovnosť
$$
(a^2-b^2)(c^2-d^2)+(b^2-d^2)(c^2-a^2)=2021.
$$
[Ľavú stranu upravte na súčin štyroch mnohočlenov.
Kompletné riešenie: \pdfklink{71-C-I-6}{https://skmo.sk/dokument.php?id=3925\#page=16}.]

Na každej stene kocky je napísané kladné celé číslo. Ku
každému vrcholu je pripísaný súčin troch čísel na priľahlých
stenách. Súčet ôsmich čísel pri vrcholoch je $1001$. Určte
všetky možné hodnoty súčtu čísel na stenách.
[31. Ak sú $a$, $b$ čísla na prednej a zadnej stene, $c$, $d$ čísla na hornej a dolnej stene a napokon $e$, $f$ čísla na ľavej a
pravej stene, tak roznásobením súčinu $(a+b)(c+d)(e+f)$
dostaneme osem sčítancov, ktorými sú práve čísla pripísané
vrcholom kocky (v~každom vrchole sa stretajú tri steny, po
jednej z popísaných troch dvojíc stien). Podľa zadania je tak
súčin troch čísel $a+b$, $c+d$ a~$e+f$ väčších ako 1 rovný číslu
1001, čo je súčin troch prvočísel 7, 11 a 13. Platí teda
$\{a+b,c+d,e+f\}=\{7,11,13\}$, takže
$a+b+c+d+e+f=7+11+13=31$.]

Určte počet všetkých trojíc kladných celých čísel $a$, $b$, $c$, pre ktoré platí
$a+ab+abc+ac+c=2017$.
[K~obom stranám rovnice pričítajte číslo 1, aby ste potom ľavú
stranu mohli rozložiť na súčin dvoch mnohočlenov.
Kompletné riešenie: \pdfklink{67-B-I-4}{https://skmo.sk/dokument.php?id=2578\#page=4}.]

\endnávod
}

{%%%%%   C-I-5
Dokážeme, že žiadne prirodzené číslo sa už neobjaví.

Po prvom kroku dostaneme čísla
$\sqrt{2}$, $\sqrt{3}$ a $\sqrt{6}$.
Po druhom kroku dostaneme čísla
$\sqrt{2\cdot3}={\localcolor\Cyan\sqrt{6}}$,
$\sqrt{2\cdot6}=2{\localcolor\Cyan\sqrt{3}}$ a
$\sqrt{3\cdot6}=3{\localcolor\Cyan\sqrt{2}}$.
Vidíme, že je to \uv{až na násobky a poradie} rovnaká
trojica čísel ako po prvom kroku (vyznačili sme ich modro).
Vysvetlíme, že trojicu takéhoto typu dostaneme aj po každom
ďalšom kroku.\fnote{Poznamenajme, že na
výsledok žiadneho kroku zrejme nemá vplyv, v~akom poradí sú
aktuálne čísla na tabuli zapísané.}

Predpokladajme teda, že po určitom počte krokov sú na tabuli
zapísané tri čísla tvaru $r\sqrt2$, $s\sqrt3$ a $t\sqrt6$,
pričom $r$, $s$, $t$ sú vhodné prirodzené čísla. Potom po
nasledujúcom kroku budú na tabuli čísla
$$\eqalign{
(r\sqrt2)\cdot(s\sqrt3)&=rs\sqrt6,\cr
(s\sqrt3)\cdot(t\sqrt6)&=3st\sqrt2,\cr
(t\sqrt6)\cdot(r\sqrt2)&=2rt\sqrt3,\cr
}$$
teda opäť čísla tvaru $r'\sqrt2$, $s'\sqrt3$ a $t'\sqrt6$, tentoraz
s~prirodzenými číslami $r'=3st$, $s'=2rt$ a $t'=rs$.

Dokázali sme, že po každom kroku, počnúc prvým, bude na tabuli
trojica čísel $r\sqrt2$, $s\sqrt3$ a $t\sqrt6$ s~vhodnými
prirodzenými číslami $r$, $s$, $t$ (meniacimi sa po každom kroku).
Keďže čísla $\sqrt2$, $\sqrt3$ a $\sqrt6$
sú iracionálne (pozri úlohu N3), žiadny z~ich násobkov
$r\sqrt2$, $s\sqrt3$ a $t\sqrt6$ nie je celé číslo.
Tým je tvrdenie z~úvodu riešenia dokázané.


\ineriesenie
K rovnakému záveru ako v~prvom riešení dospejeme, keď si najskôr
uvedomíme, že každé číslo, ktoré sa kedykoľvek na tabuli objaví, bude tvaru
$\sqrt{2^x\cdot3^y}$, pričom $x$ a $y$ sú celé nezáporné čísla.
Zrejme to platí na začiatku: $1=\sqrt{2^0\cdot3^0}$,
$\sqrt2=\sqrt{2^1\cdot3^0}$ a $\sqrt3=\sqrt{2^0\cdot3^1}$.
Potom už stačí pri každom kroku (trikrát) využiť poznatok, že
súčin dvoch čísel uvažovaného tvaru, povedzme $\sqrt{2^x\cdot3^y}$
a $\sqrt{2^u\cdot3^v}$, je opäť číslo toho istého tvaru:
$$
\sqrt{2^x\cdot3^y}\cdot\sqrt{2^u\cdot3^v}=
\sqrt{2^{x+u}\cdot3^{u+v}}.
\tag1
$$

Zaujíma nás otázka, kedy je číslo tvaru $\sqrt{2^x\cdot3^y}$ celé.
Je takmer zrejmé, že to nastane práve vtedy, keď oba exponenty $x$ a $y$
sú párne čísla.\fnote{Tento poznatok nie je nutné v inak
úplnom riešení zdôvodňovať. Napriek tomu ho vo všeobecnejšej podobe uvádzame
v návodnej úlohe N2 aj s dôkazom.}
Aj keď to využijeme až v poslednej vete riešenia,
sme už teraz motivovaní k tomu, aby sme skúmali parity
exponentov $x$ a~$y$ v~zápisoch $\sqrt{2^x\cdot3^y}$ všetkých čísel,
ktoré sa budú
na tabuli postupne objavovať. Výhodne na to použijeme
symboly $N$ a $P$ pre všetky nepárne, resp. všetky
párne čísla. Jasný význam potom majú rovnosti $N+N=P$, $N+P=N$,
$P+N=N$ a~$P+P=P$, kde v~každej z~nich môžu symboly
$N$, $P$ na rôznych miestach reprezentovať rôzne čísla (príslušné parity).

Po prvom kroku máme na tabuli čísla
$a=\sqrt{2}$, $b=\sqrt{3}$ a $c=\sqrt{6}$, pri ktorých
sú dvojice $(x,y)$ exponentov v~zápisoch $\sqrt{2^x\cdot3^y}$
rovné postupne $(1,0)$, $(0,1)$ a $(1,1)$.
Použitím symbolov $N$ a $P$ tak zapíšeme, že po prvom
kroku sú na tabuli tri čísla $a=\sqrt{2^N\cdot3^P}$,
$b=\sqrt{2^P\cdot3^N}$ a $c=\sqrt{2^N\cdot3^N}$.
Pre podobný zápis troch čísel $ab$, $bc$ a $ca$,
ktoré budú na tabuli po druhom kroku, už konkrétne hodnoty $a$,
$b$, $c$ nepotrebujeme. Podľa pravidla \thetag1 a pravidiel zo záveru
predchádzajúceho odseku totiž platí
$$\eqalign{
ab&=\sqrt{2^N\cdot3^P}\cdot\sqrt{2^P\cdot3^N}=
\sqrt{2^{N+P}\cdot3^{P+N}}=\sqrt{2^N\cdot3^N},\cr
bc&=\sqrt{2^P\cdot3^N}\cdot\sqrt{2^N\cdot3^N}=
\sqrt{2^{P+N}\cdot3^{N+N}}=\sqrt{2^N\cdot3^P},\cr
ca&=\sqrt{2^N\cdot3^N}\cdot\sqrt{2^N\cdot3^P}=
\sqrt{2^{N+N}\cdot3^{N+P}}=\sqrt{2^P\cdot3^N}.
}$$
Vidíme, že po druhom kroku -- rovnako ako po prvom kroku -- sú na
tabuli opäť tri čísla rôznych typov $\sqrt{2^N\cdot3^P}$,
$\sqrt{2^P\cdot3^N}$ a $\sqrt{2^N\cdot3^N}$. Z~trojice posledných výpočtov
je jasné, že rovnaké tri typy čísel budú na tabuli
aj po treťom kroku, po štvrtom kroku, atď., t.\,j. po
ľubovoľnom konečnom počte krokov. Po žiadnom kroku sa preto na tabuli
neobjaví číslo štvrtého typu $\sqrt{2^P\cdot3^P}$, a teda
ani žiadne prirodzené číslo.

\poznamka
Obe podané riešenia sú v~podstate založené na rovnakej myšlienke. Vysvetľujú
to rovnosti, ktoré dostaneme čiastočným odmocnením výrazov, s~ktorými sme pracovali v~druhom riešení:
$$\eqalign{
\sqrt{2^{N}\cdot3^{P}}&=\sqrt{2^{2u+1}\cdot3^{2v}}=
\bigl(2^u\cdot3^v\bigr)\sqrt{2},\cr
\sqrt{2^{P}\cdot3^{N}}&=\sqrt{2^{2w}\cdot3^{2x+1}}=
\bigl(2^w\cdot3^x\bigr)\sqrt{3},\cr
\sqrt{2^{N}\cdot3^{N}}&=\sqrt{2^{2y+1}\cdot3^{2z+1}}=
\bigl(2^y\cdot3^z\bigr)\sqrt{6}.
}$$
Po každom kroku tak dostaneme trojicu čísel tvaru
$r\sqrt2$, $s\sqrt3$ a $t\sqrt6$, ako sme vysvetlili v~prvom
riešení.

\návody

Uvažujme prirodzené čísla od $1$ do $10$ (vrátane).
Koľko najviac z~ich môžeme medzi sebou vynásobiť, aby
druhá odmocnina z ich súčinu bola rovná prirodzenému číslu?
[9. Prvočíslo $7$ v~súčine byť nemôže, malo by medzi jeho prvočiniteľmi
jediný výskyt. Súčin deviatich ostatných čísel je rovný
$2^{8}\cdot3^4\cdot5^2$, čo je $(2^{4}\cdot3^2\cdot5)^2$.]

Nech $n$ je prirodzené číslo. Ukážte, že číslo $\sqrt{n}$ je
celé práve vtedy, keď v~rozklade čísla~$n$ na
prvočinitele
má každé prvočíslo párny počet výskytov.
[Ak je kladné číslo $k=\sqrt{n}$ celé, z~rovnosti
$k^2=n$ vyplýva, že počet výskytov každého prvočísla
v~rozklade čísla $n$ je dvojnásobkom jeho výskytov
v~rozklade čísla $k$, a
teda číslo párne. Ak je naopak počet výskytov každého
prvočísla v~rozklade čísla~$n$ párny, potom to celé číslo, ktoré má vo svojom
rozklade polovičné počty týchto výskytov,
je zrejme rovné~$\sqrt{n}$.]

a) Dokážte, že rovnosť $a^2=2b^2$ neplatí pre žiadne
prirodzené čísla $a$, $b$.\hfil\break
b) Dokážte, že číslo $\sqrt2$ je iracionálne, t.\,j. že sa
nerovná žiadnemu zlomku~$a/b$ s~prirodzenými číslami $a$, $b$.
\hfil\break
c) Dokážte, že čísla $\sqrt3$ a $\sqrt6$ sú iracionálne.
\hfil\break
[a) V~rozklade čísel $a^2$ a $2b^2$ na prvočinitele má
prvočíslo 2 párny, resp. nepárny počet výskytov, takže $a^2\ne 2b^2$.
b) Z~rovnosti $\sqrt{2}=a/b$ by vyplynula rovnosť $a^2=2b^2$ z~časti~a).
c) Keby sa zlomok $a/b$ rovnal jednému z~čísel $\sqrt3$ alebo
$\sqrt6$, platilo by $a^2=3b^2$, resp. $a^2=6b^2$. Počet výskytov
prvočísla 3 v~rozklade čísla $a^2$ je párny, v~rozklade
oboch čísel $3b^2$ aj $6b^2$ je nepárny, teda $a^2\ne 3b^2$ aj
$a^2\ne 6b^2$.]

\D
Dokážte, že pre každé prirodzené číslo $n$ je číslo $\sqrt{n}$ buď
celé, alebo iracionálne.
[Stačí ukázať, že ak číslo $\sqrt{n}$ nie je iracionálne, t.\,j. je
rovné niektorému zlomku~$a/b$ s~prirodzenými číslami $a$ a $b$, tak
$\sqrt{n}$ je celé číslo. Naozaj, z~rovnosti $\sqrt{n}=a/b$ upravenej
na $a^2=nb^2$ vyplýva, že počet výskytov každého prvočísla
v~rozklade čísla~$n$ je párny, pretože je rozdielom dvoch
párnych počtov jeho výskytov v~rozkladoch $a^2$ a~$b^2$.
Podľa N2 to znamená, že $\sqrt{n}$ je celé číslo.]

Prirodzené číslo $n$ je také, že číslo $6n^2+5n+1$ je druhou
mocninou prirodzeného čísla. Dokážte, že aj obe čísla $2n+1$
a $3n+1$ sú druhými mocninami prirodzených čísel.
[Nech $2n+1=a$, $3n+1=b$ a $6n^2+5n+1=c^2$, pričom $a,b,c\in\Bbb N$.
Všimnime si, že čísla $a$, $b$ sú nesúdeliteľné (lebo je s~nimi
zrejme nesúdeliteľný ich rozdiel $b-a=n$) a~pritom platí
$ab=c^2$. Z~toho vyplýva: Každé prvočíslo deliace $a$ alebo $b$ delí iba jedno
z~nich a má v~rozklade tohto čísla rovnaký počet výskytov ako v~rozklade
čísla $c^2$, čo je párne číslo. Obe čísla $a$, $b$ tak sú
druhými mocninami. (Príklad existuje: $n=40$, vtedy
$2n+1=9^2$ a~$3n+1=11^2$.)]

Na tabuli je napísaných $n$ rôznych prirodzených čísel od $1$ do
$n$. V~jednom kroku zotrieme nejaké dve čísla a namiesto nich
napíšeme veľkosť (t.\,j. absolútnu hodnotu) ich rozdielu.
Pokračujeme tak dlho, kým na tabuli nezostane jediné
číslo. Pre ktoré čísla $n$ bude toto posledné číslo nepárne
nezávisle od toho, v~akom poradí budeme čísla zotierať?
[Pre tie prirodzené $n$, ktoré po delení štyrmi
dávajú zvyšok~$1$ alebo $2$, teda $n$ rovné 1, 2, 5, 6, 9, 10, atď.
Sledujme, ako sa zmení aktuálny počet $k$ nepárnych čísel
napísaných na tabuli po následnom kroku. Ak pri ňom zotrieme dve
nepárne čísla, napíšeme namiesto nich číslo párne, teda počet $k$ sa
zmenší o 2. Ak zotrieme jedno číslo párne a druhé nepárne,
napíšeme namiesto nich číslo nepárne, a tak sa počet $k$ nezmení.
K zmene $k$ nedôjde ani vo zvyšnom prípade, keď zotrieme dve párne
čísla, lebo namiesto nich napíšeme číslo párne. Celkovo vidíme,
že po žiadnom kroku nezmeníme paritu čísla $k$. Posledné číslo
teda vyjde nepárne práve vtedy, keď je na úvod na tabuli
nepárny počet nepárnych čísel (bez ohľadu na postup zotierania).
Počet nepárnych čísel od~1 do~$n$ je v~prípade $n=4k$ rovný $2k$,
v~prípadoch $n=4k+1$ a~$n=4k+2$ je rovný $2k+1$, napokon
v~prípade $n=4k+3$ je rovný $2k+2$.]

\endnávod
}

{%%%%%   C-I-6
Dokážeme, že pravouhlé trojuholníky $AKN$ a $DNM$ sú
podobné. Keďže uhol $KNM$ je pravý, uhly $KNA$ a $DNM$
sa dopĺňajú do $90^\circ$. To isté ale platí o~uhloch $DNM$ a $NMD$
pravouhlého trojuholníka $DNM$. Dostávame tak zhodnosť ostrých uhlov
$KNA$ a $NMD$, a tak sú trojuholníky $AKN$ a $DNM$ podľa vety~\emph{uu}
naozaj podobné. Pomer ich podobnosti určíme ako pomer
dĺžok ich prepôn $|NM|:|KN|$, ktorý je ako pomer
$|KL|:|LM|$ podľa zadania rovný~$3:1$.
\inspsc{c73i.61}{.8333}%

Analogicky sa zdôvodní vzájomná podobnosť všetkých
štyroch pravouhlých trojuholníkov $AKN$, $DNM$, $CML$ a $BLK$, ktoré
\uv{obklopujú} obdĺžnik $KLMN$. Trojuholníky $AKN$ a $CML$
(rovnako ako $DNM$ a $BLK$) sú dokonca zhodné, pretože ich
prepony sú protiľahlými stranami obdĺžnika.

Bez ujmy na všeobecnosti sme dĺžku úsečky $AK$ označili
na \obr{} číslom $1$ a dĺžku úsečky $AN$ písmenom $x$.
Z~dokázaných podobností a zhodností potom máme
$|DN|=3$, $|DM|=3x$ a $|CM|=1$, a teda $|AB|=|CD|=3x+1$ a
$|BC|=|AD|=x+3$. Dosadením do zadaného pomeru $|AB|:|BC|=2:1$
dostaneme rovnicu $(3x+1)=2(x+3)$ s~jediným riešením $x=5$,
ktorému zodpovedá $|AB|=16$ a $|BC|=8$. Obdĺžnik $ABCD$
s obsahom $S_{ABCD}=16\cdot8=128$ je zjednotením obdĺžnika $KLMN$
s~trojuholníkmi $AKN$, $CML$ (tie majú dokopy obsah $1\cdot5=5$) a trojuholníkmi
$DNM$, $BLK$ (s celkovým obsahom $3\cdot15=45$). Preto
pre obsah obdĺžnika $KLMN$ platí
$S_{KLMN}=128-5-45=78$.\fnote{Iný výpočet:
$|KN|=\sqrt{1^2+5^2}=\sqrt{26}$, $|KL|=3|KN|=3\sqrt{26}$,
$S_{KLMN}=|KL|\cdot|KN|=78.$}
Získavame tak odpoveď $S_{KLMN}:S_{ABCD}=78:128=39:64$.

\návody

Vo štvorci $ABCD$ zvoľme vo vnútri strany $AB$ ľubovoľne
bod $K$ a vo vnútri strany $BC$ ľubovoľne bod $L$. Na polpriamke
$CD$ zvoľme bod $M$ tak, aby $|\uhol KLM|=90^\circ$. Dokážte, že
trojuholníky $BLK$ a $CML$ sú podobné.
[Ukážte, že oba trojuholníky majú zhodné vnútorné uhly.]

Na stranách $AB$, $BC$, $CD$, $DA$ štvorca $ABCD$
postupne zvolíme body $K$, $L$, $M$, $N$ tak, že
$|AK|:|KB|=|BL|:|LC|=|CM|:|MD|=|DN|:|NA|=2:1$.
a)~Dokážte, že $KLMN$ je štvorec.
b) Určte pomer obsahov štvorcov $KLMN$ a~$ABCD$.
[a)~Podľa zadania majú pravouhlé trojuholníky $AKN$, $BLK$, $CMN$ a~$DNM$
zhodné dvojice odvesien, takže sú zhodné podľa vety {\it sus}.
Majú teda zhodné aj prepony a dvojice ostrých vnútorných uhlov.
Štvoruholník $KLMN$ tak má všetky strany zhodné a všetky vnútorné uhly
pravé -- napríklad rovnosť $|\uhol KLM|=90^{\circ}$ vyplýva z~toho, že
ostré uhly $BLK$ a $MLC$ sa dopĺňajú do~$90^{\circ}$.
b) $5:9$. Nech $|AB|=3$. Potom $S_{ABCD}=9$ a~zhodné pravouhlé
trojuholníky $AKN$, $BLK$, $CMN$ a~$DNM$ majú odvesny dĺžok $1$ a~$2$.
Obsah každého z nich je $1$, teda $S_{KLMN}=9-4\cdot1=5$.]

\D
Dve tetrisové kocky zostavené zo štvorcov
o~rozmeroch $1\times1$ sa dotýkajú v~bodoch $A$, $B$,~$C$ ako na
\obr{}. Spočítajte $|AB|$.
\inspsc{c73i.62}{.8333}%
[\,$|AB|=5/4$. Na obrázku vidíme dva pravouhlé trojuholníky
s~preponami $AB$ a $BC$, ktoré majú jednu odvesnu dĺžky 1 a
zhodné k~nej priľahlé vnútorné uhly. Tieto dva trojuholníky sú teda
podľa vety $usu$ zhodné. Označme $x$ dĺžku ich
druhej odvesny. Potom $|AB|=|BC|=2-x$ a Pytagorova veta
dáva rovnicu $1^2+x^2=(2-x)^2$, odkiaľ $x=3/4$, a~preto
$|AB|=2-x=5/4$.]

Označme $E$ stred základne $AB$ lichobežníka $ABCD$, v~ktorom platí
$|AB|:|CD|={3:1}$. Uhlopriečka~$AC$ pretína úsečky $ED$, $BD$ postupne
v~bodoch $F$, $ G$. Určte postupný pomer
$|AF|:|FG|:|GC|$.
[$12:3:5$. Hľadajte podobné trojuholníky. Zdôvodnite, že
$\triangle ABG \sim \triangle CDG$ a $\triangle AEF \sim
\triangle CDF$. Z~prvej podobnosti vyplýva $|AG|:|CG|=3:1$,
z druhej $|AF|:|FC|=3:2$. Zvyšok je \uv{trocha
počítania}. Kompletné riešenie: \pdfklink{64-C-I-4}{https://skmo.sk/dokument.php?id=1338\#page=4}.]

Na \obr{} sú vyznačené uhly medzi uhlopriečkou a stranou
v troch pravouholníkoch $3\times1$, $2\times1$ a $1\times1$.
Dokážte, že súčet týchto troch uhlov je rovný $90^\circ$.
\inspsc{c73i.63}{.8333}%
[Trikovo zobrazíme vykreslenú uhlopriečku obdĺžnika $3\times1$ podľa
osi danej jeho \uv{dolnou} stranou dĺžky 3. Zdôvodnite, že
táto nová úsečka je preponou pravouhlého rovnoramenného
trojuholníka, ktorého jedna z~odvesien je vykreslená uhlopriečka
obdĺžnika $2\times1$. Zistíme tak, že súčet dvoch z troch
zadaných uhlov je $45^\circ$.]

\endnávod
}

{%%%%%   A-S-1
a) Dokážeme sporom, že nemožno získať žiadnu postupnosť,
ktorá obsahuje trikrát číslo 20, teda ani tú zo zadania úlohy.

Pripusťme naopak, že k~niektorému číslu $\overline{a_1a_2 \dots a_9}$
s~navzájom rôznymi ciframi existujú tri indexy
$1 \leq i < j < k~\leq 7$ také, že platí
$$
a_i+a_{i+1}+a_{i+2}=a_j+a_{j+1}+a_{j+2}=a_k+a_{k+1}+a_{k+2}=20.
\tag1
$$
Keďže súčet šiestich rôznych cifier je najviac
$9+8+7+6+5+4=39<2\cdot20$,
krajná trojica sčítancov
$(a_i,a_{i+1},a_{i+2})$ a $(a_k,a_{k+1},a_{k+2})$
sa musí \uv{prekrývať}. Platí teda $k\leqq i+2$, čo
vzhľadom na $i<j<k$ znamená, že $k=i+2$, a teda $j=i+1$.
Prvá rovnosť v~\thetag1 tak prejde
na $a_i+a_{i+1}+a_{i+2}=a_{i+1}+a_{i+2}+a_{i+3}$, odkiaľ
$a_i=a_{i+3}$, čo je spor. Tým je sľúbený dôkaz hotový.

\smallskip
b) Áno, vyhovuje napríklad číslo $849\,751\,623$. Súčty trojíc jeho
susedných cifier sú totiž (zľava doprava)
21, 20, 21, 13, 12, 9, 11.

\poznamka
Hoci je podané riešenie úplné, vysvetlíme, ako
príklad vyhovujúceho čísla pre časť b) nájsť.
Zistíme dokonca, že okrem uvedeného čísla $849\,751\,623$
vyhovuje už len jeho \uv{zrkadlová} kópia $326\,157\,948$.

Z~úvah podobných tým z~časti a) nášho riešenia vyplýva, že potrebné
súčty rovné číslu 21 nemôžu dávať ani dve disjunktné trojice cifier,
ani dve trojice s dvoma spoločnými ciframi~-- tým hovorme
ďalej susedná trojica. Dve trojice so súčtom 21 teda majú jednu
spoločnú cifru, takže obe susedia s rovnakou trojicou s menším
súčtom.

Pre každé dve susedné trojice cifier platí, že ich súčty sa
líšia o~rozdiel tých dvoch cifier, ktoré ležia iba v~jednej z~oboch
trojíc. Keďže takýto rozdiel je najviac rovný $9-1=8$ a súčty rôzne od 21 sú podľa zadania 9, 11, 12, 13 a 20,
trojica so súčtom 21 môže susediť jedine s~trojicami so súčtom 13 alebo 20. To môže nastať iba vtedy, ak je jedna z oboch trojíc
so súčtom~21 \uv{na kraji}, t.\,j. susedí iba s~jednou trojicou.

Rozoberme podrobne situáciu, keď jedna z~trojíc so súčtom 21 je
\emph{prvá zľava}. Podľa našich úvah vtedy
pre hľadané číslo $\overline{a_1a_2 \dots a_9}$ nastane jeden
z~prípadov:
\item{(i)} $a_1+a_2+a_3=21$, $a_2+a_3+a_4=13$, $a_3+a_4+a_5=21$, $a_4+a_5+a_6=20$,
\item{(ii)} $a_1+a_2+a_3=21$, $a_2+a_3+a_4=20$, $a_3+a_4+a_5=21$, $a_4+a_5+a_6=13$.

Prípad (i) ľahko vylúčime, pretože vtedy $a_1-a_4=8$, odkiaľ
$a_4=1$, a preto z~rovnosti $a_3+a_4+a_5=21$ vyplýva $a_3+a_5=20$,
čo je zrejmý spor.

V~prípade (ii) máme $a_3-a_6=8$, čiže $a_3=9$ a $a_6=1$, takže
zadané rovnosti môžeme po dosadení zjednodušiť na
$$
a_1+a_2=a_4+a_5=12\quad\hbox{a}\quad a_2+a_4=11.
$$
Keďže cifra 9 už tu nevystupuje, z dvoch súčtov
rovných~12 vyplýva, že $\{a_1, a_2\}$ a~$\{a_4, a_5\}$ sú v~niektorom poradí množiny $\{4, 8\}$ a
$\{5, 7\}$. Rovnosť $a_2+a_4=11$ potom vedie k~záveru, že
$\{a_2,a_4\}=\{4,7\}$. Prvých 6 cifier hľadaného deväťciferného
čísla je tak buď $849\,751$, alebo $579\,481$. Teraz už
k~týmto dvom začiatkom začneme skúšať dopĺňať sprava
vo vhodnom poradí cifry 2, 3, 6 tak, aby sme dostali nové
trojice susedných cifier s doteraz chýbajúcimi súčtami 9, 11 a 12.
Nájdeme tak jediné vyhovujúce číslo
$849\,751\,623$ (neukončené doplňovania vedú k~$849\,751\,3??$,
$579\,481\,26?$ a $579\,481\,3??$).\fnote{{Namiesto takého
skúšania je možné postupovať nasledovne: Keďže súčet siedmich čísel zo
zadania~b) je rovný 107, pre každé vyhovujúce číslo
$\overline{a_1a_2 \dots a_9}$ musí platiť
$2(a_1+a_9)+(a_2+a_8)=3\cdot(1+2+\ldots+9)-107=28$.
Pre $a_1=8$ a $a_2=4$ odtiaľ dostávame
$2a_9+a_8=8$, čo vzhľadom na $\{a_7,a_8,a_9\}=\{2,3,6\}$
už zrejme znamená, že nutne platí $\overline{a_7a_8a_9}=623$.
Pre $a_1=5$ a $a_2=7$ vychádza $2a_9+a_8=11$, čo však
s~ciframi $a_8,a_9\in\{2,3,6\}$ nie je možné splniť.}}

Druhú situáciu, keď trojica so súčtom 21 je \emph{prvá sprava}, nie je
nutné rozoberať. Od takého vyhovujúceho čísla
$\overline{a_1a_2 \dots a_9}$ totiž môžeme prejsť k~vyhovujúcemu číslu $\overline{a_9a_8 \dots a_1}$,
ktoré je podľa predchádzajúceho rozboru rovné $849\,751\,623$.
Druhej situácii tak zodpovedá jediné vyhovujúce číslo $326\,157\,948$.

\ineriesenie
Pre časť a) úlohy uvedieme iný dôkaz sporom.

Pripusťme teda, že súčty $S_i=a_i+a_{i+1}+a_{i+2}$,
pričom $1\leq i\leq7$, majú pre niektoré vytvorené číslo
$\overline{a_1a_2 \dots a_9}$ vzostupne usporiadané
hodnoty 9, 11, 12, 13, 20, 20, 20. Z~rovnosti
$$
S_1+S_4+S_7=1+2+\ldots+9=45
$$
a nerovností $11+12+13<45<20+20+9$ vyplýva, že práve jeden z~troch súčtov $S_1$, $S_4$, $S_7$ je rovný 20, teda zvyšné dva
súčty sú zrejme 12 a 13. Platí tak
$$
\{S_1,S_4,S_7\}=\{12,13,20\}.
\tag2
$$
Podľa \thetag2 rozlíšime ďalej tri prípady, pri ktorých využijeme
to, že vďaka rovnosti ${S_i-S_{i+1}}={a_i-a_{i+4}}$ a rôznosti cifier
$a_1,\ldots,a_9$ platí
$$
0<|S_i-S_{i+1}|<9\quad(1\leq i\leq6).
\tag3
$$

\item{(i)} V~prípade $S_1=20$ podľa \thetag3 platí $0<|S_2-20|<9$, čo je v~spore s~tým, že vzhľadom na \thetag2 je $S_2$ jedno z~čísel 9, 11, 20.
\item{(ii)} V~prípade $S_4=20$ dostaneme spor ako v~(i), kde $S_2$ zameníme za $S_3$.
\item{(iii)} V~prípade $S_7=20$ stačí podobne v~(i) zameniť $S_2$ za $S_6$.

Tým je dôkaz sporom hotový.

\poznamka
K práve dokončenému dôkazu sporom dodajme
nasledujúce. Pretože tri zo súčtov $S_i$ majú rovnakú hodnotu 20,
možno potrebný spor získať z nerovností \thetag3 aj bez použitia výsledku \thetag2
o hodnotách $S_1$, $S_4$, $S_7$. Naozaj, vďaka \thetag3 môžu
tri čísla~20 susediť v~sedmici $(S_1,S_2,S_3,S_4,S_5,S_6,S_7)$
iba s dvoma číslami 12 a 13. Je však zrejmé, že každá trojica
po dvoch nesusedných členov má tú vlastnosť, že jej členy susedia dokopy
s~aspoň troma ďalšími členmi.\fnote{Ak je totiž v každej
z oboch \uv{medzier} medzi členmi danej trojice iba po jednom člene,
leží aspoň jeden ďalší člen pred prvým alebo
za posledným členom tejto trojice.} Tým je nový dôkaz sporom hotový.

\poznamka
Ukážme, že aj úvahy z~druhého dôkazu sporom
je možné využiť pri hľadaní všetkých čísel
$\overline{a_1a_2 \dots a_9}$, ktoré vyhovujú
časti b) zadania. Pre ľubovoľné z~ich
znova označíme $S_i=a_i+a_{i+1}+a_{i+2}$, kde $1\leq i\leq7$.
Nerovnosti \thetag3 pritom budeme využívať bez odkazov.

Podľa zadaných hodnôt 9, 11, 12, 13, 20, 21, 21 súčtov $S_i$
tentoraz zistíme, že $\{S_1,S_4,S_7\}$ je jedna z~množín
$\{12,13,20\}$ alebo $\{11,13,21\}$. Štvorica zvyšných súčtov
$(S_2,S_3,S_5,S_6)$ je tak (až na poradie) jedna zo štvoríc
$(9,11,21,21)$ alebo $(9,12,20,21)$. Keďže navyše platí
$$
|(S_2+S_5)-(S_3+S_6)|=|a_2-a_8|<9,
$$
každá z dvojíc $(S_2,S_5)$ a $(S_3,S_6)$ zrejme obsahuje číslo
20 alebo 21, pritom iba jedna z~nich aj číslo 9. Bez ujmy
na všeobecnosti predpokladajme, že touto dvojicou s číslom 9 je $(S_2,S_5)$ (inak
cifry východiskového čísla zapíšeme v~opačnom poradí). Ďalej
rozlíšime dva prípady.

\smallskip
$\bullet$ Prípad $S_2=9$. Vtedy ako vieme platí $S_5\geqq 20$,
a keďže $S_3<S_2+9<20$, taktiež $S_6\geqq20$.
Vzhľadom na $S_5\ne S_6$ tak máme $\{S_5,S_6\}=\{20,21\}$, a preto
$\{S_1,S_4,S_7\}=\{11,13,21\}$, teda $S_3$ je \uv{zvyšná} hodnota
12. Teraz z~$S_2=9$ a $S_3=12$ vyplýva, že 21 sa nerovná ani $S_1$
ani $S_4$, a preto $21=S_7$, odkiaľ $S_6=20$ a $S_5=21$.
Podľa poslednej rovnosti je $S_4>12$, a preto $S_4=13$ a $S_1=11$.

Zistili sme, že v~prípade $S_2=9$ platí
$$
(S_1,S_2,\dots,S_7)=(11,9,12,13,21,20,21).
$$
Odtiaľ vychádza $a_7-a_4=S_{5}-S_4=8$, takže $a_4=1$ a $a_7=9$, teda
z~$S_2=9$ vyplýva $a_2+a_3=8$, a preto v~dôsledku $S_1=11$ je
$a_1=3$. Umiestnenie cifier 1, 3 a 9 tak poznáme, zvyšné cifry 2, 4,
5, 6, 7, 8 sú v~niektorom poradí riešením sústavy rovníc
$$
a_2+a_3=8,\ a_3+a_5=11,\ a_5+a_6=12,\ a_6+a_8=11,\
a_8+a_9=12.
$$
Vidíme, že cifra 2 je nutne $a_2$, odkiaľ postupne $a_3=6$,
$a_5=5$, $a_6=7$, $a_8=4$ a $a_9=8$. Dostali sme vyhovujúce číslo
$326\,157\,948$. Druhé vyhovujúce číslo s~opačným poradím cifier
je $849\,751\,623$.

\smallskip
$\bullet$ Prípad $S_5=9$. Vtedy ako vieme platí $S_2\geqq 20$,
a keďže $S_6<S_5+9<20$, taktiež $S_3\geqq20$. Vzhľadom na $S_2\ne
S_3$ tak máme $\{S_2,S_3\}=\{20,21\}$, a preto opäť
$\{S_1,S_4,S_7\}=\{11,13,21\}$, teda $S_6$ je \uv{zvyšná} hodnota
12. Teraz z~$S_5=9$ a $S_6=12$ vyplýva, že 21 sa nerovná ani $S_4$
ani $S_7$, a preto $21=S_1$, odkiaľ $S_2=20$ a $S_3=21$.
Podľa poslednej rovnosti je $S_4>12$, a preto $S_4=13$ a $S_7=11$.

Zistili sme, že v~prípade $S_5=9$ platí
$$
(S_1,S_2,\dots,S_7)=(21,20,21,13,9,12,11).
$$
Odtiaľ vychádza $a_3-a_6=S_3-S_4=8$, takže $a_3=9$ a $a_6=1$, teda
z~$S_6=12$ vyplýva $a_7+a_8=11$, a preto v~dôsledku $S_7=11$ je
$a_9=0$, a to je spor. Žiadne vyhovujúce číslo so súčtom $S_5=9$
preto nevyhovuje.


\schemaABC
Za úplné riešenie dajte 6 bodov, z toho 3 body za časť a) a 3 body
za časť b).
\smallskip
\noindent V~neúplných riešeniach časti a) oceňte
čiastočné kroky z vyššie popísaných postupov nasledovne:
\item{A1.} Sformulovaná hypotéza o~tom, že nemožno mať tri trojice so súčtom 20: 1 bod, len pokiaľ riešiteľ nevyrieši časť b), inak 0 bodov.
\item{A2.} Dve trojice so súčtom 20 nemôžu byť disjunktné (s~dôkazom): 1 bod.
\item{A3.} Dve trojice so súčtom 20 nemôžu byť susedné, t.\,j. mať dve spoločné cifry (s~dôkazom): 1~bod.
\item{A4.} Žiadna trojica so súčtom 20 nemôže susediť s trojicou so súčtom 9 ani 11 (s dôkazom): 1~bod.

\noindent
Celkom potom za neúplné riešenie časti a) dajte $\rm\max(A1,A2+A3,A3+A4)$ bodov.

\smallskip
Ako sme uviedli v~komentári, riešenie časti b) je úplné aj v~prípade,
keď riešiteľ číslo $849\,751\,623$ alebo $326\,157\,948$
uvedie bez vysvetlenia, ako k~nemu prišiel. Za neúplné riešenie
časti b) dajte 1~bod, ak riešiteľ napríklad dokáže, že dve trojice so
súčtom 21 majú spoločnú práve jednu cifru a
že trojice, ktoré s~niektorou z~nich susedia, majú súčet 13 alebo 20.
Ak riešiteľ odvodí, ako musia vyzerať šesťciferné začiatky
alebo konce všetkých vyhovujúcich čísel, dajte za časť b) 2 body.
\endschema
}

{%%%%%   A-S-2
Vyhovujúce mnohočleny $P(x)$ majú mať tvar
$P(x) = ax^2+bx+c$, pričom $a$, $b$, $c$ sú celé čísla.
Ak by pre koeficient pri $x^2$ platilo $a<1$,
prvá zadaná nerovnosť by neplatila pre veľké hodnoty $x$.
Podobne druhá nerovnosť vylučuje
možnosť, že platí $a>2$. Ostatné dva prípady
$a\in\{1,2\}$ posúdime jednotlivo.

Pre $a=1$ prepíšeme zadané nerovnosti na tvar
$$
(b-2)x + (c+2023) > 0\quad\hbox{a}\quad x^2 -bx - c > 0.
$$
V~prípade $b\ne2$ má prvá nerovnosť nenulový koeficient
pri $x$, takže nemôže byť splnená pre všetky $x$, nech už je $c$ akékoľvek.
Musí preto
platiť $b=2$ a prvá nerovnosť potom prejde na $c>-2023$.
Druhá nerovnosť platí pre všetky $x$ práve vtedy, keď
trojčlen $x^2 - bx - c$ s~kladným koeficientom pri $x^2$
má záporný diskriminant, teda práve vtedy, keď $b^2 + 4c<0$.
To vďaka $b=2$ prechádza na $c<-1$. Dokopy
dostávame, že v~prípade $a=1$ je platnosť oboch nerovností pre
všetky $x$ ekvivalentná s~dvojicou podmienok $b=2$ a $-1>c>-2023$,
ktoré zrejme spĺňa 2021 trojčlenov $P(x)$.

Analogicky budeme postupovať aj v prípade $a=2$, len to zapíšeme
stručnejšie. Prepísané nerovnosti majú tentoraz tvar
$$
x^2+(b-2)x + (c+2023) > 0\quad\hbox{a}\quad bx + c < 0.
$$
Druhá nerovnosť platí pre všetky $x$ práve vtedy, keď $b=0$ a $c<0$.
Platnosť prvej nerovnosti pre všetky $x$
opäť vyjadríme podmienkou záporného diskriminantu.
Tento výraz $(b-2)^2-4(c+2023)$ je po dosadení $b=0$ záporný
práve vtedy, keď $c > -2022$. V~prípade $a=2$ tak vyhovujú trojčleny
$P(x)$, pre ktoré $b=0$ a $0>c> -2022$. Aj tých je zrejme 2021.

\zaver
Hľadaný počet mnohočlenov je rovný $2\cdot2021=4042$.

\schemaABC
Za úplné riešenie dajte 6 bodov. V~neúplných riešeniach oceňte
čiastočné kroky nasledovne:
\item{A1.} Konštatovanie, že nutne platí $a\in\{1,2\}$ (možno brať za zrejmý dôsledok známych vlastností kvadratickej funkcie): 2 body.
\item{A2.} Určenie $b$ (možno brať za zrejmé) a vymedzenie $c$ pri obidvoch lineárnych nerovnostiach: celkom 1 bod.
\item{A3.} Vymedzenie $c$ pri dvoch kvadratických nerovnostiach: po 1 bode za každú z~nich.
\item{A4.} Určenie správneho počtu mnohočlenov: 1 bod.

Celkom potom dajte $\rm A1+A2+A3+A4$ bodov. Ak sa riešiteľ
zaoberá iba prípadmi $a\in\{1,2\}$ a~nenapíše, že to stačí,
dajte najviac 4 body.
\endschema
}

{%%%%%   A-S-3
Označme ešte $M$ stred úsečky $XB$. Ukážeme, že ťažnica $OM$ trojuholníka $OXB$
leží na osi úsečky $BY$.\fnote{Táto hypotéza nie je tak
prekvapujúca, lebo vzhľadom na zrejmú rovnosť $|OB|=|OY|$ je
dokazovaná rovnosť $|GB|=|GY|$ ekvivalentná s~tým, že priamka $OG$
je osou úsečky $BY$.} Keďže ťažisko $G$ tejto ťažnici patrí, bude
tým želaná rovnosť $|GB|=|GY|$ dokázaná.

Keďže bod $Y$ leží na Tálesovej kružnici nad priemerom $AB$ so
stredom $O$, platí $|OY| = |OB|$ a uhol $AYB$ je pravý.
Bod $M$ je tak stredom
prepony $XB$ pravouhlého trojuholníka $XBY$ a ako taký je aj stredom kružnice
jemu opísanej. Platí tak ${|MY| = |MB|}$. To spolu s~$|OY| = |OB|$
znamená, že oba krajné body $O$,~$M$ ťažnice~$OM$ ležia na osi úsečky~$BY$. Tým je tvrdenie úlohy dokázané.
\inspsc{a73ii.31}{.8333}%

\poznamka
Podaný výklad môžeme mierne obmieňať úvahami o stredných priečkach
trojuholníkov $BAX$, $BAY$ alebo $BXY$. Ich zapojením môžeme
napríklad dokazovať tieto tvrdenia (v~zátvorkách naznačíme ako):
\item{$\triangleright$} Priamka $OM$ (kde $M$ je stred $XB$) je osou úsečky $BY$. (Úsečka $OM$ je strednou priečkou trojuholníka $BAX$, teda $OM\parallel AX\perp BY$.)
\item{$\triangleright$} Os úsečky $BY$ rozpoľuje obe úsečky $AB$ a $XB$. (Stredné priečky oboch pravouhlých trojuholníkov $BAY$ a $BXY$, ktoré sú rovnobežné s~$AY$, a teda kolmé na $BY$, ležia na osi ich spoločnej odvesny $BY$.)
\inspsc{a73ii.32}{.8333}%

\ineriesenie
Použitím druhej ťažnice $XJ$ trojuholníka $XOB$ ($J$ je stred strany~$OB$)
ukážeme ako v~prvom riešení, že úsečka $OG$ leží na osi úsečky $BY$.
Bod $O$ na nej leží vďaka zrejmej rovnosti $|OB|=|OY|$,
stačí teda iba overiť, že platí $OG\perp BY$, čiže
$OG\parallel AY$, pretože uhol $AYB$ je pravý podľa Tálesovej vety.

Podľa známej polohy ťažiska $G$ na ťažnici $XJ$ platí $|JG|=\frac13|JX|$
a okrem toho je zrejme aj $|JO|=\frac13|JA|$. Dokopy to znamená, že
úsečka $OG$ je rovnoľahlá s~úsečkou~$AX$ podľa stredu~$J$.
Platí teda $OG\parallel AX$. Tým je želaný vzťah
$OG\parallel AY$ overený, pretože $X$ je vnútorný bod úsečky $AY$.
\inspsc{a73ii.33}{.8333}%

\ineriesenie
Ukážeme, že na~riešenie úlohy je možné využiť aj tretiu ťažnicu $BN$
trojuholníka~$XOB$, pričom $N$ je stred jeho strany $XO$.

Označme ešte $K$ stred úsečky $AX$ a $L$ priesečník polpriamky $BN$
s~úsečkou~$AX$. Pre strednú priečku $KN$ trojuholníka $XAO$ platí
$|KN|=\frac12|AO|=\frac14|AB|$ a $KN\parallel AO$, čiže $KN\parallel
AB$. V~dôsledku toho je úsečka $KN$ obrazom úsečky $AB$
v rovnoľahlosti so stredom $L$ a koeficientom $1/4$.
Platí tak $|LN|=\frac14|LB|$,
odkiaľ $|LN|=\frac13|BN|$. Túto dĺžku má aj úsek $NG$ ťažnice $BN$,
takže dokopy dostávame
$$
|LG|=|LN|+|NG|=\tfrac13|BN|+\tfrac13|BN|=|GB|.
\tag1
$$
Všimnime si teraz trojuholník $BLY$. Ten má pri vrchole $Y$ pravý uhol
vďaka Tálesovej kružnici nad priemerom $AB$. Stred jeho prepony
$BL$ je však podľa \thetag1 práve bod $G$, teda podľa Tálesovej vety
má aj úsečka $YG$ dĺžku $\frac23|BN|$.
Tým je rovnosť $|YG|=|GB|$ dokázaná.
\inspsc{a73ii.34}{.8333}%

\schemaABC
Za úplné riešenie dajte 6 bodov. V~neúplných postupoch podľa
vzorových riešení alebo poznámky oceňte
čiastočné kroky nasledovne (nové body označujeme rovnako ako
v~textoch riešenia):
\item{A1.} Zápis hypotézy, že na osi úsečky $BY$ leží nielen ťažisko $G$, ale celá ťažnica $OM$:~1~bod.
\item{A2.} Za hypotézu z~$\rm A1$ spolu s~konštatovaním, že $|OB|=|OY|$: 2 body.
\item{A3.} Dôkaz rovnosti $|MB|=|MY|$: 3 body.
\item{A4.} Dôkaz rovnobežnosti $OM\parallel AX$: 2~body.
\item{A5.} Dôkaz kolmosti $OM\perp BY$: 3 body.
\item{B1.} Vysvetlenie, prečo stačí dokázať $OG\perp BY$: 2~body.
\item{B2.} Dôkaz rovnobežnosti $OG\parallel AX$: 2~body.
\item{C1.} Dôkaz tvrdenia, že bod $G$ je stredom úsečky $BL$: 4 body.

\noindent Celkom potom dajte
$\rm\max(A1, A2, A1 + A3, A1 + A4, A1 + A5, B1, B1 + B2, C1) $
bodov.
\endschema
}

{%%%%%   A-II-1
Pred uvedením riešenia poznamenajme, že
oba súčty čísel v~zadaných štvoriciach $(24, 24, 25, 25)$ a~$(20,23,26,29)$
sú rovné tomu istému číslu 98. Ako ukážeme v~riešení časti~b), je to
najvyššia možná hodnota takého súčtu, ktorú navyše dosiahneme
práve pri tých tabuľkách, ktoré majú číslo 9 uprostred a~čísla 1, 2, 3, 4 v~rohoch. Tento poznatok zároveň uľahčuje konštrukciu potrebného príkladu tabuľky na riešenie časti~a).

\smallskip
a) Áno, napríklad pre nasledujúcu tabuľku:
\inspscno{a73iii.11}{.8333}%

Je ľahké overiť, že
súčty čísel vo štvorcoch $2\times 2$ majú požadované hodnoty
24, 24, 25 a 25.

b) Nie. Pre dôkaz uvažujme hodnoty súčtov $S$, ktoré
dostaneme sčítaním štyroch súčtov zo štvorcov $2\times2$ každej
vyplnenej tabuľky. Do súčtu $S$ prispieva jedno číslo tabuľky
štyrikrát (nazveme ho \uv{stredovým}), štyri jej čísla
dvakrát (čísla \uv{postranné}) a zvyšné štyri čísla jedenkrát
(čísla \uv{rohové}). Preto súčet $S$ bude najväčší možný,
ak najväčšie číslo 9 bude stredové, štyri menšie čísla 8, 7,
6, 5 postranné a štyri najmenšie čísla 4, 3, 2, 1 rohové.
Len pri takom rozmiestnení čísel v~tabuľke dosiahneme hodnotu
$$
S=4\cdot 9+2\cdot(8+7+6+5)+(4+3+2+1)=98.
$$
Keďže v~každom štvorci $2\times2$ sú okrem čísla
stredového dve čísla postranné a jedno číslo rohové,
je v~prípade $S=98$ súčet týchto štyroch čísel aspoň
$9+5+6+1=21$, a teda sa nemôže rovnať 20. Keďže
však pre štvoricu čísel zo zadania~b) platí $20+23+26+29=98$,
nie je to štvorica súčtov čísel zo štvorcov $2\times2$
žiadnej vyplnenej tabuľky.

\poznamka
Práve podané riešenie časti b) je možné skrátiť použitím
jednoduchého výsledku z~úvodnej časti nasledujúceho riešenia,
že štvorec so súčtom čísel 29 musí byť vyplnený číslami
9, 8, 7, 5. V~tejto situácii totiž pre uvažovaný súčet $S$ z~prvého
riešenia dostávame odhad
$$
S\leqq 4\cdot 9+2\cdot(8+7+6+4)+(5+3+2+1)=97,
$$
teda $S$ nemôže mať potrebnú hodnotu $20+23+26+29=98$.

\ineriesenie
Zápornú odpoveď pre časť b) dokážeme sporom
odlišným spôsobom, pri ktorom sa zaobídeme bez poznatku
z~úvodného odseku k~prvému riešeniu.
Za tým účelom dva štvorce $2\times2$ tabuľky $3\times3$ nazveme \uv{protiľahlými},
ak majú spoločné práve jedno políčko (uprostred tabuľky).

Pripusťme, že vyplnenie tabuľky pre štvoricu súčtov $(20,23,26,29)$
existuje. Keďže $9+8+7+6=30$, ležia vo štvorci $2\times2$ so súčtom 29
práve čísla 9, 8, 7, 5. Vyberme dva protiľahlé štvorce tak,
aby jeden z~nich mal súčet čísel $S_1=29$; súčet v~protiľahlom
štvorci označíme $S_2$ a ako $a$, $b$ (resp. $c$, $d$)
označíme čísla v~rohoch celej tabuľky, ktoré sú (resp. nie sú)
do súčtov $S_1$, $S_2$ zahrnuté, povedzme $a$ do $S_1$ a~$b$
do~$S_2$:
\inspscno{a73iii.12}{.8333}%

Pre číslo $s$ uprostred tabuľky potom zrejme platí
$$
s=S_1+S_2+c+d-(1+2+\ldots+9)=29+S_2+c+d-45=S_2+c+d-16.
$$
Keby platilo $S_2\geqq23$, z~poslednej rovnosti by sme mali
$s\geq 7+c+d$, čo je spor, pretože $s\leq9$ a $c+d\geq3$. Platí tak
nutne $S_2=20$, teda zvyšné dva protiľahlé štvorce majú súčty
$S_3=26$ a $S_4=23$. Z~analogickej rovnosti
$$
s=S_3+S_4+a+b-(1+2+\ldots+9)=26+23+a+b-45=a+b+4
$$
vzhľadom na $a\geqq 5$ (lebo $a$ je jedno z~čísel 9, 8, 7, 5)
vyplýva $s>9$, čo je konečný spor.

\schemaABC
Za úplné riešenie úlohy dajte 6 bodov, z toho 2 body za časť a),
kde stačí teda uviesť príklad jednej vyplnenej tabuľky, a 4 body za
časť b). V~neúplných riešeniach časti b) oceňte čiastočné kroky
nasledovne (uvažujeme len vyhovujúce tabuľky a štvorce $2\times2$
v~nich):

\smallskip
\item{A1.} Zdôvodnenie, prečo v~tabuľke je číslo 9 stredové: 1 bod.
\item{A2.} Zdôvodnenie, prečo v~tabuľke sú čísla 5, 6, 7, 8 postrannými, resp. čísla 1, 2, 3, 4 rohovými: 1~bod.
\item{B1.} Zdôvodnenie, prečo protiľahlými štvorcami sú tie so súčtami 29 a 20, resp. 23 a 26: 1 bod.
\item{B2.} Vyjadrenie stredového čísla pomocou súčtov pre dva protiľahlé štvorce a dvoch čísel, ktoré do nich nepatria: 1 bod.
\item{C1.} Zdôvodnenie, prečo štvorec so súčtom 29 je vyplnený číslami 9, 8, 7, 5: 1~bod.

\smallskip\noindent
Celkom potom za časť b) dajte $\rm \max(A1+A2,B1+B2)+C1$ bodov.
\endschema
}

{%%%%%   A-II-2
\def\D{{\mathop{\text D\null}\nolimits}}%
Ukážeme, že úlohe vyhovujú všetky dvojice kladných celých čísel
$(k,n)$.

Pre $n=1$ a ľubovoľné $k$ zvolíme $(a,b)=(k,k)$.
Potom platí $\D(a+k,b)=\D(2k,k)=k$ a rovnako tak
aj $\D(a,b) = \D(k,k) = k$, teda požadovaná rovnosť je splnená.

Pre $n>1$ a ľubovoľné $k$ platí $nk-k>0$, a tak môžeme zvoliť
$(a,b)=(nk-k,nk)$. Potom platí
$$
\D(a+k,b) =\D(nk,nk)=nk
$$
a rovnako tak aj
$$
n\cdot\D(a,b)= n\cdot\D(k(n-1),kn) = nk\cdot\D(n-1,n)=nk,
$$
pričom sme v~poslednej rovnosti využili fakt, že po sebe idúce prirodzené
čísla $n-1$ a~$n$ sú nesúdeliteľné. Želaná rovnosť je teda opäť splnená,
čím je celý dôkaz ukončený.

\poznamka
Uvedené riešenie je úplné, ale neposkytuje žiadny návod, ako sa k nemu
dopracovať. Naznačíme jeden možný spôsob, ako príklady
potrebných dvojíc čísel $(a,b)$ hľadať.

Riešme úlohu najprv pre $k=1$ a ľubovoľné $n$, keď
máme nájsť príklad dvojice $(a,b)$ s~vlastnosťou
$$
\D(a+1,b)=n\cdot\D(a,b).
\tag1
$$
Z~tejto rovnosti vyplýva, že $\D(a,b)\mid \D(a+1,b)$, takže číslo
$\D(a,b)$ musí byť nielen deliteľom čísel $a$ a $b$, ale aj
deliteľom čísla $a+1$. Čísla $a$ a $a+1$ sú však nesúdeliteľné,
takže musí platiť $\D(a,b)=1$. Vtedy \thetag1 prejde na $\D(a+1,b)=n$,
takže hľadáme príklad nesúdeliteľných čísel $a$, $b$ s~vlastnosťou
$\D(a+1,b)=n$. Nájsť taký príklad je ľahké: v~prípade $n>1$
vyhovuje dvojica $(a,b)=(n-1,n)$, v~prípade $n=1$ dvojica
$(a,b)=(1,1)$.

Prechod od hodnoty $k=1$ k~všeobecnému $k>1$ založíme na
nasledujúcom pozorovaní, platnom pre každé pevné $n$:
Ak pre nejaké čísla $a$, $b$ platí \thetag1,
tak pre čísla $a'=ka$ a $b'=kb$ platí rovnosť
$$
\D(a'+k,b')=n\cdot\D(a',b').
\tag2
$$
Overiť rovnosť \thetag2 za predpokladu \thetag1 je ľahké:
$$\eqalign{
\D(a'+k,b')&=\D(ka+k,kb)=k\cdot\D(a+1,b)=k\cdot(n\cdot\D(a,b) )=\cr
&=n\cdot(k\cdot\D(a,b))=n\cdot\D(ka,kb)=n\cdot\D(a',b').}
$$
Z dvojice $(a,b)$, ktorá je príkladom riešenia rovnice \thetag1
s~hodnotou $k=1$, sme tak dokázali \uv{vyrobiť} príklad riešenia
$(a',b')$ rovnice \thetag2 so všeobecnou hodnotou $k$.

\schemaABC
Za úplné riešenie dajte 6 bodov. V~neúplných riešeniach oceňte
čiastočné závery nasledovne:
\item{A0.} Existencia čísel $a$, $b$ pre konečne veľa dvojíc $(k,n)$ alebo hypotéza o~ich existencii pre ľubovoľnú dvojicu $(k,n)$: 0 bodov.
\item{A1.} Existencia čísel $a$, $b$ pre nekonečne veľa dvojíc $(k,n)$, v ktorých $n \ne1$: (typicky pre dvojice $(1,n)$): 1 bod.
\item{A2.} Existencia čísel $a$, $b$ pre nekonečne veľa dvojíc $(k,n)$, v ktorých $k$ aj $n$ nadobúdajú nekonečne veľa hodnôt (napríklad všetky prípady $k=n$) : 2~body.
\item{A3.} Existencia čísel $a$, $b$ chýba iba pre konečný počet hodnôt $k$ či konečný počet hodnôt $n$ (typicky vynechanie či chybné riešenie prípadu $n=1$): 5 bodov.
\item{B1.} Zdôvodnenie, prečo stačí riešiť iba prípad $k=1$ (pozri komentár za riešením): 2~body.

\noindent
Celkom potom dajte $\rm \max(A1,A2,A3,B1)$ bodov.
\endschema
}

{%%%%%   A-II-3
Kružnice so stredmi $O_1$ a $O_2$ majú spoločnú tetivu $PQ$.
Priamka $O_1O_2$ je tak osou tejto úsečky a pretne ju v~jej strede.
Preto priamka $O_1O_2$ nemôže prechádzať vrcholom $A$,
pretože ten leží na priamke $PQ$, avšak nie vo vnútri úsečky~$PQ$.
Upresnime ešte, že vďaka ostrouhlosti trojuholníka $ABC$ oba
stredy $O_1$, $O_2$ zrejme ležia vo vnútri
polroviny $BCP$.\fnote{Uhly $BPQ$ a
$CPQ$ sú totiž oba ostré, pretože sú zhodné postupne
s~uhlami $BCA$ a $CBA$.}

Aby sme dokázali implikáciu zo zadania úlohy,
predpokladajme najprv, že priamka~$O_1O_2$
prechádza vrcholom $C$ (\obr).
\inspdf{a73iii_31.pdf}%

V~trojuholníku $PQC$ potom os strany $PQ$ prechádza vrcholom $C$, teda tento
trojuholník je rovnoramenný s~hlavným vrcholom $C$. Preto platí
$$
|\uhol BQA|=|\uhol CQP|=|\uhol CPQ|=|\uhol CPA|=|\uhol CBA|,
$$
pričom v~poslednom kroku sme využili zhodnosť obvodových uhlov
nad oblúkom $AC$ kružnice~$k$. Trojuholník $BQA$ je tak rovnoramenný
s~hlavným vrcholom $A$. Oba body $A$ a~$O_1$ preto ležia na
osi úsečky $BQ$, odkiaľ vyplýva
$$
|\uhol BAO_1| = 90^\circ-|\uhol ABQ|=90^\circ-|\uhol CQP|=|\uhol BCO_1|.
$$
Úsečku $BO_1$ teda vidno z~bodov $A$ a~$C$ pod rovnakým
uhlom, a štvorica bodov $B$, $O_1$, $C$ a~$A$ teda leží na jednej
kružnici.
Tým sme dokázali potrebný záver, že bod~$O_1$ leží na kružnici~$k$.

\smallskip
V druhom prípade, keď priamka $O_1O_2$ prechádza vrcholom $B$,
z~analogického postupu vyplýva, že na kružnici $k$ leží bod $O_2$.
(Môžeme však tiež navzájom vymeniť označenie vrcholov $B$ a $C$.)
Tým je dôkaz implikácie zo zadania úlohy hotový.

\ineriesenie
Odlišným spôsobom posúdime vyššie rozoberaný prípad,
keď priamka $O_1O_2$ prechádza vrcholom $C$, takže je osou
súmernosti rovnoramenného trojuholníka $PQC$. Za tým účelom
uvažujme priesečník polpriamky $CO_2$ s~kružnicou $k$,
ktorý označíme~$X$ ako na \obr{}.
Určite stačí ukázať, že priamka~$AX$ je osou úsečky $BQ$, pretože
s~prihliadnutím k~tomu, že priamka $CX$ je osou úsečky $PQ$,
potom už bude platiť $X=O_1$, a~teda $O_1\in k$.

Keďže polpriamka $CX$ je osou uhla $PCQ$, čiže $PCB$, je bod
$X$ stredom toho oblúka $BP$ kružnice $k$, ktorý neprechádza
bodom $C$. Z~toho vyplýva zhodnosť štyroch obvodových uhlov~$PCX$,
$BCX$, $PAX$ a $BAX$. Označme ešte $R$ stred úsečky~$PQ$ a~$Y$ priesečník úsečiek $BQ$ a~$AX$. Potom porovnaním vnútorných uhlov
podfarbených trojuholníkov~$AQY$ a~$CQR$ zisťujeme, že
uhol $AYQ$ je rovnako ako uhol $CRQ$ pravý.
Os $AX$ uhla $BAQ$ je teda kolmá na úsečku $BQ$,
a preto priamka $AX$ je osou tejto úsečky, ako sme sľúbili
ukázať.
\inspdf{a73iii_32.pdf}%

Dodajme, že práve podaný výklad je možné obmeniť napríklad tak,
že na odvodenie kolmosti $AX\perp BQ$ využijeme štvoruholník $X\!RQY$
alebo $AY\!RC$, pri ktorých je možné zo zhodnosti vhodných uhlov
ľahko nahliadnuť, že sú tetivové.

\schemaABC
Za úplné riešenie dajte 6 bodov, z toho 1 bod za konštatovanie (aj bez dôkazu), že na priamke~$O_1O_2$ nemôže ležať
vrchol $A$, a 5 bodov za vyriešenie situácie, keď priamka
$O_1O_2$ prechádza vrcholom~$B$ alebo~$C$. Ďalšie pokyny zapisujeme
len pre prípad, keď priamka $O_1O_2$ prechádza vrcholom~$C$ ako
v~oboch podaných riešeniach.

V~neúplných riešeniach 5-bodovej časti oceňte
čiastočné kroky nasledovne. Tolerujte pritom absenciu úvodnej
zmienky o~tom, že stredy~$O_1$ a~$O_2$ ležia vo vnútri polroviny
$BCP$.

Za postup podobný prvému riešeniu dajte
1 bod za dôkaz rovnoramennosti trojuholníka $PQC$ a~2~body za
dôkaz rovnoramennosti trojuholníka $BQA$. Ak riešiteľ využíva tieto
alebo iné z~nich vyplývajúce poznatky bez dôkazov, dajte
najviac 3 body z 5 možných bodov.

Za postup podobný druhému riešeniu dajte:
\item{$\triangleright$} 1 bod za zavedenie priesečníka $X$ s~vyjadreným úmyslom dokázať rovnosť $O_1=X$.
\item{$\triangleright$} 1 bod za dôkaz, že zavedený bod $X$ je stredom oblúka $BP$.
\item{$\triangleright$} 2 body za dôkaz, že úsečky $AX$ a $BQ$ sú navzájom kolmé.

Pokiaľ riešiteľ využíva tieto alebo iné z nich vyplývajúce poznatky
bez dôkazov, dajte najviac 3~body z~5~možných bodov.
\endschema
}

{%%%%%   A-II-4
Označme $x_1, x_2,\dots, x_{74}$ čísla zo zadania. Pre každé $i$
z~predpokladu $x_i\in\langle4,10\rangle$ zrejme
vyplýva $(x_i-4)(10-x_i)\geqq0$,
čo po roznásobení dáva $x_i^2\leqq 14x_i-40$.
Sčítaním týchto nerovností pre všetky $i=1,2,\ldots,74$
s~prihliadnutím k~zadanému súčtu $x_1+x_2+\ldots+x_{74}=356$
získame
$$
  x_1 ^2 + x_2 ^2 + \ldots + x_{74} ^2 \leq 14 (x_1 + x_2 + \dots + x_{74}) -
40 \cdot 74 = 14 \cdot 356 - 40 \cdot 74 = 2024.
$$
Tým je dokázaná nerovnosť $x_1 ^2+x_2 ^2+\ldots+x_{74} ^2\leqq2024$.
Rovnosť v~nej nastane práve vtedy, keď je každé číslo $x_i$ rovné štyrom či desiatim.
Ak ukážeme, že takáto skupina čísel $x_1, x_2,\dots, x_{74}\in\{4,10\}$
so súčtom rovným 356 existuje, budeme z~riešením hotoví.

Nech je hľadaná skupina 74 čísel zostavená z~$c$ štvoriek a $d$
desiatok. Neznáme $c$ a~$d$ tak majú spĺňať sústavu rovníc
$$\eqalign{
c+d &=74, \cr
4c+10d&= 356.
}$$
Jednoduchým výpočtom nájdeme \emph{jediné} riešenie $(c,d)=(64,10)$. Skupina
74~čísel zložená z~64~štvoriek a 10~desiatok tak zaručuje, že 2024 je
skutočne hľadaná najväčšia možná hodnota skúmaného súčtu.

\ineriesenie
Každú skupinu 74 čísel
$x_1,x_2,\dots,x_{74}\in\langle4,10\rangle$ so súčtom 356
nazveme \emph{prípustnou skupinou}. Označíme pre ňu ako $S$ súčet
$x_1^2+x_2^2+\ldots+x_{74}^2$ a ako $p$ počet tých indexov $i$,
pre ktoré platí $x_i=4$ alebo $x_i=10$. Záver, že najväčšia možná
hodnota $S$ existuje\fnote{Pojednáme o~tom
ešte v~poznámke za týmto riešením.}
a je rovná 2024, vyplynie z nasledujúcich dvoch tvrdení,
ako vysvetlíme ešte pred ich dôkazmi.

\smallskip
\item{1)} {\sl Prípustná skupina s~hodnotou $p\geqq73$ je (až na poradie čísel)
jediná. Pozostáva z~$10$~desiatok
a~$64$~štvoriek a~súčet~$S$ má pre ňu hodnotu $2024$.}
\item{2)} {\sl Pre každú prípustnú skupinu s~hodnotou $p<73$ existuje
prípustná skupina, ktorá má obe hodnoty $S$ a $p$ väčšie.}

\smallskip\noindent
Podajme teraz sľúbené vysvetlenie: Podľa 1) stačí dokázať
nerovnosť $S<2024$ pre ľubovoľnú prípustnú skupinu s~hodnotou
$p<73$. Ak na ňu uplatníme opakovane záver~2), celkom najviac
$(73-p)$-krát, dostaneme sa od východiskovej skupiny k~prípustnej skupine
s~hodnotou $p\geqq73$ a väčšou hodnotou $S$, ktorá je podľa~1)
rovná práve číslu~2024.

\smallskip
\emph{Dôkaz} 1). Nebudeme opakovať dôkaz toho, čo sme
zistili v~predchádzajúcom riešení: Prípustná skupina s~hodnotou $p=74$ je
jediná, je zložená z~10 desiatok a 64 štvoriek a~súčet~$S$ má pre
ňu hodnotu 2024. Tvrdenie 1) tak platí, pokiaľ neexistuje
žiadna prípustná skupina s~hodnotou $p=73$. Posledné dokážeme
sporom.

Pripusťme teda, že prípustná skupina s hodnotou $p=73$ existuje.
Taká skupina je teda zložená z~$d$ desiatok, $73-d$ štvoriek
a jedného čísla $x$, kde $4<x<10$. Z~rovnosti $10d+(73-d)\cdot4+x=356$
vyplýva $x=64-6d$. Platí teda
$$
4<64-6d<10,\quad\hbox{čiže}\quad 54<6d<60.
$$
To je už želaný spor, lebo žiadny násobok šiestich v intervale
$(54,60)$ neexistuje.\fnote{Alternatívny dôkaz:
Ak je 356 rovné súčtu 73 čísel, z ktorých každé je 4 alebo 10, a nejakého
reálneho čísla~$x$, potom zrejme $x$ je celé a modulo 6 vzhľadom
na $10\equiv4$ dostávame $356\equiv 73\cdot4+x$, t.\,j. $2\equiv 4+x$,
čiže $x\equiv 4$. Odtiaľ za predpokladu $x\in\langle4,10\rangle$ už vyplýva $x=4$
alebo $x=10$, teda prípad $p=73$ je naozaj vylúčený.}

\smallskip
\emph{Dôkaz} 2). Nech prípustná skupina $x_1,x_2,\dots,x_{74}$ má
hodnotu~$p$ menšiu ako~73. Aspoň dva členy tejto skupiny sa teda
nerovnajú ani 4 ani 10. Vyberme preto dva členy
$x_i\leqq x_j$ s~indexmi $i\ne j$, ktoré oba ležia
v~otvorenom intervale $(4,10)$. Položme $c=\min(x_i-4,10-x_j)>0$
a v~uvažovanej prípustnej skupine nahraďme člen~$x_i$ menším
číslom $x_i'=x_i-c$ a člen~$x_j$ väčším číslom $x_j'=x_j+c$.
Touto zmenou zostane určite zachovaný súčet 356 všetkých 74 čísel.
Navyše vďaka výberu $c$ nastáva v~platných nerovnostiach $c\leqq x_i-4$
a $c\leqq 10-x_j$ aspoň jedna rovnosť, teda oba
nové členy $x_i'$ a $x_j'$ ležia v~intervale $\langle4,10\rangle$ a
aspoň jeden z~nich je rovný 4 alebo 10.
Nová skupina čísel je teda prípustná a má oproti pôvodnej skupine
väčšiu hodnotu $p$ počtu všetkých zastúpených čísel 4 a 10.
Zostáva overiť, že väčšiu hodnotu má aj súčet~$S$.
Naozaj, súčet štvorcov dvoch pozmenených členov
sa zväčší vďaka nerovnostiam $c>0$ a $x_j-x_i\geqq0$ o~hodnotu
$$
(x_i-c)^2+(x_j+c)^2-\bigl(x_i^2 + x_j^2\bigr)=
2c(x_j - x_i)+2c^2>0.
$$
Tým je dôkaz tvrdenia 2) ukončený.

\poznamka
Keďže všetkých prípustných skupín 74 čísel je nekonečne
veľa, nie je existencia najväčšej možnej hodnoty $S$ samozrejmá, ani
keď ukážeme, že množina všetkých hodnôt $S$ je zhora ohraničená.
Bez dôkazu tejto existencie (možno k nemu využiť jeden výsledok
matematické analýzy, totiž Weierstrassovu vetu o spojitej funkcii
niekoľkých premenných, ktorá je definovaná na kompaktnej
množine) nemožno podané riešenie viesť zjednodušeným spôsobom,
pri ktorom sa zaoberáme iba otázkou, ako musí vyzerať každá
prípustná skupina, pre ktorú je súčet $S$ najväčší možný.

\schemaABC
Za úplné riešenie dajte 6 bodov. V~neúplných riešeniach oceňte čiastkové
kroky nasledovne:
\item{A0.} Správna odpoveď bez zdôvodnenia: 0 bodov.
\item{A1.} Dôkaz nerovnosti $x_1^2+x_2^2+\dots+x_{74}^2\leqq2024$: 5 bodov.
\item{A2.} Uvedenie vyhovujúceho príkladu, keď $x_1^2+x_2^2+\dots+x_{74}^2=2024$ (stačí aj uhádnutie): 1~bod.
\item{B1.} Dôkaz tvrdenia 1) : 2~body, po 1 bode za prípady $p=73$ a $p=74$.
\item{B2.} Dôkaz tvrdenia 2): 3 body.
\item{B3.} Zdôvodnenie, prečo pre každú prípustnú skupinu s~hodnotou $p<73$ sa nájde prípustná skupina s~väčším súčtom $S$ a hodnotou $p\geq73$: 4 body.

\noindent
Celkom potom dajte $\rm\max[A1+A2,B1+\max(B2,B3)]$ bodov. Za
riešenie, ktoré využíva nedokázanú existenciu najväčšieho možného
súčtu, dajte najviac 4 body.
\endschema
}

{%%%%% A-III-1
\def\D{{\mathop{\text D\null}\nolimits}}
\def\n{{\mathop{\text n\null}\nolimits}}
Bez ujmy na všeobecnosti predpokladajme, že platí napríklad
$$
\D(a,b)\cdot\n(b,c)=\left(\D(b,c)\cdot\n(c,a)\right)\cdot\left(\D(c,a)\cdot\n(a,b)\right)=s,
\tag1
$$
pričom $s$ je vhodné prirodzené číslo. Použitím známeho vzťahu
$\D(x,y)\cdot\n(x,y)=xy$, ktorý platí pre všetky prirodzené čísla $x$
a~$y$, dostaneme
$$\eqalign{
(abc)^2&=(ab)\cdot(bc)\cdot(ca)=
(\D(a,b)\cdot\n(a,b))\cdot(\D(b,c)\cdot\n(b,c))\cdot(\D(c,a)\cdot \n(c,a))=\cr
&=(\D(a,b)\cdot\n(b,c))\cdot
(\D(b,c)\cdot\n(c,a)\cdot\D(c,a)\cdot\n(a,b))=s\cdot s=s^2,
}$$
a teda po odmocnení $abc=s$. Teraz si všimneme, že
$$
\D(a,b)\cdot\n(b,c)=s=a\cdot(bc)=a\cdot \D(b,c)\cdot\n(b,c),
$$
odkiaľ po vydelení $\n(b,c)$ dostávame $\D(a,b)=a\cdot \D(b,c)$.
Z poslednej rovnosti už vidíme, že $a\mid \D(a,b)\mid b$,
a teda $b$ je násobkom $a$,
ako sme potrebovali dokázať.

\ineriesenie
Označme $v_p(x)$ exponent prvočísla $p$ v~prvočíselnom
rozklade daného prirodzeného čísla $x$. Predpokladajme opäť, že
platí rovnosť \thetag1 ako v~prvom riešení.
Potom pre každé prvočíslo $p$ máme
$$
v_p\left(\D(a,b)\cdot\n(b,c)\right)=v_p\left(\D(b,c)\cdot
\n(c,a)\cdot\D(c,a)\cdot\n(a,b)\right).
$$
Pokiaľ pri pevnom $p$ položíme
$\alpha=v_p(a)$, $\beta=v_p(b)$ a $\gamma=v_p(c)$, môžeme zapísanú
rovnosť opakovaným použitím pravidla $v_p(xy)=v_p(x)+v_p(y)$
a známych vzorcov
$$
v_p(\D(x, y))=\min(v_p(x), v_p(y)),\quad
v_p(\n(x,y)) =\max(v_p(x), v_p(y))
$$
prepísať ako
$$
\min(\alpha,\beta)+\max(\beta,\gamma)
=\min(\beta,\gamma)+\max(\gamma,\alpha)+\min(\gamma,\alpha)+\max(\alpha,\beta).
$$
To sa dá vzhľadom na $\max(\gamma,\alpha)+\min(\gamma,\alpha)=\alpha+\gamma$ ešte
zjednodušiť na
$$
\min(\alpha,\beta)+\max(\beta,\gamma)=
\min(\beta,\gamma)+(\alpha+\gamma)+\max(\alpha,\beta).
$$
Teraz si všimneme, že ak by platilo $\alpha>\beta$, získali by sme po
sčítaní zrejmých nerovností $\min(\alpha,\beta)<\max(\alpha,\beta)$,
$\max(\beta,\gamma)<\alpha+\gamma$ a $0\leqq\min(\beta,\gamma)$
spor s odvodenou rovnosťou. Preto platí $\alpha\leq\beta$.

Ukázali sme, že pre ľubovoľné prvočíslo $p$ platí
$v_p(a) \leq v_p(b)$, odkiaľ už vyplýva, že $b$ je násobkom $a$.
}

{%%%%% A-III-2
Zo zadaných rovností striedavých uhlov vyplýva $PC \parallel AD$ a
$PD \parallel BC$.
Keďže priamky $PC$ a $PD$ sú rôzne, a teda rôznobežné, platí,
že aj priamky $AD$ a~$BC$ sú rôznobežné. Označme~$X$ ich priesečník.
Štvoruholník $PCXD$ je potom zrejme rovnobežník, a preto platí aj
$|\uhol CPD|=|\uhol CXD|$ (\obr.).
\inspdf{a73iv_21.pdf}%

V štvoruholníku $AXCP$ platí $AX\parallel CP$
a~$|\uhol PAX|=|\uhol CXA|$, jedná sa preto o~rovnoramenný lichobežník.
Os jeho základne $CP$, na ktorej leží aj stred $O$ zo zadania, je teda
zhodná s~osou druhej základne $AX$. Platí tak $|OA|=|OX|$.
Analogicky použitím rovnoramenného lichobežníka $BXDP$ získame
rovnosť $|OB|=|OX|$. Dokopy dostávame $|OA| = |OX| = |OB|$,
a úloha je tak vyriešená.
}

{%%%%% A-III-3
Ukážeme, že hľadané najväčšie číslo $n$ je rovné 99.
V~priebehu riešenia budeme typy tetramín z~\obrr1{} nazývať
postupne O, T, L a I.

V~prvej časti riešenia uvedieme príklad sady 100 tetramín, ktorú
nie je možné do tabuľky $20\times20$ vyhovujúcim spôsobom umiestniť. Bude
to potom zrejme znamenať, že žiadne prirodzené číslo $n\geqq100$
požadovanú vlastnosť nemá.

Zoberme sadu 100 tetramín zloženú z~99 tetramín typu O~a jedného
tetramina typu~T. Ak zafarbíme tabuľku $20\times20$
ako šachovnicu, bude v~nej rovnaký počet bielych aj čiernych políčok,
konkrétne $20^2:2=200$.
Ak potom umiestnime do tabuľky najprv akokoľvek našich
99~tetramín typu~O, bude nimi pokrytých 198~bielych a 198~čiernych políčok,
lebo každé z~nich pokryje 2~biele a 2~čierne políčka. Bez pokrytia
tak zostanú niektoré 2~biele a~niektoré 2~čierne políčka, ktoré však nemožno pokryť
zvyšným tetraminom typu~T~- jeho umiestnením totiž vždy pokryjeme
3~políčka rovnakej farby. Tým je prvá časť riešenia hotová.

\smallskip
V druhej časti riešenia dokážeme, že číslo $n=99$ požadovanú
vlastnosť má. Popíšeme totiž postup, ako je možné ľubovoľnú sadu
99~tetramín do tabuľky $20\times20$ vyhovujúcim spôsobom umiestniť.
Využijeme na to \obr.
\inspdf{a73iv_31.pdf}%

Z~\obrr1{} vľavo vidíme, ako štyrmi tetraminami typu T vyplniť
štvorec $4\times 4$ a ako každými dvoma tetraminami
jedného z troch typov O, L, I~vyplniť obdĺžnik~$4\times 2$.
Všetky také štvorce a všetky také obdĺžniky,
ktoré môžeme z~danej sady 99 tetramín zostaviť,
budeme potom do tabuľky postupne ukladať (najprv všetky štvorce,
až potom obdĺžniky) po vrstvách výšky štyroch riadkov.
Keďže sú riadky tvorené 20 políčkami, čo je násobok štyroch,
bude každá nová vrstva započatá, len ak je predchádzajúca vrstva
vyplnená bezo zvyšku. Uvedomme si, že v~okamihu, keď
už žiadny obdĺžnik $4\times2$ na uloženie do tabuľky
nemáme k~dispozícii, platia nasledujúce skutočnosti.

\smallskip
\item{$\bullet$} Sada všetkých zvyšných, \tj. doposiaľ neumiestnených tetramín,
ktorú označíme $\Cal Z$, je podmnožinou sady
$1\times$O, $1\times$L, $1\times$I, $3\times$T.
Pre počet $k$ tetramín v~sade $\Cal Z$ tak platí $k\leqq6$.
\item{$\bullet$} Keďže sme $99-k$ tetraminami v~uložených
štvorcoch a obdĺžnikoch pokryli celkom
$4(99-k)=400-4(k+1)$ políčok tabuľky, nepokryté tak
zostalo $4(k+1)$ jej políčok.
\item{$\bullet$} Obdĺžnikmi a štvorcami sme postupne pokrývali vrstvy tabuľky
o~80 políčkach bezo zvyškov, preto doposiaľ nepokryté políčka
o~počte $4(k+1)$, ktorý je vďaka $k\leqq6$ menší ako 80,
tvoria časť poslednej vrstvy tabuľky, teda jej podtabuľku
s~rozmermi $4\times(k+1)$.
\item{$\bullet$} Počet $99-k$ tetramín v~umiestnených štvorcoch a obdĺžnikoch je nutne párne
číslo, teda číslo $k\leqq6$ je nepárne, a preto $k\in\{1,3,5\}$.

\smallskip\noindent
Zostáva nám tak umiestniť $k$ tetramín tvoriacich sadu $\Cal Z$ do tabuľky
$4\times(k+1)$. Rozlíšime tri prípady podľa hodnoty
$k\in\{1,3,5\}$, pritom pre $k=3$ a $k=5$ využijeme
rozdelenie tabuľky $4\times(k+1)$ so zastúpením útvarov A, B a
C, ktoré sú vykreslené na \obrr1{} vpravo. Uplatníme pritom tieto
zrejmé poznatky: tetramino ktoréhokoľvek typu možno umiestniť do
tabuľky $4\times2$ aj do útvaru A,
zatiaľ čo do útvarov B a C možno umiestniť tetramino každého
z~typov O, L, T.

{\it Prípad $k=1$}. Do príslušnej tabuľky $4\times 2$ umiestnime
to tetramino, ktoré tvorí celú sadu $\Cal Z$.

{\it Prípad $k=3$.} Do útvaru A umiestnime jedno tetramino
zo sady $\Cal Z$, pritom dáme prednosť typu~I, ak je zastúpený.
Zvyšné dve tetraminá zo sady $\Cal Z$ umiestnime
po jednom do útvarov B a C, nech už sú ktoréhokoľvek typu O, L, T.

{\it Prípad $k=5$}. Súprava $\Cal Z$ potom obsahuje dve alebo tri
tetraminá typu T. Dve z~nich umiestnime podľa obrázka. Ostatné tri
tetraminá zo sady $\Cal Z$ umiestnime po jednom do útvarov~A, B a~C,
pritom do útvaru A typ I, ak je v~$\Cal Z$ zastúpený.

Tým je aj druhá časť riešenia hotová.
}

{%%%%% A-III-4
Dokážeme, že hľadané najväčšie $n$ je rovné 1.

V~prvej časti riešenia ukážeme, že jeden vyhovujúci pár je možné vždy utvoriť.
Podľa zadania sú počty dievčat páčiacich sa jednotlivým
chlapcom všetky čísla od 1 do 10 (v~nejakom poradí).
To isté platí aj pre počty chlapcov páčiacich sa jednotlivým dievčatám.
Jednému chlapcovi sa tak páči všetkých 10 dievčat a jednému dievčaťu všetkých 10 chlapcov, takže tento chlapec a toto dievča tvoria vyhovujúci pár.

V druhej časti riešenia uvedieme príklad párty
spĺňajúci podmienky zadania, pri ktorom neexistujú
dva disjunktné páry s obojstranným
zaľúbením. Pre jeho konštrukciu rozostavme chlapcov
do horného radu a dievčatá pod nich do spodného radu ako na \obr{}.
\inspdf{a73iv_41.pdf}%

Predpokladajme, že platí to, čo je na obrázku vyznačené šípkami
pre šiesteho chlapca aj šieste dievča zľava:
Každému chlapcovi sa páči práve dievča pod ním
spolu so všetkými dievčatami napravo od nej, zatiaľ čo každému dievčaťu sa páči
práve chlapec prvý zľava spolu so všetkými chlapcami
napravo od chlapca nad týmto dievčaťom.
Potom sa zrejme každému chlapcovi páči iný počet dievčat a každému dievčaťu iný počet
chlapcov, avšak vo všetkých (desiatich) pároch s obojstranným zaľúbením
je zastúpený jediný chlapec~-- ten prvý zľava. Preto žiadne dva
také páry nie sú disjunktné.
}

{%%%%% A-III-5
\def\ent#1{\left\lfloor #1\right\rfloor}%
Zaveďme výhodne funkciu $f(x)=3x-\ent{2x}-\ent{x}$.
Postupnosť zo zadania je potom určená členom $a_1$
a~pravidlom $a_{k+1} = f(a_k)$.
Preto zo zrejmých rovností $f(1)=f(0)=0$ a~$f(1/2) = 1/2$
vyplýva, že táto postupnosť je od nejakého člena konštantná,
ak je v~nej zastúpené číslo $1$ alebo $1/2$.

Keďže na intervale $0<x<1/2$ platí $f(x) = 3x$,
je zastúpenie čísla $1$, resp. $1/2$ v~našej postupnosti
čísel $a_k$ zaručené, pokiaľ platí $a_1=1/3^\alpha$,
resp. $a_1=1/(2\cdot 3^\alpha)$, pričom $\alpha$ označuje (tu aj ďalej)
nezáporné celé číslo. Vidíme tak, že medzi hľadané prirodzené čísla $n$
patria všetky hodnoty $n=3^\alpha$ a $n=2\cdot 3^\alpha$.
Vo zvyšku riešenia ukážeme, že iné vyhovujúce~$n$ neexistujú.

Na to bude stačiť, keď pre každé vyhovujúce $n$ nájdeme $\alpha$
s~vlastnosťou $n\mid 2\cdot 3^\alpha$. Zafixujme teda jedno
vyhovujúce $n$ a okrem postupnosti $(a_k)$ s~prvým členom $a_1=1/n$
uvážme novú postupnosť $(b_k)$ s členmi $b_k=n\cdot a_k$.
Potom $b_1=1$ a rekurentný vzťah $a_{k+1}=f(a_k)$ po
vynásobení číslom $n$ prejde
na
$$
b_{k+1}=3b_k-n\ent{\frac{2b_k}{n}}-n\ent{\frac{b_k}{n}}.
$$
Odtiaľ indukciou vyplýva, že každé číslo $b_k$ je celé a navyše
platí kongruencia (tu aj v~ďalšom odseku modulo $n$)
$b_{k+1}\equiv 3b_k$, odkiaľ (opäť indukciou) $b_{k}\equiv3^{k-1}$
pre každé~$k$.

Keďže $n$ je vyhovujúce, je postupnosť $(a_k)$
od nejakého člena konštantná, takže to isté platí
aj pre postupnosť $(b_k)$. Pre isté $\alpha$ tak máme
$b_{\alpha+2}=b_{\alpha+1}$, čo spolu s $b_{\alpha+2}\equiv 3b_{\alpha+1}$ vedie
k~$2b_{\alpha+1}\equiv0$. Dokopy s~$b_{\alpha+1}\equiv3^{\alpha}$ už
dostávame $2\cdot3^{\alpha}\equiv0$, čiže $n\mid2\cdot3^{\alpha}$, ako
sme sľúbili ukázať.

\zaver
Hľadané čísla $n$ sú práve tvarov $n=3^\alpha$ a
$n=2\cdot 3^\alpha$, kde $\alpha$ je nezáporné celé číslo.

\def\zl#1{\left\{#1\right\}}
\def\ce#1{\left\lfloor#1\right\rfloor}
\def\Ekv{\Leftrightarrow}
\ineriesenie
Uvažujme znovu funkciu $f(t)=3t-\ce{2t}-\ce{t}$. Nájdeme dokonca všetky
\emph{reálne} čísla $a_1$, pre ktoré je postupnosť
$(a_k)_{k=1}^{\infty}$, v ktorej platí
$a_{k+1}=f(a_k)$ pre každé $k\geqq1$, od istého člena konštantná.
To zrejme nastane práve vtedy, keď bude platiť $f(a_k)=a_k$ pre niektoré
$k\geq1$, práve vtedy potom totiž budeme mať
$a_i=a_k$ pre každé~$i\geqq k$.

Pre analýzu podmienky $f(a_k)=a_k$ odvodíme najprv
potrebné vlastnosti funkcie~$f$. Označme za tým učelom
$\zl{t}=t-\ce{t}$.\fnote{Hodnota $\zl{t}$ sa bežne nazýva
\emph{zlomková časť} daného reálneho čísla $t$.}
Predpis pre funkciu $f$ upravíme takto:
$$
f(t)=3t-\ce{2t}-\ce{t}=(2t-\ce{2t})+(t-\ce{t})=\zl{2t}+\zl{t} .
\tag1
$$
Vidíme, že funkcia $f$ má periódu 1, takže pre ľubovoľné
$t\in\Bbb R$ a $m\in\Bbb Z$ platí $f(t)=f(t+m)$. Odtiaľ voľbou
$t=\zl{2x}+\zl{x}$ a $m=\ce{2x}+\ce{x}$ dostaneme z~\thetag1
pre hodnotu $f(f(x))$ s~ľubovoľným $x\in\Bbb R$ vyjadrenie
$$
f(f(x))=f(\zl{2x}+\zl{x})
=f(\zl{2x}+\zl{x}+\ce{2x}+\ce{x})=f(2x+x)=f(3x).
\tag2
$$

Zistime teraz, pre ktoré $t\in\Bbb R$ platí rovnosť $f(t)=t$.
Vďaka \thetag1 máme
$$
f(t)=t\Ekv \zl{2t}+\zl{t}=t \Ekv \zl{2t}=t-\zl{t}\Ekv
\zl{2t}=\ce{t}.
$$
Keďže $\zl{2t}\in\langle0,1)$ a $\ce{t}\in\Bbb Z$,
rovnosť $f(t)=t$ nastane práve vtedy, keď bude platiť
$\zl{2t}=\ce{t}=0$. To zrejme spĺňajú jedine čísla $t=0$ a
$t=1/2$.

Vzhľadom na posledný záver ešte určíme tie čísla $t\in\Bbb R$, pre ktoré
je hodnota~$f(t)$ rovná 0 alebo $1/2$.
Všimnime si najskôr, že podľa \thetag1 platí $f(t)=3t$ pre každé
$t\in\langle0,1/2)$ a $f(t)=3t-1$ pre každé $t\in\langle1/2,1)$.
Odtiaľ pre všeobecné $t\in\Bbb R$ vzhľadom na $f(t)=f(\zl{t})$ a
$\zl{t}\in\langle0,1)$ zrejme vyplývajú ekvivalencie
$$
f(t)=0\Ekv \zl{t}=0\quad\hbox{a}\quad
f(t)=1/2\Ekv \zl{t}=1/6\lor\zl{t}=1/2.
\tag3
$$

Vďaka vykonaným úvahám o~funkcii $f$ rovnosť $f(a_k)=a_k$ zo~záveru
prvého odseku nastane práve vtedy, keď $a_k$ (a teda aj
$a_{k+1}$) bude jedno z~čísel 0 alebo~$1/2$. Využijeme rovnosti~\thetag2,
podľa ktorých platí $f(a_k)=f(3^{k-1}a_1)$, takže hľadáme práve tie~$a_1$,
pre ktoré je hodnota $f(3^{k-1}a_1)$ rovná 0 alebo $1/2$.
To je podľa \thetag3 ekvivalentné s~tým, že hodnota $\zl{3^{k-1}a_1}$
je jedno z~čísel 0, $1/6$ alebo $1/2$. Pri označení
$m=\ce{3^{k-1}a_1}$ tak všetky hľadané~$a_1$ sú práve tie
čísla, ktoré majú pre niektoré celé $k\geq1$ a $m$ jedno
z~vyjadrení
$$
a_1=\frac{m}{3^{k-1}},\quad
a_1=\frac{6m+1}{2\cdot3^{k}},\quad
a_1=\frac{2m+1}{2\cdot3^{k-1}}.
$$
Dodajme ešte, že množiny čísel určené prvou a treťou rovnosťou
sú zrejme disjunktné. Druhú rovnosť je možné v odpovedi vynechať,
lebo určuje len čísla, ktoré sú tiež určené treťou rovnosťou,
ako vyplýva z úpravy
$$
\frac{6m+1}{2\cdot3^k}=\frac{2\cdot3m+1}{2\cdot3^{(k+1)-1}}.
$$
}

{%%%%% A-III-6
Ukážeme, že hľadaný trojuholník je jediný, má strany dĺžok 21, 28, 35
a~dotyčné dve kružnice majú polomer rovný prvočíslu 5.
\let\ro=\varrho

V~ľubovoľnom pravouhlom trojuholníku $ABC$ s~preponou $AB$ označíme
$a=|BC|$, $b=|AC|$ a $c=|AB|$. Zrejme existujú dve zhodné kružnice
so všetkými dotykmi popísanými v~zadaní. Označme ich $k_1(S_1,r)$ a
$k_2(S_2,r)$ tak, aby sa $k_1$ dotýkala $AC$ (a teda $k_2$ zase
$BC$) a body ich dotyku s~$AB$ označme $T_1$, resp. $T_2$
ako
na \obr{}. Uvažujme ešte kružnicu $k(S,\varrho)$ vpísanú
trojuholníku $ABC$ a označme $D$ bod dotyku $k$ s~$AB$.
\inspdf{a73iv_61.pdf}%

V~prvej časti riešenia ukážeme, že polomer $r$ kružníc $k_1$ a $k_2$
je určený rovnosťou
$$
r=\frac{c(a+b-c)}{2(a+b)}.
\tag1
$$
V rovnoľahlosti so stredom v~bode $A$ s~koeficientom $r/\ro$,
ktorá zobrazí $k$ na $k_1$, prejde bod~$D$ do bodu $T_1$.
Platí tak $|AT_1|=|AD|\cdot r/\ro$. Analogicky platí
aj rovnosť $|BT_2|=|BD|\cdot r/\ro$.
Keďže z~obdĺžnika $T_1T_2S_2S_1$ vyplýva $|T_1T_2|=|S_1S_2|=2r$,
po dosadení do pravej strany zrejmej rovnosti
$c=|AT_1|+|T_1T_2|+|BT_2|$ s prihliadnutím na $|AD|+|BD|=c$
dostaneme
$$
c=|AD|\cdot\frac{r}{\ro}+2r+ |BD|\cdot\frac{r}{\ro}=
2r+(AD| + |BD|)\cdot\frac{r}{\ro}=
2r+\frac{cr}{\ro}.
$$
Odtiaľ pre polomer $r$ vychádza vyjadrenie
$$
r =\frac{c\ro}{c+2\ro}.
$$
Po dosadení známeho vzorca $\ro=(a+b-c)/2$ a jednoduchej úprave
už získame avizovanú rovnosť \thetag1.

\smallskip
Predpokladajme teraz, že dĺžky $a$, $b$, $c$, $r$ sú vyjadrené
celými číslami a položme $k=\D(a,b,c)$. Potom $a=ka_1$, $b=kb_1$ a
$c=kc_1$, pričom vďaka Pytagorovej rovnosti $a_1^2+b_1^2=c_1^2$
sú čísla $a_1$, $b_1$, $c_1$ dokonca po dvoch nesúdeliteľné a
prvé dve z~ich majú rôznu paritu\fnote{Súčet štvorcov
dvoch nepárnych čísel je párne číslo, ktoré nie je deliteľné štyrmi,
teda to nie je štvorec.}. Tretie číslo $c$ je teda nepárne,
a preto číslo $a_1+b_1-c_1$ je párne. Dosadením
$a=ka_1$, $b=kb_1$ a $c=kc_1$ do \thetag1
a následnom krátení číslom $k$ obdržíme
$$
r=\frac{kc_1(a_1+b_1-c_1)}{2(a_1+b_1)}=
\frac{k}{a_1+b_1}\cdot c_1 \cdot\frac{a_1+b_1-c_1}{2}.
\tag2
$$
Dokážme, že nielen druhý, ale aj prvý zlomok na pravej strane \thetag2
má celočíselnú hodnotu. Ak totiž
nejaké prvočíslo $p$ delí nepárne číslo $a_1+b_1$, tak
nemôže platiť $p \mid c_1$ (a~teda ani $p\mid a_1+b_1-c_1$),
inak by vďaka rovnosti $(a_1+b_1)^2=c_1^2+2a_1b_1$
platilo $p \mid a_1b_1$, čo by spolu s~$p\mid a_1+b_1$ znamenalo
$p \mid a_1$ a $p \mid b_1$, teda spor s~nesúdeliteľnosťou čísel
$a_1, b_1$. Číslo $a_1+b_1$ je tak nesúdeliteľné ako
s~$c_1$, tak s~$a_1+b_1-c_1$, a preto vďaka celočíselnosti $r$
je v~jeho vyjadrení \thetag2 aj prvý zlomok celočíselný.

Až teraz doplňme predpoklad, že $r$ je prvočíslo.
To je podľa \thetag2 súčinom troch prirodzených čísel. Vzhľadom na
$c_1>a_1\geq1$ tak nutne platí $c_1 = r$ a dva zlomky na pravej
strane \thetag2 majú hodnotu 1.
Platia teda rovnosti $k=a_1+b_1$ a $a_1+b_1-c_1=2$, čiže $k=a_1+b_1=c_1+2=r+2$. Z~toho vyplýva
$$
2a_1b_1=(a_1+b_1)^2-c_1^2=(r+2)^2-r^2=4r+4.
$$
Po vydelení dvoma dostávame
$$
a_1b_1=2r+2=2(a_1+b_1-2)+2=2a_1+2b_1-2.
$$
Poslednú rovnosť $a_1b_1=2a_1+2b_1-2$ je možné prepísať na tvar
$(a_1-2)(b_1-2)=2$. Odtiaľ ľahko vyplýva, že $\{a_1,b_1\}=\{3,4\}$,
a teda $c_1=r=a_1+b_1-2=5$ a $k=r+2=7$. Pre pravouhlý trojuholník so
stranami $7\cdot3$, $7\cdot4$ a $7\cdot5$ potom naozaj platí
$r=5$. Tým je tvrdenie z~úvodnej vety riešenia dokázané.
}

{%%%%%   B-S-1
Dokážeme, že nutne platí $b=c$, teda medzi číslami $a$, $b$, $c$
môžu byť najviac dve rôzne. Počet 2 dosiahneme napríklad v~prípade
čísel $a=2$ a~$b=c=1$, ktoré zrejme majú požadovanú vlastnosť.

Predpokladajme teda, že prirodzené čísla $a$, $b$, $c$ vyhovujú
zadaniu úlohy. To je možné vyjadriť tromi podmienkami:

\smallskip
\item{1.} $a\mid b+c$, čiže $b+c=ma$ pre nejaké prirodzené číslo $m$,
\item{2.} $b\mid a+b$, čiže $b\mid a$,
\item{3.} $c\mid a+c$, čiže $c\mid a$.

\smallskip\noindent
Z~$b\mid a$ vyplýva, že číslo $b$ delí oba členy na pravej strane
upravenej rovnosti $c=ma-b$, a teda aj $b\mid c$.
Podobne z~$c\mid a$ a~$b=ma-c$ vyplýva $c\mid b$. Pre
prirodzené čísla $b$,~$c$ vzťahy $b\mid c$ a~$c\mid b$
už znamenajú, že $b=c$, ako sme sľúbili dokázať.

\ineriesenie
Namiesto rovnosti $b=c$ z~prvého riešenia
dokážeme len, že prirodzené čísla $a$, $b$ a~$c$, pre ktoré platí
$a\mid b+c$, $b\mid a$ a~$c\mid a$, nemôžu byť navzájom rôzne.

Predpokladajme teda, že platí $a\ne b$ a~$a\ne c$. Potom zo vzťahov
$b\mid a$ a~$c\mid a$ vyplývajú nerovnosti $a\geq 2b$ a~$a\geq 2c$.
Ich sčítaním dostaneme $a+a\geq 2b+2c$, čiže $a\geq b+c$, kde vďaka vzťahu
$a\mid b+c$ musí nastať rovnosť $a=b+c$. Musia preto tiež platiť
obe rovnosti $a=2b$ a~$a=2c$ (inak by aspoň jedna
z~nerovností $a\geq 2b$, $a\geq 2c$ bola ostrá a~platilo by potom $a>b+c$),
z ktorých však vyplýva $b=c$. Preto
všetky tri nerovnosti $a\ne b$, $a\ne c$ a~$b\ne c$ nemôžu
platiť súčasne.

\poznamka
Aj keď to zadanie úlohy nevyžaduje, ukážeme (dokonca dvoma spôsobmi),
že všetky vyhovujúce trojice $(a,b,c)$ sú tvaru $(b,b,b)$
a~$(2b,b,b)$, kde $b$ je ľubovoľné prirodzené číslo.

Pri prvom postupe využijeme rovnosť $b=c$, ktorú sme dokázali
v~prvom riešení. Vďaka nej sa podmienky $b+c=ma$, $b\mid a$ a~$c\mid a$
zredukujú na vzťahy $2b=ma$ a~$b\mid a$. Keďže z~$b\mid a$ vyplýva $2b\leq 2a$,
v~rovnosti $2b=ma$ musí byť $m=1$ alebo $m=2$. Trojica
$(a,b,c)$ sa podľa toho rovná $(2b,b,b)$, resp. $(b,b,b)$.

Pri druhom postupe vyjadríme podmienky $a\mid b+c$, $b\mid a$ a~$c\mid a$
vzťahmi
$$
b+c\in\{a,2a,3a,4a,\dots\}\quad\hbox{a}\quad
b,c\in\bigl\{a,\tfrac12a,\tfrac13a,\tfrac14a,\dots \bigr\}.
$$
Z druhého vzťahu vyplýva $b+c\leqq2a$, teda podľa prvého vzťahu buď platí
$b+c=a$, a~to práve
keď $b=c=\frac12a$,
alebo platí $b+c=2a$, a~to práve keď $b=c=a$.
Tým je potrebný záver dokázaný.

\schemaABC
Za úplné riešenie dajte 6 bodov, z toho 1 bod za správnu odpoveď
a~5 bodov za dôkaz tvrdenia, že $a$, $b$, $c$ nemôžu byť tri rôzne
čísla. Bod za správnu odpoveď však priznajte, len keď je k~nej
uvedená aspoň jedna vyhovujúca trojica $(a,b,c)$,
napríklad $(2,1,1)$ alebo $(2b,b,b)$. Z~5~bodov za dôkaz dajte
1~bod za zápis (napríklad aj slovný) všetkých troch deliteľností $a\mid b+c$, $b\mid a$,
$c\mid a$ a~ďalší 1 bod za aspoň jednu z~nerovností
$a\leq b+c$, $b\leq a$, $c\leq a$. Za všetky tri nerovnosti
a~výsledok $b+c\leq 2a$ sčítania posledných dvoch z~nich dajte 3 body
(pozri druhý postup z~poznámky). Pri postupe podobnom tomu z~iného
riešenia dajte 2~body za aspoň jednu z~nerovností $a\geqq 2b$ alebo
$a\geqq 2c$ a~ďalší 1~bod za ich sčítanie.
\endschema
}

{%%%%%   B-S-2
\let\phi=\varphi
Popíšeme niekoľko postupov riešenia oboch častí a) aj b).
Bez komentára v~nich budeme využívať známe
rovnosti $|AS|=|BS|=|CS|=|DS|$. Zdôraznime ešte, že
podmienka $|AB|>|BC|$ zo zadania úlohy zaručuje,
že bod $E$ leží na predĺžení uhlopriečky~$AC$ za bod
$C$.

\smallskip\noindent
{\it Časť a), postup $1$.}
Bod~$C$ leží na prepone $SE$ pravouhlého trojuholníka $SBE$.
Súčasne leží na osi jeho odvesny $BE$, pretože $|CB|=|CE|$ podľa
zadania. Z~Tálesovej vety teda vyplýva, že kružnica opísaná tomuto
trojuholníku má stred práve v~bode~$C$. Platia preto rovnosti
$|CB|=|CE|=|CS|=|BS|$,
odkiaľ vyplýva, že trojuholník~$BSC$ je rovnostranný. Hľadaná veľkosť
uhla $BSC$ je preto $60^\circ$.

\smallskip\noindent
{\it Časť a), postup $2$.}
Uhly~$EBS$ a~$CBA$ sú pravé, a~teda zhodné, a~uhol $CBS$ je ich
spoločnou časťou, a~preto aj uhly $EBC$ a~$SBA$ sú zhodné.
Z~rovnoramenných trojuholníkov~$BEC$ a~$ABS$ však vyplýva
$|\angle EBC|=|\angle CEB|$ a~$|\angle SBA|=|\angle BAS|$.
Platí teda $|\angle CEB|=|\angle BAS|$, čiže
$|\angle AEB|=|\angle BAE|$, a~preto je aj trojuholník $EAB$
rovnoramenný a~platí $|BE|=|AB|$. Rovnoramenné trojuholníky $BEC$ a~$ABS$
sú teda podľa vety $usu$ zhodné. Platí preto $|BC|=|BS|=|CS|$,
trojuholník $BSC$ je teda rovnostranný, odkiaľ $|\angle BSC|=60^\circ$.

\smallskip\noindent
{\it Časť a), postup $3$.}
Označme $\phi$ veľkosť uhlov pri základni $BE$
rovnoramenného trojuholníka $BEC$. Vzhľadom na $|\angle EBS|=90\st$ je potom $90\st-\phi$ veľkosť uhlov pri základni~$BC$
rovnoramenného trojuholníka $BCS$, a~preto jeho tretí uhol~$BSC$
má veľkosť~$2\phi$. Trojuholník $SBE$ tak má vnútorné uhly
veľkostí $90\st$, $\phi$ a~$2\phi$, platí teda
$\phi+2\phi=90\st$, odkiaľ $\phi=30\st$,
a~preto $|\angle BSC|=2\phi=60^\circ$.
\inspsc{b73ii.21}{.8333}%

\smallskip\noindent
{\it Časť b), postup $1$.}
Uhol $FSD$ je rovnako ako uhol $EBD$
pravý, pretože $FS\parallel EB$ podľa zadania.
Z~riešenia časti a) vieme,
že $BSC$ je rovnostranný trojuholník. Rovnoramenný
trojuholník $CDS$ má preto vnútorné uhly veľkostí
$30\st$, $30\st$ a~$120\st$. Odtiaľ pre vnútorné uhly
trojuholníka $CFS$ vychádza
$$
|\uhol FCS|=|\uhol DCS|=30\st\quad\hbox{a}\quad
|\uhol CSF|=|\uhol CSD|-|\uhol FSD|=120\st-90\st=30\st,
$$
takže $CFS$ je rovnoramenný trojuholník a~platí $|SF|=|CF|$.
Namiesto rovnosti $|DF|=2\,|CF|$
teda stačí dokázať rovnosť $|DF|=2\,|SF|$.
Tá však ako je známe vyplýva z~pravouhlého trojuholníka $FDS$,
lebo~$|\uhol FDS|=30\st$.\fnote{Vyplýva to z~pozorovania,
že rovnostranný trojuholník je svojou výškou rozdelený na dva zhodné trojuholníky s uhlami
$30\st$, $60\st$, $90\st$. Je možné tiež
priamo využiť z~toho vyplývajúcu rovnosť~$\sin 30\st=\frac12$.}

\smallskip\noindent
{\it Časť b), postup $2$.}
Ukážme, že pravouhlý trojuholník $FDS$ je možné využiť aj inak.
Označme $d=|CS|=|BS|=|DS|$. Z~riešenia časti a) vieme, že aj $|BC|=d$,
čo spolu s~$|DB|=2d$ vedie podľa~Pytagorovej vety k~rovnosti
$|DC|=d\sqrt{3}$. Z~podobnosti pravouhlých trojuholníkov $FDS$ a~$BDC$
máme $|DF|/|DS|=|DB|/|DC|$, odkiaľ pre dĺžku úsečky $DF$
vychádza
$$
|DF|=\frac{|DB|\cdot|DS|}{|DC|}=
\frac{2d\cdot d}{d\sqrt3}=\frac23\cdot d\sqrt3=\frac23\,|DC|.
$$
To už zrejme znamená, že platí $|DF|=2\,|CF|$, ako sme mali dokázať.

\smallskip\noindent
{\it Časť b), postup $3$.}
Označme~$M$ priesečník priamok $CD$ a~$BE$.
Keďže~$S$ je stred úsečky $BD$ a~$SF\parallel BM$, je úsečka
$SF$ stredná priečka trojuholníka $BMD$, teda platí $|DF|=|FM|$. Našou
úlohou je preto ukázať, že $C$ je stred úsečky $FM$. Na to stačí
overiť, že sú zhodné trojuholníky $CFS$ a~$CME$. Tie však majú zhodné
vrcholové uhly pri spoločnom vrchole~$C$ a~rovnako zhodné
striedavé uhly pri vrcholoch $F$ a~$M$; napokon sú zhodné
aj ich strany $CS$ a~$CE$, ako sme ukázali v~riešeniach časti~a).
Podľa vety $usu$ sú teda trojuholníky $CFS$ a~$CME$ naozaj zhodné.
\inspsc{b73ii.22}{.8333}%

\smallskip\noindent
{\it Časť b), postup $4$.}
Označme $G$ stred úsečky $DE$. Potom $SG$ je stredná priečka trojuholníka
$BDE$ a teda okrem $FS\parallel BE$ platí aj $SG\parallel BE$.
Preto bod $F$ leží na priečke~$SG$, ktorá je zároveň
ťažnicou trojuholníka $SED$. Aj úsečka $DC$ je jeho ťažnicou,
lebo $|CS|=|CE|$ podľa riešenia časti~a). Priesečník $F$ týchto dvoch
ťažníc je teda ťažiskom trojuholníka $SED$, a preto platí
$|DF|=2\,|CF|$, ako sme mali dokázať.
\inspsc{b73ii.23}{.8333}%

\smallskip\noindent
{\it Časť b), postup $5$.}
Označme $H$ priesečník priamky $SF$ so stranou $AB$.
Zo stredovej súmernosti obdĺžnika $ABCD$ a~z~podobnosti
trojuholníkov $AHS$ a~$ABE$ (veta $uu$) vyplýva
$$
\frac{|DF|}{|CF|}=\frac{|BH|}{|AH|}=\frac{|BA|}{|AH|}-1=
\frac{|EA|}{|AS|}-1=\frac{|ES|}{|AS|}.
$$
Podľa riešenia časti a) však platí $|EC|=|CS|=|SA|$, takže
$|ES|/|AS|=2$, a~preto tiež $|DF|/|CF|=2$, ako sme mali
dokázať.
\inspsc{b73ii.24}{.8333}%

\smallskip\noindent
{\it Časť b), postup $6$.}
Označme $J$ stred úsečky $BS$.
Podľa riešenia časti a) je trojuholník $BSC$ rovnostranný,
takže úsečka $CJ$ je jeho výška.
Následne z~$CJ\perp BD$ a~$FS\perp BD$ vyplýva, že trojuholníky
$CDJ$ a~$FDS$ sú podľa vety $uu$ podobné. Preto zo
zrejmej rovnosti $|DS|=\frac23|DJ|$ vyplýva tiež
$|DF|=\frac23|DC|$. To už znamená, že naozaj platí $|DF|=2\,|CF|$.
\inspsc{b73ii.25}{.8333}%

\schemaABC
Za úplné riešenie dajte 6 bodov, po 3 bodoch za každú z~častí a) a~b).
V prípade neúplných postupov oceňte čiastkové kroky nasledovne:

\smallskip\noindent
\item{A1.} Dôkaz rovnosti $|AB|=|BE|$: 1 bod.
\item{A2.} Dôkaz rovnosti $|SC|=|CE|$ (alebo tvrdenie, že $C$ je stred kružnice opísanej trojuholníku~$BSE$): 2~body.
\item{A3.} Ak je riešenie časti a) založené na výpočtoch uhlov (ako v~našom postupe 3), dajte:
\itemitem{$\triangleright$} 1 bod za označenie neznámej veľkosti (povedzme $\varphi$) jedného z~uhlov pri vrchole $B$, $C$, $S$ alebo $E$ s~cieľom vyjadriť pomocou $\varphi$ veľkosti ďalších uhlov potrebných na výpočet tejto neznámej.
\itemitem{$\triangleright$} 1 bod za spomínané vyjadrenia vedúce k~zostaveniu rovnice pre jednu neznámu $\varphi$ (na základe súčtu $180\st$ vnútorných uhlov vhodného trojuholníka alebo rovnosti $|\uhol EBS|=90\st$ alebo rovnoramennosti jedného z~trojuholníkov $BEC$ alebo $BCS$).
\itemitem{$\triangleright$} 1 bod za vyriešenie zostavenej rovnice a~určenie hodnoty $|\uhol BSC|=60\st$.
\item{} Za zavedenie viac ako jednej neznámej žiadny bod neudeľujte, pokiaľ nie je buď ich elimináciou získaná rovnica s~1~neznámou (za 2 body), alebo je zapísaná sústava rovníc, ktorá má jediné riešenie (taktiež za 2 body).
\item{B1.} Dôkaz, že trojuholník $CFS$ je rovnoramenný: 1 bod.
\item{B2.} Dôkaz rovnosti $|DF|=2\,|SF|$: 1 bod.
\item{B3.} Myšlienka využiť podobnosť trojuholníkov $DFS$ a~$DBC$ na~porovnanie dĺžok ich strán: 1~bod.
\item{B4.} Dokreslenie aspoň jedného z~bodov $M$, $G$, $H$, $J$: 1 bod.

\smallskip\noindent
Za časť a) dajte $\rm\max(A1,A2,A3)$ bodov, za časť b) potom
$\rm\max(B1+B2,B3,B4)$ bodov. Absenciu zmienky o~polohe bodu~$E$
na polpriamke opačnej k~polpriamke $CA$ tolerujte.
\endschema
}

{%%%%%   B-S-3
Ukážeme, že čísla 1 a~3 sú jediné možné hodnoty
daného výrazu. Zdôraznime, že vďaka nenulovosti čísel
$a$, $b$, $c$ platí $a^2+b^2+c^2>0$, takže daný zlomok má zmysel
a~naše výpočty jeho hodnôt budú korektné.

Upravujme najprv prvú zo zadaných rovností:
$$\eqalign{
a^2(b+c)&=b^2(c+a),\cr
a^2b+a^2c&=b^2c+b^2a,\cr
a^2b-b^2a&=b^2c-a^2c,\cr
ab(a-b)&=c(b-a)(b+a),\cr
ab(a-b)&=(b-a)(cb+ca),\cr
(a-b)(ab&+bc+ca)=0.\cr
}$$
Analogickou úpravou druhej rovnosti dostaneme
$$
(b-c)(ab+bc+ca)=0.
$$
Vidíme, že ak $ab+bc+ca \ne 0$, tak platí $a-b=0$ aj $b-c=0$,
čo znamená $a=b=c$. Je teda nutne $ab+bc+ca=0$ alebo $a=b=c$.
Pozrime sa na hodnotu daného výrazu v~každom z~oboch prípadov.

V~prípade $ab+bc+ca=0$ platí
$$
\frac{(a+b+c)^2}{a^2+b^2+c^2}=
\frac{a^2+b^2+c^2+2(ab+bc+ca)}{a^2+b^2+c^2}=
\frac{a^2+b^2+c^2}{a^2+b^2+c^2}=1,
$$
zatiaľ čo v~prípade $a=b=c$ vychádza
$$
\frac{(a+b+c)^2}{a^2+b^2+c^2}=\frac{(3a)^2}{a^2+a^2+a^2}=
\frac{9a^2}{3a^2}=3.
$$
Oba prípady sú možné, napríklad prvý nastane pre
$(a,b,c)=(3,6,{-2})$ a~druhý pre $(a,b,c)=(1,1,1)$.\fnote{Prvý
uvedený príklad má jednoduchšiu alternatívu $(a,b,c)=(2,2,{-1})$,
ktorú možno ľahko nájsť, keď v~rovnosti $ab+bc+ca=0$
položíme $b=a$. Dostaneme tak rovnosť $a(a+2c)=0$, takže stačí vziať
$b=a={-2c}$, kde $c\ne0$.}
(Overenie rovností zo zadania je pre trojicu $(3,6,{-2})$
ľahké, pre trojicu $(1,1,1)$ je triviálne.)

\ineriesenie
Kratším postupom odvodíme podmienku, že platí $ab+bc+ca=0$ alebo
$a=b=c$. (Potom už je možné riešenie dokončiť ako pôvodne.)

Ak odčítame súčin $(b+c)(c+a)(a+b)$ od každého z troch výrazov
v~rovnostiach
$$
a^2(b+c)=b^2(c+a)=c^2(a+b),
$$
dostaneme po vyňatí spoločných činiteľov rovnosti
$$
(b+c)\bigl[a^2- (a+b)(c+a)\bigr]=
(c+a)\bigl[b^2- (a+b)(b+c)\bigr]=
(a+b)\bigl[c^2-(b+c)(c+a)\bigr].
$$
Výrazy v~hranatých zátvorkách sa rovnajú tomu istému výrazu
$-(ab+bc+ca)$, takže platí
$$
-(b+c)(ab+bc+ca)=-(c+a)(ab+bc+ca)=-(a+b)(ab+bc+ca).
$$
Odtiaľ v~prípade $ab+bc+ca\ne0$ vychádza $b+c=c+a=a+b$, a~teda $a=b=c$,
ako sme sľúbili odvodiť.

\schemaABC
Za úplné riešenie dajte 6 bodov, tolerujte pritom absenciu
úvodnej zmienky o~tom, že zadaný zlomok má zmysel.
V prípade neúplných postupov oceňte čiastkové kroky nasledovne:

\smallskip
\item{A1.} Odvodenie rozkladu $(a-b)(ab+bc+ca)=0$ alebo rozkladu analogického: 2~body.
\item{A2.} Zdôvodnenie, že platí $a=b=c$ alebo $ab+bc+ca=0$: 3 body.
\item{A3.} Určenie hodnoty výrazu v~prípade $ab+bc+ca=0$: 1 bod.
\item{A4.} Uvedenie príkladu nenulových čísel $a$, $b$, $c$ spĺňajúcich okrem rovností zo zadania aj rovnosť $ab+bc+ca=0$: 1~bod.
\item{A5.} Určenie hodnoty výrazu v~prípade $a=b=c$: 1 bod.

\smallskip\noindent
Celkom potom dajte $\rm\max(A1,A2)+A3+A4+A5$ bodov.

Za akékoľvek úpravy zadaných rovností, ktoré nevedú k~záveru
z~A1 ani A2, žiadne body neudeľujte. Ak by riešiteľ
podľa doterajších pokynov nemal získať žiadny bod, dajte
1~bod za uvedenie hypotézy, že jediné možné hodnoty daného zlomku
sú 1 a~3, ak sú obe doložené príkladom trojice $(a,b,c)$ nenulových čísel,
ktorá spĺňa rovnosti zo zadania.
\endschema
}

{%%%%%   B-II-1
Tvrdenie úlohy dokážeme sporom. Pripusťme teda, že čísla $a$, $b$
nie sú rozmiestnené striedavo. To zrejme znamená, že
sa v~kruhu vedľa seba nachádzajú niekde dve $a$ a súčasne niekde
dve $b$.

Vyberme dve susedné $a$. Ak začneme od tejto
dvojica $aa$ putovať po kruhu jedným smerom, narazíme na číslo
$b$ (inak by v~kruhu boli samé $a$). Akonáhle sa to stane, budeme
mať susednú trojicu $aab$. Podľa zadania potom platí $a\mid a+b$,
odkiaľ $a\mid b$. Analogickou úvahou nájdeme trojicu $bba$,
z ktorej dostaneme $b\mid a$. Pre prirodzené čísla $a$, $b$ tak platí
$a\mid b$ aj $b\mid a$, a teda $a=b$, čo je v spore s~tým, že čísla $a$, $b$ sú rôzne.
Tým je celé riešenie úlohy hotové.


\poznamky
\item{1.}
Bez ujmy na všeobecnosti sme od začiatku dôkazu sporom mohli
predpokladať, že platí napríklad $a<b$. Potom by sme vystačili
len s~úvahou o trojici~$bba$ vedúcou k~spornému záveru $b\mid a$.
\item{2.}
Ukážme navyše, že pre čísla $a$, $b$ spĺňajúce
zadanie úlohy musí platiť
buď $b=2a$, alebo $a=2b$. Vzhľadom na symetriu stačí
v~prípade $a<b$ dokázať rovnosť $b=2a$. Pri striedavom rozmiestnení čísel
z~trojice $aba$ máme $b\mid 2a$, takže $2a=kb$ pre
vhodné prirodzené číslo $k$. Z~predpokladu $a<b$ však vyplýva
$kb=2a<2b$, odkiaľ $k<2$, čiže $k=1$, a preto $2a=kb=b$, ako sme
chceli dokázať.

\schemaABC
Za úplné riešenie dajte 6 bodov. Neúplné riešenia, ktoré sa zaoberajú
prípadom, keď rozmiestnenie čísel $a$, $b$ nie je striedavé a keď
riešiteľ vopred nevyberie, ktorá z~nerovností $a<b$ alebo $b<a$
platí, hodnoťte nasledovne:

\smallskip
\item{A1.} Konštatovanie, že musia byť vedľa seba dve $a$ aj dve $b$: 1 bod.
\item{A2.} Konštatovanie, že musí existovať ako úsek $aab$ (alebo $baa$), tak aj úsek $abb$ (alebo $bba$): 3 body.
\item{A3.} Korektné odvodenie oboch deliteľností $a\mid a+b$ a $b\mid a+b$: 4 body.
\item{A4.} Dosiahnutie sporu s~prípadnou drobnou argumentačnou chybou: 5 bodov.

\smallskip\noindent
Celkom potom dajte $\rm\max(A1,A2,A3,A4)$ bodov. Ak riešiteľ vopred vyberie, že z~nerovností $a<b$ alebo $a>b$ platí
napríklad $a<b$, potom stačí uviesť v~A1 iba $bb$, v~A2 iba $bba$
a v~A3 iba $b\mid a+b$ (pozri poznámku 1 za riešením).
Ak postupuje riešiteľ nami neuvedeným spôsobom, hodnoťte jeho
čiastkové kroky obdobne.
\endschema
}

{%%%%%   B-II-2
Podať riešenie zadanej úlohy je možné v~mnohých obmenách, preto
najprv opíšeme niektoré ich spoločné prvky a možnosti.

\smallskip
\item{1.}
Možno si ľahko všimnúť, že tvrdenie úlohy platí práve vtedy, keď sa rovnajú niektoré dve z~čísel $a$, $b$, $c$. Preto môžeme
celé riešenie pojať tak, že z predpokladu rôznosti
čísel $a$, $b$, $c$ odvodíme spor. Je však tiež možné podať
nepriamy dôkaz rovnosti $b=c$ (prvé riešenie)
alebo rovnosti $a=b$ (v~poznámke za prvým riešením).
\item{2.}
Je tiež vidieť, že rovnosť prvého a druhého zlomku zo zadania
nastane určite v~prípade, keď platí $b=c$. Preto sa oplatí túto rovnosť
algebraicky upravovať s~cieľom, aby prešla na tvar $(b-c)(\dots)=0$.
Ukážeme ďalej, že takto vyjde rovnosť
$$
(b-c) (a-b-c-d) = 0.
\tag1
$$
Analogicky z~rovnosti druhého a tretieho zlomku dostaneme
$$
(a-b)(a+b-c+d)=0.
\tag2
$$
(Upraviť rovnosť prvého a tretieho zlomku podobne pekne nemožno.)
V~prvom riešení vystačíme s~odvodením rovnosti \thetag{1}, v druhom riešení
získame obe rovnosti \thetag{1} a \thetag{2} zjednodušeným výpočtom a potom už
ľahko celé riešenie dokončíme.

\smallskip\noindent
Začneme odvodením rovnosti \thetag{1} z~komentára:
$$
\eqalign{
\frac{a-b}{c+d}&=\frac{a-c}{b+d},\cr
(a-b)(b+d)&=(a-c)(c+d),\cr
ab+ad-b^2-bd&=ac+ad-c^2-cd,\cr
(ab-ac)-(b^2-c^2)-(bd-cd)&=0, \cr
a(b-c)-(b-c)(b+c)-d(b-c)&=0, \cr
(b-c)(a-b-c-d)&=0.
}$$
Posúďme najskôr prípad, keď $b-c\ne0$. Z~rovnosti \thetag{1} vtedy vyplýva
$$
a-b-c-d=0,
\tag3
$$
odkiaľ $a-b=c+d$ a $a-c=b+d$. Prvé dva zadané zlomky
tak majú spoločnú hodnotu~1, ktorú musí mať aj tretí zlomok:
$$
\frac{b-c}{a+d}=1.
$$
Odtiaľ máme $b-c=a+d$, čo prepíšeme ako
$$
a-b+c+d=0.
\tag4
$$
Sčítaním rovností \thetag{3} a \thetag{4} dostaneme $2a-2b=0$, čiže
$a-b=0$, čo je v spore s~tým, že zadané zlomky majú hodnotu 1.
Posudzovaný prípad $b-c\ne0$ je tak vylúčený, a~preto platí $b-c=0$.
Tretí zlomok zo zadania je teda naozaj rovný nule.

\poznamka
Ukážme stručnejšie, že analogicky z~rovnosti \thetag{2} vyplýva $a=b$.
V~prípade $a\ne b$ z~\thetag{2} máme $a+b-c+d=0$, takže
druhý a tretí zlomok zo zadania majú spoločnú hodnotu ${-1}$,
a preto ju má aj prvý zlomok. Pre jeho čitateľ a menovateľ tak platí
$a-b=-(c+d)$, čo spolu s~$a+b-c+d=0$ vedie k~záveru $b=c$,
ktorý je však určenou hodnotou ${-1}$ vylúčený.
Tým je dôkaz rovnosti $a=b$ ukončený.

\def\a{A}
\def\b{B}
\def\c{C}
\ineriesenie
Zaveďme tri nové (podľa zadania nenulové) čísla
$$
A=a+d,\quad B=b+d,\quad C=c+d.
\tag5
$$
Vďaka rovnosti $\a-\b=a-b$, $\a-\c=a-c$, $\b-\c=b-c$
dostávame
$$
\frac{\a-\b}{\c}=\frac{\a-\c}{\b}=\frac{\b-\c}{\a}.
$$
Takto zjednodušené rovnosti teraz upravíme súčasne:
$$\displaylines{
\eqalign{
\frac{\a-\b}{\c}&=\frac{\a-\c}{\b},\cr
(\a-\b)\b&=(\a-\c)\c,\cr
\a\b-\b^2&=\a\c-\c^2,
}
\hfil\hskip15pt
\eqalign{
\frac{\a-\c}{\b}&=\frac{\b-\c}{\a},\cr
(\a-\c)\a&=(\b-\c)\b,\cr
\a^2-\a\c&=\b^2-\b\c,\cr
}
\cr
\eqalign{
(\a\b-\a\c)-(\b^2-\c^2)&=0,\cr
\a(\b-\c)-(\b-\c)(\b+\c)&=0,\cr
(\b-\c)(\a-\b-\c)&=0,
}
\hfil
\eqalign{
(\a^2-\b^2)-(\a\c-\b\c)&=0,\cr
(\a-\b)(\a+\b)-\c(\a-\b)&=0,\cr
(\a-\b)(\a+\b-\c)&=0.}
}
$$
Odtiaľ po dosiahnutí vzťahov \thetag{5} už dostaneme rovnosť z~komentára
$$
(b-c)(a-b-c-d)=0\quad\thetag{1}\qquad\hbox{a}\qquad
(a-b)(a+b-c+d)=0\quad\thetag{2}.
$$
Ak je $b-c=0$ alebo $a-b=0$, tvrdenie úlohy platí. V~opačnom
prípade z~rovností \thetag{1} a~\thetag{2} vyplýva
$$
a-b-c-d=0\quad\hbox{a}\quad a+b-c+d=0,
$$
čo po sčítaní dáva $2a-2c=0$, čiže $a-c=0$, takže aj vtedy
tvrdenie úlohy platí.

\schemaABC
Za úplné riešenie dajte 6 bodov. Neúplné riešenia hodnoťte takto:

\smallskip
\item{A1.} Konštatovanie, že stačí dokázať rovnosť dvoch z troch čísel $a$, $b$, $c$: 1 bod.
\item{A2.} Odvodenie jednej z~rovností \thetag{1} alebo \thetag{2}: 3 body.
\item{A3.} Odvodenie oboch rovností \thetag{1} a \thetag{2}: 4 body.
\item{A4.} Dôkaz tvrdenia, že v~prípade troch rôznych čísel $a$, $b$, $c$ by spoločná hodnota zadaných zlomkov bola rovná 1 alebo $-1$: 4 body.
\item{A5.} Riešenie s~drobným argumentačným nedostatkom: 5 bodov.

\smallskip\noindent
Celkom potom dajte $\rm\max(A1,A2,A3,A4,A5)$ bodov.
Ak postupuje riešiteľ nami neuvedeným spôsobom, hodnoťte jeho
čiastkové kroky obdobne.
\endschema
}

{%%%%%   B-II-3
Dokážeme, že bod~$C$ je ťažiskom trojuholníka $BFD$, z čoho už vyplynie,
že priamka $BC$ naozaj rozpoľuje jeho stranu~$DF$.

Najskôr si všimneme, že z~rovnobežnosti základní $AB$ a $CD$
vyplýva zhodnosť striedavých uhlov $PAB$ a $PED$ (\obr).
Keďže zhodné sú aj vrcholové uhly $APB$ a~$EPD$,
trojuholníky $PAB$ a $PED$ sú podobné, a teda vďaka rovnosti $|PB|=|PD|$
dokonca zhodné. Preto platí $|DE|=|BA|=6$.
Bod~$E$ zrejme leží na polpriamke $DC$,\fnote{Zhodné
trojuholníky $PAB$ a $PED$ sú totiž súmerne združené
podľa stredu $P$, rovnako ako rovnobežné polpriamky
$BA$ a $DC$.} takže $|DC|=4$, čo spolu s~$|DE|=6$ znamená, že
bod~$C$ leží na úsečke~$DE$ a pritom $|DC|=2\,|CE|$.
\inspsc{b73iii.31}{.8333}%

Teraz prejdime k~trojuholníku $BDF$. Keďže bod $P$ je stred
jeho strany~$BD$ a~$PE\parallel DF$, je úsečka $PE$ jeho strednou priečkou,
teda bod $E$ je stred strany~$BF$. Úsečka~$DE$ je preto
ťažnicou, takže jej bod $C$ určený rovnosťou $|DC|=2|CE|$ je naozaj
ťažiskom tohto trojuholníka, ako sme sľúbili dokázať.

\poznamka
Ukážeme, že po dôkaze zhodnosti trojuholníkov $PAB$ a~$PED$ je možné
zvyšok riešenia dokončiť inak. Vďaka spomínanej zhodnosti je totiž
štvoruholník $ABED$ rovnobežník, pretože jeho strany $AB$ a $ED$
sú zhodné a rovnobežné. Ďalej pokračujme nasledovne:

\smallskip
\item{$\triangleright$} Keďže $ABED$ je rovnobežník, je aj štvoruholník $AEFD$
rovnobežník, a to vďaka rovnobežnosti svojich protiľahlých strán.
Z oboch rovnobežníkov tak dostávame $|BE|=|AD|=|EF|$, čo znamená,
že bod $E$ je stred úsečky $BF$.
\item{$\triangleright$} V~rovnobežníku $ABED$ platí $|DE|=|AB|=6$, v~lichobežníku $ABCD$
máme $|DC|=4$. Dokopy to znamená,
že bod $C$ leží na úsečke $DE$ a $|CE|=6-4=2$, a~preto
$|DC|=2\,|CE|$.

\smallskip\noindent
Podľa prvej odvodenej vlastnosti je úsečka $DE$ ťažnica trojuholníka
$BFD$, a~preto bod~$C$ je vďaka druhej vlastnosti jeho
ťažiskom.\fnote{K tomuto záveru sme tentoraz nevyužili stred
$P$ strany $BD$.} Priamka $BC$ tak skutočne rozpoľuje stranu~$DF$, ako
sme mali dokázať.

\ineriesenie
Rovnako ako v~prvom riešení dokážeme, že $|AP|=|EP|$
a že bod~$C$ leží na úsečke~$DE$ tak, že $|EC|=2$.
Keďže $PE$ je stredná priečka trojuholníka $BDF$
rovnobežná s~jeho stranou~$DF$, stačí dokázať, že
priamka $BC$ rozpoľuje úsečku~$PE$. Potom totiž bude rozpoľovať
aj úsečku~$DF$, pretože trojuholník $BDF$ je dvojnásobným zväčšením trojuholníka $BPE$.\fnote{Toto zväčšenie sa v geometrii nazýva \emph{rovnoľahlosť} so stredom $B$ a koeficientom 2.}

Označme preto $X$ priesečník úsečiek $BC$, $PE$ a ukážme,
že naozaj platí $|PX|=|EX|$. Keďže trojuholník $ECX$ je podobný
trojuholníku $ABX$ podľa vety $uu$,
z~rovnosti $|AB|=3\,|EC|$ vyplýva tiež $|AX|=3\,|EX|$.
Odtiaľ $|AX|=\frac34|AE|$ a $|EX|=\frac14|AE|$, podobne
z~$|AP|=|EP|$ máme $|AP|=\frac12|AE|$, a preto
$$
|PX|=|AX|-|AP|=\tfrac34|AE|-\tfrac12|AE|=\tfrac14|AE|=|EX|,
$$
teda~$|PX|=|EX|$, ako sme sľúbili ukázať.

\schemaABC
Za úplné riešenie dajte 6 bodov. Neúplné postupy hodnoťte nasledovne:

\smallskip
\item{A1.} Konštatovanie, že tvrdenie úlohy bude dokázané, keď ukážeme, že bod $C$ je ťažiskom trojuholníka~$BDF$: 1 bod.
\item{B1.} Určenie polohy bodu $C$ na úsečke $DE$ (napríklad rovnosťou $|CE|=2$): 2 body.
\item{B2.} Dôkaz poznatku, že bod $E$ je stredom úsečky $BF$: 2 body.
\item{B3.} Dokončenie dôkazu, že bod $C$ je ťažiskom trojuholníka~$BDF$: 1 bod.
\item{C1.} Dôkaz zhodnosti trojuholníkov $PAB$ a~$PED$: 1 bod.
\item{C2.} Dôkaz poznatku, že $ABED$ je rovnobežník: 2 body. Ak je tento poznatok iba konštatovaný (prípadne označený za zrejmý), dajte iba 1 bod.
\item{D1.} Zavedenie priesečníka $X$ úsečiek $BC$, $PE$ so zámerom dokázať rovnosť $|PX|=|EX|$: 1 bod.
\item{D2.} Dôkaz rovnosti $|AX|=3\,|EX|$: 1 bod.
\item{D3.} Dôkaz rovnosti $|PX|=|EX|$: 1 bod.
\item{D4.} Dokončenie dôkazu použitím odvodeného poznatku, že priamka~$BC$ rozpoľuje strednú priečku~$PE$ trojuholníka~$BDF$: 1 bod.

\smallskip\noindent
Celkom potom dajte
$\rm\max\bigl(A1+B1+B2+B3,\,A1+\max(C1,C2)+B2+B3,\,
B1+D1+D2+D3+D4\bigr)$ bodov.
Ak riešiteľ postupuje nami neuvedeným
spôsobom, hodnoťte jeho čiastkové kroky obdobne.
\endschema
}

{%%%%%   B-II-4
Ukážeme, že hľadaný počet je $2\cdot (2024!)^2$.
Každú správne vyplnenú tabuľku budeme ďalej nazývať
len~\uv{tabuľkou}, výhodne označíme $n=2024$ a~vyriešime úlohu
pre tabuľku $n\times n$ so všeobecne zvoleným $n$,
keď počet správnych vyplnení vyjde $2\cdot(n!)^2$.

Keďže súčty čísel v~riadkoch tabuľky sú podľa
zadania navzájom rôzne, chýba medzi nimi jediná z~$n+1$
možných hodnôt $0,1,\dots,n$ takých súčtov. To isté platí
aj pre súčty čísel v~jednotlivých stĺpcoch tabuľky. Z~toho
vyplýva, že ako medzi riadkovými, tak medzi stĺpcovými súčtami
je zastúpené aspoň jedno z~čísel $0$ alebo $n$. Znamená to,
že v~tabuľke vždy nájdeme riadok so samými~0 alebo samými~1 aj takýto stĺpec.
Zrejme však nemôžu byť v~jednom smere niekde samé~0 a v~druhom smere niekde samé~1. Práve jedno z~čísel tak vypĺňa
celý riadok aj celý stĺpec, takže riadkové aj stĺpcové súčty
sú vždy buď $0,1,\dots,n-1$, alebo $1,2,\dots,n$.
Podľa toho všetky tabuľky rozlíšime na dva {\it druhy}.

Vysvetlíme teraz, prečo tabuliek oboch druhov je rovnaký počet.
Naozaj, ak v~tabuľke jedného druhu všetky čísla
\uv{prevrátime} (z~0 na~1 a z~1 na~0), dostaneme zrejme
tabuľku druhého druhu.\fnote{Ľubovoľný riadkový aj stĺpcový
súčet totiž zmení svoju hodnotu~$s$ na hodnotu $n-s$,
lebo každé zapísané číslo $c$ sme zamenili číslom $1-c$.} Preto sa ďalej
budeme venovať iba určeniu počtu tých tabuliek, ktorých
riadkové aj stĺpcové súčty sú $0,1,\dots,n-1$.

Ďalej v dvoch etapách dokážeme, že ak jednotlivé súčty $0,1,\dots,n-1$
vopred ľubovoľne priradíme ako konkrétnym riadkom, tak konkrétnym stĺpcom,
bude zodpovedajúca tabuľka existovať a navyše bude jediná.
Keďže na priradenie jednotlivých súčtov $0,1,\dots,n-1$
konkrétnym riadkom aj konkrétnym stĺpcom máme práve $n!\cdot
n!=(n!)^2$ možností, počet tabuliek so súčtami
$0,1,\dots,n-1$ potom bude rovný $(n!)^2$. Ako už vieme,
rovnaký je aj počet tabuliek so súčtami $1,2,\dots,n$.
Hľadaný celkový počet tabuliek tak bude naozaj
rovný $2\cdot(n!)^2$.

\emph{Etapa $1$}. Predpokladajme najskôr, že súčty $0,1,\dots,n-1$ sú
v tomto poradí predpísané riadkom tabuľky zhora nadol a jej
stĺpcom zľava doprava.
Vysvetlíme, prečo pre tento špeciálny prípad je
vyplnenie tabuľky jediné možné
a vyzerá ako na \obr{}.
\inspsc{b73iii.41}{.8333}%

S~vypĺňaním tabuľky začneme v ľavom spodnom
rohu a budeme postupovať k~stredu tabuľky \uv{po špirále}
po krokoch, ktoré zodpovedajú farebne rozlíšeným úsekom
vykreslenej čiary. Tak v prvom kroku
pôjde o \uv{okrajové} políčka celej tabuľky: stĺpec aj riadok pre
súčet 0 musíme vyplniť samými nulami a potom prázdnych $n-1$
políčok v stĺpci aj riadku pre súčet $n-1$ musíme vyplniť samými
jednotkami. V druhom kroku obdobne musíme vyplniť všetky
okrajové polia zo zatiaľ nevyplneného \uv{zvyšku} tabuľky, atď.
Nakoniec tak dôjdeme k~jedinému možnému vyhovujúcemu vyplneniu
celej tabuľky.

\emph{Etapa $2$}. Uveďme najskôr zrejmé pravidlá, ktoré budeme v tejto časti
potrebovať: Ak zmeníme v~akejkoľvek vyplnenej tabuľke
poradie jej riadkov,
dôjde k rovnakej zmene poradia pôvodných hodnôt riadkových súčtov,
zatiaľ čo stĺpcové súčty nezmenia ani svoje hodnoty, ani svoje
poradie. Analogické pravidlo platí aj pre zmeny poradia stĺpcov.

Predpokladajme teraz, že súčty $0,1,\dots,n-1$ sú predpísané
ako riadkom tabuľky v~ľubovoľnom poradí, tak aj jej stĺpcom
v ľubovoľnom poradí. Potom jej vyhovujúce vyplnenie dostaneme,
keď v tabuľke vyplnenej ako na obrázku vyššie pozmeníme príslušnými
spôsobmi najskôr poradie riadkov (ak je to vôbec potrebné)
a potom (opäť len v prípade nutnosti) aj poradie stĺpcov. Rovnako ľahko
vysvetlíme, prečo je takto zostrojené vyhovujúce vyplnenie jediné:
Z každej vyhovujúcej tabuľky musíme opačnými zmenami poradia riadkov a~stĺpcov
dostať tabuľku z obrázka, pretože tá je, ako už vieme,
pre tento špeciálny prípad jediná možná. Tým je potrebné tvrdenie
dokázané.

\poznamka
Ukážme pre zaujímavosť, ako je možné postup
vypĺňania tabuľky s riadkovými aj stĺpcovými súčtami
$0,1,\dots,n-1$ v ľubovoľných dvoch poradiach formalizovať.

V prvom kroku uvážime dva riadky a dva stĺpce tabuľky s predpísanými
súčtami~0 a~$n-1$. Riadok aj stĺpec pre súčet~0 musíme vyplniť
(namiesto toho ďalej píšeme len \uv{vyplníme}) samými~0 a
potom prázdnych $n-1$ miest v~riadku aj stĺpci pre súčet $n-1$
vyplníme samými~1.
V~každom ďalšom $i$-tom kroku, kým $2i\leqq n$, predpokladáme,
že sú vyplnené práve riadky a stĺpce pre všetky súčty $s<i-1$ a
$s>n-i$ a pritom
v~každom zo zostávajúcich $n-2i+2$ riadkov a $n-2i+2$ stĺpcov už máme
$i-1$~krát~0 a~$i-1$~krát~1, a~teda $n-2i+2$ voľných miest.
Najprv tieto miesta v~riadku aj stĺpci pre súčet~$i-1$
vyplníme samými~0, potom zvyšné voľné miesta (v~počte $n-2i+1$)
v~riadku aj stĺpci pre súčet $n-i$ doplníme samými~1
(jedine tak v~nich dosiahneme súčtu $(i-1)+(n-2i+1)=n-i$).
V prípade nepárneho $n=2t+1$ uskutočníme ešte jeden
krok s poradovým číslom $t+1$,
keď dáme 0 na posledné voľné miesto -- to leží
v~riadku aj stĺpci pre súčet $t$, v ktorých už je
$t$~krát~0 a $t$~krát~1. Tak dostaneme
požadovanú tabuľku $n\times n$ pre párne aj nepárne $n$.


\ineriesenie
Prvú časť pôvodného riešenia (o existencii tabuliek $n\times n$
dvoch druhov a rovnosti ich počtov) opakovať nebudeme.
Iba iným spôsobom odvodíme, že pre počet $P(n)$ všetkých tabuliek
$n\times n$ so súčtami $0,1,\dots,n-1$, ktorý je ako vieme rovný
počtu týchto tabuliek so súčtami $1,2,\dots,n$, platí
$P(n)=(n!)^2$ pre každé $n\geqq1$.

Všetky tabuľky $n\times n$ s~daným $n>1$ a súčtami $0,1,\dots,n-1$
rozdelíme do $n\cdot n=n^2$ skupín podľa toho,
ktorý riadok a ktorý stĺpec je v~tabuľke zostavený zo samých~$0$.
Ak tento riadok a tento stĺpec z~tabuľky vyškrtneme,
zostane nám tabuľka $(n-1)\times(n-1)$, ktorej riadkové súčty aj
stĺpcové súčty sú čísla $1,2,\dots,n-1$. Naopak z~každej tabuľky
$(n-1)\times(n-1)$ so súčtami $1,2,\dots,n-1$,
ktorých je $P(n-1)$, zostavíme $n^2$ rôznych tabuliek $n\times n$
so súčtami $0,1,\dots,n-1$, keď v~nej ľubovoľne (pred prvý riadok, medzi dva
susedné riadky, alebo za posledný riadok) urobíme miesto
pre nový riadok, rovnako tak potom urobíme miesto pre nový stĺpec
a nakoniec nový riadok aj stĺpec vyplníme samými~$0$.
Preto pre každé $n>1$ platí $P(n)=n^2P(n-1)$, čo spolu
so zrejmou hodnotou $P(1)=1$ už vedie k vzorcu $P(n)=(n!)^2$,
ktorý sme sľúbili odvodiť.

\schemaABC
Za úplné riešenie dajte 6 bodov. Neúplné riešenia
hodnoťte nasledovne:

\smallskip
\item{A1.} Hypotéza, že riadkové aj stĺpcové súčty vyplnenej tabuľky tvoria rovnakú skupinu čísel, a to buď $0,1,\dots,2023$, alebo $1,2,\dots,2024$: 1 bod.
\item{A2.} Dôkaz hypotézy z~A1: 1 bod. Ak dôkaz vychádza z~poznatku, že riadkové aj stĺpcové súčty tvoria rovnakú skupinu čísel, ani tento poznatok nemožno považovať za zrejmý a musí byť zdôvodnený, najjednoduchšie úvahou o~dvojakom sčítaní všetkých čísel tabuľky (po riadkoch a po stĺpcoch).
\item{B1.} Určenie počtu možných rozmiestnení riadkových a stĺpcových súčtov (pre jeden z dvoch druhov tabuliek): 1 bod.
\item{B2.} Hypotéza, že akákoľvek konkretizácia jednotlivých riadkových a stĺpcových súčtov vždy určuje práve jednu tabuľku: 1 bod.
\item{B3.} Dôkaz hypotézy z~B2: 1~bod.
\item{C1.} Zdôvodnenie, prečo pre jedno riešiteľom vybrané rozloženie riadkových a stĺpcových súčtov existuje práve jedno vyplnenie tabuľky: 1 bod.
\item{C2.} Zdôvodnenie, prečo z jednej tabuľky z C1 možno dostať permutáciami riadkov a stĺpcov všetky tabuľky jedného druhu: 1 bod.
\item{D1.} Odvodenie vzťahov medzi počtami tabuliek $n\times n$ pre rôzne $n$ (tabuliek či už jedného, alebo oboch možných druhov dokopy), ktoré umožňujú rekurentný výpočet výsledku: 3~body.
\item{E1.} Uvedenie správneho výsledku: 1 bod.

\smallskip\noindent
Celkom potom dajte
$\rm A1+A2+\max(B1+B2+B3,C1+C2+B1,\,D1)+E1$ bodov.
Pri správnom postupe aj výsledku s menšími argumentačnými
nedostatkami dajte 5 bodov, rovnako ako pri inak správnom postupe
s~numerickou chybou. Ak postupuje riešiteľ nami neuvedeným spôsobom,
hodnoťte jeho čiastkové kroky obdobne.
\endschema
}

{%%%%%   C-S-1
Označme $s$ súčet všetkých Pažítkových známok a~$p$ ich počet.
Ani jedno z~týchto čísel nepoznáme, ale vieme, že platí
$$
\frac{s}{p}=3, \quad\hbox{čiže}\quad s=3p.
$$
Podľa druhej vety zadania zostavíme
pre neznáme $s$, $p$ ďalšiu rovnicu a upravíme ju:
$$
\frac{s-3\cdot5}{p-3}=2 \quad \Leftrightarrow \quad s-15=2p-6
\quad \Leftrightarrow \quad s=2p+9.
$$
Odtiaľ po dosadení $s=3p$ dostaneme rovnicu $3p=2p+9$ s~jediným
riešením $p=9$. Súčet známok je tak rovný $s=3p=27$.

Okrem troch pätiek teda Pažítka dostal ešte ďalších šesť známok so
súčtom rovným $27-3\cdot5=12$. Určite to nemohlo byť ani šesť
jednotiek (so súčtom iba $6$), ani päť jednotiek,
pretože šiesta známka by potom bola $12-5\cdot1=7$.
Štyri jednotky však Pažítka dostať mohol, jeho zvyšné dve známky
so súčtom $12-4=8$ by potom boli buď dve štvorky, alebo trojka
a~päťka.

\zaver
Najväčší možný počet Pažítkových jednotiek je 4.


\schemaABC
Za úplné riešenie dajte 6 bodov. V~prípade neúplných postupov oceňte
čiastkové kroky nasledovne:

\smallskip
\item{A0.} Uvedenie správnej odpovede bez zdôvodnenia: 0 bodov.
\item{A1.} Uvedenie správnej odpovede spolu s~vyhovujúcim príkladom sady známok: 1 bod.
\item{A2.} Prepis slovného zadania úlohy do matematickej symboliky (najčastejšie do sústavy dvoch rovníc o~dvoch neznámych): 1 bod.
\item{A3.} Vyriešenie matematickej úlohy z~bodu A2 alebo iné zdôvodnenie vedúce k~záveru, že Pažítka dostal $9$~známok (resp. $6$ známok, pokiaľ neuvažujeme tri päťky): 4 body.
\item{A4.} Zdôvodnenie, prečo Pažítka nemohol dostať viac ako 4 jednotky: 5 bodov.

\smallskip\noindent
Celkom potom dajte $\rm\max(A2,A3,A4)+A1$ bodov. V~úplnom riešení
musí byť zdôvodnené, že situácia so štyrmi jednotkami môže
skutočne nastať (najčastejšie príkladom sady známok, ako je uvedené
v~bode~A1).
\endschema
}

{%%%%%   C-S-2
Uvažujme najskôr os vnútorného uhla pri vrchole $D$.
Tá pretne základňu~$AB$ v~bode, ktorý
označíme $P$. Os uhla príslušný uhol rozpoľuje, takže modro
vyznačené uhly~$ADP$ a~$CDP$ sú zhodné (\obr). Vďaka
$AB\parallel CD$ je s~nimi zhodný aj uhol $APD$ striedavý k~uhlu~$CDP$.
Z~rovnosti $|\uhol ADP|=|\uhol APD|$ vyplýva, že trojuholník $ADP$ je
rovnoramenný so základňou~$DP$.
\inspsc{c73ii.3}{.8333}%

Podľa zadania os vnútorného uhla pri vrchole $C$ pretína
základňu $AB$ v~tom istom bode~$P$. Analogicky ako vyššie
sa zdôvodní zhodnosť zeleno vyznačených uhlov, a teda
aj rovnoramennosť trojuholníka $BCP$ so základňou $CP$.

Využitím oboch nájdených rovnoramenných trojuholníkov už dostávame
$$
|AD|+|BC|=|AP|+|BP|=|AB|.
$$

\schemaABC
Za úplné riešenie dajte 6 bodov. V~prípade neúplných postupov oceňte
čiastkové kroky nasledovne:

\smallskip
\item{A1.} Náčrt, v ktorom sa osi vnútorných uhlov pri vrcholoch $C$ a $D$ pretnú na základni $AB$ a~v~ktorom sú vyznačené oba páry zhodných uhlov pri vrcholoch $C$ aj $D$ (tolerujte, keď zhodnosť týchto uhlov nie je zapísaná v texte): 1 bod.
\item{A2.} Pozorovanie, že $|\uhol APD|=|\uhol PDC|$ alebo že $|\uhol BPC|=|\uhol PCD|$: 2 body.
\item{A3.} Dôkaz, že platí aspoň jedna z rovností $|AD|=|AP|$ alebo $|BC|=|BP|$ (tiež že aspoň jeden z~trojuholníkov $ADP$, $BCP$ je rovnoramenný): 3~body.
\item{A4.} Dôkaz, že platia obe rovnosti $|AD|=|AP|$ aj $|BC|=|BP|$ (tiež že oba trojuholníky $ADP$, $BCP$ sú rovnoramenné): 4 body.
\item{B1.} Záznam myšlienky, že na dôkaz tvrdenia by stačilo, keby platilo $|AD|=|AP|$ a $|BC|=|BP|$: 2~body.

\smallskip\noindent
Celkom potom dajte $\rm\max(A1,A2,A3,A4)+B1$ bodov.
\endschema
}

{%%%%%   C-S-3
a) Šesť rovnakých čísel je napríklad v~tejto
tabuľke spĺňajúcej podmienky úlohy:
$$
\vbox{\def\pol#1{\hbox to 15 bp{\hss#1\hss}\vrule}%
\halign{\vrule height12bp depth 3bp\pol#&&\pol#\cr
\noalign{\hrule}
8 & 1 & 1 \cr\noalign{\hrule}
1 & 8 & 1 \cr\noalign{\hrule}
1 & 1 & 8 \cr\noalign{\hrule}
}}
$$
Keby v~tabuľke bolo aspoň sedem rovnakých čísel, obsahoval by
niektorý riadok tri z~týchto siedmich rovnakých čísel. Súčet troch
rovnakých celých čísel však nemôže byť rovný $10$.

\zaver
Najväčší možný počet rovnakých čísel v~tabuľke je 6.


\smallskip
b) Tabuľka môže obsahovať šesť rôznych čísel, jeden z~mnohých
príkladov\fnote{Možno ukázať, že každá vyhovujúca tabuľka
so šiestimi rôznymi číslami je vyplnená niektorou z devätíc čísel
$(1,1,2,2,3,3,5,6,7)$, $(1,1,2,2,3,4,5,5,7)$,
$(1,1,2,3,3,4,5,5,6)$ alebo $(1,2,2,3,3,4,4,5,6)$.}
vyzerá takto:
$$
\vbox{\def\pol#1{\hbox to 15 bp{\hss#1\hss}\vrule}%
\halign{\vrule height12bp depth 3bp\pol#&&\pol#\cr
\noalign{\hrule}
7 & 1 & 2 \cr\noalign{\hrule}
1 & 6 & 3 \cr\noalign{\hrule}
2 & 3 & 5 \cr\noalign{\hrule}
}}
$$
Zdôvodníme ďalej, že viac ako šesť rôznych čísel tabuľka obsahovať
nemôže.

V~tabuľke sa môžu nachádzať iba celé čísla od $1$ do $8$.
Keby v~nej totiž bolo číslo väčšie ako 8,
súčet čísel v riadku s týmto číslom by bol aspoň
$9+1+1$, čo je viac ako~10.

Vieme teda, že keby bolo v~tabuľke aspoň sedem rôznych čísel,
niektorým z nich by muselo byť jedno z čísel 7 alebo 8. Obe
možnosti teraz posúdime oddelene.

\smallskip
\item{$\triangleright$} Prítomnosť čísla 8 v~tabuľke znamená, že
v~riadku a~stĺpci s~číslom~$8$
je po dvoch číslach~$1$, takže v~celej tabuľke je
najviac 6~rôznych čísel (okrem $8$ a~$1$ to môžu byť
len čísla zo 4 políčok mimo spomínaného riadka a~stĺpca).

\item{$\triangleright$} Prítomnosť čísla $7$ v~tabuľke znamená, že
v~riadku aj stĺpci s~číslom $7$ sú v~oboch prípadoch čísla
$1$ a~$2$.
Z~ostatných štyroch čísel tabuľky označíme ako $x$
to z~nich, ktoré leží v~riadku jedného z~dvoch spomínaných čísel 1
a~súčasne v~stĺpci druhého z~nich-- pozri tabuľku, v ktorej
sú zvyšné tri čísla označené krúžkami.
$$\vbox{\def\pol#1{\hbox to 15 bp{\hss$#1$\hss}\vrule}%
\halign{\vrule height12bp depth 3bp\pol#&&\pol#\cr
\noalign{\hrule}
1 & 7 & 2 \cr\noalign{\hrule}
{\bullet} & 2 & \circ \cr\noalign{\hrule}
x & 1 & {\bullet} \cr\noalign{\hrule}
}}$$
V~riadku aj stĺpci s~číslami $x$ a~$1$ musí ako tretie byť číslo $9-x$.
Dva plné krúžky v~tabuľke tak označujú
rovnaké číslo, teda v~tabuľke je opäť najviac šesť rôznych čísel.

\zaver
Najväčší možný počet rôznych čísel v~tabuľke je 6.

\ineriesenie
Odlišným spôsobom zdôvodníme, prečo tabuľka nemôže
obsahovať sedem alebo viac rôznych čísel.

Súčet všetkých čísel v~tabuľke je zrejme $3\cdot10=30$. Súčet
siedmich {rôznych} kladných celých čísel je aspoň
$1+2+\ldots+7=28$. Ak teda nájdeme v~tabuľke
sedem rôznych čísel, {budú mať zvyšné dve čísla súčet
najviac $30-28=2$, takže sa budú obe rovnať číslu 1 a~ostatných
sedem rôznych čísel v~tabuľke so súčtom 28 budú nutne všetky
čísla od 1 do 7.}
Zostáva tak posúdiť, či je možné tabuľku správne vyplniť
číslami $1,1,1,2,3,4,5,6,7$. Na to už stačí len využiť
rovnaký poznatok ako v~prvom riešení:
V~riadku aj stĺpci s~číslom~$7$ sú v~oboch prípadoch
čísla $1$ a~$2$. To však odporuje tomu, že číslo 2
má byť v~tabuľke iba raz.

\poznamka
Ukážme, že existenciu vyhovujúcej tabuľky s číslami
$1,1,1,2,3,4,5,6,7$ môžeme vylúčiť aj inak.

V~žiadnom riadku ani stĺpci takejto tabuľky nemôžu byť dve čísla
$1$, pretože $1+1+7<10$. Preto v~každom riadku aj stĺpci je po
jednom čísle $1$. Akékoľvek číslo $x\in\{2,3,\dots,7\}$
preto má vo svojom riadku aj stĺpci jedno číslo $1$, teda
tretím číslom tam je v~oboch prípadoch číslo $9-x$.
Žiadne číslo $9-x$ však nie je k~dispozícii dvakrát.
$$\vbox{\def\pol#1{\hbox to 15 bp{\hss$#1$\hss}\vrule}%
\halign{\vrule height12bp depth 3bp\pol#&&\pol#\cr
\noalign{\hrule}
\circ & 1 & \circ \cr\noalign{\hrule}
\circ & {\bullet} & 1 \cr\noalign{\hrule}
1 & x & {\bullet} \cr\noalign{\hrule}
}}$$


\schemaABC
Za úplné riešenie dajte 6 bodov, z toho 2 body za časť a) a~4 body
za časť b). V~prípade neúplných postupov oceňte
čiastkové kroky nasledovne (uvažujeme len tabuľky vyhovujúce zadaniu):

\smallskip
\item{A1.} Príklad tabuľky so šiestimi rovnakými číslami: 1 bod.
\item{A2.} Zdôvodnenie, prečo tabuľka nemôže obsahovať viac ako šesť rovnakých čísel: 1 bod.

\smallskip\noindent
Celkom za časť a) potom dajte $\rm A1+A2$ bodov.

\smallskip
\item{B1.} Príklad tabuľky so šiestimi rôznymi číslami: 1 bod.
\item{B2.} Zdôvodnenie, prečo tabuľka nemôže obsahovať viac ako sedem rôznych čísel: 1 bod.
\item{B3.} Zdôvodnenie, prečo tabuľka s~číslom 8 obsahuje najviac šesť rôznych čísel: 1 bod.
\item{B4.} Zdôvodnenie, prečo tabuľka s~číslom 7 obsahuje najviac šesť rôznych čísel: 2 body.
\item{B5.} Zdôvodnenie, prečo tabuľka s~(aspoň) siedmimi rôznymi číslami by musela byť vyplnená číslami 1, 1, 1, 2, 3, 4, 5, 6, 7: 2 body.
\item{B6.} Zdôvodnenie, prečo tabuľka nemôže obsahovať viac ako šesť rôznych čísel: 3 body.

\smallskip\noindent
Celkom za časť b) potom dajte $\rm B1+\max(B2,B3,B4,B5,B6)$ bodov.\endschema
}

{%%%%%   C-II-1
Medzi každými šiestimi po sebe idúcimi číslami sú práve tri párne (tie
do súčtu prispejú trikrát prvočíslom 2) a práve dve deliteľné tromi,
z~ktorých jedno je nepárne (to prispeje prvočíslom 3) a
druhé je párne (jeho príspevok prvočíslom 2 už sme započítali).
Zvyšné dve z týchto šiestich čísel prispejú prvočíslami $p$ a $q$,
z ktorých každé je väčšie ako 3.

a) Jediné prvočísla $p$, $q$ väčšie ako 3, pre ktoré platí
$2+2+2+3+p+q=23$, čiže $p+q=14$, sú $p=q=7$.
To by však znamenalo, že medzi šiestimi po sebe idúcimi číslami
sú dve deliteľné siedmimi. Takáto situácia nastať nemôže,
pretože dva rôzne násobky siedmich sa líšia najmenej o sedem.
Súčet $23$ nám teda vyjsť nemôže.

b) Áno, môže. Vyhovuje napríklad šestica čísel
$121$, $122$, $123$, $124$, $125$ a $126$,
ktorých najmenšie prvočíselné delitele sú postupne
$11$, $2$, $3$, $2$, $5$ a $2$, takže naozaj
dávajú požadovaný súčet $25$.

\poznamka
Opíšeme, ako hľadať čísla v~časti b). Využijeme na to
úvahy z~úvodu nášho riešenia. Jediné prvočísla $p$, $q$ väčšie ako 3,
pre ktoré platí $2+2+2+3+p+q=25$, čiže $p+q=16$, sú $p=5$ a $q=11$
(alebo naopak). V~hľadanej šestici sa tak musí nachádzať číslo,
ktorého najmenší prvočíselný deliteľ je $11$.

Ak zaradíme do šestice rovno číslo $11$, budú v~nej
z~násobkov $5$ prítomné čísla $10$ alebo~$15$
(môžu tam byť aj obe, ale žiadne iné). Ani jedno z týchto dvoch čísel
nám ale nemôže prispieť prvočíslom $5$, pretože majú
menšie prvočíselné delitele $2$, resp. $3$.

Druhé najmenšie číslo, ktoré môže prispieť prvočíslom $11$,
je $11^2=121$. V~jeho \uv{okolí} už nájdeme vyhovujúcich
šesť čísel. Keďže $119=7\cdot17$, toto číslo zaradiť
nemôžeme. Prichádza tak do úvahy šestica začínajúca číslom $120$
alebo šestica začínajúca číslom $121$. Tieto dve šestice (obe
s~číslom $125=5^3$) vyhovujú.

Ďalej časti b) vyhovujú napríklad všetky šestice obsahujúce
ako číslo $143=11\cdot13$, tak aj číslo $145=5\cdot29$.


\schemaABC
Za úplné riešenie dajte 6 bodov. Časť a) je hodnotená 3 bodmi,
časť b) tiež 3 bodmi. Neúplné riešenia hodnoťte nasledovne:

\smallskip
\item{A1.} Konštatovanie, že v~každej šestici po sebe idúcich čísel sa nachádzajú tri párne čísla, ktoré prispejú trikrát prvočíslom $2$: 1 bod.
\item{A2.} Konštatovanie, že v~každej šestici po sebe idúcich čísel sa nachádza jedno nepárne číslo deliteľné tromi, ktoré prispeje prvočíslom $3$: 1 bod.
\item{A3.} Zdôvodnenie, prečo prípad $p=q=7$ nie je možný: 1 bod.
\item{B1.} Nájdenie ľubovoľnej vyhovujúcej šestice v~časti b): 3 body.\hfil\break
Tolerujte, ak je namiesto overenia iba konštatované, že navrhnutá šestica \uv{zrejme} vyhovuje.
\item{B2.} Hľadanie vyhovujúcej šestice v~časti b) v~\uv{okolí} čísla $121$ (prípadne inej vhodnej mocniny prvočísla 11 alebo 5), ktoré kvôli numerickej chybe skončí neúspešne: 2 body.
\item{B3.} Pozorovanie, že v~časti b) je potrebné nájsť prvočísla $p$, $q$ väčšie ako 3 také, že $p+q=16$: 1 bod.

\smallskip\noindent
Celkom dajte $\rm A1 + A2 +A3 +\max(B1,B2,B3)$ bodov.
Pokiaľ riešiteľ postupuje nami neuvedeným spôsobom, hodnoťte jeho
čiastkové kroky obdobne.
\endschema
}

{%%%%%   C-II-2
Áno, je to možné. Jednotlivé kroky pre štvorice čísel zapísané v~stĺpcoch sú znázornené šípkou.
Preškrtnuté číslo je vždy nahradené súčinom dvoch podfarbených čísel.
%%% makro pre C-II-2 rocnik 73
\def\cancel#1#2{\setbox0=\hbox{#2}%
    \dimen0=\wd0 \dimen1=\ht0 \advance\dimen1 by\dp0
    \hbox{\rlap{#2}\vbox to0pt{\kern\dp0 \csname cancel:\string#1\endcsname \vss}\kern\wd0}%
}
{\lccode`\?=`\p \lccode`\!=`\t \lowercase{\gdef\ignorept#1?!{#1}}}
\def\cancelB#1{\expandafter\ignorept\the\dimen#1 }
\sdef{cancel:/}{\pdfliteral{q 1 w 1 J \cancelcolor\space 0 0 m \cancelB0 \cancelB1 l S Q}}
%\def\cancelcolor{1 0 0 RG}
\def\cancelcolor{0 0 1 RG}
%\def\green{\setrgbcolor{0 1 0}}
\def\podfarbi#1{\setbox0=\hbox{#1}\leavevmode
{\localcolor\rlap{\Yellow\strut\vrule width\wd0}\box0}}
%\newcommand{\g}[1]{\colorbox{lightgray}{#1}}
\let\g=\podfarbi
\def\ra{\rightarrow}
\def\c#1{\cancel /{$#1$}}
$$
\vbox
{\halign{#&&\hbox to 26pt{\hss#\unskip\hss}\cr
     \g{$\sqrt{11}$} & & \g{$\sqrt{11}$} & & $\sqrt{11}$ & & $\c{\sqrt{11}}$ & $\ra $ & \g{$132$} & & \g{$132$} & & $132$ \cr
     \g{$\sqrt{12}$} & & \g{$\sqrt{12}$} & & $\c{\sqrt{12}}$ & $\ra$ & $132$ & & \g{ $132$} & & \g{$132$} & & $132$ \cr
     $\sqrt{13}$ & & $\c{\sqrt{13}}$ & $\ra$ & \g{$\sqrt{132}$} & & \g{$\sqrt{132}$} & & $\sqrt{132}$ & & $\c{\sqrt{132}}$ & $\ra$ & $132^2$ \cr
     $\c{\sqrt{14}}$ & $\ra$ & $\sqrt{132}$ & & \g{$\sqrt{132}$} & & \g{$\sqrt{132}$} & & $\c{\sqrt{132}}$ & $\ra$ & $132^2$ & & $132^2$ \cr
}}
$$

\schemaABC
Plný počet 6 bodov dajte tým riešeniam,
z ktorých je jasné, ako máme každý krok vykonať,
aby sme nakoniec dostali štvoricu celých čísel.
Neúplné riešenia hodnoťte nasledovne:

\smallskip
\item{A1.} Tvrdenie, že celé čísla sú dosiahnuteľné, bez snahy vysvetliť, akým spôsobom: 0 bodov.
\item{A2.} Rozklad čísel na prvočinitele a/alebo ich čiastočné odmocnenie: 0 bodov.
\item{A3.} Opis série krokov, po ktorých získame dve rovnaké čísla: 1 bod.
\item{A4.} Opis série krokov, po ktorých získame jedno celé číslo a tri necelé čísla: 2 body.
\item{A5.} Opis série krokov, po ktorých získame dve celé čísla a dve necelé čísla: 4 body.
\item{A6.} Hypotéza, že štyri celé čísla sú dosiahnuteľné, so slovným opisom krokov, z ktorého je jasný zámer, avšak v nejakom kroku tento opis nemá jednoznačnú interpretáciu a niektorá z možných interpretácií nevedie k cieľu: 5 bodov.
\item{B1.} Myšlienka, že akonáhle získame dve \emph{celé} čísla, ľahko potom dôjdeme k~cieľu: 2~body.
\item{B2.} Myšlienka, že akonáhle získame dve \emph{rovnaké} čísla, ich súčin bude celé číslo: 2 body.

\smallskip\noindent
Celkom potom dajte
$\rm\max\bigl(B1+\max(B2,A3,A4),A5,A6\bigr)$ bodov.
Čiastočné body dajte aj v prípade, že v~riešení je
konštatované, že všetky štyri celé čísla sú (asi)
nedosiahnuteľné. Pokiaľ však riešiteľ konštatuje, že
nedosiahnuteľnosť štyroch celých čísel svojim postupom dokázal,
dajte najviac 3~body. Ak postupuje riešiteľ nami neuvedeným
spôsobom, hodnoťte jeho čiastkové kroky obdobne.
\endschema
}

{%%%%%   C-II-3
Úloha úzko nadväzuje na úlohu z~domáceho kola s rovnako zadanými
obdĺžnikmi $ABCD$ a $KLMN$. V~priebehu jej riešenia sme odvodili
poznatok, že štyri pravouhlé trojuholníky $ANK$, $DMN$, $CLM$, $BKL$
sú navzájom podobné, pritom každý z nich
má dĺžky odvesien v~pomere $5:1$,
konkrétne $|AN|:|AK|=5:1$. Riešitelia krajského kola sa môžu
na tento výsledok odvolať, uveďme však
aj teraz jeho odvodenie podľa \obr{}. Postup sa nám
totiž bude hodiť na obdobné posúdenie dvojice obdĺžnikov
$KLMN$ a $WXYZ$.
\inspsc{c73i.61}{.8333}%

Avizovaná podobnosť pravouhlých trojuholníkov $ANK$, $DMN$, $CLM$ a $BKL$,
ktoré \uv{obklopujú} obdĺžnik $KLMN$, vyplýva podľa vety $uu$
zo zhodností ich vnútorných uhlov. Napríklad vyznačené
uhly $KNA$ a~$DNM$ sa dopĺňajú do $90^{\circ}$, rovnako ako uhly
$DNM$ a~$NMD$, odkiaľ dostávame zhodnosť uhlov $KNA$ a~$NMD$.
Analogicky sa zdôvodnia aj ostatné potrebné zhodnosti uhlov.

Ďalej sme v~riešení úlohy určili pomery podobností
spomínaných štyroch trojuholníkov: trojuholníky $ANK$ a $CLM$ sú dokonca zhodné
(majú totiž zhodné prepony), oproti nim majú trojuholníky $DMN$ a $BKL$
strany trikrát dlhšie (trikrát dlhšie sú totiž podľa zadania
ich prepony). Pri voľbe jednotky dĺžky $1=|AK|$ a označení
$x=|AN|$ tak máme $|CM|=1$, $|DN|=3$ a $|DM|=3x$, a teda
$|AB|=|CD|=3x+1$ a~$|BC|=|AD|=x+3$. Dosadením do zadaného
pomeru $|AB|:|BC|=2:1$ dostaneme rovnicu $(3x+1)=2(x+3)$
s~jediným riešením $x=5$. Preto naozaj platí napríklad
$|AN|:|AK|=5:1$. Týmto uzatvárame pripomenutie
postupu z~domáceho kola.

Teraz vyššie vykonané úvahy využijeme pre dvojicu obdĺžnikov
$KLMN$ a~$WXYZ$. Znovu tu tak máme štvoricu
podobných pravouhlých trojuholníkov $KWZ$, $LXW$, $MYX$ a $NZY$
\uv{obklopujúcich} obdĺžnik $WXYZ$. Hľadaný pomer $|XY|:|YZ|$,
zapísaný ako pomer $|WZ|:|XW|$, je tak vlastne pomerom
dĺžok prepôn dvoch podobných trojuholníkov $KWZ$ a $LXW$.
Nájdeme ho ďalej ako pomer $|KZ|:|LW|$
dĺžok zodpovedajúcich si odvesien týchto dvoch trojuholníkov.
\inspsc{c73iii.31}{.8333}%

Zo zadania vyplýva, že strana $AB$ obdĺžnika $ABCD$ musí byť
rovnobežná so stranou~$ZW$ obdĺžnika $WXYZ$, pretože je zrejme vylúčené, aby
platilo $AB\parallel ZY$. Preto sú striedavé uhly $KWZ$ a~$BKL$ zhodné,
takže pravouhlé trojuholníky $KWZ$ a~$BKL$ sú podobné podľa vety $uu$.
Na obrázku je tak dokonca osem navzájom podobných pravouhlých
trojuholníkov, každý s~dĺžkami odvesien v~skôr určenom pomere $5:1$.

Pri označení $y=|KZ|$ a $z=|LW|$ tak máme $|KW|=5y$
a $|LX|=5z$. To dáva $|KL|=5y+z$ a $|LM|=5z+y$ (lebo
$|MX|=|KZ|$ zo zhodnosti trojuholníkov $MYX$ a $KWZ$).
Podmienku $|KL|:|LM|=3:1$ teda môžeme prepísať ako rovnosť
$5y+z=3(5z+y)$, z ktorej ľahko vyplýva $y=7z$. Podľa našich úvah
tak pre hľadaný pomer vychádza
$$
|XY|:|YZ|=|WZ|:|XW|=|KZ|:|LW|=y:z=7:1.
$$

\schemaABC
Za úplné riešenie dajte 6 bodov. Za také treba považovať aj
riešenie, keď poznatky z~A1 a A3 sú označené za známe (z~domáceho kola)
a poznatok z~A2 za analógiu poznatku z~A1 (existujú však úplné
postupy, ktoré poznatok z~A3 nevyužívajú priamo, ale skryto
v~nejakej rovnici). Neúplné riešenia hodnoťte nasledovne, pritom body za
A1, A2 a A3 dajte aj v~prípade opísanom v~predchádzajúcej vete:

\smallskip
\item{A1.} Dôkaz podobnosti niektorých dvoch susedných trojuholníkov $ANK$, $BKL$, $CLM$, $DMN$ (obklopujúcich obdĺžnik $KLMN$): 1 bod.
\item{A2.} Dôkaz podobnosti niektorých dvoch susedných trojuholníkov $KWZ$, $LXW$, $MYX$, $NZY$ (obklopujúcich obdĺžnik $WXYZ$): 1 bod.
\item{A3.} Podložený výpočet pomeru dĺžok niektorých dvoch strán jedného trojuholníka z~A1 (najčastejšie odvesien, keď vyjde $5:1$): 2 body.
\item{A4.} Dôkaz podobnosti dvoch trojuholníkov, z~ktorých jeden je uvedený v~A1 a druhý v~A2: 1 bod.
\item{B1.} Podložené zostavenie jednej alebo viacerých lineárnych rovníc, ktoré umožňujú spočítať pomer podobnosti medzi väčšími a menšími trojuholníkmi z~A2: 5 bodov. \hfil\break
Nestačí teda vypísať napríklad sústavu kvadratických rovníc, ktoré možno získať bez použitia podobnosti použitím Pytagorovej vety pre rôzne zastúpené pravouhlé trojuholníky.

\smallskip\noindent
Celkom potom dajte
$\rm\max\bigl(\max(A1,A2,A3)+A4,\,B1\bigr)$
bodov. Tolerujte, ak riešiteľ zdôvodní podobnosť nejakej
dvojice trojuholníkov a potom podobnosti, ktoré možno zdôvodniť rovnakým
postupom, označí za analogické. Ak postupuje riešiteľ nami neuvedeným
spôsobom, hodnoťte jeho čiastkové kroky obdobne.
\endschema
}

{%%%%%   C-II-4
Všimnime si, že vďaka rovnostiam zo zadania platí
$$
(a+b)^2=(a^2+b^2)+2ab=(c^2+d^2)+2cd=(c+d)^2.
$$
Keďže obe čísla $a+b$, $c+d$ sú kladné,
po odmocnení dostávame $a+b=c+d$.

Pri podobnej úvahe pre rozdiely $a-b$ a $c-d$ musíme
byť o~niečo opatrnejší. Z~rovností
$$
(a-b)^2=(a^2+b^2)-2ab=(c^2+d^2)-2cd=(c-d)^2
$$
po odmocnení tentoraz dostávame $|a-b|=|c-d|$, takže nastane
aspoň jeden z~prípadov (i) $a-b=c-d$ alebo (ii) $a-b=d-c$.

V obidvoch prípadoch môžeme získanú rovnosť sčítať s rovnosťou
$a+b=c+d$, čím dostaneme v~prípade (i) rovnosť $2a=2c$, t.\,j. $a=c$,
zatiaľ čo v~prípade (ii) vyjde $2a=2d$, t.\,j. $a=d$.
Z~rovnosti $a+b=c+d$ potom v~prípade (i) vďaka $a=c$ platí $b=d$,
zatiaľ čo v~prípade (ii) vďaka $a=d$ platí $b=c$.
V~každom prípade sa tak dvojica $a$, $b$ v~niektorom poradí rovná
dvojici $c$, $d$, a preto každá vyhovujúca štvorica
musí byť v tvare $(a,b,a,b)$ alebo $(a,b,b,a)$.

Na druhej strane všetky štvorice $(a,b,a,b)$ aj $(a,b,b,a)$
zrejme spĺňajú obe rovnosti zo zadania. Zostáva preto
určiť počet týchto štvoríc. Ak platí $a=b$, jedná sa o~štvorice
toho istého tvaru $(a,a,a,a)$ a tých je $1\,000$.
Ak naopak $a\ne b$, sú štvorice $(a,b,a,b)$ a~$(a,b,b,a)$ rôzne.
Keďže máme $1\,000$ možností pre výber $a$ a potom 999 možností
pre výber $b$, počet štvoríc, v ktorých $a\ne b$, je rovný
$2\cdot1\,000\cdot999=1\,998\,000$. Celkom tak existuje
práve $1\,999\,000$ vyhovujúcich štvoríc.

\ineriesenie
Z~prvej zadanej rovnosti vyjadríme $d=ab/c$, čo dosadíme do druhej
rovnosti, ktorú ďalej upravíme nasledovne:
$$\eqalign{
a^2+b^2&=c^2+(ab/c)^2\quad|\cdot c^2,\cr
c^2(a^2+b^2)&=c^4+a^2b^2,\cr
c^2a^2+c^2b^2&=c^4+a^2b^2,\cr
c^2a^2-c^4&=a^2b^2-c^2b^2,\cr
c^2\bigl(a^2-c^2\bigr)&=b^2\bigl(a^2-c^2\bigr),\cr
\bigl(a^2-c^2\bigr)\bigl(c^2-b^2\bigr)&=0.
}$$
Keďže čísla $a$, $b$, $c$, $d$ sú kladné, podľa
odvodenej rovnosti sa číslo $c$ rovná niektorému z~čísel $a$ alebo
$b$. Tomu druhému sa vďaka $d=ab/c$ potom rovná číslo $d$.

Znova sme došli k~záveru, že každá vyhovujúca štvorica musí byť
tvaru $(a,b,a,b)$ alebo $(a,b,b,a)$, pričom akákoľvek
taká štvorica zadaniu zrejme vyhovuje. Štvoríc tvaru $(a,b,a,b)$
je $1000^2$ (pripúšťame aj možnosť $a=b$). Rovnako tak štvoríc
tvaru $(a,b,b,a)$ je $1000^2$ (pripúšťame aj možnosť $a=b$).
Keďže však v~súčte $1000^2+1000^2$ je každá z~$1000$ štvoríc
$(a,a,a,a)$ (možnosť $a=b$) započítaná dvakrát,
vyhovujúcich štvoríc je práve $2\cdot 1000^2-1000=1\,999\,000$.

\poznamka
Zdôvodnenie, prečo každá vyhovujúca štvorica $(a,b,c,d)$ musí
byť v tvare $(a,b,a,b)$ alebo $(a,b,b,a)$, možno podať
tiež geometricky. Za tým účelom uvážime dva pravouhlé trojuholníky,
jeden s odvesnami dĺžok
$a$ a $b$, druhý s odvesnami dĺžok $c$ a~$d$. Podľa prvej zadanej
rovnosti majú tieto dva trojuholníky rovnaké obsahy, podľa druhej rovnosti
majú zhodné prepony. Majú tak aj zhodné výšky na preponu.
Využime teraz známu konštrukciu pravouhlého trojuholníka
podľa zadanej prepony a zadanej výšky. Vďaka súmernosti použitej
Tálesovej kružnice podľa osi prepony sú naše dva trojuholníky zhodné.
Preto platí $(a,b)=(c,d)$ alebo $(a,b)=(d,c)$, ako sme chceli
ukázať.
\inspsc{c73iii.41}{.8333}%

\schemaABC
Za úplné riešenie dajte 6 bodov. Za také považujte aj riešenia,
v ktorých po odvodení jediných možných tvarov $(a,b,a,b)$
a $(a,b,b,a)$ chýba zmienka o~tom, že všetky také štvorice
rovnosti zo zadania
zrejme spĺňajú. Neúplné riešenia hodnoťte nasledovne:

\smallskip
\item{A0.} Vyjadrenie jedného z~čísel $a$, $b$, $c$, $d$ z~jednej rovnosti a dosadenie do druhej rovnosti: 0 bodov.
\item{A1.} Udeľte 2 body, ak riešiteľ splní čokoľvek z~nasledujúceho:
\itemitem{$\triangleright$} Odvodenie niektorej z~rovností $(a+b)^2=(c+d)^2$, $a+b=c+d$ alebo $(a-b)^2=(c-d )^2$.
\itemitem{$\triangleright$} Pozorovanie, že ak pevne zvolíme hodnotu súčinu $ab$, hodnota $a^2+b^2$ klesá s~tým, čím bližšie sú čísla $a$ a $b$ k~sebe.
\itemitem{$\triangleright$} Označenie $A=a^2$, $B=b^2$, $C=c^2$, $D=d^2$ a prevedenie pôvodných rovností na rovnosti $AB=CD$ a $A+B=C+D$.
\item{A2.} Odvodenie oboch rovností $(a+b)^2=(c+d)^2$ a $(a-b)^2=(c-d)^2$: 3 body.
\item{A3.} Vyjadrenie jednej neznámej z~rovnice $ab=cd$ a dosadenie do druhej rovnice s~vyjadreným zámerom riešiť kvadratickú rovnicu pre druhú mocninu vhodnej neznámej (v~našom druhom riešení sa jedná o~kvadratickú rovnicu pre $c^2$ s~koreňmi $a^2$, $b^2$): 3 body.
\item{A4.} Zdôvodnenie, prečo niektoré z~čísel $a$, $b$ je rovné niektorému z~čísel $c$, $d$, prípadne odvodenie rovnosti, z ktorej to ihneď vyplýva, napríklad rovnosti $(c^2-a^2)(c^2-b^2)=0$: 4~body.\hfil\break
Tolerujte pritom, keď riešiteľ odvodí (typicky zlým odmocnením rovnosti $(a-b)^2=(c-d)^2$) iba jeden z dvoch možných prípadov. Tento krok je možné tiež splniť vyriešením kvadratickej rovnice z~A3.
\item{A5.} Dôkaz, že vyhovujúca štvorica musí byť v tvare $(a,b,a,b)$ alebo $(a,b,b,a)$ (nezáleží, či je záver sformulovaný pomocou štvoríc alebo je zapísaná alebo slovne popísaná rovnosť $\{a,b\}=\{c,d\}$):~5~bodov.\hfil\break
Tento dôkaz môže byť vykonaný postupmi z oboch riešení aj geometrickým postupom z poznámky.
\item{B1.} Pozorovanie, že rovnostiam zo zadania vyhovujú nielen štvorice tvaru $(a,b,a,b)$, ale aj štvorice tvaru $(a,b,b,a)$: 1 bod.
\item{B2.} Správne vyčíslenie počtu štvoríc tvarov $(a,b,a,b)$ a $(a,b,b,a)$ spolu: 1 bod.

\smallskip\noindent
\noindent Celkom potom dajte
$\rm\max(A1, A2, A3, A4, A5, B1) + B2$ bodov. Ak postupuje riešiteľ nami
neuvedeným spôsobom, hodnoťte jeho čiastkové kroky obdobne.

\endschema
}

{%%%%%   vyberko, den 1, priklad 1
...}

{%%%%%   vyberko, den 1, priklad 2
...}

{%%%%%   vyberko, den 1, priklad 3
...}

{%%%%%   vyberko, den 1, priklad 4
...}

{%%%%%   vyberko, den 2, priklad 1
...}

{%%%%%   vyberko, den 2, priklad 2
...}

{%%%%%   vyberko, den 2, priklad 3
...}

{%%%%%   vyberko, den 2, priklad 4
...}

{%%%%%   vyberko, den 3, priklad 1
...}

{%%%%%   vyberko, den 3, priklad 2
...}

{%%%%%   vyberko, den 3, priklad 3
...}

{%%%%%   vyberko, den 3, priklad 4
...}

{%%%%%   vyberko, den 4, priklad 1
...}

{%%%%%   vyberko, den 4, priklad 2
...}

{%%%%%   vyberko, den 4, priklad 3
...}

{%%%%%   vyberko, den 4, priklad 4
...}

{%%%%%   vyberko, den 5, priklad 1
...}

{%%%%%   vyberko, den 5, priklad 2
...}

{%%%%%   vyberko, den 5, priklad 3
...}

{%%%%%   vyberko, den 5, priklad 4
...}

{%%%%%   trojstretnutie, priklad 1
...}

{%%%%%   trojstretnutie, priklad 2
...}

{%%%%%   trojstretnutie, priklad 3
...}

{%%%%%   trojstretnutie, priklad 4
...}

{%%%%%   trojstretnutie, priklad 5
...}

{%%%%%   trojstretnutie, priklad 6
...}

{%%%%%   IMO, priklad 1
...}

{%%%%%   IMO, priklad 2
...}

{%%%%%   IMO, priklad 3
...}

{%%%%%   IMO, priklad 4
...}

{%%%%%   IMO, priklad 5
...}

{%%%%%   IMO, priklad 6
...}

{%%%%%   MEMO, priklad 1
...}

{%%%%%   MEMO, priklad 2
...}

{%%%%%   MEMO, priklad 3
...}

{%%%%%   MEMO, priklad 4
...}

{%%%%%   MEMO, priklad t1
...}

{%%%%%   MEMO, priklad t2
...}

{%%%%%   MEMO, priklad t3
...}

{%%%%%   MEMO, priklad t4
...}

{%%%%%   MEMO, priklad t5
...}

{%%%%%   MEMO, priklad t6
...}

{%%%%%   MEMO, priklad t7
...}

{%%%%%   MEMO, priklad t8
...} 