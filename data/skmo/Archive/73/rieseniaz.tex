{%%%%%   Z4-I-1
...}

{%%%%%   Z4-I-2
...}

{%%%%%   Z4-I-3
...}

{%%%%%   Z4-I-4
...}

{%%%%%   Z4-I-5
...}

{%%%%%   Z4-I-6
...}

{%%%%% Z5-I-1
Všetko budeme počítať v~decimetroch.
Preteky sú dlhé 20\,240\,dm a~opísaný zajacov trojskok meria $35 + 15 + 61 = 111$\,dm.
Delením so zvyškom zistíme, že
$$
20\,240 = 182 \cdot 111 + 38 .
$$
Teda po 182 trojskokoch zostáva zajacovi do cieľa 38\,dm.

Ďalší odraz vychádza na ľavú nohu, skočí 35\,dm a do cieľa zostávajú 3\,dm.
Ďalší odraz vychádza na pravú nohu a týmto skokom preskočí cieľom.

Celkom zajac urobí $182\cdot3 + 2 = 548$ skokov, pred cieľovým skokom sa odrazí pravou nohou.
}

{%%%%% Z5-I-2
V prvej vrstve bolo 16 kociek.
Z~uvedených pohľadov vyplýva, že v tretej vrstve boli 2 kocky.
Počet kociek v druhej vrstve nie je možné jednoznačne určiť, najviac ich však mohlo byť $3\cdot3=9$.
Táto možnosť je znázornená na nasledujúcom obrázku, v~ktorom dva zadané priemety zodpovedajú pohľadu spredu a~sprava:
\insp{z5-I-2a.eps}%


\noindent
Naozaj nie je možné pridať jedinú kocku, aby sa nezmenil niektorý z~daných priemetov.
Zuzka použila najviac $16+9+2 =27$ kociek.

K najmenšiemu možnému počtu kociek je možné dospieť tak, že sa z druhej vrstvy postupne odoberá čo najviac kociek bez toho, aby sa zmenil niektorý z daných priemetov.
Jedna taká možnosť vyzerá takto:
\insp{z5-I-2b.eps}%


\noindent
Menej ako štyri kocky v druhej vrstve byť nemôžu.
Dve kocky, na ktorých stoja kocky z tretej vrstvy, sú pri pohľade sprava v zákryte.
Preto musia byť použité ďalšie dve kocky, aby tento priemet súhlasil so zadaním.
Zuzka použila najmenej $16+4+2 =22$ kociek.

\poznamka
Pri možnosti s najmenším počtom kociek môžu byť kocky v druhej vrstve rozmiestnené rôzne, avšak nie ľubovoľne.
Iné možné rozmiestnenie je toto:
\insp{z5-I-2c.eps}%


Úpravami v druhej vrstve je možné mať stavbu s akýmkoľvek počtom kociek v~rozsahu od 22 do 27 kociek.
Tu je jedna možnosť postavená z ~23 kociek:
\insp{z5-I-2d.eps}%


}

{%%%%% Z5-I-3
Záhony s rovnakou zeleninou označíme rovnakým písmenom, záhony s~rôznymi zeleninami rôznymi písmenami.
Aby záhony s rovnakou zeleninou nesusedili, musia byť vysadené takto:
\insp{z5-I-3a.eps}%


Do záhonov A~môže Katka zasadiť ľubovoľnú z troch plodín, do záhonov B ľubovoľnú zo zvyšných dvoch plodín a do C poslednú zvyšnú plodinu.
Katka teda môže záhony vysadiť $3\cdot2=6$ spôsobmi.

\poznamka
Všetky možné priradenia plodín záhonom sú vypísané tu:
$$
\begintable
   cesnak\|A|A\|C|C\|B|B\cr
    mrkva\|B|C\|A|B\|C|A\cr
reďkvička\|C|B\|B|A\|A|C\endtable
$$
Ak je cesnak na záhone A, potom mrkva môže byť buď na záhone B, alebo C, a reďkvička na zostávajúcom záhone.
Takto je postupne vyčerpaných všetkých $3\cdot2=6$ možností.
Každé dva susedné stĺpce v~tabuľke sa líšia prehodením dvoch písmen a~žiadny stĺpec sa neopakuje.
}

{%%%%% Z5-I-4
Pred dažďom mal pán Malina vo svojom sude $3\cdot16=48$ litrov vody, čo je o~32 litrov viac než v~sude pána Jahodu.

Do oboch sudov napršalo rovnako, teda rozdiel množstva vody v sudoch po daždi bol opäť 32 litrov ako pred dažďom.
Po daždi bolo v~sude pána Malinu dvakrát viac vody ako v~sude pána Jahodu, teda v Jahodovom sude bolo 32 a v~Malinovom sude 64 litrov vody.

Do každého suda napršalo 16 litrov vody ($48-32=64-48=16$).

\poznamka
Úlohu je možné znázorniť pomocou úsečiek
\insp{z5-I-4.eps}%


\noindent
či zapísať rovnicou
$$
2\cdot16+2n = 3\cdot16 +n,
$$
pričom $n$ označuje množstvo napršanej vody.
}

{%%%%% Z5-I-5
Cena farby v~každej časti zodpovedá počtu štvorcov, ktorým je táto časť spoločná.
To je rovnaké, ako by všetky štvorce boli samostatne vyfarbené farbou rovnakej ceny.

Každý štvorec má obsah $4\cdot4=16\cm^2$, štvorce sú štyri a~1\,cm$^2$ farby stojí 1~euro.
Teda farba na vyfarbenie ornamentu bude stáť $16\cdot4=64$ eur.

\ineriesenie
Rôznymi odtieňmi sivej rozlíšime, koľkým štvorcom sú jednotlivé časti ornamentu spoločné.
Ornament rozdelíme na štvorčeky so stranou 1\,cm:
\insp{z5-I-5a.eps}%


Všetkým štyrom štvorcom je spoločný 1 štvorček,
trom štvorcom prislúcha 6 štvorčekov,
dvom štvorcom 12 štvorčekov a~jednému štvorcu 18.
Teda farba na vyfarbenie ornamentu bude stáť
$$
1\cdot4+6\cdot3+12\cdot2+18\cdot1=64\ \text{eur}.
$$
}

{%%%%%   Z5-I-6
Pre názornosť si úlohu napíšeme ako písomné odčítanie, ktorého výsledok je 28\,926 a kde hviezdičky na každom riadku zastupujú cifry od 1 do 5:
$$
\alggg{&*&*&*&*&* \\ -&*&*&*&*&*}{&2&8&9&2&6}
$$
Cifru na mieste jednotiek je možné dostať jedine takto (s~prechodom cez desiatku):
$$
\alggg{&*&*&*&*&1 \\ -&*&*&*&*&5}{&2&8&9&2&6}
$$
So zvyšnými použiteľnými ciframi možno cifru na mieste desiatok dostať jedine niektorým z nasledujúcich spôsobov:
$$
\alggg{&*&*&*&4&1 \\ -&*&*&*&1&5}{&2&8&9&2&6}
\qquad
\alggg{&*&*&*&5&1 \\ -&*&*&*&2&5}{&2&8&9&2&6}
$$

\begin{enumerate}\alphatrue
\item
V~prvom prípade by lístok musel byť rozstrihnutý takto
$1|234|5$
a~kartičky preskladané $5|234|1$ a~$234|1|5$.
V~tomto prípade rozdiel vyhovuje zadaniu:
$$
\alggg{&5&2&3&4&1 \\ -&2&3&4&1&5}{&2&8&9&2&6}
$$
\item
V druhom prípade by lístok musel byť rozstrihnutý takto
$1|2|34|5$, čo sú tri (a nie dva) strihy.
Tento prípad nevyhovuje zadaniu.
\end{enumerate}

Lucka lístok rozstrihla medzi ciframi 1 a~2 a~medzi ciframi 4 a~5.

\poznamka
Nutnosť strihu medzi ciframi 1 a~2 vyplýva už z~prvého postrehu.
Nutnosť strihu medzi ciframi 4 a~5 (a~následného preskladania) vyplýva z~tejto úvahy:
ak by pred cifrou 5 bola cifra 4, tak by sme museli vedieť doplniť:
$$
\alggg{&*&*&*&*&1 \\ -&*&*&*&4&5}{&2&8&9&2&6}
$$
V~takom prípade by v~menšenci na mieste desiatok musela byť cifra 7, ktorú však nemáme k~dispozícii.

\ineriesenie
Lístok je možné na dvakrát rozstrihnúť šiestimi spôsobmi:
$$
1|2|345,\quad 1|23|45,\quad 1|234|5,\quad 12|3|45,\quad 12|34|5,\quad 123|4|5.
$$
Vzniknuté tri kartičky je možné preskladať šiestimi spôsobmi, čo schematicky (vzostupne) zapíšeme takto:
$$
A|B|C,\quad A|C|B,\quad B|A|C,\quad B|C|A,\quad C|A|B,\quad C|B|A.
$$
Prebratím všetkých možností je možné nájsť rozstrihania a preskladania, ktoré vyhovujú zadaniu.

Namiesto skúmania možných rozdielov (ktorých je pre každé rozstrihanie 15) je možné postupovať tak, že k príslušnej šestici čísel pripočítame požadovaný rozdiel 28926 a overíme, či je medzi výsledkami niektoré číslo z~danej šestice.
Najväčšie číslo, ktoré je možné z~cifier od 1 do 5 vytvoriť, je 54321.
Teda najväčšie číslo, ku ktorému má zmysel požadovaný rozdiel pripočítať, je 25395.
Tým sa skúšanie podstatne obmedzuje a vyzerá nasledovne
(opakujúce sa či vopred zamietnuté možnosti píšeme do zátvoriek):

\begin{enumerate}\alphatrue
\item
Po rozstrihaní $1|2|345$ je možné zostaviť čísla
$$
12345,\quad 13452,\quad 21345,\quad 23451,\quad 34512,\quad 34521.
$$
Po pripočítaní 28926 postupne dostávame
$$
41271,\quad 42378,\quad 50271, \quad 52377,\quad (>54321,\quad\dots)
$$
Žiadna možnosť nevyhovuje.
\item
Po rozstrihaní $1|23|45$ je možné zostaviť čísla
$$
12345,\quad 14523,\quad 23145,\quad 23451,\quad 45123,\quad 45231.
$$
Po pripočítaní 28926 postupne dostávame
$$
(41271),\quad 43449,\quad 52071,\quad (52377,\quad >54321,\quad\dots)
$$
Žiadna možnosť nevyhovuje.
\item
Po rozstrihaní $1|234|5$ je možné zostaviť čísla
$$
12345,\quad 15234,\quad 23415,\quad 23451,\quad 51234,\quad 52341.
$$
Po pripočítaní 28926 postupne dostávame
$$
(41271),\quad 44160,\quad 52341,\quad (52377,\quad >54321,\quad\dots)
$$
Vyhovuje možnosť $23451+28926=52341$.
\item
Po rozstrihaní $12|3|45$ je možné zostaviť čísla
$$
12345,\quad 12453,\quad 31245,\quad 34512,\quad 45123,\quad 45312.
$$
Po pripočítaní 28926 postupne dostávame
$$
(41271),\quad 41379,\quad (>54321,\quad\dots)
$$
Žiadna možnosť nevyhovuje.
\item
Po rozstrihaní $12|34|5$ je možné zostaviť čísla
$$
12345,\quad 12534,\quad 34125,\quad 34512,\quad 51234,\quad 53412.
$$
Po pripočítaní 28926 postupne dostávame
$$
(41271),\quad 41460,\quad (>54321,\quad\dots)
$$
Žiadna možnosť nevyhovuje.
\item
Po rozstrihaní $123|4|5$ je možné zostaviť čísla
$$
12345,\quad 12354,\quad 41235,\quad 45123,\quad 51234,\quad 54123.
$$
Po pripočítaní 28926 postupne dostávame
$$
(41271),\quad 41280,\quad (>54321,\quad\dots)
$$
Žiadna možnosť nevyhovuje.
\end{enumerate}
Jediná vyhovujúca možnosť vychádza v~prípade c):
Lucka lístok rozstrihla medzi ciframi 1 a~2 a~medzi ciframi 4 a~5.

\poznamka
V~prípade a) je rozdiel najväčšieho a~najmenšieho čísla z~danej šestice 22176.
Všetky ostatné rozdiely sú menšie, preto požadovaný rozdiel 28926 dostať nemožno a~nebolo nutné čokoľvek ďalej skúšať.
}

{%%%%%   Z6-I-1
Úlohu môžeme výhodne riešiť odzadu:
\begin{itemize}
\item Pred druhým (posledným) kolom výmen mali Pavol a~Rasťo o~sedem guľôčok menej, zatiaľ čo Jaro o~14 guľôčok viac.
Teda Pavol a~Rasťo mali $25-7=18$ guľôčok, zatiaľ čo Jaro ich mal $25+14=39$.
\item Pred prvým kolom výmen mal Rasťo dvojnásobok guľôčok (polovicu dal Jarovi, zostala mu polovica) a~Pavol tri polovice guľôčok (tretinu dal Jarovi, ostali mu dve tretiny).
Teda Rasťo mal $2\cdot18=36$ guľôčok a~Pavol $\frac32\cdot18=27$ guľôčok.
\item Počas prerozdeľovania bol celkový počet guľôčok stále rovnaký.
Súčet po druhom, resp. prvom kole výmen bol $25+25+25 =18+18+39 =75$ guľôčok.
Pred prvou výmenou (po hre) mal Rasťo 36 a~Pavol 27 guľôčok.
Teda po hre mal Jaro
$$
75-36-27=12\ \text{guľôčok}.
$$
\end{itemize}

\poznamka
Znázornenie predchádzajúcich úvah, príp. kontrola výsledkov môže vyzerať takto:
$$
\begintable
\hfill stavy\|Rasťo|Pavol|Jaro\|výmeny\hfill\crthick
\hfill po 2. kole výmen\|25|25|25\|\crthick
\||$\uparrow 7$|$\downarrow 7$\|Jaro Pavlovi\cr
\|$\uparrow 7$||$\downarrow 7$\|Jaro Rasťovi\hfill\crthick
\hfill po 1. kole výmen\|18|18|39\|\crthick
\|$\downarrow 18$||$\uparrow 18$\|Rasťo Jarovi\hfill\cr
\||$\downarrow 9$|$\uparrow 9$\|Pavol Jarovi\crthick
\hfill po hre\|36|27|12\|\endtable
$$
}

{%%%%% Z6-I-2
Štvoruholníky s rozličnými obsahmi možno rozlíšiť nasledovne:

\smallskip
a) Niektorá uhlopriečka štvoruholníka je rovnobežná so stranou štvorca.
Označme vrcholy štvorca a~štvoruholníka tak, že uhlopriečka $LN$ je rovnobežná so stranami $AB$ a~$CD$:
\insp{z6-I-2a.eps}%


\noindent
Obsah trojuholníka $LNK$ je polovicou obsahu obdĺžnika $LNAB$, obsah trojuholníka $LNM$ je polovicou obsahu obdĺžnika $LNDC$.
Trojuholníky $LNK$ a~$LNM$ tvoria dokopy celý štvoruholník, obdĺžniky $LNAB$ a~$LNDC$ tvoria dokopy daný štvorec.
Obsah štvorca je $6\cdot6=36\,(\Cm^2)$, obsah štvoruholníka $KLMN$ je polovičný:
$$
36:2=18\,(\Cm^2).
$$
Nie je podstatné, či rovnobežné úsečky uvažujeme vodorovne alebo zvisle.
Aj umiestnenie bodov $K$ a~$M$ na stranách štvorca nehrá v~predchádzajúcej úvahe žiadnu podstatnú úlohu.

\smallskip
b) Žiadna uhlopriečka štvoruholníka nie je rovnobežná so stranou štvorca.
V tomto prípade rozlišujeme dve možnosti:
\insp{z6-I-2b.eps}%


\noindent
Naozaj platí, že zámena ktoréhokoľvek vrcholu v ktoromkoľvek z týchto štvoruholníkov dáva štvoruholník, ktorého uhlopriečka je rovnobežná so stranou štvorca (teda prípad diskutovaný vyššie).
Uvedené štvoruholníky majú navzájom rôzne obsahy, a~tie vyjadríme ako rozdiel obsahu celého štvorca a~obsahov štyroch trojuholníkových rožkov.
Obsah štvorca je $6\cdot6=36\,(\Cm^2)$.
Rožky sú trojakého typu a~ich obsahy sú
$$
\frac12\cdot2\cdot2=2\,(\Cm^2),\quad
\frac12\cdot2\cdot4=4\,(\Cm^2),\quad
\frac12\cdot4\cdot4=8\,(\Cm^2).
$$
Obsahy prvého a~druhého štvoruholníka sú
$$
36-4\cdot4 =20\,(\Cm^2),\quad
36-2\cdot2-2\cdot8 =16\,(\Cm^2).
$$

\smallskip
Karolína mohla dostať štvoruholníky s~obsahmi 16, 18 a~20\,(cm$^2$).

\ineriesenie
Na každej strane štvorca Karolína vyberala jeden z dvoch modrých bodov ako vrchol štvoruholníka.
Takto mohla zostrojiť celkom $2\cdot2\cdot2\cdot2=16$ štvoruholníkov.
Mnohé z nich sú však navzájom zhodné, teda majú rovnaké obsahy.
Navzájom nezhodné štvoruholníky, ktoré mohla Karolína zostrojiť, sú štyri:
\insp{z6-I-2c.eps}%


\noindent
Naozaj platí, že zámena ktoréhokoľvek vrcholu v ktoromkoľvek štvoruholníku dáva štvoruholník, ktorý je zhodný s iným z týchto štvoruholníkov.

Obsah každého štvoruholníka je možné vyjadriť ako rozdiel obsahu celého štvorca a~obsahov štyroch trojuholníkových rožkov.
Pri treťom a~štvrtom štvoruholníku odčítame rovnaké rožky (iba iným spôsobom), preto sú obsahy týchto štvoruholníkov rovnaké.
Stačí teda preveriť prvé tri štvoruholníky.
Ich obsahy sú vypočítané v~predchádzajúcom riešení.

Karolína mohla dostať štvoruholníky s~obsahmi 16, 18 a~20\,(cm$^2$).

\poznamky
Prípadné zhodnosti v~predchádzajúcej úvahe patria medzi súmernosti štvorca, t.\,j. zobrazenia, ktoré zachovávajú daný štvorec.
Tie zahŕňajú osové súmernosti, stredovú súmernosť a~otáčanie o~90\st.
Štvorec má celkom osem súmerností.
}

{%%%%% Z6-I-3
Opísané čísla je možné zostaviť tak, že z~desiatich dostupných cifier sa odstránia dve a~zvyšných osem sa usporiada zostupne.
To je rovnaké, ako keby sa z~desaťciferného čísla 9876543210 odstránili dve cifry:

\begin{itemize}
\item
Ak by sme z tohto čísla ako prvú cifru odstránili 9, tak by zostalo deväť možností, ktorú cifru odstrániť ako druhú.
\item
Ak by sme ako prvú cifru odstránili 8, tak by zostalo osem možností, ktorú cifru odstrániť ako druhú
(odstránenie dvojice 8 a~9 je zahrnuté v~predchádzajúcom prípade).
\item
Ak by sme ako prvú cifru odstránili 7, tak by zostalo sedem možností, ktorú cifru odstrániť ako druhú
(odstránenie dvojíc 7, 8 a ~7, 9 je zahrnuté v~predchádzajúcich prípadoch).
\item
V~podobnom duchu uvažujeme až do konca\dots
\item
Ak by sme ako prvú cifru odstránili 1, tak by zostala jediná možnosť, ktorú cifru odstrániť ako druhú, a to 0.
\end{itemize}
Celkový počet možností, teda počet všetkých čísel vyhovujúcich zadaniu, je
$$
9+8+7+6+5+4+3+2+1 =45.
$$

\ineriesenie
Počet možností, ako odstrániť dve cifry z desiatich, je možné určiť nasledovne:

Prvú cifru z desiatich je možné vybrať desiatimi spôsobmi.
Druhú cifru zo zvyšných deviatich cifier je možné vybrať deviatimi spôsobmi.
To dáva celkovo $10\cdot9=90$ možností, ako vybrať dve cifry z desiatich s ohľadom na poradie výberu.
Toto poradie nás však nezaujíma -- nie je podstatné, v~akom poradí cifry vyberáme, ale ktoré vyberieme.
Teda celkový počet možností je polovičný:
$$
\frac12\cdot10\cdot9 =45.
$$

\poznamka
Predchádzajúce dvojité vyjadrenie toho istého výsledku možno zovšeobecniť takto:
$$
(n-1)+(n-2)+\cdots+2+1 = \frac12\cdot n\cdot(n-1).
$$
Toto číslo vyjadruje počet všetkých dvojíc, ktoré je možné utvoriť z~$n$ prvkov
(bez ohľadu na poradie výberu).
}

{%%%%% Z6-I-4
Podľa známeho postupu je druhý medzivýsledok 3175 súčinom prvého činiteľa a~druhej cifry druhého činiteľa.
Ako súčin trojciferného a~jednociferného čísla je možné číslo 3175 vyjadriť jediným spôsobom, a to $635\cdot5$ (prvočíselný rozklad je $3175=5\cdot5\cdot127$).
Teda môžeme doplniť:
$$
\vbox{\let\\=\cr
\halign{&\hbox to1.0em{\hss$#$\hss}\\
    &&6&3&5 \\
\x &&*&5&* \\
\noalign{\vskip4pt\hrule\vskip4pt}
   &*&*&*&* \\
   3&1&7&5&& \\
   *&*&*&&& \\
\noalign{\vskip4pt\hrule\vskip4pt}
  *&*&6&*&* \\
}}
$$

Tretí medzivýsledok je trojciferné číslo, ktoré je súčinom 635 a~prvej cifry druhého činiteľa.
Jediným trojciferným násobkom čísla 635 je samo toto číslo.
Teda môžeme doplniť:
$$
\vbox{\let\\=\cr
\halign{&\hbox to1.0em{\hss$#$\hss}\\
    &&6&3&5 \\
\x &&1&5&* \\
\noalign{\vskip4pt\hrule\vskip4pt}
   &*&*&*&* \\
   3&1&7&5&& \\
   6&3&5&&& \\
\noalign{\vskip4pt\hrule\vskip4pt}
  *&*&6&*&* \\
}}
$$

Prvý medzivýsledok je štvorciferné číslo, ktoré je súčinom 635 a~tretej cifry druhého činiteľa.
Štvorciferné násobky čísla 635 sú tieto:
$$
\gathered
635\cdot2=1270,\quad
635\cdot3=1905,\quad
635\cdot4=2540,\quad
635\cdot5=3175,\\
635\cdot6=3810,\quad
635\cdot7=4445,\quad
635\cdot8=5080,\quad
635\cdot9=5715.
\endgathered
$$
Aby tretia cifra vo výslednom súčine bola 6, musí byť druhá cifra prvého medzivýsledku buď 3, alebo 4 (podľa toho, či došlo k prechodu cez desiatku alebo nie).
Medzi vyššie uvedenými kandidátmi spĺňa túto podmienku iba číslo 4445.
Teda môžeme doplniť a~dopočítať výsledok:
$$
\vbox{\let\\=\cr
\halign{&\hbox to1.0em{\hss$#$\hss}\\
    &&6&3&5 \\
\x &&1&5&7 \\
\noalign{\vskip4pt\hrule\vskip4pt}
   &4&4&4&5 \\
   3&1&7&5&& \\
   6&3&5&&& \\
\noalign{\vskip4pt\hrule\vskip4pt}
  9&9&6&9&5 \\
}}
$$
}

{%%%%% Z6-I-5
V obvodoch prvého, druhého a štvrtého útvaru sú započítané vždy dve z~troch strán základného trojuholníka, a to tak, že
každá z dvoch strán je započítaná dvakrát
a~v~obvodoch rôznych útvarov sú zahrnuté rôzne dvojice strán.

V obvode tretieho útvaru sú započítané všetky strany základného trojuholníka, a~to opäť každá dvakrát.

Teda súčet obvodov prvého, druhého a štvrtého útvaru je rovný dvojnásobku obvodu tretieho útvaru.
To znamená, že neznámy obvod štvrtého útvaru je rovný rozdielu obvodov prvého a~druhého útvaru od dvojnásobku obvodu tretieho:
$$
2\cdot14{,}7 -8 -11{,}4 =29{,}4 -19{,}4 =10\,(\Cm).
$$

\poznamky
Zo zadania je možné odvodiť veľkosti strán základného trojuholníka, ktoré na tento účel označíme $a$, $b$, $c$:
\insp{z6-I-5a.eps}%


\noindent
Rozdiel obvodov tretieho a~prvého útvaru je rovný dvojnásobku $c$, teda
$$
c =\frac12(14{,}7-8) =3{,}35\,(\Cm).
$$
Rozdiel obvodov tretieho a~druhého útvaru je rovný dvojnásobku $b$, teda
$$
b =\frac12(14{,}7-11{,}4) =1{,}65\,(\Cm).
$$
Z toho a zo známych obvodov prvých troch útvarov je možné vyjadriť veľkosť poslednej strany základného trojuholníka:
$$
a =\frac82 -1{,}65
=\frac{11{,}4}2 -3{,}35
=\frac{14{,}7}2 -1{,}65 -3{,}35
=2{,}35\,(\Cm).
$$

Hodnoty $a$, $b$, $c$ spĺňajú trojuholníkové nerovnosti ($1{,}65 + 2{,}35 > 3{,}35$ atď.), teda trojuholník so stranami týchto dĺžok naozaj existuje a~(až na mierku) vyzerá takto:
\insp{z6-I-5b.eps}%


Z~uvedeného môžeme pre kontrolu odvodiť obvod štvrtého útvaru:
$$
2(b+c) =2(1{,}65+3{,}35) = 10\,(\Cm).
$$
}

{%%%%% Z6-I-6
Detí bolo sedem, teda každé mohlo odohrať najviac šesť hier.
František odohral šesť hier, takže hral s každým z prítomných detí.

Alex odohral jednu hru, takže nehral s nikým iným ako s Františkom.
Eva odohrala päť hier, takže hrala so všetkými okrem Alexa.

Barbora odohrala dve hry, takže nehrala s nikým iným ako s Františkom a Evou.
Dana odohrala štyri hry, takže hrala so všetkými okrem Alexa a Barbory.

Cyril odohral tri hry, a to s Františkom, Evou a Danou.

S Gabikou hrali František, Eva a Dana, teda Gabika odohrala tri hry.

\poznamka
Predchádzajúce závery sú prehľadne znázornené v~tabuľke:
$$
\begintable
  \|A|B|C|D|E|F|G\crthick
A\|$-$|||||$*$|\cr
B\||$-$|||$*$|$*$|\cr
C\|||$-$|$*$|$*$|$*$|\cr
D\|||$*$|$-$|$*$|$*$|$*$\cr
E\||$*$|$*$|$*$|$-$|$*$|$*$\cr
F\|$*$|$*$|$*$|$*$|$*$|$-$|$*$\cr
G\||||$*$|$*$|$*$|$-$\endtable
$$
}

{%%%%% Z7-I-1
Najmenej zrniek napočítala Cilka, najviac Ajka.
Rozdiel výsledkov Cilky a~Barborky je 14, rozdiel výsledkov Barborky a~Ajky je 3, rozdiel výsledkov Cilky a~Ajky je 17.
Ak by Danielov odhad súhlasil s niektorým z týchto troch výsledkov, tak by spomínaný súčet rozdielov bol určený iba dvoma sčítancami:
\begin{itemize}
\item Ak by Danielov odhad súhlasil s~výsledkom Cilky, tak by súčet rozdielov od zvyšných výsledkov bol $14+17=31$.
\item Ak by Danielov odhad súhlasil s~výsledkom Barborky, tak by súčet rozdielov od zvyšných výsledkov bol $14+3=17$.
\item Ak by Danielov odhad súhlasil s~výsledkom Ajky, tak by súčet rozdielov od zvyšných výsledkov bol $17+3=20$.
\end{itemize}

\noindent
Pre iné hodnoty Danielovho odhadu platí:
\begin{itemize}
\item Ak by Danielov odhad bol menší ako výsledok Cilky, tak by súčet rozdielov od všetkých výsledkov bol väčší ako 31.
\item Ak by Danielov odhad bol medzi výsledkami Cilky a~Barborky, tak by súčet rozdielov od všetkých výsledkov bol medzi 31 a~17.
\item Ak by Danielov odhad bol medzi výsledkami Barborky a~Ajky, tak by súčet rozdielov od všetkých výsledkov bol medzi 17 a~20.
\item Ak by Danielov odhad bol väčší ako výsledok Ajky, tak by súčet rozdielov od všetkých výsledkov bol väčší ako 20.
\end{itemize}

\noindent
Teda Danielov odhad mohol byť medzi výsledkami Cilky a Barborky, alebo byť väčší ako výsledok Ajky.
Postupným skúšaním v~rámci týchto obmedzení odhalíme nasledujúce dve možnosti:
\begin{itemize}
\item Ak by Danielov odhad bol 873\,451\,215, tak by súčet rozdielov od výsledkov troch kamarátok bol $2+12+15=29$.
\item Ak by Danielov odhad bol 873\,451\,233, tak by súčet rozdielov od výsledkov troch kamarátok bol $20+6+3=29$.
\end{itemize}

\noindent
Daniel odhadoval počet zrniek v pieskovisku buď na 873\,451\,215, alebo 873\,451\,233.

\poznamka
Pre úplnosť dopĺňame niekoľko ďalších hodnôt v~rámci daných obmedzení (v~Danielovom odhade píšeme len posledné dvojčíslie):
$$
\begintable
Danielov odhad\|$<13$|\it 13|14|15|16|17|\dots\ \cr
súčet rozdielov\|$>31$|\it 31|30|29|28|27|\dots\ \endtable
$$
$$
\begintable
\dots\!|26|\it 27|28|29|\it 30|31|32|33|$>33$\cr
\dots\!|18|\it 17|18|19|\it 20|23|26|29|$>29$\endtable
$$

Predchádzajúce skúšanie je možné nahradiť niekoľkými výpočtami.
Napr. za predpokladu, že Danielov odhad je medzi výsledkami Cilky a Barborky, je súčet rozdielov od výsledkov troch kamarátok rovný
$$
(d-13)+(27-d)+(30-d) = 44-d,
$$
kde $d$ je posledné dvojčíslie Danielovho odhadu.
Tento súčet je rovný 29 práve vtedy, keď $d=15$, čo zodpovedá jednej z~uvedených možností.
}

{%%%%% Z7-I-2
Počet rýb, ktoré ulovil pán Delfín, bol násobkom 9 a~nepresahoval 70.
Pán Delfín teda mohol uloviť
$$
9,\quad 18,\quad 27,\quad 36,\quad 45,\quad 54,\quad 63
$$
rýb.
Počet rýb, ktoré ulovil pán Žralok, bol násobkom 17 a nepresahoval 70.
Pán Žralok teda mohol uloviť
$$
17,\quad 34,\quad 51,\quad 68
$$
rýb.
Dokopy ulovili 70 rýb, čo je možné z~uvedených čísel vyjadriť jedine ako
$$
70=36+34.
$$

Pán Delfín ulovil 36 rýb.

\poznamka
pre predchádzajúci výsledok nie je nutné skúšať všetky možné súčty (ktorých je $7\cdot4=28$).
Stačí prebrať štyri možné počty rýb pána Žraloka, zistiť, koľko rýb by zodpovedalo pánovi Delfínovi (rozdiel od 70) a overiť, či je tento počet deliteľný 9:

$$
\begintable
$\check{z}$\|17|34|51|68\crthick
$d=70-\check{z}$\|53|36|19|2\cr
$9\,\vert\,d$ ?\|nie|áno|nie|nie\endtable
$$
}

{%%%%% Z7-I-3
Uvažujme súčet prvých dvoch čísel.
Pokiaľ k~tomuto súčtu pripočíta Eva (autorka úlohy) tretie číslo, dostane 15, ak to isté číslo odpočíta, dostane 10.
Rozdiel $15-10=5$ teda zodpovedá dvojnásobku tretieho čísla.
Teda tretie číslo je $5:2=2{,}5$.

Uvažujme súčet prvého a~tretieho čísla.
Ak k tomuto súčtu Eva pripočíta druhé číslo, dostane 15, ak to isté číslo odpočíta, dostane 8.
Rozdiel $15-8=7$ teda zodpovedá dvojnásobku druhého čísla.
Teda druhé číslo je $7:2=3{,}5$.

Súčet všetkých troch čísel je 15.
Teda prvé číslo je $15-2{,}5-3{,}5=9$.

Eva si myslí čísla 9, 3,5 a ~2,5.

\poznamka
Úlohu možno vyjadriť pomocou sústavy rovníc
$$
a+b+c=15,\quad
a+b-c=10,\quad
a-b+c=8,
$$
pričom $a$, $b$ a~$c$ je postupne prvé, druhé a~tretie číslo.
Predchádzajúce úvahy možno zapísať nasledovne:
$$
\gathered
2c =(a+b+c)-(a+b-c) =15-10 =5,\quad\text{teda}\quad c=2{,}5, \\
2b =(a+b+c)-(a-b+c) =15-8 =7,\quad\text{teda}\quad b=3{,}5, \\
a =(a+b+c)-b-c =15-2{,}5-3{,}5 =9.
\endgathered
$$
}

{%%%%% Z7-I-4
Číslo 2024 potrebujeme vyjadriť ako súčin troch prirodzených čísel, z ktorých dve sú väčšie alebo rovné 18 a líšia sa najviac o 10.
Vyhovujúce možnosti preberieme na základe prvočíselného rozkladu čísla 2024, ktorý vyzerá nasledovne:
$$
2024 =2\cdot2\cdot2\cdot11\cdot23.
$$

Prvočísla 11 a 23 patria k vekom tety a strýka, navyše každé niekomu inému; ak by to tak nebolo, potom by z~daných prvočísel nebolo možné vytvoriť dva činitele väčšie ako 18.
Navyše žiadny z~vekov nemôže byť väčší ako $2\cdot23=46$ rokov; najbližší vyšší možný vek je $2\cdot2\cdot23=92$ rokov a~v~takom prípade by z~daných prvočísel nebolo možné vytvoriť činiteľa, ktorý by sa od 92 líšil najviac o~10.
Teda stačí prebrať nasledujúce možnosti:
\begin{itemize}
\item
Ak by mala teta alebo strýko 23 rokov, tak by druhý musel mať medzi 18 a~33 rokmi.
Z~možných činiteľov tomuto obmedzeniu vyhovuje iba $2\cdot11=22$.
Teda teta mohla mať 22 a strýko 23 rokov.
V~tomto prípade by na oslave boli $2\cdot2=4$ psy.
\item
Ak by mala teta alebo strýko $2\cdot23=46$ rokov, tak by druhý musel mať medzi 36 a~56 rokmi.
Z~možných činiteľov tomuto obmedzeniu vyhovuje iba $2\cdot2\cdot11=44$.
Teda teta mohla mať 44 a strýko 46 rokov.
V tomto prípade by na oslave bol 1~pes.
\end{itemize}

Na oslave bol buď jeden pes, alebo štyri psy.

\poznamka
Súčin vekov tety a~strýka bol väčší ako $18\cdot18=324$.
Teda počet psov na oslave bol rovný nanajvýš celej časti podielu $2024:324$, čo je 6.
Z~daných prvočísel je možné zostaviť iba čísla 1, 2 a~4.
Rozbor možností je možné založiť na týchto troch prípadoch:
\begin{itemize}
\item Prípady s~1 alebo so 4 psami možné sú, viď predchádzajúce riešenie.
\item Prípad s 2 psami možný nie je.
Súčin vekov tety a~strýka by bol $2\cdot2\cdot11\cdot23$, čo je možné ako súčin dvoch činiteľov väčších ako 18 vyjadriť buď $22\cdot46$, alebo $44\cdot23$.
V oboch rozkladoch je však rozdiel činiteľov väčší ako 10.
\end{itemize}
}

{%%%%% Z7-I-5
Jediné možné umiestnenie štvorca v~rámci trojuholníka je ako na nasledujúcom obrázku.
Dodatočné delenie strán na tretiny a~spojenie zodpovedajúcich bodov priečkami rozdelí daný trojuholník na 9 zhodných trojuholníčkov:
\insp{z7-I-5a.eps}%


Štvorec pozostáva zo štyroch trojuholníčkov, teda pomer obsahu štvorca a~daného trojuholníka je $4:9$.
Trojuholník má obsah 36\,cm$^2$, teda obsah štvorca je
$$
\frac49\cdot36 =4\cdot4 =16\,(\Cm^2).
$$

\poznamky
Obsah pravouhlého trojuholníka je polovicou obsahu doplneného obdĺžnika.
Predchádzajúcu myšlienku s tretinovým delením možno rozvíjať aj podľa nasledujúceho obrázku:
\insp{z7-I-5b.eps}%


Z~uvedeného o.\,i. vyplýva, že odvesny daného trojuholníka sú v~pomere $1:2$.

V~rámci daného trojuholníka je možné objaviť niekoľko navzájom podobných trojuholníkov.
Aj tento postreh je možné využiť na riešenie úlohy.
}

{%%%%% Z7-I-6
Súčet vekov siedmich trpaslíkov, ktorí prišli Nošteka navštíviť, je 1443 dní.
Ak by trpaslíci boli rozdelení tak, že v jednej skupine je sám Noštek a v druhej skupine všetci ostatní, tak by Noštek mal 1443 dní.
Viac dní Noštek mať nemôže.

Súčet vekov siedmich trpaslíkov, ktorí prišli Nošteka navštíviť, tvorí nepárny počet dní.
Aby súčet vekov všetkých ôsmich trpaslíkov bol deliteľný dvoma, musel byť Noštekov počet dní tiež nepárny.
Aby bolo možné rozdeliť trpaslíkov do dvoch skupín s rovnakými vekmi, musí sa dať zo siedmich známych vekov vyjadriť ako polovica súčtu všetkých, tak polovica tohto súčtu bez veku Nošteka.
Postupne vzostupne preberieme možnosti, kým nenájdeme vyhovujúcu.
V~nasledujúcej tabuľke označuje $N$ vek Nošteka a~$S$ súčet vekov všetkých ôsmich trpaslíkov.
Rozhodovanie na poslednom riadku je vysvetlené nižšie:
$$
\begintable
$N$\|1|3|5|7|9|11|13|\dots\crthick
$S$\|1444|1446|1448|1450|1452|1454|1456|\dots\cr
$\frac12S$\|722|723|724|725|726|727|728|\dots\cr
$\frac12S - N$\|721|720|719|718|717|716|715|\dots\crthick
?\|č|s|č|s|č|č|OK|\dots\endtable
$$

Okrem dvoch trpaslíkov, ktorí prišli Nošteka navštíviť, sú veky všetkých ostatných deliteľné 10.
Tieto dva výnimočné veky končia ciframi 5 a~8.
Zo siedmich známych vekov trpaslíkov teda možno vyjadriť iba súčty končiace ciframi 0, 5, 8 a~3 (podľa toho, či je zahrnuté žiadne, jedno alebo obe spomínané čísla).
Tento postreh vylučuje možnosti zodpovedajúce $N=1$, 5, 9, 11 (a~mnohé ďalšie).
Preberieme ostatné možnosti:

\begin{itemize}
\item
V~prípade $N=3$ by v~súčte $\frac12S=723$ museli byť zahrnuté obe čísla 105 a~168:
$$
723=105+168+450.
$$
Sčítanec 450 by sa musel dať vyjadriť pomocou čísel 120, 140, 210, 280 a~420.
Postupným skúšaním (napr. od najväčšieho z~dostupných čísel) zisťujeme, že to nie je možné.
\item
V prípade $N=7$ by v~súčte $\frac12S-N=718$ muselo byť zahrnuté číslo 168:
$$
718=168+550.
$$
Sčítanec 550 by sa musel dať vyjadriť pomocou čísel 120, 140, 210, 280 a~420.
Postupným skúšaním (napr. od najväčšieho z~dostupných čísel) zisťujeme, že to nie je možné.
\item
V~prípade $N=13$ by v~súčte $\frac12S=728$ muselo byť zahrnuté číslo 168:
$$
728=168+560.
$$
Sčítanec 560 by sa musel dať vyjadriť pomocou čísel 120, 140, 210, 280 a~420.
Skúšaním zisťujeme $560=140+420$, teda sme našli najmenšie vyhovujúce riešenie.
\end{itemize}
Noštek mohol mať najmenej 13 a najviac 1443 dní.
\poznamky
Rozdelenie do skupín s~13-denným Noštekom (t.\,j. rozdelenie zodpovedajúce súčtu $\frac12S=728$) vyzeralo takto:
$$
140+168+420 = 13 + 105+120+210+280 .
$$

V~prípadoch $N=3$, resp. 7 je možné zamerať sa na druhý z~diskutovaných súčtov (teda 720, resp. 725), ktorý taktiež z~dostupných čísel vyjadriť nemožno.

Vo vylučovaní možností je možné použiť ďalšie vlastnosti daných čísel.
Napr. nemožnosť vyjadrenia čísla 450 v~prípade $N=3$ možno zdôvodniť nasledovne.
Číslo 450 je deliteľné 10, ale nie je deliteľné 20.
Z~dostupných čísel má rovnakú vlastnosť iba číslo 210, teda vo vyjadrení 450 musí byť zahrnuté:
$$
450=210+240.
$$
Sčítanec 240 by sa musel dať vyjadriť pomocou zvyšných čísel 120, 140, 280 a ~420.
Keďže čísla 280 a~420 sú väčšie ako 240, prichádza do úvahy iba súčet $120+140$.
Ten je však tiež väčší ako 240, takže vyjadrenie 450 nie je možné.
}

{%%%%% Z8-I-1
Ako počty detí, tak ich tohtoročné prírastky sú vyjadrené prirodzenými číslami.
Vzhľadom na to, že $10\,\%=10/100=1/10$, musel byť pôvodný počet všetkých detí násobkom 10.
Podobne platí, že pôvodný počet chlapcov bol násobkom 20, pretože $5\,\%=5/100=1/20$,
a~pôvodný počet dievčat bol násobkom 5, pretože $20\,\%=20/100=1/5$.

Keďže chlapcov bolo o~30 viac ako dievčat, najmenšie možné pôvodné počty boli nasledujúce
($ch$, resp. $d$ označuje pôvodný počet chlapcov, resp. dievčat):
$$
\begintable
$ch$\|40|60|80|100|120|\dots\cr
$d$\|10|30|50|70|90|\dots\crthick
$ch+d$\|50|90|130|170|210|\dots\endtable
$$

Po započítaní tohtoročných prírastkov dostávame nasledujúci prehľad:
$$
\begintable
$1{,}05ch$\|42|63|84|105|126|\dots\cr
$1{,}2d$\|12|36|60|84|108|\dots\crthick
$1{,}05ch+1{,}2d$\|54|99|144|189|234|\dots\cr
$1{,}1(ch+d)$\|55|99|143|187|231|\dots\endtable
$$

Súčet nových počtov chlapcov a~dievčat je rovný pôvodnému súčtu zväčšenému o~10\,\% práve v~druhom prípade;
so zväčšujúcimi sa číslami sa rozdiel medzi týmito dvoma hodnotami len zväčšuje.
Tento rok je v skautskom oddiele 99 detí.

\poznamka
Problém je možné zapísať pomocou rovnice
$$
1{,}05ch+1{,}2d =1{,}1(ch+d) ,
$$
kde $ch=d+30$.
Po dosadení a~úpravách dostávame
$$
\aligned
2{,}25d+31{,}5 &=2{,}2d+33 , \\
0{,}05d &=1{,}5 , \\
d &=30 ,
\endaligned
$$
čo zodpovedá druhému z vyššie uvedených prípadov.
}

{%%%%% Z8-I-2
Po prvom trhaní mal Adam $4=2\cdot2$ kúsky papiera.
V~každom ďalšom kroku trhal každý kúsok buď na $4=2\cdot2$, alebo na $10=2\cdot5$ menších kúskov.
Teda po každom trhaní obsahoval prvočíselný rozklad aktuálneho počtu kúskov iba čísla 2 a~5.
Zastúpenie týchto dvoch prvočísel v rozklade môže byť rozmanité, avšak nie ľubovoľné.
Určite obsahuje aspoň dve 2 (z~prvého trhania) a~ku každej 5 prislúcha jedna 2 (z~trhania na 10 kúskov).
V~každom prípade rozdiel počtu 5 a~počtu 2 je párne číslo (ktoré je dvojnásobkom počtu trhaní na 4 kúsky).

Prvočíselný rozklad čísla 20\,000 vyzerá nasledovne:
$$
20\,000 =2\cdot10\,000 =2\cdot10^4 =2^5\cdot5^4 .
$$
V~rozklade je počet 2 o~jedna väčší ako počet 5.
Adam teda nemohol natrhať 20\,000 kúskov.

\ineriesenie
Keď Adam roztrhá nejaký kúsok na 4 menšie kúsky, celkový počet kúskov sa zväčší o~3.
Keď Adam roztrhá nejaký kúsok na 10 menších kúskov, celkový počet kúskov sa zväčší o~9.
Platí, že počet kúskov po každom trhaní dáva rovnaký zvyšok po delení tromi.
Na začiatku mal Adam jeden kus papiera, teda po každom trhaní bol zvyšok po delení aktuálneho počtu kúskov tromi rovný 1.

Avšak zvyšok po delení $20\,000:3$ je rovný 2 (najbližšie číslo deliteľné 3 je 20\,001).
Adam teda nemohol natrhať 20\,000 kúskov.
}

{%%%%% Z8-I-3
Doplníme priesečník priamok $CF$ a~$AE$, ktorý označíme $G$:
\insp{z8-I-3a.eps}%


V~pravidelnom päťuholníku sú všetky strany navzájom zhodné a~rovnako tak všetky uhlopriečky.
Teda Patova cesta $EB$ je zhodná s~uhlopriečkou $EC$.
Postupne ukážeme, že úsečka $EC$ je zhodná s~$EG$ a~že úsečka $AF$ je zhodná s~$AG$.
K~týmto tvrdeniam sa dopracujeme porovnaním niekoľkých uhlov, ktoré si označíme podľa nasledujúceho obrázka.
V~tomto označení zohľadňujeme, že trojuholníky $BAE$ a~$BEC$ sú rovnoramenné, teda že uhly pri ich základniach sú zhodné:
\insp{z8-I-3b.eps}%


Päťuholník pozostáva z troch trojuholníkov, teda súčet veľkostí vnútorných uhlov päťuholníka je $3\cdot180\st=540\st$.
Pravidelný päťuholník má všetky vnútorné uhly zhodné, teda veľkosť vnútorného uhla pravidelného päťuholníka je $540\st:5=108\st$.
V~našom označení tak dostávame
$$
\alpha=\beta+\gamma =108\st.
$$
Súčet uhlov v~trojuholníku $BAE$ je $\alpha+2\beta=180\st$, teda
$$
\beta=\frac12\left(180\st-108\st\right)=36\st.
$$
Z toho a~z predchádzajúceho vyjadrenia dostávame
$$
\gamma=108\st-36\st=72\st.
$$
Súčet uhlov v~trojuholníku $BEC$ je $2\gamma+\delta=180\st$, teda
$$
\delta=180\st-2\cdot72\st=36\st.
$$

Z~uvedeného vyplýva, že uhly $BEC$ a~$BEG$ sú zhodné.
Navyše úsečky $CG$ a~$EB$ sú podľa zadania kolmé, teda trojuholník $CEG$ je rovnoramenný s~hlavným vrcholom $E$.
Z~toho vyplýva, že úsečky $EC$ a~$EG$ sú zhodné a~rovnako tak uhly pri základni $CG$.

Pravidelný päťuholník je súmerný (okrem iného) podľa osi úsečky $AB$, teda úsečky $AB$ a~$EC$ sú rovnobežné.
Súhlasné uhly $ECG$ a~$AFG$ sú zhodné, preto aj trojuholník $FAG$ je rovnoramenný s~hlavným vrcholom $A$.
Z~toho vyplýva, že úsečky $AF$ a~$AG$ sú zhodné.

Dokopy dostávame
$$
|EA|+|AF| =|EA|+|AG| =|EG| =|EC| =|EB|,
$$
teda Mat a~Pat zametali rovnako dlhé úseky.
\insp{z8-I-3c.eps}%


\poznamka
Pravidelnému päťuholníku je možné vždy opísať kružnicu.
Vzhľadom k~tej kružnici sú uhly $BEC$ a~$BEA$ obvodovými uhlami, ktoré patria navzájom zhodným tetivám $BC$ a~$BA$, a preto sú zhodné.
Tento (nesamozrejmý) poznatok môže významne skrátiť argumentáciu v predchádzajúcom riešení.
Navyše obvodový uhol je polovicou uhla stredového, ktorý je v našom prípade rovný $360\st:5=72\st$.
Vzhľadom k~predchádzajúcemu označeniu teda naozaj platí $\beta=\delta=72\st:2=36\st$.
\insp{z8-I-3d.eps}%

}

{%%%%% Z8-I-4
Hynkov príklad môžeme prepísať ako
$$
@ + 11\cdot\# +111\cdot*+ 1\,111\cdot\& +11\,111\cdot\$ =\ ?
$$
Druhý a štvrtý sčítanec je deliteľný 11.
Koeficient 111 pri treťom sčítanci a~11\,111 pri piatom sčítanci dáva po delení 11 zvyšok 1.
Teda pôvodný súčet je deliteľný 11 práve vtedy, keď súčet $@+*+\$$ je deliteľný 11.

Z~dostupných čísel je možné číslo 11 (či jeho násobok) vyjadriť jedine ako $2+4+5$.
Teda znaky @, $*$, \$ predstavujú čísla 2, 4, 5 v~nejakom poradí.
Na znaky \# a~\& zostávajú čísla 1 a~3 v~nejakom poradí.

Najmenší súčet dostaneme, ak znakom \$, \&, $*$, \#, @ postupne priradíme najmenšie možné čísla v~rámci predchádzajúcich obmedzení:
$$
5+33+444+1\,111+22\,222 =23\,815. \tag{1}
$$
Najväčší súčet dostaneme, ak znakom \$, \&, $*$, \#, @ postupne priradíme najväčšie možné čísla v~rámci predchádzajúcich obmedzení:
$$
2+11+444+3\,333+55\,555 =59\,345. \tag{2}
$$

Výsledkom Hynkovho príkladu môže byť najmenej 23\,815 a~najviac 59\,345.

\poznamka
Počet možností, ako piatim znakom priradiť päť cifier, je $5\cdot4\cdot3\cdot2\cdot1 =120$.
Aj bez úvodných postrehov je možné určiť najmenší a najväčší súčet bez toho, aby sa museli preberať všetky možnosti.
Napr. najväčší možný súčet, ktorý je možné z~daných cifier dostať bez nároku na deliteľnosť 11, zodpovedá priradeniu
$\$=5$, $\&=4$, $*=3$, $\#=2$, $@=1$, čo skrátene zapíšeme ako $(5,4,3,2,1)$.
Možné súčty zostupne zodpovedajú priradeniam
$$
\aligned
& (5,4,3,2,1),\ (5,4,3,1,2),\ (5,4,2,3,1),\ \ (5,4,2, 1,3),\ (5,4,1,3,2),\ (5,4,1,2,3),\\
& (5,3,4,2,1),\ \ (5,3,4,1,2),\ \ (5,3,2,4,1),\ \ (5,3,2, 1,4),\ \\dots\\
\endaligned
$$
Postupným výpočtom zodpovedajúcich súčtov a overením ich deliteľnosti 11 je možné odhaliť najväčšiu vyhovujúcu možnosť.
Riešenie (2) zodpovedá 8. možnosti v~tejto postupnosti.

Pri hľadaní najmenšieho možného Hynkovho súčtu je možné postupovať podobne, počnúc priradením $(1,2,3,4,5)$.
Riešenie (1) zodpovedá 27. možnosti v~príslušnej postupnosti.
}

{%%%%% Z8-I-5
Trojuholníky $CDH$ a~$CHI$ majú spoločnú stranu $CH$, teda majú rovnakú výšku zo spoločného vrcholu $C$.
Tieto trojuholníky majú rovnaký obsah, teda úsečky $DH$ a~$HI$ sú zhodné.
Trojuholníky $CHI$ a~$FIH$ majú spoločnú stranu $HI$, teda majú rovnakú výšku zo spoločného vrcholu $I$.
Tieto trojuholníky majú rovnaký obsah, teda úsečky $CH$ a~$HF$ sú zhodné.

Predchádzajúce závery znamenajú, že $H$ je stredom úsečiek $DI$ a~$CF$, čo sú uhlopriečky štvoruholníka $CDFI$.
Tento štvoruholník je teda rovnobežníkom, ktorý je uhlopriečkami rozdelený na štyri trojuholníky s rovnakým obsahom.
Obsah rovnobežníka $CDFI$ je teda rovný štvornásobku obsahu trojuholníka $CDH$.

Z uvedeného vyplýva, že priamky $AC$ a~$FI$ sú rovnobežné, teda aj štvoruholník $AFID$ je rovnobežníkom.
Tento rovnobežník má s~rovnobežníkom $CDFI$ spoločný trojuholník $DFI$, ktorý tvorí polovicu ako prvého, tak druhého rovnobežníka.
Obsah rovnobežníka $AFID$ je teda rovnaký ako obsah rovnobežníka $CDFI$.

Obsah štvoruholníka $AFHD$ môžeme vyjadriť takto:
$$
\aligned
S_{AFHD} &=S_{AFID}-S_{FIH} = \\
&=S_{CDFI}-S_{CDH} = \\
&=4\cdot S_{CDH}-S_{CDH} =3\cdot8 =24\,(\text{dm}^2).
\endaligned
$$
\insp{z8-I-5a.eps}%


\poznamky
Štvoruholník $AFHD$ je vlastne lichobežníkom.
Body $B$ a~$E$ nehrajú pri riešení úlohy žiadnu rolu.
Z~úvodných postrehov vyplýva niekoľko ďalších skutočností, ktoré je možné použiť pri riešení úlohy:

Úsečky $DI$, $IF$ a~$FD$ sú strednými priečkami trojuholníka $AGC$, a~tie rozdeľujú tento trojuholník na štyri zhodné trojuholníky.
Obsah každého z týchto trojuholníkov sa rovná dvojnásobku obsahu referenčného trojuholníka $CDH$ (na obrázku vyznačené ako $S$).
$S_{AFHD}=3\cdot S_{CDH}$.
\insp{z8-I-5b.eps}%


Trojuholníky $CDH$ a~$CAF$ sú podobné s~koeficientom 2.
Obsah trojuholníka $CAF$ je preto štvornásobkom obsahu trojuholníka $CDH$.
$S_{AFHD}=S_{CAF}-S_{CDH}={3\cdot S_{CDH}}$.
}

{%%%%% Z8-I-6
Vzhľadom na to, že každé rohové políčko vystupuje v dvoch súčtoch, snažíme sa do týchto políčok umiestniť najmenšie možné čísla a~nejako doplniť zvyšok.
Po chvíli skúšania je možné odhaliť napr. nasledujúce vyplnenie, v~ktorom je súčet čísel pozdĺž každej strany rovný 12:
$$\begintable
1|9|2\cr
8|7|6\cr
3|5|4\endtable
$$

Vyplnenie s menšími súčtami sa nájsť nedarí, a to preto, že to nie je možné.
Najmenší možný súčet pozdĺž strany so sčítancom 9 je $1+2+9=12$.
Teda číslo 9 by muselo byť uprostred tabuľky a zvyšné čísla pozdĺž strán.
Najmenší možný súčet pozdĺž strany so sčítancom 8 je $1+2+8=11$.
Teda menší súčet dosiahnuť nemožno a~rozmýšľame nad doplnením tabuľky, v~ktorej pozdĺž jednej strany sú čísla 1, 2 a~8.
Pozdĺž protiľahlej strany by boli tri zo zvyšných piatich čísel.
Najmenšie možné čísla sú 3, 4 a~5, ktorých súčet je však $3+4+5=12$ a nie 11.

Najmenšia možná hodnota súčtu v~Adamovej tabuľke je 12.

\poznamka
Každé zo štyroch rohových políčok prispieva do dvoch súčtov, každé zo zvyšných štyroch políčok pozdĺž strán prispieva do jedného súčtu.
Teda súčet všetkých štyroch súčtov pozdĺž strán je aspoň
$$
2\cdot(1+2+3+4)+(5+6+7+8) =46.
$$
Požiadavka rovnosti súčtov pozdĺž strán znamená, že predchádzajúci súčet by musel byť deliteľný štyrmi.
Najbližšie väčšie číslo deliteľné štyrmi je 48.
Teda najmenšia možná hodnota súčtu v~Adamovej tabuľke je $48:4=12$.
Vyššie uvedená tabuľka ukazuje, že také vyplnenie je možné.
}

{%%%%% Z9-I-1
Ak by Matovi do chalúpky chýbal trojnásobok ubehnutej vzdialenosti, tak by bol v štvrtine.
Tento prípad by nastal, keby mal zdolanú polovicu vzdialenosti, ktorú doteraz ubehol.
Teda Mat sa v~danom okamihu nachádzal v polovici medzi rybníkom a~chalúpkou.
Ak by Patovi do chalúpky chýbala tretina ubehnutej vzdialenosti, tak by bol v troch štvrtinách.
Tento prípad by nastal, keby mal zdolaný dvojnásobok vzdialenosti, ktorú doteraz ubehol.
Teda Pat sa v~danom okamihu nachádzal v troch osminách medzi rybníkom a chalúpkou.
V danom okamihu bol bližšie k chalúpke Mat.
\insp{z9-i-1.eps}%

\poznamky
Ak $m$ a~$p$ postupne označujú Matovu a~Patovu vzdialenosť od rybníka v~danom okamihu a~$c$ označuje celú vzdialenosť medzi rybníkom a~chalúpkou, tak informácie zo zadania doslovne zapíšeme takto:
$$
\frac12m+3\left(\frac12m\right) =c,\quad
2p+\frac13\left(2p\right)=c.
$$
Odtiaľ jednoducho dostávame $2m=c$, teda $m=\frac12c$, a~$\frac83p=c$, teda $p=\frac38c$.
}

{%%%%% Z9-I-2
Využijeme to, že uhlopriečky v~rovnobežníku sa navzájom rozpoľujú a~v~kosoštvorci sú navyše kolmé.
Označme priesečník uhlopriečok $AC$ a~$BD$ ako $E$ a~stred úsečky $EC$ ako $R$.
Úsečka $SR$ je strednou priečkou trojuholníka $DEC$, ktorá je rovnobežná so stranou $DE$.
Úsečka $SR$ je preto kolmá na $AC$.
V pravouhlom trojuholníku $ARS$ poznáme veľkosť prepony $|AS|=7\cm$ a~veľkosť odvesny $|AR|=\frac34|AC|=6\cm$.
Vzhľadom na to, že $7>6$, takýto trojuholník naozaj existuje a možno ho zostrojiť napr. takto:
\begin{itemize}
\item úsečka $AR$ s~veľkosťou 6\,cm,
\item kolmica na úsečku $AR$ idúca bodom $R$,
\item kružnica so stredom v~bode $A$ a~polomerom 7\,cm,
\item bod $S$ je priesečníkom kolmice a~kružnice.
\end{itemize}
\noindent
Zvyšné vrcholy kosoštvorca možno zostrojiť nasledovne:
\begin{itemize}
\item bod $C$ leží na polpriamke $AR$ vo vzdialenosti 8\,cm od $A$,
\item bod $D$ je stredovo súmerný s~bodom $C$ podľa stredu $S$,
\item bod $B$ je osovo súmerný s~bodom $D$ podľa osi $AC$.
\insp{z9-i-2a.eps}%
\end{itemize}

\poznamky
Hlavnú pozornosť venujeme konštrukcii trojuholníka $ARS$.
Konštrukcie súmerných bodov $D$ a~$B$ považujeme za dobre známe, teda detailne nerozpisujeme.
Priesečníky vo štvrtom kroku konštrukcie sú dva;
druhá možnosť vedie k tomu istému riešeniu s inak označenými vrcholmi.
Čiastkové konštrukcie je možné realizovať rôzne.
Napr. pre danú úsečku $AC$ je možné body $E$ a~$R$ postupne zostrojiť ako stredy úsečiek $AC$ a~$EC$,
bod $D$ je možné zostrojiť ako priesečník priamky $CS$ s~kolmicou na $AC$ idúcou bodom $E$,
a pod.
Pravouhlý trojuholník $ARS$ je možné zostrojiť aj takto:
\begin{itemize}
\item úsečka $AS$ s~veľkosťou 7\,cm,
\item kružnica so stredom v~bode $A$ a~polomerom 6\,cm,
\item kružnica s~priemerom $AS$,
\item bod $R$ je priesečníkom kružníc.
\insp{z9-i-2b.eps}%
\end{itemize}

\noindent
Podľa Tálesovej vety je uhol pri vrchole $R$ naozaj pravý.
Trojuholníky $ACB$ a~$ACD$ vyzerajú takmer rovnostranne, ale nie sú.
Najmä neplatí, že sa kružnica v~prvom obrázku dotýka úsečky $CD$, aj keď to tak môže vyzerať.

\ineriesenie
Využijeme to, že úsečky $AS$ a~$DE$ sú ťažnicami trojuholníka $ACD$.
Navyše si uvedomujeme, že uhlopriečky v kosoštvorci sú navzájom kolmé.
Označme priesečník uhlopriečok $AC$ a~$BD$ ako $E$ a~priesečník ťažníc $AS$ a~$DE$, t.\,j. ťažisko, ako $T$.
Ťažisko leží v tretine každej ťažnice, bližšie k~strane trojuholníka.
V~pravouhlom trojuholníku $AET$ teda poznáme veľkosť prepony $|AT|=\frac23|AS|=\frac{14}3\cm$ a~veľkosť odvesny $|AE|=\frac12|AC|=4\cm$.
Vzhľadom na to, že $\frac{14}3>4$, trojuholník $AET$ naozaj existuje a~je ho možné zostrojiť podobne ako trojuholník $ARS$ v~riešení uvedenom vyššie.
Zvyšné vrcholy kosoštvorca možno zostrojiť napr. takto:
\begin{itemize}
\item bod $D$ leží na polpriamke $ET$ vo vzdialenosti $3|ET|$ od $E$,
\item bod $B$ je stredovo súmerný s~bodom $D$ podľa stredu $E$,
\item bod $C$ je stredovo súmerný s~bodom $A$ podľa stredu $E$.
\insp{z9-i-2c.eps}%
\end{itemize}

\poznamky
V~uvedenom riešení je potrebné rozdeliť danú úsečku na tretiny.
Korektné riešenie tejto podúlohy považujeme za dobre známe, teda detailne nerozpisujeme.
Predchádzajúca poznámka súvisí s~faktom, že trojuholníky $AET$ a~$ARS$ sú podobné s~koeficientom podobnosti $2/3$.
Vhodným rozšírením diskutovaného útvaru je možné objaviť ďalšie vzťahy a na nich založiť alternatívne konštrukcie.
Ak body $A'$, $B'$, $D'$, $E'$ označujú postupne body súmerné s~$A$, $B$, $D$, $E$ podľa stredu $C$, tak napr.
štvoruholník $ACB'D$ je rovnobežníkom,
štvoruholník $DEE'B'$ je obdĺžnikom,
trojuholník $AE'B'$ je podobný s~trojuholníkom $ARS$ (teda aj s~$AET$)
a pod.
\insp{z9-i-2d.eps}%
}

{%%%%% Z9-I-3
Pracujeme so zaokrúhlenými číslami, teda skutočný priemerný bodový zisk mohol byť v~rozsahu od 10,35 (vrátane) po 10,45 (toto číslo sa už zaokrúhľuje na 10,5).
Ak $n$ označuje počet účastníkov súťaže a~$c$ celkový súčet bodov získaných všetkými súťažiacimi, tak predchádzajúcu podmienku zapíšeme takto:
$$
10{,}35 \le \frac{c}{n} < 10{,}45 . \tag{1}
$$
Podobnou úvahou zisťujeme, že ďalšia podmienka zo zadania znamená
$$
10{,}55 \le \frac{c+4}{n} < 10{,}65 . \tag{2}
$$
Vzhľadom na to, že $\frac{c+4}n=\frac{c}n+\frac4n$ a~že sčítanec $\frac{c}n$ je ohraničený v~(1), z~podmienky (2) postupne dostávame
$$
\eqalignno{
10{,}55-\frac{c}n &\le \frac4n < 10{,}65-\frac{c}n , & \cr
0{,}1 &< \frac4n < 0{,}3. & (3) }
$$
Ďalšími ekvivalentnými úpravami nájdeme ohraničenie pre $n$:
$$
\aligned
10 &> \frac{n}4 > \frac{10}3 , \\
40 &> n > \frac{40}3.
\endaligned
$$
Týmto ohraničeniam vyhovujú všetky prirodzené čísla od 14 do 39.

Súťaže sa zúčastnilo najmenej 14 a najviac 39 detí.

\poznamky
Všimnite si, že v~ohraničeniach (3) sú obe nerovnosti ostré:
pre spodný odhad odčítame od 10,55 najväčšiu možnú hodnotu $\frac{c}n$, a~tá je ostro menšia ako 10,45;
pre horný odhad odčítame od hodnoty ostro menšej ako 10,65 najmenšiu možnú hodnotu $\frac{c}n$, a~tá je 10,35.
Informácia o maximálnom počte bodov, ktoré môže získať každý súťažiaci, je nadbytočná.

\odst{Náznak iného riešenia}
K~možným počtom súťažiacich sa dá dopracovať aj skúšaním možností.
Za predpokladu, že hodnoty priemerných bodových ziskov sú presné, by podmienky (1) a~(2) boli nahradené rovnosťami
$$
\frac{c}n=10{,}4,\quad
\frac{c+4}n=10{,}6.
$$
Dosadením prvej rovnosti do druhej a~úpravou dostávame
$\frac4n =0{,}2$, teda $n=20$.
To je vyhovujúci počet súťažiacich a ostatné vyhovujúce možnosti možno nájsť skúšaním okolitých čísel a overovaním podmienok zo zadania.
Nie je nutné postupovať úplne systematicky, stačí nájsť hraničné hodnoty, pre ktoré podmienky platia, ale pre nasledovníka, resp. predchodcu neplatia.
Napr. overenie pre horné ohraničenie počtu súťažiacich vyzerá nasledovne:
\begin{itemize}
\item Podmienka (1) pre $n=39$ a~jej postupné úpravy dávajú
$$
\gathered
10{,}35 \le \frac{c}{39} < 10{,}45 , \\
403{,}65 \le c < 407{,}55 , \\
407{,}65 \le c+4 < 411{,}55 , \\
10{,}4525541 \le \frac{c+4}{39} < 10{,}5525641 .
\endgathered
$$
Tieto ohraničenia nie sú v spore s ohraničeniami (2), iba ich spresňujú.
Teda počet súťažiacich mohol byť 39.
\item Podmienka (1) pre $n=40$ a~jej postupné úpravy dávajú
$$
\gathered
10{,}35 \le \frac{c}{40} < 10{,}45 , \\
414 \le c < 418 , \\
418 \le c+4 < 422 , \\
10{,}45 \le \frac{c+4}{40} < 10{,}55 .
\endgathered
$$
Tieto ohraničenia sú v spore s ohraničeniami (2) -- žiadne číslo nie je ostro menšie ako 10,55 a súčasne väčšie alebo rovné 10,55.
Teda počet súťažiacich nemohol byť 40.
\end{itemize}
}

{%%%%% Z9-I-4
Označme $x$ a~$y$ Karolove dvojciferné čísla a~povedzme, že cifry vymenil v~prvom čísle.
Ak cifry čísla $x$ označíme $a$ a~$b$, tak platí
$$
(10a+b)y-(10b+a)y=4248.
$$
Po úprave dostávame $9(a-b)y=4248$, čiže $(a-b)y=472$.
Odtiaľ vyplýva, že číslo $y$ je dvojciferným deliteľom čísla 472.

Prvočíselný rozklad čísla 472 je $472=2^3\cdot59$, teda jeho jediným dvojciferným deliteľom je 59.
To znamená, že $a-b=8$.
Tejto podmienke vyhovujú nasledujúce dve možnosti:
\begin{itemize}
\item $a=8$, $b=0$, teda $(10a+b)y =80\cdot59 =4720$,
\item $a=9$, $b=1$, teda $(10a+b)y =91\cdot59 =5369$.
\end{itemize}

Karolovi malo správne vyjsť buď 4720, alebo 5369.

\poznamka
Pri zámene cifier v~prvom prípade dostávame 08.
To síce dáva vyhovujúce riešenie ($80\cdot59-8\cdot59=4248$), ale nejedná sa o~dvojciferné číslo.
Nechce sa veriť, že by si Karol tento fakt aj pri svojej nepozornosti nevšimol.
Vylúčenie tejto možnosti preto nepovažujeme za chybu.
}

{%%%%% Z9-I-5
Vzhľadom na smery odvesien trojuholníka $ABC$ opíšeme trojuholníku $A'B'C'$ obdĺžnik ako na nasledujúcom obrázku.
Pri zvyčajnom označení $a=|BC|$ a~$b=|AC|$ majú strany tohto obdĺžnika veľkosti $3a$ a~$3b$:
\insp{z9-I-5.eps}%


Každá zo strán trojuholníka $A'B'C'$ je preponou pravouhlého trojuholníka, ktorého hlavný vrchol leží v~niektorom z~vrcholov opísaného obdĺžnika.
Podľa Pytagorovej vety tak postupne odvodzujeme, že
$$
\aligned
\vert{}A'B'\vert{}^2 &=a^2+(3b)^2 =a^2+9b^2,\\
\vert{}B'C'\vert{}^2 &=(3a)^2+(2b)^2 =9a^2+4b^2,\\
\vert{}A'C'\vert{}^2 &=(2a)^2+b^2 =4a^2+b^2.
\endaligned
$$
Súčtom týchto troch vyjadrení a~s pomocou Pytagorovej vety v~trojuholníku $ABC$ skutočne dostávame
$$
|A'B'|^2 + |B'C'|^2 + |C'A'|^2 =14a^2+14b^2 =14\cdot |AB|^2.
$$
}

{%%%%% Z9-I-6
Najmenší štvorec s vrcholmi v uzlových bodoch má rozmery $1\x 1$, najväčší má rozmery $4\x 4$.
V~rámci týchto obmedzení nájdeme ďalšie prípady, ktoré rozlíšime nasledovne.
\begin{itemize}
\item Štvorce so stranami rovnobežnými vzhľadom k~sieti:
\insp{z9-i-6a.eps}%
\item Štvorce so stranami uhlopriečnymi vzhľadom k~sieti:
\insp{z9-i-6b.eps}%
\item Ostatné štvorce:
\insp{z9-i-6c.eps}%
\end{itemize}
Pre samotné počítanie štvorcov istého typu nie je podstatné ich natočenie, ale iba koľko jednotiek zaberajú vo vodorovnom a zvislom smere.
Teda štvorcov typu $B$ je v~sieti rovnaký počet ako štvorcov typu $E$.
Štvorcov typu $C$ je v~sieti rovnaký počet ako štvorcov typu $G$, resp. $H$.
Štvorcov typu $D$ je v~sieti rovnaký počet ako štvorcov typu $F$, čo je rovnako ako štvorcov typu $I$, resp. $J$.
Stačí teda spočítať štvorce štyroch typov:
\begin{itemize}
\item Štvorec typu $A$ môžeme vo zvislom smere umiestniť štyrmi spôsobmi, vo vodorovnom smere 2\,023 spôsobmi.
Takých štvorcov je
$$
4\cdot2\,023 =8\,092 .
$$
\item Štvorec typu $B$ (resp. $E$) môžeme vo zvislom smere umiestniť tromi spôsobmi, vo vodorovnom smere 2\,022 spôsobmi.
Takých štvorcov je
$$
3\cdot2\,022 =6\,066 .
$$
\item Štvorec typu $C$ (resp. $G$, $H$) môžeme vo zvislom smere umiestniť dvoma spôsobmi, vo vodorovnom smere 2\,021 spôsobmi.
Takých štvorcov je
$$
2\cdot2\,021 =4\,042 .
$$
\item Štvorec typu $D$ (resp. $F$, $I$, $J$) môžeme vo zvislom smere umiestniť jediným spôsobom, vo vodorovnom smere 2\,020 spôsobmi.
Takých štvorcov je
$$
1\cdot2\,020 =2\,020 .
$$
\end{itemize}
Počet štvorcov, ktorých všetky vrcholy sú uzlovými bodmi siete, je
$$
1\cdot8\,092 + 2\cdot6\,066 + 3\cdot4\,042 + 4\cdot2\,020 =40\,430 .
$$
}

{%%%%%   Z4-II-1
...}

{%%%%%   Z4-II-2
...}

{%%%%%   Z4-II-3
...}

{%%%%%   Z5-II-1
Delencami môžu byť jedine dané čísla, teda čísla od 1 do 5.
Deliteľmi môžu byť jedine súčty daných čísel, teda čísla od 3 do 9.
Delenie bezo zvyšku je možné uvedeným spôsobom zostaviť jedine takto:
$$
	3:(1+2), \quad
	4:(1+3), \quad
	5:(1+4), \quad
	5:(2+3).
$$
Poradie sčítancov v~zátvorkách nie je podstatné, vo všetkých prípadoch je podiel rovný~1.

Doplnenie zvyšných dvoch čísel do zvyšných dvoch políčok dáva nasledujúce príklady:
$$
\begin{aligned}
	4\cdot5-3:(1+2) &=19, \\
	2\cdot5-4:(1+3) &= 9, \\
	2\cdot3-5:(1+4) &= 5, \\
	1\cdot4-5:(2+3) &= 3.
\end{aligned}
$$
Poradie činiteľov nie je podstatné.
Deti teda mohli dostať výsledky 19, 9, 5, alebo 3.

\hodnotenie
Po 1~bode za každé správne doplnenie s~výsledkom;
2~body za úplnosť komentára.
\endhodnotenie
}

{%%%%%   Z5-II-2
Poradie jaskyniarov sa postupne mení takto:
$$
\begin{aligned}
	\text{vstup: } & \text{A, B, C, D, E, F, G} \\
	\text{1. dvere: } & \text{B, C, D, E, F, G, A} \\
	\text{2. dvere: } & \text{C, D, E, F, G, A, B} \\
	\vdots \ & \\
	\text{7. dvere: } & \text{A, B, C, D, E, F, G} \\
	\text{8. dvere: } & \text{B, C, D, E, F, G, A} \\
	\text{9. dvere: } & \text{C, D, E, F, G, A, B} \\
	\vdots \ & \\
	\text{14. dvere: } & \text{A, B, C, D, E, F, G} \\
	\text{15. dvere: } & \text{B, C, D, E, F, G, A} \\
	\text{16. dvere: } & \text{C, D, E, F, G, A, B} \\
	\vdots \ &
\end{aligned}
$$
V~každých siedmych dverách je ich poradie rovnaké.

Najbližší násobok siedmich menší ako 100 je 98 ($7\cdot14=98$).
Teda 98. dverami (rovnako ako 7., 14. atď.) prechádzajú jaskyniari v pôvodnom poradí.
Ďalšie, 99. dvere (rovnako ako 1., 8., 15. atď.) otvára Adam.
Ďalšie, sté dvere (rovnako ako 2., 9., 16. atď.) otvára Ben.

\hodnotenie
2~body za pozorovanie, že poradie jaskyniarov sa opakuje s~periódou zodpovedajúcou ich počtu;
2~body za odhalenie pôvodného poradia v~98. dverách;
2~body za doriešenie úlohy.
\endhodnotenie}

{%%%%%   Z5-II-3
Veľkosť (ľavej vodorovnej) čiarkovanej úsečky je rovnaká ako súčet veľkostí bodkovanej úsečky a ~ strany menšieho štvorca.
Teda veľkosť čiarkovanej úsečky je dvojnásobná vzhľadom na veľkosť bodkovanej úsečky, resp. strana väčšieho štvorca je dvojnásobná vzhľadom na stranu menšieho štvorca.
Celý útvar tak možno rozdeliť na políčka zhodné s menším bielym štvorcom:
\insp{z5-ii-3a.eps}%

Väčší biely štvorec je tvorený 4 takými políčkami.
Bielych políčok je dokopy 5 ($1+4=5$).
Všetkých políčok je 36 ($6\cdot6=36$).
Sivých políčok je 31 ($36-5=31$).

Sivá časť má obsah 62\,cm$^2$, teda obsah jedného políčka je 2\,cm$^2$ ($62:31=2$).
Bielych políčok je 5, teda biela časť obrazu má obsah 10\,cm$^2$ ($5\cdot2=10$).

\hodnotenie
1~bod za poznatok o~pomere strán väčšieho a mešieho bieleho štvorca v základnom obrázku; 1~bod za rozdelenie útvaru a~počty bielych a~sivých políčok;
2~body za obsah jedného políčka;
2~body za obsah bielej časti obrazu.
\endhodnotenie}

{%%%%%   Z6-II-1
Danino číslo (súčet Alexovho a~Barborinho čísla) bolo o~30 väčšie ako Cyrilovo (dvojnásobok daného čísla).

Ak by dané číslo bolo menšie ako 50, tak by Barborino číslo bolo nula (zaokrúhlenie na stovky).
V takom prípade by Danino číslo bolo rovnaké ako Alexovo, a to nemôže byť väčšie ako Cyrilovo číslo.
Teda dané číslo bolo aspoň 50.

Zaokrúhlené Alexovo a~Barborino číslo, a~ preto aj~Danino číslo, bolo násobkom desiatich.
Tiež rozdiel Daninho a Cyrilovho čísla bol násobkom desiatich.
Teda dané číslo bolo násobkom piatich.

Pre čísla od 50 do 99, ktoré sú násobkami piatich, preberieme všetky možnosti:
\bgroup
\def\ctr#1{\hfil\quad\!#1\quad\!\hfil}
$$
\begintable
Číslo\|50|55|60|65|70|75|80|85|90|95\crthick
Alex\|50|60|60|70|70|80|80|90|90|100\cr
Barbora\|100|100|100|100|100|100|100|100|100|100\cr
Cyril\|100|110|120|130|140|150|160|170|180|190\cr
Dana\|150|160|160|170|170|180|180|190|190|200\crthick
Eva\|50|50|40|40|\bf 30|\bf 30|20|20|10|10\endtable
$$
\egroup

Žiaci mohli dostať buď číslo 70, alebo 75.

\hodnotenie
2 body za čiastočné pozorovania týkajúce sa daného čísla; po 1 bode za
každú správnu možnosť; 2 body za úplnosť komentára.
\endhodnotenie
}

{%%%%%   Z6-II-2
Najprv určíme počet vodorovných radov útvaru.
Tých je toľko, koľko je celých čísel od 4 do 12, a to je 9.

Obsah trojuholníka $ABC$ je polovica z obdĺžnika tvoreného 9 radmi po 12 štvorcoch.
Obsah trojuholníka $ABC$ je
$$
\frac12\cdot12\cdot9 =54\,(\Cm^2) .
$$

\hodnotenie
3 body za počet radov útvaru; 3 body za obsah trojuholníka.
\endhodnotenie
}

{%%%%%   Z6-II-3
Najprv porovnáme výkonnosti matematikov rôznych úrovní:
\begin{itemize}
\item Matematik 3.~úrovne vyrieši za deň dvojnásobok toho, čo matematik 2.~úrovne.
\item Matematik 4.~úrovne vyrieši za deň dvojnásobok toho, čo matematik 3.~úrovne, t.\,j.~štvornásobok toho, čo matematik 2.~úrovne.
\item Matematik 5.~úrovne vyrieši za deň dvojnásobok toho, čo matematik 4.~úrovne, t.\,j.~štvornásobok toho, čo matematik 3.~úrovne, t.\,j.~osemnásobok toho, čo matematik 2.~úrovne.
\end{itemize}

Traja matematici 5.~úrovne vyriešia za deň toľko, čo 24 matematikov 2.~úrovne ($3\cdot8=24$).
To je podľa zadania o~1000 príkladov viac, než vyriešia štyria matematici 2.~úrovne.
Teda 20 matematikov 2.~úrovne vyrieši za deň 1000 príkladov ($24-4=20$).

Jeden matematik 2.~úrovne vyrieši za deň 50 príkladov ($1000:20=50$).

\hodnotenie
2 body za porovnanie matematikov rôznych úrovní; 2 body za pomocné výpočty; 2 body za výsledok.
\endhodnotenie
}

{%%%%%   Z7-II-1
Deti zatlieskali, keď padlo číslo 2, 4, alebo 6.
Deti zadupali, keď padlo číslo 3, alebo 6.

Pretože deti dupali celkom trikrát, mohli padnúť jedine nasledujúce pätice:
$$
6,6,6,*,*; \quad 6,6,3,*,*; \quad 6,3,3,*,*; \quad 3,3,3,*,*,
$$
kde $*$ značí čísla rôzne od 3 a~6.

Keďže deti tlieskali celkom trikrát, boli v pätici práve tri párne čísla.
Posledné z~vyššie uvedených pätíc takto doplniť nemožno, zvyšné tri doplniť možno.
Aby súčet čísel bol rovný 20, je doplnenie každej pätice určené jednoznačne:
$$
6,6,6,1,1; \quad 6,6,3,4,1; \quad 6,3,3,4,4.
$$

To sú všetky možné pätice čísel, ktoré mohli deťom padnúť (bez ohľadu na poradie).

\hodnotenie
2 body za prvú správnu päticu a potom po 1 bode za každú ďalšiu; 2 body za úplnosť komentára.
\endhodnotenie
}

{%%%%%   Z7-II-2
Potrebujeme určiť počet radov útvaru, t.\,j. výšku trojuholníka $ABC$ vzhľadom na stranu $AB$.
Tento údaj je ukrytý v~informácii o~obvode útvaru.

Spodný aj horný rad (s jedným štvorcom) prispievajú do obvodu útvaru 3\,cm, ostatné rady (s~dvoma štvorcami) prispievajú 4\,cm.
Obvod útvaru bez príspevku spodného a~horného štvorca je 172\,cm ($178-2\cdot3=172$).
Teda počet radov s dvoma štvorcami je 43 ($172:4=43$) a radov celkom je 45 ($2+43=45$).

Obsah trojuholníka $ABC$ je
$$
\frac12\cdot1\cdot45 =22{,}5\,(\Cm^2).
$$

\hodnotenie
2 body za čiastkové postrehy súvisiace s obvodom útvaru; 2 body za
počet radov útvaru; 2 body za obsah trojuholníka.
\endhodnotenie
}

{%%%%%   Z7-II-3
Z druhej informácie o priemernom umiestnení vyplýva, že dievčatá boli buď na druhom a štvrtom mieste a chlapci boli na prvom, treťom a piatom mieste alebo že dievčatá boli na prvom a poslednom mieste. Žiadne z dievčat však nemohlo byť posledné, na základe tretej informácie. Teda posledný musel byť Tomáš a dievčatá sa umiestnili na druhom a štvrtom mieste. 

Odtiaľ a zo štvrtej informácie vyplýva, že Lukáš bol prvý a Anna štvrtá (aby vôbec bol nejaký priestor medzi nimi).
Jurko bol teda tretí a Fiona druhá.

Cieľové poradie detí bolo
$$
\text{1. Lukáš,\quad 2. Fiona,\quad 3. Jurko,\quad 4. Anna,\quad 5. Tomáš.}
$$

\hodnotenie
2 body za zistenie, že sa striedajú dievčatá a chlapci; 2 body za cieľové poradie
detí; 2 body za kvalitu komentára.
\endhodnotenie
}

{%%%%%   Z8-II-1
V oboch prípadoch najprv zistíme, ako sa príslušná vlastnosť medzi postupne tvorenými číslami opakuje:

\smallskip
a)
Číslo je deliteľné tromi práve vtedy, keď je jeho ciferný súčet deliteľný tromi.
Do ciferných súčtov prispievajú opakovane čísla od 1 do 9, a~tie majú po delení tromi zvyšky opakovane 1, 2, 0.
Vo vytvorených číslach sa tak zvyšok po delení ciferného súčtu troma opakuje po trojiciach takto:
$$
1, 0, 0; 1, 0, 0; \dots
$$
V~každej takej trojici sú dve čísla deliteľné tromi.


Najbližší násobok troch menší ako 2024 je 2022 $(=3\cdot674)$.
Teda na prvých 2022 riadkoch je 1348 čísel deliteľných tromi (674 trojíc po dvoch číslach deliteľných tromi).
Číslo na 2023.~riadku deliteľné tromi nie je, číslo na 2024. riadku deliteľné tromi je.

Čísla deliteľné tromi sú na 1349 riadkoch.

\smallskip
b)
Číslo je deliteľné štyrmi práve vtedy, keď je jeho posledné dvojčíslie deliteľné štyrmi.
Vo vytvorených číslach sa posledné dvojčíslia opakujú takto:
$$
\_1, 12, 23, 34, 45, 56, 67, 78, 89; 91, 12, 23, \dots
$$
Až na úplne prvé číslo, ktoré je jednociferné, sa posledné dvojčíslia tvorených čísel opakujú po deviatich.
V~každej devätici sú dve čísla deliteľné štyrmi, a to 12 a~56.

Najbližší násobok deviatich menší ako 2024 je 2016 $(=9\cdot224)$.
Teda na prvých 2016 riadkoch je 448 čísel deliteľných štyrmi (224 devätíc po dvoch číslach deliteľných štyrmi).
Na zvyšných ôsmich riadkoch sú koncové dvojčíslia 91, 12, 23, 34, 45, 56, 67, 78, medzi ktorými sú dve deliteľné štyrmi.

Čísla deliteľné štyrmi sú na 450 riadkoch.

\poznamka
Aj v prípade b) je možné sledovať postupnosť zvyškov po delení štyrmi.
Tie sa pri tvorených číslach opakujú po deväticiach takto:
$$
1, 0, 3, 2; 1, 0, 3, 2, 1; \dots
$$
V~každej takej devätici je jedno číslo deliteľné štyrmi.
Ďalší postup by bol podobný vyššie uvedenému.

\hodnotenie
V každom z oboch prípadov dajte 1 bod za odhalenie opakujúceho sa
vzoru; 1 bod za výsledok; 1 bod za kvalitu komentára.
\endhodnotenie
}

{%%%%%   Z8-II-2
Pomerné zastúpenie červených lodí medzi všetkými hračkami je
$$
\frac{75}{100}\cdot\frac14 =\frac34\cdot\frac14 =\frac3{16}.
$$
Pomerné zastúpenie červených áut medzi všetkými hračkami je
$$
\frac{40}{100}\cdot\frac34 =\frac25\cdot\frac34 =\frac3{10}.
$$
Pomerné zastúpenie červených hračiek medzi všetkými je
$$
\frac3{16}+\frac3{10}
=\frac{15+24}{80}
=\frac{39}{80}.
$$
Počet všetkých hračiek je teda deliteľný 80.

Ak by všetkých hračiek bolo 80, bolo by červených 39 a ostatných 41.
V~takom prípade by červených hračiek bolo o~2 menej ako tých s~inou farbou.
Červených je však o~10 menej ako ostatných a~$10=5\cdot2$.
Teda všetkých hračiek je $5\cdot80=400$.

V~obchode je 400 hračiek.

\poznamka
Úvahu záverečnej časti riešenia možno nahradiť rovnicou
$$
\frac{39}{80}h =\frac{41}{80}h -10,
$$
kde $h$ označuje počet hračiek.
Úpravami dostávame $\frac{2}{80}h=10$, a teda $h=400$.
\hodnotenie
3 body za pomerné zastúpenie červených hračiek medzi všetkými hračkami; 3 body za počet hračiek. 
\endhodnotenie
}

{%%%%%   Z8-II-3
Pri rozbore úlohy využijeme to, že deltoid je osovo súmerný podľa jednej svojej uhlopriečky, a teda má navzájom kolmé uhlopriečky.
Kosoštvorec možno chápať ako špeciálny {deltoid}, ktorý má všetky strany navzájom zhodné a je súmerný podľa oboch uhlopriečok.

Deltoid $ABCD$ zo zadania je osovo súmerný podľa uhlopriečky $AC$.
Delenie deltoidu opíšeme pomocou bodov $E$, $F$, $G$, $H$, $I$ takých, že $E$ leží na uhlopriečke $AC$, body $F$, $G$ ležia na stranách $AB$, $BC$ a~body $H$, $I$ sú osovo súmerné vzhľadom na $G$, $F$ podľa priamky $AC$.
Nezávisle na polohe bodu $E$ na uhlopriečke $AC$ sú štvoruholníky $AFEI$ a~$EGCH$ deltoidy (príp. kosoštvorce) a~štvoruholníky $EFBG$ a~$EHDI$ sú zhodné.
\insp{z8-ii-3a.eps}%

V kosoštvorci sú každé dva protiľahlé vnútorné uhly zhodné.
Keďže uhly $BCD$ a~$BAD$ zhodné nie sú, štvoruholníky $AFEI$ a~$EGCH$ nemôžu byť zhodné kosoštvorce.
Uvažujeme teda o~takom delení, aby zhodnými kosoštvorcami boli štvoruholníky $EFBG$ a~$EHDI$.
Vzhľadom na to, že uhlopriečky v kosoštvorci sú jeho osami súmernosti, je také delenie určené jednoznačne --
v~kosoštvorci $EFBG$ je uhlopriečka $BE$ osou uhla $ABC$ a~uhlopriečka $GF$ je osou úsečky $BE$, v~kosoštvorci $EHDI$ je tomu obdobne.
\insp{z8-ii-3b.eps}%

\smallskip
Konštrukcia deltoidu:
    \item{1)} úsečka $AC$ dĺžky 15\,cm,
    \item{2)} kružnica so stredom $A$ a~polomerom 11\,cm,
    \item{3)} kružnica so stredom $C$ a~polomerom 6\,cm,
    \item{4)} body $B$, $D$ sú priesečníky kružníc 2) a~3).
\insp{z8-ii-3c.eps}%
\smallskip

Konštrukcia delenia:
    \item{5)} os uhla $ABC$,
    \item{6)} bod $E$ je priesečníkom priamky 5) s~úsečkou $AC$,
    \item{7)} os úsečky $BE$,
    \item{8)} body $F$, $G$ sú postupne priesečníky priamky 7) so stranami $AB$, $BC$,
    \item{9)} kolmica na priamku $AC$ idúca bodmi $F$, $G$,
    \item{10)} body $H$, $I$ sú postupne priesečníky kolmíc 9) so stranami $CD$, $DA$,
    \item{11)} štvoruholníky $AFEI$, $EGCH$, $EFBG$, $EHDI$.
\insp{z8-ii-3d.eps}%

\poznamka
Uvedená konštrukcia je odvodená z predchádzajúceho rozboru úlohy.
V~útvare sú ďalšie vzťahy, ktoré je možné tiež pri konštrukcii použiť.
Napr. platí, že body $F$, $G$, $H$, $I$ ležia na kružnici so stredom v~bode $E$.

\hodnotenie
3 body za rozbor úlohy a určenie podstatných vzťahov; 1 bod za konštrukciu deltoidu $ABCD$; 2 body za konštrukciu delenia. Pre zisk plného počtu
bodov nie je potrebné vysvetľovať, či existuje iné riešenie.
\endhodnotenie

}

{%%%%%   Z9-II-1
Označme $d$ a~$ch$ počty dievčat a~chlapcov počas prehliadky.
Vzťahy zo zadania je možné zapísať ako
$$
ch =2(d-15), \quad
d-15 =5(ch-45) ,
\tag{$*$}
$$

Počet dievčat po prehliadke vyjadrený z~prvého vzťahu je $d-15=\frac12 ch$.
Spoločne s druhým vzťahom dostávame rovnicu s neznámou $ch$,
$$
\frac12ch =5(ch-45),
$$
ktorú vyriešime:
$$
\begin{aligned}
	ch &=10ch-450, \\
	9ch &=450, \\
	ch &=50 .
\end{aligned}
$$

Odtiaľ dostávame $\frac12ch =5(ch-45) =25$, čo zodpovedá $d-15$.
Teda $d=25+15=40$.
Počas prehliadky bolo v~miesnosti múzea 40 dievčat.

\poznamky
Sústavu rovníc ($*$) je možné riešiť rôznymi spôsobmi.
Napr. dosadenie prvej rovnice do druhej dáva rovnicu s~neznámu $d$, ktorú vyriešime nasledovne:
$$
\begin{aligned}
	d-15 &= 5(2(d-15) -45), \\
	d-15 &= 10d -375, \\
	9d &= 360, \\
	d &=40.
\end{aligned}
$$

Iné riešenie môže spočívať v uvedomení si faktu, že počet dievčat po tom, čo ich časť odišla, bol deliteľný piatimi.
Pretože ich odišlo 15, bol taktiež počet dievčat počas prehliadky deliteľný piatimi.
Je teda možné postupne za $d$ dosadzovať násobky piatich väčšie ako 15, z~prvej rovnice v~($*$) vypočítať $ch$ a~overiť, či platí druhá rovnica.
Jedinou vyhovujúcou možnosťou je $d=40$.

\hodnotenie
2~body za formuláciu vzťahov ($*$);
2~body za doriešenie sústavy rovníc;
2~body za kvalitu komentára.
Pri skúšaní možností zvážte úplnosť komentára,
náhodne odhalené nezdôvodnené riešenie hodnoťte 2~bodmi.
\endhodnotenie}

{%%%%%   Z9-II-2
V rohoch obdĺžnika sú zhodné štvorce so stranou dĺžky 10\,cm.
Teda zvislá strana obdĺžnika meria $18+10=28$\,(cm).

Na stranách obdĺžnika je iba jedna neznáma úsečka (avšak dvakrát), a~tu označíme $x$.
Zo zhodností čiastkových trojuholníkov vieme určiť aj časti uhlopriečky obdĺžnika.
Táto uhlopriečka delí obdĺžnik na dva zhodné pravouhlé trojuholníky, ktorého strany sú popísané takto:
\insp{z9-ii-2a.eps}%

Podľa Pytagorovej vety platí
$$
(18+x)^2 = 28^2 +(10+x)^2 .
\tag{$*$}
$$
Po umocnení a~úpravách dostávame:
$$
\eqalignno{
	18^2 +36x +x^2 &= 28^2 +100 +20x +x^2, & \cr
	16x &= 28^2 +100 -18^2 = & (**) \cr
	    &= 784 +100 -324 = 560. & \cr
}
$$
Teda $x =560/16 = 35$ a~vodorovná strana obdĺžnika meria $35+10=45$\,(cm).

Rozmery obdĺžnika sú 28\,cm a ~45\,cm.

\ineriesenie
Ďalej používame rovnaké značenie ako v predchádzajúcom riešení.
Obsah daného obdĺžnika v~cm$^2$ je rovný $28(10+x)$.
Obdĺžnik pozostáva z desiatich menších častí, ktoré všetky majú aspoň jeden pravý uhol a aspoň jednu stranu dĺžky 10\,cm.
Z~týchto častí je možné zložiť obdĺžnik ako na obrázku:
\insp{z9-ii-2b.eps}%

Obsah tohto obdĺžnika v~cm$^2$ je rovný $20(28+x)$.
Preusporiadaním častí sa však obsah nezmenil.
Takže dostávame rovnicu,
$$
28(10+x) = 20(28+x),
\tag{$*{*}*$}
$$
ktorú vyriešime:
$$
\begin{aligned}
  280+28x &=560+20x, \\
	8x &= 280, \\
	x &=35.
\end{aligned}
$$
Vodorovná strana obdĺžnika meria $35+10=45$\,(cm); rozmery obdĺžnika sú 28\,cm a ~ 45\,cm.

\hodnotenie
1~bod za zvislú stranu obdĺžnika;
1~bod za vyjadrenie ďalších častí pomocou neznámej a~formulácii ($*$) alebo ($*{*}*$);
2~body za doriešenie a~výsledok;
2~body za kvalitu komentára.
\endhodnotenie

\poznamky
Všetky čísla v~($**$) sú deliteľné štyrmi,
čo pre druhé mocniny vidíme bez umocnenia takto: $28^2=(2\cdot14)^2=4\cdot14^2$, $18^2=(2\cdot9)^2=4\cdot9^2$.
Rovnica ($**$) je teda ekvivalentná s~$4x =14^2+25-9^2$.
Pri ručnom počítaní je toto vyjadrenie výhodnejšie.

Namiesto obdĺžnika v druhom riešení úlohy sa dá obmedziť na jeho polovicu, tzn. trojuholník ako vyššie.
V~tomto duchu by všetky diskutované obsahy boli polovičné, teda namiesto rovnice ($*{*}*$) by sme odvodili ekvivalentnú rovnicu
$14(10+x)=10(28+x)$.
Pravú stranu tejto rovnice je možné interpretovať ako
$$
S=\frac12\cdot r\cdot o,
$$
kde $o=2(28+x)$ je obvod trojuholníka a~$r=10$ polomer jemu vpísanej kružnice
(stredom tejto kružnice je spoločný bod vyznačených úsečiek vo vnútri trojuholníka).
Tento vzorec možno nájsť v ~ povolených tabuľkách, teda na ňom založené riešenie by malo byť považované za správne.
}

{%%%%%   Z9-II-3
Z~daných číslic Iveta vytvorila 4 jednociferné čísla.
Všetky tieto čísla zaberajú v~Ivetinom dlhom čísle 4 miesta.

Z~daných číslic Iveta vytvorila $4^2=16$ dvojciferných čísel.
Všetky tieto čísla zaberajú v~Ivetinom dlhom čísle $2\cdot16=32$ miest.
Posledná číslica posledného dvojciferného čísla je na 36. mieste ($4+32=36$).

Z~daných číslic Iveta vytvorila $4^3=64$ trojciferných čísel.
Všetky tieto čísla zaberajú v~Ivetinom dlhom čísle $3\cdot64=192$ miest.
Posledná číslica posledného trojciferného čísla je na 228. mieste ($36+192=228$).

Z~daných číslic Iveta vytvorila $4^4=256$ štvorciferných čísel.
Všetky tieto čísla zaberajú v~Ivetinom dlhom čísle $4\cdot256=1024$ miest.
Posledná číslica posledného štvorciferného čísla je na 1252. mieste ($228+1024=1252$).

Do 1286. miesta chýba 34 miest, a~tie sú obsadené päťcifernými číslami.
Pritom $34:5$ je 6 a zvyšok 4.
Teda hľadaná číslica je 4. číslicou v~7. päťcifernom čísle, ktoré Iveta vytvorila.
Tieto čísla zoradené vzostupne sú:
$$
11111, 11113, 11115, 11117, 11131, 11133, 111\underline{3}5.
$$

V~Ivetinom mimoriadne dlhom čísle je na 1286. mieste číslica 3.

\hodnotenie
2~body za počty jedno-, dvoj-, troj- a štvorciferných čísel;
2~body za počty miest týchto čísel v~Ivetinom dlhom čísle;
2~body za určenie hľadanej číslice a~kvalitu komentára.
\endhodnotenie}

{%%%%%   Z9-II-4
Po premiestnení guľôčok je jedno vrecúško prázdne a v ostatných štyroch sú ich rovnaké počty.
Dokopy je guľôčok 52, teda po premiestnení je v~neprázdnych vrecúškach po 13 guľôčok ($52:4=13$).
Preto pôvodne nemohlo byť v žiadnom vrecúšku viac ako 13 guľôčok.

\smallskip
a)
Možné pôvodné počty guľôčok vo vreciach zodpovedajú možným vyjadreniam čísla 52 ako súčtu piatich navzájom rôznych nezáporných celých čísel, ktoré nie sú väčšie ako 13.
Skúšaním je možné nájsť tieto možnosti:
$$
52\ =\
13+12+11+10+6\ =\
13+12+11+9+7\ =\
13+12+10+9+8.
$$

\smallskip
b)
Ak by v~žiadnom vrecúšku nebolo viac ako 12 guľôčok, potom by vo všetkých dokopy bolo najviac $12+11+10+9+8=50$ guľôčok.
Teda v niektorom vrecúšku muselo byť 13 guľôčok.

Ak by v~žiadnom vrecúšku nebolo 12 guľôčok, potom by vo všetkých dohromady bolo najviac $13+11+10+9+8=51$ guľôčok.
Teda v niektorom vrecúšku muselo byť 12 guľôčok.


\poznamka
Rozdelenia uvedené v~časti a) sú jediné možné.
To je možné zdôvodniť systematickým dosadzovaním postupne sa zmenšujúcich navzájom rôznych sčítancov nie väčších ako 13 a kontrolou ich súčtu.
V~kazdej z~uvedených možností sa naozaj vyskytuje číslo 12.

\hodnotenie
1~bod za nejaké vyhovujúce rozdelenie;
2~body za zdôvodnenie, že v~žiadnom vrecúšku nebolo viac ako 13 guľôčok;
3~body za zdôvodnenie, že v~niektorom vrecúšku bolo 12 guľôčok.
Riešenia časti b) založené na predchádzajúcej poznámke môžu byť hodnotené plným počtom bodov v~závislosti od úplnosti komentára k~časti a).
Riešenia obsahujúce všetky možnosti bez zdôvodnenia ohodnoťte 2~bodmi.
\endhodnotenie}

{%%%%%   Z9-III-1
Na vyriešenie úlohy nie je nutné súčty vyčísľovať, stačí rozpoznať vzťahy medzi nimi.
Súčet prvých dvanástich čísel označme $s$, t.\,j.
$$
1+2+\cdots+11+12 =s.
$$
Súčet 2. až 13. čísla je o~12 väčší ako $s$, pretože každý zo sčítancov je o~1 väčší ako v~predchádzajúcom prípade:
$$
2+3+\cdots+12+13 =s+12 .
$$
Súčet 3. až 14. čísla je opäť o~12 väčší ako predchádzajúci súčet, teda o~$2\cdot 12$ väčší ako $s$:
$$
3+4+\cdots+13+14 =(s+12)+12 =s+2\cdot12 .
$$
Súčet dvanástich po sebe idúcich čísel počnúc $k$ je $s+(k-1)\cdot12$.
Odtiaľ vyplýva, že ak súčet dvanástich po sebe idúcich čísel je $s+12n$, potom posledné z~týchto čísel je $12+n$.

Lukáš mohol získať 708 bodov navyše, pričom $708=12\cdot59$.
Posledné zo sčítaných čísel by teda muselo byť $12+59=71$.
Súťažných úloh bolo 71.

\ineriesenie
Súčet prvých dvanástich čísel je
$$
1+2+\cdots+11+12 =13\cdot6 =78 . \tag{1}
$$
Súčet dvanástich po sebe idúcich čísel počnúc $k$ je
$$
k+(k+1)+\cdots+(k+10)+(k+11) =(2k+11)\cdot6 . \tag{2}
$$

Lukáš mohol získať navyše 708 bodov, celkom teda mohol mať $78+708=786$ bodov.
Úpravami rovnosti $(2k+11)\cdot6 =786$ dostávame $k=60$.
Posledné zo sčítaných čísel na ľavej strane rovnosti (2) by muselo byť $60+11=71$.
Súťažných úloh bolo 71.

\poznamka
V~úpravách (1) a~(2) využívame to, že súčet prvého a~posledného čísla je rovnaký ako súčet druhého a~predposledného atď. a~že takýchto dvojíc je 6.
Bez tohto postrehu je možné súčet (2) vyjadriť pomocou (1) ako $12k +78-12 =12k+66$.

\hodnotenie
2~body za čiastkové postrehy týkajúce sa súčtu dvanástich po sebe idúcich čísel;
2~body za formuláciu problému pomocou neznámej ($12n=708$ u~prvého riešenia, resp. $(2k+11)\cdot6=786$ u~druhého);
2~body za doriešenie a~kvalitu komentára.
\endhodnotenie
}

{%%%%%   Z9-III-2
Vnútorné uhly trojuholníka $ABC$ označíme zvyčajným spôsobom, t.\,j. $\alpha$, $\beta$, $\gamma$.
Osi vnútorných uhlov trojuholníka $ABC$ sú osami súmerností týchto uhlov, teda platí
$|\uhol BAD| =|\uhol DAC| =\frac\alpha2$ atď.
Trojuholníky $DEF$, $DGH$, $DIJ$ sú rovnostranné a~body $A$, $B$, $C$ sú stredy strán protiľahlých spoločnému vrcholu $D$.
Priamky $DA$, $DB$, $DC$ sú teda osami súmerností týchto trojuholníkov a~platí
$|\uhol EDA| = |\uhol ADF| = 30\st$ atď.
(pozri obrázok na ďalšej strane).

Veľkosť uhla $ADC$ je $30\st+51\st+30\st =111\st$ a~pre súčet vnútorných uhlov v~trojuholníku $ADC$ platí
$$
\frac\alpha2+111\st+\frac\gamma2 =180\st , \tag{1}
$$
čo po úprave dáva $\alpha+\gamma=138\st$.
Súčasne pre vnútorné uhly trojuholníka $ABC$ platí
$$
\alpha+\beta+\gamma=180\st, \tag{2}
$$
čiže $\alpha+\gamma=180\st-\beta$.
Z~dvojakého vyjadrenia $\alpha+\gamma$ dostávame $138\st =180\st-\beta$, teda
$$
\beta=42\st .
$$

Podobne vyjadríme veľkosť uhla $BDC$ ako $30\st+66\st+30\st =126\st$ a~zo súčtu vnútorných uhlov v~trojuholníku $BDC$
$$
\frac\beta2+126\st+\frac\gamma2 =180\st \tag{3}
$$
určíme $\beta+\gamma=108\st$.
Zároveň z~(2) vyplýva $\beta+\gamma=180\st-\alpha$.
Dokopy dostávame $108\st =180\st-\alpha$, teda
$$
\alpha=72\st .
$$

Dosadením $\alpha$ alebo $\beta$ do niektorej z predchádzajúcich rovností vyjadríme zvyšný uhol $\gamma$.
Napr. dosadením do (2) dostávame $72\st+42\st+\gamma=180\st$, a~teda
$$
\gamma=66\st .
$$

Veľkosti vnútorných uhlov trojuholníka $ABC$ sú 72\st, 42\st a~66\st.
\insp{z9-iii-2a.eps}%

\poznamky
Okrem podmienok (1) a~(3) je možné použiť aj obdobnú podmienku vzhľadom na~trojuholník $ADB$.
Na tento účel je potrebné vyjadriť uhol $FDG$ ako doplnkový uhol do plného uhla s~vrcholom $D$, ktorého veľkosť je $360\st-3\cdot60\st-51\st-66\st =63\st$.
Potom veľkosť uhla $ADB$ je $30\st+63\st+30\st =123\st$ a~pre súčet vnútorných uhlov v~trojuholníku $ADB$ platí
$$
\frac\alpha2+123\st+\frac\beta2 =180\st . \tag{4}
$$

Ktorékoľvek tri zo štyroch rovníc (1) až (4) určujú jednoznačne riešenie a~takto možno úlohu tiež riešiť.
Napr. sústava rovníc (1), (3), (4) je ekvivalentná sústave
$$
\alpha+\gamma =138\st, \quad
\beta+\gamma =108\st, \quad
\alpha+\beta =114\st.
$$
Odčítaním tretej rovnice od prvej dostávame $\gamma-\beta=24\st$.
Odčítaním tejto rovnice od druhej dostávame $2\beta=84\st$, teda $\beta=42\st$.
Po dosadení tohto výsledku do druhej, resp. tretie rovnice určíme $\gamma=66\st$, resp. $\alpha=72\st$.

\hodnotenie
2~body za úvodné pozorovania súvisiace s osami uhlov;
2~body za určenie pomocných uhlov a~zostavenie rovníc;
2~body za doriešenie a~kvalitu komentára.
\endhodnotenie
}

{%%%%%   Z9-III-3
Situáciu zo zadania si znázorníme na obrázku:
\insp{z9-iii-3a.eps}%

Úsečka $KU$ je strednou priečkou trojuholníka $ABC$.
Trojuholníky $KNU$ a~$KUC$ majú spoločnú stranu $KU$ a~zhodné výšky na túto stranu, teda trojuholníky $KNU$ a~$KUC$ majú rovnaký obsah (a to nezávisle od polohy bodu $N$ na úsečke $ AB$).

Štvoruholník $UMLK$ je časťou päťuholníka $MLKNU$ a~pomer ich obsahov je $3:7$.
Teda pomer obsahov štvoruholníka $UMLK$ a~trojuholníka $KNU$, resp. štvoruholníka $UMLK$ a~trojuholníka $KUC$ je $3:4$.
Štvoruholník $UMLK$ je časťou trojuholníka $KUC$, teda pomer obsahov trojuholníkov $LMC$ a~$KUC$ je $1:4$.

Úsečky $LM$ a~$KU$ sú rovnobežné, teda trojuholníky $LMC$ a~$KUC$ sú podobné.
Pomer veľkostí zodpovedajúcich dvojíc úsečiek je preto $1:2$, teda aj hľadaný pomer veľkostí úsečiek $LM$ a~$KU$ je $1:2$.

\poznamka
Pozorovanie z~prvého odseku predchádzajúceho riešenia je možné využiť tiež tak, že bod $N$ umiestnime do stredu úsečky $AB$:
\insp{z9-iii-3b.eps}%

Trojuholník $ABC$ je tak strednými priečkami $KN$, $NU$, $UK$ rozdelený na štyri navzájom zhodné trojuholníky.
Platí, že pomer obsahov štvoruholníka $BUKA$ a~trojuholníka $ABC$ je $3:4$.
Naopak, ak je $KU\|AB$ a~pomer obsahov $BUKA$ a~$ABC$ je $3:4$, potom $KU$ je strednou priečkou trojuholníka $ABC$.
Podobnou úvahou vzhľadom na štvoruholník $UMLK$ a~trojuholník $KUC$ možno odvodiť, že $LM$ je strednou priečkou trojuholníka $KUC$.

Prikladáme ilustráciu, v ktorej sú všetky vyššie spomínané pomery dobre zrejmé:
\insp{z9-iii-3c.eps}%

\hodnotenie
1~bod za pozorovanie, že trojuholníky $KNU$ a~$KUC$ majú rovnaký obsah;
3~body za ďalšie čiastkové závery týkajúce sa pomerov obsahov;
2~body za výsledok a~kvalitu komentára.
\endhodnotenie
}

{%%%%%   Z9-III-4
Pre akékoľvek číslo $a$ je možné z~druhej podmienky vyjadriť $b$ a následne z~prvej podmienky $c$.
Pre akékoľvek prirodzené číslo $a$ je $b=a+34$ tiež prirodzeným číslom, avšak $c =\frac{b^2}a$ prirodzené číslo byť nemusí.
Po úprave
$$
c %=\frac{b^2}{a}
=\frac{(a+34)^2}{a}
=\frac{a^2+68a+34^2}{a}
=a+68+\frac{34^2}{a}
\tag{1}
$$
zisťujeme, že $c$ je prirodzeným číslom práve vtedy, keď $a$ je deliteľom čísla $34^2$.

Trojciferné delitele čísla $34^2=2^2\cdot17^2$ sú dva, a to $17^2=289$ a~$2\cdot17^2=578$.
To sú teda jediné možné hodnoty $a$.
Pre tieto možnosti vyjadríme $b$ a~$c$ a~overíme, či dostaneme trojciferné čísla:

\smallskip
\item{$\bullet$} Pre $a=289$ je $b=a+34=323$ a~$c=a+68+\frac{34^2}{a}=a+72=361$.
\item{$\bullet$} Pre $a=578$ je $b=a+34=612$ a~$c=a+68+\frac{34^2}{a}=a+70=648$.

\smallskip\noindent
V~oboch prípadoch sú všetky čísla trojciferné.
Trojice čísel $a$, $b$, $c$ vyhovujúce podmienkam zo zadania sú 289, 323, 361 a~578, 612, 648.

\hodnotenie
2~body za vyjadrenie (1) alebo analogickú úpravu;
2~body za trojciferné delitele čísla $34^2$ a~zodpovedajúci rozbor možností;
2~body za výsledok a~kvalitu komentára.
\endhodnotenie
}



