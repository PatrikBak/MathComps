{%%%%%   Z4-I-1
...}
\podpis{...}

{%%%%%   Z4-I-2
...}
\podpis{...}

{%%%%%   Z4-I-3
...}
\podpis{...}

{%%%%%   Z4-I-4
...}
\podpis{...}

{%%%%%   Z4-I-5
...}
\podpis{...}

{%%%%%   Z4-I-6
...}
\podpis{...}

{%%%%%   Z5-I-1
Zajac sa zúčastnil na pretekoch dlhých 2024 metrov.
Zo štartovej čiary sa odrazil ľavou nohou a po celý čas pretekov pravidelne striedal ľavú, pravú a obe nohy.
Keď sa zajac odrazil ľavou nohou, skočil 35\,dm, keď sa odrazil pravou nohou, skočil 15\,dm, a keď sa odrazil oboma nohami, skočil 61\,dm.

Koľko skokov zajac urobil, kým dorazil do cieľa?
Ktorou nohou sa odrazil pri cieľovom skoku?
}
\podpis{Libuše Hozová}

{%%%%%   Z5-I-2
Zuzka postavila zo šestnástich rovnako veľkých kociek štvorec s rozmermi $4\times 4$ kocky.
S ďalšími rovnako veľkými kockami pokračovala v stavaní.
Kocky na seba stavala tak, že každé dve susedné kocky mali spoločnú celú stenu.
Výsledná stavba vyzerala z dvoch rôznych strán ako na nasledujúcom obrázku.

Zistite, koľko najviac a~koľko najmenej kociek mohla Zuzka na svoju stavbu použiť.
\insp{z5-I-2.eps}%
}
\podpis{Erika Novotná}

{%%%%%   Z5-I-3
Katka mala v záhradke päť záhonov rozmiestnených ako na obrázku.
Do záhonov chcela zasadiť cesnak, mrkvu a reďkvičku tak, aby na každom záhone bol len jeden druh zeleniny a aby žiadne dva záhony s rovnakou zeleninou nesusedili.
Koľkými rôznymi spôsobmi mohla Katka do záhonov zasadiť cesnak, mrkvu a reďkvičku?
\insp{z5-I-3.eps}%
}
\podpis{Libuše Hozová}

{%%%%%   Z5-I-4
V záhradkárskej osade mal pán Jahoda vo svojom sude 16 litrov vody.
Sused pán Malina mal vo svojom sude trikrát viac vody ako pán Jahoda.
Začalo pršať a do oboch sudov napršalo rovnaké množstvo vody.
Po daždi pán Malina zistil, že má v sude dvakrát viac vody ako pán Jahoda.

Koľko litrov vody napršalo do každého suda?

}
\podpis{Libuše Hozová}

{%%%%%   Z5-I-5
Zo štyroch zhodných štvorcov bol vytvorený ornament ako na obrázku.
Strany štvorcov sú dlhé 4\,cm, sú navzájom rovnobežné alebo kolmé a pretínajú sa buď vo svojich štvrtinách, alebo poloviciach.
Libor chcel ornament vyfarbiť a~zistil, že farba na 1\,cm$^2$ každého súvislého poľa ho bude stáť toľko eur, koľkým štvorcom je toto pole spoločné.

Koľko eur bude stáť farba na vyfarbenie celého ornamentu?
\insp{z5-I-5.eps}%
}
\podpis{Karel Pazourek}

{%%%%%   Z5-I-6
Lucka napísala na lístok číslo 12345 a~dvakrát ho medzi ciframi rozstrihla.
Získala tak tri menšie kartičky s tromi číslami.
Tieto kartičky zoradila dvoma spôsobmi, čím dostala dve rôzne päťciferné čísla.
Rozdiel týchto dvoch čísel bol 28\,926.

Medzi ktorými ciframi Lucka lístok rozstrihla?
}
\podpis{Michaela Petrová}

{%%%%%   Z6-I-1
Kamaráti Jaro, Pavol a Rasťo hrali guľôčky.
Jarovi sa veľmi nedarilo, takže mal po hre najmenej guľôčok zo všetkých.
Chlapcom to bolo ľúto, preto dal Rasťo Jarovi polovicu všetkých svojich guľôčok a~Pavol tretinu tých svojich.
Teraz mal najviac guľôčok Jaro, a tak svojim dvom kamarátom vrátil každému sedem guľôčok.
Po týchto výmenách mali všetci rovnako, a to po 25 guľôčok.

Koľko guľôčok mal po hre (pred výmenami) Jaro?
}
\podpis{Michaela Petrová}

{%%%%%   Z6-I-2
Karolína narysovala štvorec so stranou 6\,cm.
Na každej strane štvorca vyznačila modrou farbou dva body, ktorými rozdelila príslušnú stranu na tri zhodné časti.
Potom zostrojila štvoruholník, ktorý mal všetky vrcholy modrej farby a~ktorého žiadne dva vrcholy neležali na rovnakej strane štvorca.

Aké obsahy štvoruholníkov mohla Karolína dostať?
Nájdite všetky možnosti.
}
\podpis{Libuše Hozová}

{%%%%%   Z6-I-3
V~osemcifernom čísle je každá jeho cifra (okrem poslednej) väčšia ako nasledujúca cifra.

Koľko je všetkých takýchto čísel?
}
\podpis{Iveta Jančigová}

{%%%%%   Z6-I-4
V~nasledujúcom písomnom násobení dvoch trojciferných čísel sú cifry zastúpené hviezdičkami.
Na miesta hviezdičiek doplňte cifry tak, aby bol výpočet správny:
$$
\begin{array}{cccccc}
 & & * & * & * & \\
 & & * & * & * & \\
\hline
 & * & * & * & * & \\
3 & 1 & 7 & 5 & & \\
* & * & * & & & \\
\hline
* & * & 6 & * & * & \\
\end{array}
$$
}
\podpis{Libuše Hozová}

{%%%%%   Z6-I-5
Peťo zložil z navzájom zhodných trojuholníkov niekoľko rovinných útvarov, pozri obrázok.
Obvody prvých troch sú postupne 8\,cm, 11,4\,cm a~14,7\,cm.

Určte obvod štvrtého útvaru.
\insp{z6-I-5.eps}%
}
\podpis{Eva Semerádová}

{%%%%%   Z6-I-6
Alex, Barbora, Cyril, Dana, Eva, František a Gabika sa stali na svojich školách víťazmi v stolnom tenise
a~zišli sa na dvojdňovom turnaji o~celkového víťaza.
Každé z týchto siedmich detí malo počas turnaja zohrať jeden zápas s každým iným.
Prvý deň turnaja hral Alex jeden zápas, Barbora hrala dva zápasy, Cyril tri, Dana štyri, Eva päť zápasov a František šesť.

Koľko zápasov hrala prvý deň Gabika?
}
\podpis{Libuše Hozová}

{%%%%%   Z7-I-1
Ajka, Barborka, Cilka a Daniel sa dohadovali o počte zrniek piesku na ich pieskovisku.
Daniel povedal kamarátkam svoj odhad a tie sa ho rozhodli overiť.
Ajka napočítala 873\,451\,230, Barborka 873\,451\,227 a~Cilka 873\,451\,213 zrniek piesku.
Súčet (kladných) rozdielov týchto troch výsledkov od Danielovho odhadu bol 29.

Koľko zrniek piesku mohol odhadnúť Daniel?
Uveďte všetky možnosti.
}
\podpis{Veronika Bachratá}

{%%%%%   Z7-I-2


Pán Delfín a pán Žralok boli zdatní rybári.
Raz dokopy ulovili 70 rýb.
Päť devätín z rýb, ktoré ulovil pán Delfín, boli pstruhy.
Dve sedemnástiny z rýb, ktoré ulovil pán Žralok, boli kapry.

Koľko rýb ulovil pán Delfín?
}
\podpis{Libuše Hozová}

{%%%%%   Z7-I-3
Myslím si tri čísla.
Keď ich sčítam, dostanem 15.
Keď od súčtu prvých dvoch čísel odčítam tretie, dostanem 10.
Keď od súčtu prvého a~tretieho čísla odčítam druhé, dostanem~8.

Ktoré čísla si myslím?
}
\podpis{Eva Semerádová}

{%%%%%   Z7-I-4
Anetkin strýko má narodeniny v ten istý deň v roku ako Anetkina teta.
Strýko je starší ako teta, nie však o viac ako desať rokov, a obaja sú plnoletí.
Na poslednej oslave si Anetka uvedomila, že keď vynásobí ich oslavované veky a výsledný súčin ešte vynásobí počtom psov, ktorí sa na oslave zišli, dostane číslo 2024.

Koľko psov mohlo byť na tejto oslave?
Určte všetky možnosti.
}
\podpis{Michaela Petrová}

{%%%%%   Z7-I-5
Pravouhlý trojuholník má obsah 36\,m$^2$.
V~ňom je umiestnený štvorec tak, že dve strany štvorca sú časťami dvoch strán trojuholníka a~jeden vrchol štvorca je v~tretine najdlhšej strany.

Určite obsah tohto štvorca.
}
\podpis{Erika Novotná}

{%%%%%   Z7-I-6
Trpaslíci počítajú svoje veky v dňoch, takže každý deň oslavujú narodeniny.
U~trpaslíka Nošteka sa zišlo sedem trpaslíkov s vekmi 105, 120, 140, 168, 210, 280 a~420 dní.
Počas oslavy sa všetkým ôsmim trpaslíkom podarilo rozdeliť do dvoch skupín s rovnakými súčtami vekov.

Koľko najmenej a~koľko najviac dní mohol mať trpaslík Noštek?
}
\podpis{Erika Novotná}

{%%%%%   Z8-I-1
V minulom roku bolo v našom skautskom oddiele o 30 chlapcov viac ako dievčat.
Tento rok sa počet detí v~oddiele zväčšil o~10\,\%, pričom počet chlapcov sa zväčšil o~5\,\% a~počet dievčat sa zväčšil o~20\,\%.

Koľko detí máme tento rok v~oddiele?
}
\podpis{Libuše Hozová}

{%%%%%   Z8-I-2
Adam mal papier, ktorý bol natoľko veľký, že by sa z neho dalo natrhať niekoľko desiatok tisíc kúskov.
Najprv papier roztrhal na štyri kúsky.
Každý z týchto kúskov vzal a roztrhal buď na štyri, alebo na desať kúskov.
Rovnakým spôsobom pokračoval ďalej: každý novovzniknutý kúsok roztrhal buď na štyri, alebo na desať menších kúskov.

Rozhodnite a~vysvetlite, či môže Adam týmto spôsobom natrhať presne 20\,000 kúskov.
}
\podpis{Iveta Jančigová}

{%%%%%   Z8-I-3
V športovom areáli tvorili stanovištia $A$, $B$, $C$, $D$, $E$ vrcholy pravidelného päťuholníka.
Tieto stanovištia boli pospájané priamymi cestami.
Na ceste z~$A$ do $B$ bola fontána $F$, ktorú so stanovišťom $C$ spájala cesta kolmá na cestu z~$B$ do $E$.
Pat a~Mat sa zišli na stanovišti $E$ a~rozhodli sa zamiesť niektoré cesty.
Pat pozametal cestu z $E$ do $B$.
Mat pozametal cestu z~$E$ do $A$ a~ešte z~$A$ do $F$.

Určte, ktorý z~nich zametal dlhší úsek.
%Určte rozdiel úsekov zametených Patom a~Matom.
}
\podpis{Libuše Hozová}

{%%%%%   Z8-I-4
Hynek napísal nasledujúci príklad s piatimi záhadnými sčítancami:
$$
@ + \#\# + *** + \&\&\&\& + \$\$\$\$\$ =\ ?
$$
Prezradil, že znaky $@, \#, *, \&, \text{\textdollar}$ predstavujú navzájom rôzne cifry 1, 2, 3, 4, 5 a~že výsledný súčet je deliteľný jedenástimi.

Ktoré najmenšie a~ktoré najväčšie číslo môže byť výsledkom Hynkovho príkladu?
}
\podpis{Erika Novotná}

{%%%%%   Z8-I-5
Trojuholník $ABC$ je rozdelený úsečkami ako na obrázku.
Úsečky $DE$ a $AB$ sú rovnobežné.
Trojuholníky $CDH$, $CHI$, $CIE$, $FIH$ majú rovnaký obsah, a to 8\,dm$^2$.

Zistite obsah štvoruholníka $AFHD$.
\insp{z8-I-5.eps}%
}
\podpis{Eva Semerádová}

{%%%%%   Z8-I-6
Adam vpísal do tabuľky $3\times 3$ čísla od 1 po 9 ako na obrázku:
$$
\begin{array}{ccc}
7 & 6 & 4 \\
1 & 2 & 8 \\
9 & 3 & 5 \\
\end{array}
$$
Pre toto vyplnenie platí, že súčet čísel troch políčok pozdĺž každej strany je stále rovnaký.
Adam zistil, že čísla do tabuľky je možné vyplniť aj inak, bez toho, aby pokazil vlastnosť s rovnakými súčtami pozdĺž strán.

Akú najmenšiu hodnotu môže mať tento súčet?
Uveďte príklad tabuľky s~najmenším súčtom pozdĺž strán a~vysvetlite, prečo menší súčet byť nemôže.
}
\podpis{Josef Tkadlec}

{%%%%%   Z9-I-1
Pat a~Mat sa vykúpali v~rybníku a~potom si dali preteky do svojej chalúpky.
V istom okamihu platilo, že keby Mat mal zdolanú polovicu vzdialenosti, ktorú doteraz ubehol, chýbal by mu do chalúpky trojnásobok onej polovičnej vzdialenosti.
V tom istom okamihu platilo, že keby Pat mal zdolaný dvojnásobok vzdialenosti, ktorú doteraz ubehol, chýbala by mu do chalúpky tretina onej dvojnásobnej vzdialenosti.

Kto bol v~danom okamihu bližšie k chalúpke?
}
\podpis{Libuše Hozová}

{%%%%%   Z9-I-2
Zostrojte kosoštvorec $ABCD$, v ktorom platí $|AC| = 8\cm$ a~$|AS| = 7\cm$, kde $S$ je stredom strany $CD$.
}
\podpis{Karel Pazourek}

{%%%%%   Z9-I-3
V~základnej škole U~Troch dubov, kam chodí aj Žigmund, každoročne organizujú vedomostnú súťaž, v ktorej každý súťažiaci môže získať najviac 15 bodov.
Tento rok bol priemerný bodový zisk súťažiacich zaokrúhlený na desatiny rovný 10,4.
Žigmund si po súťaži uvedomil, že niektoré otázky si zle prečítal a odpovedal na niečo iné.
Mohol tak mať o~4 body viac a~priemerný bodový zisk zaokrúhlený na desatiny by sa tým zvýšil na 10,6.

Určte, koľko najmenej a~koľko najviac detí mohlo tento rok súťažiť.
}
\podpis{Michaela Petrová}

{%%%%%   Z9-I-4
Karol mal vynásobiť dve dvojciferné čísla.
Z~nepozornosti vymenil poradie cifier v~jednom z~činiteľov a~dostal súčin, ktorý bol o~4\,248 menší ako správny výsledok.

Koľko malo Karolovi správne vyjsť?
}
\podpis{Libuše Hozová}

{%%%%%   Z9-I-5
Trojuholník $ABC$ je pravouhlý s pravým uhlom pri vrchole $C$.
Body $A'$, $B'$, $C'$ sú obrazy bodov $A$, $B$, $C$ postupne v stredových súmernostiach so stredmi $C$, $A$, $B$.
Dokážte, že platí
$$
|A’B’|^2 + |B’C’|^2 + |C’A’|^2 = 14\cdot |AB|^2 .
$$
}
\podpis{Jaroslav Zhouf}

{%%%%%   Z9-I-6
Nižšie je naznačená časť štvorcovej siete pozostávajúca zo 4 riadkov a~2023 stĺpcov.

Určte počet štvorcov, ktorých všetky vrcholy sú mrežovými bodmi tejto štvorcovej siete.
\insp{z9-I-6.eps}%
}
\podpis{Karel Pazourek}

{%%%%%   Z4-II-1
...}
\podpis{...}

{%%%%%   Z4-II-2
...}
\podpis{...}

{%%%%%   Z4-II-3
...}
\podpis{...}

{%%%%%   Z5-II-1
Deti dostali príklad s piatimi prázdnymi políčkami:
$$
\square \cdot \square - \square : \left( \square + \square \right) =\, ?
$$
Do každého políčka mali vpísať jedno z čísel 1, 2, 3, 4, 5 tak, aby každé číslo použili iba raz a aby delenie vychádzalo bezo zvyšku.

Nájdite všetky výsledky, ktoré mohli deti dostať.}
\podpis{Michaela Petrová}

{%%%%%   Z5-II-2
Albert, Ben, Cyril, Daniel, Erik, Filip a Gabriel vstúpili v tomto poradí po jednom za sebou do jaskyne s mnohými dverami.
Svoje poradie menia len pri prechode dverami, a to takto:
prvý zo zástupu otvorí dvere, podrží ich všetkým ostatným a potom sa zaradí na koniec.
Prvé dvere teda otvára Albert, druhé Ben atď.

Kto otvorí sté dvere?}
\podpis{Karel Pazourek}

{%%%%%   Z5-II-3
Na obraze sú dva biele štvorce v sivom poli.
Strany obrazu a~strany oboch štvorcov sú orientované vodorovne, alebo zvisle:
\insp{z5-ii-3.eps}%

Bodkovane vyznačené vzdialenosti menšieho štvorca od strán obrazu sú rovnaké ako veľkosť strany tohto menšieho štvorca.

Čiarkovane vyznačené vzdialenosti väčšieho štvorca od strán obrazu sú rovnaké ako veľkosť strany tohto väčšieho štvorca.
Obsah sivej časti obrazu je 62\,cm$^2$.

Určite obsah bielej časti obrazu.}
\podpis{Eva Semerádová}

{%%%%%   Z6-II-1
Žiaci dostali prirodzené číslo menšie ako 100.
Alex dané číslo zaokrúhlil na desiatky.
Barbora dané číslo zaokrúhlila na stovky.
Cyril dané číslo vynásobil dvoma.
Dana zaokrúhlené Alexove a Barborine číslo sčítala.
Eva od Daninho čísla odpočítala Cyrilovo číslo.
František oznámil Evin výsledok, a~ten bol 30.

Ktoré číslo mohli žiaci na začiatku dostať?
Určite všetky možnosti.}
\podpis{Eva Semerádová}

{%%%%%   Z6-II-2
Zo zhodných štvorcov so stranou dĺžky 1\,cm je zložený útvar s nasledujúcimi vlastnosťami:

\smallskip
\item{$\bullet$} Útvar je tvorený radmi susediacich štvorcov.
\item{$\bullet$} Spodný rad pozostáva z dvanástich štvorcov.
\item{$\bullet$} Každý vyšší rad začína zľava rovnako ako rad pod ním, má však o~jeden štvorec menej.
\item{$\bullet$} Horný rad tvoria štyri štvorce.

\smallskip\noindent
Vrcholy spodnej strany útvaru sú označené $A$ a~$B$, pravý vrchol hornej strany je označený $C$.

Určite obsah trojuholníka $ABC$.
\insp{z6-ii-2.eps}%
}
\podpis{Erika Novotná, Karel Pazourek}

{%%%%%   Z6-II-3
V ~krajine Binária žijú matematici niekoľkých úrovní.
Všetci matematici rovnakej úrovne vyriešia za deň rovnaký počet príkladov.
Matematik vyššej úrovne vyrieši za deň dvojnásobné množstvo príkladov ako matematik predchádzajúcej úrovne (teda napríklad matematik štvrtej úrovne vyrieši za deň dvakrát viac príkladov ako matematik tretej úrovne).

Traja matematici piatej úrovne vyriešia za deň o~tisíc príkladov viac ako štyria matematici druhej úrovne.

Koľko príkladov vyrieši za deň jeden matematik druhej úrovne?}
\podpis{Karel Pazourek, Michaela Petrová}

{%%%%%   Z7-II-1
Deti hádzali klasickou hracou kockou s číslami od 1 do 6.
Keď padlo párne číslo, tak zatlieskali.
Keď padlo číslo deliteľné tromi, tak zadupali.
Keď nastali oba prípady, urobili oboje, v ostatných prípadoch nerobili nič.
Počas piatich hodov deti celkom trikrát zatlieskali a trikrát zadupali.
Súčet čísel, ktoré postupne padli, bol dvadsať.

Nájdite všetky možné pätice čísel, ktoré mohli padnúť, bez ohľadu na ich poradie.}
\podpis{Eva Semerádová}

{%%%%%   Z7-II-2
Zo zhodných štvorcov so stranou dĺžky 1\,cm je zložený útvar s nasledujúcimi vlastnosťami:

\smallskip
\item{$\bullet$} Spodný a~horný rad pozostáva z~jedného štvorca.
\item{$\bullet$} Ostatné rady majú po dvoch susediacich štvorcoch.
\item{$\bullet$} Všetky stĺpce sú tvorené dvoma susediacimi štvorcami.
\item{$\bullet$} Obvod útvaru je 178\,cm.

\smallskip\noindent
Vrcholy spodnej strany útvaru sú označené $A$ a~$B$, pravý vrchol hornej strany je označený $C$.

Určite obsah trojuholníka $ABC$.
\insp{z7-ii-2.eps}%
}
\podpis{K. Pazourek}

{%%%%%   Z7-II-3
Na pretekoch sa spolu stretlo päť detí, dievčatá Anna a Fiona a chlapci Jurko, Lukáš a Tomáš.
O~cieľovom poradí máme tieto informácie:

\smallskip
\item{$\bullet$} Žiadne dve deti sa neumiestnili na rovnakom mieste.
\item{$\bullet$} Priemerné umiestnenie dievčat bolo rovnaké ako priemerné umiestnenie chlapcov.
\item{$\bullet$} Obe dievčatá sa umiestnili pred Tomášom.
\item{$\bullet$} Jurko sa umiestnil medzi Annou a Lukášom.

\smallskip\noindent
Určite cieľové poradie detí.}
\podpis{Erika Novotná}

{%%%%%   Z8-II-1
Počítačový program vypisoval po riadkoch čísla tvorené ciframi od 1 do 9.
Cifry boli použité opakovane v~prirodzenom poradí.
Na každom riadku bolo o~jednu cifru viac ako na riadku predchádzajúcom, na prvom riadku bola 1:
$$
\aligned
1 \\
12 \\
123 \\
\vdots\ \ \\
123456789 \\
1234567891 \\
12345678912 \\
\vdots\ \
\endaligned
$$
Výpis bol ukončený číslom na 2024. riadku.

Zistite, na koľkých riadkoch boli čísla deliteľné
a) tromi,
b) štyrmi.}
\podpis{Patrik Bak}

{%%%%%   Z8-II-2
Medzi hračkami v~obchode sú iba lode a~autá.
Lode tvoria štvrtinu hračiek.
75\,\%~lodí a~40\,\% áut je červených.
Červených hračiek je o~10 menej ako tých s inou farbou.

Koľko hračiek je v~obchode?}
\podpis{Eva Semerádová}

{%%%%%   Z8-II-3
Deltoid je konvexný štvoruholník, ktorý má dve dvojice zhodných susedných strán.

Zostrojte deltoid $ABCD$ so stranami $AB$ a~$AD$ dĺžky 11\,cm, so stranami $CB$ a~$CD$ dĺžky 6\,cm a~s~uhlopriečkou $AC$ dĺžky 15\,cm.
Rozdeľte deltoid $ABCD$ na štyri štvoruholníky tak, aby dva z nich boli deltoidy a dva boli zhodné kosoštvorce.
Konštrukciu opíšte a~zdôvodnite.

\poznamka
Stačí nájsť a opísať jedno takéto delenie deltoidu $ABCD$.}
\podpis{Karel Pazourek}

{%%%%%   Z9-II-1
Na prehliadke sa v miestnosti múzea zišla skupina chlapcov a dievčat.
Počas prehliadky nikto nerušil a neodchádzal.
Po prehliadke odišlo 15 dievčat a v miestnosti tak zostalo dvakrát viac chlapcov ako dievčat.
Následne odišlo 45 chlapcov a miestnosti zostalo päťkrát viac dievčat ako chlapcov.

Koľko dievčat bolo v~miestnosti múzea počas prehliadky?}
\podpis{Libuše Hozová}

{%%%%%   Z9-II-2
Obdĺžnik na obrázku je rozdelený na dva zhodné štvorce, štyri zhodné menšie pravouhlé trojuholníky a ~ štyri zhodné väčšie pravouhlé trojuholníky.
Veľkosti niektorých strán v~cm sú vyznačené na obrázku.

Vypočítajte rozmery obdĺžnika.
\insp{z9-ii-2.eps}%
}
\podpis{Mária Dományová}

{%%%%%   Z9-II-3
Iveta postupne vypisovala prirodzené čísla tvorené číslicami 1, 3, 5, 7.
Žiadne iné číslice nepoužila, postupovala vzostupne od najmenšieho čísla a~žiadne číslo nezabudla.
Čísla písala bezprostredne za sebou a~tak zostavovala jedno mimoriadne dlhé číslo:
$$
1357111315173133\dots
$$

Ktorá číslica je v~tomto čísle na 1286. mieste?}
\podpis{Erika Novotná}

{%%%%%   Z9-II-4
V piatich vrecúškach je spolu 52 guľôčok.
V~žiadnych dvoch vrecúškach nie je rovnaký počet guľôčok, niektoré vrecúško môže byť i~prázdne.
Všetky guľôčky z akéhokoľvek (neprázdneho) vrecúška je možné premiestniť do ostatných štyroch vrecúšok tak, že v nich budú rovnaké počty guľôčok.
\item{a)} Nájdite nejaké rozdelenie gulôčok do vrecúšok, ktoré má všetky uvedené vlastnosti.
\item{b)} Ukážte, že pri akomkoľvek rozdelení s~uvedenými vlastnosťami je v~niektorom vrecúšku práve 12 gulôčok.\endgraf}
\podpis{Jaroslav Zhouf}

{%%%%%   Z9-III-1
V~základnej škole Orechová, kam chodí aj Lukáš, organizujú vedomostnú súťaž s vopred zoradenými úlohami.
Každá správne vyriešená úloha je hodnotená toľkými bodmi, aké je jej poradie.
Každá neriešená či nie celkom správne vyriešená úloha je hodnotená 0 bodmi.
Lukáš správne vyriešil len prvých 12 úloh.
Ak by správne vyriešil len posledných 12 úloh, získal by o~708 bodov viac.

Koľko bolo súťažných úloh?}
\podpis{Michaela Petrová}

{%%%%%   Z9-III-2
Útvar na obrázku pozostáva z~trojuholníka $ABC$ a~troch rovnostranných trojuholníkov $DEF$, $DGH$ a~$DIJ$.
Bod $D$ je priesečníkom osí vnútorných uhlov trojuholníka $ABC$, vrcholy $A$, $B$, $C$ sú postupne stredy strán $EF$, $GH$, $IJ$.
Veľkosť uhla $EDJ$ je $51\st$, veľkosť uhla $HDI$ je $66\st$.

Určte veľkosti vnútorných uhlov trojuholníka $ABC$.
\insp{z9-iii-2.eps}%
}
\podpis{K. Pazourek}

{%%%%%   Z9-III-3
Na stranách všeobecného trojuholníka $ABC$ sú dané body $K$, $L$, $M$, $N$, $U$:
\item{$\bullet$} bod $K$ je stredom strany $AC$,
\item{$\bullet$} bod $U$ je stredom strany $BC$,
\item{$\bullet$} body $L$ a~$M$ ležia postupne na úsečkách $CK$ a~$CU$ tak, že $LM\parallel KU$,
\item{$\bullet$} bod $N$ leží na úsečke $AB$ tak, že $|AN|:|AB|=3:7$,
\item{$\bullet$} pomer obsahov mnohouholníkov $UMLK$ a~$MLKNU$ je $3:7$.

\noindent
Určte pomer veľkostí úsečiek $LM$ a~$KU$.}
\podpis{Mária Dományová}

{%%%%%   Z9-III-4
Nájdite všetky trojice trojciferných prirodzených čísel $a$, $b$, $c$, pre ktoré platí
$$
b^2 =a\cdot c, \quad b =a+34 .
$$}
\podpis{Erika Novotná}

