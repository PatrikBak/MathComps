{%%%%% A-I-1
Na párty sa zišlo 20 osôb, z~toho 10 chlapcov a 10 dievčat. Každému sa páči práve $k$~osôb opačného pohlavia. Je vždy možné vytvoriť pár, v ktorom sa obom páči ten druhý? Riešte a)~pre $k=5$, b)~pre $k=6$.
}
\podpis{Josef Tkadlec}

{%%%%% A-I-2
Z~cifier 1 až 9 vytvoríme deväťciferné číslo s~navzájom rôznymi ciframi. Potom vypočítame súčty všetkých trojíc po sebe idúcich cifier a zapíšeme týchto sedem súčtov vzostupne. Rozhodnite, či možno takto získať postupnosť
\itemitem{a)} 11, 15, 16, 18, 19, 21, 22,
\itemitem{b)} 11, 15, 16, 18, 19, 21, 23.
}
\podpis{Patrik Bak}

{%%%%% A-I-3
Daný je trojuholník $ABC$ s ťažiskom $T$. Nad úsečkami $BT$ a~$CT$ sú zostrojené pravouhlé rovnoramenné trojuholníky $BTK$ a $CTL$ rovnako ako na \obr{}. Označme $D$ stred strany $BC$ a $E$ stred úsečky~$KL$. Určte všetky možné hodnoty pomeru $|AT|/|DE|$.
\inspsc{a73i.30}{.8333}%
}
\podpis{Michal Rolínek}

{%%%%% A-I-4
O nepárnom prvočísle $p$ povieme, že je \emph{špeciálne}, ak súčet všetkých prvočísel menších ako $p$ je násobkom $p$. Existujú dve po sebe idúce prvočísla, ktoré sú špeciálne?
}
\podpis{Jaroslav Zhouf}

{%%%%% A-I-5
Rozhodnite, či existuje neprázdna podmnožina políčok tabuľky $7\times 7$ s nasledujúcou vlastnosťou: Pre každé z~\emph{tetramín} na \obr{}
\inspdf{a73i_50.pdf}%
možno túto podmnožinu vyplniť bez prekrývania výlučne jeho kópiami.
Jednotlivé kópie môžeme ľubovoľne otáčať a preklápať.}
\podpis{Michal Rolínek}

{%%%%% A-I-6
Pre reálne čísla $a$, $b$, $c$, $d$ z~intervalu $\langle1,2\rangle$ platí $(a + c)(b + d) = 8$. Dokážte, že
$$\frac{1}{a^2+b^2-1} + \frac{1}{b^2+c^2-1} + \frac{1}{c^2+d^2- 1} + \frac{1}{d^2+a^2-1} \geq 1,$$
a určte, kedy nastane rovnosť.}
\podpis{Zdeněk Pezlar}

{%%%%% B-I-1
Koľko neprázdnych podmnožín množiny $\{0,1,\ldots,9\}$ má súčet prvkov deliteľný tromi?}
\podpis{Eliška Macáková}

{%%%%% B-I-2
Pre reálne čísla $a$, $b$, $c$ platí
$$\frac{a}{b+c} = \frac{b}{c+a} = \frac{c}{a+b}.$$
Určte všetky možné hodnoty výrazu
$$\frac{a^3+b^3+c^3}{(b+c)^3 + (c+a)^3 + (a+b)^3}.
$$
}
\podpis{Michal Rolínek}

{%%%%% B-I-3
Nech $E$ je stred strany $AD$ pravouholníka $ABCD$. Predpokladajme, že päta $F$ kolmice z~vrcholu $B$ na priamku $CE$ leží vnútri úsečky $CE$ (\obr) a~označme $G$ pätu kolmice z~bodu~$F$ na stranu $AD$. Dokážte, že priamka $CE$ rozpoľuje uhol $AFG$.
\inspsc{b73i.30}{.8333}%
}
\podpis{Jaroslav Švrček}

{%%%%% B-I-4
Rozhodnite, či existuje pätica kladných celých čísel
\itemitem{(i)} $a$, $a$, $a$, $a$, $b$ ($a \ne b$),
\itemitem{(ii)} $a$, $a$, $b$, $b$, $c$ ($a \ne b \ne c \ne a$),
\endgraf
v ktorých je každé z týchto čísel deliteľom súčtu každých troch zo zvyšných štyroch čísel.}
\podpis{Jaroslav Zhouf}

{%%%%% B-I-5
V pravouhlom trojuholníku je pomer polomeru kružnice vpísanej k polomeru kružnice opísanej $2:5$. Dokážte, že dĺžka jednej z jeho strán je aritmetickým priemerom dĺžok zvyšných dvoch strán.}
\podpis{Mária Dományová}

{%%%%% B-I-6
Rozhodnite, či možno štvorcovú tabuľku
$4\times4$ vyplniť navzájom rôznymi prirodzenými číslami od~$1$ do $16$ tak,
že v každom riadku aj každom stĺpci existuje číslo, ktorého sedemnásobok je súčtom zvyšných troch čísel.}
\podpis{Jaromír Šimša}

{%%%%% C-I-1
Existuje desať po sebe idúcich prirodzených čísel, ktoré sú postupne deliteľné číslami $9$,~$7$, $5$, $3$, $1$, $1$, $3$, $5$, $7$, $9$?}
\podpis{Jaroslav Zhouf}

{%%%%% C-I-2
Pre stred~$M$ prepony $AC$ pravouhlého trojuholníka $ABC$ platí $|BC| = |CM|$. Dokážte, že kružnice opísané trojuholníkom $ABC$ a $ABM$ majú rovnaké polomery.}
\podpis{Michal Pecho}

{%%%%% C-I-3
Uvažujme 20 výrokov:
\item{$\bullet$} Mám práve jednu sestru.
\item{$\bullet$} Mám práve jedného brata.
\item{$\bullet$} Mám práve dve sestry.
\item{$\bullet$} Mám práve dvoch bratov.
\item{$\bullet$} ...
\item{$\bullet$} Mám práve desať sestier.
\item{$\bullet$} Mám práve desať bratov.

\itemitem{a)} Každý zo štyroch súrodencov vyslovil iný z týchto 20 výrokov. Je možné, že všetci štyria mali pravdu?
\itemitem{b)} Nájdite najväčšie prirodzené číslo $n$ také, že každý z $n$ súrodencov môže vysloviť iný z týchto 20 výrokov a mať pravdu.

(Všetci súrodenci majú rovnakých rodičov.)}
\podpis{Josef Tkadlec}

{%%%%% C-I-4
Koľko usporiadaných štvoríc kladných celých čísel $(a,b,c,d)$ so súčtom~100 spĺňa rovnice
$$
(a+b)(c+d) = (b+c)(a+d) = (a+c)(b+d)?
$$
}
\podpis{Patrik Bak}

{%%%%% C-I-5
Na tabuli sú napísané čísla $1$, $\sqrt{2}$, $\sqrt{3}$. V každom kroku čísla $a$, $b$, $c$ napísané na tabuli zotrieme a nahradíme ich súčinmi $ab$, $bc$, $ca$. Zistite, či po niekoľkých krokoch bude znovu niektoré z~čísel napísaných na tabuli prirodzené.}
\podpis{Jaroslav Zhouf}

{%%%%% C-I-6
Daný je obdĺžnik $ABCD$, pričom ${|AB|:|BC|}=2:1$. Na jeho stranách $AB$, $BC$, $CD$, $DA$ sú dané postupne body $K$, $L$, $M$, $N$ tak, že $KLMN$ je obdĺžnik, v~ktorom $|KL|:|LM|=3:1$ (\obr). Vypočítajte pomer obsahov obdĺžnikov $ABCD$ a $KLMN$.
\inspsc{c73i.60}{.8333}
}
\podpis{Josef Tkadlec}

{%%%%% A-S-1
Z cifier 1 až 9 vytvoríme deväťciferné číslo s navzájom rôznymi ciframi. Potom vypočítame súčet každej trojice susedných cifier a týchto sedem súčtov zapíšeme vzostupne. Rozhodnite, či takto môžeme získať postupnosť
\ite a) 9, 11, 12, 13, 20, 20, 20,
\ite b) 9, 11, 12, 13, 20, 21, 21.
}
\podpis{Martin Melicher}

{%%%%% A-S-2
Určte počet všetkých kvadratických mnohočlenov $P(x)$ s celočíselnými koeficientmi takých, že pre každé reálne číslo $x$ platí
$$
x^2 + 2x - 2023 < P(x) < 2x^2.
$$
}
\podpis{Ján Mazák, Michal Rolínek}

{%%%%% A-S-3
Vnútri polkruhu nad priemerom $AB$ so stredom $O$ uvažujme ľubovoľný bod $X$.
Označme $G$ ťažisko trojuholníka $XOB$ a $Y$ priesečník polpriamky $AX$ s hranicou polkruhu rôzny od $A$.
Dokážte, že $|YG|=|GB|$.}
\podpis{Jiří Blažek, Josef Tkadlec}

{%%%%% A-II-1
Tabuľku $3\times 3$ vyplníme navzájom rôznymi prirodzenými číslami od 1 do 9. Potom vypočítame súčet čísel v~každom zo štyroch štvorcov $2\times 2$ a~tieto štyri súčty zapíšeme vzostupne. Rozhodnite, či tak môžeme získať postupnosť
\ite a) 24, 24, 25, 25,
\ite b) 20, 23, 26, 29.
}
\podpis{Tomáš Bárta}

{%%%%% A-II-2
Určte všetky dvojice $(k,n)$ kladných celých čísel, pre ktoré existujú kladné celé čísla $a$, $b$ také, že platí
$$
D(a+k, b) = n\cdot(a, b),
$$
pričom $D(x,y)$ označuje najväčší spoločný deliteľ kladných celých čísel $x$ a $y$.
}
\podpis{Jaromír Šimša}

{%%%%% A-II-3
Nech $k$ je kružnica opísaná danému ostrouhlému trojuholníku $ABC$. Uvažujme vnútorný bod~$P$ toho oblúka $BC$ kružnice $k$, ktorý neobsahuje bod $A$. Označme $Q$ priesečník úsečiek~$AP$ a $BC$, ďalej $O_1$ a~$O_2$ stredy kružníc opísaných postupne trojuholníkom $BPQ$ a~$CPQ$. Dokážte, že ak priamka $O_1O_2$ prechádza niektorým vrcholom trojuholníka $ABC$, tak jeden z~bodov $O_1$, $O_2$ leží na kružnici~$k$.}
\podpis{Michal Janík}

{%%%%% A-II-4
Súčet $74$ (nie nutne rôznych) reálnych čísel z~intervalu $\langle4,10\rangle$ je $356$.
Určte najväčšiu možnú hodnotu súčtu ich druhých mocnín.}
\podpis{Zdeněk Pezlar}

{%%%%% A-III-1
Nech $a$, $b$, $c$ sú kladné celé čísla, pre ktoré sa jedna z~hodnôt
$$
D(a,b)\cdot n(b,c), \qquad D(b,c)\cdot n(c,a), \qquad D(c,a)\cdot n(a,b)
$$
rovná súčinu zvyšných dvoch. Dokážte, že niektoré z čísel $a$, $b$, $c$ je násobkom
iného z~ich.

(Symbol $D(x,y)$, resp. $n(x,y)$ označuje najväčší spoločný deliteľ, resp. najmenší spoločný násobok kladných celých
čísel $x$, $y$.)}
\podpis{Jaroslav Švrček, Josef Tkadlec}

{%%%%% A-III-2
Vnútorný bod $P$ konvexného štvoruholníka $ABCD$ spĺňa rovnosti
$$
|\angle PAD|=|\angle ADP|=|\angle CBP|=|\angle PCB|=|\angle CPD|.
$$
Nech~$O$ je stred kružnice opísanej trojuholníku~$CPD$. Dokážte, že $|OA|=|OB|$.}
\podpis{Patrik Bak}

{%%%%% A-III-3
Určte najväčšie prirodzené číslo $n$ s~vlastnosťou: Ľubovoľnú sadu $n$~tetramín, z~ktorých každé je jedného zo štyroch tvarov na \obr,
možno bez prekrývania umiestniť do tabuľky $20\times 20$ tak, že každé tetramino pokrýva práve štyri políčka tabuľky.
(Jednotlivé tetraminá môžeme ľubovoľne otáčať a preklápať.)
\inspdf{a73iii_30.pdf}%
}
\podpis{Josef Tkadlec}

{%%%%% A-III-4
Na párty sa zišlo 10 chlapcov a 10 dievčat. Každému chlapcovi sa páči iný kladný počet dievčat. Každému dievčaťu sa páči iný kladný počet chlapcov. Určte najväčšie celé číslo~$n$ s~nasledujúcou vlastnosťou:
Vždy je možné utvoriť aspoň $n$ disjunktných párov, v ktorých sa obom páči ten druhý.}
\podpis{Josef Tkadlec}

{%%%%% A-III-5
Postupnosť reálnych čísel $\bigl(a_k\bigr)_{k=1}^{\infty}$ spĺňa pre každý index $k\geqq1$ rovnosť
$$
a_{k+1}=3a_k-\lfloor{2a_k}\rfloor-\lfloor{a_k}\rfloor.
$$
Určte všetky prirodzené čísla $n$, pre ktoré je takáto postupnosť s~prvým členom $a_1=1/n$ od istého člena konštantná.

\smallskip
(Zápisom $\lfloor{x}\rfloor$ rozumieme najväčšie celé číslo, ktoré neprevyšuje dané reálne číslo~$x$.)}
\podpis{Jaromír Šimša}

{%%%%% A-III-6
Nájdite všetky pravouhlé trojuholníky s~celočíselnými dĺžkami strán, do ktorých sa dajú vpísať dve zhodné kružnice s prvočíselným polomerom,
ktoré majú vonkajší dotyk, obe sa dotýkajú prepony a každá z~nich sa dotýka inej odvesny.}
\podpis{Jaromír Šimša}

{%%%%% B-S-1
Prirodzené čísla $a$, $b$, $c$ sú umiestnené do kruhu ako na obrázku, pričom každé číslo je deliteľom súčtu dvoch čísel s ním susediacich. Koľko najviac z~čísel $a$, $b$, $c$ môže byť rôznych?
\inspdf{b73s_1.pdf}%
}
\podpis{Josef Tkadlec}

{%%%%% B-S-2
Nech $ABCD$ je obdĺžnik so stredom $S$ a dlhšou stranou $AB$. Kolmica na priamku~$BD$ prechádzajúca
vrcholom $B$ pretne priamku~$AC$ v~bode~$E$. Rovnobežka s~priamkou~$BE$ vedená stredom~$S$ pretne stranu~$CD$ v~bode~$F$. Predpokladajme, že $|CE|=|BC|$.
\ite a) Určte veľkosť uhla $BSC$.
\ite b) Dokážte, že $|DF|=2\,|CF|$.
}
\podpis{Jaroslav Švrček}

{%%%%% B-S-3
Pre nenulové reálne čísla $a$, $b$, $c$ platí
$$
a^2(b+c)=b^2(c+a)=c^2(a+b).
$$
Určte všetky možné hodnoty výrazu
$$
\frac{(a+b+c)^2}{a^2+b^2+c^2}.
$$
}
\podpis{Jaromír Šimša}

{%%%%% B-II-1
Patrik vybral dve rôzne kladné celé čísla $a$, $b$, každé napísal na 10 kariet a všetkých 20~kariet rozmiestnil po obvode kruhu.
Všimol si, že každé číslo je teraz deliteľom súčtu dvoch čísel na susedných kartách.
Dokážte, že čísla na kartách sa ob jedno striedajú.}
\podpis{Josef Tkadlec}

{%%%%% B-II-2
Reálne čísla $a$, $b$, $c$, $d$ spĺňajú rovnosti
$$
\frac{a-b}{c+d} = \frac{a-c}{b+d} = \frac{b-c}{a+d}.
$$
Dokážte, že tieto zlomky sú rovné nule.}
\podpis{Zdeněk Pezlar}

{%%%%% B-II-3
Daný je lichobežník $ABCD$ so základňami $AB$ a $CD$, ktorých dĺžky sú postupne 6 a~4. Označme $P$ stred uhlopriečky $BD$ a $E$ priesečník priamok $AP$ a $CD$. Rovnobežka s~priamkou $AP$ prechádzajúca vrcholom $D$ pretína priamku $BE$ v~bode~$F$. Dokážte, že priamka $BC$ rozpoľuje úsečku $DF$.}
\podpis{Jaroslav Švrček}

{%%%%% B-II-4
Koľkými rôznymi spôsobmi môžeme vyplniť tabuľku $2024 \times 2024$ číslami $0$ a $1$ tak, aby súčty čísel v jednotlivých riadkoch boli navzájom rôzne a aj súčty čísel v~jednotlivých stĺpcoch boli navzájom rôzne?}
\podpis{Eliška Macáková}

{%%%%% C-S-1
Pažítkov priemer známok je presne~$3$. Keby sme tri z Pažítkových pätiek do priemeru nezapočítali, bol by priemer jeho známok presne~$2$. Určte najväčší počet jednotiek, ktoré mohol Pažítka dostať. (Možné známky sú $1$, $2$, $3$, $4$, $5$.)}
\podpis{Patrik Bak}

{%%%%% C-S-2
V~lichobežníku $ABCD$, pričom $AB \parallel CD$, sa osi vnútorných uhlov pri vrcholoch $C$ a~$D$ pretínajú na úsečke $AB$. Dokážte, že platí $|AD|+|BC|=|AB|$.}
\podpis{Patrik Bak}

{%%%%% C-S-3
Uvažujme tabuľky $3\times 3$ vyplnené kladnými celými číslami tak, že súčet čísel v~každom riadku aj každom stĺpci je 10. Koľko najviac čísel v takejto tabuľke môže byť
\ite a) rovnakých,
\ite b) rôznych?
}
\podpis{Ján Mazák}

{%%%%% C-II-1
Pre každé zo šiestich po sebe idúcich prirodzených čísel väčších ako 1 určíme najmenšie prvočíslo, ktoré ho delí, a potom týchto šesť prvočísel sčítame. Môže nám vyjsť
\ite a) 23,
\ite b) 25?
}
\podpis{Eliška Macáková}

{%%%%% C-II-2
Na tabuli sú napísané štyri čísla $\sqrt{11}$, $\sqrt{12}$, $\sqrt{13}$, $\sqrt{14}$. V~každom kroku jedno číslo z~tabule zotrieme a nahradíme ho súčinom niektorých dvoch zo zvyšných troch čísel. Zistite, či je možné postupovať
tak, aby po niekoľkých krokoch boli na tabuli len celé čísla.}
\podpis{Josef Tkadlec}

{%%%%% C-II-3
Daný je obdĺžnik $ABCD$, pričom ${|AB|:|BC|}=2:1$. Na jeho stranách $AB$, $BC$, $CD$, $DA$ sú dané postupne body $K$, $L$, $M$, $N$ tak, že $KLMN$ je obdĺžnik, v ktorom $|KL|:|LM|=3:1$. Na jeho stranách $KL$, $LM$, $MN$, $NK$ sú opäť dané postupne body $W$, $X$, $Y$, $Z$ tak, že $WXYZ$ je obdĺžnik, ktorého strany sú rovnobežné so stranami obdĺžnika $ABCD$. Vypočítajte pomer $|XY|:|YZ|$.}
\podpis{Pavel Calábek}

{%%%%% C-II-4
Určte počet usporiadaných štvoríc $(a,b,c,d)$ čísel z~množiny $\{1,2,3,\dots,1000\}$, ktoré súčasne spĺňajú rovnosti
$$
ab=cd,\qquad a^2+b^2=c^2+d^2.
$$
}
\podpis{Ján Mazák}

{%%%%%   vyberko, den 1, priklad 1
...}
\podpis{...}

{%%%%%   vyberko, den 1, priklad 2
...}
\podpis{...}

{%%%%%   vyberko, den 1, priklad 3
...}
\podpis{...}

{%%%%%   vyberko, den 1, priklad 4
...}
\podpis{...}

{%%%%%   vyberko, den 2, priklad 1
...}
\podpis{...}

{%%%%%   vyberko, den 2, priklad 2
...}
\podpis{...}

{%%%%%   vyberko, den 2, priklad 3
...}
\podpis{...}

{%%%%%   vyberko, den 2, priklad 4
...}
\podpis{...}

{%%%%%   vyberko, den 3, priklad 1
...}
\podpis{...}

{%%%%%   vyberko, den 3, priklad 2
...}
\podpis{...}

{%%%%%   vyberko, den 3, priklad 3
...}
\podpis{...}

{%%%%%   vyberko, den 3, priklad 4
...}
\podpis{...}

{%%%%%   vyberko, den 4, priklad 1
...}
\podpis{...}

{%%%%%   vyberko, den 4, priklad 2
...}
\podpis{...}

{%%%%%   vyberko, den 4, priklad 3
...}
\podpis{...}

{%%%%%   vyberko, den 4, priklad 4
...}
\podpis{...}

{%%%%%   vyberko, den 5, priklad 1
...}
\podpis{...}

{%%%%%   vyberko, den 5, priklad 2
...}
\podpis{...}

{%%%%%   vyberko, den 5, priklad 3
...}
\podpis{...}

{%%%%%   vyberko, den 5, priklad 4
...}
\podpis{...}

{%%%%%   trojstretnutie, priklad 1
Rozhodnite, či existuje 2024 navzájom rôznych kladných celých čísel s nasledovnou vlastnosťou: Keď uvážime všetky možné podiely dvoch rôznych čísel (uvažujeme $a/b$ aj $b/a$), tak dostaneme čísla s~konečnými desatinnými rozvojmi (za desatinnou čiarkou) navzájom rôznych nenulových dĺžok.}
\podpis{Patrik Bak, Slovensko}

{%%%%%   trojstretnutie, priklad 2
Pre kladné celé číslo $n$ povieme, že \emph{$n$-konfigurácia} pozostáva z množín (nie nutne rôznych) $A_{i,j}$ pre všetky celočíselné dvojice indexov $(i, j)$, kde  $1\le i,j\le n$. Hovoríme, že $n$-konfigurácia je \emph{sladká}, ak pre každú dvojicu indexov $(i, j)$ spĺňajúcu  $1\le i\le n-1$ a $1\le j \le n$ platí $A_{i,j} \subseteq A_{i+1,j}$ a $A_{j,i} \subseteq A_{j,i+1}$. Nech $f(n, k)$ označuje počet sladkých $n$-konfigurácií, pre ktoré $A_{n,n}\subseteq \{1,2,\ldots, k\}$. Rozhodnite, ktoré číslo je väčšie: $f(2024,2024^2)$ alebo $f(2024^2,2024)$.}
\podpis{Wojciech Nadara, Poľsko}

{%%%%%   trojstretnutie, priklad 3
Nech $ABC$ je trojuholník a $D$ bod na strane $BC$.
Body $E$, $F$ ležia postupne na priamkach $AB$, $AC$ tak, že
bod $B$ leží medzi bodmi $A$ a $E$ spĺňajúc $|BE|=|BD|$ a~bod $C$ leží medzi bodmi $A$ a $F$ spĺňajúc $|CF|=|CD|$.
Nech $P$ je taký bod, pre ktorý je $D$ stredom kružnice vpísanej trojuholníku $PEF$. Dokážte, že bod $P$ leží vo vnútornej oblasti kružnice $\Omega$ opísanej trojuholníku $ABC$ alebo na nej.}
\podpis{Josef Tkadlec, ČR}

{%%%%%   trojstretnutie, priklad 4
Nech $ABCD$ je konvexný štvoruholník, pre ktorý platí $|AB| = |BC| = |CD|$. Body $X$, $Y$ ležia postupne na polpriamkach $CA$, $BD$ tak, že $|BX| = |CY|$. Nech $P$, $Q$, $R$, $S$ sú postupne stredy úsečiek $BX$, $CY$, $XD$, $YA$. Dokážte, že body $P$, $Q$, $R$, $S$ ležia na kružnici.}
\podpis{Michal Pecho, SR}

{%%%%%   trojstretnutie, priklad 5
Nech $\alpha \ne 0$ je reálne číslo. Nájdite všetky funkcie $f \colon \Bbb{R} \to\Bbb{R}$ spĺňajúce
$$
f(x^2+y^2) = f(x-y)f(x+y)+\alpha yf(y)
$$
pre všetky $x$, $y\in\Bbb{R}$.}
\podpis{Walther Janous, Rakúsko}

{%%%%%   trojstretnutie, priklad 6
Rozhodnite, či existuje nekonečne veľa trojíc $(a, b, c)$ kladných celých čísel takých, že $p$ delí $\lfloor{(a+b\sqrt{2024})^p}\rfloor - c$ pre každé prvočíslo $p$.}
\podpis{Walther Janous, Rakúsko}

{%%%%%   IMO, priklad 1
...}
\podpis{...}

{%%%%%   IMO, priklad 2
...}
\podpis{...}

{%%%%%   IMO, priklad 3
...}
\podpis{...}

{%%%%%   IMO, priklad 4
...}
\podpis{...}

{%%%%%   IMO, priklad 5
...}
\podpis{...}

{%%%%%   IMO, priklad 6
...}
\podpis{...}

{%%%%%   MEMO, priklad 1
Určte všetky $k\in\Bbb{N}_0$, pre ktoré existuje funkcia $f \colon \Bbb{N}_0 \to \Bbb{N}_0$ taká, že $f(2024)=k$ a
$$
f(f(n)) \le f(n+1) - f(n)
$$
platí pre všetky $n \in \Bbb{N}_0$.

\poznamka
Symbolom $\Bbb{N}_0$ značíme množinu nezáporných celých čísel.}
\podpis{Rakúsko}

{%%%%%   MEMO, priklad 2
Na nekonečnej tabuli sa nachádza list papiera (taký ako tento). Marvin si tajne zvolí konvexný $2024$-uholník, ktorý sa nachádza celý na papieri. Tigrica chce nájsť vrcholy $2024$-uholníka $P$. V~každom kroku Tigrica narysuje na tabuľu priamku $g$, ktorá je celá mimo papier, potom Marvin odpovie priamkou $h$ rovnobežnou s priamkou $g$, ktorá je zo všetkých rovnobežných priamok prechádzajúcich aspoň jedným vrcholom $2024$-uholníka $P$ najbližšie k priamke $g$. Dokážte, že existuje kladné celé číslo $n$ také, že Tigrica vie vždy nájsť vrcholy $2024$-uholníka $P$ za najviac $n$ krokoch.}
\podpis{Maďarsko}

{%%%%%   MEMO, priklad 3
Nech $ABC$ je ostrouhlý rôznostranný trojuholník.
Zvoľme kružnicu $\omega$ prechádzajúcu bodmi $B$ a $C$, ktorá pretne druhýkrát úsečky $AB$ a $AC$ postupne v bodoch $D\ne A$ a $E\ne A$.
Nech $F$ je priesečník priamok $BE$ a $CD$.
Nech $G$ je bod na kružnici opísanej trojuholníku $ABF$ taký, že $GB$ je dotyčnicou kružnice $\omega$.
Podobne, nech $H$ je bod na kružnici opísanej trojuholníku $ACF$ taký, že $HC$ je dotyčnicou kružnice $\omega$.
Dokážte, že existuje taký bod $T\ne A$, ktorý nezávisí na voľbe kružnice $\omega$, že kružnica opísaná trojuholníku $AGH$ prechádza bodom $T$.}
\podpis{Patrik Bak, Slovensko}

{%%%%%   MEMO, priklad 4
Pre ľubovoľné kladné celé číslo $n$ označme $\sigma(n)$ súčet kladných deliteľov čísla $n$.
Určte všetky mnohočleny $P(x)$ s celočíselnými koeficientmi také, že $P(k)$ je deliteľné číslom $\sigma(k)$ pre všetky kladné celé čísla $k$.}
\podpis{Rakúsko}

{%%%%%   MEMO, priklad t1
Uvažujme dve nekonečné postupnosti reálnych čísel $a_0,a_1,a_2,\ldots$ a $b_0,b_1,b_2,\ldots$ také, že  $a_0 = 0$, $b_0 = 0$ a
$$
a_{k+1} = b_k, \qquad b_{k+1} = \frac{a_k b_k + a_k + 1}{b_k + 1}
$$
pre každé celé číslo $k \geq 0$.
Dokážte, že platí $a_{2024} + b_{2024} \geq 88$.}
\podpis{Marián Poturnay, Slovensko}

{%%%%%   MEMO, priklad t2
Nájdite všetky funkcie $f \colon \Bbb R \to \Bbb R$ také, že
$$
\postdisplaypenalty=10000
yf(x+1) = f(x+y-f(x)) + f(x)f(f(y))
$$
platí pre všetky $x, y \in \Bbb R$.}
\podpis{Chorvátsko}

{%%%%%   MEMO, priklad t3
Na brehu rieky Tisa sedí v rade $2024$ matematikov. Každý z nich pracuje na práve jednej výskumnej téme, pričom ak dvaja matematici pracujú na rovnakej téme, tak na nej pracujú aj všetci matematici sediaci medzi nimi.

Marvin chce zistiť pre každú dvojicu matematikov, či pracujú na rovnakej téme. Môže sa spýtať ľubovoľného matematika nasledovnú otázku "Koľko z týchto $2024$ matematikov pracuje na tvojej téme?" Marvin sa pýta otázky postupne, takže kým sa spýta ďalšiu otázku, tak vie odpovede na všetky predošlé otázky.

Určte najmenšie kladné celé číslo $k$ také, že Marvin vie splniť svoj cieľ za použitia najviac $k$ otázok.}
\podpis{Maďarsko}

{%%%%%   MEMO, priklad t4
Konečná postupnosť $x_1, x_2, \ldots, x_r$ kladných celých čísel je \emph{palindróm}, ak platí $x_i=x_{r+1-i}$ pre všetky celé čísla $1\leq i \leq r$.

Nech $a_1, a_2, \ldots$ je nekonečná postupnosť kladných celých čísel. Pre kladné celé číslo $j\geq 2$ označme $a[j]$ konečnú podpostupnosť $a_1, a_2, \ldots, a_{j-1}$. Predpokladajme, že existuje (rýdzo) rastúca nekonečná postupnosť $b_1, b_2, \ldots$ kladných celých čísel taká, že pre každé kladné celé číslo $n$ je podpostupnosť $a[b_n]$ palindróm a platí $b_{n+2}\leq b_{n+1}+b_n$. Dokážte, že existuje kladné celé číslo $T$ také, že platí $a_i=a_{i+T}$ pre každé kladné celé číslo $i$.}
\podpis{Chorvátsko}

{%%%%%   MEMO, priklad t5
Nech $ABC$ je trojuholník, pre ktorý platí $|\angle BAC| = 60^\circ$.
Nech $D$ je bod na priamke $AC$ taký, že platí $|AB| = |AD|$, pričom $A$ leží medzi $C$ a $D$.
Predpokladajme, že na kružnici opísanej trojuholníku $DBC$ ležia body $E \ne F$ také, že platí $|AE| =  |AF| = |BC|$.
Dokážte, že priamka $EF$ prechádza stredom kružnice opísanej trojuholníku $ABC$.}
\podpis{Marián Poturnay, Slovensko}

{%%%%%   MEMO, priklad t6
Nech $ABC$ je ostrouhlý trojuholník.
Nech $M$ je stred úsečky $BC$.
Nech $I$, $J$, $K$ sú postupne stredy kružníc vpísaných trojuholníkom $ABC$, $ABM$, $ACM$.
Nech $P$, $Q$ sú body postupne na priamkach $MK$, $MJ$ také, že platí $|\angle AJP| = |\angle ABC|$ a $|\angle AKQ| = |\angle BCA|$.
Nech $R$ je priesečník priamok $CP$ a $BQ$.
Dokážte, že priamky $IR$ a $BC$ sú na seba kolmé.}
\podpis{Michal Pecho, Slovensko}

{%%%%%   MEMO, priklad t7
{\it Zlepenie} kladných celých čísel prebieha tak, že ich zápisy v desiatkovej sústave zapíšeme postupne za sebou a výsledok interpretujeme ako zápis jedného kladného celého čísla v desiatkovej sústave.

Nájdite všetky kladné celé čísla $k$, pre ktoré existuje celé číslo $N_k$ s nasledovnou vlastnosťou: pre všetky $n \ge N_k$ možno v nejakom poradí zlepiť čísla $1,2,\dots,n$ tak, aby výsledné číslo bolo deliteľné číslom $k$.

\poznamka
Zápis v desiatkovej sústave sa nemôže začínať nulou.

{\it Príklad.} Zlepením čísel $15$, $14$, $7$ v tomto poradí získame číslo $15\,147$.
}
\podpis{Patrik Bak, Slovensko}

{%%%%%   MEMO, priklad t8
Nech $k$ je kladné celé číslo a nech $a_1, a_2, \ldots$ je nekonečná postupnosť kladných celých čísel taká, že platí $$a_ia_{i+1} \mid k-a_i^2$$
pre všetky celé čísla $i\geq 1.$
Dokážte, že existuje kladné celé číslo $M$ také, že platí $a_n=a_{n+1}$ pre všetky celé čísla $n \geq M$.}
\podpis{Chorvátsko}

{%%%%%   CPSJ, priklad 1
Na začiatku sú na tabuli napísané čísla $1$ a $2$. V každom kroku si vyberieme kladné reálne číslo $x$ a nahradíme dvojicu $(a,b)$ čísel napísaných na tabuli dvojicou
$$\left(a+\frac{x}{b},b+\frac{x}{a}\right).$$
Je možné (po konečnom počte krokov) dosiahnuť stav, v ktorom sú na tabuli napísané čísla $2$ a $3$?}
\podpis{Poľsko}

{%%%%%   CPSJ, priklad 2
Pre koľko (neprázdnych) podmnožín množiny $\{1,2,3,4,\dots,11\}$ je súčin jej prvkov treťou mocninou kladného celého čísla?}
\podpis{Tomáš Bárta}

{%%%%%   CPSJ, priklad 3
Nech $ABCD$ je konvexný štvoruholník, v ktorom $|AB|=|BD|=|DC|$ a ${AB\perp BD\perp DC}$. Označme $M$ stred strany $BC$. Dokážte, že
$|\uhol BAM|+|\uhol DCA|=45^\circ.$}
\podpis{Jaroslav Švrček}

{%%%%%   CPSJ, priklad 4
Celé čísla $a$, $b$, $c$ spĺňajú $a + b + c = 1$ a $ab + bc + ca < abc$. Dokážte, že
$$ab + bc + ca < 2abc.$$}
\podpis{Poľsko}

{%%%%%   CPSJ, priklad 5
Pre kladné celé číslo $n$ označme $S(n)$ súčet číslic v jeho desiatkovom zápise. Nájdite najmenšie kladné celé číslo $n$, pre ktoré platí
$4 S(n) = 3 S(2 n).$}
\podpis{Eliška Macáková}

{%%%%%   CPSJ, priklad t1
Označme $G$ těžiště trojúhelníku $ABC$. Nechť $D$ je čtvrtý vrchol rovnoběžníku $AGDB$. Dokažte, že $BG \parallel CD$.}
\podpis{Patrik Bak}

{%%%%%   CPSJ, priklad t2
Mezi trojicemi $(a,b,c)$ přirozených čísel splňujících
$$
\left(a+14\sqrt{3}\right)\left(b-14c\sqrt{3}\right)=2024
$$
určete tu s největší hodnotou $a$.}
\podpis{Mária Dományová}

{%%%%%   CPSJ, priklad t3
Wyznacz miary k\ą{}tów wewn\ę{}trznych we wszystkich trójk\ą{}tach równoramiennych, które można podzielić na dwa trójk\ą{}ty równoramienne o~roz\l{}\ą{}cznych wn\ę{}trzach. }
\podpis{Jaroslav Švrček}

{%%%%%   CPSJ, priklad t4
Ile jest dodatnich liczb ca\l{}kowitych $n$ mniejszych od $2024$ i podzielnych przez $\lfloor \sqrt{n}\rfloor-1$? Symbol $\lfloor x\rfloor$ oznacza najwi\ę{}ksz\ą{} liczb\ę{} ca\l{}kowit\ą{} nie wi\ę{}ksz\ą{} od $x$.

\nopagebreak\smallskip\noindent
Na przyk\l{}ad $n=8$ spe\l{}nia dany warunek, gdyż $8$ jest podzielne przez $\lfloor\sqrt{8}\rfloor-1=2-1=1$, ale $n=9$ go nie spe\l{}nia, gdyż $9$ nie jest podzielne przez $\lfloor \sqrt{9}\rfloor -1=2$.}
\podpis{Patrik Bak}

{%%%%%   CPSJ, priklad t5
Existuje celé číslo $n\geq 1$ také, že keď cifry čísla $2^n$, zapísaného v desiatkovej sústave, napíšeme v opačnom poradí, dostaneme \emph{inú} celočíselnú mocninu dvojky?}
\podpis{Poľsko}

{%%%%%   CPSJ, priklad t6
V každom políčku obdĺžnikovej tabuľky je kladné celé číslo. Pre každé políčko tabuľky platí, že číslo v ňom je rovné celkovému počtu rôznych hodnôt v políčkach, ktoré sú s ním v rovnakom riadku alebo stĺpci (vrátane seba samého). Nájdite všetky tabuľky s takouto vlastnosťou.}
\podpis{Poľsko}

{%%%%%   EGMO, priklad 1
...}
\podpis{...}

{%%%%%   EGMO, priklad 2
...}
\podpis{...}

{%%%%%   EGMO, priklad 3
...}
\podpis{...}

{%%%%%   EGMO, priklad 4
...}
\podpis{...}

{%%%%%   EGMO, priklad 5
...}
\podpis{...}

{%%%%%   EGMO, priklad 6
...}
\podpis{...}

{%%%%%   vyberko C, den 1, priklad 1
Kladné celé číslo nazveme vyvážené, ak je počet rôznych prvočísel, ktoré toto číslo delia, rovný počtu cifier tohto čísla. Napríklad číslo $385 = 5 \cdot 7 \cdot  11$ je vyvážené, ale $275 = 5^2 \cdot  11$ vyvážené nie je. Dokážte, že existuje iba konečne veľa vyvážených čísel.}
\podpis{...}

{%%%%%   vyberko C, den 1, priklad 2
Na kružnici ležia body $A$, $B$, $C$, $D$ a $E$ v tomto poradí tak, že platí $|\angle ABE| = |\angle BEC| = |\angle ECD| = 45^{\circ}$. Dokážte, že platí $$|AB|^2 + |CE|^2 = |BE|^2 + |CD|^2.$$}
\podpis{...}

{%%%%%   vyberko C, den 1, priklad 3
$3m$ loptičiek očíslovaných 1, 1, 1, 2, 2, 2, 3, 3, 3, $\ldots$, $m$, $m$, $m$ je rozmiestnených do $8$ krabičiek tak, že každé dve krabičky obsahujú loptičku s rovnakým číslom. Nájdite najmenšie $m$, pre ktoré to mohlo nastať.}
\podpis{...}

{%%%%%   vyberko C, den 1, priklad 4
Nech $n\geq 2$ je kladné celé číslo a nech $a_1,a_2,...,a_n\in\langle0,1\rangle$ sú reálne čísla. Nájdite najväčšiu možnú hodnotu najmenšieho z čísel
$$\align
    a_1&-a_1a_2,\\
     a_2&-a_2a_3,\\
     &\,\,\,\vdots\\
     a_n&-a_na_1.
\endalign
$$
}
\podpis{...}

{%%%%%   vyberko C, den 1, priklad 5
V rovnobežníku $ABCD$ leží bod $P$ tak, že platí $|PC| = |BC|$. Ukážte, že priamka $BP$ je kolmá na priamku, ktorá spája stredy úsečiek $AP$ a $CD$.}
\podpis{...} 