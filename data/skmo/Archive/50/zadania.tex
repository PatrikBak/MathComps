{%%%%%   A-I-1
V~urne sú len biele a~čierne guľôčky, ktorých počet zaokrúhlený
na stovky je 1\,000. Pravdepodobnosť vytiahnutia dvoch čiernych guľôčok
je o~$\frac{17}{43}$ väčšia ako pravdepodobnosť vytiahnutia dvoch bielych
guľôčok. Koľko bielych a~koľko čiernych guľôčok je v~urne?
(Pravdepodobnosť vytiahnutia ktorejkoľvek guľôčky je rovnaká.)}
\podpis{P. Černek}

{%%%%%   A-I-2
Nech $a_1$, $a_2$ sú prirodzené čísla a~nech pre každé prirodzené $n\ge2$
je číslo $a_{n+1}$ o~1 väčšie ako najväčší nepárny deliteľ súčtu
$a_n+a_{n-1}$. Dokážte, že postupnosť $a_1,a_2,a_3,\dots$ je od
určitého člena počínajúc periodická. Nájdite všetky také
postupnosti, ktoré sú periodické už od prvého člena.}
\podpis{J. Bábeľa}

{%%%%%   A-I-3
V~rovine je daný ostrouhlý trojuholník $ABC$. Päty výšok z~vrcholov
$A$, $B$ označme postupne $A_1$, $B_1$. Dotyčnice kružnice opísanej
trojuholníku $CA_1B_1$ zostrojené v~bodoch $A_1$, $B_1$ sa pretínajú v~bode
$M$. Dokážte, že kružnice opísané trojuholníkom $AMB_1$, $BMA_1$, $CA_1B_1$
prechádzajú jedným bodom.}
\podpis{J. Švrček}

{%%%%%   A-I-4
V~obore reálnych čísel riešte sústavu nerovníc
$$
  \align
    \sin{x} + \cos{y} &\ge \sqrt2,\\
    \sin{y} + \cos{z} &\ge \sqrt2,\\
    \sin{z} + \cos{x} &\ge \sqrt2.
  \endalign
$$}
\podpis{J. Švrček}

{%%%%%   A-I-5
Nájdite všetky funkcie $f:\Bbb R\to\Bbb R$ také, že pre všetky
reálne čísla $x$, $y$ platí
$$
  x^2 + y^2 + 2f(xy) = f(x+y) \cdot \bigl( f(x)+f(y) \bigr).
$$}
\podpis{E. Kováč}

{%%%%%   A-I-6
Zostrojte lichobežník $ABCD$, ak sú dané dĺžky jeho ramien
$|BC|=4{,}5\cm$, $|DA|=3\cm$ a~veľkosť $75\st$ uhla, ktorý
zvierajú priamky $BC$ a~$AD$, a~ak navyše platí $|AB|\cdot|CD|=|AC|^2$.}
\podpis{E. Kováč}

{%%%%%   B-I-1
V~obore kladných čísel riešte sústavu rovníc
$$
  \align
    3x + y_{10} =& 598{,}6, \\
    x_{10} + 2y =& 723{,}4,
  \endalign
$$
pričom $x_{10}$ a~$y_{10}$ označujú postupne čísla $x$ a~$y$ zaokrúhlené na
desiatky.}
\podpis{S. Bednářová}

{%%%%%   B-I-2
Na povrchu kocky $ABCDEFGH$ je zostrojená lomená čiara zložená zo
štyroch zhodných úsečiek v~stenách $ABFE$, $BCGF$, $CDHG$ a~$GHEF$,
ktorá vychádza z~vrcholu~$A$ a~končí vo vrchole~$E$.
Určte, v~akom pomere delí táto lomená čiara hranu $CG$.}
\podpis{J. Zhouf}

{%%%%%   B-I-3
Do každého poľa štvorcovej tabuľky $n\times n$ vpíšeme jedno
z~čísel $1,2,\dots,n$ tak, aby v~každom riadku aj~v~každom stĺpci boli
buď všetky čísla rovnaké, alebo všetky navzájom rôzne. Príkladom pre $n=5$
je nasledujúca tabuľka:
$$
\begin{array}{|c|c|c|c|c|}
\hline
5 & 4 & 1 & 2 & 3 \\
\hline
3 & 3 & 3 & 3 & 3 \\
\hline
4 & 1 & 2 & 5 & 3 \\
\hline
1 & 2 & 5 & 4 & 3 \\
\hline
2 & 5 & 4 & 1 & 3 \\
\hline
\end{array}
$$
Označme $S$ súčet všetkých čísel tabuľky. Koľko rôznych hodnôt~$S$ pre
dané $n$ existuje?}
\podpis{J. Šimša}

{%%%%%   B-I-4
Nech $k$ je kružnica opísaná trojuholníku $ABC$, $D$ je priesečník
ťažnice na stranu~$AB$ s~kružnicou $k$. Dotyčnice ku kružnici $k$ v~bodoch
$A$, $B$, $C$, $D$ vytvárajú štvoruholník $PQRS$. Zistite, pre ktoré
trojuholníky $ABC$ je štvoruholník $PQRS$ tetivový.}
\podpis{J. Földes}

{%%%%%   B-I-5
Určte všetky polynómy $P(x)$ také, že pre každé reálne číslo $x$
je splnená rovnosť
$$
  \postdisplaypenalty 10000
  P(2x)=8P(x)+(x-2)^2.
$$}
\podpis{P. Černek}

{%%%%%   B-I-6
Zostrojte trojuholník $ABC$ s~obsahom $18\cm^2$ a~nasledujúcou
vlastnosťou: obvod každého pravouholníka $KLMN$, ktorého vrcholy $K$, $L$
ležia na úsečke~$AB$ a~body $M$, $N$ postupne na úsečkách $BC$,
$AC$, je rovný trom pätinám obvodu trojuholníka $ABC$.}
\podpis{S. Bednářová}

{%%%%%   C-I-1
Nájdite všetky trojciferné čísla $n$ také, že posledné trojčíslie
čísla $n^2$ je zhodné s~číslom $n$.}
\podpis{J. Zhouf}

{%%%%%   C-I-2
Zostrojte lichobežník, ak sú dané dĺžky $9\cm$ a~$12\cm$ jeho
uhlopriečok, dĺžka $8\cm$ strednej priečky a~vzdialenosť $2\cm$ stredov
uhlopriečok.}
\podpis{E. Kováč}

{%%%%%   C-I-3
Nájdite všetky dvojice prirodzených čísel $a$, $b$, pre ktoré platí
$$
  n(a,b) + D(a,b) = 63,
$$
kde $n(a,b)$ označuje najmenší spoločný násobok a~$D(a,b)$ najväčší
spoločný deliteľ čísel $a$, $b$.}
\podpis{L. Boček}

{%%%%%   C-I-4
Dokážte, že pre dĺžky $a$, $b$, $c$ strán ľubovoľného trojuholníka
platí
$$
  {(a^2+b^2)c^2-(a^2-b^2)^2\over abc^2}\le2.
$$
Pre ktoré trojuholníky nastane v~predchádzajúcom vzťahu rovnosť?}
\podpis{J. Šimša}

{%%%%%   C-I-5
Tridsať maturantov jedného gymnázia si podalo prihlášku na ďalšie
štúdium na niektorú zo~šiestich fakúlt Slovenskej technickej univerzity.
Využili možnosť podať viac prihlášok, a~tak polovica
žiakov podala prihlášku aspoň na tri fakulty. Tretina študentov
si podala prihlášku na viac ako tri fakulty. Na fakultu architektúry sa
vzhľadom na talentové prijímacie skúšky nehlásil nikto. Dokážte,
že na niektorú zo~zvyšných piatich fakúlt sa prihlásilo menej ako
dvadsať študentov.}
\podpis{P. Hliněný}

{%%%%%   C-I-6
Do danej kružnice s~polomerom~$r$ vpíšte lichobežník $ABCD$ s~kratšou
základňou~$CD$ a~priesečníkom uhlopriečok~$E$ tak, aby platilo
$|BC|=|CD|$ a~$|AE|=r$.}
\podpis{P. Leischner}

{%%%%%   A-S-1
Nájdite všetky reálne čísla~$p$, pre ktoré má sústava nerovníc
$$
\align
25+2x^2&\leq 13y+10z-p,\\
25+3y^2&\leq 6z+10x,\\
25+4z^2&\leq 6x+5y+p
\endalign
$$
s~neznámymi $x$, $y$, $z$ riešenie v~obore reálnych čísel.}
\podpis{J. Švrček}

{%%%%%   A-S-2
Je daný lichobežník $ABCD$ so základňou~$AB$ dĺžky~$a$, v~ktorom
sú oba uhly $ABC$ a~$ADB$ pravé. Na strane~$AB$ leží bod~$M$
taký, že úsečka~$MD$ je kolmá na $AC$ a~úsečka~$MC$ je kolmá
na $BD$. Určte dĺžky ostatných strán lichobežníka.}
\podpis{J. Zhouf}

{%%%%%   A-S-3
Nájdite všetky štvorciferné čísla $\overline{abcd}$, ktoré sú
deliteľné každým z~dvojciferných čísel $\overline{ab}$,
$\overline{bc}$, $\overline{cd}$, pričom cifry $a$, $b$, $c$,
$d$ sú nepárne a~nie všetky rovnaké.}
\podpis{J. Šimša}

{%%%%%   A-II-1
Nájdite najmenšie štvorciferné číslo $n$, pre ktoré má sústava
$$
\align
x^3+y^3+y^2x+x^2y&=n,\\
x^2+y^2+x+y&=n+1
\endalign
$$
iba celočíselné riešenia.}
\podpis{J. Zhouf}

{%%%%%   A-II-2
Určte všetky reálne čísla $s$ a~$t$, pre ktoré je grafom funkcie
$$
f(x)=\frac{x^2-4x+s}{t|x-1|+x+7}
$$
lomená čiara zložená z~dvoch polpriamok.}
\podpis{P. Černek}

{%%%%%   A-II-3
Je daná kružnica $k(S,r)$ a~na nej body $M$, $N$ také, že uhol
$MSN$ je ostrý. Ľubovoľným bodom~$X$ menšieho z~oblúkov~$MN$ veďme
rovnobežku s~priamkou~$MS$ a~označme $Y$ jej priesečník s~úsečkou~$SN$.
Zostrojte taký bod~$X$, pre ktorý je obsah trojuholníka
$SXY$ maximálny.}
\podpis{P. Černek}

{%%%%%   A-II-4
Určte všetky funkcie $f:\Bbb R\to\Bbb R$ také, že pre
všetky reálne čísla $x$ a~$y$ platí
$$
f\bigl(x^2+f(y)\bigr)=(x-y)^2\cdot f(x+y).
$$}
\podpis{P. Calábek}

{%%%%%   A-III-1
Určte všetky mnohočleny $P(x)$ s~reálnymi koeficientmi také, že pre všetky
reálne čísla~$x$ platí
$$
\bigl(P(x)\bigr)^{2}+P(-x)=P(x^2)+P(x).
$$}
\podpis{P. Calábek}

{%%%%%   A-III-2
V~rovine je daný trojuholník $PQX$, pričom $|PQ|=3\cm$,
$|PX|=2{,}6\cm$, $|QX|=3{,}8\cm$. Zostrojte pravouhlý
trojuholník $ABC$ tak, aby sa jeho vpísaná kružnica dotýkala
prepony~$AB$ v~bode~$P$, odvesny~$BC$ v~bode~$Q$ a~aby bod~$X$
ležal na priamke~$AC$.}
\podpis{J. Šimša}

{%%%%%   A-III-3
Nájdite všetky trojice reálnych čísel $a$, $b$, $c$, pre ktoré
je množinou všetkých riešení nerovnice
$$
\sqrt{2x^2+ax+b}>x-c
$$
s~neznámou~$x$ množina $(-\infty,0\rangle\cup(1,\infty)$.}
\podpis{P. Černek}

{%%%%%   A-III-4
V~istom jazyku je $n$~písmen. Skupina písmen (napísaných za
sebou) je slovom práve vtedy, keď sa medzi žiadnymi dvoma rovnakými
písmenami nenachádzajú dve rovnaké písmená. Určte počet všetkých slov
maximálnej dĺžky.}
\podpis{K. Černeková}

{%%%%%   A-III-5
Z~papiera bol vystrihnutý rovnoramenný lichobežník $C_1AB_2C_2$
s~kratšou základňou~$B_2C_2$. Pätu kolmice zo stredu~$D$ ramena~$C_1C_2$
na základňu~$AC_1$ označíme $B_1$. Po prehnutí papiera
pozdĺž úsečiek $DB_1$, $AD$ a~$AC_2$ sa body $C_1$, $C_2$
premiestnili v~priestore do jedného bodu~$C$ a~body $B_1$, $B_2$
do bodu~$B$. Vznikol tak model štvorstena $ABCD$ s~objemom
$64\cm^{3}$. Určte dĺžky strán pôvodného lichobežníka.}
\podpis{P. Leischner}

{%%%%%   A-III-6
Dané sú prirodzené čísla $a_1,a_2,\dots,a_n$ a~funkcia
$f:\Bbb Z\to\Bbb R$ taká, že $f(x)=1$ pre každé celé $x<0$ a
$$
f(x)=1-f(x-a_1)\,f(x-a_2)\cdots f(x-a_n)
$$
pre každé celé $x\geq0$. Dokážte, že existujú prirodzené čísla
$s$ a~$t$ také, že pre každé celé $x>s$ platí $f(x+t)=f(x)$.}
\podpis{P. Kaňovský}

{%%%%%   B-S-1
Nájdite všetky trojciferné čísla~$n$, ktorých druhá mocnina končí
rovnakým trojčíslím ako druhá mocnina čísla $3n-2$.}
\podpis{J. Šimša}

{%%%%%   B-S-2
Je daný tetivový štvoruholník $ABCD$. Označme $E$ priesečník priamok
$BC$ a~$AD$. Ak leží priesečník uhlopriečok $AC$ a~$BD$ na
osi uhla $AEB$, tak je trojuholník $ABE$ rovnoramenný. Dokážte.}
\podpis{E. Kováč}

{%%%%%   B-S-3
Určte mnohočleny $P$ a~$Q$ také, že pre všetky reálne čísla
$x$ platí
$$
Q(x^2)=(x+1)^4 - x\bigl(P(x)\bigr)^2.
$$}
\podpis{P. Černek}

{%%%%%   B-II-1
Určte všetky reálne čísla~$p$ také, že pre ľubovoľné kladné
čísla $x$, $y$ platí nerovnosť
$$
\frac{x^3+py^3}{x+y}\geq xy.
$$}
\podpis{J. Bábeľa}

{%%%%%   B-II-2
Daný je trojuholník $ABC$. Zostrojte rovnobežník $KLMN$ tak, aby
jeho vrcholy $K$ a~$L$ ležali na strane~$AB$, vrchol~$M$ na
strane~$BC$, vrchol~$N$ na strane~$AC$ a~aby trojuholníky
$AKN$, $LBM$ a~$NMC$ mali rovnaké obsahy.}
\podpis{J. Šimša}

{%%%%%   B-II-3
Určte všetky prirodzené čísla~$n$, pre ktoré je podiel
$$
\frac{(n^2)_{10}}{(n_{10})^2}
$$
celé číslo. Zápis~$z_{10}$ označuje číslo, ktoré vznikne
zaokrúhlením čísla~$z$ na desiatky.}
\podpis{S. Bednářová}

{%%%%%   B-II-4
Nájdite všetky ostrouhlé trojuholníky $ABC$, ktorých ťažisko~$T$
splýva s~priesečníkom výšok trojuholníka $PQR$, pričom body $P$,
$Q$, $R$ sú postupne priesečníky polpriamok opačných k~polpriamkam
$TA$, $TB$, $TC$ s~kružnicou opísanou trojuholníku $ABC$.}
\podpis{J. Földes}

{%%%%%   C-S-1
Nájdite všetky trojice $a$, $b$, $c$ prirodzených čísel pre
ktoré súčasne platí
$$
n(ab,c)=2^8,\qquad n(bc,a)=2^9,\qquad n(ca,b)=2^{11},
$$
kde $n(x,y)$ označuje najmenší spoločný násobok prirodzených
čísel $x$ a~$y$.}
\podpis{P. Černek}

{%%%%%   C-S-2
V~rovine je daný štvorec $ABCD$. Kružnica~$k$ prechádza bodmi $A$,
$B$ a~dotýka sa priamky~$CD$. Označme $M$ ($M\ne B$) priesečník
kružnice~$k$ a~strany~$BC$. Určte pomer $|CM|:|BM|$.}
\podpis{J. Švrček}

{%%%%%   C-S-3
Pre ktoré dvojciferné čísla~$n$ je číslo $n^{3}-n$ deliteľné číslom
sto?}
\podpis{J. Zhouf}

{%%%%%   C-II-1
Nájdite všetky dvojice prirodzených čísel $a$, $b$, pre ktoré
platí
$$
   a + b + D(a,b) + n(a,b) = 50,
$$
kde $D(a,b)$ označuje najväčší spoločný
deliteľ a~$n(a,b)$ najmenší spoločný násobok
prirodzených čísel $a$, $b$.}
\podpis{J. Šimša}

{%%%%%   C-II-2
Kružnice $k(S,r)$ a~$l(O,R)$ sa zvonku dotýkajú v~bode $T$. Ich
spoločná dotyčnica v~bode~$T$ pretína ich vonkajšiu spoločnú dotyčnicu
v~bode~$M$. Dokážte, že trojuholník $SOM$ je pravouhlý a~vyjadrite jeho
obsah pomocou polomerov $r$, $R$ daných kružníc.}
\podpis{P. Leischner}

{%%%%%   C-II-3
Nájdite všetky dvojice kladných čísel $x$, $y$, ktoré sú
riešením sústavy rovníc
$$\align
x\cdot y_{10}&=195{,}6,\\
y\cdot x_{10}&=241{,}7.
\endalign$$
Zápis~$z_{10}$ označuje číslo, ktoré vznikne
zaokrúhlením čísla~$z$ na desiatky.}
\podpis{S. Bednářová}

{%%%%%   C-II-4
Zostrojte taký trojuholník $ABC$,
pre ktorý platí, že výška a~ťažnica z~vrcholu~$C$
delia ťažnicu z~vrcholu~$A$ na tri zhodné úsečky, ak je daná
dĺžka strany~$AB$ a~veľkosť výšky z~vrcholu~$C$.}
\podpis{J. Földes}

{%%%%%   vyberko, den 1, priklad 1
Dané sú dve kružnice $k_1$ a~$k_2$, ktoré
sa pretínajú v~dvoch rôznych
bodoch $A$ a~$B$. Priamka~$p$ prechádzajúca bodom~$A$
pretína kružnice $k_1$ a~$k_2$ postupne v~bodoch $C$ a~$D$.
Označme $P$ a~$Q$ projekcie bodu~$B$ zostrojené postupne na
dotyčnicu ku kružnici~$k_1$ v~bode~$C$ a~na dotyčnicu
ku kružnici~$k_2$ v~bode~$D$. Dokážte, že priamka~$PQ$
je dotyčnicou kružnice~$k_3$ zostrojenej nad priemerom~$AB$.}
\podpis{Mgr. Richard Kollár, František Kardoš:???}

{%%%%%   vyberko, den 1, priklad 2
Dané je prirodzené číslo~$n$, $n \ge 2$ a~reálne čísla $x_i$,
$0 \le x_i \le 1$, kde $i = 1,2, \dots, n$.
Dokážte, že platí
$$
(x_1 + x_2 + \dots + x_n) - (x_1x_2 + x_2x_3 + \dots + x_nx_1) \le
\left\lfloor \dfrac n2 \right\rfloor
$$
a~zistite, kedy nastáva rovnosť!}
\podpis{Mgr. Richard Kollár, František Kardoš:???}

{%%%%%   vyberko, den 1, priklad 3
V~bode $(1,1)$ súradnicovej sústavy je umiestnený
balík piesku. Ďalej sa pohybuje vždy podľa
jedného z~nasledujúcich pravidiel:
\item z~bodu $(a,b)$ môže prejsť do bodu $(2a,b)$ alebo do bodu
 $(a,2b)$;
\item z~bodu $(a,b)$ môže prejsť
do bodu $(a-b, b)$, ak $a> b$ alebo do bodu $(a, b-a)$, ak $a<b$.

Do ktorých bodov súradnicovej sústavy sa môže
balík dostať?}
\podpis{Mgr. Richard Kollár, František Kardoš:???}

{%%%%%   vyberko, den 1, priklad 4
Zistite, či medzi prvými $100\,000\,001$ členmi {\it Fibonacciho postupnosti} existuje číslo končiace štyrmi nulami!}
\podpis{Mgr. Richard Kollár, František Kardoš:???}

{%%%%%   vyberko, den 2, priklad 1
Je daných $2n+1$ kladných čísel takých, že rozdiel
medzi súčtom ľubovoľných $n+1$ daných čísel a~súčtom zvyšných $n$~čísel je
kladný. Dokážte, že pre súčin~$B$ všetkých $\binom{2n+1}{n+1}$ takýchto rozdielov a~súčin~$A$
všetkých $2n+1$ daných čísel platí
$$
B^{n+1}\leq A^{\binom{2n}n}.
$$}
\podpis{Mgr. Ján Bábeľa, Tomáš Jurík:???}

{%%%%%   vyberko, den 2, priklad 2
Ak
$$
\frac a{b-c}+\frac b{c-a}+\frac c{a-b}=0,
$$
potom
$$
\frac a{(b-c)^2}+\frac b{(c-a)^2}+\frac c{(a-b)^2}=0.
$$
Dokážte!}
\podpis{Mgr. Ján Bábeľa, Tomáš Jurík:???}

{%%%%%   vyberko, den 2, priklad 3
Uvažujme $n$~rovnakých mincí, ktoré ležia na stole
a~vytvárajú uzavretú reťaz (každé dve susedné
sa dotýkajú). Koľko otáčok vykoná minca
rovnakých rozmerov, ak s~ňou obídeme (bez kĺzania)
celú reťaz, pričom predpokladáme, že pohybujúca
sa minca sa neustále dotýka niektorej z~daných mincí a~pri
svojom pohybe sa dotkne každej z~daných mincí? Ako sa odpoveď
zmení, ak bude mať táto minca $k$-krát väčší polomer ako mince v~reťazi?}
\podpis{Mgr. Ján Bábeľa, Tomáš Jurík:???}

{%%%%%   vyberko, den 2, priklad 4
Dokážte, že na povrchu devätnásťstena,
ktorý je opísaný guli s~polomerom~$10$, existujú dva body,
ktorých vzdialenosť je väčšia ako~21.}
\podpis{Mgr. Ján Bábeľa, Tomáš Jurík:???}

{%%%%%   vyberko, den 3, priklad 1
Tri kružnice rovnakého polomeru rovného~$t$ prechádzajú jedným bodom~$T$,
všetky sú vnútri trojuholníka $ABC$ a~každá z~nich sa dotýka dvoch jeho strán.
Označme $r$ polomer vpísanej a~$R$ polomer opísanej kružnice trojuholníku $ABC$.
Dokážte, že:
\ite (i) $t={rR\over R+r}$;
\ite (ii) $T$ leží na priamke prechádzajúcej cez stred vpísanej a~stred
opísanej kružnice trojuholníka $ABC$.}
\podpis{Ján Špakula:???}

{%%%%%   vyberko, den 3, priklad 2
Nech funkcia~$f$ je definovaná nasledovne:
\item Ak $n=p^k > 1$ je mocnina prvočísla~$p$, potom $f(n) =
  n+1$.
\item Ak $n=p_1^{k_1}\cdots p_r^{k_r}$ ($r>1$) je súčin mocnín navzájom
rôznych prvočísel, potom $f(n) = p_1^{k_1} + \cdots +
p_r^{k_r}$.
Pre každé $m>1$ skonštruujme postupnosť $a_0, a_1, a_2, \dots$ takú, že
$a_0=m$ a $a_{j+1} = f(a_j)$ pre $j\geq0$. Označme $g(m)$ najmenšie číslo v~tejto
postupnosti. Určte hodnoty $g(m)$ pre každé $m>1$.}
\podpis{Ján Špakula:???}

{%%%%%   vyberko, den 3, priklad 3
Uvažujme reálne čísla spĺňajúce podmienku $a\ge b\ge c>0$. Dokážte, že
$$
  \frac{a^2-b^2}c + \frac{c^2-b^2}a + \frac{a^2-c^2}b \ge 3a - 4b +c.
$$}
\podpis{Ján Špakula:???}

{%%%%%   vyberko, den 3, priklad 4
Dokážte, že pre každé prirodzené číslo $n\ge 6$ existuje množina~$M$
obsahujúca $n$~bodov v~rovine takých, že pre každý bod~$P$ z~množiny~$M$
existujú aspoň 3~body v~$M$ vo vzdialenosti~$1$ od $P$!}
\podpis{Ján Špakula:???}

{%%%%%   vyberko, den 4, priklad 1
Dokážte, že pre každé prirodzené číslo~$n$ platí
$$
  (2n + 1)^n \ge (2n)^n + (2n - 1)^n.
$$}
\podpis{Juraj Földes:???}

{%%%%%   vyberko, den 4, priklad 2
Všetky steny konvexného mnohostena sú trojuholníky. Dokážte, že môžeme
nafarbiť každú hranu tohto mnohostena červenou alebo modrou farbou
tak, aby sa z~každého vrcholu dalo dostať do každého iného vrcholu iba
po modrých a~zároveň iba po červených hranách!}
\podpis{Juraj Földes:???}

{%%%%%   vyberko, den 4, priklad 3
Množina~$T_0$ pozostáva zo všetkých čísel tvaru $(2^k)!$, kde $k = 0, 1,
2, \dots$. Pre každé prirodzené číslo $p = 1, 2, \dots , 2001$ označme
$T_p$ množinu čísel, ktoré dostaneme ako súčty niekoľkých (aj jedného)
rôznych čísel z~$T_{p - 1}$. Dokážte, že existuje prirodzené číslo,
ktoré nepatrí do $T_{2001}$.}
\podpis{Juraj Földes:???}

{%%%%%   vyberko, den 4, priklad 4
Dokážte, že ak v~konvexnom päťuholníku $ABCDE$ platí $|\uhol ABC| =
|\uhol ADE|$ a~$|\uhol AEC| = |\uhol ADB|$, tak $|\uhol BAC| = |\uhol
DAE|$.}
\podpis{Juraj Földes:???}

{%%%%%   vyberko, den 5, priklad 1
Nech $u:\Bbb R\to\Bbb R$ je funkcia. Dokážte, že nasledujúce dve podmienky
sú ekvivalentné:
\ite (A)
Buď $u(x)=1$ pre každé $x\in\Bbb R$, alebo je funkcia~$u$ striktne
monotónna a~pre všetky $x,y\in\Bbb R$ platí
$$
  u(x+y) = u(x) u(y).
$$
\ite (B)
Existuje striktne monotónna funkcia $f:\Bbb R\to\Bbb R$ taká, že
pre všetky $x,y\in\Bbb R$ platí
$$
  f(x+y) = f(x) u(y) + f(y).
$$}
\podpis{Eugen Kováč:???}

{%%%%%   vyberko, den 5, priklad 2
Daný je trojuholník $ABC$ a~jeho vnútorný bod~$M$. Dokážte, že platí
nerovnosť
$$
 \min{\{ |MA|, |MB|, |MC| \}} + |MA| + |MB| + |MC|
  < |AB| + |BC| + |AC|.
$$}
\podpis{Eugen Kováč:???}

{%%%%%   vyberko, den 5, priklad 3
Dokážte, že pre každé prirodzené číslo~$n$, $n\ge 2$ existuje $n$-prvková
množina~$S$ celých čísel taká, že pre všetky $a,b\in S$, $a\ne b$ je číslo~$ab$
deliteľné číslom $(a-b)^2$.}
\podpis{Eugen Kováč:???}

{%%%%%   vyberko, den 5, priklad 4
Nech $m$, $n$ sú prirodzené čísla také, že $m\ge n\ge 2$. Dokážte, že počet
všetkých polynómov stupňa $2n-1$ s~navzájom rôznymi koeficientmi z~množiny
$\{1,2,\dots,2m\}$, ktoré sú deliteľné polynómom
$ x^{n-1} +\dots+ x + 1 $,  je
$$
  2^n n! \left[ 4 \binom{m+1}{n+1} - 3 \binom{m}{n} \right].
$$}
\podpis{Eugen Kováč:???}

{%%%%%   trojstretnutie, priklad 1
Dokážte, že pre ľubovoľné kladné čísla $a_1,a_2,\dots ,
  a_n$ ($n\ge 2$) platí nerovnosť
$$
(a_1^3+1)(a_2^3+1)\dots (a_n^3+1)\geq (a_1^2a_2+1)(a_2^2a_3+1)\dots
    (a_n^2a_1+1).
$$}
\podpis{P. Kaňovský}

{%%%%%   trojstretnutie, priklad 2
Trojuholník $ABC$ má ostré vnútorné uhly pri vrcholoch $A$ a~$B$.
Nad stranami $AC$ a~$BC$ sú trojuholníku zvonku pripísané
rovnoramenné trojuholníky $AC\!D$ a~$BC\!E$ so základňami $AC$
a~$BC$ tak, že $|\uhol ADC|=|\uhol ABC|$ a~súčasne $|\uhol
BEC|=|\uhol BAC|$. Označme $S$ stred opísanej kružnice
trojuholníku $ABC$. Dokážte, že dĺžka lomenej čiary $DSE$ sa rovná
obvodu trojuholníka $ABC$ práve vtedy, keď je uhol $AC\!B$ pravý!}
\podpis{J. Šimša}

{%%%%%   trojstretnutie, priklad 3
Pre ľubovoľné prirodzené čísla $n$, $k$ spĺňajúce podmienky
$\frac12n<k\leq \frac23n$ nájdite najmenší počet políčok, ktoré môžeme
obsadiť na štvorcovej šachovnici $n\times n$ tak, aby v~žiadnom
riadku ani v~žiadnom stĺpci šachovnice neexistovalo $k$~voľných
(\tj.~neobsadených) susedných políčok!}
\podpis{J. Šimša}

{%%%%%   trojstretnutie, priklad 4
V~rovine sú dané body $A$, $B$ $(A\ne B)$. V~tejto
  rovine uvažujme ľubovoľný trojuholník $ABC$ s~vlastnosťou: Na jeho
  stranách $BC$, $C\!A$ existujú postupne body $D$, $E$, pre ktoré platí
\ite (i) $\dfrac{|BD|}{|BC|}=\dfrac{|CE|}{|CA|}=\dfrac13$;
\ite (ii) body $A$, $B$, $D$, $E$ ležia (v~tomto poradí) na
          jednej kružnici.
  Určte množinu priesečníkov priamok $AD$ a~$BE$ pre všetky trojuholníky
  $ABC$ s~danou vlastnosťou!}
\podpis{J. Švrček}

{%%%%%   trojstretnutie, priklad 5
Určte všetky funkcie $f:\Bbb R \to \Bbb R$ vyhovujúce rovnici
$$
f(x^2+y)+f\bigl(f(x)-y\bigr)=2f\bigl(f(x)\bigr)+2y^2
$$
  pre všetky $x,y\in \Bbb R$.}
\podpis{P. Kaňovský}

{%%%%%   trojstretnutie, priklad 6
V~priestore je daná karteziánska sústava súradníc. Každý bod
s~celočíselnými súradnicami nazveme {\it mrežovým}. Ofarbime
$2\,000$ mrežových bodov na modro a~iných $2\,000$~mrežových bodov
na červeno tak, aby žiadne dve modročervené úsečky nemali spoločný
vnútorný bod. (Úsečku nazývame modročervenou, pokiaľ je jeden jej
krajný bod ofarbený na modro a~druhý na červeno.) Uvažujme najmenší
kváder s~hranami rovnobežnými s~osami súradníc, ktorý
obsahuje všetky ofarbené body.
\ite (a) Dokážte, že kváder obsahuje aspoň
         $500\,000$~mrežových bodov.
\ite (b) Uveďte príklad popísaného ofarbenia, keď uvažovaný
         kváder obsahuje maximálne $8\,000\,000$~mrežových bodov.
}
\podpis{J. Šimša}

{%%%%%   IMO, priklad 1
Nech $O$ je stred  kružnice opísanej ostrouhlému
trojuholníku $ABC$. Bod $P$ strany $BC$ je pätou
výšky z vrcholu $A$.

Predpokladajme, že $|\uhol BCA| \geq
|\uhol ABC| + 30^{\circ}$.

Dokážte, že  $|\uhol CAB| + |\uhol COP| < 90^{\circ}.$}
\podpis{Južná Kórea}

{%%%%%   IMO, priklad 2
Dokážte, že nerovnosť
$$
\frac{a}{\sqrt{a^2+8bc}} + \frac{b}{\sqrt{b^2+8ca}} + \frac{c}{\sqrt{c^2+8ab}}
\geq 1
$$
platí pre  všetky  kladné reálne čísla $a,
b, c$.}
\podpis{Južná Kórea}

{%%%%%   IMO, priklad 3
Matematickej súťaže sa zúčastnilo 21 dievčat a
21 chlapcov.
\item{$\bullet$}
Každý súťažiaci vyriešil nanajvýš
šesť úloh.
\item{$\bullet$}
Pre každé dievča a každého chlapca existuje
aspoň jedna úloha, ktorú vyriešili obaja.

\noindent
Dokážte, že existuje úloha, ktorú vyriešili
aspoň tri dievčatá a aspoň traja chlapci!}
\podpis{Nemecko}

{%%%%%   IMO, priklad 4
Nech $n$ je nepárne číslo väčšie ako 1
a nech $k_1, k_2, \dots, k_n$ sú dané celé čísla.
Pre každú z~$n!$ permutácií $a=(a_1, a_2,\dots, a_n)$
množiny $\{1, 2, \dots, n\}$ nech
$$
S(a) = \sum_{i=1}^n k_i a_i.
$$
Dokážte, že existujú dve permutácie $b$ a $c$,
$b \ne c$  také, že $n!$ je deliteľom $S(b)-S(c)$.}
\podpis{Kanada}

{%%%%%   IMO, priklad 5
V trojuholníku $ABC$ nech $AP$ rozpoľuje uhol $BAC$,
pričom $P$ leží na strane $BC$ a nech $BQ$
rozpoľuje uhol $ABC$, pričom $Q$ leží na strane
$CA$.

Je známe, že $|\uhol BAC| = 60^{\circ}$ a že
$|AB|+|BP| = |AQ|+|QB|$.

Aké sú možné veľkosti uhlov trojuholníka
$ABC$?}
\podpis{Izrael}

{%%%%%   IMO, priklad 6
Nech pre celé čísla $a, b, c, d$ platí
$a>b>c>d>0$. Predpokladajme, že
$$
ac+bd = (b+d+a-c)(b+d-a+c).
$$
Dokážte, že $ab+cd$ nie je prvočíslo!}
\podpis{Rusko}

