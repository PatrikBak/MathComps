{%%%%%   A-I-1
...}

{%%%%%   A-I-2
...}

{%%%%%   A-I-3
...}

{%%%%%   A-I-4
Po jednoduchej úprave dostávame:
$$\dsize\frac {a+1}b +\frac {b+1}a = \frac
{a^2+b^2+a+b}{ab} .\tag 1 $$
Číslo $D$ delí aj $a$ aj $b$, preto $D^2$ delí $ab$.
Označme preto $a=Da_1$ a $b=Db_1$, kde $a_1, b_1$ sú nesúdeliteľné
prirodzené čísla.
Výraz (1) má potom (po vykrátení) tvar:
$$\dsize \frac
{Da_1^2+Db_1^2+a_1+b_1}{Da_1b_1} .\tag 2 $$
Aby bol výraz (2) prirodzené číslo, musí byť čitateľ deliteľný
menovateľom, a teda aj jeho všetkými deliteľmi. Preto musí byť
čitateľ deliteľný $D$,
$$D | Da_1^2+Db_1^2+a_1+b_1 .$$
Číslo $D$ zrejme delí čísla $Da_1^2$ a $Db_1^2$, preto musí platiť
$$D | a_1 + b_1 . \tag 3 $$
Keďže čísla $a_1,b_1,D$ sú prirodzené a platí (3), musí zároveň byť
$$D \le a_1+b_1 , \tag 4$$
čo po prenásobení D ($D > 0$) dáva
$$D^2 \le a + b. $$
Po odmocnení (obe strany sú kladné):
$$D \le \sqrt{a+b} .\tag 5$$


Ešte musíme zistiť, či niekedy nastane  v (5) rovnosť.
Zrejme práve vtedy, keď nastáva v (4). Preto musí byť $D = a_1 + b_1$.
Aby bola zároveň splnená podmienka $D < a < b$, musí platiť
$1 < a_1 < b_1 $. Voľme preto $a_1 = 2$ a $b_1 = 5$. Potom musí byť $D
= 2 + 5 = 7$, čiže $a  = Da_1 = 14$ a $ b = Db_1 = 35$. Poľahky sa
presvedčíme, že v tomto prípade rov\-nosť~(5) skutočne nastáva.
}

{%%%%%   A-I-5
...}

{%%%%%   A-I-6
...}

{%%%%%   B-I-1
...}

{%%%%%   B-I-2
...}

{%%%%%   B-I-3
...}

{%%%%%   B-I-4
...}

{%%%%%   B-I-5
\fontplace
\tpoint A; \tpoint B; \bpoint C; \bpoint D;
\tpoint R; \lBpoint T; \lpoint V; \rBpoint O;
\tpoint a; \rpoint r; \tpoint r; \cpoint\ssize45\st;
[12] %%\vskip-.5\baselineskip
\hfil\Obr

Situáciu znázorňuje \obr, v~ktorom $T$ je bod
dotyku dotyčnice $BV$, $O$ je stred gule, $|AB|=|BC|=|AC|=2|AD|=a$,
$R$ kolmý priemet bodu $T$ do roviny $ABC$, $|BT|=|BA|=a$ (dotyčnice),
$|TR|=|BR|=\frac12{a\sqrt2}$.
\vskip -.5\baselineskip
\inspicture{}
\smallskip
Hľadaný polomer $r$ vypočítame
z~pravouhlého lichobežníka $ARTO$, ktorého stra\-nu~$|AR|$ vypočítame
z~pravouhlého trojuholníka $ADR$, v ktorom poznáme $|AD|=\frac
12a$, $|DR|=|BD|-|BR|=\frac12{a\sqrt3}-\frac12{a\sqrt2}$. Napokon
$|AR|^2=\frac12{a^2}(3-\sqrt 6)$,
$r=\frac12a\big({2\sqrt2-\sqrt3}\big)$.
}

{%%%%%   B-I-6
...}

{%%%%%   C-I-1
...}

{%%%%%   C-I-2
...}

{%%%%%   C-I-3
...}

{%%%%%   C-I-4
...}

{%%%%%   C-I-5
Ak je $A=\overline{pqrs}$ ciferný zápis hľadaného čísla, potom číslo
s~vpísanou
cifrou $k$ môžeme rozložiť na súčet
$$
\overline{pqkrs}=\overline{pq0rs}+k\cdot 100.
$$
Pretože druhý sčítanec je deliteľný číslom $k$, možno vlastnosť čísla
$A$ vyjadriť takto:
Päťciferné číslo $B$ so~zápisom $B=\overline{pq0rs}$ je
deliteľné každým z~čísel $2, 3, 4,~ \dots, 9$~, alebo každým
z~čísel 40, 9 a~7. Číslo $B$ je násobkom čísla štyridsať, práve keď je
násobkom štyridsiatich dvojčíslie $\overline{rs}$, t\.j\.~
$\overline{rs}\in \{00, 40, 80\}$. Rozlíšime jednotlivé prípady.

\goodbreak
\item{a)} $\overline{rs}=00$. Číslo $B=\overline{pq\,000}$ je
teda deliteľné číslami 9 a~7 jedine v~prípade $\overline{pq}=63$.
Dostávame prvé riešenie $A=6\,300$.

\item{b)} $\overline{rs}=40$. Číslo $B=\overline{pq\,040}$ je
deliteľné deviatimi, práve keď $p+q=5$ alebo $p+q=14$. V~prvom
prípade
$$
B=1\,000(10p+5-p)+40=7(1\,285p+720)+5p,
$$
čo nie je násobkom čísla sedem pre žiadnu cifru $p\leqq 5$.
V druhom prípade
$$
B=1\,000(10p+14-p)+40=7(1\,285p+2\,005)+5(p+1),
$$
čo je násobkom čísla sedem jedine pre $p=6$. Vtedy $q=8$ a~$A=6\,840$.

\item{c)} $\overline{rs}=80$. Číslo $B=\overline{pq\,080}$ je
deliteľné deviatimi, práve keď $p+q=1$ alebo $p+q=10$. V~prvom
prípade $B=10\,080$, čo je násobkom čísla sedem, takže máme riešenie
$A=1\,080$. V~druhom prípade
$$
B=1\,000(10p+10-p)+80=7(1\,285p+1\,440)+5p,
$$
čo je násobkom čísla sedem jedine pre $p=7$. Vtedy $q=3$ a~$A=7\,380$.

%%\noindent
{\it Odpoveď:} Hľadané čísla $A$ sú 1\,080, 6\,300, 6\,840 a~7\,380.
}

{%%%%%   C-I-6
...}

{%%%%%   A-S-1
...}

{%%%%%   A-S-2
...}

{%%%%%   A-S-3
...}

{%%%%%   A-II-1
Pre $n=1$ také číslo samozrejme neexistuje, pre $n=2$ je príkladom
takého čísla čís\-lo~$13$.
Pre $n=3$ a~$n=4$ také čísla neexistujú, pretože
ani žiadne z~čísel 301, 211, 121, 103, ani žiadne z~čísel 3\,001,
2\,101, 2\,011, 1\,201, 1\,111, 1\,021, 1\,003  nie je násobkom
trinástich. (Vypísali sme všetky nepárne troj- a štvorciferné čísla
s~ciferným súčtom~4.)
Pre väč\-šie~$n$ sa takéto číslo vždy dá nájsť. Stačí uvažovať
$n$-ciferné číslo
$1001+10^{n-4}\cdot 1001$. Toto číslo je evidentne deliteľné 13-timi
(lebo $1001=13\cdot 77$), pre $n\ge 5$ je aj nepárne (končí sa na
jednotku) a jeho ciferný súčet je zrejme 4.

\noindent
{\bf Iné riešenie.\ }
Najprv obdobne ako v prvom type riešenia ukážeme, že pre $n=2$ také
číslo existuje a pre $n=1,3,4$ neexistuje.
Všimnime si teraz, aké zvyšky po delení trinástimi dávajú
jednotlivé mocniny 1, 10, $10^2$, $10^3$, $10^4$, \dots:
$$\matrix
\text{číslo:\ }&1&10&10^2&10^3&10^4&10^5&10^6&10^7&10^8&\dots\\
\text{jeho zvyšok:\ }&1&10&9&12\ &3&4&1&10\ &9&\dots\endmatrix
$$
Vidíme, že zvyšok čísla $10^n$ po delení trinástimi závisí len od toho,
aký je zvyšok čís\-la~$n$ po delení šiestimi (tvrdenie sa ľahko dokáže
matematickou indukciou):
$$
\matrix
\text{$n$ je tvaru:\ }&6k&6k+1&6k+2&6k+3&6k+4&6k+5\\
\text{zvyšok čísla $10^n$:\ }&1&10&9&12&3&4\endmatrix
$$
S~pomocou tejto tabuľky nájdeme príklady $n$-ciferných čísel
s potrebnou vlastnosťou pre~každé prirodzené $n\geqq5$:
$$\matrix
\text{počet cifier\ }&\text{číslo\ }&\text{rovné\ }&
\text{dáva rovnaký zvyšok ako}\\
6k\ (k\geqq1)&100\dots01101&10^{6k-1}+1101&
4+12+9+1=26\\
6k+1\ (k\geqq1)&200\dots011&2\cdot10^{6k}+11&
2\cdot1+11=13\\
6k+2\ (k\geqq0)&100\dots03&10^{6k+1}+3&
10+3=13\\
6k+3\ (k\geqq1)&100\dots0101001&10^{6k+2}+101001&
9+4+12+1=26\\
6k+4\ (k\geqq1)&100\dots010011&10^{6k+3}+10011&
12+3+11=26\\
6k+5\ (k\geqq0)&100\dots01011&10^{6k+4}+1011&
3+12+11=26\\
\endmatrix$$


{\it Odpoveď:} $n=2, n\geqq5$.
}

{%%%%%   A-II-2
...}

{%%%%%   A-II-3
Označme $S$
stred úsečky $AB$, $k$ kružnicu opísanú troju\-hol\-ní\-ku $ABC$,
$Z$ priesečník priamky $CS$ s $k$ ($Z\nq C$)
(\obr).
Pretože $OT\perp CZ$, je $T$ stredom tetivy
$CZ$, naviac $|SC| = 3\cdot |ST|$ (ťažisko), takže $|SZ| = |ST|$.
Dvojakým vyjadrením mocnosti bodu $S$ ku~kružnici $k$
(možno použiť aj podobnosť trojuholníkov $AZS$ a~$CBS$)
máme
$$|SA|\cdot |SB| = |SZ|\cdot |SC|,$$ alebo
$|SA|^2 = \dsize\frac 13|SC|\cdot |SC|$, čiže
$|SC| = \sqrt 3|SA|$.
Bod~$C$ leží teda na kružnici $k_1 (S;\sqrt 3|SA|)$.

\vskip5cm


Naopak, označme
$A_0, B_0$ priesečníky priamky $AB$ s~kruž\-ni\-cou
$k_1$, $E, F$ prie\-seč\-ní\-ky osi úsečky $AB$ s kružnicou $k_1$,
a nech  $C$ je ľubovoľný bod kružnice $k_1$ rôzny od~$A_0, B_0, E,
F$.
Potom z mocnosti bodu $S$ ku~kružnici $k$
(opäť možno úspešne použiť aj podobnosť trojuholníkov)
vyplýva
$$|SZ| = \dsize\frac {|SA|^2}{|SC|}= \frac {|SA|}{\sqrt 3} = \frac
{|SC|}3 = |ST| ,$$
teda $T$ je stred tetivy $CZ$, a preto buď $O=T$ alebo $OT \perp CZ$.
Prípad $O=T$ môže nastať len pre $T$ z osi úsečky $AB$, t\.j\.
$CS\perp AB$, čiže $C\in \{E, F\}$, čo sme však vylúčili.
Vzniknutý trojuholník $ABC$ teda vyhovuje podmienkam úlohy.


{\it Odpoveď:}
Hľadané geometrické miesto bodov je kružnica $k_1$ $(S;\sqrt 3|SA|)$
bez~bodov $A_0, B_0, E, F$.
}

{%%%%%   A-II-4
...}

{%%%%%   A-III-1
a)
Postupujme sporom.
Nech existuje také prirodzené číslo $k$, že
$$G(k-1)=G(k)=G(k+1) = A.$$
%%Zrejme (z Lemy) $A\in \{ 0,1,\dots, k-1\}$.
Potom zo zadania platia nasledujúce vzťahy
$$\alignat{3}
A &=\,& G(k+1) &=\,& k+1 - G(G(k)) &= k+1 -G(A), \\
A &=\,&\ G(k)\ \ &=\,& k - G(G(k-1)) &= k - G(A).
\endalignat$$
Z toho ale $k+1 = G(A) + A = k$,
čo je hľadaný spor, pretože $k+1\ne k$.

Dokázali sme, že neexistuje také $k$, pre
ktoré $G(k-1) = G(k) = G(k+1)$.

b)
Po krátkom pozorovaní prvých členov postupnosti si môžeme
všimnúť, že pre~malé~$n$ je rozdiel $G(n)-G(n-1)$ buď 0
alebo 1. Toto tvrdenie (z ktorého už jednoducho vyplýva tvrdenie b))
pre každé prirodzené číslo $n$ dokážeme matematickou indukciou.

$1^\circ$\
Pre $n=1$ platí zo zadania $G(0)=0$, $G(1) = 1 - G(G(0)) = 1$,
čiže
$G(1) - G(0) = 1$, tvrdenie platí.

$2^\circ$\
Nech $G(k) - G(k-1) \in \{0,1\}$ pre každé prirodzené
$k \le n$. Odtiaľ predovšetkým vyplýva, že $0\le G(k)\le k$ pre každé
$k\le n$, pretože $G(0)=0$.
Ďalej  $G(n+1) - G(n) = 1 + G(G(n-1)) - G(G(n))$.
Ak $G(n-1) = G(n)$, potom aj $G(G(n-1))=G(G(n))$, a teda
$G(n+1) - G(n) = 1$.
V opačnom prípade $G(n) = G(n-1) + 1$, z čoho potom $G(G(n-1)) - G(G(n)) =
G(a) - G(a+1)$, kde $a = G(n-1)$ je nezáporné celé číslo
neprevyšujúce $n-1$, preto môžeme opäť použiť
indukčný predpoklad, t\.j\.
$G(a) - G(a+1) \in \{\m 1, 0\}$. To znamená, že
$G(n+1) - G(n) = 1 + G(a) - G(a+1) \in \{ 0, 1\}$.
Tým sme dôkaz indukciou a zároveň dôkaz tvrdenia zo zadania ukončili.
}

{%%%%%   A-III-2
...}

{%%%%%   A-III-3
...}

{%%%%%   A-III-4
...}

{%%%%%   A-III-5
...}

{%%%%%   A-III-6
...}

{%%%%%   B-S-1
...}

{%%%%%   B-S-2
...}

{%%%%%   B-S-3
...}

{%%%%%   B-II-1
...}

{%%%%%   B-II-2
...}

{%%%%%   B-II-3
Pripusťme, že by pre niektoré číslo~ $r$ mala daná rovnica dva
celočíselné korene $a$, $b$.
Tretí koreň~ $c$ musí byť tiež
ale celý, pretože $a+b+c=1\,996$. Všetky tri čísla $a$, $b$, $c$
nemôžu byť nepárne, keď je ich súčet párny. Ich
súčin je však nepárne číslo~ 1\,995, čo nie je možné. Rovnica môže
mať teda nanajvýš jeden celočíselný koreň.
}

{%%%%%   B-II-4
...}

{%%%%%   C-S-1
...}

{%%%%%   C-S-2
...}

{%%%%%   C-S-3
...}

{%%%%%   C-II-1
...}

{%%%%%   C-II-2
...}

{%%%%%   C-II-3
...}

{%%%%%   C-II-4
...}

{%%%%%   vyberko, den 1, priklad 1
...}

{%%%%%   vyberko, den 1, priklad 2
...}

{%%%%%   vyberko, den 1, priklad 3
...}

{%%%%%   vyberko, den 1, priklad 4
...}

{%%%%%   vyberko, den 2, priklad 1
...}

{%%%%%   vyberko, den 2, priklad 2
...}

{%%%%%   vyberko, den 2, priklad 3
...}

{%%%%%   vyberko, den 2, priklad 4
...}

{%%%%%   vyberko, den 3, priklad 1
...}

{%%%%%   vyberko, den 3, priklad 2
...}

{%%%%%   vyberko, den 3, priklad 3
...}

{%%%%%   vyberko, den 3, priklad 4
...}

{%%%%%   vyberko, den 4, priklad 1
...}

{%%%%%   vyberko, den 4, priklad 2
...}

{%%%%%   vyberko, den 4, priklad 3
...}

{%%%%%   vyberko, den 4, priklad 4
...}

{%%%%%   vyberko, den 5, priklad 1
...}

{%%%%%   vyberko, den 5, priklad 2
...}

{%%%%%   vyberko, den 5, priklad 3
...}

{%%%%%   vyberko, den 5, priklad 4
...}

{%%%%%   trojstretnutie, priklad 1
...}

{%%%%%   trojstretnutie, priklad 2
...}

{%%%%%   trojstretnutie, priklad 3
...}

{%%%%%   trojstretnutie, priklad 4
...}

{%%%%%   trojstretnutie, priklad 5
...}

{%%%%%   trojstretnutie, priklad 6
...}

{%%%%%   IMO, priklad 1
...}

{%%%%%   IMO, priklad 2
...}

{%%%%%   IMO, priklad 3
...}

{%%%%%   IMO, priklad 4
...}

{%%%%%   IMO, priklad 5
...}

{%%%%%   IMO, priklad 6
...}

{%%%%%   MEMO, priklad 1
...}

{%%%%%   MEMO, priklad 2
...}

{%%%%%   MEMO, priklad 3
...}

{%%%%%   MEMO, priklad 4
...}

{%%%%%   MEMO, priklad t1
...}

{%%%%%   MEMO, priklad t2
...}

{%%%%%   MEMO, priklad t3
...}

{%%%%%   MEMO, priklad t4
...}

{%%%%%   MEMO, priklad t5
...}

{%%%%%   MEMO, priklad t6
...}

{%%%%%   MEMO, priklad t7
...}

{%%%%%   MEMO, priklad t8
...} 