{%%%%%   A-I-1
Stĺpce šachovnice $8\times 8$ označme zľava doprava
číslami $1, 2,\dots, 8$, riadky označme rovnakými číslami zdola nahor.
Do každého políčka zapíšeme súčet čísel príslušného
riadku a~stĺpca. Vyberieme 8~ políčok tak, aby žiadne dve z~nich neboli
ani v rovnakom riadku, ani v rovnakom stĺpci. Aký je
\ite a) najväčší možný súčet,
\ite b) najväčší možný súčin,
\ite c) najmenší možný súčet druhých mocnín
\endgraf\noindent
čísel na vybraných poliach?}
\podpis{J. Zhouf}

{%%%%%   A-I-2
Na stranách $AB$, $BC$ a~$CA$ daného trojuholníka $ABC$ sú
zvolené po rade body $K$, $L$ a~$M$ tak, že platí
$$
0<\frac{|AK|}{|AB|}=\frac{|BL|}{|BC|}=\frac{|CM|}{|CA|}<1.
$$
Dokážte, že ak je trojuholník $KLM$ rovnostranný, potom je
rovnostranný aj trojuholník $ABC$.}
\podpis{J. Šimša}

{%%%%%   A-I-3
Postupnosť prirodzených čísel $a_1, a_2, a_3, \dots$ spĺňa
pre každé prirodzené $n\ge1$ tri rovnosti:
$$
\align
a_{n}+a_{2n}&=a_{3n},\\
a_{n}+a_{3n-1}&=a_{2n}+a_{2n-1},\\
a_{n}+a_{3n+1}&=a_{2n}+a_{2n+1} .
\endalign
$$
Pritom vieme, že všetky štyri členy $a_1$, $a_{14}$, $a_{17}$
a~$a_{21}$ sú prvočísla. Dokážte rovnosť
$a_{1995}=a_{2000}$.}
\podpis{J. Šimša}

{%%%%%   A-I-4
Dokážte, že ak pre prirodzené čísla $a$, $b$ je aj číslo
$$
\frac {a+1}b+\frac {b+1}a
$$
prirodzené, potom pre najväčší spoločný deliteľ $D$ čísel $a$, $b$
platí nerovnosť $D\le\sqrt{a+b}$. Môže nastať rovnosť v~prípade,
že $D<a<b$?}
\podpis{M. Niepel}

{%%%%%   A-I-5
Nájdite všetky funkcie $f\colon{\Bbb N}\to{\Bbb Z}$
spĺňajúce pre každé $x,y\in{\Bbb N}$ rovnosť
$$
f(xy)=f(x)+f(y)-f(D(x,y)),
$$
kde $D(x,y)$ je najväčší spoločný deliteľ čísel $x$, $y$,
ak viete, že platí $f(p)=p$ pre každé
prvočíslo~$p$.}
\podpis{P. Hliněný}

{%%%%%   A-I-6
V~priestore je daný trojuholník $ABC$ so~stranami
$|AB|=|AC|=10\cm$ a~$|BC|=12\cm$. Nájdite množinu všetkých
bodov~$D$, pre ktoré spojnica stredu~$O$ gule opísanej štvorstenu
$ABCD$ s~ťažiskom~$T$ tohto štvorstena je priamka kolmá na rovinu
$ADT$.}
\podpis{P. Leischner}

{%%%%%   B-I-1
Zistite, pre ktoré reálne čísla~$p$ má rovnica
$$
x^3+px^2+2px=3p+1
$$
tri rôzne reálne korene $x_1$, $x_2$ a~$x_3$ také, že
$x_1x_2=x_3^2$.}
\podpis{J. Šimša}

{%%%%%   B-I-2
V~rovine je daný trojuholník $ABC$, v~ktorom $|\uhol BAC|=105\st$,
$|\uhol ABC|=55\st$ a~$|AB|=6\cm$. Na
strane~$BC$ zostrojte body $X$, $Y$ ($|BX|<|BY|$) a~na strane~$AC$ body $M$, $N$ ($|AM|<|AN|$) tak, aby štvoruholníky $ABXM$
a~$MXYN$ boli tetivové a~im opísané kružnice  mali rovnaký polomer
ako kružnica opísaná trojuholníku $NYC$.}
\podpis{P. Černek}

{%%%%%   B-I-3
Ak zvolíme ľubovoľne 11 rôznych dvojciferných čísel, vždy z~nich možno
vybrať dve skupiny čísel, ktoré majú rovnaký počet prvkov, neobsahujú
žiaden spoločný prvok a~dávajú rovnaký súčet. Dokážte.}
\podpis{A. Vrba}

{%%%%%   B-I-4
Číslo $2n^4+n^3+50$ je deliteľné šiestimi práve  pre tie prirodzené čísla
$n$, pre ktoré je číslo $2\cdot 4^n+3^n+50$ deliteľné trinástimi.
Dokážte.}
\podpis{J. Šimša}

{%%%%%   B-I-5
Daný je trojboký ihlan $ABCV$, ktorého podstavou je rovnostranný
trojuholník $ABC$ s~dĺžkou strany $a$. Priamky $AV$, $BV$ a~$CV$
majú od roviny podstavy rovnakú odchýlku~$45^\circ$. Určte
polomer gule, ktorá sa dotýka roviny $ABC$ v~bode $A$, a~aj
priamky~$VB$. (Odchýlkou priamky od roviny rozumieme uhol, ktorý
priamka zviera so svojím kolmým priemetom do tejto roviny.)}
\podpis{R. Kollár}

{%%%%%   B-I-6
Umiestnite v~rovine 7~navzájom rôznych bodov a~7~navzájom rôznych priamok
tak, aby každými dvoma z~týchto bodov prechádzala jedna z~týchto priamok
a~aby sa každé dve z~týchto priamok pretínali v~jednom z~týchto bodov.
Preveďte diskusiu.}
\podpis{P. Hliněný}

{%%%%%   C-I-1
V~rovnostrannom trojuholníku $ABC$ so stranou dĺžky~$a$
označme $K$, $L$, $M$ po rade stredy strán $AB$, $BC$, $CA$.
Vnútri alebo na obvode trojuholníka $ABC$ je zvolený bod~$S$.
Dokážte, že platí rovnosť
$$
|AS|^2+|BS|^2+|C\!S|^2=|KS|^2+|LS|^2+|MS|^2+\frac34a^2.
$$
}
\podpis{J. Zhouf}

{%%%%%   C-I-2
Rozhodnite, či možno množinu čísel $\{1, 2,~ \dots, 1995\}$
rozdeliť na dve skupiny tak, aby v~prvej skupine bolo
\ite a) dvakrát
\ite b) trikrát
\ite c) štyrikrát

viac čísel ako v~druhej skupine, a~aby súčty čísel v~oboch
skupinách boli rovnaké.}
\podpis{J. Zhouf}

{%%%%%   C-I-3
Zostrojte lichobežník $ABCD$  ($AB\parallel CD$)
s~pravým uhlom pri vrchole~$A$, ak $|AC|=5\cm$, $|BD|=7\cm$
a~uhlopriečka~$AC$ delí obsah $ABCD$ na dve časti v~pomere
$2:1$.}
\podpis{J. Švrček}

{%%%%%   C-I-4
Určte všetky dvojice $x$, $y$ prirodzených čísel, pre ktoré
súčasne platí:
\ite a) $2\,100<xy<2\,500\,$;
\ite b) $0{,}85<\dfrac xy<0{,}9\,$;
\ite c) podiel $\dfrac{y+x}{y-x}$ je celočíselný.
}
\podpis{J. Zhouf}

{%%%%%   C-I-5
Určte všetky štvorciferné čísla~$A$, ktoré majú pre
každé $k=2, 3, 4, \dots, 9$ túto vlastnosť:
Ak vpíšeme cifru~$k$ medzi prostredné cifry čísla~$A$, dostaneme
päťciferné číslo, ktoré je násobkom čísla~$k$.}
\podpis{J. Šimša}

{%%%%%   C-I-6
Určte dĺžku prepony pravouhlého trojuholníka, ak poznáte
polomer~$r$ kružnice vpísanej a~polomer~$R$ kružnice pripísanej
k~prepone tohto trojuholníka (\tj. kružnice, ktorá sa dotýka
zvonku prepony a~predĺženia oboch odvesien
trojuholníka).}
\podpis{P. Leischner}

{%%%%%   A-S-1
Nájdite všetky také dvojice celých čísel $a$, $b$, pre ktoré sú obe
čísla
$$
\frac {a+1}b + \frac {b+1}a  ,\qquad
\frac {a^2}b + \frac {b^2}a
$$
celé.}
\podpis{R. Kollár}

{%%%%%   A-S-2
Nájdite najväčšie reálne číslo~$q$, pre ktoré je nerovnosť
$2^n\ge 1+nq^n$ splnená pre každé prirodzené číslo $n\ge 2$.}
\podpis{J. Šimša}

{%%%%%   A-S-3
Popíšte konštrukciu rovnoramenného trojuholníka $ABC$ so základňou
$AB$, v~ktorom $|OA| = 9\cm$ a~$|OB| = 3\cm$, kde $O$ je stred
kružnice pripísanej k~strane~$BC$ trojuholníka $ABC$ (\tj. kružnice,
ktorá sa dotýka zvonku strany~$BC$ a~predĺžení strán $AB$ a~$AC$).}
\podpis{A. Vrba}

{%%%%%   A-II-1
Určte, pre ktoré prirodzené čísla~$n$ existuje
nepárne $n$-ciferné číslo, ktoré je deliteľné trinástimi a~má
ciferný súčet rovný štyrom.}
\podpis{J. Šimša}

{%%%%%   A-II-2
Dané sú dve kružnice $k_1(S_1,r_1)$ a~$k_2(S_2,r_2)$, $r_1<r_2$,
ktoré sa zvonku dotýkajú v~bode~$F$.
Nech $t$ je ich spoločná vonkajšia dotyčnica,
jej body dotyku s~kružnicami $k_1$, k$_2$ označme po rade $A$, $B$.
Veďme teraz inú dotyčnicu ku kružnici~$k_1$ rovnobežnú s~priamkou~$t$.
Jej dotykový bod s~kružnicou~$k_1$ označme~$C$ a~priesečníky s~kružnicou~$k_2$ označme
$D$ a~$E$.
Dokážte, že bod~$F$ a~stredy kružníc opísaných trojuholníkom
$ABC$ a~$ADE$ ležia na jednej priamke.}
\podpis{M. Niepel}

{%%%%%   A-II-3
V~rovine je daná úsečka~$AB$. Nájdite všetky body~$C$ tejto roviny
také, že pre stred~$O$ kružnice opísanej trojuholníku $ABC$ a~jeho
ťažisko~$T$ platí:
$O\neq T$, $OT \perp CT$.}
\podpis{M. Engliš}

{%%%%%   A-II-4
Deti sa v~tábore delili do družín
nasledujúcim spôsobom:
Vedúci určil spomedzi detí niekoľko náčelníkov.
Každý náčelník si do svojej družiny
vzal všetkých svojich kamarátov z~tábora (kamarátstvo je vzájomné).
Napočudovanie to vyšlo {\it dobre}, teda tak, že sa náčelníci
nemuseli o~žiadne dieťa hádať, žiadne dieťa nezvýšilo a~žiadni dvaja
náčelníci neboli kamaráti.
Druhýkrát vedúci určil iný počet náčelníkov.
Mohlo rozdelenie detí rovnakým spôsobom opäť dopadnúť {\it dobre}?}
\podpis{P. Hliněný}

{%%%%%   A-III-1
Pre postupnosť $G(n)$ ($n=0,1,2,\dots$) celých čísel platí:
$$
\align
G(0) & = 0 \\
G(n) & = n - G(G(n-1)) \quad (n=1,2,3,\dots)
\endalign
$$
Potom dokážte, že
\ite a) neexistuje prirodzené číslo $k$ také, že $G(k-1) = G(k) = G(k+1)$,
\ite b) pre každé prirodzené číslo $k$ platí $G(k)\ge G(k-1)$.
}
\podpis{M. Engliš}

{%%%%%   A-III-2
V~priestore je daný ostrouhlý
trojuholník $ABC$ s~výškami $AP$, $BQ$ a~$CR$. Dokážte, že
pre každý vnútorný bod~$V$ trojuholníka $PQR$ existuje štvorsten $ABCD$
taký, že bod~$V$ má zo všetkých bodov steny $ABC$ najväčšiu
vzdialenosť (po povrchu štvorstena) od~bodu~$D$.}
\podpis{P. Černek, J. Šimša}

{%%%%%   A-III-3
Daných je  šesť trojprvkových podmnožín konečnej množiny~$X$.
Dokážte, že potom prvky množiny~$X$ možno ofarbiť dvoma farbami tak,
aby žiadna zo šiestich daných podmnožín nebola
jednofarebná, \tj. nemala všetky tri prvky rovnakej
farby.}
\podpis{P. Hliněný}

{%%%%%   A-III-4
Daný je ostrý uhol $XCY$ a na jeho ramenách $CX$, $CY$ po rade body $A$, $B$
tak, že $|CX|< |CA| = |CB| < |CY|$. Popíšte konštrukciu priamky, ktorá
pretína rameno $CX$ a~úsečky $AB$, $BC$ po rade v~bodoch $K$, $L$ a~$M$
tak, že platí
$$
|KA|\cdot|YB| = |XA|\cdot |MB| = |LA|\cdot |LB| \ne 0 .
$$
}
\podpis{P. Černek}

{%%%%%   A-III-5
Pre ktoré celé čísla~$k$ existuje funkcia  $f\colon {\Bbb N} \to {\Bbb Z}$
spĺňajúca
\ite a) $f(1995)=1996$,
\ite b) $f(x\cdot y)=f(x)+f(y)+k\cdot f(D(x,y))$ pre každú dvojicu
prirodzených čísel $x$ a~$y$?

\noindent
($D(x,y)$ označuje najväčší spoločný deliteľ čísel $x$ a $y$.)}
\podpis{P. Hliněný}

{%%%%%   A-III-6
Na stranách $AB$, $BC$ a~$CA$ daného trojuholníka $ABC$ sú
dané po rade body $K$, $L$ a~$M$ tak, že platí
$$
\frac{|AK|}{|AB|}=\frac{|BL|}{|BC|}=\frac{|C\!M|}{|C\!A|}=\frac13.
$$
Ak sú kružnice opísané trojuholníkom $AKM$, $BLK$ a~$CML$ zhodné,
potom aj kružnice vpísané týmto trom trojuholníkom sú zhodné.
Dokážte.}
\podpis{J. Šimša}

{%%%%%   B-S-1
Nájdite všetky prirodzené čísla~$n$, pre ktoré je číslo
$1\overbrace{99\dots9}^{n}6$  deliteľné trinástimi.}
\podpis{A. Vrba, J. Šimša}

{%%%%%   B-S-2
Do kružnice je vpísaný štvorec $ABCD$. Ľubovoľným bodom~$M$
uhlopriečky~$AC$ je vedená tetiva~$KL$ rovnobežná so stranou~$AB$.
Dokážte, že $|KM|^2+|ML|^2=|AB|^2$.}
\podpis{P. Leischner}

{%%%%%   B-S-3
Nájdite 1996 navzájom rôznych celých čísel $a_1,a_2,\dots,a_{1996}$ tak, aby medzi súčtami všetkých dvojíc
$a_i+a_j$ ($1\le i<j\le1\,996$) bolo
\ite a) čo najviac rôznych čísel,
\ite b) čo najmenej rôznych čísel.
}
\podpis{R. Kollár}

{%%%%%   B-II-1
Nájdite všetky prirodzené čísla~$n$, pre ktoré je číslo
$5^n-3^n+2$ deliteľné siedmimi.}
\podpis{A. Vrba, J. Šimša}

{%%%%%   B-II-2
Body dotyku dotyčníc vedených z~bodu~$V$ ku kružnici~$k$
označme $A$, $B$. Zostrojte  sečnicu kružnice~$k$ tak, aby
prechádzala bodom~$V$ a~kružnicu~$k$ pretínala v~bodoch $C$ a~$D$,
kde $|AC|=|BD|$.}
\podpis{J. Švrček}

{%%%%%   B-II-3
Dokážte, že rovnica $x^3-1\,996x^2+rx+1\,995=0$ má pre každý
reálny koeficient~$r$ nanajvýš jeden celočíselný koreň.}
\podpis{A. Vrba}

{%%%%%   B-II-4
Daný je  trojboký ihlan $ABCV$ s~podstavou $ABC$
($|AB|=8\cm$, $|AC|=|BC|=5\cm$), ktorého bočné steny
majú od roviny podstavy odchýlku $45\st$ a~päta~$P$ jeho výšky
spustenej z~vrcholu~$V$ leží vnútri podstavy. Vypočítajte veľkosť
výšky~$VP$.}
\podpis{P. Leischner}

{%%%%%   C-S-1
Rozložte všetkými možnými spôsobmi číslo $1996$ na súčet
niekoľkých (aspoň dvoch) po sebe idúcich prirodzených čísel.}
\podpis{J. Zhouf}

{%%%%%   C-S-2
Vo vnútri rovnostranného trojuholníka $ABC$ je daný bod~$D$, ktorým
sú postupne vedené rovnobežky $KL$, $MN$, $PQ$ so stranami
$AB$, $BC$, $CA$ ako na \obr. Bod~$D$ je zvolený tak, že
vzniknutý šesťuholník $QMLPNK$ má pravé uhly pri vrcholoch $M$
a~$N$. Určte pomer obsahov šesťuholníka $QMLPNK$ a~trojuholníka
$ABC$.
\insp{c45.1}%
}
\podpis{J. Zhouf}

{%%%%%   C-S-3
Určte všetky päťciferné čísla~$A$ s~nasledujúcou vlastnosťou:
ak zapíšeme za sebou (zľava doprava) zvyšky, ktoré dáva číslo~$A$ po delení
číslami $2$, $3$, $4$, $5$ a~$6$, dostaneme opäť pôvodné číslo~$A$.}
\podpis{J. Šimša}

{%%%%%   C-II-1
Zistite, pre ktoré prirodzené čísla~$n$ možno
rozdeliť množinu čísel $\{1, 2, \dots, n\}$ na dve skupiny tak, aby
v~prvej skupine bolo trikrát viac čísel ako v~druhej, a~aby
súčty čísel v~oboch skupinách boli rovnaké.}
\podpis{R. Kollár}

{%%%%%   C-II-2
Určte všetky body~$S$ daného štvorca $ABCD$, pre ktoré
platí
$$
|SA|\cdot|SC|=|SB|\cdot|SD|.
$$
}
\podpis{J. Zhouf}

{%%%%%   C-II-3
Nájdite najmenšie päťciferné číslo $\overline{abcde}$,
ktorého všetky číslice sú nenulové, a~ktoré je deliteľné každým
z~čísel  $\overline{e}$, $\overline{de}$, $\overline{cde}$,
$\overline{bcde}$, $\overline{abcde}$.}
\podpis{J. Zhouf}

{%%%%%   C-II-4
V~rovine sú dané 3 rôzne body $A$, $B$ a~$M$, ktoré neležia na jednej
priamke. V~polrovine $ABM$ zostrojte kružnice $k_1$
a~$k_2$, ktoré sa dotýkajú priamky~$AB$ po rade v~bodoch $A$ a~$B$,
dotýkajú sa zvonku v~nejakom bode~$T$ a~ich spoločná dotyčnica v~tomto bode
prechádza bodom~$M$.}
\podpis{P. Leischner}

{%%%%%   vyberko, den 1, priklad 1
...}
\podpis{...}

{%%%%%   vyberko, den 1, priklad 2
...}
\podpis{...}

{%%%%%   vyberko, den 1, priklad 3
...}
\podpis{...}

{%%%%%   vyberko, den 1, priklad 4
...}
\podpis{...}

{%%%%%   vyberko, den 2, priklad 1
...}
\podpis{...}

{%%%%%   vyberko, den 2, priklad 2
...}
\podpis{...}

{%%%%%   vyberko, den 2, priklad 3
...}
\podpis{...}

{%%%%%   vyberko, den 2, priklad 4
...}
\podpis{...}

{%%%%%   vyberko, den 3, priklad 1
...}
\podpis{...}

{%%%%%   vyberko, den 3, priklad 2
...}
\podpis{...}

{%%%%%   vyberko, den 3, priklad 3
...}
\podpis{...}

{%%%%%   vyberko, den 3, priklad 4
...}
\podpis{...}

{%%%%%   vyberko, den 4, priklad 1
...}
\podpis{...}

{%%%%%   vyberko, den 4, priklad 2
...}
\podpis{...}

{%%%%%   vyberko, den 4, priklad 3
...}
\podpis{...}

{%%%%%   vyberko, den 4, priklad 4
...}
\podpis{...}

{%%%%%   vyberko, den 5, priklad 1
...}
\podpis{...}

{%%%%%   vyberko, den 5, priklad 2
...}
\podpis{...}

{%%%%%   vyberko, den 5, priklad 3
...}
\podpis{...}

{%%%%%   vyberko, den 5, priklad 4
...}
\podpis{...}

{%%%%%   trojstretnutie, priklad 1
...}
\podpis{...}

{%%%%%   trojstretnutie, priklad 2
...}
\podpis{...}

{%%%%%   trojstretnutie, priklad 3
...}
\podpis{...}

{%%%%%   trojstretnutie, priklad 4
...}
\podpis{...}

{%%%%%   trojstretnutie, priklad 5
...}
\podpis{...}

{%%%%%   trojstretnutie, priklad 6
...}
\podpis{...}

{%%%%%   IMO, priklad 1
...}
\podpis{...}

{%%%%%   IMO, priklad 2
...}
\podpis{...}

{%%%%%   IMO, priklad 3
...}
\podpis{...}

{%%%%%   IMO, priklad 4
...}
\podpis{...}

{%%%%%   IMO, priklad 5
...}
\podpis{...}

{%%%%%   IMO, priklad 6
...}
\podpis{...}

{%%%%%   MEMO, priklad 1
}
\podpis{}

{%%%%%   MEMO, priklad 2
}
\podpis{}

{%%%%%   MEMO, priklad 3
}
\podpis{}

{%%%%%   MEMO, priklad 4
}
\podpis{}

{%%%%%   MEMO, priklad t1
}
\podpis{}

{%%%%%   MEMO, priklad t2
}
\podpis{}

{%%%%%   MEMO, priklad t3
}
\podpis{}

{%%%%%   MEMO, priklad t4
}
\podpis{}

{%%%%%   MEMO, priklad t5
}
\podpis{}

{%%%%%   MEMO, priklad t6
}
\podpis{}

{%%%%%   MEMO, priklad t7
}
\podpis{}

{%%%%%   MEMO, priklad t8
}
\podpis{}
