{%%%%%   Z4-I-1
Doplňte čísla do súčinovej pyramídy na \obr. Každé číslo v~tehličkách (od druhého riadku) je rovné súčinu čísel v~tehličkách bezprostredne nad ním. V~prvom riadku sú iba jednociferné čísla.
\insp{z59i.41}%
}
\podpis{M. Kollár}

{%%%%%   Z4-I-2
Janka delila lentilky sebe a~sestre Danke. Sebe dala 11 lentiliek, Danke~2. Potom sebe 10 a~Danke~4. Potom sebe 9 a~Danke~6...týmto spôsobom sebe dala vždy o~1 menej a~Danke o~2 viac ako predtým. V~delení pokračovala, až sa jej lentilky minuli. Na svoje i~Dankino počudovanie dievčatá zistili, že majú lentiliek rovnako. Koľko bolo všetkých lentiliek?}
\podpis{M. Dillingerová}

{%%%%%   Z4-I-3
Doplň čísla od $1$ do $12$ (každé práve raz) do tabuľky. V~jednotlivých riadkoch sú príklady na sčítanie, odčítanie, násobenie a~delenie a~tie musia byť všetky splnené. Navyše v~každom riadku sú dopĺňané čísla usporiadané zľava doprava od najväčšieho po najmenšie.
$$
\begin{array}{|c|c|c|c|c|c|c|}
\hline
\hspace{2em} & + & \hspace{2em} & + & \hspace{2em} & = & 33 \\
\hline
\hspace{2em} & - & \hspace{2em} & - & \hspace{2em} & = & 0 \\
\hline
\hspace{2em} & \cdot & \hspace{2em} & \cdot & \hspace{2em} & = & 35 \\
\hline
\hspace{2em} & : & \hspace{2em} & : & \hspace{2em} & = & 1 \\
\hline
\end{array}
$$
}
\podpis{M. Kollár}

{%%%%%   Z4-I-4
Jurko chodí domov zo školy okolo školského latkového plota. V~pondelok sa rozhodol, že cestou zo školy bude na každú druhú latku robiť značku bielou kriedou. V~utorok, stredu i~štvrtok to zopakoval s~každou druhou ešte neoznačenou latkou. V~piatok ráno zistil, že mu ostalo už iba 7~latiek neoznačených. Koľko najmenej a~koľko najviac latiek mohol mať plot školy?}
\podpis{M. Dillingerová}

{%%%%%   Z4-I-5
Samo má tri staršie sestry. Každá robí niečo iné a~každá chová jedno zvieratko. Ľudka hrá tenis. Najmladšia je klaviristka. Ivana neštuduje jazyky. Mária nie je prostredná. Na klavíri sedáva veľká čierna mačka. Prostredná má papagája. Tretím zvieraťom je pes.
Vypíš, ako sa volá najstaršia sestra, čo robí a~aké zviera chová.}
\podpis{M. Dillingerová}

{%%%%%   Z4-I-6
Martin má vystrihnutý z~papiera jeden obdĺžnik s~rozmermi $2\cm$ a~$6\cm$. Okrem toho má ešte jeden obdĺžnik a~jeden štvorec. Zo všetkých troch útvarov vie bez prekrývania a~dier zložiť jeden veľký štvorec. Aké rozmery môžu mať jeho útvary? Nájdi dve riešenia.}
\podpis{M. Dillingerová}

{%%%%%   Z5-I-1
Húsenica Leona spadla doprostred štvorcovej siete. Rozhodla sa, že polezie "do špirály" tak, ako je naznačené na \obr{}; na žiadnom štvorčeku nebude dvakrát a~žiaden štvorček nevynechá.
\insp{z59.1}%

Z~prvého štvorčeka na druhý liezla smerom na východ, z~druhého na tretí smerom na sever, z~tretieho na štvrtý smerom na západ, zo štvrtého na piaty tiež na západ, z~piateho na šiesty na juh...Ktorým smerom liezla z~81. na 82. štvorček?
}
\podpis{M. Petrová}

{%%%%%   Z5-I-2
Miša si z~papiera vystrihla dva rovnaké štvorce, jeden obdĺžnik s~rozmermi $10\cm\times24\cm$ a~ešte jeden obdĺžnik. Aké rozmery mohol mať tento obdĺžnik, ak sa zo všetkých štyroch útvarov dal zložiť štvorec bez toho, aby sa jednotlivé diely prekrývali? Takých obdĺžnikov sa dá nájsť niekoľko, uveď aspoň štyri.}
\podpis{L. Šimůnek}

{%%%%%   Z5-I-3
Vyrieš nasledujúci algebrogram a~nájdi všetky riešenia. Rovnaké písmená nahraď rovnakými číslicami, rôzne rôznymi.
$$
\begin{array}{ccccc}
 & O & S & E & L \\
 & & S & E & L \\
 & & & E & L \\
 & & & & L \\
\hline
1 & 0 & 0 & 3 & 4 \\
\end{array}
$$}
\podpis{M. Volfová}

{%%%%%   Z5-I-4
Nina dostala od pani učiteľky nasledujúce kartičky:
\insp{z59.2}%

Má z~nich zostaviť príklad pre svojich spolužiakov, pričom každú kartičku použije práve raz. Pomôž Nine a~zostav jeden taký príklad tak, aby každé delenie vyšlo bezo zvyšku. Aký bude výsledok?
}
\podpis{M. Petrová}

{%%%%%   Z5-I-5
Našich 84 žiakov išlo do kina. Lístok síce stál 2€, ale každý 12. žiak mal polovičnú zľavu a~každý 35. vstup zdarma. Koľko stálo vstupné pre všetkých žiakov?}
\podpis{M. Volfová}

{%%%%%   Z5-I-6
Chlapci našli starý plán mínového poľa (tabuľka). Čísla sú na políčkach, kde žiadne míny nie sú, a~udávajú počet zamínovaných susedných políčok. Urči, koľko je v~poli mín spolu a~kde sú. (Políčka susedia práve vtedy, keď majú spoločný vrchol alebo stranu.)
$$
\begin{array}{|c|c|c|c|c|c|}
\hline
 1 &  & 2 &  & 2 \\
\hline
  & 3 &  & 3 &  \\
\hline
  3 &  &  &  & 3 \\
\hline
   & 2 &  &  &  \\
\hline
   &  &  & 2 &  \\
\hline
\end{array}
$$}
\podpis{M. Volfová}

{%%%%%   Z6-I-1
Janko s~Marienkou chodia k~babičke, ktorá má cukráreň a~predáva perníky. Obe deti jej samozrejme pomáhajú, hlavne so zdobením. Kým babička ozdobí päť perníkov, ozdobí Marienka tri a~Janko dva. Pri poslednej návšteve ozdobili všetci traja spolu päť plných tácok. Marienka s~babičkou zdobili po celý čas, Janko najprv zdobil a~potom zrovnával perníky po dvanásť na jednu tácku a~odnášal ich do komory. Všetci traja v~ten istý okamih začali i~skončili.
\itemitem{1.} Koľko perníkov ozdobil Janko?
\itemitem{2.} Ako dlho im trvala celá práca, ak babička ozdobí jeden perník za 4~minúty?
\itemitem{3.} Ako dlho pomáhal Janko zdobiť?
}
\podpis{M. Petrová}

{%%%%%   Z6-I-2
Štvormiestny PIN kód Rasťovho mobilu je zaujímavý:
\itemitem{$\bullet$} jednotlivé číslice sú prvočísla,
\itemitem{$\bullet$} 1. a 2. číslica v~tomto poradí vytvorí prvočíslo,
\itemitem{$\bullet$} 2. a 3. číslica v~tomto poradí vytvorí prvočíslo,
\itemitem{$\bullet$} 3. a 4. číslica v~tomto poradí vytvorí prvočíslo.

\noindent
Rasťo zabudol svoj PIN kód, ale pamätá si všetky vyššie uvedené vlastnosti. Snaží sa zaktivovať vypnutý mobil. Ktoré čísla by mal vyskúšať?
}
\podpis{M. Petrová}

{%%%%%   Z6-I-3
Na \obr{} je útvar zložený zo siedmich rovnakých štvoruholníkových dielikov stavebnice. Aký je obvod tohto útvaru, ak obvod jedného štvoruholníkového dielika je $17\cm$?
\insp{z59.3}%
}
\podpis{K. Pazourek}

{%%%%%   Z6-I-4
Ocko sa rozhodol, že bude dávať svojmu synovi Mojmírovi raz za mesiac vreckové. Prvé vreckové dostal Mojmír v~januári. Ocko každý mesiac vreckové zvyšoval vždy o~4€. Keby Mojmír neutrácal, mal by po dvanástom vreckovom pred Vianocami 900€. Koľko eur dostal Mojmír v~januári ako prvé vreckové?}
\podpis{L. Hozová}

{%%%%%   Z6-I-5
Doplň miesto hviezdičiek čísla tak, aby súčet výsledkov nasledujúcich dvoch  príkladov bol $5\,842$.
$$
\begin{array}{rrrr}
 & * & 2 & * & 7 \\
 & 3 & * & 4 & * \\
\hline
 & 4 & * & 0 & 0 \\
\end{array}
\qquad
\begin{array}{rrrr}
 & 2 & * & 9 & * \\
- & * & 2 & * & 4 \\
\hline
 & * & 5 & 4 & * \\
\end{array}
$$
}
\podpis{M. Dillingerová}

{%%%%%   Z6-I-6
Na školskú olympiádu vytvorili žiaci stupne víťazov z~drevených kociek, pozri \obr. Koľko kociek použili celkom?
Zostavené stupne natreli po celom povrchu (okrem podstavy) na bielo a~po vyhlásení výsledkov svoj výtvor rozobrali. Koľko kociek malo 6, koľko 5, 4, 3, 2, 1 či žiadnu stenu bielu?
\insp{z59.4}%
}
\podpis{M. Dillingerová, M. Volfová}

{%%%%%   Z7-I-1
Do predajne vína sa v~noci dostal kocúr. Vyskočil na policu, na ktorej boli v~dlhom rade vyrovnané fľaše s~vínom -- prvá tretina fliaš skraja stála po 8 €, nasledujúca tretina fliaš stála po 6{,}5 € a~posledná tretina po 5 €. Najprv kocúr zhodil na zem fľašu za 8 €, ktorá stála úplne na začiatku radu, a~potom postupoval ďalej a~zhadzoval bez vynechania jednu fľašu za druhou. Než ho to prestalo baviť, zhodil 25 fliaš a~tie sa všetky rozbili. Ráno majiteľ ľutoval, že kocúr nezačal so zhadzovaním na druhom okraji police. Aj keby totiž rozbil rovnako veľa fliaš, bola by škoda o~33 € nižšia. Koľko fliaš bolo pôvodne na polici?
}
\podpis{L. Šimůnek}

{%%%%%   Z7-I-2
Na tabuli sú napísané tri prirodzené čísla $a$, $b$, $c$, pre ktoré platí:
\itemitem{$\bullet$} najväčší spoločný deliteľ čísel $a$, $b$ je $15$,
\itemitem{$\bullet$} najväčší spoločný deliteľ čísel $b$, $c$ je $6$,
\itemitem{$\bullet$} súčin čísel $b$, $c$ je $1\,800$,
\itemitem{$\bullet$} najmenší spoločný násobok čísel $a$, $b$ je $3\,150$

Ktoré sú to čísla?
}
\podpis{L. Šimůnek}

{%%%%%   Z7-I-3
V~štvoruholníku $KLMN$ poznáme vyznačené uhly (\obr{}) a~vieme, že platí $|KN|=|LM|$.
Zisti veľkosť uhla $KNM$.
\insp{z59.5}%
}
\podpis{L. Hozová}

{%%%%%   Z7-I-4
Kocka bola zložená z~64 kocôčok s~hranou dĺžky $2\cm$. Potom bolo niekoľko kocôčok z~viditeľnej strany odobraných, pozri 
\obr.
\insp{z59.6}%
  
\itemitem{1.} Urči objem a~povrch získaného telesa.
\itemitem{2.} Teleso bolo na celom povrchu natreté červenou farbou, potom rozobrané na pôvodné kocôčky. Koľko z~nich malo 6, koľko 5, 4, 3, 2, 1 či žiadnu stenu červenú?
}
\podpis{M. Volfová}

{%%%%%   Z7-I-5
Na číselnej osi (\obr) sú znázornené čísla $12x$ a $\m4x$. Znázorni na tejto osi nulu a~číslo $x$.
\insp{z59.7}%
}
\podpis{M. Petrová}

{%%%%%   Z7-I-6
Doplň miesto hviezdičiek čísla tak, aby súčet výsledkov nasledujúcich dvoch príkladov bol $5\,842$.
$$
\begin{array}{rrrr}
 & * & 2 & * & 7 \\
 & 3 & * & 4 & * \\
\hline
 & 4 & * & 0 & * \\
\end{array}
\qquad
\begin{array}{rrrr}
 & 2 & * & 9 & * \\
- & * & 2 & 5 & 4 \\
\hline
 & * & 5 & * & * \\
\end{array}
$$
Úloha má viac riešení, urči aspoň dve.
}
\podpis{M. Dillingerová}

{%%%%%   Z8-I-1
Napíšte číslo $75$ ako súčet niekoľko bezprostredne po sebe idúcich prirodzených čísel. Nájdite aspoň štyri riešenia.}
\podpis{M. Volfová}

{%%%%%   Z8-I-2
Tri kamarátky sa zišli na chalupe a~vyrazili na huby. Našli spolu 55~hríbov. Po návrate si urobili praženicu, rozdelili ju na štyri rovnaké porcie a~pozvali na ňu kamaráta Petra. Ľuba dala na praženicu šesť zo svojich hríbov, Marienka osem a~Šárka päť. Každej po tom zostal rovnaký počet hríbov. Peter im daroval bonboniéru, v~ktorej bolo 38~bonbónov, a~povedal, že sa majú spravodlivo rozdeliť podľa toho, akým dielom prispeli na jeho jedlo.
\itemitem{1.} Koľko hríbov našla každá?
\itemitem{2.} Ako si mali podľa Petra bonbóny podeliť? Určte koľko bonbónov si mali jednotlivé kamarátky vziať.
}
\podpis{M. Volfová}

{%%%%%   Z8-I-3
Sedadlá v~divadelnej sále sú rozdelené do troch kategórií podľa ich vzdialenosti od javiska. "I.\,miesta" sú najbližšie k~javisku, tvoria dve pätiny kapacity sály a~predávajú sa za 11 €. "II.\,miesta" tvoria ďalšie dve pätiny kapacity sály a~predávajú sa za 10 €. Ostatné "III.\,miesta"  sa predávajú za 9 €. Pred zahájením predpredaja na slávnostnú premiéru bolo rozdaných 150 vstupeniek zadarmo pozvaným hosťom. Vstupenky boli rozdávané postupne od predných miest sály dozadu. Všetky ostatné vstupenky sa potom predali. Keby sa však voľné vstupenky rozdávali postupne od zadných miest dopredu, bola by tržba o~216 € väčšia. Koľko miest bolo v~sále?
}
\podpis{L. Šimůnek}

{%%%%%   Z8-I-4
Dostali sme kocku, ktorá mala dĺžku hrany vyjadrenú v~centimetroch celým číslom. Všetky jej steny sme nafarbili na červeno a~potom sme ju rozrezali bezo zvyšku na kocôčky s~hranou dĺžky $1\cm$. 
\itemitem{$\bullet$} Lukáš tvrdí, že kocôčok s~dvomi nafarbenými stenami je desaťkrát viac než tých s~tromi nafarbenými stenami.
\itemitem{$\bullet$} Martina tvrdí, že kocôčok s~dvomi nafarbenými stenami je pätnásťkrát viac než tých s~tromi nafarbenými stenami.

\noindent
Pravdu má však iba jeden -- kto? Koľko cm merala hrana pôvodnej kocky?
}
\podpis{L. Šimůnek}

{%%%%%   Z8-I-5
Zo štvorca so stranou dĺžky $6\cm$ odstrihneme od každého vrcholu zhodný rovnoramenný pravouhlý trojuholník tak, aby sa obsah štvorca zmenšil o~32\%. Zistite dĺžku odvesien odstrihnutých trojuholníkov.}
\podpis{M. Krejčová}

{%%%%%   Z8-I-6
V~dvoch miestnostiach vzdelávacieho centra sa konali prednášky. Priemerný vek ôsmich ľudí prítomných v~prvej miestnosti bol 20~rokov, priemerný vek dvanástich ľudí prítomných v~druhej miestnosti bol 45~rokov. Počas prednášky odišiel jeden účastník a~tým sa priemerný vek všetkých osôb v~oboch miestnostiach zvýšil o~jeden rok. Koľko rokov mal účastník, ktorý odišiel?}
\podpis{L. Hozová}

{%%%%%   Z9-I-1
Dostal som zadané dve prirodzené čísla. Potom som ich obe zaokrúhlil na desiatky. Ktoré čísla som dostal zadané, ak viete, že súčasne platí:
\itemitem{$\bullet$} podiel zaokrúhlených čísel je rovnaký ako podiel pôvodných čísel,
\itemitem{$\bullet$} súčin zaokrúhlených čísel je o~295 väčší než súčin pôvodných čísel,
\itemitem{$\bullet$} súčet zaokrúhlených čísel je o~6 väčší než súčet pôvodných čísel.
}
\podpis{L. Šimůnek}

{%%%%%   Z9-I-2
Pat a~Mat boli na výlete. Vyšli ráno po ôsmej hodine v~čase, keď veľká a~malá ručička na Patových hodinkách ležali na opačných polpriamkach. Na opačných polpriamkach ležali ručičky Patových hodiniek aj v~čase, keď sa obaja priatelia pred poludním vrátili. Mat dobu trvania výletu meral stopkami. Určite aj vy s~presnosťou na sekundy, ako dlho trval výlet. Predpokladajte, že Patove hodinky a~Matove stopky išli presne.}
\podpis{M. Volfová}

{%%%%%   Z9-I-3
Na \obr{} je kocka s~hranou dĺžky $2\cm$ tvorená ôsmimi kocôčkami s~hranou dĺžky $1\cm$. Osem stien kocôčok je nafarbených na čierno, ostatné sú biele. Pritom sa z~nich dá zložiť kocka, ktorej povrch je biely. Koľkými spôsobmi môžu byť kocôčky nafarbené? Predpokladajte, že rovnako nafarbené kocôčky nedokážeme odlíšiť, možno ich teda zamieňať.
\insp{z59.8}%
}
\podpis{K. Pazourek}

{%%%%%   Z9-I-4
Adam a~Eva dostali košík, v~ktorom bolo 31~jabĺk. Prvý deň zjedla Eva tri štvrtiny toho, čo zjedol Adam. Druhý deň zjedla Eva dve tretiny toho, čo zjedol druhý deň Adam. Druhý deň večer bol košík prázdny. Koľko jabĺk zjedla z~košíka Eva? (Adam i~Eva jablká jedia celé a~nedelia si ich.)}
\podpis{L. Hozová}

{%%%%%   Z9-I-5
Vodič preváža mlieko v~cisterne tvaru valca. Priemer podstavy je $180\cm$, dĺžka cisterny je $4\,\text{m}$. Koľko hl mlieka je v~cisterne, ak je naplnená do troch štvrtín priemeru (\obr{})?
\insp{z59.9}%
}
\podpis{M. Krejčová}

{%%%%%   Z9-I-6
V~lichobežníku $ABCD$ so základňami $AB$ a~$CD$ dĺžok $7\cm$ a~$4\cm$ sú body $S$ a~$T$ stredy strán $AD$ a~$BC$, pozri \obr{}. Bod~$X$ je priesečník úsečiek $AC$ a~$ST$, bod~$Y$ je priesečník úsečky~$AB$ a~priamky~$DX$. Obsah štvoruholníka $AYCD$ je $12\cm^2$. Vypočítajte obsah lichobežníka $ABCD$.
\insp{z59.10}%
}
\podpis{M. Dillingerová}

{%%%%%   Z4-II-1
...}
\podpis{...}

{%%%%%   Z4-II-2
...}
\podpis{...}

{%%%%%   Z4-II-3
...}
\podpis{...}

{%%%%%   Z5-II-1
Matúš a~jeho kamaráti išli na Štefana koledovať. Okrem jabĺčok, orieškov a~perníčkov dostal každý z~chlapcov aj pomaranče. Jaro dostal 1~pomaranč, Milan tiež. Po dvoch pomarančoch dostali Rado, Patrik, Michal a~Dušan. Matúš dostal dokonca štyri pomaranče, čo bolo najviac zo všetkých chlapcov. Ostatní chlapci dostali po troch pomarančoch. Koľko chlapcov išlo koledovať, keď všetci spolu dostali 23 pomarančov?}
\podpis{M. Volfová}

{%%%%%   Z5-II-2
Bzdocha Jozefína dopadla na stôl doprostred štvorcovej siete tvorenej 81 štvorcami  -- viď \obr. Rozhodla sa, že zo siete nezlezie na stôl priamo, ale nasledujúcim spôsobom: najprv jeden štvorček na juh, potom jeden na východ, potom dva na sever, dva na západ a~opäť jeden na juh, jeden na východ, dva na sever, dva na západ...Na ktorom štvorčeku bola tesne pred tým, než zliezla z~tejto siete na stôl? Po koľkých štvorčekoch tejto siete liezla?
\insp{z59ii.1}%
}
\podpis{M. Petrová}

{%%%%%   Z5-II-3
Jurko má paličky dĺžok $2\cm$, $3\cm$, $3\cm$, $3\cm$, $4\cm$, $5\cm$, $5\cm$, $5\cm$, $6\cm$, $6\cm$ a~$9\cm$. Skladá z~nich strany trojuholníkov. Žiadna palička nie je súčasťou strany dvoch alebo viac trojuholníkov. Jurko môže použiť toľko paličiek, koľko chce, ale nesmie ich lámať a~každá použitá palička musí ležať celá na obvode trojuholníka. Jurko tvrdí, že sa jeho paličky dajú použiť na poskladanie strán troch trojuholníkov s~rovnakými obvodmi. Má pravdu? Aký najväčší obvod by také trojuholníky mali?}
\podpis{M. Dillingerová}

{%%%%%   Z6-II-1
Vpíšte do krúžkov čísla $1$, $2$, $3$, $4$, $5$, $6$, $7$ a~$8$ tak, aby žiadne dve po sebe idúce čísla neboli
spojené čiarou a~v~sivých krúžkoch boli nepárne čísla.
\insp{z59ii.61}%
%Určte obsah obdĺžnika, keď viete, že jeho šírka je rovná $\frac23$ jeho dĺžky 
%a~obvod meria $148\cm$.
}
\podpis{S. Bednářová}

{%%%%%   Z6-II-2
Myslím si štvorciferné číslo, ktorého každá číslica je iná. Keď škrtnem posledné dve číslice v~tomto čísle, dostanem prvočíslo. Rovnako dostanem prvočíslo aj v~prípade, keď vyškrtnem z~mysleného čísla druhú a~štvrtú číslicu, a~dokonca aj v~prípade, keď z~neho vyškrtnem prostredné dve číslice. Moje myslené číslo ale prvočíslo nie je -- dá sa bezo zvyšku deliť tromi. Čísel, ktoré majú tieto vlastnosti, je viac. To moje je najväčšie z~nich. Ktoré číslo si myslím?}
\podpis{M. Petrová}

{%%%%%   Z6-II-3
Krabička tvaru kocky s~hranou dĺžky $4\cm$ je úplne naplnená uloženými malými kockami s~hranou dĺžky $1\cm$. Vymysli všetky rôzne krabičky také, ktoré majú štvorcové dno a~všetky kocky sa do každej z~nich presne vojdú. Napíš ich rozmery.}
\podpis{M. Krejčová}

{%%%%%   Z7-II-1
Kremienok a~Chocholúšik našli debničku s~pokladom. Každý z~nich si nabral do jedného vrecka strieborné mince a~do druhého vrecka zlaté mince. Kremienok mal na pravom vrecku dieru a~cestou polovicu zlatiek stratil. Chocholúšik mal dieru na ľavom vrecku a~cestou domov stratil polovicu strieborniakov. Doma venoval Chocholúšik tretinu svojich zlatiek Kremienkovi a~Kremienok štvrtinu svojich strieborniakov Chocholúšikovi. Každý potom mal presne 12~zlatiek a~18~strieborniakov. Koľko zlatiek a~koľko strieborniakov si vzal každý z~nich z~nájdeného pokladu?}
\podpis{M. Dillingerová}

{%%%%%   Z7-II-2
Na tabuli sú napísané tri prirodzené čísla $x$, $y$, $z$. Urči ktoré, ak vieš, že súčasne platí:
\itemitem{$\bullet$} $x$ je z~nich najväčšie,
\itemitem{$\bullet$} najmenší spoločný násobok čísel $x$ a~$y$ je $200$,
\itemitem{$\bullet$} najmenší spoločný násobok čísel $y$ a~$z$ je $300$,
\itemitem{$\bullet$} najmenší spoločný násobok čísel $x$ a~$z$ je $120$.
}
\podpis{L. Šimůnek}

{%%%%%   Z7-II-3
%%Pravidelná šesťcípa hviezda $ABCDEFGHIJKL$ so stredom~$S$, znázornená na \obr{}, vznikla zjednotením dvoch %%rovnostranných trojuholníkov, z~ktorých každý mal obsah $36\cm^2$. Vypočítaj obsah štvoruholníka $SBDF$.
Pravidelná šesťcípa hviezda $ABCDEFGHIJKL$ so stredom~$S$, znázornená na obrázku, vznikla zjednotením dvoch rovnostranných trojuholníkov, z~ktorých každý mal obsah $72\cm^2$. Vypočítaj obsah štvoruholníka $ABCF$.
\insp{z59ii.71}
}
\podpis{S. Bednářová}

{%%%%%   Z8-II-1
Priemerný vek rodiny Gebuľových, ktorú tvorí otec, mama a~niekoľko detí, je 18 rokov. Otec má 38 rokov a~priemerný vek rodiny bez neho je 14. Koľko detí majú Gebuľovci?}
\podpis{L. Hozová}

{%%%%%   Z8-II-2
Koľko existuje šesťciferných prirodzených čísel deliteľných bezo zvyšku $45$ takých, ktoré majú na mieste stotisícok číslicu~$1$, na mieste tisícok číslicu~$2$ a~na mieste desiatok číslicu~$3$?}
\podpis{L. Šimůnek}

{%%%%%   Z8-II-3
Na obrázku je šesťuholník $ABEFGD$. Štvoruholníky $ABCD$ a~$EFGC$ sú zhodné obdĺžniky a~štvoruholník $BEGD$ je tiež obdĺžnik. Určte pomer obsahov bielej a~sivej časti šesťuholníka, ak $|AB|=5\cm$ a~trojuholník $BEC$ je rovnostranný.
\insp{z59ii.81}%
}
\podpis{K. Pazourek}

{%%%%%   Z9-II-1
Doplňte do prázdnych políčok na \obr{} čísla tak, aby v~každom políčku bol súčet čísel zo všetkých s~ním priamo susediacich políčok, ktoré sú svetlejšie ako dopĺňané. To znamená, že vo svetlosivom políčku je súčet čísel zo susedných bielych políčok, a~v~tmavosivom políčku je súčet čísel zo susedných svetlosivých políčok.
\insp{z59ii.2}%
}
\podpis{S. Bednářová}

{%%%%%   Z9-II-2
Šárka naplnila pohár a~hrnček, ktorý mal dvakrát väčší objem ako pohár, vodou s~džúsom. Pomer vody a~džúsu bol v~pohári $2:1$ a~v~hrnčeku $4:1$. Potom preliala obsah pohára i~obsah hrnčeka do džbánu. Aký bol pomer vody a~džúsu v~džbáne?}
\podpis{L. Hozová}

{%%%%%   Z9-II-3
Dostal som zadané dve dvojciferné prirodzené čísla. Potom som ich obe zaokrúhlil na desiatky. Určite, ktoré čísla som mal zadané, ak viete, že súčasne platí:
\itemitem{$\bullet$} rozdiel zaokrúhlených čísiel je rovnaký ako rozdiel pôvodných čísiel,
\itemitem{$\bullet$} súčin zaokrúhlených čísiel je o~$184$ väčší než súčin pôvodných čísiel.
}
\podpis{L. Šimůnek}

{%%%%%   Z9-II-4
Do rovnostranného trojuholníka $ABC$ je vpísaný pravidelný šesťuholník $KLMNOP$ tak, že body $K$, $L$ ležia na strane~$AB$, body $M$, $N$ ležia na strane $BC$ a~body $O$, $P$ ležia na strane~$AC$. Vypočítajte obsah šesťuholníka $KLMNOP$, ak obsah trojuholníka $ABC$ je $60\cm^2$.}
\podpis{K. Pazourek}

{%%%%%   Z9-III-1
Pani učiteľka potrebovala vymyslieť príklady na rovnice do písomky. Preto si vypísala všetky rovnice tvaru
$$
\heartsuit\cdot x + \clubsuit = 13,
$$
kde $\heartsuit$ a~$\clubsuit$ sú jednociferné prirodzené čísla. Zo všetkých rovníc vybrala tie, ktorých koreň bol~$3$. V~písomke dala do každej skupiny jednu z~nich. Koľko najviac skupín mohla mať?
}
\podpis{K. Pazourek}

{%%%%%   Z9-III-2
Počet domácich a~dochádzajúcich žiakov školy je v~pomere $3:1$. Domáci chodia pešo. U~dochádzajúcich je pomer počtu tých, čo využívajú verejnú dopravu, a~tých, čo jazdia sami na bicykli alebo s~rodičmi autom, $3:2$. U~verejnej dopravy je pomer tých, čo jazdia vlakom, a~tých, čo autobusom, $7:5$. Ďalej vieme, že pomer tých, čo dochádzajú na bicykli, k~počtu tých, ktorých vozia rodičia autom, je $5:3$. O~koľko viac žiakov dochádza vlakom ako vozia rodičia, keď verejnou dopravou ich dochádza~24? Koľko je žiakov školy?}
\podpis{M. Volfová}

{%%%%%   Z9-III-3
Dostali sme kocku, ktorá mala dĺžku hrany vyjadrenú v~centimetroch celým číslom väčším ako~$2$. Všetky jej steny sme nafarbili na žlto a~potom sme ju rozrezali bezo zvyšku na kocôčky s~hranou dĺžky $1\cm$. Tieto kocôčky sme roztriedili na štyri kôpky. Na prvej boli kocôčky s~jednou žltou stenou, na druhej s~dvomi žltými stenami a~na tretej s~tromi. Na štvrtej kôpke potom boli kocôčky bez žltej steny.
Určte dĺžku hrany pôvodnej kocky, ak viete, že aspoň jedno z~nasledujúcich tvrdení je pravdivé:
  \ite{$\bullet$} Počty kocôčok v~prvej a~štvrtej kôpke boli v~pomere $4:9$.
  \ite{$\bullet$} Na prvej kôpke bolo trikrát viac kocôčok ako na druhej.
%Určte dĺžku hrany pôvodnej kocky, ak viete, že na prvej kôpke bolo trikrát viac kocôčok ako na druhej.
}
\podpis{L. Šimůnek}

{%%%%%   Z9-III-4
V~rovnostrannom trojuholníku $ABC$ leží pravidelný šesťuholník $KLMNOP$ tak, že body $K$, $M$, $O$ sú postupne stredmi strán $AB$, $BC$ a~$AC$. Vypočítajte obsah šesťuholníka $KLMNOP$, ak obsah trojuholníka $ABC$ je $60\cm^2$.}
\podpis{K. Pazourek}

