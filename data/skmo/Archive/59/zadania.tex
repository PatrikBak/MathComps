{%%%%%   A-I-1
V~obore reálnych čísel riešte sústavu rovníc
$$
\align
\sqrt{x^2 - y}&= z - 1,\\
\sqrt{y^2 - z}&= x - 1,\\
\sqrt{z^2 - x}&= y - 1.
\endalign
$$}
\podpis{Radek Horenský}

{%%%%%   A-I-2
Do kosoštvorca $ABCD$ je vpísaná kružnica. Uvažujme jej ľubovoľnú dotyčnicu pretínajúcu obe strany $BC$, $CD$ a~označme postupne $R$, $S$ jej priesečníky s~priamkami $AB$, $AD$. Dokážte, že hodnota súčinu $|BR|\cdot|DS|$ od voľby dotyčnice nezávisí.}
\podpis{Leo Boček}

{%%%%%   A-I-3
Na tabuli sú napísané čísla $1,2,\dots,33$. V~jednom kroku zvolíme na tabuli dve čísla, z~ktorých jedno je deliteľom druhého, obe zotrieme a~na tabuľu napíšeme ich (celočíselný) podiel. Takto pokračujeme, kým na tabuli nezostanú iba čísla, z~ktorých žiadne nie je deliteľom iného. (V~jednom kroku môžeme zotrieť aj dve rovnaké čísla a~nahradiť ich číslom~$1$.) Najmenej koľko čísel môže na tabuli zostať?}
\podpis{Peter Novotný}

{%%%%%   A-I-4
V~ľubovoľnom ostrouhlom rôznostrannom trojuholníku $ABC$ označme $O$, $V$ a~$S$ postupne stred kružnice opísanej, priesečník výšok a~stred kružnice vpísanej. Dokážte, že os úsečky~$OV$ prechádza bodom~$S$ práve vtedy, keď jeden
vnútorný uhol trojuholníka $ABC$ má veľkosť $60\st$.}
\podpis{Tomáš Jurík}

{%%%%%   A-I-5
V~nádrži je $r_0$~rýb, spoločný úlovok $n$~rybárov. Prichádzajú pre svoj podiel jednotlivo. Každý si myslí, že sa dostavil ako prvý, a~aby si vzal presne $n$-tinu aktuálneho počtu rýb v~nádrži, musí predtým jednu z~rýb pustiť späť
do mora. Určte najmenšie možné číslo~$r_0$ v~závislosti od daného $n\ge2$, keď aj posledný rybár si aspoň jednu rybu odnesie.}
\podpis{Dag Hrubý}

{%%%%%   A-I-6
Pre dané prvočíslo~$p$ určte počet (všetkých) usporiadaných trojíc $(a,b,c)$ čísel z~množiny $\{1,2,3,\dots,2p^2\}$, ktoré spĺňajú vzťah
$$
{[a, c] + [b, c]\over a~+ b}={p^2 + 1\over p^2 + 2}\cdot c,
$$
pričom $[x, y]$ označuje najmenší spoločný násobok čísel $x$ a~$y$.}
\podpis{Tomáš Jurík}

{%%%%%   B-I-1
Na stole ležia tri kôpky zápaliek: v~jednej 2\,009, v~druhej 2\,010 a~v~poslednej 2\,011. Hráč, ktorý je na ťahu, zvolí dve kôpky a~z~každej z~nich odoberie po jednej zápalke. V~hre sa pravidelne striedajú dvaja hráči. Hra končí, akonáhle niektorá kôpka zmizne. Vyhráva ten hráč, ktorý urobil posledný ťah. Popíšte stratégiu jedného z~hráčov, ktorá mu zaručí výhru.}
\podpis{Ján Mazák}

{%%%%%   B-I-2
Na tabuli je napísané štvorciferné číslo, ktoré má presne šesť kladných deliteľov, z~ktorých práve dva sú jednociferné a~práve dva dvojciferné. Väčší z~dvojciferných deliteľov je druhou mocninou prirodzeného čísla. Určte všetky čísla, ktoré môžu byť na tabuli napísané.}
\podpis{Peter Novotný}

{%%%%%   B-I-3
V~rovine je daná úsečka~$AB$. Zostrojte rovnobežník $ABCD$, pre ktorého stredy strán $AB$, $CD$, $DA$ označené postupne $K$, $L$, $M$ platí: body $A$, $B$, $L$, $D$ ležia na jednej kružnici a~aj body $K$, $L$, $D$, $M$ ležia na jednej kružnici.}
\podpis{Jaroslav Švrček}

{%%%%%   B-I-4
Nájdite 2\,009 po sebe idúcich štvorciferných čísel, ktorých súčet je súčinom troch po sebe idúcich prirodzených čísel.}
\podpis{Radek Horenský}

{%%%%%   B-I-5
Vnútri kratšieho oblúka~$AB$ kružnice opísanej rovnostrannému trojuholníku $ABC$ je zvolený bod~$D$. Tetiva~$CD$ pretína stranu~$AB$ v~bode~$E$. Dokážte, že trojuholník so stranami dĺžok $|AE|$, $|BE|$, $|CE|$ je podobný s~trojuholníkom $ABD$.}
\podpis{Pavel Leischner}

{%%%%%   B-I-6
Reálne čísla $a$, $b$ majú túto vlastnosť: rovnica $x^2-ax+b-1=0$  má v~množine reálnych čísel dva rôzne korene, ktorých rozdiel je kladným koreňom rovnice $x^2-ax+b+1=0$.
\ite a) Dokážte nerovnosť $b>3$.
\ite b) Pomocou $b$ vyjadrite korene oboch rovníc.}
\podpis{Jaromír Šimša}

{%%%%%   C-I-1
Erika a~Klárka hrali hru "slovný logik" s~týmito pravidlami: Hráč~$A$ si myslí slovo zložené z~piatich rôznych písmen. Hráč~$B$ vysloví ľubovoľné slovo zložené z~piatich rôznych písmen a~hráč~$A$ mu prezradí, koľko písmen uhádol na správnej pozícii a~koľko na nesprávnej. Písmená považujeme za rôzne, aj keď sa líšia iba mäkčeňom alebo dĺžňom (napríklad písmena~A, Á sú rôzne). Keby si hráč~$A$ myslel napríklad slovo LOĎKA a~$B$ by vyslovil slovo KOLÁČ, odpovie hráč~$A$, že jedno písmeno uhádol hráč~$B$ na správnej pozícii a~dve na nesprávnej. Skrátene oznámi "$1+2$", lebo sa naozaj obe slová zhodujú iba v~písmene~O vrátane pozície (druhej zľava) a~v~písmenách K a~L, ktorých pozície sú odlišné. Erika si myslela slovo z~piatich rôznych písmen a~Klárka vyslovila slová KABÁT, STRUK, SKOBA, CESTA a~ZÁPAL. Erika na tieto slová v~danom poradí odpovedala $0+3$, $0+2$, $1+2$, $2+0$ a~$1+2$. Zistite, aké slovo si Erika mohla myslieť.}
\podpis{Peter Novotný}

{%%%%%   C-I-2
Vrcholom~$C$ pravouholníka $ABCD$ veďte priamky $p$ a~$q$, ktoré majú s~daným pravouholníkom spoločný iba bod~$C$, pričom priamka~$p$ má od bodu~$A$ najväčšiu možnú vzdialenosť a~priamka~$q$ vymedzuje s~priamkami $AB$, $AD$ trojuholník s~čo najmenším obsahom.}
\podpis{Leo Boček}

{%%%%%   C-I-3
Určte všetky reálne čísla~$x$, ktoré vyhovujú rovnici $4x - 2 \lfloor x\rfloor= 5$. (Symbol $\lfloor x\rfloor$ označuje najväčšie celé číslo, ktoré nie je väčšie ako číslo~$x$, tzv. dolnú celú časť reálneho čísla~$x$.)}
\podpis{Jaroslav Švrček}

{%%%%%   C-I-4
Kružnica $k(S;r)$ sa dotýka priamky~$AB$ v~bode~$A$. Kružnica $l(T;s)$ sa dotýka priamky~$AB$ v~bode~$B$ a~pretína kružnicu~$k$ v~krajných bodoch $C$, $D$ jej priemeru. Vyjadrite dĺžku~$a$ úsečky~$AB$ pomocou polomerov $r$, $s$.
Dokážte ďalej, že priesečník~$M$ priamok $CD$, $AB$ je stredom úsečky~$AB$.}
\podpis{Leo Boček}

{%%%%%   C-I-5
Dokážte, že pre ľubovoľné kladné reálne čísla $a$, $b$ platí
$$
\sqrt{ab}\le{2(a^2+3ab+b^2)\over5(a+b)}\le{a+b\over2},
$$
a~pre každú z~oboch nerovností zistite, kedy prechádza na rovnosť.}
\podpis{Ján Mazák}

{%%%%%   C-I-6
Nájdite všetky prirodzené čísla, ktoré nie sú deliteľné desiatimi a~ktoré vo svojom dekadickom zápise majú niekde vedľa seba dve nuly, po ktorých vyškrtnutí sa pôvodné číslo 89-krát zmenší.}
\podpis{Jaromír Šimša}

{%%%%%   A-S-1
V~obore reálnych čísel riešte sústavu rovníc
$$
\align
\sqrt{x-y^2}&=z-1,\\
\sqrt{y-z^2}&=x-1,\\
\sqrt{z-x^2}&=y-1.
\endalign
$$
}
\podpis{Radek Horenský}

{%%%%%   A-S-2
Nájdite všetky možné hodnoty podielu
$$
\frac{r+\rho}{a+b},
$$
pričom $r$ je polomer kružnice opísanej a~$\rho$ polomer kružnice vpísanej
pravouhlému trojuholníku s~odvesnami dĺžok $a$ a~$b$.}
\podpis{Tomáš Jurík}

{%%%%%   A-S-3
Na tabuli sú napísané čísla $1,2,\dots,33$. V~jednom kroku zvolíme
niekoľko čísel napísaných na tabuli (aspoň dve), ktorých súčin je druhou
mocninou prirodzeného čísla, zvolené čísla zotrieme
a~na tabuľu napíšeme druhú odmocninu z~ich súčinu. Takto pokračujeme,
až na tabuli ostanú iba také čísla, že súčin žiadnych z~nich
nie je druhou mocninou. Koľko najmenej čísel môže na tabuli ostať?}
\podpis{Peter Novotný}

{%%%%%   A-II-1
Dokážte, že rovnica $x^2+p|x|=qx-1$ s~reálnymi parametrami $p$, $q$
má v~obore reálnych čísel štyri riešenia práve vtedy, keď platí
$p+|q|+2<0$.}
\podpis{Jaromír Šimša}

{%%%%%   A-II-2
Daný je rovnobežník $ABCD$ s~tupým uhlom $ABC$. Na jeho uhlopriečke~$AC$
v~polrovine $BDC$ zvoľme bod~$P$ tak, aby platilo $|\angle
BPD|=|\angle ABC|$. Dokážte, že priamka~$CD$ je dotyčnicou ku kružnici
opísanej trojuholníku $BCP$ práve vtedy, keď úsečky $AB$ a~$BD$ sú zhodné.}
\podpis{Jaroslav Švrček}

{%%%%%   A-II-3
Určte všetky celé kladné čísla $m$, $n$ také, že $n$ delí $2m-1$
a~$m$ delí $2n-1$.}
\podpis{Tomáš Szaniszlo}

{%%%%%   A-II-4
V~ľubovoľnom trojuholníku $ABC$ označme $O$ stred kružnice vpísanej, $P$
stred kružnice pripísanej ku strane~$BC$ a~$D$ priesečník osi uhla
$CAB$ so stranou~$BC$. Dokážte, že platí
$$
\frac{2}{|AD|}=\frac{1}{|AO|}+\frac{1}{|AP|}.
$$
(Kružnica pripísaná ku strane~$BC$ je taká kružnica, ktorá sa dotýka jednak strany~$BC$,
jednak oboch polpriamok opačných k~polpriamkam $BA$ a~$CA$.)}
\podpis{Pavel Leischner}

{%%%%%   A-III-1
Určte všetky dvojice celých kladných čísel $a$, $b$, pre ktoré platí
$$
4^a+4a^2+4=b^2.
$$
}
\podpis{Martin Panák}

{%%%%%   A-III-2
Kruhový terč s~polomerom $12\cm$ zasiahlo 19~výstrelov. Dokážte, že
vzdialenosť niektorých dvoch zásahov je menšia ako $7\cm$.}
\podpis{Vojtech Bálint, Jaromír Šimša}

{%%%%%   A-III-3
Rumburak uniesol na svoj hrad 31 členov strany~$A$, 28~členov strany~$B$,
23~členov strany~$C$, 19~členov strany~$D$ a~každého zavrel do
samostatnej cely. Po práci sa občas mohli prechádzať po dvore a~rozprávať sa.
Akonáhle sa spolu začali rozprávať traja členovia troch rôznych strán, Rumburak
ich za trest preregistroval do štvrtej strany. (Nikdy sa
spolu nerozprávali viac ako traja unesení.)
\ite a) Mohlo sa stať, že po určitom čase boli všetci unesení
členmi jednej strany? Ktorej?
\ite b) Určte všetky štvorice celých kladných čísel,
ktorých súčet je~$101$
a~ktoré ako počty unesených členov štyroch strán umožňujú, aby
sa Rumburakovým pričinením stali časom všetci členmi jednej strany.}
\podpis{Vojtech Bálint, Jaromír Šimša}

{%%%%%   A-III-4
Je daná kružnica~$k$ s~tetivou~$AC$, ktorá nie je priemerom.
Na jej dotyčnici vedenej bodom~$A$ zvolíme bod $X\ne A$
a~označíme $D$ priesečník kružnice~$k$ s~vnútrom úsečky~$XC$
(ak existuje). Trojuholník $ACD$ doplníme na lichobežník
$ABCD$ vpísaný do kružnice~$k$. Určte množinu priesečníkov priamok $BC$
a~$AD$ prislúchajúcich všetkým takým lichobežníkom.}
\podpis{Pavel Leischner}

{%%%%%   A-III-5
Na tabuli sú napísané čísla $1,2,\dots,33$. V~jednom kroku zvolíme
dve čísla napísané na tabuli, ktorých súčin je druhou
mocninou prirodzeného čísla, obe zvolené čísla zotrieme
a~na tabuľu napíšeme druhú odmocninu z~ich súčinu. Takto pokračujeme,
až na tabuli ostanú iba také čísla, že súčin žiadnych dvoch z~nich
nie je druhou mocninou. (V~jednom kroku môžeme zotrieť aj dve
rovnaké čísla a~nahradiť ich tým istým číslom.)
Dokážte, že na tabuli ostane aspoň 16~čísel.}
\podpis{Peter Novotný}

{%%%%%   A-III-6
Nájdite minimum výrazu
$$
\frac{a+b+c}{2}-\frac{[a,b]+[b,c]+[c,a]}{a+b+c},
$$
pričom premenné $a$, $b$, $c$ sú ľubovoľné celé čísla väčšie ako~$1$
a~$[x,y]$ označuje najmenší spoločný násobok čísel $x$, $y$.}
\podpis{Tomáš Jurík}

{%%%%%   B-S-1
Určte všetky hodnoty reálnych parametrov $p$, $q$, pre ktoré má každá z~rovníc
$$
x(x-p)=3+q, \quad x(x+p)=3-q
$$
v~obore reálnych čísel dva rôzne korene, ktorých aritmetický priemer je jedným z~koreňov zvyšnej rovnice.}
\podpis{Jaromír Šimša}

{%%%%%   B-S-2
Dané sú dĺžky odvesien $a=|BC|$, $b=|AC|$ pravouhlého trojuholníka $ABC$,
pričom $a>b$. Označme $D$ stred prepony~$AB$ a~$E$ ($E\ne C$) priesečník strany~$BC$ s~kružnicou
opísanou trojuholníku $ADC$. Vypočítajte obsah trojuholníka $EAD$.}
\podpis{Pavel Novotný}

{%%%%%   B-S-3
Určte všetky dvojice celých kladných čísel $m$, $n$, pre ktoré platí $37+27^m=n^3$.}
\podpis{Martin Panák}

{%%%%%   B-II-1
Kružnica $l(T;s)$ prechádza stredom kružnice $k(S;2\cm)$. Kružnica $m(U;t)$ sa zvonka dotýka kružníc $k$ a~$l$, pričom $US\perp ST$. Polomery $s$ a~$t$ vyjadrené v~centimetroch sú celé čísla. Určte ich.
}
\podpis{Pavel Leischner}

{%%%%%   B-II-2
V~matematickej súťaži bolo zadaných 7~úloh a~za každú z~nich mohol súťažiaci získať 0, 1 alebo 2~body. Súťaže sa zúčastnilo 60~žiakov. Za každú úlohu bolo udelených aspoň 95~bodov. Dokážte, že medzi súťažiacimi nájdeme dvoch takých, že každú z~úloh vyriešil aspoň jeden z~nich za 2~body.
}
\podpis{Ján Mazák}

{%%%%%   B-II-3
V~rovine je daný rovnobežník $ABCD$. Označme postupne $K$, $L$, $M$ stredy strán $AB$, $CD$, $AD$. Predpokladajme, že body
$A$, $B$, $L$, $D$ ležia na jednej kružnici a~súčasne aj body $K$, $L$, $D$, $M$ ležia na jednej kružnici. Dokážte, že $|AC|=2\cdot|AD|$.
}
\podpis{Jaroslav Švrček}

{%%%%%   B-II-4
Číslo~$n$ je súčinom štyroch prvočísel. Ak každé z~týchto prvočísel zväčšíme o~$1$ a~vzniknuté štyri čísla vynásobíme, dostaneme číslo o~$2\,886$ väčšie ako pôvodné číslo~$n$. Určte všetky také~$n$.
}
\podpis{Jaromír Šimša}

{%%%%%   C-S-1
Ak zväčšíme čitateľ aj menovateľ istého zlomku o~$1$, dostaneme zlomok o~hodnotu~$1/20$ väčší.
Ak urobíme s~väčším zlomkom rovnakú operáciu, dostaneme zlomok o~hodnotu~$1/12$ väčší,
ako bola hodnota zlomku na začiatku. Určte všetky tri zlomky.}
\podpis{Jaromír Šimša}

{%%%%%   C-S-2
Kružnice $k(S;6\cm)$ a~$l(O;4\cm)$ majú vnútorný dotyk v~bode~$B$.
Určte dĺžky strán trojuholníka $ABC$, pričom bod~$A$ je priesečník priamky~$OB$
s~kružnicou~$k$ a~bod~$C$ je priesečník kružnice~$k$ s~dotyčnicou z~bodu~$A$
ku kružnici~$l$.}
\podpis{Pavel Leischner}

{%%%%%   C-S-3
Nájdite všetky dvojice nezáporných celých čísel $a$, $b$, pre ktoré platí
$$
a^2 + b + 2 = a + b^2.
$$}
\podpis{Ján Mazák}

{%%%%%   C-II-1
Dokážte, že pre ľubovoľné celé čísla $n$ a~$k$ väčšie ako $1$
je číslo $n^{k+2} - n^k$ deliteľné dvanástimi.
}
\podpis{Vojtech Bálint}

{%%%%%   C-II-2
Dokážte, že pre ľubovoľné čísla $a$, $b$ z~intervalu $\langle 1,\infty)$
platí nerovnosť
$$
(a^2+1)(b^2+1) - (a-1)^2 (b-1)^2 \ge 4
$$
a~zistite, kedy nastane rovnosť.
}
\podpis{Jaromír Šimša}

{%%%%%   C-II-3
Daná je kružnica~$k$ so stredom~$S$. Kružnica~$l$ má väčší polomer ako
kružnica~$k$, prechádza jej stredom a~pretína ju v~bodoch $M$ a~$N$.
Priamka, ktorá prechádza bodom~$N$
a~je rovnobežná s~priamkou~$MS$, vytína na kružniciach
tetivy $NP$ a~$NQ$. Dokážte, že trojuholník $MPQ$ je rovnoramenný.
}
\podpis{Tomáš Jurík}

{%%%%%   C-II-4
Určte všetky dvojice reálnych čísel $x$, $y$, ktoré vyhovujú sústave rovníc
$$
\align
\lfloor x+y\rfloor &=2\,010,\\
\lfloor x\rfloor -y&=p,
\endalign
$$
ak a) $p = 2$, b) $p = 3$.

Symbol $\lfloor x\rfloor$ označuje najväčšie celé číslo, ktoré nie je väčšie
ako dané reálne číslo~$x$ (tzv. dolná celá časť reálneho čísla~$x$).
}
\podpis{Jaroslav Švrček}

{%%%%%   vyberko, den 1, priklad 1
Množina~$M$ obsahuje všetky také prirodzené čísla, ktoré sa dajú napísať v~tvare ${n^2+n}$ pre nejaké prirodzené číslo~$n$. Dokážte, že pre každé prirodzené číslo $k>1$ existujú $a_1, a_2, \dots, a_k, b \in M$ také, že $a_1<a_2<\cdots<a_k<b$ a~$a_1+a_2+\cdots +a_k=b$.
}
\podpis{Ondrej Budáč, Tomáš Kocák:mathlinks, http://www.artofproblemsolving.com/Forum/viewtopic.php?t=77842}

{%%%%%   vyberko, den 1, priklad 2
V~rovine je daný trojuholník $ABC$ a~jeho opísaná kružnica~$k$. Kružnica~$l$ so stredom~$O$ sa dotýka kružnice~$k$ v~bode~$P$ a~úsečky~$BC$ v~bode~$Q$. Vieme, že bod~$P$ leží na oblúku kružnice~$k$ nad tetivou~$BC$, ktorý neobsahuje bod~$A$. Dokážte, že ak platí $|\uhol CAO| = |\uhol BAO|$, potom aj $|\uhol PAO| = |\uhol QAO|$.
}
\podpis{Ondrej Budáč, Tomáš Kocák: mathlinks, http://www.artofproblemsolving.com/Forum/viewtopic.php?t=259941}

{%%%%%   vyberko, den 1, priklad 3
Nájdite všetky funkcie $f\colon \Bbb R \to \Bbb R$ také, ktoré pre všetky reálne $x$, $y$ spĺňajú
$$
f(f(x)+y)=f(x^2-y)+4f(x)y.
$$
}
\podpis{Ondrej Budáč, Tomáš Kocák:mathlinks, http://www.mathlinks.ro/viewtopic.php?t=5429}

{%%%%%   vyberko, den 1, priklad 4
Nech $n$ je nepárne celé číslo väčšie ako $1$ a~nech $c_1,c_2,\dots,c_n$ sú kladné celé čísla. Pre každú permutáciu $a=(a_1,a_2,\dots,a_n)$ čísel $1,2,\dots,n$ definujeme
$$
S(a)=\sum_{i=1}^nc_ia_i.
$$
Dokážte, že existujú dve rôzne permutácie $a$, $b$ čísel $1,2, \dots,n$ také, že rozdiel ${S(a)-S(b)}$ je deliteľný číslom $n!=1\cdot2\cdot\dots\cdot n$.
}
\podpis{Ondrej Budáč, Tomáš Kocák:IMO Shortlist 2001, C2}

{%%%%%   vyberko, den 2, priklad 1
Trojposchodové schodisko so šírkou dva je vyrobené z~12 jednotkových kociek. Určte, pre ktoré $n$ sa dá kocka s~rozmermi $n \times n \times n$ poskladať iba pomocou takýchto schodísk.}
\podpis{Hana Budáčová, Michal Prusák:IMO Shortlist 2000, C2}

{%%%%%   vyberko, den 2, priklad 2
Pre postupnosť $\{a_n\}_{n\ge 1}$ kladných celých čísel platí, že $a_1 = 1$ a pre $n\ge 2$ je
$$
a_n = \begin{cases}
a_{n-1} - n & \text{ak } a_{n-1} > n, \\
a_{n-1} + n & \text{ak } a_{n-1} \le n.
\end{cases}
$$
Nech $S$ je množina takých indexov $n$, že $a_n=2\,010$.
\itemitem{a)}Dokážte, že $S$ obsahuje nekonečne veľa prvkov.
\itemitem{b)}Nájdite najmenšie číslo patriace do~$S$.
}
\podpis{Hana Budáčová, Michal Prusák:IMO Shortlist 1993, 1}

{%%%%%   vyberko, den 2, priklad 3
Trojuholník $ABC$ je vpísaný do kružnice~$k$. Osi jeho vnútorných uhlov pretínajú kružnicu~$k$ druhýkrát v~bodoch  $A'$, $B'$ a~$C'$. Dokážte, že obsah trojuholníka $A'B'C'$ je väčší alebo rovný obsahu trojuholníka $ABC$.
}
\podpis{Hana Budáčová, Michal Prusák:IMO Shortlist 1988}

{%%%%%   vyberko, den 2, priklad 4
Pre štyri kladné celé čísla platí, že druhá mocnina súčtu ľubovoľných dvoch z~nich je deliteľná súčinom zvyšných dvoch. Dokážte, že aspoň tri čísla spomedzi nich musia byť rovnaké.
}
\podpis{Hana Budáčová, Michal Prusák:Rusko 1999, Fifth round, Grade 11, Problem 5}

{%%%%%   vyberko, den 3, priklad 1
Uvažujme prvočíslo~$p$ väčšie ako $5$. Nech $a$, $b$ a~$c$ sú také celé čísla, že rozdiel žiadnych dvoch z~nich nie je deliteľný prvočíslom~$p$. Ďalej $i$, $j$ a~$k$ sú nezáporné celé čísla, pričom výraz $i+j+k$ je deliteľný číslom $p-1$. Navyše platí, že pre ľubovoľné celé číslo~$x$ je výraz
$$
(x-a)(x-b)(x-c)[(x-a)^{i}(x-b)^{j}(x-c)^{k}-1]
$$
deliteľný prvočíslom~$p$. Dokážte, že každé z~čísel $i$, $j$ a~$k$ je deliteľné číslom $p-1$.
}
\podpis{Martin Potočný, Michal Takács:USA TST 2009 c.8 (http://www.mathlinks.ro/viewtopic.php?p=1566058#1566058)}

{%%%%%   vyberko, den 3, priklad 2
Majme ľubovoľný trojuholník $ABC$. Kružnica vpísaná do tohto trojuholníka sa dotýka strany~$AB$ v~bode~$Z$ a~strany~$AC$ v~bode~$Y$. Úsečky $BY$ a~$CZ$ sa pretínajú v~bode~$G$. Označíme $R$ a~$S$ také body v~rovine, že štvoruholníky $BCYR$ a~$BCSZ$ sú rovnobežníky. Dokážte, že veľkosti úsečiek $GR$ a~$GS$ sú rovnaké.
%{\it Zadanie bude zverejnené po IMO 2010.}
}
\podpis{Martin Potočný, Michal Takács:IMO Shortlist 2009, G3}

{%%%%%   vyberko, den 3, priklad 3
Univerzitu navštevuje $2^{n+1}$ študentov, pričom  $n\ge2$ a~žiadni
dvaja študenti nie sú rovnako starí. Na univerzite funguje $2^n$
korešpondenčných seminárov pre talentovanú mládež. Každý seminár má za
vedúcich (organizátorov) niekoľko dobrovoľníkov z~radov
študentov univerzity. Žiadne dva semináre nevznikli v~tom istom čase,
vznikali postupne. Každý študent môže robiť vedúceho vo viacerých
seminároch, ale len ak sa tým neporušuje jedno dôležité pravidlo.
Nemôže existovať spomedzi ním organizovaných seminárov dvojica AKS a~BKS a~dvojica od neho mladších študentov $a$ a~$b$ takých, že $a$ je
mladší od $b$, $a$ robí vedúceho v~AKS, $b$ robí vedúceho v~BKS a~zároveň BKS je novší ako AKS. Dokážte, že aspoň jeden
korešpondenčný seminár trpí nedostatkom vedúcich, teda že ich nemá
viac ako~$4n$.
}
\podpis{Martin Potočný, Michal Takács:IMO Shortlist 2008, C6}

{%%%%%   vyberko, den 4, priklad 1
Nech $a$, $b$, $c$, $d$ sú (v~tomto poradí) dĺžky strán $AB$, $BC$, $CD$, $DA$ dotyčnicového štvoruholníka $ABCD$. Dokážte, že platí
$$
{a^2\over a+b}+{c^2\over c+d}\ge {a+c\over 2}.
$$
}
\podpis{Jakub Beran, Ján Mazák:vlastna Mazo}

{%%%%%   vyberko, den 4, priklad 2
Nájdite všetky prvočísla~$p$, pre ktoré je číslo~$p^3$ deliteľom čísla
$$
1^{3p}+2^{3p}+3^{3p}+\dots+(p-1)^{3p}.
$$
}
\podpis{Jakub Beran, Ján Mazák:Polsko, Oboz naukowy OM 2002, 8/23}

{%%%%%   vyberko, den 4, priklad 3
Konvexný štvoruholník $ABCD$ je vpísaný do kružnice. Priamky $AB$ a~$CD$ sa pretínajú v~bode~$P$,
priamky $AD$ a~$BC$ sa pretínajú v~bode~$Q$. Dokážte, že
$$
|PQ|^2=|PA|\cdot |PB| + |QB|\cdot |QC|.
$$
}
\podpis{Jakub Beran, Ján Mazák:Polsko, Oboz naukowy OM 2002, 5/7}

{%%%%%   vyberko, den 4, priklad 4
Na šachovnici $n\times n$ je obvod (a~nič iné) obtiahnutý červenou.
Dvaja hráči, Aladár a~Boris, hrajú takúto hru: Hráč si v~každom ťahu zvolí políčko šachovnice
a~obtiahne červenou jednu jeho stranu (ktorá ešte nebola červená). Tým vznikne medzi dvoma susednými políčkami nepriechodná hranica.
Hra končí, keď je šachovnica červenými hranicami rozdelená na dve časti.
Hráč, ktorý šachovnicu takto rozdelil, prehráva. Začína Aladár. Určte, ktorý hráč dokáže pre dané $n$ vždy vyhrať a~popíšte jeho víťaznú stratégiu.
}
\podpis{Jakub Beran, Ján Mazák:Ukrainian Mathematical Competition 2008/9, Final Round, Problem 10.7}

{%%%%%   vyberko, den 5, priklad 1
Na stole leží v~jednom dlhom rade vedľa seba 2009 kariet. Každá karta je z~jednej strany zlatá, z~druhej čierna. Na začiatku sú všetky karty otočené zlatou farbou nahor. Dvaja hráči (stojaci na tej istej strane stola) hrajú hru, pričom sa striedajú v~ťahoch. V~každom ťahu hráč zvolí blok tesne za sebou idúcich 50~kariet, z~ktorých karta ležiaca najviac vľavo je otočená zlatou farbou nahor, a~všetkých 50 kariet otočí (teda karty, ktoré boli zhora zlaté, budú čierne, a~naopak). Hráč, ktorý už nemôže urobiť žiadny ťah, prehrá.
\itemitem{a)} Musí hra po konečnom počte ťahov vždy skončiť?
\itemitem{b)} Existuje víťazná stratégia pre začínajúceho hráča?
%{\it Zadanie bude zverejnené po IMO 2010.}
}
\podpis{Peter Novotný, Erika Trojáková:IMO Shortlist 2009, C1}

{%%%%%   vyberko, den 5, priklad 2
Daný je lichobežník $ABCD$ s~rovnobežnými stranami $AB$, $CD$, pričom $|AB|>|CD|$. Body $K$, $L$ ležia postupne vnútri úsečiek $AB$, $CD$ tak, že $|AK|/|KB|=|DL|/|LC|$. Body $P$, $Q$ ležia vnútri úsečky $KL$, pričom
$$
|\uhol APB|=|\uhol BCD|\qquad\text{a}\qquad|\uhol CQD|=|\uhol ABC|.
$$
Dokážte, že body $P$, $Q$, $B$, $C$ ležia na jednej kružnici.
}
\podpis{Peter Novotný, Erika Trojáková:Shortlist 2006, G2}

{%%%%%   vyberko, den 5, priklad 3
Nech $a$, $b$, $c$, $d$ sú kladné reálne čísla také, že
$$
abcd=1\qquad\text{a}\qquad a+b+c+d > \frac ab+\frac bc+\frac cd+\frac da.
$$
Dokážte, že
$$
a+b+c+d < \frac ba+\frac cb+\frac dc+\frac ad.
$$
}
\podpis{Peter Novotný, Erika Trojáková:Shortlist 2008, A5}

{%%%%%   vyberko, den 3, priklad 4
...}
\podpis{...}

{%%%%%   vyberko, den 5, priklad 4
...}
\podpis{...}

{%%%%%   trojstretnutie, priklad 1
Určte všetky trojice $(a,b,c)$ kladných reálnych čísel, ktoré sú riešením sústavy rovníc
$$
\aligned
   a\sqrt{b}-c &= a,\\
   b\sqrt{c}-a &= b,\\
   c\sqrt{a}-b &= c.
\endaligned
$$}
\podpis{Michal Takács}

{%%%%%   trojstretnutie, priklad 2
Uvažujme ľubovoľných 60~bodov v~kruhu s~polomerom~$1$. Dokážte, že na obvode kruhu existuje taký bod, že súčet jeho vzdialeností od všetkých 60~bodov nie je väčší ako $80$.}
\podpis{Jaromír Šimša}

{%%%%%   trojstretnutie, priklad 3
Nech $p$ je prvočíslo. Dokážte, že možno zvoliť $p^3$ políčok na šachovnici s~rozmermi $p^2\times p^2$ tak, že stredy žiadnych štyroch zvolených políčok nie sú vrcholmi obdĺžnika so stranami rovnobežnými s~okrajmi šachovnice.}
\podpis{Bart{\l}omiej Bzd{\accent157 e}ga}

{%%%%%   trojstretnutie, priklad 4
Nájdite najväčšie celé číslo~$k$, pre ktoré je pravdivé nasledujúce tvrdenie:
Daných je ľubovoľných 2010 nedegenerovaných trojuholníkov. V~každom trojuholníku sú jeho strany ofarbené tak, že jedna je modrá, jedna je červená a~jedna biela. Pre každú farbu osobitne usporiadame dĺžky strán. Dostaneme
$$
\begin{align*}
b_1\le b_2 \le \dots \le b_{2010} &\quad \text{pre dĺžky modrých strán},\\
r_1\le r_2 \le \dots \le r_{2010} &\quad \text{pre dĺžky červených strán},\\
w_1\le w_2 \le \dots \le w_{2010} &\quad \text{pre dĺžky bielych strán}.
\end{align*}
$$
Potom existuje aspoň $k$ indexov $j$ takých, že môžeme utvoriť nedegenerovaný trojuholník so stranami dĺžok $b_j$, $r_j$, $w_j$.}
\podpis{Michal Rolínek}

{%%%%%   trojstretnutie, priklad 5
Pre kladné reálne čísla $x$, $y$, $z$ platí $x+y+z\ge 6$. Nájdite najmenšiu možnú hodnotu výrazu
$$
x^2+y^2+z^2+\frac{x}{y^2+z+1}+\frac{y}{z^2+x+1}+\frac{z}{x^2+y+1}.
$$}
\podpis{Ján Mazák}

{%%%%%   trojstretnutie, priklad 6
Nech $ABCD$ je konvexný štvoruholník, pričom
$$
|AB|+|CD|=\sqrt{2}\cdot|AC| \quad \text{a} \quad |BC|+|DA|=\sqrt{2}\cdot |BD|.
$$
Dokážte, že $ABCD$ rovnobežník.}
\podpis{Jaromír Šimša}

{%%%%%   IMO, priklad 1
Určte všetky funkcie $f \colon \Bbb R \to \Bbb R$ také, že rovnosť
$$
  f\bigl(\lfloor x\rfloor y\bigr) = f(x)\bigl\lfloor f(y)\bigr\rfloor
$$
platí pre všetky $x,y \in \Bbb R$.
(Symbol $\lfloor z\rfloor$ označuje najväčšie celé číslo, ktoré nie je väčšie ako~$z$.)}
\podpis{Francúzsko}

{%%%%%   IMO, priklad 2
Označme $I$ stred vpísanej kružnice a~$\Gamma$ opísanú kružnicu trojuholníka $ABC$. Priamka~$AI$ pretína kružnicu $\Gamma$ v~bode~$D$ ($D\ne A$).  Nech $E$ je bod na oblúku $BDC$ a~$F$ bod na strane~$BC$, pričom
$$
|\angle BAF| = |\angle CAE| < \tfrac{1}{2}|\angle BAC|.
$$
Označme $G$ stred úsečky~$IF$. Dokážte, že priamky $DG$ a~$EI$ sa pretínajú na kružnici~$\Gamma$.}
\podpis{Hongkong}

{%%%%%   IMO, priklad 3
Nech $\Bbb N$ je množina všetkých kladných celých čísel. Určte všetky funkcie $g\colon\Bbb N \to \Bbb N$ také, že
$$
\bigl(g(m)+n\bigr)\bigl(m+g(n)\bigr)
$$
je štvorcom celého čísla pre všetky  $m,n \in \Bbb N$.}
\podpis{USA}

{%%%%%   IMO, priklad 4
Nech $P$ je vnútorný bod trojuholníka $ABC$. Priamky $AP$, $BP$, $CP$
pretínajú kružnicu~$\Gamma$ opísanú trojuholníku $ABC$ postupne v~bodoch $K$, $L$, $M$ (rôznych od $A$, $B$, $C$). Dotyčnica ku kružnici~$\Gamma$ v~bode~$C$ pretína priamku $AB$ v~bode~$S$. Predpokladajme, že $|SC|=|SP|$. Dokážte, že $|MK|=|ML|$.}
\podpis{Poľsko}

{%%%%%   IMO, priklad 5
V~každej zo šiestich truhlíc $B_1$, $B_2$, $B_3$, $B_4$, $B_5$, $B_6$
je na začiatku jedna minca. Dovolené sú dva typy operácií:
  \itemitem{I:}Zvolíme neprázdnu truhlicu~$B_j$ pre nejaké $1 \le j \le 5$.
    Odoberieme jednu mincu z~$B_j$ a~pridáme dve mince do $B_{j+1}$.
  \itemitem{II:}Zvolíme neprázdnu truhlicu~$B_k$ pre nejaké $1 \le k \le 4$.
    Odoberieme jednu mincu z~$B_k$ a~vymeníme navzájom obsah truhlíc $B_{k+1}$ a~$B_{k+2}$ (ktoré môžu byť aj prázdne).
\noindent
Zistite, či existuje konečná postupnosť operácií taká, že po jej vykonaní budú truhlice
$B_1$, $B_2$, $B_3$, $B_4$, $B_5$ prázdne a~truhlica~$B_6$ bude obsahovať presne
$2010^{2010^{2010}}$ mincí.  (Platí $a^{b^{c}}=a^{(b^{c})}$.)}
\podpis{Holandsko}

{%%%%%   IMO, priklad 6
Nech $a_1,a_2,a_3,\dots$ je postupnosť kladných reálnych čísel. Predpokladajme, že
existuje kladné celé číslo~$s$ také, že pre všetky $n>s$ platí
$$
  a_n=\max\{a_k+a_{n-k} \,\,;\,\, 1\le k\leq n-1\}.
$$
Dokážte, že existujú kladné celé čísla $l$ a~$N$ také, že $l\le s$ a~pre všetky $n\ge N$ platí $a_n=a_l+a_{n-l}$.}
\podpis{Irán}

{%%%%%   MEMO, priklad 1
Nájdite všetky funkcie $f\colon\Bbb R\to\Bbb R$ také, že pre všetky $x,y\in\Bbb R$ platí
$$
f(x+y) + f(x)f(y) = f(xy) + (y+1)f(x) + (x+1)f(y).
$$}
\podpis{Česká rep., Pavel Calábek}

{%%%%%   MEMO, priklad 2
Na tabuli sú napísané všetky kladné delitele celého kladného čísla~$N$. Dvaja hráči $A$ a~$B$ hrajú nasledovnú hru, pričom sa pravidelne striedajú v~ťahoch: V~prvom ťahu hráč~$A$ zmaže číslo~$N$. Ak posledné zmazané číslo je~$d$, potom hráč, ktorý je na ťahu, zmaže deliteľa čísla~$d$ alebo násobok čísla~$d$. Hráč, ktorý nemôže urobiť ťah, prehráva.
Určte všetky čísla~$N$, pre ktoré hráč~$A$ môže vyhrať bez ohľadu na ťahy hráča~$B$.}
\podpis{Poľsko}

{%%%%%   MEMO, priklad 3
Je daný tetivový štvoruholník $ABCD$ a~na jeho uhlopriečke~$AC$ bod~$E$ taký, že $|AD|=|AE|$ a~$|CB|=|CE|$.
Nech $M$ je stred kružnice~$k$ opísanej trojuholníku $BDE$.
Kružnica~$k$ pretína priamku~$AC$ v~bodoch $E$ a~$F$.
Dokážte, že priamky $FM$, $AD$ a~$BC$ sa pretínajú v~jednom bode.}
\podpis{Švajčiarsko}

{%%%%%   MEMO, priklad 4
Nájdite všetky kladné celé čísla~$n$, ktoré vyhovujú obom nasledujúcim podmienkam:
\itemitem{(i)}číslo~$n$ má aspoň štyri kladné delitele;
\itemitem{(ii)}ak $a$ a~$b$ sú delitele čísla~$n$, pre ktoré platí $1 < a < b < n$, potom číslo $b - a$ tiež delí~$n$.
\endgraf
}
\podpis{Slovinsko}

{%%%%%   MEMO, priklad t1
Sú dané tri rastúce postupnosti
$$
a_1,\ a_2,\ a_3,\ \dots,\qquad b_1,\ b_2,\ b_3,\ \dots,\qquad c_1,\ c_2,\ c_3,\ \dots
$$
celých kladných čísel. Každé celé kladné číslo je členom práve jednej z~týchto postupností. Pre každé celé kladné číslo~$n$ sú splnené podmienky:
\itemitem{(i)}$c_{a_n}=b_n+1$;
\itemitem{(ii)}$a_{n+1}>b_n$;
\itemitem{(iii)}číslo $c_{n+1}c_n-(n+1)c_{n+1}-nc_n$ je párne.
\endgraf\noindent
Nájdite $a_{2010}$, $b_{2010}$ a $c_{2010}$.}
\podpis{Litva}

{%%%%%   MEMO, priklad t2
Pre každé celé číslo $n\ge2$ určte najväčšie možné reálne číslo~$C_n$ také, že
pre všetky kladné reálne čísla $a_1, \dots, a_n$ platí
$$
\postdisplaypenalty=10000
\frac{a_1^2+\cdots+a_n^2}n \ge \left(\frac{a_1+\cdots+a_n}n\right)^2 + C_n\cdot(a_1-a_n)^2.
$$}
\podpis{Švajčiarsko}

{%%%%%   MEMO, priklad t3
V~každom vrchole pravidelného $n$-uholníka je veža. V~tom istom okamihu každá veža
vystrelí na jednu z~dvoch susedných veží a~zasiahne ju. {\it Výsledkom streľby\/} nazveme množinu všetkých zasiahnutých veží, pričom nerozlišujeme, či bola veža zasiahnutá raz alebo dvakrát.
Označme $P(n)$ počet všetkých možných výsledkov streľby (pre dané~$n$).
Dokážte, že pre každé celé $k\ge3$ sú čísla $P(k)$ a~$P(k+1)$ nesúdeliteľné.}
\podpis{Česká rep., Martin Panák}

{%%%%%   MEMO, priklad t4
Nech $n$ je celé kladné číslo. Štvorec $ABCD$ je rozdelený na $n^2$ jednotkových štvorcov. Každý z~týchto štvorcov je ďalej rozdelený na dva trojuholníky uhlopriečkou rovnobežnou s~úsečkou~$BD$. Niektoré z~vrcholov malých štvorcov sú zafarbené na červeno tak, že každý z~$2n^2$ vytvorených trojuholníkov má aspoň jeden červený vrchol. Nájdite najmenší možný počet červených vrcholov.}
\podpis{Slovinsko}

{%%%%%   MEMO, priklad t5
Kružnica vpísaná trojuholníku $ABC$ sa dotýka strán $BC$, $CA$ a~$AB$ postupne v~bodoch $D$, $E$ a~$F$.
Nech $K$ je bod súmerný s~bodom~$D$ podľa stredu vpísanej kružnice. Priamky $DE$ a~$FK$ sa pretínajú v~bode~$S$. Dokážte, že priamka~$AS$ je rovnobežná s~$BC$.}
\podpis{Poľsko}

{%%%%%   MEMO, priklad t6
Nech $A$, $B$, $C$, $D$, $E$ sú také body, že $ABCD$ je tetivový štvoruholník a~$ABDE$ je rovnobežník. Uhlopriečky $AC$ a~$BD$ sa pretínajú v~bode~$S$ a~polpriamky $AB$ a~$DC$ v~bode~$F$. Dokážte, že $|\uhol AFS|=|\uhol ECD|$.}
\podpis{Chorvátsko}

{%%%%%   MEMO, priklad t7
Pre celé nezáporné číslo~$n$ definujme $a_n$ ako číslo, ktorého dekadický zápis má tvar
$$
1\underbrace{0\dots0}_{n}2\underbrace{0\dots0}_{n}2\underbrace{0\dots0}_{n}1.
$$
Dokážte, že $a_n/3$ sa dá vždy vyjadriť ako súčet tretích mocnín dvoch celých kladných čísel, ale nikdy sa nedá vyjadriť ako súčet druhých mocnín dvoch celých čísel.}
\podpis{Švajčiarsko}

{%%%%%   MEMO, priklad t8
Je dané celé kladné číslo~$n$, ktoré nie je mocninou čísla~$2$. Ukážte, že existuje celé kladné číslo~$m$ s~nasledujúcimi dvoma vlastnosťami:
\itemitem{(i)}$m$ je súčinom dvoch po sebe idúcich celých kladných čísel;
\itemitem{(ii)}dekadický zápis čísla~$m$ pozostáva z~dvoch identických blokov $n$~číslic.
\endgraf
}
\podpis{Poľsko} 