{%%%%%   A-I-1
Vypíšme, aké hodnoty môžu nadobúdať prvé členy uvedenej postupnosti.
Dostaneme
$$
\gather
x_1 = 1, \quad x_2\in \{-1, 1\}, \quad x_3\in \{-2, 0, 2\}, \quad
x_4\in \{-4, -2, 0, 2, 4\}, \\
x_5\in \{-8, -6, -4, -2, 0, 2, 4, 6, 8\} .
\endgather
$$
Všimnime si, že všetky členy, ktoré sme vypísali, sú celé čísla.
Ďalej je zrejmé, že pre $i > 2$ je každý člen~$x_i$ párne číslo.
(Ďalšie pozorovanie je, že ak nájdeme postupnosť, pre ktorú
$x_i = a$ pre nejaké číslo~$a$ a~dané $i>1$, tak existuje
aj postupnosť, pre ktorú $x_i = -a$.)

Zistime, akú najväčšiu a~akú najmenšiu hodnotu môže nadobúdať
číslo~$x_n$ (v~závislosti od~$n$). Označme $a_i$ najväčšiu
hodnotu, ktorú môže nadobúdať člen~$x_i$. Pretože postupností
dĺžky~$i$ spĺňajúcich dané vlastnosti je len konečný počet,
maximum~$a_i$ existuje a~je zrejme kladné. K~číslu~$a_i$ musí pre
každé $i>1$ existovať postupnosť $x_1, \dots, x_{i - 1}$, pre
ktorú platí
$$
a_i=\pm x_{i-1}\pm\dots\pm x_1\le|x_{i-1}|+\dots+|x_1|\le
a_{i-1}+\dots+a_1.         \tag1
$$
Vieme, že $a_1 = 1$, $a_2 = 1$, $a_3 = 2$. Pomocou predchádzajúceho
vzťahu dokážme, že $a_i=2^{i - 2}$ pre každé~$i > 1$.
Dôkaz urobíme matematickou indukciou vzhľadom na~$i$.

$1^0$ Tvrdenie platí pre $i= 1$ ($a_1=1$) a~$i= 2$ ($a_2=1$).

$2^0$ Predpokladajme, že tvrdenie platí pre každé $k$, $2\le k~\le i-1$,
a~dokážme, že tvrdenie platí aj pre $k = i$. Z~odhadu $(1)$,
indukčného predpokladu a~vzorca pre súčet geometrického radu dostaneme
$$
a_i\le a_{i-1}+\dots+a_1=2^{i-3}+\dots+2+1+1=
\frac{2^{i-2}-1}{2-1}+1=2^{i-2}.
$$
Uvažujme postupnosť $x_1=1$ a~$x_i=x_{i-1}+\dots+x_1$
pre každé $i>1$. V~tomto prípade bude podľa predchádzajúceho platiť
$x_i=2^{i-2}$, takže $a_i=2^{i-2}$ pre každé $i>1$.

Podobne dokážeme, že najmenšia hodnota, akú môže~$x_n$ nadobudnúť, je~${-2}^{n-2}$.

\smallskip
Zistili sme, že pre každé $n>1$ leží člen~$x_n$ ľubovoľnej
uvažovanej postupnosti v~množine $\{-2^{n-2}, -2^{n-2} + 2,
-2^{n-2} + 4, \dots ,2^{n - 2}\}$, ktorú označíme $\mm M_n$.
Dokážme nakoniec, že $x_n$ môže pre $n > 1$ nadobúdať ľubovoľnú
hodnotu z~množiny~$\mm M_n$.

Voľme znamienka nasledujúcim spôsobom: $x_i = x_{i - 1} + \dots +
x_{1}$ pre $i < n$. Pre takú postupnosť platí $x_i = 2^{i-2}$
pre $1< i~< n$. Dokážme, že v~rovnosti
$x_n={\pm2}^{n-3}\pm2^{n-2}\pm\dots\pm1\pm1$ možno znamienka vybrať
tak, aby sa hodnota~$x_n$ rovnala ľubovoľne zvolenému číslu 
z~množiny~$\mm M_n$.
Dôkaz urobíme opäť matematickou indukciou.

$1^0$ Tvrdenie platí pre $n = 2$ ($-1=-x_1$ a~$1=+x_1$, lebo
$x_1=1$) a~$n=3$ ($-2=-1-1$, $0=-1+1$, $2=1+1$).

$2^0$ Predpokladajme, že tvrdenie platí pre $k \le n-1$, kde $n\ge4$.
Dokážme tvrdenie pre $k = n$. Zvoľme ľubovoľné číslo~$a$ z~množiny~$\mm M_n$.
Dokážeme, že existuje taká voľba znamienok $+$ a~$-$,
že $a = \pm 2^{n - 3} \pm 2^{n - 4} \pm \dots \pm 1 \pm 1$.
Rozoberieme dve možnosti.

1. {$a \ge 0$}. Pretože $a \in\mm M_n$, je $a - 2^{n - 3}$ párne
celé číslo z~intervalu $\langle -2^{n - 3}, 2^{n - 3}\rangle$, 
a~teda $a - 2^{n - 3} \in\mm M_{n - 1}$. Z~indukčného predpokladu
vyplýva, že existuje voľba znamienok $+$ a~$-$ taká, že $a -
2^{n-3} = \pm 2^{n - 4} \pm 2^{n - 5} \pm \dots \pm 1 \pm 1$.
Potom $a = 2^{n - 3} \pm 2^{n - 4} \pm 2^{n - 5} \pm \dots \pm 1
\pm 1$, čo sme chceli dokázať.

2. {$a < 0$}. Podobne ako v~predchádzajúcom prípade dokážeme, že
$a$ sa dá napísať v~tvare $a = -2^{n - 3} \pm 2^{n - 4} \pm
2^{n-5} \pm \dots \pm 1 \pm 1$.

Tým sme dokázali, že všetky hodnoty~$x_n$ tvoria práve množinu~$\mm M_n$.}

{%%%%%   A-I-2
\fontplace
\tpoint A;
\tpoint B;
\tpoint C;
\tpoint\xy-.7,0 K;
\tpoint\xy.7,0 L;
\bpoint\xy.5,0 P;
\bpoint\xy-.5,0 Q;
\rBpoint\xy1,0 S_{\!A};
\bpoint S_B;
\lBpoint S_C;
\lpoint p;
[1] \hfil\Obr

\fontplace
\tpoint A;
\tpoint B;
\tpoint C;
\tpoint K;
\tpoint L;
\rpoint M;
\rpoint P;
\bpoint\xy.5,0 Q;
\lBpoint S_C;
\rbpoint p;
[2] \hfil\Obr

Keď zvolíme veľkosť $r_B>0$ polomeru kružnice~$k_B$,
sú už tým obe ďalšie kružnice $k_A$, $k_C$ určené. K~ich
zostrojeniu využijeme základné vlastnosti dotyčníc kružníc.
Predpokladajme, že kružnice $k_A$, $k_B$, $k_C$ majú vlastnosti
popísané v~zadaní. Ak označíme napr.~$K$ priesečník vnútornej
spoločnej dotyčnice kružníc $k_A$ a~$k_B$ (v~bode~$P$ ich vonkajšieho
dotyku) s~priamkou~$p$, ktorá je spoločnou vonkajšou dotyčnicou všetkých
troch kružníc, musia byť $|KA| = |KP|$ a~$|KB| = |KP|$ (\obr). To
znamená, že bod~$K$ je stredom úsečky~$AB$ a~súčasne bod~$P$ leží
na Tálesovej kružnici nad priemerom~$AB$. Keď poznáme bod~$P$, ľahko
už zostrojíme kružnicu~$k_A$, o~ktorej vieme, že sa dotýka priamky~$p$
v~bode~$A$. Analogicky zostrojíme kružnicu~$k_C$.
\midinsert
\inspicture{}
\endinsert
Máme zistiť, pre ktoré hodnoty~$r_B$ je trojuholník~$BPQ$ rovnoramenný.
Pretože body dotyku $P$, $Q$ kružnice~$k_B$ s~oboma susednými
kružnicami ležia vnútri opačných polrovín určených priamkou~$BS_B$,
sú oba uhly $BPQ$ a~$BQP$ ostré (príslušné stredové uhly sú
menšie ako $180\st$). Ak teda náhodou vyjde trojuholník~$PBQ$ tupouhlý,
môže byť rovnoramenný, len keď $|BP|=|BQ|$. V~takom prípade je
ale zo súmernosti zrejmé, že $|AB|=|BC|$, \tj. $h=1$. Trojuholník~$BPQ$
je potom rovnoramenný pre každé $r_B>0$.

Predpokladajme ďalej, že $h\ne1$. V~takom prípade môžeme
predpokladať, že trojuholník~$BPQ$ je ostrouhlý (inak podľa predchádzajúceho
odstavca nemôže byť rovnoramenný). Ak je rovnoramenný,
buď $|PQ| = |BQ|$, alebo $|PQ|=|BP|$. Predpokladajme,
že napr.~$|PQ|=|BQ|$ (ako ukážeme neskôr, druhý prípad
možno riešiť využitím súmernosti).
Trojuholník~$BPQ$ je súmerný podľa spojnice~$S_BS_C$ stredov oboch
kružníc, ktorá prechádza bodom dotyku~$Q$ oboch kružníc 
a~priesečníkom~$K$ dotyčníc $KB$, $KP$. Označme ešte $L$ priesečník
spoločnej vnútornej dotyčnice kružníc $k_C$ a~$k_B$ s~priamkou~$p$ ($L$~je
stred úsečky~$BC$, \obr) a~$M$ priesečník oboch dotyčníc $KP$ a~$LQ$
\inspicture{}
(ten je obrazom bodu~$L$ v~uvedenej osovej súmernosti).
Trojuholník~$KLM$ je teda rovnoramenný so stranami
$|KL|=|KM|=(1+h)/2$, $|ML|=2|LQ|=h$, jeho obvod je $1+2h$.
Veľkosť polomeru~$r_B$ vpísanej kružnice vypočítame pomocou obsahu.
Pre obsah~$S$ trojuholníka~$KLM$ platí
$$
 S~= \frac{1}{2} h \sqrt{\left(\frac{1 + h}{2}\right)^{\!\!2} -
 \left(\frac{h}{2}\right)^{\!\!2}} = \frac{1}{4}h\sqrt{1 + 2h}
$$
a~súčasne
$$
S=\frac12r_B(1+2h).
$$
Odtiaľ vychádza
$$
r_B = \frac{h}{2\sqrt{2h+1}}. \tag1
$$
Ak je naopak $r_B$ dané vzťahom~$(1)$, môžeme zostrojiť
rovnoramenný trojuholník~$KLM$ s~ramenami $KL$ a~$KM$ dĺžky
$(1+h)/2$ a~základňou~$ML$, $|ML|=h$, pričom jeho vpísaná
kružnica~$k_B$ sa bude dotýkať ramena~$KL$ v~bode~$B$. Označme
$P$, $Q$ postupne body dotyku kružnice~$k_B$ so stranami $KM$ 
a~$LM$. Pretože $K$ je stred úsečky~$AB$, platí $|KA| = |KB| = |KP|$.
To znamená, že kružnica~$k_A$ dotýkajúca sa priamky~$p$
v~bode~$A$ a~prechádzajúca bodom~$P$ sa bude dotýkať kružnice~$k_B$
v~bode~$P$. Analogicky zostrojíme aj kružnicu~$k_C$ dotýkajúcu sa
priamky~$p$ v~bode~$C$ a~prechádzajúcu bodom~$Q$. Zo súmernosti
trojuholníka~$KLM$ podľa priamky~$KQ$ vyplýva, že $|PQ| = |BQ|$.
Tým je prvý prípad vyriešený.

V~prípade rovnosti $|PQ|=|BP|$ môžeme postupovať úplne rovnako.
Jednoduchšie ale bude, keď zmeníme mierku pôvodného obrázku
v~pomeru~$1:h$, takže bude $|AB|=h'={1}/{h}$, $|BC|=1$. Keď
naviac vymeníme označenie bodov $A$ a~$C$, tak sa z~rovnosti $|BP|
= |PQ|$ stane rovnosť $|BQ|=|PQ|$. Podľa predchádzajúceho potom pre
veľkosť polomeru $r'_B=(1/h)r_B$ dostaneme
$$
\gather
\frac{1}{h}r_B =r'_B=\frac{h'}{2\sqrt{2h'+1}}=
\frac{\frac{1}{h}}{2\sqrt{2\frac{1}{h} + 1}},  \\
\intext{\tj.}
r_B = \frac{h}{2\sqrt{2h + h^2}} .             \tag2
\endgather
$$
Alebo sme mohli riešiť úlohu o~niečo všeobecnejšie za predpokladu
$|AB|=a$, $|BC|=b$, potom by sme namiesto vzťahu~$(1)$ dostali
$$
r_B = \frac{b\sqrt{a^2+2ab}}{2(a+2b)} .    \tag1$'$
$$
Pre $a=1$, $b=h$ vyjde za predpokladu $|PQ| = |BQ|$ pôvodný
vzťah~$(1)$, zatiaľ čo pre $|PQ| = |BP|$ vymeníme označenie bodov $A$,
$C$ (a~tým aj bodov $P$, $Q$) a~do vzťahu~$(1')$ dosadíme $a=h$,
$b=1$. Dostaneme tak vzťah~$(2)$.

\zaver
Pre $h=1$ je trojuholník~$BPQ$ rovnoramenný pre ľubovoľné
$r_B>0$. Pre $h\ne1$ je trojuholník~$BPQ$ rovnoramenný pre $r_B$ určené
vzťahom~$(1)$ ($|PQ|=|BQ|$) alebo pre $r_B$ určené vzťahom~$(2)$
($|PQ|=|BP|$).}

{%%%%%   A-I-3
Najprv ukážeme, že žiadna hodnota skúmaného výrazu~$V$ nie je
menšia ako~$1$. Použijeme nerovnosti medzi
aritmetickým a~geometrickým priemerom ($x + y \ge 2
\sqrt{\mathstrut xy}$) pre všetky dvojice kladných čísel $x$,
$y$ z~množiny $\{a^4,b^4,c^4\}$.
$$
\align
V=&\frac{a^4+b^4+c^4}{a^2b^2+a^2c^2+b^2c^2}=
\frac{1}{2}\cdot
\frac{(a^4+b^4)+(b^4+c^4)+(c^4+a^4)}{a^2b^2+a^2c^2+b^2c^2}\ge\\
\ge&\frac{1}{2}\cdot
\frac{2\cdot(a^2b^2+b^2c^2+c^2a^2)}{a^2b^2+a^2c^2+b^2c^2}=1.
\endalign
$$
Teraz ukážeme, že každá hodnota~$V$ je menšia ako~$2$. Z~Herónovho vzorca pre obsah~$S$ trojuholníka so stranami $a$, $b$, $c$
vieme, že
$$
S^2 = s(s- a)(s - b)(s - c),
$$
kde $s = (a + b +c)/2$.
Po dosadení za~$s$ a~roznásobení dostaneme
$$
0 < 16S^2 = - a^4 - b^4 - c^4 + 2a^2b^2 + 2b^2c^2 + 2c^2a^2.
$$
Odtiaľ
$$
a^4+b^4+c^4<2a^2b^2+2b^2c^2+2c^2a^2,\quad
\text{\tj.}\quad
\frac{a^4+b^4+c^4}{a^2b^2+b^2c^2+c^2a^2}<2.
$$
Zhrňme výsledok úvah prvých dvoch odstavcov. Zistili sme, že
$V\in\left<1,2\right)$. Ukážme, že všetky hodnoty~$V$ zaplnia
celý interval $V\in\left<1,2\right)$. Zvolíme ľubovoľnú hodnotu
$k\in\left<1,2\right)$ a~nájdeme trojuholník, pre ktorý má výraz~$V$
hodnotu~$k$. Uvažujme trojuholník so stranami $a$, $1$, $1$,
ktorý podľa trojuholníkovej nerovnosti existuje práve vtedy, keď
$0<a<2$. Zistíme, pre ktoré~$a$ platí $V = k$, preto vyriešime
rovnicu
$$
\frac{a^4 + 2}{2a^2 + 1} = k~\tag{1}
$$
s~neznámou~$a$. Po substitúcii $a^2 = b$ dostaneme kvadratickú
rovnicu $b^2 - 2kb + 2 - k~= 0$ s~neznámou~$b$. Jej
diskriminant je rovný $D = 4k^2 - 4(2 - k) = 4(k^2 + k~- 2)$. Na
to, aby mala rovnica riešenie, musí byť diskriminant nezáporný,
teda musí platiť $k^2 + k~- 2 \ge 0$. Táto nerovnosť je splnená
pre $k \in \left(-\infty , -2\right> \cup \left<1,\infty\right)$,
teda aj pre uvažované $k\in\<1,2\)$. Potrebujeme ešte dokázať, že
skúmaná rovnica má aspoň jeden koreň~$b$ v~intervale $(0,4)$,
lebo $b = a^2$ a~$a\in(0,2)$. Všimnime si, že pre oba korene
$b_{1,2}$ platí
$$
\align
b_{1,2} =\frac{2k \pm 2\sqrt{k^2 + k~- 2}}{2} =&
            k\pm \sqrt{k^2 + k~- 2} \le \\
\le&k + \sqrt{k^2 + k~- 2}<k + \sqrt{k^2} = 2k < 4,
\endalign
$$
pričom sme využili nerovnosť $k < 2$. Na druhej strane pre koreň~$b_1$
(so znamienkom~$+$ pred~$\sqrt D$) platí
$$
b_{1} = k~+ \sqrt{k^2 + k~- 2} \ge k~> 0 .
$$
Tým sme ukázali, že $0 < b_1 < 4$. Existuje teda číslo $a =
\sqrt{b_1}$ spĺňajúce rovnicu~$(1)$.

\ineriesenie
Opakovaným dosadzovaním dĺžok strán konkrétnych trojuholníkov
dospejeme k~hypotéze, že $1\le V<2$. Dokazujme najprv dolný odhad
$1\le V$, ktorý je ekvivalentný s~nerovnosťou
$$
a^2b^2+a^2c^2+b^2c^2\le a^4+b^4+c^4.
$$
Je to bikvadratická nerovnica s~premennou~$a$, takže po
substitúcii $a^2= t$ dostaneme kvadratickú nerovnicu
$$
  0 \le t^2  - t(b^2 + c^2) + b^4 + c^4 - b^2c^2 .    \tag2
$$
Jej diskriminant je $D = (b^2 + c^2)^2 - 4(b^4 + c^4 - b^2c^2) =
-3(b^4 + c^4 - 2b^2c^2) = -3 (b^2 - c^2)^2 \le 0$. Pretože naviac
je koeficient pri $t^2$ na pravej strane~$(2)$ kladný, je
nerovnica~$(2)$ splnená pre všetky reálne čísla $b$, $c$ a~$t$.
Tým je nerovnosť $V\ge1$ dokázaná.

Prejdeme k~nerovnosti $V<2$. Danú nerovnicu prenásobme kladným
menovateľom, dostaneme
$$
 2(a^2b^2 + a^2c^2 + b^2c^2)  >   a^4 + b^4 + c^4 .
$$
Je to opäť bikvadratická nerovnica s~premennou~$a$. Po
substitúcii $t = a^2$ prejde nerovnica do tvaru $t^2 - 2t(b^2 +
c^2) + b^4 + c^4 - 2b^2c^2 < 0$. Jej diskriminant je $D = 4(b^2
+ c^2)^2 - 4(b^4 + c^4 - 2b^2c^2) = 16b^2c^2$. Pretože koeficient
pri $t^2$ je kladný, je riešením tejto nerovnice interval určený
nerovnosťami
$$
\frac{2(b^2 + c^2) - \sqrt{D}}{2}<t<
 \frac{2(b^2 + c^2) -\sqrt{D}}{2},
$$
čiže
$$
(b - c)^2<t<(b + c)^2 .
$$
Tieto nerovnosti platia, pretože $t=a^2$ a~$|b-c|<a<b+c$ podľa
trojuholníkových nerovností. Tým je nerovnosť $V<2$ dokázaná.

Že hodnoty~$V$ zaplnia celý interval $\<1,2\)$, dokážeme rovnako
ako v~prvom riešení.

\smallskip
{\it Iný dôkaz nerovnosti $V<2$.}
Vyjdeme z~trojuholníkovej nerovnosti $|a-b|<c < a+b$. Po umocnení
na druhú a~následnej úprave dostaneme $-2ab<c^2 - a^2 - b^2 <
2ab$, \tj. $|c^2 - a^2 - b^2|< 2ab$. Po ďalšom umocnení na druhú
dostaneme
$$
c^4 + b^4 + a^4 - 2c^2a^2 - 2c^2b^2 + 2a^2b^2 < 4a^2b^2 ,
$$
čiže
$$
c^4 + b^4 + a^4 < 2c^2a^2 + 2c^2b^2 + 2a^2b^2 .
$$
Odtiaľ už vyplýva, že $V < 2$.

\medskip
{\it Poznámky\/}.
Všimnime si, že podobné trojuholníky majú rovnakú hodnotu
výrazu~$V$. Skutočne, ak $a$, $b$, $c$ sú strany
trojuholníka, sú $ka$, $kb$, $kc$ pre každé reálne $k>0$
stranami podobného trojuholníka a~platí
$$
 \frac{(ka)^4 + (kb)^4 + (kc)^4}{(ka)^2(kb)^2 + (ka)^2(kc)^2 +
 (kb)^2(kc)^2} =  \frac{a^4 + b^4 + c^4}{a^2b^2 + a^2c^2 + b^2c^2} .
$$
To znamená, že bez ujmy na všeobecnosti môžeme predpokladať, že $c = 1$.
Máme teda skúmať obor hodnôt výrazu
$$
   \frac{a^4 + b^4 + 1}{a^2b^2 + a^2 + b^2}
$$
za predpokladu $|a-b|<1<a+b$, čo zjednodušuje a~sprehľadňuje
výpočty.

\smallskip
V~druhej časti riešenia sme mali zistiť obor hodnôt funkcie
$f(a) = (a^4 + 2)/(2a^2 + 1)$. Zrejme $f(1)=1$ a~$f(0)=2$.
Zo spojitosti funkcie~$f$ vyplýva, že na intervale $\(0,1\>$
nadobúda všetky hodnoty z~intervalu $\<1,2\)$.

\smallskip
Dôkladným rozborom uvedených dôkazov zistíme, že nerovnosť
$V \ge 1$ platí pre všetky reálne čísla $a$, $b$, $c$, z~ktorých
aspoň dve sú nenulové.}

{%%%%%   A-I-4
Pretože $(abc)_z$ je číslo $az^2+bz+c$, máme zistiť, kedy všeobecne
platí ekvivalencia $n\deli c + 3b - 4a$, práve vtedy, keď $n \deli
az^2+bz + c$. V~nej sú $a$, $b$, $c$ ľubovoľné číslice pri
základe~$z$, \tj. čísla z~množiny $\{0,1,\dots,z-1\}$. Všimnime
si, že $z-1\ge4$, lebo predpokladáme, že $z\ge5$.

Keď zvolíme $a = b = c = 1$, dostaneme, že $n \deli 0$ práve vtedy, keď
$n\deli z^2 + z~+ 1$. Pretože nula je deliteľná každým celým
číslom, musí platiť $n \deli z^2 + z+ 1$. Keď zvolíme $a = 1$,
$b=0$ a~$c = 4$, dostaneme, že $n \deli 0$ práve vtedy, keď $n \deli
z^2 +4$. Podobnou úvahou ako vyššie zistíme, že $n \deli z^2+4$.
Ak nejaké číslo delí dve čísla, musí deliť aj ich najväčší
spoločný deliteľ, teda $n \deli \nsd(z^2 + 4, z^2 + z+ 1)$.
Tento spoločný deliteľ nájdeme pomocou Euklidovho algoritmu.
$$
\align
\nsd(&z^2+4,z^2+z+1)=\\
=&\nsd\bigl(z^2+4,z^2+z+1-(z^2+4)\bigr)=\nsd(z^2+4,z-3)=\\
=&\nsd\bigl(z^2+4-z(z-3),z-3\bigr)=\nsd(4+3z,z-3)=\\
=&\nsd\bigl(4+3z-3(z-3),z-3\bigr)=\nsd(13,z-3).
\endalign
$$
Zistili sme, že $n \deli 13$. Pretože $n > 1$, nutne $n=13$.
Ak má niektoré~$n$ požadovanú vlastnosť, je to nutne číslo $n=13$.
Dokážme, že číslo~$13$ skutočne danú vlastnosť má. Odvodená nutná
podmienka $n\deli\nsd(13,z-3)$ je pre $n=13$ splnená napr.~pre
$z=16$. Daná ekvivalencia má potom tvar $13 \deli c + 3b - 4a$
práve vtedy, keď $13\deli a\cdot 16^2 + b\cdot 16 + c$. Dokážeme
silnejšiu vlastnosť, že totiž čísla $a\cdot 16^2+b\cdot 16 + c$ 
a~$c + 3b - 4a$ dávajú po delení trinástimi rovnaký zvyšok, teda že
ich rozdiel je deliteľný trinástimi.
$$
(a\cdot 16^2+b\cdot16 + c)-(c + 3b - 4a)=260a+13b=13(20+b).
$$
Úloha má jediné riešenie $n = 13$.

\poznamka
Podobne ako v~závere riešenia môžeme dokázať, že uvedené kritérium
deliteľnosti pre $n = 13$ platí aj v~ľubovoľnej číselnej sústave so
základom $z=13k + 3$.}

{%%%%%   A-I-5
\fontplace
\tpoint A;
\tpoint B;
\blpoint C;
\bpoint D;
\rtpoint K;
\rpoint\toleft2pt L;
\lbpoint M;
\rpoint A_0;
\bpoint\xy-1,.7 C_0;
[3] \hfil\Obr

\fontplace
\tpoint A;
\tpoint B;
\blpoint C;
\bpoint D;
\rtpoint K;
\rpoint\xy-1,.5 L;
\lbpoint M;
\rpoint\down\unit P;
\rpoint A_0;
\bpoint\xy-1,.7 C_0;
\tpoint\xy2,0 C_1;
\rpoint D_1;
[4] \hfil\Obr

\fontplace
\bpoint C_1;
\rtpoint K;
\rpoint\toleft2pt L;
\rpoint\xy-.7,-.7 M;
\bpoint\xy-1,.7 C_0;
\tpoint C_0'; \bpoint C_1';
[5] \hfil\Obr

\fontplace
\rtpoint K;
\rBpoint L;
\rpoint\xy-.7,-.9 M;
\tpoint\xy1.5,0 C_0;
\tpoint\xy.5,0 C_0';
[6] \hfil\Obr

Ak je $ABCD$ ľubovoľný štvorec, ktorý spĺňa podmienky úlohy,
bude rovnakým podmienkam vyhovovať aj štvorec, ktorý dostaneme
osovou súmernosťou podľa priamky~$MK$. Hľadaná množina bude teda
osovo súmerná podľa tejto priamky a~nám stačí určiť tú jej
časť, ktorá leží v~jednej z~oboch polrovín s~hraničnou priamkou~$MK$.

Okrem ľubovoľného štvorca~$ABCD$, ktorý spĺňa podmienky úlohy,
uvažujme štvorec~$A_0B_0C_0D_0$ s~uhlopriečkou $B_0D_0=KM$ ($B_0
\equiv K$, $D_0 \equiv M$), pričom vrchol~$C_0$ leží v~rovnakej
polrovine ohraničenej priamkou~$KM$ ako vrchol~$C$ štvorca~$ABCD$ (\obr).
(Vrchol~$C_0$ zrejme rovnako patrí do hľadanej množiny.)
\midinsert
\line{\qquad\inspicture-!\hss \inspicture-!\quad}
\endinsert
Pretože trojuholníky $KLB$ a~$MLD$ sú podobné podľa vety~$uu$, delí bod~$L$ uhlopriečky oboch štvorcov v~rovnakom pomere
$$
|BL|:|LD| = |KL|:|LM| = \text{konšt.}
$$
Veľkosť uhla~$LCD$ ($|\uh LCD|=|\uh LC_0M|$) je určená polohou
bodu~$L$ na úsečke~$MK$, má teda konštantnú veľkosť, takže bod~$C$
leží na rovnakom oblúku~$\gamma$ kružnice opísanej trojuholníku~$LC_0M$
nad tetivou~$LM$ ako bod~$C_0$. Naviac kružnica opísaná
trojuholníku~$A_0KL$ je zhodná s~kružnicou opísanou trojuholníku~$C_0ML$, pretože
v~jednej z~nich vidno tetivu~$A_0L$ z~bodu~$K$ pod uhlom~$45\st$
a~v~druhej tetivu~$C_0L$ zhodnej dĺžky pod rovnakým uhlom
z~bodu~$M$.

Pretože bod~$M$ leží na strane~$CD$, zrejme $|\uh LMC|\ge|\uh
LDC|=45\st$ (pokiaľ $M\ne D$, je to vonkajší uhol trojuholníka~$DML$,
ktorý má pri vrchole~$D$ uhol~$45^{\circ}$). Pretože uhol~$LMC_0$
meria práve~$45^{\circ}$, leží bod~$C$ na časti oblúka~$\gamma$ medzi bodmi $C_0$ a~$M$.

Ďalej si všimnime, že vrchol~$D$ štvorca~$ABCD$ leží na oblúku,
z~ktorého vidno úsečku~$LM$ pod uhlom~$45\st{}$ v~polrovine
opačnej ku~$KMC$. Zostrojme bod~$P$ (\obr), ktorý leží na
priesečníku priamky~$AD$ a~kolmice na~priamku~$MK$ v~bode~$L$. Body
$M$, $D$, $L$ a~$P$ ležia na Tálesovej kružnici s~priemerom~$MP$, 
a~pretože $|\uh MPL|=|\uh MDL|=45\st$, je trojuholník~$MPL$ rovnoramenný
pravouhlý. To znamená, že bod~$P$ je jednoznačne určený polohou
bodu~$L$ na úsečke~$MK$. 
(Bod~$P$ vznikne otočením bodu~$M$ okolo stredu~$L$
o~$90\st$, pretože bod~$L$ ako bod uhlopriečky~$BD$ má od priamok
$CD$ a~$DA$ rovnakú vzdialenosť, priamka~$DA$ je teda v~spomenutom
otočení obrazom priamky~$CD$; odtiaľ taktiež vyplýva rovnosť
$|LM|=|LP|$.)

Bod~$D$ preto musí ležať na oblúku~$\delta$ Tálesovej
polkružnice nad priemerom~$MP$ v~polrovine opačnej k~$PML$,
súčasne ale polpriamka~$DP$ (ktorá obsahuje vrchol~$A$) nesmie
pretnúť úsečku~$LK$. Odtiaľ vyplýva, že vrchol~$D$ môže ležať len
v~tej časti spomenutej polkružnice nad priemerom~$MP$, ktorá leží 
v~polrovine~$PKL$. Pritom je zrejmé, že priamka~$PK$ túto
polkružnicu pretne v~ďalšom bode rôznom od~$P$ práve vtedy, keď
$|KL|>|LM|$ (pre $|KL|=|LM|$ bude $KP$ dotyčnicou kružnice nad
priemerom~$MP$). Keď označíme v~takom prípade $D_1$ priesečník $KP$
s~polkružnicou~$\delta$ a~$C_1$ priesečník polpriamky~$D_1M$ 
s~oblúkom~$\gamma$, je zrejmé, že vrchol~$C$ padne do časti~$C_0C_1$
oblúka~$\gamma$. V~opačnom prípade, \tj. pre
$|KL|\le|LM|$, vyplnia zrejme vrcholy~$C$ celú časť~$C_0M$
oblúka~$\gamma$.

Naozaj. Zvoľme ľubovoľný bod~$C$ na časti~$C_0C_1$ oblúka~$\gamma$
v~prvom prípade, resp.~na $C_0M$ v~druhom prípade.
Priamka~$CM$ pretne oblúk~$\delta$ v~bode, ktorý označíme~$D$.
Vrchol~$A$ potom zostrojíme ako priesečník polpriamky~$DP$
s~Tálesovou kružnicou nad priemerom~$PK$ (v~prvom prípade máme
zaručené, že bude ležať v~polrovine~$PKA_0$, a~nie v~opačnej).
Pretože, ako už vieme, sú kružnice opísané trojuholníkom $LC_0M$
a~$LA_0K$ zhodné, zistíme ľahko z~príslušných obvodových uhlov,
že $|\uh DAL|=|\uh DCL|$, takže trojuholníky $DAL$ a~$DCL$ sú zhodné,
teda $|DA|=|DC|$. Pretože polpriamka~$DL$ pretína úsečku~$MK$
v~bode~$L$, pretne polpriamka~$AK$ polpriamku~$DL$ v~bode~$B$ za
bodom~$K$, pričom trojuholník~$DAB$ je rovnoramenný pravouhlý. $ABCD$ je teda
štvorec, ktorý spĺňa podmienky úlohy.
\inspicture{}

\zaver
Hľadanou množinou vrcholov~$C$ štvorcov~$ABCD$ je pre $|ML|<|LK|$
oblúk~$C_0C_1$ kružnice opísanej trojuholníku~$MLC_0$ a~oblúk s~ním
osovo súmerný podľa danej priamky~$MK$ (\obr),
pre $|ML|\ge|LK|$ je
to oblúk~$C_0M$ rovnakej kružnice a~oblúk s~ním osovo súmerný
podľa priamky~$MK$ (\obr).
\inspicture{}}

{%%%%%   A-I-6
Označme~$p_i$ pravdepodobnosť, že
vyhrá hráč~$A$, pričom figúrka stojí na $i$-tom políčku.
Dostaneme sústavu rovníc
$$
\align
p_2 &= \frac{1}{3} + \frac{2}{3}p_3, \\
p_3 &= \frac{1}{3}p_2 + \frac{2}{3}p_4, \\
p_4 &= \frac{1}{3}p_3 + \frac{2}{3}p_5 ,\\
p_5 &= \frac{1}{3}p_4 .
\endalign
$$
Postupným dosadzovaním z~jednej rovnice do druhej dostaneme riešenie
$p_2 = 15/31$. Pretože na začiatku stojí figúrka na
políčku číslo~$2$, je pravdepodobnosť výhry hráča~$A$
rovná~$15/31$. Podobným spôsobom zostavíme a~vyriešime
systém rovníc pre hráča~$B$ a~dostaneme tak, že hráč~$B$
vyhrá s~pravdepodobnosťou $16/31$. Pravdepodobnosť
remízy je teda~$0$. Tým je úloha vyriešená.}

{%%%%%   B-I-1
Každý štvormiestny palindróm $p=\overline{abba}$ sa dá zapísať v~tvare
$$
p=a\cdot 1001+b\cdot 110,
$$
kde $a\in \{ 1,2,\dots ,9\}$ a~$b\in \{ 0,1,2,\dots ,9\}$.
Potom druhá mocnina čísla $\overline{abba}$ má tvar
$$
\align
p^2=&a^2\cdot 1\,002\,001+2ab\cdot 110\,110+b^2\cdot 12\,100=\\
   =&a^2\cdot 10^6+2ab\cdot 10^5+(b^2+2ab)\cdot 10^4+\\
    &+(2a^2+2b^2)\cdot 10^3+(b^2+2ab)\cdot 10^2+2ab\cdot 10^1+a^2.
\endalign
$$
Posledná číslica čísla~$p^2$ je teda rovnaká ako posledná číslica
čísla~$a^2$.

Pre $a\geq 4$ je číslo~$p^2$ nutne osemmiestne. Jeho prvá číslica
je rovná jednej z~hodnôt $c$, $c+1$, $c+2$, kde $c$ je prvá číslica
dvojmiestneho čísla~$a^2$. (Maximálny prenos z~nižšieho rádu je
rovný číslu~$2$.) Ak je ale dané číslo opäť palindróm, je jeho
prvá aj posledná číslica rovnaká. Porovnaním prvej a~poslednej
číslice u~čísel $16$, $25$, $36$, $49$, $64$, $81$ vidíme, že žiadne z~nich
nie je tvaru $\overline{c(c+2)}$, $\overline{c(c+1)}$ alebo
$\overline{cc}$.

Ak $a=3$ a~$b\geq 2$, je číslo~$p^2$ opäť osemmiestne, jeho
posledná číslica je~$9$ a~prvá je~$1$, nejedná sa teda o~palindróm.

Vo všetkých ostatných prípadoch je číslo~$p^2$ sedemmiestne. Pretože
$a^2$ je iba jednomiestne a~zápis čísla~$p^2$ je symetrický,
musia byť nutne všetky tri hodnoty $2ab$, $2ab+b^2$, $2a^2+2b^2$
menšie ako~$10$, aby nedošlo k~prenosu do vyššieho rádu. Diskutujme
tri prípady:
\item{$\bullet$} $a=3$: nerovnici $2\cdot 3^2+2b^2<10$ nevyhovuje žiadne~$b$,
\item{$\bullet$} $a=2$: nerovnici $2\cdot 2^2+2b^2<10$ vyhovuje iba $b=0$,
\item{$\bullet$} $a=1$: nerovnici $2\cdot 1^2+2b^2<10$ vyhovuje iba $b=0$, $b=1$.

\zaver
Najväčším štvormiestnym palindrómom spĺňajúcim
podmienky úlohy je číslo~$2\,002$.}

{%%%%%   B-I-2
Keď pripočítame k~prvej rovnici trojnásobok rovnice druhej, získame
rovnicu
$$
x^3+3x^2y+3xy^2+y^3=27z^3.
$$
Jej úpravou dostaneme
$$
(x+y)^3=(3z)^3,\quad \hbox{ \tj. }\quad x+y=3z.
$$
Dosadením tohto výrazu do ľavej strany druhej rovnice sústavy
dostaneme
$$
x^2y+xy^2=xy(x+y)=3xyz,\quad \hbox{ \tj. }\quad 3xyz=6z^3.
$$
Rozlíšime dva prípady.

Ak $z=0$, je posledná rovnica splnená pre všetky $x,y\in
{\Bbb R}$. Z~prvej rovnice sústavy získame $x^3+y^3=0$, \tj. 
$y=-x$. Riešením je každá trojica $(t,-t,0)$, kde $t$ je ľubovoľné
reálne číslo.

Ak $z\not= 0$, tak $xy=2z^2$. Spolu s~rovnicou $x+y=3z$
dostávame sústavu
$$
\align
x+y&=3z,\\
 xy&=2z^2
\endalign
$$
dvoch rovníc o~dvoch neznámych $x$, $y$ s~parametrom~$z$. Elimináciou
napr.~neznámej~$y$ dostaneme kvadratickú rovnicu
$$
x^2-3zx+2z^2=0.
$$
Zo vzťahov medzi koreňmi a~koeficientmi kvadratickej rovnice získame
riešenie v~tvare $x=z$, $y=2z$ alebo $x=2z$, $y=z$. Riešením je teda
každá trojica $(t,2t,t)$ a~$(2t,t,t)$, kde $t$ je ľubovoľné
reálne číslo (rôzne od nuly).

\zaver
Sústava má riešenie $(t,2t,t)$ a~$(2t,t,t)$ pre
každé~$t\not= 0$, $(t,-t,0)$ pre každé~$t$ a~žiadne iné riešenie nemá.

\ineriesenie
Prvú rovnicu vynásobíme dvoma a~odčítame od
nej trojnásobok rovnice druhej (vylúčime tak neznámu~$z$).
Získame rovnicu
$$
2x^3+2y^3-3x^2y-3xy^2=0.
$$
Ľavú stranu rovnice postupne upravíme na tvar
$$
\align
2(x+y)(x^2-xy+y^2)-3(x+y)xy&=0,\\
       (x+y)(2x^2-5xy+2y^2)&=0,\\
          (x+y)(2x-y)(x-2y)&=0.
\endalign
$$
Môžu teda nastať tri prípady:
\item{$\bullet$} $x+y=0$, potom $y=-x$. Dosadením do prvej rovnice sústavy
dostaneme $9z^3=x^3+(-x)^3=0$, \tj. $z=0$.
\item{$\bullet$} $2x-y=0$, potom $y=2x$. Dosadením do prvej rovnice sústavy
dostaneme $9z^3=x^3+(2x)^3=9x^3$, \tj. $z=x$.
\item{$\bullet$} $x-2y=0$, potom $x=2y$. Dosadením do prvej rovnice sústavy
dostaneme $9z^3=(2y)^3+y^3=9y^3$, \tj. $z=y$.

\zaver
Riešením sú všetky trojice $(t,-t,0)$,
$(t,2t,t)$ a~$(2t,t,t)$, kde $t$ je ľubovoľné reálne číslo.}

{%%%%%   B-I-3
\fontplace
\thickmuskip3mu
\tpoint A'; \tpoint B=B'; \bpoint C=C';
\tpoint A;  \tpoint V=V';
\tpoint c; \lBpoint a; \lBpoint b;
\lpoint v; \rBpoint b;
[1] \hfil\Obr

\fontplace
\tpoint A'; \tpoint B'; \bpoint C';
\tpoint A;  \lbpoint V;
\tpoint c; \tpoint c; \lBpoint a; \bpoint x;
\lpoint v; \rBpoint b;
[2] \hfil\Obr

\fontplace
\rBpoint A; \tpoint B; \tpoint C; \bpoint\xy.9,0 A';
\tpoint a; \bpoint c; \bpoint c;
\lpoint b; \lBpoint b;
[3] \hfil\Obr

Bez ujmy na všeobecnosti predpokladajme, že platí $a\geq b$.
Ak je obsah trojuholníka~$A'B'C'$ so stranami
dĺžok $a$, $b$, $2c$ rovný dvojnásobnému obsahu trojuholníka~$ABC$ so
stranami dĺžok $a$, $b$, $c$, sú výšky $CV$ a~$C'V'$ týchto
trojuholníkov zhodné. Trojuholníky $ACV$ a~$A'C'V'$ sú teda
zhodné podľa vety~{\it Ssu}, preto môžeme oba trojuholníky $ABC$
a~$A'B'C'$ premiestniť tak, aby platilo $B=B'$, $C=C'$
a~$V=V'$; potom už však nemôže platiť $A=A'$. Ak je poloha bodov
$A$ a~$A'$ na priamke~$BV$? Pretože $b=|AC|=|A'C|$, je trojuholník~$AA'C$
rovnoramenný a~jeho základňa~$AA'$ má stred v~bode~$V$ (\obr).
\inspicture{}
Predpoklad $a\geq b$ znamená, že $|AC|=|A'C|\leq |BC|$,
takže bod~$B$ neleží na úsečke~$AA'$; pretože $|AB|=c$
a~$|A'B|=2c$, leží bod~$B$ na polpriamke opačnej k~$AA'$ tak, že
bod~$A$ je stredom úsečky~$A'B$.
Z~pravouhlých trojuholníkov $AVC$ a~$BVC$ vyplýva
$$
\align
v^2&=a^2-\Bigl(\frac32c\Bigr)^2,\\
v^2&=b^2-\Bigl(\frac12c\Bigr)^2.
\endalign
$$
Porovnaním pravých strán dostaneme po úprave
$$
a^2-b^2=2c^2.
$$
Ukázali sme tak, že ak k~danému trojuholníku~$ABC$ existuje
trojuholník so stranami $a$, $b$, $2c$ a~obsahom~$2S$, tak pre
dĺžky $a$, $b$, $c$ musí byť splnená rovnosť $|{a^2-b^2}|=2c^2$.

\smallskip
Predpokladajme naopak, že pre veľkosti strán $a$, $b$, $c$
trojuholníka~$ABC$ platí $|{a^2-b^2}|=2c^2$. Najprv ukážeme, že
trojuholník so stranami $a$, $b$, $2c$ existuje, \tj. že platí
trojuholníková nerovnosť
$$
a+b>2c>|a-b|.
$$
Pre trojuholník~$ABC$ platí trojuholníková nerovnosť
$a+b>c>|a-b|$. Preto platí $2c>c>|a-b|$. Keď ďalej vynásobíme obe
strany nerovnosti $c>|a-b|$ kladným výrazom $a+b$, obdržíme nerovnosť
$$
c(a+b)>|a^2-b^2|=2c^2,
$$
z~ktorej po delení~$c$ vyplýva nerovnosť
$$
a+b>2c.
$$
Predpokladajme teraz, že v~trojuholníku~$A'B'C'$ so stranami $a$,
$b$, $2c$ platí rovnosť $2c^2=a^2-b^2$ (opäť bez ujmy na
všeobecnosti predpokladáme, že $a>b$~-- tu nemôže byť $a=b$,
pretože by bolo $c=0$).
Vysvetlíme, prečo päta~$V$ výšky z~vrcholu~$C'$ na stranu~$A'B'$
padne dovnútra tejto strany (a~nie na jej predĺženie). K~tomu
stačí ukázať, že trojuholník~$A'B'C'$ má ostré vnútorné uhly 
pri vrcholoch $A'$ aj $B'$ (\obr). Uhol~$A'B'C'$ je menší ako uhol~$B'A'C'$,
\inspicture{}
lebo predpokladáme, že $a>b$. Uhol~$B'A'C'$ je ostrý práve vtedy,
keď platí nerovnosť $|B'C'|^2 < |A'B'|^2 + |A'C'|^2$,
čiže $a^2<4c^2+b^2$. Posledná nerovnosť je ale zaručená
rovnosťou $a^2=b^2+2c^2$.
Z~pravouhlých trojuholníkov $A'VC'$ a~$B'VC'$ vyplýva, že pre dĺžky
$x=|A'V|$ a~$v=|C'V|$ platí
$$
\align
v^2&=b^2-x^2,\\
v^2&=a^2-(2c-x)^2.
\endalign
$$
Porovnaním pravých strán dostaneme po úprave
$$
4cx=4c^2-(a^2-b^2)
$$
a~dosadením za $a^2-b^2$ vyjde
$$
4cx=4c^2-2c^2=2c^2,\quad \hbox{ \tj. } x=\frac12c.
$$
Keď označíme $A$ (v~súlade s~prvou časťou) stred strany~$A'B'$, platí
$$
|AC'|=|A'C'|=b,
$$
teda trojuholník~$AB'C'$ má strany dĺžok $a$, $b$, $c$ a~obsah
rovný polovici obsahu trojuholníka~$A'B'C'$. Tým sme dokázali
opačnú implikáciu.

\ineriesenie
Z~Herónovho vzorca pre obsah~$S_1$
trojuholníka~$ABC$ a~pre obsah~$S_2$ trojuholník~$A'B'C'$ máme
$$
\align
S_1&=\frac14\sqrt{\bigl((a+b)^2-c^2\bigr) \bigl(c^2-(a-b)^2\bigr)},\\
S_2&=\frac14\sqrt{\bigl((a+b)^2-4c^2\bigr)\bigl(4c^2-(a-b)^2\bigr) }.
\endalign
$$
Z~podmienky $S_2=2S_1$ vyplýva
$$
\bigl((a+b)^2-4c^2\bigr) \bigl(4c^2-(a-b)^2\bigr)
=4\bigl((a+b)^2-c^2\bigr)\bigl(c^2-(a-b)^2\bigr).
$$
Z~tejto podmienky po úprave dostaneme
$$
(a^2-b^2)^2=4c^4,\quad \hbox{\tj. }
\left| a^2-b^2\right| =2c^2.
$$
Prevedené úpravy sú ekvivalentné, preto je možné celý postup
obrátiť. Z~rovnosti $| a^2-b^2| =2c^2$ vyplýva,
že trojuholník~$A'B'C'$ má dvakrát väčší obsah ako trojuholník~$ABC$.
Existencia trojuholníkov sa dá dokázať rovnakým postupom ako
v~prvom riešení.

\ineriesenie
Uvažujme úsečku~$BC$ dĺžky~$a$ ($a>b$)
a~kružnicu~$k$ so stredom v~bode~$C$ a~polomerom~$b$ (\obr).
\inspicture{}
V~rovnakej polrovine (s~hraničnou priamkou~$BC$) uvažujme body $A$
a~$A'$, pre ktoré platí $|AB|=c$, $|A'B|=2c$. Ak ležia body $B$, $A$
a~$A'$ na jednej priamke, potom obsah trojuholníka~$A'BC$ je
dvojnásobkom obsahu trojuholníka~$ABC$. Z~mocnosti bodu~$B$ ku
kružnici~$k$ vyplýva
$$
|BA|\cdot |BA'|=2c^2=a^2-b^2.
$$
Ak je naopak splnená posledná rovnosť, pretne polpriamka opačná
k~$AB$ kružnicu~$k$ v~bode, ktorého vzdialenosť od bodu~$B$ je
rovná~$2c$, týmto bodom je však~$A'$. Odtiaľ už vyplýva tvrdenie pre
obsahy trojuholníkov. Existencia trojuholníkov sa dá dokázať rovnakým
postupom ako v~prvom riešení.}

{%%%%%   B-I-4
Keď sčítame všetky tri čísla vzniknutej trojice,
dostaneme
$$
(r+5q)+(3r-5p)+(2q-3p)=4r+7q-8p=3(r+2q-3p)+(p+q+r).
$$
Toto číslo dáva po delení tromi rovnaký zvyšok ako číslo
$(p+q+r)$, \tj. zvyšok po delení tromi súčtu čísel v~trojici
zostáva rovnaký. Pre trojicu $(1,3,7)$ je zvyšok rovný dvom
($1+3+7=11=3\cdot 3+2$).
Súčet troch po sebe idúcich celých čísel je však deliteľný tromi,
takže dáva zvyšok nula. Vyplýva to z~rovnosti
$k+(k+1)+(k+2)=3(k+1)$.

\zaver
Po konečnom počte {\it krokov} nemôžeme z~trojice
$(1,3,7)$ dostať trojicu po sebe idúcich celých čísel.

\ineriesenie
Skúmajme, ako sa mení parita trojice čísel
v~nasledujúcich {\it krokoch}. Na začiatku sú všetky tri čísla
nepárne. Postupne dostávame
$$
(n,n,n)\rightarrow (p,p,n)\rightarrow (n,n,p)\rightarrow (n,n,n)\rightarrow
\cdots
$$
Pretože sa parita čísel pravidelne mení podľa danej schémy,
nemôžeme z~trojice nepárnych čísel dostať trojicu $(p,n,p)$,
resp.~$(n,p,n)$, ktoré reprezentujú všetky trojice po sebe
idúcich čísel (za párnym číslom nasleduje nepárne a~naopak).

\ineriesenie
\podla{Miroslava Jagoša}
Zistíme, ktorá trojica je bezprostredným predchodcom trojice $(k,k+1,k+2)$.
Riešime sústavu rovníc
$$
r+5q=k, \qquad 3r-5p=k+1, \qquad 2q-3p=k+2.
$$
Z~druhej a tretej rovnice máme $9r-10q=\m2k-7$ a~po pripočítaní dvojnásobku prvej
rovnice dostaneme $11r=\m7$, teda $r$ nie je celé. Pretože z~celočíselnej trojice vznikne
opäť celočíselná trojica, vyplýva odtiaľ, že trojica $(k,k+1,k+2)$ nemôže vzniknúť po konečnom počte {\it krokov} zo žiadnej celočíselnej trojice, teda ani z trojice $(1,3,7)$.}

{%%%%%   B-I-5
\fontplace
\tpoint A; \tpoint B; \bpoint C; \bpoint D;
\rtpoint S; \rbpoint X; \lBpoint P;
\lpoint t; \rBpoint k_1; \bpoint k_2;
[4] \hfil\Obr

Pretože úsečka~$AD$ je priemerom kružnice~$k_1$, je uhol~$APD$ pravý (\obr).
\inspicture{}
Uvažujme spoločnú dotyčnicu~$t$ oboch kružníc prechádzajúcu bodom~$P$.
Označme postupne $S$ a~$X$ priesečníky dotyčnice~$t$ s~úsečkami $AB$ 
a~$CD$. Priamka~$AB$ je ale tiež spoločnou dotyčnicou oboch kružníc.
Platí preto $|SA|=|SP|=|SB|$. Bod~$S$ je preto stredom Tálesovej
kružnice zostrojenej nad stranou~$AB$ ako priemerom. Uhol~$APB$ je
preto rovnako ako uhol~$APD$ pravý a~bod~$P$ je teda vnútorným
bodom úsečky~$BD$.
Trojuholník~$BPS$ je rovnoramenný so základňou~$BP$, pre jeho
uhly teda platí $|\uh SBP|=|\uh SPB|$. Uhol~$SPB$ má naviac
rovnakú veľkosť ako uhol~$DPX$ (dvojice vrcholových uhlov).
Platí preto $|\uh ABP|=|\uh DPX|$. Súčasne však je uhol~$XPC$
uhlom úsekovým pre tetivu~$CP$ kružnice~$k_2$. Z~rovnosti
obvodového a~úsekového uhla máme $|\uh PBC|=|\uh XPC|$.
Celkovo dostávame
$$
|\uh ABC|=|\uh ABP|+|\uh PBC|=|\uh DPX|+|\uh
XPC|=|\uh DPC|,
$$
čo sme chceli dokázať.}

{%%%%%   B-I-6
\fontplace
\rtpoint O; \tpoint u; \rpoint v;
\tpoint 1; \rpoint 1;
[5] \hfil\Obr

\fontplace
\rtpoint O; \tpoint x; \rpoint y;
\tpoint 1; \lbpoint\down\unit 1;
\tpoint u;
[6] \hfil\Obr

Nulové body výrazu $x^2-ux$ sú $x=0$ a~$x=u$. Pretože podľa
zadania platí $u>0$, rozdelíme reálnu os na tri navzájom
disjunktné intervaly $I_1=(-\infty ,0)$, $I_2=\langle 0,u
\rangle$ a~$I_3=(u,\infty )$.

Na intervaloch  $I_1$ a~$I_3$ riešime kvadratickú rovnicu 
$$
x^2-(u-v)x-1=0. \tag1
$$
Táto rovnica má kladný diskriminant $(u-v)^2+4$, a~teda dva rôzne
reálne korene
$$
\align
x_1&=\frac{u-v-\sqrt{(u-v)^2+4}}{2},\\
x_2&=\frac{u-v+\sqrt{(u-v)^2+4}}{2} .
\endalign
$$
Pretože $\sqrt{(u-v)^2+4}>|u-v|$, platí $x_1<0$ a~$x_2>0$.
Znamená to, že číslo~$x_1$ je vždy riešením rovnice~(1), lebo
$I_1=(-\infty,0)$, zatiaľ čo číslo~$x_2$ je riešením rovnice~(1) práve vtedy,
keď platí $x_2\in I_3$, čiže $x_2>u$.

Na intervale~$I_2$ riešime kvadratickú rovnicu
$$
x^2-(u+v)x+1=0.
$$
Táto rovnica má diskriminant $D=(u+v)^2-4$ a~prípadné reálne
korene
$$
\align
x_3&=\frac{u+v-\sqrt{(u+v)^2-4}}{2},\\
x_4&=\frac{u+v+\sqrt{(u+v)^2-4}}{2}.
\endalign
$$
Zo zadania vyplýva, že aspoň jeden z~koreňov $x_3$,
$x_4$ musí byť riešením rovnice~(1) (ležiacim v~intervale~$I_2$).
Preto hlavne musí byť diskriminant~$D$ nezáporný, z~čoho vyplýva
podmienka $|u+v|\geqq2$. Pretože naviac $\sqrt{(u+v)^2-4}<|u+v|$,
majú oba korene $x_3$, $x_4$ rovnaké znamienko ako súčet $u+v$.
Spolu to znamená, že musí platiť $u+v\geqq2$
(v~prípade $u+v\leqq-2$ by totiž žiadne z~čísel $x_3$, $x_4$
neležalo v~$I_2$). Za podmienky $u+v\geqq2$ ale platí
$0<x_3\leqq x_4$, takže zo zadania vyplýva, že
v~intervale $I_2=\langle 0,u\rangle$ leží číslo~$x_3$
(a~prípadne aj číslo~$x_4$).

Z~doterajších úvah vyplýva, že našou úlohou je odpovedať na otázku, kedy
za podmienok
$$
u>0\quad\text{a}\quad u+v\geqq2
\tag2
$$
nastane niektorý z~týchto prípadov:
\ite a) $x_2\notin I_3$, $\{x_3,x_4\}\subset I_2$, $x_3\ne x_4$;
\ite b) $x_2\in I_3$, $x_3=x_4\in I_2$;
\ite c) $x_2\in I_3$, $x_3\in I_2$, $x_4\notin I_2$.

\smallskip
a) Zistíme, kedy sú splnené jednotlivé podmienky, ktoré tento
prípad vymedzujú (pre lepší prehľad ich v~texte uvádzame čiernymi
bodmi).

$\bullet$ $x_2\notin I_3$, čiže $x_2\leqq u$. Po
úprave získame nerovnosť
$$
\sqrt{(u-v)^2+4}\leqq u+v,
$$
ktorej pravá
strana je podľa~(2) kladná, takže obe strany môžeme umocniť
na druhú. Po ďalšej jednoduchej úprave dostaneme podmienku
$uv\geqq1$. Preto platí
$$
x_2\notin I_3\quad\Longleftrightarrow\quad uv\geqq1.
$$

$\bullet$ $\{x_3,x_4\}\subset I_2$. Ako vieme, za podmienok~(2)
platí $0<x_3\leqq x_4$, stačí preto iba skúmať nerovnosť
$x_4\leqq u$, čiže $\sqrt{(u+v)^2-4}\leqq u-v$. Posledná
nerovnosť môže platiť jedine vtedy, keď $u\geqq v$. Potom po
umocnení strán skúmanej nerovnosti a~následnej úprave dostaneme
podmienku $uv\leqq1$. Preto platí
$$
\{x_3,x_4\}\subset I_2\quad\Longleftrightarrow\quad
u\geqq v\ \land\  uv\leqq1.
$$

$\bullet$ $x_3\ne x_4$. Zo skoršieho odvodenia podmienky $u+v\geqq2$
je jasné, že rovnosť $x_3=x_4$ nastane práve vtedy, keď $u+v=2$. Za
podmienok~(2) teda platí
$$
x_3\ne x_4\quad\Longleftrightarrow\quad u+v>2.
$$

Zhrnieme teraz všetky podmienky pre skúmaný prípad a). Z~nerovností
$uv\geqq1$ a~$uv\leqq1$ vyplýva $uv=1$, čiže $v=1/u$. Zostávajúce
podmienky majú potom tvar $u\geqq 1/u$ a~$u+1/u>2$ a~sú zrejme
obe splnené práve vtedy, keď $u>1$. Hľadané body~$[u,v]$ v~prípade~a)
teda tvoria časť hyperboly $v=1/u$ určenú obmedzením $u>1$ (\obr).
\inspicture{}

\smallskip
b) Z~predchádzajúceho rozboru prípadu~a) vyplýva, že za podmienok~(2)
platia ekvivalencie
$$
\gather
x_2\in I_3\ \Longleftrightarrow\ uv<1,\quad
x_4\in I_2\ \Longleftrightarrow\ u\geqq v\ \land\  uv\leqq1,\\
x_3=x_4\ \Longleftrightarrow\ u+v=2.
\endgather
$$
Vidíme, že v~prípade~b) musí platiť $v=2-u$. Vtedy majú zostávajúce
podmienky tvar $(2-u)u<1$ a~$u\geqq 2-u$ a~sú zrejme obe splnené práve vtedy,
keď $u>1$. Hľadané body~$[u,v]$ v~prípade~b) teda tvoria
polpriamku určenú rovnicou $v=2-u$ a~obmedzením $u>1$.

\smallskip
c) Podmienka $x_3\in I_2$ sa dá vyjadriť nerovnosťou
$x_3\leqq u$, ktorá je ekvivalentná s~nerovnosťou
$\sqrt{(u+v)^2-4}\geqq v-u$. Tá je splnená triviálne, pokiaľ
$u\geqq v$. Ako sme ale ukázali skôr, v~prípade $u\geqq v$
platí nielen $x_3\in I_2$, ale aj $x_4\in I_2$, čo prípad~c)
vylučuje. V~prípade~c) teda nutne platí $u<v$ a~z~nerovnosti
$\sqrt{(u+v)^2-4}\geqq v-u$ po umocnení a~úprave dostaneme
podmienku $uv\geqq1$. Ako ale vieme, z~poslednej nerovnosti vyplýva
$x_2\notin I_3$, takže prípad~c) nemôže nikdy nastať.

\zaver
Množinou všetkých bodov vyhovujúcich zadaniu je časť
hyperboly $v=1/u$ a~časť priamky $v=2-u$, v~oboch prípadoch
časti určené podmienkou $u>1$.


\ineriesenie
Rovnicu možno riešiť tiež graficky. Skúmame,
kedy budú mať grafy funkcií $f(x)=|x^2-ux|$
a~$g(x)=1-vx$ práve tri spoločné body (\obr).
\inspicture{}
Graf funkcie~$f$ je zložený z~časti paraboly, grafom funkcie~$g$
je priamka prechádzajúca bodom~$[0,1]$. Aby táto priamka mala
s~grafom~$f(x)$ spoločné práve tri body, musí byť buď dotyčnicou
paraboly na intervale~$(0,u)$ (potom $u+v=2$, odvodenie je
analogické ako v~predchádzajúcom riešení~-- pomocou diskriminantu), alebo
musí prechádzať bodom~$[u,0]$ a~súčasne pretínať graf funkcie~$f$
vo vnútornom bode intervalu~$(0,u)$. Keď dosadíme súradnice
bodu~$[u,0]$ do rovnice priamky~$g$, dostaneme $0=1-vu$, \tj. $uv=1$.
Rovnako ako v~predchádzajúcom riešení musí platiť $u>1$, čo
môžeme overiť nájdením druhého priesečníku priamky s~parabolou.}

{%%%%%   C-I-1
Najväčší možný súčet by vytvorila pätica čísel $54\,321$, $54\,321$,
$54\,321$, $54\,321$, $54\,321$. Keďže majú byť čísla navzájom rôzne,
pokúsime sa zmeniť túto päticu tak, aby sa nenarušilo trojčíslie~$543$,
\tj. aby zmena súčtu bola čo najmenšia. Tak ale budú ešte
dve z~piatich čísel rovnaké, pretože z~číslic $1$, $2$ je možné zostaviť
iba štyri rôzne dvojčíslia $11$, $12$, $21$, $22$. Zmeníme preto jedno
trojčíslie~$543$ na $542$ tak, že zameníme číslicu~$2$ číslicou~$3$ na
mieste desiatok. Rovnako tak na mieste jednotiek nemôže byť všetkých päť
jednotiek, pretože by posledné trojčíslie najmenej troch päťmiestnych
čísel bolo~$321$. Vymeníme preto číslicu~$1$ z~miesta jednotiek
s~číslicou~$2$ z~miesta stoviek a~to preto, aby zmena súčtu pätice
čísel bola čo najmenšia. Po týchto výmenách môžu byť posledné
dvojčíslia piatich čísel tieto: $31$, $22$, $21$, $21$, $11$, alebo $31$, $21$, $21$,
$21$, $12$, alebo $32$, $21$, $21$, $21$, $11$. Snažíme sa teraz rozmiestniť tieto
dvojčíslia za trojčíslia $543$, $543$, $543$, $543$, $542$. Zistíme, že
vyhovuje iba prvá pätica dvojčíslí. Hľadaná pätica
päťmiestnych čísel s~najväčším možným súčtom je  $54\,331$, $54\,322$,
$54\,321$, $54\,311$, $54\,221$.}

{%%%%%   C-I-2
\fontplace
\tpoint A; \tpoint B; \bpoint C;
\bpoint\xy.5,0 D; \tpoint\xy.5,0 E; \bpoint F;
\rtpoint P; \rBpoint\xy.5,.5 Q; \rpoint R;
[1]
\hfil\Obr

Zo zadania vieme, že $|PR|=|CQ|$, preto aj $|QR|=|CP|$ (\obr).
Úsečka~$DE$ je strednou priečkou trojuholníka~$CPB$, preto
\inspicture{}
$|DE|=|CP|/2$, a teda tiež $|DE|=|QR|/2$. Pretože $DE \parallel QR$, nemôžu byť
úsečky $RE$ a~$QD$ rovnobežné (inak by bol $REDQ$ rovnobežník
a~platilo by $|DE|=|QR|$). Preto sa priamky $RE$ a~$QD$ pretínajú
v~bode, ktorý je na obrázku označený ako~$F$, a~úsečka~$DE$ je
strednou priečkou trojuholníkov $CPB$ a~$QRF$, ktorých strany $CP$ 
a~$QR$ ležia na jednej priamke. Preto je vzdialenosť bodov $F$ a~$B$ od
priamky~$CR$ rovnaká, čiže priamky $CR$ a~$FB$ sú rovnobežné, 
a~teda priamka~$FB$ je (rovnako ako priamka~$CR$) kolmá na priamku~$AB$.}

{%%%%%   C-I-3
Nech naše pôvodné úspory sú $x$~Sk a~nech ročná úroková miera
v~banke~$A$ (resp.~v~banke~$B$) je $p\,\%$ (resp.~$q\,\%$), \tj. vklad v~banke~$A$
(resp.~v~banke~$B$) narastie po jednom roku $a$-krát (resp.~$b$-krát), kde
$a=1+p/100$ a $b=1+q/100$. Podľa zadania
platí
$$
\gather
\left({\frac{5}{6} \cdot x}\right) \cdot a~+ \left({\frac{1}{6}
\cdot x} \right) \cdot b = 67\,000,\cr
\left[ {\left({\frac{5}{6} \cdot x} \right) \cdot a} \right] \cdot 
a+ \left[{\left({\frac{1}{6} \cdot x} \right) \cdot b} \right]
\cdot b = 74\,900,\cr
\left({\frac{1}{6} \cdot x} \right) \cdot a+
 \left({\frac{5}{6} \cdot x} \right) \cdot b = 71\,000,
\endgather
$$
a~po úprave
$$
\gather
5 \cdot \frac{{xa}}{6} + \frac{{xb}}{6} = 67\,000,\cr
5 \cdot \frac{{xa}}{6} \cdot a~+ \frac{{xb}}{6} \cdot b=74\,900,\cr
\frac{{xa}}{6} + 5 \cdot \frac{{xb}}{6} =71\,000.
\endgather
$$
Keď označíme $u=xa/6$ a~$v=xb/6$,
prejdú prvá a~tretia rovnica na sústavu
$$
\align
5u +  v~&= 67\,000,\cr
 u~+ 5v &= 71\,000,
\endalign
$$
z~ktorej vychádza $u=11\,000$ a~$v=12\,000$. Pretože $a=6u/x$ 
a~$b=6v/x$, dá sa druhá rovnica
sústavy zapísať ako
$$
\frac{5}{6} \cdot x \cdot \frac{{36u^2 }}{{x^2 }} + \frac{1}{6}
\cdot x \cdot \frac{{36v^2 }}{{x^2 }} = 74\,900,
$$
alebo aj
$$
\frac{{30u^2+6v^2 }}{x} = 74\,900,
$$
odkiaľ pre $u = 11\,000$ a~$v = 12\,000$ vychádza $x = 60\,000$, preto
$$
\align
a~&= \frac{{6u}}{x} = \frac{{66\,000}}{{60\,000}} = 1{,}1,\cr
b &= \frac{{6v}}{x} = \frac{{72\,000}}{{60\,000}} = 1{,}2.
\endalign
$$
Hľadaná čiastka je preto rovná 
$$
\align
\left(\frac{1}{6} \cdot x \cdot a^2
 + \frac{5}{6} \cdot x \cdot b^2\right)\text{ Sk}=&
({10\,000 \cdot 1{,}1^2+50\,000 \cdot1{,}2^2}){\text{ Sk}}=\\
=&84\,100\text{ Sk}.
\endalign
$$}

{%%%%%   C-I-4
\fontplace
\tpoint A; \tpoint B; \bpoint C; \bpoint D;
\tpoint B'; \tpoint E;
\tpoint c; \tpoint a-c; \lpoint v;
\lBpoint c; \bpoint c; \rBpoint c;
[2] \hfil\Obr

{\it Rozbor\/}.
Ak označíme $|AB|=a$, $|CD|=c$
\inspicture{}
a~výšku lichobežníka~$v$ (\obr), môžeme pre jeho obsah~$S$ písať
$$
S~= \frac{1}{2}(a~+ c)v.
$$
Obsah trojuholníka~$AED$ je podľa zadania rovný
$$
\frac{{|AE| \cdot v}}{2} = \frac{1}{2} \cdot
S~= \frac{1}{2} \cdot \frac{1}{2}(a~+ c)v,
$$
odkiaľ vyplýva, že $|AE|=(a+c)/2$ (\tj. úsečka~$AE$ má dĺžku rovnakú ako stredná priečka
lichobežníka~$ABCD$). Pretože bod~$E$ leží na úsečke~$AB$, platí
$$
|EB| = |AB| - |AE| =
a- \frac{1}{2}(a~+ c) = \frac{1}{2}(a~- c),
$$
takže $a > c$. Ak označíme $B'$ bod úsečky~$AB$, pre ktorý
$|AB'|=c$, bude $|B'B|=a-c$, a~pretože hľadaný lichobežník~$ABCD$
je rovnoramenný, je rovnoramenný aj trojuholník~$B'BC$, takže stred~$E$
úsečky~$B'B$ je zároveň pätou výšky z~vrcholu~$C$ na základňu~$AB$ (\obrr1).
Pomocou Pytagorovej vety vypočítame, že
$$
c = \sqrt {|DE|^2  - v^2 }
= \sqrt {5^2  - 3^2}\,{\text{cm}} =4\,\text{cm}.
$$

\noindent
{\it Popis konštrukcie\/}:
\ite1.  $\triangle DEC$; $|DC| = 4\cm$, $|CE| =3\cm$, $|\uh ECD|=90\st$;
\ite2.  $p$;  $p\parallel CD$, $E\in p$;
\ite3.  $k(D,4\cm)$, $l(C,4\cm)$;
\ite4.  $A$;  $A\in p\cap k$, uhol~$ADC$ je tupý;
\ite5.  $B$; $B\in p\cap l$, uhol~$BCD$ je tupý.

\noindent
Úloha má jediné riešenie.}

{%%%%%   C-I-5
Nech číslo $n=\overline{abcd}=1\,000a + 100b + 10c + d$,
kde $a,b,c,d\in\{0, 1, \dots, 9\}$, $a\ne 0$. Číslo~$m + n$
je štvormiestne, preto je číslo~$m$ najviac štvormiestne. Rozoberieme
jednotlivé prípady podľa počtu číslic~$m$.

\smallskip
1. Číslo~$m$ je jednomiestne, \tj. $m = \overline x  = x$, kde $x
\in\{1, 2, \dots, 9\}$. Podľa zadania úlohy jednak
$$
m + n = 1\,000a + 100b + 10c + d + x,
$$
jednak
$$
m + n = 1\,000d + 100c + 10b + a.
$$
Odtiaľ postupne dostaneme
$$
\gather
1\,000a + 100b + 10c + d + x = 1\,000d + 100c + 10b + a,\\
x = 999(d - a) + 90(c - b).
\endgather
$$
Pravá strana poslednej rovnosti je deliteľná deviatimi, preto môže
byť jedine $x = 9$. Po dosadení
tejto hodnoty do rovnosti a~vykrátení deviatimi vychádza
$$
\gather
1 = 111(d - a) + 10(c - b),\\
10(b - c) + 1 = 111(d - a).
\endgather
$$
Z~nerovností  $ - 9 \le b - c \le 9$ vyplýva 
$ - 89 \le 10(b - c) + 1 \le 91$. Medzi číslami $ - 89$ a~$91$ je
jediný násobok~$111$, a~to číslo~$0$. Rovnosť $10(b - c) + 1 = 0$
však nie je splnená. Žiadne jednomiestne číslo~$m$ teda nie je riešením
danej úlohy.

\smallskip
2.  Číslo~$m$ je dvojmiestne, \tj. $m = \overline {xx}  = 10x + x
= 11x$, kde $x\in\{1, 2, \dots, 9\}$. Analogicky ako
v~predchádzajúcom prípade môžeme postupne písať
$$
\gather
1\,000a + 100b + 10c + d + 11x = 1\,000d + 100c + 10b + a,\\
11x = 999(d - a) + 90(c - b).
\endgather
$$
Pravá strana poslednej rovnosti je deliteľná deviatimi, preto môže
byť jedine $x = 9$. Potom
$$
\gather
11 = 111(d - a) + 10(c - b),\\
10(b - c) + 11 = 111(d - a).
\endgather
$$
Tu máme $ - 79 \le 10(b - c) + 11 \le 101$, odkiaľ
vyplýva jediná možnosť ${10(b - c) + 11}=0$, ktorá však neplatí pre
žiadne číslice $b$, $c$. Žiadne dvojmiestne číslo~$m$ teda nie je
riešením danej úlohy.

\smallskip
3.  Číslo~$m$ je trojmiestne, \tj. $m = \overline {xxx}  = 100x +
10x + x = 111x$, kde $x \in \{1, 2, \dots, 9\}$.
Opäť môžeme písať
$$
\gather
1\,000a + 100b + 10c + d + 111x = 1\,000d + 100c + 10b + a,\\
111x = 999(d - a) + 90(c - b),\\
37x = 333(d - a) + 30(c - b).
\endgather
$$
Pravá strana poslednej rovnosti je deliteľná tromi a~číslo~$37$ nie je
deliteľné tromi, preto musí byť $x = 3$, alebo $x = 6$, alebo $x=9$.

Nech $x = 3$. Potom
$$
\gather
37 = 111(d - a) + 10(c - b),\\
10(b - c) + 37 = 111(d - a).
\endgather
$$
Tu máme $ - 53 \le 10(b - c) + 37 \le 127$, odkiaľ
buď $10(b - c) + 37 = 0$, alebo $10(b - c) + 37 = 111$. Ani jedna
z~posledných dvoch rovností však nie je splnená pre žiadne číslice
$b$,~$c$.

Nech $x=6$. Potom
$$
\gather
74 = 111(d - a) + 10(c - b),\\
10(b - c) + 74 = 111(d - a).
\endgather
$$
Tu máme $ - 16 \le 10(b - c) + 74 \le 164$, odkiaľ
buď $10(b - c) + 74 = 0$, alebo $10(b - c) + 74 = 111$. Ani jedna
z~posledných dvoch rovností však nie je splnená pre žiadne číslice
$b$,~$c$.

Nech $x=9$. Potom
$$
\gather
111 = 111(d - a) + 10(c - b),\\
10(b - c) = 111(d - a~- 1).
\endgather
$$
Tu máme $ - 90 \le 10(b - c) \le 90$,
odkiaľ jedine $10(b - c) = 0$ a~$111(d-a-1) = 0$,
\tj. jedine  $c - b = 0$ a~$d - a~= 1$. Riešením danej úlohy sú
teda čísla $n \in \left\{ \overline {1\,bb2} ,\overline
{2\,bb3},\overline {3\,bb4},\overline {4\,bb5},\overline
{5\,bb6},\overline {6\,bb7},\overline {7\,bb8},\overline {8\,bb9}
\right\}$ pre $b \in \{0,1,\dots,9\}$, \tj. celkom $80$~čísel.
Číslo~$m$ je rovné $999$.

\smallskip
4. Číslo~$m$ je štvormiestne, \tj. $m = \overline {xxxx}  = 1\,111x$,
kde $x \in \{1, 2, \dots, 9\}$.
Opäť môžeme písať
$$
1\,111x = 999(d - a) + 90(c - b).
$$
Opäť môže byť jedine $x = 9$, čo dáva rovnosť
$$
10(b - c) + 1\,111 = 111(d - a).
$$
Platí jednak $10(b - c) + 1\,111 \ge 1\,111 - 90
= 1\,021$, jednak $111(d - a) \le 999$. Preto
žiadne štvormiestne číslo~$m$ nie je riešením danej úlohy.

\zaver
Úloha má $80$~riešení, a~to čísla $m = 999$ a
$$
\align
n \in \bigl\{\overline {1\,bb2},\overline {2\,bb3},&
\overline {3\,bb4},\overline {4\,bb5},\overline {5\,bb6}
,\overline {6\,bb7},\overline {7\,bb8},\overline {8\,bb9}\bigr\}\\
 &\text{pre $b \in \{ 0,1,\dots,9 \}$}.
\endalign
$$}

{%%%%%   C-I-6
\fontplace
\rbpoint\xy.5,-.4 S; \bpoint\xy1,0 C_1; \lpoint C_2;
\tpoint A_1; \tpoint B_1; \tpoint X_1; \bpoint A_2; \tpoint B_2;
\rpoint p; \lBpoint k; \rBpoint l;
[3] \hfil\Obr

{\it Rozbor\/}.
Predpokladajme, že požadovaný trojuholník~$ABC$ je zostrojený.
Stred kružnice vpísanej ľubovoľnému trojuholníku leží na osiach
jeho vnútorných uhlov. Podľa zadania leží stred kružnice~$k$
na ťažnici~$t_c$ trojuholníka~$ABC$, preto os vnútorného uhla pri
vrchole~$C$ splýva s~ťažnicou~$t_c$. Trojuholník~$ABC$ je teda
rovnoramenný so základňou~$AB$ (\obr). Keď leží stred~$S$ kružnice~$k$
s~polomerom~$r$ vo štvrtine ťažnice~$t_c$, leží teda vo
vzdialenosti~$r$ od strany~$AB$ a~vo vzdialenosti~$3r$ od vrcholu~$C$.
(Bod~$S$ nemôže mať od vrcholu~$C$ vzdialenosť $r/3$,
lebo by bod~$C$ ležal vo vnútornej oblasti kružnice~$k$, ktorá je
však trojuholníku~$ABC$ vpísaná, takže body $A$, $B$, $C$ ležia
v~jej vonkajšej oblasti.) Bod~$C$ je teda priesečníkom priamky~$p$
a~kružnice~$l$ so stredom~$S$ a~polomerom~$3r$.
\inspicture{}

\noindent
{\it Popis konštrukcie\/}:
\ite1.  dané: $k(S,r)$, $p$;
\ite2.  $l(S,3r)$;
\ite3.  $C$; $ C\in p\cap l$;
\ite4.  $X$; $X\in\to CS$, $|XC|= 4r$;
\ite5.  $x$; $x\bot XC$, $ X\in x$;
\ite6.  dotyčnice $a$, $b$ z~bodu~$C$ ku~$k$ (napr.~pomocou
        Tálesovej kružnice nad priemerom~$CS$);
\ite7.  $A$, $B$; $ A\in x\cap b$, $ B\in x\cap a$.

\noindent
Diskusia pre prípad, že poradie vrcholov $A$, $B$, $C$ je proti
smeru pohybu hodinových ručičiek:
\ite{} Úloha má dve riešenia  $\iff  |Sp| < 3r$;
\ite{} úloha má jedno riešenie  $\iff  |Sp| = 3r$;
\ite{} úloha nemá žiadne riešenie $ \iff  |Sp| > 3r$.}

{%%%%%   A-S-1
Pre ľubovoľne vybraných dvadsať prirodzených čísel
$$
x_1<x_2<\dots<x_{20}
$$
odhadneme, koľko medzi nimi môže byť súčtových trojíc, teda
trojíc $\{x_i,x_j,x_k\}$ spĺňajúcich podmienky $1\leqq
i<j<k\leqq20$ 
a~$x_i+x_j=x_k$, a~to najprv
pri pevnom indexe $k\in\{3,4,\dots,20\}$. Nech sú to trojice
$\{x_{i_1},x_{j_1},x_k\}$, $\{x_{i_2},x_{j_2},x_k\}$, \dots,
$\{x_{i_p},x_{j_p},x_k\}$. Potom čísla
$$
x_{i_1},x_{j_1},x_{i_2},x_{j_2},\dots,x_{i_p},x_{j_p}
$$
sú navzájom rôzne a~všetky ležia v~množine
$\{x_1,x_2,\dots,x_{k-1}\}$, takže pre
ich počet~$2p$ platí odhad $2p\leqq k-1$, odkiaľ
$p\leqq\lfloor(k-1)/2\rfloor$ (kde $\lfloor a\rfloor$ značí celú časť čísla~$a$).
Preto počet všetkých súčtových trojíc nemôže byť číslo väčšie ako
súčet
$$
\sum_{k=3}^{20}\left\lfloor\frac{k-1}{2}\right\rfloor=
1+1+2+2+3+3+\dots+9+9=90.
$$
Príklad množiny $\mm M=\{1,2,\dots,20\}$ ukazuje, že počet $90$~súčtových
trojíc je dosiahnuteľný, lebo pri každom
$k\in\{3,4,\dots,20\}$ môžeme za číslo~$i$ vybrať ľubovoľné číslo
z~množiny $\{1,2,\dots,\lfloor(k-1)/2\rfloor\}$; zodpovedajúce celé číslo
$j=k-i$ potom skutočne spĺňa nerovnosti $i<j<k$, takže
$\{i,j,k\}$ je súčtová trojica ležiaca v~$\mm M$.}

{%%%%%   A-S-2
\fontplace
\bpoint O; \tpoint S_1; \ltpoint\xy-1,0 S_2;
\bpoint P; \bpoint R; \ltpoint K; \tpoint Q;
\rBpoint r_1; \lBpoint r_2;
\rBpoint k_1; \lBpoint\down.7\unit k_2;
[7] \hfil\Obr

\fontplace
\rBpoint M; \tpoint\xy-1,0 S_1; \ltpoint\xy-1,0 S_2;
\bpoint P; \bpoint R; \rtpoint\xy.5,0 L; \tpoint Q;
\rBpoint k_1; \lBpoint\down.7\unit k_2;
[8] \hfil\Obr

\fontplace
\bpoint O; \tpoint S_1; \tpoint S_2;
\bpoint P; \bpoint R; \ltpoint\xy-.5,0 K; \tpoint Q;
[9] \hfil\Obr

Zo súmernosti spoločných dotyčníc vyplýva, že body
dotyku $P$ a~$Q$ sú súmerne združené podľa priamky~$S_1S_2$,
takže platí $PQ\perp S_1S_2$. Priamka~$PQ$ preto bude dotyčnicou ku
kružnici~$k_2$, keď ukážeme, že priesečník~$K$ priamok $PQ$ 
a~$S_1S_2$ leží na kružnici~$k_2$ (\obr). Označme ešte~$O$
\inspicture{}
priesečník oboch dotyčníc s~priamkou~$S_1S_2$ a~$R$~bod dotyku dotyčnice~$PO$
s~kružnicou~$k_2$. Z~podobných pravouhlých trojuholníkov $S_1OP$ a~$S_2OR$
vyplýva pomer
$$
\frac{r_1}{r_2}=\frac{|S_1P|}{|S_2R|}=
\frac{|S_1O|}{|S_2O|}=\frac{|S_1O|}{|S_1O|-r_1},
\quad\text{odkiaľ}\quad |S_1O|=\frac{r_1^2}{r_1-r_2}.
$$
Z~Euklidovej vety o~odvesne~$S_1P$ trojuholníka~$S_1OP$ preto vyplýva, že
$$
r_1^2=|S_1P|^2=|S_1K|\cdot|S_1O|=|S_1K|
\,\frac{r_1^2}{r_1-r_2},
$$
teda $|S_1K|=r_1-r_2$, a~preto $|S_2K|=|S_1S_2|-|S_1K|=
r_1-(r_1-r_2)=r_2$. To znamená, že bod~$K$ skutočne leží
na kružnici~$k_2$ a~dôkaz tvrdenia je hotový.

\ineriesenie
Označme $L$~priesečník kružnice~$k_2$ s~úsečkou~$S_1S_2$, $M$~pätu
kolmice spustenej z~bodu~$S_2$ na úsečku~$S_1P$ a~$R$~bod dotyku
kružnice~$k_2$ s~tou spoločnou dotyčnicou, ktorá prechádza bodom~$P$
(\obr). Pretože $S_2RPM$ je pravouholník, platí
\inspicture{}
$|MP|=|S_2R|=r_2$, a~preto $|S_1M|=|S_1P|-|MP|=r_1-r_2$. Rovnakú
dĺžku $r_1-r_2$ má tiež úsečka~$S_1L$, lebo $r_1=|S_1S_2|$ 
a~$r_2=|S_2L|$. Trojuholníky $S_1MS_2$ a~$S_1LP$ majú teda zhodné
uhly pri vrchole~$S_1$ aj priľahlé strany, sú preto zhodné podľa
vety~$sus$. Platí teda nielen $S_1M\perp S_2M$, ale aj
$S_1L\perp PL$. Bod~$L$ však leží na kružnici~$k_2$, takže priamka~$PL$
je jej dotyčnicou, ktorá s~ohľadom na súmernosť prechádza taktiež
bodom~$Q$. Dôkaz je skončený.

\ineriesenie
Označme $O$~priesečník oboch dotyčníc, $K$~pätu kolmice z~bodu~$P$ na~$OS_1$
(vzhľadom na~súmernosť oboch dotyčníc podľa spojnice~$S_1S_2$
je to priesečník $PQ$ s~$OS_1$) a~$R$~pätu kolmice
\inspicture{}
z~bodu~$S_2$ na~$OP$ (\obr). Pretože $|S_1P|=|S_1S_2|=r_1$, platí
$|\uh S_1PS_2|=|\uh S_1S_2P|$, preto
$$
\align
|\uh S_2PK|=&90\st-|\uh S_1S_2P|=\\
           =&90\st-|\uh S_1PS_2|=|\uh S_2PR|,
\endalign
$$
takže pravouhlé trojuholníky $KS_2P$ a~$RS_2P$ sa zhodujú v~prepone~$S_2P$
a~priľahlom uhle pri vrchole~$P$. Preto $|S_2K|=|S_2R|$
a~kružnica so stredom~$S_2$ a~polomerom $r_2=|S_2R|$ sa dotýka
spojnice~$PQ$ v~bode~$K$.}

{%%%%%   A-S-3
Ľavú stranu druhej rovnice upravíme na súčin.
$$
x^3-2x^2-px+2p=x^2(x-2)-p(x-2)=(x-2)(x^2-p).
$$
Pre spoločný koreň~$x$ oboch rovníc teda platí $x=2$ alebo
$x^2=p$. V~prvom prípade po dosadení do prvej rovnice dostaneme
$$
2^3+2^2-36\cdot2-p=0,\quad\text{čiže}\quad p=\m60;
$$
v~druhom prípade môžeme prvú rovnicu zjednodušiť na tvar
$x^3-36x=0$, odkiaľ vyplýva $x=0$ alebo $x=\pm6$, a~preto
z~podmienky $p=x^2$ vychádza $p=0$ resp.~$p=36$.

Dodajme, že po nájdení rozkladu ľavej strany druhej rovnice sme
mohli vypísať jej korene $x_1=2$, $x_{2,3}=\pm\sqrt{p}$ a~po ich
postupnom dosadení do prvej rovnice určiť hľadané hodnoty
$p=\m60$, $p=0$ a~$p=36$.

\ineriesenie
Z~prvej rovnice ľahko vyjadríme
$p=x^3+x^2-36x$ a~dosadením do druhej rovnice dostaneme rovnicu
$$
x^3-2x^2-(x^3+x^2-36x)x+2(x^3+x^2-36x)=0.
$$
(bez parametra~$p$), ktorú musí spĺňať spoločný koreň oboch
pôvodných rovníc.
Po úprave dostaneme rovnicu $x^4-2x^3-36x^2+72x=0$, ktorej korene
ľahko určíme (sú to totiž celé čísla) napríklad postupným
rozkladom
$$
\align
x^4-2x^3-36x^2+72x&=x[x^2(x-2)-36(x-2)]=x(x-2)(x^2-36)=\\
                  &=x(x-2)(x-6)(x+6).
\endalign
$$
Vidíme, že spoločným koreňom musí byť jedno z~čísel $x_1=0$,
$x_2=2$, $x_3=6$, $x_4={-6}$. Keď ich dosadíme do pôvodných rovníc,
ihneď zistíme príslušné hodnoty~$p$; sú to čísla $0$, ${-60}$
a~$36$ (posledné zodpovedá obom koreňom $x_{3,4}=\pm6$).

\ineriesenie
Spoločné korene mnohočlenov
$$
P_1(x)=x^3+x^2-36x-p,\quad
P_2(x)=x^3-2x^2-px+2p
$$
(ak vôbec existujú) sú korene mnohočlena, ktorý je najväčším
spoločným deliteľom mnohočlenov $P_1$ a~$P_2$. Nájdeme ho
Euklidovým algoritmom postupného delenia zo zvyškom. V~prvých
dvoch krokoch dostaneme ako zvyšky mnohočleny
$$\align
P_3(x)&=P_1(x)-P_2(x)=3x^2+(p-36) x-3p,\\
P_4(x)&=P_2(x)-\left(\frac{x}{3}+\frac{30-p}{9}\right)P_3(x)=
        (p-36)\left(\frac{(p-30) x}{9}-\frac{p}{3}\right).
\endalign
$$
V~prípade, keď $p=36$, je algoritmus skončený; najväčší spoločný
deliteľ je vtedy rovný $P_3(x)=3x^2-3\cdot36=3(x-6)(x+6)$, takže
mnohočleny $P_1$, $P_2$ majú dva spoločné korene $x=\pm6$. Ďalej
preto predpokladajme, že $p\ne36$. Jediný kandidát na spoločný
koreň mnohočlenov $P_1$, $P_2$ je koreň mnohočlena~$P_4$, teda číslo
$x=3p/(p-30)$. Stačí iba zistiť, kedy je toto číslo koreňom
mnohočlena~$P_3$. Pretože
$$
P_3\left(\frac{3p}{p-30}\right)=
\frac{9p(p+60)}{(p-30)^2},
$$
majú požadovanú vlastnosť iba hodnoty $p=0$ a~$p=\m60$ (ktorým
zodpovedá~spoločný koreň $x=0$ resp.~$x=2$).}

{%%%%%   A-II-1
Pretože v~zápise dvojmiestneho čísla vystupuje
číslica~$4$, nutne platí $z\geqq5$. Z~rozvinutých zápisov
$(1001)_z=z^3+1$ a~$(41)_z=4z+1$ vyplýva, že hľadáme práve tie
prirodzené $z\geqq5$, pre ktoré je číslo $z^3+1$ násobkom čísla
$4z+1$. Pomocou Euklidovho algoritmu nájdeme ich najväčší
spoločný deliteľ. Môžeme postupovať tak, že najprv vydelíme oba
výrazy ako mnohočleny a~potom sa "zbavíme" zlomkov.
$$
\align
    z^3+1&=\Bigl(\frac14z^2-\frac{1}{4^2}z+\frac{1}{4^3}\Bigr)(4z+1)
          +\frac{63}{4^3},\qquad/\cdot4^3\\
4^3(z^3+1)&=(16z^2-4z+1)(4z+1)+63.        \tag1
\endalign
$$
Pretože čísla $4$ a~$4z+1$ sú nesúdeliteľné, vidíme odtiaľ,
že číslo $4z+1$ delí číslo $z^3+1$, práve vtedy, keď delí číslo
$63$, teda práve vtedy, keď $4z+1\in\{1,3,7,9,21,63\}$.
Z~podmienky $z\geqq5$ však vyplýva $4z+1\geqq21$, takže
$4z+1=21$ (rovnica $4z+1=63$ nemá celočíselné riešenie) a~$z=5$.

\poznamka
Rozklad~$(1)$ tiež ľahko odhalíme, keď využijeme známy vzorec
$a^3+b^3=(a+b)(a^2-ab+b^2)$. Podľa neho môžeme rovno písať
$$
4^3(z^3+1)=(4^3z^3+1)+63=(4z+1)(16z^2-4z+1)+63.
$$}

{%%%%%   A-II-2
\fontplace
\tpoint A; \tpoint B; \bpoint\xy.5,0 C;
\tpoint S;
\rpoint X; \lpoint\xy-.5,0 Y;
\cpoint\a; \cpoint\b;
\tpoint90\st-\a; \tpoint90\st-\b;
[10] \hfil\Obr

\fontplace
\tpoint A; \tpoint B; \bpoint C;
\tpoint S; \rbpoint\xy1,0 S_0;
\rpoint X; \lpoint\xy-.5,0 Y;
\tpoint X_0; \tpoint\xy2,0 Y_0; \tpoint C_0;
\cpoint; \cpoint;
[11] \hfil\Obr

\fontplace
\tpoint A; \tpoint B; \bpoint C;
\tpoint S;
\rpoint X; \lpoint\xy-.5,0 Y;
\cpoint\a; \cpoint\b;
\cpoint\a; \cpoint\down.5\unit\b;
[12] \hfil\Obr

Vnútorné uhly trojuholníka~$ABC$ označme ako zvyčajne $\al$, $\be$,
$\ga$.
Podľa vety o~obvodovom a~stredovom uhle v~kružnici opísanej trojuholníku~$ASC$
\inspicture{}
platí (\obr) $|\uh SXC|=2\al$, takže uhol pri základni~$SC$
rovnoramenného trojuholníka~$SCX$ má veľkosť
$|\uh XSC|=(180\st-2\al)/2=90\st-\al$ (využili
sme predpoklad, že $\al$ je ostrý uhol). Analogicky
sa odvodí rovnosť $|\uh YSC|=90\st-\be$.
Pretože uhly pri vrcholoch $A$ a~$C$ trojuholníka~$ASC$ sú ostré, je
stred~$X$ vnútorným bodom uhla~$ASC$; podobne je stred~$Y$
vnútorným bodom uhla~$BSC$. Preto možno vyjadriť
veľkosť uhla~$XSY$ ako súčet veľkostí uhlov $XSC$ a~$YSC$.
$$
|\uh XSY|=|\uh XSC|+|\uh YSC|=(90\st-\al)+(90\st-\be)=\ga.
$$
Ak označíme ešte $\om=|\uh ASC|$, potom pre polomery kružníc
opísaných trojuholníkom $ASC$ a~$BSC$ platia vzťahy
$$
|SX|=\frac{|AC|}{2\sin\om}\quad\text{a}\quad
|SY|=\frac{|BC|}{2\sin(180\st-\om)}=\frac{|BC|}{2\sin\om},
$$
ktoré spolu so skôr určenou veľkosťou uhla~$XSY$
vedú k~nasledujúcej závislosti medzi obsahmi $S_{SXY}$ 
a~$S_{ABC}$ trojuholníkov $SXY$ a~$ABC$:
$$
S_{SXY}=\frac{|SX|\cdot|SY|\cdot\sin|\uh XSY|}{2}=
\frac{|AC|\cdot|BC|\cdot\sin\ga}{8\sin^2\om}=
\frac{S_{ABC}}{4\sin^2\om}.
$$
Odtiaľ vyplýva nerovnosť $S_{SXY}\geqq S_{ABC}/4$, pričom
rovnosť nastane práve vtedy, keď $\sin\om=1$, čiže $\om=90\st$.
Obsah trojuholníka~$SXY$ je preto najmenší práve vtedy, keď je bod~$S$
pätou výšky z~vrcholu~$C$ na stranu~$AB$. (Táto päta je vnútorným
bodom strany~$AB$ vďaka podmienke, že trojuholník~$ABC$ je ostrouhlý.)

\ineriesenie
Spojnica~$XY$ stredov oboch opísaných
kružníc pretína spoločnú tetivu~$CS$ v~jej stredu~$S_0$ a~kolmé
priemety $X_0$, $Y_0$ bodov $X$, $Y$ na stranu~$AB$ sú stredmi
úsečiek $AS$, $SB$ (\obr). Preto
$|X_0Y_0|=|AB|/2$ a~pre obsah trojuholníka~$SXY$ platí
$$
\align
S_{SXY}=&\frac12|XY|\cdot|S_0S|\ge\frac12|X_0Y_0|\cdot|S_0S|=\\
       =&\frac12\cdot\frac12|AB|\cdot\frac12|CS|
       \ge\frac14\cdot \frac12 |AB|\cdot|CC_0|=\frac14 S_{ABC},
\endalign
$$
kde $CC_0$~je výška trojuholníka~$ABC$. Rovnosť v~prvej z~predchádzajúcich
dvoch nerovností nastane práve vtedy, keď $XY\parallel AB$, \tj. práve vtedy,
keď $CS\perp AB$, čiže $S=C_0$. A~práve vtedy prejde na rovnosť
aj druhá nerovnosť.
\midinsert
\centerline{\inspicture-!\hss\inspicture-!}
\endinsert

\ineriesenie
Priesečníky $C$ a~$S$ kružníc opísaných
trojuholníkom $ASC$ a~$BSC$ sú súmerne združené podľa priamky~$XY$,
takže pre veľkosť uhla~$SXY$ platí (\obr)
$$
|\uh SXY|=\frac12|\uh SXC|=\frac12\cdot2|\uh BAC|=|\uh BAC|,
$$
podobne $|\uh SYX|=|\uh ABC|$. Preto sú trojuholníky $SXY$ a~$CAB$
podobné podľa vety~$uu$, takže ich obsahy $S_{SXY}$ 
a~$S_{ABC}$ sú pomocou koeficientu podobnosti $k=|XS|:|AC|$
zviazané rovnosťou $S_{SXY}=k^2 S_{ABC}$. Pretože úsečka~$AC$
je tetivou kružnice s~polomerom~$|XS|$, platí nerovnosť $|AC|\leqq
2|XS|$, čiže $k\geqq 1/2$; rovnosť $k=1/2$ pritom
nastane len vtedy, keď je strana~$AC$ priemerom kružnice opísanej
trojuholníku~$ASC$, čo je ekvivalentné s~podmienkou $CS\perp AB$. Tým je
dokázaná nerovnosť $S_{SXY}\geqq S_{ABC}/4$ aj nájdená
podmienka, kedy nastane rovnosť.}

{%%%%%   A-II-3
Rovnice danej sústavy majú zmysel len vtedy, keď sú
čísla $x$, $y$, $z$ kladné a~rôzne od~$1$. Pre také čísla
$x$, $y$, $z$ (iné ďalej neuvažujeme) dostávame odlogaritmovaním
ekvivalentnú sústavu rovníc
$$
y+z=x^p,\quad z+x=y^p,\quad x+y=z^p.
\tag1
$$
Ukážeme najprv, že v~obore $\mm M=(0,1)\cup(1,\infty)$ je každé
riešenie sústavy~$(1)$ tvorené trojicou rovnakých čísel. Využijeme
na to známy poznatok, že pre $p\geqq 0$ je funkcia $f(t)=t^p$
na množine kladných čísel~$t$ neklesajúca (presnejšie rastúca pre
$p>0$ a~konštantná pre $p=0$). Predpokladajme naopak, že pre niektoré
riešenie~$(x,y,z)$ platí napríklad $x<y$.
Odpočítaním prvých dvoch rovníc z~$(1)$ dostaneme $y-x=x^p-y^p$.
Z~predpokladu $x<y$ ale vyplýva $x^p\le y^p$, takže $y-x>0$ 
a~zároveň $x^p-y^p\le 0$, čo je v~spore s~predchádzajúcou rovnosťou.
Podobne odvodíme spor aj v~prípade, keď $x>y$, a~v~prípadoch, keď
$x\ne z$ resp.~$y\ne z$ (sústava~$(1)$ je totiž v~neznámych $x$,
$y$, $z$ symetrická).

Sústava~$(1)$ sa preto redukuje na rovnicu $x+x=x^p$, ktorú máme
riešiť v~obore $\mm M=(0,1)\cup(1,\infty)$. Pretože $x\ne0$,
dostávame po delení číslom~$x$ ekvivalentnú rovnicu $2=x^{p-1}$.
Táto rovnica nemá riešenie pre $p=1$, pre $p=0$ má jediné riešenie
$x=1/2$, pre prirodzené $p\geqq2$ má jediné riešenie
$x=2^{1/(p-1)}$, ktoré možno pre $p\geqq3$ zapísať ako
$x=\root{p-1}\of{2}$. (Čísla $1/2$ aj $2^{1/(p-1)}$
zrejme patria do $\mm M$.)

\odpoved
Daná sústava má pre $p=0$ jediné riešenie
$x=y=z=1/2$, pre $p=1$ nemá riešenie, pre prirodzené $p\geqq2$
má jediné riešenie $x=y=z=2^{1/(p-1)}$.}

{%%%%%   A-II-4
Na úvod si všimneme, že v~dôsledku rovnosti $x_1=1$ platí
$$
\aligned
x_2&=x_{1}^{\pm1}=1^{\pm1}=1,\\
x_3&=x_{2}^{\pm1}+x_{1}^{\pm1}=1^{\pm1}+1^{\pm1}=2.
\endaligned              \tag1
$$
Pretože pre každé $x>0$ sú obe čísla $x^{+1}$, $x^{-1}$ kladné,
vyplýva odtiaľ jednoducho matematickou indukciou, že nerovnosť $x_n\geqq1$ je splnená
pre každé~$n$. Ale ak $x\geqq1$, potom $0<x^{-1}\leqq x^{1}$,
a~preto pre každé $n\geqq4$ platia odhady
$$
\postdisplaypenalty 10000
1+1+\frac{1}{2}+\frac{1}{x_4}+\dots+\frac{1}{x_{n-1}}\leqq
x_n\leqq  1+1+2+x_4+\dots+x_{n-1},
\tag2
$$
ktoré využijeme vo všetkých troch častiach riešenia.

\smallskip
a) Dokážeme (sporom), že existuje~$k$, pre ktoré $x_k>10^3$.
Predpokladajme naopak, že pre každé~$k$ platí opačná nerovnosť
$x_k\leqq10^3$. Z~ľavej nerovnosti v~$(2)$ potom pre každé $n>4$
vyplýva odhad
$$
\postdisplaypenalty 10000
x_n\geqq\frac{5}{2}+
\underbrace{10^{-3}+10^{-3}+\dots+10^{-3}}_{(n-4)\,\text{krát}}
=\frac{5}{2}+(n-4)\cdot10^{-3}.
$$
Odtiaľ ale vyplýva, že $x_n>10^3$ pre každé $n>10^6+4$, čo je spor.

\smallskip
b) Dokážeme najskôr, že z~pravých nerovností v~$(2)$ vyplýva
odhad $x_n\leqq2^{n-2}$ pre každé $n\geqq2$. Využijeme indukciu.
Pre $n=2$ aj pre $n=3$ platí podľa~$(1)$ rovnosť $x_n=2^{n-2}$;
nech $n\geqq4$ a~nech pre všetky $k\in\{2,3,\dots,n-1\}$ platí
$x_k\leqq2^{k-2}$, potom z~$(2)$ dostávame
$$
x_n\leqq 1+(1+2+2^2+\dots+2^{n-3})=1+(2^{n-2}-1)=2^{n-2}.
$$
Tým je dôkaz indukciou ukončený.
Keď dosadíme odhady $x_n\leqq2^{n-2}$ do ľavej nerovnosti v~($2$),
vyjde nám pre hodnotu~$x_{10^6}$ dolný odhad
$$
x_{10^6}\geqq
2+\frac{1}{2}+\frac{1}{2^2}+\dots+\frac{1}{2^{10^6-3}}=
3-\frac{1}{2^{10^6-3}}.
$$
To spolu s~príkladom vyhovujúcej postupnosti
$$
\align
x_n&=x_{n-1}+x_{n-2}+\dots+x_1\quad(n\in\{2,3,\dots,10^6-1\}),\\
x_{10^6}&=x_{10^6-1}^{-1}+x_{10^6-2}^{-1}+\dots+
x_2^{-1}+x_1^{-1},
\endalign
$$
v~ktorej $x_n=2^{n-2}$ pre každé $n\in\{2,3,\dots,10^6-1\}$ 
a~$x_{10^6}=3-2^{3-10^6}$, ukazuje, že najmenšia možná hodnota
člena~$x_{10^6}$ je rovná $3-2^{3-10^6}$.

\smallskip
c) Predpokladajme, že nerovnosť $x_n<4$ platí okrem troch hodnôt
$n\in\{1,2,3\}$ ešte pre niektoré ďalšie~$n$, ktoré označíme
$n_4,n_5,n_6,\dots$ tak, že $4\leqq n_4<n_5<n_6<\dots$ (zatiaľ
ešte nevieme, či ide o~konečnú alebo nekonečnú postupnosť).
Ukážme, že pre každé také~$n_k$ sú vo všetkých exponentoch
príslušnej rovnosti
$$
x_{n_k}=1+1+2^{\pm1}+x_4^{\pm1}+x_5^{\pm1}+\dots+x_{n_k-1}^{\pm1}
$$
vybrané znamienka "mínus". Pre mocninu~$2^{\pm1}$ je to zrejmé,
lebo $x_{n_k}<4$; z~rovnakej nerovnosti ďalej
vyplýva, že znamienko v~exponente ktorejkoľvek mocniny~$x_j^{\pm1}$ ($4\leqq
j\leqq n_k-1$) musí byť vybrané tak, aby platilo
$$
x_j^{\pm1}<4-\frac52=\frac32.
$$
Táto nerovnosť však môže byť splnená iba so znamienkom
"mínus", lebo podľa~$(2)$ máme $x_j\geqq5/2$. Tým je
tvrdenie o~výbere znamienok dokázané. Porovnaním dvoch za sebou
idúcich rovností
$$
\align
x_{n_k}&=\frac52+\frac{1}{x_4}+\frac{1}{x_5}+\dots+
\frac{1}{x_{n_k-1}},\\
x_{n_{k+1}}&=\frac52+\frac{1}{x_4}+\frac{1}{x_5}+\dots+
\frac{1}{x_{n_k-1}}+\frac{1}{x_{n_k}}+\dots+\frac{1}{x_{n_{k+1}-1}}
\endalign
$$
dostaneme pre všetky $k\geqq4$ nerovnosti
$$
x_{n_{k+1}}\geqq x_{n_k}+\frac{1}{x_{n_k}},
$$
ktoré s~prihliadnutím k~tomu, že funkcia $f(t)=t+{1}/{t}$ je na
intervale $t\in\langle1,\infty)$ rastúca a~že $x_{n_4}\geqq2{,}5$,
vedú postupne k~odhadom
$$\gather
x_{n_5}\geqq f(2{,}5)=2{,}9,\enspace
x_{n_6}\geqq f(2{,}9)>3{,}24,\enspace
 x_{n_7} > f(3{,}24)>3{,}54,\\
x_{n_8}>f(3{,}54)>3{,}82,\enspace  x_{n_9}\geqq f(3{,}82)>4{,}08.
\endgather
$$
Posledná nerovnosť je ale v~spore s~podmienkou $x_{n_k}<4$
určujúcou výber indexov~$n_k$. Preto nerovnosť $x_n<4$ nemôže
platiť pre deväť indexov~$n$.}

{%%%%%   A-III-1
Pretože druhú rovnicu môžeme upraviť na tvar $xy(x+y)=-2$,
upravme podobne aj prvú rovnicu: $(x+y)^2-3xy=7$. Pre čísla
$s=x+y$, $p=xy$ tak dostávame ekvivalentnú sústavu
$$
\aligned
s^2-3p&=7,\\
    sp&=-2,
\endaligned \tag1
$$
ktorá po vyjadrení $p=-2/s$ (zrejme nemôže byť $s=0$) z~druhej
rovnice vedie na
kubickú rovnicu $s^3-7s+6=0$. Tá má celočíselné
korene $s_1=1$, $s_2=2$ a~$s_3={-3}$. Nájdeným hodnotám~$s$
zodpovedajú tieto hodnoty súčinu $p=xy$: $p_1={-2}$, $p_2={-1}$,
$p_3=2/3$. Čísla $x$, $y$ tvoria dvojicu koreňov kvadratickej
rovnice $t^2-st+p=0$, takže sa jedná o~jednu z~rovníc
$$
t^2-t-2=0,\quad t^2-2t-1=0,\quad t^2+3t+\frac23=0.
$$
Ich riešením dostaneme (všetkých) šesť riešení danej sústavy
$$
\gather
\{x,y\}=\{-1,2\},\quad
\{x,y\}=\bigl\{1+\sqrt2,1-\sqrt2\bigr\},\\
\{x,y\}=\left\{\frac{-9+\sqrt{57}}{6},\frac{-9-\sqrt{57}}{6}\right\}.
\endgather
$$}

{%%%%%   A-III-2
\fontplace
\tpoint A; \tpoint B; \bpoint C;
\lBpoint D; \rBpoint E; \tpoint F;
\bpoint P; \bpoint Q; \rpoint R;
\tpoint a; \rBpoint a; \tpoint b; \lBpoint b;
\lBpoint c; \rBpoint c;
\lBpoint x; \lBpoint x; \bpoint x;
\rBpoint y; \rBpoint y; \bpoint y;
\tpoint z; \tpoint z; \rpoint\xy.5,0 z;
\tpoint S_1; \tpoint S_2; \lpoint S_3;
\rpoint\down\unit G;
[13] \hfil\Obr

Predpokladajme, že spomenuté štvoruholníky majú uvedenú vlastnosť.
Zo súmernosti dotyčníc z~daného bodu k~danej kružnici
vyplýva, že strany trojuholníka~$ABC$ sú rozdelené bodmi $D$, $E$, $F$
a~bodmi dotyku kružníc vpísaných uvažovaným štvoruholníkom na úseky
dĺžok, ktoré označíme podľa \obr. Sú na ňom tiež vyznačené
body $P$, $Q$, $R$ vzájomného dotyku spomenutých kružníc. Naším
cieľom je dokázať rovnosti $x=y=z$ a~$a=b=c$.
\inspicture{}

Pre úseky dotyčníc z~bodu~$A$ ku kružniciam pri strane~$BC$
platia rovnosti $a+2z=|AP|=a+2y$, odkiaľ ihneď vyplýva $y=z$; z~dôvodov
symetrie teda naozaj platí $x=y=z$. (Všade ďalej budeme písať $x$
namiesto $y$ a~$z$.) Všimnime si
teraz trojuholníky $AEG$ a~$AFG$. Majú spoločnú stranu~$AG$ a~zhodné
strany $AF$ a~$AE$ (dĺžky $a+x$). Tiež ich tretie strany $EG$
a~$FG$ sú zhodné, lebo
$$
|EG|=|EQ|+|QG|=x+|RG|=|FR|+|RG|=|FG|.
$$
Trojuholníky $AEG$ a~$AFG$ sú teda zhodné podľa vety~$sss$, takže
uhly $BAD$ a~$CAD$ sú zhodné a~polpriamka~$AD$ je osou
uhla~$BAC$. Ako vieme, os uhla trojuholníka pretína protiľahlú
stranu v~pomere dĺžok priľahlých strán. V~našom prípade to
znamená, že
$$
\frac{a+2x+b}{a+2x+c}=\frac{b+x}{c+x}.
$$
Jednoduchou úpravou dostaneme rovnosť $(b-c)(a+x)=0$, z~ktorej
vidíme, že $b=c$. Z~dôvodov symetrie teda platí $a=b=c$
a~celý dôkaz je hotový.

\ineriesenie
Označme $S_1$, $S_2$, $S_3$ stredy vpísaných kružníc (\obrr1).
Rovnako ako v~predchádzajúcom riešení si najprv všimneme, že platí
$x=y=z$ a~že trojuholníky $AEG$ a~$AFG$ sú zhodné. K~tomu
sme využili rovnosť $|GQ|=|GR|$, z~ktorej vyplýva, že podľa vety~$sss$
sú zhodné aj trojuholníky $S_1QG$ a~$S_1RG$. Keďže
$R\in S_1S_2$ a~$Q\in S_1S_3$, zo súmernosti podľa
osi~$AD$ teraz vyplýva, že priamky $AB$ a~$S_1S_2$ zvierajú rovnaký
uhol ako priamky $AC$ a~$S_1S_3$, a~pretože kolmé priemety úsečiek
$S_1S_2$ a~$S_1S_3$ na zodpovedajúce priamky~$AB$, resp.~$AC$ sú
zhodné (majú dĺžku~$2x$), platí $|S_1S_2|=|S_1S_3|$. Analogicky
$|S_1S_2|=|S_2S_3|$, takže trojuholník~$S_1S_2S_3$ je rovnostranný.
Odtiaľ pre polomery $r_1$, $r_2$ a~$r_3$ vpísaných kružníc vyplýva
$r_1+r_2=r_2+r_3=r_3+r_1$, čiže $r_1=r_2=r_3$. Kružnice sú
teda zhodné, takže $AB\parallel S_1S_2$, $BC\parallel S_2S_3$
a~$CA\parallel S_3S_1$ a~trojuholník~$ABC$ je rovnostranný.

\poznamka
K~dokončeniu predchádzajúceho dôkazu môžeme úvahu o~dĺžkach
úsečiek~$S_iS_j$ nahradiť úvahou o~tzv.~orientovaných
uhloch medzi priamkami. Orientovaný uhol~$\<p,q\>$ priamok $p$,
$q$ (v~tomto poradí) je uhol, o~ktorý musíme v~kladnom smere
otočiť priamku~$q$, aby bola rovnobežná s~priamkou~$p$. Pritom
$\<p,q\>=\<q,p\>$ práve vtedy, keď $\<p,q\>$ je násobok~$90\st$. Zo
súmernosti podľa osí $AD$, $BE$ a~$CF$ tak postupne dostávame
$\<S_1S_3,AC\>=\<AB,S_1S_2\>=\<S_2S_3,BC\>=\<AC,S_1S_3\>$.
Pretože obe zodpovedajúce kružnice so stredmi $S_1$, $S_3$ majú
spoločnú dotyčnicu~$AC$ a~ležia v~rovnakej polrovine určenej
priamkou~$AC$, znamená to, že $S_1S_3$ a~$AC$ sú rovnobežné.}

{%%%%%   A-III-3
Pokiaľ sa nám podarí zostaviť podľa daného pravidla
$(k+3)$-člennú postupnosť
$$
x_1=1,\ x_2,\ x_3,\ \dots,\ x_{k-1},\ x_k=x_{k+1}=x_{k+2}=x_{k+3}=12,
$$
môžeme všetky nasledujúce členy $x_{k+4},x_{k+5},x_{k+6},\dots$
definovať tak, aby sa tiež rovnali číslu~$12$. Skutočne, vzhľadom na
matematickú indukciu stačí ukázať, ako s~vytýčeným cieľom
vybrať znamienka v~rovnosti určujúcej člen~$x_{k+4}$. Položme
$$
\align
x_{k+4}=&\p12(k+3)-12(k+2)-12(k+1)+12k\pm\\
        &\pm(k-1)x_{k-1}\pm(k-2)x_{k-2}\pm\dots\pm x_1,
\endalign
$$
pritom znamienka v~druhom riadku vyberieme presne také,
aké boli v~súčte určujúcom člen $x_k=12$.
Potom sa súčet v~druhom riadku rovná~$12$, takže vychádza
$$
x_{k+4}=12(k+3)-12(k+2)-12(k+1)+12k+12=12.
$$
Vhodný príklad pre $k=8$ je $x_1=x_2=x_3=1$,
$$
%\medmuskip3mu
\align
x_4&=3-2+1=2,\\
x_5&=4\cdot2-3-2+1=4,\\
x_6&=5\cdot4-4\cdot2-3-2-1=6,\\
x_7&=6\cdot6-5\cdot4-4\cdot2+3-2+1=10,\\
x_8&=7\cdot10-6\cdot6-5\cdot4-4\cdot2+3+2+1=12,\\
x_9&=8\cdot12-7\cdot10-6\cdot6+5\cdot4+4\cdot2-3-2-1=12,\\
x_{10}&=9\cdot12-8\cdot12-7\cdot10+6\cdot6+5\cdot4+4\cdot2+3+2+1=12,\\
x_{11}&=10\cdot12+9\cdot12-8\cdot12-7\cdot10-6\cdot6-5\cdot4
        +4\cdot2-3+2-1=12.
\endalign
$$
Dokázali sme, že jedna z~uvažovaných postupností má iba
prvých sedem členov rôznych od čísla~$12$.}

{%%%%%   A-III-4
\fontplace
\lBpoint A; \trpoint B; \tlpoint\xy-.5,0 C;
\trpoint D; \rpoint\down\unit O;
\tpoint\xy1,0 K; \rbpoint S;
\bpoint o; \lpoint p; \rBpoint k;
[14] \hfil\Obr

{\it Rozbor\/}. Predpokladajme, že trojuholník~$ABC$ má všetky požadované
vlastnosti a~označme~$D$ priesečník kružnice~$k$ opísanej trojuholníku~$ABC$
s~polpriamkou opačnou k~ramenu~$KA$ daného uhla~$AKS$ (\obr).
Polpriamka~$AD$ rozpoľuje uhol~$BAC$, preto sú uhly $BAD$ a~$CAD$
zhodné, takže sú zhodné aj tetivy $BD$ a~$CD$ kružnice~$k$. Bod~$S$
je preto stredom základne~$BC$ rovnoramenného trojuholníka~$BCD$,
takže uhol~$BSD$ je pravý. To znamená, že bod~$D$ leží na
priamke~$p$, ktorá prechádza bodom~$S$ kolmo na dané rameno~$KS$.
Stred~$O$ kružnice~$k$ leží jednak na priamke~$p$ (osi tetivy~$BC$),
jednak na priamke~$o$, ktorá je osou tetivy~$AD$.
\inspicture{}

\konstrukcia
Pre daný uhol~$AKS$ najprv preložíme bodom~$S$ priamku~$p$ kolmú
na rameno~$KS$. Potom zostrojíme priesečník~$D$ priamky~$p$
s~polpriamkou opačnou k~ramenu~$KA$. Ďalej zostrojíme os~$o$
úsečky~$AD$ a~jej priesečník s~priamkou~$p$ označíme~$O$. Konečne
zostrojíme kružnicu~$k$ so stredom~$O$ a~polomerom $r=|OA|$
$(=|OD|)$ a~jej priesečníky s~priamkou~$KS$ označíme $B$ a~$C$.

\smallskip
{\it Dôkaz správnosti konštrukcie\/}.
Ukážeme, že zostrojený trojuholník~$ABC$ má všetky požadované
vlastnosti. Z~posledného kroku konštrukcie vyplýva, že body $B$, $C$
ležia na priamke~$KS$ a~že bod~$S$ je stredom úsečky~$BC$. Pretože
na priamke~$p$, osi úsečky~$BC$, leží aj bod~$D$, platí
$|BD|=|CD|$, a~preto $|\uh BAD|=|\uh CAD|$ (lebo všetky
body $A$, $B$, $C$, $D$ ležia na kružnici~$k$.) Polpriamka~$AD$ je
teda osou uhla~$BAC$ a~bod~$K$ je jej priesečníkom s~úsečkou~$BC$.

\smallskip
{\it Diskusia\/}.
Vysvetlíme, prečo pre daný tupý uhol~$AKS$ je hľadaný trojuholník~$ABC$
jediný (keď neberieme do úvahy možnosť zameniť označenie vrcholov $B$
a~$C$). Pretože je uhol~$AKS$ tupý, bod~$D$ z~našej konštrukcie
zrejme existuje a~priamky $p$ a~$o$ sú rôznobežné, takže
aj bod~$O$ je určený jednoznačne. Zostáva zdôvodniť, prečo
kružnica~$k$ pretne priamku~$KS$ v~dvoch bodoch. Pretože bod~$K$
je vnútorným bodom základne~$AD$ rovnoramenného trojuholníka~$ADO$, platí
$|OK|<|OA|=|OD|=r$, teda bod~$K$ leží vo vnútornej oblasti
kružnice~$k$ a~priamka~$KS$ je nutne jej sečnicou.}

{%%%%%   A-III-5
Hľadané dvojmiestne čísla $A$, $B$ majú tvar $A=az+b$ a~$B=bz+a$,
kde $a$, $b$ sú ich (nenulové!) číslice, takže
$a,b\in\{1,2,\dots,z-1\}$. Kvadratická rovnica z~textu úlohy má
dvojnásobný koreň~$x_0$ práve vtedy, keď platí $2x_0=A$ a~$x_0^2=B$.
Z~týchto rovností vyplýva, že číslo~$x_0$ je kladné a~celé.
Vzhľadom na nerovnosť $x_0^2=B<z^2$ ($B$ je totiž dvojmiestne,
zatiaľ čo $z^2$ je trojmiestne) naviac platí $x_0<z$, odkiaľ
$A=2x_0<2z$, takže číslo~$A$ má ako prvú číslicu jednotku.
Platí teda $a=1$ a~z~rovností $2x_0=z+b$ a~$x_0^2=bz+1$
vylúčením~$x_0$ dostaneme pre číslicu~$b$ kvadratickú rovnicu
$(b-z)^2=4$ s~dvoma koreňmi $b_1=z-2$ a~$b_2=z+2$. Za číslicu~$b$
možno však zobrať iba prvú z~nich, takže nutne $b=z-2$.

Dokázali sme, že čísla $A$, $B$ musia byť tvaru
$A=z+(z-2)=2z-2$ a~$B=({z-2})z+1=(z-1)^2$; v~sústave so základom~$z$
teda majú zápisy $A=\overline{1(z-2)}$ a~$B=\overline{(z-2)1}$.
Urobme ešte skúšku. Kvadratická rovnica
$x^2-(2z-2)x+(z-1)^2=0$ má naozaj dvojnásobný koreň $x_0=z-1$,
lebo jej ľavá strana je rovná $(x-z+1)^2$.

\poznamka
Kľúčovú rovnosť $a=1$ možno odvodiť aj bez úvahy 
o~dvojnásobnom koreni~$x_0$ skúmanej rovnice, keď zapíšeme
podmienku, že jej diskriminant $A^2-4B$ je rovný nule:
$$
0=A^2-4B=(az+b)^2-4(bz+a)=b^2+2z(a-2)b+a(az^2-4).
$$
Posledný výraz môže mať nulovú hodnotu len vtedy, keď je činiteľ
$a-2$ záporný, lebo $b\geqq1$, $a\geqq1$, $z\geqq3$ 
a~$az^2-4\geqq3^2-4=5$. Z~nerovnosti $a-2<0$ už však vyplýva
$a=1$. Pre také $a$ dostávame rovnicu $0=b^2-2zb+(z^2-4)$ 
a~záver je rovnaký ako v~uvedenom riešení.}

{%%%%%   A-III-6
K~daným kladným číslam $a$, $b$, $c$ spĺňajúcim
podmienku $abc=1$ zapíšeme AG-nerovnosť pre trojicu čísel
$a/b$, $a/b$ a~$b/c$.
$$
\frac13\left(\frac{a}{b}+\frac{a}{b}+\frac{b}{c}\right)\geqq
\root3\of{\dfrac{a}{b}\cdot\dfrac{a}{b}\cdot\dfrac{b}{c}}=
\root3\of{\dfrac{a^2}{bc}}=\root3\of{\dfrac{a^3}{abc}}=a.
$$
Platí teda odhad
$$
\dfrac{2a}{3b}+\dfrac{b}{3c}\geqq a.
$$
Z~rovnakého dôvodu platia aj odhady
$$
\dfrac{2b}{3c}+\dfrac{c}{3a}\geqq b\quad\text{a}\quad
\dfrac{2c}{3a}+\dfrac{a}{3b}\geqq c.
$$
Sčítaním týchto troch odhadov dostaneme dokazovanú nerovnosť.

\ineriesenie
Ak platí pre kladné čísla $a$, $b$, $c$ rovnosť $abc=1$, potom
$\max\{a,b,c\}\geqq1$ a~$\min\{a,b,c\}\leqq1$. Pretože dokazovaná
nerovnosť sa nezmení, keď nahradíme trojicu $(a,b,c)$ trojicou
$(b,c,a)$ alebo trojicou $(c,a,b)$, budeme predpokladať, že čísla
$a$ a~$c$ sú z~trojice $(a,b,c)$
najmenšie a~najväčšie (v~nejakom poradí), takže platí
$$
(a-1)(1-c)\geqq0.
\tag1
$$
Do dokazovanej nerovnosti dosadíme $c=a^{-1}b^{-1}$
a~urobíme niekoľko ekvivalentných úprav.
$$
\align
ab^{-1}+ab^2+a^{-2}b^{-1}&\geqq a+b+a^{-1}b^{-1},\quad
/\cdot a^2b\\
a^3+a^3b^3+1&\geqq a^3b+a^2b^2+a,\\
a^3b^3-a^2b^2-a^3b+a^3-a+1&\geqq0,\\
a^3(b^3-b^2-b+1)+(a^3-a^2)b^2-(a-1)&\geqq0,\\
a^3(b-1)^2(b+1)+(a-1)(ab-1)(ab+1)&\geqq0.
\endalign
$$
Posledná nerovnosť platí, lebo
$(a-1)(ab-1)=(a-1)(ab-abc)=ab(a-1)(1-c)$ a~taký
súčin je podľa~$(1)$ nezáporný.

\ineriesenie
Pre ľubovoľné kladné čísla $A$, $B$, $C$ sú
trojice $(A^2,B^2,C^2)$ a~$(A,B,C)$ tzv.~súhlasne
usporiadané, teda platí nerovnosť
$$
A^2\cdot A+B^2\cdot B+C^2\cdot C\geqq
A^2\cdot B+B^2\cdot C+C^2\cdot A.
\tag2
$$
Dokážme~$(2)$ bezprostredne. Úpravou dostávame nerovnosť
$$
(A-C)^2(A+C)+(B-C)(B^2-A^2)\geqq 0,
$$
ktorá zrejme platí, pokiaľ $B=\max\{A,B,C\}$, čo možno vždy
dosiahnuť cyklickou permutáciou danej trojice čísel.
Keď zvolíme v~dokázanej nerovnosti~$(2)$ hodnoty
$A=\root{3}\of{a/b}$, $B=\root{3}\of{b/c}$
a~$C=\root{3}\of{c/a}$, dostaneme nerovnosť
$$
\frac ab+\frac bc+\frac ca\geqq
\root{3}\of{\frac{a^2}{bc}}+\root{3}\of{\frac{b^2}{ca}}+
\root{3}\of{\frac{c^2}{ab}},
$$
z~ktorej za predpokladu $abc=1$ vyplýva dokazovaná nerovnosť.}

{%%%%%   B-S-1
Ľubovoľné z~uvažovaných päťmiestnych čísel má v~desiatkovej sústave
zápis tvaru~$\overline{abcba}$. Jeho rozvinutím a~úpravou získame
rovnosť
$$
\overline{abcba}=10\,001a+1\,010b+100c=101(99a+10b+c)+2a-c.
$$
Odtiaľ vyplýva, že skúmané číslo je deliteľné~$101$ práve vtedy, keď
$2a-c=0$ (pre ľubovoľné číslice $a$, $c$ totiž iste platí
$|2a-c|<101$). Z~rovnosti $2a=c$ vyplýva $a\leqq4$, a~pretože
hľadáme čo najväčšie také číslo, zvolíme jeho prvú číslicu~$a=4$,
ktorej odpovedá číslica~$c=8$. Pretože číslica~$b$
nemá na deliteľnosť číslom~$101$ vplyv, zvolíme ju čo najväčšiu:
$b=9$. Hľadané číslo je teda~$49\,894$.}

{%%%%%   B-S-2
\fontplace
\tpoint A; \tpoint B; \lpoint C; \bpoint D;
\tpoint K; \tpoint L; \lbpoint M; \bpoint\xy-1,0 N;
\ltpoint\xy-1,.3 P; \lBpoint\down\unit Q; \lbpoint R;
[7] \hfil\Obr

Stredy strán štvoruholníka~$ABCD$ označme $K$, $L$, $M$, $N$ podľa
\obr.
\inspicture{}
Pretože úsečky $KL$ a~$MN$ sú postupne stredné priečky trojuholníkov
$ABC$ a~$ACD$, platí $KL\parallel AC\parallel MN$. Obdobne
platí $LM\parallel BD\parallel KN$, takže $KLMN$ je rovnobežník 
a~bod~$Q$ rozpoľuje úsečku~$KM$. Všimnime si teraz trojuholník~$KMN$. Stredom~$Q$
jeho strany~$KM$ prechádza podľa predpokladu úlohy uhlopriečka~$BD$,
ktorá je, ako vieme, rovnobežná s~druhou stranou~$KN$. Preto
aj stred~$R$ tretej strany~$MN$ leží na uhlopriečke~$BD$. Pretože
úsečka~$MN$ je rovnoľahlá s~úsečkou~$CA$ podľa stredu~$D$, rozpoľuje
uhlopriečka~$BD$ nielen úsečku~$MN$ (v~bode~$R$), ale aj úsečku~$AC$
(v~zodpovedajúcom bode~$P$).}

{%%%%%   B-S-3
Dané prirodzené čísla označme podľa veľkosti
$x_1<x_2<x_3<x_4<x_5$. Pretože platí
$$
x_1+x_2<x_1+x_3<x_1+x_4<x_1+x_5<
x_2+x_5<x_3+x_5<x_4+x_5,
$$
je medzi všetkými súčtami $x_i+x_j$ aspoň sedem rôznych hodnôt.
Nevypísané zostali iba tri z~možných súčtov, a~to súčty
$x_2+x_3$, $x_2+x_4$ a~$x_3+x_4$. Preto pre počet~$p$ možných hodnôt
uvažovaných súčtov platí $7\le p\le 10$.
Pre každú z~hodnôt
$p\in\{7,8,9,10\}$ uvedieme príklad päťprvkovej množiny $\mm M_p$
prirodzených čísel, pre ktorú uvažované súčty nadobúdajú práve $p$~rôznych
hodnôt (ich množinu označíme $\mm S_p$):
$$
\aligned
\mm M_7=&\{1,2,3,4,5\},\cr
\mm M_8=&\{1,2,3,4,6\},\cr
\mm M_9=&\{1,2,3,4,7\},\cr
\mm M_{10}=&\{1,2,3,5,8\},
\endaligned \qquad
\aligned
   \mm S_7=&\{3,4,5,6,7,8,9\};\cr
   \mm S_8=&\{3,4,5,6,7,8,9,10\};\cr
   \mm S_9=&\{3,4,5,6,7,8,9,10,11\};\cr
\mm S_{10}=&\{3,4,5,6,7,8,9,10,11,13\}.\cr
\endaligned
$$}

{%%%%%   B-II-1
Medzi $109$ po sebe idúcimi päťmiestnymi číslami
$$
10\,902,\ 10\,903,\dots, 10\,999,\  11\,000,\dots,
11\,009,\ 11\,010
$$
nie je žiadny palindróm (je možné uviesť aj iné vyhovujúce
príklady $109$~päťmiestnych čísel,
my sme vypísali skupinu najmenších z~nich).
Najmenší a~najväčší päťmiestny palindróm sú čísla $10\,001$
a~$99\,999$; pred číslom~$10\,001$ je len jedno päťmiestne číslo,
za číslom~$99\,999$ už dokonca žiadne také číslo nie je. Ukážeme
teraz, že za každým päťmiestnym palindrómom
$x$, $x\ne99\,999$,
nasleduje päťmiestny palindróm $x+100$ alebo $x+110$ alebo $x+11$.
Skutočne, ak $x=\overline{abcba}$, tak v~prípade $c\ne9$
je palindrómom číslo $x+100=\overline{ab(c+1)ba}$, v~prípade $c=9\ne b$
je palindrómom číslo $x+110=\overline{a(b+1)0(b+1)a}$ a v~prípade $c=b=9$
(keď nutne $a\ne9$) je palindrómom číslo
$x+11=\overline{(a+1)000(a+1)}$.

\odpoved
Hľadaný najväčší počet čísel je rovný~$109$.}

{%%%%%   B-II-2
\fontplace
\tpoint A; \tpoint B; \bpoint C;
\tpoint K; \lBpoint L; \rBpoint M;
\tpoint kc; \tpoint (1-k)c;
\lBpoint(1-k)a; \lBpoint ka;
\rBpoint kb; \rBpoint(1-k)b;
[8] \hfil\Obr

\fontplace
\rpoint A; \lpoint B; \bpoint C;
\tpoint\down\unit K; \tpoint\down\unit K';
\lbpoint L; \lbpoint L'; \rBpoint M;
[9] \hfil a)

\fontplace
\rpoint A; \lpoint B; \bpoint C;
\tpoint\down\unit K; \tpoint\down\unit K';
\lbpoint L; \rbpoint M; \rbpoint M';
[10] \hfil b)

Pretože uhly $KLC$, $KMC$ a~$LCM$ sú pravé~(\obr), je štvoruholník~$KLCM$
pravouholník a~trojuholníky $AKM$ a~$KBL$ sú podobné s~trojuholníkom~$ABC$.
Označme ako obyčajne $a=|BC|$, $b=|AC|$, $c=|AB|$ 
a~položme $|AK|=kc$, kde $0<k<1$. Potom však $|KB|=(1-k)c$ a~zo spomenutej
podobnosti trojuholníkov dostávame vyjadrenie $|AM|=kb$, $|LC|=|KM|=ka$,
$|BL|=(1-k)a$ a~$|MC|=|KL|=(1-k)b$. Preto platí
$$
\align
S_{ABLM}&=S_{ABC}-S_{LMC}=\frac12ab-\frac12\cdot ka\cdot(1-k)b=\\
        &=\frac12ab(1-k+k^2)=\frac12ab
\left(\Bigl(k-\frac12\Bigr)^2+\frac34\right)\geqq\\
        &\geqq\frac12ab\cdot\frac34=\frac34S_{ABC},
\endalign
$$
pričom rovnosť $S_{ABLM}=\frac34S_{ABC}$ nastane práve vtedy, keď
$k=1/2$, teda práve vtedy, keď je bod~$K$ stredom prepony~$AB$.
\inspicture{}

\ineriesenie
Štvoruholník~$ABLM$ má minimálny obsah práve vtedy,
keď má maximálny obsah trojuholník~$LMC$, ktorý je "polovicou"
pravouholníka~$KLCM$. Stačí preto ukázať, že obsah~$S_{KLCM}$ je
maximálny práve vtedy, keď je bod~$K$ stredom prepony~$AB$ (keď zrejme
$S_{KLCM}=S_{ABC}/2$). Ak je bod~$K$ vybraný tak, že
$|AK|<|AB|/2$, je úsečka~$KL$ strednou priečkou lichobežníka~$AK'L'C$,
ktorý má o~$S_{K'L'B}$ menší obsah ako trojuholník~$ABC$
(\obr a),
takže platí
$$
S_{KLCM}=\frac12S_{AK'L'C}<\frac12S_{ABC}.
$$
\midinsert
\centerline{\inspicture-!\hss\inspicture-!}
\centerline\Obr
\endinsert

Ak naopak $|AK|>\frac12|AB|$, využijeme obdobný lichobežník~$BK'M'C$
(\obrr1b) a~usúdime, že platí
$$
S_{KLCM}=\frac12S_{BK'M'C}<\frac12S_{ABC}.
$$
Tým je tvrdenie o~maximálnom obsahu~$S_{KLCM}$ dokázané.

\odpoved
Štvoruholník~$ABLM$ má najmenší možný obsah práve vtedy,
keď bod~$K$ leží uprostred prepony~$AB$.}

{%%%%%   B-II-3
\fontplace
\thickmuskip3mu \medmuskip2mu
\rtpoint O; \tpoint x; \lpoint y;
\tpoint V;
\bpoint\xy21,9 \ssize y=(3-p)x;
\rtpoint\xy-1,-1 \ssize y=(-3-p)x;
\bpoint \ssize y=(x-1)^2;
[11] \hfil a) $p=2$

\fontplace
\thickmuskip3mu \medmuskip2mu
\rtpoint O; \tpoint x; \lpoint y;
\tpoint T_1=V;
\bpoint \ssize y=(3-p)x;
\rtpoint\xy-2,0 \ssize y=(-3-p)x;
\bpoint \ssize y=(x-1)^2;
[12] \hfil b) $p=3$

\fontplace
\thickmuskip3mu \medmuskip2mu
\rtpoint O; \tpoint x; \lpoint y;
\tpoint V; \rpoint T_2;
\bpoint\xy7,18 \ssize y=(3-p)x;
\rtpoint\xy6,-20 \ssize y=(-3-p)x;
\lpoint\xy-2,-7 \ssize y=(x-1)^2;
[13] \hfil c) $p=1$

Aj keď danú úlohu možno riešiť názorne geometrickou úvahou
o~vzájomnej polohe paraboly $y=(x-1)^2$ a~lomenej čiary $y=3|x|-px$,
dáme najprv prednosť čisto algebraickému postupu.
Daná rovnica zrejme nemá riešenie~$x=0$. Po
odstránení absolútnej hodnoty a~jednoduchej úprave dostaneme rovnice
$$
\align
x^2+(p+1)x+1&=0\qquad\text{pre}\ x<0,\tag1\\
x^2+(p-5)x+1&=0\qquad\text{pre}\ x>0.\tag2
\endalign
$$
Pretože každá kvadratická rovnica má najviac dva rôzne korene,
hľadáme všetky tie čísla~$p$, pre ktoré má jedna z~rovníc (1), (2)
jeden koreň a~druhá dva rôzne korene (a~to vždy predpísaných
znamienok). Všimnime si, že pre každé $q\in\Bbb R$ majú reálne
korene~$x_{1,2}$
rovnice $x^2+qx+1=0$ (ak vôbec existujú) rovnaké znamienko,
ktoré je opačné ako znamienko čísla~$q$;
platí totiž $x_1x_2=1$ a~$x_1+x_2={-q}$.
Pre rovnice (1), (2) tak hlavne dostávame podmienky
$$
p+1>0\quad\text{a}\quad p-5<0,\quad\text{čiže}\quad
p\in(-1,5).
$$
Okrem toho už len požadujeme, aby pre diskriminanty oboch rovníc
$$
D_1=(p+1)^2-4,\quad D_2=(p-5)^2-4
$$
platilo buď $D_1=0$ a~$D_2>0$, alebo $D_1>0$ a~$D_2=0$. Rovnosť
$D_1=0$ platí iba pre $p\in\{-3,1\}$, rovnosť $D_2=0$
iba pre $p\in\{3,7\}$. Z~týchto štyroch hodnôt ležia v~intervale
$({-1},5)$ iba čísla $p=1$ a~$p=3$, pričom pre $p=1$
vychádza $D_2=12>0$, pre $p=3$ zasa $D_1=12>0$.

\odpoved
Hľadané hodnoty sú $p=1$ a~$p=3$.

\ineriesenie
Grafom funkcie $y=(x-1)^2$ je parabola s~vrcholom~$V[1,0]$,
grafom funkcie $y=3|x|-px$ je lomená čiara tvorená ramenami niektorého
uhla s~vrcholom~$O[0,0]$ (\obr{}a pre $p=2$).
\vadjust{\medskip
\line{\hss\inspicture-!\hss\hss\inspicture-!\hss}
\line{\hss\inspicture-!\hss}
\centerline\Obr
\medskip}%
Oba grafy majú spoločné tri body práve vtedy, keď jedno z~ramien
spomenutého uhla je dotyčnicou paraboly a~druhé je jej "sečnicou".
Pretože skúmaná parabola nemá dotyčnicu rovnobežnú s~osou~$y$,
môžeme rovnice oboch dotyčníc prechádzajúcich bodom~$[0,0]$ hľadať v~tvare
$y=kx$. Ako je známe, smernica~$k$ sa určí z~podmienky, že
rovnica $kx=(x-1)^2$ má dvojnásobný koreň, teda nulový
diskriminant. Ten má vyjadrenie $(k+2)^2-4$, takže hľadané hodnoty
sú $k_1=0$, $k_2={-4}$ a~zodpovedajúce body dotyku $T_1=V[1,0]$
a~$T_2[{-1},4]$. Z~rovníc pre smernice dotykových ramien skúmaných
uhlov $3-p=0$ a~$-3-p={-4}$ nájdeme riešenie $p_1=3$ a~$p_2=1$
a~ľahko sa presvedčíme, že druhé rameno je v~oboch prípadoch
skutočne sečnicou paraboly (\obrr1b pre $p=3$ a~\obrr1c pre $p=1$).}

{%%%%%   B-II-4
\fontplace
\tpoint A; \tpoint B; \bpoint\xy-1,0 C; \bpoint D;
\rpoint S_1; \lBpoint S_2; \tlpoint P;
\tlpoint k_1; \tlpoint k_2;
[14] \hfil\Obr

\fontplace
\tpoint A; \tpoint B; \bpoint\xy-1,0 C; \bpoint D;
\trpoint P;
\tlpoint k_1; \tlpoint k_2;
[15] \hfil\Obr

\fontplace
\tpoint A; \tpoint B; \bpoint\xy-1,0 C; \bpoint D;
\lpoint P; \tpoint E;
\tlpoint k_1; \tlpoint k_2;
[16] \hfil\Obr

Označme $S_1$ a~$S_2$ stredy uvažovaných kružníc
(\obr). Obe úsečky $S_1A$ a~$S_2B$ sú kolmé na priamku~$AB$,
\inspicture{}
sú teda rovnobežné a~striedavé uhly $PS_2B$ a~$PS_1D$
zhodné. Podľa vety 
o~obvodových a~striedavých uhloch preto platí
$$
|\uh PCB|=\frac12|\uh PS_2B|=\frac12|\uh PS_1D|=|\uh PAD|.
$$
Oba uhly $APD$ a~$ADC$ sú však pravé, takže
$$
|\uh PAD|=90^{\circ}-|\uh ADP|=|\uh CDP|.
$$
Spolu dostávame, že uhly $PCB$ a~$CDP$ sú zhodné, čo podľa
vety o~obvodovom a~úsekovom uhle znamená, že priamka~$BC$ je
dotyčnicou ku kružnici opísanej trojuholníku~$CDP$.

\ineriesenie
V~rovnoľahlosti so stredom~$P$, v~ktorej
kružnica~$k_1$ prejde na kružnicu~$k_2$, musí dotyčnica~$CD$
kružnice~$k_1$ prejsť na rovnobežnú dotyčnicu~$AB$ kružnice~$k_2$, pritom sa bod
dotyku~$D$ zobrazí do bodu dotyku~$B$. Bod~$P$ preto leží na
uhlopriečke~$BD$ (\obr). Odtiaľ vyplýva zhodnosť striedavých uhlov $CDP$
a~$PBA$ (medzi rovnobežkami $AB$ a~$CD$).
Uhol~$PBA$ je ale úsekový uhol medzi tetivou~$BP$ a~dotyčnicou~$AB$
kružnice~$k_2$, je teda zhodný s~príslušným obvodovým uhlom~$PCB$.
Uhly $CDP$ a~$PCB$ sú preto zhodné, čo sme potrebovali
dokázať (ako v~závere predchádzajúceho riešenia).
\inspicture{}

\poznamka
Podľa úlohy B--I--5 sú zhodné uhly $ABC$ a~$CPD$ (\obr).
\inspicture{}
Pretože sú zhodné aj striedavé uhly $PEB$ a~$PCD$, kde $E$ je
priesečník polpriamky~$CP$ so stranou~$AB$, možno požadovanú zhodnosť
uhlov $CDP$ a~$PCB$ odvodiť z~trojuholníkov $BCE$ a~$PDC$.}

{%%%%%   C-S-1
Označme hľadané číslo~$10a + b$, kde $a$, $b$
sú celé čísla, $a \ge 1$, $0\le b \le 9$.
Podľa zadania má platiť
$$
10a + b = |a^2  - b^2|.
$$

Predpokladajme najprv, že $a \ge b$. V~tom prípade jednoduchými
úpravami dostávame
$$
\align
        10a + b &= a^2  - b^2 ,\\
a^2  - 10a + 25 &= b^2  + b + 25,\\
     (a~- 5)^2  &= b^2  + b + 25.
\endalign
$$
Do poslednej rovnosti potom postupne dosadzujeme $b = 0$, $b = 1$,~
$\dots$, $b = 9$ a~zisťujeme, či výraz $b^2  + b + 25$ je
druhou mocninou nejakého nezáporného celého čísla. Rovnici
vyhovujú dvojice $b = 0$, $a = 0$; $b = 0$, $a = 10$; $b = 7$, $a = 14$.

\smallskip
V~prípade, keď $a < b$, obdobnými úpravami dostaneme
$$
\align
        10a + b &= b^2  - a^2 ,\\
a^2  + 10a + 25 &= b^2  - b + 25,\\
     (a~+ 5)^2  &= b^2  - b + 25
\endalign
$$
a~podobne ako v~prvom prípade získame dvojice
$b = 0$, $a = 0$; $b = 1$, $a = 0$; $b = 8$, $a = 4$.

\zaver
S~prihliadnutím k~podmienkam zadania sú riešením
úlohy tri čísla $48$, $100$, $147$.}

{%%%%%   C-S-2
\fontplace
\tpoint A; \tpoint B; \bpoint C; \bpoint D;
\tpoint\down\unit P; \lpoint\xy.5,.5 Q; \bpoint E;
\cpoint\ssize30\st; \cpoint\ssize60\st;
\cpoint\ssize30\st; \cpoint\ssize60\st;
\cpoint\xy.5,.5 \ssize30\st; \cpoint\ssize60\st;
\tpoint a; \rpoint a; \lpoint a;
\bpoint x; \rBpoint2x;
[4] \hfil\Obr

\fontplace
\tpoint A; \tpoint B; \bpoint C; \bpoint D;
\tpoint\xy.5,-.5 P; \rpoint\xy-.5,-1.2 Q;
\bpoint E; \bpoint O;
\tpoint G; \lBpoint S; \bpoint\xy-1,1 R;
[5] \hfil\Obr

\fontplace
\tpoint A; \tpoint B; \bpoint C; \bpoint D;
\point ; \rpoint\xy1,.5 P; \lpoint\xy.5,.5 Q;
\bpoint E; \bpoint X; \tpoint\down1.4\unit Y;
\bpoint\xy-.5,.5 Z; \tpoint X';
\tpoint G; \bpoint\xy-.5,.5 S; \lpoint R;
\bpoint\xy.5,.5 U;
[6] \hfil\Obr

Označme~$a$ dĺžku strany štvorca~$ABCD$. Trojuholníky $AED$,
$BAP$ a~$CBQ$ sú podobné podľa vety $uu$, pričom trojuholníky $BAP$ 
a~$CBQ$ sú dokonca zhodné (\obr). Trojuholník~$AED$ je polovicou
rovnostranného trojuholníka so stranou~$AE$.
Ak označíme $|ED| = x$, tak $|AE| = 2x$.
\inspicture{}

V~pravouhlom trojuholníku~$AED$ platí
$$
a = |AD| = \sqrt {|AE|^2 -|ED|^2 } = \sqrt {4x^2  -x^2 }  = x\sqrt 3,
$$
odkiaľ $x=\left(\sqrt 3/3\right)a$. (Veľkosť~$x$ môžeme tiež spočítať
použitím goniometrického vzorca ${x}:{a} = {|ED|}:{|AD|}= \tg30\st
= \sqrt 3/3$.)

Trojuholníky $BAP$ a~$CBQ$ sú polovicami rovnostranného
trojuholníka so stranou~$a$. Rovnostranný trojuholník so stranou
dĺžky~$a$ má výšku $\left(\sqrt 3/2\right)a$ a~jeho obsah je
$\left(\sqrt 3/4\right)a^2$. Súčet obsahov trojuholníkov $AED$, $BAP$
a~$CBQ$ je teda
$$
\frac{1}{2} \cdot \frac{\sqrt 3}{3}a \cdot a+
\frac{\sqrt 3}{4}a^2 =\frac{5\sqrt 3}{12}a^2.
$$
Keďže obsah štvorca $ABCD$ je~$a^2$, je pomer obsahov
lichobežníka $PQCE$ a~štvorca $ABCD$ rovný
$$
\frac{a^2-\frac5{12}\sqrt 3a^2}{a^2}=\frac{12 - 5\sqrt3}{12},
$$
čo je číslo menšie ako $0{,}29$.

\zaver
Obsah lichobežníka~$PQCE$ je menší ako tretina
obsahu štvorca~$ABCD$.

\medskip
Pre zaujímavosť uvedieme ešte jedno riešenie, v~ktorom ukážeme, že
skúmaný obsah sa dá odhadnúť pomocou úvah o~vzájomnej polohe
vhodných bodov (bez výpočtu dĺžok a~obsahu).

\ineriesenie
Pretože nás zaujímajú len pomery obsahov, môžeme
predpokladať, že $ABCD$ je štvorec so stranou~$1$.
V~stredovej súmernosti podľa stredu štvorca~$O$ prejdú body
$E$, $P$ a~$Q$ do bodov, ktoré označíme $G$, $R$ a~$S$ (\obr).
Z~pravouhlého trojuholníka~$AED$ s~uhlom $60\st$ pri vrchole~$E$ vyplýva
$$
|DE|=\frac12|AE|>\frac12|AD|=\frac12,
$$
takže pre obsah rovnobežníka~$AGCE$ platí nerovnosť
$$
S_{AGCE}<\frac12.
$$
Zároveň sa zdá, že zhodné lichobežníky~$RCES$ a~$AGQP$ majú
väčší obsah ako štvorec~$PQRS$. Ak to tak naozaj je, musí
byť $S_{RCES}>S_{AGCE}/3$, takže nutne platí
$$
\align
S_{PQCE}=&S_{AGCE}-S_{AGQP}=S_{AGCE}-S_{RCES}<\\
       <&\frac23 S_{AGCE}<\frac23\cdot\frac12=\frac13.
\endalign
$$
Tým bude úloha vyriešená.

\midinsert
\centerline{\inspicture-!\hss\inspicture-!}
\endinsert

Strana~$SR$ štvorca~$PQRS$ je súčasne výškou lichobežníka~$RCES$.
Preto bude nerovnosť $S_{PQRS}<S_{RCES}$ dokázaná,
keď overíme, že strana štvorca
je kratšia ako stredná priečka lichobežníka. Tou je úsečka~$XZ$,
kde $Z$ označuje stred úsečky~$SR$, čo je zároveň päta výšky
rovnostranného trojuholníka~$XYD$ (\obr). Označme~$U$ priesečník
uhlopriečky~$AC$ daného štvorca s~úsečkou~$PQ$. Týmto bodom
prechádza aj priamka~$DY$, ktorá je súmerne združená s~priamkou~$BP$
práve podľa osi~$AC$, pretože $|\uh YDA|=|\uh ABP|=30\st$. To však
znamená, že bod~$Y$, ktorý je priesečníkom $DU$ a~$XZ$, leží mimo
štvorca~$PQRS$! Preto naozaj $|XZ|=|ZY|>|QR|$.
Obsah lichobežníka~$PQCE$ je teda menší ako tretina obsahu
štvorca~$ABCD$.}

{%%%%%   C-S-3
Označme a~zapíšme v~desiatkovej sústave päť päťmiestnych čísel,
ktoré sa čítajú spredu rovnako ako zozadu a~sú zostavené
z~daných číslic:
$$
\align
\overline{a_1b_1c_1b_1a_1}=&a_1\cdot10^4+b_1\cdot10^3+c_1\cdot10^2+b_1\cdot10+a_1,\cr
\overline{a_2b_2c_2b_2a_2}=&a_2\cdot10^4+b_2\cdot10^3+c_2\cdot10^2+b_2\cdot10+a_2,\cr
\overline{a_3b_3c_3b_3a_3}=&a_3\cdot10^4+b_3\cdot10^3+c_3\cdot10^2+b_3\cdot10+a_3,\cr
\overline{a_4b_4c_4b_4a_4}=&a_4\cdot10^4+b_4\cdot10^3+c_4\cdot10^2+b_4\cdot10+a_4,\cr
\overline{a_5b_5c_5b_5a_5}=&a_5\cdot10^4+b_5\cdot10^3+c_5\cdot10^2+b_5\cdot10+a_5.
\endalign
$$
Medzi číslicami $c_1$, $c_2$, $c_3$, $c_4$, $c_5$ je práve
jedna jednotka, práve jedna dvojka, práve jedna trojka, práve
jedna štvorka a~práve jedna päťka. Keby totiž na mieste stoviek
uvažovaných piatich čísel chýbala napr.~jednotka, musela by sa na
miestach ostatných rádov vyskytovať v~nepárnom počte (päťkrát), čo
vzhľadom na symetriu uvažovaných čísel nie je možné.
Pre súčet~$S$ uvažovaných čísel teda platí
$$
\align
S=&\overline{a_1b_1c_1b_1a_1}+\overline{a_2b_2c_2b_2a_2}
  +\overline{a_3b_3c_3b_3a_3}+\overline{a_4b_4c_4b_4a_4}
  +\overline{a_5b_5c_5b_5a_5}=\\
 =&(a_1+a_2+a_3+a_4+a_5)\cdot(10^4+1)+\\
  &+(b_1+b_2+b_3+b_4+b_5)\cdot(10^3+10)+\\
  &+(c_1+c_2+c_3+c_4+c_5)\cdot10^2=\\
=&10\,001\cdot(a_1+a_2+a_3+a_4+a_5)
 +1\,010\cdot(b_1+b_2+b_3+b_4+b_5)+\\
 &+100\cdot(1+2+3+4+5)=\\
=&10\,001\cdot(a_1+a_2+a_3+a_4+a_5)+\\
 &+1\,010\cdot(b_1+b_2+b_3+b_4+b_5)+1\,500.
\endalign
$$
S~ohľadom na číslice, ktoré máme k~dispozícii, bude súčet~$S$
najmenší, keď bude
$$
\align
a_1+a_2+a_3+a_4+a_5=&1+1+2+2+3=9,\\
b_1+b_2+b_3+b_4+b_5=&5+5+4+4+3=21.
\endalign
$$
Najmenší možný súčet má teda hodnotu
$$
S_{\min}=10\,001\cdot9+1\,010\cdot21+1\,500=112\,719
$$
a~vznikne napr.~ako súčet
$$
S_{\min}=13\,131+14\,241+24\,342+25\,452+35\,553.
$$
Podobne bude súčet~$S$ najväčší, keď bude
$$
\align
a_1+a_2+a_3+a_4+a_5=&5+5+4+4+3=21,\\
b_1+b_2+b_3+b_4+b_5=&1+1+2+2+3=9.
\endalign
$$
Najväčší možný súčet má teda hodnotu
$$
S_{\max}=10\,001\cdot21+1\,010\cdot9+1\,500=220\,611
$$
a~vznikne napr.~ako súčet
$$
S_{\max}=53\,535+52\,425+42\,324+41\,214+31\,113.
$$}

{%%%%%   C-II-1
Pre každé z~dvojmiestnych prvočísiel $97$, $89$, $83$, $79$, $73$,~{\dots}
hľadáme jeho najmenší násobok, ktorý prevyšuje číslo~$2\,003$.
Vzhľadom na~to, že medzi $k$ po sebe idúcimi celými číslami je
práve jedno deliteľné~$k$, a~pretože $97\cdot21 = 2\,037$,
$89\cdot23 = 2\,047$, $83\cdot25 = 2\,075$, $79\cdot26 = 2\,054$,
musí byť $2\,003 + n\ge 2\,075$, teda $n\ge 72$. Pre také~$n$ máme
zaručené, že pre každé z~prvočísiel $97$, $89$, $83$, $79$ je medzi číslami
$2\,003$, $2\,004$, $2\,005$,~ $\dots$, $2\,003+n$ aspoň
jedno ním deliteľné.
Medzi uvedenými $73$~číslami $2\,003$ až $2\,075$ je vždy aspoň jedno deliteľné
prvočíslom~$73$, aspoň jedno deliteľné prvočíslom~$71$ atď.
Hľadané číslo~$n$ je teda~$72$.}

{%%%%%   C-II-2
\fontplace
\tpoint A; \tpoint B; \lpoint C; \bpoint D; \bpoint E; \rpoint F;
\bpoint P; \tpoint Q; \ltpoint S; \tpoint X;
[7] \hfil\Obr

V~pravidelnom šesťuholníku~$ABCDEF$ so stredom~$S$, v~ktorom $Q$
je stred strany~$AB$ a~$P$ je stred strany~$DE$, poznáme veľkosť
\inspicture{}
uhla~$PAQ$ (\obr), pretože všetky pravidelné šesťuholníky sú
navzájom podobné. V~pravouhlom trojuholníku~$APQ$ teda poznáme
dĺžku prepony~$AP$ a~veľkosti dvoch uhlov ($AQP$ je pravý uhol).
Odtiaľ vyplýva postup {\it konštrukcie\/}:
\ite 1. úsečka~$AP$;
\ite 2. Tálesova kružnica~$k$ nad priemerom~$AP$;
\ite 3. polpriamka~$AX$, ktorá zviera s~úsečkou~$AP$ uhol veľkosti~$PAQ$
(ten zostrojíme pomocou ľubovoľného pravidelného šesťuholníka);
\ite 4. bod~$Q$ ako priesečník kružnice~$k$ s~polpriamkou~$AX$;
\ite 5. stred~$S$ úsečky~$PQ$;
\ite 6. kružnica so stredom~$S$ a~polomerom~$|SQ|$;
\ite 7. pravidelný šesťuholník~$ABCDEF$.

\noindent
Úloha má dve riešenia súmerne združené podľa osi~$AP$ podľa toho,
v~ktorej polrovine s~hraničnou priamkou~$AP$ zostrojíme
polpriamku~$AX$ (bod~3 konštrukcie).}

{%%%%%   C-II-3
V~prvom prípade platí
$$
1\,000p\cdot{p\over100}+1\,000q\cdot{q\over100}
=1\,000(p+q)\cdot{p+2{,}4\over100},
$$
v~druhom prípade platí
$$
1\,000p\cdot{2p\over100}+1\,000q\cdot{2q\over100}
=1\,000(p+q)\cdot{p+5{,}8\over100}.
$$
Úpravou oboch rovníc získame sústavu
$$
\align
  p^2+q^2=&(p+2{,}4)(p+q), \tag1\\
2p^2+2q^2=&(p+5{,}8)(p+q).
\endalign
$$
Pretože ľavá strana druhej rovnice je dvojnásobkom ľavej strany
prvej rovnice, musí platiť
$$
2(p+2{,}4)(p+q)=(p+5{,}8)(p+q).
$$
Odtiaľ po vykrátení nenulovým výrazom~$p+q$ vychádza~$p=1$.
Dosadením tejto hodnoty napr.~do rovnice~(1) a~po úprave získame
kvadratickú rovnicu
$$
q^2-3{,}4q-2{,}4=0.
$$
Pretože hľadáme celočíselné korene, prepíšeme rovnicu do tvaru
$$
q(5q-17)=12
$$
a~ľahko zistíme, že medzi deliteľmi čísla~$12$ rovnici vyhovuje
jedine~$q=4$.}

{%%%%%   C-II-4
\fontplace
\tpoint\toleft\unit A; \tpoint\toright\unit B;
\bpoint C; \bpoint D; \tpoint\toright\unit E;
\tpoint\frac12(a+c); %\tpoint\frac12(a-c);
\bpoint c; \lBpoint b; \rBpoint v; \rBpoint u;
[8] \hfil\Obr

Označme~$E$ pätu kolmice spustenej z~vrcholu~$C$ na základňu~$AB$
rovnoramenného lichobežníka~$ABCD$ a~jednotlivé dĺžky úsečiek
označme takto (\obr): $|AB|=a=12$, $|BC|=b$, $|CD|=c=10$,
$|AC|=u$, $|CE|=v$. Potom $|BE|=(a-c)/2=1$,
$|AE|=(a+c)/2=11$.
\inspicture{}
Podľa Pytagorovej vety pre trojuholníky $AEC$ a~$EBC$
môžeme teda písať
$$
\gather
v^2=u^2-11^2=b^2-1^2,            \tag1   \\
\intext{alebo}
u^2-b^2=11^2-1^2=120.
\endgather
$$
Odtiaľ je vidieť, že čísla $u$ a~$b$ sú zároveň obe párne,
alebo obe nepárne, preto v~rozklade
$$
(u-b)(u+b)=120=2\cdot60=4\cdot30=6\cdot20=10\cdot12
$$
prichádzajú do úvahy len uvedené rozklady čísla~$120$ na párne
činitele.
Uvedeným rozkladom potom zodpovedajú štyri sústavy rovníc pre
neznáme $u$ a~$b$:
$$
\aligned u-b&=2,\\ u+b&=60; \endaligned\qquad
\aligned u-b&=4,\\ u+b&=30; \endaligned\qquad
\aligned u-b&=6,\\ u+b&=20; \endaligned\qquad
\aligned u-b&=10,\\ u+b&=12. \endaligned
$$
Ich riešením (najlepšie tak, že vždy odčítame druhú rovnicu od
prvej) dostaneme pre dĺžku ramena~$b$ lichobežníka~$ABCD$ štyri
možnosti, $b\in\{29,13,7,1\}$. Z~rovnosti~(1) však vidíme, že
musí byť $b>1$, úlohe teda vyhovujú len prvé tri hodnoty.

\odpoved
Možná dĺžka ramena lichobežníka je buď~$7$, alebo~$13$,
alebo~$29$.}

{%%%%%   vyberko, den 1, priklad 1
...}

{%%%%%   vyberko, den 1, priklad 2
...}

{%%%%%   vyberko, den 1, priklad 3
...}

{%%%%%   vyberko, den 1, priklad 4
...}

{%%%%%   vyberko, den 2, priklad 1
...}

{%%%%%   vyberko, den 2, priklad 2
...}

{%%%%%   vyberko, den 2, priklad 3
...}

{%%%%%   vyberko, den 2, priklad 4
...}

{%%%%%   vyberko, den 3, priklad 1
...}

{%%%%%   vyberko, den 3, priklad 2
...}

{%%%%%   vyberko, den 3, priklad 3
...}

{%%%%%   vyberko, den 3, priklad 4
...}

{%%%%%   vyberko, den 4, priklad 1
...}

{%%%%%   vyberko, den 4, priklad 2
...}

{%%%%%   vyberko, den 4, priklad 3
...}

{%%%%%   vyberko, den 4, priklad 4
...}

{%%%%%   vyberko, den 5, priklad 1
...}

{%%%%%   vyberko, den 5, priklad 2
...}

{%%%%%   vyberko, den 5, priklad 3
...}

{%%%%%   vyberko, den 5, priklad 4
...}

{%%%%%   trojstretnutie, priklad 1
Ukážeme najprv, že pre každé $i\in\{1,2,\dots,n\}$ platí
 $x_i\leq i$. Dôkaz urobíme sporom. Predpokladajme, že $x_i>i$
pre nejaké~$i$.

Ak $x_1>1$, vyplýva z~poslednej rovnice danej sústavy nerovnosť
$\max \{n,x_n\}=nx_1>n$, takže $x_n>n$. Ak ďalej pre nejaké
$i>1$ platí $x_i>i$, potom $\max \{i-1,x_{i-1}\}=(i-1)x_i>(i-1)i>i-1$.
Preto tiež $x_{i-1}>i-1$. Odtiaľ vyplýva, že ak nerovnosť
$x_i>i$ platí pre niektoré $i\in \{1,2,\dots,n\}$, potom už platí pre
každé $i\in\{1,2,\dots,n\}$. V~takom prípade má však daná sústava
tvar
$$
x_1=x_2,\enspace x_2=2x_3,\enspace \dots,\enspace
x_{n-1}=(n-1)x_n,\enspace x_n=nx_1.
$$
Vynásobením týchto rovníc dostaneme $x_1x_2\dots
x_n=n!\,x_1x_2\dots x_n$, čo neplatí pre žiadne prirodzené $n\geq
2$. Všetky~$x_i$ sú totiž kladné čísla. To je spor.

Pre všetky $i\in\{1,2,\dots,n\}$ teda $x_i\leq i$. Preto
$$
i=\max \{i,x_i\}=ix_{i+1},\qquad \text{pričom }x_{n+1}=x_1.
$$
Odtiaľ už ľahko získame jediné reálne riešenie danej sústavy
$$
x_1=x_2=\dots =x_n=1.
$$}

{%%%%%   trojstretnutie, priklad 2
\fontplace
\rpoint A; \lpoint B; \bpoint C; \lBpoint D;
\lpoint D'; \rbpoint E; \tpoint\xy-1,0 F; \trpoint K;
\cpoint\xy-.4,0 \g; \cpoint\g; \cpoint\g; \cpoint\g;
[1] \hfil\Obr

a) Označme veľkosti vnútorných uhlov daného trojuholníka~$ABC$
zvyčajným spôsobom a~uvažujme Tálesovu kružnicu zostrojenú nad
priemerom~$BC$. Vzhľadom na~to, že trojuholník~$ABC$ je
ostrouhlý, ležia päty výšok $E$, $F$ v~polrovine~$BCA$.
Z~vlastností tetivového štvoruholníka~$BCEF$ vyplýva (\obr), že
$|\uh AFE|=\gamma$ a~$|\uh AEF|=\beta$. Podobne
z~tetivového štvoruholníka~$AFDC$ vyplýva, že $|\uh DFB|=\gamma$.
Ak $K=A$, potom $|\uh DKF|=|\uh DAF|=90^{\circ}-\beta$
a~$|\uh KEF|=|\uh AEF|=\beta$.
\midinsert
\inspicture{}
\endinsert

Ak sa bod~$K$ bude spojito pohybovať po úsečke~$AF$ od
bodu~$A$ k~bodu~$F$, porastie veľkosť uhla~$DKF$ spojito
od hodnoty $90^{\circ}-\beta$ k~hodnote~$\gamma$ (v~ostrouhlom
trojuholníku je $90\st-\beta<\gamma$) a~súčasne bude veľkosť
uhla~$KEF$ spojito klesať, a~to od veľkosti
$\beta>90^{\circ}-\beta$ k~hodnote~$0^{\circ}$. Preto na úsečke~$AF$
existuje bod~$K$, pre ktorý platí $|\uh DKF|=|\uh KEF|$.

\smallskip
b) Nech $D'$ je obraz bodu~$D$ v~osovej súmernosti podľa priamky~$AB$.
Pretože $|\uh AFE|=|\uh DFB|=\gamma$, ležia body $E$,
$F$ a~$D'$ na jednej priamke.
Priamka~$KD'$ je podľa vety o~úsekovom uhle dotyčnicou kružnice~$k$
opísanej trojuholníku~$KFE$, pretože $|\uh D'KF|=|\uh DKF|=|\uh KEF|$.
Pre mocnosť bodu~$D'$ ku kružnici~$k$ platí
$$
\aligned
|KD'|^2&=|D'F|\cdot|D'E|=|D'F|(|D'F|+|FE|)=\\
       &=|D'F|^2+|D'F|\cdot|FE|.
\endaligned                      \tag1
$$
Teraz stačí využiť rovnosti $|FD|=|D'F|$ a~$|KD|=|KD'|$, ktoré
vyplývajú zo súmernosti bodov $D$ a~$D'$ podľa~$AB$ a~mocnosť
bodu~$F$ ku~kružnici s~priemerom~$AB$, podľa ktorej
$$
|EF|\cdot|FD'|=|AF|\cdot|BF|.
$$
Dosadením do~$(1)$ tak dostaneme $|KD|^2=|FD|^2+|AF|\cdot|BF|$, čo
sme chceli dokázať.}

{%%%%%   trojstretnutie, priklad 3
Zo zadania úlohy vyplýva, že niektoré dve z~čísel $p$, $q$, $r$ sú
buď nanajvýš rovné~$1$, alebo sú aspoň~$1$. Môžeme ich preto
označiť tak, že nastane jeden z~nasledujúcich dvoch prípadov:
\itemitem{(i)} $p\leq q\leq 1\leq r$,
\itemitem{(ii)} $r\leq 1\leq p\leq q$.

\smallskip
(i) Položme $a=q$, $b=1$, $c=pq$, potom platí $pa=pq=c$,
$qb=q=a$, $rc=pqr=1=b$. Trojuholníky, ktorých strany majú
veľkosti $a$, $b$, $c$ a~$pa$, $pb$, $pc$, sú teda zhodné
(ak existujú). Ukážeme, že trojuholník so stranami dĺžok $q$,
$1$, $pq$ existuje. Pretože $pq\leq q\leq 1$, stačí overiť
jedinú trojuholníkovú nerovnosť, a~to $pq+q>1$. Zo vzťahov
$$
pq=\frac1{r} \quad \hbox{a} \quad r\leq \frac52 \qquad
         \hbox{vyplýva} \qquad pq\geq \frac25.
$$
   Vzhľadom na to, že $p\leq q$, platí tiež
$$
q\geq \sqrt{\frac25}>\frac35,\qquad \hbox{a teda} \qquad
      pq+q>1.
$$

\smallskip
(ii) Položme opäť $a=q$, $b=1$, $c=pq$. Ukážeme, že aj v~tomto
prípade existuje trojuholník so stranami dĺžok $q$, $1$, $pq$.
Pretože teraz $pq\geq q\geq 1$, stačí overiť nerovnosť $pq<q+1$.
Z~nerovnosti $p\leq q$ vyplýva $\sqrt{pq}\leq q$; stačí preto
overiť silnejšiu nerovnosť $pq<\sqrt{pq}+1$, \tj. že $t=\sqrt{pq}$
spĺňa kvadratickú nerovnosť $t^2-t-1<0$, alebo že
$\m3/2<\sqrt{pq}<5/2$.
Zo vzťahov
$$
\postdisplaypenalty 10000
pq=\frac1{r} \quad \hbox{a} \quad r\geq \frac25 \qquad
         \hbox{vyplýva} \qquad \sqrt{pq}\leq \sqrt{\frac52}<\frac52,
$$
a~naviac $\sqrt{pq}\geq 1$.
Tým je dôkaz hotový.}

{%%%%%   trojstretnutie, priklad 4
Označme dĺžky strán trojuholníka~$ABC$ zvyčajným spôsobom $a$,
$b$, $c$ a~$D$ stred strany~$AB$. Z~mocnosti bodu~$C$ ku
kružnici opísanej štvoruholníku~$ABXY$ dostaneme $|CA|\cdot
|CY|=|CB|\cdot |CX|$, teda $a\cdot |CX|=b\cdot |CY|$. Z~C\`evovej
vety potom vyplýva
$$
\frac{|AD|\cdot |BX|\cdot |CY|}{|DB|\cdot |XC|\cdot |YA|}=
\frac{|BX|\cdot |CY|}{|XC|\cdot |YA|}=1,
$$
takže dosadením $a\cdot |CX|=b\cdot |CY|$ dostávame ďalej
$a\cdot |BX|=b\cdot |AY|$. Sčítaním rovností
$$
a\cdot |CX|=b\cdot |CY|
\qquad \hbox{a}\qquad
a\cdot |BX|=b\cdot |AY|
$$
dostaneme $a^2=b^2$, čiže $a=b$. Trojuholník~$ABC$ je teda
rovnoramenný.

\ineriesenie
Ak je štvoruholník~$ABXY$ tetivový, sú
trojuholníky $ABC$ a~$XYC$ podobné~({\it uu\/}), platí preto
$a\cdot |CX|=b\cdot|CY|$.
Označme $S_{EFG}$ obsah trojuholníka~$EFG$.
Pre obsahy trojuholníkov zrejme platí
$$
\frac{S_{APC}}{S_{APY}}=\frac{|AC|}{|AY|}=\frac{b}{|AY|}\qquad \hbox{a}
    \qquad \frac{S_{BPC}}{S_{BPX}}=\frac{|BC|}{|BX|}=\frac{a}{|BX|}.
$$
Pretože bod~$P$ leží na ťažnici z~vrcholu~$C$ trojuholníka~$ABC$,
platí tiež $S_{APC}=S_{BPC}$, čo s~oboma predchádzajúcimi
vzťahmi dáva
$$
\frac{b}{|AY|}\,S_{APY}=\frac{a}{|BX|}\,S_{BPX}.
$$
Z~rovnosti obvodových uhlov $AXB$ a~$AYB$ a~ďalej z~rovnosti
vrcholových uhlov pri vrchole~$P$ vyplýva podobnosť trojuholníkov
$APY$ a~$BPX$~({\it uu\/}). Platí teda
$S_{APY}:S_{BPX}=|AY|^2:|BX|^2$, čo v~spojení s~predchádzajúcim
vzťahom dáva $a\cdot |BX|=b\cdot |AY|$. Ďalej pokračujeme ako 
v~predchádzajúcom riešení.}

{%%%%%   trojstretnutie, priklad 5
Ukážeme, že podmienkam úlohy vyhovujú všetky prirodzené
čísla~$n$, ktoré sú mocninami čísla~$2$, \tj. všetky prirodzené
čísla tvaru $n=2^m$, kde $m$ je prirodzené číslo.
Pre každé $k\in \{1,2,\dots ,2^m-1\}$ platí
$$
{2^m \choose k}=\frac{2^m\cdot (2^m-1)\cdot\dots\cdot (2^m-k+1)}
                    {1\cdot 2\cdot \dots \cdot k}.     \eqno{(1)}
$$
Ľubovoľné prirodzené číslo $r\in \{1,\dots ,k-1\}$ možno zapísať
v~tvare~$2^{\alpha}\ell$, kde $\ell$ je nepárne číslo a~$\alpha<m$ je
celé nezáporné číslo. Preto každý zo zlomkov
$$
\frac{2^m-r}{r}=\frac{2^{m-\alpha}-\ell}{\ell}
$$
má po skrátení v~čitateli aj v~menovateli nepárne čísla. Podobne
aj číslo~$k$ možno zapísať v~tvare~$2^{\alpha}\ell$, preto
zlomok na pravej strane rovnosti
$$
\frac{2^m}k=\frac{2^{m-\alpha}}{\ell}
$$
má v~čitateli párne a~v~menovateli nepárne číslo. Súčin
všetkých týchto zlomkov pre $r=1,2,\dots,k$ je rovný kombinačnému
číslu~$(1)$, ktoré je preto párne. Tým sme dokázali, že každé
kombinačné číslo tvaru~$(1)$ je párne.

Nech naopak $n$ nie je mocninou čísla~$2$, \tj. $n=c\cdot 2^m$, kde
$c\geq 3$ je nepárne číslo. Ukážeme, že kombinačné číslo
$$
{c\cdot 2^{m} \choose 2^{m}}=\frac{c\cdot 2^m(c\cdot 2^{m}-1)\cdot \dots
     \cdot (c\cdot 2^m-2^m+1)}{1\cdot 2\cdot 3 \cdot \dots \cdot 2^{m}}
      \eqno{(2)}
$$
je nepárne. Podobne ako skôr ukážeme, že pre všetky
$r\in\{1,2,\dots ,2^m-1\}$ má každý zo zlomkov
$$
\frac{c\cdot 2^m-r}{r}\qquad \hbox{a tiež} \qquad
\frac{c\cdot 2^m}{2^m}=c
$$
po skrátení v~čitateli aj v~menovateli nepárne čísla. Súčin všetkých týchto zlomkov
je rovný kombinačnému číslu~$(2)$, ktoré je preto nepárne.

\smallskip
Danej úlohe vyhovujú všetky prirodzené čísla~$n$, ktorá sú
mocninou čísla~$2$.}

{%%%%%   trojstretnutie, priklad 6
Pre každé $c\in\Bbb R$ je funkcia $f(x)=x+c$ riešením danej
funkcionálnej rovnice (obe jej strany sú potom rovné $x+y+2c$).
Ukážeme, že iné riešenia daná rovnica nemá.
Najprv dokážeme, že funkcia~$f$ je surjektívna. Voľbou
$y=-f(x)$ v~danej rovnici dostaneme
$$
f(0)-2x=f\left(f\bigl(-f(x)\bigr)-x\right).
$$
Pretože každé reálne číslo možno vyjadriť v~tvare $f(0)-2x$,
existuje pre každé $y\in\Bbb R$ také $z\in\Bbb R$, že platí $y=f(z)$.
Špeciálne potom existuje $a\in\Bbb R$, pre ktoré platí $f(a)=0$. Voľbou
$x=a$ v~danej funkcionálnej rovnici dostaneme
$$
f(y)=2a+f\bigl(f(y)-a\bigr) \qquad \hbox{\tj.} \qquad
    f(y)-a=f\bigl(f(y)-a\bigr)+a.
$$
Pretože funkcia~$f$ je surjektívna, existuje pre každé $x\in\Bbb R$
také $y\in\Bbb R$, že $x=f(y)-a$. Odtiaľ vyplýva, že pre
každé~$x$ reálne platí $x=f(x)+a$, \tj. $f(x)=x-a$.
Tým je úloha vyriešená.}

{%%%%%   IMO, priklad 1
Vytvorme množinu všetkých rozdielov $\mm D=\{x-y: x,y\in\mm  A\}$.
Pretože $\mm A$ má $101$~prvkov, obsahuje $\mm D$ okrem nuly
najviac $2\cdot\binom{101}{2}=101\cdot100=10\,100$ ďalších
(kladných aj záporných) čísel. Všimnime si, že dve z~uvažovaných
množín $\mm A_i$, $\mm A_j$ sú disjunktné práve vtedy, keď $x+t_i\ne
y+t_j$ pre ľubovoľné $x,y\in\mm A$, teda práve vtedy, keď
$t_i-t_j\notin\mm D$. Našou úlohou je preto vybrať čísla
$t_1,t_2,\dots,t_{100}\in\mm S$ tak, aby žiadny ich rozdiel
nepadol do "zakázanej" množiny~$\mm D$.

Spomenutý výber urobíme induktívne. Prvé číslo~$t_1$ vyberieme 
z~$\mm S$ ľubovoľne. Predpokladajme, že sme už pre niektoré
$k\leqq99$ vybrali čísla $t_1,t_2,\dots,t_{k}\in\mm S$ tak, že
$t_i-t_{j}\notin\mm D$ pre ľubovoľné rôzne
$i,j\in\{1,2,\dots,k\}$ (pre $k=1$ je to splnené triviálne).
Číslo~$t_{k+1}$ musíme v~$\mm S$ zvoliť tak, aby platilo
$t_{k+1}-t_{i}\notin\mm  D$ pre každé $i\in\{1,2,\dots,k\}$. Pre
pevné~$i$ tak má číslo~$t_{k+1}$ práve toľko "zakázaných" hodnôt~$t_i+d$,
koľko je všetkých čísel $d\in\mm D$. Tých je, ako vieme,
najviac $1+101\cdot100=10\,101$. Pre všetky
$i\in\{1,2,\dots,k\}$ tak celkom dostaneme najviac
$k\cdot10\,101$ zakázaných hodnôt, čo je najviac
$99\cdot10\,101=999\,999$ čísel. V~množine~$\mm S$ je však $10^6$~čísel,
takže výber čísla~$t_{k+1}$ je možný.

\poznamka
Hodnota $|\mm S|=10^6$ je zbytočne veľká (v~predchádzajúcom riešení sme
zanedbali skutočnosť, že v~množine~$\mm D$ leží s~každým číslom 
aj číslo opačné). Dá sa ukázať, že pre ľubovoľnú $k$-prvkovú
podmnožinu~$\mm A$ množiny $\mm S=\{1,2,\dots,n\}$ platí, že ak je $m$
prirodzené číslo také, že
$$
n>(m-1)\left(\binom k2+1\right),
$$
existujú v~množine~$\mm S$ čísla $t_1,t_2,\dots,t_m$ také, že
množiny $\mm A_j=\{x+t_j:x\in\mm A\}$ ($j=1,2,\dots,m$) sú
navzájom disjunktné. (Pre $k=101$ stačí teda uvažovať množinu
$\mm S=\{1,2,\dots,500\,051\}.)$}

{%%%%%   IMO, priklad 2
\podla{Jana Moláčka}
Ukážeme, že riešeniami sú práve všetky dvojice~$(a,b)$ tvaru
$(8k^4-k,2k)$, $(k,2k)$ a~$(k,1)$, kde $k\in\Bbb N$ je ľubovoľné.
Hľadáme prirodzené čísla $a$, $b$, $n$, pre ktoré platí
$$
\frac{a^2}{2ab^2-b^3+1}=n.
\tag1
$$
Rovnosť~$(1)$ možno upraviť na tvar kvadratickej rovnice
$$
a^2-2nb^{2}a+n(b^3-1)=0.
$$
s~neznámou~$a$. Jej koreňmi sú čísla
$$
a_{1,2}=nb^2\pm\sqrt{\big(nb^2\big)^2-n\big(b^3-1\big)}.
\tag2
$$
Pretože jeden z~koreňov~$a_{1,2}$ je rovný hľadanému
prirodzenému číslu~$a$, odmocnenec vo vzťahu~$(2)$ musí byť
štvorec, teda musí mať tvar
$$
\big(nb^2\big)^2-n\big(b^3-1\big)=d^2        \tag3
$$
pre vhodné celé $d\geqq0$. Také číslo~$d$ určite existuje,
ak $b=1$ (potom $b^3-1=0$, takže $d=nb^2$). Zaoberajme sa ale
najprv náročnejším prípadom, keď $b>1$. Ukážme, že pre
také~$b$ z~$(3)$ vyplývajú odhady
$$
nb^2-\frac{b+1}{2}<d<nb^2-\frac{b-1}{2}.     \tag4
$$
Pretože oba krajné výrazy sú kladné, môžeme obe nerovnosti
umocniť; po dosadení~$d^2$ a~jednoduchých algebraických úpravách
dostaneme dvojicu nerovností
$$
(b+1)^2<4n(b^2+1)\quad\text{a}\quad
(b-1)^2+4n(b^2-1)>0,
$$
ktoré zrejme platia, lebo $n\geqq1$ a~$b>1$. Tým sú odhady~$(4)$
dokázané.

Všimnime si teraz, že rozdiel oboch krajných výrazov v~$(4)$ je rovný~$1$.
Pre nepárne~$b$ by sa tieto výrazy dokonca rovnali dvom
po sebe idúcim prirodzeným číslam, takže by žiadne celé~$d$
spĺňajúce podmienku~$(4)$ neexistovalo. Číslo~$b$ je preto párne,
teda $b=2k$ pre vhodné $k\in\Bbb N$; jediné celé~$d$ vyhovujúce
nerovnostiam~$(4)$ je potom tvaru
$$
d=nb^2-\frac{b}{2}=4nk^2-k.
$$
Pre také $b$ a~$d$ prejde rovnosť~$(3)$ do tvaru
$$
(4nk^2)^2-n(8k^3-1)=(4nk^2-k)^2,
$$
z~ktorej ľahko vyplýva $n=k^2$; vzťahy~$(2)$ potom dávajú vyjadrenie
$$
a_1=8k^4-k\quad\text{a}\quad
a_2=k.
$$
Pretože obe vypočítané hodnoty~$a_{1,2}$ sú prirodzené čísla
(pre každé $k\in\Bbb N$), dostávame dve (nekonečné) skupiny riešení
$(a,b)=(8k^4-k,2k)$ a~$(a,b)=(k,2k)$, kde $k\in\Bbb N$.
Zostáva rozobrať prípad, keď $b=1$. Vtedy má zlomok zo zadania úlohy
tvar
$$
\frac{a^2}{2ab^2-b^3+1}=\frac{a^2}{2a}=\frac{a}{2},
$$
takže je rovný prirodzenému číslu práve vtedy, keď $a=2k$ pre vhodné
$k\in\Bbb N$. Treťou (a~poslednou) skupinou riešení sú teda
dvojice tvaru $(a,b)=(k,1)$, kde $k\in\Bbb N$.}

{%%%%%   IMO, priklad 3
\fontplace
\tpoint A; \tpoint B; \lpoint C; \bpoint D;
\bpoint E; \rpoint F;
\tpoint M; \bpoint N; \lpoint\xy.5,0 P;
[1] \hfil\Obr

\fontplace
\rtpoint A; \ltpoint B;
\tpoint M; \rpoint P; \bpoint P';
[3] \hfil\Obr

\fontplace
\tpoint A; \tpoint B; \lpoint C; \bpoint D;
\bpoint E; \rpoint F;
\tpoint M; \bpoint N;
\tpoint\vv a; \rBpoint\vv b; \lBpoint\vv c;
\bpoint\vv d; \rBpoint\vv e; \lBpoint\vv f;
[2] \hfil\Obr

\fontplace
\tpoint P; \tpoint Q; \rBpoint R;
\tpoint M;
\bpoint\vv x; \lBpoint\vv y; \rBpoint\vv z;
\lpoint\vv y-\vv z;
[4] \hfil\Obr

Označme $A$, $B$, $C$, $D$, $E$, $F$ vrcholy daného šesťuholníka
a~uvažujme tri uhlopriečky $AD$, $BE$ a~$FC$. Niektoré dve z~nich
nutne zvierajú uhol aspoň~$60\st$. Bez ujmy na všeobecnosti
predpokladajme, že sú to uhlopriečky $AD$ a~$BE$. Označme $P$
ich priesečník a~$M$, $N$ stredy protiľahlých strán $AB$ a~$DE$
(\obr).

\midinsert
\centerline{\inspicture-!\hss\inspicture-!}
\endinsert
%\inspicture{}
%\inspicture{}
%\twocpictures

Pretože $|\uh APB|=|\uh EPD|\ge60\st$, leží bod~$P$ vnútri
kružnice opísanej rovnostrannému trojuholníku~$ABP'$ (\obr), takže platí
$|MP|\le|MP'|=(\sqrt3/2)|AB|$ s~rovnosťou práve vtedy, keď $P=P'$,
čiže práve vtedy, keď je trojuholník~$ABP$ rovnostranný. Podobne odvodíme,
že $|NP|\le(\sqrt3/2)|DE|$ s~rovnosťou práve vtedy, keď
je trojuholník~$DEP$ rovnostranný. Pre vzdialenosť stredov oboch protiľahlých
strán $AB$, $DE$ tak dostávame odhad
$$
|MN|\le|MP|+|NP|\le\tfrac12\sqrt3(|AB|+|DE|).
$$
Z~predpokladov úlohy teda vyplýva, že oba trojuholníky~$ABP$ a~$DEP$ sú
rovnostranné a~uhlopriečky $AD$, $BE$ zvierajú uhol~$60\st$.

Zostávajúca uhlopriečka~$CF$ musí s~jednou z~uhlopriečok $AD$, $BE$
zvierať uhol aspoň~$60\st$. Opäť môžeme bez ujmy na všeobecnosti
predpokladať, že sa jedná napr.~o~uhlopriečku~$AD$, priesečník
uhlopriečok $CE$, $AD$ označme~$Q$. Úplne rovnako ako v~predchádzajúcom
prípade zistíme, že trojuholníky $FAQ$ a~$CDQ$ sú rovnostranné.
Keď nakoniec označíme $R$ priesečník uhlopriečok $BE$ a~$CF$, ktoré
podľa predchádzajúceho zvierajú nutne uhol~$60\st$, zistíme, že aj trojuholníky
$BCR$ a~$EFR$ sú rovnostranné. Odtiaľ vyplýva tvrdenie úlohy.

\ineriesenie
Označme $A$, $B$, $C$, $D$, $E$, $F$ vrcholy daného šesťuholníka
a~$\vv a=\vv{AB},\vv b=\vv{BC},\dots,\vv f=\vv{FA}$ vektory
určené jeho stranami, pritom $\vv a+\vv b+\dots+\vv f=0$. Ak označíme
\inspicture{}
$M$, $N$ stredy protiľahlých strán $AB$, $DE$, môžeme príslušný
vektor~$\vv{MN}$ vyjadriť dvoma spôsobmi (\obr).
$$
\vv{MN}=\tfrac12\vv a+\vv b+\vv c+\tfrac12\vv d
\quad\text{a}\quad
\vv{MN}=-\tfrac12\vv a-\vv f-\vv e-\tfrac12\vv d,
$$
odkiaľ
$$
\vv{MN}=\tfrac12(\vv b+\vv c-\vv e-\vv f).      \tag1
$$

Podľa predpokladu platí
$$
|\vv{MN}|=\frac{\sqrt3}2|\vv a+\vv d|
         \ge\frac{\sqrt3}2|\vv a-\vv d|.        \tag2
$$
Položme $\vv x=\vv a-\vv d$, $\vv y=\vv c-\vv f$, $\vv z=\vv
e-\vv b$, z~(1) a~(2) tak dostaneme
$$
\align
|\vv y-\vv z|&\ge\sqrt3|\vv x|\\
\intext{a~podobne}
|\vv z-\vv x|&\ge\sqrt3|\vv y|,\\
|\vv x-\vv y|&\ge\sqrt3|\vv z|.
\endalign
$$
Práve uvedené nerovnosti môžeme pomocou
skalárnych súčinov ekvivalentne prepísať ako
$$
\align
|\vv y|^2-2(\vv y,\vv z)+|\vv z|^2&\ge3|\vv x|^2,\\
|\vv z|^2-2(\vv z,\vv x)+|\vv x|^2&\ge3|\vv y|^2,\\
|\vv x|^2-2(\vv x,\vv y)+|\vv y|^2&\ge3|\vv z|^2.\\
\endalign
$$
Sčítaním všetkých troch nerovností vyjde
$$
-|\vv x|^2-|\vv y|^2-|\vv z|^2
-2(\vv y,\vv z)
-2(\vv z,\vv x)
-2(\vv x,\vv y)\ge0,
$$
čiže $-|\vv x+\vv y+\vv z|^2\ge0$. Odtiaľ však vyplýva, že $\vv
x+\vv y+\vv z=0$ a~že vo všetkých predchádzajúcich nerovnostiach platí
rovnosť. Teda jednak
$$
\align
|\vv y-\vv z|&=\sqrt3|\vv x|,\\
|\vv z-\vv x|&=\sqrt3|\vv y|,\\
|\vv x-\vv y|&=\sqrt3|\vv z|,
\endalign
$$
jednak vďaka rovnosti v~$(2)$ a~v~ďalších dvoch analogických
nerovnostiach aj ${\vv a\parallel\vv d\parallel\vv x}$,
$\vv c\parallel\vv f\parallel\vv y$, $\vv e\parallel\vv
b\parallel\vv z$.

Keď zostrojíme trojuholník~$PQR$ tak, že $\vv{PQ}=\vv
x$, $\vv{QR}=\vv y$, $\vv{RP}=\vv z$ (čo môžeme vďaka rovnosti
$\vv x+\vv y+\vv z=0$), bude niektorý z~jeho vnútorných uhlov mať
veľkosť aspoň~$60\st$. Nech je to napr.~uhol~$PRQ$ (\obr). Pre
stred~$M$ strany~$PQ$ potom platí $|MR|=|\vv y-\vv
z|/2=\sqrt3|\vv x|/2= \sqrt3|PQ|/2$, čo znamená, že trojuholník~$PQR$
je rovnostranný. Pre vnútorné uhly daného šesťuholníka to
vzhľadom na dokázanú rovnobežnosť jeho protiľahlých strán 
s~prislúchajúcimi vektormi $\vv x$, $\vv y$ a~$\vv z$ znamená, že
všetky jeho vnútorné uhly majú veľkosť~$120\st$.
\inspicture{}

\poznamka
Z~uvedeného riešenia je zrejmé, že ľubovoľný šesťuholník spĺňajúci
predpoklady úlohy dostaneme tak, že z~niektorého rovnostranného
trojuholníka "odrežeme" pri každom jeho vrchole zhodný rovnostranný
trojuholník.}

{%%%%%   IMO, priklad 4
\fontplace
\rpoint A; \lpoint B; \lBpoint C; \bpoint D;
\lpoint P; \tpoint\xy1,.4 Q; \tpoint R;
\brpoint\down\unit X_1; \bpoint X_2;
\cpoint\a; \cpoint\g;
[5] \hfil\Obr

\fontplace
\rpoint A; \lpoint B; \lBpoint C; \bpoint D;
\lpoint P; \tpoint\xy1,.4 Q; \tpoint R;
\tpoint\xy-1,-1 O; \rtpoint\xy1,-.5 X;
\bpoint M; \tpoint N;
\cpoint; \rBpoint k;
[6] \hfil\Obr

Označme po rade $X_1$, $X_2$ priesečníky osi uhla~$ABC$ a~osi uhla~$ADC$
s~uhlopriečkou~$AC$ daného tetivového štvoruholníka~$ABCD$
(\obr). Zo známej vlastnosti osi uhla vyplýva jednak
$|AX_1|/|CX_1|=|AB|/|CB|$ (v~trojuholníku~$ABC$), jednak
$|AX_2|/|CX_2|=|AD|/|CD|$ (v~trojuholníku~$ACD$). Osi uhlov $ABC$ a~$ADC$
sa teda pretnú na uhlopriečke~$AC$ práve vtedy, keď $X_1=X_2$, čiže
práve vtedy, keď $|AD|\cdot|CB|=|AB|\cdot|CD|$.
Podľa Tálesovej vety ležia päty $P$ a~$Q$ na kružnici
s~priemerom~$CD$, takže pre veľkosť tetivy~$PQ$ tejto kružnice platí
$$
|PQ|=|CD|\sin|\uh PCQ|=|CD|\sin\gamma,
$$
kde $\gamma=\uh ACB$ (bez ohľadu na to, či päta~$P$ padne
dovnútra strany~$BC$ alebo nie). Podobne ležia päty $R$ a~$Q$ na
kružnici s~priemerom~$AD$, takže pre veľkosť tetivy~$QR$ tejto
kružnice platí
$$
|QR|=|AD|\sin|\uh RAQ|=|AD|\sin\a,
$$
kde $\a=\uh BAC$. Vidíme teda, že rovnosť $|PQ|=|QR|$ je
ekvivalentná s~rovnosťou $|CD|\sin\g=|AD|\sin\a$, čo je podľa
sínusovej vety pre trojuholník~$ABC$ ekvivalentné s~rovnosťou
$|AD|\cdot|CB|=|AB|\cdot|CD|$. Tým je tvrdenie úlohy dokázané.
\midinsert\inspicture{}\endinsert

\ineriesenie
\podla{Mareka Krčála}
Označme $M$ a~$N$ body, v~ktorých os uhla~$ABC$, resp.~os uhla~$ADC$
pretne kružnicu~$k$ opísanú danému tetivovému štvoruholníku~$ABCD$ (\obr).
Vzhľadom na to, že každý z~bodov $M$, $N$ rozpoľuje
príslušný oblúk~$AC$ kružnice~$k$, je $MN$ osou uhlopriečky~$AC$
a~zároveň priemerom kružnice~$k$. Označme $O$ stred úsečky~$AC$.
Bez ujmy na všeobecnosti predpokladajme, že uhol~$BAD$ nie je tupý
(inak by sme vymenili označenie vrcholov $A$ a~$C$), takže
päta~$P$ kolmice z~bodu~$D$ na $BC$ padne mimo úsečku~$BC$.
Z~vlastností tetivového štvoruholníka vyplýva $|\uh DCP|=|\uh BAD|$
a~z~rovnosti obvodových uhlov nad tetivou~$BD$ rovnosť $|\uh
BMD|=|\uh BAD|$. Trojuholníky $ARD$ a~$CPD$ sú teda podobné.
\inspicture{}

Predpokladajme, že priesečník~$X$ oboch spomenutých osí uhlov leží na
priamke~$AC$. Pretože $MN$ je priemer kružnice~$k$, podľa
Tálesovej vety $|\uh XDM|=|\uh NDM|=90\st$, takže štvoruholník~$OXDM$
je tetivový. Teda tiež $|\uh XOD|=|\uh XMD|=|\uh BMD|$
a~vidíme, že trojuholník~$OQD$ je podobný s~trojuholníkmi $ARD$ a~$CPD$.
Uvažujme zobrazenie, ktoré vznikne zložením otočenia okolo
stredu~$D$ o~uhol $90\st-|\uh BAD|$ a~rovnoľahlosti so
stredom~$D$ a~koeficientom $|DR|/|DA|$. Toto zobrazenie zobrazí
bod~$A$ do bodu~$R$, bod~$C$ do bodu~$P$ a~bod~$O$ do bodu~$Q$.
Pretože $O$ je stred úsečky~$AC$, je jeho obraz v~tomto
zobrazení, teda bod~$Q$, stredom úsečky~$PR$, ktorá je obrazom
úsečky~$AC$.

Obrátene, ak je $Q$ stred úsečky~$PR$, je obrazom bodu~$O$ 
v~uvedenom zobrazení, takže trojuholník~$OQD$ je podobný s~trojuholníkmi $ARD$ 
a~$CPD$ (tie sú podobné vždy). Ak teraz označíme $X$ priesečník
priamky~$BM$ s~uhlopriečkou~$AC$, bude $XDMO$ tetivový, a~teda
veľkosť uhla~$XDM$ bude~$90\st$. Odtiaľ vyplýva, že bod~$X$ leží na osi~$ND$ uhla~$ADC$.}

{%%%%%   IMO, priklad 5
Obe strany dokazovanej nerovnosti nezmenia hodnotu, keď
od všetkých členov~$x_i$ odpočítame rovnaké číslo~$c$. Keď vyberieme za~$c$
aritmetický priemer danej $n$-tice členov~$x_i$, bude
"posunutá" postupnosť členov $x_i:=x_i-c$ spĺňať podmienku
$$
\sum_{i=1}^{n} x_i=0.          \tag1
$$
Dodajme, že spomenuté "posunutie" zachová tiež usporiadanie
čísel~$x_i$ podľa veľkosti a~nezmení ani nič na tom, či
dotyčná $n$-tica tvorila aritmetickú postupnosť alebo nie.
Za predpokladu~$(1)$ upravíme oba súčty z~dokazovanej nerovnosti na
$$
\align
\sum_{i=1}^n\sum_{j=1}^n|x_i-x_j|
&=2\sum_{1\leqq i<j\leqq n}(x_j-x_i)=\\
&=2\sum_{i=1}^{n}
(\underbrace{(1+\cdots+1)}_{\text{$(i-1)$ krát}}
-\underbrace{(1+\cdots+1)}_{\text{$(n-i)$ krát}})x_i=\\
        &=2\sum_{i=1}^{n}(2i-n-1)x_i,\\
\sum_{i=1}^n\sum_{j=1}^n(x_i-x_j)^2
&=n\sum_{i=1}^n x_i^2-2\sum_{i=1}^n x_i\sum_{j=1}^n x_j
  +n\sum_{j=1}^n x_j^2=\\
&=2n\sum_{i=1}^{n} x_i^2.
\endalign
$$
Po dosadení a~krátení štyrmi zistíme, že máme dokázať nerovnosť
$$
\biggl(\sum_{i=1}^{n}(2i-n-1)x_i\biggr)^{\!2}\leqq
\frac{(n^2-1)n}{3}\sum_{i=1}^{n} x_i^2.
\tag2
$$
Ukážme, že~$(2)$ je Cauchyho nerovnosť
$$
\biggl(\sum_{i=1}^{n}x_iy_i\biggr)^{\!2}\leqq
\sum_{i=1}^{n} x_i^2\sum_{i=1}^{n} y_i^2
\tag3
$$
pre $n$-ticu členov $y_i=2i-n-1$, $i=1,2,\dots,n$. Naozaj, pre
takú $n$-ticu platí
$$
\postdisplaypenalty 10000
\align
\sum_{i=1}^{n} y_i^2&=\sum_{i=1}^{n}(2i-n-1)^2=
4\sum_{i=1}^{n}i^2-4(n+1)\sum_{i=1}^{n}i+n(n+1)^2=\\
&=4\cdot\frac{n(n+1)(2n+1)}{6}-4(n+1)\cdot\frac{n(n+1)}{2}+n(n+1)^2=\\
&=\frac{(n^2-1)n}{3}.
\endalign
$$
Tým je dôkaz nerovnosti~$(2)$ hotový.

Ako je dobre známe, rovnosť v~Cauchyho nerovnosti~$(2)$ nastane
práve vtedy, keď existuje reálne číslo~$p$, pre ktoré platí $n$-tica
rovností
$$
x_i=py_i=p(2i-n-1)     \quad(i=1,2,\dots,n). \tag4
$$
Overme, že túto podmienku za predpokladu~$(1)$ spĺňajú práve tie
konečné postupnosti $x_1,x_2,\dots,x_n$, ktoré sú aritmetické.
Naozaj,
Ak platia rovnosti~$(4)$, je konečná postupnosť $x_1,x_2,\dots,x_n$
aritmetická s~diferenciou~$2p$. Naopak, ak je postupnosť
$x_1,x_2,\dots,x_n$ aritmetická a~$d$ je jej diferencia,
tak pre každé $i=1,2,\dots,n$ platí rovnosť $x_i=x_1+(i-1)d$  
a~súčet všetkých členov~$x_i$ je daný vzťahom
$$
\sum_{i=1}^{n} x_i=\frac{n(x_1+x_n)}{2}.
$$
Podmienka~$(1)$ preto znamená, že $x_1+x_n=0$, čiže
$x_1+x_1+(n-1)d=0$. Odtiaľ dostávame $x_1=d(1-n)/2$, preto
členy~$x_i$ majú pre každé~$i$ vyjadrenie
$$
x_i=\frac{d(1-n)}{2}+(i-1)d=\frac{(1-n+2i-2)d}{2}=
\frac{(2i-n-1)d}{2},
$$
čo je~$(4)$ pre $p=d/2$.}

{%%%%%   IMO, priklad 6
Pripomeňme si najskôr vlastnosti mocnín $n^1,n^2,\dots,n^k,\dots$
pri delení prvočíslom~$q$. Ak je $n$ celé číslo nesúdeliteľné s~$q$,
tak $n^{q-1}\equiv1\pmod q$ (tzv.~malá Fermatova veta),
naviac množina tých prirodzených~$k$, pre ktoré $n^{k}\equiv1\pmod
q$, je tvorená všetkými násobkami najmenšieho z~nich (čo je buď
číslo $q-1$, alebo niektorý jeho deliteľ).
Uvažujme preto rozklad
$$
p^p-1=(p-1)S,\quad\text{kde }S=p^{p-1}+p^{p-2}+\cdots+p+1,   \tag1
$$
a~za "kandidáta" na vhodné prvočíslo~$q$ vyberme niektoré
z~prvočísiel deliacich súčet~$S$
(neskôr upresníme, akú doplňujúcu vlastnosť
prvočiniteľa~$q$ čísla~$S$ budeme ešte potrebovať a~prečo
také~$q$ vôbec existuje). Pretože $q\deli S$ a~$S\deli(p^p-1)$,
platí $q\deli(p^p-1)$, \tj. $p^p\equiv1\pmod q$.

Pripusťme, že pre vybrané~$q$ tvrdenie úlohy neplatí, teda
existuje celé~$n$ s~vlastnosťou $n^p\equiv p\pmod q$. Umocnením
tejto kongruencie na~$p$ dostaneme $n^{p^2}\equiv p^p\pmod q$,
čo spolu s~kongruenciou zo záveru predchádzajúceho odstavca znamená, že
$n^{p^2}\equiv1\pmod q$. Číslo~$n$ je teda nesúdeliteľné
s~číslom~$q$ a~podľa poznatkov pripomenutých v~úvode riešenia vieme,
že najmenšie prirodzené~$k$ s~vlastnosťou $n^k\equiv1\pmod q$ musí
byť deliteľom čísla~$p^2$, teda jedno z~čísel $1$, $p$, $p^2$. Toto
číslo musí byť súčasne deliteľom čísla $q-1$ (malá Fermatova
veta), takže to nebude číslo~$p^2$, pokiaľ nebude číslo~$p^2$ deliť
číslo $q-1$, teda pokiaľ $q\not\equiv1\pmod{p^2}$. To je práve tá
doplňujúca vlastnosť prvočísla~$q$, ktorú sme skôr
spomenuli. Odložme na chvíľu dôkaz existencie takého prvočísla~$q$
a~dokončime úvahy o~mocninách čísla~$n$.

Pokiaľ teda $q\not\equiv1\pmod{p^2}$, platí kongruencia
$n^k\equiv1\pmod q$ pre $k=1$ alebo pre $k=p$, v~oboch prípadoch
máme $n^p\equiv1\pmod q$. Porovnaním s~kongruenciou $n^p\equiv
p\pmod q$ potom dostaneme $p\equiv1\pmod q$, takže každá z~$p$~mocnín~$p^j$
zo súčtu~$S$ je kongruentná s~číslom~$1$ (modulo~$q$),
takže $S\equiv p\pmod q$. Pretože však $q\deli S$, platí
$S\equiv0\pmod q$. Porovnaním vychádza $p\equiv0\pmod q$, čo je
spor s~tým, že $p\equiv1\pmod q$. Preto žiadne celé~$n$
s~vlastnosťou $n^p\equiv p\pmod q$ neexistuje, pokiaľ spĺňa
prvočíslo~$q$ podmienku $q\not\equiv1\pmod{p^2}$. Existenciu
takého prvočiniteľa~$q$ (z~rozkladu čísla~$S$) teraz dokážeme.

Určme zvyšok súčtu~$S$ pri delení číslom~$p^2$. Pretože
$p^2\deli p^j$ ($j\geqq2$), platí
$$
S=p^{p-1}+p^{p-2}+\cdots+p+1\equiv0+0+\cdots+0+p+1\pmod {p^2},
$$
teda $S\equiv p+1\not\equiv1\pmod{p^2}$. Odtiaľ už vyplýva, že
aspoň jeden z~prvočiniteľov~$q_j$ čísla $S=q_1q_2\dots q_r$ nie je
kongruentný s~$1$ (modulo~$p^2$). (Vynásobením $r$~kongruencií
$q_j\equiv1\pmod{p^2}$ by sme totiž dostali
$S\equiv1\pmod{p^2}$.)

Dôkaz je hotový a~úloha vyriešená.}

