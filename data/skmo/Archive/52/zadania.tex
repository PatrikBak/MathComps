{%%%%%   A-I-1
Postupnosť celých čísel~$(x_n)_{n=1}^\infty$ s~prvým členom~$x_1=1$ spĺňa podmienku
$$
x_n=\pm x_{n-1}\pm\cdots\pm x_1
$$
s~vhodnou voľbou znamienok "$+$" a~"$-$" pre ľubovoľné~$n>1$, napríklad $x_2=-x_1$, $x_3=-x_2+x_1$, $x_4=x_3-x_2-x_1$,~$\dots$ Pre dané~$n$ určte všetky možné hodnoty~$x_n$.}
\podpis{J. Földes}

{%%%%%   A-I-2
Na priamke~$p$ sú dané rôzne body $A$, $B$, $C$ v~tomto poradí, kde $|AB|=1$ a~$|BC|=h$. Uvažujme kružnice $k_A$, $k_B$, $k_C$, ktoré sa dotýkajú priamky~$p$ postupne v~bodoch $A$, $B$, $C$. Kružnice $k_A$, $k_B$ majú pritom vonkajší dotyk v~bode~$P$ a~kružnice $k_B$, $k_C$ majú vonkajší dotyk v~bode~$Q$. Určte všetky také hodnoty polomeru kružnice~$k_B$, pre ktoré je trojuholník~$BPQ$ rovnoramenný.}
\podpis{J. Zhouf}

{%%%%%   A-I-3
Určte všetky možné hodnoty výrazu
$$
\frac{a^4+b^4+c^4}{a^2b^2+a^2c^2+b^2c^2},
$$
kde $a$, $b$, $c$ sú dĺžky strán trojuholníka.}
\podpis{P. Kaňovský}

{%%%%%   A-I-4
Určte všetky prirodzené čísla~$n>1$ také, že v~niektorej číselnej sústave so základom $z\geqq5$ platí nasledovné kritérium deliteľnosti:
Trojmiestne číslo~$(abc)_z$ je deliteľné číslom~$n$ práve vtedy, keď je číslom~$n$ deliteľné číslo $c+3b-4a$.}
\podpis{P. Černek}

{%%%%%   A-I-5
V~rovine sú dané tri rôzne body $K$, $L$, $M$, ktoré v~tomto poradí ležia na priamke. V~tejto rovine nájdite množinu všetkých vrcholov~$C$ štvorcov~$ABCD$ takých, že bod~$K$ leží na strane~$AB$, bod~$L$ na uhlopriečke~$BD$ a~bod~$M$ na strane~$CD$.}
\podpis{J. Šimša}

{%%%%%   A-I-6
Hráči $A$ a~$B$ hrajú na doske zloženej zo šiestich polí očíslovaných $1,2,\dots,6$ nasledujúcu hru. Na začiatku je umiestnená na pole s~číslom~$2$ figúrka a~potom sa hádže bežnou hracou kockou. Ak padne číslo deliteľné tromi, posunie sa figúrka na pole s~číslom o~jedna menším, inak na pole s~číslom o~jedna väčším. Hra končí víťazstvom hráča $A$ (resp.~$B$), ak sa dostane figúrka na pole s~číslom~$1$ (resp.~$6$). S~akou pravdepodobnosťou zvíťazí hráč~$A$?}
\podpis{P. Černek}

{%%%%%   B-I-1
{\it Palindrómom\/} rozumieme prirodzené číslo, ktoré sa číta rovnako odpredu aj odzadu, napr.~$16\,261$. Nájdite najväčší štvormiestny palindróm, ktorého druhá mocnina je tiež palindrómom.}
\podpis{E. Kováč}

{%%%%%   B-I-2
Nájdite všetky trojice reálnych čísel $(x,y,z)$ vyhovujúcich sústave rovníc
$$
\eqalign{
x^3+y^3&=9z^3,\cr
x^2y+y^2x&=6z^3.
}
$$
}
\podpis{J. Zhouf}

{%%%%%   B-I-3
Je daný trojuholník so stranami dĺžok $a$, $b$, $c$ a~obsahom~$S$. Dokážte, že rovnosť $2c^2=|a^2-b^2|$ platí práve vtedy, keď existuje trojuholník so stranami dĺžok $a$, $b$, $2c$ a~obsahom~$2S$.}
\podpis{P. Černek}

{%%%%%   B-I-4
{\it Krokom} budeme rozumieť nahradenie usporiadanej trojice celých čísel $(p,q,r)$ trojicou $(r+5q,3r-5p,2q-3p)$. Rozhodnite, či existuje celé číslo~$k$ také, že z~trojice $(1,3,7)$ vznikne po konečnom počte krokov trojica $(k,k+1,k+2)$.}
\podpis{P. Černek}

{%%%%%   B-I-5
V~rovine je daný pravouhlý lichobežník~$ABCD$ s~dlhšou základňou~$AB$ a~pravým uhlom pri vrchole~$A$. Kružnica~$k_1$ zostrojená nad stranou~$AD$ ako priemerom a~kružnica~$k_2$, ktorá prechádza vrcholmi $B$, $C$ a~dotýka sa priamky~$AB$, majú vonkajší dotyk v~bode~$P$. Dokážte, že uhly $CPD$ a~$ABC$ sú zhodné.}
\podpis{J. Švrček}

{%%%%%   B-I-6
V~karteziánskej sústave súradníc $Ouv$ znázornite množinu všetkých bodov $[u,v]$, kde $u>0$, pre ktoré má rovnica
$$
|x^2-ux|+vx-1=0
$$
s~neznámou~$x$ práve tri rôzne reálne riešenia.}
\podpis{J. Šimša}

{%%%%%   C-I-1
Z~piatich jednotiek, piatich dvojok, piatich trojok, piatich štvoriek a piatich pätiek zostavte päť navzájom rôznych päťmiestnych čísel tak, aby ich súčet bol čo najväčší.}
\podpis{J. Šimša}

{%%%%%   C-I-2
Je daný trojuholník~$ABC$ s~ostrými vnútornými uhlami pri vrcholoch $A$ a~$B$. Označme $Q$ priesečník ťažnice~$AD$ s~výškou~$CP$ a~$E$ pätu kolmice z~bodu~$D$ na stranu~$AB$. Ďalej nech $R$ je taký bod na polpriamke opačnej k~$PC$, že $|PR| = |CQ|$. Dokážte, že priamky $AD$ a~$RE$ sú rôznobežné a~že ich priesečník leží na kolmici k~priamke~$AB$ prechádzajúcej bodom~$B$.}
\podpis{J. Švrček}

{%%%%%   C-I-3
Predpokladajme, že každá z~dvoch bánk $A$ a~$B$ bude mať počas nasledujúcich dvoch rokov stálu ročnú úrokovú mieru. Keby sme uložili $5/6$ našich úspor v~banke~$A$ a~zvyšok v~banke~$B$, vzrástli by naše úspory po jednom roku na 67\,000~Sk a~po dvoch rokoch na 74\,900~Sk. Keby sme však uložili $5/6$ našich úspor v~banke~$B$ a~zvyšok v~banke~$A$, vzrástli by naše úspory po jednom roku na 71\,000~Sk. Na akú čiastku by sa v~takom prípade zvýšili naše úspory po dvoch rokoch?}
\podpis{J. Šimša}

{%%%%%   C-I-4
Zostrojte lichobežník~$ABCD$ s~výškou~$3\cm$ a~zhodnými stranami $BC$, $CD$ a~$DA$, pre ktorý platí: Na základni~$AB$ existuje bod~$E$ taký, že úsečka~$DE$ má dĺžku~$5\cm$ a~delí lichobežník na dve časti s~rovnakými obsahmi.}
\podpis{E. Kováč}

{%%%%%   C-I-5
K~prirodzenému číslu~$m$ zapísanému rovnakými číslicami sme pripočítali štvormiestne prirodzené číslo~$n$. Získali sme štvormiestne číslo s~opačným poradím číslic ako má číslo~$n$. Určte všetky také dvojice čísel $m$ a~$n$.}
\podpis{J. Zhouf}

{%%%%%   C-I-6
V~rovine je daná priamka~$p$ a kružnica~$k$. Zostrojte taký trojuholník~$ABC$, aby $k$ bola kružnicou jemu vpísanou, aby jej stred ležal v~jednej štvrtine jeho ťažnice na stranu~$AB$ a~aby vrchol~$C$ ležal na priamke~$p$. Urobte diskusiu o~počte riešení v~závislosti na vzájomnej polohe priamky~$p$ a kružnice~$k$.}
\podpis{P. Černek}

{%%%%%   A-S-1
Hovoríme, že tri navzájom rôzne prirodzené čísla tvoria súčtovú trojicu, ak súčet prvých dvoch z~nich sa rovná tretiemu číslu. Zistite, aký najväčší počet súčtových trojíc sa môže nachádzať v~množine dvadsiatich prirodzených čísel.}
\podpis{P. Černek}

{%%%%%   A-S-2
V~rovine sú dané kružnice $k_1(S_1,r_1)$ a~$k_2(S_2,r_2)$ tak, že $S_2\in k_1$ a~$r_1>r_2$. Spoločné dotyčnice oboch kružníc sa dotýkajú kružnice~$k_1$ v~bodoch $P$ a~$Q$. Dokážte, že priamka~$PQ$ sa dotýka kružnice~$k_2$.}
\podpis{J. Földes}

{%%%%%   A-S-3
Zistite, pre ktoré reálne čísla~$p$ majú rovnice
$$
\align
x^3+x^2-36x-p&=0,\\
x^3-2x^2-px+2p&=0\\
\endalign
$$
spoločný koreň.}
\podpis{P. Černek}

{%%%%%   A-II-1
Nájdite základy~$z$ všetkých číselných sústav, v~ktorých je štvormiestne číslo~$(1001)_z$ deliteľné dvojmiestnym číslom $(41)_z$.}
\podpis{P. Černek}

{%%%%%   A-II-2
Vnútri strany~$AB$ daného ostrouhlého trojuholníka~$ABC$ nájdite bod~$S$ tak, aby trojuholník~$SXY$, kde $X$ a~$Y$ sú postupne stredy kružníc opísaných trojuholníkom $ASC$ a~$BSC$, mal najmenší možný obsah.}
\podpis{P. Černek}

{%%%%%   A-II-3
V~obore reálnych čísel riešte sústavu rovníc
$$\align
\log_x(y+z)&=p,\\
\log_y(z+x)&=p,\\
\log_z(x+y)&=p\\
\endalign
$$
s~neznámymi $x$, $y$, $z$ a~nezáporným celočíselným parametrom~$p$.}
\podpis{J. Švrček}

{%%%%%   A-II-4
Postupnosť~$(x_n)_{n=1}^{\infty}$ s~prvým členom $x_1=1$ spĺňa pre každé~$n>1$ podmienku
$$
x_n=x_{n-1}^{\pm1}+x_{n-2}^{\pm1}+\dots+x_1^{\pm1}
$$
s~vhodnou voľbou znamienok "$+$" a~"$-$" v~exponentoch mocnín.
\ite a) Rozhodnite, či niektorý člen takej postupnosti musí byť väčší ako~$1\,000$.
\ite b) Zistite najmenšiu možnú hodnotu člena $x_{1\,000\,000}$.
\ite c) Dokážte, že nerovnosť $x_n<4$ nemôže platiť pre deväť členov $x_n$ takej postupnosti.}
\podpis{J. Földes}

{%%%%%   A-III-1
V~obore reálnych čísel riešte sústavu rovníc
$$
\postdisplaypenalty 10000
\align
x^2-xy+y^2&=7,\\
x^2y+xy^2&=-2.
\endalign
$$
}
\podpis{J. Földes}

{%%%%%   A-III-2
Vnútri strán $BC$, $CA$, $AB$ daného trojuholníka~$ABC$ zvolíme postupne body $D$, $E$, $F$ tak, aby sa úsečky $AD$, $BE$, $CF$ preťali v~jednom bode, ktorý označíme~$G$. Ak je možné štvoruholníkom $AFGE$, $BDGF$, $CEGD$ vpísať kružnice, z~ktorých každé dve majú vonkajší dotyk, potom je trojuholník~$ABC$ rovnostranný. Dokážte.}
\podpis{M. Tancer}

{%%%%%   A-III-3
Postupnosť~$(x_n)_{n=1}^{\infty}$ s~prvým členom~$x_1=1$ spĺňa pre každé~$n>1$ podmienku
$$
x_n=\pm (n-1)x_{n-1}\pm (n-2)x_{n-2}\pm\dots\pm2x_2\pm x_1
$$
s~vhodnou voľbou znamienok "$+$" a~"$-$". Rozhodnite, či je možné, aby nerovnosť ${x_n\ne12}$ platila len pre konečne veľa indexov~$n$.}
\podpis{P. Černek}

{%%%%%   A-III-4
V~rovine je daný tupý uhol~$AKS$. Zostrojte trojuholník~$ABC$ tak, aby jeho strana~$BC$ ležala na priamke~$KS$, aby bod~$S$ bol jej stredom a~bod~$K$ jej priesečníkom s~osou protiľahlého uhla~$BAC$.}
\podpis{P. Leischner}

{%%%%%   A-III-5
Ukážte, že v~číselnej sústave s~ľubovoľným základom~$z\ge3$ existujú dvojmiestne čísla $A$ a~$B$, ktoré sa líšia len poradím svojich číslic a~majú túto vlastnosť:
Kvadratická rovnica $x^2-Ax+B=0$ má v~obore reálnych čísel dvojnásobný koreň. Dokážte tiež, že pre daný základ~$z$ je taká dvojica $A$, $B$ jediná. Napríklad v~desiatkovej sústave ($z=10$) sú to jedine čísla $A=18$ a~$B=81$.}
\podpis{J. Šimša}

{%%%%%   A-III-6
Ak súčin kladných čísel $a$, $b$, $c$ je rovný~$1$, potom platí
$$
\frac ab+\frac bc+\frac ca\geqq a+b+c.
$$
Dokážte.}
\podpis{P. Kaňovský}

{%%%%%   B-S-1
Nájdite najväčšie päťmiestne prirodzené číslo, ktoré je deliteľné číslom~$101$ a~ktoré sa číta odpredu rovnako ako odzadu.}
\podpis{J. Šimša}

{%%%%%   B-S-2
Je daný konvexný štvoruholník~$ABCD$. Označme $P$ priesečník jeho uhlopriečok a~$Q$ priesečník spojníc stredov jeho protiľahlých strán. Ak bod~$Q$ leží na uhlopriečke~$BD$, je bod~$P$ stredom uhlopriečky~$AC$. Dokážte.}
\podpis{E. Kováč}

{%%%%%   B-S-3
Koľko rôznych výsledkov môžeme dostať, ak sčítame každé dve z~daných piatich rôznych prirodzených čísel? Pre každý možný počet uveďte príklad takej pätice čísel.}
\podpis{P. Černek}

{%%%%%   B-II-1
Určte najväčší počet po sebe idúcich päťmiestnych prirodzených čísel, medzi ktorými nie je žiadny palindróm (\tj. číslo, ktoré sa číta odpredu rovnako ako odzadu).}
\podpis{J. Šimša}

{%%%%%   B-II-2
V~rovine je daný pravouhlý trojuholník~$ABC$. Nech $K$ je ľubovoľný bod prepony~$AB$. Kružnica zostrojená nad úsečkou~$CK$ ako nad priemerom pretne odvesny $BC$ a~$CA$ vo vnútorných bodoch, ktoré označíme postupne $L$ a~$M$. Rozhodnite, pre ktorý bod~$K$ má štvoruholník~$ABLM$ najmenší možný obsah.}
\podpis{J. Švrček}

{%%%%%   B-II-3
Určte všetky reálne čísla~$p$, pre ktoré má rovnica
$$
(x-1)^2=3|x|-px
$$
práve tri rôzne riešenia v~obore reálnych čísel.}
\podpis{J. Šimša}

{%%%%%   B-II-4
V~rovine je daný pravouhlý lichobežník~$ABCD$ s~dlhšou základňou~$AB$ a~pravým uhlom pri vrchole~$A$. Označme $k_1$ kružnicu zostrojenú nad stranou~$AD$ ako nad priemerom a~$k_2$ kružnicu, ktorá prechádza bodmi $B$, $C$ a~dotýka sa priamky~$AB$. Ak majú kružnice $k_1$, $k_2$ vonkajší dotyk v~bode~$P$, je priamka~$BC$ dotyčnicou kružnice opísanej trojuholníku~$CDP$. Dokážte.}
\podpis{J. Švrček}

{%%%%%   C-S-1
Ak od ľubovoľného aspoň dvojmiestneho prirodzeného čísla odtrhneme číslicu na mieste jednotiek, dostaneme číslo o~jednu číslicu "kratšie". Nájdite všetky pôvodné čísla, ktoré sa rovnajú absolútnej hodnote rozdielu druhej mocniny "kratšieho" čísla a~druhej mocniny odtrhnutej číslice.}
\podpis{J. Zhouf}

{%%%%%   C-S-2
Na strane~$CD$ štvorca~$ABCD$ je zvolený bod~$E$ tak, že uhol~$DAE$ má veľkosť~$30^{\circ}$. Bod~$P$ je pätou kolmice vedenej bodom~$B$ na priamku~$AE$, bod~$Q$ pätou kolmice vedenej bodom~$C$ na priamku~$BP$. Rozhodnite, či je obsah lichobežníka~$PQCE$ menší ako tretina obsahu štvorca~$ABCD$.}
\podpis{L. Boček}

{%%%%%   C-S-3
Z~piatich jednotiek, piatich dvojok, piatich trojok, piatich štvoriek a piatich pätiek zostavíme päť päťmiestnych čísel, ktoré sa čítajú odpredu rovnako ako odzadu (napr.~$32\,223$), a~potom tieto čísla sčítame. Akú najmenšiu a~akú najväčšiu hodnotu môže mať výsledný súčet?}
\podpis{J. Šimša}

{%%%%%   C-II-1
Nájdite najmenšie prirodzené číslo~$n$, pre ktoré je súčin
$$
\postdisplaypenalty 10000
2\,003\cdot2\,004\cdot2\,005\cdot\dots\cdot(2\,003+n)
$$
deliteľný všetkými dvojmiestnymi prvočíslami.}
\podpis{J. Šimša}

{%%%%%   C-II-2
V~rovine je daná úsečka~$AP$. Zostrojte pravidelný šesťuholník~$ABCDEF$ tak, aby bod~$P$ bol stredom jeho strany~$DE$.}
\podpis{J. Švrček}

{%%%%%   C-II-3
Keby Karol požičal jednému známemu $p$~tisíc~Sk s~úrokom~$p\,\%$ a~druhému známemu $q$~tisíc~Sk s~úrokom~$q\,\%$, kde $p$ a~$q$ sú celé čísla, priniesli by mu obe pôžičky taký istý zisk, ako keby jednej osobe požičal celkovú čiastku s~úrokom~$(p + 2{,}4)\,\%$. Keby požičal jednému známemu $p$~tisíc~Sk s~úrokom~$2p\,\%$ a~druhému známemu $q$~tisíc~Sk s~úrokom $2q\,\%$, priniesli by mu tieto pôžičky rovnaký zisk, ako keby jednej osobe požičal celkovú čiastku s~úrokom~$(p + 5{,}8)\,\%$. Určte čísla $p$ a~$q$.}
\podpis{J. Šimša, J. Zhouf}

{%%%%%   C-II-4
Určte dĺžku ramien rovnoramenného lichobežníka so základňami dĺžok $10$ a~$12$ tak, aby dĺžky všetkých jeho strán aj uhlopriečok boli vyjadrené celými číslami.}
\podpis{P. Černek}

{%%%%%   vyberko, den 1, priklad 1
Nájdite všetky polynómy~$f(x)$ s~reálnymi koeficientmi také, že pre každé
reálne číslo~$x$ platí
$$ (x+2003) f(x-2002) = (x-2001) f(x).$$}
\podpis{Peter Novotný:???}

{%%%%%   vyberko, den 1, priklad 2
Ak máme usporiadanú štvoricu $(x,y,z,w)$, tak z~nej môžeme v~jednom kroku
vyrobiť usporiadanú štvoricu $(xy,yz,zw,wx)$.
Začali sme s~usporiadanou štvoricou reálnych čísel
$(x_0,y_0,z_0,w_0)$ a~po konečnom počte krokov sme dostali tú istú
usporiadanú štvoricu. S~akou usporiadanou štvoricou sme mohli začať?}
\podpis{Peter Novotný:???}

{%%%%%   vyberko, den 1, priklad 3
Nájdite najväčšie $p \in \Bbb R$ také, že pre každý trojuholník
so stranami $a$, $b$, $c$ a~obsahom~$S$ je splnená nerovnosť
$$ a^2 + b^2 \ge pc^2 +3S.$$}
\podpis{Peter Novotný:???}

{%%%%%   vyberko, den 1, priklad 4
Dve kružnice $k_1$ a $k_2$ majú vonkajší dotyk v~bode~$K$. Obe majú vnútorný dotyk
s~kružnicou~$k$ postupne v~bodoch $A_1$ a $A_2$. Nech $P$ je jeden z~priesečníkov
kružnice~$k$ so spoločnou dotyčnicou kružníc $k_1$ a $k_2$ vedenou bodom~$K$.
Pre $i=1,2$ označme $B_i$ priesečník priamky~$PA_i$ s~kružnicou~$k_i$ (rôzny od~$A_i$).
Dokážte, že priamka~$B_1B_2$ je spoločnou dotyčnicou kružníc $k_1$ a $k_2$.}
\podpis{Peter Novotný:???}

{%%%%%   vyberko, den 2, priklad 1
V~ostrouhlom trojuholníku~$ABC$ označme~$H$ pätu výšky vedenej z~vrcholu~$A$. Nech $P$ je
ľubovoľný vnútorný bod trojuholníka~$ABC$ a~nech $D$, $E$ a~$Q$ sú po rade päty výšok z~neho
vedených na strany $AB$, $AC$ a~$AH$. Dokážte, že platí
$$
\bigl||AB|\cdot|AD|-|AC|\cdot|AE|\bigr|=|BC|\cdot|PQ|.
$$}
\podpis{Mgr. Richard Kollár:???}

{%%%%%   vyberko, den 2, priklad 2
Nech $a_0,a_1,a_2,~\dots$ je postupnosť kladných reálnych čísel spĺňajúca podmienku
$a_{i-1}a_{i+1}\le a_i^2$ pre každé $i\ge1$. Dokážte, že pre každé $n>1$ platí
$$
\frac{a_0+\cdots+a_n}{n+1}\cdot\frac{a_1+\cdots+a_{n-1}}{n-1}\ge
\frac{a_0+\cdots+a_{n-1}}{n}\cdot\frac{a_1+\cdots+a_n}{n}.
$$}
\podpis{Mgr. Richard Kollár:???}

{%%%%%   vyberko, den 2, priklad 3
Pre každé prirodzené číslo~$n$ nech $a_n=0$ alebo~$1$, podľa toho, či je počet jednotiek
v~zápise čísla~$n$ v~dvojkovej sústave párny~($0$) alebo nepárny~($1$).
Dokážte, že neexistujú prirodzené čísla $k$ a~$m$ také, že
$$
a_{k+j} = a_{k+m+j} = a_{k+2m+j}
$$
pre každé $0\le j\le m-1$.}
\podpis{Mgr. Richard Kollár:???}

{%%%%%   vyberko, den 3, priklad 1
Nájdite všetky reálne čísla $x$, $y$, pre ktoré platí
$$
\align
   (x-1)(y^2+6) &= y(x^2+1),\\
   (y-1)(x^2+6) &= x(y^2+1).
\endalign
$$}
\podpis{Ján Mazák:???}

{%%%%%   vyberko, den 3, priklad 2
Nech $n$ je nepárne prirodzené číslo. Šachovnica $n\times n$ má políčka zafarbené striedavo
bielou a čiernou farbou, pričom rohy sú čierne. {\it Tromino\/} v~tvare~L je tvorené troma
susediacimi políčkami. Pre aké hodnoty~$n$ je možné pokryť všetky čierne políčka šachovnice
neprekrývajúcimi sa trominami? Ak je to možné, aký je minimálny počet potrebných tromín?}
\podpis{Ján Mazák:???}

{%%%%%   vyberko, den 3, priklad 3
Aký ciferný súčet môže mať štvorec (\tj. druhá mocnina) celého čísla?}
\podpis{Ján Mazák:???}

{%%%%%   vyberko, den 3, priklad 4
Daný je pravouhlý trojuholník~$ABC$ s~pravým uhlom pri vrchole~$C$, pričom $|BC|<|AC|$.
Označme $I$~stred vpísanej kružnice, $O$~stred strany~$AB$. Na stranách $AC$, $AB$ zvoľme
po rade body $Q$, $P$ tak, aby platilo $|CQ|=|CB|=|BP|$. Priesečník osi uhla~$ACB$
a~priamky~$QP$ označme~$S$. Kolmica na $QI$ prechádzajúca bodom~$S$ pretína $BQ$ v~bode~$R$.
Dokážte, že body $I$, $R$, $O$ ležia na jednej priamke.}
\podpis{Ján Mazák:???}

{%%%%%   vyberko, den 4, priklad 1
Nájdite všetky celočíselné riešenia rovnice
$$
\frac{13}{x^2} + \frac{1996}{y^2} = \frac{z}{1997}.
$$}
\podpis{Juraj Földes:???}

{%%%%%   vyberko, den 4, priklad 2
Do štvorstena je vpísaná guľa, ktorá sa jednej steny dotýka v~priesečníku výšok, druhej sa
dotýka v~ťažisku a~tretej sa dotýka v~strede vpísanej kružnice. Dokážte, že potom je
štvorsten pravidelný.}
\podpis{Juraj Földes:???}

{%%%%%   vyberko, den 4, priklad 3
Nájdite súčet štvorcov prvých $100$~členov aritmetickej postupnosti, o~ktorej viete, že
súčet prvých $100$~členov je~$\m1$ a~súčet druhého, štvrtého, šiesteho,... a~stého člena
je~$\p1$.}
\podpis{Juraj Földes:???}

{%%%%%   vyberko, den 4, priklad 4
Nájdite všetky reálne riešenia rovnice
$$
\sqrt{x^2-p} + 2\sqrt{x^2-1} = x,
$$
kde $p$ je reálny parameter.}
\podpis{Juraj Földes:???}

{%%%%%   vyberko, den 5, priklad 1
Fibonacciho postupnosť~$F_n$ je definovaná nasledovne: $F_1=1$, $F_2=1$, $F_{n+2}=F_{n+1}+F_n$.
Určte najväčší spoločný deliteľ čísel $F_{2005}$ a $F_{3005}$.}
\podpis{Radovan Bauer:???}

{%%%%%   vyberko, den 5, priklad 2
Konečná postupnosť celých čísel $a_0,a_1,\dots,a_n$ sa nazýva kvadratická, ak pre každé
$i\in\{1,2,\dots,n\}$ platí $|a_i-a_{i-1}|=i^2$.
\ite (a) Dokážte, že pre ľubovoľné dve celé čísla $b$ a~$c$ existuje prirodzené číslo~$n$
a~kvadratická postupnosť taká, že platí $a_0=b$ a $a_n=c$.
\ite (b) Nájdite najmenšie prirodzené číslo~$n$, pre ktoré existuje kvadratická postupnosť
s~$a_0=0$ a~$a_n=2004$.}
\podpis{Radovan Bauer:???}

{%%%%%   vyberko, den 5, priklad 3
Nech $a$, $b$ sú dané prirodzené čísla, nech $f$ je funkcia z~$\Bbb R$ do~$\Bbb R$ taká, že
platí $|f(x)|\le1$ a
$$
f(x+a+b) + f(x) = f(x+a) + f(x+b)
$$
pre všetky $x\in\Bbb R$. Dokážte, že $f$ je periodická funkcia (\tj. existuje také~$p$, že
$f(x+p)=f(x)$ pre všetky $x\in\Bbb R$).}
\podpis{Radovan Bauer:???}

{%%%%%   vyberko, den 5, priklad 4
Dve kružnice $k_1$ a~$k_2$ sa pretínajú v~bodoch $P$ a~$Q$. Zvoľme si ľubovoľne body
$A_1$, $B_1$ na kružnici~$k_1$. Priamky $A_1P$ a $B_1P$ pretínajú kružnicu~$k_2$ okrem
bodu~$P$ v~bodoch $A_2$, $B_2$ a~priamky $A_1B_1$, $A_2B_2$ sa pretínajú v~bode~$C$.
Dokážte, že keď meníme polohu bodov $A_1$, $B_1$, tak stredy kružníc opísaných
trojuholníku~$A_1A_2C$ ležia na kružnici.}
\podpis{Radovan Bauer:???}


{%%%%%   trojstretnutie, priklad 1
Nech $n\geq 2$ je prirodzené číslo. V~obore reálnych čísel vyriešte sústavu rovníc
$$
\align
   \max\{1,x_1\}   =& x_2,\\
   \max\{2,x_2\}   =& x_3,\\
                   \vdots\phantom{x} & \\
\max\{n-1,x_{n-1}\}=& (n-1)x_n,\\
   \max\{n,x_n\}   =& nx_1.\
\endalign
$$}
\podpis{...}

{%%%%%   trojstretnutie, priklad 2
Daný je ostrouhlý trojuholník~$ABC$, v~ktorom veľkosť vnútorného uhla pri vrchole~$B$ je
väčšia ako $45^{\circ}$. Nech $D$, $E$, $F$ sú po rade päty výšok z~vrcholov $A$, $B$, $C$
a~nech $K$ je taký bod úsečky~$AF$, že platí $|\uh DKF|=|\uh KEF|$. Dokážte, že
\ite a) taký bod $K$ vždy existuje;
\ite b) platí rovnosť $|KD|^2=|FD|^2+|AF|\cdot |BF|$.}
\podpis{...}

{%%%%%   trojstretnutie, priklad 3
Ak pre čísla $p$, $q$, $r$ z~intervalu $\langle 2/5,5/2 \rangle$
platí $pqr=1$, potom existujú dva trojuholníky s~rovnakým obsahom,
pričom jeden má strany $a$, $b$, $c$ a~druhý má strany $pa$, $qb$, $rc$. Dokážte.}
\podpis{...}

{%%%%%   trojstretnutie, priklad 4
Daný je trojuholník~$ABC$ a~v~jeho vnútri bod~$P$ ležiaci na ťažnici z~vrcholu~$C$. Označme $X$
priesečník priamky~$AP$ so stranou~$BC$ a~$Y$ priesečník priamky~$BP$ so stranou~$AC$. Dokážte,
že ak je štvoruholník~$ABXY$ tetivový, potom je trojuholník~$ABC$ rovnoramenný.}
\podpis{...}

{%%%%%   trojstretnutie, priklad 5
Určte všetky prirodzené čísla $n\geq 2$, pre ktoré sú všetky binomické koeficienty
$$
{n \choose 1},\ {n \choose 2},\ \dots ,\ {n \choose n-1}
$$
párne čísla.}
\podpis{...}

{%%%%%   trojstretnutie, priklad 6
Nájdite všetky funkcie $f:\Bbb R \to \Bbb R$, ktoré pre všetky $x,y\in\Bbb R$ spĺňajú vzťah
$$
f\bigl(f(x)+y\bigr)=2x+f\bigl(f(y)-x\bigr).
$$}
\podpis{...}

{%%%%%   IMO, priklad 1
Nech $A$ je podmnožina množiny $S=\{1,2,\dots,1\,000\,000\}$
obsahujúca práve $101$~prvkov. Dokážte, že v~$S$ existujú čísla
$t_1,t_2,\dots,t_{100}$ také, že množiny
$$
A_j=\{x+t_j: x\in A\} \quad \text{pre }j=1,2,\dots,100
$$
sú po dvoch disjunktné.}
\podpis{Brazília}

{%%%%%   IMO, priklad 2
Určte všetky dvojice prirodzených čísel $(a,b)$ také, že
$$
a^2\over2ab^2-b^3+1
$$
je prirodzené číslo.}
\podpis{Bulharsko}

{%%%%%   IMO, priklad 3
Je daný konvexný šesťuholník, ktorého ľubovoľné dve protiľahlé strany
majú nasledujúcu vlastnosť: vzdialenosť ich stredov je
$\sqrt3/2$~násobok súčtu ich dĺžok. Dokážte, že všetky uhly
daného šesťuholníka sú rovnaké.

(Konvexný šesťuholník~$ABCDEF$ má tri dvojice protiľahlých strán:
$AB$ a~$DE$, $BC$ a~$EF$, $CD$ a~$FA$.)}
\podpis{Poľsko}

{%%%%%   IMO, priklad 4
Nech~$ABCD$ je tetivový štvoruholník. Označme postupne $P$, $Q$
a~$R$ päty kolmíc z~bodu~$D$ na priamky $BC$, $CA$ a~$AB$.
Dokážte, že $|PQ|=|QR|$ práve vtedy, keď sa osi uhlov $ABC$ a~$ADC$
pretínajú na priamke~$AC$.}
\podpis{Fínsko}

{%%%%%   IMO, priklad 5
Nech $n$ je prirodzené číslo a~$x_1,x_2,\dots,x_n$ reálne čísla
také, že $x_1\le x_2\le \dots\le x_n$.
\ite{} \hskip-2pt(a)
Dokážte, že
$$
\biggl(\sum_{i=1}^n\sum_{j=1}^n|x_i-x_j|\biggr)^{\!2}\le
{2(n^2-1)\over3}\sum_{i=1}^n\sum_{j=1}^n(x_i-x_j)^2.
$$
\ite{} \hskip-2pt(b)
Ukážte, že rovnosť platí práve vtedy, keď $x_1,x_2,\dots,x_n$ je
aritmetická postupnosť.

}
\podpis{Írsko}

{%%%%%   IMO, priklad 6
Nech $p$ je prvočíslo. Dokážte, že existuje prvočíslo~$q$
také, že pre žiadne celé číslo~$n$ nie je číslo $n^p-p$ deliteľné~$q$.}
\podpis{Francúzsko}

