{%%%%%   Z4-I-1
...}

{%%%%%   Z4-I-2
...}

{%%%%%   Z4-I-3
...}

{%%%%%   Z4-I-4
...}

{%%%%%   Z4-I-5
...}

{%%%%%   Z4-I-6
...}

{%%%%%   Z5-I-1
\napad
Koľko z~osemnástich skladieb zaznie v~jeden deň a~na koľko z~nich sa nedostane?

\riesenie
Medzi odohraním prvej rannej a~poslednej večernej skladby uplynie ${22-8}=14$ hodín, každý deň sa preto hrá pätnásť skladieb.
Vo zvonkohre je nastavených osemnásť skladieb, teda o~tri viac.

Na skladbu, ktorú počuli hrdinovia úlohy v~pondelok o~15.~hodine, sa v~utorok dostalo o~3~hodiny neskôr, teda o~18.~hodine.
V~stredu o~ďalšie 3~hodiny neskôr, teda o~21.~hodine.

Vo štvrtok sledovaná skladba nezaznela vôbec: od posledného jej uvedenia hrala zo sedemnástich ostatných skladieb jedna v~stredu o~22.~hodine, pätnásť vo štvrtok, jedna v~piatok o~8.~hodine a~na sledovanú skladbu sa dostalo až o~9.~hodine.

Nasledujúci deň, v~sobotu, zaznela sledovaná skladba o~3~hodiny neskôr, teda o~12.~hodine.
V~nedeľu o~ďalšie 3~hodiny neskôr, teda o~15.~hodine.

Daný týždeň hrala sledovaná skladba napoludnie jedine v~sobotu, Oľga a~Ľuboš prišli na nádvorie druhý raz v~sobotu.

\poznamka
Predchádzajúca diskusia môže byť nahradená programom zvonkohry pre daný týždeň.
Skladby označujeme číslami 1 až 18, pričom 1 označuje tú, ktorú počuli Oľga s~Ľubošom:
\bgroup
\def\ctr#1{\hfil\hskip.3em #1\hskip.3em \hfil}
$$
\begintable
\|\ 8\,h|\ 9\,h|10\,h|11\,h|12\,h|13\,h|14\,h|15\,h|16\,h|17\,h|18\,h|19\,h|20\,h|21\,h|22\,h\crthick
po\|||||||\dots|\bf1|2|3|4|5|6|7|8\cr
ut\|9|10|11|12|13|14|15|16|17|18|\bf1|2|3|4|5\cr
st\|6|7|8|9|10|11|12|13|14|15|16|17|18|\bf1|2\cr
št\|3|4|5|6|7|8|9|10|11|12|13|14|15|16|17\cr
pi\|18|\bf1|2|3|4|5|6|7|8|9|10|11|12|13|14\cr
so\|15|16|17|18|\bf1|2|3|4|5|6|7|8|9|10|11\cr
ne\|12|13|14|15|16|17|18|\bf1|\dots||||||\endtable
$$
\egroup
}

{%%%%%   Z5-I-2
\napad
Môžu rôzne vyplnenia rohových políčok viesť k~rovnakému súčtu v~kruhu?


\riesenie
Čísla 2, 4, 6 a~8 možno do rohových políčok napísať mnohými spôsobmi.
Pre rôzne spôsoby však môžeme nakoniec dostať ten istý súčet v~kruhu, ako napr. v~nasledujúcich prípadoch:
\figure{z5-I-2a}%


\noindent
Prvý a~druhý prípad sa líšia pootočením o~90\st, druhý a~tretí sú súmerné podľa vodorovnej osi štvorca a~pod.
Rôzne hodnoty v~kruhu teda dostaneme iba vtedy, keď čísla v~rohových políčkach nie sú nijako súmerné.

Skúšanie všetkých možností si preto môžeme zjednodušiť tým, že uvažujeme jedno z~čísel, napr. 2, v~jednom konkrétnom políčku, napr. vľavo hore.
Ostatné čísla dopĺňame tak, aby žiadne dve vyplnenia neboli súmerné.
Pritom jediná súmernosť, ktorá zachováva 2 v~ľavom hornom políčku, je súmernosť podľa uhlopriečky prechádzajúcej týmto políčkom.
S~týmito požiadavkami máme nasledujúce tri riešenia:
\figure{z5-I-2b}%


V~kruhu môžu byť napísané čísla 84, 96 a~100.

\poznamka
Pre každé vyplnenie rohových políčok môžeme nájsť 7 ďalších, ktoré sú s~ním nejako súmerné.
Všetky vyplnenia teda možno rozdeliť do osmíc, ktoré určite majú rovnaké súčty v~kruhu.
Pritom čísla do štyroch rohových políčok môžeme vyplniť spolu 24~spôsobmi.
V~kruhu tak môžu byť napísané nanajvýš $24:8=3$ rôzne čísla.
Skúšaním určíme, ktoré čísla to sú, a~že sú navzájom rôzne.
}

{%%%%%   Z5-I-3
\napad
Určte dĺžku strany malého štvorca.

\riesenie
Obvod malého štvorca je päťkrát menší ako obvod dlaždice, preto aj jeho strana je päťkrát menšia ako strana dlaždice.
Strana malého štvorca teda meria
$$
10:5=2\,(\text{dm}).
$$
Pritom dĺžka strany dlaždice je rovná súčtu dĺžok strany malého štvorca a~dvoch kratších strán obdĺžnika.
Kratšia strana obdĺžnika teda meria
$$
(10-2):2=4\,(\text{dm}).
$$
Súčasne dĺžka strany dlaždice je rovná súčtu dĺžok kratšej a~dlhšej strany obdĺžnika.
Dlhšia strana obdĺžnika teda meria
$$
10-4=6\,(\text{dm}).
$$
Rozmery obdĺžnikov sú $4\,\text{dm}\times 6\,\text{dm}$.
}

{%%%%%   Z5-I-4
\napad
Počítajte po trojiciach.

\riesenie
Trojica smriečok, borovička a~jedlička stála dokopy
$$
22+25+33=80\,\text{\euro}.
$$

Predavač takých trojíc za celý deň predal $3\,600:80=45$.

Celkom teda predal $3\cdot 45=135$ stromčekov.
}

{%%%%%   Z5-I-5
\napad
Začnite od prvej hviezdičky.

\bgroup
\let\times=\cdot

\riesenie
Postupne dosadíme všetky možné cifry na mieste prvej hviezdičky, vypočítame súčin a~preveríme ostatné požiadavky:
\begin{itemize}
\item $42\times 18=756$, výsledok je menší ako $2\,000$; nevyhovuje.
\item $42\times 28=1\,176$, výsledok je menší ako $2\,000$; nevyhovuje.
\item $42\times 38=1\,596$, výsledok je menší ako $2\,000$; nevyhovuje.
\item $42\times 48=2\,016$, súčet $4+0+1+6=11$ je nepárny; vyhovuje.
\item $42\times 58=2\,436$, súčet $5+4+3+6=18$ je párny; nevyhovuje.
\item $42\times 68=2\,856$, súčet $6+8+5+6=25$ je nepárny; vyhovuje.
\item $42\times 78=3\,276$, výsledok je väčší ako $2\,999$; nevyhovuje.
\item zvyšné dva súčiny sú ešte väčšie; nevyhovujú.
\end{itemize}

Úloha má dve riešenia, a~to
$$
42\times 48=2\,016
\quad\text{a}\quad
42\times 68=2\,856.
$$
\egroup
}

{%%%%%   Z5-I-6
\napad
Začnite s~vrcholmi, ktoré sú spoločné obom trojuholníkom.

\riesenie
Pomenujme vrcholy ako na nasledujúcom obrázku a~všimnime si, že body $A$ a~$C$ sú vždy od zvyšných troch bodov rovnako vzdialené (zodpovedajúce úsečky tvoria strany rovnostranných trojuholníkov).
Preto kružnica so stredom v~bode~$A$ prechádzajúca bodom~$B$ prechádza aj bodmi $C$ a~$D$.
Kružnica so stredom v~bode~$A$ vyhovujúca Jarkiným požiadavkám je teda jediná.
Podobne existuje jediná vyhovujúca kružnica so stredom v~bode~$C$.

Kružnica so stredom v~bode~$B$ prechádzajúca bodom~$A$ prechádza aj bodom~$C$, ďalšia kružnica prechádza bodom~$D$.
Kružnice so stredom v~bode~$B$ vyhovujúce Jarkiným požiadavkám sú teda dve.
Podobne existujú dve vyhovujúce kružnice so stredom v~bode~$D$.

Spolu existuje šesť vyhovujúcich kružníc:
\figure{z5-I-6a}%


}

{%%%%%   Z6-I-1
Aby bol súčet uvedeného tvaru, nemohlo mať žiadne z~čísel pred/za desatinnou čiarkou viac ako dve cifry.
Teda ako Janino, tak Dávidovo číslo bolo typu $**{,}**$ alebo $*{,}*$.
Keďže výsledné číslo malo na mieste stotín nenulovú cifru, muselo mať aspoň jedno z~čísel na mieste stotín~-- a~teda aj~na mieste desiatok~-- nenulovú cifru.
Keby aj druhé z~čísel malo rovnaké vlastnosti, bol by výsledok väčší ako $11{,}11$.
Druhé z~čísel preto bolo typu $*{,}*$ a~celý príklad môžeme zapísať takto:
$$
\alggg{*&*{,}&*&*\\&*{,}&*&}{1&1{,}&1&1}
$$

Keďže iba prvé číslo má na mieste desiatok a~stotín nenulové cifry, musia byť obe tieto cifry 1.
Celý príklad potom vyzerá nasledovne:
$$
\alggg{1&*{,}&*&1\\&*{,}&*&}{1&1{,}&1&1}
$$
Keďže Dávidovo číslo bolo zapísané navzájom rôznymi ciframi, je prvé číslo Janino a~druhé Dávidovo.
Zaujíma nás, ktoré najväčšie číslo mohol napísať Dávid, čiže ktoré najmenšie číslo mohla napísať Jana:
\begin{itemize}
\item Najmenšie mysliteľné Janino číslo je $10{,}01$, čo ale nevyhovuje podmienke, že práve dve cifry sú rovnaké.
\item Ďalšie mysliteľné Janino číslo je $10{,}11$, čo nevyhovuje z~rovnakého dôvodu.
\item Ďalšie mysliteľné Janino číslo je $10{,}21$, v~tomto prípade by Dávidovo číslo bolo $0{,}9$ a~táto možnosť vyhovuje všetkým podmienkam zo zadania.
\end{itemize}
Najväčšie číslo, ktoré mohol Dávid napísať, bolo číslo $0{,}9$.
}

{%%%%%   Z6-I-2
Načrtneme si, ako Kockorádova záhrada vyzerala (čísla označujú poradie dláždených chodníkov):
\figure{z6-I-2}%


Prvý a~tretí chodník majú rovnaké rozmery, teda aj rovnakú plochu.
Pri treťom chodníku sa spotrebovalo menej dlažby ako pri prvom preto, lebo tretí chodník sa v~mieste kríženia s~druhým chodníkom nedláždil (tam už bola dlažba položená).
Táto spoločná časť druhého a~tretieho chodníka je štvorec, ktorého strana zodpovedá šírke chodníka.
Na vydláždenie tohto štvorca bolo spotrebovaných
$$
228-219=9\,(\text{m}^2)
$$
dlažby.
Štvorec s~obsahom 9\,m$^2$ má stranu dlhú 3\,m, šírka všetkých troch chodníkov je preto rovná 3\,m.
Z~toho a~z~množstva dlažby použitej na jednotlivé chodníky môžeme určiť rozmery záhrady:

Na prvý chodník sa spotrebovalo 228\,m$^2$ dlažby, čo je skutočný obsah plochy chodníka.
Dĺžka záhrady je
$$
228:3=76\,(\text{m}).
$$
Na druhý chodník sa spotrebovalo 117\,m$^2$ dlažby, čo je o~9\,m$^2$ menej ako skutočný obsah plochy chodníka (spoločný štvorec s~prvým chodníkom bol už vydláždený).
Šírka záhrady je
$$
(117+9):3=126:3=42\,(\text{m}).
$$
Rozmery Kockorádovej záhrady sú 76\,m a~42\,m.

\poznamka
Ak riešiteľ pri výpočte šírky záhrady nepripočíta oných 9\,m$^2$, obdrží chybný rozmer 39\,m.
Také riešenie hodnoťte nanajvýš stupňom "dobre".
}

{%%%%%   Z6-I-3
\napad
Koľký článok odzadu má ako prvý obe nohy oblečené?

\riesenie
Naznačme niekoľko posledných článkov mnohonožky Mirky (zľava doprava), horný riadok predstavuje ľavé nohy, spodný riadok pravé.
Oblečené nohy vyznačujeme čierno, bosé nohy bielo, a~to tak dlho, kým nie sú na jednom článku obe nohy oblečené -- potom sa vzor opakuje:
\figure{z6-I-3a}%


Keby mala mnohonožka 15~článkov, boli by na 8~článkoch obe nohy bosé.
Pokračujeme ďalej, kým nedostaneme 14~článkov s~oboma nohami bosými:
\figure{z6-I-3b}%


Pokračujeme ďalej, kým je počet článkov s~oboma nohami bosými rovnaký:
\figure{z6-I-3c}%


Mnohonožka Mirka mala buď 26, alebo 27 článkov, teda buď 52, alebo 54 nôh.
}

{%%%%%   Z6-I-4
\napad
Spočítajte bratov každého dieťaťa.

\riesenie
V~každej skupine sú len deti s~rovnakým počtom bratov a~počet bratov každého dieťaťa je zo zadania známy.
Preto sa deti mohli rozdeliť jediným možným spôsobom.
Stačí postupne určiť počty bratov každého dieťaťa a~utvoriť zodpovedajúce skupiny.
\begin{itemize}
\item Alica a~Betka majú jedného brata, Cyril žiadneho.
\item Erika a~Gabika majú dvoch bratov, Dávid a~Filip jedného.
\item Iveta má jedného brata, Hugo žiadneho.
\item Ján, Karol a~Lukáš majú dvoch bratov.
\end{itemize}
\noindent
Deti sa teda rozdelili do troch skupín:
\begin{itemize}
\item Erika, Gabika, Ján, Karol, Lukáš.
\item Alica, Betka, Dávid, Filip, Iveta.
\item Cyril, Hugo.
\end{itemize}

}

{%%%%%   Z6-I-5
\napad
Začnite v~niektorom rohovom štvorčeku.

\riesenie
Ak je ľavý horný štvorček vyfarbený nejakou farbou, musia byť okolité tri štvorčeky vyfarbené navzájom rôznymi farbami.
Juro musel použiť aspoň 4~farby, ktoré budeme označovať ciframi od 1 do 4:
\figure{z6-I-5a}%


Teraz potrebujeme zistiť, či štyri farby stačia na vyfarbenie zvyšku siete podľa uvedených pravidiel, alebo nie.
Postupne zistíme, že štyri farby naozaj stačia:
\figure{z6-I-5b}%


}

{%%%%%   Z6-I-6
V~nevyplnených bielych políčkach môžu byť napísané iba také dvojice celých čísel, ktorých súčin je 55 a~z~ktorých každé je väčšie ako~1.
Tomu vyhovuje iba dvojica 5 a~11 a~tým sú tiež určené čísla v~ostatných nevyplnených políčkach:
v~svetlo sivých políčkach dostávame $9\cdot5=45$ a~$9\cdot11=99$,
v~najtmavšom potom $55\cdot45\cdot99=245\,025$.
\figure{z6-I-6a}%


\poznamka
Čísla 5 a~11 možno do obrázka vyplniť dvojakým spôsobom, čo je
v~tejto úlohe nepodstatné.
Na hodnotenie nemá vplyv, či riešiteľ uvažuje obe možnosti, alebo iba jednu.
}

{%%%%%   Z7-I-1
Lomená čiara, ktorá rozdeľuje štvorec na dva útvary, leží celá vnútri štvorca, s~jeho obvodom má spoločné iba koncové body.
Vzhľadom na to, že nové útvary majú mať rovnaký obvod, musia mať rovnakú aj tú jeho časť, ktorá je na obvode štvorca.
Koncové body deliacej čiary preto musia byť súmerné podľa stredu štvorca.

Daný štvorec má obvod 16\,cm a~obvod dvoch nových útvarov má byť dokopy 32\,cm.
Deliaca čiara je spoločná obom útvarom, dvojnásobok jej dĺžky preto zodpovedá rozdielu $32-16=16$\,(cm).
Deliaca čiara musí byť dlhá 8\,cm.

Teraz už neostáva iné ako skúšať:
\figure{z7-I-1a}%


\poznamka
Akékoľvek iné rozdelenie je zhodné s~niektorým z~predchádzajúcich.
}

{%%%%%   Z7-I-2
Podľa zadaných výrokov si predstavíme, odkiaľ kto mohol prísť, ak všetci hovorili pravdu:
\begin{itemize}
\item zo severu Jozef alebo Zdeno,
\item z~východu Karol alebo Zdeno,
\item z~juhu Mojmír,
\item zo západu Karol alebo Zdeno.
\end{itemize}
\noindent
Teraz zvažujeme, ktoré výroky mohli byť nepravdivé:
\begin{itemize}
\item Keby klamal Mojmír, museli by všetci ostatní vravieť pravdu, a~to by znamenalo, že z~juhu neprišiel nikto.
Mojmírova výpoveď teda bola pravdivá.
\item Keby klamal Zdeno, musel by prísť z~juhu, čo by znamenalo, že klamal aj Mojmír.
Zdenova výpoveď teda bola pravdivá.
\item Keby klamal Karol, musel by prísť zo severu alebo z~juhu. To by v~prvom prípade znamenalo, že klamal aj Jozef, v~druhom prípade, že klamal aj Mojmír.
Karolova výpoveď teda bola pravdivá.
\end{itemize}
Mojmír, Zdeno a~Karol vraveli pravdu, klamal teda Jozef (čo skutočne nie je s~ničím v~rozpore).
Zo severu preto prišiel Zdeno, z~juhu prišiel Mojmír.

\poznamka
Jozef a~Karol prišli jeden z~východu a~jeden zo západu; z~uvedeného sa nedá presne určiť, kto prišiel odkiaľ.
}

{%%%%%   Z7-I-3
\napad
Koľko tortičiek má rovnakú cenu ako tri veterníky?

\riesenie
Za tie isté peniaze možno kúpiť o~tretinu viac tortičiek ako veterníkov, tzn. za cenu 3~veterníkov možno kúpiť 4~tortičky.
Veterník je o~0,40 € drahší ako tortička, 3~veterníky teda stoja o~1,20 € viac ako 3~tortičky.
Z~uvedeného vyplýva, že 1,20 € plus cena 3~tortičiek zodpovedá cene 4~tortičiek.
Preto tortička stojí 1,20 €.
Veterník stojí o~0,40 € viac, teda 1,60 €.

(Pre kontrolu: za 9,60 € môžu dievčatá kúpiť $9{,}60:1{,}60=6$ veterníkov alebo $9{,}60:1{,}20=8$ tortičiek.)

\poznamka
Pre všetkých, ktorí vedia vyjadriť podmienky zo zadania pomocou neznámych, dodávame:

Ak cenu tortičiek označíme $t$ (\euro), tak veterník stojí $t+0{,}40$ (\euro).
Ak počet veterníkov, ktoré možno kúpiť za všetky peniaze, označíme $k$, tak tortičiek možno za rovnaké peniaze kúpiť $\frac43k$.
Zo zadania vieme, že
$$
k\cdot(t+0{,}40)=\frac43k\cdot t.
$$
Z~toho vyplýva, že $3t+1{,}20=4t$, teda $t=1{,}20$, resp. $t+0{,}40=1{,}60$.
}

{%%%%%   Z7-I-4
\napad
Začnite s~druhou cifrou druhého činiteľa.

\riesenie
Číslo $4\,949$ je súčinom prvého činiteľa a~druhej cifry druhého činiteľa.
Pritom $4\,949=707\cdot 7$ a~pri delení čísla $4\,949$ inou cifrou ako 7 celočíselný trojciferný výsledok nedostaneme. Preto druhá cifra druhého činiteľa je~7:
$$
\vbox{\let\\=\cr
\halign{&\hbox to1.0em{\hss$#$\hss}\cr
&&&7&0&7\\
&\times&&*&7&*\\
\noalign{\vskip4pt\hrule\vskip4pt}
&&*&*&*&*\\
&4&9&4&9&\\
&*&*&*&&\\
\noalign{\vskip4pt\hrule\vskip4pt}
*&*&*&4&*&*\\
}}
$$

Posledný pomocný súčin je súčinom 707 a~prvej cifry druhého činiteľa.
Pritom tento súčin je trojciferný, teda táto cifra môže byť jedine 1:
$$
\vbox{\let\\=\cr
\halign{&\hbox to1.0em{\hss$#$\hss}\cr
&&&7&0&7\\
&\times&&1&7&*\\
\noalign{\vskip4pt\hrule\vskip4pt}
&&*&*&*&*\\
&4&9&4&9&\\
&7&0&7&&\\
\noalign{\vskip4pt\hrule\vskip4pt}
*&*&*&4&*&*\\
}}
$$

Prvý pomocný súčin je súčinom 707 a~tretej cifry druhého činiteľa.
Pritom tento súčin je štvorciferný, takže táto cifra musí byť väčšia ako~1.
Po dosadení všetkých cifier od 2 do 9 a~dopočítaní príkladu zisťujeme, že štvrtá cifra vo výsledku je rovná~4 len vtedy, keď neznáma cifra je~6.

Úloha má jediné riešenie:
$$
\vbox{\let\\=\cr
\halign{&\hbox to1.0em{\hss$#$\hss}\cr
&&&7&0&7\\
&\times&&1&7&6\\
\noalign{\vskip4pt\hrule\vskip4pt}
&&4&2&4&2\\
&4&9&4&9&\\
&7&0&7&&\\
\noalign{\vskip4pt\hrule\vskip4pt}
1&2&4&4&3&2\\
}}
$$

\poznamka
Záverečné skúšanie je možné urýchliť tým, že sa najskôr zameriame na druhú cifru prvého pomocného súčinu -- z~uvedeného možno ľahko vyvodiť, že to môže byť jedine~2.
}

{%%%%%   Z7-I-5
\napad
Uvedomte si, ako zostrojíte tretí vrchol trojuholníka, keď poznáte dva jeho vrcholy a~veľkosti dvoch zvyšných strán.

\riesenie
Úsečka~$AB$ je základňou rovnoramenného trojuholníka $ABX$, teda $|XA|=|XB|$.
Všetky body~$X$ s~touto vlastnosťou tvoria os úsečky~$AB$, \tj. kolmicu idúcu stredom úsečky~$AB$.

Ak je úsečka~$BC$ základňou rovnoramenného trojuholníka $BCX$, tak bod~$X$ musí ležať na osi úsečky~$BC$.
V~takom prípade je bod~$X$ priesečníkom osí úsečiek $AB$ a~$BC$ (na obrázku označený ako $X_1$).
\figure{z7-I-5b}%

Ak je úsečka~$BC$ ramenom rovnoramenného trojuholníka $BCX$ a~úsečka~$BX$ jeho základňou, tak $|CB|=|CX|$.
Všetky body~$X$ s~touto vlastnosťou tvoria kružnicu, ktorá má stred v~bode~$C$ a~prechádza bodom~$B$.
V~takom prípade je bod~$X$ priesečníkom tejto kružnice a~osi úsečky~$AB$ (dve možnosti, na obrázku označené $X_2$ a~$X_3$).

Ak je úsečka~$BC$ ramenom rovnoramenného trojuholníka $BCX$ a~úsečka~$CX$ jeho základňou, tak $|BC|=|BX|$.
Všetky body~$X$ s~touto vlastnosťou tvoria kružnicu, ktorá má stred v~bode~$B$ a~prechádza bodom~$C$.
V~takom prípade je bod~$X$ priesečníkom tejto kružnice a~osi úsečky~$AB$ (dve možnosti, na obrázku označené $X_4$ a~$X_5$).

Úloha má celkom päť riešení vyznačených na obrázku.
% \midinsert
% \figure{z7-I-5b}
% \endinsert

\poznamky
Zvolená mierka nemá vplyv na hodnotenie úlohy, zato však venujte pozornosť konštrukcii osi úsečky.

Os úsečky~$BC$ prechádza spoločnými bodmi vyznačených kružníc.
Body $X_4$ a~$X_5$ možno zostrojiť aj ako priesečníky kružnice so stredom v~bode~$B$ prechádzajúcej bodom~$C$ a~kružnice s~tým istým polomerom a~stredom v~bode~$A$.
}

{%%%%%   Z7-I-6
Medzi číslami 900 a~1\,001 má najväčší ciferný súčet číslo 999, a~to 27;
väčšími súčtami sa zaoberať nemusíme.

Medzi číslami 1 a~27 má najväčší ciferný súčet číslo 19, a~to 10;
väčšími súčtami sa zaoberať nemusíme.

Medzi číslami 1 a~10 majú ciferný súčet 1 iba čísla 1 a~10;
ostatnými číslami sa zaoberať nemusíme.

Teraz odzadu určíme všetky riešenia
(v~prvom stĺpci je 1 a~v~každom ďalšom stĺpci sú čísla, ktorých ciferný súčet je rovný číslu v~stĺpci predchádzajúcom):

\bgroup
\def\ctr#1{\hfil\quad#1\quad}
\def\tstrut{\vrule height 11pt depth 5pt width 0pt}
$$
\begintable
1|1|1|1\,000\nr
\omit|\omit|\omit\hrulefill|\omit\hrulefill\nr
||10|901\nr
\omit|\omit|\omit|\omit\hrulefill\nr
|||910\nr
\omit|\omit\hrulefill|\omit\hrulefill|\omit\hrulefill\nr
|10|19|919\nr
\omit|\omit|\omit|\omit\hrulefill\nr
|||928\nr
\omit|\omit|\omit|\omit\hrulefill\nr
|||937\nr
\omit|\omit|\omit|\omit\hrulefill\nr
|||946\nr
\omit|\omit|\omit|\omit\hrulefill\nr
|||955\nr
\omit|\omit|\omit|\omit\hrulefill\nr
|||964\nr
\omit|\omit|\omit|\omit\hrulefill\nr
|||973\nr
\omit|\omit|\omit|\omit\hrulefill\nr
|||982\nr
\omit|\omit|\omit|\omit\hrulefill\nr
|||991\endtable
$$
\egroup
Čísel s~uvedenými vlastnosťami je spolu 12.
}

{%%%%%   Z8-I-1
\napad
Akú časť všetkých nájdených orieškov doniesli veveričky domov?

\riesenie
Ak množstvo orieškov, ktoré našla Pizizubka, označíme $x$, tak Ryšavka našla $2x$ orieškov a~Uška $3x$ orieškov.
\begin{itemize}
\lineskiplimit 1pt \lineskip 2pt
\item Pizizubka zjedla polovicu svojich orieškov, zvýšilo jej $\frac12{x}$ orieškov.
\item Ryšavka zjedla tretinu svojich orieškov, zvýšilo jej $\frac23\cdot2x=\frac43 x$ orieškov.
\item Uška zjedla štvrtinu svojich orieškov, zvýšilo jej $\frac34\cdot3x=\frac94 x$ orieškov.
\end{itemize}
\noindent
Všetkým veveričkám spolu zvýšilo
$$
\left(\frac12+\frac43+\frac94\right)x=\frac{49}{12}x
$$ orieškov, čo je podľa zadania rovné 196.
Takže
$$
\aligned
\frac{49}{12}x&=196,\\
\frac{x}{12}&=4,\\
x&=48.
\endaligned
$$
Pizizubka našla 48~orieškov, Ryšavka $2\cdot48=96$ orieškov a~Uška $3\cdot48=144$ orieškov.
}

{%%%%%   Z8-I-2
\napad
Koľkokrát je každé číslo započítané do celkového súčtu?

\riesenie
Číslo na každej stene je započítané celkom v~štyroch čiastočných súčtoch
(každá stena sa počíta raz ako prostredná a~trikrát ako susedná).
Preto je aj vo výslednom súčte každé z~čísel započítané štyrikrát.
Výsledný súčet teda nadobúda hodnotu
$$
4\cdot(1+2+3+4+5+6+7+8)=4\cdot36=144,
$$
a~to nezávisle na tom, ako boli čísla na stenách osemstena napísané.
%\def\tstrut{\vrule height 11.5pt depth 5pt width 0pt}
}

{%%%%%   Z8-I-3
\napad
Začnite s~Petrom.

\riesenie
Keďže výkonnosť Petra bola 10, musel nastrieľať v~prvých troch kolách po 10~bodov.
Keďže súčet hodnotení v~3.~kole bol 32~bodov, musel Zdeno v~tomto kole trafiť 8~bodov.
Keďže súčet hodnotení v~4.~kole bol 32~bodov, musel byť súčet hodnotení Michala a~Zdena v~tomto kole 17~bodov.
Keďže v~žiadnom kole nemali žiadni dvaja chlapci rovnaké hodnotenie, mohli mať v~tomto kole
\begin{enumerate}\alphatrue
\item buď Michal 9 a~Zdeno 8~bodov,
\item alebo Michal 8 a~Zdeno 9~bodov.
\end{enumerate}

\smallskip
Predpokladajme možnosť~a) a~pokúsme sa doplniť tabuľku
$$\begintable
\|1. kolo|2. kolo|3. kolo|4. kolo\|výkonnosť\crthick
Peter\|10|10|10|5\|10\cr
Juraj\|||9|10\|7,5\cr
Michal\|||5|9\|8\cr
Zdeno\|||8|8\|8,5\crthick
celkom\|32|32|32|32\|$-$\endtable
$$
Aby výkonnosť Zdena bola $8{,}5=17:2$, musel v~prvých dvoch kolách trafiť po 9~bodov.
Aby výkonnosť Michala bola $8=16:2$ a~aby v~žiadnom kole nemal rovnaké hodnotenie ako Zdeno, musel v~prvých dvoch kolách trafiť po 8~bodov.
Aby súčet hodnotení v~1. aj~2.~kole bol 32~bodov, musel Juraj v~týchto dvoch kolách trafiť po 5~bodov.
V~takom prípade by však jeho výkonnosť nebola 7,5 (ale iba 7).
Možnosť~a) preto nemohla nastať.

\smallskip
Predpokladajme možnosť~b) a~pokúsme sa doplniť tabuľku
$$\begintable
\|1. kolo|2. kolo|3. kolo|4. kolo\|výkonnosť\crthick
Peter\|10|10|10|5\|10\cr
Juraj\|||9|10\|7,5\cr
Michal\|||5|8\|8\cr
Zdeno\|||8|9\|8,5\crthick
celkom\|32|32|32|32\|$-$\endtable
$$
Aby výkonnosť Juraja bola $7{,}5=15:2$,
musel v~jednom z~prvých dvoch kôl trafiť 6 a~v~druhom 6 alebo menej bodov.
Aby výkonnosť Michala bola $8=16:2$ a~aby v~žiadnom kole nemal rovnaké hodnotenie ako Peter, musel v~jednom z~prvých dvoch kôl trafiť 8 a~v druhom 8 alebo 9~bodov.

Keby Juraj trafil 6~bodov v~rovnakom kole ako Michal~8, tak by Zdeno v~rovnakom kole musel trafiť 8~bodov (aby bol súčet hodnotení v~tomto kole rovný 32~bodov).
To by však Michal a~Zdeno mali rovnaké hodnotenie, preto táto možnosť nastať nemohla.

Juraj teda musel trafiť 6~bodov v~inom kole ako Michal~8.
Predpokladajme, že sa tak stalo v~1.~kole a~pokúsme sa doplniť tabuľku
(z~predchádzajúceho vyplýva, že Michal v~tom istom kole musel trafiť 9~bodov)
$$\begintable
\|1. kolo|2. kolo|3. kolo|4. kolo\|výkonnosť\crthick
Peter\|10|10|10|5\|10\cr
Juraj\|6||9|10\|7,5\cr
Michal\|9|8|5|8\|8\cr
Zdeno\|||8|9\|8,5\crthick
celkom\|32|32|32|32\|$-$\endtable
$$
Aby súčet hodnotení v~1.~kole bol 32~bodov, musel Zdeno v~tomto kole trafiť 7~bodov.
Aby výkonnosť Zdena bola $8{,}5=17:2$, musel v~druhom kole trafiť 9~bodov.
Aby súčet hodnotení v~3.~kole bol 32~bodov, musel Juraj v~tomto kole trafiť 5~bodov.
Keďže 5 je menšie ako 6, súhlasí výkonnosť Juraja so zadaním.
Našli sme jedno vyhovujúce riešenie úlohy:
$$\begintable
\|1. kolo|2. kolo|3. kolo|4. kolo\|výkonnosť\crthick
Peter\|10|10|10|5\|10\cr
Juraj\|6|5|9|10\|7,5\cr
Michal\|9|8|5|8\|8\cr
Zdeno\|7|9|8|9\|8,5\crthick
celkom\|32|32|32|32\|$-$\endtable
$$

Juraj však mohol trafiť 6~bodov v~2.~kole.
V~takom prípade by výsledná tabuľka mala vymenené hodnotenie pri~1. a~2.~kole.
}

{%%%%%   Z8-I-4
Stred úsečky~$BE$, ktorým podľa zadania prechádza priamka~$DF$, označíme~$G$.
Úsečka~$FG$ je strednou priečkou trojuholníka $BCE$, ktorá je rovnobežná so stranou~$BC$.
Preto štvoruholník $GBCD$ je rovnobežníkom, a~teda platí, že úsečky $EG$, $GB$, $DC$ a~$AE$ sú navzájom zhodné.
\figure{z8-I-4a}%


Lichobežník $ABCD$ tak môžeme rozdeliť na štyri trojuholníky $AED$, $DCE$, $EGC$ a~$GBC$ s rovnakým obsahom (prvé tri trojuholníky sú dokonca navzájom zhodné).
Obsah lichobežníka je preto rovný štvornásobku obsahu trojuholníka $CDE$, \tj.
$$4\cdot 3=12\,\cm^2.
$$

\poznamka
Trojuholníky $DFC$ a~$GFE$ sú zhodné, preto má rovnobežník $AECD$ rovnaký obsah ako trojuholník $AGD$, a~ten je zhodný s~trojuholníkom $EBC$.
(V~oboch prípadoch možno zhodnosť trojuholníkov zdôvodniť niekoľkými spôsobmi, napr. podľa vety {\it usu\/}.)
Obsah lichobežníka $ABCD$ je preto rovný dvojnásobku obsahu rovnobežníka $AECD$, a~ten je rovný dvojnásobku obsahu trojuholníka $CDE$.

Z~uvedeného tiež vyplýva, že obsah trojuholníka $EBC$ je štvornásobkom obsahu trojuholníka $DFC$, a~ten je rovný polovici obsahu trojuholníka $CDE$.
}

{%%%%%   Z8-I-5
Keď sa k~zákuskom dostal Ján, bolo ich 6 rovnakého druhu, a~to karamelových kociek --
keby to boli kokosky alebo laskonky, muselo by kociek byť viac ako 6 a~zákuskov celkom by potom bolo viac ako 10.
Preto karamelových kociek pôvodne bolo aspoň 6 a~mamička priniesla
\begin{itemize}
\item buď 1 kokosku, 3 laskonky a~6 kociek,
\item alebo 1 kokosku, 2 laskonky a~7~kociek.
\end{itemize}
Prvá možnosť nie je vyhovujúca~-- aby Jozef aj Jakub mali každý dva zákusky rôznych druhov, musel by aspoň jeden z~nich vybrať aj kocku, a~to by ich potom na Jána neostalo 6.

Druhá možnosť je vyhovujúca~-- jeden z~prvých dvoch chlapcov si vybral kokosku a~laskonku, druhý laskonku a~kocku, na Jána zvýšilo 6~kociek.

Mamička priniesla 1~kokosku, 2~laskonky a~7~karamelových kociek.
}

{%%%%%   Z8-I-6
\napad
Aký je najmenší spoločný násobok troch čísel, z~ktorých jedno je deliteľom iného?

\riesenie
Číslo 30 má celkom 8~deliteľov, ktoré môžu byť dosadené za $A$ a~$B$.
\figure{z8-I-6a}%


Číslo~$C$ je najmenším spoločným násobkom 13 a~$A$, číslo~$E$ je najmenším spoločným násobkom $C$ a~30, teda $E$ je najmenším spoločným násobkom čísel 13, $A$ a~30.
Keďže $A$ je deliteľom čísla~30, je $E$ najmenším spoločným násobkom čísel 13 a~30, \tj. $390=13\cdot 30$.

Podobne možno zdôvodniť, že bez ohľadu na hodnotu~$B$ je $F$ najmenším spoločným násobkom čísel $14=2\cdot7$ a~$30=2\cdot3\cdot5$, \tj. $210=7\cdot30$.

Číslo $G$ v~najspodnejšej tehličke preto môže byť jedine najmenším spoločným násobkom čísel 390 a~210, \tj. $2\,730=7\cdot13\cdot30$.
\figure{z8-I-6b}%


\poznamka
Keby sme uvažovali všetky možné dvojice čísel, ktorých najmenší spoločný násobok je 30, dostali by sme celkom 27~možností.
Dopĺňaním jednotlivých prípadov za $A$ a~$B$ si každý skôr či neskôr všimne, že čísla $E$, $F$, a~teda aj~$G$ sú stále rovnaké.
}

{%%%%%   Z9-I-1
\napad
Môže byť číslo v~kruhu menšie ako 20, resp.\ väčšie ako 20\,000?

\riesenie
Pre dostatočne veľké číslo v~niektorom z~rohových políčok vonkajšieho štvorca môžu byť zodpovedajúce súčiny vo vnútornom štvorci väčšie ako ľubovoľné vopred zvolené číslo, pričom ľahko dokážeme zabezpečiť, aby každé z~čísel 2, 4, 6 a~8 bolo použité.
Preto aj súčet v~kruhu môže byť ľubovoľne veľký.

\smallskip
Zistíme, aký najmenší súčet môže byť v~kruhu.
Určite to nemôže byť žiadne z~predpísaných čísel: ani to najväčšie z~nich (8) totiž nie je väčšie ako súčet zvyšných troch ($2+4+6=12$).
Súčet v~kruhu preto musí byť väčší alebo rovný súčtu všetkých predpísaných čísel, \tj. $2+4+6+8=20$.

Keby súčet v~kruhu bol 20, tak by predpísané čísla museli byť v~štyroch susediacich políčkach malého štvorca.
Keďže jedno z~týchto čísel je~6, muselo by byť jedno zo susediacich políčok vonkajšieho štvorca 6 alebo 3, a~to by tiež musel byť deliteľ druhého susediaceho políčka vnútorného štvorca.
Žiadne z~čísel 2, 4 a~8 však takého deliteľa nemá, preto táto možnosť nastať nemôže.

Súčet v~kruhu preto musí byť väčší alebo rovný~21.
Nasledujúce vyplnenie dokazuje, že najmenšie číslo, ktoré môže byť v~kruhu napísané, je~21.
\figure{z9-I-1a}%



\poznamka
Všimnite si, že číslo v~kruhu je vždy súčinom súčtov dvojíc protiľahlých čísel vonkajšieho štvorca.
Tento poznatok možno tiež využiť pri riešení úlohy.
}

{%%%%%   Z9-I-2
Skratka mohla ústiť do náučného chodníka buď v~úseku medzi $A$ a~$B$, alebo v~úseku medzi $B$ a~$C$.
\figure{z9-I-2a}%


Skratka bola dlhá 1\,500\,m a~s~jej využitím bola cesta z~$A$ do $B$ (podľa 3.~informácie) dlhá 1\,700\,m.
Vzdialenosť $B$ od ústia skratky preto bola v~oboch prípadoch rovnaká, a~to
$$
1\,700-1\,500=200\,(\text{m}).
$$

Z~1. a~2. informácie vyplýva, že cesta z~$A$ do $B$ po náučnom chodníku bola dlhá
$$
7\,700-5\,800=1\,900\,(\text{m}).
$$

Ak dĺžku úseku náučného chodníka medzi $B$ a~$C$ označíme~$y$, tak s~využitím 4.~informácie dostávame dve rôzne rovnice podľa toho, kam ústi skratka:
\begin{enumerate}\alphatrue
\item Pre ústie v~úseku medzi $A$ a~$B$ platí
$$
\aligned
1\,900+2y+200+1\,500&=8\,800,\\
2y&=5\,200,\\
y&=2\,600\,(\text{m}).
\endaligned
$$
V~tomto prípade bol celý náučný chodník dlhý
$$
1\,900+2\,600=4\,500\,(\text{m}).
$$
\item Pre ústie v~úseku medzi $B$ a~$C$ platí
$$
\aligned
1\,900+2y-200+1\,500&=8\,800,\\
2y&=5\,600,\\
y&=2\,800\,(\text{m}).
\endaligned
$$
V~tomto prípade bol celý náučný chodník dlhý
$$
1\,900+2\,800=4\,700\,(\text{m}).
$$
\end{enumerate}

\poznamka
Ak dĺžku úseku náučného chodníka medzi $A$ a~$B$ označíme~$x$, medzi $B$ a~ústím skratky~$z$ a~dĺžku červenej cesty medzi $A$ a~$C$ označíme~$w$, tak informácie zo zadania možno zapísať pomocou rovníc takto:
\begin{enumerate}
\item $w+y+x=7\,700$,
\item $y+w=5\,800$,
\item $1\,500+z=1\,700$,
\item $x+2y+z+1\,500=8\,800$, resp. $x+2y-z+1\,500=8\,800$.
\end{enumerate}
V~predchádzajúcom je uvedené postupné riešenie tejto sústavy pre neznáme $z$, $x$ a~$y$ (a~následné vyjadrenie súčtu $x+y$).
Z~druhej rovnice možno vyjadriť aj hodnotu neznámej~$w$.
}

{%%%%%   Z9-I-3
\napad
Aký je vzťah medzi polomerom loptičky, polomerom kružnice vyznačenej hladinou a~vzdialenosťou stredu loptičky od hladiny?

\riesenie
Nasledujúci obrázok znázorňuje rez loptičky, ktorý prechádza jej stredom (bod~$S$) a~je kolmý na hladinu (priamka~$AB$).
Bod~$C$ je pätou kolmice z~bodu~$S$ na hladinu a~bod~$D$ je najvyšším bodom loptičky nad hladinou.
\figure{z9-I-3}%


Zo zadania vieme, že $|AC|=4$\,cm a~$|CD|=2$\,cm.
Polomer loptičky $|SA|=|SD|$ označíme~$r$.
Podľa Pytagorovej vety v~pravouhlom trojuholníku $ACS$ dostávame:
$$
\aligned
r^2&=4^2+(r-2)^2,\\
r^2&=16+r^2-4r+4,\\
4r&=20,\\
r&=5.
\endaligned
$$
Júliina loptička mala priemer 10\,cm.
}

{%%%%%   Z9-I-4
Ak myslené päťciferné číslo označíme~$x$, tak v~prvých troch riadkoch boli napísané čísla
$x+\frac12x=\frac32x$, $x+\frac15x=\frac65x$ a~$x+\frac19x=\frac{10}9x$.
Súčet v~štvrtom riadku bol rovný
$$
\left( \frac32+\frac65+\frac{10}9 \right)x=\frac{343}{90}x.
$$
Tento výsledok má byť treťou mocninou istého prirodzeného čísla, takže je sám prirodzeným číslom.
Keďže čísla 343 a~90 sú nesúdeliteľné, musí byť $x$ násobkom~90.
Keďže 343 je treťou mocninou~7, musí byť $\frac1{90}{x}$ treťou mocninou nejakého prirodzeného čísla.

Najmenší násobok~90, ktorý je päťciferný, je $10\,080=90\cdot112$; preto $\frac1{90}{x}\ge 112$.
Najmenšou treťou mocninou nejakého prirodzeného čísla, ktorá je väčšia alebo rovná 112, je $125=5^3$; preto $\frac1{90}{x}=125$.

Najmenšie číslo, ktoré si Katka mohla myslieť, je $90\cdot125=11\,250$.

}

{%%%%%   Z9-I-5
\napad
Začnite kritickým tunelčekom.

\riesenie
Komôrky budeme označovať krúžkami, tunelčeky čiarami.
Začneme kritickým tunelčekom, ktorého zasypaním sa domček rozdelí na dve oddelené časti.
Ak komôrky na koncoch tohto tunelčeka označíme $A$ a~$B$, tak
každá komôrka patrí do práve jednej z~nasledujúcich dvoch skupín:
\begin{enumerate}\alphatrue
\item komôrka~$A$ a~všetky komôrky, do ktorých sa z~nej možno dostať bez použitia tunelčeka~$AB$,
\item komôrka~$B$ a~všetky komôrky, do ktorých sa z~nej možno dostať bez použitia tunelčeka~$BA$.
\end{enumerate}
To znamená, že žiadna komôrka z~jednej skupiny nie je spojená tunelčekom so žiadnou komôrkou z~druhej skupiny.
Teraz určíme, koľko najmenej komôrok môže byť v~jednej skupine, aby boli splnené ostatné podmienky:
\begin{itemize}
\item
Aby z~komôrky~$A$ viedli tri tunelčeky, musia byť v~skupine~a) aspoň dve ďalšie komôrky, ktoré označíme $C$ a~$D$.
Tri komôrky v~skupine však nestačia -- dajú sa spojiť jedine $C$ a~$D$, a~to by z~$C$ a~$D$ viedli iba dva tunelčeky.
\item
Preto v~skupine~a) musí byť aspoň jedna ďalšia komôrka, ktorú označíme~$E$.
Štyri komôrky však tiež nestačia -- $E$ sa dá spojiť jedine s~$C$ a~$D$, a~to by z~$E$ viedli iba dva tunelčeky.
\item
Preto v~skupine~a) musí byť aspoň jedna ďalšia komôrka, ktorú označíme~$F$.
Päť komôrok v~jednej skupine už stačí -- komôrky môžu byť pospájané napr. takto:
\figure{z9-I-5}%
\end{itemize}


Domček mal najmenej 10~komôrok.
}

{%%%%%   Z9-I-6
\napad
Rozdeľte si úlohu na etapy.

\riesenie
Štvorec sa postupne preklápa okolo bodov na úsečke~$AB$ (na nasledujúcich obrázkoch to sú body $K$, $M_1$, $R_2$, $A_3$, atď.).
V~každej etape sa bod~$R$ pohybuje po časti kružnice, ktorej stred je v~niektorom z~vyznačených bodov a~polomer je rovný buď strane, alebo uhlopriečke štvorca.
\figure{z9-I-6a}%


Časti kružníc sú väčšinou štvrťkružnice (čo zodpovedá veľkosti vnútorného uhla štvorca), iba v~krajných bodoch úsečky to sú trištvrtekružnice (čo zodpovedá veľkosti vonkajšieho uhla štvorca).

Pre narysovanie celej stopy bodu~$R$ potrebujeme stredy kružníc ($K=K_1$, $M_1=M_2$, $R_2=R_3$ atď.), ktoré sú na úsečke~$AB$ po 2\,cm.
Spoločné body kružníc ($R_1=K_2$, $R_2=R_3$, $K_3=R_4$ atď.) ležia v~mrežových bodoch štvorčekovej siete so stranou 2\,cm.
\figure{z9-I-6b}%
}

{%%%%%   Z4-II-1
...}

{%%%%%   Z4-II-2
...}

{%%%%%   Z4-II-3
...}

{%%%%% Z5-II-1
Otec štyrikrát prehral, takže musel strýkovi zaplatiť $4\cdot8=32$ eur.

Otec však vyhral toľkokrát, že aj po zaplatení týchto 32~eur získal 24~eur.
Jeho celková výhra bola $32+24=56$ eur, vyhral teda $56:8=7$ partií.

Otec sedemkrát vyhral, štyrikrát prehral a~päťkrát remizoval, so strýkom teda zohral $7+4+5=16$ partií.

\hodnotenie
2~body za určenie otcovej celkovej výhry;
2~body za počet otcových vyhraných partií;
2~body za počet všetkých zohraných partií.
\endhodnotenie
}

{%%%%% Z5-II-2
Veverička zjedla v~diétny deň jeden oriešok, v~normálny deň teda zjedla tri oriešky.
Oba typy dní sa pravidelne striedali, preto sa typ dňa, ktorým 19-denné obdobie začínalo,
opakoval celkom 10-krát, druhý typ sa opakoval 9-krát.

Keďže nevieme, či sledované obdobie začínalo diétnym, alebo normálnym dňom, musíme uvážiť obe možnosti:
\begin{enumerate}\alphatrue
\item ak sa začínalo diétnym dňom, zjedla veverička $10\cdot1+9\cdot3=37$ orieškov,
\item ak sa začínalo normálnym dňom, zjedla veverička $10\cdot3+9\cdot1=39$ orieškov.
\end{enumerate}

Veverička zjedla najmenej 37 a~najviac 39~orieškov.

\hodnotenie
1~bod za počet orieškov zjedených v~normálny deň;
po 2~bodoch za celkový počet zjedených orieškov pri každej z~možností;
1~bod za záver.
\endhodnotenie
}

{%%%%% Z5-II-3
Keby strany $AB$ a~$AC$ boli rovnako dlhé, kružnica so stredom v~bode~$A$ prechádzajúca bodom~$B$ by prechádzala aj bodom~$C$.
V~takom prípade by Ema zostrojila jedinú kružnicu so stredom v~bode~$A$.

Keby strany $AB$ a~$AC$ boli rôzne dlhé, kružnica so stredom v~bode~$A$ prechádzajúca jedným z~bodov $B$ a~$C$ by neprechádzala tým druhým.
V~takom prípade by Ema zostrojila dve kružnice so stredom v~bode~$A$.

Obdobné prípady môžu nastať aj pre kružnice so stredom v~bode~$C$.
Kružnice so stredom v~bode~$B$ budú určite dve, keďže zo zadania vieme, že strany $BA$ a~$BC$ sú rôzne dlhé.

Ak by strany trojuholníka $ABC$ boli navzájom rôzne, tak by Ema zostrojila $2+2+2=6$ kružníc.
Ak by strana~$AC$ bola zhodná s~jednou zo zvyšných dvoch strán, tak by Ema zostrojila $1+2+2=5$ kružníc.
Strana $AC$ preto musí byť dlhá buď 3\,cm, alebo 4\,cm.
\insp{z5-II-3.eps}%

\hodnotenie
Po 1~bode za každú z~dvoch vyhovujúcich možností;
3~body za zdôvodnenie správneho počtu kružníc;
1~bod za vysvetlenie, že iné možnosti nie sú.

Zdôvodnenie správneho počtu kružníc iba pri jednej z~dvoch vyhovujúcich možností považujte za postačujúce.
\endhodnotenie
}

{%%%%% Z6-II-1
Súčet vekov všetkých detí bol
$$
2+3+4+5+6+8=28,
$$
teda súčet vekov detí z~každej rodiny bol $28:2=14$.

Práve v~jednej rodine bolo najstaršie -- osemročné dieťa.
Môžeme teda uvažovať, ako vyjadriť súčet~14 pomocou uvedených čísel tak, aby jeden zo sčítancov bol~8.
To možno buď ako $14=8+6$, alebo $14=8+4+2$.
Mohli teda nastať nasledujúce dve možnosti:
\begin{enumerate}\alphatrue
\item deti z~jednej rodiny boli staré 8 a~6 rokov, deti z~druhej rodiny boli staré 5, 4, 3 a~2~roky,
\item deti z~jednej rodiny boli staré 8, 4 a~2 roky, deti z~druhej rodiny boli staré 6, 5, a~3~roky.
\end{enumerate}

\hodnotenie
2~body za určenie súčtu vekov detí z~každej rodiny;
po 2~bodoch za určenie každej z~vyhovujúcich možností.

Riešenia s~oboma možnosťami, ale bez akéhokoľvek zdôvodnenia hodnoťte nanajvýš 3~bodmi.
\endhodnotenie
}

{%%%%% Z6-II-2
Prvý deň Pat vykopal 40\,cm.

Druhý deň sa Mat dokopal do $3\cdot 40=120$\,(cm), vykopal teda $120-40=80$\,(cm).

Tretí deň vykopal Pat tiež 80\,cm, dokopal sa do $120+80=200$\,(cm).
V~tom okamihu bola jama o~50\,cm väčšia ako on, Pat teda meral $200-50=150$\,(cm).

\hodnotenie
2~body za hĺbku jamy druhý deň;
3~body za hĺbku jamy tretí deň;
1~bod za výšku Pata.
\endhodnotenie
}

{%%%%% Z6-II-3
Plocha obdĺžnikového pozemku bola 6~árov, \tj. 600\,m$^2$.
Jedna jej strana merala 20\,m, druhá preto merala $600:20=30$\,(m).
Strana jedného štvorcového pozemku bola dlhá 20\,m, jeho plocha bola $20\cdot 20=400$\,(m$^2$), \tj. 4~áre.
Strana druhého štvorcového pozemku bola dlhá 30\,m, jeho plocha bola $30\cdot 30=900$\,(m$^2$), \tj. 9~árov.

Všetky tri pozemky mali dokopy $6+4+9=19$~(árov).
Celkom záhradkári zaplatili 57€, zoranie jedného áru teda vyšlo na $57:19=3$€.

Záhradkár s~obdĺžnikovým pozemkom zaplatil $6\cdot 3=18$€,
záhradkár s~menším štvorcovým pozemkom zaplatil $4\cdot 3=12$€ a~záhradkár s~väčším štvorcovým pozemkom zaplatil $9\cdot 3=27$€.

\hodnotenie
1~bod za druhú stranu obdĺžnikového pozemku;
1~bod za plochy štvorcových pozemkov;
2~body za jednotkovú cenu orby;
2~body za platby jednotlivých záhradkárov.
\endhodnotenie
}

{%%%%% Z7-II-1
1.
Vo štvrtok po výbere 40~dukátov bol mešec prázdny, pred výberom v~ňom teda bolo oných 40~dukátov.

V~stredu po výbere (pred nočným zdvojnásobením) bolo v~mešci $40:2=20$ dukátov, pred výberom v~ňom teda bolo $20+40=60$ dukátov.

V~utorok po výbere bolo v~mešci $60:2=30$ dukátov, pred výberom v~ňom teda bolo $30+40=70$ dukátov.

V~pondelok Marienka do mešca vložila $70:2=35$ dukátov.

\smallskip
2.
Aby každý deň pred výberom bol počet dukátov v~mešci rovnaký, muselo by nočné zdvojnásobenie vyrovnať výber 40~dukátov.
To znamená, že týchto 40~dukátov by muselo byť práve polovicou počtu dukátov pred výberom,
teda pred výberom by tam muselo byť 80~dukátov.
Aby v~utorok pred výberom bolo v~mešci 80~dukátov, musela by Marienka v~pondelok do mešca vložiť 40~dukátov.

\ineriesenie
1.
Ak Marienkin pondelkový vklad označíme~$z$, tak
\begin{itemize}
\item v~utorok po výbere bolo v~mešci $2z-40$ dukátov,
\item v~stredu po výbere bolo v~mešci $4z-80-40=4z-120$ dukátov,
\item vo štvrtok po výbere bolo v~mešci $8z-240-40=8z-280$ dukátov.
\end{itemize}
Vo štvrtok po výbere bol mešec prázdny, tzn. $8z=280$, teda $z=35$.
V~pondelok Marienka do mešca vložila 35~dukátov.

\smallskip
2.
Aby každý deň pred výberom bol počet dukátov v~mešci rovnaký, musel by byť počet dukátov rovnaký aj po výbere, resp. pred nočným zdvojnásobením.
Porovnaním týchto hodnôt napr. v~pondelok a~v~utorok dostávame
$z=2z-40$, teda $z=40$.
V~pondelok mala Marienka do mešca vložiť 40~dukátov.

\hodnotenie
Po 3~bodoch za každú časť úlohy (z~toho 2~body za čiastočné kroky v~zvolenom postupe a~1~bod za výsledok).
\endhodnotenie
}

{%%%%% Z7-II-2
V~každom riadku platí, že číslo v~jeho pravom krajnom políčku je o~2~väčšie ako číslo v~jeho ľavom krajnom políčku.
Súčet čísel v~pravom krajnom stĺpci je teda o~${3\cdot2}=6$ väčší ako súčet čísel v~ľavom krajnom stĺpci.
Číslo v~dolnom políčku prostredného stĺpca je o~3 väčšie ako číslo v~jeho prostrednom políčku.
Tieto rozdiely sú v~mriežke zvýraznené inými odtieňmi sivej.
\insp{z7-II-2b.eps}%

Súčet čísel všetkých sivých políčok v~pravej časti mriežky je preto o~$6+3=9$ väčší ako súčet čísel všetkých sivých políčok v~jej ľavej časti.
Avšak súčet všetkých čísel v~pravej časti má byť o~100 väčší ako súčet všetkých čísel v~ľavej časti.
Preto musí byť v~jedinom nezvýraznenom políčku číslo $100-9=91$.
V~prostrednom políčku mriežky potom musí byť číslo $91+3=94$.

\hodnotenie
3~body za určenie vhodných podmnožín v~dvoch oddelených častiach mriežky a~stanovenie rozdielu súčtov v~týchto podmnožinách;
2~body za vypočítanie čísla v~neoznačenom políčku (alebo iného pomocného čísla);
1~bod za určenie čísla v~prostrednom políčku.
\endhodnotenie

\ineriesenie
Ak hľadané číslo v~prostrednom políčku označíme~$x$, tak čísla v~ostatných políčkach sú:
\insp{z7-II-2c.eps}%

Súčet všetkých čísel v~ľavej časti mriežky je $4x-3$, súčet všetkých čísel v~pravej časti je $5x+3$.
Pritom druhý výraz má byť o~100 väčší ako prvý.
Z~toho dostávame rovnicu
$$
4x-3+100=5x+3,
$$
ktorej riešením je $x=94$.
Číslo v~prostrednom políčku je 94.

\hodnotenie
3~body za zostavenie rovnice, pričom neznámou je jedno z~čísel v~mriežke;
3~body za vypočítanie čísla v~prostrednom políčku.
\endhodnotenie

\poznamka
Pri inom výbere neznámej dôjdeme k~inej rovnici s~obdobným riešením.
Ak napr. $y$ označuje číslo v~ľavom hornom políčku, tak dostaneme
$$
\begin{aligned}
\!y+(y+3)+(y+4)+(y+6)+100&=\!(y+1)+(y+2)+(y+5)+(y+7)+(y+8), \\
4y+113&=5y+23, \\
y&=90.
\end{aligned}
$$
Číslo v~prostrednom políčku je potom určené ako $y+4=94$.
}

{%%%%% Z7-II-3
Možné rozmery kvádrov určíme pomocou rozkladu zadaných objemov na súčin troch rôznych prirodzených čísel:
$$
\begin{gathered}
12 =1\cdot2\cdot6 =1\cdot3\cdot4, \\
30 =1\cdot2\cdot15 =1\cdot3\cdot10 =1\cdot5\cdot6 =2\cdot3\cdot5.
\end{gathered}
$$
V~týchto rozkladoch hľadáme spoločné dvojice čísel, ktoré predstavujú rozmery lepených stien.
Také dvojice sú práve tri a~zodpovedajú nasledujúcim možnostiam:
\begin{enumerate}\alphatrue
\item Pre spoločnú stenu $1\cm\times 2\cm$ je tretí rozmer nového kvádra $6+15=21$\,(cm).
\item Pre spoločnú stenu $1\cm\times 3\cm$ je tretí rozmer nového kvádra $4+10=14$\,(cm).
\item Pre spoločnú stenu $1\cm\times 6\cm$ je tretí rozmer nového kvádra $2+5=7$\,(cm).
\end{enumerate}

Nový kváder mohol mať rozmery $1\cm\times 2\cm\times21\cm$, $1\cm\times 3\cm\times14\cm$, alebo $1\cm\times 6\cm\times7\cm$.

\ineriesenie
Súčet objemov oboch kvádrov je 42\,cm$^3$.
Rovnako ako pri pôvodných kvádroch sú rozmery nového kvádra v~centimetroch vyjadrené celými číslami.
Na rozdiel od pôvodných kvádrov však tieto rozmery nemusia byť navzájom rôzne.
Možné rozmery nového kvádra určíme pomocou rozkladu jeho objemu na súčin troch prirodzených čísel:
$$
42 =1\cdot1\cdot42 =1\cdot2\cdot21 =1\cdot3\cdot14 =1\cdot6\cdot7 =2\cdot3\cdot7.
$$
Dvojice čísel v~týchto rozkladoch predstavujúce spoločnú stenu musia pozostávať z~rôznych čísel, ktorých súčin musí byť deliteľom čísla 12 (objem menšieho z~pôvodných kvádrov).
Také dvojice sú štyri, a~tie zodpovedajú nasledujúcim možnostiam:
\begin{enumerate}\alphatrue
\item Pre spoločnú stenu $1\cm\times 2\cm$ by tretí rozmer menšieho, resp. väčšieho kvádra bol $12:2=6$\,(cm), resp. $30:2=15$\,(cm).
\item Pre spoločnú stenu $1\cm\times 3\cm$ by tretí rozmer menšieho, resp. väčšieho kvádra bol $12:3=4$\,(cm), resp. $30:3=10$\,(cm).
\item Pre spoločnú stenu $1\cm\times 6\cm$ by tretí rozmer menšieho, resp. väčšieho kvádra bol $12:6=2$\,(cm), resp. $30:6=5$\,(cm).
\item Pre spoločnú stenu $2\cm\times 3\cm$ by tretí rozmer menšieho, resp. väčšieho kvádra bol $12:6=2$\,(cm), resp. $30:6=5$\,(cm).
\end{enumerate}
V~prípade d) by menší z~pôvodných kvádrov nemal rôzne dĺžky strán, ostatné možnosti vyhovujú všetkým požiadavkám.
Nový kváder mohol mať rozmery $1\cm\times 2\cm\times21\cm$, $1\cm\times 3\cm\times14\cm$, alebo $1\cm\times 6\cm\times7\cm$.

\hodnotenie
2~body za určenie všetkých rozkladov;
3~body za diskusiu a~výber vyhovujúcich možností;
1~bod za záver.
\endhodnotenie
}

{%%%%% Z8-II-1
Súčin dvoch krajných cifier je 40, a~to je možné iba ako $40=5\cdot 8$.
Rozdiel týchto cifier je rovný $8-5=3$.
Súčin dvoch vnútorných cifier je 18, a~to je možné buď ako $18=2\cdot 9$, alebo ako $18=3\cdot 6$.
V~prvom prípade je rozdiel $9-2=7$, čo je rôzne od rozdielu krajných cifier.
V~druhom prípade je rozdiel $6-3=3$, čo súhlasí s~rozdielom krajných cifier.
Z~prvých troch podmienok teda vyplýva, že krajné cifry sú 5 a~8, vnútorné cifry sú 3 a~6.
Také čísla sú štyri:
$$
5368,\quad 8635,\quad 8365,\quad 5638. \tag{1}
$$

Aby bol rozdiel mysleného čísla a~opačne napísaného čísla najväčší možný, musí byť na mieste tisícok, resp.
stoviek väčšia z~dvoch možných cifier.
Monika premýšľala o~čísle 8635.

\hodnotenie
1~bod za dvojicu krajných cifier;
1~bod za možné dvojice vnútorných cifier;
2~body za určenie správnej dvojice vnútorných cifier a~štyroch možností (1);
2~body za určenie vyhovujúcej možnosti.
\endhodnotenie

\poznamka
Medzi číslami (1) sú opačne napísané čísla v~prvej dvojici a~čísla v~druhej dvojici.
Záverečnú časť úlohy možno spraviť porovnaním štyroch možných rozdielov:
$$
\begin{aligned}
8635-5368&=3267,\quad 5368-8635=-3267,\\
8365-5638&=2727,\quad 5638-8365=-2727.
\end{aligned}
$$
}

{%%%%% Z8-II-2
Z~Matovho vysvetlenia vyplýva, že jeho výška je presne v~strede medzi hĺbkou celej jamy a~jej polovicou.
Teda 90\,cm je rovné trom štvrtinám hĺbky celej jamy.
Mat kopal jamu hlbokú $\frac43\cdot 90=120$\,(cm).

\hodnotenie
3~body za poznatok, že 90\,cm zodpovedá $\frac34$ hĺbky celej jamy;
3~body za výpočet hĺbky jamy.
\endhodnotenie

\ineriesenie
Ak označíme $j$ hĺbku celej jamy, tak podľa Matovho vysvetlenia môžeme vzdialenosť vrchu Matovej hlavy od povrchu zeme vyjadriť ako
$$
j-90=90-\frac12j. \tag{1}
$$
Z~toho dostávame $\frac32j=180$, teda $j=120$.
Mat kopal jamu hlbokú 120\,cm.

\hodnotenie
3~body za vyjadrenie informácií zo zadania pomocou neznámej $j$;
3~body za výpočet hĺbky jamy.
\endhodnotenie

\poznamka
Ak označíme $v$ vzdialenosť vrchu Matovej hlavy od povrchu zeme a~$j$ hĺbku celej jamy, tak Matovo vysvetlenie môžeme zapísať napr. ako
$$
90=\frac12j+v,\quad 90=j-v.
$$
Obvyklými úpravami možno z~tejto sústavy získať rovnicu ekvivalentnú s~(1).
}

{%%%%% Z8-II-3
Ramená uhla $ABC$ sú súmerné podľa jeho osi.
Preto bod~$A'$ leží na polpriamke $BC$, a~to tak, že $|BA'|=|BA|=5$\,cm.

Koncové body úsečky~$BC$ sú súmerné podľa jej stredu, preto $B'=C$.

Body $C$ a~$C'$ sú súmerné podľa osi úsečky~$AB$, preto je úsečka~$CC'$ kolmá na túto priamku.
Navyše je trojuholník $ABC$ pravouhlý s~pravým uhlom pri vrchole~$B$,
preto body $A$, $B$, $C$, $C'$ tvoria vrcholy obdĺžnika.
\insp{z8-II-3.eps}%

Trojuholník $A'B'C'$ je teda pravouhlý s~pravým uhlom pri vrchole~$B'$.
Jeho odvesny majú veľkosti
$$
\begin{aligned}
\vert A'B'\vert&=\vert BB'\vert-\vert BA'\vert=12-5=7\,(\Cm), \\
\vert B'C'\vert&=\vert AB\vert=5\,(\Cm).
\end{aligned}
$$
Obsah trojuholníka $A'B'C'$ je $\frac12\cdot7\cdot5=17{,}5\,(\Cm^2)$.

\hodnotenie
Po 1~bode za spresnenie polôh bodov $A'$, $B'$, $C'$;
2~body za poznatok, že trojuholník $A'B'C'$ je pravouhlý;
1~bod za veľkosti odvesien a~obsah.
\endhodnotenie
}

{%%%%% Z9-II-1
Podľa prvej podmienky vieme doplniť iba prostredné políčko v~prvom riadku, $4+8=12$,
a~prostredné políčko v treťom riadku, $10-12=\m2$.

Ďalšie čísla priamo doplniť nevieme, ale môžeme si pomôcť neznámou a~rovnicou.
Ak napr. číslo v~prvom políčku v~treťom riadku označíme~$x$, tak podľa prvej podmienky bude v~treťom políčku v~tom istom riadku $\m x-2$.
\insp{z9-II-1a.eps}%

Podľa druhej podmienky dostávame
$$
\begin{aligned}
4+10-x-2&=8+10+x, \\
2x&=-6, \\
x&=-3.
\end{aligned}
$$

Po dosadení vieme doplniť aj zvyšné čísla v~druhom riadku a~dostávame nasledujúce jednoznačné riešenie:
\insp{z9-II-1b.eps}%


\hodnotenie
Po 1~bode za doplnenie hodnôt 12 a~$\m2$;
3~body za zostavenie a~vyriešenie rovnice;
1~bod za doplnenie zvyšných čísel.

Riešenie pomocou rovnice nie je nevyhnutné, dá sa odhaliť napr. postupným skúšaním a~vysvetlením, že úloha viac riešení nemá.
Naopak, označením viacerých čísel z~prázdnych políčok neznámymi možno úlohu riešiť pomocou viacerých rovníc o~viacerých neznámych.
Navrhované hodnotenie prispôsobte žiackemu riešeniu vzhľadom na jeho úplnosť a~kvalitu komentára.
\endhodnotenie
}

{%%%%% Z9-II-2
Všetkých štvorciferných čísel obsahujúcich všetky uvedené cifry je 24:
\bgroup
\thinsize=0pt
\thicksize=0pt
%\def\ctr#1{\hfil\quad#1\quad\hfil}
$$
\begintable
1\,234|1\,243|1\,324|1\,342|1\,423|1\,432\cr
2\,134|2\,143|2\,314|2\,341|2\,413|2\,431\cr
3\,124|3\,142|3\,214|3\,241|3\,412|3\,421\cr
4\,123|4\,132|4\,213|4\,231|4\,312|4\,321\endtable
$$
\egroup
Medzi týmito 24 číslami sa na každom mieste opakuje každá zo 4 cifier práve 6-krát (${6\cdot4}=24$).
Súčet všetkých cifier ako na mieste jednotiek, tak na mieste desiatok, stoviek aj tisícok je rovný
$$
6\cdot(1+2+3+4)=60.
$$
Súčet všetkých uvedených čísel je preto rovný
$$
60+10\cdot60+100\cdot60+1\,000\cdot60=66\,660.
$$

Keďže Patovi pôvodne vyšlo 58\,126, musí byť súčet dvoch chýbajúcich sčítancov rovný
$$
66\,660-58\,126=8\,534.
$$
Keďže všetky čísla pozostávajú z~cifier menších ako 5, nedochádza pri sčítaní ktorýchkoľvek dvoch nikde k~prechodu cez desiatku.
Cifry na jednotlivých miestach čísla 8\,534 možno preto získať nasledovne:
\begin{itemize}
\item $8=4+4$,
\item $5=2+3$ (možnosť $1+4$ vylučujeme, keďže potom by jeden zo sčítancov mal na dvoch miestach 4),
\item $3=1+2$,
\item $4=1+3$ (možnosť $2+2$ vylučujeme, keďže potom by jeden zo sčítancov mal na dvoch miestach 2).
\end{itemize}
Číslo 8\,534 možno vyjadriť jedine ako súčet čísel 4\,213 a~4\,321.
A~to sú práve čísla, na ktoré Pat pôvodne zabudol.

\hodnotenie
3~body za určenie správneho súčtu 66\,660;
1~bod za určenie rozdielu 8\,534;
2~body za určenie pôvodne chýbajúcich čísel 4\,213 a~4\,321.

\poznamky
Počet všetkých štvorciferných čísel obsahujúcich štyri rôzne cifry je rovný
počtu všetkých permutácií štvorprvkovej množiny, a~tých je $4\cdot3\cdot2\cdot1=24$.

Celkový správny súčet možno odvodiť aj zoskupovaním vhodných sčítancov:
napr. súčet každého čísla s~číslom napísaným opačne je vždy 5\,555
(napr. $1\,234+4\,321=5\,555$) a~takých dvojíc je zrejme~12; súčet všetkých
uvažovaných čísel teda je $12\cdot5\,555=66\,660$.
\endhodnotenie
}

{%%%%% Z9-II-3
Počet čiernych potkanov na štarte označíme~$x$, počet bielych potkanov na štarte označíme~$y$.
Do cieľa tak došlo $0{,}56x$ čiernych potkanov a~$0{,}84y$ bielych potkanov a~podľa zadania je
$0{,}56x:0{,}84y=1:2$.
Potrebujeme zistiť pomer $x:y$.

Predchádzajúcu rovnosť môžeme zapísať ako
$$
\frac{0{,}56x}{0{,}84y}=\frac12,
$$
čo je ekvivalentné s
$$
\frac{x}{y}=\frac12\cdot\frac{84}{56}=\frac{42}{56}=\frac34.
$$
Pomer čiernych a~bielych potkanov na štarte bol $3:4$.

\hodnotenie
3~body za odvodenie úvodnej rovnosti alebo podobného vzťahu;
3~body za vyjadrenie pomeru $x:y$.
\endhodnotenie
}

{%%%%% Z9-II-4
Pri každom preklopení opisuje bod~$Q$ štvrťkružnicu so stredom v~niektorom z~vrcholov štvorca a~s~polomerom, ktorý sa postupne zmenšuje o~dĺžku strany štvorca.
Aby bod~$Q$ po piatich preklopeniach splynul s~niektorým vrcholom štvorca, musí byť úsečka~$MQ$ päťnásobkom strany štvorca.
\insp{z9-II-4a.eps}%


Dĺžky štvrťkružníc sú v~rovnakých pomeroch ako ich polomery.
Pritom polomery všetkých štvrťkružníc sú celočíselnými násobkami polomeru najmenšej (piatej) štvrťkružnice.
Ak jej dĺžku označíme~$d$, tak súčet dĺžok všetkých piatich štvrťkružníc je
$$
d+2d+3d+4d+5d=15d, \tag{1}
$$
čo je trojnásobok dĺžky najväčšej (prvej) štvrťkružnice.
Zo zadania vieme, že prvá štvrťkružnice je dlhá 5\,cm.
Súčet (1), teda dĺžka stopy opísanej bodom~$Q$, je
$$
3\cdot5=15\,(\Cm).
$$

\hodnotenie
2~body za určenie $MQ$ ako päťnásobku strany štvorca;
2~body za vyjadrenie súčtu~(1);
2~body za doriešenie a~vyjadrenie v~cm.

\poznamka
Vyjadrenie $d$ pomocou dĺžky strany štvorca, ozn. $a$, je $d=\frac12\pi a$.
Súčet (1) potom môže byť napísaný takto:
$$
\frac12\pi a\cdot(1+2+3+4+5)=
\frac{15}2\pi a=3\cdot\frac52\pi a.
$$
Na~určenie súčtu v~cm nepotrebujeme poznať ani $a$, ani $d$.
\endhodnotenie
}

{%%%%%   Z9-III-1
Informácie zo zadania môžeme pomocou jednej neznámej zapísať napr. takto:
$$
\begintable
\|dnes|skôr\crthick
Michaela\|$2x$|$84-2x$\cr
Jana\|$84-2x$|$x$\endtable
$$
Rozdiel vekov je v~oboch riadkoch rovnaký (rozdiel medzi \uv{dnes}
a~\uv{skôr}), čo vedie na rovnicu
$$
2x-84+2x =84-2x-x,
$$
ktorej riešenie je
$$
\aligned
7x &=168, \\
x &=24.
\endaligned
$$
Z~toho dostávame $2x=48$ a~$84-2x=36$, tzn. Michaela má 48~rokov a~Jana 36~rokov.

\ineriesenie
Vzťahy zo zadania je možné s~viacerými neznámymi vyjadriť napr. takto:
$$
\begintable
\|dnes|skôr\crthick
Michaela\|$2x$|$y$\cr
Jana\|$y$|$x$\endtable
$$
Michaela a~Jana majú dokopy 84~rokov a~rozdiel vekov je v~oboch riadkoch rovnaký.
To vedie na sústavu dvoch rovníc
$$
\aligned
2x+y &=84, \\
2x-y &=y-x,
\endaligned
$$
ktorá je ekvivalentná sústave
$$
\aligned
4x+2y &=168, \\
3x&=2y.
\endaligned
$$
Dosadením druhej rovnice do prvej dostávame
$$
\postdisplaypenalty 10000
\aligned
7x &=168, \\
x &=24.
\endaligned
$$
Z~toho vyplýva $2x=48$ a~$y=\frac32\cdot24=36$, to znamená, že Michaela má 48~rokov a~Jana 36~rokov.

\hodnotenie
2~body za vyjadrenie vzťahov obsiahnutých v~úvodnej tabuľke;
2~body za zostavenie a~vyriešenie rovnice, resp. sústavy rovníc;
po 1~bode za dopočítanie vekov Michaely a~Jany.

\poznamka
Za neznámu môže byť zvolený aj rozdiel medzi \uv{dnes} a~\uv{skôr} alebo
rozdiel medzi vekmi Michaely a~Jany; v~takom prípade by úvodné vzťahy spolu
s~ostatnými podmienkami viedli na inú sústavu rovníc s~obdobným riešením,
teda aj hodnotením.

Zo zadania vyplýva, že Michaela je staršia ako Jana a~že Michaelin vek je párne
číslo. Súčet ich vekov je 84~rokov, takže namiesto uvedeného riešenia rovníc
je možné postupne skúšať, ktorá z~nasledujúcich možností vyhovuje
ostatným požiadavkám zo zadania:
$$
\begintable
Michaela\|44|46|48|\dots\cr
Jana\|40|38|36|\dots\endtable
$$
Takto možno ľahko odhaliť, že vyhovujúca je tretia možnosť.
Bez zdôvodnenia, že sa jedná o~jedinú možnosť, hodnoťte také riešenie nanajvýš 4~bodmi.
Za zdôvodnenie jednoznačnosti dajte ďalšie 2~body podľa kvality komentára.
\endhodnotenie
}

{%%%%%   Z9-III-2
Rozlíšime tri prípady podľa toho, kde mohla sedieť Zlatovláska:

\noindent
1. Ak by Zlatovláska sedela v~prvom kresle, tak by
\begin{itemize}
\item princezná v~prvom kresle hovorila pravdu,
\item princezná v~druhom kresle hovorila pravdu,
\item princezná v~treťom kresle klamala.
\end{itemize}
\noindent
2. Ak by Zlatovláska sedela v~druhom kresle, tak by
\begin{itemize}
\item princezná v~prvom kresle hovorila pravdu,
\item princezná v~druhom kresle klamala,
\item princezná v~treťom kresle klamala.
\end{itemize}
\noindent
3. Ak by Zlatovláska sedela v~treťom kresle, tak by
\begin{itemize}
\item princezná v~prvom kresle klamala,
\item princezná v~druhom kresle hovorila pravdu,
\item princezná v~treťom kresle hovorila pravdu.
\end{itemize}

V~prvom a~v~treťom prípade by klamala jedna princezná, v~druhom prípade
by klamali dve princezné. Keďže muškina rada Janovi pomohla odhaliť
Zlatovlásku, musela mu muška prezradiť, že klamali dve princezné. Tomu
zodpovedá druhý prípad, čiže Zlatovláska sedela v~druhom kresle.
(Ak by muška tvrdila, že klamala jedna princezná, Jano by nedokázal
rozhodnúť medzi prvým a~tretím prípadom.)

\hodnotenie
4~body za rozbor všetkých prípadov;
2~body za určenie jediného možného prípadu a~umiestnenie Zlatovlásky.
\endhodnotenie

\poznamka
Predchádzajúci rozbor možností možno pojať opačne, \tj. podľa pravdivosti, resp. nepravdivosti jednotlivých výrokov vyvodzovať, kde by mala sedieť Zlatovláska. Takto by sa muselo uvažovať celkom osem prípadov:
\begin{itemize}
\item Výroky princezien v~prvom a~v~treťom kresle sú navzájom v~rozpore, teda tieto dve princezné nemôžu obe súčasne vravieť pravdu -- tým sú vylúčené dva prípady.
\item Z~rovnakého dôvodu nemôžu princezné v~prvom a~v~treťom kresle obe súčasne klamať~-- tým sú vylúčené ďalšie dva prípady.
\item Nie je možné, aby súčasne princezná v~druhom kresle klamala a~princezná v~treťom kresle hovorila pravdu (to by boli obe Zlatovlásky) -- tým je vylúčený ďalší prípad.
\end{itemize}
Zvyšné tri prípady zodpovedajú práve prípadom v~uvedenom riešení.
}

{%%%%%   Z9-III-3
Označme počet obrancov $p$ a~uvažujme odzadu.
Posledný $p$-ty obranca si vzal $p$~dukátov, a~tým boli rozobrané
všetky dukáty. Predposledný $(p-1)$-ty obranca si vzal $p-1$ dukátov
a~sedminu aktuálneho zvyšku, ktorý označme~$z$.
Keďže obaja obrancovia dostali rovnako, platí
$$
p-1+\frac17\cdot z=p. \tag{1}
$$
Z~toho dostávame $\frac17\cdot z=1$, teda $z=7$. Súčasne však platí, že
posledný obranca mal k~dispozícii šesť sedmín tohto zvyšku. To znamená,
že
$$
p=\frac 67\cdot z, \tag{2}
$$
teda $p=6$.
Ak sa obrancovia rozdelili o~všetky dukáty a~všetci dostali rovnako,
muselo ich byť šesť, každý dostal šesť dukátov, všetkých dukátov preto
bolo $6\cdot6=36$. Keďže sme doposiaľ uvažovali len niektoré informácie
zo zadania, je nutné overiť, že také delenie je naozaj možné:
% \newpage
$$
\begintable
\|vzal|zostalo\crthick
prvý\|\hfill$1+\frac17(36-1)=6$|\hfill$36-6=30$\cr
druhý\|\hfill$2+\frac17(30-2)=6$|\hfill$30-6=24$\cr
tretí\|\hfill$3+\frac17(24-3)=6$|\hfill$24-6=18$\cr
štvrtý\|\hfill$4+\frac17(18-4)=6$|\hfill$18-6=12$\cr
piaty\|\hfill$5+\frac17(12-5)=6$|\hfill$12-6=6$ \cr
šiesty\|\hfill$6+\frac17(6-6)=6$|\hfill$6-6=0$
\endtable
$$
O~odmenu sa delilo šesť obrancov.

\hodnotenie
3~body za vzťah (1) a~určenie $z=7$;
2~body za vzťah (2) a~výsledok $p=6$;
1~bod za overenie.
\endhodnotenie

\ineriesenie
Označme počet všetkých dukátov $l$ a~uvažujme odpredu.
Prvý obranca si vzal jeden dukát a~sedminu zvyšku, teda si vzal
$$
1+\frac17(l-1)=\frac17(l+6) \tag{3}
$$
dukátov a~počet dukátov sa tak zmenšil na $\frac67(l-1)$.
Druhý obranca si vzal dva dukáty a~sedminu nového zvyšku, teda si vzal
$$
2+\frac17\Big(\frac67(l-1)-2\Big)=\frac1{49}(6l+78) \tag{4}
$$
dukátov.
Keďže obaja obrancovia dostali rovnako, platí
$$
\frac17(l+6)=\frac1{49}(6l+78). \tag{5}
$$
Z~toho dostávame
$$
\aligned
7l+42 &=6l+78, \\
l&=36.
\endaligned
$$
Rozpísaním ako v~tabuľke pri predchádzajúcom postupe sa overí, že sme našli vyhovujúce riešenie a~že obrancov bolo šesť.

\hodnotenie
Po 1~bode za vzťah (3), resp. (4) a~jeho prípadnú úpravu;
2~body za zostavenie a~vyriešenie rovnice (5);
2~body za overenie a~výsledok.
\endhodnotenie

\ineriesenie
Označme počet obrancov $p$.
Posledný $p$-ty obranca si vzal $p$ dukátov, a~tým boli rozobrané
všetky dukáty.
Všetci obrancovia dostali rovnako, každý preto dostal $p$~dukátov.
Obrancovia sa teda delili celkom o~$p^2$ dukátov.

Prvý obranca si vzal jeden dukát a~sedminu zvyšku, takže platí
$$
1+\frac17(p^2-1)=p, \tag{6}
$$
čiže
$$
p^2-1=7(p-1).
$$
Ľavú stranu v~tejto rovnici možno vyjadriť ako $p^2-1=(p+1)(p-1)$.
Zo zadania vyplýva, že $p>1$, teda $p-1>0$ a~predchádzajúca rovnica
je ekvivalentná s~rovnicou
$$
p+1=7,
$$
čiže $p=6$.
Rozpísaním ako v~tabuľke pri prvom postupe sa overí,
že sme našli vyhovujúce riešenie.
O~odmenu sa delilo šesť obrancov.

\hodnotenie
2~body za vyjadrenie počtu všetkých dukátov pomocou~$p$;
3~body za zostavenie a~vyriešenie rovnice~(6);
1~bod za overenie.

S~poznatkom o~počte všetkých dukátov možno skúšaním zistiť, že
najmenšie~$p$, pre ktoré sa dajú dukáty rozdeliť podľa uvedených pravidiel,
je $p=6$. Také riešenie bez zdôvodnenia, že sa jedná o~jedinú možnosť,
hodnoťte nanajvýš 4~bodmi. Za zdôvodnenie jednoznačnosti dajte ďalšie
2~body podľa kvality komentára.
\endhodnotenie

\ineriesenie
Označme počet všetkých dukátov~$l$.
Prvý obranca si vzal jeden dukát a~sedminu zvyšku, takže onen zvyšok $l-1$ musel byť deliteľný siedmimi.
Číslo~$l$ je preto tvaru
$$
l=7k+1 \tag{7}
$$
pre nejaké kladné celočíselné~$k$.
Prvý obranca si teda vzal
$$
1+\frac17(l-1)=1+k
$$
dukátov.
Všetci obrancovia dostali rovnako a~posledný dostal práve toľko dukátov, aké bolo jeho poradové číslo.
To znamená, že obrancov bolo $k+1$ a~každý dostal $k+1$ dukátov, celkom sa tak delili
o~$$
l=(k+1)^2=k^2+2k+1 \tag{8}
$$
dukátov.
Porovnaním (7) a~(8) dostávame rovnicu
$$
7k+1=k^2+2k+1, \tag{9}
$$
čiže $5k=k^2$.
Zo zadania vyplýva, že obrancovia boli aspoň dvaja, teda $k>0$ a~predchádzajúca rovnica je ekvivalentná s~$k=5$.
Rozpísaním ako v~tabuľke pri prvom postupe sa overí, že sme našli vyhovujúce riešenie.
O~odmenu sa delilo šesť obrancov.

\hodnotenie
1~bod za vzťah~(7);
2~body za vzťah~(8);
2~body za zostavenie a~vyriešenie rovnice~(9);
1~bod za overenie.

S~poznatkom~(7) možno skúšaním zistiť, že najmenšie~$k$, pre ktoré sa dajú dukáty rozdeliť podľa uvedených pravidiel, je $k=5$.
Také riešenie bez zdôvodnenia, že sa jedná o~jedinú možnosť, hodnoťte nanajvýš 4~bodmi.
Za zdôvodnenie jednoznačnosti dajte ďalšie 2~body podľa kvality komentára.
\endhodnotenie
}

{%%%%%   Z9-III-4
Stred kružnice opísanej trojuholníku $ABC$ označme~$S$.
Vzdialenosť stredu~$S$ od každého z~bodov $A$, $B$, $C$ je rovná hľadanému polomeru, ktorý označme~$r$.
Trojuholníky $ACS$, $BCS$ a~$ABS$ sú teda rovnoramenné.
Najskôr ukážeme, že trojuholník $ABS$ je pravouhlý s~pravým uhlom pri vrchole~$S$.

Stred~$S$ leží na osi základne~$AB$, ktorá je zároveň osou súmernosti trojuholníka $ABC$.
Rovnoramenné trojuholníky $ACS$ a~$BCS$ sú preto zhodné.
Potom aj uhly $SAC$, $ACS$, $SCB$ a~$CBS$ sú navzájom zhodné s~veľkosťami
$$
\alpha=\frac12\cdot 45\st=22\st30'. \tag{1}
$$
Súčet veľkostí vnútorných uhlov $SAC$ a~$ACS$ v~trojuholníku $ACS$ je rovný $45\st$, a~to je tiež veľkosť vonkajšieho uhla pri~vrchole $S$ (na obr. označeného $\omega$).
Tento uhol je polovicou uhla $ASB$, preto je uhol $ASB$ pravý.
\insp{z9-III-4.eps}%


V~pravouhlom trojuholníku $ABS$ majú obe odvesny dĺžku~$r$ a~prepona~$AB$ je dlhá 6\,cm.
Podľa Pytagorovej vety platí
$$
r^2+r^2=6^2,
$$
teda $r=\sqrt{18}=3\sqrt2$\,(cm).
Kružnica opísaná trojuholníku $ABC$ má polomer $3\sqrt2$\,cm.

\hodnotenie
1~bod za poznatok o~zhodnosti trojuholníkov $ACS$ a~$BCS$, resp. uhlov $SAC$, $ACS$, $SCB$ a~$CBS$;
3~body za odvodenie, že uhol $ASB$ je pravý;
2~body za výpočet polomeru
(za správnu odpoveď považujte ktorúkoľvek z~vyššie uvedených hodnôt).

Prvé 4~body dajte aj~v~prípade, že veľkosť uhla $ASB$ je odvodená s~odkazom na vetu o~stredovom a~obvodovom uhle.

Za použitie Pytagorovej vety v~trojuholníku $ABS$ bez zdôvodnenia, prečo je tento trojuholník pravouhlý, dajte nanajvýš 1~bod.
\endhodnotenie

\poznamky
Po výpočte (1) sa možno k~veľkosti uhla $ASB$ dopočítať rozličnými spôsobmi.
Napr. určením veľkostí vnútorných uhlov pri základni v~rovnoramennom trojuholníku $ABC$,
$$
\frac12(180\st-45\st)=67\st30',
$$
potom vyjadrením veľkostí vnútorných uhlov pri základni v~rovnoramennom trojuholníku $ABS$,
$$
\beta=67\st30'-22\st30'=45\st,
$$
odkiaľ je veľkosť uhla $ASB$ vypočítaná ako
$$
\delta=180\st-2\cdot45\st=90\st.
$$

Iné odvodenie môže byť založené na určení veľkostí zvyšných vnútorných uhlov
v~zhodných rovnoramenných trojuholníkoch $ACS$ a~$BCS$,
$$
\gamma=180\st-2\cdot22\st30'=135\st,
$$
odkiaľ je veľkosť uhla $ASB$ vypočítaná ako
$$
\delta=360\st-2\cdot135\st=90\st.
$$
\insp{z9-III-4a.eps}%
}


