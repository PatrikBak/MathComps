{%%%%% A-I-1
Nájdite všetky prvočísla $p$, pre ktoré existuje prirodzené číslo~$n$ také,
že $p^n+1$ je treťou mocninou niektorého prirodzeného čísla.}
\podpis{Ján Mazák, Róbert Tóth}

{%%%%% A-I-2
Máme $n^2$ prázdnych škatúľ; každá z~nich má štvorcové dno.
Výška aj šírka každej škatule je prirodzené číslo z~množiny $\{1,2, \dots, n\}$.
Každé dve škatule sa líšia aspoň v~jednom z~týchto
dvoch rozmerov. Jednu škatuľu je dovolené vložiť do druhej, ak má oba
rozmery menšie a~aspoň jeden z~rozmerov má aspoň o~2 menší.
Takto môžeme vytvoriť postupnosť škatúľ vložených navzájom do seba
(\tj.~prvá škatuľa je vnútri druhej, druhá škatuľa je vnútri tretej atď.).
Každú takúto sadu uložíme na inú poličku. Určte najmenší možný počet
poličiek potrebný na uskladnenie všetkých $n^2$ škatúľ.}
\podpis{Peter Novotný}

{%%%%% A-I-3
Daný je ostrouhlý trojuholník $ABC$ s~výškami $AK$, $BL$, $CM$.
Dokážte, že trojuholník $ABC$ je rovnoramenný práve vtedy, keď platí
rovnosť
$$
|AM|+|BK|+|CL| = |AL|+|BM|+|CK|.
$$}
\podpis{Jaromír Šimša}

{%%%%% A-I-4
Nájdite všetky funkcie $f \colon \Bbb N \to \Bbb N$, ktoré majú
pre každé prirodzené číslo $m$ nasledujúcu vlastnosť: ak označíme
$d_1, d_2, \dots, d_n$ všetky delitele čísla~$m$, platí
$$
f (d_1) \cdot f (d_2) \cdot \ldots \cdot f (d_n) = m.
$$}
\podpis{Pavel Calábek}

{%%%%% A-I-5
Vnútri základne~$AB$ rovnoramenného trojuholníka $ABC$ leží bod~$D$.
Zvoľme bod~$E$ tak, aby $ADEC$ bol rovnobežník. Na polpriamke opačnej k~$ED$ leží
bod~$F$ taký, že $|EB| = |EF|$. Dokážte, že dĺžka tetivy, ktorú
vytína priamka~$BE$ na~kružnici opísanej trojuholníku $ABF$, je
dvojnásobkom dĺžky úsečky~$AC$.}
\podpis{Jan Kuchařík, Patrik Bak}

{%%%%% A-I-6
Vyriešte v~obore reálnych čísel sústavu rovníc
$$
\align
k-x^2=&y, \\
k-y^2=&z, \\
k-z^2=&u, \\
k-u^2=&x
\endalign
$$
s~reálnym parametrom~$k$ z~intervalu $\langle 0, 1 \rangle$.}
\podpis{Jaroslav Švrček}

{%%%%% B-I-1
Každému vrcholu pravidelného 66-uholníka priradíme jedno z~čísel~1 alebo~${-1}$.
Ku každej úsečke spájajúcej dva jeho vrcholy (strane či uhlopriečke) potom
pripíšeme súčin čísel v~jej krajných bodoch a~všetky čísla pri jednotlivých
úsečkách sčítame. Určte najmenšiu možnú a~najmenšiu nezápornú hodnotu takéhoto
súčtu.}
\podpis{Pavel Calábek}

{%%%%% B-I-2
Určte všetky dvojice $(a, b)$ reálnych parametrov, pre ktoré má sústava rovníc
$$
\align
|x| + y &= a,\\
2|y| - x &= b
\endalign
$$
práve tri riešenia v~obore reálnych čísel, a~pre každú z~nich tieto riešenia určte.}
\podpis{Jaroslav Švrček}

{%%%%% B-I-3
Na kružnici~$k$ sú zvolené body $A$, $B$, $C$, $D$, $E$ (v~tomto poradí) tak, že
platí $|AB| = |CD| = |DE|$. Dokážte, že ťažiská trojuholníkov $ABD$, $ACD$ a~$BDE$
ležia na kružnici sústrednej s~kružnicou~$k$.}
\podpis{Tomáš Jurík}

{%%%%% B-I-4
Nájdite všetky osemciferné čísla ${*}2{*}0{*}1{*}6$ so štyrmi neznámymi {\it nepárnymi\/} ciframi
vyznačenými hviezdičkami, ktoré sú deliteľné číslom~2\,016.}
\podpis{Jaromír Šimša}

{%%%%% B-I-5
Daný je pravouhlý trojuholník $ABC$ s~preponou~$AB$. Označme $D$ pätu jeho
výšky z~vrcholu~$C$ a~$M$, $N$ priesečníky osí uhlov $ADC$, $BDC$ so stranami $AC$, $BC$.
Dokážte, že platí
$$
2|AM|\cdot|BN| = |MN|^2.
$$}
\podpis{Jaroslav Švrček}

{%%%%% B-I-6
Určte všetky reálne čísla $r$ také, že nerovnosť $a^3 + ab + b^3\ge a^2 + b^2$
platí pre všetky dvojice reálnych čísel $a$, $b$, ktoré sú väčšie alebo rovné~$r$.}
\podpis{Ján Mazák}

{%%%%% C-I-1
Dokážte, že pre ľubovoľné reálne číslo $a$ platí nerovnosť
$$
a^2+\frac{1}{a^2-a+1}\geq a+1.
$$
Určte, kedy nastáva rovnosť.}
\podpis{Jaroslav Švrček}

{%%%%% C-I-2
Nájdite najväčšie prirodzené číslo~$d$, ktoré má tú vlastnosť, že pre
ľubovoľné prirodzené číslo~$n$ je hodnota výrazu
$$
V(n)=n^{4}+11n^{2}-12
$$
deliteľná číslom~$d$.}
\podpis{Aleš Kobza}

{%%%%% C-I-3
Päta výšky z~vrcholu $C$ v~trojuholníku $ABC$ delí stranu~$AB$
v~pomere $1:2$. Dokážte, že pri zvyčajnom označení
dĺžok strán trojuholníka $ABC$ platí nerovnosť
$$
3|a-b|<c.
$$}
\podpis{Jaroslav Švrček}

{%%%%% C-I-4
Nájdite všetky trojčleny $P(x)=ax^2+bx+c$ s~celočíselnými
koeficientmi $a$, $b$, $c$, pre ktoré platí $P(1)<P(2)<P(3)$ a~zároveň
$$\bigl(P(1)\bigr)^2+\bigl(P(2)\bigr)^2+\bigl(P(3)\bigr)^2=22.$$}
\podpis{Tomáš Jurík}

{%%%%% C-I-5
V~danom trojuholníku $ABC$ zvoľme vnútri strany~$AC$ body $K$, $M$
a~vnútri strany~$BC$ body $L$, $N$ tak, že
$$
|AK|=|KM|=|MC|, \quad |BL|=|LN|=|NC|.
$$
Ďalej označme $E$ priesečník uhlopriečok lichobežníka $ABLK$,
$F$ priesečník uhlopriečok lichobežníka $KLNM$
a~$G$ priesečník uhlopriečok lichobežníka $ABNM$.
Dokážte, že body~$E$, $F$ a~$G$ ležia na ťažnici trojuholníka~$ABC$ z~vrcholu~$C$
a~určte pomer ${|GF|:|EF|}$.}
\podpis{Šárka Gergelitsová}

{%%%%% C-I-6
\ite a) Marienka rozmiestni do vrcholov pravidelného osemuholníka rôzne počty
od jedného po osem cukríkov. Peter si potom môže vybrať, ktoré tri kôpky
cukríkov dá Marienke, ostatné si ponechá. Jedinou podmienkou je, že tieto tri
kôpky ležia vo vrcholoch rovnoramenného trojuholníka. Marienka chce rozmiestniť cukríky tak, aby ich dostala čo najviac, nech už Peter
trojicu vrcholov vyberie akokoľvek. Koľko ich tak Marienka zaručene získa?
\ite b) Rovnakú úlohu vyriešte aj pre pravidelný deväťuholník, do ktorého
vrcholov rozmiestni Marienka 1 až~9~cukríkov. 

(Medzi rovnoramenné trojuholníky zaraďujeme aj trojuholníky rovnostranné.)}
\podpis{Jaromír Šimša}

{%%%%% A-S-1
Zistite, aké najmenšie kladné celé číslo možno vložiť medzi dvojčíslia
$20$ a~$16$ tak, aby výsledné číslo bolo násobkom čísla $2016$.}
\podpis{Radek Horenský}

{%%%%% A-S-2
Nájdite všetky kladné celé čísla~$n$, pre ktoré sa dajú čísla $1, 2, \dots, n$ rozdeliť
do troch disjunktných neprázdnych množín s~navzájom rôznymi počtami prvkov tak,
že v~ľubovoľnej dvojici množín má tá s~menším počtom prvkov väčší
súčet svojich prvkov.}
\podpis{Martin Panák}

{%%%%% A-S-3
Body $D$ a~$E$ sú (v~tomto poradí) pätami výšok z~vrcholov $B$ a~$C$
ostrouhlého trojuholníka $ABC$.
Predpokladajme, že platí $|AE| \cdot |AD| = |BE| \cdot |CD|$. Akú
najmenšiu veľkosť môže mať uhol $BAC$?}
\podpis{Patrik Bak}

{%%%%% A-II-1
Nájdite všetky trojice celých čísel $(a, b, c)$ také, že každý zo zlomkov
$$
\frac {a}{b+c}, \quad \frac {b}{c+a}, \quad \frac {c}{a+b}
$$
má celočíselnú hodnotu.
}
\podpis{Jaroslav Švrček}

{%%%%% A-II-2
Daná je kružnica~$p$ so stredom~$K$ prechádzajúca bodom~$M$
a~polkružnica~$q$ nad priemerom~$KM$. Ľubovoľným bodom~$L$ vnútri úsečky~$KM$ vedieme kolmicu na~$KM$.
Tá pretne polkružnicu~$q$ v~bode~$Q$ a~kružnicu~$p$ v~bodoch $P_1$,
$P_2$ tak, že~${|P_1Q| > |P_2Q|}$. Priamka~$MQ$ pretína kružnicu~$p$ ešte
v~bode~$R\ne M$.
Dokážte, že pre obsahy $S_1$ a~$S_2$ trojuholníkov $MP_1Q$ a~$P_2RQ$
platí $1 <S_1: S_2 <3+ \sqrt8$.}
\podpis{Šárka Gergelitsová}

{%%%%% A-II-3
V~závislosti od reálneho parametra~$k$ určte počet riešení sústavy rovníc
%V~obore reálnych čísel vyriešte sústavu rovníc
$$
\aligned
x^2+kxy+y^2&= z, \\
y^2+kyz+z^2&= x, \\
z^2+kzx+x^2&= y
\endaligned
$$
v~obore reálnych čísel.
%s~reálnym parametrom~$k$.
}
\podpis{Patrik Bak}

{%%%%% A-II-4
Daný je ostrouhlý trojuholník $ABC$ s~výškou~$AD$. Osi uhlov $BAD$, $CAD$
pretínajú stranu~$BC$ postupne v~bodoch $E$, $F$. Kružnica opísaná trojuholníku $AEF$
pretína strany $AB$, $AC$ postupne v~bodoch $G$, $H$. Dokážte, že priamky $EH$, $FG$
a~$AD$ sa pretínajú v~jednom bode.}
\podpis{Patrik Bak}

{%%%%% A-III-1
Na kôpke leží 100 diamantov, z~ktorých 50 je pravých a~50~falošných.
Pozvali sme svojrázneho znalca, ktorý jediný dokáže rozpoznať, ktoré sú
ktoré. Zakaždým, keď mu ukážeme nejakú trojicu diamantov, dva vyberie a~(pravdivo) povie, koľko
z~nich je pravých. Rozhodnite, či môžeme zaručene odhaliť všetky pravé
diamanty bez ohľadu na to, ako znalec volí posudzované dvojice.}
\podpis{Michal Rolínek, Josef Tkadlec}

{%%%%% A-III-2
Nájdite všetky dvojice reálnych čísel $k$, $l$ také, že nerovnosť
$$
ka^2 + lb^2 > c^2
$$
platí pre dĺžky strán $a$, $b$, $c$ ľubovoľného trojuholníka.}
\podpis{Patrik Bak}

{%%%%% A-III-3
Nájdite všetky funkcie $f \colon \Bbb R \to \Bbb R$ také, že pre všetky
reálne čísla $x$, $y$ platí
$$
f(y-xy) = f(x) y+(x-1)^2f (y).
$$
}
\podpis{Pavel Calábek}

{%%%%% A-III-4
Každej postupnosti zloženej z~$n$ núl a~$n$ jednotiek priradíme číslo,
ktoré je počtom
%maximálnych
úsekov rovnakých cifier v~nej.
(Napríklad postupnosť $00111001$ má 4~také úseky $00$, $111$, $00$,~$1$.) Pre
dané $n$ sčítame všetky čísla priradené jednotlivým takým
postupnostiam. Dokážte, že výsledný súčet je rovný
$$(n+1)\binom{2n}n.$$}
\podpis{Patrik Bak}

{%%%%% A-III-5
Daný je ostrouhlý trojuholník $ABC$ s~priesečníkom výšok~$H$.
Os uhla $BHC$ pretína stranu~$BC$ v~bode~$D$.
Označme postupne $E$ a~$F$ obrazy bodu~$D$ v~osových súmernostiach
podľa priamok $AB$ a~$AC$.
Dokážte, že kružnica opísaná trojuholníku $AEF$ prechádza stredom kružnicového
oblúka~$BAC$.}
\podpis{Patrik Bak}

{%%%%% A-III-6
Dané je nenulové celé číslo~$k$. Dokážte, že rovnici
$$
k~= \frac {x^2-xy+2y^2}{x+y}
$$
vyhovuje nepárny počet
usporiadaných dvojíc celých čísel $(x, y)$ práve vtedy, keď $k$ je deliteľné siedmimi.}
\podpis{Patrik Bak}

{%%%%% B-S-1
Na tabuli je napísaných päť navzájom rôznych kladných čísel.
Určte najväčší možný počet dvojíc z~nich vytvorených,
v~ktorých je súčet oboch prvkov rovný jednému z~piatich čísel napísaných na tabuli.}
\podpis{Michal Rolínek}

{%%%%% B-S-2
Na odvesnách $AC$ a~$BC$ daného pravouhlého trojuholníka $ABC$
určte postupne body $K$ a~$L$ tak, aby súčet
$$
|AK|^2+|KL|^2+|LB|^2
$$
nadobúdal najmenšiu možnú hodnotu a~vyjadrite ju pomocou $c=|AB|$.}
\podpis{Jaroslav Švrček}

{%%%%% B-S-3
Napíšeme za sebou 1000 prirodzených čísel, ktoré majú vlastnosti:
súčet
každých siedmich po sebe zapísaných čísel je 2017, na 123. mieste je číslo~123, na
234.~mieste číslo 234 a~na 345.~mieste číslo~345. Určte súčet štyroch čísel na
456., 567., 678. a~789.~mieste.}
\podpis{Jaroslav Zhouf}

{%%%%% B-II-1
Nájdite všetky dvojice prirodzených čísel $a$,~$b$, pre ktoré
platí
$$
a+\frac{66}{a}=b+\frac{66}b.
$$
}
\podpis{Jaromír Šimša}

{%%%%% B-II-2
Na kružnici $k$ je vyznačený konečný počet jej oblúkov, pričom ich zjednotenie pokrýva celú kružnicu~$k$ a~každý oblúk má dĺžku menšiu ako je obvod kružnice~$k$. Dokážte, že existuje taká podmnožina~$\mm M$ danej množiny oblúkov, že súčet dĺžok všetkých oblúkov z~$\mm M$ nie je väčší ako dvojnásobok obvodu kružnice~$k$ a~zároveň zjednotenie oblúkov z~$\mm M$ pokrýva celú kružnicu~$k$.
%V~rovine je daných 2017 takých bodov, že z~každej trojice možno vybrať dva, ktorých vzdialenosť je menšia ako~1. Dokážte, že existuje kruh s~polomerom~1, ktorý obsahuje aspoň 1009 daných bodov.
}
\podpis{Vojtech Bálint}

{%%%%% B-II-3
V~rovine sú dané kružnice $k$ a~$l$, ktoré sa pretínajú v~bodoch
$E$ a~$F$. Dotyčnica ku kružnici~$l$ zostrojená v~bode~$E$ pretína kružnicu~$k$
v~bode~$H$ ($H\ne E$). Na oblúku~$EH$ kružnice~$k$, ktorý neobsahuje bod~$F$,
zvoľme bod~$C$ ($E\ne C\ne H$) a~priesečník priamky~$CE$ s~kružnicou~$l$
označme~$D$ ($D\ne E$). Dokážte, že trojuholníky $DEF$ a~$CHF$ sú podobné.}
\podpis{Šárka Gergelitsová}

{%%%%% B-II-4
Určte všetky hodnoty reálneho parametra~$p$ tak, aby rovnica
$$
2017\cdot\Big|1-\big|1-|1-x|\big|\Big|=2016x+p
$$
mala práve tri riešenia v~obore reálnych čísel.}
\podpis{Jaroslav Švrček}

{%%%%% C-S-1
Nájdite všetky riešenia rovnice
$$
1=\frac{|3x-7|-|9-2x|}{|x+2|}.
$$
}
\podpis{Vojtech Bálint}

{%%%%% C-S-2
Označme $\mm M$ množinu všetkých hodnôt výrazu
$V(n)=n^{4}+11n^{2}-12$,
pričom $n$ je nepárne prirodzené číslo. Nájdite všetky možné zvyšky po
delení číslom~$48$, ktoré dávajú prvky množiny~$\mm M$.}
\podpis{Aleš Kobza}

{%%%%% C-S-3
Päta~$P$ výšky z~vrcholu~$C$ v~trojuholníku $ABC$ delí stranu~$AB$
v~pomere ${|AP|:|PB|}=1:3$. V~rovnakom pomere sú aj obsahy štvorcov nad jeho stranami
$AC$ a~$BC$. Dokážte, že trojuholník $ABC$ je pravouhlý.}
\podpis{Leo Boček}

{%%%%% C-II-1
Nájdite všetky mnohočleny $P(x)=ax^2+bx+c$ s~celočíselnými koeficientmi
spĺňajúce
$$
1<P(1)<P(2)<P(3)\qquad \text{a~súčasne}\qquad \frac{P(1)\cdot P(2)\cdot P(3)}4=17^2.
$$
}
\podpis{Tomáš Jurík}

{%%%%% C-II-2
Štvorcovú tabuľku $6\times 6$ zaplníme všetkými celými číslami od
1 do 36.
\ite a) Uveďte príklad takého zaplnenia tabuľky, že súčet každých
dvoch čísel v~rovnakom riadku či v~rovnakom stĺpci je väčší ako~11.
\ite b) Dokážte, že pri ľubovoľnom zaplnení tabuľky sa v~niektorom riadku alebo stĺpci nájdu
dve čísla, ktorých súčet neprevyšuje~12.\endgraf}
\podpis{Jaromír Šimša}

{%%%%% C-II-3
Dokážte, že obdĺžnik s~rozmermi $32\times120$ sa dá zakryť siedmimi zhodnými štvorcami so stranou~30.}
\podpis{Vojtech Bálint}

{%%%%% C-II-4
Dokážte, že pre všetky kladné reálne čísla $ a\le b\le c$ platí
$$
(-a+b+c)\Big(\frac{1}{a}+\frac{1}{b}+\frac{1}{c}\Big)\geq3.
$$
}
\podpis{Šárka Gergelitsová}

{%%%%%   vyberko, den 1, priklad 1
Tri zajace hodnotia kvalitu $2n+1$ mrkiev. Každý zajac priradí mrkvám ohodnotenia
od~$0$~po~$2n$, každé práve raz (každej mrkve práve jedno).
Celkové ohodnotenie mrkvy je súčtom troch ohodnotení danej mrkvy zajacmi.
Mrkva~$M_1$ je lepšia ako mrkva~$M_2$, ak aspoň dva zajace ohodnotili mrkvu
$M_1$ ostro väčším ohodnotením ako $M_2$.
Ukážte, že každá mrkva je lepšia ako práve $n$ iných mrkiev práve vtedy,
keď majú všetky mrkvy rovnaké celkové ohodnotenie.}
\podpis{Matej Králik, Slavomír Hanzely:TMO 2014 shortlist, https://artofproblemsolving.com/community/c6h1433709p8103868}

{%%%%%   vyberko, den 1, priklad 2
Dokážte, že v~každom trojuholníku so stranami $a$, $b$, $c$ a~polomerom opísanej kružnice~$R$
platí
$$9R^2 \ge a^2 + b^2 + c^2.$$}
\podpis{Matej Králik, Slavomír Hanzely:Rumunsko shortlist 2008, https://artofproblemsolving.com/community/c6h1095500p4906189}

{%%%%%   vyberko, den 1, priklad 3
Nech $ABC$ je trojuholník s~vpísanou kružnicou~$k$.
Označme body dotyku $k$ so stranami $BC$ a~$AC$ postupne $D_1$ a~$E_1$.
Body $D_2$ a~$E_2$ ležia na stranách $BC$ a~$AC$, pričom platí
${|CD_2|=|BD_1|}$ a~${|CE_2|=|AE_1|}$. Priesečník $AD_2$ a~$BE_2$ označme~$P$.
Kružnica~$k$ pretína $AD_2$ v~dvoch bodoch, z~ktorých ten bližší k~vrcholu~$A$
označíme~$Q$. Dokážte, že $|AQ|=|D_2P|$.}
\podpis{Matej Králik, Slavomír Hanzely:USAMO 2001}

{%%%%%   vyberko, den 1, priklad 4
Nájdite všetky funkcie $f\colon \Bbb{Z}\to\Bbb{Z}$, pre ktoré platí
$$f(a^3+b^3+c^3)=f^3(a)+f^3(b)+f^3(c).$$}
\podpis{Matej Králik, Slavomír Hanzely:https://artofproblemsolving.com/community/c6h1304692p6967857}

{%%%%%   vyberko, den 2, priklad 1
Pre ľubovoľné prirodzené číslo~$k$ označíme $S(k)$ jeho ciferný súčet. Nájdite všetky polynómy~$P$ s~celočíselnými koeficientmi také, že pre všetky prirodzené čísla $n\ge 2017$ platí $P(n)>0$ a $$S(P(n))=P(S(n)).$$}
\podpis{Martin Vodička:IMO shortlist 2016, N1}

{%%%%%   vyberko, den 2, priklad 2
Nájdite všetky prirodzené čísla~$n$ s~nasledujúcou vlastnosťou: Pre ľubovoľné reálne čísla $a_1,a_2,\dots,a_n$ a $b_1,b_2,\dots,b_n$ spĺňajúce $|a_i|+|b_i|=1$ pre všetky $1\le i\le n$ existujú čísla $x_1,x_2,\dots,x_n$ také, že
$$|x_i|=1\quad\text{pre všetky $1\le i\le n$} \qquad\text{a}\qquad
\Big |\sum_{i=1}^na_ix_i\Big |+\Big |\sum_{i=1}^nb_ix_i\Big |\le 1.$$
}
\podpis{Martin Vodička:IMO shortlist 2016, A3}

{%%%%%   vyberko, den 2, priklad 3
Nech $ABC$ je rovnoramenný trojuholník s~$|AB|=|AC|>|BC|$ a~nech $I$ je stred vpísanej kružnice do tohto trojuholníka. Polpriamka~$BI$ pretína stranu~$AC$ v~bode~$D$ a~priamka kolmá na $AC$ prechádzajúca bodom~$D$ pretína $AI$ v~bode~$E$. Označme $J$ obraz bodu~$I$ v~osovej súmernosti podľa priamky~$AC$. Dokážte, že body $B$, $E$, $D$, $J$ ležia na jednej kružnici.}
\podpis{Martin Vodička:IMO shortlist 2016, G4}

{%%%%%   vyberko, den 3, priklad 1
Postupnosť $\{a_n\}_{n=1}^\infty$ spĺňa $a_1 = 0$ a pre každé $i\ge1$ platí $|a_{i+1}| = |a_i + 1|$. Dokážte, že pre všetky prirodzené čísla $n$ platí
$$
\frac{a_1+a_2+\dots+a_n}{n} \ge -\frac12.
$$
}
\podpis{Peter Novotný, Peter Súkeník:Austrália 2003}

{%%%%%   vyberko, den 3, priklad 2
V~hoteli sa nachádza $n\ge3$ izieb, ktoré sú rozmiestnené pozdĺž kruhovej chodby. V~každej izbe je prepínač svetla. Keď ho prepneme, zmeníme stav svetla (zo zapnutého na vypnutý a~naopak) v~danej izbe a tiež v~oboch susedných izbách. Pre ktoré hodnoty~$n$ je možné vždy (bez ohľadu na počiatočný stav svetiel) pomocou konečného počtu prepnutí zhasnúť všetky svetlá?
}
\podpis{Peter Novotný, Peter Súkeník:Peter Novotný}

{%%%%%   vyberko, den 3, priklad 3
Daný je trojuholník $ABC$. Priamka rovnobežná so stranou~$BC$ pretína strany $AB$ a~$AC$ postupne v~bodoch $P$ a~$Q$. Nech $M$ je vnútorný bod trojuholníka $APQ$. Úsečky $MB$ a~$MC$ pretínajú úsečku~$PQ$ postupne v~bodoch $E$ a~$F$. Označme $N$ druhý priesečník kružníc opísaných trojuholníkom $PMF$ a~$QME$. Dokážte, že body $A$,~$M$,~$N$ ležia na jednej priamke.}
\podpis{Peter Novotný, Peter Súkeník:2016 Sharygin Geometry Olympiad}

{%%%%%   vyberko, den 3, priklad 4
Nech $p$, $q$ sú prvočísla, pričom $q\ne2$. Dokážte, že celé číslo $x$ spĺňajúce
$$
q \mid (x+1)^p-x^p
$$
existuje práve vtedy, keď $$q \equiv 1 \pmod p.$$
}
\podpis{Peter Novotný, Peter Súkeník:Irán 2016}

{%%%%%   vyberko, den 4, priklad 1
Nech $ABCD$ je tetivový štvoruholník s~opísanou kružnicou~$k$ a~nech $r$ a~$s$ sú postupne obrazy priamky~$AB$ v~osovej súmernosti podľa osi vnútorného uhla $CAD$ a~$CBD$. Priamky $r$ a~$s$ sa pretínajú v~bode~$P$, ktorý sa nachádza vo vonkajšej oblasti kružnice~$k$. Označme $O$ stred kružnice~$k$. Dokážte, že priamky $OP$ a~$CD$ sú na seba kolmé.}
\podpis{Jozef Rajník, Samuel Sládek:Brazilian Math Olympiad 2008, Problem 4, https://artofproblemsolving.com/community/c6h235749p1297651}

{%%%%%   vyberko, den 4, priklad 2
Nájdite najväčší možný konečný počet koreňov rovnice
$$|x - a_1| + |x - a_2| + \dots + |x - a_{42}| = |x - b_1| + |x - b_2| + \dots + |x - b_{42}|,
$$
kde $a_1,a_2,\dots,a_{42},b_1,b_2,\dots,b_{42}$ sú po dvoch rôzne reálne čísla.
}
\podpis{Jozef Rajník, Samuel Sládek:ARO 2005 - problem 11.1, https://artofproblemsolving.com/community/c6h35314p220220}

{%%%%%   vyberko, den 4, priklad 3
Uvažujme štvorčekovú mriežku $n \times n$, kde $n$ je kladné celé číslo. V~jednom ťahu vyberieme štyri políčka, ktoré ležia na prieniku dvoch riadkov a~dvoch stĺpcov a~na každé z~nich položíme žetón. Tento ťah môžeme spraviť len vtedy, keď aspoň jedno zo štyroch vybraných políčok je prázdne. V~závislosti od čísla $n$ určte najväčší počet ťahov, ktorý môžeme spraviť, ak začíname s~prázdnou mriežkou.}
\podpis{Jozef Rajník, Samuel Sládek:http://www.hexagon.edu.vn/images/resources/upload/dec3c0b23f6d5bffa9c661616b1658fd/problemsolvingmethods in combinatorics an approach to olympiad_1377958817.pdf, str. 52}

{%%%%%   vyberko, den 4, priklad 4
Nájdite všetky kladné celé čísla~$n$ také, že $\varphi(n)$ delí $n^2 + 3$.
(Číslo $\varphi(n)$ označuje Eulerovu funkciu, \tj. počet kladných celých čísel neprevyšujúcich~$n$, ktoré sú s~číslom~$n$ nesúdeliteľné.)
}
\podpis{Jozef Rajník, Samuel Sládek:http://vjimc.osu.cz/j26/j26problems1.pdf}

{%%%%%   vyberko, den 5, priklad 1
Je daný rôznostranný ostrouhlý trojuholník $ABC$. Nech $H$ je jeho ortocentrum a~$O$ stred kružnice opísanej. Ďalej nech $B' = BH \cap AC$, $C' = CH \cap AB$, $P = AH \cap B'C'$ a~$T = AO \cap BC$. Označme $M$ stred strany~$BC$. Dokážte, že $MH \parallel TP$.}
\podpis{Patrik Bak, Tomáš Kekeňák:Rusko 2008}

{%%%%%   vyberko, den 5, priklad 2
Nájdite všetky kladné celé čísla~$n$ také, že do buniek obdĺžnikovej tabuľky vieme napísať všetky kladné delitele čísla $n$ tak, že sú splnené nasledovné podmienky:
 \ite{$\bullet$} všetky bunky obsahujú navzájom rôzne delitele;
 \ite{$\bullet$} súčty vo všetkých riadkoch sú rovnaké;
 \ite{$\bullet$} súčty vo všetkých stĺpcoch sú rovnaké.
}
\podpis{Patrik Bak, Tomáš Kekeňák:ISL 2016, C2}

{%%%%%   vyberko, den 5, priklad 3
Nájdite najmenšiu reálnu konštantu~$C$ takú, že pre ľubovoľné kladné reálne čísla $a_1$, $a_2$, $a_3$, $a_4$ a~$a_5$ (nie nutne rôzne) vieme zvoliť rôzne indexy $i$, $j$, $k$ a~$l$ také, že $$\left|\frac{a_i}{a_j}-\frac{a_k}{a_l}\right| \le C.$$}
\podpis{Patrik Bak, Tomáš Kekeňák:ISL 2016, A2}

{%%%%%   vyberko, den 2, priklad 4
...}
\podpis{...}

{%%%%%   vyberko, den 5, priklad 4
...}
\podpis{...}

{%%%%%   trojstretnutie, priklad 1
Nájdite všetky kladné reálne čísla~$c$ s~nasledujúcou vlastnosťou: Existuje nekonečne veľa dvojíc kladných celých čísel $(n,m)$ takých, že $n\ge m+c\sqrt{m-1}+1$ a~medzi číslami $n,n+1,\dots, 2n-m$ sa nenachádza žiadna druhá mocnina celého čísla.}
\podpis{Patrik Bak}

{%%%%%   trojstretnutie, priklad 2
Daný je ostrouhlý trojuholník $ABC$ s~opísanou kružnicou~$\omega$. Na oblúku~$BC$ kružnice~$\omega$ neobsahujúcom bod~$A$ je daný bod~$D$. Bod~$E$ leží vnútri trojuholníka $ABC$ mimo priamky~$AD$ a~spĺňa $|\uhel DBE|=|\uhel ACB|$ a~$|\uhel DCE|=|\uhel ABC|$. Uvažujme bod~$F$ na priamke~$AD$ taký, že $EF\parallel BC$ a~bod $G\ne A$ na kružnici~$\omega$ taký, že $|AF|=|FG|$. Dokážte, že body $D$, $E$, $F$, $G$ ležia na jednej kružnici.}
\podpis{Patrik Bak}

{%%%%%   trojstretnutie, priklad 3
Nech $k$ je kladné celé číslo. Na tabuli je napísaná konečná postupnosť celých čísel $x_1,x_2,\dots,x_n$. Maťko a~Kubko hrajú hru, ktorá prebieha v~nasledujúcich kolách:
\ite $\bullet$ V~každom kole najskôr Maťko rozdelí postupnosť, ktorá je na tabuli, na dve alebo viac súvislých podpostupností (t.\,j. takých, že každá podpostupnosť pozostáva z~po sebe idúcich členov pôvodnej postupnosti). Pritom ak je počet podpostupností väčší ako 2, súčet čísel v~každej z~nich musí byť násobkom~$k$.
\ite $\bullet$ Následne Kubko vyberie jednu z~podpostupností a~všetky ostatné podpostupnosti z~tabule zotrie.
\endgraf\noindent
Hra končí v~momente, keď na tabuli zostane napísané iba jedno číslo. Dokážte, že Maťko môže svoje ťahy voliť tak, že nezávisle na ťahoch Kubka hra skončí po nanajvýš $3k$ kolách.}
\podpis{Poľsko}

{%%%%%   trojstretnutie, priklad 4
Daný je trojuholník $ABC$. Priamka~$l$ je rovnobežná so stranou~$BC$ a~pretína strany $AB$, $AC$ postupne v~bodoch $D$, $E$ a~kružnicu opísanú trojuholníku $ABC$ v~bodoch $F$, $G$ tak, že body $F$,~$D$, $E$, $G$ ležia na priamke~$l$ v~tomto poradí. Kružnice opísané trojuholníkom $FEB$ a~$DCG$ sa pretínajú v~bodoch $P$ a~$Q$. Dokážte, že body $A$, $P$, $Q$ ležia na jednej priamke.}
\podpis{Patrik Bak}

{%%%%%   trojstretnutie, priklad 5
Každé zo~$4n^2$ políčok tabuľky $2n\times 2n$ ($n\ge 1$) je ofarbené buď načerveno, alebo namodro. Hovoríme, že množina štyroch navzájom rôznych políčok tabuľky je {\it pekná}, ak tieto políčka možno označiť $A$, $B$, $C$, $D$ tak, že $A$~a~$B$ ležia v~tom istom riadku, $C$~a~$D$ ležia v~tom istom riadku, $A$~a~$C$ ležia v~tom istom stĺpci, $B$~a~$D$ ležia v~tom istom stĺpci, políčka $A$ a~$D$ sú modré a~políčka $B$ a~$C$ sú červené. Určte najväčší možný počet pekných množín v~takej tabuľke.}
\podpis{Poľsko}

{%%%%%   trojstretnutie, priklad 6
Nájdite všetky funkcie $f\colon (0,+\infty)\to\Bbb{R}$ také, že
$$ f(x)-f(x+y)=f\left(\frac xy\right)f(x+y)
\qquad \text{ pre všetky $x,y>0$.}
$$}
\podpis{Rakúsko}

{%%%%%   IMO, priklad 1
Pre dané celé číslo $a_0 > 1$ definujeme postupnosť $a_0,a_1,a_2,\dots$ tak, že pre všetky $n \ge 0$ platí
$$
a_{n+1} = \begin{cases}
\sqrt{a_n} & \text{ak } \sqrt{a_n} \text{ je celé číslo}, \\
a_n + 3 & \text{inak.}
\end{cases}
$$
Určte všetky hodnoty~$a_0$, pre ktoré existuje také číslo~$A$, že pre nekonečne veľa indexov $n$ platí~$a_n = A$.}
\podpis{Južná Afrika}

{%%%%%   IMO, priklad 2
Označme $\Bbb R$ množinu reálnych čísel. Určte všetky funkcie $f  \colon  \Bbb R \to \Bbb R$ také, že pre všetky reálne čísla $x$ a~$y$ platí
$$
f \left( f(x) f(y) \right) + f(x+y) = f(xy).
$$}
\podpis{Albánsko}

{%%%%%   IMO, priklad 3
Poľovník a~neviditeľný zajac hrajú hru v~euklidovskej rovine. Zajacova počiatočná poloha~$A_0$ a~poľovníkova počiatočná poloha~$B_0$ sú zhodné. Po $n-1$ kolách sa zajac nachádza v~bode~$A_{n-1}$ a~poľovník v~bode~$B_{n-1}$. V~$n$-tom kole sa postupne udejú tieto tri veci:
\ite{i)}
Zajac sa neviditeľne presunie do bodu~$A_n$, pričom vzdialenosť medzi $A_{n-1}$ a~$A_n$ je presne~$1$.
\ite{ii)}
Sledovacie zariadenie ukáže poľovníkovi bod~$P_n$. Jediná záruka, ktorú sledovacie zariadenie poľovníkovi poskytuje, je, že vzdialenosť medzi bodmi $P_n$ a~$A_n$ je nanajvýš~$1$.
\ite{iii)}
Poľovník sa viditeľne presunie do bodu~$B_n$, pričom vzdialenosť medzi $B_{n-1}$ a~$B_n$ je presne~$1$.
\endgraf\noindent
Dokáže poľovník vždy (\tj. bez ohľadu na to, ako sa presúva zajac a~bez ohľadu na to, aké body ukazuje sledovacie zariadenie) voliť svoje ťahy tak, že po $10^9$ kolách má istotu, že vzdialenosť medzi ním a~zajacom je nanajvýš $100$?}
\podpis{Rakúsko}

{%%%%%   IMO, priklad 4
Na kružnici~$\Omega$ sú dané dva rôzne body $R$ a~$S$, pričom $RS$ nie je jej priemerom. Označme~$l$ dotyčnicu kružnice~$\Omega$ v~bode~$R$. Nech $T$ je taký bod, že $S$ je stredom úsečky~$RT$. Na kratšom oblúku~$RS$ kružnice~$\Omega$ je zvolený bod~$J$ tak, že kružnica~$\Gamma$ opísaná trojuholníku  $JST$ pretína priamku~$l$ v~dvoch rôznych bodoch. Označme $A$ ten priesečník $\Gamma$ s~$l$, ktorý je bližšie k~bodu~$R$. Priamka~$AJ$ pretína kružnicu~$\Omega$ v~bode~$K$ ($K\ne J$). Dokážte, že priamka~$KT$ je dotyčnicou kružnice~$\Gamma$.}
\podpis{Luxembursko}

{%%%%%   IMO, priklad 5
Dané je celé číslo $N \ge 2$. Skupina $N(N+1)$ futbalistov, z~ktorých žiadni dvaja nemajú rovnakú výšku, stojí v~rade. Tréner Ján chce odstrániť $N(N-1)$ futbalistov z~tohto radu tak, aby nový rad pozostávajúci zo zvyšných $2N$ futbalistov spĺňal nasledovných $N$~podmienok:
\item nikto nestojí medzi dvoma najvyššími futbalistami,
\item nikto nestojí medzi tretím a~štvrtým najvyšším futbalistom,
\item atď...
\item nikto nestojí medzi dvoma najnižšími futbalistami.

Dokážte, že je to vždy možné.}
\podpis{Rusko}

{%%%%%   IMO, priklad 6
Usporiadaná dvojica celých čísel $(x,y)$  sa nazýva {\it primitívny mrežový bod}, keď najväčší spoločný deliteľ čísel $x$ a~$y$ je~$1$. Dokážte, že pre ľubovoľnú konečnú množinu~$\mn S$ primitívnych mrežových bodov existuje kladné celé číslo~$n$ a~celé čísla $a_0,a_1,\dots,a_n$ také, že pre všetky $(x,y)$ z~$S$ platí
$$
\postdisplaypenalty=10000
a_0 x^n + a_1 x^{n-1} y + a_2 x^{n-2} y^2 + \cdots + a_{n-1} x y^{n-1} + a_n y^n = 1.
$$}
\podpis{USA}

{%%%%%   MEMO, priklad 1
Nájdite všetky funkcie $f \colon \Bbb{R} \to \Bbb{R}$ spĺňajúce
$$
f(x^2+f(x)f(y))=x f(x+y)
$$
pre všetky reálne čísla $x$ a~$y$.
}
\podpis{Slovensko, Patrik Bak}

{%%%%%   MEMO, priklad 2
Nech $n \ge 3$ je celé číslo. Označenie $n$~vrcholov, $n$~strán a~vnútra pravidelného $n$-uholníka $2n+1$ rôznymi celými číslami nazývame {\it priemerné}, ak sú splnené nasledovné podmienky:
\ite{$\bullet$} Každá strana je označená aritmetickým priemerom čísel na jej koncových vrcholoch.
\ite{$\bullet$} Vnútro $n$-uholníka je označené aritmetickým priemerom všetkých čísel na jeho vrcholoch.
\endgraf\noindent
Nájdite všetky celé čísla $n \ge 3$ také, že existuje priemerné označenie pravidelného $n$-uholníka pozostávajúce z~$2n+1$ po sebe idúcich celých čísel.
}
\podpis{Česká rep.}

{%%%%%   MEMO, priklad 3
Je daný konvexný päťuholník $ABCDE$. Označme $P$ priesečník priamok $CE$ a~$BD$. Dokážte, že ak  platí $|\uhol PAD| = |\uhol ACB|$ a~$|\uhol CAP| = |\uhol EDA|$, tak bod~$P$ leží na priamke určenej stredmi kružníc opísaných trojuholníkom $ABC$ a~$ADE$.
}
\podpis{Slovensko, Patrik Bak}

{%%%%%   MEMO, priklad 4
Nájdite najmenšiu hodnotu výrazu
$$|2^m-181^n|,$$
pričom $m$ a~$n$ sú kladné celé čísla.}
\podpis{Nemecko}

{%%%%%   MEMO, priklad t1
Nájdite všetky dvojice polynómov $(P,Q)$ s~reálnymi koeficientmi spĺňajúce $$P(x+Q(y))=Q(x+P(y))$$ pre všetky reálne čísla $x$ a~$y$.
}
\podpis{Poľsko}

{%%%%%   MEMO, priklad t2
Nájdite najmenšiu možnú reálnu konštantu~$C$ takú, že nerovnosť
$$
|x^3+y^3+z^3+1| \le C |x^5+y^5+z^5+1|
$$
platí pre všetky reálne čísla $x$, $y$, $z$ spĺňajúce $x+y+z=\m1$.
}
\podpis{Rakúsko}

{%%%%%   MEMO, priklad t3
Na každom políčku tabuľky $2017 \times 2017$ je lampa. Každá lampa je buď zapnutá, alebo vypnutá. Lampu nazývame {\it tônistá}, ak má párny počet zapnutých susedných lámp. Aký je najmenší možný počet tônistých lámp? (Dve lampy sú susedné, ak sa nachádzajú vedľa seba v~rovnakom riadku alebo v~rovnakom stĺpci danej tabuľky.)
}
\podpis{Rakúsko}

{%%%%%   MEMO, priklad t4
Nech $n \ge 3$ je celé číslo. Postupnosť $P_1, P_2, \dots, P_n$ rôznych bodov v~rovine nazývame {\it prešibaná}, ak žiadne tri body postupnosti neležia na jednej priamke, lomená čiara $P_1 P_2 \dots P_n$ nepretína samu seba a~pre každé $i=1,2,\dots,n-2$ je trojuholník $P_iP_{i+1}P_{i+2}$ orientovaný proti smeru hodinových ručičiek. Pre každé celé číslo $n \ge 3$ nájdite najväčšie celé číslo~$k$ s~nasledovnou vlastnosťou: existuje $n$ rôznych bodov $A_1, A_2, \dots, A_n$ ležiacich v~jednej rovine takých, že existuje $k$ rôznych permutácií $\sigma \colon \{1,2,\dots,n\} \to \{1,2,\dots,n\}$, pre ktoré postupnosť $A_{\sigma(1)},A_{\sigma(2)},\dots, A_{\sigma(n)}$ je prešibaná. (Lomená čiara $P_1 P_2 \dots P_n$ pozostáva z~úsečiek $P_1P_2,P_2P_3,\dots,P_{n-1}P_n$.)
}
\podpis{Poľsko}

{%%%%%   MEMO, priklad t5
Nech $ABC$ je ostrouhlý trojuholník taký, že $|AB|>|AC|$ a~$\Gamma$ je kružnica jemu opísaná. Stred kratšieho oblúka~$BC$ kružnice~$\Gamma$ označme~$M$ a~priesečník polpriamok $AC$ a~$BM$ označme~$D$. Nech $E \ne C$ je priesečník vnútornej osi uhla $ACB$ a~kružnice opísanej trojuholníku $BDC$. Predpokladajme, že bod~$E$ leží vo vnútri trojuholníka $ABC$ a~existuje priesečník~$N$ priamky~$DE$ s~kružnicou~$\Gamma$ taký, že $E$ je stred úsečky~$DN$. Dokážte, že $N$ je stred úsečky~$I_BI_C$, kde $I_B$, $I_C$ sú postupne stredy kružníc pripísaných $ABC$ oproti vrcholom $B$ a~$C$.
}
\podpis{Chorvátsko}

{%%%%%   MEMO, priklad t6
Nech $\Gamma$ je kružnica so stredom v~bode~$O$ opísaná ostrouhlému trojuholníku $ABC$, pre ktorý platí, že $|AB|\ne|AC|$. Dotyčnice ku kružnici~$\Gamma$ v~bodoch $B$ a~$C$ sa pretínajú v~bode~$D$. Nech priamka~$AO$ pretína $BC$ v~bode~$E$. Označme $M$ stred úsečky~$BC$ a~$N \ne A$ priesečník priamky~$AM$ a~kružnice~$\Gamma$. Nech $F \ne A$ je bod ležiaci na $\Gamma$ taký, že $A$, $M$, $E$, $F$ ležia na jednej kružnici. Dokážte, že priamka~$FN$ rozpoľuje úsečku~$MD$.
}
\podpis{Slovensko, Patrik Bak}

{%%%%%   MEMO, priklad t7
Nájdite všetky kladné celé čísla $n\ge 2$ také, že existuje usporiadanie $x_0,x_1,\dots,x_{n-1}$ čísel $0,1, \dots, n-1$ také, že súčty
$$
x_0,\quad x_0+x_1,\quad \dots,\quad x_0+x_1+\ldots+x_{n-1}
$$
dávajú navzájom rôzne zvyšky po delení~$n$.
}
\podpis{Poľsko}

{%%%%%   MEMO, priklad t8
Pre celé číslo $n \ge 3$ definujeme postupnosť $\alpha_1,\alpha_2, \dots, \alpha_k$ ako postupnosť exponentov v~prvočíselnom rozklade $n!=p_1^{\alpha_1}p_2^{\alpha_2} \dots p_k^{\alpha_k}$, kde $p_1< p_2< \dots <p_k$ sú prvočísla.
Nájdite všetky celé čísla $n \ge 3$, pre ktoré je $\alpha_1,\alpha_2, \dots, \alpha_k$ geometrická postupnosť.
}
\podpis{Rakúsko}

{%%%%%   CPSJ, priklad 1
Nájdite najväčšie celé číslo~$n\ge 3$, pre ktoré existuje $n$-ciferné číslo $\overline{a_1a_2a_3\ldots a_n}$ s~nenulovými
ciframi $a_1$, $a_2$ a~$a_n$, ktoré je deliteľné číslom $\overline{a_2a_3\ldots a_n}$.}
\podpis{Jaromír Šimša}

{%%%%%   CPSJ, priklad 2
Daný je trojuholník $ABC$, pričom $|AB|+|AC|=3 \cdot |BC|$. Označme $D$, $E$ také body, že $BCDA$ a~$CBEA$ sú rovnobežníky. Na stranách $AC$ a~$AB$ sú postupne zvolené body $F$ a~$G$ tak, že $|AF|=|AG|=|BC|$. Dokážte, že priamky $DF$ a~$EG$ sa pretínajú na úsečke~$BC$.}
\podpis{Patrik Bak}

{%%%%%   CPSJ, priklad 3
Dokážte, že pre všetky reálne čísla $x$, $y$ platí
$$(x^2+1)(y^2+1) \ge 2(xy-1)(x+y).$$
Pre ktoré celé čísla $x$, $y$ nastáva rovnosť?}
\podpis{Patrik Bak}

{%%%%%   CPSJ, priklad 4
Daný je pravouhlý trojuholník $ABC$ s~obvodom~$2$, pričom $|\uhol ACB|=90^\circ$. Bod~$S$ je stredom kružnice pripísanej k~strane~$AB$ daného trojuholníka a~$H$ je priesečník výšok trojuholníka $ABS$. Určte najmenšiu možnú dĺžku úsečky~$HS$.}
\podpis{Jerzy Bednarczuk}

{%%%%%   CPSJ, priklad 5
V~každom políčku tabuľky $(mn+1)\times (mn+1)$ je vpísané reálne číslo z~intevalu $\langle0,1\rangle$. Pritom súčet čísel v~každom štvorcovom výseku tabuľky s~rozmermi $n\times n$ je rovný $n$. Určte, aký najväčší môže byť súčet všetkých čísel v~tabuľke.}
\podpis{\L{}ukasz Bożyk}

{%%%%%   CPSJ, priklad t1
Rozhodněte, zda existují prvočísla $p$, $q$, $r$ taková, že
$$(p^2+p)(q^2+q)(r^2+r)$$
je druhou mocninou některého celého čísla.}
\podpis{Kamil Rychlewicz}

{%%%%%   CPSJ, priklad t2
Rozhodněte, zda existuje konvexní šestiúhelník, jehož všechny strany mají délky větší než 1 a všech devět jeho úhlopříček má délky menší než 2.}
\podpis{Vojtech Bálint}

{%%%%%   CPSJ, priklad t3
Ile jest 8-cyfrowych liczb postaci ${*}2{*}0{*}1{*}7$, gdzie cztery nieznane cyfry zast\ą{}piono gwiazdkami, które s\ą{} podzielne przez $7$?}
\podpis{Peter Novotný}

{%%%%%   CPSJ, priklad t4
Bolek narysowa\l{} na tablicy trapez $ABCD$ ($AB\parallel CD$), a~w~nim jego lini\ę{} środkow\ą{}~$EF$. Punkt przeci\ę{}cia jego przek\ą{}tnych $AC$, $BD$ oznaczy\l{} przez~$P$, a~jego rzut prostok\ą{}tny na prost\ą{}~$AB$ oznaczy\l{} przez~$Q$. Lolek, chc\ą{}c dokuczyć Bolkowi, zmaza\l{} z~tablicy wszystko oprócz odcinków $EF$ i~$PQ$. Gdy Bolek to zobaczy\l{}, chcia\l{} uzupe\l{}nić rysunek i~dorysować wyjściowy trapez, ale nie wiedzia\l{} jak to zrobić. Czy umiesz pomóc Bolkowi?}
\podpis{Libuše Hozová, Jaroslav Švrček}

{%%%%%   CPSJ, priklad t5
Do každého políčka štvorcovej tabuľky $100\times 100$ vpíšeme číslo $1$, $2$ alebo $3$. Uvažujme všetky podtabuľky $m\times n$, pričom $m\ge 2$ a~$n\ge 2$. Podtabuľku nazveme {\it vyrovnaná}, ak má vo svojich rohových políčkach štyri rovnaké čísla. Pre čo najväčšie číslo~$k$ dokážte, že vždy môžeme nájsť $k$ vyrovnaných podtabuliek, z~ktorých žiadne dve sa neprekrývajú, \tj. nemajú spoločné políčko.}
\podpis{Jaromír Šimša}

{%%%%%   CPSJ, priklad t6
Na tabuli je napísaných 100 navzájom rôznych kladných reálnych čísel, pričom pre ľubovoľné tri rôzne čísla $a$, $b$, $c$ je číslo $a^2+bc$ celé. Dokážte, že pre ľubovoľné dve čísla $x$, $y$ z~tabule je číslo $\frac xy$ racionálne.}
\podpis{Dominik Burek}
