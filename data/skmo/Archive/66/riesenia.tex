{%%%%%   A-I-1
Predpokladajme, že pre prirodzené číslo~$a$ platí $p^n+1 = a^3$
(zjavne $a~\ge 2$). Túto rovnosť upravíme tak, aby bolo možné jednu
stranu rozložiť na súčin:
$$
p^n = a^3-1 = (a-1)(a^2+a+1).
$$
Z~tohto rozkladu vyplýva, že ak $a> 2$, sú čísla $a-1$ aj~$a^2+a+1$
mocninami prvočísla~$p$ (s~kladnými celočíselnými exponentmi).

V~prípade $a>2$ tak platí $a-1=p^k$, čiže
$a= p^k+1$ pre kladné celé číslo~$k$, čo po dosadení
do trojčlena $a^2+a+1$ dáva hodnotu $p^{2k}+3p^k+3$. Keďže $a-1=p^k<{a^2+a+1}$,
je určená hodnota trojčlena $a^2+a+1$ vyššou mocninou prvočísla~$p$,
zaručene preto platí
$$
p^{k} \mid p^{2k}+3p^k+3,\quad\text{čiže}\quad p^{k}\mid3.
$$

Z~toho vyplýva, že musí byť $p=3$ a~$k=1$, a~teda $a=p^k+1=4$.
Číslo ${a^2+a+1}=21$ však nie je mocninou troch, a~tak v~prípade $a>2$
žiadne prvočíslo~$p$ rovnici $p^n+1=a^3$ nevyhovuje, nech je exponent $n$ zvolený akokoľvek.

Pre $a= 2$ dostávame rovnicu $p^n = 7$, preto je $p = 7$ jediné vyhovujúce
prvočíslo.


\návody
Určte všetky trojice $(a,b,c)$ prirodzených čísel, pre ktoré platí
$$
2^a+4^b=8^c.
$$
[62--B--I--1]

Nájdite všetky dvojice prirodzených čísel $a$, $b$ také, že $ab = a+b$.
[Všetky členy dáme na ľavú stranu a~rozložíme ju na súčin: $(a-1)(b-1)= 1$.
Je iba jeden spôsob, ako napísať číslo $1$ na pravej strane ako súčin
dvoch nezáporných celých čísel, preto $a~= b = 2$.]

Nájdite všetky prvočísla $p$, pre ktoré existuje prirodzené číslo $x$ také,
že $p^5+4 = x^2$.
[Zrejme $x>2$ a~z~upravenej rovnice $p^5=(x-2)(x+2)$
vzhľadom na $x+2>x-2>0$ vyplýva, že dvojica $(x+2,x-2)$ je buď $(p^5,1)$, $(p^4,p)$,
alebo $(p^3,p^2)$. V~prvom prípade je $p^5-1=4$, v druhom $p^4-p=4$, avšak
žiadna z~oboch rovníc nemá prvočíselné riešenie. Ostáva možnosť $p^3-p^2=4$, ktorá dáva $p=2$ a~$x=6$.]

\D
Nájdite všetky dvojice prvočísel $p$, $q$, pre ktoré existuje prirodzené
číslo~$a$ také, že
$$
\frac{pq}{p+q}=\frac{a^2+1}{a+1}.
$$
[62--A--I--1]

Nájdite všetky dvojice prirodzených čísel $x$, $y$ také, že
$$\displaystyle {xy^2\over x+y}$$ je prvočíslo.
[58--A--I--3]

Dokážte, že pre žiadne prirodzené číslo $n$ nie je číslo $27^n - n^{27}$ prvočíslom.
[57--A--III--4]

Nájdite všetky celé čísla $n$, pre ktoré je $n^4-3n^2+9$ prvočíslo.
[61--A--III--1]
\endnávod}

{%%%%%   A-I-2
Ukážeme, že hľadaný minimálny počet poličiek je rovný $3n-2$. Táto
odpoveď je zrejme správna pre $n=1$ a~pre $n=2$, lebo pre také
$n$ platí $n^2=3n-2$ a~súčasne každá z~$n^2$ škatúľ musí byť
v~takom prípade očividne uložená na inej poličke. V~celom ďalšom riešení budeme preto
predpokladať, že $n\ge3$.

Škatuliam priradíme tabuľku $n \times n$~-- škatuli so šírkou~$w$
a~výškou~$h$ v~nej bude zodpovedať bod so~súradnicami $(w, h)$.

Žiadne dve škatule z~množiny $\mm S= \{(w, h)\: n \le w+h \le n+2\}$ (tri
najdlhšie diagonály, na \obr\ pre $n=7$) nemôžu byť na jednej poličke.
Ak totiž sú $(w, h)$ a~$(w', h')$ dve škatule na jednej poličke, pre
ktoré platí $w<w'$ a~$h<h'$, platí aj~$w+1 \le w'$ a~${h+1} \le h'$
a~navyše musí byť $w+2 \le w'$ alebo $h+2 \le h'$. V~oboch prípadoch
dostaneme sčítaním $w+h+3 \le w'+ h'$. Ak teda $(w, h) \in\mm S$, bude
$w'+ h' \ge {w+h+3} \ge n+3$, čiže $(w', h') \notin\mm S$.
Keďže $|\mm S| = 3n-2$, poličiek musí byť aspoň $3n-2$.
\inspinsp{a66.1}{a66.2}%

Teraz ukážeme, ako rozdeliť škatule na poličky tak, aby stačilo $3n-2$
poličiek. Možný postup pre $n=7$ je zrejmý z~\obr\ (sady do seba vložených
škatúľ sú vyznačené šípkami) a~možno ho priamo zovšeobecniť pre prípad ľubovoľného
$n\ge3$. Aj~vtedy dve škatule $(w, h)$ a~$(w', h')$ dáme na rovnakú poličku práve
vtedy, keď $2 (h'-h) = w'-w$.
Inými slovami, začneme s~"najväčšími" škatuľami $(n-1, h)$, $(n, h)$ pre
$h= 1,2, \dots, n$ a~$(w, n)$ pre $w~= 1, 2, \dots, n-2$ a~každú z~týchto
$3n-2$ škatúľ položíme na zvláštnu poličku. Potom na každej poličke urobíme
nasledujúci algoritmus: Pozrieme sa na poslednú škatuľu položenú na
poličku, nech je to $(w, h)$. Ak $w-2 \ge 1$ a~$h-1 \ge 1$, pridáme na
poličku škatuľu $(w-2, h-1)$ (\tj.~vložíme ju do škatule na
poličke) a~krok algoritmu zopakujeme (inak skončíme). Zadané pravidlá sú pre každú
poličku splnené triviálne. Hodnota $3n-2$ je preto naozaj dosiahnuteľná.

Ostáva vysvetliť, prečo je každá z~$n^2$ škatúľ uložená na nejakej
poličke. Pre danú škatuľu $(w,h)$ určíme poličku, na ktorej sa ocitne,
postupným súčasným zväčšovaním $w$ o~2 a~$h$ o~1, až nakoniec dostaneme
jednu z~$3n-2$ najväčších škatúľ so~šírkou $w\in\{n-1,n\}$ či výškou
$h=n$. Pri tomto postupe zväčšovania dostaneme podľa našej
procedúry postupnosť do seba vložených škatúľ uložených na poličke,
ktorú sme na úvod priradili spomenutej najväčšej škatuli.
Počet $3n-2$ škatúľ je preto naozaj postačujúci.

\poznamka
Je zaujímavé, že existujú aj iné vyhovujúce uloženia všetkých $n^2$ škatúľ
s~rovnakou sadou $3n-2$ najväčších škatúľ na jednotlivých poličkách,
ktorú sme využili v~našom riešení~--
pre $n=7$ je jedno také odlišné uloženie znázornené na \obr.
\insp{a66.3}%


\návody
Dve políčka na šachovnici nazveme susedné, ak majú spoločnú stranu.
Koľko najviac políčok možno vybrať na šachovnici $2n \times 2n$ tak, aby
žiadne dve z~vybraných políčok neboli susedné?
[Polovicu. Celá šachovnica sa dá rozdeliť na disjunktné dvojice
susedných políčok a~z~každej dvojice môžeme vybrať maximálne jedno políčko.
Polovica sa dá dosiahnuť napr. výberom všetkých políčok jednej farby pri
klasickom ofarbení šachovnice.]

Vyriešte zadanú úlohu pre $n = 4$ a~$n = 5$. Nestačí uhádnuť minimálny
počet poličiek: treba jednak ukázať, že je možné škatule na určený
počet poličiek uložiť, jednak zdôvodniť, že menší počet poličiek nebude
stačit.

Pre $n \in \{3,4\}$ nájdite čo najväčšiu sadu škatúľ takú, že žiadna
z~nich sa nedá vložiť do inej. Skúste svoju sadu zovšeobecniť pre väčšie~$n$.

\medskip\everypar{}
Výborným doplňujúcim materiálom (vhodným aj pre samostatnú prípravu
žiakov) je kapitola 3.2 v~Zbierke KMS dostupnej na stránke {\tt
http://kms.sk/\allowbreak zbierka}.
\endnávod
}

{%%%%%   A-I-3
Body $K$, $L$, $M$ sú definované ako päty výšok a~my
potrebujeme vyjadriť kolmosť tak, aby sa s~ňou dobre pracovalo; keďže
dokazované tvrdenie pracuje s~dĺžkami úsečiek, a~nie s~uhlami, zapíšeme určujúcu vlastnosť bodov $K$, $L$, $M$ pomocou dĺžok úsečiek.

Keďže priamky $AB$ a~$CM$ sú na seba
kolmé,
je $|AC|^2-|BC|^2 = |AM|^2-|BM|^2$. (To vyplýva
z~Pytagorovej vety použitej na pravouhlé trojuholníky $AMC$ a~$BMC$,
lebo $|AC|^2=|AM|^2+|CM|^2$, $|BC|^2 = |BM|^2+|CM|^2$.)
Z~tejto rovnosti hneď vyplýva, že pri štandardnom označení dĺžok strán
trojuholníka $ABC$ platí
$$
|AM|-|BM| =\frac{b^2-a^2}{|AM|+|BM|}= {b^2-a^2 \over c}
$$
a~analogicky
$$
|BK|-|CK| = {c^2-b^2 \over a} \qquad \text{a} \qquad |CL|-|AL| = {a^2-c^2 \over b}.
$$
Rovnosť zo zadania sa teda dá prepísať ako
$$
\align
{a^2-b^2 \over c}+{b^2-c^2 \over a}+{c^2-a^2 \over b} =& 0, \\
ab (a^2-b^2)+bc (b^2-c^2)+ca (c^2-a^2) =& 0.
\endalign
$$

Pokúsme sa výraz na ľavej strane rozložiť na súčin. Budeme sa naň
pozerať ako na kubický mnohočlen~$P$ v~premennej~$a$.
Je jednoduché overiť, že pre rovnoramenný trojuholník je ľavá strana nulová,
preto $P (b) = 0$ a~$P (c) = 0$.
Poznáme teda dva korene mnohočlena~$P$. Po vydelení $P(a)$
zodpovedajúcimi koreňovými činiteľmi (teda výrazom $(a-b)(a-c)$)
už zostane iba lineárny mnohočlen $a(b-c)+b^2-c^2$, ktorý ľahko
rozložíme na súčin. Keď to všetko zhrnieme, vidíme, že rovnosť zo zadania je ekvivalentná s~rovnosťou
$$
(a-b)(b-c)(c-a)(a+b+c) = 0.
$$
Je jasné, že získaná rovnosť platí práve vtedy, keď je trojuholník $ABC$ rovnoramenný.


\ineriesenie
Predpokladajme najskôr, že trojuholník $ABC$ je rovnoramenný, že teda
napríklad bez ujmy na všeobecnosti $|AC| = |BC|$. Z~osovej súmernosti
trojuholníka~$ABC$ potom vyplývajú rovnosti $|AM| = |BM|$, $|AL| = |BK|$
a~$|CL| = |CK|$, takže rovnosť zo zadania je splnená.

Predpokladajme teraz naopak, že platí
rovnosť zo zadania. Prepíšeme ju na tvar
$$
(|AM|-|BM|)+(|BK|-|CK|)+(|CL|-|AL|) = 0. \tag1
$$
Pri štandardnom označení dĺžok strán a~vnútorných
uhlov trojuholníka~$ABC$ s~polomerom~$r$ kružnice opísanej platí
$$
|AM|-|BM| = b \cos \al-a~\cos \be
= 2r \sin \be \cos \al-2r \sin \al \cos \be =
2r \sin (\beta- \al),
$$
a~preto z~rovnosti \thetag1 vyplýva
$$
\sin (\beta-\al)+\sin (\ga- \be)+\sin (\al- \be) = 0.\tag2
$$

Vďaka goniometrickému pravidlu
$$
x+y+z~= 0\quad \Rightarrow\quad \sin x+\sin y+\sin
z~= -4 \sin \frac {x} {2} \sin \frac {y} {2} \sin \frac {z} {2},\tag3
$$
ktoré za okamih dokážeme, z~rovnosti \thetag2 vyplýva
$$
\sin \frac {\be-\al} {2} \sin \frac {\ga- \be} {2} \sin \frac {\al- \be} {2} = 0.
$$
Taká rovnosť zrejme nastane, len keď sa niektoré dva
z~vnútorných uhlov $\al$, $\be$, $\ga$ trojuholníka~$ABC$ rovnajú, čiže trojuholník $ABC$ je rovnoramenný.

Ostáva teda overiť pravidlo \thetag3. Za predpokladu $x+y+z= 0$
platí
$$
\postdisplaypenalty 10000
\align
\sin x+\sin y+\sin z=& 2 \sin \frac {x+y} {2} \cos \frac {x-y} {2}-\sin (x+y) = \\
=&2 \sin \frac {x+y} {2} \cos \frac {x-y} {2}-2 \sin \frac {x+y} {2} \cos \frac {x+y} {2} =\\
=&2 \sin \frac {x+y} {2} \Bigl (\cos \frac {x-y} {2}-\cos \frac {x+y} {2} \Bigr) = \\
=&2 \sin \frac {x+y} {2} \cdot (-2) \cdot \sin \frac {(x-y)+(x+y)} {4} \sin \frac {(x-y)-(x+y)} {4} = \\
=&{-4}\sin \frac {\m z} {2} \sin \frac {x} {2} \sin \frac {-y} {2} =
-4 \sin \frac {x} {2} \sin \frac {y} {2} \sin \frac {z} {2}.
\endalign
$$
Dôkaz je hotový.

\ineriesenie
Ešte jedným algebraickým postupom ukážeme, že za predpokladu
$$
|AM|+|BK|+|CL|=|AL|+|BM|+|CK|
\tag1
$$
je trojuholník $ABC$ rovnoramenný. Rovnako ako v~prvom riešení na to
najskôr využijeme rovnosti
$$
\aligned
|AM|^2-|BM|^2&=|AC|^2-|BC|^2,\\
|BK|^2-|CK|^2&=|AB|^2-|AC|^2,\\
|CL|^2-|AL|^2&=|BC|^2-|AB|^2,
\endaligned
$$
ktorých sčítaním dostaneme po úprave
$$
|AM|^2+|BK|^2+|CL|^2=|AL|^2+|BM|^2+|CK|^2.
\tag2
$$
Keďže výšky trojuholníka prechádzajú jedným bodom, platí podľa Cevovej
vety tiež rovnosť
$$
|AM|\cdot|BK|\cdot|CL|=|AL|\cdot|BM|\cdot|CK|.
\tag3$$
Povšimneme si ešte, že vďaka algebraickej identite
$$
(u+v+w)^2=u^2+v^2+w^2+2(uv+vw+wu)
$$
vyplýva z~(1) a~(2) posledná potrebná rovnosť
$$
|AM|\cdot|BK|+|BK|\cdot|CL|+|CL|\cdot|AM|=
|AL|\cdot|BM|+|BM|\cdot|CK|+|CK|\cdot|AL|.
\tag4
$$

Teraz z~rovností (1), (3), (4) učiníme rozhodujúci algebraický
záver: keďže kubická rovnica s~trojicou koreňov
$$
|AM|,\ |BK|,\ |CL|
$$
je zároveň kubickou rovnicou s~trojicou koreňov
$$
|AL|,\ |BM|,\ |CK|,
$$
sú obe trojice koreňov rovnaké (až na poradie, v~akom sme
korene vypísali).

Z~predpokladu (1) sme tak odvodili, že úsečka~$AM$ je zhodná
s~jednou z~úsečiek $AL$, $BM$, $CK$. Je zrejmé, že tieto
tri možnosti nastanú práve v~prípadoch, keď platí
$|AB|=|AC|$, $|AC|=|BC|$, resp. $|AB|=|BC|$.

\návody
Dokážte, že priamky $AB$ a~$CD$ ležiace v~jednej rovine sú na seba
kolmé práve vtedy, keď platí $|AC|^2-|BC|^2 = |AD|^2- |BD|^2$.
[Stačí dokázať, že ak $A$, $B$ sú dva rôzne body
v~rovine a~$k$ je reálne číslo, je množinou bodov~$X$ takých, že
$|AX|^2- |BX|^2 = k$, priamka kolmá na priamku~$AB$.
Na dôkaz tohto tvrdenia stačí uvažovať pätu~$P$ kolmice z~bodu~$X$ na
priamku~$AB$ a~použiť Pytagorovu vetu v~trojuholníkoch $APX$ a~$BPX$;
dostaneme $|AX|^2- |BX|^2 = (|AP|^2+|PX|^2)-(|BP|^2+|PX|^2) =
|AP|^2- |BP|^2$.
Ak je hodnota $|AX|^2- |BX|^2$ konštantná, je päta kolmice z~$X$ na
$AB$ vždy rovnaká pre všetky body~$X$, leží teda na priamke kolmej
na $AB$.
Otočením úvahy hneď vidíme, že každý bod tejto priamky má požadovanú
vlastnosť.]

Dokážte, že pre ľubovoľný mnohočlen $P(x)$ a~ľubovoľné reálne číslo~$r$ platí nasledujúce tvrdenie: ak $P(r) = 0$, je mnohočlen $P(x)$ deliteľný dvojčlenom $x-r$.
[Po vydelení $P (x)$ činiteľom $x-r$ dostaneme podiel $Q (x)$ a~zvyšok
$R (x)$, ktorý má menší stupeň ako $x-r$, musí to teda byť konštantný
mnohočlen.
Po dosadení $x = r$ do rovnosti $P (x) = (x-r) Q (x)+R (x)$ máme $R (r) =0$,
čiže zvyšok je nulový mnohočlen.]

Rozložte na súčin výraz $bc^2-b^2c+ca^2-c^2a+ab^2-a^2b$. [$(a-b)(b-c)(c-a)$. Považujte
daný výraz za mnohočlen v~premennej~$a$ a~všimnite si, že $b$ aj~$c$ sú jeho korene. Alebo trochu prácnejšie: $bc^2-b^2c+ca^2-c^2a+ab^2-a^2b=bc({c-b})+ca(a-c)+ab(b-a)
=bc(c-b)-abc+abc+ca(a-c)+ab(b-a)=bc\*(c-b-a)+ca(b+a-c)+ab(b-a)=c(b+a-c)(a-b)+ab(b-a)
={(a-b)(bc-c^2+ca-ab)}=(a-b)(b-c)(c-a)$.]

\D
Vnútri strany $AB$ ľubovoľného trojuholníka $ABC$ leží bod~$D$.
Dokážte, že platí
$$
|AB| \cdot |CD|^2+|AB| \cdot |AD| \cdot |BD|
= |BC|^2 \cdot |AD|+|AC|^2 \cdot |BD|.
$$
[Tento fakt je známy ako Stewartova veta.
Pre trojuholníky $ADC$, $BDC$ zapíšte kosínusovú vetu s~uhlami $ADC$, $BDC$ a~využite na elimináciu
ich kosínusov to, že sú to dve navzájom opačné čísla.]

Dokážte, že ak pre reálne čísla $a$, $b$, $c$ platí $a^2 b^4+b^2c^4+
c^2a^4 = a^4b^2+b^4c^2+c^4a^2$, majú niektoré dve z~čísel $a$, $b$, $c$
rovnaké absolútne hodnoty. [$L-P=(a^2-b^2)(b^2-c^2)\*(c^2-a^2)$]
\endnávod
}

{%%%%%   A-I-4
Ukážeme, že jediným riešením je funkcia $f$ taká, že
$$
f (m) = \cases
p,&\hbox {ak $m$ je netriviálnou mocninou prvočísla~$p$, \tj. $m=p^k$, $k\ge1$}, \\
1 &\hbox {v~ostatných prípadoch}.
\endcases
$$

Číslo $1$ má jediného prirodzeného deliteľa $1$, preto zo zadanej rovnosti pre $m=1$ vyplýva $f (1) = 1$.

Nech $m = p$ je prvočíslo. Potom musí platiť
$$
f (1) \cdot f (p) = p, \quad \hbox {čiže} \quad f (p) = p.
$$
Pre $m = p^2$ teda dostaneme
$$
f (1) \cdot f (p) \cdot f (p^2) = p^2, \quad \hbox {čiže} \quad f (p^2) = p
$$
a~všeobecne pre prirodzené $k>1$ a~$m=p^k$
$$
f (1) \cdot f (p) \cdot f (p^2) \cdot \ldots \cdot f (p^k)=p^k.
$$
Matematickou indukciou podľa $k$ tak ľahko ukážeme, že pre
všetky prirodzené čísla~$k$ musí platiť $f (p^k)= p$.

Uvažujme teraz
prirodzené číslo~$m$ s~aspoň dvoma
rôznymi prvočíselnými deliteľmi, ktorého prvočíselný rozklad je
$m = p_1^{\alpha_1} p_2^{\alpha_2} \dots p_k^{\alpha_k}$, pričom $k\ge2$
a~$\a_i\ge1$ pre každé $i~\in \{1, 2, \dots, k\}$.
Medzi deliteľmi čísla~$m$ sú určite všetky mocniny
$$
p_1,\ p_1^2,\ \dots,\ p_1^{\alpha_1},\ \
p_2,\ p_2^2,\ \dots,\ p_2^{\alpha_2},\ \ \dots,\ \
p_k,\ p_k^2,\ \dots,\ p_k^{\alpha_k}
$$
jeho jednotlivých prvočiniteľov, pritom len samotný súčin prislúchajúcich
funkčných hodnôt
$$
\align
&f(p_1) f(p_1^2) \dots f(p_1^{\alpha_1})
f(p_2) f(p_2^2) \dots f(p_2^{\alpha_2}) \dots
f(p_k) f(p_k^2) \dots f(p_k^{\alpha_k})=\\
&=\underbrace{p_1p_1\dots p_1}_{\a_1}
\underbrace{p_2p_2\dots p_2}_{\a_2}\dots
\underbrace{p_kp_k\dots p_k}_{\a_k}
=p_1^{\alpha_1} p_2^{\alpha_2} \dots p_k^{\alpha_k}=m
\endalign
$$
dáva, ako vidíme, hodnotu $m$. To znamená, že súčin všetkých ďalších funkčných
hodnôt zvyšných deliteľov (z~ktorých ani jeden nie je netriviálnou mocninou prvočísla)
vrátane hodnoty $f(m)$ musí byť rovný~1, takže všetky činitele nevynímajúc $f(m)$ sa rovnajú~1.

Tým je hľadaná funkcie~$f$ jednoznačne určená a~zároveň je z~poslednej rovnosti
zrejmé, že má požadované vlastnosti.


\návody
Dokážte matematickou indukciou, že každé prirodzené číslo väčšie ako
$1$ sa dá rozložiť na súčin prvočísel.

Určte hodnoty funkcie~$f$, ktorá spĺňa podmienku zo zadania, pre
mocniny trojky.

Nájdite všetky funkcie $f \: \Bbb N \to \Bbb N$, ktoré majú pre každé
prirodzené číslo~$m$ nasledujúcu vlastnosť:
ak označíme $d_1, d_2, \dots, d_n$ všetky delitele čísla~$m$, platí
$$
f (d_1) \cdot f (d_2) \cdot \ldots \cdot f (d_n) = 2^n.
$$
[Matematickou indukciou možno dokázať, že $f (x) = 2$ pre každé prirodzené číslo~$x$.
Ak označíme delitele čísla~$m$ vzostupne podľa veľkosti $1 = d_1 <
d_2 <\dots <d_n = m$, máme
z~indukčného predpokladu $2^{n-1} f (m) = 2^n$ čiže $f (m)= 2$.]

\D
Určte všetky funkcie $f\colon\Bbb Z \to \Bbb Z$ také, že pre všetky celé
čísla $x$, $y$ platí
$$
f\bigl(f(x)+y\bigr)=x+f(y+2\,006).
$$
[56--A--I--6]

Označme $\Bbb N$ množinu všetkých prirodzených čísel a~uvažujme všetky
funkcie \hbox{$f:\Bbb N\to \Bbb N$} také, že pre ľubovoľné $x,y\in \Bbb N$ platí
$$
f\bigl(xf(y)\bigr)=yf(x).
$$
Určte najmenšiu možnú hodnotu $f(2\,007)$.
[56--A--III--3]
\endnávod
}

{%%%%%   A-I-5
Označme $S$ stred kružnice opísanej trojuholníku $ABF$ a~$K$, $L$, $M$
päty kolmíc z~bodu $S$ na priamky $AB$, $BF$, $BE$.
Ako ľahko overíme,
uhol $ABF$ je tupý, bod~$S$ leží na osi~$CK$ strany~$AB$ vnútri polroviny
opačnej k~$CEB$ a~bod~$E$ leží vnútri úsečky~$SL$, takže je vnútorným
bodom úsečky~$BM$ (\obr).
Keďže bod~$M$ je stredom uvažovanej tetivy a~$|AC|=|BC|$, stačí ukázať,
že $|BM| = |BC|$.
\insp{a66.4}%

Označme $\alpha$ a~$\beta$ veľkosti uhlov pri základniach
rovnoramenných trojuholníkov $ABC$ a~$BFE$.
Keďže $|\angle BDF| = \alpha$, z~trojuholníka $DBF$ máme
$$
|\angle CBM| = 180^\circ- 2(\alpha+\beta). \tag1
$$
Z~rovnobežnosti $CE\parallel AB$ ďalej vyplýva, že aj $|\angle BCE| = \alpha$,
a~zo zrejmej podobnosti pravouhlých trojuholníkov $SEM\sim BEL$ máme $|\angle ESM| = \beta$.
A~keďže $CE$ je rovnako ako $AB$ kolmé na~$SK$, je štvoruholník $CEMS$ tetivový, takže
$|\angle ECM| = |\angle ESM| =\beta$.
Spolu teda pre veľkosti uhlov $BCM$ a~$BMC$ dostávame
$$
|\angle BCM|= |\angle BCE| + |\angle ECM| = \alpha+\beta
$$
a~podľa~(1) môžeme dopočítať
$$
|\angle BMC|= 180^\circ-|\angle CBM|-|\angle BCM| =2(\alpha+\beta)-(\alpha+\beta) = \alpha+\beta.
$$
v~trojuholníku $BCM$ teda platí $|\angle BMC| = |\angle BCM|$. Preto $|BM| = |BC|$.
\insp{a66.5}%

\ineriesenie
Využijeme súmernosť podľa osi~$o$ úsečky~$BF$. Na
priamke~$o$ leží bod~$E$ aj~stred~$S$ kružnice~$k$ opísanej trojuholníku $ABF$.
Tetiva, ktorú vytína priamka~$BE$ v~kružnici~$k$, sa v~uvedenej
osovej súmernosti zobrazí na úsečku $FH$, pričom $H$ je priesečník priamky
$FE$ s~kružnicou~$k$ rôzny od~$F$ (\obr).

Keďže $|DE|=|AC|$, treba dokázať, že $|HD|+|DF| = 2|AC|$.

Označme $|AC| = |BC| = a$, $|AB| = c$, $|AD| = d$, $|CD| = b$.

Štvoruholník $DBEC$ je vďaka zhodným uhlom $CBD$ a~$EDB$ osovo súmerný,
je to teda rovnoramenný lichobežník alebo pravouholník,
preto platí aj $|BE|=|CD|=b$ a
$$
\align
|DF| = &|DE|+|EF| = |AC|+|BE|=\\ =&|AC|+|CD| = a+b.\tag1
\endalign
$$
Treba teda dokázať, že
$$
|HD|=2|AC|-|DF| = 2a-(a+b)=a-b. \tag2
$$

Z~mocnosti bodu~$D$ ku kružnici~$k$ vyplýva
$$
|HD| \cdot |DF| = |AD| \cdot |BD|=d(c-d),
$$
čo je podľa (1) ekvivalentné rovnosti
$$
|HD|={d(c-d)\over a+b}. \tag3
$$
Porovnaním s~\thetag2 tak dostávame, že má platiť
$$
d (c-d) = a^2-b^2. \tag4
$$

Pre $D=K$, pričom $K$ označuje stred úsečky~$AB$ (a~zároveň pätu kolmice
z~bodu~$C$ na~$AB$), je $c=2d$ a posledná rovnosť sa tak zmení
na~Pytagorovu vetu pre trojuholník $AKC$. Pre $D\ne K$ potom z~Pytagorovej vety
použitej na trojuholníky~$DKC$ a~$AKC$ vyplýva
$$
\postdisplaypenalty 10000
\align
a^2-b^2&= (|AK|^2+|KC|^2)-(|DK|^2+|KC|^2) = |AK|^2-|DK|^2= \\
&= \bigl||AK|-|DK|\bigr|(|AK|+|DK|) = d(c-d),
\endalign
$$
bez ohľadu na polohu bodu~$D$ vnútri~$AB$.
Tým je dokázaná rovnosť~(4), teda je dokázané aj~tvrdenie úlohy.

\poznamka
Rovnosť \thetag4 je priamym dôsledkom Ptolemaiovej vety
použitej v~tetivovom lichobežníku $DBEC$, ktorého obe uhlopriečky
majú dĺžku~$a$, ramená dĺžku~$b$ a~základne dĺžky $c-d$ a~$d$.

\ineriesenie
Tak ako v~predošlom riešení využijeme bod~$H$ a~dokážeme, že
$|HF|=2|AC|$. Pridáme však
ďalšie dva body: bod~$I$, ktorý je obrazom bodu~$A$ v~stredovej súmernosti
podľa~$C$, a~bod~$J$, ktorý je stredom úsečky $BI$. Body $C$, $E$, $J$
sú kolineárne, pretože úsečka~$CJ$ je strednou priečkou v~trojuholníku
$ABI$, a~teda rovnobežná s~úsečkou~$AB$ (\obr).
Dokážeme, že $AHFI$ je rovnobežník; z~toho hneď vyplynie dokazované
tvrdenie. Keďže $AI \parallel HF$, stačí dokázať, že $AH \parallel IF$.
\insp{a66.6}%

Keďže $|CA| = |CB| = |CI|$, leží bod~$B$ na Tálesovej kružnici
nad priemerom~$AI$, takže uhol $ABI$ je pravý.
Bod~$E$ leží na priesečníku osí úsečiek $BF$ a~$BI$, preto je stredom
kružnice opísanej trojuholníku $BFI$. Pre uhol $FIB$ nad tetivou~$BF$
tejto kružnice tak máme $|\angle FIB| =\tfrac12 |\angle FEB|$
(využívame, že bod~$F$ leží v~polrovine
opačnej k~$BIA$, čo platí vďaka tomu, že os~$SE$ úsečky~$BF$ pretína
polpriamku opačnú k~$BA$). Platí teda
$$
\align
|\angle FIA| & = |\angle BIA|+|\angle FIB| = (90^\circ-|\angle IAB|)+\tfrac12 |\angle FEB| = \\
& = 90^\circ-|\angle IAB|+\tfrac12(180^\circ-2 |\angle BFE|) = 180^\circ-|\angle IAB|-|\angle BFH| = \\
& = 180^\circ-|\angle IAB|-|\angle BAH| = 180^\circ-|\angle IAH|,
\endalign
$$
čiže $AH \parallel IF$. Tým je tvrdenie úlohy dokázané.

\návody
Pomocou počítania veľkostí uhlov dokážte, že výšky v~ostrouhlom
trojuholníku $ABC$ sa pretínajú v~jednom bode. [Označme postupne $D$ a~$E$
päty výšok z~vrcholov $A$ a~$B$, ďalej $P$ priesečník úsečiek $AD$
a~$BE$ a~$X$ priesečník $CP$ a~$AB$. Dokážeme, že priamka~$CP$ je kolmá
na $AB$. Štvoruholníky $ABDE$ a~$CDPE$ sú tetivové, pretože ich vrcholy
ležia na Tálesových kružniciach s~priemermi $AB$ a~$CP$. Preto uhly
$BAD$, $BED$, $PCD$ majú všetky rovnakú veľkosť $90^\circ-|\uhol ABC|$. Uhol
$CXB$, ktorý dopočítame zo známych veľkostí zvyšných uhlov
v~trojuholníku $CXB$, je teda pravý.]

Daný je ostrouhlý trojuholník $ABC$ s~pätami výšok $D$, $E$, $F$ ležiacimi postupne na stranách $AB$, $BC$, $CA$.
Obraz bodu $F$ v~stredovej súmernosti podľa stredu strany~$AB$ leží na priamke~$DE$. Určte veľkosť uhla $BAC$.
[57--A--II--3]

Daný je ostrouhlý trojuholník $ABC$ taký, že $|AC|\ne |BC|$.
Vnútri jeho strán $BC$ a~$AC$ uvažujme body $D$ a~$E$, pre ktoré je
$ABDE$ tetivový štvoruholník. Priesečník jeho uhlopriečok $AD$ a~$BE$
označme~$P$. Dokážte, že ak sú priamky $CP$ a~$AB$ navzájom kolmé, tak $P$
je priesečníkom výšok trojuholníka $ABC$.
[\hbox{56--A--III--5};
Označme $M$ pätu výšky z~vrcholu~$C$; bod~$P$ leží na
úsečke~$CM$. Uvažujme úsečku~$B'C$, ktorá je obrazom úsečky~$BC$
v~osovej súmernosti s~osou~$CM$. Uhly $CBP$, $CB'P$ sú vďaka symetrii
zhodné. Body $A$, $B$, $D$, $E$ ležia podľa zadania na kružnici, preto
uhly $CAP$ a~$CBP$ sú zhodné. Z~bodov $A$ a~$B'$ je vidno úsečku~$CP$ pod
rovnakým uhlom, navyše sú v~rovnakej polrovine vzhľadom na~priamku~$CP$, a~teda $PCAB'$ je tetivový štvoruholník. Preto uhly $B'AP$,
$B'CP$, $BCP$ majú všetky zhodnú veľkosť $90^\circ-\beta$. Ostáva
dopočítať veľkosti uhlov v~trojuholníku $ADB$ a~vidíme, že uhol $ADB$
je pravý, preto $P$ je priesečník výšok. Kľúčová idea tohto riešenia~--
využitie vhodnej osovej súmernosti~-- je veľmi užitočná aj~v~súťažnej úlohe.]

Daný je trojuholník $ABC$ a~vnútri neho bod~$P$. Označme $X$ priesečník
priamky~$AP$ so stranou~$BC$ a~$Y$ priesečník priamky~$BP$ so stranou~$AC$.
Dokážte, že štvoruholník $ABXY$ je tetivový práve vtedy, keď druhý
priesečník (rôzny od bodu~$C$) kružníc opísaných trojuholníkom $ACX$ a~$BCY$ leží na
priamke~$CP$.
[55--A--II--3]

Vnútri strany~$BC$ ostrouhlého trojuholníka $ABC$ zvoľme bod~$D$ a~na úsečke~$AD$ bod~$P$ tak, aby neležal na ťažnici z~vrcholu~$C$. Priamka tejto ťažnice pretne kružnicu opísanú trojuholníku $CPD$ v~bode, ktorý označíme~$K$ ($K\ne C$).
Dokážte, že kružnica opísaná trojuholníku $AKP$ prechádza okrem bodu~$A$ ďalším pevným bodom, ktorý od výberu bodov $D$ a~$P$ nezávisí.
[58--A--II--4]

\D
V~tetivovom štvoruholníku $ABCD$ označme $L$, $M$ stredy kružníc
vpísaných postupne do trojuholníkov $BCA$, $BCD$. Ďalej označme $R$
priesečník kolmíc vedených z~bodov $L$ a~$M$ postupne na priamky $AC$
a~$BD$. Dokážte, že trojuholník $LMR$ je rovnoramenný.
[56--A--III--2]

V~rovine, v~ktorej je daná úsečka~$AB$, uvažujme trojuholníky $XYZ$
také, že $X$ je vnútorným bodom úsečky~$AB$, trojuholníky $XBY$ a~$XZA$
sú podobné ($\triangle XBY\sim\triangle XZA$)
a~body $A$, $B$, $Y$, $Z$ ležia v~tomto poradí na
kružnici. Nájdite množinu stredov všetkých úsečiek~$YZ$.
[63--A--III--2]

V~ostrouhlom trojuholníku $ABC$, v~ktorom $|AC|\ne|BC|$, označme $D$ a~$E$ päty výšok z~vrcholov $A$ a~$B$. Nech $V$ je priesečník výšok trojuholníka $ABC$, bod~$F$ je priesečník priamok $AB$ a~$DE$ a~bod~$S$ je stred strany~$AB$. Ďalej nech $K$ je priesečník kružníc opísaných trojuholníkom $BDS$ a~$AES$ rôzny od bodu~$S$.
\item{a)} Dokážte, že body $D$, $E$, $V$, $K$ ležia na jednej kružnici.
\item{b)} Dokážte, že body $F$, $V$, $K$ ležia na jednej priamke.
\endgraf\indent
[57--A--III--2]
\medskip\everypar{}
Výborným materiálom (vhodným aj pre samostatnú prípravu žiakov) sú
kapitoly 2.1 a~2.2 v Zbierke KMS dostupnej na stránke {\tt http://kms.sk/zbierka}.
\endnávod

}

{%%%%%   A-I-6
Odčítaním tretej rovnice danej sústavy od prvej rovnice dostaneme
$$
z^2-x^2 = (z-x)(z+x) = y-u.
\tag1
$$
Podobne z~druhej a~štvrtej rovnice danej sústavy vyjde
$$
y^2-u^2 = (y-u)(y+u) = x-z.
\tag2
$$
Zo vzťahov (1) a~(2) potom vyplýva $x = z$
práve vtedy, keď $y = u$. Preto ďalej rozlíšime dva prípady označené
ako a) a~b).

\smallskip
a) Predpokladajme najskôr, že $x = z$ a~$ y = u$, \tj. riešenie danej
sústavy rovníc budeme hľadať v~tvare usporiadaných štvoríc
$(x, y, z, u) = (x, y, x, y)$ s~neznámymi $x$~a~$y$. V~tomto prípade môžeme
pôvodnú sústavu rovníc redukovať na sústavu dvoch rovníc
s~neznámymi $x$, $y$ v~tvare
$$
\align
k-x^2=&y, \\
k-y^2=&x.
\endalign
$$
Odčítaním jednej rovnice od druhej dôjdeme po jednoduchej úprave
k~nasledujúcej rovnici v~súčinovom tvare:
$$
\postdisplaypenalty 10000
(y-x)(y+x-1) = 0.
$$
Takže nastáva jedna z~dvoch
(ako sa ukáže, nie úplne disjunktívnych) možností.

Pri prvej možnosti $y-x=0$ prechádza
redukovaná sústava po dosadení $y=x$ na jedinú kvadratickú rovnicu
$$
x^2+x-k= 0.
$$
Táto rovnica má pre ľubovoľnú
hodnotu parametra $k\in\langle 0, 1 \rangle$ (ako je vymedzené
v~zadaní) dva rôzne reálne korene,
a~to
$$
x_{1,2} = \frac {-1\pm\sqrt{4k+1}}{2}.
$$
Zadaná úloha má preto najmenej dve riešenia
$$
x_1 = y_1 = z_1 = u_1 = \frac{\m1+\sqrt{4k+1}}{2},\quad
x_2 = y_2 = z_2 = u_2 = \frac{\m1-\sqrt{4k+1}}{2}.
\tag3
$$

Pri druhej možnosti $x+y-1=0$ sa dosadením $y =1-x$ dostaneme
od redukovanej sústavy ku kvadratickej rovnici
$$
x^2-x+(1-k) = 0.
$$
Keďže jej diskriminant je rovný $4k-3$,
má uvedená rovnica reálne korene
iba v~prípade, že $k\ge 3/4$. Týmito koreňmi sú reálne čísla
$$
x_3 = \frac{1+\sqrt{4k-3}}{2}\quad\text{a} \quad
x_4 = \frac{1- \sqrt{4k-3}}{2}.
$$
Zodpovedajúce hodnoty $y=1-x$ sú potom
$$
y_3 = \frac{1- \sqrt {4k-3}}2\quad\text{a} \quad
y_4 = \frac{1+ \sqrt {4k-3}}2.
$$

Pre najmenšiu hodnotu $k=3/4$ však platí $x_3=y_3=x_4=y_4=1/2$, takže žiadne
nové riešenie nezahrnuté do rozboru prvej možnosti $x=y$ nedostávame~--
prichádzame znova k~prvému riešeniu z~(3), ktoré samozrejme
zodpovedá parametru $k=3/4$. Pre ostatné možné hodnoty parametra~$k$ dané odvodenou podmienkou $k>3/4$ a~obmedzením $k\leqq1$
má zadaná úloha ďalšie dve riešenia
$$
(x, y, z, u) = (x_3, y_3, x_3, y_3)\quad\text{a}\quad
(x, y, z, u) = (x_4, y_4, x_4, y_4).
\tag4
$$
Tieto dve riešenia sú vďaka nerovnostiam $x_3\ne x_4$,
$x_3\ne y_3$ a~$x_4\ne y_4$ naozaj navzájom rôzne
a~odlišné od oboch riešení z~(3).

\smallskip
b) Teraz predpokladajme, že $x\ne z$ a~$ y \ne u$.
V~takom prípade môžeme po dosadení $x-z$ zo vzťahu (2)
do ľavej strany rovnice (1) následne vydeliť obe jej strany
nenulovým číslom $y-u$ a~získať tak rovnicu
$$
(x+z)(y+u) =-1.
\tag5
$$
Využijeme ju ďalej k~jedinému zrejmému záveru: {\sl všetky štyri
čísla $x$, $y$, $z$, $u$ nemôžu byť záporné} (v~opačnom prípade
by totiž ľavá strana~(5) bola kladná). Preto je {\it niektoré\/} z~čísel
$x$, $y$, $z$, $u$ {\it nezáporné}. Ak však ukážeme, že vďaka zadanému predpokladu
$k\in\langle0,1\rangle$ vyplývajú z~rovníc pôvodnej sústavy implikácie
$$
x\geqq0\ \Rightarrow\
y\geqq0\ \Rightarrow\
z\geqq0\ \Rightarrow\
u\geqq0\ \Rightarrow\
x\geqq0,
\tag6
$$
bude odvodený záver o~nezápornosti {\it niektorého\/} z~čísel $x$, $y$, $z$, $u$
znamenať nezápornosť {\it všetkých štyroch}, a~tak sa dostaneme do sporu s~už
zmieneným dôsledkom rovnice~(5). V~prípade~b) tak zadaná úloha
nebude mať žiadne riešenie.

Ostáva dokázať napríklad prvú z~implikácií (6),
dôkazy ostatných troch totiž budú analogické.
Nech teda $x\ge0$. Zo štvrtej rovnice
pôvodnej sústavy vyplýva nerovnosť $x \le k$, takže vzhľadom na
$k\le1$ máme $0\leqq x\leqq k\leqq1$. To však znamená, že
$x^2\leqq k$, pretože $t^2\leqq t$ pre každé
$t\in\langle0,1\rangle$. Z~odvodenej nerovnosti $x^2\leqq k$ už
podľa prvej rovnice sústavy vyplýva $y\geqq0$, ako sme chceli
dokázať. Rozbor prípadu~b) je tak hotový (s~negatívnym záverom).

\zaver
V~prípade $0\leqq k\leqq3/4$ má zadaná sústava práve dve riešenia,
ktoré sú dané vzorcami (3), a~v~prípade $3/4<k\leqq1$
má práve štyri riešenia, ktoré sú určené vzorcami (3) a~(4).

\ineriesenie
Uvažujme funkciu $f(x)=k-x^2$, pomocou ktorej je možné danú sústavu
rovníc zapísať ako $f(x)=y$, $f(y)=z$, $f(z)=u$, $f(u)=x$.
Ak postupne dosadíme do štvrtej rovnice tretiu, druhú a~nakoniec prvú,
dostaneme $f(f(f(f(x))))=x$, čiže $f^4 (x) = x$.
(Symbolom $f^k$ budeme označovať {\it $k$-tu iteráciu\/} funkcie~$f$.
V~úlohe teda hľadáme práve také reálne čísla~$x$,
ktoré sú {\it pevnými bodmi\/} štvrtej iterácie danej funkcie~$f$.)

Vyriešime všeobecnejšiu úlohu: za toho istého predpokladu $0\leqq k\leqq1$ ako
v~súťažnej úlohe nájdeme všetky reálne~$x$ také, že
$f^{2n}(x)=x$ pre dané prirodzené číslo~$n$. Predpokladajme, že
číslo~$x$ túto vlastnosť má, a~poďme o~ňom zistiť viac. Predtým
si však všimnime, že s~číslom~$x$ vyhovuje rovnici
$f^{2n}(x)=x$ zrejme aj každé číslo $f^i(x)$, pričom $i\in\{1,2,3,\dots,2n-1\}$.

Začneme veľmi užitočným pozorovaním:
funkcia~$f$ je rastúca na intervale $({-\infty}, 0\rangle$
a~klesajúca na intervale $\langle0,\infty)$, takže $f(0)$ je jej
maximálna hodnota. Ukážeme, že pre každé hľadané $x$
z~predchádzajúceho odseku platí:

\medskip
\ite(i)
{\sl ak $x\geqq0$, sú čísla $f(x),f^2 (x),\dots,f^{2n-1}(x)$
nezáporné;}
\ite(ii)
{\sl ak $x<0$, sú čísla $f(x), f^2(x),\dots, f^{2n-1}(x)$
záporné.}

\medskip
Nech teda $x\geqq0$. Keďže $x = f^{2n} (x)$, patrí $x$ do oboru hodnôt funkcie~$f$, preto
$0 \le x \le f (0)$. Na intervale $\langle 0, f (0) \rangle$ je funkcia $f$
klesajúca, preto $f (x)\ge f (f (0))$.
Pritom $f(f(0))=f(k)=k-k^2\geqq0$ (pretože $k\in \langle 0,1 \rangle$),
a~teda aj~$f(x)\geqq0$. Opakovaním tejto úvahy dostaneme tvrdenie~(i).

Tvrdenie~(ii) dokážeme sporom. Ak by nebolo pravdivé,
spolu s~$x<0$ by pre nejaké $i\in\{1,2,\dots,2n-1\}$
platilo $f^i(x)\ge 0$. Potom by však podľa dokázaného tvrdenia~(i)~--
s~číslom~$x$ zameneným za číslo $f^i(x)$~--
pre všetky $j\in\{i+1,i+2,\dots,\allowbreak i+2n-1\}$
platilo $f^j (x) \ge 0$, a~to je pre $j = 2n$ spor:
$0>x=f^{2n}(x)\ge 0$.

\smallskip
Teraz už sme na riešenie rovnice $f^{2n} (x) = x$ úplne pripravení.
Kvôli prehľadnosti najskôr uvedieme výpis prípadov, na ktoré celý
postup rozdelíme:

a) $f(x)=x$;\quad b) $f(x)\ne x$ a~$ x<0$;\quad
c) $f(x)\ne x$ a~$ x\geqq0$.

\smallskip
a) Korene kvadratickej rovnice $f(x)=x$ nás (rovnako ako v~prvom
riešení) privedú k~dvom riešeniam $x_{1,2}=\frac12(\m1\pm\sqrt{4k+1})$.

\smallskip
b) Z~predpokladu $x<0$ podľa tvrdenia (ii)
vyplýva, že všetky čísla $f^i(x)$ sú
záporné. Keďže funkcia~$f$ je na $({-\infty},0)$ rastúca
a~predpoklad $f(x)\ne x$ znamená, že buď $f(x)<x$ alebo $f(x)>x$,
v~prípade $f(x)<x$ dostaneme postupne $f^2(x)<f(x)$,
$f^3(x)<f^2(x)$, atď. až $f^{2n}(x)<f^{2n-1}(x)$, teda spolu
$$
x>f(x)>f^2(x)>f^3(x)>\dots>f^{2n}(x),
$$
zatiaľ čo v~prípade $f(x)>x$ dostaneme podobne
$$
x<f(x)<f^2(x)<f^3(x)<\dots<f^{2n}(x).
$$
V~oboch prípadoch sme sa dostali ku sporu s~rovnosťou
$f^{2n}(x)=x$, takže v~prípade~b) žiadne riešenie poslednej rovnice
neexistuje.

\smallskip
c) Z~predpokladu $x>0$ podľa tvrdenia (i) vyplýva,
že všetky čísla $f^i(x)$ sú
nezáporné. Keďže funkcia~$f$ je na intervale $\langle0,\infty)$
klesajúca, je predpoklad $f(x)\ne x$ na odvodenie
záverov podobných tým z~prípadu~b) nedostačujúci, budeme k~nim
potrebovať porovnanie čísel $f^2(x)$ a~$x$, rozlíšime preto
tri možnosti $f^2(x)<x$, $f^2(x)=x$ a~$f^2(x)>x$.

Ak $f^2(x)<x$, dostávame postupne $f^3(x)>f(x)$,
$f^4(x)<f^2(x)$, atď. až dôjdeme k~spornému záveru, že
$$
x>f^2(x)>f^4(x)>\dots>f^{2n}(x).
$$
Podobne z~$f^2(x)>x$ odvodíme sporný záver
$$
x<f^2(x)<f^4(x)<\dots<f^{2n}(x).
$$

Ostáva tak jediná možnosť, že
$f^2(x)=x$.\footnote{Zdôraznime, že sme sa v~tomto okamihu
"zbavili" prirodzeného čísla~$n$. Dokázali sme totiž, že každé
nezáporné riešenie rovnice $f^{2n}(x)=x$ musí byť riešením rovnice
$f^2(x)=x$ (ak nie je dokonca samo riešením rovnice $f(x)=x$).}
Mnohočlen $$f^2 (x)-x=k-({k-x^2})^2-x$$ je síce štvrtého stupňa,
pomôže nám však, že poznáme dva z~jeho koreňov: ak totiž platí $f(x)=x$,
platí aj~$f^2 (x)=x$. Keďže korene mnohočlena $f(x)-x=k-x^2-x$,
\tj. čísla $x_{1,2}=\frac12(\m1\pm\sqrt{4k+1})$, sú pre ľubovoľné
$k\in\langle0,1\rangle$ reálne a~navzájom rôzne
(lebo $x_1\geqq0>x_2$), prichádzame k~záveru, že mnohočlen $f^2(x)-x$ musí byť
násobkom mnohočlena $f(x)-x$. Rutinným vydelením
získame potrebný rozklad
$$
\underbrace{k-(k-x^2)^2-x}_{f^2(x)-x}=
\underbrace{(k-x^2-x)}_{f(x)-x}(x^2-x+1-k).
$$

Prípadné nezáporné korene rovnice $f^2(x)=x$ (rôzne od $x_1$) teda
nájdeme riešením kvadratickej rovnice $x^2-x+1-k=0$. Sú to čísla
$x_{3,4}=\frac12(1\pm\sqrt{4k-3})$, a~to iba v~prípade, keď
$3/4<k\leqq1$~-- pozri diskusiu v~prvom riešení, ktorú tu vynecháme.
To sú aj jediné riešenia rovnice $f^{2n}(x)=x$ v~prípade~c).

Zhrnieme výsledky našich úvah: Rovnica $f^{2n}(x)=x$ pre funkciu
$f(x)=k-x^2$ s~parametrom $k\in\langle 0, 1 \rangle$ a~daným prirodzeným číslom~$n$
má v~obore reálnych čísel v~prípade $0\leqq k\leqq3/4$
práve dve riešenia
$x_{1,2}=\frac12(\m1\pm\sqrt{4k+1})$, v~prípade $3/4<k\leqq1$ potom
riešenia práve štyri~-- okrem uvedených $x_{1,2}$ ešte
$x_{3,4}=\frac12(1\pm\sqrt{4k-3})$. (Množina koreňov je tak
nezávislá na stupni $2n$ danej iterácie.\footnote{Dodajme, že
podanú metódu riešenia nemožno uplatniť na rovnicu $f^n(x)=x$, keď je
$n$ nepárne číslo väčšie ako~1 -- vtedy zlyháva podaný rozbor prípadu~c).
Numerické výpočty ukazujú, že
také rovnice majú okrem koreňov $x_{1,2}$
ďalšie kladné korene, ktoré závisia od hodnoty~$n$.})


\návody
Určte všetky hodnoty reálneho parametra~$p$, pre ktoré má rovnica
$p(x^2+x+4) = 2x$ práve jedno reálne riešenie.
[Daná kvadratická rovnica má práve jedno reálne riešenie vtedy, keď je
diskriminant rovný nule, čiže keď $(p-2)^2-16p^2 = 0$. To nastane
pre $p \in \{-2/3, 2/5\}$. Pre $p=0$ dostaneme lineárnu rovnicu
tiež s~jedným riešením.]

V~obore reálnych čísel riešte sústavu rovníc
$$\eqalign{
 ax+y&=2,\cr
  x-y&=2a,\cr
  x+y&=1\cr}
$$
s~neznámymi $x$, $y$ a~reálnym parametrom~$a$.
[58--B--S--1]

V~obore reálnych čísel vyriešte sústavu
$$
\align
\sqrt{x^2+y^2}&=z+1,\\
\sqrt{y^2+z^2}&=x+1,\\
\sqrt{z^2+x^2}&=y+1.
\endalign
$$
[60--B--I--1]

V~obore reálnych čísel riešte sústavu rovníc
$$
\postdisplaypenalty=10000
\align
\sqrt{x^2 - y}&= z - 1,\\
\sqrt{y^2 - z}&= x - 1,\\
\sqrt{z^2 - x}&= y - 1.
\endalign
$$
[59--A--I--1]

V~obore reálnych čísel riešte sústavu rovníc
$$
\align
\sqrt{x-y^2}&=z-1,\\
\sqrt{y-z^2}&=x-1,\\
\sqrt{z-x^2}&=y-1.
\endalign
$$
[59--A--S--1]

Určte všetky trojice $(a,b,c)$ kladných reálnych čísel, ktoré sú riešením sústavy rovníc
$$
\aligned
   a\sqrt{b}-c &= a,\\
   b\sqrt{c}-a &= b,\\
   c\sqrt{a}-b &= c.
\endaligned
$$
[59--CPS--1]

Vyriešte sústavu rovníc $k+x^2 = y$, $k+y^2 = x$ s~reálnym parametrom~$k$.
[Rovnice od seba odčítajte, výslednú rovnosť upravte na súčinový tvar
$(x-y)(x+y+1)=0$ a~rozlíšte, ktorý z~činiteľov je nulový. Pri zhrňovaní výsledkov
nezabudnite, že pre niektoré~$k$ môžu byť oba činitele nulové.]

\D
V~obore reálnych čísel riešte sústavu rovníc
$$\eqalign{
  x+y&=1,\cr
  x-y&=a,\cr
  -4ax+4y&=z^2+4
}$$
s~neznámymi $x$, $y$, $z$ a~reálnym parametrom~$a$.
[58--B--II--1]

V~obore reálnych čísel vyriešte sústavu rovníc
$$
\align
a(b^2 + c) &= c(c + ab),\\
b(c^2 + a) &= a(a + bc),\\
c(a^2 + b) &= b(b + ca).
\endalign
$$
[64--A--III--4]

Určte všetky trojice $(x,y,z)$ kladných reálnych čísel, ktoré sú riešením sústavy rovníc
$$
\eqalign{2x^3&=2y(x^2+1)-(z^2+1),\cr
2y^4&=3z(y^2+1)-2(x^2+1),\cr
2z^5&=4x(z^2+1)-3(y^2+1).\cr}
$$
[57--CPS--1]
\endnávod
}

{%%%%%   B-I-1
Predpokladajme, že niektorým $k$ vrcholom ($0\le k\le66$)
pravidelného 66\spojovnik{}uholníka je priradené číslo~$1$ a~ostatným
$66-k$ vrcholom číslo~${\m1}$. Ku každej úsečke spájajúcej dva
vrcholy s~číslami~$1$ pripíšeme
podľa podmienok úlohy číslo~$1$. Takých úsečiek je spolu $\frac12k(k-1)$.
Podobne aj ku každej úsečke, ktorej oba krajné body
majú priradené číslo~${\m1}$, pripíšeme číslo~$1$. Takých úsečiek je
spolu $\frac12(66-k)(65-k)$.
Napokon každej zo zvyšných úsečiek (a~tých je $k(66-k)$)
pripíšeme číslo~$\m1$.

Hodnota~$S$ súčtu všetkých čísel pripísaných k~jednotlivým úsečkám
(stranám či uhlopriečkam uvažovaného 66-uholníka) je teda
$$
\align
S=&\frac{k(k-1)}2+\frac{(66-k)(65-k)}2-k(66-k)=\\
=&2k^2-2\cdot66k+65\cdot33=\\
=&2(k-33)^2-33.
\endalign
$$
Vidíme, že vždy platí $S\ge\m33$, pričom rovnosť zrejme nastane pre $k=33$.

Teraz zistíme, pre ktoré $k$ ($0\le k\le66$) je splnená nerovnosť $S\ge0$.
Budú to práve tie~$k$, pre ktoré platí $|k-33|\ge\sqrt{33/2}>4$,
takže pre nezáporné hodnoty~$S$ bude
určite platiť $S\ge2\cdot5^2-33=17$. Rovnosť nastane, keď bude
$|k-33|=5$, čiže $k\in\{28, 38\}$.

\zaver
Skúmaný výraz nadobúda najmenšiu hodnotu $S=\m33$ pre $k=33$.
Najmenšia možná nezáporná hodnota skúmaného súčtu je $S=17$
pre $k=28$ či $k=38$.

\návody
Určte najmenšiu celočíselnú hodnotu výrazu
$U=|x-\sqrt{3}|+\sqrt{5}$, pričom $x$ je ľubovoľné reálne číslo.
[Najmenšia celočíselná hodnota výrazu $U$ je $3$ (pre
$x=\sqrt{3}-\sqrt{5}+3$ či pre $x=\sqrt{3}+\sqrt{5}-3$).]

Určte najmenšiu hodnotu výrazu $V=3x^2+4x+5$, pričom $x$ je
ľubovoľné reálne číslo, a~nájdite tiež jeho najmenšiu celočíselnú
hodnotu.
[Najmenšia možná hodnota výrazu $V$ je $\frac{11}3$, je dosiahnutá
pre $x=\m\frac23$. Najmenšia celočíselná hodnota skúmaného výrazu
je $V=4$ a~je dosiahnutá pre $x=\m1$ či $x=\m\frac13$.]

Každému vrcholu pravidelného 63-uholníka priradíme jedno z~čísel $1$ alebo $\m1$.
Ku každej jeho strane
pripíšeme súčin čísel v~jej vrcholoch a~všetky čísla pri jednotlivých stranách
sčítame. Nájdite najmenšiu možnú nezápornú hodnotu takého súčtu.
\hbox{[63--B--I--1]}

\D
Nech pre $x_i \in \{\m1, 1\}$ platí $x_1x_2+x_2x_3+\dots
+x_{n-1}x_n+x_nx_1=0$. Dokážte, že $n$ je párne číslo.
[Každý z~$n$ sčítancov v~danom súčte je rovný buď~1, alebo $\m1$. Jedna
polovica z~nich musí byť preto rovná~$1$ a~druhá $\m1$.
Číslo~$n$ je teda nutne párne.]
\endnávod
}

{%%%%%   B-I-2
Namiesto bežného algebraického postupu dáme prednosť grafickej
metóde riešenia danej sústavy rovníc. Za tým účelom ju prepíšeme
na tvar $y={\m|x|}+a$,
$x=2|y|-b$. Z~grafov oboch týchto závislostí medzi premennými $x$
a~$y$ (pri pevných číslach $a$ a~$b$), ktorých príklady (pre
zvolené $a$, $b$) sú vykreslené\footnote{Keď budeme meniť hodnoty
parametrov $a$ a~$b$,
budú sa oba grafy posúvať v~smere príslušnej súradnicovej osi.
Predstava týchto dvoch (navzájom nezávislých) pohybov nás
oprávňuje k~záverom o~možných vzájomných polohách oboch grafov,
a~ďalej ich v~tomto riešení uvádzame bez ďalšieho vysvetľovania.}
na \obr, vidíme (s~ohľadom na uhly, ktoré zvierajú dotyčné
polpriamky so súradnicovými osami), že v~prípade, keď obe čísla
$a$, $b$ sú {\it nekladné}, nemá zadaná sústava rovníc žiadne
riešenie s~výnimkou prípadu $a=b=0$, keď je riešenie jediné ($x=y=0$).
\insp{b66.1}%

Ďalej vidíme, že v~prípade $a>0$,
$b\leqq0$ má sústava nanajvýš dve riešenia, rovnako ako v~prípade
$a\leqq0$, $b>0$.\footnote{Pre každý z~oboch prípadov sami
nakreslite situácie, keď sústava má 0, 1, alebo 2~riešenia.}
Napokon vo zvyšnom prípade $a>0$, $b>0$ nie je ťažké usúdiť,
že sústava bude mať práve tri riešenia jedine vtedy, keď grafy
oboch závislostí budú mať jednu zo vzájomných polôh znázornených
na obrázkoch \obrnum\ a~\obrnum.
\inspinspblizko{b66.2}{b66.3}%
Na nich sú vyjadrené aj zodpovedajúce
čísla $a$ a~$b$ pomocou kladného parametra~$k$. Tri
spoločné body oboch grafov sú na oboch obrázkoch vyznačené
prázdnymi krúžkami; ich súradnice $(x,y)$, ktoré ľahko dopočítame,
sú potom hľadanými trojicami riešení zadanej sústavy.

\zaver
Daná sústava rovníc má práve tri riešenia pre každú
dvojicu parametrov $(a,b)=(k, 2k)$ a~pre každú dvojicu parametrov
$(a,b)=(k,k)$, pričom $k$ je ľubovoľné kladné reálne číslo. V~prvom
prípade má daná sústava riešenie
$(0,k)$, $({\m\frac43k},{\m\frac13k})$ a~$(4k,{\m3k})$. V~druhom
prípade má daná sústava riešenie $({\m k}, 0)$,
$(\frac13k,\frac23k)$ a~$(3k,\m2k)$.

\ineriesenie
I~keď je predchádzajúca grafická metóda riešenia
z~hľadiska matematickej exaktnosti plnohodnotná a~podané riešenie možno
tak oprávnene hodnotiť ako úplné, uveďme pre porovnanie
aj prácnejší algebraický postup.

Ako zvyčajne sa zbavíme
absolútnych hodnôt v~zadaných rovniciach rozlíšením štyroch
prípadov a~v~každom z~nich zodpovedajúcu
sústavu dvoch lineárnych rovníc
rutinným postupom najskôr vyriešime v~celom obore~$\Bbb R$
(nehľadiac teda zatiaľ na podmienky vymedzujúce daný prípad):
$$
\def\strut{\vrule width0pt height10pt depth 3pt\relax}
\matrix\format\strut\l&&\quad\l\\
(1)& x\geqq0,\ y\geqq0\:& x_1=\frac13{(2a-b)}, &y_1=\frac13{(a+b)},\\
(2)& x\geqq0,\ y<0 \:& x_2=2a+b, &y_2=-a-b,\\
(3)& x<0, \ y\geqq0\:& x_3=-2a+b, &y_3=-a+b,\\
(4)& x<0, \ y<0 \:& x_4=\frac13{(-2a-b)}, &y_4=\frac13{(a-b)}.
\endmatrix
$$

Keďže jednotlivé prípady (1)~--~(4)
sa navzájom vylučujú, je našou úlohou zistiť, pre aké dvojice
$(a,b)$ práve tri z~nájdených dvojíc $(x_i,y_i)$ spĺňajú príslušnú podmienku,
ktorá daný prípad vymedzuje. Kvôli prehľadnosti ďalšieho výkladu uveďme,
že najskôr vysvetlíme, prečo nevyhovujú jednotlivé situácie $a<0$, $b<0$,
$a=0$, $b=0$ (stačí vždy ukázať dva z~prípadov (1)~--~(4), ktoré
sú vylúčené), a~potom sa budeme venovať zvyšnej situácii, keď platí
$a>0$ a~zároveň $b>0$.

Ak je $a<0$, z~rovnice $|x|+y=a$ vyplýva $y<0$, čo vylučuje prípady
(1) a~(3). Ak je $b<0$, z~rovnice $2|y|-x=b$ vyplýva $x>0$, čo
vylučuje prípady (3) a~(4).

V~situácii, keď $a=0$, vypíšeme pre jednotlivé
prípady podmienky na číslo~$b$, za ktorých dvojica $(x_i,y_i)$
vymedzujúcej podmienke vyhovuje:
$$
\def\:{\hbox{:\enspace }}
(1)\:b=0,\quad(2)\:b>0,\quad(3)\:b\in\emptyset,\quad(4)\:b>0.
$$
Vidíme, že sústava má nanajvýš dve riešenia.
Podobné podmienky na číslo~$a$ v~situácii $b=0$ vyzerajú takto:
$$
\def\:{\hbox{:\enspace }}
(1)\:a\geqq0,\quad(2)\:a>0,\quad(3)\:a\in\emptyset,\quad(4)\:a\in\emptyset,
$$
preto aj teraz má sústava nanajvýš dve riešenia.

V~poslednej situácii, keď platí $a>0$ a~$b>0$, najskôr vypíšeme, ktoré
z~hodnôt $x_i$ a~$y_i$ majú "príslušné" znamienka automaticky:
$$
y_1>0,\quad x_2>0,\quad y_2<0,\quad x_4<0.
$$
(To mimochodom znamená, že dvojica $(x_2,y_2)$ je vždy riešením).
O~príslušnosti znamienok ostatných hodnôt $x_i$ a~$y_i$
zrejme rozhoduje, ako veľké
je (kladné) číslo~$b$ v~porovnaní s~(kladnými) číslami $a$ a~$2a$
(pre ktoré platí $a<2a$). Možné porovnania teraz rozlíšime a~pri každom z~nich
vypíšeme, ktoré z~prípadov (1)~--~(4) vedú k~riešeniu:
\item{(i)}
$b<a$:
vyhovujú prípady (1) a~(2).
\item{(ii)}
$b=a$:
vyhovujú prípady (1), (2) a~(3).
\item{(iii)}
$a<b<2a$:
vyhovujú prípady (1), (2), (3), (4).
\item{(iv)}
$b=2a$:
vyhovujú prípady (1), (2) a~(4).
\item{(v)}
$2a<b$:
vyhovujú prípady (2) a~(4).

Zisťujeme tak, že zadaná sústava rovníc má práve tri riešenia
jedine vtedy, keď platí $b=a>0$ alebo $b=2a>0$; výpis týchto riešení
dostaneme zo vzorcov pre hodnoty $x_i$ a~$y_i$. Ak dosadíme $b=a$
do príslušných z~nich, dostaneme v~prvom prípade trojicu riešení
$$
(x_1,y_1)=\bigl(\tfrac13 a,\tfrac23a\bigr),\
(x_2,y_2)=(3a,-2a),\
(x_3,y_3)=(- a, 0);
$$
pre druhý prípad dostaneme podobným dosadením $b=2a$ trojicu
$$
(x_1,y_1)=(0,a),\
(x_2,y_2)=(4a,-3a),\
(x_4,y_4)=\bigl(-\tfrac43 a,-\tfrac13 a\bigr).
$$

\návody
V~karteziánskej sústave súradníc $Oxy$ znázornite množinu
všetkých bodov v~rovine, ktorých súradnice $[x,y]$ vyhovujú rovnostiam
\item{a)}$|x|-y=1,$
b) $x+|y|=2,$
c) $|x|+|y|=3,$
d) $|x+y|+|x-y|=4,$
e) $\bigl||x+1|-1\bigr|=y,$
\item{f)}$\bigl||y-1|+1\bigr|=x-1.$

Určte všetky hodnoty reálneho parametra~$a$, pre ktoré má
sústava rovníc
$$
\align
|x|+|y| &= 1,\\
x-2y &= a.
\endalign
$$
riešenie v~obore reálnych čísel. Uveďte diskusiu vzhľadom na parameter~$a$.
[Z~grafov oboch vzťahov vyplýva, že daná sústava rovníc má riešenie pre
každé $a\in \langle \m2;2 \rangle$. Pre $a=\m2$ a~$a=2$ má
sústava práve jedno riešenie, postupne $(0;1)$ a~$(0;\m1)$. Pre každé
$a\in(\m2;2)$ má daná sústava práve dve riešenia.]

\D
V~obore reálnych čísel vyriešte sústavu rovníc s~neznámymi
$x$, $y$
$$
\align
ax+y &= 1,\\
|x|+y &= a,
\endalign
$$
pričom $a$ je reálny parameter. Uveďte diskusiu vzhľadom na parameter~$a$.
[16--C--II--4]

Použitím grafickej metódy a~ďalej potom výpočtom určte všetky
reálne riešenia sústavy rovníc
$$
\aligned
|x|+|y-1| &= 1,\\
|x-1|+|y| &= p,\
\endaligned
$$
pričom $p$ je reálny parameter.
[13--A--II--3]
\endnávod
}

{%%%%%   B-I-3
Vzhľadom na podmienky úlohy je $|\uh CBD|=|\uh ADB|=|\uh EAD|$, pretože
všetky tri uvažované uhly vytínajú na kružnici~$k$ zhodné tetivy.
Keďže $|\uh CBD|=|\uh ADB|$, je $AD\parallel BC$.

Tetivový štvoruholník ${ABCD}$ je teda rovnoramenný lichobežník
či pravouholník, v~ktorom sú (zhodné)
trojuholníky $DAB$ a~$ADC$ súmerne združené podľa spoločnej osi~$o_1$
strán $AD$, $BC$. Tá však prechádza stredom~$O$ kružnice~$k$.
V~uvedenej súmernosti si tak zodpovedajú aj ťažiská $K$ a~$L$ oboch zhodných
trojuholníkov $DAB$ a~$ADC$ (\obr). Os úsečky~$KL$ teda prechádza
stredom~$O$ kružnice~$k$.
Navyše oba body $K$ a~$L$ sú rôzne, pretože zodpovedajúce si ťažnice
z~(rôznych) vrcholov $B$ a~$C$ sa pretínajú v~strede spoločnej strany~$AD$
oboch zhodných trojuholníkov, zatiaľ čo ťažiská sú vnútornými bodmi oboch úsečiek.
\insp{b66.4}%

Analogicky dokážeme, že aj~${ABDE}$ je rovnoramenný lichobežník
či pravouholník. Pre ťažiská $K$, $M$ zhodných trojuholníkov $DAB$
a~$BED$ preto platí, že aj os úsečky~$KM$ prechádza stredom~$O$ kružnice~$k$. Odtiaľ
$|OL|=|OK|=|OM|>0$ (body $K$ a~$L$ sú rôzne), takže ťažiská všetkých troch uvažovaných trojuholníkov
ležia na kružnici sústrednej s~$k$.
Tým je dôkaz ukončený.


\návody
Dokážte tvrdenie: Ramená dvoch uhlov, ktorých vrcholy ležia na
jednej kružnici, vytínajú na tejto kružnici zhodné tetivy
práve vtedy, keď sú oba uhly zhodné.
[Využite vzťah medzi obvodovým a~stredovým uhlom.]

Os vnútorného uhla pri vrchole~$C$ v~trojuholníku $ABC$
pretína jemu opísanú kružnicu v~bode, ktorý
je stredom toho jej oblúka, ktorý neobsahuje bod~$C$. Dokážte.
[Využite výsledok predošlého tvrdenia.]

Daný je tetivový päťuholník ${ABCDE}$, v~ktorom
$|AE|=|AB|$ a~$|BC|=|DE|$. Dokážte, že priesečníky výšok (ortocentrá)
trojuholníkov $BCD$, $CDE$ a~bod~$A$ sú vrcholmi rovnoramenného trojuholníka.
[Dokážte, že ${BCDE}$ je rovnoramenný lichobežník či pravouholník, a~využite
osovú súmernosť daného päťuholníka podľa spoločnej osi úsečiek $BE$ a~$CD$.]
\endnávod
}

{%%%%%   B-I-4
Keďže $2\,016=2^5\cdot3^2\cdot7$, hľadáme práve tie
čísla $n=\overline{a2b0c1d6}$ s~nepárnymi ciframi $a$, $b$, $c$
a~$d$, ktoré sú zároveň deliteľné číslami $32=2^5$, $9=3^2$ a~$7$.

Číslo $n=\overline{a2b0c1d6}$ je deliteľné číslom~$32$ práve vtedy, keď je číslom~$32$
deliteľné jeho posledné päťčíslie
$$
\overline{0c1d6}=1\,000c+100+10d+6=32(31c+3)+8(c+d+1)+2(d+1).
$$
Podľa deliteľnosti ôsmimi vidíme, že číslo $d+1$ musí byť deliteľné
štyrmi, a~tak je buď $d=3$, alebo $d=7$. V~prípade $d=3$ je naše
päťčíslie rovné $32(31c+3)+8(c+5)$, takže je deliteľné~$32$ práve
vtedy, keď je číslo $c+5$ deliteľné $4$, teda $c\in\{3, 7\}$. V~prípade
$d=7$ je však ono päťčíslie rovné $32(31c+3)+8(c+10)$, takže
nie je deliteľné~$32$, pretože $c$ je nepárna cifra, a~tak je číslo
$c+10$ nepárne, čiže číslo $8(c+10)$ nie je deliteľné~$32$.
Podmienku deliteľnosti číslom $32$ tak spĺňajú práve tie~$n$,
ktoré za dvojicu nepárnych cifier $(c,d)$ majú $(3, 3)$ alebo $(7, 3)$.

Číslo $n=\overline{a2b0c1d6}$ je deliteľné deviatimi
práve vtedy, keď je číslom~$9$ deliteľný jeho ciferný súčet
$$
a+2+b+0+c+1+d+6=a+b+c+d+9.
$$
Z~toho dostaneme možné hodnoty súčtu $a+b$ pre obe už určené
dvojice $(c,d)$. Vezmeme pritom navyše do úvahy,
že súčet $a+b$ musí byť číslo párne, pretože obe cifry
$a$ a~$b$ sú nepárne. Pre prvú dvojicu $(c,d)=(3, 3)$ tak
vychádza $a+b=12$, teda $\{a,b\}=\{3, 9\}$ alebo $\{5, 7\}$.
Pre druhú dvojicu $(c,d)=(7, 3)$ dostaneme $a+b=8$,
teda buď $\{a,b\}=\{1, 7\}$, alebo $\{a,b\}=\{3, 5\}$.
Prichádzame k~záveru, že iba osem čísel požadovaného tvaru
je deliteľných oboma číslami $32$ a~$9$. Jedná sa o~čísla
$$
\gather
32\,903\,136,\ 92\,303\,136,\
52\,703\,136,\ 72\,503\,136,\\
12\,707\,136,\ 72\,107\,136,\
32\,507\,136,\ 52\,307\,136.
\endgather
$$

Deliteľnosť posledným činiteľom~$7$ tak ostáva posúdiť iba
pri týchto ôsmich kandidátoch na riešenie našej úlohy. Je jednoduché
sa presvedčiť, že iba prvé a~posledné z~vypísaných čísel
sú deliteľné siedmimi. Možno to spraviť rýchlo priamym delením
(ak máme po ruke kalkulačku), alebo namiesto toho využiť
menej známe kritérium deliteľnosti siedmimi,
podľa ktorého akékoľvek osemciferné číslo
$$
\overline{a_7a_6a_5a_4a_3a_2a_1a_0}=
a_0+a_1\cdot10^{1}+a_2\cdot10^{2}+a_3\cdot10^{3}+
a_4\cdot10^{4}+a_5\cdot10^{5}+a_6\cdot10^{6}+a_7\cdot10^{7}
$$
dáva po delení siedmimi taký istý zvyšok ako súčet
$$
s=a_0+3a_1+2a_2+6a_3+4a_4+5a_5+a_6+3a_7.
$$
(Koeficient pri každej cifre $a_k$ je rovný zvyšku po delení
siedmimi prislúchajúcej mocniny~$10^k$.)

\odpoved
Vyhovujú práve dve čísla
32\,903\,136 a~52\,307\,136.

\návody
Známe kritériá deliteľnosti číslami $2$, $4$ a~$8$ zovšeobecnite
na kritérium deliteľnosti číslom $2^k$ pre ľubovoľné prirodzené
číslo~$k$: Celé číslo zapísané v~desiatkovej sústave je deliteľné
číslom~$2^k$ práve vtedy, keď je také jeho posledné $k$-číslie.
[Rozdiel čísla a~jeho posledného $k$-číslia je číslo, ktorého zápis
končí $k$ nulami, takže je deliteľné číslom $10^k$, a~teda
aj~číslom~$2^k$. To znamená, že akékoľvek číslo a~jeho posledné
$k$-číslie dávajú po delení číslom~$2^k$ vždy taký istý zvyšok.]

Pripomeňte si známy poznatok, že dané prirodzené číslo
a~jeho ciferný súčet dávajú rovnaké zvyšky ako po delení tromi,
tak po delení deviatimi. Platí to aj po delení číslom $3^3=27$?
[Neplatí, napríklad číslom~$27$ je deliteľné samo číslo~$27$, avšak
jeho ciferný súčet $2+7=9$ číslom~$27$ deliteľný nie je. Neplatí ani
opačná implikácia: napríklad číslo~$1\,899$ s~ciferným súčtom~$27$
týmto číslom deliteľné nie je.]

\D
Dokážte, že o~deliteľnosti siedmimi akéhokoľvek čísla~$N$
zapísaného v~desiatkovej sústave možno rozhodovať pomocou "kódu"
$132645$ takto:
Šesťciferné číslo $N=\overline{a_5a_4a_3a_2a_1a_0}$
dáva po delení siedmimi taký istý zvyšok ako súčet
$$
s=1a_0+3a_1+2a_2+6a_3+4a_4+5a_5
$$
(cifry vypisujeme odzadu a~pred ne ako koeficienty pripisujeme
jednotlivé cifry onoho kódu).
Ak má číslo~$N$ menej ako šesť cifier, napíšeme
príslušný kratší súčet, ak má číslo~$N$ naopak viac ako šesť cifier,
kódové cifry opakujeme s~periódou~$6$, teda
$$
s=a_0+3a_1+2a_2+6a_3+4a_4+5a_5+a_6+3a_7+2a_8+6a_9+\dots
$$

Dokážte, že pre každé celé číslo~$n$ je číslo, ktoré
dostaneme, keď zapíšeme $3^n$ jednotiek za sebou,
násobkom čísla $3^n$. [Použite matematickú
indukciu: pre $n=1$ (aj~pre $n=2$) tvrdenie zrejme platí (použite
ciferný súčet). Pri druhom indukčnom kroku si všimnite, že číslo
z~$3^{n+1}$~jednotiek je násobkom čísla z~$3^n$~jednotiek,
pritom príslušný podiel je číslo tvaru $10\dots010\dots01$,
teda číslo deliteľné tromi (má totiž ciferný súčet~$3$).]
\endnávod
}

{%%%%%   B-I-5
Osi $DM$ a~$DN$ pravých uhlov pri vrchole~$D$ spolu s~výškou~$CD$ rozdeľujú
priamy uhol pri vrchole~$D$ na štyri zhodné uhly veľkosti~45\st.
Zároveň vidíme, že uhol $MDN$ je pravý, takže body $C$, $M$, $D$, $N$
ležia na Tálesovej kružnici s~priemerom~$MN$.

Ak označíme zvyčajným spôsobom uhly pri vrcholoch $A$ a~$B$ trojuholníka~$ABC$,
je zároveň $|\uh ACD|=90\st-\alpha=\beta$ a~$|\uh BCD|=90\st-\beta=\alpha$ (\obr).
Z~toho vyplýva podobnosť trojuholníkov $CDM\sim BDN$ a~$ADM\sim CDN$, takže
$$
\frac{|MD|}{|ND|}=\frac{|CM|}{|BN|}\quad\hbox{a}\quad
\frac{|MD|}{|ND|}=\frac{|AM|}{|CN|}.
$$
Porovnaním pravých strán dostaneme
$$
\belowdisplayskip 0pt
|AM|\cdot|BN|=|CM|\cdot |CN|. \tag1
$$
\insp{b66.5}%

Keďže obvodové uhly nad tetivami $CM$ a~$CN$ sú zhodné, je $|CM|=|CN|$.
Použitím Pytagorovej vety v~pravouhlom (rovnoramennom) trojuholníku $CMN$ tak
dostaneme
$$
2|CM|^2=|CM|^2+|CN|^2=|MN|^2
$$
a~dosadením do rovnosti~(1) vyjde
$$
2 |AM|\cdot |BN|=2 |CM|\cdot |CN|=2\cdot |CM|^2=|MN|^2.
$$
Tým je tvrdenie úlohy dokázané.


\poznamka
Ukážeme ešte jeden spôsob odvodenia kľúčovej rovnosti~(1).
Pravouhlé trojuholníky $ACD$ a~$CBD$ sú podobné, pretože
$|\uh BCD|=90\st-|\uh ACD|=|\uh CAD|$. To však znamená, že
osi $DM$ a~$DN$ oboch vnútorných uhlov z~vrcholu~$D$, ktoré si v~tejto podobnosti
zodpovedajú, delia protiľahlé strany
v~rovnakom pomere. Platí teda
$$
|AM|:|CM|=|CN|:|BN|, \quad \text{\tj.} \quad |AM|\cdot |BN|=|CM|\cdot |CN|.
$$


\návody
Zopakujte si základné vlastnosti a~vzťahy medzi {\it
obvodovým}, {\it stredovým\/} a~{\it úsekovým\/} uhlom a~ďalej
metódy (postupy), ako ukázať, že štyri a~viac bodov leží na jednej
kružnici.
[Možno doporučiť o.\,i. časopis Matematika~-- Fyzika~-- Informatika ({\tt
mfi.upol.cz}), roč.~24, č.~5, článok "Čtyři body na kružnici".]

Dokážte, že tetivy jednej kružnice prislúchajúce zhodným obvodovým uhlom sú zhodné.
[Použite sínusovú vetu.]

Daný je pravouhlý trojuholník $ABC$, v~ktorom $D$ označuje pätu
výšky z~vrcholu~$C$ na jeho preponu~$AB$. Ďalej nech $K$, $L$ sú
body na jeho odvesnách postupne $BC$, $AC$, pre ktoré platí
$2|BK|=|CK|$ a~$2|CL|=|AL|$. Dokážte, že body $K$, $C$, $L$ a~$D$
ležia na jednej kružnici. [Využite podobnosť pravouhlých trojuholníkov $ADC$
a~$CDB$, z~ktorej potom vyplýva aj podobnosť trojuholníkov $CLD$ a~$BKD$.]

\D
Daný je pravouhlý trojuholník $ABC$ s~preponou~$AB$. Označme $D$ pätu výšky z~vrcholu~$C$. Nech $Q$, $R$ a~$P$ sú postupne stredy úsečiek $AD$, $BD$ a~$CD$. Dokážte, že platí
$$
|\uh APB|+|\uh QCR|=180^{\circ}.
$$
[65--CPS juniorov--T1]
\endnávod
}

{%%%%%   B-I-6
Keďže
$$
a^3+b^3=(a+b)(a^2-ab+b^2),
$$
môžeme danú nerovnosť prepísať na ekvivalentný tvar
$$
(a+b-1)(a^2-ab+b^2)\geq 0. \eqno{(1)}
$$

Vzhľadom na to, že pre ľubovoľné reálne čísla $a$, $b$ platí
$$
a^2-ab+b^2=(a-\tfrac12b)^2+\tfrac34b^2\ge 0
$$
(s~rovnosťou práve vtedy, keď $a=b=0$), je nerovnosť~(1) splnená práve vtedy, keď
$$
a+b\ge 1\quad\text{alebo}\quad a=b=0.
$$

Z~podmienky $a+b\ge 1$ tak vyplýva, že nerovnosť~(1) je zaručene splnená
pre všetky dvojice reálnych čísel $a$, $b$, ktoré
sú väčšie alebo rovné~$\frac12$, takže požadovanú vlastnosť má každé $r\ge\frac12$.
Nemá ju však žiadne $r<\frac12$, pretože k~takému~$r$ možno zvoliť kladné čísla
$a=b=\max(r,\frac13)$, ktoré sú síce väčšie alebo rovné~$r$, avšak nerovnosť~(1)
nespĺňajú, lebo pre ne neplatí ani $a+b\ge1$, ani $a=b=0$.

\zaver
Danej úlohe vyhovujú práve všetky reálne čísla $r\ge \frac12$.

\návody
Dokážte, že pre každé nepárne číslo $n$ a~pre ľubovoľné
reálne čísla $a$, $b$ platí
$$
a^n+b^n=(a+b)(a^{n-1}-a^{n-2}b+\dots-ab^{n-2}+b^{n-1}).
$$

Znázornite v~karteziánskej sústave súradníc všetky body,
ktorých súradnice $x$, $y$ vyhovujú vzťahom
\item{a)} $(x-y\ge 1)\wedge (2x+y\le 1)$,
\item{b)} $(\min(x,y)\ge r)\wedge (\max(x,y)\le R)$, pričom $r<R$
sú dané reálne čísla.

Dokážte, že pre ľubovoľné kladné reálne čísla $a$, $b$, $c$ platí
$$
\def\zl#1#2{\frac1{#1^2-#1#2+#2^2}}
\zl ab+\zl bc+\zl ca\le\frac1{a^2}+\frac1{b^2}+\frac1{c^2}.
$$
Určte, kedy nastáva rovnosť.
[64--B--I--6]
\endnávod
}

{%%%%%   C-I-1
Zadaná nerovnosť patrí do veľkej skupiny
nerovností, ktoré možno dokazovať (často opakovaným) využitím poznatku,
že druhá mocnina akéhokoľvek reálneho čísla je nezáporná.
Na jeho základe overíme najskôr, že menovateľ $a^2-a+1$ zlomku,
ktorý je v~danej nerovnosti zastúpený, je vždy kladný.
Po úprave dvojčlena $a^2-a$,
ktorej hovoríme {\it doplnenie na štvorec}, totiž dostaneme
$$
a^2-a+1=\Bigl(a^2-a+\frac14\Bigr)+\frac34
=\Bigl(a-\frac12\Bigr)^{\!2}+\frac34\geqq\frac34>0.
$$

Menovateľ $a^2-a+1$ je teda kladný, a~keď ním obe strany dokazovanej nerovnosti vynásobíme, dostaneme
ekvivalentnú nerovnosť
$$
a^2(a^2-a+1)+1\geqq(a+1)(a^2-a+1).
$$
Po roznásobení a~zlúčení rovnakých mocnín $a$ dôjdeme k~nerovnosti
$$
a^4-2a^3+a^2\geqq0,
$$
ktorá však platí, pretože jej ľavá strana má rozklad
$a^2(a-1)^2$ s~nezápornými činiteľmi $a^2$ a~$(a-1)^2$~--
opäť účinkuje poznatok spomenutý v~úvode riešenia.
Tým je pôvodná nerovnosť pre každé reálne číslo~$a$ dokázaná.

Zároveň sme zistili, že rovnosť vo výslednej, a~teda
aj v~pôvodnej nerovnosti nastane
práve vtedy, keď platí $a^2(a-1)^2=0$, teda jedine vtedy, keď $a=0$ alebo $a=1$.

\ineriesenie
Danú nerovnosť môžeme prepísať na tvar
$$
(a^2-a+1)+\frac{1}{a^2-a+1}\geqq2,\quad\text{čiže}\quad
u+\frac{1}{u}\geqq2,
$$
pričom $u=a^2-a+1$. Je dobre známe, že posledná nerovnosť platí pre
každé {\it kladné\/} reálne číslo~$u$ a~že prechádza v~rovnosť
jedine pre $u=1$. Vyplýva to (opäť v~dôsledku nezápornosti každej
druhej mocniny) buď priamo z~vyjadrenia
$$
u+\frac{1}{u}=\biggl(\sqrt{u}-\frac{1}{\sqrt{u}}\biggr)^{\!\!2}+2,
$$
alebo voľbou $x=u$ a~$y=1/u$ vo~všeobecnejšej preslávenej
nerovnosti $\frac12(x+y)\geqq\sqrt{xy}$ medzi aritmetickým
a~geometrickým priemerom ľubovoľných dvoch {\it nezáporných\/} čísel
$x$ a~$y$, ktorá sama je dôsledkom obdobného vyjadrenia
rozdielu oboch dotyčných priemerov
$$
\tfrac12(x+y)-\sqrt{xy}=\frac12\bigl(\sqrt{x}-\sqrt{y}\bigr)^2\geqq0.
$$

Tak či onak stačí na~dôkaz pôvodnej nerovnosti overiť,
že výraz $u=a^2-a+1$ je kladný pre každé reálne
číslo~$a$. To možno spraviť rovnako ako v~prvom riešení, alebo
prepísať nerovnosť $a^2-a+1>0$ na tvar
$$
a(a-1)>-1
$$
a~spraviť krátku diskusiu: Posledná nerovnosť platí ako pre každé $a\geqq1$,
tak pre každé $a\leqq0$, lebo v~oboch prípadoch máme
dokonca $a(a-1)\geqq0$; pre zvyšné hodnoty~$a$,
teda pre $a\in(0, 1)$, je súčin $a(a-1)$ síce záporný,
avšak určite väčší ako ${\m1}$, pretože oba činitele $a$, $a-1$
majú absolútnu hodnotu menšiu ako~$1$. Prepísaná nerovnosť je tak
dokázaná pre každé reálne číslo~$a$, a~tým je podmienka pre použitie
nerovnosti $u+1/u\geqq2$ pre $u=a^2+a+1$ overená.

Ako sme už uviedli, rovnosť $u+1/u=2$ nastane jedine pre $u=1$.
Pre rovnosť v~nerovnosti zo zadania úlohy tak dostávame podmienku
$a^2-a+1=1$, čiže $a(a-1)=0$, čo je splnené iba
pre $a=0$ a~pre $a=1$.




\návody
Dokážte, že pre ľubovoľné reálne čísla $x$, $y$ a~$z$ platia
nerovnosti
\item{a)} $2xyz\leqq x^2+y^2z^2$,\quad b) $(x^2-y^2)^2\geqq 4xy(x-y)^2$.
\endgraf
[a) $P-L=(x-yz)^2$, b) $L-P=(x-y)^4$]

Dokážte, že pre ľubovoľné kladné čísla $a$, $b$
platí nerovnosť $a/b^2+b/a^2\geqq 1/a+1/b$.
[Vynásobte $a^2b^2$, vydeľte $a+b$ a~upravte na $(a-b)^2\geqq0$.]

\D
Použitím nerovnosti $u+1/u\geqq2$ ($\forall u>0$) dokážte, že pre
ľubovoľné kladné číslo $a$ platí
\item{a)} $\dfrac{a^2+3}{\sqrt{a^2+2}}>2$,\quad b) $\dfrac{2a^2+1}{\sqrt{4a^2+1}}>1$.
\endgraf
[Voľte $u=\sqrt{a^2+2}$ v~prípade~a), $u=\sqrt{4a^2+1}$
v~prípade~b) a~v~oboch prípadoch využite, že $u\ne1$.]

Dokážte, že pre ľubovoľné kladné čísla $a$, $b$, $c$,
$d$ platí $$(ab+cd)\Bigl(\frac1{ac}+\frac1{bd}\Bigr)\geqq4.$$
[$L=\left(\frc{b}{c}+\frc{c}{b}\right)+
\left(\frc{a}{d}+\frc{d}{a}\right)\geqq2+2=4$]

Dokážte, že pre ľubovoľné čísla $a$, $b$ z~intervalu $\langle 1,\infty)$
platí nerovnosť
$$
(a^2+1)(b^2+1) - (a-1)^2 (b-1)^2 \ge 4
$$
a~zistite, kedy nastane rovnosť.
[59--C--II--2]

Nájdite všetky reálne čísla $x$ a~$y$, pre ktoré výraz
$2x^2+y^2-2xy+2x+4$ nadobúda svoju
najmenšiu hodnotu. [65--C--I--3, časť a)]
\endnávod
}

{%%%%%   C-I-2
Vypočítajme najskôr hodnoty $V(n)$ pre niekoľko najmenších
prirodzených čísel~$n$ a~ich rozklady na súčin prvočísel zapíšme do
tabuľky:
$$
\vbox{\let\\=\cr \halign{&\hss$#$\hss&\quad#\cr
n&&V(n)\\ \noalign{\vskip3pt\hrule\vskip3pt}
1&&0\\
2&&48=2^4\cdot3\\
3&&168=2^3\cdot3\cdot7\\
4&&420=2^2\cdot3\cdot5\cdot7\\
}}
$$
Z~toho vidíme, že hľadaný deliteľ~$d$ všetkých čísel~$V(n)$
musí byť deliteľom čísla $2^2\cdot3=12$, spĺňa teda
nerovnosť $d\le12$. Preto ak ukážeme, že
číslo $d=12$ zadaniu vyhovuje, \tj. že $V(n)$ je násobkom čísla~$12$
pre {\it každé\/} prirodzené~$n$, budeme s~riešením hotoví.

Úprava
$$
V(n)=n^4+11n^2-12=(12n^2-12)+(n^4-n^2),
$$
pri ktorej sme z~výrazu $V(n)$ "vyčlenili" dvojčlen $12n^2-12$,
ktorý je zrejmým násobkom čísla~$12$, redukuje
našu úlohu na overenie deliteľnosti číslom~12 (teda deliteľnosti
číslami $3$ a~$4$) dvojčlena $n^4-n^2$. Využijeme na to jeho rozklad
$$
n^4-n^2=n^2(n^2-1)=(n-1)n^2(n+1).
$$
Pre každé celé $n$ je tak výraz $n^4-n^2$ určite deliteľný tromi
(také je totiž jedno z~troch po sebe idúcich celých čísel
$n-1$, $n$, $n+1$) a~súčasne aj deliteľný štyrmi (zaručuje
to v~prípade párneho~$n$ činiteľ~$n^2$, v~prípade nepárneho~$n$ dva
párne činitele $n-1$ a~$n+1$).

Dodajme, že deliteľnosť výrazu $V(n)$ číslom~$12$ možno dokázať
aj inými spôsobmi, napríklad môžeme využiť rozklad
$$
V(n)=n^4+11n^2-12=(n^2+12)(n^2-1)
$$
alebo prejsť k~dvojčlenu $n^4+11n^2$ a~podobne.

\odpoved
Hľadané číslo $d$ je rovné $12$.

\návody
Dokážte, že v~nekonečnom rade čísel
$$
1\cdot2\cdot3,\ 2\cdot3\cdot4,\ 3\cdot4\cdot5,\ 4\cdot5\cdot6,\ \dots,
$$
je číslo prvé deliteľom všetkých čísel ďalších.
[Využite to, že z~dvoch, resp. troch po sebe idúcich čísel je
vždy niektoré číslo deliteľné dvoma, resp. troma.]

Nájdite všetky celé $d>1$, pri ktorých hodnoty výrazov
$U(n)=n^3+17n^2-1$ a~$V(n)=n^3+4n^2+12$ dávajú po delení číslom~$d$ rovnaké zvyšky, nech je celé číslo~$n$ zvolené akokoľvek.
[Vyhovuje jedine $d=13$. Hľadané $d$ sú práve tie, ktoré delia rozdiel
${U(n)-V(n)}=13n^2-13=13(n-1)(n+1)$ pre každé celé~$n$. Aby sme
ukázali, že (zrejme vyhovujúce) $d=13$ je jediné, dosaďme do
rozdielu $U(n)-V(n)$ hodnotu $n=d$: číslo~$d$ je s~číslami $d-1$ a~$d+1$
nesúdeliteľné, takže delí súčin $13(d-1)(d+1)$ jedine vtedy, keď delí
činiteľ $13$, teda keď $d=13$.]

\D
Pre ktoré prirodzené čísla~$n$ nie je výraz $V(n)=n^4+11n^2-12$
násobkom ôsmich?
[Výraz $V(n)=(n-1)(n+1)(n^2+12)$ je určite násobkom ôsmich v~prípade
nepárneho~$n$, lebo $n-1$ a~$n+1$ sú dve po sebe idúce párne čísla,
takže jedno z~nich je deliteľné štyrmi, a~súčin oboch je tak
násobkom ôsmich. Keďže pre párne $n$ je súčin $(n-1)(n+1)$
nepárny, hľadáme práve tie $n$ tvaru $n=2k$,
pre ktoré nie je deliteľný ôsmimi výraz $n^2+12=4(k^2+3)$,
čo nastane práve vtedy, keď $k$ je párne. Hľadané $n$ sú teda
práve tie, ktoré sú deliteľné štyrmi.]

Dokážte, že pre ľubovoľné celé čísla $n$ a~$k$ väčšie ako $1$
je číslo $n^{k+2} - n^k$ deliteľné dvanástimi. [59--C--II--1]

Dokážte, že výrazy $23x + y$, $19x + 3y$ sú deliteľné číslom~$50$ pre rovnaké dvojice
prirodzených čísel $x$,~$y$. [60--C--I--2]

Určte všetky celé čísla~$n$, pre ktoré $2n^3-3n^2+n+3$ je prvočíslo. [62--C--I--5]

Dokážte, že pre každé nepárne prirodzené číslo~$n$ je súčet $n^4 + 2n^2 + 2\,013$
deliteľný číslom~$96$. [63--C--I--5]
\endnávod
}

{%%%%%   C-I-3
Päta~$D$ uvažovanej výšky je podľa zadania
tým vnútorným bodom strany~$AB$, pre ktorý platí $|AD|=2|BD|$ alebo
$|BD|=2|AD|$. Obe možnosti sú znázornené na \obr{}
s~popisom dĺžok strán $AC$, $BC$ a~oboch úsekov rozdelenej strany~$AB$.
\insp{c66.1}%

Pytagorova veta pre pravouhlé trojuholníky
$ACD$ a~$BCD$ vedie k~dvojakému vyjadreniu druhej mocniny spoločnej odvesny~$CD$,
pričom v~situácii naľavo dostaneme
$$
|CD|^2=b^2-(\tfrac23c)^2=a^2-(\tfrac13c)^2,
$$
odkiaľ po jednoduchej úprave poslednej rovnosti dostaneme vzťah
$$
3(b^2-a^2)=c^2.
$$
Pre druhú situáciu vychádza analogicky
$$
3(a^2-b^2)=c^2.
$$
Závery pre obe možnosti možno zapísať jednotne ako rovnosť
s~absolútnou hodnotou
$$
3|a^2-b^2|=c^2.
$$

Ak použijeme rozklad $|a^2-b^2|=|a-b|(a+b)$ a~nerovnosť
$c<a+b$ (ktorú ako je známe spĺňajú dĺžky strán každého trojuholníka $ABC$),
dostaneme z~odvodenej rovnosti
$$
3|a-b|c<3|a-b|(a+b)=c^2,
$$
odkiaľ po vydelení kladnou hodnotou~$c$ dostaneme $3|a-b|<c$, ako
sme mali dokázať.

Zdôraznime, že nerovnosť
$3|a-b|c<3|a-b|(a+b)$ sme správne zapísali ako ostrú~--
v~prípade $a=b$ by síce prešla na rovnosť,
avšak podľa nášho odvodenia by potom platilo $c^2=0$, čo odporuje tomu,
že sa jedná o~dĺžku strany trojuholníka.

\ineriesenie
Nerovnosť, ktorú máme dokázať, možno po vydelení tromi zapísať
bez absolútnej hodnoty ako dvojicu nerovností
$$
-\tfrac13 c<a-b<\tfrac13 c.
$$
Opäť ako v~pôvodnom riešení rozlíšime dve možnosti pre polohu
päty~$D$ uvažovanej výšky a~ukážeme, že vypísanú dvojicu
nerovností možno upresniť na tvar
$$
-\tfrac13 c<a-b<0,\quad\text{respektíve}\quad 0<a-b<\tfrac13 c,
$$
podľa toho, či nastáva situácia z~ľavej či pravej časti \obrr1.

Pre situáciu z~\obrr1{} naľavo prepíšeme
avizované nerovnosti ${\m\frac13}c<a-b<0$ ako
$a<b<a+\frac13 c$ a~odvodíme ich z~pomocného trojuholníka~$ACE$, pričom $E$
je stred úsečky~$AD$, takže body $D$ a~$E$ delia stranu~$AB$ na
tri zhodné úseky dĺžky $\frac13 c$.
\insp{c66.2}%

V~\obr{} sme rovno vyznačili, že úsečka~$EC$ má dĺžku~$a$ ako
úsečka~$BC$, a~to v~dôsledku zhodnosti trojuholníkov $BCD$ a~$ECD$ podľa
vety $sus$. Preto je pravá z~nerovností $a<b<a+\frac13 c$
porovnaním dĺžok strán trojuholníka~$ACE$, ktoré má všeobecnú platnosť.

Ľavú nerovnosť $a<b$ odvodíme z~druhého všeobecného poznatku, že
totiž v~každom trojuholníku {\it oproti väčšiemu vnútornému uhlu leží
dlhšia strana}. Stačí
nám teda zdôvodniť, prečo pre uhly vyznačené na \obrr1{} platí
$|\uhol CAE|<|\uhol AEC|$. To je však jednoduché:
zatiaľ čo uhol $CAE$ je vďaka pravouhlému trojuholníku $ACD$ ostrý,
uhol $AEC$ je naopak tupý, pretože k~nemu vedľajší uhol $CED$ je
ostrý vďaka pravouhlému trojuholníku~$CED$.

Pre prípad situácie z~\obrr2{} napravo možno predchádzajúci postup
zopakovať s~novým bodom~$E$, tentoraz stredom úsečky~$BD$.
Môžeme však vďaka súmernosti podľa osi~$AB$ konštatovať,
že z~dokázaných nerovností ${\m\frac13}c<a-b<0$ pre situáciu naľavo
vyplývajú nerovnosti ${\m\frac13}c<b-a<0$ pre situáciu napravo,
z~ktorých po vynásobení číslom~${\m1}$ dostaneme práve nerovnosti
$0<a-b<\tfrac13 c$, ktoré sme mali v~druhej situácii dokázať.


\návody
Pripomeňte si, ktoré nerovnosti spĺňajú medzi sebou
dĺžky strán ľubovoľného trojuholníka (a~ktorým preto hovoríme {\it
trojuholníkové\/}). Z~nich potom odvoďte známe pravidlo
$\al<\be\Rightarrow a<b$ o~porovnaní veľkostí
vnútorných uhlov a~dĺžok protiľahlých strán v~ľubovoľnom trojuholníku~$ABC$.
[Ak je $\al<\be$, môžeme nájsť vnútorný bod~$X$ strany~$AC$, pre ktorý
platí $|\uhel ABX|=\al$, a~teda $|AX|=|BX|$, takže z~trojuholníkovej
nerovnosti $|BC|<|BX|+|XC|$ už vyplýva $a<b$.]

Ak je $D$ vnútorný bod úsečky~$AB$, tak pre každý bod~$X$
kolmice vedenej bodom~$D$ na priamku~$AB$ má výraz $|AX|^2-|BX|^2$
tú istú hodnotu (rovnú hodnote $|AD|^2-|BD|^2$). Dokážte. [Použite
Pytagorovu vetu pre trojuholníky $ADX$ a~$BDX$.]

\D
Odvoďte nerovnosť, ktorá je zovšeobecnením nerovnosti zo
zadania súťažnej úlohy pre prípad, keď päta výšky z~vrcholu~$C$
trojuholníka~$ABC$ rozdeľuje jeho stranu~$AB$ v~pomere $1:p$,
pričom $p$ je dané kladné číslo rôzne od $1$. [$(p+1)|a-b|<|p-1|c$.]

Pre každý bod~$M$ vnútri daného rovnostranného
trojuholníka $ABC$ označme $M_a$, $M_b$, $M_c$ jeho kolmé priemety postupne na
strany $BC$, $AC$, $AB$. Dokážte rovnosť $|AM_b|+|BM_c|+|CM_a|=
|AM_c|+|BM_a|+|CM_b|$. [Najskôr trikrát použite výsledok úlohy N2
s~bodom $X=M$
a~z toho vyplývajúce vyjadrenia rozdielov $|AM|^2-|BM|^2$,
$|BM|^2-|CM|^2$, $|CM|^2-|AM|^2$ jednotlivo upravte a~potom
sčítajte.]
\endnávod
}

{%%%%%   C-I-4
Keďže $a$, $b$, $c$ sú podľa zadania celé čísla,
sú také aj hodnoty $P(1)$, $P(2)$ a~$P(3)$. Ich druhé
mocniny, čiže čísla
$P(1)^2$, $P(2)^2$ a~$P(3)^2$,
sú preto druhými mocninami celých čísel, teda tri (nie nutne rôzne)
čísla z~množiny $\{0, 1, 4, 9, 16, 25, \dots\}$. Ich súčet je podľa
zadania rovný~$22$, takže každý z~troch sčítancov je menší ako šieste
možné číslo~$25$. Akými spôsobmi možno vôbec zostaviť
súčet~$22$ z~troch čísel vybraných z~množiny $\{0, 1, 4, 9, 16\}$?

Systematickým rozborom rýchlo zistíme, že rozklad čísla~$22$
na súčet troch druhých mocnín je (až na poradie sčítancov) iba jeden,
a to $22=4+9+9$.
Dve z~čísel $P(1)$, $P(2)$ a~$P(3)$ majú teda absolútnu hodnotu~$3$
a~tretie~$2$, a~keďže $P(1)<P(2)<P(3)$, musí nutne platiť $P(1)={\m3}$, $P(3)=3$
a~$P(2)\in\{{\m2}, 2\}$. Pre každú z~oboch vyhovujúcich trojíc
$$
\bigl(P(1),P(2),P(3)\bigr)=(-3,-2, 3)\quad\text{a}\quad
\bigl(P(1),P(2),P(3)\bigr)=(-3, 2, 3)
$$
určíme koeficienty $a$, $b$, $c$ príslušného trojčlena $P(x)$ tak,
že nájdené hodnoty dosadíme do pravých strán rovníc
$$
\aligned
a+b+c&=P(1),\\
4a+2b+c&=P(2),\\
9a+3b+c&=P(3)
\endaligned
$$
a~výslednú sústavu troch rovníc s~neznámymi $a$, $b$, $c$ vyriešime.
Tento jednoduchý výpočet tu vynecháme, v~oboch
prípadoch vyjdú celočíselné trojice $(a,b,c)$, ktoré zapíšeme rovno
ako koeficienty trojčlenov, ktoré sú jedinými dvoma riešeniami
danej úlohy:
$$
P_1(x)=2x^2-5x\quad\text{a}\quad
P_2(x)=-2x^2+11x-12.
$$



\návody
Určte všetky dvojčleny $P(x)=ax+b$, pre ktoré platí
$P(2)=3$ a~$P(3)=2$. [Jediný dvojčlen $P(x)=5-x$, pretože sústava
rovníc $2a+b=P(2)=3$, $3a+b=P(3)=2$ má jediné riešenie $a={\m1}$
a~$b=5$.]

Určte všetky trojčleny $P(x)=ax^2+bx+c$, pre ktoré platí
$P(1)=4$, $P(2)=9$ a~$P(3)=18$. [Jediný trojčlen $P(x)=2x^2-x+3$,
pretože sústava rovníc $a+b+c=4$, $4a+2b+c=9$, $9a+3b+c=18$
má jediné riešenie $a=2$, $b={\m1}$, $c=3$.]

Určte všetky dvojčleny $P(x)=ax+b$ s~celočíselnými
koeficientmi $a$ a~$b$, pre ktoré platí $P(1)<P(2)$
a~$P(1)^2+P(2)^2=5$. [Vyhovujú práve štyri
dvojčleny $x+0$, $3x-4$, $x-3$ a~$3x-5$. Číslo $5$ sa dá zapísať jediným
spôsobom ako súčet dvoch druhých mocnín celých čísel,
keď neberieme ohľad na poradie sčítancov, a to $5=(\pm1)^2+(\pm2)^2$.
Preto čísla $P(1)<P(2)$ tvoria jednu z~dvojíc $(1,2)$, $({\m1},2)$,
$(\m2,\m1)$, $(\m2,1)$; pre každú z~nich vypočítajte koeficienty
$a$, $b$ postupom k~úlohe~N1.]

\D
Pre ktoré trojčleny $P(x)=ax^2+bx+c$
platí rovnosť $P(4)=P(1)-3P(2)+3P(3)$? [Pre všetky.
Presvedčte sa dosadením, že obe strany dotyčnej rovnosti
sú rovné $16a+4b+c$.]

Koeficienty $a$, $b$, $c$ trojčlena
$P(x)=ax^2+bx+c$ sú reálne čísla, pritom každá z~troch jeho
hodnôt $P(1)$, $P(2)$ a~$P(3)$ je celým číslom. Vyplýva z~toho, že
aj čísla $a$, $b$, $c$ sú celé, alebo je nutne celé aspoň
niektoré z~nich (ktoré)? [Nevyplýva, uvážte príklad
trojčlena $P(x)=\frac12x^2+\frac12x+1$: z~vyjadrenia
$P(x)=\frac12x(x+1)+1$ vyplýva, že $P(x)$ je celým číslom pre každé
celé~$x$, pretože súčin $x(x+1)$ je vtedy deliteľný dvoma.
Vo všeobecnej situácii je iba koeficient~$c$ nutne celé číslo;
vyplýva to z~vyjadrenia $c=P(0)=3P(1)-3P(2)+P(3)$.]
\endnávod
}

{%%%%%   C-I-5
Pred samotným riešením odvodíme dôležité vlastnosti všeobecného
lichobežníka $RSTU$: {\sl Ak označíme $X$ a~$Y$ postupne stredy
základní $RS$ a~$TU$, tak na úsečke~$XY$ leží priesečník~$P$
uhlopriečok $RT$ a~$SU$, a~to tak, že $|PX|:|PY|=|RS|:|TU|$.
Na priamke~$XY$ leží tiež priesečník $Q$ predĺžených ramien $RU$
a~$ST$, a~to tak, že $|QX|:|QY|=|RS|:|TU|$} (\obr).
\insp{c66.3}%

Napriek tomu, že sa podľa obrázka zdá, že bod~$P$ na úsečke~$XY$
naozaj leží, musíme tento poznatok {\it dokázať}, teda odvodiť
argumentáciou nezávislou na presnosti nášho rysovania. Na to určite
stačí preukázať, že obe úsečky $PX$, $PY$ zvierajú s~priamkou~$RT$
zhodné uhly (na obrázku vyznačené otáznikmi). Všimnime si, že
tieto úsečky sú ťažnicami trojuholníkov $RSP$ a~$TUP$, ktoré sa
zhodujú vo vnútorných uhloch (vyznačených oblúčikmi)
pri rovnobežných stranách $RS$ a~$TU$, takže sa jedná o~trojuholníky
podobné, a~to s~koeficientom $k=|TU|/|RS|$. Ako vieme\niedorocenky{ (úloha~N2)},
s~rovnakým koeficientom platí aj podobnosť "polovíc"
týchto trojuholníkov vyťatých ich ťažnicami, presnejšie
podobnosť $RXP\sim TYP$. Z~nej už želaná
zhodnosť uhlov $RPX$ a~$TPY$ aj želaná
rovnosť $|PY|=k|PX|$ (vďaka rovnakému koeficientu) vyplýva.
%
Všetko o~bode~$P$ je tak dokázané; podobne sa
overia aj obe vlastnosti bodu~$Q$~-- ukáže sa, že úsečky $QX$
a~$QY$ zvierajú ten istý uhol s~priamkou~$RQ$ a~ich dĺžky sú zviazané
rovnosťou $|QY|=k|QX|$, a~to vďaka tomu, že $QX$ a~$QY$ sú
ťažnice v~dvoch navzájom podobných trojuholníkoch $RSQ$ a~$UTQ$.

Dokázané vlastnosti všeobecného lichobežníka nám umožnia celkom
ľahko vyriešiť zadanú úlohu. Situácia je znázornená na \obr.
Okrem pomenovaných bodov sme tam ešte označili $S_1$, $S_2$, $S_3$
stredy úsečiek $AB$, $KL$ a~$MN$.
\insp{c66.4}%
Keďže trojuholníky $ABC$, $KLC$ a~$MNC$ sú navzájom podobné (podľa
vety $sus$), platí $|AB|:|KL|:|MN|=|AC|:|KC|:|MC|=3:2:1$.
Podľa zhodných vnútorných uhlov spomenutých troch trojuholníkov platí
tiež $AB\parallel KL\parallel MN$.
Štvoruholníky $ABLK$, $KLNM$ a~$ABNM$ tak sú
naozaj lichobežníky (ako je prezradené v~zadaní) so základňami
$AB$, $KL$ a~$MN$, ktorých dĺžky sú v~už odvodenom pomere
$3:2:1$. Navyše predĺžené ramená všetkých troch lichobežníkov sa
pretínajú v~bode~$C$, ktorým preto podľa dokázanej vlastnosti
prechádzajú priamky $S_1S_2$, $S_2S_3$ (a~$S_1S_3$), takže sa jedná
o~jednu priamku, na ktorej body $S_1$, $S_2$, $S_3$ a~$C$ ležia
v~uvedenom poradí tak, že $|S_1C|:|S_2C|:|S_3C|={3:2:1}$. Z~toho vyplýva
$|S_1S_2|=|S_2S_3|$ $(=|S_3C|)$, takže bod~$S_2$ je stredom úsečky~
$S_1S_3$. Na nej (opäť podľa dokázaného tvrdenia) ležia aj body
$E$, $F$ a~$G$, pričom pre bod~$E$ medzi bodmi $S_1$, $S_2$ platí
$|ES_1|:|ES_2|=3:2$, pre bod~$F$ medzi bodmi $S_2$, $S_3$ platí
$|FS_2|:|FS_3|=2:1$ a~napokon pre bod~$G$ medzi bodmi
$S_1$, $S_3$ platí $|GS_1|:|GS_3|=3:1$. Tieto delenia troch úsečiek sme
znázornili na \obr, kam sme zapísali aj dĺžky vzniknutých úsekov pri
voľbe jednotky $1=|S_1S_2|=|S_2S_3|$ (pri ktorej $|S_1S_3|=2$).
\insp{c66.5}%

Keďže
$$
|S_1F|=|S_1S_2|+|S_2F|=1+\tfrac23=\tfrac53>\tfrac32=|S_1G|,
$$
platí $|GF|=|S_1F|-|S_1G|=\frac53-\frac32=\frac16$, čo spolu
s~rovnosťou $|EF|=|ES_2|+|S_2F|=\frac25+\frac23=\frac{16}{15}$ už vedie
k~určeniu hľadaného pomeru
$$
|GF|:|EF|=\tfrac16:\tfrac{16}{15}=5:32.
$$


\návody
Zopakujte si, čo viete o~podobnosti dvoch trojuholníkov z~učiva
základnej školy: Podobnosť $\triangle A_1B_1C_1\sim \triangle
A_2B_2C_2$ s~koeficientom~$k$ znamená, že pre zvyčajne označené
dĺžky strán a~veľkosti vnútorných uhlov oboch trojuholníkov platia rovnosti
$a_2=ka_1$, $b_2=kb_1$, $c_2=kc_1$, $\al_2=\al_1$,
$\be_2=\be_1$, $\ga_2=\ga_1$. Stačí na to, aby platilo
(i) $a_2:b_2:c_2=a_1:b_1:c_1$ (veta~$sss$) alebo
(ii) $\al_2=\al_1$ a~$\be_2=\be_1$ (veta~$uu$) alebo
(iii) $a_2:a_1=b_2:b_1$ a~$\ga_2=\ga_1$ (veta~$sus$).

Nech $A_1B_1C_1$ a~$A_2B_2C_2$ sú
ľubovoľné dva podobné trojuholníky ($\triangle A_1B_1C_1\sim \triangle
A_2B_2C_2$). Označme $S_1$, $S_2$ postupne stredy strán $A_1B_1$,
$A_2B_2$. Dokážte podobnosť $\triangle A_1S_1C_1\sim \triangle
A_2S_2C_2$ a~dokážte, že má rovnaký koeficient ako
pôvodná podobnosť $\triangle A_1B_1C_1\sim \triangle
A_2B_2C_2$. [Podobnosť $\triangle A_1S_1C_1\sim \triangle
A_2S_2C_2$ platí vďaka vete $sus$, pretože vnútorné uhly oboch trojuholníkov
pri vrcholoch $A_1$, $A_2$ sú zhodné a~pre dĺžky strán, ktoré ich
zvierajú, platí $|A_2S_2|:|A_1S_1|=
\bigl(\frac12|A_2B_2|\bigr):\bigl(\frac12|A_1B_1|\bigr)=
|A_2B_2|:|A_1B_1|=|A_2C_2|:|A_1C_1|$. Predpokladaná aj dokázaná podobnosť
majú taký istý koeficient $k=|A_2B_2|/|A_1B_1|$.]

Dokážte, že ľubovoľná spojnica ramien daného lichobežníka $ABCD$,
ktorá je rovnobežná s~jeho základňami $AB\parallel CD$, je úsečka, ktorej
stred leží na spojnici stredov oboch základní. Potom dokážte, že
priesečník uhlopriečok~$P$ je stredom tej zo spomenutých spojníc ramien,
ktorá týmto priesečníkom prechádza. [Použite najskôr
výsledok úlohy~N2 pre podobné trojuholníky so spoločným vrcholom,
ktorým je priesečník predĺžených ramien, a~protiľahlými stranami,
ktorými sú jednak základňa lichobežníka, jednak uvažovaná spojnica ramien.
Na dôkaz vlastnosti priesečníka~$P$ označte~$E\in BC$, $F\in AD$
krajné body prislúchajúcej spojnice ramien
a~využite to, že podobnosť trojuholníkov $APF$, $ACD$ má taký istý koeficient
ako podobnosť trojuholníkov $BEP$, $BCD$.]

\D
Vnútri strán $AB$, $AC$ daného trojuholníka $ABC$ sú zvolené postupne
body $E$, $F$, pričom $EF \parallel BC$. Úsečka~$EF$
je potom rozdelená bodom~$D$ tak, že platí
$p = |ED|:|DF| = |BE|:|EA|$.
\item{a)} Ukážte, že pomer obsahov trojuholníkov $ABC$ a~$ABD$ je pre $p=2:3$ rovnaký
ako pre $p=3:2$.
\item{b)} Zdôvodnite, prečo pomer obsahov trojuholníkov $ABC$ a~$ABD$ má hodnotu
aspoň~$4$. [65--C--I--4]

Označme $E$ stred základne $AB$ lichobežníka $ABCD$, v~ktorom platí
$|AB|:|CD|={3:1}$. Uhlopriečka~$AC$ pretína úsečky $ED$, $BD$ postupne
v~bodoch $F$, $ G$. Určte postupný pomer
$|AF|:|FG|:|GC|$.
[64--C--I--4]
\endnávod
}

{%%%%%   C-I-6
Marienka musí počítať s~tým, že Peter vyberie takú trojicu vrcholov
rovnoramenného trojuholníka, v~ktorých je dokopy čo najmenej cukríkov.
Preto je v~záujme Marienky rozmiestniť cukríky tak, aby bol spomenutý
minimálny počet čo najväčší.

a) Ukážeme najskôr, že nech Marienka rozmiestni cukríky do vrcholov
pravidelného osemuholníka v~počtoch 1 až~8 akokoľvek, Peter vždy
nájde rovnoramenný trojuholník, v~ktorého vrcholoch je dokopy {\it
nanajvýš desať\/} cukríkov.

Najskôr si rozmyslime (použite napríklad \obr, kde sú vyznačené
všetky tri typy rovnoramenných trojuholníkov), že každé dva
rôzne vrcholy pravidelného osemuholníka sú vrcholmi
buď dvoch, alebo štyroch uvažovaných rovnoramenných trojuholníkov.
Rozlíšime pritom, ktorú zo vzdialeností\footnote{Máme na mysli počet
strán mnohouholníka na kratšej z~oboch ciest po jeho obvode,
ktoré dané dva vrcholy spájajú.} 1, 2, 3, 4 dané dva vrcholy majú,
podľa toho je počet oných trojuholníkov postupne 2, 4, 2, 2 (overte sami).
Ak teda vyberie Peter najskôr tie dva vrcholy,
do ktorých Marienka rozmiestni 1 a~2 cukríky, má potom na výber tretieho
vrcholu aspoň dve možnosti, takže sa môže vyhnúť
prípadnému výberu, keď by Marienka získala $1+2+8$
cukríkov, a~zvoliť vždy výber,
keď Marienka dostane nanajvýš $1+2+7=10$ cukríkov.
Opísali sme teda Petrovu stratégiu,
pri ktorej Marienka nezíska viac ako 10~cukríkov.
\inspinsp{c66.6}{c66.7}%

V~druhej časti riešenia úlohy~a) poradíme Marienke jedno konkrétne
rozmiestnenie, pri ktorom 10~cukríkov zaručene získa, nech Peter urobí
výber rovnoramenného trojuholníka akokoľvek. Pôjde o~rozmiestnenie, keď do
jednotlivých vrcholov postupne v~jednom smere po obvode útvaru
rozmiestnime 1, 3, 7, 4, 2, 5, 8 a~6 cukríkov (\obr).

Teraz je nutné overiť, že súčet čísel pri vrcholoch
každého rovnoramenného trojuholníka na \obrr1{} je rovný aspoň~10.
Pri kontrole stačí testovať iba súčty, ktoré sú
podľa dvoch najmenších sčítancov tvaru $1+2+x$, $1+3+x$ či $2+3+x$
(ostatné súčty sú určite aspoň~10).
Z~\obrr1{} vidíme, že sa jedná práve o~súčty
$$
1+2+8,\ 1+2+7,\ 1+3+7,\ 1+3+6,\ 2+3+8,\ 2+3+6,
$$
z~ktorých žiadny nie je menší ako~10. Marienka tak má stratégiu, ktorá
jej prináša zisk aspoň 10~cukríkov (uvedený príklad rozmiestnenia je iba
jeden z~mnohých možných). Z~prvej časti riešenia pritom vyplýva, že Marienka
zisk viac ako 10~cukríkov zaručený nemá.

\smallskip
b) Pri riešení druhej úlohy už nebudeme opakovať komentáre
o~stratégiách Petra a~Marienky a~cukríky pri vrcholoch zameníme ich
počtami, \tj. číslami.

V~prvej časti budeme predpokladať, že k~vrcholom pravidelného
deväťuholníka sú pripísané čísla od~1 do~9 akokoľvek, a~ukážeme,
že pre súčet~$s$ troch čísel pri vrcholoch vhodného rovnoramenného
trojuholníka platí $s\leqq10$.
\insp{c66.10}%

Pri pravidelnom deväťuholníku je počet rovnoramenných trojuholníkov
s~danými dvoma vrcholmi buď rovný~1 (jedná sa o~rovnostranný trojuholník),
alebo je rovný~3~-- podľa ich vzdialenosti 1, 2, 3, 4 je tento počet totiž postupne 3, 3, 1, 3
(sami overte, na \obr{} sú vyznačené všetky štyri typy rovnoramenných trojuholníkov).
Navyše budeme potrebovať ešte
poznatok, že pri oných troch rovnoramenných trojuholníkoch s~dvoma pevnými
vrcholmi tvoria ich tretie vrcholy vždy rovnostranný trojuholník.
Vyplýva to z~\obr{} pre tri situácie, keď vzdialenosť dvoch pevných
vrcholov $K$ a~$L$ je~1, resp.~2, resp.~4 (všetky rovnoramenné
trojuholníky $KLM_i$ sú vykreslené).
\insp{c66.8}%

Teraz už môžeme dokázať nerovnosť $s\leqq10$ pri ľubovoľnom
očíslovaní vrcholov, ako sme sľúbili v~úvodnej vete riešenia časti~b).

Ak je trojuholník s~číslami 1,~2,~3 rovnostranný, vyberieme ho
a~dostaneme dokonca $s=1+2+3=6$; v~opačnom prípade sú
(podľa predchádzajúcich úvah) čísla 1 a~2
alebo čísla 1 a~3 pri vrcholoch troch rovnoramenných trojuholníkov.
Znamená to, že potom medzi skúmanými súčtami
sú tri súčty tvaru $1+2+x$ alebo tri súčty tvaru
$1+3+x$. Môžeme sa teda vyhnúť dvom prípadným
súčtom pre $x=9$ a~$x=8$ a~vybrať iba ten zo súčtov,
v~ktorom bude $x\leqq7$. S~výnimkou jediného prípadu $s=1+3+7=11$
dostaneme vždy $s\le10$.

Keby však náš postup predsa len skončil hodnotou $s=11$,
znamenalo by to, že čísla 1 a~2 stoja pri vrcholoch rovnostranného
trojuholníka (inak by sme vybrali súčet $1+2+x$ s~$x\le7$)
a~navyše existujú tri rovnoramenné trojuholníky so súčtami
$1+3+7$, $1+3+8$ a~$1+3+9$.
Tieto tri trojuholníky s~dvoma spoločnými vrcholmi (s~číslami 1 a~3) však
podľa odseku ukončeného \obrr1{} zaručujú, že
trojuholník so súčtom ${7+8+9}$ je rovnostranný, takže tretí vrchol
rovnostranného trojuholníka, pri ktorého vrcholoch stoja čísla 1 a~2, nemôže byť
žiadne z~čísel 7, 8, 9. Našli sme teda rovnostranný trojuholník so súčtom
$s\le1+2+6=9$.

Dokázali sme tak, že v~prípade, keď trojuholník
so súčtom $1+2+3$ nie je rovnostranný, existuje vždy
rovnoramenný trojuholník so súčtom nanajvýš~10. Prvá časť riešenia je
preto hotová.

V~druhej časti riešenia úlohy~b) uvedieme príklad očíslovania vrcholov pravidelného deväťuholníka (opäť jedného
z~mnohých), keď súčet troch čísel pri vrcholoch každého rovnoramenného
trojuholníka je aspoň~10. Vrcholy
v~jednom smere po obvode označíme postupne číslami
1, 6, 4, 2, 9, 5, 7, 8 a~3 (\obr).
\insp{c66.9}%

Aj tentoraz stačí otestovať súčty tvaru $1+2+x$, $1+3+x$, $2+3+x$,
čo sú podľa \obrr1{} práve súčty
$$
1+2+7,\ 1+3+6,\ 1+3+8,\ 1+3+9,\ 2+3+6,\ 2+3+8,\ 2+3+9,
$$
medzi ktorými naozaj nie je žiadny menší ako~10.

\odpoved
Pre obe úlohy a) a~b) platí, že najväčší počet
cukríkov, ktoré Marienka môže zaručene získať, je rovný~10.


\návody
K~vrcholom pravidelného sedemuholníka pripíšeme
čísla od 1 do 7 v~akomkoľvek poradí.
Dokážte, že súčet troch čísel pri vrcholoch
niektorého rovnoramenného trojuholníka je menší ako~9. [Uvážte dva
vrcholy $X$ a~$Y$ s~číslami 1 a~2 a~rozborom všetkých možností overte,
že existujú vždy tri rovnoramenné trojuholníky $XYZ$ s~vhodnou voľbou
tretieho vrcholu~$Z$. Vyberieme z~nich to $Z$, pri ktorom nie je
ani číslo 7, ani číslo 6. Súčet čísel pri vrcholoch príslušného
trojuholníka $XYZ$ je potom nanajvýš $1+2+5=8$.]

Ostane všeobecne platné tvrdenie z~úlohy N1, keď v~ňom
záverečné číslo~9 zameníme číslom~8? [Nie. Pripíšte vrcholom
v~jednom smere po obvode postupne čísla 1, 3, 4, 2, 5, 6 a~7. Potom
súčet troch čísel pri vrcholoch každého rovnoramenného trojuholníka bude
aspoň~8. Uvedomte si, že pri overovaní posledného poznatku
(aj pre iné rozmiestnenie čísel ako nami uvedené) stačí
overiť, že sú {\it rôznostranné\/} tie dva trojuholníky,
ktoré majú pri svojich vrcholoch trojice čísel $(1,2,3)$ a~$(1,2,4)$.]

\D
Každý vrchol pravidelného devätnásťuholníka je ofarbený jednou
zo šiestich farieb. Dokážte, že niektorý tupouhlý trojuholník má všetky vrcholy ofarbené rovnakou farbou.
[62--C--S--3]

Rozhodnite, či z~ľubovoľných siedmich vrcholov daného pravidelného
19-uholníka možno vždy vybrať štyri, ktoré sú vrcholmi lichobežníka. [62--C--I--4]
\endnávod
}

{%%%%%   A-S-1
Číslo $2\,016$ je násobkom $9$, preto výsledné číslo musí mať
ciferný súčet deliteľný~$9$.
To nastane práve vtedy, keď aj vložené číslo bude mať ciferný súčet
deliteľný~$9$, čiže to bude násobok deviatich.
Vyskúšame postupne kladné násobky čísla~9 od najmenšieho:
čísla $9$, $18$, $27$ nevyhovujú (namiesto priameho delenia číslom 2\,016
sa stačí presvedčiť, že čísla 20\,916, 201\,816 ani
202\,716 nie sú~-- podľa svojich posledných štvorčíslí~-- deliteľné číslom~16,
zato číslo 2\,016 áno), ale $203\,616 = 2\,016 \cdot 101$.
Hľadaným najmenším číslom je teda~$36$.

\poznamka
Keďže 2\,016 je násobkom šestnástich, možno postupné hľadanie
najmenšieho vyhovujúceho čísla založiť tiež na nasledujúcom zrejmom poznatku:
číslo s~posledným dvojčíslím~16 (\tj.~číslo tvaru $100k+16$) je deliteľné
šestnástimi práve vtedy, keď je jeho predposledné dvojčíslie (teda posledné dvojčíslie
príslušného~$k$) deliteľné štyrmi.


\nobreak\medskip\petit\noindent
Za úplné riešenie dajte 6~bodov (aj v~prípade, že
neobsahuje úvahy o~deliteľnosti číslami 9 ani 16: je možné napr. postupne
vkladať čísla $1, 2, \dots, 36$).
Neúplné riešenie: 3~body za zdôvodnenie toho, že vložené číslo
musí byť deliteľné deviatimi, resp. štyrmi; 1~bod za overenie, že číslo $36$
spĺňa podmienku zo zadania.

\endpetit
\bigbreak
}

{%%%%%   A-S-2
Malé čísla $n$ vylúčime postupne nasledujúcimi úvahami.

Vytvorené tri množiny musia mať navzájom rôzne počty prvkov, a~to
je možné len pre $n\ge1+2+3= 6$.

Pre $n = 6$ bude v~najmenšej množine iba jeden prvok, preto súčet
jej prvkov bude nanajvýš~$6$. Pritom súčet piatich čísel mimo túto
množinu je aspoň~$15$, takže súčet prvkov niektorej zo
zvyšných dvoch množín je aspoň~$8$, teda väčší ako číslo z~jednoprvkovej
množiny, čo odporuje požiadavkám úlohy. Podobnou
úvahou vylúčime aj čísla $n = 7$ a~$n = 8$, pre ktoré tiež
jedno číslo musí tvoriť celú jednu množinu.

Pre $n=9$ je hľadané rozdelenie napr. $\{9, 8\}$, $\{7, 6, 3\}$, $\{5, 4, 2, 1\}$.

Pre $n = 10$ musia byť v~najmenšej množine dva prvky, ich súčet je
nanajvýš~$19$. Pritom súčet ostatných čísel je $36$, a~keďže
súčty prvkov zvyšných dvoch množín nie sú rovnaké, musí jeden
z~nich byť aspoň $19$, čo dáva spor. Podobne odvodíme spor
aj~pre $n = 11$.

Teraz opíšeme vyhovujúce rozdelenie pre každé číslo $n\ge12$,
ktoré podľa delenia tromi so zvyškom zapíšeme v~tvare $n=3k+r$,
pričom $k\ge4$ a~$r\in\{0, 1, 2\}$. Počty prvkov troch množín zvolíme
v~rastúcom poradí $k-1$, $k$ a~$k+r+1$, pričom do prvej množiny~$\mm M_1$
zaradíme $k-1$ najväčších čísel z~$\{1, 2, \dots,n\}$ (teda čísla
od $n-k+2$ po~$n$ vrátane), druhú množinu~$\mm M_2$ potom zostavíme
z~$k$ predchádzajúcich najväčších čísel (teda z~čísel od $n-2k+2$ po
$n-k+1$) a~zvyšných $k+r+1$ najmenších čísel napokon vytvorí množinu~$\mm M_3$.

Keďže všetky čísla z~$\mm M_1$ sú väčšie ako všetky čísla
z~množiny~$\mm M_2$, ktorá má iba o~1 prvok menej ako $\mm M_1$, stačí
ukázať, že súčet troch najväčších čísel z~$\mm M_1$ je väčší ako
súčet štyroch najmenších čísel z~$\mm M_2$, a~bude jasné, že taká
nerovnosť platí aj pre súčty všetkých čísel z~$\mm M_1$ a~$\mm M_2$.
Spomenuté tri a~štyri čísla naozaj existujú (lebo $k\ge4$)
a~potrebná nerovnosť $n+(n-1)+(n-2)>(n-2k+2)+(n-2k+3)+(n-2k+4)+{(n-2k+5)}$
je ekvivalentná nerovnosti $8k-n>17$, ktorá po dosadení $n=3k+r$ prejde
na nerovnosť $5k>17+r$. Tá platí, lebo $5k\ge20$ a~$17+r\le19$.
Podobne vysvetlíme, že množina~$\mm M_2$ má väčší súčet prvkov ako
množina~$\mm M_3$, ktorá má oproti $\mm M_2$ o~$r+1$ prvkov viac:
dve najväčšie čísla v~$\mm M_2$ majú určite súčet väčší ako $r+3$
najmenších čísel v~$\mm M_3$, lebo
$(n-k+1)+(n-k)=2(3k+r-k)+1=4k+2r+1\ge17>15= 1+2+3+4+5$.

Úlohe vyhovujú všetky celé čísla $n \ge 12$ a~$n = 9$.

\ineriesenie
Ukážeme, ako vylúčiť malé nevyhovujúce čísla $n$ všeobecnejšou
úvahou. Predpokladajme teda, že pre nejaké kladné celé číslo~$n$ požadované
rozdelenie existuje.
Označme jednotlivé množiny $\mm M_1$, $\mm M_2$ a~$\mm M_3$ v~poradí
podľa ich rastúcej mohutnosti.
Ak označíme $p$ počet prvkov v~množine~$\mm M_1$,
ktorá má súčet prvkov najväčší,
bude platiť $p+(p+1)+(p+2)\le n$, teda
$$
p\le \frac n3-1. \tag1
% \label{ineq1}
$$

Druhú dôležitú nerovnosť získame pozorovaním, že súčet $s_1$
prvkov v~$\mm M_1$ je nanajvýš $n + (n-1) + \dots + (n-p+1)=\frac12p(2n-p+1)$
a~zároveň musí byť väčší ako súčty prvkov $s_2$ a~$s_3$
vo zvyšných dvoch množinách.
Vzhľadom na celočíselnosť týchto troch súčtov dostávame
$$
1+2+\dots +n = s_1 + s_2 +s_3 \le s_1 + (s_1-1) + (s_1-2) = 3s_1-3,
$$
teda $\frac16n(n+1)+1\le s_1$.
Porovnaním oboch odhadov pre $s_1$ dostaneme nerovnosť
$$
\tfrac16n(n+1)+1
%\le s_1
\le\tfrac12p(2n-p+1),
$$
čiže
$$
n^2+n(1-6p)+3p^2-3p+6\le 0.\tag2
% \label{ineq2}
$$

Ak je $p=1$, platí podľa (1) $n\ge6$ a~(2) sa redukuje na
nerovnosť $n^2-5n+6\le0$, ktorá však pre žiadne $n\ge6$
neplatí.

Ak je $p=2$, platí podľa (1) $n\ge9$ a~(2) sa redukuje na
nerovnosť $n^2-11n+12\le0$, ktorá neplatí pre žiadne $n\ge10$.
Prípad $p=2$ je tak možný jedine pre $n=9$, ktoré skutočne
vyhovuje (pozri pôvodné riešenie). Pre ďalšie vyhovujúce $n$ preto musí byť
$p\ge 3$, teda podľa~\thetag1 $n\ge12$.

Teraz iným spôsobom ukážeme,
že množiny $\mm M_1$, $\mm M_2$, $\mm M_3$ zostrojené
v~pôvodnom riešení pre každé $n\ge12$
vyhovujú, a~to priamym vyjadrením prislúchajúcich rozdielov
$s_1-s_2$ a~$s_2-s_3$, o~ktorých máme ukázať, že sú oba kladné.
Využijeme na to opäť vyjadrenie $n=3k+r$
a~prvky navrhnutých množín zapíšeme do riadkov v~{\it klesajúcom\/}
poradí:
$$
\postdisplaypenalty=10000
\matrix
\mm M_1\:&3k+r&3k+r-1&\dots&2k+r+2&&&\\
\mm M_2\:&2k+r+1&2k+r&\dots&k+r+3&k+r+2&&\\
\mm M_3\:&k+r+1&k+r&\dots&r+3&r+2&r+1&\dots\\
\endmatrix
$$
(Tri bodky v~poslednom riadku znamenajú čísla $r,\dots,1$ v~prípade
$r>0$.)

Všimnime si, že pod sebou zapísané prvky množín $\mm M_1$ a~$\mm M_2$
majú ten istý rozdiel, rovný číslu $k-1$. Takých dvojíc je $k-1$, pritom
v~množine $\mm M_2$ je "navyše" posledné ($k$-te) číslo $k+r+2$.
To vedie k~prvému zo vzorcov
$$
s_1-s_2=(k-1)^2-(k+r+2),\quad
s_2-s_3=k^2-\sum_{j=1}^{r+1}j,
$$
druhý vzorec sa dokáže podobnou úvahou o~prvkoch množín $\mm M_2$ a~$\mm M_3$,
tiež zapísaných pod sebou. Že sú oba získané rozdiely za
predpokladov $k\ge4$ a~$r\in\{0, 1, 2\}$ kladné, je zrejmé:
$$
\aligned
s_1-s_2&\ge(k-1)^2-(k+4)=k(k-3)-3\ge4\cdot1-3=1,\\
s_2-s_3&\ge k^2-(1+2+3)\ge4^2-6=10.
\endaligned
$$

\nobreak\medskip\petit\noindent
Za úplné riešenie dajte 6~bodov rozdelených takto:

\noindent
2~body za dôkaz neexistencie rozdelenia pre $n <12$, $n \ne 9$
(ak dôkaz pre niektoré $n$ chýba, dajte za túto časť nanajvýš 1~bod; ak
riešiteľ vylúči iba prípady $n <6$, tak 0~bodov);

\noindent
1~bod za dôkaz existencie vhodného rozdelenia pre $n = 9$;

\noindent
3~body za dôkaz existencie vhodného rozdelenia pre $n \ge 12$
(v~neúplnom riešení nanajvýš 1~bod za popis vyhovujúceho rozdelenia,
ak chýba zdôvodnenie správnosti).

\endpetit
\bigbreak
}

{%%%%%   A-S-3
Po dosadení dĺžok úsekov $AE$, $AD$, $BE$, $CD$ vyjadrených
pomocou dĺžok $b$ a~$c$ strán trojuholníka $ABC$ a~kosínusu uhla
$\alpha=|\uhel BAC|$ do rovnosti zo zadania dostaneme
$$
b \cos \alpha \cdot c \cos \alpha = (c-b \cos \alpha) (b-c \cos \alpha),
$$
čiže $bc = (b^2+c^2) \cos \alpha$. Odtiaľ
$$
\cos\al=\frac{bc}{b^2+c^2}\le\frac12,
$$
pričom posledná nerovnosť je pre ľubovoľné kladné čísla $b$ a~$c$
ekvivalentná zrejmej nerovnosti $(b-c)^2\ge0$ (možno sa tiež
odvolať na AG-nerovnosť pre dvojicu čísel $b^2$ a~$c^2$).
Dokázali sme tak, že pre uhol~$\al$ ľubovoľného trojuholníka~$ABC$
vyhovujúceho zadaniu úlohy platí $\cos\al\le1/2$, čiže
$\al\ge60\st$. Keďže k~týmto vyhovujúcim trojuholníkom zrejme
patrí aj každý rovnostranný trojuholník (v~ktorom sú totiž $AE$, $AD$, $BE$,
$CD$ štyri zhodné úsečky), je $\al=60\st$ hľadané minimum.

\ineriesenie
Označme dĺžky úsekov $AE$, $EB$, $AD$, $DC$ postupne $p$, $q$, $r$,
$s$. Podľa predpokladu zo zadania platí $pr = qs$.
Z~mocnosti bodu~$A$ k~Tálesovej kružnici nad priemerom~$BC$, ktorá
prechádza bodmi $D$ a~$E$, pre dĺžky týchto úsekov dostaneme
$$
p (p+q) = r (r+s).
$$
Po vyjadrení $s= pr / q$ z~prvého vzťahu, dosadení do druhého
a~po zjednodušení dostaneme $pq = r^2$.
Označme $S$ stred strany~$AB$ a~$c$ jej dĺžku. Platí
$$
r^2 = pq = |AE| \cdot |BE| = \Big(\frac c2+|SE|\Big) \Big(\frac c2-|SE|\Big) = \frac{c^2}4-|SE|^2,
$$
preto $r^2 \le c^2/4$ čiže $r/c\leqq1/2$.
Podiel $r/c=|AD|/|AB|$ je
však z~pravouhlého trojuholníka~$ABD$ rovný kosínusu skúmaného uhla
$BAC$, takže sme odvodili rovnakú nerovnosť ako v~prvom
riešení, ktorého záver už opakovať nebudeme.

\ineriesenie
Ešte jedným, trochu prácnejším algebraickým výpočtom odvodíme
kľúčovú nerovnosť $\cos \al\le1/2$ pre uhol $\al=|\uhel BAC|$
každého trojuholníka~$ABC$ vyhovujúceho zadaniu úlohy. Vyjadríme všeobecne
dĺžky úsekov $AE$, $AD$, $BE$, $CD$ pomocou dĺžok
strán trojuholníka $ABC$ (štandardne označených $a$, $b$, $c$).
Vďaka kosínusovej vete platí
$$
|AE| = |AC| \cdot \cos \alpha = b \cdot
\frac {c^2+b^2-a^2} {2bc} = \frac {c^2+b^2-a^2} {2c}
$$
a~analogicky tiež
$$
|AD| = \frac {b^2+c^2-a^2} {2b}, \qquad |BE| = \frac {c^2+a^2-b^2} {2c}, \qquad
|CD| = \frac {b^2+a^2-c^2} {2b}.
$$
Dosadením získaných rovností do vzťahu zo zadania a~jeho následným
vynásobením spoločným menovateľom~$4bc$ dostaneme polynomickú
rovnosť, ktorú teraz zapíšeme a~upravíme:
$$
\aligned
\bigl(b^2+c^2-a^2\bigr)^2&=\bigl(a^2-b^2+c^2\bigr)\bigl(a^2+b^2-c^2\bigr),\\
a^4+b^4+c^4-2a^2\bigl(b^2+c^2\bigr)+2b^2c^2&=a^4-\bigl(b^2-c^2\bigr)^2,\\
% \bigl(b^2+c^2\bigr)^2-2a^2\bigl(b^2+c^2\bigr)+a^4&=a^4-\bigl(b^2-c^2\bigr)^2,\\
% \bigl(b^2+c^2\bigr)^2+\bigl(b^2-c^2\bigr)^2&=2a^2\bigl(b^2+c^2\bigr),\\
2(b^4+c^4)&=2a^2(b^2+c^2),\\
a^2&=\frac{b^4+c^4}{b^2+c^2}.
\endaligned
$$
Takto určenú hodnotu $a^2$ dosadíme do už skôr použitého
vyjadrenia $\cos\al$:
$$
\cos \alpha =
\frac {b^2+c^2-a^2} {2bc} = \frac {b^2+c^2- \dfrac {b^4+c^4} {b^2+c^2}} {2bc}
= \frac {bc} {b^2+c^2}
\leq \frac12.
$$


\nobreak\medskip\petit\noindent
Za úplné riešenie dajte 6 bodov, z~toho 5 bodov za
zdôvodnenie $\alpha \ge 60^\circ$ a~1 bod za výslovnú zmienku o~tom,
že rovnostranný trojuholník vyhovuje zadaniu.
Neúplné riešenie: za odvodenie nerovnosti $\cos \alpha \le 1/2$
dávajte 4~body; za odvodenie rovnosti, z~ktorej je možné bez
ďalších geometrických úvah zhora odhadnúť veľkosť
$\cos \alpha$, nanajvýš 2~body (ak tento odhad chýba). Za triviálne
použitie kosínusovej vety, z~ktorého riešiteľ neodvodil vzťah
umožňujúcí horný odhad veľkosti uhla $\alpha$, je 0 bodov.

\endpetit
\bigbreak
}

{%%%%%   A-II-1
Uvažovaná trojica zlomkov je symetrická v~tom zmysle, že ak nahradíme trojicu
celých čísel $(a, b, c)$ ich ľubovoľnou permutáciou, dostaneme zasa
(až na poradie) tú istú trojicu zlomkov. To isté platí, keď nahradíme čísla $a$, $b$, $c$
číslami opačnými. Táto skutočnosť nám v~ďalšom zjednoduší rozbor prípadov.

Predpokladajme teda, že čísla $a$, $b$, $c$ sú také, že všetky tri
uvažované zlomky majú celočíselnú hodnotu.
Ak sa medzi nimi nachádza nula, stačí bez ujmy na
všeobecnosti vyšetriť prípad $a=0$.
Po dosadení do uvažovaných zlomkov dostávame, že zlomky $b / c$
a~$c / b$ majú celočíselnú hodnotu. Z~toho vyplýva, že $b$ aj~$c$ sú
nenulové a~$|b|\ge |c|$ a~zároveň aj $|c| \ge |b|$, preto $c = {\pm b}$. Navyše
číslo $b+c$ je menovateľom prvého zlomku, preto $b+c \ne 0$, takže
musí byť $b=c$. Celkovo tak dostávame (zjavne
vyhovujúce) trojice $(0, c, c)$ a~ich permutácie pre každé nenulové
celé číslo~$c$.

Ostáva vyriešiť prípad, keď $abc \ne 0$.

Vzhľadom na pozorovanie z~prvého odseku budeme predpokladať, že aspoň
dve z~čísel $a$, $b$, $c$ sú kladné. Keby boli kladné všetky tri,
ležal by zlomok, ktorý má v~čitateli najmenšie z~čísel $a$, $b$, $c$, medzi $0$ a~$1$, takže by nemohol mať celočíselnú hodnotu.

Nech teda $a$, $b$ sú kladné čísla a~$c =\m d$ pre kladné~$d$.
Po dosadení do zadania dostaneme, že zlomky
$$
{a~\over d-b},\quad {b \over d-a},\quad {d \over a+b}
$$
majú celočíselnú hodnotu. Z~posledného z~nich je jasné, že $d \ge a+b$.
Preto má prvý zlomok kladný menovateľ, a~keďže jeho hodnota je celé
číslo, musí platiť $a~\ge d-b$, čiže $d \le a+b$.
Odtiaľ nutne $d = a+b$, čiže $c=\m a-b$ a~dostávame tak v~súhrne
trojice $(a, b, c)$ nenulových čísel, pre ktoré platí $a+b+c = 0$. Všetky
také trojice vyhovujú, lebo hodnota všetkých troch uvažovaných zlomkov
je pre ne rovná~${-1}$.

\odpoved
Úlohe vyhovujú všetky trojice $(0, c, c)$, $(c, 0, c)$ a~$(c, c, 0)$,
pričom $c$ je nenulové celé číslo, a~všetky trojice $(a, b, c)$ nenulových celých čísel,
pre ktoré platí $a+b+c = 0$.

\nobreak\medskip\petit\noindent
Za úplné riešenie dajte 6~bodov.
Ak chýba zmienka o~niektorej
z~vyhovujúcich trojíc (napr. sa nikde nespomína iná permutácia
trojice $(0, c, c)$), dajte nanajvýš 5~bodov.
Neúplné riešenie: za úplné riešenie prípadu, keď je jedno z~čísel $a$,
$b$, $c$ rovné~$0$, dajte 2~body; za vyriešenie prípadu, keď $a$, $b$,
$c$ sú nenulové, dajte 4~body, z~toho 1~bod za púhe vylúčenie prípadov,
keď všetky čísla $a$, $b$, $c$ sú kladné, či záporné.
Za pozorovanie z~prvého odseku o~permutáciách trojíc a~zmene znamienok,
ak nevedú k~vyriešeniu jedného z~uvedených dvoch prípadov, body nedávajte.
Za nájdenie všetkých riešení (bez zdôvodnenia, prečo iné už neexistujú) dajte 1~bod.
\endpetit
\bigbreak
}

{%%%%%   A-II-2
Kružnica, ktorej časťou je polkružnica~$q$, je obrazom kružnice~$p$
v~rovnoľahlosti so stredom~$M$ a~koeficientom~$1/2$, takže bod~$Q$
je stredom úsečky~$RM$.
Keďže oba trojuholníky, pomer ktorých obsahov nás zaujíma, majú navyše
zhodné uhly pri spoločnom vrchole~$Q$, je
$$
{S_1 \over S_2} ={\frac 12|P_1Q|\cdot|MQ|\sin|\uhel P_1QM|\over
\frac 12|P_2Q|\cdot|RQ|\sin|\uhel P_2QR|}
={|P_1Q|\over|P_2Q|}.
$$

Označme $|KM| = r$, $|ML| = x$, $|P_1L| = d_1$, $|QL| = d_2$ (\obr).
Zo súmernosti bodov $P_1$, $P_2$ podľa $KM$ potom vyplýva $|P_1L|=|P_2L|$,
takže $|P_2Q|=d_1-d_2$. Ak označíme $M'$ druhý krajný bod priemeru kružnice~$p$
s~krajným bodom~$M$, je trojuholník $M'MP_1$ pravouhlý.
Z~Euklidovej vety pre jeho výšku~$P_1L$ dostávame
$d_1^2 = x (2r-x)$ a~podobne pre výšku~$QL$ pravouhlého trojuholníka~$KQM$ platí
$d_2^2 = x (r-x)$. Odtiaľ
$$
\align
{S_1 \over S_2} &={|P_1Q|\over|P_2Q|} = {d_1+d_2 \over d_1-d_2} = {(d_1+d_2)^2 \over
d_1^2-d_2^2} =\\ &= {x (2r-x)+x (r-x)+2 \sqrt {x (2r-x) \cdot x (r-x)} \over rx}
 = {3r-2x+2 \sqrt {(2r-x) (r-x)} \over r}.
\endalign
$$

Na výsledný výraz sa možno pozerať ako na funkciu premennej $x$ s~parametrom~$r$.
Na intervale $\langle0, r \rangle$ je táto funkcia klesajúca
(obe funkcie $3r-2x$ aj $(2r-x)(r-x)$ sú klesajúce), preto nadobúda
maximum $3+2 \sqrt 2$ pre $x = 0$ a~minimum~1 pre $x = r$. Keďže podľa
zadania $x \in (0, r)$, platí $1 <S_1: S_2 <3+2 \sqrt 2=3+\sqrt8$.
\insp{a66.7}%

\ineriesenie
Body $P_2$, $R$, $P_1$, $M$ ležia v~tomto poradí na kružnici~$p$, preto
majú uhly $MP_1P_2$ a~$MRP_2$ rovnakú veľkosť. Zhodné sú
aj vrcholové uhly $P_1QM$ a~$P_2QR$. Podľa vety {\it uu\/} sú teda
trojuholníky $MP_1Q$ a~$P_2RQ$ podobné, takže pomer ich obsahov je
štvorcom pomeru dĺžok prislúchajúcich si strán $MQ$ a~$P_2Q$. Keďže
$3+\sqrt8=\bigl(1+\sqrt2\bigr){}^2$,
je naším cieľom dokázať nerovnosti
$$
1<\frac{|MQ|}{|P_2Q|}<1+\sqrt2. \tag1
$$

Označme opäť $x=|LM|$ a~$r$ polomer kružnice~$p$, takže
$|KL|=r-x$ (\obrr1). Využijeme to, že trojuholník $KQM$ je
pravouhlý, preto pre jeho výšku~$LQ$ a~odvesnu~$MQ$ podľa
Euklidových viet platí $|LQ|=\sqrt{x(r-x)}$ a~$|MQ|=\sqrt{rx}$.
Aj trojuholník $KLP_2$ je pravouhlý, preto podľa Pytagorovej vety
$|LP_2|=\sqrt{r^2-(r-x)^2}=\sqrt{x(2r-x)}$. Dostávame tak
$$
\align
\frac{|MQ|}{|P_2Q|}&=\frac{|MQ|}{|LP_2|-|LQ|}=
\frac{\sqrt{rx}}{\sqrt{x(2r-x)}-\sqrt{x(r-x)}}=
\frac{\sqrt{r}}{\sqrt{2r-x}-\sqrt{r-x}}=\\
&=\frac{\sqrt{2r-x}+\sqrt{r-x}}{\sqrt{r}}.
\endalign
$$
Vďaka nerovnostiam $0<x<r$ pre čitateľ posledného zlomku platia
odhady
$$
\sqrt{2r-r}+\sqrt{r-r}<\sqrt{2r-x}+\sqrt{r-x}<\sqrt{2r}+\sqrt{r},
$$
ktoré už bezprostredne vedú k~nerovnostiam~(1).


\poznamka
Nerovnosť $S_1 : S_2>1$ vyplýva rovno z~predpokladanej nerovnosti
$|P_1 Q| > |P_2 Q|$, pretože potom zrejme platí
$S_1>S_{LMP_1}=S_{LMP_2}>S_{MQP_2}=S_2$, lebo $|RQ|=|QM|$.

\nobreak\medskip\petit\noindent
Za úplné riešenie dajte 6~bodov, z~toho 2~body za dôkaz nerovnosti $1 <
S_1: S_2$ a~4~body za dôkaz druhej nerovnosti $S_1: S_2 <3+ \sqrt {8}$.
Riešenia s~drobnými nedostatkami v~dôkazoch (chýba potrebná zmienka
o~ekvivalencii nerovností; numerická chyba v~závere dôkazu a~pod.)
hodnoťte 5~bodmi.

Hodnotenie neúplných riešení:
\item{$\triangleright$} 1~bod za vyjadrenie hodnoty $S_1: S_2$ pomocou dĺžok úsečiek~--
či už úvahou o~rovnoľahlosti (prvé riešenie), na základe podobnosti
trojuholníkov $MP_1Q$ a~$P_2RQ$ (druhé riešenie) alebo inak (ak chýba
zmienka o~dôsledkoch podobnosti či rovnoľahlosti pre výpočet pomeru
obsahov, body nedávajte);
\item{$\triangleright$} ďalší 1~bod za vyjadrenie pomeru $S_1: S_2$ ako funkcie jednej
reálnej premennej;
\item{$\triangleright$} nanajvýš 2~body za riešenie, kde nie je preukázaná ani jedna zo
zadaných nerovností.

\endpetit
\bigbreak
}

{%%%%%   A-II-3
Danú sústavu vyriešime a~vypíšeme všetky riešenia, aby sme niektoré riešenia nepočítali viackrát.

Najskôr zvážime možnosť $x = y = z$. V~takom prípade sa sústava redukuje na jedinú
rovnicu $(k+2) x^2 = x$. Riešením danej sústavy je trojica $(0,0,0)$ pre ľubovoľné~$k$
a~navyše trojica $\big({1 \over k+2}, {1 \over k+2}, {1 \over k+2} \big)$ v~prípade
$k\ne-2$.

Vráťme sa k~danej sústave rovníc.
Odčítaním druhej rovnice od prvej dostávame
$$
(x^2-z^2)+ky (x-z)= z-x,
$$
čiže
$$
(x-z) (x+z+ky+1)= 0. \tag 1
$$
Podobne odčítaním tretej rovnice od druhej vyjde
$$
(y-x) (y+x+kz+1) = 0. \tag 2
$$

V~prípade $x \ne y\ne z\ne x$ sa tak rovnice (1) a~(2)
redukujú na
$$
\align
x+z+ky+1&= 0, \\
y+x+kz+1&= 0.
\endalign
$$
Odčítaním týchto dvoch rovníc dostaneme $(y-z)(k-1) = 0$, takže musí byť
$k=1$ a~$x+y+z=-1$. To však nemôže platiť, keďže pre $k=1$ vychádza
$$
\postdisplaypenalty 10000
z= x^2+xy+y^2 = \left (x+\frac {y}{2} \right)^{\!2}+\frac {3y^2}{4} \ge 0
$$
a~podobne $x \ge 0$ a~$y \ge 0$, takže spolu $x+y+z\ge0$.

Zistili sme, že v~každom riešení zadanej sústavy majú niektoré dve
neznáme rovnakú hodnotu.
Keďže sústava je cyklická, budeme ďalej predpokladať $x\ne y = z$,
pretože prípad $x = y = z$ sme už vyriešili.
Z~rovnice~(1) za týchto predpokladov vyplýva, že $x+y+ky+1 = 0$, čiže
$x = -(k+1)y -1$,
a~pôvodná sústava sa tým redukuje na jedinú rovnicu
$$
(k+2)y^2 +(k+1)y +1 = 0. \tag 3
$$

Ešte dodajme, že riešením rovnice~(3) nedostaneme žiadne z~už nájdených riešení,
pretože rovnosť $x=y$, čiže $y = -(k+1)y -1$ je možná len pre $k\ne-2$ a~dáva
$x = y = z=-{1/(k+2)}$, čo medzi riešenia danej sústavy nepatrí.

Pre $k=\m2$ je rovnica~(3) lineárna s~jediným riešením $y = 1$, pre~ktoré dopočítame $x = 0$.
Riešeniami danej sústavy sú tri permutácie trojice $(0,1,1)$.

Pre $k\ne\m2$ je rovnica \thetag3 kvadratická a~má reálne riešenie práve vtedy, keď
$$
D = (k+1)^2-4 (k+2) = k^2-2k-7 \ge 0,
$$
čiže práve vtedy, keď $k\notin(1-2 \sqrt2, 1+2 \sqrt2)$,
ako zistíme vyriešením výslednej kvadratickej nerovnice pre~$k$. Riešenie
je teda jediné pre $k=1\pm2 \sqrt 2$, pričom
$$
y_0 =-{k+1 \over 2 (k+2)}=1\mp\sqrt2 \qquad \hbox {a} \qquad
x_0 = {(k+1)^2 \over2 (k+2)}-1=1.
$$
Riešením pôvodnej sústavy tak sú tri permutácie trojice $(x_0, y_0, y_0)$.

Pre $k\in (-\infty,-2)\cup(-2,1-2 \sqrt2) \cup (1+2 \sqrt2, \infty)$
má kvadratická rovnica~(3) dve rôzne riešenia
$$
y_{1,2} = {-k-1 \pm \sqrt {k^2-2k-7} \over 2 (k+2)},
$$
pre ktoré dostaneme dve rôzne hodnoty $x_{1,2} = \m(k+1)y_{1,2} -1$.
Riešením pôvodnej sústavy tak sú tri permutácie trojice $(x_1, y_1, y_1)$
a~tri permutácie trojice $(x_2, y_2, y_2)$.

V~záverečnej tabuľke uvádzame celkový počet riešení danej sústavy
v~závislosti od~$k$:
$$
\vbox
{\let\\=\cr
\halign{\hss#\unskip\hss&&\quad\hss#\unskip\hss\cr
interval pre $k$&$(0,0,0)$&$\big({1 \over k+2}, {1 \over
k+2}, {1 \over k+2}\big)$&rovnica \thetag{3}&celkom \\ \noalign{\vskip2pt\hrule\vskip2pt}
$(- \infty,-2)$&1&1& 6&8 \\
$-2$&1&0& 3&4 \\
$(- 2, 1-2 \sqrt2)$&1&1& 6&8 \\
$1-2 \sqrt2$&1&1& 3&5 \\
$(1-2 \sqrt 2, 1+2 \sqrt 2)$&1&1& 0&2 \\
$1+2 \sqrt 2$&1&1& 3&5 \\
$(1+2 \sqrt2, \infty)$&1&1& 6&8\\
}}
$$

\nobreak\medskip\petit\noindent
Za úplné riešenie dajte 6~bodov, z~toho 1~bod za vyriešenie prípadu
$x = y = z$, 2~body za dôkaz neexistencie riešenia v~prípade $x \ne y$,
$y \ne z$, $z\ne x$ a~3~body za zvyšný prípad (z~toho 1~bod za
vyriešenie situácie $k=\m2$). Za drobné nedostatky (nesprávne určené
korene trojčlena $k^2-2k-7$, nesprávne dopočítaná hodnota~$x$
z~hodnoty~$y$ a~pod.) strhnite dokopy iba 1~bod.
Vypisovať všetky trojice riešiace danú sústavu nie je
v~úplnom riešení nevyhnutné, ak je v~postupe
zdôvodnené, že trojice zodpovedajúce riešeniam rovnice~(3) nie sú tvorené tromi
rovnakými číslami. Ak toto zdôvodnenie v~inak úplnom riešení chýba, dajte 5~bodov.
Ak jediná chyba
pri určení počtu riešení je zabudnutie permutácií premenných, strhnite 1~bod.

Za uhádnutie riešenia (i~keď overené skúškou) nedávajte žiadne body,
ak nie je úplne vyriešený ani jeden zo spomenutých troch prípadov
(vrátane zdôvodnenia, že iné riešenia už neexistujú). Postup, keď
riešiteľ pre niektorú hodnotu~$k$ uhádne množinu riešení a~dokáže, že
viac riešení pre dané $k$ neexistuje (napr. pomocou redukcie sústavy na
polynomickú rovnicu pre jednu z~neznámych): ak nie je ani naznačené,
ako možno pomocou tohto postupu určiť, ako sa mení počet riešení
sústavy v~závislosti od $k$, dajte 1~bod; inak nanajvýš 6~bodov
v~závislosti na tom, koľko z~podstatných intervalov pre $k$ (pozri tabuľku
v~závere uvedeného riešenia) umožnil tento postup vyriešiť.

\endpetit
\bigbreak
}

{%%%%%   A-II-4
Označme $K$ priesečník úsečiek $FG$ a~$AE$ a~$L$ priesečník $EH$ a~$AF$ (\obr).
Z~rovnosti obvodových uhlov nad tetivou~$AF$ (body $A$, $G$, $E$, $F$
ležia v~tomto poradí na kružnici opísanej trojuholníku~$AEF$) vyplýva, že
$$
|\uhel AGF |=|\uhel AEF |= 90^\circ-|\uhel DAE |= 90^\circ-|\uhel GAE |,
$$
čiže
$$
|\uhel AGF |+|\uhel GAE |=90^\circ,
$$
takže
$$
|\uhel AKG |= 180^\circ-(|\uhel AGF |+|\uhel GAE |) = 90^\circ.
$$
Úsečka~$FK$ je teda výškou trojuholníka $AEF$.
Analogicky dokážeme, že aj $EL$ je jeho výškou,
a~preto priesečník priamok $FK$ a~$EL$ je ortocentrom trojuholníka $AEF$,
ktorým prirodzene prechádza aj jeho tretia výška~$AD$.
\insp{a66.8}%

\ineriesenie
Označme $M$ ďalší priesečník polpriamky~$AD$ s~kružnicou~$k$
opísanou trojuholníku $AEF$ (\obrr1).
Keďže $AE$ je osou uhla $GAM$, sú zhodné obvodové uhly $GAE$ a~$EAM$
v~kružnici~$k$, a~preto sú zhodné aj jej tetivy $GE$ a~$EM$,
a~teda aj obvodové uhly $GFE$ a~$EFM$.
Priesečník
priamok $FG$ a~$AM$ je teda obrazom bodu~$M$ v~osovej súmernosti s~osou~$EF$.
Rovnakú úvahu môžeme spraviť aj pre priesečník priamok $EH$ a~$AM$, preto
musia byť oba priesečníky totožné.


\nobreak\medskip\petit\noindent
Za úplné riešenie dajte 6~bodov. Okrem postupu v~prvom riešení
možno využiť aj rovnosť obvodových uhlov nad tetivou~$GE$ a~ukázať,
že body $A$, $K$, $D$, $F$ ležia na kružnici s~priemerom~$AF$. Za hypotézu (nezdôvodnené
pozorovanie), že $FG \perp AE$ (resp. $EH \perp AF$), dávajte 2~body, ak
riešiteľ zdôvodní, že to stačí na vyriešenie úlohy (napr.
úvahou o~priesečníku výšok trojuholníka $AEF$), inak iba 1~bod. Riešiteľovi, ktorý
pracuje s~bodom~$M$ z~druhého riešenia a~zdôvodní, že $E$ je
stred oblúka~$GM$ (alebo že $F$ je stredom oblúka~$HM$),
nedostane sa však ďalej, dajte 2~body (samotné zavedenie bodu~$M$ nebodujte).
\endpetit
\bigbreak
}

{%%%%%   A-III-1
Ukážeme stratégiu znalca, pri ktorej sa nám odhaliť všetkých 50~pravých
diamantov nepodarí. Znalec si zapamätá jeden pravý diamant~$P$ a~jeden falošný
diamant~$F$ (napríklad prvý pravý a~prvý falošný diamant, ktoré sú mu
predložené). Kedykoľvek sa ho pýtame na trojicu, v~ktorej je práve jeden
diamant z~dvojice $P$ a~$F$, vyjadrí sa o~zvyšných dvoch.
Ak sú v~trojici $P$ aj~$F$, tak ich oba vyberie a~pravdivo o~nich
povie, že jeden z~nich je pravý. Ak v~trojici nie je ani~$P$, ani~$F$,
postupuje ľubovoľne.

Pri opísanej stratégii sa nedá zistiť, ktorý z~kameňov $P$ a~$F$ je pravý a~ktorý
falošný, pretože ich žiadna zo znalcových odpovedí nerozlíši.
}

{%%%%%   A-III-2
Predpokladajme, že daná nerovnosť pre nejakú dvojicu $k$, $l$ platí
pre dĺžky strán $a$, $b$, $c$ ľubovoľného trojuholníka.
Keď do nej dosadíme $a= 1$, $c = 1$ a~ľubovoľné kladné $b<2$ (trojuholník
s~takýmito dĺžkami strán zjavne existuje), dostaneme nerovnosť $k+lb^2>1$.
Keby bolo $k<1$, ľahko by sme našli $b>0$ dostatočne malé na to, aby táto
nerovnosť už neplatila, preto musí byť $k\ge 1$.
Analogicky musí byť $l \ge 1$.

V~karteziánskej súradnicovej sústave zvoľme body $A[-1,0]$,
$B[1,0]$, $C[x, y]$; body $A$, $B$, $C$ sú vrcholmi trojuholníka pre
ľubovoľné reálne $x$ a~ľubovoľné reálne $y \ne 0$ a~až na podobnosť tak možno
umiestniť ľubovoľný trojuholník. Po dosadení dĺžok strán
trojuholníka~$ABC$ (ľahko ich vypočítame z~Pytagorovej vety) prejde
zadaná nerovnosť na tvar
$$
k\big((x-1)^2+y^2\big)+l\big((x+1)^2+y^2\big)> 4,
$$
čiže
$$
(k+l) x^2+2 (l-k) x+k+l-4>-(k+l)y^2. \tag1
$$
Tá musí platiť pre každé $x$ a~ľubovoľné $y \ne 0$.
Pritom pre pevné~$x$ dokážeme voľbou hodnoty~$y$ dosiahnuť ľubovoľné záporné
hodnoty výrazu $V(y) =\m(k+l) y^2$ na pravej strane predchádzajúcej nerovnosti, lebo
(ako už vieme) $k+l>0$. Preto musí pre každé~$x$ platiť
$$
(k+l)x^2+2 (l-k) x+(k+l-4) \ge 0, \tag2
$$
čo vzhľadom na kladný koeficient pri mocnine~$x^2$ na ľavej strane
nastane práve vtedy, keď príslušný diskriminant $D =4(l-k)^2-4(k+l-4)(k+l)$
nie je kladný. Nerovnosť $D\le0$ ľahko upravíme na ekvivalentnú podmienku
$kl\ge k+l$.

Zistili sme teda, že ak daná nerovnosť platí pre ľubovoľnú trojicu
dĺžok strán trojuholníka, spĺňajú čísla $k$ a~$l$ okrem nerovností
$k\ge 1$ a~$l \ge 1$ aj~podmienku
$$
kl \ge k+l . \tag3
$$

Z~úvahy o~diskriminante naopak vyplýva, že ak čísla $k\ge1$ a~$l\ge1$
podmienku~(3) spĺňajú, platí nerovnosť \thetag{2} pre každé reálne~$x$.
Keďže pre také $k$, $l$ je hodnota~$V(y)$ pre každé $y \ne 0$
záporná, vyplýva z~platnosti \thetag{2} pre ľubovoľné~$x$ nerovnosť~\thetag{1}
pre všetky prípustné dvojice $x$, $y$, a~tá už je
ekvivalentná zadanej vlastnosti. Uvedená podmienka je teda
aj postačujúca.

Konjunkcia nerovností $k\ge1$, $l\ge1$ a~$kl\ge k+l$ je teda podmienka nutná
i~postačujúca. Keďže z~tretej nerovnosti vyplýva $k\ne1$ (rovnako ako $l\ne1$),
môžeme hľadanú množinu vyhovujúcich dvojíc $(k,l)$ zapísať zrejme takto:
$$
\{(k,l)\: k>1\land l\ge k/(k-1)\}.
$$


\ineriesenie
Predpokladajme, že daná nerovnosť pre nejakú dvojicu $k$, $l$ platí
pre dĺžky strán $a$, $b$, $c$ ľubovoľného trojuholníka.
Z~platnosti nerovnosti pre trojuholník, v~ktorom $a=1$, $b = 1$,
$c = \sqrt2$, vyplýva, že $k+l> 2$, čiže aspoň jedno z~čísel $k$, $l$
musí byť väčšie ako~$1$. Nech je teda napríklad $k>1$
(prípad $l>1$ sa posúdi analogicky).

Podľa kosínusovej vety platí $c^2 = a^2+b^2-2ab \cos \gamma$.
Dosadením do zadanej nerovnosti dostaneme po úprave ekvivalentnú
nerovnosť
$$
a^2 (k-1)+b^2 (l-1)+2ab \cos \gamma> 0.
$$

Vďaka predpokladu $k>1$ možno pre ľubovoľné $\gamma\in(90\st,180\st)$
zvoliť trojuholník s~tupým uhlom $\gamma$ a~so stranami $a={\m\cos\gamma}>0$ a~$b=k-1$.
Dosadením do poslednej nerovnosti po jednoduchej úprave vidíme,
že pre čísla $k$ a~$l$ musí platiť
${(k-1)(l-1)}> \cos^2 \gamma$.
Pritom pre $\gamma\in(90^\circ, 180^\circ)$
môže výraz $\cos^2 \gamma$ nadobúdať každú hodnotu
z~intervalu $(0,1)$. Aby posledná nerovnosť platila pre všetky uvedené
hodnoty uhla~$\gamma$, musí nutne platiť $(k-1) (l-1) \ge 1$. A~keďže $k>1$,
musí byť aj $l>1$.

Dokážeme, že spolu s~oboma podmienkami $k>1$ a~$l>1$ je odvodená podmienka
$$
(k-1) (l-1) \ge 1 \tag4
$$
aj postačujúca. Pri ich splnení sú čísla $a^2(k-1)$ a~$b^2(l-1)$
kladné, a~platí tak pre ne nerovnosť medzi aritmetickým a~geometrickým
priemerom, preto
$$
\align
a^2 (k-1)+b^2 (l-1)+2ab \cos \gamma \ge& 2ab \sqrt {(k-1) (l-1)}+2ab\cos \gamma\ge\\
\ge& 2ab+2ab \cos \gamma =2ab (1+ \cos \gamma)> 0,
\endalign
$$
čím je dôkaz ukončený.

Opíšme tentoraz hľadanú množinu vyhovujúcich dvojíc $(k,l)$ geometricky,
a~to bodmi so súradnicami $[k,l]$ v~karteziánskej sústave súradníc $Okl$.
Rovnosť v~(4) popisuje rovnoosovú hyperbolu so stredom v~bode $[1,1]$
a~asymptotami s~rovnicami $k=1$ a~${l=1}$. Preto nerovnosť~(4) spolu
s~podmienkami $k>1$ a~$l>1$ určuje časť prvého kvadrantu "nad" tou vetvou hyperboly,
ktorá v~ňom celá leží, pritom samotné body vetvy do určenej množiny,
ktorú sme mali nájsť, taktiež patria.


\ineriesenie
Vysvetlime najskôr, že dvojica reálnych čísel $(k,l)$ má
požadovanú vlastnosť práve vtedy, keď pre ľubovoľné kladné čísla $a$,
$b$ platí
$$
ka^2+lb^2\ge(a+b)^2. \tag5
$$
Táto podmienka je určite postačujúca, lebo pre strany $a$, $b$, $c$
každého trojuholníka platí $a+b>c>0$, a~teda aj $(a+b)^2>c^2$, čo spolu
s~(5) vedie na nerovnosť zo zadania úlohy. Keby naopak pre niektoré
kladné čísla $a$, $b$ nerovnosť~(5) neplatila, bola by sústava
nerovností
$$
ka^2+lb^2<x^2<(a+b)^2
$$
splnená pre každé $x$ z~nejakého otvoreného intervalu s~pravým
krajným bodom $a+b$, a~tak by sme v~ňom určite našli $x=c$ väčšie
ako $|a-b|$, pretože $|a-b|<a+b$. Trojuholník so stranami $a$, $b$, $c$ by potom
nespĺňal nerovnosť zo zadania úlohy. Naša podmienka
spojená s~nerovnosťou~(5) je tak nielen postačujúca,
ale aj nutná na to, aby dvojica čísel $(k,l)$ vyhovovala zadaniu.

Ak upravíme nerovnosť~(5), ktorá má platiť pre všetky $a,b>0$, na tvar
$$
(k-1)a^2+(l-1)b^2\ge 2ab, \tag6
$$
vidíme, že je nutne $k>1$, pretože v~prípade $k\le1$ by ľavá
strana~(6) pri pevnom $b>0$ bola v~premennej $a\in(0,\infty)$ zhora
ohraničená, zatiaľ čo pravá strana nie. Rovnako tak je nutne $l>1$.
Preto možno nerovnosť (6) upraviť na tvar
$$
\bigl(a\sqrt{k-1}-b\sqrt{l-1}\bigr)^2\ge2\bigl(1-\sqrt{(k-1)(l-1)}\bigr)ab.
$$

Ukážeme, že za predpokladu $k,l>1$ posledná nerovnosť platí
pre všetky $a,b>0$ práve vtedy, keď je $(k-1)(l-1)\ge1$. Nutnosť
tejto podmienky vyplýva po dosadení (kladných) hodnôt $a=\sqrt{l-1}$
a~$b=\sqrt{k-1}$, jej postačujúcosť je zrejmá
z~toho, že pravá strana potom bude nekladná, zatiaľ čo ľavá strana je nezáporná.
Dospeli sme tak k~rovnakému vymedzeniu
vyhovujúcich dvojíc $(k,l)$ ako v~predchádzajúcom riešení.
}

{%%%%%   A-III-3
Dosadením $x=1$ dostaneme, že pre všetky reálne $y$ platí
$f(0)=f(1)y$, teda nutne $f(0)=f(1)=0$.
Dosadením $y=1$ do danej rovnice potom pre každé~$x$ dostaneme
$$
f(1-x)=f(x).
$$

Nech $t$ je ľubovoľné reálne číslo, dosadením $x=1-t$ dostaneme
$$
f(ty)=f(1-t)y+t^2f(y)=f(t)y+t^2f(y) \tag1
% \label{eq3.1}
$$
pre každé reálne~$y$.
Zámenou premenných $t$ a~$y$ ďalej získame
$$
f(y)t+y^2f(t)=f(yt)=f(ty)=f(t)y+t^2f(y),
$$
takže $f(t)(y^2-y)=f(y)(t^2-t)$, čo pre $y=2$ dáva
$$
f(t)=\tfrac12f(2)(t^2-t),\quad t\in\Bbb R.
$$

Dosadením do pôvodnej rovnice zo zadania ľahko overíme, že konštanta $a=f(2)/2$
v~odvodenom predpise $f(x)=ax(x-1)$ môže byť ľubovoľné reálne číslo:
$$
\align
f(x)y+(x-1)^2f(y)&= ax(x-1)y+(x-1)^2ay(y-1)=\cr
&=a(x-1)y(x+xy-x-y+1)=\cr
&=
a(1-x)y\bigl((1-x)y-1\bigr)=f\bigl((1-x)y\bigr)=f(y-xy).
\endalign
$$
}

{%%%%%   A-III-4
Uvažujme konkrétnu postupnosť a~počítajme zľava, koľko úsekov
obsahuje. Nový úsek započítame po jeho ukončení, čiže keď narazíme na
zmenu cifry alebo na pravý okraj. Počet úsekov v~postupnosti je teda
o~jedna väčší ako počet tých jej cifier, ktorým predchádza odlišná cifra
(budeme hovoriť, že v~miestach takých cifier nastáva {\it zmena úseku\/}).
Namiesto počítania úsekov v~jednotlivých postupnostiach budeme počítať
postupnosti, ktoré v~danom mieste zmenu úseku obsahujú.

Možných miest pre zmenu úseku je $2n-1$ (všetky miesta okrem prvého),
pre voľbu cifry v~mieste zmeny úseku sú dve možnosti, pričom predchádzajúca
cifra je tým jednoznačne určená.
Na zvyšných miestach je v~každej takej postupnosti ľubovoľne rozmiestnených
$n-1$ jednotiek a~$n-1$ núl, daná zmena úseku sa preto nachádza
v~${2n-2 \choose n-1}$ rôznych postupnostiach. Celkovo tak máme $2 (2n-1)$
rôznych zmien úsekov a~každá je obsiahnutá v~${2n-2 \choose n-1}$
postupnostiach. K~tomu je nutné pripočítať jednotku za každú postupnosť
kvôli úsekom končiacim na pravom okraji; postupností je pritom
${2n \choose n}$. Spolu tak pre výsledný súčet dostávame
$$
\align
2 (2n-1) \binom {2n-2} {n-1}+\binom {2n} {n}
&=2n \binom {2n-1} {n}+{2n \choose n}
=2n \binom {2n-1} {n-1}+{2n \choose n}=\\
&= n {2n \choose n}+{2n \choose n} = (n+1) {2n \choose n},
\endalign
$$
pričom sme dvakrát využili zrejmú identitu
$$
k{m\choose k} = m {m-1 \choose k-1}.
$$

\ineriesenie
Využijeme úvodnú úvahu z~predchádzajúceho riešenia, z~ktorej
vyplýva, že výsledný súčet je rovný súčtu počtu možných postupností
a~počtu dvojíc $01$ a~$10$ v~nich dokopy obsiahnutých.
Zo symetrie je jasné, že stačí určiť len
počet dvojíc~$01$ a~výsledok vynásobiť dvoma.

Uvažujme postupnosť, ktorá obsahuje presne $k$ dvojíc $01$. Tieto
dvojice rozdeľujú zvyšok postupnosti na $k+1$ úsekov, pričom v~každom je
nezáporný počet núl a~jednotiek v~jednoznačne určenom poradí (vždy najskôr
prípadné jednotky a~potom prípadné nuly).
Stačí preto určiť počet spôsobov rozmiestnenia $n-k$ núl
a~$n-k$ jednotiek do $k+1$ úsekov; podľa známeho vzorca pre kombinácie
s~opakovaním je to ${n \choose k}$ možností pre nuly a~${n \choose k}$
možností pre jednotky, čiže celkom ${n \choose k}^{2}$ možností.
Celkový súčet je teda
$$
2\sum_{k=0}^n k\binom nk^{\!2}+\binom{2n}n. \tag1
$$

Na záverečný dôkaz, že (1) dáva požadovaný výsledok, využijeme známu
identitu $\sum_{k=0}^n \binom nk^{\!2}=\binom{2n}n$ a~symetriu
$\binom nk=\binom n{n-k}$ kombinačných čísel:
$$
\align
2\sum_{k=0}^n k\binom nk^{\!2}
=&\sum_{k=0}^n k\binom nk^{\!2}+\sum_{k=0}^nk\binom n{n-k}^{\!2}=\\
=&\sum_{k=0}^n k\binom nk^{\!2}+\sum_{k=0}^n(n-k)\binom nk^{\!2}=
% \\=&
n\sum_{k=0}^n \binom nk^{\!2}
=n\binom{2n}n.
% +\binom{2n}n=(n+1)\binom{2n}n.
\endalign
$$
Dosadením do (1) tak pre hľadaný súčet dostávame $n\binom{2n}n+\binom{2n}n=(n+1)\binom{2n}n$.

\ineriesenie
Pre $n = 1$ je tvrdenie zrejmé. Nech $n \ge 2$. Pre každý možný úsek rovnakých cifier
spočítame, v~koľkých postupnostiach sa na danom mieste nachádza. Zrejme
stačí uvažovať iba úseky núl a~výsledok potom vynásobiť dvoma.

Vezmime teda úsek $k$~núl, pričom $1 \le k\le n$. Ak sa tento úsek nachádza na
jednom z~oboch okrajov, ohraničuje ho práve jedna jednotka a~na
zvyšných $2n-k-1$ miestach je ľubovoľne rozmiestnených $n-k$ núl a~$n-1$ jednotiek.
Ak sa úsek nachádza na jednej zo zvyšných $2n-k-1$ pozícií, je
ohraničený z~každej strany jednotkou a~na zvyšných $2n-k-2$ miestach je ľubovoľne
rozmiestnených $n-k$ núl a~$n-2$ jednotiek. Príspevok~$p_k$ úsekov tvorených
$k$~nulami do skúmaného súčtu je teda
$$
\align
p_k&= 2 \binom {2n-k-1} {n-1}+(2n-k-1) \binom {2n-k-2} {n-2} = \cr
&= 2 \binom {2n-k-1} {n-1}+(n-1) \binom {2n-k-1} {n-1} = (n+1)\binom {2n-k-1} {n-1}.
\endalign
$$

Na určenie celkového súčtu $2(p_1+p_2+\ldots+p_n)$ potrebujeme vypočítať
$$
S_n = \binom {n-1} {n-1}+\binom {n} {n-1}+\ldots+\binom {2n-2} {n-1}.
$$
Súčet na pravej strane udáva počet všetkých $n$-prvkových podmnožín množiny
$\{1,2,\dots,\allowbreak 2n-1\}$, keď ich budeme počítať roztriedené do skupín podľa
ich najväčšieho prvku, ktorým je jedno z~čísel $n, n+1,\dots, 2n-1$.
Preto platí $S_n=\binom{2n-1}{n}$,
takže hľadaný súčet je $2(n+1)\binom {2n-1} {n} = (n+1) {2n \choose n}.$
}

{%%%%%   A-III-5
Označme $k$ kružnicu opísanú trojuholníku $ABC$.
Polpriamka~$AH$ pretína kružnicu~$k$ v~bode $H'\ne A$, o~ktorom je
známe, že je obrazom bodu~$H$ v~osovej súmernosti podľa strany~$BC$.
Preto priamka~$H'D$ (ako obraz osi~$HD$ uhla~$BHC$)
je osou uhla $BH'C$ a~na základe známej vlastnosti
osi uhla prechádza stredom~$G$ oblúka~$BAC$ (\obr).

Označme zvyčajným spôsobom $\alpha$, $\beta$, $\gamma$ veľkosti vnútorných
uhlov trojuholníka $ABC$.
Keďže body $B$, $G$, $A$, $H'$ ležia na kružnici~$k$ (a~body $G$, $A$
na oblúku $BAC$ kružnice~$k$),
platí $|\angle BGD| =|\angle BGH'| = |\angle BAH'| = 90^\circ-\beta$.
Z~definície bodu~$E$ potom vyplýva $|\angle BED| =|\angle BDE| = 90^\circ-\beta$,
a~keďže body $E$ aj~$G$ ležia v~polrovine~$BCA$, ležia body $B$, $D$, $G$, $E$
na kružnici. Ich poradie závisí na veľkostiach uhlov $EBD$ a~$GBD$:
pokiaľ $|\angle EBD| > |\angle GBD|$, čiže
$2 \beta> 90^\circ-\frac12\alpha$ (bod~$G$ je stredom oblúka $BAC$), bude ich poradie
$B$, $D$, $G$, $E$, inak $B$, $D$, $E$,~$G$ (pre $2\beta=90\st-\frac12\alpha$,
čiže $\gamma=3\beta$, ale vyjde $E=G$, v~tom prípade je však
tvrdenie úlohy splnené triviálne). Podľa toho je potom buď
$|\angle EGD|=180\st-|\angle EBD|=180\st-2\beta$ (\obrr1),
alebo $|\angle EGD|=|\angle EBD|=2\beta$
(\obr).
\inspinspmedzera{a66.9}{a66.10}{\!}%

Analogicky možno ukázať, že aj body $C$, $D$, $G$, $F$ ležia na kružnici,
a~to práve v~tomto poradí, ak $2\gamma> 90^\circ-\frac12\alpha$, inak
v~poradí $C$, $D$, $F$, $G$ (pri $F=G$ je tvrdenie úlohy určite splnené).
Pre veľkosť uhla $DGF$ potom podľa toho platí buď
$|\angle DGF|=180\st-2\gamma$, alebo $|\angle DGF|=2\gamma$.

Z~podmienok $2\beta> 90^\circ-\frac12\alpha$
a~$2\gamma> 90^\circ-\frac12\alpha$ pritom musí byť splnená aspoň jedna, inak
by bolo $2\beta+2\gamma \le 2({90^\circ-\frac12\alpha})=\beta+\gamma$.

V~každom prípade môžeme z~oboch tetivových štvoruholníkov so spoločnou stranou~$DG$
vyjadriť veľkosť uhla~$EGF$. Pritom veľkosť uhla~$EAF$ poznáme, z~definície
bodov $E$ a~$F$ totiž vyplýva
$$
|\angle EAF|=|\angle EAD|+|\angle DAF| = 2|\angle BAD|+2|\angle DAC| = 2|\angle BAC| = 2\alpha
$$
a~zároveň vidíme, že priamka~$EF$ oddeľuje body $A$ a~$D$, pretože uhol~$\alpha$ je ostrý.
Musíme preto rozobrať tri prípady.

1. Štvoruholníky $BDGE$ a~$CDGF$ sú tetivové, takže platí
$$
|\angle EGF|=|\angle EGD|+|\angle DGF| = 180^\circ-2 \beta+180^\circ-2 \gamma = 2\alpha
= |\angle EAF|.
$$
To znamená, že je konvexný
aj štvoruholník $DFGE$, takže jeho uhlopriečka~$EF$ oddeľuje protiľahlé vrcholy $D$ a~$G$.
Preto oba body $G$ aj $A$ ležia v~jednej polrovine vzhľadom
na~$EF$. Z~vety o~obvodových uhloch tak vyplýva, že body $E$, $G$, $A$,
$F$ ležia na kružnici.

2. Štvoruholníky $BDEG$ a~$CDGF$ sú tetivové.
V~tomto prípade $|\angle EGD| =2\beta$ a~$|\angle DGF|=180^\circ-2\gamma$,
a~keďže $2\beta+2\gamma>180\st$, je $|\angle EGD|>|\angle DGF|$, takže bod~$G$
leží v~polrovine $EFD$ (\obrr1) a~platí
$$
|\angle EGF| = |\angle EGD|-|\angle DGF| =2\beta-(180\st-2\gamma)= 180^\circ-2 \alpha,
$$
čiže $|\angle EAF|+|\angle EGF| = 180^\circ$, čo spolu s~tým, že body $A$ a~$G$ sú
v~rôznych polrovinách vzhľadom na~priamku~$EF$, implikuje, že body $F$,
$A$, $E$, $G$ ležia na jednej kružnici.

3. Štvoruholníky $BDGE$ a~$CDFG$ sú tetivové. Tento prípad je
ekvivalentný predošlému, keď zameníme $B$ s~$C$ a~$E$ s~$F$.

{\def\pi{180\st}
\poznamky
Ak budeme pracovať s~orientovanými uhlami dvoch priamok, môžeme
sa vyhnúť rozboru uvedených troch prípadov. Zistenú skutočnosť,
že body $B$, $D$, $E$, $G$ ležia na kružnici, možno charakterizovať
rovnosťou (orientovaných) uhlov $\widehat{EGD}=\widehat{EBD}$ (samozrejme
počítame modulo~$\pi$) a~podobne pre druhú kružnicu $\widehat{DGF}=\widehat{DCF}$.
Je teda $\widehat{EGF}=\widehat{EGD}+\widehat{DGF}=\widehat{EBD}+\widehat{DCF}=
\pi-2\beta+\pi-2\gamma=2\alpha$, čo sme potrebovali dokázať,
lebo z~definície bodov $E$ a~$F$ zrejme $\widehat{EAF}=\widehat{EAD}+\widehat{DAF}=2\alpha$.
}
\smallskip
Ukážeme ešte, že kľúčový poznatok celého riešenia o~štvoriciach bodov $(B,D,E,G)$
a~$(C,D,F,G)$ možno dokázať aj iným, trigonometrickým postupom.

Označme $|\angle BGD| = \varphi$, $|\angle DGC| = \psi$ a~$P$ pätu
výšky~$AH$ (\obr).
Keďže $HD$ je os uhla $BHC$, platí
$$
{|BD|\over |CD|} = {|BH|\over |CH|} = {|PH| \over \sin|\angle PBH|}: {|PH| \over
\sin |\angle PCH|} = {\sin (90^\circ-\beta) \over \sin (90^\circ-\gamma)}.
$$
Z~rovnosti $|GB|=|GC|$ a~z~dvojakého vyjadrenia pomeru obsahov
trojuholníkov $BGD$ a~$CGD$ vyplýva
$$
{\sin\phi \over \sin\psi} = {|BD| \over |CD|}.
$$
Pritom $\varphi+\psi = \alpha = (90^\circ-\gamma)+(90^\circ-\beta)$.
Spojením oboch rovností tak pre $\phi\in(0\st,\alpha)$ dostávame rovnicu
$$
{\sin \varphi \over \sin(\alpha- \phi)}={\sin (90^\circ-\beta) \over \sin (90^\circ-\gamma)}.
$$
Podiel na ľavej strane je v~uvedenom intervale rastúca funkcia premennej~$\phi$,
takže rovnica má (pre daný trojuholník) jediné riešenie.
Zjavne $\phi = 90^\circ-\beta$ v~danom intervale leží a~rovnici vyhovuje.
Preto tiež $\psi = 90^\circ-\gamma$.
\insp{a66.11}%

Trojuholník $DEB$ je rovnoramenný so základňou~$DE$, preto $|\angle
BED| = |\angle BDE| = 90^\circ-\beta = |\angle BGD|$. Body $E$, $G$
pritom ležia v~tej istej polrovine vzhľadom na priamku~$BD$, preto ležia
body $B$, $D$, $E$, $G$ na kružnici. Rovnaké tvrdenie o~bodoch
$D$, $C$, $F$, $G$ vyplýva z~dokázanej rovnosti $\psi=90^\circ-\gamma$.

\ineriesenie
Z~definície bodov $E$ a~$F$ je zrejmé, že pre (orientovaný) uhol $EAF$ platí
$\widehat{EAF}=2\alpha$. Označme postupne $H_2$, $H_3$ body súmerne združené s~priesečníkom
výšok~$H$ daného trojuholníka podľa jeho strán $AC$, resp. $AB$. Tie, ako je známe,
ležia na kružnici~$k$ opísanej trojuholníku~$ABC$ (\obr).

Keďže $DH$ je os uhla $BHC$ a~$CH\parallel DE$, je $|\angle H_3ED|=|\angle HDE|
=|\angle DHC|=\frac12|\angle BHC|=90\st-\frac12\alpha$. Rovnakú veľkosť
má však aj orientovaný uhol $\widehat{CH_3G}$ nad oblúkom~$CG$,
lebo ten je polovicou oblúka $CAB$, preto $\widehat{CH_3G}=\widehat{DEH_3}$.
A~keďže $CH_3\parallel DE$, znamená to, že body $G$, $H_3$ a~$E$
ležia na jednej priamke. Podobne ležia na jednej priamke aj body $G$, $H_2$ a~$F$, takže
$\widehat{EGF}=\widehat{H_3GH_2}=\widehat{H_3AH_2}
=\widehat{H_3AH}+\widehat{HAH_2}=2\alpha=\widehat{EAF}$, čo sme
chceli dokázať.
\insp{a66.12}%

\poznamka
\Obrr1{} zvádza k~jednoduchému záveru, že uhol $EGF$ ľahko dopočítame
z~vnútorných uhlov štvoruholníka $EDFG$:
$$
|\angle EGF| =360^{\circ}- |\angle GED|-|\angle DFG|-|\angle EDF|
=360^{\circ}-4(90\st-\tfrac12\alpha)=2\alpha.
$$
Predpoklady úlohy však konvexnosť štvoruholníka $EDFG$ bohužiaľ nezaručujú
(\obr).
\insp{a66.13}%
}

{%%%%%   A-III-6
Po vynásobení oboch strán danej rovnice výrazom $x+y$ dostaneme rovnicu
$$
x^2-xy+2y^2 = k(x+y).\tag1
$$
Každé riešenie $(x, y)$ danej rovnice je aj riešením rovnice~(1), tej
však môžu navyše vyhovovať aj~dvojice, pre ktoré $x+y = 0$, \tj.~$y={-x}$.

Dvojica $(x,-x)$ je riešením (1) práve vtedy, keď platí $x^2+x^2+2x^2=k\cdot0$,
čiže $x = 0$. Rovnica (1) má preto práve o~jedno riešenie viac ako
daná rovnica, a~tak stačí dokázať, že rovnica~(1) má párny počet celočíselných
riešení práve vtedy, keď $7\mid k$.

Rovnicu (1) ekvivalentne upravíme na tvar kvadratickej
rovnice
$$
x^2-x (y+k)+2y^2-ky = 0 \tag2
$$
s~neznámou~$x$. Pre jej diskriminant platí
$$
\align
D(y) = (y+k)^2-4 (2y^2-ky) = k^2+6ky-7y^2 =&
(k-y) (k+7y)=\tag3\\
=&{-7}(y-\tfrac37k)^2+\tfrac{16}7k^2,
\endalign
$$
čo je pre každé $k$ zhora ohraničená kvadratická funkcia. Rovnica~(2) má
teda pre každé celé~$k$ nezáporný diskriminant~$D(y)$
iba pre konečne veľa celých čísel~$y$, a~môže tak mať iba konečný počet
celočíselných riešení $(x, y)$.

Ak je $D(y)>0$ pre nejaké celé číslo~$y$, má rovnica~(2) práve dve reálne
riešenia, ktoré môžu byť celočíselné len súčasne, keďže ich súčet
$y+k$ je celé číslo.
Pre každé také~$y$ má teda (2) vždy párny počet riešení.

Z~vyjadrenia (3) vidíme, že $D(y)=0$ pre $y=k$ alebo pre $y=\m\frac17k$.
V~prvom prípade sa rovnica~(2) redukuje na rovnicu $(x-k)^2=0$
s~dvojnásobným koreňom $x=k$, rovnica~(1)
má teda jediné riešenie $(k, k)$ so zložkou $y=k$.
V~druhom prípade je $y$ celočíselné práve vtedy, keď je číslo~$k$ deliteľné siedmimi,
a~potom má rovnica~(2) dvojnásobný koreň $x=\frac37k$, teda $(\frac37k,{-\frac17}k)$
je jediné riešenie rovnice~(1) so zložkou $y={-\frac17}k$.
Navyše obe riešenia $(k, k)$ aj $(\frac37k, -\frac17k)$ sú rôzne, keďže $k\ne 0$.

Vidíme teda, že rovnica~(1) má párny počet celočíselných riešení práve vtedy, keď
je číslo~$k$ deliteľné siedmimi. Tým je tvrdenie úlohy dokázané.


\ineriesenie
Rovnako ako v~prvom riešení skúmame, kedy má rovnica (1) párny počet celočíselných riešení.

Rovnicu (1) upravíme na tvar
$$
7(2x-y-k)^2+(7y-3k)^2=16k^2. \tag4
$$
Keďže pre dané $k$ zrejme existuje iba konečný počet celých čísel $a$, $b$ takých, že
$$
7a^2+b^2=16k^2, \tag5
$$
má rovnica~(4) iba konečný počet celočíselných riešení, ktoré získame riešením sústav
$$
\align
2x-y-k=&a, \tag6\\
7y- 3k=&b, \tag7
\endalign
$$
prislúchajúcich všetkým celočíselným riešeniam $(a,b)$ rovnice~(5). Z~ich tvaru vyplýva,
že zložky $a$, $b$ majú vždy rovnakú paritu.
Vďaka tomu vidíme,
že ak má rovnica~(7) celočíselné riešenie~$y$, ktoré tak má rovnakú paritu
nielen ako číslo $k+b$, ale aj ako číslo $k+a$, je číslo $y+k+a$ párne,
a~teda aj rovnica~(6) má pre dotyčné $k$ a~$b$ celočíselné riešenie~$x$.

Z~poslednej úvahy vyplýva, že dve sústavy (6) a~(7), ktoré zodpovedajú \uv{združeným} riešeniam
$(a,b)$ a~$({-a}, b)$ (pričom $a\ne0$) rovnice~(5), majú buď po jednom riešení
(s rovnakým~$y$ a~rôznymi~$x$), alebo žiadne riešenie nemajú. Jediné takto nezdružené
celočíselné riešenia rovnice~(5) sú zrejme $(0, 4k)$ a~$(0,{-4k})$ (pripomeňme,
že $k\ne0$ podľa zadania), takže parita počtu celočíselných riešení rovnice~(4)
je zhodná s~paritou celkového počtu celočíselných riešení dvoch prislúchajúcich
sústav (6) a~(7); pre $(a,b)=(0,4k)$ to je $(x,y)=(k,k)$, pre $(a,b)=(0,{-4k})$
vyhovuje $(x,y)=(\frac37 k,{-\frac17} k)$. Preto je skúmaná parita párna
práve vtedy, keď 7 delí~$k$. K~hotovému dôkazu dodajme, že sme pri našich úvahách
mlčky využívali zrejmý poznatok, že rôznym dvojiciam $(a,b)$ zodpovedajú rôzne
riešenia $(x,y)$ sústav (6) a~(7).

\poznamka
Rovnica (1), ako je vidno aj z~jej upraveného tvaru~(4),
je pri nenulovom~$k$ rovnicou elipsy
so stredom $(\frac57k,\frac37k)$, a~ak je $k= 7m$ pre $m$~celé, je s~každým bodom
$(x, y)$ bodom elipsy aj bod $(10m - x, 6m - y)$, takže mrežové body tu
vystupujú v~dvojiciach, je ich teda párny počet.
}

{%%%%%   B-S-1
Označme $a_1<a_2<a_3<a_4<a_5$ kladné čísla napísané na tabuli.
Najmenšie čísla $a_1$ ani $a_2$ zrejme nemôžu byť súčtom
žiadnych dvoch na tabuli napísaných čísel. Číslo $a_3$ sa dá ako
súčet niektorej dvojice dostať nanajvýš jedným spôsobom, a~to $a_3=a_1+a_2$.
Číslo~$a_4$ síce môžeme dostať tromi spôsobmi: $a_1+a_2$,
$a_2+a_3$ alebo $a_1+a_3$, ale keďže je $a_1+a_2<a_1+a_3<a_2+a_3$,
môže byť $a_4$ súčtom nanajvýš jednej z~dvojíc
$\{a_1,a_2\}$, $\{a_1,a_3\}$, $\{a_2,a_3\}$. Napokon
číslo~$a_5$ možno takto získať nanajvýš dvoma spôsobmi. Keby sme totiž
$a_5$ získali tromi spôsobmi ako súčet dvoch čísel napísaných na tabuli,
bolo by aspoň jedno z~čísel $a_1$, $a_2$, $a_3$, $a_4$ sčítancom
v~dvoch rôznych súčtoch, čo nie je možné.

Spolu sme tak zistili, že počet vyhovujúcich dvojíc
nikdy neprevýši súčet $0+0+1+1+2$, ktorý je rovný štyrom.

Príkladom pätice čísel napísaných na tabuli, pre ktorý súčty štyroch (z~nich
vytvorených) dvojíc sú tiež na tabuli uvedené, je $a_i=i$ pre
$i\in\{1, 2, 3, 4, 5\}$, keď $1+2=3$, $1+3=4$, $1+4=5$ a~$2+3=5$.

\zaver
Z~daných piatich rôznych kladných čísel možno zostaviť nanajvýš štyri dvojice
požadovanej vlastnosti.


\nobreak\medskip\petit\noindent
Za úplné riešenie dajte 6~bodov.
Za získanie odhadu, že vhodných dvojíc nie je viac ako štyri, dajte 4~body.
(Za odhad, že dvojíc je nanajvýš šesť, pretože najväčšie číslo v~dvojici
byť nemôže, body nedávajte.)
Za príklad pätice čísel so štyrmi vyhovujúcimi dvojicami dajte 2~body,
a~to aj v~prípade, že sa riešiteľovi nepodarí horný odhad dokázať.

\endpetit
\bigbreak}

{%%%%%   B-S-2
V~súlade s~\obr{} označme $x=|CL|$, $y=|CK|$, potom $|BL|=a-x$, a~$|AK|=b-y$,
pričom $a$, $b$ sú postupne dĺžky odvesien $BC$, $AC$.
\insp{b66.6}%

Použitím Pytagorovej vety v~pravouhlom trojuholníku $KLC$ dostaneme $|KL|^2=x^2+y^2$,
takže skúmaný súčet môžeme upraviť
nasledujúcim spôsobom:
$$
\align
|AK|^2+|KL|^2+|LB|^2=&(b-y)^2+x^2+y^2+(a-x)^2=\\
=&2x^2+2y^2-2ax-2by+a^2+b^2=\\
=&2\Bigl(x-\frac{a}2\Bigr)^{\!2}+2\Bigl(y-\frac{b}2\Bigr)^{\!2}+\frac{a^2+b^2}2=\\
=&2\Bigl(x-\frac{a}2\Bigr)^{\!2}+2\Bigl(y-\frac{b}2\Bigr)^{\!2}+\frac{c^2}2.
\endalign
$$

Vďaka nezápornosti druhých mocnín
z~toho vidíme, že skúmaný výraz nadobúda svoju najmenšiu hodnotu, konkrétne~$\frac12c^2$,
práve vtedy, keď $x=\frac12a$ a~súčasne $y=\frac12b$, teda práve vtedy, keď
body $K$, $L$ sú postupne stredmi odvesien $AC$, $BC$ daného pravouhlého
trojuholníka $ABC$.

\zaver
Najmenšia možná hodnota skúmaného súčtu je rovná $\frac12c^2$.
Túto hodnotu dostaneme práve vtedy, keď body $K$, $L$ budú postupne stredmi
odvesien $AC$, $BC$ daného pravouhlého trojuholníka.

\nobreak\medskip\petit\noindent
Za úplné riešenie dajte 6~bodov, z~toho
2~body za vyjadrenie skúmaného súčtu pomocou dvoch vhodne zvolených parametrov.
Za následnú úpravu na tvar s~druhými mocninami, z~ktorého je hľadaná
najmenšia hodnota a~tiež príslušná poloha bodov $K$
a~$L$ zrejmá, dajte ďalšie 3~body. Za uvedenie správnej odpovedi
pre polohu oboch hľadaných bodov $K$, $L$ a~vyjadrenie najmenšej hodnoty
skúmaného výrazu pomocou~$c$ dajte 1~bod.

\endpetit
\bigbreak
}

{%%%%%   B-S-3
Označme po sebe napísané čísla $a_1,a_2,\dots ,a_{1000}$. Vzhľadom
na to, že súčet každých siedmich po sebe napísaných čísel je rovný tomu istému
číslu 2\,017, porovnaním dvoch takých súčtov so~šiestimi spoločnými sčítancami
$$
a_i+a_{i+1}+\cdots+a_{i+5}+a_{i+6}=a_{i+1}+a_{i+2}+\cdots+a_{i+6}+a_{i+7}
$$
dôjdeme k~záveru, že rovnosť $a_i=a_{i+7}$ platí pre každý prípustný index~$i$, takže
uvažovaná číselná postupnosť~$(a_i)$ má periódu dĺžky~7. Platí
teda $a_1=a_8={a_{15}=\dots}$, $a_2=a_9=a_{16}=\dots$, a pod. Keďže
indexy 123, 234, 345, 456, 567, 678, 789 členov tejto postupnosti dávajú
v~určitom poradí všetkých sedem možných zvyškov po delení siedmimi, je súčet členov
s~týmito siedmimi indexmi rovný súčtu ľubovoľných siedmich po sebe idúcich
členov tejto postupnosti, \tj.~2\,017. Z~toho už pre
$a_{123}=123$, $a_{234}=234$ a~$a_{345}=345$ bezprostredne vyplýva
$$
a_{456}+a_{567}+a_{678}+a_{789}=2\,017-(a_{123}+a_{234}+a_{345})=2\,017-702=1\,315.
$$

\zaver
Hľadaný súčet je 1\,315.

\nobreak\medskip\petit\noindent
Za úplné riešenie dajte 6~bodov, z~toho 2~body za objav, že postupnosť má periódu
dĺžky~7 (aj bez zdôvodnenia zo vzorového riešenia, ktoré možno totiž považovať za zrejmé).
Ďalšie 2~body dajte za poznatok, že sedem zadaných indexov má rôzne zvyšky
po delení siedmimi, 1~bod potom za jeho dôsledok, že sedem dotyčných členov dáva
súčet 2\,017, a~zvyšný 1~bod dajte za dopočítanie hľadaného súčtu.

\endpetit
\bigbreak
}

{%%%%% B-II-1
Anulovaním pravej strany upravíme danú rovnicu na tvar
$$
a-b+66\Big(\frac1a-\frac1b\Big)=(a-b)\Big(1-\frac{66}{ab}\Big)=\frac{1}{ab}(a-b)(ab-66)=0.
$$
Z~toho vyplýva, že hľadané dvojice $(a,b)$ prirodzených čísel sú práve tie,
pre ktoré platí $a=b$ alebo $ab=66$.

Úlohe teda vyhovuje nekonečne veľa dvojíc prirodzených čísel tvaru $(a,b)=(k,k)$,
pričom $k$ je ľubovoľné prirodzené číslo, a~keďže
číslo $66=2\cdot3\cdot11$ má osem deliteľov, tak aj osem dvojíc
$(a,b)\in\{(1, 66),(2, 33),(3, 22),(6, 11),(11, 6),(22, 3),(33, 2),(66, 1)\}$.

\nobreak\medskip\petit\noindent
Za úplné riešenie dajte 6~bodov.
Za prosté konštatovanie,
že úlohe vyhovujú dvojice $a=b$ (bez uvedenia ďalších 8~riešení),
dajte 1~bod. Ak žiak upraví danú rovnicu na súčinový tvar,
avšak neuvedie ďalších 8~konkrétnych riešení, dajte 3~body.
V~prípade,
že žiak neuvedie 1, resp. 2 (či 3) symetrické dvojice z~výpisu
konkrétnych riešení, strhnite 1, resp. 2 body.
\endpetit
\bigbreak
}

{%%%%%   B-II-2
Koncové body daných oblúkov delia kružnicu~$k$ na konečný počet častí. Ak každá taká časť je pokrytá najviac dvoma z~daných oblúkov, tak
súčet dĺžok oblúkov je najviac dvojnásobkom obvodu kružnice~$k$.

V~opačnom prípade existuje bod $T\in k$, ktorý je vnútorným bodom (aspoň) troch z~daných kružnicových oblúkov.
Označme $O$ stred kružnice $k$. Po kružnici budeme postupovať v~kladnom smere a~označíme
tri oblúky obsahujúce $T$ zoradené podľa ich počiatočného bodu $XX'$, $YY'$, $ZZ'$, pričom
$|\uhol XOT| \ge |\uhol YOT| \ge |\uhol ZOT|$. (V~zmysle orientácie sú teda body $X$, $Y$, $Z$ "pred" bodom~$T$
a body $X'$, $Y'$, $Z'$ "za" bodom~$T$.) Rozlíšime tri prípady podľa toho, ktorý z~trojice uhlov $TOX'$, $TOY'$, $TOZ'$ má maximálnu veľkosť:
\ite{$\triangleright$} Ak je to uhol $TOX'$, tak oblúk~$XX'$ pokrýva obidva zvyšné
oblúky $YY'$, $ZZ'$, takže obidva tieto oblúky môžeme vynechať, lebo ostatné oblúky pokryjú
celý obvod kružnice.
\ite{$\triangleright$} Ak je to uhol $TOY'$, tak zjednotenie oblúkov $XX'$ a~$YY'$
pokrýva oblúk~$ZZ'$, takže tento môžeme vynechať, lebo ostatné oblúky pokryjú celý
obvod kružnice.
\ite{$\triangleright$} Ak je to uhol $TOZ'$, tak zjednotenie oblúkov $XX'$ a~$ZZ'$
pokrýva oblúk~$YY'$, takže tento môžeme vynechať, lebo ostatné oblúky pokryjú celý
obvod kružnice.

Dokázali sme, že ak existuje časť, ktorá je pokrytá aspoň tromi z~daných oblúkov, tak z~danej množiny oblúkov môžeme vynechať aspoň jeden oblúk.
Konečným počtom opakovaní tohto postupu dostaneme hľadanú množinu~$\mm M$.

\nobreak\medskip\petit\noindent
Za úplné riešenie dajte 6~bodov, z~toho 2~body za úvahu o~existencii bodu pokrytého tromi oblúkmi, po jednom bode za uvedenie každého z~troch možných prípadov polohy trojice oblúkov a~1~bod za záver.

Ak žiak neuvažuje o~maximálnom spomedzi uhlov $TOX'$, $TOY'$, $TOZ'$ a~rozoberá jednotlivé možné usporiadania trojice bodov $X'$, $Y'$, $Z'$ spomedzi všetkých šiestich možností, chýbajúce možnosti adekvátne penalizujte.
\endpetit
\bigbreak
}

{%%%%%   B-II-3
Z~rovnosti obvodových uhlov nad tetivou~$HF$ kružnice~$k$
vyplýva $|\uhol HCF|=|\uhol HEF|$. Uhol $HEF$ je zároveň úsekovým uhlom
prislúchajúcim tetive~$EF$ kružnice~$l$, ktorý je však zhodný s~obvodovým uhlom
$EDF$ (\obr). Celkovo tak platí
$$
|\uhol HCF|=|\uhol HEF|=|\uhol EDF|. \eqno{(1)}
$$
\insp{b66.7}%

Vzhľadom na to, že $CEFH$ je tetivový štvoruholník, je jeho vnútorný
uhol pri vrchole~$H$ zhodný s~vonkajším uhlom pri~jeho protiľahlom vrchole~$E$.
Platí teda
$$
|\uhol CHF|=|\uhol DEF|. \eqno{(2)}
$$
Z~rovností (1) a~(2) vyplýva na základe vety {\it uu\/} podobnosť
trojuholníkov $DEF$ a~$CHF$.
Tým je dôkaz hotový.

\nobreak\medskip\petit\noindent
Za úplné riešenie dajte 6~bodov, z~toho 2, resp. 4~body
dajte za dôkaz zhodnosti jednej, resp. oboch dvojíc prislúchajúcich si
uhlov v~dvojici podobných trojuholníkov (akýmkoľvek správnym spôsobom).
Posledné 2~body dajte za využitie zhodnosti uvedenej dvojice
uhlov na dôkaz podobnosti oboch trojuholníkov.

\endpetit
\bigbreak
}

{%%%%%   B-II-4
Danú rovnicu upravíme na tvar
$$
\big|1-|1-|1-x||\big|=\frac{2\,016}{2\,017}\,x+\frac{p}{2\,017}. \eqno{(1)}
$$
Pri riešení využijeme
grafickú metódu. Grafom funkcie $y=f(x)=\big|1-|1-|1-x||\big|$
na ľavej strane rovnice~(1) je čiara lomená v~bodoch
$A=[-1, 0]$, $B=[0, 1]$, $C=[1, 0]$, $D=[2, 1]$ a~$E=[0, 3]$
(znázornená na \obr), ktorá pozostáva zo štyroch zhodných úsečiek a~dvoch polpriamok.
Grafom funkcie~$g$
na pravej strane~(1) je priamka $y=kx+q$
so smernicou $k=\frac{2\,016}{2\,017}<1$ prechádzajúca bodom $[0,q]$, pričom
$q=p/2\,017$ (smernica je natoľko "blízka" jednotke, že sme museli
pre priamku~$\gamma$, ktorá na \obrr1{} znázorňuje graf funkcie~$g$, zvoliť
smernicu o~pár percent menšiu, aby sa výsledná priamka nejavila ako rovnobežka
s~časťami grafu funkcie~$f$).
\insp{b66.8}%

Počet reálnych riešení rovnice~(1) zrejme zodpovedá počtu spoločných bodov
grafov oboch funkcií $f$ a~$g$.
Keďže všetky priamky, na ktorých ležia časti lomenej čiary grafu funkcie~$f$,
majú smernicu~${\pm1}$, nie je ťažké zistiť, koľko priesečníkov pre dané~$q$
oba grafy majú.
Vyšetríme teraz počet priesečníkov oboch grafov podľa hodnoty~$x_0$,
v~ktorej priamka~$\gamma$ grafu funkcie~$g$ pretína súradnicovú os~$x$.

Pre $x_0>3$ zrejme žiadny priesečník neexistuje.
Pre $x_0=3$ pretína $\gamma$ graf funkcie~$f$ v~jedinom bode~$E$.
Pre $x_0<3$ pretína $\gamma$ vnútro polpriamky grafu~$f$
s~počiatkom v~bode~$E$ práve raz. Pre $x_0<-1$ pretína $\gamma$ vnútro aj druhej polpriamky
grafu~$f$, ktorá má počiatok v~bode~$A$, takže pre také~$x_0$ už máme dva
spoločné body.
Pre $x_0\in\langle\m1,3)$ má ale priamka~$\gamma$ s~lomenou čiarou $ABCDE$ práve dva potrebné priesečníky
jedine vtedy, keď prechádza jej vrcholmi $D$, $C$ alebo $A$, a~napokon pre $x_0<-1$
práve jeden potrebný priesečník jedine vtedy, keď prechádza vrcholom~$B$.

To znamená, že grafy oboch funkcií majú práve tri spoločné body jedine vtedy, keď
priamka~$\gamma$ prechádza niektorým zo štyroch bodov $A$, $B$, $C$, $D$ na \obrr1.
Postupným dosadením súradníc týchto štyroch bodov do rovnice
$$y=kx+q=\frac{2\,016}{2\,017}+\frac{p}{2\,017}$$ priamky~$\gamma$
určíme hľadané hodnoty parametra~$p$.

\odpoved
Daná rovnica má práve tri reálne riešenia práve vtedy, keď
$$p\in\{{-2\,016},{-2\,015},\allowbreak2\,016, 2\,017\}.$$

\poznamka
Lomená čiara grafu funkcie~$f$ rozdeľuje rovinu na dve (neohraničené) oblasti,
pričom priamka $\gamma$ pre ľubovoľnú hodnotu parametra~$q$ pretína oblasť
"nad" lomenou čiarou v~niekoľkých úsečkách, a~tie majú s~hranicou tej oblasti
spoločný párny počet bodov, niektoré krajné body dvoch úsečiek však môžu
práve v~prípade vrcholov uvedenej lomenej čiary splynúť v~jediný. Z~tejto
úvahy tak vyplýva, že iba tie priamky, ktoré prechádzajú niektorým z~vrcholov
$A$, $B$, $C$, $D$, $E$ lomenej čiary, s~ňou môžu mať nepárny počet priesečníkov,
v~prípade vrcholu~$E$ ale iba jeden.

\nobreak\medskip\petit\noindent
Za úplné riešenie dajte 6~bodov, z~toho 3~body za nákres grafu funkcie
s~tromi absolútnymi hodnotami.
Konštatovanie o~vyhovujúcich vzájomných polohách lomenej čiary a~priamky
možno považovať (rovnako ako vo vzorovom riešení) za zrejmé, a~tak ho oceňte 2~bodmi,
napokon 1~bod dajte za dopočítanie hodnôt~$p$. Ak chýba vo výpise vyhovujúcich
polôh $k$ z~nich, strhnite $k$~bodov, ak $k\le2$; v~prípade $k\ge3$
možno celkom získať nanajvýš 3~body (za správny graf podľa prvej vety pokynov).

\endpetit
\bigbreak
}

{%%%%%   C-S-1
Zo zadania vyplýva, že $x\ne -2$. Po vynásobení výrazom $|x+2|$ dostávame
rovnicu $|x+2|=|3x-7|-|9-2x|$, ktorú teraz vyriešime.
Nulové body troch absolútnych hodnôt s~neznámou rozdeľujú reálnu os
na štyri intervaly, v~ktorých má každý z~prislúchajúcich dvojčlenov stále rovnaké znamienko. V~každom z~týchto intervalov už teda môžeme
riešiť zodpovedajúcu rovnicu bez absolútnych hodnôt.
\item{$\triangleright$} $x\in (\m\infty, \m2)$: dostávame rovnicu
$\m x-2=\m3x+7-(9-2x)$, ktorá po úprave prejde na identitu $0=0$.
Všetky čísla zo skúmaného intervalu pôvodnej rovnici vyhovujú.
\item{$\triangleright$} $x\in \langle \m2,\frac73)$:
dostávame rovnicu $x+2=\m3x+7-(9-2x)$, čiže $2x=\m4$ s~jediným riešením $x=\m2$,
ktoré, ako už vieme, pôvodnej rovnici nevyhovuje.
\item{$\triangleright$} $x\in\langle \frac73,\frac92)$: dostávame rovnicu
$x+2=3x-7-(9-2x)$ s~riešením $x=\frac92$, ktoré však v~uvažovanom intervale neleží.
\item{$\triangleright$} $x\in\langle \frac92, \infty)$: dostávame rovnicu $x+2=3x-7-(\m9+2x)$,
ktorá po úprave prejde na identitu $0=0$. Vyhovujú všetky $x$ z~tohto intervalu.

\zaver
Všetky riešenia úlohy tvoria množinu $(\m\infty,\m2)\cup\langle\frac 92,\infty)$.

\nobreak\medskip\petit\noindent
Za úplné riešenie dajte 6~bodov. Za správny postup (snahu o~odstraňovanie absolútnych hodnôt)
dajte 1~bod. Za vyšetrenie rovnice v~každom z~intervalov 1~bod. Záver (explicitné uvedenie
množiny riešení) 1~bod. Za každý zle spočítaný nulový bod strhnite 2~body
(maximálne však 5~bodov). S~rovnicou možno pracovať aj v~pôvodnom tvare s~neodstráneným zlomkom.

\endpetit
\bigbreak
}

{%%%%%   C-S-2
Najskôr vypočítame prislúchajúce hodnoty výrazu~$V$ pre niekoľko prvých
nepárnych čísel:
$$
\vbox{\halign{\hss\strut$#$\hss&\quad \hss$#$&${}=#$\hss\cr
n &\multispan2\quad\hss $V(n)$\hss\cr
\noalign{\hrule}
1 &\multispan2\quad\hss 0\hss\cr
3 & 168 & 3\cdot48+24\cr
5 & 888 & 18\cdot48+24\cr
7 & 2\,928 & 61\cdot48\cr
9 & 7\,440 & 155\cdot48\cr
}}
$$
Medzi hľadané zvyšky teda patria čísla $0$ a~$24$. Ukážeme, že iné
zvyšky už možné nie sú. Na to stačí dokázať, že pre každé nepárne
číslo~$n$ platí $24\mid V(n)$.
Z~domáceho kola vieme, že pre každé prirodzené číslo~$n$
platí $12\mid V(n)$, teda aj $3\mid V(n)$.
Keďže čísla $3$ a~$8$ sú nesúdeliteľné, stačí ukázať,
že pre každé nepárne číslo~$n$ platí $8\mid V(n)$.
Využijeme pritom rozklad daného výrazu na súčin
$$
V(n)=n^{4}+11n^{2}-12=(n^{2}-1)(n^{2}+12)=(n-1)(n+1)(n^{2}+12). \tag1
% \label{sou=00010Din}
$$

Ľubovoľné nepárne prirodzené číslo~$n$ možno zapísať v~tvare $n=2k-1$,
pričom $k\in\Bbb{N}$. Pre také~$n$ potom dostávame
$$
V(2k-1)=[(2k-1)-1][(2k-1)+1][(2k-1)^{2}+12]=4(k-1)k(4k^{2}-4k+13),
$$
a~keďže súčin $(k-1)k$ dvoch po sebe idúcich
celých čísel je deliteľný dvoma, je celý výraz deliteľný ôsmimi.

\zaver
Daný výraz môže dávať po delení číslom~$48$ práve len zvyšky~$0$
a~$24$.

\poznamka
Poznatok, že $8\mid V(n)$ pre každé nepárne~$n$, možno dokázať aj inak,
bez použitia rozkladu~(1). Ak je totiž $n=2k-1$, pričom $k\in\Bbb N$, tak číslo
$$
n^2=(2k-1)^2=4k^2-4k+1=4k(k-1)+1
$$
dáva po delení ôsmimi (vďaka tomu, že jedno z~čísel $k$, $k-1$ je párne) zvyšok~1,
a~teda rovnaký zvyšok dáva aj~číslo~$n^4$ (ako druhá mocnina nepárneho čísla~$n^2$).
Platí teda $n^2=8u+1$ a~$n^4=8v+1$ pre vhodné celé $u$ a~$v$, takže hodnota výrazu
$$
V(2k-1)=(8v+1)+11(8u+1)-12=8(v+11u)
$$
je naozaj násobkom ôsmich.

Pripojme aj podobný dôkaz poznatku $3\mid V(n)$ z~domáceho kola.
Pre čísla~$n$ deliteľné tromi je to zrejmé, ostatné $n$ sú tvaru $n=3k\pm1$, takže číslo
$$
n^2=(3k\pm1)^2=9k^2\pm6k+1=3k(3k\pm2)+1
$$
dáva po delení tromi zvyšok~1, rovnako tak aj číslo $n^4=(n^2)^2$.
Dosadenie $n^2=3u+1$ a~$n^4=3v+1$ do výrazu $V(n)$ už priamo vedie k~záveru, že $3\mid V(n)$.

\nobreak\medskip\petit\noindent
Za úplné riešenie dajte 6~bodov. Za využitie faktu z~domáceho kola ($12\mid
V(n)$ pre ľubovoľné~$n$) na odvodenie (konštatovanie) toho, že zvyšky môžu
byť iba $0$, $12$, $24$ a~$36$ dajte 1~bod.
Za snahu dokázať, že $8\mid V(n)$ pre $n$ nepárne, dajte 1~bod,
za dôkaz tohto tvrdenia dajte 2~body.
Za záver, že
z~$3\mid V(n)$ a~$8\mid V(n)$ vyplýva
$24\mid V(n)$ dajte 1~bod (nesúdeliteľnosť čísel $3$ a~$8$ musí byť
spomenutá; záver typu \uv{z~$12\mid V(n)$ a~$8\mid V(n)$ zrejme vyplýva, že
$24\mid V(n)$} nestačí, ak nie je napr. dodané, že 24 je najmenší spoločný
násobok čísel 8 a~12). Za overenie, že zvyšky $0$ a~$24$ sú naozaj možné,
dajte 1~bod (stačí uviesť napr. len hodnoty $V(1)$ a~$V(3)$).

\endpetit
\bigbreak
}

{%%%%%   C-S-3
Označme $d$ dĺžku úsečky~$AP$ a~$v$ dĺžku výšky~$CP$ trojuholníka $ABC$.
Dĺžky jeho strán označíme zvyčajným spôsobom $a$, $b$, $c$.
Zo zadania teda vyplýva $|PB|=3d$ (\obr).
\insp{c66.11}%

Použitím Pytagorovej vety v~trojuholníkoch $APC$ a~$PBC$ dostávame rovnosti
$b^2=d^2+v^2$ a~$a^2 = 9d^2+ v^2$.
Z~druhého predpokladu úlohy potom vyplýva rovnosť $a^2=3b^2$, čiže $9d^2+v^2=3d^2+3v^2$,
odkiaľ $v^2=3d^2$. Dosadením do prvých dvoch rovností tak dostávame
$a^2=12d^2$ a~$b^2=4d^2$.
A~keďže $c=4d$, čiže $c^2=16d^2$, dokázali sme, že pre dĺžky strán
trojuholníka~$ABC$ platí $a^2+b^2=c^2$.

Trojuholník $ABC$ je preto podľa obrátenej Pytagorovej vety pravouhlý.

\poznamka
Ak zvážime pomocný pravouhlý trojuholník s~odvesnami $a$ a~$b$, tak pre jeho preponu~$c'$
podľa Pytagorovej vety platí $c'{}^2=a^2+b^2$. Porovnaním s~odvodenou rovnosťou
$c^2=a^2+b^2$ tak dostávame $c'=c$, takže pôvodný trojuholník je podľa vety $sss$
zhodný s~trojuholníkom pomocným, a~je teda skutočne pravouhlý.
Môžeme tolerovať názor, že samotná Pytagorova veta udáva nielen
nutnú, ale aj postačujúcu podmienku na to, aby bol daný trojuholník pravouhlý.

\nobreak\medskip\petit\noindent
Za úplné riešenie dajte 6~bodov. Za napísanie Pytagorovej vety v~oboch
trojuholníkoch $APC$ a~$PBC$ dajte jeden bod.
Za odvodenie vzťahu $v^2=3d^2$ dajte 2~body, za jeho dôsledok $a^2+b^2=c^2$ potom ďalší 1~bod.
Za uvedenie záveru, že potom
je trojuholník pravouhlý, dajte dva body.
Za rôzne manipulácie s~rovnicami bez dosiahnutia vzťahu vedúceho k~správnemu
dôkazu body neudeľujte.

\endpetit
}

{%%%%%   C-II-1
Rovnosť zo zadania je ekvivalentná rovnosti
$P(1)\cdot P(2)\cdot P(3) =4\cdot 17^2$,
takže čísla $P(1)$, $P(2)$, $P(3)$ môžu
byť iba z~množiny deliteľov čísla $4\cdot 17^2$ väčších ako 1:
$$
2<4<17<2\cdot 17<4\cdot 17<17^2<2\cdot 17^2<4\cdot 17^2.
% \left\{2, 4, 17, 2\cdot 17, 4\cdot 17, 17^2, 2\cdot 17^2, 4\cdot 17^2\right\}.
$$

Ak by platilo $P(1)\ge 4$, bol by súčin $P(1)P(2)P(3)$ aspoň
$4\cdot17\cdot (2\cdot17)=8\cdot 17^2$, čo nevyhovuje zadaniu. Preto $P(1)=2$ a~tak
je nutne $P(2)=17$, pretože keby bolo $P(2)=4$, musel by byť daný súčin $4\cdot 17^2$ deliteľný
číslom $P(1)P(2)=8$, čo neplatí, a~pre $P(2)\ge 2\cdot 17$ by bol súčin
$P(1)P(2)P(3)$ opäť príliš veľký.
Pre tretiu neznámu hodnotu~$P(3)$ potom vychádza $P(3)=4\cdot 17^2/(2\cdot 17)=2\cdot 17$.

Hľadané koeficienty $a$, $b$, $c$ tak sú práve také celé čísla,
ktoré vyhovujú sústave
$$
\align
P(1) = \hphantom{2}a~+ \hphantom{2}b + c&=2,\\
P(2) = 4a + 2b + c &=17,\\
P(3) = 9a + 3b + c &=34.
\endalign
$$
Jej vyriešením dostaneme $a=1$, $b=12$, $c=-11$.

\zaver
Úlohe vyhovuje jediný mnohočlen $P(x)=x^2+12x-11$.

\nobreak\medskip\petit\noindent
Za úplné riešenie dajte 6~bodov.
Za konštatovanie faktu, že $P(1)$, $P(2)$, $P(3)$
sú z~množiny deliteľov čísla $4\cdot 17$ dajte 1~bod, za
vypísanie všetkých deliteľov body neudeľujte. Za odvodenie hodnôt, ktorým sa musia
rovnať čísla $P(1)$, $P(2)$, $P(3)$ prideľte 3~body. Za zostavenie (1~bod)
a~vyriešenie (1~bod) sústavy potom záverečné 2~body (tieto body dajte aj v~prípade,
keď hodnoty $P(1)$, $P(2)$, $P(3)$ sú určené chybne, no sústava s~nimi
je zostavená a~vyriešená správne).
\endpetit
\bigbreak
}

{%%%%%   C-II-2
a) Aby sme dosiahli požadované rozmiestnenie čísel v~tabuľke,
nesmú v~žiadnom riadku ani stĺpci spolu zostať
dve z~čísel nanajvýš rovných šiestim. Preto
jednu z~mnohých vyhovujúcich tabuliek zostavíme, keď čísla od~1
do~6 vpíšeme zhora nadol do políčok jednej uhlopriečky a~ďalej
budeme postupne zdola nahor brať rady políčok rovnobežných s~druhou uhlopriečkou
a~do voľných miest každej z~nich vpisovať zhora nadol
zvyšné čísla 7, 8~atď. až~36:
$$
\vbox
{\offinterlineskip\everycr{\noalign{\hrule}}\let\par\cr\obeylines %
\halign{\vrule#&&\hbox to 16pt{\hss#\unskip\hss}\vrule height 11.2pt depth 4.8pt\cr
& 1&35&33&29&25&19\cr
&36& 2&30&26&20&15\cr
&34&31& 3&21&16&11\cr
&32&27&22& 4&12& 9\cr
&28&23&17&13& 5& 7\cr
&24&18&14&10& 8& 6\cr
}}
$$
Najmenšie súčty dvoch čísel z~jednotlivých riadkov (zhora nadol) sú
$$
1+19,\ 2+15,\ 3+11,\ 4+9,\ 5+7,\ 6+8
$$
a~z~jednotlivých stĺpcov (zľava doprava)
$$
1+24,\ 2+18,\ 3+14,\ 4+10,\ 5+8,\ 6+7.
$$

Rýchlejší opis príkladu vyhovujúcej tabuľky a~jeho jednoduchšiu kontrolu dostaneme,
keď do tabuľky vpíšeme iba čísla od~1 do~12, ako vidíme nižšie. Rozmiestnenie
čísel od~13 do~36 do prázdnych políčok už zrejme môže byť ľubovoľné~-- dve najmenšie
čísla v~každom riadku aj stĺpci sú totiž práve tie od~1 do~12.
$$
\vbox
{\offinterlineskip\everycr{\noalign{\hrule}}\let\par\cr\obeylines %
\halign{\vrule#&&\hbox to 16pt{\hss#\unskip\hss}\vrule height 11.2pt depth 4.8pt\cr
&1 &11& & & &\cr
&12& 2& & & &\cr
& & & 3&9 & &\cr
& & &10&4 & &\cr
& & & & &5&7\cr
& & & & &8&6\cr
}}
$$


b) Ak sú dve z~čísel od 1 do 6 v~rovnakom riadku alebo v~rovnakom
stĺpci, ich súčet neprevýši dokonca ani číslo $6+5=11$.
V~opačnom prípade sú čísla od~1 do~6 rozmiestnené vo všetkých riadkoch
a~všetkých stĺpcoch, takže číslo~7 je v rovnakom riadku s~číslom~$x$
a~v~rovnakom stĺpci s~číslom~$y$, pričom $x$ a~$y$ sú dve {\it
rôzne\/} čísla od~1 do~6. Potom menšie z~čísel $7+x$ a~$7+y$
neprevýši menšie z~čísel $7+6$ a~$7+5$, teda číslo~12. Tým je tvrdenie
dokázané.


\nobreak\medskip\petit\noindent
Za úplné riešenie dajte 6~bodov.
Každú z~častí ohodnoťte tromi bodmi. V~časti~a) stačí uviesť akýkoľvek príklad,
najmenšie súčty dvoch čísel v~jednotlivých riadkoch a~stĺpcoch nie je nutné vypisovať,
tiež nie je nutné určovať konkrétne pozície niekoľkých najvyšších čísel, ak je to zdôvodnené~--
napríklad v~úplnom príklade zo vzorového riešenia nezáleží na umiestnení čísel od 25 do 36
na príslušných dvanásť políčok. V~časti~b) dajte 1~bod za úvahu o~rozmiestnení najmenších
čísel od 1 do 6; ďalšie body nestrhávajte, keď riešiteľ ďalej uvažuje iba umiestnenie
týchto šiestich čísel na jednu z~uhlopriečok tabuľky.
\endpetit
\bigbreak
}

{%%%%%   C-II-3
Štyrmi štvorcami so~stranou~30 zrejme zakryjeme obdĺžnik $30\times120$.
Zvyšnú časť $2\times120$ rozdelíme na tri zhodné časti, konkrétne obdĺžniky $2\times40$,
a~ukážeme, ako každý z~nich (rovnako) pokryť jedným z~troch zvyšných štvorcov so stranou~30.
Dosiahneme to, keď štvorec položíme na obdĺžnik tak, že obe uhlopriečky štvorca
budú ležať na osiach súmernosti dotyčného obdĺžnika. Stačí potom ukázať, že obdĺžnik
so~stranou~2 vpísaný do štvorca podľa \obr\ má druhú stranu dlhšiu ako~40.
Jej dĺžka je zrejme
\insp{c66.12}%
$30\sqrt2-2$ (od uhlopriečky štvorca odčítame na každej strane~1
ako veľkosť výšky pravouhlého trojuholníka so~stranami 2, $\sqrt2$,~$\sqrt2$, pozri zväčšenú
časť \obrr1), takže stačí ukázať, že $30\sqrt2-2\ge40$. To je
ekvivalentné s~nerovnosťou $5\sqrt2\ge7$, čiže $50\ge49$, čo je splnené.
Daný obdĺžnik $32\times120$ teda naozaj možno zakryť siedmimi štvorcami so stranou~30.

\nobreak\medskip\petit\noindent
Za úplné riešenie dajte 6~bodov.
Za redukciu úlohy na pokrytie obdĺžnika $2\times 120$ tromi štvorcami so stranou $30$
dajte 1~bod. Za uvažovanie vpísaných obdĺžnikov $2\times\text{niečo}$ dajte tiež 1~bod. Za
výpočet ich dlhšej strany 2~body. Úvahu, že na pokrytie obdĺžnika $2\times40$ či
celého $2\times120$
postačuje platnosť nerovnosti $30\sqrt2-2>40$ či $90\sqrt2-6>120$ oceňte jedným bodom a~jej
dôkaz tiež jedným bodom.
\endpetit
\bigbreak
}

{%%%%%   C-II-4
Nerovnosť vynásobíme kladným výrazom $abc$ a~po roznásobení ju postupne
(ekvivalentne) upravíme:
$$
\align
-a(bc+ac+ab)+b(bc+ac+ab)+c(bc+ac+ab)&\ge 3abc,\\
-abc-a^2c-a^2b+b^2c+abc+ab^2+bc^2+ac^2+abc&\ge 3abc,\\
(b^2c-abc)+(bc^2-abc)+(ac^2-a^2c)+(ab^2-a^2b)&\ge 0,\\
bc(b-a)+bc(c-a)+ac(c-a)+ab(b-a)&\ge 0.\\
\endalign
$$
Vzhľadom na~predpoklad $0<a\le b\le c$ je výsledná, a~teda aj~pôvodná nerovnosť splnená.

\ineriesenie
Dokazovanú nerovnosť postupne upravíme, pričom využijeme známu
nerovnosť $\frc bc+\frc cb\ge 2$, ktorá je pre kladné čísla $b$, $c$ ekvivalentná
s~nerovnosťou $({b-c)^2}\ge0$:
$$
\align
(-a+b+c)
\Big(\frac{1}{a}+\frac{1}{b}+\frac{1}{c}\Big)
=&1+\Big(\frac{b}{a}-\frac{a}{b}\Big)+\Big(\frac{c}{a}-\frac{a}{c}\Big)
+\Big(\frac{b}{c}+\frac{c}{b}\Big)\ge\\
\ge&
1+\frac{b^2-a^2}{ab}+\frac{c^2-a^2}{ac}+2\ge 3,
\endalign
$$
pretože zrejme platí aj $a^2\le b^2\le c^2$.

\ineriesenie
Podľa predpokladov úlohy platia nerovnosti
$$
\m a+b+c\ge c\quad\text{a}\quad \frac1a+\frac1b+\frac1c\ge\frac2b+\frac1c.
$$
Obe nerovnosti (s~kladnými stranami) medzi sebou vynásobíme a~získame tak
$$
(\m a+b+c)\Big(\frac1a+\frac1b+\frac1c\Big)\ge c\Big(\frac2b+\frac1c\Big)
=1+\frac{2c}{b}\ge3,
$$
pretože $c/b\ge1$ podľa zadania.

\nobreak\medskip\petit\noindent
Za úplné riešenie dajte 6~bodov.
Za násobenie nerovnice výrazom, o~ktorom
nie je povedané, že je kladný, strhnite 1~bod. Za použitie nerovnosti $\frc
xy+\frc yx\ge2$ bez konštatovania kladnosti $x$ a~$y$ strhnite taktiež bod.
\endpetit
}

{%%%%%   vyberko, den 1, priklad 1
...}

{%%%%%   vyberko, den 1, priklad 2
...}

{%%%%%   vyberko, den 1, priklad 3
...}

{%%%%%   vyberko, den 1, priklad 4
...}

{%%%%%   vyberko, den 2, priklad 1
...}

{%%%%%   vyberko, den 2, priklad 2
...}

{%%%%%   vyberko, den 2, priklad 3
...}

{%%%%%   vyberko, den 2, priklad 4
...}

{%%%%%   vyberko, den 3, priklad 1
...}

{%%%%%   vyberko, den 3, priklad 2
...}

{%%%%%   vyberko, den 3, priklad 3
...}

{%%%%%   vyberko, den 3, priklad 4
...}

{%%%%%   vyberko, den 4, priklad 1
...}

{%%%%%   vyberko, den 4, priklad 2
...}

{%%%%%   vyberko, den 4, priklad 3
...}

{%%%%%   vyberko, den 4, priklad 4
...}

{%%%%%   vyberko, den 5, priklad 1
...}

{%%%%%   vyberko, den 5, priklad 2
...}

{%%%%%   vyberko, den 5, priklad 3
...}

{%%%%%   vyberko, den 5, priklad 4
...}

{%%%%%   trojstretnutie, priklad 1
...}

{%%%%%   trojstretnutie, priklad 2
...}

{%%%%%   trojstretnutie, priklad 3
...}

{%%%%%   trojstretnutie, priklad 4
...}

{%%%%%   trojstretnutie, priklad 5
...}

{%%%%%   trojstretnutie, priklad 6
...}

{%%%%%   IMO, priklad 1
Riešenie rozdelíme na niekoľko častí.
Najprv predpokladajme, že $3\mid a_1$. Indukciou dokážeme, že potom aj každý ďalší člen je deliteľný 3: Ak $3\mid a_n$, tak ${3\mid a_n+3}$. A tak isto ak $a_n$ je štvorec a $3\mid a_n$, tak aj $3\mid \sqrt{a_n}$.

Zoberme štvorec deliteľný $3$, ktorý je väčší ako $a_1$, \tj. nájdime prirodzené $k$ také, že $9k^2>a_1$. Indukciou ľahko dokážeme, že $a_n\le 9k^2$ pre všetky $n$. Totiž pre prvý člen to platí, a ak $a_n<9k^2$, tak dostávame, že $a_{n+1}\le a_n+3\le 9k^2$, pretože $a_n$ je deliteľné~3. No a v prípade, že $a_n=9k^2$, je $a_{n+1}=3k<9k^2$, čím je dôkaz indukciou ukončený.

To ale znamená, že naša postupnosť je ohraničená -- a nachádza sa v nej len konečne veľa rôznych hodnôt, a preto aspoň jedna z nich sa v nej musí vyskytovať nekonečne veľakrát.

Teraz predpokladajme, že $a_1$ nie je deliteľné 3. Najprv predpokladajme, že $a_n\equiv 2\pmod{3}$ pre nejaký index $n$. Keďže žiadna druhá mocnina nedáva zvyšok 2 po delení~3, od tohto momentu bude každý ďalší člen o 3 väčší a celá postupnosť (od $a_n$)  bude rastúca. Preto nebude existovať číslo, ktoré je v danej postupnosti nekonečne veľakrát.

Teraz ukážeme, že ak $a_1$ nie je deliteľné 3, tak existuje taký index $n$, pre ktorý bude $a_n\equiv 2\pmod{3}$. V prípade, že $a_1\equiv 2\pmod{3}$, máme tvrdenie dokázané hneď. Tiež to platí pre $a_1=4$, nakoľko dostávame $a_2=2$, takže to platí pre $a_1\le 4$. Ďalej to dokážeme indukciou. Nech tvrdenie platí pre všetky $a_1\le 3k+1$. Dokážeme, že tvrdenie platí aj pre všetky $a_1\le 3k+4$. Zrejme jediný zaujímavý prípad je $a_1=3k+4$, nakoľko pre $a_1=3k+2$ to je triviálne.

Postupnosť bude stále rásť o 3, až kým \uv{nenarazí} na nejaký štvorec. Vieme, že $3k+4<(3k+1)^2\Leftrightarrow 9k^2+3k>3$, keďže $k\ge 1$. A keďže postupnosť nadobúda postupne všetky čísla so zvyškom 1 po delení 3 až kým nenarazí na štvorec, je jasné, že keď narazí na štvorec, tak to bude štvorec čísla, ktoré nie je deliteľné 3 a ktoré je nanajvýš $3k+1$. A zároveň celý čas platí, že $a_n>1$, lebo ak $a_n>1$, tak aj $\sqrt{a_n}>1$. To znamená, že existuje nejaký index $m$, pre ktorý je $1<a_m\le 3k+1$ a $3\nmid a_m$. Teraz už len využijeme indukčný predpoklad a dostávame, že tvrdenie platí aj pre $a_1=3k+4$.

Keď to zhrnieme, dostávame, že ak $3\nmid a_1$, tak postupnosť bude od istého člena rastúca, a preto tvrdenie neplatí.
Naopak pre všetky $a_1$ deliteľné 3 máme dokázanú platnosť tvrdenia. Záver je ten, že vyhovujú práve tie $a_1$, ktoré sú deliteľné 3.}

{%%%%%   IMO, priklad 2
Dosaďme do rovnosti zo zadania také $x$, $y$, pre ktoré platí $xy=x+y$, čiže $y=\frc{x}{(x-1)}$. Dostaneme
$$
f\left(f(x)f\left(\frac{x}{x-1}\right)\right)=0\qquad\text{pre všetky $x\ne1$.}
\tag1
$$
V prípade, že $f(x)=0$ pre nejaké $x\ne1$, zo vzťahu (1) dostávame, že aj $f(0)=0$. Potom po dosadení $[0,y]$ máme $f(y)=0$ pre všetky reálne $y$. Skúškou ľahko overíme, že táto funkcia vyhovuje.
Ďalej môžeme predpokladať, že $f(x)\ne0$ pre $x\ne1$. Na druhej strane z~(1) vidíme, že naša funkcia nejaký nulový bod má (v bodoch $f(x)f(\frc{x}{(x-1)})$). To znamená, že nulová hodnota funkcie~$f$ musí byť práve v~bode~1, čiže $f(1)=0$.

Po dosadení $x=0$ do (1) máme $f(f(0)^2)=0$, čiže nutne $f(0)^2=1$. Navyše si môžeme všimnúť, že ak vyhovuje funkcia $f$, tak vyhovuje aj funkcia $-f$. Preto môžeme bez ujmy na všeobecnosti predpokladať, že $f(0)=-1$.

Teraz môžeme dosadiť $[x,1]$, dostaneme
$$
-1+f(x+1)=f(x).
\tag2
$$
Z toho indukciou ľahko ukážeme, že $f(n)=n-1$ pre všetky celé čísla $n$.

Následne ukážeme, že funkcia $f$ je prostá. Predpokladajme, že $f(a)=f(b)$ pre nejaké $a\ne b$. Z (2) vyplýva, že potom aj $f(a+N+1)=f(b+N)+1$ pre ľubovoľné prirodzené číslo~$N$. Zvolíme $x_0$, $y_0$ tak, aby $x_0+y_0=a+N+1$ a $x_0y_0={b+N}$. Podľa Vi\`etových vzťahov sú hľadané $x_0$, $y_0$ riešenia kvadratickej rovnice $x^2-{(a+N)x}+{(b+N)}=0$. Tá má dve reálne riešenia, ak jej diskriminant je kladný, \tj. ak ${(a+N+1)^2}+4b-4N>0$. Je jasné, že nie je problém zvoliť dostatočne veľké $N$ tak, aby to platilo, stačí zjavne $N>-b$.

Po dosadení $[x_0,y_0]$ do rovnosti zo zadania dostaneme, že $f(f(x_0)f(y_0))=-1$. Keďže jediný bod, v ktorom sa nadobúda funkčná hodnota 0, je 1, zo vzťahu (2) vyplýva, že jediný bod, v ktorom sa nadobúda funkčná hodnota $-1$, je 0. Preto $f(x_0)f(y_0)=0$, čiže buď $f(x_0)=0$, alebo $f(y_0)=0$. Bez ujmy na všeobecnosti $f(x_0)=0$, a preto $x_0=1$. To ale znamená, že $a+N+1=y_0+1$ a $b+N=y_0$, čiže $a=b$, čo je spor. Preto je funkcia $f$ prostá.

Teraz dosaďme do rovnosti zo zadania $[t,-t]$. Postupne s využitím (2) a prostosti dostávame
$$
\align
f(f(t)f(-t))-1 &= f(-t^2),\\
f(f(t)f(-t)) &= f(-t^2+1),\\
f(t)f(-t) &= -t^2+1.
\endalign
$$
Ďalej dosadíme $[t,1-t]$ a máme s využitím prostosti a (2)
$$
\align
f(f(t)f(1-t)) &= f(t-t^2),\\
f(t)f(1-t)&=t-t^2,\\
f(t)f(-t)+f(t)&=t-t^2.
\endalign
$$
Porovnaním ostatných dvoch vzťahov dostávame $f(t)=t-1$ pre všetky reálne~$t$. Nezabudnime na to, že sme na začiatku predpokladali, že $f(0)$ je záporné. Preto dostávame ďalšie dve riešenia, a to $f(x)=x-1$ a $f(x)=1-x$. Skúškou overíme, že aj tieto vyhovujú.

\odpoved
Vyhovujúce funkcie sú $f(x)=x-1$, $f(x)=1-x$ a $f(x)=0$.}

{%%%%%   IMO, priklad 3
Keby bola odpoveď \uv{áno}, tak by mal poľovník stratégiu, ktorá funguje bez ohľadu na to, ako sa zajac hýbe a čo ukazuje sledovacie zariadenie. My ukážeme opak. Ukážeme, že ak má poľovník smolu na to, čo mu ukáže sledovacie zariadenie, tak neexistuje stratégia, ktorá mu zabezpečí, že bude po $10^9$ kolách vzdialený od zajaca najviac 100.

Označme $d_n$ vzdialenosť medzi zajacom a poľovníkom po $n$ kolách. Ak sa niekedy stane, že $d_n\ge 100$ pre $n<10^9$, tak zajac vie ľahko túto vzdialenosť zachovať. Stačí, ak sa vždy posunie o 1 smerom od poľovníka. Poľovník túto vzdialenosť potom nevie zmenšiť a zajac vyhrá.

Ukážeme, že ak $d_n<100$, tak bez ohľadu na to, akú stratégiu má poľovník, zajac má spôsob, ktorým zväčší $d_n^2$ aspoň o~$\frac12$ každých 200 kôl (ak sledovacie zariadenie ukazuje šťastne v~prospech zajaca). Týmto spôsobom dosiahne $d_n^2$ hodnotu 10\,000 za $4\cdot 10^6<10^9$ kôl a zajac vyhrá.

Predpokladajme, že poľovník je v~bode~$B_n$ a zajac v~bode~$A_n$. Zajac sa rozhodne v~tomto momente poľovníkovi ukázať. Tým samozrejme len situáciu poľovníkovi uľahčí. Nech $r$ je priamka~$A_nB_n$. Zostrojíme pravouhlý trojuholník $A_nZY_1$ s~pravým uhlom pri vrchole~$Z$ tak, že $Z$ leží na polpriamke opačnej k~$A_nB_n$. Navyše nech $|A_nY_1|=200$ a~$|ZY_1|=1$. Ďalej nech $Y_2$ je obraz $Y_1$ v~stredovej súmernosti podľa $Z$ ako na \obr.
\insp{mmo66.1}%

Zajacov plán je veľmi jednoduchý: Zvolí si jeden z bodov $Y_1$ a $Y_2$ a skáče 200 kôl rovno k nemu. Ak bude mať poľovník smolu, tak sledovacie zariadenie ukáže vždy bod na priamke $r$, ktorá je kolmá projekcia zajacovej polohy na priamku. To znamená, že po 200 kolách ukáže bod $Z$.

Čo môže poľovník robiť? Môže sa posunúť o 200 po priamke $r$ do bodu $B'$. Tak sa vzdialenosť medzi poľovníkom a zajacom zväčší na dĺžku úsečky $B'Y_1$. V skutočnosti poľovník nevie urobiť nič lepšie. Vždy totiž skončí niekde \uv{naľavo} od bodu~$B'$. Ak sa dostane nad priamku~$r$, tak môže mať smolu, že si zajac vybral bod~$Y_2$, ktorý je pod priamkou~$r$ a~určite jeho vzdialenosť bude väčšia ako dĺžka úsečky~$B'Y_1$. Ak skončí pod priamkou~$r$, tak mohol mať smolu, že zajac je v~bode~$Y_1$. Skrátka, bez ohľadu na to, čo spraví, môže mať smolu a~zajac sa vzdiali na aspoň $|B'Y_1|=|B'Y_2|=y$.

Teraz skúsime odhadnúť $y^2$. Vieme, že $|A_nB_n|=d_n$. Nech $A'$ je taký bod na polpriamke~$A_nZ$, že $|A_nA'|=200$ a~označme $\varepsilon=|ZA'|$. Zrejme $|A'B'|=d_n$. Potom
$$
y^2=1+|B'Z'|^2=1+(d_n-\varepsilon)^2.
$$
Vieme tiež, že
$$
\varepsilon=200-|A_nZ|^2=200-\sqrt{200^2-1}=\frac1{200+\sqrt{200^2-1}}>\frac1{400}.
$$
Z vyjadrenia pre $\varepsilon$ ľahko zistíme, že $\varepsilon$ je riešenie kvadratickej rovnice $x^2-400x+1=0$, a teda $\varepsilon^2+1=400\varepsilon$. Potom
$$
y^2=1+d_n^2-2d_n\varepsilon+\varepsilon^2=d_n^2+\varepsilon(400-2d_n).
$$
Keďže $\varepsilon>\frc1{400}$ a predpokladali sme, že $d_n<100$, máme $y^2>d_n^2+\frac12$. To znamená, že sme ukázali, že bez ohľadu na to, akú stratégiu má poľovník, sa môže stať, že $d_{n+200}^2>d_n^2+\frac12$. Preto zajac vyhrá.}

{%%%%%   IMO, priklad 4
%Nejaký normálny obrázok k tejto geometrii treba
Z obvodových uhlov na kružniciach $\Omega$ a $\Gamma$ máme  $|\uhol KRS|=|\uhol KJS|=|\uhol ATS|$. Na druhej strane $RA$ je dotyčnica k $\Omega$ a z úsekového uhla máme rovnosť $|\uhol SKR|=|\uhol SRA|$. Preto sú trojuholníky $ART$ a $SKR$ podobné podľa vety $uu$ (\obr) a platí $$\frac{|RT|}{|RK|}=\frac{|AT|}{|SR|}=\frac{|AT|}{|ST|}.$$
Využili sme, že $|SR|=|ST|$. Z rovnosti, ktorú sme dostali a vzťahu $|\uhol SKR|=|\uhol SRA|$ vyplýva, že trojuholníky $AST$ a $TKR$ sú podobné podľa vety $sus$. Z tejto podobnosti máme rovnosť $|\uhol SAT|=|\uhol KTR|$. A z toho vyplýva, že $KT$ je dotyčnica ku $\Gamma$ v~bode~$T$, čo sme chceli dokázať.\insp{mmo66.2}%
}

{%%%%%   IMO, priklad 5
Rozdelíme rad do $N$ blokov tak, že v každom bloku je $N+1$ za sebou stojacich ľudí. Ukážeme, že môžeme z každého bloku odstrániť $N-1$ ľudí tak, aby sme splnili podmienky.

Najprv výšky jednotlivých ľudí dáme do tabuľky $(N+1)\times N$ tak, že do prvého stĺpca dáme výšky ľudí z prvého bloku, ale zoradíme ich od najväčšej po najmenšiu. Všeobecne do $i$-teho stĺpca dáme výšky ľudí z $i$-teho bloku, zoradené od najväčšej po najmenšiu.

Následne budeme vymieňať stĺpce nasledujúcim spôsobom: Najprv si vyberieme stĺpec, ktorého číslo v druhom riadku je najväčšie. Tento stĺpec dáme úplne naľavo. Potom zo zvyšných stĺpcov vyberieme ten, ktorého tretie číslo je najväčšie, ten bude druhý zľava, atď. Všeobecne ako $k$-ty stĺpec zľava dáme ten, ktorý má na pozícii $k+1$ zo zvyšných stĺpcov najväčšie číslo.

Dostaneme tak nasledujúcu tabuľku:

$$
\matrix
    \boldmath{x_{1,1}} && x_{1,2} && x_{1,3} & \dots  & x_{1,N-1} && x_{1,N} \\
    \boldmath{\vee} && \vee && \vee &  & \vee && \vee \\
    \boldmath{x_{2,1}} &\boldmath{>}&\boldmath{ x_{2,2}} && x_{2,3} & \dots  & x_{2,N-1} && x_{2,N} \\
    \vee && \boldmath{\vee} && \vee &  & \vee && \vee \\
    x_{3,1} && \boldmath{x_{3,2}} &\boldmath{>}& \boldmath{x_{3,3}} & \dots  & x_{3,N-1} && x_{3,N} \\
    \vee && \vee && \boldmath{\vee} &  & \vee && \vee \\
    \vdots && \vdots && \vdots & \ddots & \vdots && \vdots \\
    \vee && \vee && \vee &  & \boldmath{\vee} && \vee \\
    x_{N,1} && x_{N,2} && x_{N,3} & \dots  & \boldmath{x_{N,N-1}} & \boldmath{>}& \boldmath{x_{N,N}} \\
    \vee && \vee && \vee &  & \vee && \boldmath\vee \\
    x_{N+1,1} && x_{N+1,2} && x_{N+1,3} & \dots  & x_{N+1,N-1} && \boldmath {x_{N+1,N}} \\
\endmatrix
$$
Teraz vyberieme tých ľudí, ktorých výšky sú v tabuľke znázornené tučne. Vďaka tomu, ako sme usporiadali stĺpce, ich máme zoradených tak, ako naznačujú znamienka. Vidíme, že dvojica s poradiami $2k-1$ a $2k$ je v jednom stĺpci. No v každom stĺpci sú stále ľudia z jedného bloku, a keďže sme z každého stĺpca vybrali len dvoch ľudí, medzi nimi nikto nie je. To znamená, že tento výber vyhovuje podmienkam a máme to, čo sme chceli.}

{%%%%%   IMO, priklad 6
Označme $(x_1,y_1)$, $(x_2,y_2)$, \dots, $(x_k,y_k)$ jednotlivé body množiny $\mn S$. Tvrdenie dokážeme indukciou vzhľadom na $k$. Pre $k=0$ je to triviálne, vyhovuje ľubovoľný polynóm, napr. $P(x,y)=x+y$.

Ďalej predpokladajme, že pre $k-1$ tvrdenie platí. Ukážeme, že platí aj pre $k$. Najprv ukážeme, že môžeme bez ujmy na všeobecnosti predpokladať, že $(x_k,y_k)=(1,0)$. Ak totiž úlohu vyriešime pre tento prípad, tak nasledujúcim spôsobom ju vieme vyriešiť aj v ostatných prípadoch:

Keďže sú $x_k$ a $y_k$ nesúdeliteľné, tak existujú celé čísla $c$, $d$ také, že $cx_k+dy_k=1$. Položme $p_i=cx_i+dy_i$ a $q_i=-y_kx_i+x_ky_i$ pre všetky $1\le i\le n$. Všimnime si, že platí $(p_k,q_k)=(1,0)$. Predpokladajme, že $D\mid p_i$, $D\mid q_i$ pre nejaké $1\le i\le n$. Keďže platí
$$
y_kp_i+aq_i=ay_kx_i+by_ky_i-y_kax_i+ax_ky_i=(by_k+ax_k)y_i=y_i,
$$
nutne $D\mid y_i$. Na druhej strane $x_kp_i-bq_i=x_i$, a~preto $D\mid x_i$. My však vieme, že $x_i$ a~$y_i$ sú nesúdeliteľné, preto nutne $D\mid 1$. Z toho však vyplýva, že aj $p_i$, $q_i$ sú nesúdeliteľné.

Našu úlohu s bodmi $(x_i,y_i)$ sme tak previedli na úlohu s bodmi $(p_i,q_i)$, kde posledný bod je $(1,0)$. A ak tento problém vieme vyriešiť, tak máme polynóm $P(p,q)$ taký, že $P(p_i,q_i)=0$. Teraz je jednoduché každý výskyt $p$ v tom polynóme nahradiť $cx+dy$ a každý výskyt $q$ nahradiť $-y_kx+x_ky$. Dostaneme tak určite homogénny polynóm $$Q(x,y)=P(cx+dy,-y_kx+x_ky),$$ ktorý má zrejme všetky koeficienty celočíselné a platí $Q(x_i,y_i)=P(p_i,q_i)=1$, čím je problém vyriešený.

Vráťme sa naspäť k indukcii a predpokladajme, že $(x_k,y_k)=(1,0)$.
Vieme, že z~indukčného predpokladu máme polynóm $$P(x,y)=a_0x^n+a_1x^{n-1}y+\dots+a_ny^n$$ taký, že $P(x_i,y_i)=1$ pre všetky $1\le i\le k-1$. Označme $$Q(x,y)=(y_1x-x_1y)(y_2x-x_2y)\dots(y_{k-1}x-x_{k-1}y).$$ Zjavne $Q(x_i,y_i)=0$ pre všetky $1\le i\le k-1$.

V špeciálnom prípade, keď platí $y_i=0$ pre nejaké $1\le i\le k-1$, máme $|a_0|=1$. Ľahko overíme, že v tomto špeciálnom prípade polynóm $P(x,y)^2$ vyhovuje. Inak budeme náš polynóm hľadať v tvare $$R(x,y)=P(x,y)^l-Mx^{nl-k+1}Q(x,y),$$ kde $l$, $M$ sú celé čísla, $l\ge 1$. Zjavne $R(x_i,y_i)=1$ pre všetky $1\le i\le k-1$. Potrebujeme len zariadiť, aby $R(1,0)=1$. Ak však vyjadríme $R(1,0)$, dostávame $$R(1,0)=a_0^l-My_1y_2\dots y_{k-1}.$$
Vidíme, že potrebujeme nájsť také $l$, aby $a_0^l\equiv 1\pmod{y_1y_2\dots y_{k-1}}$. Potom už triviálne zvolíme vhodné $M$. My vieme, že to platí pre $$l=\varphi(y_1y_2\dots y_{k-1})$$ za predpokladu, že $a_0$ a $y_1y_2\dots y_{k-1}$ sú nesúdeliteľné. Avšak keby platilo $D\mid a_0$, $D\mid y_i$ pre nejaké $i$, tak by platilo aj $D\mid P(x_i,y_i)=1$. To znamená, že $a_0$ a $y_i$ sú nesúdeliteľné pre všetky $1\le i\le k-1$, a preto je $a_0$ nesúdeliteľné aj s ich súčinom. To znamená, že naozaj môžeme zvoliť $l=\varphi(y_1y_2\dots y_{k-1})$ a nájdeme hľadaný polynóm $R$. Tým je dôkaz indukciou ukončený.
}

{%%%%%   MEMO, priklad 1
...}

{%%%%%   MEMO, priklad 2
...}

{%%%%%   MEMO, priklad 3
...}

{%%%%%   MEMO, priklad 4
...}

{%%%%%   MEMO, priklad t1
...}

{%%%%%   MEMO, priklad t2
...}

{%%%%%   MEMO, priklad t3
...}

{%%%%%   MEMO, priklad t4
...}

{%%%%%   MEMO, priklad t5
...}

{%%%%%   MEMO, priklad t6
...}

{%%%%%   MEMO, priklad t7
...}

{%%%%%   MEMO, priklad t8
...} 