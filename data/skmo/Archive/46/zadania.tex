{%%%%%   A-I-1
Pre každé prirodzené číslo $k$ označme $n_k$ súčin prvých $k$ prvočísel
(napr. $n_3=2\cdot3\cdot5=30$). Zistite, pre ktoré čísla $k$ je
možné zlomok
$$
3^{n_k}-1\over  n_k
$$
krátiť číslom väčším ako~$2$.}
\podpis{R. Kollár}

{%%%%%   A-I-2
Nájdite všetky dvojice mnohočlenov
$$
f(x)=x^2+ax+b,\quad g(x)=x^2+cx+d,
$$
ktoré spĺňajú tieto podmienky:
\itemitem{(1)} Každý z mnohočlenov $f$, $g$ má dva rôzne reálne korene.
\itemitem{(2)} Ak je $s$ ľubovoľný koreň $f$, potom aj $g(s)$ je koreň $f$.
\itemitem{(3)} Ak je $s$ ľubovoľný koreň $g$, potom aj $f(s)$ je koreň $g$.
\endgraf
}
\podpis{J. Šimša}

{%%%%%   A-I-3
V~ľubovoľnom trojuholníku $ABC$ označme $a$, $b$, $c$ dĺžky jeho strán
a~$t_a$, $t_b$, $t_c$ dĺžky jeho ťažníc zvyčajným spôsobom.
Zistite, či je niektorá z~nerovností
$$
a<\frac{b+c}2, \quad t_a>\frac{t_b+t_c}2
$$
dôsledkom druhej, alebo či dokonca nejde o~dve ekvivalentné
nerovnosti.}
\podpis{J. Šimša}

{%%%%%   A-I-4
Do daného kruhového odseku sú vpísané kružnice $k_1$, $k_2$. Kružnica~$k_1$ sa dotýka oblúka odseku v~bode~$A$ a~základne odseku v~bode~$B$.
Kružnica~$k_2$ sa dotýka oblúka odseku v~bode~$C$ a~základne odseku
v~bode~$D$.
\itemitem{a)} Dokážte, že body $A$, $B$, $C$, $D$ ležia na jednej kružnici.
\itemitem{b)} Uvažujme všetky také dvojice kružníc $k_1$, $k_2$,
ktoré sa navyše vzájomne dotýkajú. Aký útvar vyplnia body ich dotyku?
}
\podpis{J. Zhouf}

{%%%%%   A-I-5
Vo vnútri pravidelného štvorstena $ABCD$
sú dané body $E$, $F$ tak, že žiadne
štyri z~bodov $A$, $B$, $C$, $D$, $E$, $F$ neležia v~jednej rovine.
Štvorsten $ABCD$ je bezo zvyšku rozrezaný na niekoľko menších
štvorstenov, ktorých vrcholy tvoria množinu $\{A,B,C,D,E,F\}$. Určte
všetky možné počty menších štvorstenov, na ktoré sa dá daný štvorsten
uvedeným spôsobom rozrezať.}
\podpis{P. Hliněný}

{%%%%%   A-I-6
Každá z~uhlopriečok pravidelného $n$-uholníka ($n\ge5$) je ofarbená
jednou z~dvoch farieb (modrou alebo červenou). Je povolené postupné
prefarbovanie uhlopriečok tak, že v~každom kroku vyberieme jeden
vrchol a~zmeníme farby všetkých uhlopriečok, ktoré z~neho vychádzajú
(z~modrej na červenú a~naopak).  Rozhodnite, či možno vždy
uhlopriečky prefarbiť tak, aby existovala
\itemitem{a)} lomená čiara,
\itemitem{b)} uzavretá lomená čiara
\endgraf
\noindent
zložená iba z~modrých uhlopriečok a~prechádzajúca každým vrcholom
$n$-uholníka práve raz.
}
\podpis{J. Kratochvíl}

{%%%%%   B-I-1
Pre ktoré prirodzené čísla~$n$ možno v~pravidelnom $n$-uholníku nájsť
uzavretú lomenú čiaru zloženú z~$n$~uhlopriečok $n$-uholníka tak, aby
prechádzala všetkými vrcholmi $n$-uholníka, a~aby
každé dve z~týchto uhlopriečok mali spoločný bod?}
\podpis{P. Hliněný}

{%%%%%   B-I-2
Nájdite všetky kvadratické funkcie, ktoré zobrazia interval
$\langle 2,5\rangle$ na interval $\langle 15,27 \rangle$,
a~ktorých graf prechádza počiatkom súradnicového systému.}
\podpis{P. Černek}

{%%%%%   B-I-3
Koľko 24-ciferných prirodzených čísel, ktorých dekadický zápis obsahuje
22~cifier~$1$ a~dve cifry~$2$, je deliteľných siedmimi?}
\podpis{T. Hecht}

{%%%%%   B-I-4
V~obore reálnych čísel riešte sústavu rovníc
$$
\align
x+y+z&=3,\\
\frac1x+\frac1y+\frac1z&=0,\\
\frac xy+\frac yz+\frac zx&=\frac yx+\frac zy+\frac xz.
\endalign
$$
}
\podpis{J. Šimša}

{%%%%%   B-I-5
V~rovnobežníku $ABCD$ označme $E$ a~$F$ po rade stredy strán $BC$ a~$CD$.
Veďte rovnobežku s~priamkou~$BD$, ktorá pretína obvod
štvoruholníka v~bodoch $K$, $L$ tak, aby úsečka~$KL$ bola
rozdelená úsečkami $AE$, $AC$, $AF$ na štyri zhodné úseky.}
\podpis{J. Zhouf}

{%%%%%   B-I-6
Nad stranami ostrouhlého trojuholníka $ABC$ sú zvonku zostrojené
polkružnice. Označme po rade $K$, $L$, $M$ priesečníky predĺžených
výšok trojuholníka z~vrcholov $A$, $B$, $C$ s~týmito polkružnicami. Dokážte,
že obrazec $AMBKCL$ tvorí plášť štvorstena (trojbokého ihlanu
s~podstavou $ABC$).}
\podpis{P. Leischner}

{%%%%%   C-I-1
Číslo $4\,896$ je deliteľné ako svojím prvým dvojčíslím ($48$),
tak aj svojím posledným dvojčíslím ($96$). Koľko je štvorciferných čísel
s~touto vlastnosťou, ktoré sú navyše deliteľné~ $17$-timi?}
\podpis{J. Šimša}

{%%%%%   C-I-2
Každá strana konvexného štvoruholníka $ABCD$ je dvoma bodmi rozdelená
na~tri zhodné úsečky (\obr). Ukážte, že štvoruholníky $KLMN$
a~$PQRS$ majú rovnaký obsah.
\insp{c46.1}%
}
\podpis{J. Zhouf}

{%%%%%   C-I-3
V~pravouhlom trojuholníku $ABC$ je $K$ stred prepony~$AB$ a~bod~$M$ leží
na odvesne~$AC$ tak, že $|AM|=2|MC|$. Dokážte, že uhly $MKC$ a~$ABM$ sú
zhodné.}
\podpis{Prevzatá úloha}

{%%%%%   C-I-4
Na dvore sa hrali Karol, Miro a~Jaro.
Karol si myslel dve dvojciferné čísla. Mirovi prezradil ich
rozdiel. Ten správne našiel všetkých deväť takých dvojíc. Jarovi
Karol prezradil súčin oboch čísel. Jaro správne určil všetkých osem
dvojíc s~uvedeným súčinom. Ktoré čísla si Karol myslel?}
\podpis{P. Černek}

{%%%%%   C-I-5
V~pravidelnom trojbokom ihlane $ABCV$ je dĺžka bočnej hrany
$|AV|=5\cm$, dĺžka hrany podstavy $|AB|=4\sqrt3\cm$. Body $K$,
$L$, $M$ sú päty kolmíc vedených vnútorným bodom~$X$ podstavy
$ABC$ na bočné hrany $AV$, $BV$, $CV$. Ako je potrebné voliť bod~$X$,
aby guľová plocha prechádzajúca bodmi $K$, $L$, $M$ a~$X$ mala
čo najmenší priemer? Vypočítajte tento priemer.}
\podpis{P. Leischner}

{%%%%%   C-I-6
Nájdite všetky trojice celých čísel $x$, $y$, $z$, pre ktoré platí
$x+yz=y+xz=z+xy=6$.}
\podpis{J. Zhouf}

{%%%%%   A-S-1
Určte všetky dvojice prvočísel $p$ a~$q$, pre ktoré
platí $5^p=6+7q$.}
\podpis{J. Šimša}

{%%%%%   A-S-2
Daná je tetiva~$UV$ kružnice~$k$. Označme $L_1$ a~$L_2$
priesečníky osi úsečky~$UV$ s~kružnicou~$k$. Do~kruhového odseku
vymedzeného tetivou~$UV$ a~oblúkom~$UL_1V$ sú vpísané dve kružnice
dotýkajúce sa oblúka aj tetivy a~pretínajúce sa v~dvoch
rôznych bodoch $M$ a~$N$. Dokážte, že priamka~$MN$ prechádza
bodom~$L_2$.}
\podpis{P. Hliněný}

{%%%%%   A-S-3
Dokážte, že ak pre reálne čísla $a$, $b$, $c$ platí $a+b+c=1$,
tak
$$
2(a^2+b^2+c^2)+ab+bc+ca\ge1.
$$
}
\podpis{R. Kollár}

{%%%%%   A-II-1
Ak je súčet $5^n+3^n+1$ prvočíslo, potom je prirodzené
číslo~$n$ deliteľné dvanástimi. Dokážte.}
\podpis{J. Šimša}

{%%%%%   A-II-2
Určte, pre ktoré veľkosti uhla $DAB$ možno do
kosoštvorca $ABCD$ vpísať dve kružnice $k_1(S_1,r_1)$
a~$k_2(S_2,r_2)$ s~týmito vlastnosťami: Kružnice $k_1$ a~$k_2$
majú vonkajší dotyk, $r_2=2r_1$, kružnica~$k_1$ sa dotýka ramien uhla
$DAB$, kružnica~$k_2$ sa dotýka ramien uhla $BCD$ a~obidve kružnice
ležia v~danom kosoštvorci.}
\podpis{J. Zhouf}

{%%%%%   A-II-3
Postupnosť čísel $(a_n)_{n=1}^{\infty}$ je definovaná
rekurentne:
$$
\align
a_1=&2, \\
a_n=&2(n+a_{n-1}) \  \text{ pre každé $n\ge2$.}
\endalign
$$
Dokážte, že nerovnosť $a_n\le2^{n+2}$ platí pre každé prirodzené
číslo~$n$.}
\podpis{J. Kratochvíl}

{%%%%%   A-II-4
Nájdite mnohosten s~najmenším počtom vrcholov taký, že
žiadne tri jeho steny nemajú rovnaký počet hrán.}
\podpis{J.Kratochvíl}

{%%%%%   A-III-1
Označme strany a~uhly  trojuholníka $ABC$ zvyčajným spôsobom.
Dokážte, že z~rovnosti $\alpha=3\beta$ vyplýva $(a^2-b^2)(a-b)=bc^2$.
Rozhodnite, či tiež naopak z~rovnosti $({a^2-b^2})(a-b)=bc^2$
vyplýva $\alpha=3\beta$.}
\podpis{J. Šimša}

{%%%%%   A-III-2
Každá strana aj uhlopriečka pravidelného $n$-uholníka,
kde $n\ge3$ je nepárne, je ofarbená buď modrou, alebo červenou
farbou. Je povolené prefarbovať tieto úsečky len tak, že
v~každom kroku vyberieme jeden vrchol a~zmeníme farby všetkých úsečiek,
ktoré z~neho vychádzajú (z~modrej na červenú a~naopak). Dokážte, že
každé začiatočné ofarbenie možno týmto postupom zmeniť tak, aby nakoniec
z~každého vrcholu vychádzal párny počet modrých úsečiek. Dokážte
tiež, že takéto výsledné ofarbenie je jednoznačne určené začiatočným
ofarbením.
}
\podpis{J. Kratochvíl}

{%%%%%   A-III-3
Štvorsten $ABCD$ je bezo zvyšku rozdelený na päť konvexných
mnohostenov tak, že žiadna jeho stena nie je rozdelená a~prienik
každých dvoch z~piatich vzniknutých mnohostenov je buď spoločný vrchol,
alebo spoločná hrana, alebo spoločná stena. Aký je najmenší možný
súčet počtov stien týchto piatich mnohostenov?}
\podpis{P. Hliněný}

{%%%%%   A-III-4
Dokážte, že existuje rastúca postupnosť $(a_n)_{n=1}^{\infty}$
prirodzených čísel taká, že pre každé prirodzené číslo $k\ge2$
postupnosť $(k+a_n)_{n=1}^{\infty}$ obsahuje len konečne veľa
prvočísel.

Rozhodnite, či existuje rastúca postupnosť $(a_n)_{n=1}^{\infty}$
prirodzených čísel taká, že pre každé celé číslo $k\ge0$
postupnosť $(k+a_n)_{n=1}^{\infty}$ obsahuje len konečne veľa
prvočísel.}
\podpis{R. Kollár}

{%%%%%   A-III-5
Pre každé prirodzené číslo $n\ge2$ určte najväčšiu
hodnotu výrazu
$$
V_n=\sin x_1\cos x_2+\sin x_2\cos x_3+\dots+\sin x_{n-1}\cos x_n+
\sin x_n\cos x_1,
$$
kde $x_1,x_2,\dots,x_n$  sú ľubovoľné reálne čísla.
}
\podpis{J. Švrček}

{%%%%%   A-III-6
Daný je rovnobežník $ABCD$ taký, že $ABD$ je ostrouhlý
trojuholník a~$|\uhol BAD|=45\st$. Vo vnútri strán rovnobežníka
možno rôznymi spôsobmi vybrať body $K\in AB$, $L\in BC$, $M\in CD$
a~$N\in DA$ tak, aby $KLMN$ bol tetivový štvoruholník, ktorého
opísaná kružnica má rovnaký polomer ako obidve kružnice opísané
trojuholníkom $ANK$ a~$CLM$. Nájdite množinu priesečníkov uhlopriečok
všetkých takých štvoruholníkov $KLMN$.
}
\podpis{J. Šimša}

{%%%%%   B-S-1
Ak je päťciferné číslo $6AB73$ deliteľné číslom~$99$,
tak je deliteľné aj číslom~$19$. Dokážte.}
\podpis{P. Černek}

{%%%%%   B-S-2
Rovnica $x^3+ax^2+bx+c=0$, kde $a$, $b$ a~$c$ sú celé
čísla, má koreň $x=1-\sqrt2$. Dokážte, že potom
platí $a-2b+5c=0$.}
\podpis{P. Černek}

{%%%%%   B-S-3
Daný je rovnobežník $ABCD$ s dĺžkami strán $a=|AB|$, $b=|BC|$
a~uhlom $\alpha=|\uhol DAB|$. Popíšte konštrukciu priamky, ktorá delí
rovnobežník na dva štvoruholníky, ktorým možno vpísať kružnicu.
Preveďte diskusiu o~počte riešení vzhľadom na $a$, $b$ a~$\alpha$.}
\podpis{P. Černek}

{%%%%%   B-II-1
V~obore reálnych čísel riešte sústavu rovníc
$$
\align
3x+14=&y^2+z^2,\\
3y+14=&z^2+x^2,\\
3z+14=&x^2+y^2.
\endalign
$$
}
\podpis{P. Černek}

{%%%%%   B-II-2
Určte, pre ktoré reálne čísla $p$ má funkcia
$f(x)=x^3-px^2+1\,997$ na~intervale $\langle 0,1\rangle$ minimum
v~bode $x=1$.}
\podpis{P. Černek}

{%%%%%   B-II-3
Nech $ABCD$ je lichobežník ($AB\parallel CD$), ktorého
uhlopriečky sú navzájom kolmé. Dokážte nerovnosť
$|AB|+|CD|<|BC|+|DA|$.
}
\podpis{J. Švrček}

{%%%%%   B-II-4
Učiteľ napísal na tabuľu štyri navzájom rôzne nenulové
cifry. Žiaci mali sčítať všetky tie trojciferné čísla vytvorené
z~cifier na tabuli, v~ktorých sa žiadna cifra neopakuje. Jankovi
vyšiel nesprávny výsledok $12\,497$, pretože síce čísla správne
sčítal, ale na jedno zabudol. Ktoré číslo to bolo? Aké štyri
cifry boli napísané na tabuli?}
\podpis{P. Černek}

{%%%%%   C-S-1
Os prepony~$AB$ pravouhlého trojuholníka $ABC$ pretne
odvesnu~$AC$ v~bode~$M$, pre ktorý platí $|AM|=2|CM|$. Určte
veľkosti uhlov trojuholníka $ABC$.}
\podpis{P. Leischner}

{%%%%%   C-S-2
Karol požiadal Mira: "Mysli si dve rôzne dvojciferné
čísla a~prezraď mi ich súčet." Keď sa tak stalo, Karol
správne zistil, že existujú práve štyri také dvojice (na
poradí čísel v~dvojici nezáleží). Miro ďalej Karolovi napovedal,
že väčšie z~oboch čísel je prvočíslo. Určte všetky dvojice čísel,
ktoré si Miro mohol myslieť.}
\podpis{J. Šimša}

{%%%%%   C-S-3
Nájdite všetky štvorice prirodzených čísel, pre ktoré
platí: Súčet súčinu každých dvoch čísel zo štvorice so súčinom
zostávajúcich dvoch čísel je rovný~$51$.}
\podpis{J. Zhouf}

{%%%%%   C-II-1
V~štvorcifernom čísle sú rovnaké prvé dve cifry, a~tiež
posledné dve cifry. Určte toto číslo, ak viete, že je
druhou mocninou prirodzeného čísla.}
\podpis{ČS MO}

{%%%%%   C-II-2
Daný je pravouhlý trojuholník $ABC$. Na prepone~$AB$
zostrojte bod~$X$ a~na odvesne~$BC$ bod~$Y$ tak, aby bolo možné
štvoruholníku $AXYC$ opísať aj vpísať kružnicu.}
\podpis{P. Leischner}

{%%%%%   C-II-3
Janko s~Marienkou požiadali svoju mamičku, aby si zvolila dve rôzne
dvojciferné čísla. Mamička potom Jankovi prezradila ich rozdiel
a~Marienke ich súčet. Janko správne zistil, že práve 63 takých
dvojíc dáva daný rozdiel. Aj Marienka našla správne všetkých
40~ dvojíc s~daným súčtom. Ktoré čísla mamička zvolila?}
\podpis{J. Šimša}

{%%%%%   C-II-4
Vo štvorci $ABCD$ je $R$ stred strany~$CD$ a~$Q$ priesečník
uhlopriečky~$BD$ s~priamkou~$AR$. Na strane~$BC$ zvoľte bod~$P$
tak, aby úsečka~$PQ$ rozdelila lichobežník $ABCR$ na dva
štvoruholníky s~rovnakým obsahom.}
\podpis{J. Švrček}

{%%%%%   vyberko, den 1, priklad 1
...}
\podpis{...}

{%%%%%   vyberko, den 1, priklad 2
...}
\podpis{...}

{%%%%%   vyberko, den 1, priklad 3
...}
\podpis{...}

{%%%%%   vyberko, den 1, priklad 4
...}
\podpis{...}

{%%%%%   vyberko, den 2, priklad 1
...}
\podpis{...}

{%%%%%   vyberko, den 2, priklad 2
...}
\podpis{...}

{%%%%%   vyberko, den 2, priklad 3
...}
\podpis{...}

{%%%%%   vyberko, den 2, priklad 4
...}
\podpis{...}

{%%%%%   vyberko, den 3, priklad 1
...}
\podpis{...}

{%%%%%   vyberko, den 3, priklad 2
...}
\podpis{...}

{%%%%%   vyberko, den 3, priklad 3
...}
\podpis{...}

{%%%%%   vyberko, den 3, priklad 4
...}
\podpis{...}

{%%%%%   vyberko, den 4, priklad 1
...}
\podpis{...}

{%%%%%   vyberko, den 4, priklad 2
...}
\podpis{...}

{%%%%%   vyberko, den 4, priklad 3
...}
\podpis{...}

{%%%%%   vyberko, den 4, priklad 4
...}
\podpis{...}

{%%%%%   vyberko, den 5, priklad 1
...}
\podpis{...}

{%%%%%   vyberko, den 5, priklad 2
...}
\podpis{...}

{%%%%%   vyberko, den 5, priklad 3
...}
\podpis{...}

{%%%%%   vyberko, den 5, priklad 4
...}
\podpis{...}

{%%%%%   trojstretnutie, priklad 1
...}
\podpis{...}

{%%%%%   trojstretnutie, priklad 2
...}
\podpis{...}

{%%%%%   trojstretnutie, priklad 3
...}
\podpis{...}

{%%%%%   trojstretnutie, priklad 4
...}
\podpis{...}

{%%%%%   trojstretnutie, priklad 5
...}
\podpis{...}

{%%%%%   trojstretnutie, priklad 6
...}
\podpis{...}

{%%%%%   IMO, priklad 1
...}
\podpis{...}

{%%%%%   IMO, priklad 2
...}
\podpis{...}

{%%%%%   IMO, priklad 3
...}
\podpis{...}

{%%%%%   IMO, priklad 4
...}
\podpis{...}

{%%%%%   IMO, priklad 5
...}
\podpis{...}

{%%%%%   IMO, priklad 6
...}
\podpis{...}

{%%%%%   MEMO, priklad 1
}
\podpis{}

{%%%%%   MEMO, priklad 2
}
\podpis{}

{%%%%%   MEMO, priklad 3
}
\podpis{}

{%%%%%   MEMO, priklad 4
}
\podpis{}

{%%%%%   MEMO, priklad t1
}
\podpis{}

{%%%%%   MEMO, priklad t2
}
\podpis{}

{%%%%%   MEMO, priklad t3
}
\podpis{}

{%%%%%   MEMO, priklad t4
}
\podpis{}

{%%%%%   MEMO, priklad t5
}
\podpis{}

{%%%%%   MEMO, priklad t6
}
\podpis{}

{%%%%%   MEMO, priklad t7
}
\podpis{}

{%%%%%   MEMO, priklad t8
}
\podpis{}
