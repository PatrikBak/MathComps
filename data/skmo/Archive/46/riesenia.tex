{%%%%%   A-I-1
...}

{%%%%%   A-I-2
...}

{%%%%%   A-I-3
...}

{%%%%%   A-I-4
...}

{%%%%%   A-I-5
...}

{%%%%%   A-I-6
...}

{%%%%%   B-I-1
Dokážeme, že hľadanú uzavretú lomenú čiaru možno nájsť vždy v
pravidelnom $(2n+1)$-uholníku ($n\ge 2$), a že je nemožno nájsť
v pravidelnom $2n$-uholníku ($n\ge 2$).

Uvažujme teda na začiatok $(2n+1)$-uholník
a označme jeho vrcholy po~rade $A_0, A_1, \dots, A_{2n}$.
Ďalej nech pre $k>2n$ platí $A_k\equiv A_i$, kde
$i\in \{ 0, 1, \dots, 2n\}$ je zvyšok čísla $k$ po delení číslom $2n+1$.
Uvažujme uzavretú lomenú čiaru
$A_0A_nA_{2n}\dots A_{(2n+1)n}$, spojené sú vždy vrcholy, ktorých
indexy sa líšia o~$n$. Zrejme pre $n\ge 2$ je každá
úsečka $A_{kn}A_{(k+1)n}$ uhlopriečka, rovnako je zrejmé, že
všetky body tejto lomenej čiary sú navzájom rôzne (až na prvý a~posledný).
Ak by totiž platilo $kn \equiv ln \pmod {2n+1}$,
$0\le k, l \le 2n$, tak $k\equiv l \pmod {2n+1}$,
pretože čísla $n$ a $2n+1$ sú nesúdeliteľné.

Každá z uvedených uhlopriečok rozdeľuje všetky vrcholy
$(2n+1)$-uholníka s~vý\-nim\-kou svojich koncových bodov do dvoch skupín.
Vo vnútri oboch týchto skupín sa indexy bodov líšia najviac o $n-1$, z
čoho potom vyplýva, že ľubovoľné dve uhlopriečky uvažovanej lomenej čiary sa
pretínajú (vzdialenosť susedných vrcholov v nej je vždy práve $n$).

Teraz dokážeme, že pre pravidelný $2n$-uholník ($n\ge 2$)
hľadaná lomená čiara neexistuje. Označme jeho vrcholy $A_1, A_2,
\dots, A_{2n}$.
Nech $A_1A_k$ je jedna z uhlopriečok tvoriacich hľadanú lomenú čiaru.
Nech druhá uhlopriečka  (patriaca lomenej čiare)
vychádzajúca z vrcholu $A_1$ vedie do~bodu~$A_l$
a druhá vychádzajúca z vrcholu $A_k$ nech vedie do vrcholu $A_m$.
Ak by $l<k<m$, potom by sa tieto dve uhlopriečky nepretínali (rovnako v
prípade $l>k>m$), preto zrejme musia ležať "na jednej strane", čiže buď
$m, l \in {\Cal A}= \{2,3,\dots, k-1\}$ alebo
$m, l \in {\Cal B}= \{k+1, k+2, \dots, 2n\}$.
Bez~ujmy na~vše\-obec\-nos\-ti nech $m$ a $l$ patria do $\Cal A$. Pokračujme
v lomenej čiare z~bodu~$A_l$,
zrejme druhý koniec tejto uhlopriečky musí ležať
v $\Cal B$, inak by táto uhlopriečka nepretínala uhlo\-prieč\-ku~$A_1A_k$.
Z koncového vrcholu vedieme opäť uhlopriečku do $\Cal A$ atď\.
Časom musíme skončiť vo vrchole $A_m$, a to tak, že navštívime postupne
všetky vrcholy množín~$\Cal A$ aj $\Cal B$. Keďže však sa táto časť
lomenej čiary začína aj končí v~mno\-ži\-ne~$\Cal A$ a~jej vrcholy ležia
striedavo v~mno\-ži\-nách~$\Cal A$ a~$\Cal B$ musí $\Cal A$ obsahovať
o~jeden vrchol viac ako~$\Cal B$. Preto je počet vrcholov celého
$2n$-uholníka rovný $|A|+|B|+2$ (vrcholy $A_1$ a $A_k$), čo je zrejme nepárne
číslo.
}

{%%%%%   B-I-2
...}

{%%%%%   B-I-3
...}

{%%%%%   B-I-4
Označme rovnice
$$
\align
x+y+z&=3, \tag 1\\
\frac1x+\frac1y+\frac1z&=0, \tag 2\\
\frac xy+\frac yz+\frac zx&=\frac yx+\frac zy+\frac xz. \tag 3
\endalign
$$
Uvažujme rovnicu (3). Po jej úprave na spoločného me\-no\-va\-teľa,
vynásobení a prenesení všetkých členov na jednu stranu rovnice
dostaneme
$$(x-z)xz + (z-y)zy+(y-x)yx=0,$$
čo dáva ďalej
$$(y-x)(x-z)(z-y)=0.$$

Bez ujmy na všeobecnosti nech $x=y$. Po dosadení do (1) a (2)
dostávame sústavu:
$$2x+z=3\quad (4)  \qquad \dfrac 2x + \frac 1z = 0 \quad (5)$$
Po vyjadrení $z$ z (5) a dosadení do (4) máme $x=y=2, z= - 1$.
Vzhľadom na~ekvi\-va\-lent\-nosť použitých úprav sú zrejme táto usporiadaná
trojica a~jej permutácie $[2,- 1,2]$ a $[- 1, 2, 2]$ jedinými
riešeniami sústavy.
}

{%%%%%   B-I-5
...}

{%%%%%   B-I-6
...}

{%%%%%   C-I-1
...}

{%%%%%   C-I-2
...}

{%%%%%   C-I-3
...}

{%%%%%   C-I-4
...}

{%%%%%   C-I-5
...}

{%%%%%   C-I-6
Odčítaním druhej a~tretej rovnice od prvej dostaneme nutné
podmienky  $(x-y)(1-z)=0$, $(x-z)(1-y)=0$, takže buď je $x=y=z$,
alebo $x=y=1$, alebo $x=z=1$, alebo $y=z=1$. Ak je $x=y=z$, musí
ešte platiť $x(x+1)=6$, preto $x=2$ alebo $x=-3$. Ak je $x=y=1$,
je nutne $z=5$. Riešením sú práve tieto usporiadané trojice:
$(2,2,2)$, $(-3,-3,-3)$, $(1,1,5)$, $(1,5,1)$ a~$(5,1,1)$.
}

{%%%%%   A-S-1
...}

{%%%%%   A-S-2
...}

{%%%%%   A-S-3
...}

{%%%%%   A-II-1
Zrejme $S_n=5^n+3^n+1\geq9$ pre každé prirodzené $n$. Z~tabuliek
zvyškov mocnín $5^n$ a~$3^n$ po delení číslami $3$, $5$ a~$7$
zistíme, že $3\deli S_n$ pre každé $n=2k+1$, $5\deli S_n$ pre
každé $n=4k+2$ a~$7\deli S_n$ pre každé $n=6k+2$ a~každé $n=6k+4$.
Preto ak je $S_n$ prvočíslo, nemôžu po delení čísla~$n$
dvanástimi vyjsť zvyšky 1, 3, 5, 7, 9, 11, ani zvyšky 2, 6, 10, ani
zvyšky 2, 4, 8, 10, takže toto delenie vyjde bezo zvyšku.

(Dodajme pre zaujímavosť, že číslo $S_{12}=244\,672\,067$
je prvočíslo.)
}

{%%%%%   A-II-2
...}

{%%%%%   A-II-3
...}

{%%%%%   A-II-4
...}

{%%%%%   A-III-1
...}

{%%%%%   A-III-2
Nazvime vrcholy $n$-uholníka nasledovne:

{\it párny} -- ak z neho vychádza párny počet
modrých úsečiek (teda aj párny počet červených);

{\it nepárny} -- ak z neho vychádza nepárny počet modrých úsečiek
(a nepárny počet červených).

Susedom vrcholu $n$-uholníka nazvime
každý ďalší vrchol $n$-uholníka, s ktorým je spojený úsečkou, teda
každý jeho ďalší vrchol.
Počet nepárnych vrcholov v začiatočnom ofarbení je zrejme párny,
pretože do počtu modrých úsečiek vychádzajúcich z vrcholov $n$-uholníka
je každá modrá úsečka započítaná dvakrát (keďže $n$ je nepárne, počet
párnych vrcholov je potom nepárny).

Ďalej si uvedomíme, že výsledok prefarbovania nezávisí od poradia
vrcholov, podľa ktorých prefarbujeme, ale len od toho, koľkokrát
podľa ktorého vrcholu prefarbujeme (stačí uvážiť, že vlastne záleží len
na počte prefarbení každej úsečky a nie na poradí prefarbovania).
Keďže farba každej úsečky závisí len od parity počtu zmien jej
ofarbenia, nemá zmysel v žiadnom vrchole prefarbovať úsečky viac ako raz.
Pretože $n$ je nepárne,
párny vrchol zostane aj po~jeho prefarbení párny a nepárny zostane nepárny.
Na~druhej strane, pri~prefarbovaní každého vrcholu počet modrých úsečiek
vychádzajúcich zo~všetkých jeho susedov zmení paritu, a teda všetky
ostatné párne vrcholy sa zmenia na nepárne, kým
všetky nepárne sa zmenia na párne.

Preto na dosiahnutie ofarbenia, pri ktorom budú všetky vrcholy párne,
potrebujeme prefarbiť nepárny počet susedov každého nepárneho vrcholu a
párny počet susedov každého párneho vrcholu.
Ľubovoľné dva párne resp\. nepárne vrcholy teda musia mať rovnakú
paritu počtu prefarbených susedov, teda buď prefarbíme obidva alebo ani
jeden. Preto možno dosiahnuť potrebné ofarbenie len tak, že prefarbíme
buď všetky párne vrcholy, všetky nepárne vrcholy, alebo úplne všetky
vrcholy (už sme dokázali, že párnych vrcholov je nepárny počet
a~nepárnych je párny počet). Z týchto možností vyhovujú prvé dve. Ak
napríklad prefarbíme všetky nepárne vrcholy, každý párny
vrchol má párny počet prefarbených susedov (nepárne vrcholy), čiže
zostane párny, a každý nepárny vrchol má nepárny počet prefarbených
susedov (nepárne vrcholy okrem neho), čiže sa zmení na párny.
Podobnú úvahu možno použiť aj keď prefarbíme všetky párne vrcholy.

V druhej časti zrejme stačí ukázať, že žiadne ofarbenie, v ktorom sú
všetky vrcholy párne, nemožno prefarbiť na iné takéto ofarbenie.
Na základe úvah z prvej časti vieme, že sa to dá dosiahnuť len tak, že
prefarbíme všetky párne alebo všetky nepárne vrcholy. Kým v prvom
prípade prefarbíme všetky vrcholy, a teda dostaneme pôvodné ofarbenie,
v druhom prípade neprefarbíme nič.
Preto je výsledné ofarbenie len s párnymi
vrcholmi jednoznačne určené začiatočným ofarbením.

\ineriesenie
Znovu si najprv uvedomíme, že výsledok prefarbovania nezávisí od~poradia
vrcholov, podľa ktorých prefarbujeme, ale len od toho, koľkokrát
podľa ktorého vrcholu prefarbujeme. Ak sú $A_1,A_2,\ldots,A_n$
vrcholy daného $n$-uholníka, označme $p_i$ počet prevedených
prefarbení vzhľadom na~vrchol $A_i$ a~$p=\tsum_{i=1}^np_i$ ich
celkový súčet. Úsečka $A_iA_j$ zmení
farbu, práve keď prevedieme prefarbenie vzhľadom k~jednému
z~vrcholov $A_i$ alebo $A_j$. Vo~výslednom prefarbení teda úsečka
$A_iA_j$ zmení farbu, práve keď $p_i+p_j\equiv 1\pmod 2$. Počet
modrých úsečiek vychádzajúcich z~vrcholu~$A_i$ má vo~výslednom
ofarbení rovnakú paritu ako v~počiatočnom ofarbení, práve keď
$$
\sum\limits_{j\ne i}(p_i+p_j)=(n-1)p_i+\sum\limits_{j\ne
i}p_j\equiv p-p_i\equiv 0\pmod 2.
$$
%%(využívame skutočnosť, že $n$ je nepárne).
Ukážeme, že každé počiatočné ofarbenie možno prefarbiť tak, aby
z~každého vrchola vychádzal párny počet modrých úsečiek: Budeme
postupne prefarbovať $n$-uholník
vzhľadom k~tým vrcholom, z~ktorých v~pôvodnom
ofarbení vychádzal nepárny počet modrých úsečiek (bude teda $p_i=1$,
pokiaľ v~pôvodnom ofarbení vychádzal z~vrcholu~$A_i$ nepárny počet
modrých úsečiek, $p_i=0$ v~opačnom prípade).

Označme $a_i$ počet modrých úsečiek vychádzajúcich
z~vrcholu~$A_i$, potom $\sum\limits_{i=1}^na_i$ je
rovné dvojnásobku celkového
počtu modrých úsečiek, čiže nepárnych $a_i$ je párny počet.
Pre~práve popísané prefarbovanie je $p_i=1$ pre párny počet vrcholov
$A_i$ a~$p=\sum\limits_{i=1}^np_{i}$ je párne. Pre~vrchol $A_i$,
z~ktorého v~pôvodnom ofarbení vychádzal párny počet modrých úsečiek, je
$p_i=0$, teda parita počtu modrých úsečiek z~neho vychádzajúcich sa
po~prefarbení nezmení. Pre vrchol $A_i$, z~ktorého
v~pôvodnom ofarbení vychádzal nepárny počet modrých úsečiek, však je
$p_i=1$, takže $p-p_i\equiv 1\pmod 2$, t.j.~nepárny počet modrých
úsečiek z~neho vychádzajúcich sa po~prefarbení zmení na párny, čo sme
chceli dosiahnuť.

Keby sa niektoré počiatočné ofarbenie dalo prefarbiť na dve rôzne
ofarbenia, v~ktorých by z~každého vrcholu vychádzal párny počet
modrých úsečiek, bolo by tiež možné jedno z~týchto "párnych"
ofarbení prefarbiť na druhé. Ukážeme, že to nejde. Predpokladajme
teda, že máme ofarbenie $\varPi$, v~ktorom z~každého vrcholu
vychádza párny počet modrých úsečiek. Predpokladajme ďalej, že
po~prefarbení, pri ktorom voči vrcholu~$A_i$ prefarbujeme $p_i$-krát
($i=1,2,\ldots,n$), dostaneme iné ofarbenie $\varOmega$, v~ktorom
opäť z~každého vrcholu vychádza párny počet modrých úsečiek. Je teda
$p-p_i\equiv 0\pmod 2$, pre~každé $i=1,2,\ldots,n$, a~teda
$p_i\equiv p_j\pmod 2$, alebo $p_i+p_j\equiv 0\pmod 2$ pre~každé
$i,j=1,2,\ldots,n$. Potom ale žiadna úsečka po~prefarbení
nezmenila farbu. Ofarbenia $\varPi$ a~$\varOmega$ sú teda totožné,
čiže ku~každému ofarbeniu existuje jediné, ktoré z~neho
možno dostať popísaným spôsobom, a~v~ktorom z~každého vrcholu vychádza
párny počet modrých úsečiek.
}

{%%%%%   A-III-3
...}

{%%%%%   A-III-4
Stačí zvoliť postupnosť $a_n=(n!)^3$. Pre $k\ge 2$ rovnako ako
pre $k=0$ tvrdenie o~postupnosti $(k+a_n)$ zrejme
platí, pre $k=1$ stačí využiť vzťah $x^3+1=(x+1)(x^2-x+1)$.
}

{%%%%%   A-III-5
Ak odhadneme zhora každý sčítanec súčtu $V_n$ podľa
nerovnosti $ab\le\frac12(a^2+b^2)$, ktorá zrejme platí pre
ľubovoľné reálne čísla $a$, $b$, dostaneme odhad
$$
V_n\leqq\frac{\sin^2x_1+\cos^2x_2}{2}+
        \frac{\sin^2x_2+\cos^2x_3}{2}+\dots+
        \frac{\sin^2x_n+\cos^2x_1}{2}=\frac{n}{2}.
$$

Skúmaný výraz nadobúda najväčšiu hodnotu~$\frac12 n$, lebo ako
ľahko nahliadneme, pre~hod\-no\-ty $x_1=x_2=\dots=x_n=\frac14\pi$ vyjde
$V_n=\frac12n$.
}

{%%%%%   A-III-6
...}

{%%%%%   B-S-1
Najprv nájdeme všetky čísla tvaru $6A\,B73$, ktoré sú deliteľné 99-timi.
Potom uká\-že\-me, že všetky tieto čísla sú deliteľné 19-timi
(ukáže sa, že takéto číslo je len jedno).

Číslo $6A\,B73$ môžeme zapísať ako
$$6A\,B73 = 60\,073+1\,000\cdot A+100\cdot B =
60\,073+100\cdot (10\cdot A+B).$$
Toto číslo je deliteľné 99-mi práve
vtedy, keď je deliteľné 99-mi číslo $10\cdot A + B + 79$, pretože
$$60\,073+100\cdot (10\cdot A+B) =
99\cdot (606 + (10\cdot A + B)) + 10\cdot A + B + 79 .$$
Číslo $10\cdot A + B + 79$ je prirodzené číslo z intervalu
$\langle 79,\, 178 \rangle$. Ak má byť deliteľné 99-timi, musí zrejme byť
rovné 99, čo nastáva len v prípade $A=2$ a $B=0$. Vtedy však
$6A\,B73=62\,073 = 19\cdot 3\,267$, a tým je tvrdenie v zadaní dokázané.
}

{%%%%%   B-S-2
Po dosadení čísla $x=1 - \sqrt 2$ do danej rovnice a jednoduchej úprave
dostávame
$$(5 + 2a + b) \cdot \sqrt 2 = 7 + 3a + b + c .$$
Potom musí platiť:
$$(5 + 2a + b) =  0; \qquad \qquad 7 + 3a + b + c = 0.$$
(Ak by $5+2a+b\ne 0$, tak číslo $\sqrt 2$ je rovné podielu dvoch celých
čísel, čiže je racionálne, to však nie je pravda.)
Z týchto dvoch rovníc vyjadríme $b$ a $c$ pomocou $a$:
$b=\m 5 - 2a$ a~$c=\m 2 - a$. Potom
$$a-2b+ 5c = a - 2\cdot(\m 5 - 2a) + 5\cdot(\m 2 - a) = 0 .$$
Tým je tvrdenie dokázané.
}

{%%%%%   B-S-3
...}

{%%%%%   B-II-1
Odčítaním prvej rovnice od druhej a~druhej rovnice od tretej dostaneme
po~úprave
$$
(x-y)(3+x+y) = 0, \qquad
(y-z)(3+y+z) = 0 .
$$
Z~týchto dvoch rovníc vyplýva, že musí platiť $x = y$ alebo $3+x+y =
0$, a~zároveň $y = z$ alebo $3+y+z = 0$. Prebratím jednotlivých
možností zistíme, že môžu nastať len nasledujúce dva prípady:
buď sa všetky tri čísla $x$, $y$ a~$z$ navzájom rovnajú, alebo
sú dve z~nich rovnaké (bez ujmy na všeobecnosti napr.~$x=y$); tretie
číslo potom vypočítame zo vzťahu $z={-y}-3$. Ostatné riešenia dostaneme
permutáciou takto nájdenej trojice $(x,y,z)$.

$\bullet$~
Ak $x = y = z$, tak je daná sústava ekvivalentná s~(jedinou)
kvadratickou rovnicou $2x^2-3x-14=0$ s~koreňmi $x_1=\frac 72$
a~$x_2={-2}$. Odtiaľ vychádzajú dve riešenia: $x = y = z~= \frac 72$
a~$x = y = z~= {-2} $.

$\bullet$~
Ak $x = y$ a~$z = {-y}- 3$, tak je daná sústava ekvivalentná s~(jedinou)
kvadratickou rovnicou $2y^2+3y-5=0$ s~koreňmi $y_1={-\frac52}$ a~$y_2=1$.
Odtiaľ vychádzajú dve riešenia
$x = y = {-\frac 52}$, $z = {-\frac 12}$
a~$x = y = 1$, $z = {-4} $ a~permutáciou trojice $(x,y,z)$
dostaneme ďalšie štyri riešenia
$x = z~= {-\frac 52}$, $y = {-\frac 12}$;
$x = z~= 1$, $y = {-4}$;
$y = z~= {-\frac 52}$, $x = {-\frac 12}$;
$y = z~= 1$, $x = {-4}$.

Všetkých osem nájdených riešení spĺňa dané rovnice (skúška nie je
potrebná, pretože z~uvedeného postupu vidno, že nájdené trojice
$x$, $y$ a~$z$ danej sústave skutočne vyhovujú).
}

{%%%%%   B-II-2
...}

{%%%%%   B-II-3
...}

{%%%%%   B-II-4
...}

{%%%%%   C-S-1
...}

{%%%%%   C-S-2
...}

{%%%%%   C-S-3
...}

{%%%%%   C-II-1
Číslo  $1\,000a+ 100a+10b +b= 11(100a+b)$ má
byť druhou mocninou, preto  musí byť číslo  $100a+b$
deliteľné číslom~ 11  a~podiel $\frac1{11}(100a+b) = 9a+
\frac1{11}(a+b)$ musí byť druhou mocninou prirodzeného čísla.
Vzhľadom na~to, že $a$ a~$b$ ($a\ne0$) sú číslice, musí
byť $a+b = 11$, a~pretože $9a + 1$ má byť druhou mocninou, vyjde
$a = 7$. Hľadané číslo je $7\,744 = 88^{2}$.
}

{%%%%%   C-II-2
...}

{%%%%%   C-II-3
...}

{%%%%%   C-II-4
...}

{%%%%%   vyberko, den 1, priklad 1
...}

{%%%%%   vyberko, den 1, priklad 2
...}

{%%%%%   vyberko, den 1, priklad 3
...}

{%%%%%   vyberko, den 1, priklad 4
...}

{%%%%%   vyberko, den 2, priklad 1
...}

{%%%%%   vyberko, den 2, priklad 2
...}

{%%%%%   vyberko, den 2, priklad 3
...}

{%%%%%   vyberko, den 2, priklad 4
...}

{%%%%%   vyberko, den 3, priklad 1
...}

{%%%%%   vyberko, den 3, priklad 2
...}

{%%%%%   vyberko, den 3, priklad 3
...}

{%%%%%   vyberko, den 3, priklad 4
...}

{%%%%%   vyberko, den 4, priklad 1
...}

{%%%%%   vyberko, den 4, priklad 2
...}

{%%%%%   vyberko, den 4, priklad 3
...}

{%%%%%   vyberko, den 4, priklad 4
...}

{%%%%%   vyberko, den 5, priklad 1
...}

{%%%%%   vyberko, den 5, priklad 2
...}

{%%%%%   vyberko, den 5, priklad 3
...}

{%%%%%   vyberko, den 5, priklad 4
...}

{%%%%%   trojstretnutie, priklad 1
...}

{%%%%%   trojstretnutie, priklad 2
...}

{%%%%%   trojstretnutie, priklad 3
...}

{%%%%%   trojstretnutie, priklad 4
...}

{%%%%%   trojstretnutie, priklad 5
...}

{%%%%%   trojstretnutie, priklad 6
...}

{%%%%%   IMO, priklad 1
...}

{%%%%%   IMO, priklad 2
...}

{%%%%%   IMO, priklad 3
...}

{%%%%%   IMO, priklad 4
...}

{%%%%%   IMO, priklad 5
...}

{%%%%%   IMO, priklad 6
...}

{%%%%%   MEMO, priklad 1
...}

{%%%%%   MEMO, priklad 2
...}

{%%%%%   MEMO, priklad 3
...}

{%%%%%   MEMO, priklad 4
...}

{%%%%%   MEMO, priklad t1
...}

{%%%%%   MEMO, priklad t2
...}

{%%%%%   MEMO, priklad t3
...}

{%%%%%   MEMO, priklad t4
...}

{%%%%%   MEMO, priklad t5
...}

{%%%%%   MEMO, priklad t6
...}

{%%%%%   MEMO, priklad t7
...}

{%%%%%   MEMO, priklad t8
...} 