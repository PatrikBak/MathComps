{%%%%%   A-I-1
Nech $P(x)$, $Q(x)$ sú kvadratické mnohočleny také, že čísla $\m22$, $7$, $13$
sú tri z~koreňov rovnice $P\left(Q(x)\right)=0$.
Určte štvrtý koreň tejto rovnice.}
\podpis{P. Černek}

{%%%%%   A-I-2
Nech $K$, $L$, $M$ sú po rade vnútorné body strán $BC$, $C\!A$,
$AB$ daného trojuholníka $ABC$ také, že kružnice vpísané
dvojiciam trojuholníkov  $ABK$ a~$C\!AK$, $BC\!L$ a~$ABL$, $C\!AM$
a~$BC\!M$ majú vonkajší dotyk. Potom platí
$$
|BK|\cdot|CL|\cdot|AM|=|CK|\cdot|AL|\cdot|BM|.
$$
Dokážte.

\noindent%
{\it Poznámka}. Z~uvedenej rovnosti vyplýva na základe Cevovou
vety, že priamky $AK$, $BL$, $CM$ prechádzajú spoločným bodom.}
\podpis{J. Švrček}

{%%%%%   A-I-3
V~obore kladných reálnych čísel riešte sústavu
$$
\eqalign{\sqrt{xy}+\sqrt{xz}-x&=a,\cr
         \sqrt{yz}+\sqrt{yx}-y&=b,\cr
         \sqrt{zx}+\sqrt{zy}-z&=c,\cr}
$$
kde $a$, $b$, $c$ sú dané kladné čísla.}
\podpis{R. Horenský}

{%%%%%   A-I-4
V~rovine je daných 1\,999 zhodných trojuholníkov s~obsahom~$1$, ktoré
sú obrazmi jedného trojuholníka v~rôznych posunutiach.
Ak je prienikom všetkých daných trojuholníkov množina~$\Cal M$,
ktorá obsahuje ťažisko každého z~nich, je
obsah množiny~$\Cal M$ aspoň~$\frac19$. Dokážte.}
\podpis{M. Beneš}

{%%%%%   A-I-5
Daná je funkcia $f:\Bbb N\to\Bbb N$ taká, že $f(n)=1$, ak je $n$
nepárne, a~$f(n)=k$ pre každé párne číslo $n=2^k l$, kde $k$ je
prirodzené číslo a~$l$ číslo nepárne.
Určte najväčšie prirodzené číslo~$n$, pre ktoré platí
$$
f(1)+f(2)+\cdots+f(n)\leq 123\,456.
$$}
\podpis{S. Trávníček}

{%%%%%   A-I-6
Daný je štvorboký ihlan $ABC\!DV$ s~podstavou $ABC\!D$. Jeho hrany
$AB$, $C\!D$ sú rovnobežné a~roviny $ABV$ a~$C\!DV$
navzájom kolmé. Označme $P$
pätu výšky z~vrcholu~$V$ na  stranu~$AB$ v~trojuholníku $ABV$
a~$Q$ pätu výšky z~vrcholu~$V$ na stranu~$C\!D$
v~trojuholníku $C\!DV$. Dokážte nerovnosť
$$
|AV|^2+|BV|^2+|CV|^2+|DV|^2\geq
|PQ|^2+2(S_{ABV}+S_{C\!DV}+S_{PQV}),
$$
kde $S_{XY\!Z}$ označuje obsah trojuholníka $XY\!Z$. Zistite tiež,
kedy platí rovnosť.}
\podpis{J. Bábeľa}

{%%%%%   B-I-1
Pre ktoré reálne čísla~$t$  má  funkcia
$f(x)=5x+44+t\cdot|x-2|- 3\cdot|x-t|$
maximum rovné~0?}
\podpis{P. Černek}

{%%%%%   B-I-2
Označme $S$ stred kružnice vpísanej ľubovoľnému trojuholníku
$ABC$. Dokážte, že rovnosť $|AS|\cdot|BS|=|CS|\cdot|AB|$ platí
práve vtedy, keď je uhol $AC\!B$ pravý.}
\podpis{J. Švrček}

{%%%%%   B-I-3
Určte reálne čísla $a$, $b$, pre ktoré má sústava
$$
\align
x^{2}+y^{2}+2z^{2} &= 16,\cr
xyz^{2} + xy + z^{2} &= a,\cr
x + y + 2z &= b
\endalign
$$
v~obore reálnych čísel práve jedno riešenie.}
\podpis{J. Bábeľa}

{%%%%%   B-I-4
Sú dané kružnice $k$ a~$l$ s~rôznymi polomermi, ktoré sa
dotýkajú zvonku v~bode~$T$. Priesečníkom~$M$ dvoch ich spoločných
dotyčníc veďme sečnicu~$s$ oboch kružníc. Označme $X$ ten
z~oboch priesečníkov kružnice~$k$ so sečnicou~$s$, ktorý je vzdialenejší
od bodu~$M$. Podobne označme $Y$ ten z~oboch priesečníkov kružnice~$l$
so sečnicou~$s$, ktorý je vzdialenejší od bodu~$M$. Nech $P$ je
taký bod, že $XTY\!P$ je rovnobežník. Určte množinu bodov~$P$
zodpovedajúcich všetkým takým sečniciam~$s$.}
\podpis{J. Zhouf}

{%%%%%   B-I-5
Deväťsten $ABC\!DEFGHV$ vznikol zlepením kocky $ABC\!DEFGH$
a~pravidelného štvorbokého ihlana $EFGHV$. Na každú stenu tohto
deväťstena sme napísali číslo. Štyri z~napísaných čísel sú $25$,
$32$, $50$ a~$57$. Pre každý vrchol deväťstena $ABC\!DEFGHV$ sčítame čísla
na všetkých stenách, ktoré ho obsahujú. Dostaneme tak deväť rovnakých
súčtov. Určte zvyšných päť čísel napísaných na stenách tohto
telesa.}
\podpis{K. Černeková}

{%%%%%   B-I-6
Daný je rovnostranný trojuholník $XY\!Z$ s~ťažiskom~$T$ a~stranou
dĺžky $5\cm$. Zostrojte rovnobežník $ABC\!D$ s~obsahom $8\cm^2$
a~stranou~$AB$ dĺžky $2\cm$ tak, aby body $X$, $Y$, $Z$, $T$ ležali
postupne na priamkach $AB$, $BC$, $C\!D$, $D\!A$.}
\podpis{M. Krállová}

{%%%%%   C-I-1
Pri delení istého prirodzeného čísla číslami $19$ a~$99$
dostaneme ako zvyšky dve prvočísla. Súčet oboch neúplných podielov
sa rovná $1\,999$. Určte delené číslo.}
\podpis{J. Šimša}

{%%%%%   C-I-2
Nájdite všetky pravouhlé trojuholníky, v~ktorých
spojnica stredov vpísanej a~opísanej kružnice zviera s~preponou uhol
$45\st$.}
\podpis{M. Krállová}

{%%%%%   C-I-3
Zistite najmenšie prirodzené číslo~$k$, pre ktoré
platia jednotlivé tvrdenia a), b) a~c):
Ak obsadíme figúrkami ľubovoľných $k$~polí šachovnice $8\times8$,
budú obsadené niektoré
\itemitem{a)} tri susedné polia niektorého riadku,
\itemitem{b)} tri susedné polia niektorého šikmého radu,
\itemitem{c)} štyri susedné polia niektorého riadku alebo stĺpca.

(Šikmým radom rozumieme takú skupinu polí, ktorých
uhlopriečky jedného z~oboch smerov ležia na jednej a~tej istej priamke.)}
\podpis{J. Šimša}

{%%%%%   C-I-4
Juro zhotovil papierový model pravidelného štvorbokého
ihlana $ABC\!DV$ s~podstavou $ABC\!D$. Keď potom model
rozrezal pozdĺž štyroch hrán, bolo ho možné rozvinúť
(bez prekrytia) do roviny.
Koľko rôznych sietí daného ihlana tak mohol Juro dostať? Ukázalo
sa, že sieť, ktorú Juro dostal, mala tvar (nekonvexného)
sedemuholníka. Vypočítajte uhol $AV\!B$ v~bočnej stene ihlana.}
\podpis{P. Leischner}

{%%%%%   C-I-5
V~číselnom výraze
$$
+1+2+3-4-5-6+7+8+9-10-11-12+\cdots+595+596+597-598-599-600),
$$
v ktorom chýba ľavá zátvorka, sú postupne vypísané
všetky prirodzené čísla od $1$ do $600$; pred nimi sa pravidelne
opakujú tri znamienka $\p$ a~tri znamienka $\m$. Doplňte ľavú
zátvorku do výrazu tak, aby vyšiel výsledok~$378$.}
\podpis{P. Černek}

{%%%%%   C-I-6
Daný je pravidelný šesťuholník $KLMNOP$. Zostrojte
pravouhlý trojuholník $ABC$ s~preponou~$AB$ tak, aby jeho vrchol~$C$
ležal na úsečke~$N\!P$, body $M$, $O$, $K$ ležali postupne na
priamkach $AB$, $BC$, $C\!A$ a~aby priamka~$N\!P$ rozdelila
trojuholník $ABC$ na dve časti s~rovnakým obsahom.}
\podpis{K.~Černeková}

{%%%%%   A-S-1
Určte, pre ktoré reálne čísla~$p$ má sústava rovníc
$$
\align
(x-y)^2=&p^2,\\
x^3-y^3=&16
\endalign
$$
práve jedno riešenie v~obore reálnych čísel.}
\podpis{J. Bábeľa}

{%%%%%   A-S-2
Je daný trojuholník $ABC$. Vnútri jeho strán $BC$, $CA$, $AB$
uvažujme postupne body $K$, $L$, $M$ také, že úsečky $AK$, $BL$, $CM$ sa
pretínajú v~bode~$U$. Ak trojuholníky $AMU$ a~$KCU$ majú obsah~$P$
a~trojuholníky $MBU$ a~$CLU$ obsah~$Q$, potom $P=Q$. Dokážte.}
\podpis{J. Švrček}

{%%%%%   A-S-3
Určte najmenšie prirodzené číslo~$k$, pre ktoré platí: Ak vyberieme
ľubovoľných $k$~rôznych čísel z~množiny $\{1,2,3,\dots,2\,000\}$,
tak medzi vybranými číslami existujú dve, ktorých súčet alebo rozdiel
je~$667$.}
\podpis{J. Šimša}

{%%%%%   A-II-1
Nech $P(x)$ je kvadratický trojčlen. Určte všetky korene rovnice
$$
P(x^2+4x-7)=0,
$$
ak viete, že medzi nimi je číslo~$1$ a~aspoň jeden
koreň je dvojnásobný.}
\podpis{P. Černek}

{%%%%%   A-II-2
Daný je rovnoramenný lichobežník $UV\!ST$, v~ktorom $3|ST|<2|UV|$.
Zostrojte rovnoramenný trojuholník $ABC$ so základňou~$AB$ tak, aby
body $B$, $C$ ležali na priamke~$VS$, bod~$U$ na priamke~$AB$
a~bod~$T$ bol ťažiskom trojuholníka $ABC$.}
\podpis{P. Černek}

{%%%%%   A-II-3
Dokážte, že pre ľubovoľné kladné čísla $a$, $b$ platí nerovnosť
$$
\root3\of{a\over b}+
\root3\of{b\over a}\le
\root3\of{2(a+b)\Bigl(\frac1a+\frac1b\Bigr)}.
$$
Zistite, kedy nastane rovnosť.}
\podpis{J. Šimša}

{%%%%%   A-II-4
Určte všetky konvexné štvoruholníky $ABC\!D$ s~nasledujúcou
vlastnosťou: Vnútri štvoruholníka $ABC\!D$ existuje bod~$E$ taký,
že každá priamka, ktorá prechádza týmto bodom a~pretína strany $AB$
a~$C\!D$ vo vnútorných bodoch, delí štvoruholník $ABC\!D$ na dve časti
s~rovnakým obsahom. Svoju odpoveď zdôvodnite.}
\podpis{P. Černek, J. Švrček}

{%%%%%   A-III-1
Nech $n$ je prirodzené číslo. Dokážte, že súčet
$
4\cdot3^{2^n}+3\cdot4^{2^n}
$
je deliteľný trinástimi práve vtedy, keď $n$ je párne.}
\podpis{J. Šimša}

{%%%%%   A-III-2
Daný je rovnoramenný trojuholník $ABC$ so základňou~$AB$. Na jeho výške~$CD$
je zvolený bod~$P$ tak, že kružnice vpísané trojuholníku $ABP$
a~štvoruholníku $PECF$ sú zhodné; pritom bod~$E$ je
priesečník priamky~$AP$ so stranou~$BC$ a~$F$ priesečník priamky~$BP$
so stranou~$AC$. Dokážte, že aj kružnice vpísané trojuholníkom $ADP$
a~$BCP$ sú zhodné.}
\podpis{J. Šimša, K. Horák}

{%%%%%   A-III-3
V~rovine je daných 2\,000 zhodných trojuholníkov s~obsahom~$1$, ktoré
sú obrazmi toho istého trojuholníka v~rôznych posunutiach. Každý
z~týchto trojuholníkov obsahuje ťažiská všetkých zostávajúcich. Dokážte, že
obsah zjednotenia týchto trojuholníkov je menší ako $\frac{22}9$.}
\podpis{P. Calábek}

{%%%%%   A-III-4
Pre ktoré kvadratické funkcie $f(x)$ existuje taká kvadratická
funkcia $g(x)$, že korene rovnice $g\left(f(x)\right)=0$ sú štyri
rôzne po sebe idúce členy aritmetickej postupnosti a~súčasne sú
aj~koreňmi rovnice $f(x)g(x)=0$?}
\podpis{P. Černek}

{%%%%%   A-III-5
Monika zhotovila papierový model trojbokého ihlana, ktorého
podstavou bol pravouhlý trojuholník. Keď model rozrezala pozdĺž odvesien
podstavy a~pozdĺž ťažnice jednej zo stien, vznikol po rozvinutí do
roviny štvorec so~stranou~$a$.
% Určte pomer obsahu podstavy a povrchu ihlana.
Určte objem tohto ihlana.}
\podpis{P. Leischner}

{%%%%%   A-III-6
Nájdite všetky štvormiestne čísla $\overline{abcd}$ (v~desiatkovej
sústave), pre ktoré platí rovnosť
$$
\overline{abcd}+1=(\overline{ac}+1)(\overline{bd}+1).
$$}
\podpis{J. Zhouf}

{%%%%%   B-S-1
Pre ktoré reálne čísla $a$, $b$ je funkcia
$$
f(x)=a|x-1|+b(x-3)+|x-b|+x-1
$$
ohraničená?}
\podpis{J. Bábeľa}

{%%%%%   B-S-2
Daná je úsečka $XZ$ dĺžky $7\cm$ a~jej body $S$, $Y$ tak, že
$|XS|=2\cm$, $|YZ|=1\cm$. Zostrojte pravouhlý trojuholník $ABC$
s~preponu~$AB$ tak, aby bod~$S$ bol stredom kružnice vpísanej trojuholníku
$ABC$ a~body $X$, $Y$, $Z$ ležali postupne na priamkach $AC$, $AB$,
$BC$.}
\podpis{P. Černek}

{%%%%%   B-S-3
Do výrazu
$$
1-2+3-4+5-6+\dots+99-100
$$
sme vpísali niekoľko zátvoriek tak, že nakoniec sú v~každej dvojici
zodpovedajúcich si zátvoriek práve tri čísla a~výraz neobsahuje žiadny
súčin. Koľko rôznych výsledkov môžeme takto dostať?}
\podpis{P. Černek}

{%%%%%   B-II-1
Nájdite všetky reálne čísla $c$, pre ktoré má rovnica
$$
(c^2+c-8)(x+2)-8|x-c+2|=c|x+c+14|
$$
nekonečne veľa riešení v~obore celých čísel.}
\podpis{J. Šimša}

{%%%%%   B-II-2
Deväťsten vznikol zlepením kocky a~pravidelného
štvorbokého ihlana. Na každej stene tohto deväťstena je napísané
jedno číslo. Ich súčet je $3\,003$. Pre každú stenu~$S$
uvažovaného deväťstena sčítame čísla na všetkých stenách,
s~ktorými má $S$ spoločnú práve jednu hranu. Dostaneme tak deväť
rovnakých súčtov. Určte všetky čísla napísané na stenách
deväťstena.}
\podpis{K. Černeková}

{%%%%%   B-II-3
Daný je lichobežník $ABCD$, v~ktorom $|AB|=8\cm$ a~$|\uhol
ABC|=90\st$. Jeho obvod je $28\cm$. Polkružnica~$k$ s~priemerom~$AB$
sa dotýka strany~$CD$. Vypočítajte dĺžky zvyšných strán
daného lichobežníka, ak strana~$AB$ je jeho
\ite a) základňou,
\ite b) ramenom.}
\podpis{Smutná}

{%%%%%   B-II-4
Daný je obdĺžnik $KLMN$, $|KN|>|KL|$. Zostrojte rovnoramenný trojuholník
$ABC$ so základňou~$AB$ dĺžky $|KL|$ tak, aby jeho výška~$v_a$
obsahovala body $K$, $N$, výška~$v_b$ bod~$L$ a~výška~$v_c$ bod~$M$.
(Výškami tu rozumieme priamky.)}
\podpis{K. Černeková}

{%%%%%   C-S-1
Nájdite najmenšie prirodzené číslo~$k$, pre ktoré platí:
Ak vyberieme ľubovoľných $k$~rôznych čísel z~množiny
$\{1, 4, 7, 10, 13,\dots, 1\,999\}$, potom medzi vybranými existujú
dve rôzne čísla, ktorých súčet sa rovná $2\,000$.}
\podpis{J. Zhouf}

{%%%%%   C-S-2
Štvorec $ABC\!D$ a~obdĺžnik $AEFD$ majú takú vzájomnú
polohu, že bod~$B$ leží na kružnici vpísanej trojuholníku $AEF$.
Vypočítajte pomer dĺžky a~šírky obdĺžnika $AEFD$.}
\podpis{J. Šimša}

{%%%%%   C-S-3
Ak celé kladné číslo~$N$ vydelíme
číslom~$19$ a~získaný neúplný podiel ďalej vydelíme
číslom~$99$, vyjde nám pri druhom delení rovnaký neúplný
podiel a~rovnaký zvyšok, ako keď pôvodné číslo~$N$ vydelíme
číslom $1\,999$. Určte ako najmenšie, tak aj najväčšie také číslo~$N$.}
\podpis{J. Šimša}

{%%%%%   C-II-1
Z~dreva je vyrobených šesť zhodných pravidelných štvorbokých
ihlanov a~kocka. Stena kocky je zhodná s~podstavami ihlanov.
Určte pomer povrchu kocky a~telesa, ktoré vznikne zlepením
podstáv ihlanov so stenami kocky, ak je pomer objemov týchto
telies $1:2$.}
\podpis{P. Leischner}

{%%%%%   C-II-2
Milan zapísal za seba niekoľko prvých prirodzených čísel,
vynechal pri tom len čísla $4, 9, 14, 19, 24, 29, \dots$ Potom medzi
zapísané čísla vpísal striedavo znaky mínus a~plus, takže dostal
výraz
$$
1-2+3-5+6-7+8-10+11-12+13-15+\cdots
$$
Nakoniec ešte vpísal ľavú zátvorku za každý znak mínus a~rovnaký
počet pravých zátvoriek zapísal až na koniec výrazu:
$$
1-(2+3-(5+6-(7+8-(10+11-(12+13-(15+\cdots))))))
$$
Výsledný výraz mal hodnotu $103$. Koľko čísel bolo v~Milanovom výraze?
(Zistite všetky možnosti.)}
\podpis{P. Černek}

{%%%%%   C-II-3
Aký najväčší počet figúrok je možné rozostaviť na jednotlivé polia
hracej dosky z~obrázka tak, aby v~žiadnom šikmom rade neboli
figúrkami obsadené žiadne tri susedné polia? Nezabudnite
zdôvodniť, prečo väčší počet figúrok takto rozostaviť nemožno.
(Šikmým radom rozumieme takú skupinu polí, ktorých uhlopriečky
jedného z~oboch smerov ležia na jednej priamke.)
\insp{49-c-ii-3}
}
\podpis{J. Bábeľa}

{%%%%%   C-II-4
V~rovine sú dané body $A$, $L$, $M$ také, že $|AL|=6{,}3\cm$,
$|AM|=5{,}6\cm$, $|LM|=1{,}8\cm$. Zostrojte lichobežník $ABCD$,
ktorému sa dá vpísať kružnica, ktorá sa dotýka ramena~$BC$ v~bode~$L$
a~základne~$CD$ v~bode~$M$ (body dotyku so základňou~$AB$ a~ramenom~$AD$
lichobežníka $ABCD$ nie sú dané).}
\podpis{J. Šimša}

{%%%%%   vyberko, den 1, priklad 1
Nech $a$, $b$, $c$ sú tri prirodzené čísla s~vlastnosťami, pričom
$a^3$ je deliteľné číslom~$b$, $b^3$ je deliteľné číslom~$c$ a~$c^3$ je deliteľné číslom~$a$.
Ukážte, že $(a+b+c)^{13}$ je deliteľné číslom $abc$.}
\podpis{Ján Bábeľa, Martin Hriňák, Ján Špakula:???}

{%%%%%   vyberko, den 1, priklad 2
Dokážte, že existuje mnohočlen $p(x)$ s celočíselnými
koeficientmi taký, že pre každé $x\in\langle\frac1{10}, \frac9{10}\rangle$
platí
$$
\left|p(x) - \frac12\right| < \frac1{1000}.
$$}
\podpis{Ján Bábeľa, Martin Hriňák, Ján Špakula:???}

{%%%%%   vyberko, den 1, priklad 3
Nech $ABCDEF$ je konvexný šesťuholník taký, že
$|AB|=|BC|$,
$|CD|=|DE|$,
$|EF|=|FA|$. Dokážte, že (predĺžené) výšky trojuholníkov
$BCD$, $DEF$, $FAB$ postupne z~vrcholov $C$, $E$,
$A$ sa pretínajú v~jednom bode.}
\podpis{Ján Bábeľa, Martin Hriňák, Ján Špakula:???}

{%%%%%   vyberko, den 2, priklad 1
Majme dané postupnosti $\{a_n\}_{n=1}^\infty$,
$\{b_n\}_{n=1}^\infty$, $\{c_n\}_{n=1}^\infty$, $\{d_n\}_{n=1}^\infty$,
pre ktoré platia nasledujúce vzťahy:
$$
\begin{matrix}
a_{n+1}=a_n+b_n,\qquad& b_{n+1}=b_n+c_n,\\
c_{n+1}=c_n+d_n,\qquad& d_{n+1}=d_n+a_n.
\end{matrix}
$$
Dokážte, že ak existujú $k\geq 1$, $m\geq 1$ také, že platí
$$
\begin{matrix}
a_{k+m}=a_m, \qquad b_{k+m}=b_m,\\
c_{k+m}=c_m, \qquad d_{k+m}=d_m,
\end{matrix}
$$
tak potom $a_2=b_2=c_2=d_2=0$.}
\podpis{Ján Bábeľa, Martin Hriňák, Ján Špakula:???}

{%%%%%   vyberko, den 2, priklad 2
Nech $P$, $Q$, $R$ sú postupne stredy kružnicových oblúkov $BC$, $CA$,
$AB$ kružnice opísanej danému trojuholníku $ABC$. $K$, $L$, $M$ nech ďalej
označujú postupne stredy jeho strán $BC$, $CA$, $AB$ a~$I$ stred kružnice
tomuto trojuholníku vpísanej. Dokážte, že platí
$$|AI|\cdot|BI|\cdot|CI|=8\cdot|KP|\cdot|LQ|\cdot|MR|.$$}
\podpis{Ján Bábeľa, Martin Hriňák, Ján Špakula:???}

{%%%%%   vyberko, den 2, priklad 3
Pre $n$ prirodzené riešte v~obore reálnych čísel rovnicu
$$x_1^2+x_2^2+\dots +x_n^2-x_{n+1}=\sqrt{x_1+x_2+\dots
+x_n-x_{n+1}}-{{n+1}\over 4}.$$}
\podpis{Ján Bábeľa, Martin Hriňák, Ján Špakula:???}

{%%%%%   vyberko, den 3, priklad 1
Je daných $n$ reálnych čísel
$$
  1 \ge x_1 \ge x_2 \ge \dots x_n > 0
$$
a~reálne číslo $a\in\langle0,1\rangle$. Dokážte nerovnosť
$$
  (1 + x_1 + x_2 + \dots + x_n)^a \le
  1 + x_1^a + \frac{1}{2}(2x_2)^a + \frac{1}{3}(3x_3)^a + \dots
  \frac{1}{n}(nx_n)^a.
$$}
\podpis{Juraj Földes:???}

{%%%%%   vyberko, den 3, priklad 2
Nech $ABCD$ je štvoruholník vpísaný do kružnice so stredom~$O$. Nech
$P$ je priesečník uhlopriečok $AC$ a~$BD$. Stredy opísaných kružníc
trojuholníkom $ABP, BCP, CDP$ a~$DAP$ sú postupne $O_1$, $O_2$, $O_3$ a~$O_4$.
Dokážte, že priamky $OP$, $O_1O_3$ a~$O_2O_4$ sa pretínajú v~jednom bode.}
\podpis{Juraj Földes:???}

{%%%%%   vyberko, den 3, priklad 3
Majme číslo
$$
  N = 0 \, 2 \,  5 \,  8 \, 6 \, 5  \, 4  \, 1  \, 3  \, 9  \, 8  \,
      9  \, 7  \, 3  \,  2.
$$
Keď sa pozeráme zľava, tak sa nám číslo~$N$ rozpadne na šesť rastúcich a~klesajúcich
reťazcov cifier
$$
  0258,\, 86541, \, 139, \, 99, \, 89, \, 9732.
$$
Uvažujme len maximálne reťazce, \tj. reťazce idúce od jednej zmeny smeru (z~rastúcej
na klesajúcu alebo naopak) k~druhej. Pripusťme aby naše číslo začínalo~0,
ale aby malo každé dve susedné cifry rôzne. Aký je priemerný počet
maximálnych reťazcov v~týchto $n$-ciferných číslach?}
\podpis{Juraj Földes:???}

{%%%%%   vyberko, den 3, priklad 4
Je daných $n$ celočíselných aritmetických postupností $(a_i)_{i \in {\Bbb
Z}}$. Dokážte, že ak každé dve postupnosti majú spoločný člen, tak potom
majú všetky postupnosti spoločný člen. Dokážte, že ak môžu mať postupnosti
reálne hodnoty, tak tvrdenie nemusí platiť.}
\podpis{Juraj Földes:???}

{%%%%%   vyberko, den 4, priklad 1
Vrcholy $A$, $B$, $C$ ostrouhlého trojuholníka $ABC$ ležia postupne na
stranách  $B_1C_1$, $C_1A_1$, $A_1B_1$ trojuholníka
$A_1B_1C_1$ tak, že platí
$|\uhol ABC| = |\uhol A_1B_1C_1|$,
$|\uhol BCA| = |\uhol B_1C_1A_1|$ a~$|\uhol CAB| = |\uhol C_1A_1B_1|$.
Dokážte, že ortocentrá (priesečníky výšok) trojuholníkov $ABC$ a~$A_1B_1C_1$
sú rovnako vzdialené od stredu kružnice opísanej trojuholníku $ABC$.}
\podpis{Mgr. Richard Kollár:Bulharsko 1999, 3.kolo úloha 5}

{%%%%%   vyberko, den 4, priklad 2
V~pravouhlom súradnicovom systéme je každému mrežovému bodu (bod s~celočíselnými súradnicami)
priradené reálne číslo tak, že žiadnym dvom mrežovým bodom nie je priradené to
isté číslo. Nech $A$ je ľubovoľná neprázdna konečná množina mrežových
bodov v~tomto systéme, ktorá je stredovo symetrická
vzhľadom na počiatok~$O$ súradnicového systému, $O\notin A$.
Dokážte, že potom existuje mrežový bod~$X$ roviny taký, že ak
$A_X$ je obraz množiny~$X$ v~posunutí o~vektor $\overrightarrow{OX}$, tak
aspoň
polovica čísel priradených bodom množiny~$A_X$ je väčšia ako číslo priradené
bodu~$X$.}
\podpis{Mgr. Richard Kollár:???}

{%%%%%   vyberko, den 4, priklad 3
Označme $d(n)$ počet kladných celočíselných deliteľov prirodzeného čísla~$n$.
Nájdite všetky prirodzené čísla, pre ktoré platí
$n = (d(n))^2$.}
\podpis{Mgr. Richard Kollár:???}

{%%%%%   vyberko, den 4, priklad 4
Všetky tri vrcholy trojuholníka majú obe súradnice celočíselné, dĺžka jednej zo
 strán je $\sqrt{n}$, kde $n$ nie je deliteľné žiadnou druhou mocninou
 prvočísla.
Dokážte, že pomer polomeru vpísanej a~opísanej kružnice tomuto trojuholníku je
iracionálne číslo.}
\podpis{Mgr. Richard Kollár:???}

{%%%%%   vyberko, den 5, priklad 1
Nájdite všetky funkcie
$ h : \Bbb Z \to \Bbb Z $
také, že pre všetky celé čísla $x$, $y$ platí
$$
  h(x+y) + h(xy) = h(x) h(y) + 1 .
$$}
\podpis{Eugen Kováč:???}

{%%%%%   vyberko, den 5, priklad 2
Karty očíslované číslami $1,2,\dots,32$ sú náhodne zoradené vedľa seba.
V~jednom ťahu môžeme vybrať nejaký blok po sebe idúcich kariet, ktorých
čísla sú zoradené vzostupne alebo zostupne, a~položiť ho v~opačnom poradí.
Napríklad
$\dots 11 \underline{4 5 10} 26 8 \dots$
môže byť zmenené na
$\dots 11 \underline{10 5 4} 26 8 \dots$
Dokážte, že po najviac 58~ťahoch môžeme zoradiť karty tak, že ich čísla
budú zoradené vzostupne alebo zostupne.}
\podpis{Eugen Kováč:???}

{%%%%%   vyberko, den 5, priklad 3
Nech $ABCD$ je tetivový štvoruholník a~kladné reálne číslo~$k$.
Nech $E$ a~$F$ sú body
postupne na stranách $AB$ a~$CD$ také, že
$|AE| : |EB| = |CF| : |FD| = k$.
Nech $P$ je taký bod na úsečke~$EF$, že
$|PE| : |PF| = |AB| : |CD|$.
Dokážte, že pomer obsahov trojuholníkov $APD$ a~$BPC$ nezávisí od čísla~$k$.}
\podpis{Eugen Kováč:???}

{%%%%%   vyberko, den 5, priklad 4
Nájdite všetky dvojice $(a,b)$ reálnych čísel také, že pre každé prirodzené
číslo platí $a\lfloor bn\rfloor=b\lfloor an\rfloor$. (Pripomeňme, že $\lfloor x\rfloor$ znamená najväčšie celé
číslo, ktoré je menšie alebo rovné~$x$.)}
\podpis{Eugen Kováč:???}

{%%%%%   vyberko, den 1, priklad 4
...}
\podpis{...}

{%%%%%   vyberko, den 2, priklad 4
...}
\podpis{...}

{%%%%%   trojstretnutie, priklad 1
Dokážte, že ak kladné reálne čísla $a$, $b$, $c$ spĺňajú nerovnosť
$$
  5abc > a^3 + b^3 + c^3,
$$
potom existuje trojuholník s~dĺžkami strán $a$, $b$, $c$.}
\podpis{Bielorusko, MO 98/99}

{%%%%%   trojstretnutie, priklad 2
Daný je trojuholník $ABC$ a~jemu vpísaná kružnica~$k$.
Kružnice $k_a$, $k_b$, $k_c$ pretínajú ortogonálne kružnicu~$k$ a~úsečky
$BC$, $C\!A$, $AB$ sú (v~tomto poradí) ich tetivami.
Kružnice $k_a$, $k_b$ sa druhýkrát pretínajú v~bode~$C'$,
kružnice $k_c$, $k_a$ v~bode~$B'$
a~kružnice $k_b$, $k_c$ v~bode~$A'$.
Dokážte, že polomer kružnice opísanej trojuholníku $A'B'C'$ je
polovicou polomeru kružnice~$k$.

{\it Poznámka.} Hovoríme, že dve kružnice sa pretínajú
{\it ortogonálne}, ak ich dotyčnice v~každom spoločnom bode sú navzájom
kolmé.}
\podpis{jury MMO 99}

{%%%%%   trojstretnutie, priklad 3
Nech $n$ je prirodzené číslo. Dokážte, že
$n$ je mocninou~$2$ práve vtedy, keď existuje celé číslo~$m$ také, že
$2^n-1$ je deliteľom $m^2+9$.}
\podpis{jury MMO 98}

{%%%%%   trojstretnutie, priklad 4
Nech $P(x)$ je polynóm s~celočíselnými koeficientmi. Dokážte, že potom
polynóm
$$
  Q(x) = P(x^4) P(x^3) P(x^2) P(x) + 1
$$
nemá celočíselný koreň.}
\podpis{E. Kováč}

{%%%%%   trojstretnutie, priklad 5
Nech $ABC\!D$ je rovnoramenný lichobežník so základňami $AB$ a~$C\!D$.
Kružnica vpísaná trojuholníku $BC\!D$ sa dotýka strany $C\!D$ v~bode~$E$.
Nech $F$ je taký bod na osi uhla $D\!AC$, že priamky $EF$ a~$C\!D$ sú
navzájom kolmé. Kružnica opísaná trojuholníku $AC\!F$ pretína
priamku~$C\!D$ v~bodoch $C$ a~$G$. Dokážte, že trojuholník $AFG$ je
rovnoramenný.}
\podpis{USA, MO 98/99}

{%%%%%   trojstretnutie, priklad 6
Každé celé číslo je ofarbené jednou z~farieb
červená, modrá, zelená a~biela. Nech $x$ a~$y$ sú nepárne celé
čísla také, že $|x|\ne|y|$. Dokážte, že existujú nejaké dve prirodzené
čísla rovnakej farby, ktorých rozdiel nadobúda jednu z~hodnôt
$x$, $y$, $x+y$ alebo $x-y$.}
\podpis{jury MMO 99}

{%%%%%   IMO, priklad 1
Dve kružnice $\Gamma_1$ a~$\Gamma_2$ sa pretínajú v~dvoch bodoch $M$ a~$N$.
Nech priamka~$AB$ je ich spoločnou dotyčnicou, pričom bod~$A$ leží na
kružnici~$\Gamma_1$, bod~$B$ na $\Gamma_2$ a~navyše bod~$M$ je k~priamke~$AB$
bližšie než bod~$N$. Nech $CD$ je priamka rovnobežná s~$AB$ a~prechádzajúca
bodom~$M$, pričom bod~$C$ leží na kružnici~$\Gamma_1$ a~bod~$D$ na $\Gamma_2$.
Priamky $AC$ a~$BD$ sa pretínajú v~bode~$E$, priamky $AN$ a~$CD$ v~bode~$P$
a~priamky $BN$ a~$CD$ v~bode~$Q$. Dokážte, že $|EP|=|EQ|$.}
\podpis{Rusko}

{%%%%%   IMO, priklad 2
Nech $a$, $b$, $c$ sú kladné reálne čísla také, že $abc=1$. Dokážte nerovnosť
$$
  \( a - 1 + \frac1b \)
  \( b - 1 + \frac1c \)
  \( c - 1 + \frac1a \)
  \le 1.
$$}
\podpis{Bielorusko}

{%%%%%   IMO, priklad 3
Nech $n\ge 2$ je prirodzené číslo a~$\lambda$ je kladné reálne číslo.
Na začiatku máme na vodorovnej priamke $n$~bĺch, nie všetky v~jednom bode.
V~jednom kroku si môžeme zvoliť dve blchy v~nejakých bodoch $A$ a~$B$,
pričom $A$ je naľavo od $B$, a~blchu z~bodu~$A$ necháme skočiť do bodu~$C$,
ktorý je napravo od $B$ a~platí $|BC|:|AB|=\lambda$.

Určte všetky hodnoty~$\lambda$ také, že pre ľubovoľný bod~$M$ na danej priamke
a~ľubovoľnú začiatočnú pozíciu bĺch, existuje konečná postupnosť krokov,
po ktorej budú všetky blchy napravo od bodu~$M$.}
\podpis{USA}

{%%%%%   IMO, priklad 4
Kúzelník má sto kartičiek očíslovaných číslami $1,2,\dots,100$. Rozdelí ich
do troch krabíc (červenej, bielej a~modrej) tak, že v~každej krabici
je aspoň jedna kartička. Divák z~hľadiska vytiahne dve kartičky z~dvoch
rôznych krabíc a~nahlas oznámi súčet čísel na týchto kartičkách.
Na základe tejto informácie dokáže kúzelník určiť, z~ktorej krabice nebola
vytiahnutá žiadna kartička. Určte koľkými spôsobmi môže kúzelník rozdeliť
kartičky do krabíc tak, aby tento trik fungoval.}
\podpis{Maďarsko}

{%%%%%   IMO, priklad 5
Zistite, či existuje prirodzené číslo~$n$ také, že $n$ má práve $2000$
prvočíselných deliteľov a~$2^n+1$ je deliteľné číslom~$n$.}
\podpis{Rusko}

{%%%%%   IMO, priklad 6
Nech $AH_1$, $BH_2$, $CH_3$ sú výšky v~ostrouhlom trojuholníku $ABC$. Jemu
vpísaná kružnica sa dotýka strán $BC$, $CA$, $AB$ postupne v~bodoch $T_1$, $T_2$, $T_3$.
Uvažujme obrazy priamok $H_1H_2$, $H_2H_3$, $H_3H_1$
v~osovej súmernosti postupne podľa priamok $T_1T_2$, $T_2T_3$, $T_3T_1$.
Dokážte, že tieto obrazy vytvárajú trojuholník, ktorého vrcholy ležia na
kružnici vpísanej trojuholníku $ABC$.}
\podpis{Rusko}

