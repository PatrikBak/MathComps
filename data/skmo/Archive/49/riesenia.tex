{%%%%%   A-I-1
Vzhľadom na~to, že rovnica $P\bigl(Q(x)\bigr)=0$ má reálny koreň,
má kvadratická rovnica
$P(x)=0$ dva reálne korene
$r_1$, $r_2$ (nevylučujeme, že $r_1=r_2$).
Mnohočlen $P\bigl(Q(x)\bigr)$ možno preto zapísať v tvare
$$
P\bigl(Q(x)\bigr)=a\bigl(Q(x)-r_1\bigr)\bigl(Q(x)-r_2\bigr),
$$
kde $a$ je reálne číslo $a\ne0$.
Rovnica $P\bigl(Q(x)\bigr)=0$ má podľa zadania štyri reálne korene, preto
každá z~kvadratických rovníc $Q(x)-r_1=0$, $Q(x)-r_2=0$ musí mať
dva reálne
korene. Z~Vi\`etových vzťahov vyplýva, že súčet koreňov
v~oboch kvadratických rovniciach je rovnaký, lebo obidve rovnice majú
rovnaký koeficient
pri~lineárnom člene. Pritom tri zo štyroch
reálnych koreňov oboch kvadratických rovníc $Q(x)-r_1=0$, $Q(x)-r_2=0$
sú podľa zadania čísla $-22$, $7$,
$13$, štvrtý koreň označme $q$. Ďalej môže nastať jedna z troch možností:
\item{(i)} Jedna z~kvadratických
rovníc má korene $-22$, $7$, druhá má korene $13$ a~$q$. Potom
platí $-22+7=13+q$, teda $q=-28$.
\item{(ii)} Jedna z~kvadratických
rovníc má korene $-22$, $13$, druhá má korene $7$ a~$q$. Potom
platí $-22+13=7+q$, teda $q=-16$.
\item{(iii)} Jedna z~kvadratických
rovníc má korene $13$, $7$, druhá má korene $-22$ a~$q$. Potom
platí $13+7=-22+q$, teda $q=42$.

Je zrejmé, že v~každom z~prípadov (i), (ii), (iii) existujú príslušné
kvadratické trojčleny $P(x)$ a~$Q(x)$.
Ak má mať jedna z~kvadratických rovníc $Q(x)-r_1=0$,
$Q(x)-r_2=0$
korene $-22$, $7$ a~druhá $13$, $-28$, položíme $Q(x)=x^2+15 x$,
$r_1=(-22)\cdot7=-154$, $r_2=13\cdot(-28)=-364$,
$P(x)=(x+154)(x+364)=x^2+518 x+56\,056$. Obdobne možno postupovať v
zvyšných prípadoch.

Štvrtým koreňom rovnice $P\bigl(Q(x)\bigr)=0$ môže byť ktorékoľvek
z~čísel $-28$,
$-16$, $42$.

\ineriesenie
Úvahy o~koeficiente pri~lineárnom člene s~využitím Vi\`etových vzťahov možno
nahradiť nasledujúcou úvahou o~grafoch kvadratických funkcií.

Pretože grafy kvadratických funkcií $f_1\colon\ y=Q(x)-r_1$
a~$f_2\colon\ y=Q(x)-r_2$ majú spoločnú os súmernosti, a~pritom existujú štyri
reálne korene rovnice $P(Q(x))=0$, sú tieto korene na osi $x$ po dvoch
stredovo súmerné podľa priesečníka osi súmernosti grafov oboch funkcií
$f_1$ a~$f_2$ s~osou $x$.
Vzhľadom k~polohe daných troch
koreňov na osi $x$ možno ďalej uvažovať tri možnosti -- rovnako ako
v~predchádzajúcom riešení. Napríklad
\item{(i)} Stred súmernosti je $-7{,}5=\frac{-22+7}2$. Štvrtý
koreň leží potom na osi $x$ a~je obrazom čísla $13$ v stredovej
súmernosti podľa stredu v~bode $-7{,}5$. Štvrtým hľadaným koreňom
je teda číslo $-28$.

Podobne možno postupovať vo zvyšných dvoch prípadoch. Takýmto postupom
dospejeme k~rovnakému výsledku.
}

{%%%%%   A-I-2
...}

{%%%%%   A-I-3
Z textu úlohy vyplýva, že neznáme $x$, $y$, $z$ sú kladné čísla. Preto
môžeme danú sústavu upraviť do nasledujúceho tvaru
$$
\eqalign{
-\sqrt{x}+\sqrt{y}+\sqrt{z}&= \frac{a}{\sqrt{x}},\cr
\sqrt{x}-\sqrt{y}+\sqrt{z}&= \frac{b}{\sqrt{y}},\cr
\sqrt{x}+\sqrt{y}-\sqrt{z}&= \frac{c}{\sqrt{z}}.\cr
}
$$
Ak po dvojiciach sčítame jednotlivé rovnice predošlej sústavy,
dostaneme tak sústavu
$$
\eqalign{
\frac{b}{\sqrt{y}}+\frac{c}{\sqrt{z}}&= 2\sqrt{x},\cr
\frac{c}{\sqrt{z}}+\frac{a}{\sqrt{x}}&= 2\sqrt{y},\cr
\frac{a}{\sqrt{x}}+\frac{b}{\sqrt{y}}&= 2\sqrt{z}.\cr
}
$$
Ďalej po ľahkej úprave
$$
\eqalign{
b\sqrt{z} + c\sqrt{y} &= 2\sqrt{xyz},\cr
c\sqrt{x} + a\sqrt{z} &= 2\sqrt{xyz},\cr
a\sqrt{y} + b\sqrt{x} &= 2\sqrt{xyz}.\cr
}
$$
Odčítaním prvej a~tretej, resp\. druhej a~tretej, rovnice poslednej
sústavy ďalej získame
$$
\eqalign{
b\sqrt{z} + (c-a)\sqrt{y} &= b\sqrt{x},\cr
a\sqrt{z} - a\sqrt{y} &= (b-c)\sqrt{x}.\cr
}
$$
Obe strany prvej rovnice predchádzajúcej sústavy vynásobíme číslom $a$,
obe strany druhej rovnice potom vynásobíme číslom $-b$.
Po sčítaní obidvoch takto upravených rovníc, dostaneme
$$
a(b+c-a)\sqrt{y}= b(c+a-b)\sqrt{x},
$$
podobným spôsobom dostaneme tiež
$$
a(b+c-a)\sqrt{z}=c(a+b-c)\sqrt{x}.
$$
Ak by pre kladné čísla $a$, $b$, $c$, platil vzťah $b+c-a=0$,
potom z~predošlých dvoch rovníc vyplýva, že tiež
$a+b-c=0$, $c+a-b=0$.  Potom však
$a=b=c=0$, čo však nie je možné. Je teda
$b+c-a\ne 0$. Z~poslednej dvojice rovníc
vyjadríme
$\sqrt{y}$ a~$\sqrt{z}$ pomocou $\sqrt{x}$ nasledujúcim spôsobom:
$$
\eqalign{
\sqrt{y}&= \frac{b(c+a-b)}{a(b+c-a)}\sqrt{x},\cr
\sqrt{z}&= \frac{c(a+b-c)}{a(b+c-a)}\sqrt{x}.\cr
}
$$
Odtiaľ sa ľahko sa vidí, že výrazy $b+c-a$, $c+a-b$, $a+b-c$
sú súčasne všetky kladné alebo všetky záporné.
Po dosadení $\sqrt{y}$ a~$\sqrt{z}$ do pôvodnej sústavy rovníc
získame (po úpravách) riešenie $(x,y,z)$, kde
$$
\eqalign{
x&= \frac{a^2(b+c-a)}{(c+a-b)(a+b-c)},\cr
y&= \frac{b^2(c+a-b)}{(b+c-a)(a+b-c)},\cr
z&= \frac{c^2(a+b-c)}{(c+a-b)(b+c-a)}.\cr
}
$$
Vzhľadom k~tomu, že k~riešeniu sústavy rovníc sme dospeli výhradne po
ekvivalentných úpravách, nie je potrebné robiť skúšku správnosti.

Sústava má pritom vyššie uvedené riešenie v~obore kladných čísel práve
vtedy, keď súčasne platia nasledujúce podmienky $b+c-a>0$, $c+a-b>0$,
$a+b-c>0$, t\.j\. práve vtedy, keď kladné čísla $a$, $b$, $c$ sú
dĺžkami strán trojuholníka.
}

{%%%%%   A-I-4
Nech $A_sB_sC_s$, kde  $s\in \{1,2,\dots ,1999\}$, sú
trojuholníky vyhovujúce podmienkam úlohy
a~$(XYZ)$ nech označuje polrovinu s~hraničnou priamkou~$XY$ a~vnútorným
bodom $Z$. Každý z~daných trojuholníkov $A_sB_sC_s$ je prienikom vždy
troch polrovín $(A_sB_sC_s)$, $(B_sC_sA_s)$ a~$(C_sA_sB_s)$, preto je
(neprázdna)
množina $\mm M$ prienikom $3\cdot1\,999=5\,997$ takých polrovín. Vzhľadom
k~tomu, že polroviny $(A_sB_sC_s)$, kde $s\in \{1,2,\dots ,1999\}$, sa
navzájom líšia len posunutím, je ich prienikom polrovina $(A_iB_iC_i)$,
kde $i$ je pevný index z~množiny $\{1,2,\dots ,1999\}$. Podobne prienikom
všetkých polrovín $(B_sC_sA_s)$ je určitá polrovina $(B_jC_jA_j)$
a~prienikom všetkých polrovín $(C_sA_sB_s)$ je určitá polrovina
$(C_kA_kB_k)$, kde $j,k\in \{1,2,\dots ,1999\}$.

Množina $\mm M$ je preto prienikom troch vyššie
spomínaných polrovín $(A_iB_iC_i)$,
$(B_jC_jA_j)$ a~$(C_kA_kB_k)$, $\mm M$ je teda trojuholník $ABC$,
kde $A$ je priesečník priamok $A_iB_i$ a~$C_kA_k$, $B$ je priesečník priamok
$A_iB_i$ a~$B_jC_j$ a~napokon $C$ je priesečník priamok $B_jC_j$ a~$C_kA_k$.
Tento trojuholník je podobný všetkým trojuholníkom $A_sB_sC_s$, pričom pre
pomer podobnosti $\lambda$ platí $0<\lambda \leq 1$. (Prípad $A=B=C$ možno
podľa zadania úlohy vylúčiť.)

Vzhľadom k~tomu, že obsah trojuholníka $ABC$ je $\lambda^2$, stačí
dokázať, že $\lambda \geq \frac13$. Označme $v$ výšku z~vrcholu
$C_i$ na stranu $A_iB_i$ v~trojuholníku $A_iB_iC_i$. Pretože priamka
$A_iB_i$ je totožná s~priamkou $AB$, je vzdialenosť ťažiska $T_i$
trojuholníka $A_iB_iC_i$ od priam\-ky $AB$ rovná $\frac13v$. Podľa zadaní
obsahuje množina $\mm M$ ťažisko všetkých trojuholníkov $A_sB_sC_s$, musí
teda obsahovať ťažisko $T_i$ trojuholníka $A_iB_iC_i$.

Vzdialenosť vrcholu $C$ trojuholníka $ABC$ od jeho strany $AB$ je teda
aspoň~$\frac13v$. Porovnaním veľkostí výšok z~vrcholov $C_i$ a~$C$
v~podobných trojuholníkoch $A_iB_iC_i$ a~$ABC$ dostávame už priamo
žiadanú
nerovnosť $\lambda \geq \frac13$, t\.j\.~$\lambda^2 \geq \frac19$, čo sme
chceli dokázať.
}

{%%%%%   A-I-5
Označme
$$
S(n)=f(1)+f(2)+\cdots+f(n).
$$
Zo zadania vyplýva $S(1)=1$. Pretože $f(n)\geq 1$ pre všetky prirodzené
čísla $n$, je $S:\Bbb N\to\Bbb N$ rastúca funkcia. Ak je $n$ prirodzené
číslo
tvaru $n=2^k$, kde $k$ je prirodzené, určíme súčet $S(n)$
nasledujúcim spôsobom: Počet nepárnych
čísel, ktoré nie sú väčšie ako $n$, je $2^{k-1}$. Každé nepárne číslo
sa na súčte $S(n)$ podieľa hodnotou $1$. Počet párnych
čísel, ktoré nie sú väčšie ako $n$, je tiež
$2^{k-1}$, pritom každé párne číslo sa na súčte $S(n)$
podieľa hodnotou minimálne~$1$. Ak je naviac toto číslo deliteľné
štyrmi, podieľa sa na súčte ďalšou~$1$. Ak je ďalej číslo
deliteľné ôsmimi, podieľa sa ďalšou~$1$, atď\. (Hodnotu $S(n)$ tak
tvoríme sčítaním hodnôt $1$ \uv{po vrstvách}).
Spolu je teda
$$
\align
S(2^k)=&2^{k-1}+2^{k-1}+2^{k-2}+\cdots+2+1=2^{k-1}+
\frac{2^k-1}{2-1}=\\
=&2^k+2^{k-1}-1=3\cdot 2^{k-1}-1.
\endalign
$$

Nech $p$ je prirodzené číslo, ktoré sa dá zapísať v tvare $p=2^m s$,
kde $m$ je celé nezáporné číslo a~$s$ nepárne prirodzené číslo. Nech $k$
je prirodzené číslo také, že $p<2^k$ (teda $m<k$), a~nech $l$ je
nepárne prirodzené číslo. Potom
$$
f(2^k l+p)=f(2^k l+2^m s)=f\big(2^m(2^{k-m} l +s)\big).
$$
Číslo $2^{k-m} l +s$ je nepárne, preto
$f\big(2^m(2^{k-m}l+s)\big)=f(2^m s)=f(p)$.
Spolu teda dostávame
$f(2^k l+p)=f(p)$.

Ak sú $k$, $m$ nezáporné celé čísla, $k>m$, a~$l$ nepárne číslo, platí
podľa predchádzajúceho odstavca
$$
\align
S(2^k l+2^m)=&f(1)+f(2)+\cdots+f(2^k l)+f(2^k l +1)+f(2^k l+2)+\cdots+\\
             &+f(2^k l +2^m)=\cr
=&f(1)+f(2)+\cdots+f(2^k l)+f(1)+f(2)+\cdots+f(2^m)=\cr
=&S(2^k l)+S(2^m).
\endalign
$$
A~odtiaľ už matematickou indukciou ľahko dokážeme, že ak sú
$k_1>k_2>\cdots>k_i$ nezáporné celé čísla, potom platí
$$
S(2^{k_1}+2^{k_2}+\cdots+2^{k_i})=S(2^{k_1})+S(2^{k_2})+\cdots+
S(2^{k_i}).
$$

Najväčšie nezáporné celé číslo $k_1$ také, že
$3\cdot2^{k_1-1}-1=S(2^{k_1})\leq123\,456$, je $k_1=16$. Pritom
$S(2^{16})=98\,303$.

Najväčšie nezáporné celé číslo $k_2$ také, že
$3\cdot2^{k_2-1}-1=S(2^{k_2})\leq123\,456-98\,303=25\,153$, je $k_2=14$.
Pritom $S(2^{14})=24\,575$.

Najväčšie nezáporné celé číslo $k_3$ také, že
$3\cdot2^{k_3-1}-1=S(2^{k_3})\leq25\,153-24\,575=578$, je $k_3=8$.
Pritom $S(2^{8})=383$.

Najväčšie nezáporné celé číslo $k_4$ také, že
$3\cdot2^{k_4-1}-1=S(2^{k_4})\leq578-383=195$, je $k_4=7$. Pritom
$S(2^{7})=191$.

Najväčšie nezáporné celé číslo $k_5$ také, že
$3\cdot2^{k_5-1}-1=S(2^{k_5})\leq195-191=4$, je $k_5=1$. Pritom
$S(2^{1})=2$.

Najväčšie nezáporné celé číslo $k_6$ také, že
$S(2^{k_6})\leq4-2=2$, je $k_6=0$. Pritom
$S(2^{0})=1$.

Teda
$$
\align
S(82\,307)=&S(2^{16}+2^{14}+2^8+2^7+2+1)=\\
          =&S(2^{16})+S(2^{14})+S(2^8)+S(2^7)+S(2)+S(1)=\cr
          =&123\,455\leq123\,456.
\endalign
$$
Pritom $S(82\,308)=S(82\,307)+f(82\,308)=123\,455+2=123\,457>123\,456$.

Najväčšie prirodzené číslo $n$, pre ktoré platí $S(n)\leq123\,456$ je
$n=82\,307$.


\ineriesenie
\def\cel#1#2{\left\lfloor\frac{#1}{#2}\right\rfloor}%
\def\hcel#1{\cel{82\,304}{#1}}%
Na základe úvahy o~sčítaní hodnôt "po vrstvách" (ako v~predošlom
riešení) zistíme, že
$$
S(n)=n+\cel{n}4+\cel{n}8+\cel{n}{16}+\cdots.
$$
Pritom $\lfloor r\rfloor$ znamená celú časť reálneho čísla $r$,
čo je najväčšie celé číslo, ktoré nie je väčšie ako $r$.

Pretože pre každé reálne číslo $r$ platí
$\lfloor r\rfloor\leq r$, platí tiež
$$
S(n)\leq n+\frac{n}4+\frac{n}8+\cdots=\frac{n}2+
\frac{n}2\left(1+\frac12+\frac14+\cdots\right)=
\frac{n}2+n=\frac{3n}2.
$$
Najväčšie prirodzené číslo $n$ pre ktoré platí, že $\frac{3n}2\leq123
456$, je $n=82\,304$.
Pritom
$$
\align
S(82\,304)=&82\,304+\hcel{4}+\hcel{8}+\hcel{16}+\cdots+\\
           &+\hcel{65\,536}+\hcel{131\,072}+\cdots=\cr
 =&82\,304+20\,576+10\,288+5\,144+2\,572+1\,286+643+\cr
  &+321+160+80+40+20+10+5+2+1+0+0+\cdots=\cr
 =&123\,452.
\endalign
$$
Ďalej
$$
\eqalign{
S(82\,305)&=S(82\,304)+f(82\,305)=123\,452+1=123\,453,\cr
S(82\,306)&=S(82\,305)+f(82\,306)=123\,453+1=123\,454,\cr
S(82\,307)&=S(82\,306)+f(82\,307)=123\,454+1=123\,455,\cr
S(82\,308)&=S(82\,307)+f(82\,308)=123\,455+2=123\,457.}
$$

Najväčšie prirodzené číslo $n$, pre ktoré $S(n)\leq123\,456$, je
teda $n=82\,307$.
}

{%%%%%   A-I-6
...}

{%%%%%   B-I-1
...}

{%%%%%   B-I-2
...}

{%%%%%   B-I-3
...}

{%%%%%   B-I-4
...}

{%%%%%   B-I-5
...}

{%%%%%   B-I-6
...}

{%%%%%   C-I-1
Obe delenia hľadaného
čísla $N$ vyjadríme rovnosťami
$$
N=19a+p\quad\text{a}\quad N=99b+q,
$$
kde $a$, $b$ sú príslušné neúplné podiely a~$p$, $q$
príslušné zvyšky.
Podľa zadania sú čísla $p$, $q$ prvočísla,
pričom ako zvyšky
spĺňajú nerovnosti $p<19$ a~$q<99$. Nezáporné celé
čísla $a$, $b$
sú podľa zadania zase také, že ich súčet sa rovná
číslu
1\,999. Preto platí $b=1\,999-a$ a~z~dvojitého vyjadrenia
čísla $N$,
$$
N=19a+p=99\cdot(1\,999-a)+q,
$$
odvodíme rovnosť $118a+(p-q)=197\,901$. Pretože rozdiel
zvyškov
$p-q$ je "malé" číslo, presnejšie ${-99}<p-q<19$, je podľa
poslednej rovnosti číslo $118a$ taký násobok
čísla 118, ktorý
leží medzi číslami $197\,901-19$ a~$197\,901+99$.
Delením $197\,901:118$
zistíme, že $197\,901=1\,677\cdot118+15$. Preto nutne
platí $a=1\,677$ (takže $b=322$) a~$p-q=15$. Z~poslednej
rovnosti vyplýva, že jedno
z~prvočísel $p,q$ je nepárne a~druhé párne, teda $q=2$ a~$p=17$.
Ostáva vypočítať hodnotu $N$:
$$
N=19\cdot1\,677+17=99\cdot322+2=31\,880.
$$

{\it Odpoveď\/}: Hľadané číslo je rovné 31\,880.
}

{%%%%%   C-I-2
...}

{%%%%%   C-I-3
...}

{%%%%%   C-I-4
...}

{%%%%%   C-I-5
...}

{%%%%%   C-I-6
...}

{%%%%%   A-S-1
...}

{%%%%%   A-S-2
...}

{%%%%%   A-S-3
...}

{%%%%%   A-II-1
...}

{%%%%%   A-II-2
...}

{%%%%%   A-II-3
...}

{%%%%%   A-II-4
...}

{%%%%%   A-III-1
...}

{%%%%%   A-III-2
...}

{%%%%%   A-III-3
...}

{%%%%%   A-III-4
...}

{%%%%%   A-III-5
...}

{%%%%%   A-III-6
...}

{%%%%%   B-S-1
Uvažovaná funkcia $f$ je po častiach lineárna, preto je ohraničená na
každom ohraničenom intervale. Stačí teda funkciu $f$ vyšetriť zvlášť pre
$x\le\min(1,b)$ a~zvlášť pre $x\ge\max(1,b)$, keď majú oba výrazy
v~absolútnych hodnotách rovnaké znamienko.

a) Ak je $x\le\min(1,b)$, je
$$
f(x)=a(1-x)+b(x-3)+b-x+x-1=(b-a)x-2b+a-1.
$$
Funkcia $f$ bude na tomto intervale ohraničená práve vtedy,
keď tu bude konštantná, t\.j\.~práve vtedy, keď $a=b$.

a) Ak je $x\ge\max(1,b)$, je
$$
f(x)=a(x-1)+b(x-3)+x-b+x-1=(a+b+2)x-a+4b-1.
$$
Funkcia $f$ bude na tomto intervale ohraničená práve vtedy,
keď tu bude konštantná, t\.j\.~práve vtedy, keď $a+b=-2$.

Spojením oboch podmienok dostávame, že funkcia $f$ bude ohraničená
práve vtedy, keď bude ohraničená na oboch uvedených neohraničených
intervaloch, t\.j\.~práve keď $a=b=-1$. Pre funkciu $f$ potom
dostaneme vyjadrenie
$$
f(x)=|x+1|-|x-1|+2.
$$
Jej graf vidíme na obr.\,33.

\line{\hss\inspicture-!\hss\hss\hss\inspicture-!\hss}
\bigskip
}

{%%%%%   B-S-2
...}

{%%%%%   B-S-3
...}

{%%%%%   B-II-1
...}

{%%%%%   B-II-2
...}

{%%%%%   B-II-3
...}

{%%%%%   B-II-4
...}

{%%%%%   C-S-1
...}

{%%%%%   C-S-2
...}

{%%%%%   C-S-3
...}

{%%%%%   C-II-1
...}

{%%%%%   C-II-2
...}

{%%%%%   C-II-3
...}

{%%%%%   C-II-4
...}

{%%%%%   vyberko, den 1, priklad 1
...}

{%%%%%   vyberko, den 1, priklad 2
...}

{%%%%%   vyberko, den 1, priklad 3
...}

{%%%%%   vyberko, den 1, priklad 4
...}

{%%%%%   vyberko, den 2, priklad 1
...}

{%%%%%   vyberko, den 2, priklad 2
...}

{%%%%%   vyberko, den 2, priklad 3
...}

{%%%%%   vyberko, den 2, priklad 4
...}

{%%%%%   vyberko, den 3, priklad 1
...}

{%%%%%   vyberko, den 3, priklad 2
...}

{%%%%%   vyberko, den 3, priklad 3
...}

{%%%%%   vyberko, den 3, priklad 4
...}

{%%%%%   vyberko, den 4, priklad 1
...}

{%%%%%   vyberko, den 4, priklad 2
...}

{%%%%%   vyberko, den 4, priklad 3
...}

{%%%%%   vyberko, den 4, priklad 4
...}

{%%%%%   vyberko, den 5, priklad 1
...}

{%%%%%   vyberko, den 5, priklad 2
...}

{%%%%%   vyberko, den 5, priklad 3
...}

{%%%%%   vyberko, den 5, priklad 4
...}

{%%%%%   trojstretnutie, priklad 1
...}

{%%%%%   trojstretnutie, priklad 2
...}

{%%%%%   trojstretnutie, priklad 3
...}

{%%%%%   trojstretnutie, priklad 4
Všimnime si, že pre celé číslo $n$ platí
$ n^3 \equiv n \pmod3 $.
Jednoducho to môžeme na\-hliad\-nuť overením všetkých zvyškových tried modulo 3,
alebo priamo z~{\it Malej Fermatovej vety}. Potom tiež
$ n^4 \equiv n^2 \pmod3 $.
Avšak, ak $P(x)$ je polynóm s~celočíselnými koeficientami, tak aj
$ P(n^3) \equiv P(n) \pmod3 $
a~$ P(n^4) \equiv P(n^2) \pmod3 $.
Polynóm $Q(x)$ má zrejme celočíselné koeficienty. Z~posledných dvoch
kongruencií dostávame
$$
  Q(n) \equiv \big( P(n)P(n^2) \big)^2  + 1 \pmod3 .
$$
Ale ľahko sa overí, že pre každé celé číslo $m$ platí
$m^2+1\not\equiv 0 \pmod3$.
Takže pre každé celé číslo $n$ platí $Q(n)\not\equiv 0 \pmod3$,
z~čoho vyplýva, že polynóm $Q(x)$ nemôže mať celočíselný koreň.
Tým sme dokázali, čo bolo treba.
}

{%%%%%   trojstretnutie, priklad 5
...}

{%%%%%   trojstretnutie, priklad 6
...}

{%%%%%   IMO, priklad 1
...}

{%%%%%   IMO, priklad 2
...}

{%%%%%   IMO, priklad 3
...}

{%%%%%   IMO, priklad 4
...}

{%%%%%   IMO, priklad 5
...}

{%%%%%   IMO, priklad 6
...}

{%%%%%   MEMO, priklad 1
...}

{%%%%%   MEMO, priklad 2
...}

{%%%%%   MEMO, priklad 3
...}

{%%%%%   MEMO, priklad 4
...}

{%%%%%   MEMO, priklad t1
...}

{%%%%%   MEMO, priklad t2
...}

{%%%%%   MEMO, priklad t3
...}

{%%%%%   MEMO, priklad t4
...}

{%%%%%   MEMO, priklad t5
...}

{%%%%%   MEMO, priklad t6
...}

{%%%%%   MEMO, priklad t7
...}

{%%%%%   MEMO, priklad t8
...} 