{%%%%%   A-I-1
Ak je $S$ obsah trojuholníka so stranami $a$, $b$, $c$ a~$T$ je obsah trojuholníka
so stranami $a+b$, $b+c$, $c+a$, potom platí $T\geq4S$. Dokážte a~zistite, kedy nastane
rovnosť.}
\podpis{P. Kaňovský}

{%%%%%   A-I-2
V~obore celých čísel $x$, $y$ riešte rovnicu
$$
\left(x_5\right)^2+(y^4)_5=2xy^2+51,
$$
kde $n_5$ označuje násobok piatich najbližší k~číslu~$n$, napríklad ${(-9)_5}={-10}$.}
\podpis{P. Černek}

{%%%%%   A-I-3
V~danom trojuholníku~$ABC$ pretína os uhla~$ACB$ stranu~$AB$ v~bode~$K$ a~opísanú
kružnicu v~bode~$L$ ($L\ne C$). Označme $V$ stred kružnice vpísanej trojuholníku~$ABC$,
$S$~stred kružnice opísanej trojuholníku~$KBV$ a~Z~priesečník priamok $AB$ a~$SL$.
Dokážte, že priamka~$SK$ je dotyčnicou kružnice opísanej trojuholníku~$KLZ$.}
\podpis{J. Földes}

{%%%%%   A-I-4
Nech $n\ge2$ je dané prirodzené číslo. Pre ktoré hodnoty reálneho parametra~$p$
má sústava rovníc
$$
\align
x_1^4+\frac{2}{x_1^2}=&px_2,\\
x_2^4+\frac{2}{x_2^2}=&px_3,\\
\vdots\phantom{x} &\\
x_{n-1}^4+\frac{2}{x_{n-1}^2}=&px_n,\\
x_{n}^4+\frac{2}{x_{n}^2}=&px_1
\endalign
$$
aspoň dve riešenia v~obore reálnych čísel?}
\podpis{J. Švrček}

{%%%%%   A-I-5
Nájdite všetky mnohočleny~$P(x)$ s~reálnymi koeficientmi, ktoré pre každé reálne
číslo~$x$ spĺňajú rovnosť
$$
\postdisplaypenalty 10000
(x+1)\, P(x-1)+(x-1)\, P(x+1)=2x\, P(x).
$$
}
\podpis{E. Kováč}

{%%%%%   A-I-6
Nájdite všetky štvorsteny, ktoré majú sieť tvaru deltoidu a~práve štyri hrany danej dĺžky~$a$.
(Deltoidom rozumieme konvexný štvoruholník, ktorý je súmerný podľa jedinej zo svojich
uhlopriečok, nepatrí k~nim ani štvorec, ani kosoštvorec.)}
\podpis{P. Leischner}

{%%%%%   B-I-1
Do tabuľky $4 \times 4$ sú vpísané kladné reálne čísla tak, že súčin v~každej
pätici tvaru~\Image*{51-b-i-1}{1.2} je rovný~$1$. Zistite maximálny počet rôznych čísel zapísaných v~tabuľke.}
\podpis{P. Černek}

{%%%%%   B-I-2
Určte, koľko čísel môžeme vybrať z~množiny $\{1, 2, 3,\dots,75\,599,75\,600\}$ tak, aby
medzi nimi bolo číslo $75\,600$ a~aby pre ľubovoľné dve vybrané čísla $a$, $b$ platilo,
že $a$ je deliteľom~$b$ alebo $b$ je deliteľom~$a$. (Uveďte všetky možnosti.)}
\podpis{J. Földes}

{%%%%%   B-I-3
Nech $k$ je polkružnica zostrojená nad priemerom~$AB$, ktorá leží vo vnútri štvorca $ABCD$.
Uvažujme jej dotyčnicu~$t_1$ z~bodu~$C$ (rôznu od $BC$) a~označme $P$ jej priesečník so
stranou~$AD$. Nech $t_2$ je spoločná vonkajšia dotyčnica polkružnice~$k$ a~kružnice vpísanej
trojuholníku~$CDP$ (rôzna od $AD$). Dokážte, že priamky $t_1$ a~$t_2$ sú navzájom kolmé.}
\podpis{J. Švrček}

{%%%%%   B-I-4
Pokiaľ máme $n\ge2$ prirodzených čísel, môžeme s~nimi spraviť nasledujúcu operáciu:
Vyberieme niekoľko z~nich, ale nie všetky a~každé z~vybraných čísel nahradíme ich
aritmetickým priemerom. Zistite, či je možné pre ľubovoľnú začiatočnú $n$-ticu dostať
po konečnom počte krokov všetky čísla rovnaké, ak $n$ sa rovná
\ite a) $2\,000$,
\ite b) $35$,
\ite c) $3$,
\ite d) $17$.}
\podpis{J. Földes}

{%%%%%   B-I-5
Zistite, pre ktoré reálne čísla~$p$ má sústava
$$
\postdisplaypenalty 10000
\align
  x^2y - 2x &= p, \\
  y^2x - 2y &= 2p - p^2
\endalign
$$
práve tri riešenia v~obore reálnych čísel.}
\podpis{P. Černek}

{%%%%%   B-I-6
Je daný rovnostranný trojuholník~$MPQ$. Nájdite množinu vrcholov~$C$ všetkých
trojuholníkov~$ABC$ takých, že body $P$, $Q$ sú päty výšok z~vrcholov $A$, $B$
a~bod~$M$ je stred strany~$AB$.}
\podpis{J. Šimša}

{%%%%%   C-I-1
Dokážte, že existuje jediná číslica~$c$, pre ktorú možno nájsť jediné prirodzené
číslo~$n$ končiace číslicou~$c$ a~majúce vlastnosť, že číslo $2n+1$ je druhou
mocninou prvočísla.}
\podpis{M. Koblížková}

{%%%%%   C-I-2
V~štvoruholníku~$ABCD$ sa uhlopriečky pretínajú v~bode~$P$, uhlopriečka~$AC$
je rozdelená bodmi $P$, $N$ a~$M$ na štyri zhodné úseky ($|AP|=|PN|=|NM|=|MC|$)
a~uhlopriečka~$BD$ je rozdelená bodmi $L$, $K$ a~$P$ na štyri zhodné úseky
($|BL|=|LK|=|KP|=|PD|$). Určte pomer obsahov štvoruholníkov $KLMN$ a~$ABCD$.}
\podpis{J. Zhouf}

{%%%%%   C-I-3
Určte všetky dvojice~$(x,y)$ celých čísel, ktoré sú riešením nerovnice
$$
\frac{x}{\sqrt x}+\frac{6}{y\sqrt x}<\frac{5\sqrt y}{y}.
$$
}
\podpis{J. Zhouf}

{%%%%%   C-I-4
Jožko sa vracal z~výletu. Najprv cestoval vlakom a~potom pokračoval zo zastávky
na bicykli. Celá cesta mu trvala presne 1~hodinu 30~minút a~prešiel pri nej
vzdialenosť 60~km. Vlak išiel priemernou rýchlosťou 50~km/h. Určte, ako dlho
išiel Jožko na bicykli, keď jeho rýchlosť v~km/h je vyjadrená prirodzeným číslom
rovnako ako vzdialenosť meraná v~km, ktorú prešiel na bicykli.}
\podpis{E. Kováč}

{%%%%%   C-I-5
Zostrojte rovnoramenný trojuholník~$ABC$ so základňou~$BC$ danej dĺžky~$a$,
ak je daný stred~$P$ strany~$AB$ a~bod~$Q$ ($Q\ne P$), ktorý je pätou výšky
z~vrcholu~$B$.}
\podpis{J. Švrček}

{%%%%%   C-I-6
Istý panovník pozval na oslavu svojich narodenín 28~rytierov. Každý z~rytierov mal
medzi ostatnými práve troch nepriateľov.
\ite a) Ukážte, že panovník môže rytierov rozsadiť k~dvom stolom tak, aby každý rytier
sedel pri rovnakom stole najviac s~jedným nepriateľom.
\ite b) Ukážte, že v~prípade ľubovoľného takéhoto rozsadenia sedí pri každom stole
najviac 16~rytierov.

\noindent
(Nepriateľstvo je vzájomný vzťah: Ak $A$ je nepriateľom~$B$, tak aj $B$ je nepriateľom~$A$.)}
\podpis{J. Šimša}

{%%%%%   A-S-1
V~obore celých čísel~$x$ riešte rovnicu
$$
3(x^2)_5+(3x)_5=(3x-2)(x+2),
$$
kde $n_5$ znamená násobok piatich najbližší číslu~$n$, napr.~$({-3})_5={-5}$.}
\podpis{J. Šimša}

{%%%%%   A-S-2
Označme $S$ stred kružnice vpísanej danému trojuholníku~$ABC$ a~$P$, $Q$ päty
kolmíc z~vrcholu~$C$ k~priamkam, na ktorých ležia osi vnútorných uhlov $BAC$ a~$ABC$.
Dokážte, že priamky $AB$ a~$PQ$ sú rovnobežné.}
\podpis{J. Švrček}

{%%%%%   A-S-3
Zistite, pre ktoré reálne čísla~$p$ má sústava rovníc
$$\align
x^2+1&=(p+1)x+py-z,\\
y^2+1&=(p+1)y+pz-x,\\
z^2+1&=(p+1)z+px-y\\
\endalign
$$
s~neznámymi $x$, $y$, $z$ práve jedno riešenie v~obore reálnych čísiel.}
\podpis{E. Kováč}

{%%%%%   A-II-1
Dokážte, že pre ľubovoľné čísla $\alpha,\beta\in\langle0,\pi/2)$ platí nerovnosť
$$
\frac1{\cos\alpha}+\frac1{\cos\beta}\geq2
\sqrt{\tg\alpha+\tg\beta}.
$$
Zistite, kedy nastane rovnosť.}
\podpis{E. Kováč}

{%%%%%   A-II-2
Nájdite všetky dvojice prirodzených čísel $x$ a~$y$, pre ktoré platí
$$
x^2=4y+3\cdot n(x,y),
$$
kde $n(x, y)$ značí najmenší spoločný násobok čísel $x$ a~$y$.}
\podpis{P. Černek}

{%%%%%   A-II-3
Do kružnice~$k$ je vpísaný štvoruholník~$ABCD$, ktorého uhlopriečka~$BD$ nie je priemerom.
Dokážte, že priesečník priamok, ktoré sa kružnice~$k$ dotýkajú v~bodoch $B$ a~$D$, leží na
priamke~$AC$ práve vtedy, keď platí $|AB|\cdot|CD|=|AD|\cdot|BC|$.}
\podpis{E. Kováč}

{%%%%%   A-II-4
V~obore reálnych čísel riešte sústavu rovníc
$$
\postdisplaypenalty 10000
\align
x^2-1&=p(y+z),\\
y^2-1&=p(z+x),\\
z^2-1&=p(x+y)\\
\endalign
$$
s~neznámymi $x$, $y$, $z$ a~parametrom~$p$. Vykonajte diskusiu počtu riešení.}
\podpis{E. Kováč}

{%%%%%   A-III-1
V~obore celých čísel riešte sústavu rovníc
$$\align
(4x)_5+7y&=14,\\
(2y)_5-(3x)_7&=74,
\endalign
$$
kde $(n)_k$ značí násobok čísla~$k$ najbližší k~číslu~$n$.
}
\podpis{P. Černek}

{%%%%%   A-III-2
Uvažujme ľubovoľný rovnostranný trojuholník~$KLM$, ktorého vrcholy $K$, $L$ a~$M$ ležia postupne na stranách $AB$, $BC$ a~$CD$ daného štvorca~$ABCD$. Nájdite množinu stredov
strán všetkých takých trojuholníkov~$KLM$.}
\podpis{J. Zhouf}

{%%%%%   A-III-3
Dokážte, že prirodzené číslo~$A$ je druhou mocninou niektorého prirodzeného čísla
práve vtedy, keď pre každé prirodzené~$n$ je aspoň jeden z~rozdielov
$$
(A+1)^2-A,\,(A+2)^2-A,\,(A+3)^2-A,\,\dots,\,(A+n)^2-A
$$
deliteľný číslom~$n$.}
\podpis{P. Kaňovský}

{%%%%%   A-III-4
Nájdite všetky dvojice reálnych čísel $a$, $b$, pre ktoré má rovnica
$$
\postdisplaypenalty 10000
\frac{ax^2-24x+b}{x^2-1}=x
$$
v~obore reálnych čísel práve dve riešenia, pričom ich súčet je~$12$.
}
\podpis{P. Černek}

{%%%%%   A-III-5
V~rovine je daný trojuholník~$KLM$ a~bod~$A$ ležiaci na polpriamke opačnej k~polpriamke~$KL$.
Zostrojte pravouholník~$ABCD$, ktorého vrcholy $B$, $C$ a~$D$ ležia postupne na
priamkach $KM$, $KL$ a~$LM$.}
\podpis{P. Calábek}

{%%%%%   A-III-6
Nech $\Bbb R^+$ je množina všetkých kladných reálnych čísel. Nájdite všetky funkcie
$f:\Bbb R^{+}\to\Bbb R^+$ spĺňajúce pre ľubovoľné $x,y\in\Bbb R^+$ rovnosť
$$
f\bigl(xf(y)\bigr)=f(xy)+x.
$$}
\podpis{P. Kaňovský}

{%%%%%   B-S-1
Určte reálne číslo~$p$ tak, aby rovnica
$$
x^2+4px+5p^2+6p-16=0
$$
mala dva rôzne korene $x_1$, $x_2$ a~aby súčet $x_1^2 + x_2^2$ bol čo najmenší.}
\podpis{J. Šimša}

{%%%%%   B-S-2
Vnútri strán $BC$, $CA$, $AB$ daného ostrouhlého trojuholníka~$ABC$ sú po rade
vybrané body $X$, $Y$ a~$Z$ tak, že každému zo štvoruholníkov $ABXY$, $BCYZ$ a~$CAZX$
sa dá opísať kružnica. Dokážte, že body $X$, $Y$, $Z$ sú päty výšok trojuholníka~$ABC$.}
\podpis{E. Kováč}

{%%%%%   B-S-3
Na tabuli sú napísané čísla $1,2,...,17$. Čísla postupne zotierame, a~to tak,
že z~doposiaľ nezotretých čísel zvolíme ľubovoľné číslo~$k$ a~zotrieme všetky tie
čísla na tabuli, ktoré delia číslo $k+17$. Dokážte, že opakovaním tejto procedúry sa
nám nepodarí zotrieť všetky čísla.}
\podpis{J. Földes}

{%%%%%   B-II-1
Nájdite všetky prirodzené čísla~$n$, ktoré sú menšie ako $100$ a~majú tú vlastnosť,
že druhé mocniny čísel $7n+5$ a~$4n+3$ sa končia rovnakým dvojčíslím.}
\podpis{J. Šimša}

{%%%%%   B-II-2
V~obore reálnych čísel riešte sústavu rovníc
$$\aligned
(x^2+1)(y^2+1)+24xy&=0\\
\frac{12x}{x^2+1}+\frac{12y}{y^2+1}+1&=0.
\endaligned
$$
}
\podpis{J. Šimša}

{%%%%%   B-II-3
Vo vnútri strán $AB$, $BC$, $CD$ a~$DA$ konvexného štvoruholníka~$ABCD$ sú
postupne zvolené body $K$, $L$, $M$ a~$N$. Označme $S$ priesečník priamok $KM$ a~$LN$.
Ak je možné vpísať kružnice štvoruholníkom $AKSN$, $BLSK$, $CMSL$ a~$DNSM$, potom je
možné vpísať kružnicu aj štvoruholníku~$ABCD$. Dokážte.}
\podpis{J. Zhouf}

{%%%%%   B-II-4
Je daných $n$~nezáporných čísel. Môžeme vybrať ľubovoľné dve z~nich,
napríklad $a$ a~$b$, $a\le b$, a~zameniť ich číslami $0$ a~$b-a$. Dokážte, že opakovaním
tejto operácie je možné všetky dané čísla zmeniť na nuly práve vtedy, keď pôvodné čísla
je možné rozdeliť do dvoch skupín tak, že súčty čísel v~oboch skupinách sú rovnaké.}
\podpis{J. Földes}

{%%%%%   C-S-1
Do športového krúžku chodí 21~chlapcov. Na posledných dvoch schôdzkach nikto nechýbal,
chlapci sa zakaždým rozdelili do troch družstiev po sedem hráčov. Dokážte, že
niektorí traja chlapci boli na oboch schôdzkach spolu v~jednom družstve.}
\podpis{J. Šimša}

{%%%%%   C-S-2
V rovine je daný pravouhlý trojuholník~$ABC$ taký, že kružnica $k(A;|AC|)$ pretína
preponu~$AB$ v~jej strede~$S$. Dokážte, že kružnica opísaná trojuholníku~$BCS$
je zhodná s~kružnicou~$k$.}
\podpis{J. Švrček}

{%%%%%   C-S-3
Určte všetky dvojice prvočísiel~$(p,q)$ také, že $p>q$ a~číslo $p^2-q^2$ má najviac
štyroch deliteľov.}
\podpis{P. Calábek}

{%%%%%   C-II-1
Určte počet dvojíc~$(a,b)$ prirodzených čísel ($1\le a<b\le86$), pre ktoré je
súčin~$ab$ deliteľný tromi.}
\podpis{J. Zhouf}

{%%%%%   C-II-2
Nech kružnice zostrojené nad ramenami lichobežníka ako nad priemermi majú vonkajší dotyk.
Dokážte, že dotykový bod týchto kružníc leží na osi uhla, ktorý obe ramená
lichobežníka zvierajú.}
\podpis{J. Švrček}

{%%%%%   C-II-3
Nájdite všetky celé čísla~$x$, pre ktoré sú obe čísla $({x-3})^2-2$, $({x-7})^2+1$ prvočísla.}
\podpis{J. Šimša}

{%%%%%   C-II-4
V~rovine sú dané body $C$, $V$, $U$ také, že $|CV|=3\cm$, $|VU|=3{,}5\cm$ a~$|CU|=4{,}5\cm$.
Zostrojte ostrouhlý trojuholník~$ABC$ tak, aby bol $V$ priesečník jeho výšok a~bod~$U$ bol
súmerne združený s~bodom~$A$ podľa stredu kružnice opísanej trojuholníku~$ABC$.}
\podpis{P. Leischner}

{%%%%%   vyberko, den 1, priklad 1
Množinu troch nezáporných celých čísel $\{x,y,z\}$, $x<y<z$, nazývame {\it historickou\/},
ak $\{z-y,y-x\}=\{1776,2001\}$. Ukážte, že množinu nezáporných celých čísel môžeme napísať
ako zjednotenie po dvoch disjunktných historických množín.}
\podpis{Vladimír Marko:???}

{%%%%%   vyberko, den 1, priklad 2
Pre bod~$M$ vnútri trojuholníka~$ABC$ označme $A'$, $B'$ a~$C'$ po rade päty kolmíc spustených
z~bodu~$M$ na priamky $BC$, $CA$ a~$AB$. Definujme
$$
p(M)=\frac{|MA'|}{|MA|} \cdot \frac{|MB'|}{|MB|} \cdot \frac{|MC'|}{|MC|}.
$$
Nájdite bod~$M$, pre ktorý je $p(M)$ maximálne. Označme $\mu(ABC)$ toto maximum.
Pre ktoré trojuholníky~$ABC$ je hodnota~$\mu(ABC)$ najväčšia?}
\podpis{Vladimír Marko:???}

{%%%%%   vyberko, den 1, priklad 3
Nech $p\ge5$ je prvočíslo. Dokážte, že existuje prirodzené číslo~$a$, $1\le a\le p-2$,
také, že čísla $a^{p-1}-1$ a~$(a+1)^{p-1}-1$ nie sú deliteľné číslom~$p$.}
\podpis{Vladimír Marko:???}

{%%%%%   vyberko, den 1, priklad 4
Nájdite všetky postupnosti prirodzených čísel $a_1,\dots,a_n$, pre ktoré
$$
\frac{99}{100}=\frac{a_0}{a_1}+\frac{a_1}{a_2}+\dots+\frac{a_{n-1}}{a_n},
$$
kde $a_0=1$ a~$(a_{k+1}-1)a_{k-1}\ge a_k^2(a_k-1)$ pre $k=1,2,\dots,n-1$.}
\podpis{Vladimír Marko:???}

{%%%%%   vyberko, den 2, priklad 1
V~danom trojuholníku~$ABC$ platí $|\uh BAC|>|\uh BCA|$. Vo vnútri trojuholníka~$ABC$ je daný
bod~$P$ tak, aby $|\uh PAC|=|\uh BCA|$. Mimo trojuholníka~$ABC$ leží bod~$Q$ tak, aby
$PQ\parallel AB$ a~$BQ\parallel AC$. Nech $R$ je bod na strane~$BC$ ($R$~je od bodu~$Q$
oddelený priamkou~$AP$) taký, že $|\uh PRQ|=|\uh BCA|$. Dokážte, že kružnice opísané
trojuholníkom $ABC$ a~$PQR$ sa navzájom dotýkajú.}
\podpis{Tomáš Jurík:???}

{%%%%%   vyberko, den 2, priklad 2
Prirodzené čísla $a$, $b$, $c$ spĺňajú rovnosť
$$
\frac1a - \frac1b = \frac1c.
$$
Najväčšieho spoločného deliteľa čísel $a$, $b$, $c$ označme $h$. Dokážte, že čísla $habc$ aj
$h(b-a)$ sú druhými mocninami prirodzených čísel.}
\podpis{Tomáš Jurík:???}

{%%%%%   vyberko, den 2, priklad 3
Pre každé prirodzené číslo~$n$ dokážte nerovnosť
$$
\left( \frac{n+1}2 \right)^n \ge
(2n+1)^{n-1}\left( \frac{1\cdot2\cdot\dots\cdot n}{1\cdot3\cdot\dots\cdot(2n-1)} \right)^2.
$$}
\podpis{Tomáš Jurík:???}

{%%%%%   vyberko, den 2, priklad 4
Dve kružnice $k_1$ a~$k_2$ zo stredmi po rade v~bodoch $P$ a~$Q$ sa pretínajú v~bodoch
$S$ a~$T$. Spoločnú (vonkajšiu) dotyčnicu týchto kružníc, ktorá je bližšie k~bodu~$S$,
označíme~$p$ a~jej dotykové body s~kružnicami $k_1$, $k_2$ po rade $M$ a~$L$.
\ite (a) Druhý priesečník dotyčnice ku kružnici~$k_2$ v~bode~$S$ s~kružnicou~$k_1$
označíme~$K$. Priesečník priamok $SL$ a~$MK$ označíme~$U$. Dokážte, že priamky $MU$ a~$MS$
sú dotyčnicami kružnice opísanej trojuholníku~$STU$.
\ite (b) Označme $R$ priesečník osí strán $MS$ a~$SL$. Bod~$W$ je druhý priesečník
kružnice opísanej trojuholníku~$PQR$ s~priamkou~$RS$. Dokážte, že $S$ je stredom úsečky~$RW$.}
\podpis{Tomáš Jurík:???}

{%%%%%   vyberko, den 3, priklad 1
Šachovnica $n \times n$ je pokrytá neprekrývajúcimi sa štvorcami $2 \times 2$ a~$3 \times 3$.
Aké môže byť~n?}
\podpis{Martin Potočný:???}

{%%%%%   vyberko, den 3, priklad 2
Uhlopriečky $AC$ a~$BD$ konvexného štvoruholníka~$ABCD$ sa pretínajú v~bode~$E$.
Dokážte, že pre obsahy útvarov platí nerovnosť
$$
\sqrt{S_{ABE}} + \sqrt{S_{CDE}} \le \sqrt{S_{ABCD}}.
$$
Kedy nastáva rovnosť?}
\podpis{Martin Potočný:???}

{%%%%%   vyberko, den 3, priklad 3
Koľko je slov dĺžky~$n$ z~písmen $A$, $B$, $C$, ktoré spĺňajú obe nasledujúce podmienky?
\ite (i) Začínajú a~končia sa na~$A$;
\ite (ii) každé dve susedné písmená sú rôzne.}
\podpis{Martin Potočný:???}

{%%%%%   vyberko, den 3, priklad 4
Dané sú body $A_1$, $A_2$, $A_3$, $A_4$ na sfére opísanej pravidelnému štvorstenu
s~hranou dĺžky~$1$ také, že $|A_iA_j|<1$ pre každé $i\ne j$. Dokážte, že tieto štyri
body ležia na jednej hemisfére.}
\podpis{Martin Potočný:???}

{%%%%%   vyberko, den 4, priklad 1
Na ostrove žije $n$~domorodcov. Každí dvaja z~nich sú buď priatelia alebo nepriatelia.
Jedného dňa náčelník rozkázal všetkým obyvateľom (vrátane seba), aby si vyrobili a~nosili
náhrdelník z~mušličiek, pričom
\item{$\bullet$} ľubovoľní dvaja priatelia musia mať vo svojich náhrdelníkoch aspoň
jednu mušličku rovnakého druhu;
\item{$\bullet$} ľubovoľní dvaja nepriatelia musia mať vo svojich náhrdelníkoch
všetky mušličky rôzneho druhu.

(Je prípustný aj náhrdelník bez mušličiek.)
\ite (a) Ukážte, že domorodci mohli splniť náčelníkov rozkaz
\ite (b) Nájdite najmenší počet druhov mušličiek potrebných na to, aby domorodci
mohli určite splniť náčelníkov rozkaz.}
\podpis{František Kardoš:???}

{%%%%%   vyberko, den 4, priklad 2
Je daný trojuholník~$ABC$, v~ktorom $\beta<45^\circ$. Na strane~$BC$ leží bod~$D$ tak,
že stred kružnice vpísanej trojuholníku~$ABD$ je totožný so stredom~$O$ kružnice
opísanej trojuholníku~$ABC$. Nech~$l$ je kružnica opísaná trojuholníku~$AOC$. Označme
$P$ priesečník dotyčníc ku kružnici~$l$ v~bodoch~$A$ a~$C$, ďalej označme $Q$ priesečník
priamok $AD$ a~$CO$ a~napokon $X$ nech je priesečník priamky~$PQ$ a~dotyčnice ku~$l$
v~bode~$O$. Dokážte, že $|XO|=|XD|$.}
\podpis{František Kardoš:???}

{%%%%%   vyberko, den 4, priklad 3
Nech $A=\{a_1,a_2,\dots,a_n\}$ je množina $n$~navzájom rôznych celých čísel, $n\ge3$.
Označme $m$ a~$M$, najmenší a~najväčší prvok množiny~$A$. Predpokladajme, že existuje
taký polynóm~$p(x)$ s~celočíselnými koeficientmi, ktorý spĺňa
\ite (i) $\forall a\in A:\quad m<p(a)<M$;
\ite (ii) $\forall a\in A\setminus\{m,M\}:\quad p(m)<p(a)$.

Dokážte, že potom $n\le5$ a~existujú celé čísla $b$ a~$c$ také, že všetky prvky množiny~$A$
sú riešenia rovnice $p(x)+x^2+bx+c=0$.}
\podpis{František Kardoš:???}

{%%%%%   vyberko, den 4, priklad 4
Nech $x_1,x_2,\dots,x_n$ sú ľubovoľné reálne čísla. Dokážte nerovnosť
$$
\frac{x_1}{1+x_1^2} + \frac{x_2}{1+x_1^2+x_2^2} + \dots
+ \frac{x_n}{1+x_1^2+\dots+x_n^2} < \sqrt{n}.
$$}
\podpis{František Kardoš:???}

{%%%%%   vyberko, den 5, priklad 1
Nech $A=\{1,2,3,\dots,n\}$, kde $n$ je kladné celé číslo. Podmnožinu množiny~$A$ nazveme
{\it súvislou\/}, ak pozostáva z~jedného prvku alebo z~niekoľkých za sebou idúcich čísel.
Nájdite najväčšie celé číslo~$k$, pre ktoré $A$~obsahuje $k$~rôznych podmnožín
$A_1,\dots A_k$ takých, že prienik ľubovoľných dvoch množín $A_i$ a~$A_j$ je súvislá množina.}
\podpis{Juraj Földes:???}

{%%%%%   vyberko, den 5, priklad 2
Označme $ABC$ trojuholník s~ťažiskom~$G$. Nájdite, spolu s~dôkazom, polohu bodu~$P$ v~rovine
trojuholníka~$ABC$ tak, že výraz $|AP|\cdot|AG|+|BP|\cdot|BG|+|CP|\cdot|CG|$ má minimálnu
hodnotu. Vyjadrite toto minimum pomocou dĺžok strán trojuholníka~$ABC$.}
\podpis{Juraj Földes:???}

{%%%%%   vyberko, den 5, priklad 3
Nájdite všetky funkcie $f:\Bbb Z\to\Bbb Z$ spĺňajúce rovnosť
$$
f(m^2+f(n))=f^2(m)+n
$$
pre všetky celé čísla $m$ a~$n$.}
\podpis{Juraj Földes:???}

{%%%%%   vyberko, den 5, priklad 4
Je daná rovnica
$$
(p+2)x^2 - (p+1)y^2 + px + (p+2)y = 1,
$$
kde $p$ je dané prvočíslo tvaru $4k+3$. Dokážte, že platia nasledujúce tvrdenia.
\ite (a) Ak $(x_0,y_0)$ je riešenie rovnice, kde $x_0$ a~$y_0$ sú kladné celé čísla,
tak $p$ delí~$x_0$.
\ite (b) Daná rovnica má nekonečne veľa riešení tvaru~$(x_0,y_0)$, kde $x_0$, $y_0$ sú
kladné celé čísla.}
\podpis{Juraj Földes:???}

{%%%%%   trojstretnutie, priklad 1
Nech $a$, $b$ sú rôzne reálne čísla a~$k$, $m$
prirodzené čísla, pre ktoré platí $k+m=n\geq 3$, $k\leq 2m$ a~$m\leq
2k$. Uvažujme postupnosti $(x_1,x_2,\dots ,x_n)$, ktoré vyhovujú
nasledujúcim podmienkam:
\ite{} $k$~členov postupnosti sa rovná~$a$, pritom $x_1=a$;
\ite{} $m$~členov postupnosti sa rovná~$b$; pritom $x_n=b$;
\ite{} žiadne tri po sebe idúce členy nie sú rovnaké.

\noindent
Určte všetky možné hodnoty súčtu
$$
x_nx_1x_2+x_1x_2x_3+\dots +x_{n-2}x_{n-1}x_n+x_{n-1}x_nx_1.
$$}
\podpis{...}

{%%%%%   trojstretnutie, priklad 2
Daný je trojuholník~$ABC$, ktorého obsah je~$S$. Pre jeho dĺžky
strán $|BC|=a$, $|CA|=b$, $|AB|=c$ platí $a\leq b\leq c$. Určte
najväčšie reálne číslo~$u$ a~najmenšie reálne číslo~$v$ tak, aby
pre každý vnútorný bod~$P$ trojuholníka~$ABC$ bola splnená
nerovnosť
$$
u\leq |PD|+|PE|+|PF|\leq v,
$$
kde $D$, $E$, $F$ sú po rade priesečníky priamok $AP$, $BP$, $CP$
s~protiľahlými stranami daného trojuholníka.
(Hodnoty $u$, $v$ vyjadrite pomocou daných veličín $a$, $b$, $c$
a~$S$.)}
\podpis{...}

{%%%%%   trojstretnutie, priklad 3
Nech $\mm S=\{1,2,\dots,n\}$, kde $n$ je dané prirodzené číslo.
Určte počet všetkých funkcií $f:\mm S\to\mm S$ takých, že pre každé
$x\in\mm S$ platí $x+f^4(x)=n+1$.
(Symbol~$f^4$ označuje štvrtú iteráciu, t.\,j.~$f^4(x)=f(f(f(f(x))))$.)}
\podpis{...}

{%%%%%   trojstretnutie, priklad 4
Nech $n>1$ je prirodzené číslo a~$p$ prvočíslo také, že $n$ je
deliteľom čísla $p-1$ a~súčasne $p$~je deliteľom čísla $n^3-1$.
Dokážte, že $4p-3$ je druhou mocninou prirodzeného čísla.}
\podpis{...}

{%%%%%   trojstretnutie, priklad 5
Nech $O$ označuje stred kružnice opísanej ostrouhlému trojuholníku~$ABC$.
Body $P$, $Q$ nech sú po rade takými bodmi jeho strán
$AC$, $BC$, pre ktoré súčasne platí
$$
\frac{|AP|}{|PQ|}=\frac{|BC|}{|AB|} \qquad \text{a} \qquad
      \frac{|BQ|}{|PQ|}=\frac{|AC|}{|AB|}.
$$
Dokážte, že body $O$, $P$, $Q$ a~$C$ ležia na jednej kružnici.}
\podpis{...}

{%%%%%   trojstretnutie, priklad 6
Nech $n\geq 2$ je párne prirodzené číslo. Uvažujme polynómy tvaru
$$
P(x)=x^n+a_{n-1}x^{n-1}+\cdots +a_1x+1
$$
s~reálnymi koeficientmi, ktoré majú aspoň jeden reálny koreň.
Určte najmenšiu možnú hodnotu súčtu $a_1^2+\cdots +a_{n-1}^2$.}
\podpis{...}

{%%%%%   IMO, priklad 1
Nech $n$ je prirodzené číslo a~nech $\mm T$ je množina všetkých bodov~$(x,y)$
v~rovine, kde $x$ a~$y$ sú celé nezáporné čísla
a~$x+y<n$. Každý bod z~množiny~$\mm T$ je ofarbený buď červenou, alebo
modrou farbou. Ak je bod~$(x,y)$ červený, sú červené aj všetky body
$(x',y')\in\mm T$, pre ktoré platí $x'\leq x$ a~$y'\leq y$.
Definujme $X$-množinu ako množinu $n$~modrých bodov majúcich
rôzne $x$-ové súradnice a~$Y$-množinu ako množinu $n$~modrých
bodov majúcich rôzne $y$-ové súradnice. Dokážte, že počet
$X$-množín je rovný počtu $Y$-množín.}
\podpis{Kolumbia}

{%%%%%   IMO, priklad 2
Nech $BC$ je priemer kružnice~${\Gamma}$ so stredom~$O$. Bod~$A$
leží na kružnici~$\Gamma$ tak, že $0\st<|\uh AOB|<120\st$.
Nech $D$ je stred toho oblúka~$AB$, na ktorom neleží bod~$C$.
Priamka vedená bodom~$O$ rovnobežne s~$DA$ pretne priamku~$AC$
v~bode~$J$. Os úsečky~$OA$ pretne kružnicu~$\Gamma$ v~bodoch $E$
a~$F$. Dokážte, že bod~$J$ je stred kružnice vpísanej trojuholníku~$CEF$.}
\podpis{Južná Kórea}

{%%%%%   IMO, priklad 3
Nájdite všetky dvojice prirodzených čísel $m,n\geq3$ také, že
existuje nekonečne veľa prirodzených čísel~$a$, pre ktoré je
$$
a^m+a-1\over a^n+a^2-1
$$
celé číslo.}
\podpis{Rumunsko}

{%%%%%   IMO, priklad 4
Nech $n$ je prirodzené číslo väčšie ako~$1$. Všetky kladné delitele
čísla~$n$ označíme $d_1,d_2,\ldots,d_k$, kde
$$
1=d_1<d_2<\dots<d_k=n.
$$
Položme $D=d_1d_2+d_2d_3+\dots+d_{k-1}d_k$.
\ite(a) Dokážte, že $D<n^2$.
\ite(b) Určte všetky~$n$, pre ktoré je číslo~$D$ deliteľom
        čísla~$n^2$.}
\podpis{Rumunsko}

{%%%%%   IMO, priklad 5
Nech $\Bbb R$ označuje množinu všetkých reálnych čísel. Nájdite všetky
funkcie $f:\Bbb R\to\Bbb R$ také, že
$$
\bigl(f(x)+f(z)\bigr)\bigl(f(y)+f(t)\bigr)=f(xy-zt)+f(xt+yz)
$$
pre ľubovoľné $x,y,z,t\in\Bbb R$.}
\podpis{India}

{%%%%%   IMO, priklad 6
V~rovine sú dané kružnice $\Gamma_1,\Gamma_2,\dots,\Gamma_n$
s~polomerom~$1$, kde $n\geq3$. Ich stredy označme po rade
$O_1,O_2,\dots,O_n$. Predpokladajme, že každá priamka pretína
najviac dve z~daných kružníc. Dokážte, že
$$
\sum_{1\leq i<j\leq n}{1\over |O_iO_j|}\leq{(n-1)\pi\over4}.
$$}
\podpis{Ukrajina}

