{%%%%%   A-I-1
Číslo $1\,997^{2^{n}}-1$ je deliteľné číslom $2^{n+2}$ pre
každé prirodzené číslo $n$. Dokážte.}
\podpis{P. Kaňovský}

{%%%%%   A-I-2
Daný je ľubovoľný trojuholník $ABC$. Os vnútorného uhla $BAC$
pretne stranu~$BC$ v~bode, ktorý označíme~$U$.
Dokážte rovnosť
$$
|AU|^2=|AB|\cdot|AC|-|BU|\cdot|CU|.
$$
Môže táto rovnosť platiť, ak nahradíme bod~$U$ iným vnútorným
bodom strany~$BC$?}
\podpis{J. Šimša}

{%%%%%   A-I-3
V~istom jazyku sú len dve písmena $A$ a~$B$. Pre
slová tohto jazyka platia nasledujúce pravidlá:
\item{1)} Jediné slovo dĺžky $1$ je $A$.
\item{2)} Ľubovoľná skupina písmen $X_1X_2X_3\dots X_nX_{n+1}$, kde
$X_i=\{A,B\}$ pre~každý index $i$, tvorí slovo dĺžky $n+1$, práve keď
obsahuje aspoň jedno písmeno $A$, a~pritom nie je tvaru
$X_1X_2\dots X_nA$, kde $X_1X_2\dots X_n$ je slovo dĺžky $n$.

Nájdite
\item{a)} všetky slová dĺžky $4$,
\item{b)} vzorec pre počet $p_n$ všetkých slov dĺžky $n$.
}
\podpis{J. Zhouf}

{%%%%%   A-I-4
Daný štvorsten $ABCD$ má
zhodné protiľahlé hrany: $|AB|=|CD|=p$, $|AC|=|BD|=q$
a~$|AD|=|BC|=r$. Označme $K$ stred hrany $AB$ a~$L$ stred hrany
$CD$.
\item{a)}
Dokážte, že priamka~$KL$ je kolmá na obidve hrany $AB$ a~$CD$.
\item{b)}
Ukážte, že najmenšia možná hodnota súčtu
$|AE|^2+|EF|^2+|FC|^2$, kde $E$ a~$F$ sú ľubovoľné body priečky
$KL$, je rovná $\dfrac{2p^2+q^2+r^2}{6}$.}
\podpis{P. Leischner}

{%%%%%   A-I-5
Guľôčky siedmich rôznych farieb sú rozdelené do siedmich
vrecúšok tak, že v~každých dvoch vrecúškach môžeme nájsť po jednej
guľôčke tej istej farby. Dokážte:
\item{a)} Guľôčky niektorej farby sú zastúpené v~aspoň troch vrecúškach.
\item{b)} Pokiaľ boli rozdelené z každej farby len 3~ guľôčky, tak v~žiadnom
vrecúšku nenájdeme dve guľôčky tej istej farby.

Rozhodnite tiež, či je takéto rozdelenie guľôčok pri
podmienke z~tvrdenia~ b) vôbec možné.}
\podpis{P. Hliněný}

{%%%%%   A-I-6
Daný je pravouhlý lichobežník so základňami $a$, $c$
($a>c$) a~dlhším ramenom~$b$. Zostrojte priamku, ktorá daný
lichobežník rozdeľuje na dva navzájom podobné štvoruholníky.
Preveďte diskusiu o~počte riešení vzhľadom na~dĺžky $a$, $b$, $c$.}
\podpis{J. Švrček}

{%%%%%   B-I-1
Magický štvorec je štvorcová tabuľka prirodzených čísel, v~ktorej je
súčet všetkých čísel v~každom riadku, v~každom stĺpci aj~na oboch
uhlopriečkach rovnaký. Nájdite všetky magické štvorce $3\times3$,
pre ktoré je súčin štyroch čísel v~rohových poliach rovný $3\,465$.}
\podpis{P.Černek}

{%%%%%   B-I-2
V~rovine je daná priamka~$q$ a~bod $A$, ktorý na nej neleží.
Určte v~tejto rovine množinu stredov~$S$ všetkých štvorcov~$ABCD$
takých, že bod~$B$ leží na priamke~$q$.}
\podpis{J. Molnár}

{%%%%%   B-I-3
Dokážte, že pre každú trojicu $x$, $y$, $z$ kladných
čísel platí nerovnosť
$$
\sqrt{xyz}\left(\frac{2}{x+y}+\frac{2}{y+z}+\frac{2}{z+x}\right)
\le \sqrt{x}+\sqrt{y}+\sqrt{z}.
$$
Zistite, kedy nastane rovnosť.}
\podpis{J. Švrček}

{%%%%%   B-I-4
Daný je štvorsten, v~ktorom sú každé dve protiľahlé hrany zhodné.
Vo~vnútri štvorstena existuje bod $M$, ktorý je rovnako vzdialený od
všetkých jeho stien. Dokážte, že každá výška daného štvorstena je
rovná štvornásobku vzdialenosti bodu $M$ od jeho stien.}
\podpis{P. Leischner}

{%%%%%   B-I-5
V~obore reálnych čísel riešte sústavu rovníc
$$
\aligned
x+2y+3z&=x^2-z^2+p,\\
y+2z+3x&=y^2-x^2+p,\\
z+2x+3y&=z^2-y^2+p,\\
\endaligned
$$
kde $p$ je reálny parameter. Preveďte diskusiu o~počte riešení
vzhľadom na~parameter $p$.}
\podpis{J. Šimša}

{%%%%%   B-I-6
Aký najväčší obsah môže mať konvexný štvoruholník, v~ktorom obidve
úsečky spájajúce stredy protiľahlých strán sú zhodné a~majú
danú dĺžku~ $d$?}
\podpis{J. Zhouf}

{%%%%%   C-I-1
Pre ľubovoľné trojciferné číslo určíme jeho
zvyšky po delení číslami $2, 3, 4, \dots, 10$ a~získaných
deväť čísel potom sčítame. Zistite najmenšiu možnú hodnotu takéhoto
súčtu.}
\podpis{J. Šimša}

{%%%%%   C-I-2
Nájdite všetky trojuholníky $ABC$, pre~ktoré platí
$a+v_a=b+v_b$ pri zvyčajnom označení strán a~výšok trojuholníka.}
\podpis{P. Černek}

{%%%%%   C-I-3
Sto detí sa rozdelilo do troch družstiev $A$, $B$ a~$C$.
Potom, čo jedno dieťa prestúpilo z~$A$ do $B$, jedno z~$B$ do $C$
a~jedno z~$C$ do $A$, sa priemerná hmotnosť detí zvýšila
v~družstve $A$ o~$120$\,g, v~družstve $B$ o~$130$\,g, zatiaľ čo
v~družstve $C$ sa znížila o~$240$\,g. Koľko detí bolo
v~jednotlivých družstvách?}
\podpis{P. Černek}

{%%%%%   C-I-4
Vo vnútri daného pravouhlého rovnoramenného
trojuholníka $ABC$ s~preponou~$AB$ zvolíme ľubovoľne bod~$X$.
Ďalej zostrojíme priamky $p$ a~$q$,
ktoré prechádzajú bodom~$X$ tak, že $p\parallel AB$ a~$q\perp AB$.
Trojuholník $ABC$ vytína na priamke~$p$ úsečku~$KL$, na
priamke~$q$ úsečku $MN$. Určte všetky body~$X$, pre ktoré
platí $|KL|=2\cdot|MN|$.}
\podpis{J. Šimša}

{%%%%%   C-I-5
Riešte sústavu
$$
\align
7\lfloor x\rfloor+ \phantom[2y\phantom]=& 117{,}4,\\
5x\phantom]  + 2\lfloor y\rfloor=& \phantom{1}91{,}9,
\endalign
$$
kde $\lfloor a\rfloor$ je tzv. dolná celá časť reálneho čísla~$a$, \tj. celé číslo,
pre ktoré platí $\lfloor a\rfloor\le a<\lfloor a\rfloor+1$. Napríklad $\lfloor3{,}7\rfloor=3$
a~$\lfloor\m3{,}7\rfloor=\m4$.
}
\podpis{P. Černek}

{%%%%%   C-I-6
Zostrojte deltoid so stranami $12\cm$ a~$13\cm$, ktorý
je svojimi uhlopriečkami rozdelený na štyri trojuholníky, ktoré sú
štyrmi stenami nejakého štvorstena. Zhotovte papierový model tohoto
štvorstena.}
\podpis{S. Bednářová, P. Černek}

{%%%%%   A-S-1
Nájdite všetky trojuholníky $ABC$, pre ktoré platí
rovnosť
$$
|BC|\cdot|AX|=|AC|\cdot|BY|,
$$
kde bod $X$ je priesečníkom osi uhla $BAC$ so stranou $BC$ a~bod~
$Y$ priesečníkom osi uhla $ABC$ so stranou~$AC$.}
\podpis{P. Černek}

{%%%%%   A-S-2
V~istom jazyku sú len dva znaky $A$ a~$B$. Prípustné sú
v~ňom len také  slová, v~ktorých nestoja vedľa seba viac ako dva
rovnaké znaky. Dokážte, že počty $p_n$ všetkých prípustných slov
dĺžky $n$ možno určiť pomocou rovností $p_1=2$, $p_2=4$
a~$p_{k+2}=p_{k+1}+p_k$ pre každé prirodzené číslo~ $k$.}
\podpis{J. Zhouf}

{%%%%%   A-S-3
Dokážte, že všetky riešenia sústavy rovníc
$$
\align
ux+vy=&x^2+xy,\\
vx+uy=&y^2+xy,\\
xy+uv=&u^2+v^2
\endalign
$$
v~obore nenulových reálnych čísel majú tvar $x=y=u=v$.}
\podpis{J. Šimša}

{%%%%%   A-II-1
Číslo $1\,997^{3^n}+1$ je deliteľné číslom~ $3^{n+3}$ pre každé
prirodzené číslo~$n$. Dokážte.}
\podpis{J. Šimša}

{%%%%%   A-II-2
V~jednom rade je postavených $n$ stĺpcov dámových kameňov tak, že
medzi každými dvoma stĺpcami rovnakej výšky sa nachádza stĺpec vyšší.
(Všetky kamene majú rovnakú výšku, niektoré stĺpce môžu byť
tvorené aj~jedným kameňom.) Najvyšší stĺpec obsahuje $k$~ kameňov.
Pre dané $k$ určte najväčšiu možnú hodnotu~$n$.}
\podpis{J. Kratochvíl}

{%%%%%   A-II-3
V~obore reálnych čísel riešte sústavu rovníc
$$
%%\align
x^2-yz=a, \qquad
y^2-zx=a-1, \qquad
z^2-xy=a+1
%%\endalign
$$
s~reálnym parametrom~$a$. Preveďte diskusiu o~počte riešení.
}
\podpis{P. Černek}

{%%%%%   A-II-4
V~rovine, v~ktorej je daná úsečka~$BD$, nájdite množinu všetkých
vrcholov~$A$ konvexných štvoruholníkov $ABCD$, pre ktoré súčasne
platí:
\item{a)} stred~$O_C$ kružnice vpísanej trojuholníku $BCD$ leží na kružnici
opísanej trojuholníku $ABD$,
\item{b)} stred~$O_A$ kružnice vpísanej trojuholníku $ABD$ leží na kružnici
opísanej trojuholníku $BCD$.
}
\podpis{J. Zhouf}

{%%%%%   A-III-1
V~obore kladných reálnych čísel riešte rovnicu
$$
x\cdot\left\lfloor x\cdot\bigl\lfloor x\cdot\lfloor x\rfloor\bigr\rfloor\right\rfloor=88,
$$
kde $\lfloor a\rfloor$ je dolná celá časť reálneho čísla $a$, \tj. celé číslo,
pre ktoré platí $\lfloor a\rfloor\le a<\lfloor a\rfloor+1$. Napríklad $\lfloor3{,}7\rfloor=3$
a~$\lfloor6\rfloor=6$.}
\podpis{J. Šimša}

{%%%%%   A-III-2
Dokážte, že z~množiny ľubovoľných štrnástich rôznych prirodzených čísel
možno
pre niektoré číslo~ $k$ ($1\le k\le7$) vybrať dve disjunktné $k$-prvkové
podmnožiny $\{a_1,a_2,\dots,a_k\}$ a~$\{b_1,b_2,\dots,b_k\}$
tak, aby sa súčty
$$
A=\frac1{a_1}+\frac1{a_2}+\dots+\frac1{a_k}\quad\text{a}\quad
B=\frac1{b_1}+\frac1{b_2}+\dots+\frac1{b_k}
$$
navzájom líšili o~menej ako $0{,}001$, \tj. aby platilo
$|A-B|<0{,}001$.
}
\podpis{J. Šimša}

{%%%%%   A-III-3
Do daného štvorstena $ABCD$ je vpísaná guľa. Jej štyri dotykové
roviny, ktoré sú so stenami štvorstena rovnobežné,
z~neho odtínajú štyri menšie štvorsteny. Dokážte, že súčet dĺžok
všetkých 24 ich hrán je rovný dvojnásobku súčtu dĺžok hrán celého
štvorstena $ABCD$.}
\podpis{P. Leischner}

{%%%%%   A-III-4
Do výrazu
$$
\text{deň}^{\text{mesiac}}-\text{rok}
$$
dosadzujeme
ľubovoľný dátum roku 1998 a~potom zisťujeme najväčšiu
mocninu čísla~$3$, ktorá delí výsledné číslo. Napr. pre 21.~apríl
vychádza číslo $21^4-1\,998=192\,483=3^3\cdot7\,129$, čo je násobok
mocniny $3^3$, nie však mocniny $3^4$. Nájdite všetky dni, pre
ktoré je zodpovedajúca mocnina najväčšia.}
\podpis{R. Kollár}

{%%%%%   A-III-5
Vo vonkajšej oblasti kružnice~$k$ je daný bod~$A$. Všetky
lichobežníky, ktoré sú do kružnice~$k$ vpísané tak, že ich
predĺžené ramená sa pretínajú v~bode~$A$, majú spoločný
priesečník uhlopriečok. Dokážte.}
\podpis{P. Leischner}

{%%%%%   A-III-6
Nech $a$, $b$, $c$ sú kladné čísla. Dokážte, že trojuholník
so~stranami $a$, $b$, $c$ existuje práve vtedy, keď má sústava rovníc
$$
\frac yz+\frac zy=\frac ax,  \quad
\frac zx+\frac xz=\frac by,  \quad
\frac xy+\frac yx=\frac cz
$$
riešenie v~obore reálnych čísel.}
\podpis{P. Černek, J. Zhouf}

{%%%%%   B-S-1
Určte všetky trojice $(a,b,c)$ reálnych čísel, pre ktoré platí
$$
  \align
   a+b+c&=1,\\
   ab+bc+ca&=abc.
  \endalign
$$
}
\podpis{J. Švrček}

{%%%%%   B-S-2
Nech obidve úsečky spájajúce stredy protiľahlých strán konvexného
štvoruholníka $ABCD$ majú rovnakú dĺžku. Dokážte, že uhlopriečky
$AC$ a~$BD$ sú navzájom kolmé a~že platí rovnosť
$$
|AB|^2+|C\!D|^2=|BC|^2+|DA|^2.
$$
}
\podpis{J. Šimša}

{%%%%%   B-S-3
Nájdite všetky štvorcové tabuľky $3 \times 3$ prirodzených čísel,
v~ktorých je súčin všetkých čísel v~každom riadku, v~každom stĺpci
aj~na oboch uhlopriečkach rovnaký, a~pre ktoré platí, že súčet štyroch čísel
v~ich rohových poliach je jednociferné číslo.}
\podpis{J. Těšínský}

{%%%%%   B-II-1
V~obore reálnych čísel riešte sústavu rovníc
$$
\align
xy&=ax+ay,\\
xz&=2x+2z,\\
yz&=3y+3z,
\endalign
$$
kde $a$ je reálny parameter. Preveďte diskusiu o~počte riešení
vzhľadom na parameter~$a$.
}
\podpis{J. Zhouf}

{%%%%%   B-II-2
Popíšte konštrukciu trojuholníka $ABC$, v~ktorom pri zvyčajnom označení
platí $t_a=9\cm$, $t_b=12\cm$ a~$3c=2t_c$.}
\podpis{P. Černek}

{%%%%%   B-II-3
Daná je štvorcová tabuľka $3\times3$ prirodzených čísel,
v~ktorej je súčin všetkých čísel v~každom riadku, v~každom stĺpci aj~na
oboch uhlopriečkach rovný číslu~ $s$.
\item{a)} Dokážte, že číslo $s$ je treťou mocninou prirodzeného čísla.
\item{b)} Pokiaľ je jedno z~rohových čísel tabuľky rovné~1, je súčet
všetkých štyroch rohových čísel druhou mocninou prirodzeného čísla.
Dokážte.}
\podpis{J. Tešínský}

{%%%%%   B-II-4
V~danom ostrouhlom trojuholníku $ABC$ označme $A_1$, $B_1$ päty výšok
z~vrcholov $A$, $B$. Určte veľkosti jeho vnútorných uhlov pri
vrcholoch $B$ a~$C$, ak je veľkosť uhla $BAC$ rovná $40\st$
a~ak sú polomery vpísaných kružníc trojuholníkom $A_1B_1C$ a~$ABC$
v~pomere $1:2$.}
\podpis{P.Leischner}

{%%%%%   C-S-1
V~obore reálnych čísel riešte rovnicu
$$
\lfloor3x-5\rfloor=5x-8,
$$
kde $\lfloor a\rfloor$ je dolná celá časť reálneho čísla~$a$, \tj. celé číslo,
pre ktoré platí $\lfloor a\rfloor\le a<\lfloor a\rfloor+1$. Napríklad $\lfloor3{,}7\rfloor=3$
a~$\lfloor\m3{,}7\rfloor=\m4$.}
\podpis{P. Černek}

{%%%%%   C-S-2
Nájdite najmenšie trojciferné číslo, ktoré je deliteľné
práve polovicou z~čísel
$$
2,\ 3,\ 4,\ 6,\ 8,\ 9,\ 12,\ 16,\ 18,\ 24,\ 27,\ 36.
$$
}
\podpis{P. Černek}

{%%%%%   C-S-3
Daný je rovnoramenný pravouhlý trojuholník $ABC$ s~preponou~$AB$. Označme
$P$ stred jeho výšky z~vrcholu~$C$, $M$~priesečník priamky~$AP$
s~odvesnou~$BC$ a~$N$ priesečník priamky~$BP$ s~odvesnou~$AC$.
Dokážte, že pravouholník $KLMN$, ktorého strana~$KL$ leží na
prepone~$AB$, je štvorec.}
\podpis{J. Šimša}

{%%%%%   C-II-1
Z~troch rôznych nenulových cifier sme zostavili všetkých šesť možných
trojciferných čísel. Tieto čísla sme zoradili od najväčšieho po
najmenšie. Zistili sme, že štvrté číslo v~tomto rade je aritmetickým
priemerom prvého a~piateho čísla. Z~ktorých cifier
boli čísla zostavené? Zistite všetky možnosti.}
\podpis{J. Zhouf}

{%%%%%   C-II-2
Daný je rovnoramenný pravouhlý trojuholník $ABC$ s~preponou~$AB$.
Určte množinu všetkých bodov~$X$ tohto trojuholníka
s~nasledujúcou vlastnosťou:
Ak vedieme bodom~$X$ priamku rovnobežnú s~$AB$ a~priamku kolmú na
$AB$, vytne na nich trojuholník $ABC$ dve zhodné úsečky.}
\podpis{J. Šimša}

{%%%%%   C-II-3
Nájdite všetky kladné čísla $x$, pre ktoré je medzi desiatimi číslami
$$
\lfloor x\rfloor,\ \lfloor2x\rfloor,\ \lfloor3x\rfloor,\ \lfloor4x\rfloor,\ \lfloor5x\rfloor,\ 
\lfloor6x\rfloor,\ \lfloor7x\rfloor,\ \lfloor8x\rfloor,\ \lfloor9x\rfloor,\ \lfloor10x\rfloor
$$
práve deväť rôznych.
Symbol $\lfloor a\rfloor$ je dolná celá časť reálneho čísla~$a$, \tj. celé číslo,
pre ktoré platí $\lfloor a\rfloor\le a<\lfloor a\rfloor+1$. Napríklad $\lfloor3{,}7\rfloor=3$ a~$\lfloor4\rfloor=4$.}
\podpis{J. Šimša}

{%%%%%   C-II-4
Nájdite všetky lichobežníky $ABCD$ so základňami $AB$ a~$CD$,
pre ktoré platí:
$|AB|=6\cm$, $|CD|=4\cm$
a
$$
|BC|+d_A=|AD|+d_B=|AB|+v,
$$
kde $v$ označuje výšku lichobežníka, $d_A$ vzdialenosť bodu~$A$ od
priamky~$BC$ a~$d_B$ vzdialenosť bodu~$B$ od priamky~$AD$.}
\podpis{P. Černek}

{%%%%%   vyberko, den 1, priklad 1
...}
\podpis{...}

{%%%%%   vyberko, den 1, priklad 2
...}
\podpis{...}

{%%%%%   vyberko, den 1, priklad 3
...}
\podpis{...}

{%%%%%   vyberko, den 1, priklad 4
...}
\podpis{...}

{%%%%%   vyberko, den 2, priklad 1
...}
\podpis{...}

{%%%%%   vyberko, den 2, priklad 2
...}
\podpis{...}

{%%%%%   vyberko, den 2, priklad 3
...}
\podpis{...}

{%%%%%   vyberko, den 2, priklad 4
...}
\podpis{...}

{%%%%%   vyberko, den 3, priklad 1
...}
\podpis{...}

{%%%%%   vyberko, den 3, priklad 2
...}
\podpis{...}

{%%%%%   vyberko, den 3, priklad 3
...}
\podpis{...}

{%%%%%   vyberko, den 3, priklad 4
...}
\podpis{...}

{%%%%%   vyberko, den 4, priklad 1
...}
\podpis{...}

{%%%%%   vyberko, den 4, priklad 2
...}
\podpis{...}

{%%%%%   vyberko, den 4, priklad 3
...}
\podpis{...}

{%%%%%   vyberko, den 4, priklad 4
...}
\podpis{...}

{%%%%%   vyberko, den 5, priklad 1
...}
\podpis{...}

{%%%%%   vyberko, den 5, priklad 2
...}
\podpis{...}

{%%%%%   vyberko, den 5, priklad 3
...}
\podpis{...}

{%%%%%   vyberko, den 5, priklad 4
...}
\podpis{...}

{%%%%%   trojstretnutie, priklad 1
...}
\podpis{...}

{%%%%%   trojstretnutie, priklad 2
...}
\podpis{...}

{%%%%%   trojstretnutie, priklad 3
...}
\podpis{...}

{%%%%%   trojstretnutie, priklad 4
...}
\podpis{...}

{%%%%%   trojstretnutie, priklad 5
...}
\podpis{...}

{%%%%%   trojstretnutie, priklad 6
...}
\podpis{...}

{%%%%%   IMO, priklad 1
...}
\podpis{...}

{%%%%%   IMO, priklad 2
...}
\podpis{...}

{%%%%%   IMO, priklad 3
...}
\podpis{...}

{%%%%%   IMO, priklad 4
...}
\podpis{...}

{%%%%%   IMO, priklad 5
...}
\podpis{...}

{%%%%%   IMO, priklad 6
...}
\podpis{...}

{%%%%%   MEMO, priklad 1
}
\podpis{}

{%%%%%   MEMO, priklad 2
}
\podpis{}

{%%%%%   MEMO, priklad 3
}
\podpis{}

{%%%%%   MEMO, priklad 4
}
\podpis{}

{%%%%%   MEMO, priklad t1
}
\podpis{}

{%%%%%   MEMO, priklad t2
}
\podpis{}

{%%%%%   MEMO, priklad t3
}
\podpis{}

{%%%%%   MEMO, priklad t4
}
\podpis{}

{%%%%%   MEMO, priklad t5
}
\podpis{}

{%%%%%   MEMO, priklad t6
}
\podpis{}

{%%%%%   MEMO, priklad t7
}
\podpis{}

{%%%%%   MEMO, priklad t8
}
\podpis{}
