{%%%%%   A-I-1
Označme $k=1\,997$ a~všimnime si, že pre každé~$n$ platí
$$
k^{2^{n+1}}-1=\bigl(k^{2^{n}}\bigr)^2-1^2=
\bigl(k^{2^{n}}-1\bigr)\bigl(k^{2^{n}}+1\bigr).
\tag 1$$
To nám umožní dokazovať uvedené tvrdenie indukciou.
Začneme s~hodnotou $n=0$, lebo číslo $k-1$ je deliteľné
číslom $2^2$. Pretože číslo $k^{2^{n}}+1$ je pre každé~$n$ párne,
vyplýva z~(1), že pokiaľ číslo $k^{2^{n}}-1$ je deliteľné
číslom~$2^{n+2}$, je číslo $k^{2^{n+1}}-1$ deliteľné čís\-lom~
$2\cdot2^{n+2}$, teda číslom $2^{n+3}$. Tým je dôkaz indukciou
ukončený. Dodajme, že namiesto (1) je možné podobne využiť aj rovnosti
$$
\bigl(k^{2^{n}}-1\bigr)^2=k^{2^{n+1}}-2\cdot k^{2^{n}}+1=
\bigl(k^{2^{n+1}}-1\bigr)-2\bigl(k^{2^{n}}-1\bigr).
$$

\medskip
{\bf Iné riešenie.}
Namiesto matematickej indukcie môžeme využiť binomickú vetu
a~dokázať, že  pre každé celé číslo $k$ je rozdiel $(4k+1)^{2^n}-1$
deliteľný číslom~ $2^{n+2}$. (Odtiaľ voľbou $k=499$
dostaneme tvrdenie úlohy.) Z~binomickej vety pre exponent $2^n$
vyplýva rozklad
$$
\aligned
(4k+1)^{2^n}-1=&(4k)^{2^n}+\binom{2^n}{1}(4k)^{2^n-1}+\dots+\\
&+\binom{2^n}{j}(4k)^{2^n-j}+\dots+\binom{2^n}{2^n-1}4k.
\endaligned
\tag 2
$$

Prvý sčítanec napravo je deliteľný mocninou $2^{2^{n+1}}$,
a~teda aj~mocninou $2^{n+2}$, lebo $n+2\leqq2^{n+1}$ pre každé celé
$n\geqq0$ (ľahká indukcia). Teraz pre každé $j\in\{1,2,\dots,2^n-1\}$
zistíme, akou mocninou čísla~ 2 je deliteľné kombinačné
číslo~$\binom{2^n}{j}$. Na~to využijeme vyjadrenie
$$
\binom{2^n}{j}=\frac{2^n}{j}\cdot\frac{2^n-1}{1}\cdot
\frac{2^n-2}{2}\cdot\frac{2^n-3}{3}\cdot\dots\cdot\frac{2^n-j+1}{j-1},
\tag 3
$$
ktoré je výhodné preto, že čísla $i$ a~$2^n-i$
($1\leqq i\leqq2^n-1$) majú vo svojich
rozkladoch na~prvočinitele tú istú mocninu čísla 2.
Ak je preto $j=2^{\alpha}\ell$, kde
$0\leqq\alpha\leqq n-1$ a~$\ell$ je nepárne, je podľa (3) uvažované
číslo $\binom{2^n}{j}$ nepárnym násobkom mocniny $2^{n-\alpha}$.
Odtiaľ vyplýva, že číslom $2^{n+2}$ je deliteľný každý sčítanec na
pravej strane (2) práve vtedy, keď pre každý uvažovaný index $j$ platí
nerovnosť $n+2\leqq(n-\alpha)+2(2^n-j)$, alebo
$\alpha+2\leqq2(2^n-j)$. Pretože $\alpha+2\leqq2^{\alpha+1}$
(ľahká indukcia), stačí nám dokázať silnejšie nerovnosti
$2^{\alpha}\leqq2^n-j$. Tie ale vyplývajú z~definície čísel
$\alpha=\alpha(j)$: pretože moc\-ni\-na~ $2^{\alpha}$ delí číslo $j$,
delí aj~číslo $2^n-j$, takže ho neprevyšuje.
}

{%%%%%   A-I-2
...}

{%%%%%   A-I-3
a) Postupne nájdeme všetky slová dĺžok 2, 3 a~4.
Z~možných skupín dĺžky 2 slovami nie sú práve $BB$ (neobsahuje
žiadne $A$) a~$AA$ (je neprípustného tvaru). Takže slová dĺžky 2 sú práve
$AB$ a~$BA$. Z~nich vytvoríme neprípustné tvary dĺžky~ 3: $ABA$
a~$BAA$. Ostatné skupiny dĺžky 3 (s~výnimkou $BBB$), teda slová sú:
$AAA$, $AAB$, $ABB$, $BAB$, $BBA$. Ak pripíšeme na ich
koniec vždy písmeno $A$, dostaneme päť neprípustných tvarov dĺžky~ 4:
$AAAA$, $AABA$, $ABBA$, $BABA$, $BBAA$. Ostatné skupiny dĺžky~ 4
(s~výnimkou $BBBB$) sú teda hľadané slová: $AAAB$, $AABB$,
$ABAA$, $ABAB$, $ABBB$, $BAAA$, $BAAB$, $BABB$, $BBAB$, $BBBA$.

b) Podľa časti a) vieme, že $p_1=1$, $p_2=2$, $p_3=5$ a~$p_4=10$.
Teraz pre každé $n\geqq1$ vyjadríme počet $p_{n+1}$ pomocou počtu
$p_n$. Rozdelíme teda všetky slová dĺžky $n+1$  do~ dvoch
skupín podľa toho, ktorým písmenom končia: tých, ktoré sú tvaru
$\dots A$, je práve $2^n-p_n$ (pred~posledným $A$ stojí ľubovoľná
skupina dĺžky $n$, ktorá nie je slovo); tých, ktoré sú tvaru
$\dots B$, je práve $2^n-1$ (pred~pos\-led\-ným~$B$ stojí ľubovoľná
skupina dĺžky $n$ okrem $BB\dots B$). Preto platí
$p_{n+1}=(2^n-p_n)+(2^n-1)=2^{n+1}-1-p_n$. Z~ vyjadrení
pre $p_{n+1}$ a~$p_{n+2}$ vyplýva:
$p_{n+2}=2^{n+2}-1-p_{n+1}=2^{n+2}-1-(2^{n+1}-1-p_n)=
p_n+2^{n+1}$. Z~toho budeme určovať hodnoty $p_n$
oddelene pre párne a~pre~nepárne indexy~ $n$:
$$\align
p_{2k-1}=&p_{2k-3}+2^{2k-2}=p_{2k-5}+2^{2k-4}+2^{2k-2}=\dots=\\
=&p_1+2^2+2^4+\dots+2^{2k-4}+2^{2k-2}=\\
=&1+4+4^2+\dots+4^{k-2}+4^{k-1}=\frac{4^{k}-1}{3}, \\
\endalign
$$
$$\align
p_{2k}=&p_{2k-2}+2^{2k-1}=p_{2k-4}+2^{2k-3}+2^{2k-1}=\dots=\\
=&p_2+2^3+2^5+\dots+2^{2k-3}+2^{2k-1}=\\
=&2+2\cdot4^1+2\cdot4^2+\dots+2\cdot4^{k-2}+2\cdot4^{k-1}=
\frac{2\cdot(4^{k}-1)}{3} .
\endalign
$$
Nájdené vzorce platia pre každé $k\geqq1$. Dodajme, že ich môžeme
zapísať jednotným spôsobom
$$
p_{n}=\frac{2^{n+2}+(-1)^{n+1}-3}{6}\qquad(n=1,2,3,4,\dots)  .
$$

\medskip
{\bf Iné riešenie.}
Odlišný postup riešenia založíme na priamom (t\.j\.~ ne\-re\-ku\-rent\-nom)
popise slov daného jazyka. Nie je ťažké sa dovtípiť, že rozhodnutie
o~tom, či je daná skupina písmen~$A$, $B$
slovom, podstatne závisí na parite počtu písmen~$A$, ktorými sa
táto skupina končí.
Najprv je jasné, že skupina končiaca písmenom $B$ je slovo, práve
keď obsahuje {\it aspoň jedno\/} písmeno $A$. Skupina
$\underbrace{AA\dots A}_{\text{$q$-krát}}$
(neobsahujúca žiadne písmeno~ $B$) je zrejme slovo, práve keď je
počet $q$ jeho písmen {\it nepárny}. "Zostáva" rozhodnúť o~skupinách
$$
X_1X_2\dots X_rB\underbrace{AA\dots A}_{\text{$q$-krát}},
$$
kde $q$ a~$r$ sú ľubovoľné počty (nevylučujeme, že $r=0$).
Zapísaná skupina je slovo, práve keď buď $q$ je {\it nepárne\/}
a~$X_1X_2\dots X_rB$ slovo {\it nie je\/} (teda $X_i=B$ pre každé $i$),
alebo $q$ je {\it párne\/} a~$X_1X_2\dots X_rB$ slovo {\it je\/}
(čiže $r\geqq1$ a~$X_i=A$ pre niektoré $i$). Dôkaz poslednej
vety sa jednoducho prevedie indukciou podľa čísla $q$.

Získaný popis umožňuje priamo nielen vypísať všetky slová
danej dĺžky $n$, ale tiež určiť ich počet $p_n$, a~to pomocou
rozdelenia slov do skupín podľa počtu~$q$ písmen~$A$, ktorými
jednotlivé slová končia ($0\leqq q\leqq n$).
V~každej takejto skupine sú všetky slová tvaru
$$
X_1X_2\dots X_{n-q}\underbrace{AA\dots A}_{\text{$q$-krát}},
$$
kde v~prípade $q<n$ nutne $X_{n-q}=B$.
Pre dané nepárne $q$ je tohoto tvaru jediné slovo (odpovedajúce
počiatočnej skupine $X_1X_2\dots X_{n-q}=BB\dots B)$,
pre dané párne $q\leqq n-2$ je takýchto slov práve $2^{n-q-1}-1$
(lebo potom $X_{n-q}=B$ a~$X_1X_2\dots X_{n-q-1}$ je ľubovoľná
z~$2^{n-q-1}-1$ skupín písmen obsahujúcich aspoň jedno $A$),
napokon pre párne $q\in\{n-1,n\}$ také slovo neexistuje (takže
vyjadrenie $2^{n-q-1}-1$ platí aj~pre $q=n-1$). Ak spočítame tieto počty
pre~$q\in\{0,1,\dots,n\}$, dostaneme v~prípade nepárneho $n=2k-1$
$$
\align
p_{2k-1}&=\underbrace{(2^{2k-2}-1)}_{q=0}+\underbrace{1\strut}_{q=1}
+\underbrace{(2^{2k-4}-1)}_{q=2}+\underbrace{\strut1}_{q=3}+\dots
+\underbrace{(2^0-1)}_{q=2k-2}+
\underbrace{\strut1}_{q=2k-1}=\\
&=4^{k-1}+4^{k-2}+\dots+4+1=\frac{4^{k}-1}{3},
\endalign
$$
zatiaľ čo pre párne $n=2k$ nám vyjde
$$\align
p_{2k}&=\underbrace{(2^{2k-1}-1)}_{q=0}+\!\underbrace{\strut1}_{q=1}\!
+\underbrace{(2^{2k-3}-1)}_{q=2}+\!\underbrace{\strut1}_{q=3}\!+\dots
+\underbrace{(2^1-1)}_{q=2k-2}+\!\underbrace{\strut1}_{q=2k-1}\!+
\!\underbrace{\strut0}_{q=2k}\!=\\
&=2\cdot(4^{k-1}+4^{k-2}+\dots+4+1)=\frac{2\cdot(4^{k}-1)}{3}.
\endalign
$$
}

{%%%%%   A-I-4
...}

{%%%%%   A-I-5
...}

{%%%%%   A-I-6
...}

{%%%%%   B-I-1
...}

{%%%%%   B-I-2
Označme $P$ pätu kolmice $p$ na priamku $q$, ktorá prechádza bodom~
$A$. S~každým štvorcom daných vlastností možno uvažovať štvorec,
ktorý je s~ním symetrický podľa priamky~ $p$ a~rovnako vyhovuje
podmienkam úlohy. Odtiaľ vyplýva, že vyšetrovaná množina stredov $S$
všetkých štvorcov~$ABC\!D$ daných vlastností je rovinný útvar, ktorý je
symetrický podľa priamky~$p$. Uvažujme teraz taký štvorec $ABC\!D$
so~zvyčajným označením jeho vrcholov, kde $B \in q$.

\smallskip

\noindent
Ďalej rozlíšime tri prípady polohy jeho vrcholov $C$, $D$ v~rovine:

a) obidva vrcholy $C$, $D$ ležia v~polrovine $qA$ (obr\. \obrnum),

b) len vrchol $D$ leží v~polrovine $qA$ (obr\. \obrnum),

c) žiaden z~vrcholov $C$, $D$ neleží v~polrovine $qA$ (obr\. \obrnum).

%\midinsert
\vskip12cm
%\centerline{\inspicture-!\hss\inspicture-!}
%\centerline{\inspicture-!\hss\inspicture-!}
%\endinsert

\medbreak
V~prípade a) označme $K$, $L$ po rade päty kolmíc zo stredu $S$
uvažovaného štvorca $ABC\!D$ na priamky~ $p$, $q$. Vzhľadom k~tomu,
že platí $|\angle ASB|=|\angle KSL|=90^{\circ}$, je pravouhlý
trojuholník $BSL$ obrazom pravouhlého trojuholníka $ASK$
v~otočení so~ stredom $S$ o~uhol~ $90^{\circ}$. Preto platí
$|SK|=|SL|$. Rovnaký záver zdôvodníme podobne aj~v~prípadoch~ b)
a~c) (obr\. \obrrnum2 ~a~\obrrnum1). Stred~ $S$ uvažovaného štvorca $ABC\!D$
teda vždy leží na osi $o$ uhla $KPL$.

Naopak ku každému bodu $S$ priamky $o$ možno ľahko zostrojiť štvorec
$ABC\!D$, ktorého stredom je bod $S$ a~ktorého vrchol $B$ leží na
priamke~ $q$.

{\it Záver\/}: Hľadanou množinou bodov je dvojica navzájom
kolmých priamok~ $o$ a~$o'$, ktoré prechádzajú bodom~ $P$ a~zvierajú
s~priamkami $p$ a~$q$ uhol $45^{\circ}$ (obr\. \obrnum).

\medskip
\newpage
{\bf Iné riešenie.} Využijeme  vlastnosti obvodových uhlov, a~to
pre všetky tri vyššie popísané prípady. Aj~tu ukážeme riešenie len
pre prípad~z~obr\. \obrrnum4.

\vskip7cm

Vzhľadom k~tomu, že platí $|\angle ASB|=|\angle AP\!B|=90^{\circ}$,
môžeme štvoruholníku $AP\!BS$ opísať kružnicu. Odtiaľ na základe
vety o~obvodových uhloch dostávame $|\angle APS|=|\angle
ABS|=45^{\circ}$. Stred $S$ uvažovaného štvoruholníka~$ABC\!D$ leží
teda na osi uhla $AP\!B$.
}

{%%%%%   B-I-3
...}

{%%%%%   B-I-4
...}

{%%%%%   B-I-5
Sčítaním všetkých troch rovníc dostaneme
$$
6(x+y+z)=3p,\qquad\text{t\.j\.}\qquad
x+y+z=\frac{p}2.
$$
 Ak dosadíme túto hodnotu do ľavých strán jednotlivých rovníc
sústavy upravených podľa vzoru
$$
a+2b+3c=2(a+b+c)+c-a,
$$
dostaneme po úprave nasledujúcu sústavu rovníc (už bez parametra~$p$):
$$
z-x=x^2-z^2 ,\qquad x-y=y^2-x^2,  \qquad y-z=z^2-y^2.
$$

Z~prvej rovnice ľahko zistíme, že platí $z-x=0$ alebo $z+x+1=0$.
Podobne z~druhej, resp.~ tretej rovnice dostávame $x-y=0$ alebo
$x+y+1=0$, resp.~ $y-z=0$ alebo $y+z+1=0$. Pretože daná sústava
rovníc sa nemení cyklickou permutáciou neznámych $x$, $y$, $z$,
stačí rozlíšiť dva prípady:

(i) $x=y=z$.  Z~počiatočnej sústavy potom ľahko určíme, že jediným
    riešením je trojica
$$
(x,y,z)=\left(\frac{p}6,\frac{p}6,\frac{p}6\right).
$$

(ii) $x=y$ a~$z=-x-1$.  Pre také trojice neznámych je daná
    sústava ekvivalentná s~jedinou rovnicou $2x=p+2$ (o~tom sa
    ľahko presvedčíme dosadením), takže platí
$$
x=y=1+\frac{p}2 \qquad \hbox{a} \qquad z=-2-\frac{p}2.
$$

Permutovaním trojice
    $(x,y,z)$ v~prípade~(i) žiadne ďalšie riešenie nedostaneme,
v~prípade~(ii) sú výsledkom riešenia
 $$
\left(1+\frac{p}2,\ 1+\frac{p}2,\ -2-\frac{p}2\right),\
   \left(-2-\frac{p}2,\ 1+\frac{p}2,\ 1+\frac{p}2\right),\
   \left(1+\frac{p}2,\ -2-\frac{p}2,\ 1+\frac{p}2\right).
$$

    Riešenie danej sústavy je jediné, práve keď
$$\frac{p}6=1+\frac{p}2=-2-\frac{p}2.
$$
To nastane len pre $p=-3$. Pre každé iné~ $p$ má sústava
práve štyri riešenia, lebo žiadne dve z~čísel $\frac16p$,
$1+\frac12p$, $-2-\frac12p$ sa pre $p\ne-3$ nerovnajú.
}

{%%%%%   B-I-6
...}

{%%%%%   C-I-1
...}

{%%%%%   C-I-2
...}

{%%%%%   C-I-3
...}

{%%%%%   C-I-4
...}

{%%%%%   C-I-5
...}

{%%%%%   C-I-6
...}

{%%%%%   A-S-1
...}

{%%%%%   A-S-2
...}

{%%%%%   A-S-3
Sčítaním prvých dvoch rovníc dostaneme
$$
(u+v)(x+y)=(x+y)^2,
$$
odkiaľ vyplýva
$$x+y=0
\qquad \text{alebo} \qquad
x+y=u+v.
$$

V~prvom prípade po dosadení $y=-x$ do prvej rovnice dostaneme $x(u-v)=0$,
odkiaľ $u=v$ (lebo podľa zadania $x\ne0$).
Po dosadení $y=-x$ a~$u=v$
do~tretej rovnice dostaneme po ľahkej úprave rovnicu
$$x^2+u^2=0,$$
ktorá v~našom obore nemá riešenie.

V druhom prípade (keď $x+y=u+v$) je pravá strana prvej
rovnice rovná $x(x+y)=x(u+v)$, to je $ux+vx$.
Porovnaním
s~ľavou stranou $ux+vy$  preto (s~ohľadom na~$v\ne0$) vychádza $x=y$.
V~tomto prípade teda
$$x=y=\frac12(u+v),$$
po dosadení do~tretej rovnice
$$
\Bigl(\dfrac{u+v}{2}\Bigr)^2+uv=u^2+v^2,
$$
vyjde po úprave $(u-v)^2=0$. Z toho $u=v$, teda $x=y=u=v$.
Tým je celý dôkaz ukončený.

{\it Poznámka}: Ani v~obore {\it všetkých\/} %(teda aj~nulových)
reálnych čísel (vrátane nuly)
iné riešenia ako $x=y=u=v$ neexistujú. Dokážeme to,
keď predchádzajúci postup doplníme o~rozbor dvoch situácií:

a) $x+y=0$ a~$x=0$. Zrejme $y=0$ a z~tretej rovnice sústavy, ktorá
má teraz tvar
$$uv=u^2+v^2,$$
už vyplýva $u=v=0$.

b) $x+y=u+v$ a~$v=0$. Zrejme $x+y=u$ a~tretia rovnica sústavy má
teraz tvar
$$xy=u^2,$$
takže $xy=(x+y)^2$. Odtiaľ už vyplýva $x=y=0$,
a~preto aj~$u=0$.
}

{%%%%%   A-II-1
Tvrdenie dokážeme indukciou podľa čísla~ $n$. Môžeme
začať od hodnoty $n=0$: číslo $1\,997^{3^0}+1$ je skutočne násobkom
čísla $3^3$ ($1\,998=27\cdot74$). Ak platí podľa indukčného
predpokladu rovnosť $1\,997^{3^n}+1=3^{n+3}k_n$ pre~vhodné prirodzené
číslo $k_n$, dostaneme zo vzorca $A^3+B^3=(A+B)^3-3AB(A+B)$ pre
hodnoty $A=1\,997^{3^n}$ a~$B=1$ nasledujúce vyjadrenie:
$$
1\,997^{3^{n+1}}+1=\bigl(3^{n+3}k_n\bigr)^3-
3\cdot1\,997^{3^n}\cdot(3^{n+3}k_n)=
3^{n+4}\bigl(3^{2n+5}k_n^3-1\,997^{3^n}k_n\bigr).
$$
Tým je dôkaz ukončený.

Dodajme, že pri druhom indukčnom kroku bolo tiež možné
využiť rozklad
$$
x^{3^{n+1}}+1=\bigl(x^{3^n}\bigr)^3+1^3=\bigl(x^{3^n}+1\bigr)
                              \bigl(x^{2\cdot3^n}-x^{3^n}+1\bigr)
$$
a~vysvetliť, prečo pre $x=1\,997$ je druhý činiteľ deliteľný tromi:
čísla $1\,997^{2\cdot3^n}$ a~$1\,997^{3^n}$ totiž po delení tromi
dávajú po rade zvyšky $1$ a~$2$.
}

{%%%%%   A-II-2
...}

{%%%%%   A-II-3
Ak odčítame od prvej rovnice druhú, a~potom
od tretej rovnice prvú, dostaneme rovnosti
$$
(x-y)(x+y+z)=1\quad\text{\ \ a\ \ }\quad (z-x)(x+y+z)=1.
$$
Vyplýva z~nich, že čísla $x-y$ a~$z-x$ sú, rovnako ako $x+y+z$,
nutne rôzne od nuly a~obe sa rovnajú číslu $s=(x+y+z)^{-1}$.
Vyjadrenie $y=x-s$ a~$z=x+s$ dosadíme do pôvodnej sústavy:
$$\aligned
x^2-(x+s)(x-s)&=a,\\
(x-s)^2-x(x+s)&=a-1,\\
(x+s)^2-x(x-s)&=a+1.
\endaligned
\tag {$*$}
$$
Ľahko zistíme, že táto sústava je ekvivalentná s~dvojicou rovníc
$s^2=a$ a~$3xs=1$, z~ktorých vyplýva podmienka riešiteľnosti
$a>0$ (lebo $s\ne0$), ale aj vyjadrenie $s=\pm\sqrt{a}$
a~$x=\dfrac{1}{3s}=\pm\dfrac{1}{3\sqrt{a}}$, takže
$y=x-s=\pm\dfrac{1-3a}{3\sqrt{a}}$
a~$z=x+s=\pm\dfrac{1+3a}{3\sqrt{a}}$ (vo všetkých vzorcoch platí
vždy rovnaké znamienko). Pretože sústava ($*$) bola riešená
ekvivalentnými úpravami, nie je potrebné prevádzať skúšku.

{\it Odpoveď\/}: Pre $a\leqq0$ sústava nemá žiadne riešenie, pre $a>0$
existujú práve dve riešenia $(x,y,z)$, a~to trojice
$$
\left(\frac{1}{3\sqrt{a}},\frac{1-3a}{3\sqrt{a}},
\frac{1+3a}{3\sqrt{a}}\right)\quad\text{a}\quad
\left(\frac{-1}{3\sqrt{a}},\frac{-1+3a}{3\sqrt{a}},
\frac{-1-3a}{3\sqrt{a}}\right).
$$

\medskip
{\bf Iné riešenie.} Členy na ľavých stranách rovníc eliminujeme
tak, že sčítame {$y$-násobok} prvej rovnice so $z$-násobkom
druhej a~$x$-násobkom tretej rovnice. Dostaneme tak lineárnu
rovnicu $0=ay+(a-1)z+(a+1)x$. Podobne sčítaním $z$-násobku
prvej rovnice s~$x$-násobkom druhej a~$y$-násobkom
tretej rovnice dostaneme $0=az+(a-1)x+(a+1)y$. Zo získaných rovníc
$$
(a+1)x+ay+(a-1)z=0,\qquad (a-1)x+(a+1)y+az=0
$$
eliminujeme najprv premennú $z$ (odčítaním $(a-1)$-násobku
druhej od $a$-násobku prvej rovnice), výsledkom je
vyjadrenie $y=(1-3a)x$; potom podobnou elimináciou premennej~ $y$
dospejeme k~rovnosti $z=(1+3a)x$. Ak dosadíme tieto vyjadrenia~ $y$
a~$z$ do pôvodných rovníc, dostaneme sústavu
$$
\align
9a^2x^2=&a,\\
9a(a+1)x^2=&a+1,\\
9a(a-1)x^2=&a-1.
\endalign
$$
Tá je ekvivalentná (bez ohľadu na hodnotu parametra $a$)
s~jedinou rovnicou $9ax^2=1$. Tak dostávame podmienku riešiteľnosti
$a>0$ a~vzorce pre obidve riešenia
$$
x=\pm\frac{1}{3\sqrt{a}},\quad\
y=(1-3a)x=\pm\frac{1-3a}{3\sqrt{a}},\quad\
z=(1+3a)x=\pm\frac{1+3a}{3\sqrt{a}}.
$$
}

{%%%%%   A-II-4
Označme $\alpha=|\angle BAD|$
a~$\gamma=|\angle BC\!D|$. Platí
$$
\align
|\angle BO_C\!D|=&180^{\circ}-(|\angle O_CBD|+|\angle O_C\!DB|)=\\
=&180^{\circ}-\frac12(|\angle CBD|+|\angle C\!DB|)=
90^{\circ}+\frac12\gamma,
\endalign
$$
podobne $|\angle BO_AD|=90^{\circ}+\frac12\alpha$. Pretože body
$A$ a~$O_C$ ležia v~opačných polrovinách s~hraničnou priamkou $BD$,
je podmienka a) úlohy ekvivalentná s~tým, že súčet veľkostí uhlov
$BAD$ a~$BO_C\!D$ je $180^{\circ}$, t\.j\.~
$\alpha+(90^{\circ}+\frac12\gamma)=180^{\circ}$. Podobne
usúdime, že podmienka b) je splnená, práve keď
$\gamma+(90^{\circ}+\frac12\alpha)=180^{\circ}$. Nájdená dvojica
rovníc má jediné riešenie $\alpha=\gamma=60^{\circ}$. Preto body
$A$ a~$C$ ležia každý na inom z~dvoch kruhových oblúkov,
z~ktorých je úsečku $BD$ vidieť pod uhlom $60^{\circ}$.
Na druhej strane, ak si zvolíme ľubovoľný vnútorný bod $A$ jedného
z~týchto oblúkov, možno na druhom oblúku vybrať bod $C$ tak,
aby $ABC\!D$ bol {\it konvexný\/} štvoruholník (stačí napríklad
trojuholník $ABD$ doplniť na~rovnobežník $ABC\!D$).

{\it Odpoveď\/}: Hľadanú množinu vrcholov~ $A$ tvoria vnútorné body
dvoch kruhových oblúkov, z ktorých je úsečku $BD$ vidieť pod
uhlom $60^{\circ}$.
}

{%%%%%   A-III-1
...}

{%%%%%   A-III-2
...}

{%%%%%   A-III-3
\fontplace
\tpoint B; \tpoint C; \lBpoint D; \bpoint A;
\rBpoint K; \lBpoint L; \lbpoint M;
[19 {\rm\quad}] \hfil \Obr

Označme $\varrho$ polomer vpísanej gule a~$v_A$, $v_B$,
$v_C$, $v_D$ telesové výšky daného štvorstena (s~indexmi podľa
vrcholov, z ktorých vychádzajú). Odťatý štvor\-sten $AKLM$~ (\obr)
je rovnoľahlý podľa stredu~ $A$ s~celým štvorstenom
$ABC\!D$. Súčty dĺžok ich hrán sú preto v~ rovnakom pomere
ako ich telesové výšky zo~ spoločného vrcholu $A$, teda
v~pomere $(v_A-2\varrho):v_A$, lebo $2\varrho$ je vzdialenosť rovín $KLM$
a~$BC\!D$ (sú totiž rovnobežné a~obidve sa dotýkajú vpísanej gule).

\inspicture r(1)

Rovnakú úvahu môžeme zopakovať pre~ zvyšné  tri odťaté štvorsteny.
Našou úlohou je preto dokázať rovnosť
$$
\frac{v_A-2\varrho}{v_A}+\frac{v_B-2\varrho}{v_B}+
\frac{v_C-2\varrho}{v_C}+\frac{v_D-2\varrho}{v_D}=2,
$$
ktorá je ekvivalentná s~rovnosťou
$$
\varrho\Bigl(\frac{1}{v_A}+\frac{1}{v_B}+
\frac{1}{v_C}+\frac{1}{v_D}\Bigr)=1.
$$
Na~to nám poslúži nasledujúca úvaha o~objeme $V$
a~povrchu $S$ štvorstena $ABC\!D$. Najprv $S=S_A+S_B+S_C+S_D$
(kde $S_X$ je obsah tej steny, ktorá neobsahuje vrchol~$X$), ďalej
$$
V=\frac13 S_Av_A=\frac13 S_Bv_B=\frac13 S_Cv_C=\frac13 S_Dv_D
$$
a~konečne $V=\dfrac13\varrho S$. Podľa týchto vzorcov platí
$$
\varrho\Bigl(\frac{1}{v_A}+\frac{1}{v_B}+
\frac{1}{v_C}+\frac{1}{v_D}\Bigr)=\frac{3V}{S}
\Bigl(\frac{S_A}{3V}+\frac{S_B}{3V}+
\frac{S_C}{3V}+\frac{S_D}{3V}\Bigr)=1
$$
a~tým je dôkaz ukončený.
}

{%%%%%   A-III-4
...}

{%%%%%   A-III-5
...}

{%%%%%   A-III-6
Pravá strana prvej rovnice má rovnaké znamienko ako
neznáma $x$  ($\ne0$), ľavá strana ako súčin $yz$ ($\ne0$).
Ľubovoľné riešenie $(x,y,z)$ danej sústavy preto spĺňa podmienku
$$
xyz>0.                        \tag 1
$$
Ako vieme, kladné čísla $a$, $b$, $c$ tvoria strany niektorého
trojuholníka práve vtedy, keď je kladné každé z~ troch čísel
$$
a+b-c,\quad\ a+c-b,\quad\ b+c-a.      \tag 2
$$
Ak je $(x,y,z)$ riešenie danej sústavy, tak
$$
a+b-c=x\Bigl(\frac yz+\frac zy\Bigr)+
y\Bigl(\frac zx+\frac xz\Bigr)
-z\Bigl(\frac xy+\frac yx\Bigr)=\frac{2xy}{z},
$$
čo je podľa (1) kladné číslo. Analogicky zistíme, že
$$
a+c-b=\frac{2xz}{y}>0\qquad\text{a}\qquad
b+c-a=\frac{2yz}{x}>0.
$$

V~druhej časti riešenia naopak predpokladajme, že každé z~čísel~ (2)
je kladné a~nájdime {\it všetky\/} riešenia danej sústavy (aj~keď
by stačilo uviesť {\it jedno\/} riešenie). Pomôžu nám pri tom
predchádzajúce výpočty, podľa ktorých musí napríklad platiť
$$
(a+b-c)(a+c-b)=\frac{2xy}{z}\cdot\frac{2xz}{y}=4x^2.
$$
Táto a~ďalšie dve analogické rovnosti vedú k~vyjadreniu
$$
\left.\aligned
x&=\frac{\ep_1}{2}\sqrt{(a+b-c)(a+c-b)}\\
y&=\frac{\ep_2}{2}\sqrt{(a+b-c)(b+c-a)}\\
z&=\frac{\ep_3}{2}\sqrt{(a+c-b)(b+c-a)}
\endaligned\,\right\},                            \tag 3
$$
kde $\ep_i=\pm1$ pre $i\in\{1,2,3\}$, pritom $\ep_1\ep_2\ep_3=1$
podľa (1). Také trojice $(\ep_1,\ep_2,\ep_3)$ sú zrejme
práve štyri. Podľa (3) tak dostávame štyri trojice $(x,y,z)$.
Priamym dosadením a~rutinným výpočtom overíme, že sú to
skutočne riešenia zadanej sústavy.
}

{%%%%%   B-S-1
Druhú rovnicu upravíme na tvar
$
(1-c)ab+(a+b)c=0.
$
Podľa prvej rovnice je však $a+b=1-c$, takže odtiaľ
dostávame podmienku
$
(1-c)(ab+c)=0.
$
Pretože pôvodná sústava bola symetrická vzhľadom k~neznámym $a$,
$b$, $c$, pokúsime sa ešte upraviť činiteľ $ab+c$. Pomocou prvej
rovnice tak dostaneme
$$
ab+c=ab+(1-a-b)=a(b-1)+(1-b)=(1-a)(1-b),
$$
takže
$$
(1-c)(ab+c)=(1-a)(1-b)(1-c)=0.
$$
Odtiaľ vyplýva, že niektoré z~čísel $a$, $b$,
$c$ je nutne rovné jednej, ostatné dve z~nich sú potom (z rovnosti
$a+b+c=1$) čísla navzájom opačné. Trojica $(a,b,c)$ má teda
jeden z~tvarov
$$
(1,k,\m k),\ (k,1,\m k),\ (k,\m k,1),
$$
kde $k$ je vhodné číslo. Dosadením sa ľahko presvedčíme, že sú
to skutočne riešenia, a~to pre ľubovoľné reálne číslo~ $k$.

\medskip
{\bf Riešenie 2.} Použijeme štandardný postup pre riešenie sústav
rovníc. Z~prvej rovnice vyjadríme napríklad "neznámu" $c=1-a-b$
a~dosadíme do druhej rovnice, ktorú potom budeme riešiť vzhľadom
na~"neznámu"~ $b$ (považujúc $a$ za "parameter"). Dostaneme tak po
rutinných úpravách kvadratickú rovnicu
$$
(a-1)b^2+(a^2-2a+1)b+(a-a^2)=0.
$$
Jej koeficienty, ako ľahko vidíme, majú spoločný činiteľ
$a-1$, takže rovnicu pred riešením ešte upravíme:
$$
(a-1)\bigl[b^2+(a-1)b-a\bigr]=0.
$$
Pretože korene trojčlena v~hranatých zátvorkách sú $b_1=1$
a~$b_2=-a$, musí nastať jeden z~prípadov $a=1$, $b=1$, alebo $b=-a$.
Záver je rovnaký ako v~prvom riešení.

\medskip
{\bf Riešenie 3.} V~oboch rovniciach vystupujú výrazy, ktoré, ako
vieme, súvisia s~koe\-fi\-cien\-tami mnohočlena $P(x)=(x-a)(x-b)(x-c)$.
Tak zistíme, že ak sú obidve rovnice splnené, má mnohočlen $P(x)$
tvar $x^3-x^2+px-p$, kde $p=abc$. Vtedy platí $P(1)=1-1+p-p=0$,
takže číslo 1 musí byť jedným z~koreňov $a$, $b$, $c$ mnohočlena
$P(x)$. Záver je rovnaký ako v~prvom riešení.
}

{%%%%%   B-S-2
...}

{%%%%%   B-S-3
...}

{%%%%%   B-II-1
Z~druhej a~tretej rovnice, ktoré upravíme do~tvaru
$$
x(z-2)=2z, \qquad y(z-3)=3z,
$$
vyplýva podmienka $z\notin\{2,3\}$ a~vyjadrenie
$$
x={2z\over z-2}, \qquad y={3z\over z-3} . \tag1
$$
Dosadením do prvej rovnice sústavy dostaneme
$$
{6z^2\over (z-2)(z-3)}={2az\over z-2}+{3az\over z-3}
$$
a~po úprave
$$
z\bigl((6-5a)z+12a\bigr)=0.
$$

Odtiaľ vyplýva, že je buď $z=0$ (potom $x=y=0$), alebo (za predpokladu
$a\ne\frac65$)
$$
z={12a\over5a-6},
$$
odkiaľ podľa (1) dostávame riešenie
$$
x={12a\over a+6},\quad
y={12a\over 6-a},\quad
z={12a\over5a-6}.                          \tag2
$$
Pritom podmienka $z\notin\{2,3\}$ je ekvivalentná s podmienkou
$a\notin\{-6,6\}$. Naviac si všimnime, že pre $a=0$ dáva (2)
riešenie $x=y=z=0$. Toto riešenie zrejme sústave vyhovuje pre
ľubovoľné reálne~ $a$.

{\it Záver}. Pre $a\in\{-6,0,\frac65,6\}$ má sústava jediné
riešenie $x=y=z=0$, pre ostatné reálne~ $a$ má sústava naviac
aj~nenulové riešenie (2).

\medskip
{\bf Iné riešenie.}
Trojica $x=y=z=0$ je riešením danej sústavy. Z~druhej a~tretej
rovnice vyplýva, že pokiaľ je jedno z~čísel $x$, $y$, $z$ rovné
nule, sú nulové aj~ostatné dve. Preto ďalej predpokladajme, že
$xyz\ne0$. Potom nutne aj~$a\ne0$. Sústavu prepíšeme do~tvaru
$$
\frac1a=\frac1x+\frac1y,\qquad
\frac12=\frac1x+\frac1z,\qquad
\frac13=\frac1y+\frac1z       .
$$
To je sústava lineárnych rovníc vzhľadom k~neznámym $\dfrac1x$,
$\dfrac1y$, $\dfrac1z$. Ľahko nájdeme jej (jediné) riešenie
$$
\frac1x=\frac1{12}+\frac1{2a},\quad
\frac1y=-\frac1{12}+\frac1{2a},\quad
\frac1z=\frac5{12}-\frac1{2a}.
$$
Odtiaľ určíme trojicu
$$
x={12a\over a+6},\quad
y={12a\over 6-a},\quad
z={12a\over5a-6},                          \tag3
$$
ktorá je riešením, pokiaľ $a\notin\{-6,6,\frac65\}$.

{\it Záver}. Pre $a\in\{-6,0,\frac65,6\}$ má sústava jediné
riešenie $x=y=z=0$; pre ostatné hodnoty~ $a$ má sústava
aj~druhé riešenie (3).
}

{%%%%%   B-II-2
...}

{%%%%%   B-II-3
...}

{%%%%%   B-II-4
...}

{%%%%%   C-S-1
...}

{%%%%%   C-S-2
...}

{%%%%%   C-S-3
...}

{%%%%%   C-II-1
...}

{%%%%%   C-II-2
...}

{%%%%%   C-II-3
...}

{%%%%%   C-II-4
...}

{%%%%%   vyberko, den 1, priklad 1
...}

{%%%%%   vyberko, den 1, priklad 2
...}

{%%%%%   vyberko, den 1, priklad 3
...}

{%%%%%   vyberko, den 1, priklad 4
...}

{%%%%%   vyberko, den 2, priklad 1
...}

{%%%%%   vyberko, den 2, priklad 2
...}

{%%%%%   vyberko, den 2, priklad 3
...}

{%%%%%   vyberko, den 2, priklad 4
...}

{%%%%%   vyberko, den 3, priklad 1
...}

{%%%%%   vyberko, den 3, priklad 2
...}

{%%%%%   vyberko, den 3, priklad 3
...}

{%%%%%   vyberko, den 3, priklad 4
...}

{%%%%%   vyberko, den 4, priklad 1
...}

{%%%%%   vyberko, den 4, priklad 2
...}

{%%%%%   vyberko, den 4, priklad 3
...}

{%%%%%   vyberko, den 4, priklad 4
...}

{%%%%%   vyberko, den 5, priklad 1
...}

{%%%%%   vyberko, den 5, priklad 2
...}

{%%%%%   vyberko, den 5, priklad 3
...}

{%%%%%   vyberko, den 5, priklad 4
...}

{%%%%%   trojstretnutie, priklad 1
...}

{%%%%%   trojstretnutie, priklad 2
...}

{%%%%%   trojstretnutie, priklad 3
...}

{%%%%%   trojstretnutie, priklad 4
...}

{%%%%%   trojstretnutie, priklad 5
...}

{%%%%%   trojstretnutie, priklad 6
...}

{%%%%%   IMO, priklad 1
...}

{%%%%%   IMO, priklad 2
...}

{%%%%%   IMO, priklad 3
...}

{%%%%%   IMO, priklad 4
...}

{%%%%%   IMO, priklad 5
...}

{%%%%%   IMO, priklad 6
...}

{%%%%%   MEMO, priklad 1
...}

{%%%%%   MEMO, priklad 2
...}

{%%%%%   MEMO, priklad 3
...}

{%%%%%   MEMO, priklad 4
...}

{%%%%%   MEMO, priklad t1
...}

{%%%%%   MEMO, priklad t2
...}

{%%%%%   MEMO, priklad t3
...}

{%%%%%   MEMO, priklad t4
...}

{%%%%%   MEMO, priklad t5
...}

{%%%%%   MEMO, priklad t6
...}

{%%%%%   MEMO, priklad t7
...}

{%%%%%   MEMO, priklad t8
...} 